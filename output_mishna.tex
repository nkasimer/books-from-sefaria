\documentclass[12pt, openany]{book}
\usepackage[
paperheight=11in,
paperwidth=8.5in,
top=0.5in,
bottom=0.5in,
inner=0.7in,
outer=0.5in,
marginparsep=0.1in,
headsep=16pt
]{geometry}

\newcommand{\texttitle}{משנה פסחים}\usepackage{titlesec}
\usepackage{resources/unnumberedtotoc}

\usepackage{fancyhdr}
\pagestyle{fancy}
\fancyhf{}
\fancyhead[LO,RE]{\thepage}
\fancyhead[CO]{\chapname}
\fancyhead[CE]{\texttitle}

\usepackage{paracol}
\usepackage{anyfontsize}
\usepackage{ragged2e}
\usepackage{polyglossia}
\usepackage{multicol}
\usepackage{hyperref}

\setdefaultlanguage{hebrew}
\setotherlanguage{english}
\usepackage{fontspec}
\setmainfont{Frank Ruehl CLM}
\newfontfamily\englishfont{EB Garamond}

\newcommand{\sethebfont}{
\fontsize{10.5pt}{21.0pt} \selectfont
}

\newcommand{\hebeng}[2]{
	{\sethebfont #1\\}
	
	\begin{english}
		#2
	\end{english}
	\clearpage
}

\newcommand{\twocol}[1]{
	{\sethebfont \begin{multicols}{2}
			#1
	\end{multicols}}	
}

\newcommand{\textblock}[1]{
{\sethebfont #1\\}	
}

\setlength{\parskip}{8pt}

\newcommand{\chapname}{}
\newcommand{\sectname}{}

\newcommand{\newchap}[1]{
	\addcontentsline{toc}{chapter}{#1}
	\renewcommand{\chapname}{#1}
		\begin{center}
			\textbf{%
\fontsize{16pt}{16pt}\selectfont
				#1}
		\end{center}
}

\newcommand{\newsection}[1]{
	\addcontentsline{toc}{section}{#1}
	\renewcommand{\sectname}{#1}	
	\vspace{-\baselineskip}
	\begin{center}
		\textbf{%
\fontsize{16pt}{16pt}\selectfont
			#1}
	\end{center}
	\vspace{-\baselineskip}
	\nopagebreak
}

\begin{document}
\frontmatter
\pagenumbering{roman}

\title{\texttitle}

\author{}

\date{}

\maketitle

\begin{minipage}[b][\textheight][b]{\textwidth}\englishfont	
	\begin{english}
		\vfill
		The following book includes:
\begin{itemize}
\item[$\bullet$] Mishnah, ed. Romm, Vilna 1913
\item[$\bullet$] License: Public Domain
\item[$\bullet$] Source: \url{http://primo.nli.org.il/primo_library/libweb/action/dlDisplay.do?vid=NLI&docId=NNL_ALEPH001741739}
\end{itemize}
		It was retrieved from Sefaria on \today\space \texthebrew{(\Hebrewtoday)}.  It was typeset and formatted by Ktavi, using \LaTeX .
		\clearpage
		
	\end{english}
\end{minipage}


\tableofcontents

\clearpage
\mainmatter
\pagenumbering{arabic}

\newsection{פרק א}
\textblock{אור לארבעה עשר בודקים את החמץ לאור הנר. כל מקום שאין מכניסין בו חמץ אין צריך בדיקה ולמה אמרו שתי שורות במרתף. מקום שמכניסין בו חמץ. בית שמאי אומרים שתי שורות על פני כל המרתף. ובית הלל אומרים שתי שורות החיצונות שהן העליונות:  }
\textblock{אין חוששין שמא גררה חולדה מבית לבית. וממקום למקום. דאם כן מחצר לחצר. ומעיר לעיר אין לדבר סוף:  }
\textblock{רבי יהודה אומר בודקין אור ארבעה עשר. ובארבעה עשר שחרית ובשעת הביעור. וחכמים אומרים לא בדק אור ארבעה עשר. יבדוק בארבעה עשר. לא בדק בארבעה עשר. יבדוק בתוך המועד. לא בדק בתוך המועד יבדוק לאחר המועד ומה שמשייר יניחנו בצינעא. כדי שלא יהא צריך בדיקה אחריו:  }
\textblock{רבי מאיר אומר אוכלין כל חמש ושורפין בתחלת שש. ורבי יהודה אומר אוכלין כל ארבע ותולין כל חמש ושורפין בתחלת שש:  }
\textblock{ועוד אמר רבי יהודה שתי חלות של תודה פסולות ומונחות על גג האיצטבא. כל זמן שמונחות כל העם אוכלים. ניטלה אחת תולין. לא אוכלין. ולא שורפין. ניטלו שתיהן. התחילו כל העם שורפין. ר' גמליאל אומר חולין נאכלים כל ארבע. ותרומה כל חמש. ושורפין כתחילת שש:  }
\textblock{רבי חנינא סגן הכהנים אומר. מימיהם של כהנים לא נמנעו מלשרוף את הבשר שנטמא בוולד הטומאה. עם הבשר שנטמא באב הטומאה. אף על פי שמוסיפין טומאה על טומאתו. הוסיף ר' עקיבא ואמר מימיהם של כהנים. לא נמנעו מלהדליק את השמן שנפסל בטבול יום. בנר שנטמא בטמא מת. אף על פי שמוסיפין טומאה על טומאתו:  }
\textblock{אמר ר' מאיר מדבריהם למדנו. ששורפין תרומה טהורה עם הטמאה בפסח. אמר לו רבי יוסי אינה היא המדה. ומודים רבי אליעזר ורבי יהושע ששורפין זו לעצמה. וזו לעצמה. על מה נחלקו על התלויה ועל הטמאה. שר' אליעזר אומר תשרף זו לעצמה. וזו לעצמה. ורבי יהושע אומר שתיהן כאחת:  }
\newsection{פרק ב}
\textblock{כל שעה שמותר לאכול מאכיל לבהמה לחיה ולעופות. ומוכרו לנכרי ומותר בהנאתו. עבר זמנו. אסור בהנאתו. ולא יסיק בו תנור וכירים. רבי יהודה אומר אין ביעור חמץ אלא שריפה. וחכמים אומרים אף מפרר וזורה לרוח או מטיל לים:  }
\textblock{חמץ של נכרי שעבר עליו הפסח מותר בהנאה. ושל ישראל אסור בהנאה. שנאמר (שמות יג, ז) לא יראה לך שאור:  }
\textblock{נכרי שהלוה את ישראל על חמצו אחר הפסח מותר בהנאה וישראל שהלוה את הנכרי על חמצו אחר הפסח אסור בהנאה. חמץ שנפלה עליו מפולת הרי הוא כמבוער רבן שמעון בן גמליאל אומר כל שאין הכלב יכול לחפש אחריו:  }
\textblock{האוכל תרומת חמץ בפסח. בשוגג משלם קרן וחומש. במזיד פטור מתשלומים ומדמי עצים:  }
\textblock{אלו דברים שאדם יוצא בהן ידי חובתו בפסח בחיטים. בשעורים. בכוסמין ובשיפון. ובשבולת שועל. ויוצאין בדמאי. ובמעשר ראשון שנטלה תרומתו. ובמעשר שני והקדש שנפדו. והכהנים בחלה ובתרומה. אבל לא בטבל. ולא במעשר ראשון שלא נטלה תרומתו ולא במעשר שני והקדש שלא נפדו חלות תודה ורקיקי נזיר. עשאן לעצמו אין יוצאין בהן עשאן למכור בשוק יוצאין בהן:  }
\textblock{ואלו ירקות שאדם יוצא בהן ידי חובתו בפסח. בחזרת. ובעולשין. ובתמכא. ובחרחבינה. ובמרור יוצאין בהן. בין לחין בין יבשין. אבל לא כבושין ולא שלוקין. ולא מבושלין. ומצטרפין לכזית ויוצאין בקלח שלהן. ובדמאי ובמעשר ראשון שנטלה תרומתו. ובמעשר שני והקדש שנפדו:  }
\textblock{אין שורין את המורסן לתרנגולים אבל חולטין. האשה לא תשרה את המורסן שתוליך בידה למרחץ. אבל שפה היא בבשרה יבש לא ילעוס אדם חיטין ויניח על מכתו בפסח. מפני שהן מחמיצות:  }
\textblock{אין נותנין קמח לתוך חרוסת. או לתוך החרדל. ואם נתן יאכל מיד. ורבי מאיר אוסר. אין מבשלין את הפסח לא במשקין. ולא במי פירות אבל סכין ומטבילין אותו בהן מי תשמישו של נחתום ישפכו מפני שהן מחמיצין:  }
\newsection{פרק ג}
\textblock{אלו עוברין בפסח. כותח הבבלי ושכר המדי. וחומץ האדומי. וזיתום המצרי. וזומן של צבעים ועמילן של טבחים. וקולן של סופרים. רבי אליעזר אומר אף תכשיטי נשים. זה הכלל. כל שהוא ממין דגן הרי זה עובר בפסח. הרי אלו באזהרה ואין בהן משום כרת:  }
\textblock{בצק שבסדקי עריבה אם יש כזית במקום אחד חייב לבער. ואם לא בטל במיעוטו. וכן לענין הטומאה. אם מקפיד עליו חוצץ ואם רוצה בקיומו הרי הוא כעריבה. בצק החרש אם יש כיוצא בו שהחמיץ. הרי זה אסור:  }
\textblock{כיצד מפרישין חלה בטומאה ביום טוב רבי אליעזר אומר לא תקרא לה שם עד שתאפה. רבי יהודה בן בתירא אומר תטיל בצונן. אמר רבי יהושע לא זה הוא חמץ שמוזהרים עליו בבל יראה ובבל ימצא. אלא מפרשתה ומנחתה עד הערב. ואם החמיצה. החמיצה:  }
\textblock{רבן גמליאל אומר שלש נשים לשות כאחת ואופות בתנור אחד זו אחר זו. וחכמים אומרים שלש נשים עוסקות בבצק אחת לשה. ואחת עורכת. ואחת אופה רבי עקיבא אומר לא כל הנשים. ולא כל העצים. ולא כל התנורים שוין זה הכלל תפח תלטוש בצונן:  }
\textblock{שיאור ישרף והאוכלו פטור. סידוק ישרף. והאוכלו חייב כרת. איזהו שיאור. כקרני חגבים. סידוק שנתערבו סדקיו זה בזה. דברי רבי יהודה. וחכמים אומרים זה וזה. האוכלו חייב כרת. ואיזהו שיאור. כל שהכסיפו פניו כאדם שעמדו שערותיו:  }
\textblock{ארבעה עשר שחל להיות בשבת. מבערים את הכל מלפני השבת דברי רבי מאיר. וחכמים אומרים בזמנן. רבי אלעזר בר צדוק אומר. תרומה מלפני השבת וחולין בזמנן:  }
\textblock{ההולך לשחוט את פסחו ולמול את בנו. ולאכול סעודת אירוסין בבית חמיו . ונזכר שיש לו חמץ בתוך ביתו. אם יכול לחזור ולבער ולחזור למצותו. יחזור ויבער. ואם לאו. מבטלו בלבו להציל מן הנכרים [ספרים אחרים הגייס]. ומן הנהר. ומן הלסטים. ומן הדליקה. ומן המפולת. יבטל בלבו. ולשבות שביתת הרשות יחזור מיד:  }
\textblock{וכן מי שיצא מירושלם. ונזכר שיש בידו בשר קודש. אם עבר צופים. שורפו במקומו. ואם לאו. חוזר. ושורפו לפני הבירה מעצי המערכה. ועד כמה הן חוזרין רבי מאיר אומר זה וזה בכביצה. רבי יהודה אומר זה וזה בכזית וחכמים אומרים בשר קודש בכזית. וחמץ בכביצה:  }
\newsection{פרק ד}
\textblock{מקום שנהגו לעשות מלאכה בערבי פסחים עד חצות עושין. מקום שנהגו שלא לעשות אין עושין. ההולך ממקום שעושין למקום שאין עושין. או ממקום שאין עושין למקום שעושין. נותנין עליו חומרי מקום שיצא משם. וחומרי מקום שהלך לשם. ואל ישנה אדם מפני המחלוקת:  }
\textblock{כיוצא בו המוליך פירות שביעית ממקום שכלו. למקום שלא כלו. או ממקום שלא כלו למקום שכלו חייב לבער. רבי יהודה אומר אומרים לו צא והבא לך אף אתה:  }
\textblock{מקום שנהגו למכור בהמה דקה לעובד כוכבים מוכרין. מקום שלא נהגו למכור ({\small ספרים אחרים וירושלמי}  שנהגו שלא למכור) אין מוכרין. ובכל מקום אין מוכרין להם בהמה גסה. עגלים וסייחים שלמין ושבורין. רבי יהודה מתיר בשבורה. בן בתירה מתיר בסוס:  }
\textblock{מקום שנהגו לאכול צלי בלילי פסחים אוכלין. מקום שנהגו שלא לאכול אין אוכלין. מקום שנהגו להדליק את הנר בלילי יום הכפורים מדליקין. מקום שנהגו שלא להדליק. אין מדליקין. ומדליקין בבתי כנסיות. ובבתי מדרשות. ובמבואות האפלים. ועל גבי החולים:  }
\textblock{מקום שנהגו לעשות מלאכה בתשעה באב עושין. מקום שנהגו שלא לעשות מלאכה אין עושין. ובכל מקום תלמידי חכמים בטלים. רבן שמעון בן גמליאל אומר לעולם יעשה אדם עצמו תלמיד חכם. וחכמים אומרים ביהודה היו עושין מלאכה בערבי פסחים עד חצות. ובגליל לא היו עושין כל עיקר. והלילה. בית שמאי אוסרין. ובית הלל מתירין עד הנץ החמה:  }
\textblock{ר' מאיר אומר כל מלאכה שהתחיל בה קודם לארבעה עשר גומרה בארבעה עשר. אבל לא יתחיל בה בתחילה בארבעה עשר. אף על פי שיכול לגומרה. וחכמים אומרים שלש אומניות עושין מלאכה בערבי פסחים עד חצות. ואלו הן. החייטים. הספרים. והכובסין ר' יוסי בר יהודה אומר אף הרצענים:  }
\textblock{מושיבין שובכין לתרנגולים בארבעה עשר. ותרנגולת שברחה. מחזירין אותה למקומה. ואם מתה מושיבין אחרת תחתיה. גורפין מתחת רגלי בהמה בארבעה עשר. ובמועד מסלקין לצדדין מוליכין ומביאין כלים מבית האומן אף על פי שאינם לצורך המועד:  }
\textblock{ששה דברים עשו אנשי יריחו. על שלשה מיחו בידם. ועל שלשה לא מיחו בידם. ואלו הן. שלא מיחו בידם. מרכיבין דקלים כל היום. וכורכין את שמע. וקוצרין וגודשין לפני העומר. ולא מיחו בידם. ואלו שמיחו בידם. מתירין גמזיות של הקדש. ואוכלין מתחת הנשרים בשבת. ונותנים פאה לירק. ומיחו בידם חכמים:  }
\textblock{ששה דברים עשה חזקיה המלך. על שלשה הודו לו. ועל שלשה לא הודו לו. גירר עצמות אביו על מטה של חבלים והודו לו. כיתת נחש הנחשת והודו לו. גנז ספר רפואות והודו לו. על שלשה לא הודו לו. קיצץ דלתות של היכל ושיגרן למלך אשור ולא הודו לו. סתם מי גיחון העליון ולא הודו לו. עיבר ניסן בניסן ולא הודו לו:  }
\newsection{פרק ה}
\textblock{תמיד נשחט בשמנה ומחצה. וקרב בתשע ומחצה. בערבי פסחים נשחט בשבע ומחצה. וקרב בשמנה ומחצה. בין בחול בין בשבת. חל ערב פסח להיות בערב שבת. נשחט בשש ומחצה וקרב בשבע ומחצה. והפסח אחריו:  }
\textblock{הפסח ששחטו שלא לשמו. וקבל והלך וזרק שלא לשמו. או לשמו ושלא לשמו. או שלא לשמו ולשמו פסול. כיצד לשמו ושלא לשמו. לשם פסח ולשם שלמים. שלא לשמו ולשמו. לשם שלמים ולשם פסח:  }
\textblock{שחטו שלא לאוכליו ושלא למנוייו לערלים ולטמאים פסול. לאוכליו ושלא לאוכליו. למנוייו ושלא למנוייו. למולים ולערלים. לטמאים ולטהורים. כשר. שחטו קודם חצות פסול. משום שנאמר (ויקרא כג, ה) בין הערבים. שחטו קודם לתמיד כשר. ובלבד שיהא אחד ממרס בדמו עד שיזרק דם התמיד. ואם נזרק כשר:  }
\textblock{השוחט את הפסח על החמץ. עובר בלא תעשה. רבי יהודה אומר אף התמיד רבי שמעון אומר הפסח בארבעה עשר לשמו חייב. ושלא לשמו פטור. ושאר כל הזבחים בין לשמן ובין שלא לשמן פטור. ובמועד לשמו פטור. שלא לשמו חייב. ושאר כל הזבחים בין לשמן בין שלא לשמן חייב. חוץ מן החטאת ששחט שלא לשמה:  }
\textblock{הפסח נשחט בשלש כתות. שנאמר (שמות יב, ו) ושחטו אותו כל קהל עדת ישראל. קהל. ועדה. וישראל נכנסה כת הראשונה נתמלאת העזרה. נעלו דלתות העזרה תקעו הריעו ותקעו. הכהנים עומדים שורות שורות ובידיהם בזיכי כסף ובזיכי זהב. שורה שכולה כסף כסף. ושורה שכולה זהב זהב. לא היו מעורבין ולא היו לבזיכין שוליים שמא יניחום ויקרש הדם:  }
\textblock{שחט ישראל וקבל הכהן. נותנו לחבירו. וחבירו לחבירו. ומקבל את המלא. ומחזיר את הריקן. כהן הקרוב אצל המזבח. זורקו זריקה אחת כנגד היסוד:  }
\textblock{יצתה כת ראשונה ונכנסה כת שניה. יצתה שניה נכנסה שלישית. כמעשה הראשונה כך מעשה השניה והשלישית. קראו את ההלל אם גמרו שנו. ואם שנו שלשו. אף על פי שלא שלשו מימיהם. רבי יהודה אומר מימיהם של כת שלישית לא הגיע לאהבתי כי ישמע ה'. מפני שעמה מועטין:  }
\textblock{כמעשהו בחול כך מעשהו בשבת. אלא שהכהנים מדיחים את העזרה שלא כרצון חכמים. ר' יהודה אומר כוס היה ממלא מדם התערובות. זרקו זריקה אחת על גבי המזבח ולא הודו לו חכמים:  }
\textblock{כיצד תולין ומפשיטין. אונקליות של ברזל היו קבועים בכתלים ובעמודים שבהן תולין ומפשיטין. וכל מי שאין לו מקום לתלות ולהפשיט. מקלות דקים חלקים היו שם ומניח על כתפו ועל כתף חבירו ותולה ומפשיט. ר' אליעזר אומר ארבעה עשר שחל להיות בשבת מניח ידו על כתף חברו. ויד חברו על כתפו. ותולה ומפשיט:  }
\textblock{קרעו והוציא אימוריו. נתנו במגיס. והקטירן על גבי המזבח. יצתה כת ראשונה וישבה לה בהר הבית שניה בחיל והשלישית במקומה עומדת. חשיכה יצאו וצלו את פסחיהן:  }
\newsection{פרק ו}
\textblock{אלו דברים בפסח דוחין את השבת. שחיטתו. וזריקת דמו. ומיחוי קרביו. והקטר חלביו. אבל צלייתו והדחת קרביו. אינן דוחין את השבת. הרכבתו והבאתו מחוץ לתחום. וחתיכת יבלתו. אין דוחין את השבת. רבי אליעזר אומר דוחין:  }
\textblock{אמר רבי אליעזר והלא דין הוא. מה אם שחיטה שהיא משום מלאכה דוחה את השבת. אלו שהן משום שבות לא ידחו את השבת. אמר לו רבי יהושע יום טוב יוכיח שהתירו בו משום מלאכה ואסור בו משום שבות. אמר לו רבי אליעזר. מה זה יהושע מה ראיה רשות למצוה. השיב רבי עקיבא ואמר הזאה תוכיח. שהיא מצוה והיא משום שבות ואינה דוחה את השבת. אף אתה אל תתמה על אלו. שאף על פי שהן מצוה. והן משום שבות. לא ידחו את השבת. אמר לו רבי אליעזר ועליה אני דן. ומה אם שחיטה שהיא משום מלאכה דוחה את השבת. הזאה שהיא משום שבות אינו דין שדוחה את השבת. אמר לו רבי עקיבא או חלוף. מה אם הזאה שהיא משום שבות אינה דוחה את השבת. שחיטה שהיא משום מלאכה אינו דין שלא תדחה את השבת. אמר לו רבי אליעזר. עקיבא עקרת מה שכתוב בתורה. בין הערבים במועדו. בין בחול. בין בשבת. אמר לו רבי הבא לי מועד לאלו. כמועד לשחיטה. כלל אמר רבי עקיבא. כל מלאכה שאפשר לעשותה מערב שבת אינה דוחה את השבת. שחיטה שאי אפשר לעשותה מערב שבת. דוחה את השבת:  }
\textblock{אימתי מביא חגיגה עמו. בזמן שהוא בא בחול בטהרה ובמועט. ובזמן שהוא בא בשבת במרובה ובטומאה אין מביאין עמו חגיגה:  }
\textblock{חגיגה היתה באה מן הצאן. מן הבקר. מן הכבשים ומן העזים. מן הזכרים. ומן הנקבות. ונאכלת לשני ימים ולילה אחד:  }
\textblock{הפסח ששחטו שלא לשמו בשבת. חייב עליו חטאת. ושאר כל הזבחים ששחטן לשום פסח. אם אינן ראויין חייב ואם ראויין הן. רבי אליעזר מחייב חטאת ורבי יהושע פוטר. אמר רבי אליעזר מה אם הפסח שהוא מותר לשמו. כששינה את שמו חייב. זבחים שהן אסורין לשמן. כששינה את שמן. אינו דין שיהא חייב. אמר לו רבי יהושע לא. אם אמרת בפסח ששינהו לדבר אסור. תאמר בזבחים ששינן לדבר המותר. אמר ליה רבי אליעזר אימורי ציבור יוכיחו שהן מותרין לשמן. והשוחט לשמן חייב. אמר לו רבי יהושע לא אם אמרת באימורי ציבור שיש להן קצבה תאמר בפסח שאין לו קצבה. רבי מאיר אומר אף השוחט לשם אימורי צבור פטור:  }
\textblock{שחטו שלא לאוכליו ושלא למנוייו. לערלין ולטמאין חייב. לאוכליו ושלא לאוכליו. למנוייו ושלא למנוייו. למולין ולערלין לטהורים ולטמאים פטור. שחטו ונמצא בעל מום חייב שחטו ונמצא טריפה בסתר פטור שחטו ונודע שמשכו הבעלים את ידם. או שמתו. או שנטמאו פטור. מפני ששחט ברשות:  }
\newsection{פרק ז}
\textblock{כיצד צולין את הפסח. מביאין שפוד של רימון תוחבו מתוך פיו עד בית נקובתו. ונותן את כרעיו ואת בני מעיו לתוכו. דברי רבי יוסי הגלילי. רבי עקיבא אומר כמין בשול הוא זה. אלא תולין חוצה לו:  }
\textblock{אין צולין את הפסח לא על השפוד ולא על האסכלא. אמר רבי צדוק מעשה ברבן גמליאל שאמר לטבי עבדו צא וצלה לנו את הפסח על האסכלא. נגע בחרסו של תנור יקלף את מקומו. נטף מרוטבו על החרס וחזר עליו יטול את מקומו. נטף מרוטבו על הסולת. יקמוץ את מקומו:  }
\textblock{סכו בשמן תרומה. אם חבורת כהנים יאכלו. אם ישראל אם חי הוא ידיחנו. ואם צלי הוא יקלוף את החיצון. סכו בשמן של מעשר שני. לא יעשנו דמים על בני חבורה. שאין פודין מעשר שני בירושלם:  }
\textblock{חמשה דברים באין בטומאה. ואינן נאכלין בטומאה. העומר. ושתי הלחם. ולחם הפנים. וזבחי שלמי צבור. ושעירי ראשי חדשים. הפסח שבא בטומאה נאכל בטומאה. שלא בא מתחלתו אלא לאכילה:  }
\textblock{נטמא הבשר. והחלב קיים. אינו זורק את הדם. נטמא החלב. והבשר קיים. זורק את הדם. ובמוקדשין אינו כן. אלא אף על פי שנטמא הבשר והחלב קיים. זורק את הדם:  }
\textblock{נטמא קהל או רובו. או שהיו הכהנים טמאים והקהל טהורים. יעשה בטומאה. נטמא מיעוט הקהל. הטהורין עושין את הראשון והטמאין עושין את השני:  }
\textblock{הפסח שנזרק דמו ואחר כך נודע שהוא טמא. הציץ מרצה. נטמא הגוף אין הציץ מרצה. מפני שאמרו. הנזיר ועושה פסח. הציץ מרצה על טומאת הדם. ואין הציץ מרצה על טומאת הגוף. נטמא טומאת התהום הציץ מרצה:  }
\textblock{נטמא שלם או רובו. שורפין אותו לפני הבירה מעצי המערכה. נטמא מיעוטו. והנותר. שורפין אותו בחצרותיהן. או על גגותיהן מעצי עצמן. הציקנין שורפין אותו לפני הבירה. בשביל ליהנות מעצי המערכה:  }
\textblock{הפסח שיצא או שנטמא. ישרף מיד. נטמאו הבעלים או שמתו. תעובר צורתו וישרף בששה עשר. רבי יוחנן בן ברוקה אומר אף זה ישרף מיד לפי שאין לו אוכלין:  }
\textblock{העצמות. והגידין. והנותר. ישרפו בששה עשר. חל ששה עשר להיות בשבת. ישרפו בשבעה עשר. לפי שאינן דוחין לא את השבת ולא את יום טוב:  }
\textblock{כל הנאכל בשור הגדול יאכל בגדי הרך. וראשי כנפים. והסחוסים. השובר את העצם בפסח הטהור. הרי זה לוקה ארבעים. אבל המותיר בטהור. והשובר בטמא אינו לוקה את הארבעים:  }
\textblock{אבר שיצא מקצתו חותך עד שמגיע לעצם. וקולף עד שמגיע לפרק וחותך. ובמוקדשין. קוצץ בקופיץ שאין בו משום שבירת העצם. מן האגף ולפנים כלפנים. מן האגף ולחוץ. כלחוץ. החלונות. ועובי החומה. כלפנים:  }
\textblock{שתי חבורות שהיו אוכלות בבית אחד. אלו הופכין את פניהם הילך ואוכלין. ואלו הופכין את פניהם. הילך ואוכלין. והמיחם באמצע כשהשמש עומד למזוג קופץ את פיו. ומחזיר את פניו. עד שמגיע אצל חבורתו ואוכל. והכלה הופכת את פניה ואוכלת:  }
\newsection{פרק ח}
\textblock{האשה בזמן שהיא בבית בעלה. שחט עליה בעלה. ושחט עליה אביה. תאכל משל בעלה. הלכה רגל ראשון לעשות בבית אביה. שחט עליה אביה. ושחט עליה בעלה. תאכל במקום שהיא רוצה. יתום ששחטו עליו אפטרופסין יאכל במקום שהוא רוצה. עבד של שני שותפין. לא יאכל משל שניהן. מי שחציו עבד. וחציו בן חורין. לא יאכל משל רבו: 
}
\textblock{האומר לעבדו צא ושחוט עלי את הפסח. שחט גדי. יאכל. שחט טלה יאכל. שחט גדי וטלה. יאכל מן הראשון. שכח מה אמר לו רבו. כיצד יעשה. ישחט טלה וגדי ויאמר. אם גדי אמר לי רבי. גדי שלו. וטלה שלי ואם טלה אמר לי רבי. הטלה שלו וגדי שלי. שכח רבו מה אמר לו. שניהם יצאו לבית השריפה. ופטורין מלעשות פסח שני: 
}
\textblock{האומר לבניו הריני שוחט את הפסח על מי שיעלה מכם ראשון לירושלם. כיון שהכניס הראשון ראשו ורובו זכה בחלקו ומזכה את אחיו עמו. לעולם נימנין עליו עד שיהא בו כזית לכל אחד ואחד. נימנין ומושכין את ידיהן ממנו עד שישחט. רבי שמעון אומר. עד שיזרוק עליו את הדם: 
}
\textblock{הממנה עמו אחרים בחלקו. רשאין בני חבורה ליתן לו את שלו. והוא אוכל משלו. והן אוכלין משלהן: 
}
\textblock{זב שראה שתי ראיות. שוחטין עליו בשביעי. ראה שלש שוחטין עליו בשמיני שלו שומרת יום כנגד יום. שוחטין עליה בשני שלה. ראתה שני ימים. שוחטין עליה בשלישי והזבה שוחטין עליה בשמיני: 
}
\textblock{האונן והמפקח את הגל. וכן מי שהבטיחוהו להוציאו מבית האסורים. והחולה. והזקן. שהן יכולין לאכול כזית שוחטין עליהן. על כולן אין שוחטין עליהן בפני עצמן. שמא יביאו את הפסח לידי פסול. לפיכך אם אירע בהן פסול. פטורין מלעשות פסח שני. חוץ מן המפקח בגל שהוא טמא מתחלתו: 
}
\textblock{אין שוחטין את הפסח על היחיד. דברי רבי יהודה. ורבי יוסי מתיר. אפילו חבורה של מאה שאין יכולין לאכול כזית. אין שוחטין עליהן. ואין עושין חבורת נשים ועבדים וקטנים: 
}
\textblock{אונן טובל ואוכל את פסחו לערב. אבל לא בקדשים. השומע על מתו. והמלקט לו עצמות. טובל ואוכל בקדשים גר שנתגייר בערב פסח. בית שמאי אומרים טובל ואוכל את פסחו לערב. ובית הלל אומרים הפורש מן הערלה כפורש מן הקבר: 
}
\newsection{פרק ט}
\textblock{מי שהיה טמא או בדרך רחוקה ולא עשה את הראשון יעשה את השני. שגג או נאנס ולא עשה את הראשון יעשה את השני. אם כן למה נאמר טמא או שהיה בדרך רחוקה. שאלו פטורין מהכרת. ואלו חייבין בהכרת: }
\textblock{איזו היא דרך רחוקה. מן המודיעים ולחוץ. וכמדתה לכל רוח. דברי ר' עקיבא. ר' אליעזר אומר מאיסקופת העזרה ולחוץ. אמר ר' יוסי לפיכך נקוד על ה' לומר לא מפני שרחוקה ודאי. אלא מאיסקופת העזרה ולחוץ: }
\textblock{מה בין פסח הראשון לשני. הראשון אסור בבל יראה ובל ימצא. והשני מצה וחמץ עמו בבית. הראשון טעון הלל באכילתו. והשני אינו טעון הלל באכילתו זה וזה טעון הלל בעשייתן. ונאכלין צלי על מצה ומרורים. ודוחין את השבת: }
\textblock{הפסח שבא בטומאה לא יאכלו ממנו זבין וזבות. נדות ויולדות. ואם אכלו פטורים מכרת. רבי אליעזר פוטר אף על ביאת מקדש: }
\textblock{מה בין פסח מצרים לפסח דורות. פסח מצרים מקחו מבעשור וטעון הזאה באגודת אזוב על המשקוף ועל שתי מזוזות. ונאכל בחפזון בלילה אחד. ופסח דורות נוהג כל שבעה: }
\textblock{אמר ר' יהושע שמעתי שתמורת הפסח קריבה. ותמורת הפסח אינה קריבה. ואין לי לפרש. אמר רבי עקיבא אני אפרש. הפסח שנמצא קודם שחיטת הפסח. ירעה עד שיסתאב וימכר. ויקח בדמיו שלמים. וכן תמורתו. אחר שחיטת הפסח. קרב שלמים וכן תמורתו: }
\textblock{המפריש נקבה לפסחו. או זכר בן שתי שנים ירעה עד שיסתאב וימכר. ויפלו דמיו לנדבה. המפריש פסחו ומת. לא יביאנו בנו אחריו לשם פסח. אלא לשם שלמים: }
\textblock{הפסח שנתערב בזבחים. כולן ירעו עד שיסתאבו. וימכרו ויביא בדמי היפה שבהן ממין זה. ובדמי היפה שבהן ממין זה. ויפסיד המותר מביתו. נתערב בבכורות רבי שמעון אומר אם חבורת כהנים יאכלו: }
\textblock{חבורה שאבדה פסחה. ואמרה לאחד צא ובקש ושחוט עלינו. והלך ומצא ושחט. והם לקחו ושחטו. אם שלו נשחט ראשון הוא אוכל משלו. והם אוכלים עמו משלו. ואם שלהן נשחט ראשון. הם אוכלין משלהן והוא אוכל משלו. ואם אינו ידוע איזה מהן נשחט ראשון. או ששחטו שניהן כאחד. הוא אוכל משלו. והם אינם אוכלים עמו. ושלהן יצא לבית השריפה. ופטורין מלעשות פסח שני. אמר להן אם אחרתי. צאו ושחטו עלי. הלך ומצאו ושחט והן לקחו ושחטו. אם שלהן נשחט ראשון. הן אוכלין משלהן. והוא אוכל עמהן. ואם שלו נשחט ראשון. הוא אוכל משלו. והן אוכלין משלהן. ואם אינו ידוע איזה מהם נשחט ראשון. או ששחטו שניהם כאחד. הן אוכלין משלהן. והוא אינו אוכל עמהן. ושלו יצא לבית השריפה ופטור מלעשות פסח שני. אמר להן ואמרו לו אוכלין כולם מן הראשון. ואם אין ידוע איזה מהן נשחט ראשון שניהן יוצאין לבית השריפה. לא אמר להן. ולא אמרו לו. אינן אחראין זה לזה: }
\textblock{שתי חבורות שנתערבו פסחיהן. אלו מושכין להן אחד ואלו מושכין להן אחד. אחד מאלו בא לו אצל אלו. ואחד מאלו בא לו אצל אלו. וכך הם אומרים אם שלנו הוא הפסח הזה. ידיך משוכות משלך. ונמנית על שלנו. ואם שלך הוא הפסח הזה ידינו משוכות משלנו ונמנינו על שלך וכן חמש חבורות של חמשה חמשה ושל עשרה עשרה. מושכין להן אחד מכל חבורה וחבורה. וכן הם אומרים: }
\textblock{שנים שנתערבו פסחיהם. זה מושך לו אחד. וזה מושך לו אחד. זה ממנה עמו אחד מן השוק. וזה ממנה עמו אחד מן השוק. זה בא אצל זה. וזה בא אצל זה. וכך הם אומרים אם שלי הוא פסח זה. ידיך משוכות משלך ונמנית על שלי. ואם שלך הוא פסח זה. ידי משוכות משלי. ונמניתי על שלך: }
\newsection{פרק י}
\textblock{ערבי פסחים סמוך למנחה. לא יאכל אדם עד שתחשך. ואפילו עני שבישראל לא יאכל עד שיסב. ולא יפחתו לו מארבע כוסות של יין. ואפילו מן התמחוי:  }
\textblock{מזגו לו כוס ראשון. בית שמאי אומרים מברך על היום. ואחר כך מברך על היין. ובית הלל אומרים מברך על היין. ואחר כך מברך על היום:  }
\textblock{הביאו לפניו מטבל בחזרת. עד שמגיע לפרפרת הפת. הביאו לפניו מצה וחזרת. וחרוסת. ושני תבשילין. אף על פי שאין חרוסת מצוה. רבי אליעזר בר צדוק אומר. מצוה. ובמקדש. היו מביאים לפניו גופו של פסח:  }
\textblock{מזגו לו כוס שני. וכאן הבן שואל אביו. ואם אין דעת בבן. אביו מלמדו. מה נשתנה הלילה הזה מכל הלילות. שבכל הלילות אנו אוכלין חמץ ומצה. הלילה הזה כולו מצה. שבכל הלילות. אנו אוכלין שאר ירקות. הלילה הזה מרור. שבכל הלילות. אנו אוכלין בשר צלי. שלוק. ומבושל. הלילה הזה כולו צלי. שבכל הלילות. אנו מטבילין פעם אחת. הלילה הזה שתי פעמים. ולפי דעתו של בן. אביו מלמדו. מתחיל בגנות. ומסיים בשבח. ודורש מארמי אובד אבי. עד שיגמור כל הפרשה כולה:  }
\textblock{רבן גמליאל היה אומר. כל שלא אמר שלשה דברים אלו בפסח. לא יצא ידי חובתו. ואלו הן. פסח. מצה. ומרור. פסח. על שום שפסח המקום על בתי אבותינו במצרים. מצה. על שום שנגאלו אבותינו במצרים. מרור. על שום שמררו המצריים את חיי אבותינו במצרים. בכל דור ודור חייב אדם לראות את עצמו. כאילו הוא יצא ממצרים. שנאמר (שמות יג, ח) והגדת לבנך ביום ההוא לאמר. בעבור זה עשה ה' לי בצאתי ממצרים. לפיכך אנחנו חייבין להודות. להלל. לשבח. לפאר. לרומם להדר. לברך. לעלה ולקלס. למי שעשה לאבותינו. ולנו את כל הניסים האלו. הוציאנו מעבדות לחירות. מיגון לשמחה. ומאבל ליום טוב. ומאפילה לאור גדול ומשעבוד לגאולה ונאמר לפניו הללויה:  }
\textblock{עד היכן הוא אומר. בית שמאי אומרים. עד אם הבנים שמחה. ובית הלל אומרים עד חלמיש למעינו מים וחותם בגאולה. רבי טרפון אומר. אשר גאלנו. וגאל את אבותינו ממצרים. ולא היה חותם. רבי עקיבא אומר. כן ה' אלהינו ואלהי אבותינו יגיענו למועדים ולרגלים אחרים הבאים לקראתינו לשלום. שמחים בבנין עירך וששים בעבודתך. ונאכל שם מן הזבחים ומן הפסחים כו' עד ברוך אתה ה' גאל ישראל:  }
\textblock{מזגו לו כוס שלישי מברך על מזונו. רביעי גומר עליו את ההלל. ואומר עליו ברכת השיר. בין הכוסות הללו אם רוצה לשתות ישתה. בין שלישי לרביעי. לא ישתה:  }
\textblock{ואין מפטירין אחר הפסח אפיקומן. ישנו מקצתן יאכלו. כולן לא יאכלו. ר' יוסי אומר. נתנמנמו. יאכלו. נרדמו לא יאכלו:  }
\textblock{הפסח אחר חצות. מטמא את הידים. הפגול והנותר. מטמאין את הידים. בירך ברכת הפסח. פטר את של זבח. בירך את של זבח. לא פטר את של פסח. דברי רבי ישמעאל. ר' עקיבא אומר. לא זו פוטרת זו. ולא זו פוטרת זו:   }

\end{document}
