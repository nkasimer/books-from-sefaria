\documentclass[12pt, openany]{book}
\usepackage[
paperheight=11in,
paperwidth=8.5in,
top=0.5in,
bottom=0.5in,
inner=0.7in,
outer=0.5in,
marginparsep=0.1in,
headsep=16pt
]{geometry}

\newcommand{\texttitle}{עירובין}\usepackage{titlesec}
\usepackage{resources/unnumberedtotoc}

\usepackage{fancyhdr}
\pagestyle{fancy}
\fancyhf{}
\fancyhead[LO,RE]{\thepage}
\fancyhead[CO]{\chapname}
\fancyhead[CE]{\texttitle}

\usepackage{paracol}
\usepackage{anyfontsize}
\usepackage{ragged2e}
\usepackage{polyglossia}
\usepackage{multicol}
\usepackage{hyperref}

\setdefaultlanguage{hebrew}
\setotherlanguage{english}
\usepackage{fontspec}
\setmainfont{Frank Ruehl CLM}
\newfontfamily\englishfont{EB Garamond}

\newcommand{\sethebfont}{
\fontsize{10.5pt}{21.0pt} \selectfont
}

\newcommand{\hebeng}[2]{
	{\sethebfont #1\\}
	
	\begin{english}
		#2
	\end{english}
	\clearpage
}

\newcommand{\twocol}[1]{
	{\sethebfont \begin{multicols}{2}
			#1
	\end{multicols}}	
}

\newcommand{\textblock}[1]{
{\sethebfont #1\\}	
}

\setlength{\parskip}{8pt}

\newcommand{\chapname}{}
\newcommand{\sectname}{}

\newcommand{\newchap}[1]{
	\addcontentsline{toc}{chapter}{#1}
	\renewcommand{\chapname}{#1}
		\begin{center}
			\textbf{%
\fontsize{16pt}{16pt}\selectfont
				#1}
		\end{center}
}

\newcommand{\newsection}[1]{
	\addcontentsline{toc}{section}{#1}
	\renewcommand{\sectname}{#1}	
	\vspace{-\baselineskip}
	\begin{center}
		\textbf{%
\fontsize{16pt}{16pt}\selectfont
			#1}
	\end{center}
	\vspace{-\baselineskip}
	\nopagebreak
}

\begin{document}
\frontmatter
\pagenumbering{roman}

\title{\texttitle}

\author{}

\date{}

\maketitle

\begin{minipage}[b][\textheight][b]{\textwidth}\englishfont	
	\begin{english}
		\vfill
		The following book includes:
\begin{itemize}
\item[$\bullet$] Wikisource Talmud Bavli
\item[$\bullet$] License: CC-BY
\item[$\bullet$] Source: \url{http://he.wikisource.org/wiki/%D7%AA%D7%9C%D7%9E%D7%95%D7%93_%D7%91%D7%91%D7%9C%D7%99}
\end{itemize}
		It was retrieved from Sefaria on \today\space \texthebrew{(\Hebrewtoday)}.  It was typeset and formatted by Ktavi, using \LaTeX .
		\clearpage
		
	\end{english}
\end{minipage}


\tableofcontents

\clearpage
\mainmatter
\pagenumbering{arabic}

\newchap{פרק \hebrewnumeral{1}\quad מבוי}
\newsection{דף ב}
\textblock{}
\textblock{מתני׳ {\large\emph{מבוי}} שהוא גבוה למעלה מעשרים אמה ימעט ר' יהודה אומר אינו צריך}
\textblock{והרחב מעשר אמות ימעט ואם יש לו צורת הפתח אע"פ שהוא רחב מעשר אמות אין צריך למעט:}
\textblock{{\large\emph{גמ׳}} תנן התם סוכה שהיא גבוהה למעלה מעשרים אמה פסולה ורבי יהודה מכשיר מאי שנא גבי סוכה דתני פסולה וגבי מבוי תני תקנתא}
\textblock{סוכה דאורייתא תני פסולה מבוי דרבנן תני תקנתא}
\textblock{ואיבעית אימא דאורייתא נמי תני תקנתא אלא סוכה דנפישין מיליה פסיק ותני פסולה מבוי דלא נפישי מיליה תני תקנתא}
\textblock{אמר רב יהודה אמר רב חכמים לא למדוה אלא מפתחו של היכל ורבי יהודה לא למדה אלא מפתחו של אולם}
\textblock{דתנן פתחו של היכל גבהו עשרים אמה ורחבו עשר אמות ושל אולם גבהו ארבעים אמה ורחבו עשרים אמות}
\textblock{ושניהן מקרא אחד דרשו (ויקרא ג, ב) ושחטו פתח אהל מועד דרבנן סברי קדושת היכל לחוד וקדושת אולם לחוד וכי כתיב פתח אהל מועד אהיכל כתיב}
\textblock{ורבי יהודה סבר היכל ואולם קדושה אחת היא וכי כתיב פתח אהל מועד אתרוייהו הוא דכתיב}
\textblock{ואיבעית אימא לר' יהודה נמי קדושת אולם לחוד וקדושת היכל לחוד והכא היינו טעמא דרבי יהודה דכתיב אל פתח אולם הבית}
\textblock{ורבנן אי הוה כתב אל פתח אולם כדקאמרת השתא דכתיב אל פתח אולם הבית הבית הפתוח לאולם}
\textblock{והא כי כתיב האי במשכן כתיב}
\textblock{אשכחן משכן דאיקרי מקדש ומקדש דאיקרי משכן דאי לא תימא הכי הא דאמר רב יהודה אמר שמואל שלמים ששחטן קודם פתיחת דלתות ההיכל פסולין שנאמר (ויקרא ג, ב) ושחטו פתח אהל מועד בזמן שפתוחין ולא בזמן שהן נעולים והא כי כתיב ההיא במשכן כתיב אלא אשכחן מקדש דאיקרי משכן ומשכן דאיקרי מקדש}
\textblock{}
\textblock{בשלמא מקדש דאיקרי משכן דכתיב (ויקרא כו, ד) ונתתי (את) משכני בתוככם אלא משכן דאיקרי מקדש מנלן אילימא מדכתיב (במדבר י, כא) (ונשאו) הקהתים נושאי המקדש והקימו את המשכן עד בואם}
\newchap{פרק \hebrewnumeral{1}\quad מבוי}
\textblock{}
\textblock{ההוא בארון כתיב אלא מהכא (שמות כה, ח) ועשו לי מקדש ושכנתי בתוכם}
\textblock{בין לרבנן ובין לרבי יהודה לילפו מפתח שער החצר דכתיב (שמות כז, יח) אורך החצר מאה באמה ורחב חמשים בחמשים וקומה חמש אמות וכתיב (שמות כז, יד) וחמש עשרה אמה קלעים לכתף וכתיב (שמות לח, טו) ולכתף השנית מזה ומזה לשער החצר קלעים חמש עשרה אמה מה להלן חמש ברוחב עשרים אף כאן חמש ברוחב עשרים}
\textblock{פתח שער החצר איקרי פתח סתמא לא איקרי}
\textblock{ואיבעית אימא כי כתיב קלעים חמש עשרה אמה לכתף בגובהה הוא דכתיב}
\textblock{גובהה והא כתיב וקומה חמש אמות ההוא משפת מזבח ולמעלה}
\textblock{ורבי יהודה מפתחו של אולם גמר והא תנן והרחב מעשר אמות ימעט ולא פליג ר' יהודה}
\textblock{אמר אביי פליג בברייתא דתניא והרחב מי' אמות ימעט ר' יהודה אומר אינו צריך למעט}
\textblock{וליפלוג במתני' פליג בגובהה וה"ה לרחבה}
\textblock{ואכתי ר' יהודה מפתחו של אולם גמר והתניא מבוי שהוא גבוה מכ' אמה ימעט ורבי יהודה מכשיר עד מ' ונ' אמה ותני בר קפרא עד מאה}
\textblock{בשלמא לבר קפרא גוזמא אלא לרב יהודה מאי גוזמא בשלמא לרבי יהודה ארבעים גמר מפתחו של אולם אלא נ' מנא ליה}
\textblock{א"ר חסדא הא מתניתא אטעיתיה לרב דתניא מבוי שהוא גבוה מכ' אמה יותר מפתחו של היכל ימעט הוא סבר מדרבנן מפתחו של היכל גמרי רבי יהודה מפתחו של אולם גמר ולא היא ר' יהודה מפתחא דמלכין גמר}
\textblock{ורבנן אי מפתחו של היכל גמירי ליבעו דלתות כהיכל אלמה תנן הכשר מבוי ב"ש אומרים לחי וקורה וב"ה אומרים לחי או קורה}
\textblock{דלתות היכל לצניעות בעלמא הוא דעבידן}
\textblock{אלא מעתה לא תיהני ליה צורת הפתח דהא היכל צורת הפתח הויא לו אפילו הכי עשר אמות הוא דרויח אלמה תנן אם יש לו צורת הפתח אע"פ שרחב מעשר אמות אינו צריך למעט}
\textblock{מידי הוא טעמא אלא לרב הא מתני ליה רב יהודה לחייא בר רב קמיה דרב אינו צריך למעט וא"ל אתנייה צריך למעט}
\textblock{אלא מעתה}
\newsection{דף ג}
\textblock{לא תיהני ליה אמלתרא דהא היכל אמלתרא הויא ליה ואפי' הכי עשרים אמה הוא דגבוה דתנן חמש אמלתראות של מילה היו על גביו זו למעלה מזו וזו למעלה מזו}
\textblock{והאי מאי תיובתא דילמא כי תניא ההיא דאמלתראות באולם תניא}
\textblock{והאי מאי קושיא דילמא תבנית היכל כתבנית אולם}
\textblock{אלמה אמר רבי אילעא אמר רב רחבה ד' אע"פ שאינה בריאה ואם יש לה אמלתרא אפי' גבוהה יותר מעשרים אמה אינו צריך למעט}
\textblock{אמר רב יוסף אמלתרא מתניתא היא מאן קתני לה}
\textblock{אמר אביי והא חמא בריה דרבה בר אבוה קתני לה ותיהוי אמלתרא מתניתא ותיקשי לרב}
\textblock{אמר לך רב דל אנא מהכא מתנייתא מי לא קשיין אהדדי אלא מאי אית לך למימר תנאי היא לדידי נמי תנאי היא}
\textblock{רב נחמן בר יצחק אמר בלא רב מתנייתא אהדדי לא קשיין לרבנן קורה טעמא מאי משום היכרא והאי דקתני יתר מפתחו של היכל סימנא בעלמא}
\textblock{ורב נחמן בר יצחק הניחא אי לא סבירא ליה הא דרבה אלא אי סבירא ליה הא דרבה דאמר רבה כתיב (ויקרא כג, מג) למען ידעו דורותיכם כי בסכות הושבתי עד עשרים אמה אדם יודע שדר בסוכה למעלה מעשרים אמה אין אדם יודע משום דלא שלטא ביה עינא}
\textblock{אלמא גבי סוכה נמי בהיכרא פליגי איפלוגי בתרתי למה לי}
\textblock{צריכא דאי אשמעינן גבי סוכה בהא קאמר ר' יהודה כיון דלישיבה עבידא שלטא ביה עינא אבל מבוי דלהילוך עביד אימא מודה להו לרבנן ואי אשמעינן בהא בהא קאמרי רבנן אבל בהך אימא מודו ליה לר' יהודה צריכא}
\textblock{מאי אמלתרא רב חמא בריה דרבה בר אבוה אמר קיני כי אתא רב דימי אמר אמרי במערבא פסקי דארזא}
\textblock{מאן דאמר פסקי דארזא כ"ש קיני מ"ד קיני אבל פסקי דארזא לא}
\textblock{ומ"ד פסקי דארזא מ"ט משום דנפיש משכיה והא סוכה דנפיש משכיה וקאמרי רבנן דלא}
\textblock{אלא כיון דקא חשיב אית ליה קלא:}
\textblock{מקצת קורה בתוך עשרים ומקצת קורה למעלה מעשרים מקצת סכך בתוך עשרים ומקצת סכך למעלה מעשרים אמר רבה במבוי כשר בסוכה פסול}
\textblock{מאי שנא במבוי דכשר דאמרי' קלוש סוכה נמי לימא קלוש}
\textblock{אי קלשת הויא לה חמתה מרובה מצילתה}
\textblock{הכא נמי אי קלשת הויא לה קורה הניטלת ברוח אלא על כרחך נעשו כשפודין של מתכת הכא נמי על כרחך נעשית צילתה מרובה מחמתה}
\textblock{אמר רבא מפרזקי' סוכה דליחיד היא לא מדכר מבוי דלרבים מדכרי אהדדי}
\textblock{רבינא אמר סוכה דאורייתא אחמירו בה רבנן מבוי דרבנן לא אחמירו ביה רבנן}
\textblock{רב אדא בר מתנה מתני להא שמעתא דרבה איפכא אמר רבה במבוי פסול בסוכה כשירה}
\textblock{מאי שנא סוכה דכשירה דאמרינן קלוש במבוי נמי לימא קלוש}
\textblock{אי קלשת הוי לה קורה הניטלת ברוח הכא נמי אי קלשת הויא לה חמתה מרובה מצילתה אלא על כרחך נעשית צילתה מרובה מחמתה הכא נמי ע"כ נעשו כשפודין של מתכת}
\textblock{אמר רבא מפרזקיא סוכה דליחיד היא רמי אנפשיה ומדכר מבוי דלרבי' היא סמכי אהדדי ולא מדכרי דאמרי אינשי קדרא דבי שותפי לא חמימא ולא קרירא}
\textblock{רבינא אמר סוכה דאורייתא לא בעי חיזוק מבוי דרבנן בעי חיזוק}
\textblock{מאי הוי עלה רבה בר רב עולא אמר זה וזה פסול רבא אמר זה וזה כשר}
\textblock{חלל סוכה תנן חלל מבוי תנן}
\textblock{א"ל רב פפא לרבא תניא דמסייע לך מבוי שהוא גבוה מעשרים אמה יותר מפתחו של היכל ימעט והיכל גופו חללו עשרים}
\textblock{איתיביה רב שימי בר רב אשי לרב פפא כיצד היה עושה מניח קורה משפת עשרים ולמטה}
\textblock{אימא ולמעלה והא למטה קתני}
\textblock{הא קמ"ל דלמטה כלמעלה מה למעלה חללה עשרים אף למטה חללה עשרה:}
\textblock{אמר אביי משמיה דרב נחמן אמת סוכה ואמת מבוי באמה בת חמשה אמת כלאים באמה בת ששה}
\textblock{אמת מבוי באמה בת חמשה למאי הלכתא לגובהו ולפירצת מבוי}
\textblock{והא איכא משך מבוי בארבע אמות דלקולא}
\textblock{כמאן דאמר בארבעה טפחים}
\textblock{ואיבעית אימא בארבע אמות ורוב אמות קאמר}
\textblock{אמת סוכה באמה בת חמשה למאי הלכתא לגובהה ולדופן עקומה}
\textblock{והא איכא משך סוכה בארבע אמות דלקולא דתניא רבי אומר אומר אני כל סוכה שאין בה ד' אמות על ד' אמות פסולה}
\textblock{כרבנן דאמרי אפי' אינה מחזקת אלא ראשו ורובו ושולחנו}
\textblock{ואיבעית אימא לעולם רבי היא ורוב אמות קאמר}
\textblock{אמת כלאים באמה בת ששה למאי הילכתא לקרחת הכרם ולמחול הכרם}
\textblock{דתנן קרחת הכרם ב"ש אומרים עשרים וארבע אמות וב"ה אומרים שש עשרה אמות ומחול הכרם בית שמאי אומרים שש עשרה אמות ובית הלל אומרים שתים עשרה אמות}
\textblock{איזו היא קרחת הכרם כרם שחרב אמצעיתו אין שם שש עשרה אמות לא יביא זרע לשם היו שם שש עשרה אמה נותן לו כדי עבודתו וזורע את המותר}
\textblock{ואיזהו מחול הכרם בין כרם לגדר אין שם שתים עשרה אמה לא יביא זרע לשם היו שם שתים עשרה אמה נותן לו כדי עבודתו וזורע את השאר}
\textblock{והא איכא רצופים בארבע אמות דלקולא דתנן כרם הנטוע על פחות מארבע אמות רבי שמעון אומר אינו כרם וחכמים אומרים כרם ורואין את האמצעיים כאילו אינם}
\textblock{כרבנן דאמרי הוי כרם ואיבעי' אימא לעולם ר' שמעון ורוב אמות קאמר}
\textblock{ורבא משמיה דר"נ אמר כל אמות באמה בת ששה אלא הללו שוחקות והללו עצבות}
\textblock{מיתיבי כל אמות שאמרו חכמים באמה בת ששה ובלבד}
\newsection{דף ד}
\textblock{שלא יהו מכוונות בשלמא לרבא כי היכי דליהויין הללו שוחקות והללו עצבות אלא לאביי קשיא}
\textblock{אמר לך אביי אימא אמת כלאים באמה בת ששה}
\textblock{והא מדקתני סיפא רשב"ג אומר כל אמות שאמרו חכמים בכלאים באמה בת ששה ובלבד שלא יהו מצומצמות מכלל דתנא קמא כל אמות קאמר}
\textblock{אמר לך אביי ולאו מי איכא רשב"ג דקאי כוותי אנא דאמרי כרשב"ג}
\textblock{לאביי ודאי תנאי היא לרבא מי לימא תנאי היא}
\textblock{אמר לך רבא רשב"ג הא אתא לאשמועינן אמת כלאים לא יצמצם}
\textblock{ולימא אמת כלאים לא יצמצם באמה בת ששה למעוטי מאי לאו למעוטי אמת סוכה ואמת מבוי}
\textblock{לא למעוטי אמה יסוד ואמה סובב}
\textblock{דכתיב (יחזקאל מג, יג) ואלה מדות המזבח באמות אמה אמה וטפח וחיק האמה ואמה רחב וגבולה אל שפתה סביב זרת האחד וזה גב המזבח חיק האמה זה יסוד ואמה רחב זה סובב וגבולה אל שפתה סביב אלו הקרנות וזה גב המזבח זה מזבח הזהב:}
\textblock{אמר ר' חייא בר אשי אמר רב שיעורין חציצין ומחיצין הלכה למשה מסיני}
\textblock{שיעורין דאורייתא הוא דכתיב (דברים ח, ח) ארץ חטה ושעורה וגו' ואמר רב חנן כל הפסוק הזה לשיעורין נאמר}
\textblock{חטה לכדתנן הנכנס לבית המנוגע וכליו על כתיפיו וסנדליו וטבעותיו בידיו הוא והם טמאין מיד היה לבוש כליו וסנדליו ברגליו וטבעותיו באצבעותיו הוא טמא מיד והן טהורין עד שישהא בכדי אכילת פרס פת חיטין ולא פת שעורין מיסב ואוכל בליפתן}
\textblock{שעורה דתנן עצם כשעורה מטמא במגע ובמשא ואינו מטמא באהל}
\textblock{גפן כדי רביעית יין לנזיר}
\textblock{תאנה כגרוגרת להוצאת שבת}
\textblock{רמון כדתנן כל כלי בעלי בתים שיעורן כרימונים}
\textblock{(דברים ח, ח) ארץ זית שמן (ודבש) ארץ שכל שיעוריה כזיתים כל שיעוריה ס"ד והאיכא הני דאמרן אלא אימא ארץ שרוב שיעוריה כזיתים}
\textblock{דבש ככותבת הגסה ליום הכיפורים}
\textblock{ותיסברא שיעורין מיכתב כתיבי אלא הלכתא נינהו ואסמכינהו רבנן אקראי}
\textblock{חציצין דאורייתא נינהו דכתיב (ויקרא טו, טז) ורחץ את כל בשרו (במים) שלא יהא דבר חוצץ בין בשרו למים במים במי מקוה כל בשרו מים שכל גופו עולה בהן וכמה הן אמה על אמה ברום ג' אמות ושיערו חכמים מי מקוה מ' סאה}
\textblock{כי איצטריך הילכתא לשערו וכדרבה בר רב הונא דאמר רבה בר רב הונא נימא אחת קשורה חוצצת שלש אינן חוצצות שתים איני יודע}
\textblock{שערו נמי דאורייתא הוא דתניא ורחץ את כל בשרו את הטפל לבשרו וזהו שער}
\textblock{כי אתאי הילכתא לרובו ולמיעוטו ולמקפיד ולשאין מקפיד וכדר' יצחק}
\textblock{דאמר ר' יצחק דבר תורה רובו ומקפיד עליו חוצץ ושאינו מקפיד עליו אינו חוצץ וגזרו על רובו שאינו מקפיד משום רובו המקפיד ועל מיעוטו המקפיד משום רובו המקפיד}
\textblock{וליגזור נמי על מיעוטו שאינו מקפיד משום מיעוטו המקפיד אי נמי משום רובו שאינו מקפיד}
\textblock{היא גופה גזירה ואנן ניקום וניגזור גזירה לגזירה}
\textblock{מחיצות דאורייתא נינהו}
\textblock{דאמר מר ארון תשעה וכפורת טפח הרי כאן עשרה}
\textblock{לא צריכא לר' יהודה דאמר אמת בנין באמה בת ששה אמת כלים באמה בת חמשה}
\textblock{ולר"מ דאמר כל האמות היו בבינונית מאי איכא למימר}
\textblock{לר' מאיר כי אתאי הילכתא לגוד וללבוד ולדופן עקומה:}
\textblock{היה גבוה מעשרים אמה ובא למעטו כמה ממעט כמה ממעט כמה דצריך ליה}
\textblock{אלא רחבו בכמה רב יוסף אמר טפח אביי אמר ארבעה}
\textblock{לימא בהא קא מיפלגי דמאן דאמר טפח קסבר מותר להשתמש תחת הקורה}
\newsection{דף ה}
\textblock{ומ"ד ארבעה קסבר אסור להשתמש תחת הקורה}
\textblock{לא דכולי עלמא קסברי מותר להשתמש תחת הקורה ובהא קא מיפלגי מר סבר קורה משום היכר ומר סבר קורה משום מחיצה}
\textblock{ואיבעית אימא דכולי עלמא קורה משום היכר והכא בהיכר של מטה ובהיכר של מעלה קא מיפלגי דמר סבר אמרינן היכר של מטה כהיכר של מעלה ומר סבר לא אמרינן היכר של מטה כהיכר של מעלה}
\textblock{ואיבעית אימא דכולי עלמא אמרינן היכר של מטה כהיכר של מעלה והכא בגזירה שמא יפחות קמיפלגי:}
\textblock{היה פחות מעשרה טפחים וחקק בו להשלימו לעשרה כמה חוקק כמה חוקק כמה דצריך ליה אלא משכו בכמה רב יוסף אמר בד' אביי אמר בארבע אמות}
\textblock{לימא בדרבי אמי ורבי אסי קמיפלגי דאיתמר מבוי שנפרץ מצידו כלפי ראשו איתמר משמיה דר' אמי ור' אסי אם יש שם פס ד' מתיר בפירצה עד עשר}
\textblock{ואם לאו פחות משלשה מתיר שלשה אינו מתיר לרב יוסף אית ליה דרבי אמי לאביי לית ליה דר' אמי}
\textblock{אמר לך אביי התם סוף מבוי הכא תחלת מבוי אי איכא ארבע אמות אין אי לא לא}
\textblock{אמר אביי מנא אמינא לה דתניא אין מבוי ניתר בלחי וקורה עד שיהו בתים וחצרות פתוחין לתוכו}
\textblock{ואי בד' היכי משכחת ליה}
\textblock{וכי תימא דפתח לה בדופן האמצעי והאמר רב נחמן נקיטינן איזהו מבוי שניתר בלחי וקורה כל שארכו יתר על רחבו ובתים וחצרות פתוחין לתוכו}
\textblock{ורב יוסף דפתח ליה בקרן זוית}
\textblock{אמר אביי מנא אמינא לה דאמר רמי בר חמא אמר רב הונא לחי הבולט מדופנו של מבוי פחות מארבע אמות נידון משום לחי ואינו צריך לחי אחר להתירו ד' אמות נידון משום מבוי וצריך לחי אחר להתירו}
\textblock{ורב יוסף לאפוקי מתורת לחי עד דאיכא ארבע אמות למיהוי מבוי אפי' בארבעה טפחים נמי הוי מבוי:}
\textblock{גופא אמר רמי בר חמא אמר רב הונא לחי הבולט מדפנו של מבוי}
\textblock{פחות מארבע אמות נידון משום לחי ואין צריך לחי אחר להתירו ד' אמות נידון משום מבוי וצריך לחי אחר להתירו}
\textblock{אותו לחי היכן מעמידו אי דמוקי ליה בהדי' אוספי הוא דקא מוסיף עליה}
\textblock{אמר רב פפא דמוקי ליה לאידך גיסא רב הונא בריה דרב יהושע אמר אפילו תימא דמוקי לה בהדי' דמטפי ביה או דמבצר ביה}
\textblock{אמר רב הונא בריה דרב יהושע לא אמרן אלא במבוי שמונה אבל במבוי שבעה ניתר בעומד מרובה על הפרוץ}
\textblock{וקל וחומר מחצר ומה חצר שאינה ניתרת בלחי וקורה ניתרת בעומד מרובה על הפרוץ מבוי שניתר בלחי וקורה אינו דין שניתר בעומד מרובה על הפרוץ}
\textblock{מה לחצר שכן פרצתה בעשר תאמר במבוי שפרצתו בארבע}
\textblock{קסבר רב הונא בריה דרב יהושע מבוי נמי פרצתו בעשר למאן קאמרינן לרב הונא והא רב הונא פרצתו בד' סבירא לי'}
\textblock{רב הונא בריה דרב יהושע טעמא דנפשיה קאמר}
\textblock{רב אשי אמר אפילו תימא במבוי שמונה נמי לא צריך לחי מה נפשך אי עומד נפיש ניתר בעומד מרובה על הפרוץ ואי פרוץ נפיש נידון משום לחי}
\textblock{מאי אמרת דשוו תרוייהו כי הדדי הוה ליה ספק דבריהן וספק דבריהן להקל:}
\textblock{אמר רב חנין בר רבא אמר רב מבוי שנפרץ}
\newsection{דף ו}
\textblock{מצידו בעשר מראשו בד'}
\textblock{מאי שנא מצידו בעשר דאמר פתחא הוא מראשו נמי נימא פתחא הוא}
\textblock{א"ר הונא בריה דרב יהושע כגון שנפרץ בקרן זוית דפתחא בקרן זוית לא עבדי אינשי}
\textblock{ורב הונא אמר אחד זה וא' זה בארבעה וכן א"ל רב הונא לרב חנן בר רבא לא תפלוג עלאי דרב איקלע לדמחריא ועבד עובדא כוותי א"ל רב בקעה מצא וגדר בה גדר}
\textblock{אמר רב נחמן בר יצחק כוותיה דרב הונא מסתברא דאיתמר מבוי עקום רב אמר תורתו כמפולש ושמואל אמר תורתו כסתום}
\textblock{במאי עסקינן אילימא ביותר מעשר בהא לימא שמואל תורתו כסתום}
\textblock{אלא לאו בעשר וקאמר רב תורתו כמפולש אלמא פירצת מבוי מצידו בד'}
\textblock{ורב חנן בר רבא שאני התם דקא בקעי בה רבים}
\textblock{מכלל דרב הונא סבר אע"ג דלא בקעי בה רבים מ"ש מדר' אמי ורב אסי}
\textblock{התם דאיכא גידודי הכא דליכא גידודי:}
\textblock{ת"ר כיצד מערבין דרך רה"ר עושה צורת הפתח מכאן ולחי וקורה מכאן חנניה אומר ב"ש אומרים עושה דלת מכאן ודלת מכאן וכשהוא יוצא ונכנס נועל ב"ה אומרים עושה דלת מכאן ולחי וקורה מכאן}
\textblock{ורה"ר מי מיערבא והתניא יתר על כן א"ר יהודה}
\textblock{מי שהיו לו שני בתים משני צידי רה"ר עושה לחי מכאן ולחי מכאן או קורה מכאן וקורה מכאן ונושא ונותן באמצע אמרו לו אין מערבין רשות הרבים בכך}
\textblock{וכי תימא בכך הוא דלא מיערבא הא בדלתות מיערבא והאמר רבה בר בר חנה אמר רבי יוחנן ירושלים אילמלא דלתותיה ננעלות בלילה חייבין עליה משום רשות הרבים}
\textblock{ואמר עולא הני אבולי דמחוזא אילמלא דלתותיהן ננעלות חייבין עליהן משום רה"ר}
\textblock{אמר רב יהודה הכי קאמר כיצד מערבין מבואות המפולשין לרשות הרבים עושה צורת הפתח מכאן ולחי וקורה מכאן}
\textblock{איתמר רב אמר הילכתא כתנא קמא ושמואל אמר הלכה כחנניה}
\textblock{איבעיא להו לחנניה אליבא דבית הלל צריך לנעול או אין צריך לנעול ת"ש דאמר רב יהודה אמר שמואל אינו צריך לנעול וכן א"ר מתנה אמר שמואל אינו צריך לנעול איכא דאמרי אמר רב מתנה בדידי הוה עובדא ואמר לי שמואל אין צריך לנעול}
\textblock{בעו מיניה מרב ענן צריך לנעול או אין צריך לנעול אמר להו תא חזי הני אבולי דנהרדעא דטימן עד פלגייהו בעפרא ועייל ונפיק מר שמואל ולא אמר להו ולא מידי}
\textblock{אמר רב כהנא הנך מגופות הואי}
\textblock{כי אתא רב נחמן אמר פניוה לעפרייהו לימא קסבר רב נחמן צריך לנעול לא כיון דראויות לנעול אע"פ שאין ננעלות}
\textblock{ההוא מבוי עקום דהוה בנהרדעא רמי עליה חומריה דרב וחומריה דשמואל ואצרכוהו דלתות חומריה דרב דאמר תורתו כמפולש והאמר רב הלכה כת"ק}
\textblock{כשמואל דאמר הלכה כחנניה והאמר שמואל תורתו כסתום כרב דאמר תורתו כמפולש}
\textblock{ומי עבדינן כתרי חומרי והא תניא לעולם הלכה כבית הלל והרוצה לעשות כדברי בית שמאי עושה כדברי בית הלל עושה מקולי ב"ש ומקולי ב"ה רשע מחומרי ב"ש ומחומרי ב"ה עליו הכתוב אומר (קהלת ב, יד) הכסיל בחשך הולך אלא אי כב"ש כקוליהון וכחומריהון אי כב"ה כקוליהון וכחומריהון}
\textblock{הא גופא קשיא אמרת לעולם הלכה כב"ה והדר אמרת הרוצה לעשות כדברי ב"ש עושה}
\textblock{ל"ק כאן קודם בת קול כאן לאחר ב"ק}
\textblock{ואיבעית אימא הא והא לאחר בת קול}
\newsection{דף ז}
\textblock{ורבי יהושע היא דלא משגח בבת קול}
\textblock{ואיבעית אימא הכי קאמר כל היכא דמשכחת תרי תנאי ותרי אמוראי דפליגי אהדדי כעין מחלוקת ב"ש וב"ה לא ליעבד כי קוליה דמר וכי קוליה דמר ולא כחומריה דמר וכי חומריה דמר אלא או כי קוליה דמר וכחומריה עביד או כקוליה דמר וכחומריה עביד}
\textblock{מ"מ קשיא}
\textblock{אמר ר"נ בר יצחק כוליה כרב עבדוה דאמר רב הונא אמר רב הלכה ואין מורין כן}
\textblock{ולרב אדא בר אהבה אמר רב דאמר הלכה ומורין כן מאי איכא למימר}
\textblock{אמר רב שיזבי כי לא עבדינן כחומרי דבי תרי היכא דסתרי אהדדי}
\textblock{כגון שדרה וגולגולת דתנן השדרה והגולגולת שחסרו וכמה חסרון בשדרה בש"א שתי חוליות וב"ה אומרים חוליא אחת ובגולגולת בש"א כמלא מקדח וב"ה אומרים כדי שינטל מן החי וימות}
\textblock{ואמר רב יהודה אמר שמואל וכן לענין טריפה}
\textblock{אבל היכא דלא סתרי אהדדי עבדינן}
\textblock{והיכא דסתרי אהדדי לא עבדינן מתיב רב משרשיא מעשה ברבי עקיבא שליקט אתרוג באחד בשבט ונהג בו ב' עישורין אחד כדברי ב"ש ואחד כדברי ב"ה}
\textblock{ר' עקיבא גמריה איסתפיק ליה ולא ידע אי בית הלל בחד בשבט אמור אי בחמיסר בשבט אמור ועבד הכא לחומרא והכא לחומרא:}
\textblock{יתיב רב יוסף קמיה דרב הונא ויתיב וקאמר אמר רב יהודה אמר רב מחלוקת בסרטיא מכאן וסרטיא מכאן ופלטיא מכאן ופלטיא מכאן}
\textblock{אבל סרטיא מכאן ובקעה מכאן או בקעה מכאן ובקעה מכאן עושה צורת הפתח מכאן ולחי וקורה מכאן}
\textblock{השתא סרטיא מכאן ובקעה מכאן עושה לו צורת הפתח מכאן ולחי וקורה מכאן בקעה מכאן ובקעה מכאן מיבעיא}
\textblock{הכי קאמר סרטיא מכאן ובקעה מכאן נעשה כבקעה מכאן ובקעה מכאן}
\textblock{ומסיים בה משמיה דרב יהודה אם היה מבוי כלה לרחבה א"צ כלום}
\textblock{אמר ליה אביי לרב יוסף הא דרב יהודה דשמואל היא}
\textblock{דאי דרב קשיא דרב אדרב בתרתי דאמר רב ירמיה בר אבא אמר רב מבוי שנפרץ במלואו לחצר ונפרצה חצר כנגדו חצר מותרת ומבוי אסור ואמאי ליהוי כמבוי שכלה לרחבה}
\textblock{אמר ליה אנא לא ידענא עובדא הוה בדורא דרעותא מבוי שכלה לרחבה הוה ואתא לקמיה דרב יהודה ולא אצרכיה ולא מידי ואי קשיא משמיה דרב תיהוי משמיה דשמואל ולא קשיא מידי}
\textblock{השתא דאמר ליה רב ששת לרב שמואל בר אבא ואמרי ליה לרב יוסף בר אבא אסברא לך כאן שעירבו כאן שלא עירבו}
\textblock{דרב אדרב נמי לא קשיא כאן שעירבו בני חצר עם בני מבוי כאן שלא עירבו}
\newsection{דף ח}
\textblock{ולמאי דסליק אדעתין מעיקרא בין שעירבו ובין שלא עירבו פליגי בעירבו במאי פליגי בשלא עירבו במאי פליגי}
\textblock{בשלא עירבו פליגי בנראה מבחוץ ושוה מבפנים}
\textblock{בעירבו קמיפלגי בדרב יוסף דאמר רב יוסף לא שנו אלא שכלה לאמצע רחבה אבל כלה לצידי רחבה אסור}
\textblock{אמר רבה הא דאמרת לאמצע רחבה מותר לא אמרן אלא זה שלא כנגד זה אבל זה כנגד זה אסור}
\textblock{אמר רב משרשיא הא דאמרת זה שלא כנגד זה מותר לא אמרן אלא רחבה דרבים אבל רחבה דיחיד זימנין דמימלך עלה ובני לה בתים וה"ל כמבוי שכלה לה לצידי רחבה ואסור}
\textblock{ומנא תימרא דשני לן בין רחבה דרבים לרחבה דיחיד דאמר רבין בר רב אדא אמר רבי יצחק מעשה במבוי אחד שצידו אחד כלה לים וצידו אחד כלה לאשפה ובא מעשה לפני רבי ולא אמר בה לא היתר ולא איסור}
\textblock{איסור לא אמר בה דהא קיימי מחיצות היתר לא אמר בה חיישינן שמא תינטל אשפה ויעלה הים שרטון}
\textblock{ומי חיישינן שמא תינטל אשפה והתנן אשפה ברשות הרבים גבוה י' טפחים חלון שעל גבה זורקין לה בשבת}
\textblock{אלמא שני בין אשפה דרבים לאשפה דיחיד}
\textblock{הכא נמי שני בין רחבה דרבים לרחבה דיחיד}
\textblock{ורבנן מאי}
\textblock{אמר רב יוסף בר אבדימי תנא וחכמים אוסרין אמר רב נחמן הלכה כדברי חכמים איכא דאמרי אמר רב יוסף בר אבדימי תנא וחכמים מתירין אמר רב נחמן אין הלכה כדברי חכמים}
\textblock{מרימר פסיק לה לסורא באוזלי אמר חיישינן שמא יעלה הים שרטון}
\textblock{ההוא מבוי עקום דהוה בסורא כרוך בודיא אותיבו ביה בעקמומיתיה אמר רב חסדא הא לא כרב ולא כשמואל לרב דאמר תורתו כמפולש צורת הפתח בעי לשמואל דאמר תורתו כסתום הני מילי לחי מעליא אבל האי כיון דנשיב ביה זיקא ושדי ליה לא כלום הוא}
\textblock{ואי נעיץ ביה סיכתא וחבריה חבריה:}
\textblock{גופא אמר רב ירמיה בר אבא אמר רב מבוי שנפרץ במלואו לחצר ונפרצה חצר כנגדו חצר מותרת ומבוי אסור}
\textblock{אמר ליה רבה בר עולא לרב ביבי בר אביי רבי לא משנתנו היא זו חצר קטנה שנפרצה לגדולה גדולה מותרת וקטנה אסורה מפני שהיא כפתחה של גדולה}
\textblock{אמר ליה אי מהתם הוה אמינא הני מילי היכא דלא קא דרסי בה רבים אבל היכא דקא דרסי בה רבים אימא אפילו חצר נמי}
\textblock{והא נמי תנינא חצר שהרבים נכנסין לה בזו ויוצאין לה בזו רה"ר לטומאה ורה"י לשבת}
\textblock{אי מהתם הוה אמינא הני מילי זה שלא כנגד זה}
\textblock{אבל זה כנגד זה אימא לא}
\textblock{ולרבה דאמר זה כנגד זה אסור הא דרב במאי מוקי לה בזה שלא כנגד זה תרתי למה לי}
\textblock{אי מהתם הוה אמינא הני מילי לזרוק אבל לטלטל אימא לא קמ"ל}
\textblock{איתמר מבוי העשוי כנדל אמר אביי עושה צורת הפתח לגדול והנך כולהו מישתרו בלחי וקורה}
\textblock{א"ל רבא כמאן כשמואל דאמר תורתו כסתום למה ליה צורת הפתח ועוד הא ההוא מבוי עקום דהוה בנהרדעא וחשו לה לדרב}
\textblock{אלא אמר רבא עושה צורת הפתח לכולהו להאי גיסא ואידך גיסא מישתרו בלחי וקורה}
\textblock{אמר רב כהנא בר תחליפא משמיה דרב כהנא בר מניומי משמיה דרב כהנא בר מלכיו משמיה דרב כהנא רביה דרב ואמרי לה רב כהנא בר מלכיו היינו רב כהנא רביה דרב מבוי שצידו אחד ארוך וצידו אחד קצר פחות מארבע אמות מניח את הקורה באלכסון ארבע אמות אינו מניח את הקורה אלא כנגד הקצר רבא אמר אחד זה ואחד זה אינו מניח את הקורה אלא כנגד הקצר}
\textblock{ואימא טעמא דידי ואימא טעמא דידהו אימא טעמא דידי קורה טעמא מאי משום היכר ובאלכסון לא הוי היכר}
\textblock{ואימא טעמא דידהו קורה משום מאי משום מחיצה ובאלכסון נמי הוי מחיצה}
\textblock{אמר רב כהנא הואיל ושמעתתא דכהני היא אימא בה מילתא הא דאמרת מניח הקורה באלכסון לא אמרן אלא שאין באלכסונו יותר מעשר אבל יש באלכסונו יותר מעשר דברי הכל אינו מניח אלא כנגד הקצר}
\textblock{איבעיא להו מהו להשתמש תחת הקורה רב ורבי חייא ור' יוחנן אמרו מותר להשתמש תחת הקורה שמואל ורבי שמעון בר רבי ור"ש בן לקיש אמרו אסור להשתמש תחת הקורה}
\textblock{לימא בהא קמיפלגי דמר סבר קורה משום היכר ומר סבר קורה משום מחיצה}
\textblock{לא דכולי עלמא קורה משום היכר והכא בהא קמיפלגי דמר סבר היכירא מלגיו ומר סבר היכירא מלבר}
\textblock{ואיבעית אימא דכולי עלמא משום מחיצה והכא בהא קמיפלגי דמר סבר חודו הפנימי יורד וסותם ומר סבר חודו החיצון יורד וסותם}
\textblock{אמר רב חסדא הכל מודים בבין לחיים שאסור}
\textblock{בעא מיניה רמי בר חמא מרב חסדא נעץ שתי יתידות בשני כותלי מבוי מבחוץ והניח קורה על גביהן מהו}
\textblock{א"ל לדברי המתיר אסור לדברי האוסר מותר}
\textblock{רבא אמר לדברי האוסר נמי אסור בעינן קורה ע"ג מבוי וליכא}
\textblock{איתיביה רב אדא בר מתנה לרבא היתה קורתו}
\newsection{דף ט}
\textblock{משוכה או תלויה פחות מג' אין צריך להביא קורה אחרת ג' צריך להביא קורה אחרת רשב"ג אומר פחות מד' אין צריך להביא קורה אחרת ד' צריך להביא קורה אחרת}
\textblock{מאי לאו משוכה מבחוץ ותלויה מבפנים}
\textblock{לא אידי ואידי מבפנים משוכה מרוח אחת ותלויה משתי רוחות}
\textblock{מהו דתימא מרוח אחת אמרינן לבוד משתי רוחות לא אמרינן לבוד קמ"ל}
\textblock{רב אשי אמר משוכה והיא תלויה והיכי דמי כגון שנעץ שתי יתידות עקומות על שני כותלי מבוי שאין בגובהן ג' ואין בעקמומיתן ג' מהו דתימא או לבוד אמרינן או חבוט אמרינן לבוד וחבוט לא אמרינן קמ"ל}
\textblock{תני ר' זכאי קמיה דר' יוחנן בין לחיים ותחת הקורה נידון ככרמלית אמר ליה פוק תני לברא}
\textblock{אמר אביי מסתברא מילתיה דר' יוחנן תחת הקורה אבל בין לחיין אסור ורבא אמר בין לחיים נמי מותר}
\textblock{אמר רבא מנא אמינא לה דכי אתא רב דימי א"ר יוחנן מקום שאין בו ד' על ד' מותר לבני רה"ר ולבני רה"י לכתף עליו ובלבד שלא יחליפו}
\textblock{ואביי התם בגבוה ג'}
\textblock{אמר אביי מנא אמינא לה דאמר רב חמא בר גוריא אמר רב תוך הפתח צריך לחי אחר להתירו}
\textblock{וכי תימא דאית ביה ד' על ד' והאמר רב חנין בר רבא אמר רב תוך הפתח אע"פ שאין בו ד' על ד' צריך לחי אחר להתירו}
\textblock{ורבא התם דפתוח לכרמלית}
\textblock{אבל לרה"ר מאי שרי יציבא בארעא וגיורא בשמי שמיא}
\textblock{אין מצא מין את מינו וניעור}
\textblock{א"ל רב הונא בריה דרב יהושע לרבא ואת לא תסברא דבין לחיין אסור והאמר רבה בר בר חנה אמר ר' יוחנן מבוי שרצפו בלחיין פחות פחות מד'  באנו למחלוקת רשב"ג ורבנן}
\textblock{לרשב"ג דאמר אמרינן לבוד משתמש עד חודו הפנימי של לחי הפנימי לרבנן דאמרי לא אמרינן לבוד משתמש עד חודו הפנימי של חיצון אבל בין לחיין דכולי עלמא אסור}
\textblock{ורבא התם נמי דפתוח לכרמלית}
\textblock{אבל לרה"ר מאי שרי יציבא בארעא וגיורא בשמי שמיא אין מצא מין את מינו וניעור}
\textblock{רב אשי אמר כגון שרצפו בלחיים פחות פחות מארבעה במשך ארבע אמות}
\textblock{לרבן שמעון בן גמליאל דאמר אמרינן לבוד הוה ליה מבוי וצריך לחי אחר להתירו ולרבנן דאמרי לא אמרינן לבוד לא צריך לחי אחר להתירו}
\textblock{ולרבן שמעון בן גמליאל להוי כנראה מבחוץ ושוה מבפנים}
\textblock{מידי הוא טעמא אלא לר' יוחנן הא כי אתא רבין אמר רבי יוחנן נראה מבחוץ ושוה מבפנים אינו נידון משום לחי}
\textblock{איתמר נראה מבפנים ושוה מבחוץ נידון משום לחי נראה מבחוץ ושוה מבפנים רבי חייא ורבי שמעון ב"ר חד אמר נידון משום לחי וחד אמר אינו נידון משום לחי}
\textblock{תסתיים דר' חייא הוא דאמר נידון משום לחי דתני רבי חייא כותל שצידו אחד כנוס מחבירו בין שנראה מבחוץ ושוה מבפנים ובין שנראה מבפנים ושוה מבחוץ נידון משום לחי תסתיים}
\textblock{ורבי יוחנן מי לא שמיע ליה הא אלא שמיע ליה ולא סבר לה רבי חייא נמי לא סבר לה}
\textblock{האי מאי בשלמא ר' יוחנן לא סבר לה משום הכי לא תני לה אלא ר' חייא אי איתא דלא סבר לה למה ליה למיתנא}
\textblock{אמר רבה בר רב הונא נראה מבחוץ ושוה מבפנים נידון משום לחי אמר רבה ומותבינן אשמעתין חצר קטנה שנפרצה לגדולה גדולה מותרת וקטנה אסורה מפני שהיא כפתחה של גדולה ואם איתא קטנה נמי תשתרי בנראה מבחוץ ושוה מבפנים}
\textblock{אמר ר' זירא בנכנסין כותלי קטנה לגדולה}
\textblock{ולימא לבוד ותשתרי}
\textblock{וכי תימא דמפלגי טובא והא תני רב אדא בר אבימי קמיה דר' חנינא קטנה בעשר גדולה באחת עשרה}
\textblock{אמר רבינא במופלגין מכותל זה בשנים ומכותל זה בד'}
\textblock{ולימא לבוד מרוח אחת ותשתרי}
\newsection{דף י}
\textblock{רבי היא דאמר בעינן שני פסין דתניא חצר ניתרת בפס אחד רבי אומר בשני פסין}
\textblock{האי מאי אי אמרת בשלמא נראה מבחוץ ושוה מבפנים אינו נידון משום לחי ורבי סבר לה כר' יוסי ודרבי זירא ודרבינא ליתא משום הכי קטנה בעשר וגדולה באחת עשרה משום דר' סבר לה כר' יוסי}
\textblock{אלא אי אמרת נראה מבחוץ ושוה מבפנים נידון משום לחי ודרבי זירא ודרבינא איתא ור' לא סבר לה כר' יוסי גדולה באחת עשרה למה לי}
\textblock{ממה נפשך אי למשרייה לגדולה קאתי בעשר ושני טפחים סגיא ואי למיסרה לקטנה קאתי לאשמועינן דמפלגי טובא}
\textblock{אלא לאו ש"מ נראה מבחוץ ושוה מבפנים אינו נידון משום לחי ש"מ}
\textblock{א"ר יוסף לא שמיע לי הא שמעתתא}
\textblock{א"ל אביי את אמרת ניהלן ואהא אמרת ניהלן דאמר רמי בר אבא אמר רב הונא לחי המושך עם דפנו של מבוי פחות מד' אמות נידון משום לחי ומשתמש עם חודו הפנימי ד' אמות נידון משום מבוי ואסור להשתמש בכולו}
\textblock{ואת אמרת לן עלה שמע מינה תלת שמע מינה בין לחיין אסור ושמע מינה משך מבוי בארבע ושמע מינה נראה מבחוץ ושוה מבפנים נידון משום לחי}
\textblock{והלכתא נראה מבחוץ ושוה מבפנים נידון משום לחי תיובתא והלכתא}
\textblock{אין משום דתני רבי חייא כוותיה:}
\textblock{והרחב מעשר ימעט: אמר אביי תנא והרחב מעשר ימעט ר' יהודה אומר אינו צריך למעט ועד כמה}
\textblock{סבר רב אחי קמיה דרב יוסף למימר עד שלש עשרה אמה ושליש וקל וחומר מפסי ביראות}
\textblock{ומה פסי ביראות שהתרתה בהן פרוץ מרובה על העומד לא התרתה בהן יותר משלש עשרה אמה ושליש מבוי שלא התרתה בו פרוץ מרובה על העומד אינו דין שלא תתיר בו יותר משלש עשרה אמה ושליש}
\textblock{והיא הנותנת פסי ביראות שהתרתה בהן פרוץ מרובה על העומד לא תתיר בהן יותר משלש עשרה אמה ושליש מבוי שלא התרתה בו פרוץ מרובה על העומד תתיר בו יותר מי"ג אמות ושליש}
\textblock{אי נמי לאידך גיסא פסי ביראות דאקילת בהו חד קולא אקיל בהו קולא אחרינא מבוי כלל כלל לא}
\textblock{תני לוי מבוי שהוא רחב עשרים אמה נועץ קנה באמצעיתו ודיו הוא תני לה והוא אמר לה דאין הלכה כאותה משנה איכא דאמרי אמר שמואל משמיה דלוי אין הלכה כאותה משנה}
\textblock{אלא היכי עביד אמר שמואל משמיה דלוי}
\textblock{עושה פס גבוה עשרה במשך ארבע אמות ומעמידו לארכו של מבוי}
\textblock{אי נמי כדרב יהודה דאמר רב יהודה מבוי שהוא רחב חמש עשרה אמה מרחיק שתי אמות ועושה פס שלש אמות}
\textblock{ואמאי יעשה פס אמה ומחצה וירחיק שתי אמות ויעשה פס אמה ומחצה שמע מינה עומד מרובה על הפרוץ משתי רוחות לא הוי עומד}
\textblock{לעולם אימא לך הוי עומד ושאני הכא דאתי אוירא דהאי גיסא ואוירא דהאי גיסא ומבטל ליה}
\textblock{ויעשה פס אמה וירחיק אמה ויעשה פס אמה וירחיק אמה ויעשה פס אמה ש"מ עומד כפרוץ אסור}
\textblock{לעולם אימא לך מותר ושאני הכא דאתא אוירא דהאי גיסא ודהאי גיסא ומבטל ליה}
\textblock{וירחיק אמה ויעשה פס אמה ומחצה וירחיק אמה ויעשה פס אמה ומחצה}
\textblock{אין הכי נמי וכולי האי לא אטרחוה רבנן}
\textblock{וליחוש דלמא שביק פיתחא רבה ועייל בפיתחא זוטא אמר רב אדא בר מתנה חזקה אין אדם מניח פתח גדול ונכנס בפתח קטן}
\textblock{ומאי שנא מדרבי אמי ודרבי אסי}
\textblock{התם קא ממעט בהילוכא הכא לא קא ממעט בהילוכא}
\textblock{תנן התם עור העסלא וחלל שלו מצטרפין בטפח}
\textblock{מאי עור העסלא אמר רבה בר בר חנה א"ר יוחנן עור כיסוי של בית הכסא}
\textblock{וכמה כי אתא רב דימי אמר אצבעיים מכאן ואצבעיים מכאן ואצבעיים ריוח באמצע כי אתא רבין אמר אצבע ומחצה מכאן ואצבע ומחצה מכאן ואצבע ריוח באמצע}
\textblock{א"ל אביי לרב דימי מי פליגיתו א"ל לא הא ברברבתא הא בזוטרתא ולא פליגין}
\textblock{א"ל לאיי פליגיתו ובעומד מרובה על הפרוץ משתי רוחות פליגיתו לדידך הוי עומד משתי רוחות לרבין מרוח אחת הוי עומד משתי רוחות לא הוי עומד}
\textblock{דאי סלקא דעתך לא פליגיתו לרבין הכי איבעי ליה למימר אצבע ושליש מכאן ואצבע ושליש מכאן ואצבע ושליש ריוח באמצע}
\textblock{ואלא מאי פליגינן לדידי הכי איבעי לי למימר אצבע ושני שלישים מכאן ואצבע ושני שלישים מכאן ואצבעיים ושני שלישים ריוח באמצע}
\textblock{אלא אי איכא למימר דפלגינן בפרוץ כעומד פלגינן:}
\textblock{אם יש לו צורת הפתח אף על פי שרחב מעשר אינו צריך למעט: אשכחן צורת הפתח דמהניא ברחבו ואמלתרא דמהניא בגבהו}
\newsection{דף יא}
\textblock{איפכא מאי}
\textblock{ת"ש דתניא מבוי שהוא גבוה מעשרים אמה ימעט ואם יש לו צורת הפתח אינו צריך למעט}
\textblock{אמלתרא ברחבו מאי תא שמע דתניא מבוי שהוא גבוה מעשרים אמה ימעט והרחב מעשר ימעט ואם יש לו צורת הפתח אינו צריך למעט ואם יש לו אמלתרא אינו צריך למעט}
\textblock{מאי לאו אסיפא לא ארישא}
\textblock{מתני ליה רב יהודה לחייא בר רב קמיה דרב אינו צריך למעט א"ל אתנייה צריך למעט}
\textblock{אמר רב יוסף מדברי רבינו נלמד חצר שרובה פתחים וחלונות אינה ניתרת בצורת הפתח}
\textblock{מ"ט הואיל ויותר מעשר אוסר במבוי ופרוץ מרובה על העומד אוסר בחצר מה יותר מעשר האוסר במבוי אינו ניתר בצורת הפתח אף פרוץ מרובה על העומד האוסר בחצר אינו ניתר בצורת הפתח}
\textblock{מה ליותר מעשר האוסר במבוי שכן לא התרת בו אצל פסי ביראות לר"מ תאמר בפרוץ מרובה על העומד האוסר בחצר שכן התרת אצל פסי ביראות לדברי הכל}
\textblock{לימא מסייע ליה דפנות הללו שרובן פתחים וחלונות מותר ובלבד שיהא עומד מרובה על הפרוץ}
\textblock{שרובן ס"ד אלא אימא שריבה בהן פתחים וחלונות ובלבד שיהא עומד מרובה על הפרוץ}
\textblock{אמר רב כהנא כי תניא ההיא בפיתחי שימאי}
\textblock{מאי פיתחי שימאי פליגי בה רב רחומי ורב יוסף חד אמר דלית להו שקפי וחד אמר דלית להו תיקרה}
\textblock{ואף ר' יוחנן סבר לה להא דרב דאמר רבין בר רב אדא א"ר יצחק מעשה באדם אחד מבקעת בית חורתן שנעץ ד' קונדיסין בארבע פינות השדה ומתח זמורה עליהם ובא מעשה לפני חכמים והתירו לו לענין כלאים}
\textblock{ואמר ר"ל כדרך שהתירו לו לענין כלאים כך התירו לו לענין שבת ר' יוחנן אמר לכלאים התירו לו לענין שבת לא התירו לו}
\textblock{במאי עסקינן אילימא מן הצד והאמר רב חסדא צורת הפתח שעשאה מן הצד לא עשה ולא כלום}
\textblock{אלא על גבן ובמאי אילימא בעשר בהא לימא ר' יוחנן בשבת לא}
\textblock{אלא לאו ביתר מעשר}
\textblock{לא לעולם בעשר ומן הצד ובדרב חסדא קא מיפלגי}
\textblock{ורמי דר' יוחנן אדר' יוחנן ורמי דריש לקיש אדריש לקיש דאמר ריש לקיש משום ר' יהודה בר' חנינא}
\textblock{פיאה מותרת לענין כלאים אבל לא לשבת ור' יוחנן אמר כמחיצות לשבת דלא כך מחיצות לכלאים דלא}
\textblock{בשלמא דריש לקיש אדר"ל לא קשיא הא דידיה הא דרביה אלא דרבי יוחנן אדר' יוחנן קשיא}
\textblock{אי אמרת בשלמא התם על גבן הכא מן הצד שפיר אלא אי אמרת אידי ואידי מן הצד מאי איכא למימר}
\textblock{לעולם אידי ואידי מן הצד התם בעשר הכא ביותר מעשר}
\textblock{ומנא תימרא דשני לן בין עשר ליותר מעשר דאמר ליה רבי יוחנן לריש לקיש לא כך היה המעשה שהלך רבי יהושע אצל ר' יוחנן בן נורי ללמוד תורה אף על פי שבקי בהלכות כלאים ומצאו שיושב בין האילנות ומתח זמורה מאילן לאילן ואמר לו רבי אי גפנים כאן מהו לזרוע כאן אמר לו בעשר מותר ביותר מעשר אסור}
\textblock{במאי עסקינן אילימא על גבן יותר מעשר אסור והתניא היו שם קנין הדוקרנין ועשה להן פיאה מלמעלה אפי' ביותר מעשר מותר}
\textblock{אלא לאו מן הצד וקאמר ליה בעשר מותר יותר מעשר אסור שמע מינה:}
\textblock{גופא אמר רב חסדא צורת הפתח שעשאה מן הצד לא עשה ולא כלום}
\textblock{ואמר רב חסדא צורת הפתח שאמרו צריכה שתהא בריאה כדי להעמיד בה דלת ואפילו דלת של קשין}
\textblock{אמר ריש לקיש משום ר' ינאי צורת הפתח צריכה היכר ציר מאי היכר ציר אמר רב אויא אבקתא}
\textblock{אשכחינהו רב אחא בריה דרב אויא לתלמידי דרב אשי אמר להו אמר מר מידי בצורת הפתח אמרו ליה לא אמר ולא כלום}
\textblock{תנא צורת הפתח שאמרו קנה מכאן וקנה מכאן וקנה על גביהן צריכין ליגע או אין צריכין ליגע רב נחמן אמר אין צריכין ליגע ורב ששת אמר צריכין ליגע}
\textblock{אזל רב נחמן ועבד עובדא בי ריש גלותא כשמעתיה א"ל רב ששת לשמעיה רב גדא זיל שלוף שדינהו אזל שלף שדינהו אשכחוהו דבי ריש גלותא חבשוהו אזל רב ששת קם אבבא א"ל גדא פוק תא נפק ואתא}
\textblock{אשכחיה רב ששת לרבה בר שמואל אמר ליה תני מר מידי בצורת הפתח א"ל אין תנינא כיפה ר"מ מחייב במזוזה וחכמים פוטרין ושוין שאם יש ברגליה עשרה שהיא חייבת}
\textblock{אמר אביי הכל מודים אם גבוהה עשרה ואין ברגליה שלשה אי נמי יש ברגליה שלשה ואין גבוהה עשרה ולא כלום}
\textblock{כי פליגי ביש ברגליה ג' וגבוהה עשרה ואין רחבה ארבעה ויש בה לחוק להשלימה לארבעה}
\textblock{ר' מאיר סבר חוקקין להשלים ורבנן סברי אין חוקקין להשלים}
\textblock{א"ל אי משכחת להו לא תימא להו לבי ריש גלותא ולא מידי מהא מתניתא דכיפה:}
\textblock{{\large\emph{מתני׳}} הכשר מבוי ב"ש אומרים לחי וקורה וב"ה אומרים או לחי או קורה רבי אליעזר אומר לחיין}
\textblock{משום ר' ישמעאל אמר תלמיד אחד לפני ר"ע לא נחלקו ב"ש וב"ה על מבוי שהוא פחות מארבע אמות שהוא ניתר או בלחי או בקורה על מה נחלקו על רחב מארבע אמות ועד עשר שב"ש אומרים לחי וקורה וב"ה אומרים או לחי או קורה א"ר עקיבא על זה ועל זה נחלקו:}
\textblock{{\large\emph{גמ׳}} כמאן דלא כחנניה ולא כתנא קמא}
\textblock{אמר רב יהודה הכי קאמר הכשר מבוי סתום כיצד ב"ש אומרים לחי וקורה וב"ה אומרים או לחי או קורה}
\textblock{ב"ש אומרים לחי וקורה למימרא דקא סברי ב"ש ארבע מחיצות דאורייתא}
\textblock{לא לזרוק משלש הוא דמיחייב לטלטל עד דאיכא ארבע}
\textblock{ב"ה אומרים או לחי או קורה לימא קא סברי ב"ה שלש מחיצות דאורייתא}
\textblock{לא לזרוק משתים הוא דמיחייב לטלטל עד דאיכא שלש:}
\textblock{ר' אליעזר אומר לחיין: איבעיא להו ר' אליעזר לחיין וקורה קאמר או דילמא לחיין בלא קורה קאמר}
\textblock{ת"ש מעשה ברבי אליעזר שהלך אצל רבי יוסי בן פרידא תלמידו}
\newsection{דף יב}
\textblock{לאובלין ומצאו שיושב במבוי שאין לו אלא לחי אחד אמר לו בני עשה לחי אחר אמר לו וכי לסותמו אני צריך א"ל יסתם ומה בכך}
\textblock{אמר רשב"ג לא נחלקו ב"ש וב"ה על מבוי שהוא פחות מארבע אמות שאינו צריך כלום על מה נחלקו על רחב מארבע אמות ועד עשר שב"ש אומרים לחי וקורה וב"ה אומרים או לחי או קורה}
\textblock{קתני מיהת וכי לסותמו אני צריך אי אמרת בשלמא לחיין וקורה משום הכי אמר וכי לסותמו אני צריך אלא אי אמרת לחיין בלא קורה מאי לסותמו}
\textblock{הכי קאמר וכי לסותמו בלחיין אני צריך}
\textblock{אמר מר אמר רבן שמעון בן גמליאל לא נחלקו ב"ש וב"ה על מבוי שפחות מארבע אמות שאינו צריך כלום והא אנן תנן משום רבי ישמעאל אמר תלמיד אחד לפני ר"ע לא נחלקו ב"ש וב"ה על מבוי שהוא פחות מארבע אמות שהוא ניתר או בלחי או בקורה}
\textblock{אמר רב אשי הכי קאמר אינו צריך לחי וקורה כב"ש ולא לחיין כר' אליעזר אלא או לחי או קורה כבית הלל}
\textblock{וכמה אמר רב אחלי ואיתימא רב יחיאל עד ארבעה}
\textblock{אמר רב ששת אמר רב ירמיה בר אבא אמר רב מודים חכמים לרבי אליעזר בפסי חצר ורב נחמן אמר הלכה כר' אליעזר בפסי חצר}
\textblock{אמר רב נחמן בר יצחק מאן מודים ר' הלכה מכלל דפליגי [מאן פליג עליה] רבנן דתניא חצר ניתרת בפס אחד רבי אומר בשני פסין}
\textblock{אמר רבי אסי אמר רבי יוחנן חצר צריכה שני פסין אמר ליה רבי זירא לרבי אסי מי אמר רבי יוחנן הכי והא את הוא דאמרת משמיה דר' יוחנן פסי חצר צריכין שיהא בהן ד' וכי תימא ארבעה מכאן וארבעה מכאן}
\textblock{והתני רב אדא בר אבימי קמיה דרבי חנינא ואמרי לה קמיה דר' חנינא בר פפי קטנה בעשר וגדולה באחת עשרה}
\textblock{כי סליק רבי זירא מימי פרשה ברוח אחת בארבעה משתי רוחות משהו לכאן ומשהו לכאן}
\textblock{והדתני אדא בר אבימי רבי היא וסבר לה כרבי יוסי}
\textblock{אמר רב יוסף אמר רב יהודה אמר שמואל חצר ניתרת בפס אחד אמר ליה אביי לרב יוסף מי אמר שמואל הכי והא אמר ליה שמואל לרב חנניה בר שילא את לא תעביד עובדא אלא או ברוב דופן או בשני פסין}
\textblock{אמר ליה ואנא לא ידענא דעובדא הוה בדורה דרעותא לשון ים הנכנס לחצר הוה ואתא לקמיה דרב יהודה ולא אצרכיה אלא פס אחד}
\textblock{אמר ליה לשון ים קאמרת קל הוא שהקלו חכמים במים}
\textblock{כדבעא מיניה רבי טבלא מרב מחיצה תלויה מהו שתתיר בחורבה אמר ליה אין מחיצה תלויה מתרת אלא במים קל הוא שהקלו חכמים במים}
\textblock{מכל מקום קשיא}
\textblock{כי אתו רב פפא ורב הונא בריה דרב יהושע מבי רב פירשוה מרוח אחת בארבעה משתי רוחות משהו לכאן ומשהו לכאן}
\textblock{אמר רב פפא אי קשיא לי הא קשיא לי דאמר ליה שמואל לרב חנניה בר שילא את לא תעביד עובדא אלא או ברוב דופן או בשני פסין}
\textblock{למה לי רוב דופן בפס ארבעה סגי וכי תימא מאי ברוב דופן בדופן שבעה דבארבעה הוה ליה רוב דופן למה לי ארבעה בשלשה ומשהו סגי דהא אמר רב אחלי ואיתימא רב יחיאל עד ארבעה}
\textblock{איבעית אימא כאן בחצר כאן במבוי ואיבעית אימא דרב אחלי גופיה תנאי היא}
\textblock{תנו רבנן לשון ים הנכנס לחצר אין ממלאין הימנו בשבת אלא אם כן יש לו מחיצה גבוה י' טפחים בד"א שפירצתו ביותר מעשרה אבל עשרה אין צריך כלום}
\textblock{ממלא הוא דלא ממלאינן הא טלטולי מטלטלינן והא נפרצה חצר במלואה למקום האסור לה}
\textblock{הכא במאי עסקינן דאית ליה גידודי:}
\textblock{אמר רב יהודה מבוי שלא נשתתפו בו הכשירו בלחי הזורק לתוכו חייב הכשירו בקורה הזורק לתוכו פטור}
\textblock{מתקיף לה רב ששת טעמא דלא נשתתפו בו הא נשתתפו בו אפילו הכשירו בקורה נמי חייב וכי ככר זו עשה אותו רשות היחיד או רשות הרבים}
\textblock{והתניא חצירות של רבים ומבואות שאינן מפולשין בין עירבו ובין לא עירבו הזורק לתוכן חייב}
\textblock{אלא אי איתמר הכי איתמר אמר רב יהודה מבוי שאינו ראוי לשיתוף הכשירו בלחי הזורק לתוכו חייב הכשירו בקורה הזורק לתוכו פטור}
\textblock{אלמא קסבר לחי משום מחיצה וקורה משום היכר וכן אמר רבה לחי משום מחיצה וקורה משום היכר ורבא אמר אחד זה ואחד זה משום היכר}
\textblock{איתיביה רבי יעקב בר אבא לרבא הזורק למבוי יש לו לחי חייב אין לו לחי פטור}
\textblock{הכי קאמר אינו צריך אלא לחי הזורק לתוכו חייב לחי ודבר אחר הזורק לתוכו פטור}
\textblock{איתיביה יתר על כן אמר רבי יהודה מי שיש לו שני בתים בשני צידי רשות הרבים עושה לחי מכאן ולחי מכאן או קורה מכאן וקורה מכאן ונושא ונותן באמצע}
\textblock{אמרו לו אין מערבין רשות הרבים בכך}
\textblock{התם קסבר רבי יהודה שתי מחיצות דאורייתא}
\textblock{אמר רב יהודה אמר רב מבוי שארכו כרחבו אינו ניתר בלחי משהו אמר רב חייא בר אשי אמר רב מבוי שארכו כרחבו אינו ניתר בקורה טפח}
\textblock{אמר רבי זירא כמה מכוונן שמעתא דסבי כיון דארכו כרחבו הוה ליה חצר וחצר אינה ניתרת בלחי וקורה אלא בפס ארבעה}
\textblock{אמר רבי זירא אי קשיא לי הא קשיא לי ליהוי האי לחי כפס משהו ונשתרי}
\textblock{אישתמיטתיה הא דאמר ר' אסי אמר ר' יוחנן פסי חצר צריכין שיהא בהן ארבעה}
\textblock{אמר רב נחמן נקטינן איזהו מבוי שניתר בלחי וקורה כל שארכו יתר על רחבו ובתים וחצרות פתוחים לתוכו ואיזו היא חצר שאינה ניתרת בלחי וקורה אלא בפס ארבעה כל שמרובעת}
\textblock{מרובעת אין עגולה לא הכי קאמר אי ארכה יתר על רחבה הוה ליה מבוי ומבוי בלחי וקורה סגיא ואי לא הוה לה חצר}
\textblock{וכמה סבר שמואל למימר עד דאיכא פי שנים ברחבה אמר ליה רב הכי אמר חביבי אפילו משהו:}
\textblock{משום רבי ישמעאל אמר תלמיד אחד כו':}
\newsection{דף יג}
\textblock{ר"ע אומר על זה ועל זה נחלקו כו':}
\textblock{ר"ע היינו תנא קמא איכא בינייהו דרב אחלי ואיתימא רב יחיאל ולא מסיימי}
\textblock{תניא אמר ר"ע לא אמר ר' ישמעאל דבר זה אלא אותו תלמיד אמר דבר זה והלכה כאותו תלמיד}
\textblock{הא גופה קשיא אמרת לא א"ר ישמעאל דבר זה אלמא לית הלכתא כוותיה והדר אמרת הלכה כאותו תלמיד}
\textblock{אמר רב יהודה אמר שמואל לא אמרה ר' עקיבא אלא לחדד בה התלמידים}
\textblock{ור"נ בר יצחק אמר נראין איתמר}
\textblock{א"ר יהושע בן לוי כל מקום שאתה מוצא משום רבי ישמעאל אמר תלמיד אחד לפני ר"ע אינו אלא ר"מ ששימש את ר' ישמעאל ואת ר"ע}
\textblock{דתניא אמר ר"מ כשהייתי אצל ר' ישמעאל הייתי מטיל קנקנתום לתוך הדיו ולא אמר לי דבר כשבאתי אצל רבי עקיבא אסרה עלי}
\textblock{איני והאמר רב יהודה אמר שמואל משום ר' מאיר כשהייתי לומד אצל ר' עקיבא הייתי מטיל קנקנתום לתוך הדיו ולא אמר לי דבר וכשבאתי אצל ר' ישמעאל אמר לי בני מה מלאכתך אמרתי לו לבלר אני אמר לי בני הוי זהיר במלאכתך שמלאכתך מלאכת שמים היא שמא אתה מחסר אות אחת או מייתר אות אחת נמצאת מחריב את כל העולם כולו}
\textblock{אמרתי לו דבר אחד יש לי וקנקנתום שמו שאני מטיל לתוך הדיו אמר לי וכי מטילין קנקנתום לתוך הדיו והלא אמרה תורה (במדבר ה, כג) וכתב (במדבר ה, כג) ומחה כתב שיכול למחות}
\textblock{מאי קא"ל ומאי קא מהדר ליה}
\textblock{הכי קא"ל לא מיבעיא בחסירות וביתירות [דלא טעינא] דבקי אנא אלא אפילו מיחש לזבוב נמי דילמא אתי ויתיב אתגיה דדל"ת ומחיק ליה ומשוי ליה רי"ש דבר אחד יש לי וקנקנתום שמו שאני מטיל לתוך הדיו}
\textblock{קשיא שימוש אשימוש קשיא אסרה אאסרה}
\textblock{בשלמא שימוש אשימוש לא קשיא מעיקרא אתא לקמיה דר"ע ומדלא מצי למיקם אליביה אתא לקמיה דרבי ישמעאל וגמר גמרא והדר אתא לקמיה דר"ע וסבר סברא}
\textblock{אלא אסרה אאסרה קשיא קשיא}
\textblock{תניא רבי יהודה אומר ר"מ היה אומר לכל מטילין קנקנתום לתוך הדיו חוץ מפרשת סוטה ורבי יעקב אומר משמו חוץ מפרשת סוטה שבמקדש}
\textblock{מאי בינייהו אמר רב ירמיה למחוק לה מן התורה איכא בינייהו}
\textblock{והני תנאי כי הני תנאי דתניא אין מגילתה כשירה להשקות בה סוטה אחרת ר' אחי בר יאשיה אמר מגילתה כשירה להשקות בה סוטה אחרת}
\textblock{אמר רב פפא דילמא לא היא עד כאן לא קאמר ת"ק התם אלא כיון דאינתיק לשום רחל תו לא הדרא מינתקא לשום לאה אבל גבי תורה דסתמא מיכתבא הכי נמי דמחקינן}
\textblock{אמר רב נחמן בר יצחק דילמא לא היא עד כאן לא קאמר רבי אחי בר יאשיה התם אלא דאיכתיב מיהת לשום סוטה בעולם אבל גבי תורה דלהתלמד כתיבא הכי נמי דלא מחקינן}
\textblock{ולית ליה לרבי אחי בר יאשיה הא דתנן כתב [גט] לגרש את אשתו}
\textblock{ונמלך ומצאו בן עירו ואמר שמך כשמי ושם אשתך כשם אשתי פסול לגרש בו}
\textblock{הכי השתא התם (דברים כד, א) וכתב לה כתיב בעינן כתיבה לשמה הכא ועשה לה כתיב בעינן עשייה לשמה עשייה דידה מחיקה היא}
\textblock{א"ר אחא בר חנינא גלוי וידוע לפני מי שאמר והיה העולם שאין בדורו של רבי מאיר כמותו ומפני מה לא קבעו הלכה כמותו שלא יכלו חביריו לעמוד על סוף דעתו שהוא אומר על טמא טהור ומראה לו פנים על טהור טמא ומראה לו פנים}
\textblock{תנא לא ר"מ שמו אלא רבי נהוראי שמו ולמה נקרא שמו ר"מ שהוא מאיר עיני חכמים בהלכה ולא נהוראי שמו אלא רבי נחמיה שמו ואמרי לה רבי אלעזר בן ערך שמו ולמה נקרא שמו נהוראי שמנהיר עיני חכמים בהלכה}
\textblock{אמר רבי האי דמחדדנא מחבראי דחזיתיה לר' מאיר מאחוריה ואילו חזיתיה מקמיה הוה מחדדנא טפי דכתיב (ישעיהו ל, כ) והיו עיניך רואות את מוריך}
\textblock{א"ר אבהו א"ר יוחנן תלמיד היה לו לר"מ וסומכוס שמו שהיה אומר על כל דבר ודבר של טומאה ארבעים ושמונה טעמי טומאה ועל כל דבר ודבר של טהרה ארבעים ושמונה טעמי טהרה}
\textblock{תנא תלמיד ותיק היה ביבנה שהיה מטהר את השרץ במאה וחמשים טעמים}
\textblock{אמר רבינא אני אדון ואטהרנו ומה נחש שממית ומרבה טומאה טהור שרץ שאין ממית ומרבה טומאה לא כ"ש}
\textblock{ולא היא מעשה קוץ בעלמא קעביד}
\textblock{א"ר אבא אמר שמואל שלש שנים נחלקו ב"ש וב"ה הללו אומרים הלכה כמותנו והללו אומרים הלכה כמותנו  יצאה בת קול ואמרה אלו ואלו דברי אלהים חיים הן והלכה כב"ה}
\textblock{וכי מאחר שאלו ואלו דברי אלהים חיים מפני מה זכו ב"ה לקבוע הלכה כמותן מפני שנוחין ועלובין היו ושונין דבריהן ודברי ב"ש ולא עוד אלא שמקדימין דברי ב"ש לדבריהן}
\textblock{כאותה ששנינו מי שהיה ראשו ורובו בסוכה ושלחנו בתוך הבית בית שמאי פוסלין וב"ה מכשירין אמרו ב"ה לב"ש לא כך היה מעשה שהלכו זקני ב"ש וזקני ב"ה לבקר את ר' יוחנן בן החורנית ומצאוהו יושב ראשו ורובו בסוכה ושלחנו בתוך הבית אמרו להן בית שמאי (אי) משם ראיה אף הן אמרו לו אם כך היית נוהג לא קיימת מצות סוכה מימיך}
\textblock{ללמדך שכל המשפיל עצמו הקב"ה מגביהו וכל המגביה עצמו הקב"ה משפילו כל המחזר על הגדולה גדולה בורחת ממנו וכל הבורח מן הגדולה גדולה מחזרת אחריו וכל הדוחק את השעה שעה דוחקתו וכל הנדחה מפני שעה שעה עומדת לו}
\textblock{ת"ר שתי שנים ומחצה נחלקו ב"ש וב"ה הללו אומרים נוח לו לאדם שלא נברא יותר משנברא והללו אומרים נוח לו לאדם שנברא יותר משלא נברא נמנו וגמרו נוח לו לאדם שלא נברא יותר משנברא עכשיו שנברא יפשפש במעשיו ואמרי לה ימשמש במעשיו}
\textblock{{\large\emph{מתני׳}} הקורה שאמרו רחבה כדי לקבל אריח ואריח חצי לבנה של שלשה טפחים דייה לקורה שתהא רחבה טפח כדי לקבל אריח לרחבו}
\textblock{רחבה כדי לקבל אריח ובריאה כדי לקבל אריח רבי יהודה אומר רחבה אף על פי שאין בריאה היתה של קש ושל קנים רואין אותה כאילו היא של מתכת}
\textblock{עקומה רואין אותה כאילו היא פשוטה עגולה רואין אותה כאילו היא מרובעת כל שיש בהיקיפו שלשה טפחים יש בו רוחב טפח:}
\newsection{דף יד}
\textblock{{\large\emph{גמ׳}} טפח טפח ומחצה בעי}
\textblock{כיון דרחב לקבל טפח אידך חצי טפח מלבין ליה בטינא משהו מהאי גיסא ומשהו מהאי גיסא וקיימא}
\textblock{אמר רבה בר רב הונא קורה שאמרו צריכה שתהא בריאה כדי לקבל אריח ומעמידי קורה אינן צריכין שיהיו בריאין כדי לקבל קורה ואריח ורב חסדא אמר אחד זה ואחד זה צריכין שיהיו בריאין כדי לקבל קורה ואריח}
\textblock{אמר רב ששת הניח קורה על גבי מבוי ופרס עליה מחצלת והגביה מן הקרקע שלשה קורה אין כאן מחיצה אין כאן קורה אין כאן דהא מיכסיא מחיצה אין כאן דהויא לה מחיצה שהגדיים בוקעין בה}
\textblock{ת"ר קורה היוצאה מכותל זה ואינה נוגעת בכותל זה וכן שתי קורות אחת יוצאה מכותל זה ואחת יוצאה מכותל זה ואינן נוגעות זו בזו פחות משלשה אין צריך להביא קורה אחרת שלשה צריך להביא קורה אחרת}
\textblock{רבן שמעון בן גמליאל אומר פחות מד' אין צריך להביא קורה אחרת ארבע צריך להביא קורה אחרת}
\textblock{וכן ב' קורות המתאימות לא בזו כדי לקבל אריח ולא בזו כדי לקבל אריח אם מקבלות אריח לרחבו טפח אין צריך להביא קורה אחרת ואם לאו צריך להביא קורה אחרת}
\textblock{רשב"ג אומר אם מקבלת אריח לארכו שלשה אין צריך להביא קורה אחרת ואם לאו צריך להביא קורה אחרת}
\textblock{היו אחת למעלה ואחת למטה ר' יוסי בר' יהודה אומר רואין את העליונה כאילו היא למטה ואת התחתונה כאילו היא למעלה ובלבד שלא תהא עליונה למעלה מכ' אמה ותחתונה למטה מעשרה}
\textblock{אמר אביי ר' יוסי בר' יהודה סבר לה כאבוה בחדא ופליג עליה בחדא סבר לה כאבוה בחדא דאית ליה רואין}
\textblock{ופליג עליה בחדא דאילו ר' יהודה סבר למעלה מעשרים ור' יוסי ברבי יהודה סבר בתוך כ' אין למעלה מכ' לא:}
\textblock{ר' יהודה אומר רחבה אע"פ שאינה בריאה: מתני ליה רב יהודה לחייא בר רב קמיה דרב רחבה אע"פ שאינה בריאה א"ל אתנייה רחבה ובריאה}
\textblock{והאמר ר' אילעאי אמר רב רחבה ארבעה אע"פ שאינה בריאה רחבה ארבעה שאני:}
\textblock{היתה של קש כו': מאי קמ"ל דאמרינן רואין היינו הך}
\textblock{מהו דתימא במינה אמרינן שלא במינה לא אמרינן קמ"ל:}
\textblock{עקומה רואין אותה כאילו היא פשוטה: פשיטא קמ"ל כדרבי זירא דאמר ר' זירא היא בתוך המבוי ועקמומיתה חוץ למבוי היא בתוך עשרים ועקמומיתה למעלה מעשרים היא למעלה מעשרה ועקמומיתה למטה מעשרה רואין כל שאילו ינטל עקמומיתה ואין בין זה לזה שלשה אין צריך להביא קורה אחרת ואם לאו צריך להביא קורה אחרת}
\textblock{הא נמי פשיטא היא בתוך מבוי ועקמומיתה חוץ למבוי איצטריכא ליה מהו דתימא ליחוש דילמא אתי לאמשוכי בתרה קמ"ל:}
\textblock{עגולה רואין אותה כאילו היא מרובעת: הא תו למה לי סיפא איצטריכא ליה כל שיש בהיקיפו ג' טפחים יש בו רחב טפח}
\textblock{מנא הני מילי א"ר יוחנן אמר קרא (מלכים א ז, כג) ויעש את הים מוצק עשר באמה משפתו עד שפתו עגול סביב וחמש באמה קומתו וקו שלשים באמה יסוב אותו סביב}
\textblock{והא איכא שפתו}
\textblock{אמר רב פפא שפתו שפת פרח שושן כתיב ביה דכתיב (מלכים א ז, כו) ועביו טפח ושפתו כמעשה כוס פרח שושן אלפים בת יכיל}
\textblock{והאיכא משהו כי קא חשיב מגואי קא חשיב}
\textblock{תניא רבי חייא ים שעשה שלמה היה מחזיק מאה וחמשים מקוה טהרה מכדי מקוה כמה הוי ארבעים סאה כדתניא (ויקרא טו, טז) ורחץ (את בשרו)}
\textblock{במים במי מקוה כל בשרו מים שכל גופו עולה בהן וכמה הן אמה על אמה ברום שלש אמות ושיערו חכמים מי מקוה ארבעים סאה}
\textblock{כמה הוו להו חמש מאה גרמידי לתלת מאה מאה למאה וחמשין חמשין בארבע מאה וחמשין סגיא}
\textblock{הני מילי בריבועא ים שעשה שלמה עגול היה}
\textblock{מכדי כמה מרובע יתר על העגול רביע לארבע מאה מאה למאה עשרים וחמשה הני מאה ועשרים וחמשה הוו להו}
\textblock{תני רמי בר יחזקאל ים שעשה שלמה שלש אמות תחתונות מרובעות ושתים עליונות עגולות}
\textblock{נהי דאיפכא לא מצית אמרת דשפתו עגול כתיב אלא אימא חדא}
\textblock{לא סלקא דעתך דכתיב (מלכים א ז, כו) אלפים בת יכיל בת כמה הויא שלש סאין דכתיב (יחזקאל מה, יד) מעשר הבת מן הכור דהוה להו שיתא אלפי גריוי}
\textblock{והא כתיב (דברי הימים ב ד, ה) מחזיק בתים שלשת אלפים ההוא לגודשא}
\textblock{אמר אביי שמע מינה האי גודשה תלתא הוי ותנן נמי שידה תיבה ומגדל כוורת הקש וכוורת הקנים ובור ספינה אלכסנדרית אע"פ שיש להן שולים והן מחזיקות ארבעים סאה בלח שהן כוריים ביבש טהורין:}
\textblock{{\large\emph{מתני׳}} לחיין שאמרו גובהן עשרה טפחים ורחבן ועוביין כל שהוא ר' יוסי אומר רחבן ג' טפחים:}
\textblock{{\large\emph{גמ׳}} לחיין שאמרו כו' לימא תנן סתמא כר"א דאמר לחיין בעינן}
\textblock{לא מאי לחיין לחיין דעלמא אי הכי קורה נמי ניתני קורות ומאי קורות קורות דעלמא}
\textblock{הכי קאמר אותן לחיין שנחלקו בהן ר' אליעזר וחכמים גובהן עשרה טפחים ורוחבן ועוביין כל שהוא וכמה כל שהוא תני רבי חייא אפילו כחוט הסרבל}
\textblock{תנא עשה לחי לחצי מבוי אין לו אלא חצי מבוי פשיטא אלא אימא יש לו חצי מבוי הא נמי פשיטא מהו דתימא ליחוש דילמא אתי לאישתמושי בכוליה קמ"ל}
\textblock{אמר רבא עשה לחי למבוי והגביהו מן הקרקע שלשה או שהפליגו מן הכותל שלשה לא עשה ולא כלום אפילו לרשב"ג דאמר אמרינן לבוד הני מילי למעלה אבל למטה כיון דהויא מחיצה שהגדיין בוקעין בה לא קאמר:}
\textblock{ר' יוסי אומר רחבן ג' טפחים: אמר רב יוסף אמר רב יהודה אמר שמואל אין הלכה כר' יוסי לא בהילמי ולא בלחיין}
\textblock{אמר ליה רב הונא בר חיננא בהילמי אמרת לן בלחיין לא אמרת לן מאי שנא בהילמי דפליגי רבנן עליה לחיין נמי פליגי רבנן עליה א"ל שאני לחיין משום דקאי רבי כוותיה}
\textblock{רב רחומי מתני הכי אמר רב יהודה בריה דרב שמואל [בר שילת] משמיה דרב אין הלכה כרבי יוסי לא בהילמי ולא בלחיין א"ל אמרת אמר להו לא אמר רבא האלהים אמרה וגמירנא לה מיניה ומאי טעמא קא הדר ביה משום דר' יוסי נימוקו עמו}
\textblock{אמר ליה רבא בר רב חנן לאביי הילכתא מאי א"ל פוק חזי מאי עמא דבר}
\textblock{איכא דמתני לה אהא השותה מים לצמאו אומר שהכל נהיה בדברו ר' טרפון אומר בורא נפשות רבות וחסרונן על כל מה שבראת א"ל רב חנן לאביי הלכתא מאי א"ל פוק חזי מאי עמא דבר}
\newsection{דף טו}
\textblock{איתמר לחי העומד מאליו אביי אמר הוי לחי רבא אמר לא הוי לחי}
\textblock{היכא דלא סמכינן עליה מאתמול כולי עלמא לא פליגי דלא הוי לחי כי פליגי היכא דסמכינן עליה מאתמול אביי אמר הוי לחי דהא סמכינן עליה מאתמול רבא אמר לא הוי לחי כיון דמעיקרא לאו אדעתיה דהכי עבידי לא הוי לחי}
\textblock{קא סלקא דעתך כי היכי דפליגי בלחי פליגי נמי במחיצה}
\textblock{ת"ש העושה סוכתו בין האילנות ואילנות דפנות לה כשירה הכא במאי עסקינן שנטען מתחילה לכך אי הכי פשיטא מהו דתימא ליגזור דילמא אתי לאישתמושי באילן קמ"ל}
\textblock{ת"ש היה שם אילן או גדר או חיצת הקנים נידון משום דיומד}
\textblock{הכא נמי במאי עסקינן שעשאן מתחילה לכך אי הכי מאי קמ"ל [קמ"ל] חיצת הקנים קנה קנה פחות משלשה טפחים כדבעא מיניה אביי מרבה}
\textblock{ת"ש אילן המסיך על הארץ אם אין נופו גבוה מן הארץ ג' טפחים מטלטלין תחתיו הכא נמי במאי עסקינן שנטעו מתחילה לכך}
\textblock{אי הכי ליטלטל בכולו אלמה אמר רב הונא בריה דרב יהושע אין מטלטלין בו אלא בית סאתים}
\textblock{משום דהוי דירה שתשמישה לאויר וכל דירה שתשמישה לאויר אין מטלטלין בה אלא בית סאתים}
\textblock{ת"ש שבת בתל שהוא גבוה עשרה והוא מארבע אמות ועד בית סאתים וכן בנקע שהוא עמוק עשרה והוא מארבע אמות ועד בית סאתים וקמה קצורה ושיבולות מקיפות אותה מהלך את כולה וחוצה לה אלפים אמה}
\textblock{וכי תימא הכא נמי שעשה מתחילה לכך בשלמא קמה לחיי אלא תל ונקע מאי איכא למימר}
\textblock{אלא במחיצות כולי עלמא לא פליגי דהויא מחיצה כי פליגי בלחי אביי לטעמיה דאמר לחי משום מחיצה ומחיצה העשויה מאליה הויא מחיצה ורבא לטעמיה דאמר לחי משום היכר אי עבידא בידים הויא היכר ואי לא לא הוי היכר}
\textblock{ת"ש אבני גדר היוצאות מן הגדר מובדלות זו מזו פחות משלשה אין צריך לחי אחר שלשה צריך לחי אחר}
\textblock{הכא נמי שבנאן מתחילה לכך אי הכי פשיטא מהו דתימא למיסר בניינא הוא דעבידא קמ"ל}
\textblock{ת"ש דתני ר' חייא כותל שצידו אחד כנוס מחברו בין שנראה מבחוץ ושוה מבפנים ובין שנראה מבפנים ושוה מבחוץ נדון משום לחי}
\textblock{הכא נמי שעשאו מתחילה לכך אי הכי מאי קמ"ל הא קמ"ל נראה מבחוץ ושוה מבפנים נדון משום לחי}
\textblock{תא שמע דרב הוה יתיב בההוא מבואה הוה יתיב רב הונא קמיה אמר ליה לשמעיה זיל אייתי לי כוזא דמיא עד דאתא נפל לחיא אחוי ליה בידיה קם אדוכתיה אמר ליה רב הונא לא סבר לה מר לסמוך אדיקלא אמר דמי האי מרבנן כמאן דלא פרשי אינשי שמעתא מי סמכינן עליה מאתמול}
\textblock{טעמא דלא סמכינן הא סמכינן הוי לחי}
\textblock{לימא אביי ורבא בדלא סמכינן עליה פליגי הא סמכינן עליה הוה לחי לא ס"ד דההוא ברקא דהוה בי בר חבו דהוו פליגי בה אביי ורבא כולי שנייהו:}
\textblock{{\large\emph{מתני׳}} בכל עושין לחיין אפילו בדבר שיש בו רוח חיים ורבי מאיר אוסר ומטמא משום גולל}
\textblock{ור' מאיר מטהר וכותבין עליו גיטי נשים ור' יוסי הגלילי פוסל:}
\textblock{{\large\emph{גמ׳}} תניא ר' מאיר אומר כל דבר שיש בו רוח חיים אין עושין אותו לא דופן לסוכה ולא לחי למבוי לא פסין לביראות ולא גולל לקבר משום רבי יוסי הגלילי אמרו אף אין כותבין עליו גיטי נשים}
\textblock{מאי טעמא דרבי יוסי הגלילי דתניא (דברים כד, א) ספר אין לי אלא ספר מניין לרבות כל דבר ת"ל (דברים כד, א) וכתב לה מכל מקום אם כן מה ת"ל ספר לומר לך מה ספר דבר שאין בו רוח חיים ואינו אוכל אף כל דבר שאין בו רוח חיים ואינו אוכל}
\textblock{ורבנן מי כתיב בספר ספר כתיב לספירות דברים בעלמא הוא דאתא}
\textblock{ורבנן האי וכתב לה מאי דרשי ביה ההוא מבעי ליה בכתיבה מתגרשת ואינה מתגרשת בכסף סלקא דעתך אמינא הואיל ואיתקש יציאה להויה מה הויה בכסף אף יציאה בכסף קמ"ל}
\textblock{ור' יוסי הגלילי האי סברא מנא ליה נפקא ליה מספר כריתות ספר כורתה ואין דבר אחר כורתה}
\textblock{ורבנן האי ספר כריתות מיבעי ליה לדבר הכורת בינו לבינה לכדתניא הרי זה גיטך על מנת שלא תשתי יין על מנת שלא תלכי לבית אביך לעולם אין זה כריתות כל שלשים יום הרי זה כריתות}
\textblock{ורבי יוסי הגלילי נפקא ליה מכרת כריתות}
\textblock{ורבנן כרת כריתות לא דרשי:}
\textblock{{\large\emph{מתני׳}} שיירא שחנתה בבקעה והקיפוה כלי בהמה מטלטלין בתוכה ובלבד שיהא גדר גבוה עשרה טפחים ולא יהו פירצות יתרות על הבנין}
\textblock{כל פירצה שהיא כעשר אמות מותרת מפני שהיא כפתח יתר מכאן אסור:}
\textblock{{\large\emph{גמ׳}} איתמר פרוץ כעומד רב פפא אמר מותר רב הונא בריה דרב יהושע אמר אסור}
\textblock{רב פפא אמר מותר הכי אגמריה רחמנא למשה לא תפרוץ רובה רב הונא בריה דרב יהושע אמר אסור הכי אגמריה רחמנא למשה גדור רובה}
\textblock{תנן ולא יהו פירצות יתרות על הבנין הא כבנין מותר}
\textblock{לא תימא הא כבנין מותר אלא אימא אם בנין יתר על הפירצה מותר}
\textblock{אבל כבנין מאי אסור אי הכי ליתני לא יהו פירצות כבנין קשיא}
\textblock{ת"ש המקרה סוכתו בשפודין או בארוכות המטה אם יש ריוח ביניהן כמותן כשירה}
\textblock{הכא במאי עסקינן כשנכנס ויוצא}
\textblock{והא אפשר לצמצם}
\textblock{אמר רבי אמי במעדיף}
\textblock{רבא אמר אם היו נתונין ערב נותנו שתי שתי נותנו ערב}
\textblock{ת"ש שיירא שחנתה בבקעה והקיפוה בגמלין באוכפות}
\newsection{דף טז}
\textblock{בעביטין בשליפין בקנים בקולחות מטלטלין בתוכה ובלבד שלא יהא בין גמל לגמל כמלא גמל ובין אוכף לאוכף כמלא אוכף ובין עביט לעביט כמלא עביט}
\textblock{הכא נמי כשנכנס ויוצא}
\textblock{תא שמע נמצאת אתה אומר שלש מדות במחיצות: כל שהוא פחות משלשה צריך שלא יהא בין זה לזה שלשה כדי שלא יזדקר הגדי בבת ראש}
\textblock{כל שהוא ג' ומג' עד ד' צריך שלא יהא בין זה לזה כמלואו כדי שלא יהא פרוץ כעומד ואם היה פרוץ מרובה על העומד אף כנגד העומד אסור}
\textblock{כל שהוא ד' ומארבעה עד עשר אמות צריך שלא יהא בין זה לזה כמלואו שלא יהא פרוץ כעומד ואם היה פרוץ כעומד כנגד העומד מותר כנגד הפרוץ אסור ואם היה עומד מרובה על הפרוץ אף כנגד הפרוץ מותר}
\textblock{נפרצה ביותר מעשר אסור היו שם קנים הדוקרנים ועושה להן פיאה מלמעלה אפילו ביותר מעשר מותר}
\textblock{קתני מיהת רישא מג' ועד ד' ובלבד שלא יהא בין זה לזה כמלואו תיובתא דרב פפא}
\textblock{אמר לך רב פפא מאי מלואו נכנס ויוצא}
\textblock{הכי נמי מסתברא מדקתני אם היה פרוץ מרובה על העומד אף כנגד העומד אסור הא כעומד מותר שמע מינה}
\textblock{לימא תיהוי תיובתיה דר"ה בריה דרב יהושע אמר לך וליטעמיך אימא סיפא אם היה עומד מרובה על הפרוץ אף כנגד הפרוץ מותר הא כפרוץ אסור}
\textblock{סיפא קשיא לרב פפא רישא קשיא לר"ה בריה דרב יהושע}
\textblock{סיפא לרב פפא לא קשיא איידי דתנא רישא פרוץ מרובה על העומד תנא סיפא עומד מרובה על הפרוץ}
\textblock{רישא לרב הונא בריה דרב יהושע לא קשיא איידי דבעי למיתני סיפא עומד מרובה על הפרוץ תנא רישא פרוץ מרובה על העומד}
\textblock{בשלמא לרב פפא משום הכי לא עריב להו ותני להו}
\textblock{אלא לרב הונא בריה דרב יהושע ליערבינהו וליתננהו כל שהוא פחות משלשה ושלשה צריך שלא יהא בין זה לזה שלשה}
\textblock{משום דלא דמי פסולא דרישא לפסולא דסיפא פסולא דרישא כדי שלא יזדקר הגדי בבת אחת פסולא דסיפא שלא יהא פרוץ כעומד}
\textblock{פחות משלשה מני רבנן היא דאמרי פחות משלשה אמרינן לבוד שלשה לא אמרינן לבוד}
\textblock{אימא סיפא כל שהוא שלשה ומשלשה ועד ארבעה}
\textblock{אתאן לרבן שמעון בן גמליאל דאמר פחות מארבעה לבוד דאי רבנן משלשה ועד ארבעה שלשה וארבעה חד הוא}
\textblock{אמר אביי מדרישא רבנן סיפא נמי רבנן ומודו רבנן דכל למישרא כנגדו אי איכא מקום ארבעה חשיב ואי לא לא חשיב}
\textblock{רבא אמר מדסיפא רבן שמעון בן גמליאל רישא נמי רבן שמעון בן גמליאל וכי אמר רבן שמעון בן גמליאל אמרי' לבוד הנ"מ למעלה אבל למטה הוה ליה כמחיצה שהגדיים בוקעין בה לא אמרי' לבוד}
\textblock{תא שמע דפנות הללו שרובן פתחים וחלונות מותר ובלבד שיהא עומד מרובה על הפרוץ}
\textblock{שרובן סלקא דעתך אלא שריבה בהן פתחים וחלונות מותר ובלבד שיהא עומד מרובה על הפרוץ}
\textblock{הא כפרוץ אסור תיובתא דרב פפא תיובתא והילכתא כוותיה דרב פפא}
\textblock{תיובתא והילכתא אין משום דדייקא מתני' כוותיה דתנן לא יהיו פרצות יתירות על הבנין הא כבנין מותר:}
\textblock{{\large\emph{מתני׳}} מקיפין שלשה חבלים זה למעלה מזה וזה למעלה מזה ובלבד שלא יהו בין חבל לחבירו שלשה טפחים}
\textblock{שיעור חבלים ועוביין יתר על טפח כדי שיהא הכל עשרה טפחים}
\textblock{מקיפין בקנים ובלבד שלא יהא בין קנה לחבירו שלשה טפחים}
\textblock{בשיירא דברו דברי ר' יהודה וחכמים אומרים לא דברו בשיירא אלא בהווה}
\textblock{כל מחיצה שאינה של שתי ושל ערב אינה מחיצה דברי רבי יוסי בר' יהודה וחכמים אומרים אחד משני דברים:}
\textblock{{\large\emph{גמ׳}} אמר רב המנונא אמר רב הרי אמרו עומד מרובה על הפרוץ בשתי הוי עומד בעי רב המנונא בערב מאי}
\textblock{אמר אביי תא שמע שיעור חבלים ועוביין יתר על טפח שיהו הכל עשרה טפחים ואי איתא למה לי יתר על טפח}
\textblock{ליעביד פחות משלשה וחבל משהו פחות משלשה וחבל משהו פחות מארבעה וחבל משהו}
\textblock{ותיסברא האי פחות מארבעה היכא מוקים ליה אי מוקים ליה תתאי הוה ליה כמחיצה שהגדיים בוקעין בה}
\textblock{אי מוקים ליה עילאי אתי אוירא דהאי גיסא ודהאי גיסא ומבטל ליה}
\textblock{אי מוקים ליה במיצעי הוה ליה עומד מרובה על הפרוץ משתי רוחות שמעת מינה עומד מרובה על הפרוץ משתי רוחות הוי עומד}
\textblock{אלא רב המנונא הכי קא מיבעיא ליה כגון דאייתי מחצלת דהוי ז' ומשהו וחקק בה ג' ושבק בה ד' ומשהו ואוקמיה בפחות מג'}
\textblock{רב אשי אמר מחיצה תלויה איבעיא ליה כדבעא מיניה רבי טבלא מרב מחיצה תלויה מהו שתתיר בחורבה א"ל אין מחיצה תלויה מתרת אלא במים קל הוא שהקלו חכמים במים:}
\textblock{מקיפין בקנים וכו': בשירא אין ביחיד לא והתניא ר' יהודה אומר כל מחיצות שבת לא התירו ליחיד יותר מבית סאתים}
\textblock{כדאמר רב נחמן ואיתימא רב ביבי בר אביי לא נצרכא אלא ליתן להן כל צרכן ה"נ ליתן להן כל צרכן}
\textblock{היכא איתמר דרב נחמן ואיתימא רב ביבי בר אביי אהא דתנן כל מחיצה שאינה של שתי וערב אינה מחיצה דברי ר' יוסי בר' יהודה}
\textblock{ומי א"ר יוסי בר' יהודה הכי והתניא אחד יחיד ואחד שיירא לחבלים ומה בין יחיד לשיירא יחיד נותנין לו בית סאתים שנים נותנין להם בית סאתים ג' נעשו שיירא ונותנין להן בית שש דברי ר' יוסי ברבי יהודה}
\textblock{וחכמים אומרים אחד יחיד ואחד שיירא נותנין להן כל צרכן ובלבד שלא יהא בית סאתים פנוי}
\textblock{אמר רב נחמן ואיתימא רב ביבי בר אביי לא נצרכא אלא ליתן להן כל צרכן:}
\textblock{דרש רב נחמן משום רבינו שמואל יחיד נותנין לו בית סאתים ב' נותנין להן בית סאתים ג' נעשו שיירא ונותנין להן בית שש}
\textblock{שבקת רבנן ועבדת כר' יוסי ברבי יהודה}
\textblock{הדר אוקים רב נחמן אמורא עליה ודרש דברים שאמרתי לפניכם טעות הן בידי ברם כך אמרו יחיד נותנין לו בית סאתים שנים נותנין להן בית סאתים שלשה נעשו שיירא ונותנין להן כל צרכן}
\newsection{דף יז}
\textblock{רישא רבי יוסי ברבי יהודה וסיפא רבנן}
\textblock{אין משום דקאי אבוה בשיטתיה}
\textblock{אמר רב גידל אמר רב שלשה בחמש אסורין בשבע מותרין אמרו ליה אמר רב הכי אמר להו אורייתא נביאי וכתיבי דאמר רב הכי}
\textblock{אמר רב אשי מאי קשיא דילמא הכי קאמר הוצרכו לשש והקיפו בשבע אפי' בשבע מותרין לא הוצרכו אלא לחמש והקיפו בשבע אפי' בחמש אסורין}
\textblock{ואלא הא דקתני ובלבד שלא יהא בית סאתים פנוי מאי לאו פנוי מאדם לא פנוי מכלים}
\textblock{איתמר שלשה ומת אחד מהן שנים ונתוספו עליהן רב הונא ורבי יצחק חד אמר שבת גורמת וחד אמר דיורין גורמין}
\textblock{תסתיים דרב הונא הוא דאמר שבת גורמת דאמר רבה בעאי מרב הונא ובעאי מרב יהודה עירב דרך הפתח ונסתם הפתח דרך החלון ונסתם החלון מהו ואמר לי שבת הואיל והותרה הותרה תסתיים}
\textblock{לימא רב הונא ורבי יצחק בפלוגתא דרבי יוסי ורבי יהודה קמיפלגי דתנן חצר שנפרצה משתי רוחותיה וכן בית שנפרץ משתי רוחותיו וכן מבוי שניטלו קורותיו או לחייו מותרין לאותה שבת ואסורין לעתיד לבא דברי רבי יהודה}
\textblock{ר' יוסי אומר אם מותרין לאותה שבת מותרין לעתיד לבא ואם אסורין לעתיד לבא אסורין לאותה שבת}
\textblock{לימא רב הונא דאמר כר' יהודה ורבי יצחק דאמר כר' יוסי}
\textblock{אמר לך רב הונא אנא דאמרי אפי' לר' יוסי עד כאן לא קאמר רבי יוסי התם אלא דליתנהו למחיצות הכא איתנהו למחיצות}
\textblock{ור' יצחק אמר אנא דאמרי אפי' לר' יהודה עד כאן לא קאמר ר' יהודה התם אלא דאיתנהו לדיורין הכא ליתנהו לדיורין:}
\textblock{וחכמים אומרים אחד משני דברים: היינו ת"ק}
\textblock{איכא בינייהו יחיד ביישוב:}
\textblock{{\large\emph{מתני׳}} ארבעה דברים פטרו במחנה מביאין עצים מ"מ ופטורין מרחיצת ידים ומדמאי ומלערב:}
\textblock{{\large\emph{גמ׳}} ת"ר מחנה היוצאת למלחמת הרשות מותרין בגזל עצים יבשים ר' יהודה בן תימא אומר אף חונין בכל מקום ובמקום שנהרגו שם נקברין:}
\textblock{מותרין בגזל עצים יבשים: האי תקנתא דיהושע הוה דאמר מר עשרה תנאים התנה יהושע שיהו מרעין בחורשין ומלקטין עצים משדותיהן}
\textblock{התם בהיזמי והיגי הכא בשאר עצים}
\textblock{אי נמי התם במחוברין הכא בתלושין}
\textblock{אי נמי התם בלחין הכא ביבשים:}
\textblock{ר' יהודה בן תימא אומר אף חונין בכל מקום ובמקום שנהרגים שם נקברים: פשיטא מת מצוה הוא ומת מצוה קונה מקומו}
\textblock{לא צריכא אף על גב}
\textblock{דאית ליה קוברין דתניא איזהו מת מצוה כל שאין לו קוברין קורא ואחרים עונין אותו אין זה מת מצוה}
\textblock{ומת מצוה קנה מקומו והתניא המוציא מת מוטל בסרטיא מפניהו לימין אסרטיא או לשמאל אסרטיא}
\textblock{שדה בור ושדה ניר מפניהו לשדה בור שדה ניר ושדה זרע מפניהו לשדה ניר היו שתיהן נירות שתיהן זרועות שתיהן בורות מפניהו לכל רוח שירצה}
\textblock{אמר רב ביבי הכא במת מוטל על המיצר עסקינן מתוך שניתנה רשות לפנותו מן המיצר מפניהו לכל רוח שירצה:}
\textblock{ופטורין מרחיצת ידים: אמר אביי לא שנו אלא מים ראשונים אבל מים אחרונים חובה}
\textblock{אמר רב חייא בר אשי מפני מה אמרו מים אחרונים חובה מפני שמלח סדומית יש שמסמא את העינים}
\textblock{אמר אביי ומשתכחא כקורטא בכורא א"ל רב אחא בריה דרבא לרב אשי כייל מילחא מאי א"ל [הא] לא מיבעיא:}
\textblock{ומדמאי: דתנן מאכילין את העניים דמאי ואת אכסניא דמאי אמר רב הונא תנא בית שמאי אומרים אין מאכילין את העניים דמאי ואת אכסניא דמאי ובית הלל אומרים מאכילין את העניים דמאי ואת אכסניא דמאי:}
\textblock{ומלערב: אמרי דבי רבי ינאי לא שנו אלא עירובי חצירות אבל עירובי תחומין חייבין}
\textblock{דתני רבי חייא לוקין על עירובי תחומין דבר תורה}
\textblock{מתקיף לה רבי יונתן וכי לוקין על לאו שבאל מתקיף רב אחא בר יעקב אלא מעתה דכתיב (ויקרא יט, לא) אל תפנו אל האובות ואל הידעונים ה"נ דלא לקי}
\textblock{רבי יונתן הכי קשיא ליה לאו שניתן לאזהרת מיתת ב"ד וכל לאו שניתן לאזהרת מיתת בית דין אין לוקין עליו}
\textblock{אמר רב אשי מי כתיב אל יוציא (שמות טז, כט) אל יצא כתיב:}
\textblock{\par \par {\large\emph{הדרן עלך מבוי}}\par \par }
\textblock{}
\textblock{מתני׳ {\large\emph{עושין}} פסין לביראות}
\textblock{ארבעה דיומדין נראין כשמונה דברי ר' יהודה ר"מ אומר שמונה נראין כשנים עשר ארבעה דיומדים וארבעה פשוטין}
\textblock{גובהן עשרה טפחים ורוחבן ששה ועוביים כל שהוא וביניהן כמלא שתי רבקות של שלש שלש בקר דברי ר"מ}
\textblock{ר' יהודה אומר של ארבע קשורות ולא מותרות אחת נכנסת ואחת יוצאת}
\textblock{מותר להקריב לבאר ובלבד שתהא פרה ראשה ורובה בפנים ושותה}
\textblock{}
\textblock{מותר}
\newchap{פרק \hebrewnumeral{2}\quad עושין פסין}
\newsection{דף יח}
\textblock{}
\textblock{להרחיק כל שהוא ובלבד שירבה בפסין ר' יהודה אומר עד בית סאתים}
\textblock{אמרו לו לא אמרו בית סאתים אלא לגנה ולקרפף אבל אם היה דיר או סהר או מוקצה או חצר אפילו בית חמשת כורין אפילו בית עשרה כורין מותר ומותר להרחיק כל שהוא ובלבד שירבה בפסין:}
\textblock{{\large\emph{גמ׳}} לימא מתני' דלא כחנניא דתניא עושין פסין לבור וחבלין לשיירא וחנניא אומר חבלין לבור אבל לא פסין}
\textblock{אפי' תימא חנניא בור לחוד באר לחוד}
\textblock{איכא דאמרי מדלא קתני חנניא אומר עושין חבלין לבור ופסין לבאר מכלל דלחנניא לא שנא בור ולא שנא באר חבלין אין פסין לא לימא מתני' דלא כחנניא}
\textblock{אפילו תימא חנניא למאי דקאמר ת"ק קא מהדר ליה}
\textblock{לימא מתני' דלא כר"ע דתנן אחד באר הרבים ובור הרבים ובאר היחיד עושין להן פסין אבל בור היחיד עושין לו מחיצה גבוה עשרה טפחים דברי ר"ע}
\textblock{ואילו הכא קתני לביראות לביראות אין לבורות לא}
\textblock{אפי' תימא ר"ע באר מים חיים דפסיקא ליה לא שנא דרבים ול"ש דיחיד קתני בור מכונסין דלא פסיקא ליה לא קתני}
\textblock{לימא מתני' דלא כר' יהודה בן בבא דתנן רבי יהודה בן בבא אומר אין עושין פסין אלא לבאר הרבים בלבד ואילו הכא קתני לביראות ל"ש דרבים ול"ש דיחיד}
\textblock{אפי' תימא ר' יהודה בן בבא מאי ביראות ביראות דעלמא}
\textblock{מאי דיומדין א"ר ירמיה בן אלעזר דיו עמודין:}
\textblock{ד"יו למ"נודה שב"ח זונ"ית נתק"לקל במי"דה שלש"ה סימן:}
\textblock{תנן התם ר' יהודה אומר כל השיתין פטורין חוץ מן הדיופרא מאי דיופרא אמר עולא אילן העושה דיו פירות בשנה}
\textblock{א"ר ירמיה בן אלעזר דיו פרצוף פנים היה לו לאדם הראשון שנאמר (תהלים קלט, ה) אחור וקדם צרתני כתיב (בראשית ב, כב) ויבן ה' אלהים את הצלע וגו' רב ושמואל חד אמר פרצוף וחד אמר זנב}
\textblock{בשלמא למ"ד פרצוף היינו דכתיב אחור וקדם צרתני אלא למ"ד זנב מאי אחור וקדם צרתני}
\textblock{כדר' אמי דא"ר אמי אחור למעשה בראשית וקדם לפורענות}
\textblock{בשלמא אחור למעשה בראשית דלא איברי עד מעלי שבתא אלא וקדם לפורענות מאי היא אילימא משום קללה הא בתחילה נתקלל נחש ולבסוף נתקללה חוה ולבסוף נתקלל אדם}
\textblock{אלא למבול דכתיב (בראשית ז, כג) וימח את כל היקום אשר על פני האדמה מאדם ועד בהמה וגו'}
\textblock{בשלמא למ"ד פרצוף היינו דכתיב (בראשית ב, ז) וייצר תרין יודין אלא למ"ד זנב מאי וייצר}
\textblock{כדר"ש בן פזי דא"ר שמעון בן פזי אוי לי מיצרי אוי לי מיוצרי}
\textblock{בשלמא למאן דאמר פרצוף היינו דכתיב (בראשית ה, ב) זכר ונקבה בראם אלא למאן דאמר זנב מאי זכר ונקבה בראם}
\textblock{לכדר' אבהו דר' אבהו רמי כתיב זכר ונקבה בראם וכתיב (בראשית א, כז) (כי) בצלם אלהים ברא אותו בתחלה עלתה במחשבה לבראות שנים ולבסוף לא נברא אלא אחד}
\textblock{בשלמא למאן דאמר פרצוף היינו דכתיב (בראשית ב, כא) ויסגור בשר תחתנה אלא למאן דאמר זנב מאי ויסגור בשר תחתנה}
\textblock{אמר רב זביד ואיתימא ר' ירמיה ואיתימא רב נחמן בר יצחק לא נצרכה אלא למקום חתך}
\textblock{בשלמא למ"ד זנב היינו דכתיב ויבן אלא למ"ד פרצוף מאי ויבן}
\textblock{לכדר' שמעון בן מנסיא דדריש ר"ש בן מנסיא ויבן ה' אלהים את הצלע מלמד שקילעה הקב"ה לחוה והביאה לאדם הראשון שכן בכרכי הים קורין לקלעיתא בנייתא}
\textblock{דבר אחר ויבן ה' אלהים אמר רב חסדא ואמרי לה במתניתא תנא מלמד שבנאה הקב"ה לחוה כבניין}
\textblock{אוצר מה אוצר זה רחב מלמטה וקצר מלמעלה כדי לקבל את הפירות אף האשה רחבה מלמטה וקצרה מלמעלה כדי לקבל את הולד}
\textblock{ויביאה אל האדם מלמד שעשה הקב"ה שושבינות לאדם הראשון מכאן לגדול שיעשה שושבינות לקטן ואל ירע לו}
\textblock{ולמאן דאמר פרצוף הי מינייהו סגי ברישא אמר רב נחמן בר יצחק מסתברא דזכר סגי ברישא דתניא לא יהלך אדם אחורי אשה בדרך ואפי' היא אשתו נזדמנה על הגשר יסלקנה לצדדין וכל העובר אחורי אשה בנהר אין לו חלק לעולם הבא:}
\textblock{תנו רבנן המרצה מעות לאשה מידו לידה או מידה לידו בשביל שיסתכל בה אפילו דומה למשה רבינו שקיבל תורה מהר סיני לא ינקה מדינה של גיהנם ועליו הכתוב אומר (משלי יא, כא) יד ליד לא ינקה רע לא ינקה מדינה של גיהנם}
\textblock{אמר רב נחמן מנוח עם הארץ היה שנאמר (שופטים יג, יא) ויקם וילך מנוח אחרי אשתו}
\textblock{מתקיף לה רב נחמן בר יצחק אלא מעתה גבי אלקנה דכתיב וילך אלקנה אחרי אשתו הכי נמי וגבי אלישע דכתיב (מלכים ב ד, ל) ויקם וילך אחריה הכי נמי}
\textblock{אלא אחרי דבריה ועצתה הכא נמי אחרי דבריה ועצתה}
\textblock{אמר רב אשי ולמאי דאמר רב נחמן מנוח עם הארץ היה אפילו בי רב נמי לא קרא דכתיב (בראשית כד, סא) ותקם רבקה ונערותיה ותרכבנה על הגמלים ותלכנה אחרי האיש ולא לפני האיש}
\textblock{אמר רבי יוחנן אחרי ארי ולא אחרי אשה אחרי אשה ולא אחרי עבודת כוכבים אחורי עבודת כוכבים ולא אחורי בית הכנסת בשעה שמתפללין}
\textblock{ואמר ר' ירמיה בן אלעזר כל אותן השנים שהיה אדם הראשון בנידוי הוליד רוחין ושידין ולילין שנאמר (בראשית ה, ג) ויחי אדם שלשים ומאת שנה ויולד בדמותו כצלמו מכלל דעד האידנא לאו כצלמו אוליד}
\textblock{מיתיבי היה ר' מאיר אומר אדם הראשון חסיד גדול היה כיון שראה שנקנסה מיתה על ידו ישב בתענית מאה ושלשים שנה ופירש מן האשה מאה ושלשים שנה והעלה זרזי תאנים על בשרו מאה ושלשים שנה}
\textblock{כי קאמרינן ההוא בשכבת זרע דחזא לאונסיה}
\textblock{ואמר רבי ירמיה בן אלעזר מקצת שבחו של אדם אומרים בפניו וכולו שלא בפניו מקצת שבחו בפניו דכתיב (בראשית ז, א) כי אותך ראיתי צדיק לפני בדור הזה}
\textblock{כולו שלא בפניו דכתיב (בראשית ו, ט) נח איש צדיק תמים היה בדורותיו}
\textblock{וא"ר ירמיה בן אלעזר מאי דכתיב (בראשית ח, יא) והנה עלה זית טרף בפיה אמרה יונה לפני הקב"ה רבש"ע יהיו מזונותי מרורין כזית ומסורין בידך ואל יהיו מתוקין כדבש ותלוין ביד בשר ודם כתיב הכא טרף וכתיב התם (משלי ל, ח) הטריפני לחם חוקי}
\textblock{וא"ר ירמיה בן אלעזר כל בית שנשמעין בו דברי תורה בלילה שוב אינו נחרב שנאמר (איוב לה, י) ולא אמר איה אלוה עושי נותן זמירות בלילה}
\textblock{ואמר רבי ירמיה בן אלעזר מיום שחרב בית המקדש דיו לעולם שישתמש בשתי אותיות שנאמר (תהלים קנ, ו) כל הנשמה תהלל יה הללויה}
\textblock{ואמר רבי ירמיה בן אלעזר נתקללה בבל נתקללו שכיניה נתקללה שומרון נתברכו שכיניה נתקללה בבל נתקללו שכיניה דכתיב (ישעיהו יד, כג) ושמתיה למורש קיפוד ואגמי מים נתקללה שומרון נתברכו שכיניה דכתיב (מיכה א, ו) ושמתי שומרון לעי השדה}
\newsection{דף יט}
\textblock{למטעי כרם}
\textblock{וא"ר ירמיה בן אלעזר בא וראה שלא כמדת הקב"ה מדת בשר ודם מדת בשר ודם מתחייב אדם הריגה למלכות מטילין לו חכה לתוך פיו כדי שלא יקלל את המלך}
\textblock{מדת הקב"ה אדם מתחייב הריגה למקום שותק שנאמר (תהלים סה, ב) לך דומיה תהלה ולא עוד אלא שמשבח שנאמר תהלה ולא עוד אלא שדומה לו כאילו מקריב קרבן שנאמר (תהלים סה, ב) ולך ישולם נדר}
\textblock{היינו דא"ר יהושע בן לוי מאי דכתיב (תהלים פד, ז) עוברי בעמק הבכא מעין ישיתוהו גם ברכות יעטה מורה}
\textblock{עוברי אלו בני אדם שעוברין על רצונו של הקב"ה עמק שמעמיקין להם גיהנם הבכא שבוכין ומורידין דמעות כמעיין של שיתין גם ברכות יעטה מורה שמצדיקין עליהם את הדין ואומרים לפניו רבונו של עולם יפה דנת יפה זכית יפה חייבת ויפה תקנת גיהנם לרשעים גן עדן לצדיקים}
\textblock{איני והאמר רבי שמעון בן לקיש רשעים אפילו על פתחו של גיהנם אינם חוזרין בתשובה שנאמר (ישעיהו סו, כד) ויצאו וראו בפגרי האנשים הפושעים בי וגו' שפשעו לא נאמר אלא הפושעים שפושעים והולכין לעולם}
\textblock{ל"ק הא בפושעי ישראל הא בפושעי עובדי כוכבים}
\textblock{הכי נמי מסתברא דא"כ קשיא דר"ל אדריש לקיש דאמר ריש לקיש פושעי ישראל אין אור גיהנם שולטת בהן ק"ו ממזבח הזהב}
\textblock{מה מזבח הזהב שאין עליו אלא כעובי דינר זהב עמד כמה שנים ולא שלטה בו האור פושעי ישראל שמליאין מצות כרמון שנאמר (שיר השירים ו, ז) כפלח הרמון רקתך ואמר ר"ש בן לקיש אל תיקרי רקתך אלא ריקתיך שאפי' ריקנין שבך מליאין מצות כרמון עאכ"ו}
\textblock{אלא הא דכתיב עוברי בעמק הבכא ההוא דמחייבי ההיא שעתא בגיהנם ואתי אברהם אבינו ומסיק להו ומקבל להו בר מישראל שבא על בת עובד כוכבים דמשכה ערלתו ולא מבשקר ליה}
\textblock{מתקיף לה רב כהנא השתא דאמרת הפושעים דפשעי ואזלי אלא מעתה דכתיב המוציא והמעלה דמסיק ודמפיק הוא אלא דאסיק ואפיק הכי נמי דפשעי הוא}
\textblock{ואמר רבי ירמיה (בר) אלעזר שלשה פתחים יש לגיהנם אחד במדבר ואחד בים ואחד בירושלים במדבר דכתיב (במדבר טז, לג) וירדו הם וכל אשר להם חיים שאולה}
\textblock{בים דכתיב (יונה ב, ג) מבטן שאול שועתי שמעת קולי}
\textblock{בירושלים דכתיב (ישעיהו לא, ט) נאם ה' אשר אור לו בציון ותנור לו בירושלים ותנא דבי רבי ישמעאל אשר אור לו בציון זו גיהנם ותנור לו בירושלים זו פתחה של גיהנם}
\textblock{ותו ליכא והאמר ר' מריון אמר ר' יהושע בן לוי ואמרי לה תנא רבה בר מריון בדבי רבי יוחנן בן זכאי שתי תמרות יש בגי בן הנום ועולה עשן מביניהן וזו היא ששנינו ציני הר הברזל כשירות וזו היא פתחה של גיהנם דילמא היינו דירושלים}
\textblock{א"ר יהושע בן לוי ז' שמות יש לגיהנם ואלו הן שאול ואבדון ובאר שחת ובור שאון וטיט היון וצלמות וארץ התחתית}
\textblock{שאול דכתיב (יונה ב, ג) מבטן שאול שועתי שמעת קולי אבדון דכתיב (תהלים פח, יב) היסופר בקבר חסדך אמונתך באבדון באר שחת דכתיב (תהלים טז, י) כי לא תעזוב נפשי לשאול לא תתן חסידך לראות שחת ובור שאון וטיט היון דכתיב (תהלים מ, ג) ויעלני מבור שאון מטיט היון וצלמות דכתיב (תהלים קז, י) יושבי חשך וצלמות וארץ התחתית גמרא הוא}
\textblock{ותו ליכא והאיכא גיהנם גיא שעמוקה (בגיהנם) שהכל יורד לה על עסקי (הנם)}
\textblock{והאיכא תפתה דכתיב (ישעיהו ל, לג) כי ערוך מאתמול תפתה ההוא שכל המתפתה ביצרו יפול שם}
\textblock{גן עדן אמר ריש לקיש אם בארץ ישראל הוא בית שאן פתחו ואם בערביא בית גרם פתחו ואם בין הנהרות הוא דומסקנין פתחו בבבל אביי משתבח בפירי דמעבר ימינא רבא משתבח בפירי דהרפניא:}
\textblock{וביניהן כמלוא שתי וכו': פשיטא כיון דתנא ליה דקשורות הוו אנן ידעינן דלא הוו מותרות}
\textblock{מהו דתימא קשורות כעין קשורות אבל ממש לא קמ"ל ולא מותרות:}
\textblock{אחת נכנסת ואחת יוצאת: תנא רבקה נכנסת ורבקה יוצאת ת"ר כמה ראשה ורובה של פרה שתי אמות וכמה עוביה של פרה אמה ושני שלישי אמה}
\textblock{שהן כעשר דברי ר"מ ר' יהודה אומר כשלש עשרה אמה וכארבע עשרה אמה}
\textblock{כעשר הא עשר הויין משום דבעי למיתנא סיפא כשלש עשרה}
\textblock{כשלש עשרה טפי הויין משום דבעי למתני כארבע עשרה וכארבע עשרה הא לא הויא אמר רב פפא יתירות על שלש עשרה ואינן מגיעות לארבע עשרה}
\textblock{א"ר פפא בבור שמונה דכ"ע לא פליגי דלא בעינן פשוטין}
\textblock{בבור שתים עשרה דכ"ע לא פליגי דבעינן פשוטין}
\textblock{כי פליגי משמונה עד שתים עשרה לר' מאיר בעינן פשוטין לר' יהודה לא בעינן פשוטין}
\textblock{ורב פפא מאי קמ"ל תנינא}
\textblock{רב פפא ברייתא לא שמיע ליה וקמ"ל כברייתא:}
\textblock{ארי"ך יות"ר בת"ל חיצ"ת חצ"ר שיבש"ה סימן: בעא מיניה אביי מרבה האריך בדיומדין כשיעור פשוטין לר"מ מהו}
\textblock{א"ל תניתוה ובלבד שירבה בפסין מאי לאו דמאריך בדיומדין לא דמפיש ועביד פשוטין}
\textblock{א"ה האי ובלבד שירבה בפסין עד שירבה פסין מיבעי ליה תני עד שירבה פסין}
\textblock{א"ד א"ל תניתוה ובלבד שירבה בפסין מאי לאו דמפיש ועביד פשוטין לא דמאריך בדיומדין}
\textblock{הכי נמי מסתברא מדקתני ובלבד שירבה בפסין ש"מ}
\textblock{בעא מיניה אביי מרבה יותר משלש עשרה אמה ושליש לר' יהודה מהו פשוטין עביד או בדיומדין מאריך}
\textblock{א"ל תניתוה כמה הן מקורבין כדי ראשה ורובה של פרה וכמה מרוחקין אפי' כור ואפי' כוריים}
\textblock{ר' יהודה אומר בית סאתים מותר יותר מבית סאתים אסור אמרו לו לר' יהודה אי אתה מודה בדיר וסהר ומוקצה וחצר אפילו בת חמשת כורים ואפי' בת עשרה כורים שמותר}
\textblock{אמר להן זו מחיצה ואלו פסין}
\textblock{ואם איתא זו מחיצה וזו (היא) מחיצה מיבעי ליה}
\textblock{הכי קאמר זו תורת מחיצה עליה ופרצותיה בעשר ואלו תורת פסין עליהן ופרצותיהן בשלש עשרה אמה ושליש}
\textblock{בעא מיניה אביי מרבה תל המתלקט עשרה מתוך ארבע נידון משום דיומד או אינו נידון משום דיומד}
\textblock{א"ל תניתוה ר' שמעון בן אלעזר אומר היתה שם אבן מרובעת רואין כל שאילו תחלק ויש בה אמה לכאן ואמה לכאן נידון משום דיומד ואם לאו אינו נידון משום דיומד}
\textblock{רבי ישמעאל בנו של רבי יוחנן בן ברוקה אומר היתה שם אבן עגולה רואין כל שאילו תחקק ותחלק ויש בה אמה לכאן ואמה לכאן נידון משום דיומד ואם לאו אינו נידון משום דיומד}
\textblock{במאי קא מיפלגי מר סבר חד רואין אמרינן תרי רואין לא אמרינן ומר סבר אפי' תרי רואין נמי אמרינן}
\textblock{בעא מיניה אביי מרבה חיצת הקנים קנה קנה פחות משלשה נידון משום דיומד או לאו}
\textblock{אמר ליה תניתוה היה שם אילן או גדר או חיצת הקנים נידון משום דיומד מאי לאו קנה קנה פחות משלשה}
\textblock{לא גודריתא דקני אי הכי היינו אילן}
\textblock{ואלא מאי קנה קנה פחות משלשה היינו גדר אלא מאי אית לך למימר תרי גווני גדר הכא נמי תרי גווני אילן}
\textblock{איכא דאמרי גודריתא דקני קא מיבעיא ליה גודריתא דקני מאי א"ל תניתוה היה שם גדר או אילן או חיצת הקנים נידון משום דיומד מאי לאו גודריתא דקני}
\textblock{לא קנה קנה פחות משלשה אי הכי היינו גדר}
\textblock{ואלא מאי גודריתא דקני היינו אילן אלא מאי אית לך למימר}
\newsection{דף כ}
\textblock{תרי גווני אילן הכא נמי תרי גווני גדר}
\textblock{בעא מיניה אביי מרבה חצר שראשה נכנס לבין הפסין מהו לטלטל מתוכה לבין הפסין ומבין הפסין לתוכה א"ל מותר}
\textblock{שתים מאי אמר ליה אסור}
\textblock{אמר רב הונא שתים אסורין ואפילו עירבו גזירה שמא יאמרו עירוב מועיל לבין הפסין רבא אמר עירבו מותר}
\textblock{א"ל אביי לרבא תניא דמסייע לך חצר שראשה אחד נכנס לבין הפסין מותר לטלטל מתוכה לבין הפסין ומבין הפסין לתוכה אבל שתים אסור בד"א שלא עירבו אבל עירבו מותרין}
\textblock{לימא תיהוי תיובתא דרב הונא אמר לך רב הונא התם דהדרן וערבן}
\textblock{בעא מיניה אביי מרבה יבשו מים בשבת מהו א"ל כלום נעשית מחיצה אלא בשביל מים מים אין כאן מחיצה אין כאן}
\textblock{בעי רבין יבשו מים בשבת ובאו בשבת מהו א"ל אביי יבשו בשבת לא תיבעי לך דבעי מיניה דמר ופשיט לי דאסיר}
\textblock{באו נמי לא תיבעי לך דהוה ליה מחיצה העשויה בשבת ותניא כל מחיצה העשויה בשבת בין בשוגג בין במזיד בין באונס בין ברצון שמה מחיצה}
\textblock{ולאו אתמר עלה אמר ר"נ לא שנו אלא לזרוק אבל לטלטל לא}
\textblock{כי איתמר דר"נ אמזיד איתמר}
\textblock{אמר ר"א הזורק לבין פסי הביראות חייב (א"ל) פשיטא אי לאו מחיצה היא היכי משתרי ליה למלאות}
\textblock{לא צריכא דעבד כעין פסי ביראות ברה"ר וזרק לתוכה חייב}
\textblock{הא נמי פשיטא אי לאו דבעלמא מחיצה היא גבי בור היכי משתרי ליה לטלטלי לא צריכא אע"ג דקא בקעי בה רבים}
\textblock{ומאי קמ"ל דלא אתו רבים ומבטלי מחיצתא הא א"ר אלעזר חדא זימנא}
\textblock{דתנן רבי יהודה אומר אם היתה דרך רה"ר מפסקתן יסלקנה לצדדין וחכ"א אינו צריך רבי יוחנן ור"א דאמרי תרווייהו כאן הודיעך כחן של מחיצות}
\textblock{אי מהתם הוה אמינא כאן ולא סבירא ליה קא משמע לן כאן וסבירא ליה}
\textblock{ולימא הא ולא בעי הך חדא מכלל חבירתה איתמר:}
\textblock{מותר להקריב לבאר וכו': תנן התם לא יעמוד אדם ברשות הרבים וישתה ברשות היחיד ברשות היחיד וישתה ברשות הרבים אלא אם כן מכניס ראשו ורובו למקום שהוא שותה}
\textblock{וכן בגת}
\textblock{גבי אדם הא אמר דבעי ראשו ורובו גבי פרה מי בעינן לה ראשה ורובה או לא}
\textblock{כל היכא דקא נקיט מנא ולא נקיט לה לא תיבעי לך דבעי ראשה ורובה מלגיו כי תבעי לך היכא דנקיט מנא ונקיט לה מאי}
\textblock{א"ל תניתוה ובלבד שתהא הפרה ראשה ורובה מבפנים ושותה מאי לאו דנקיט לה ונקיט מנא לא דנקיט מנא ולא נקיט לה}
\textblock{וכי נקיט מנא ולא נקיט לה מי שרי והתניא לא ימלא אדם מים ויתן בשבת לפני בהמתו אבל ממלא הוא ושופך והיא שותה מאיליה}
\textblock{הא אתמר עלה אמר אביי הכא באבוס העומד ברשות הרבים גבוה י' טפחים ורוחב ד' וראשו אחד נכנס לבין הפסין}
\textblock{גזרה דילמא חזי ליה לאבוס דמקלקל ואתי לתקוניה ודרא ליה לדוולא בהדיה וקא מפיק מרשות היחיד לרשות הרבים}
\textblock{וכי האי גוונא מי מיחייב והאמר רב ספרא אמר ר' אמי אמר ר' יוחנן המפנה חפציו מזוית לזוית ונמלך עליהן והוציאן פטור שלא היתה עקירה משעה ראשונה לכך}
\textblock{אלא זמנין דמתקן ליה והדר מעייל ליה וקא מעייל מרשות הרבים לרשות היחיד}
\textblock{איכא דאמרי גבי אדם הא קאמרינן דסגי ליה בראשו ורובו גבי פרה מי סגי לה בראשה ורובה או לא}
\textblock{היכא דנקיט מנא ונקיט לה לא תיבעי לך דסגי לה בראשה ורובה אלא כי תיבעי לך דנקיט מנא ולא נקיט לה מאי}
\textblock{אמר ליה תניתוה ובלבד שתהא פרה ראשה ורובה מבפנים ושותה מאי לאו דנקיט מנא ולא נקיט לה לא דנקיט מנא ונקיט לה}
\textblock{והכי נמי מסתברא דאי נקיט מנא ולא נקיט לה מי שרי והתניא לא ימלא אדם מים ויתן לפני בהמתו אבל ממלא ושופך והיא שותה מאיליה}
\textblock{הא איתמר עלה אמר אביי הכא באבוס העומד ברשות הרבים גבוה עשרה טפחים ורוחב ארבעה וראשו נכנס לבין הפסין דזמנין דחזי ליה לאבוס דמקלקל ואתי לתקוניה ודרי ליה לדוולא בהדיה וקא מפיק מרשות היחיד לרשות הרבים}
\textblock{וכי האי גוונא מי מיחייב והאמר רב ספרא אמר ר' אמי אמר ר' יוחנן המפנה חפציו מזוית לזוית ונמלך עליהן והוציאן פטור שלא היתה עקירה משעה ראשונה לכך}
\textblock{אלא זמנין דמתקן ליה והדר מעייל ליה וקא מעייל ליה מרה"ר לרשות היחיד}
\textblock{תא שמע גמל שראשו ורובו מבפנים אובסין אותו מבפנים והא איבוס כמאן דנקיט מנא ונקיט לה דמיא וקא בעינן ראשה ורובה}
\textblock{אמר רב אחא בר רב הונא אמר רב ששת שאני גמל הואיל וצוארו ארוך}
\textblock{ת"ש בהמה שראשה ורובה בפנים אובסין אותה מבפנים והא אבוס כמאן דנקיט מנא ונקיט לה וקא בעינן ראשו ורובו מאי בהמה נמי דקתני גמל}
\textblock{והתניא בהמה והתניא גמל}
\textblock{מידי גבי הדדי תניא תנ"ה ר"א אוסר בגמל הואיל וצוארו ארוך}
\textblock{א"ר יצחק בר אדא לא הותרו פסי ביראות אלא לעולי רגלים בלבד והתניא לא הותרו פסי ביראות אלא לגבי בהמה בלבד מאי בהמה בהמת עולי רגלים אבל אדם}
\newsection{דף כא}
\textblock{מטפס ועולה מטפס ויורד}
\textblock{איני והאמר רב יצחק אמר רב יהודה אמר שמואל לא הותרו פסי ביראות אלא לבאר מים חיים בלבד ואי לבהמה מה לי חיים מה לי מכונסין בעינן מידי דחזי לאדם}
\textblock{גופא לא הותרו פסי ביראות אלא לבהמה בלבד אבל אדם מטפס ועולה מטפס ויורד ואם היו רחבין אפילו לאדם נמי ולא ימלא אדם מים ויתן לפני בהמתו אבל ממלא הוא ושופך לפני בהמה ושותה מאיליה}
\textblock{מתקיף לה רב ענן אם כן מה הועילו פסי ביראות מה הועילו למלאות מהן}
\textblock{אלא מה הועיל ראשה ורובה של פרה}
\textblock{אמר אביי הכא במאי עסקינן באיבוס העומד ברה"ר גבוה עשרה ורוחב ארבעה וראשו נכנס לבין הפסין וכו'}
\textblock{אמר רב ירמיה בר אבא אמר רב אין בורגנין בבבל ולא פסי ביראות בחו"ל}
\textblock{בורגנין בבבל לא דשכיחי בידקי פסי ביראות בחו"ל לא דלא שכיחי מתיבתא אבל איפכא עבדינן}
\textblock{א"ד אמר רב ירמיה בר אבא אמר רב אין בורגנין ופסי ביראות לא בבבל ולא בחו"ל בורגנין בבבל לא דשכיחי בידקי בחו"ל נמי לא דשכיחי גנבי}
\textblock{פסי ביראות בבבל לא דשכיחי מיא בחוץ לארץ נמי לא דלא שכיחי מתיבתא}
\textblock{א"ל רב חסדא למרי בריה דרב הונא בריה דרב ירמיה בר אבא אמרי אתיתו מברנש לבי כנישתא דדניאל דהוה תלתא פרסי בשבתא אמאי סמכיתו אבורגנין הא אמר אבוה דאבוה משמיה דרב אין בורגנין בבבל}
\textblock{נפק ואחוי ליה הנהו מתוותא דמבלען בשבעים אמה ושיריים:}
\textblock{אמר רב חסדא דריש מרי בר מר מאי דכתיב (תהלים קיט, צו) לכל תכלה ראיתי קץ רחבה מצותך מאד דבר זה אמרו דוד ולא פירשו אמרו איוב ולא פירשו אמרו יחזקאל ולא פירשו עד שבא זכריה בן עדו ופירשו}
\textblock{אמרו דוד ולא פירשו דכתיב לכל תכלה ראיתי קץ רחבה מצותך מאד אמרו איוב ולא פירשו דכתיב (איוב יא, ט) ארוכה מארץ מדה ורחבה מני ים}
\textblock{אמרו יחזקאל ולא פירשו דכתיב (יחזקאל ב, י) ויפרש אותה לפני והיא כתובה פנים ואחור וכתוב אליה קינים והגה והי}
\textblock{קינים זו פורענותן של צדיקים בעולם הזה וכן הוא אומר (יחזקאל לב, טז) קינה היא וקוננוה והגה זו מתן שכרן של צדיקים לעתיד לבא וכה"א (תהלים צב, ד) עלי הגיון בכנור והי זו היא פורענותן של רשעים לעתיד לבא וכן הוא אומר (יחזקאל ז, כו) הוה על הוה תבא}
\textblock{עד שבא זכריה בן עדו ופירשו דכתיב (זכריה ה, ב) ויאמר אלי מה אתה רואה ואומר אני רואה מגילה עפה ארכה עשרים באמה ורחבה עשר באמה וכי פשטת לה הויא לה עשרין בעשרין וכתיב היא כתובה פנים ואחור וכי קלפת לה כמה הויא לה ארבעין בעשרין}
\textblock{וכתיב (ישעיהו מ, יב) מי מדד בשעלו מים ושמים בזרת תכן וגו' נמצא כל העולם כולו אחד משלשת אלפים ומאתים בתורה}
\textblock{ואמר רב חסדא דריש מרי בר מר מאי דכתיב (ירמיהו כד, א) והנה שני דודאי תאנים מועדים לפני היכל ה' הדוד (ה) אחד תאנים טובות מאד כתאני}
\textblock{הבכורות והדוד (ה) אחד תאנים רעות מאד אשר לא תאכלנה מרוע}
\textblock{תאנים הטובות אלו צדיקים גמורים תאנים הרעות אלו רשעים גמורים ושמא תאמר אבד סברם ובטל סיכוים ת"ל הדודאים נתנו ריח אלו ואלו עתידין שיתנו ריח}
\textblock{דרש רבא מאי דכתיב (שיר השירים ז, יד) הדודאים נתנו ריח אלו בחורי ישראל שלא טעמו טעם חטא}
\textblock{ועל פתחינו כל מגדים אלו בנות ישראל שמגידות פתחיהן לבעליהן ל"א שאוגדות פתחיהן לבעליהן}
\textblock{חדשים גם ישנים דודי צפנתי לך אמרה כנסת ישראל לפני הקב"ה רבונו של עולם הרבה גזירות גזרתי על עצמי יותר ממה שגזרת עלי וקיימתים}
\textblock{א"ל רב חסדא לההוא מדרבנן דהוה קא מסדר אגדתא קמיה מי שמיע לך חדשים גם ישנים מהו אמר ליה אלו מצות קלות ואלו מצות חמורות}
\textblock{א"ל וכי תורה פעמים פעמים ניתנה אלא הללו מדברי תורה והללו מדברי סופרים}
\textblock{דרש רבא מאי דכתיב (קהלת יב, יב) ויותר מהמה בני הזהר עשות ספרים הרבה וגו' בני הזהר בדברי סופרים יותר מדברי תורה שדברי תורה יש בהן עשה ולא תעשה ודברי סופרים כל העובר על דברי סופרים חייב מיתה}
\textblock{שמא תאמר אם יש בהן ממש מפני מה לא נכתבו אמר קרא עשות ספרים הרבה אין קץ}
\textblock{(קהלת יב, יב) ולהג הרבה יגיעת בשר א"ר פפא בריה דרב אחא בר אדא משמיה דרב אחא בר עולא מלמד שכל המלעיג על דברי חכמים נידון בצואה רותחת}
\textblock{מתקיף לה רבא מי כתיב לעג להג כתיב אלא כל ההוגה בהן טועם טעם בשר}
\textblock{תנו רבנן מעשה בר"ע שהיה חבוש בבית האסורין והיה ר' יהושע הגרסי משרתו בכל יום ויום היו מכניסין לו מים במדה יום אחד מצאו שומר בית האסורין אמר לו היום מימך מרובין שמא לחתור בית האסורין אתה צריך שפך חציין ונתן לו חציין}
\textblock{כשבא אצל ר"ע אמר לו יהושע אין אתה יודע שזקן אני וחיי תלויין בחייך}
\textblock{סח לו כל אותו המאורע אמר לו תן לי מים שאטול ידי אמר לו לשתות אין מגיעין ליטול ידיך מגיעין אמר לו מה אעשה שחייבים עליהן מיתה מוטב אמות מיתת עצמי ולא אעבור על דעת חבירי}
\textblock{אמרו לא טעם כלום עד שהביא לו מים ונטל ידיו כששמעו חכמים בדבר אמרו מה בזקנותו כך בילדותו על אחת כמה וכמה ומה בבית האסורין כך שלא בבית האסורין על אחת כמה וכמה}
\textblock{אמר רב יהודה אמר שמואל בשעה שתיקן שלמה עירובין ונטילת ידים יצתה בת קול ואמרה (משלי כג, טו) בני אם חכם לבך ישמח לבי גם אני ואומר (משלי כז, יא) חכם בני ושמח לבי ואשיבה חרפי דבר}
\textblock{דרש רבא מאי דכתיב (שיר השירים ז, יב) לכה דודי נצא השדה נלינה בכפרים נשכימה לכרמים נראה אם פרחה הגפן פתח הסמדר הנצו הרמונים שם אתן את דודי לך}
\textblock{לכה דודי נצא השדה אמרה כנסת ישראל לפני הקב"ה רבש"ע אל תדינני כיושבי כרכים שיש בהן גזל ועריות ושבועת שוא ושבועת שקר נצא השדה בא ואראך תלמידי חכמים שעוסקין בתורה מתוך הדחק}
\textblock{נלינה בכפרים אל תקרי בכפרים אלא בכופרים בא ואראך אותם שהשפעת להן טובה והן כפרו בך}
\textblock{נשכימה לכרמים אלו בתי כנסיות ובתי מדרשות נראה אם פרחה הגפן אלו בעלי מקרא פתח הסמדר אלו בעלי משנה הנצו הרמונים אלו בעלי גמרא שם אתן את דודי לך אראך כבודי וגודלי שבח בני ובנותי}
\textblock{אמר רב המנונא מאי דכתיב (מלכים א ה, יב) וידבר שלשת אלפים משל ויהי שירו חמשה ואלף מלמד שאמר שלמה על כל דבר ודבר של תורה שלשת אלפים משל על כל דבר ודבר של סופרים חמשה ואלף טעמים}
\textblock{דרש רבא מאי דכתיב (קהלת יב, ט) ויותר שהיה קהלת חכם עוד לימד דעת את העם [ו] איזן וחקר תיקן משלים הרבה לימד דעת את העם דאגמריה בסימני טעמים ואסברה במאי דדמי ליה}
\textblock{[ו] איזן וחקר תיקן משלים הרבה אמר עולא אמר רבי אליעזר בתחילה היתה תורה דומה לכפיפה שאין לה אזנים עד שבא שלמה ועשה לה אזנים}
\textblock{קווצותיו תלתלים אמר רב חסדא אמר מר עוקבא מלמד שיש לדרוש על כל קוץ וקוץ תילי תילים של הלכות}
\textblock{שחורות כעורב במי אתה מוצאן במי}
\newsection{דף כב}
\textblock{שמשכים ומעריב עליהן לבית המדרש רבה אמר במי שמשחיר פניו עליהן כעורב}
\textblock{רבא אמר במי שמשים עצמו אכזרי על בניו ועל בני ביתו כעורב כי הא דרב אדא בר מתנא הוה קאזיל לבי רב אמרה ליה דביתהו ינוקי דידך מאי אעביד להו אמר לה מי שלימו קורמי באגמא}
\textblock{(דברים ז, י) ומשלם לשונאיו אל פניו להאבידו א"ר יהושע בן לוי אילמלא מקרא כתוב אי אפשר לאומרו כביכול כאדם שנושא משוי על פניו ומבקש להשליכו ממנו}
\textblock{לא יאחר לשונאו א"ר אילא לשונאיו הוא דלא יאחר אבל יאחר לצדיקים גמורים}
\textblock{והיינו דא"ר יהושע בן לוי מאי דכתיב (דברים ז, יא) אשר אנכי מצוך היום לעשותם היום לעשותם ולא למחר לעשותם היום לעשותם למחר לקבל שכרם}
\textblock{א"ר חגי ואיתימא ר' שמואל בר נחמני מאי דכתיב (שמות לד, ו) ארך אפים ארך אף מבעי ליה}
\textblock{אלא ארך אפים לצדיקים ארך אפים לרשעים:}
\textblock{ר' יהודה אומר עד בית סאתים וכו': איבעיא להו בור ופסין קאמר או דילמא בור ולא פסין קאמר}
\textblock{אדם נותן עיניו בבורו ולא גזרינן דילמא אתי לטלטולי יותר מבית סאתים בקרפף}
\textblock{או דילמא אדם נותן עיניו במחיצתו וגזרינן דילמא אתי לאיחלופי יותר מבית סאתים בקרפף}
\textblock{ת"ש כמה הן מקורבין כדי ראשה ורובה של פרה וכמה הן מרוחקין אפי' כור אפי' כוריים ר' יהודה אומר בית סאתים מותר יתר מבית סאתים אסור}
\textblock{אמרו לרבי יהודה אי אתה מודה בדיר וסהר מוקצה וחצר אפילו בית חמשת כורים ובית עשרת כורים שמותר}
\textblock{אמר להם זו מחיצה ואלו פסין}
\textblock{ר"ש בן אלעזר אומר בור בית סאתים אבית סאתים מותר ולא אמרו להרחיק אלא כדי ראשה ורובה של פרה}
\textblock{הא מדקאמר ר"ש בן אלעזר בור ולא פסין מכלל דרבי יהודה בור ופסין קאמר ולא היא ר' יהודה בור בלא פסין קאמר}
\textblock{אי הכי היינו דר' שמעון בן אלעזר איכא בינייהו אריך וקטין}
\textblock{כלל אמר רבי שמעון בן אלעזר כל אויר שתשמישו לדירה כגון דיר וסהר מוקצה וחצר אפילו בית חמשת כורים ובית עשרת כורים מותר}
\textblock{וכל דירה שתשמישה לאויר כגון בורגנין שבשדות בית סאתים מותר יתר מבית סאתים אסור:}
\textblock{{\large\emph{מתני׳}} ר' יהודה אומר אם היה דרך רשות הרבים מפסקתן יסלקנה לצדדין וחכמים אומרים אינו צריך:}
\textblock{{\large\emph{גמ׳}} רבי יוחנן ור' אלעזר דאמרי תרווייהו כאן הודיעך כוחן של מחיצות}
\textblock{כאן וסבירא ליה והאמר רבה בר בר חנה א"ר יוחנן ירושלים אילמלא דלתותיה ננעלות בלילה חייבין עליה משום רשות הרבים}
\textblock{אלא כאן ולא ס"ל}
\textblock{ורמי דר' יהודה אדר' יהודה ורמי דרבנן אדרבנן}
\textblock{דתניא יתר על כן א"ר יהודה מי שהיו לו שני בתים משני צידי רשות הרבים עושה לו לחי מכאן ולחי מכאן או קורה מכאן וקורה מכאן ונושא ונותן באמצע אמרו לו אין מערבין רשות הרבים בכך}
\textblock{קשיא דר' יהודה אדרבי יהודה קשיא דרבנן אדרבנן}
\textblock{דר' יהודה אדרבי יהודה לא קשיא התם דאיכא שתי מחיצות מעלייתא הכא ליכא שתי מחיצות מעלייתא}
\textblock{דרבנן אדרבנן [נמי] לא קשיא הכא איכא שם ארבע מחיצות התם ליכא שם ד' מחיצות}
\textblock{א"ר יצחק בר יוסף א"ר יוחנן ארץ ישראל אין חייבין עליה משום רה"ר יתיב רב דימי וקאמר ליה להא שמעתא א"ל אביי לרב דימי מ"ט}
\textblock{אילימא משום דמקיף לה סולמא דצור מהך גיסא ומחתנא דגדר מהך גיסא בבל נמי מקיף לה פרת מהך גיסא ודיגלת מהאי גיסא דכולא עלמא נמי מקיף אוקיינוס דילמא מעלות ומורדות קאמרת}
\textblock{א"ל קרקפנא חזיתיה לרישך בי עמודי כי א"ר יוחנן להא שמעתא}
\textblock{איתמר נמי כי אתא רבין א"ר יוחנן ואמרי לה א"ר אבהו א"ר יוחנן מעלות ומורדות שבארץ ישראל אין חייבין עליהן משום רה"ר לפי שאינן כדגלי מדבר}
\textblock{בעא מיניה רחבה מרבא תל המתלקט עשרה מתוך ארבע ורבים בוקעין בו חייבין עליו משום רה"ר או אין חייבין עליו}
\textblock{אליבא דרבנן לא תיבעי לך השתא ומה התם דניחא תשמישתיה אמרי רבנן לא אתו רבים ומבטלי לה מחיצתא הכא דלא ניחא תשמישתיה לא כל שכן}
\textblock{כי תיבעי לך אליבא דר' יהודה מאי התם הוא דניחא תשמישתיה הכא הוא דלא ניחא תשמישתיה לא אתו רבים ומבטלי מחיצתא או דילמא לא שנא א"ל חייבין}
\textblock{ואפי' עולין לו בחבל א"ל אין ואפילו במעלות בית מרון א"ל אין}
\textblock{איתיביה חצר שהרבים נכנסין לה בזו ויוצאין בזו רה"ר לטומאה ורשות היחיד לשבת}
\textblock{מני אילימא רבנן השתא ומה התם דניחא תשמישתיה אמרי רבנן לא אתו רבים ומבטלי מחיצתא הכא דלא ניחא תשמישתיה לא כ"ש}
\textblock{אלא לאו ר' יהודה היא}
\textblock{לא לעולם רבנן ורה"ר לטומאה איצטריכא ליה}
\textblock{ת"ש מבואות המפולשות בבורות בשיחין ובמערות רשות היחיד לשבת ורשות הרבים לטומאה}
\textblock{בבורות סלקא דעתך אלא לבורות רשות היחיד לשבת ורשות הרבים לטומאה}
\textblock{מני אילימא רבנן השתא ומה התם דניחא תשמישתיה אמרי לא אתו רבים ומבטלי לה הכא דלא ניחא תשמישתיה לא כ"ש אלא לאו ר' יהודה היא}
\textblock{לא לעולם רבנן ורה"ר לטומאה איצטריכא ליה}
\textblock{ת"ש שבילי בית גילגול וכיוצא בהן רשות היחיד לשבת ורה"ר לטומאה}
\textblock{ואיזהו שבילי בית גילגול אמרי דבי ר' ינאי כל שאין העבד יכול ליטול סאה של חיטין וירוץ לפני סרדיוט}
\textblock{מני אילימא רבנן השתא ומה התם דניחא תשמישתא אמרי רבנן לא אתו רבים ומבטלי לה מחיצתא הכא דלא ניחא תשמישתא לא כל שכן אלא לאו רבי יהודה היא}
\textblock{א"ל שבילי בית גילגול קאמרת יהושע אוהב ישראל היה עמד ותיקן להם דרכים וסרטיא כל היכא דניחא תשמישתא מסרה לרבים כל היכא דלא ניחא תשמישתא מסרה ליחיד:}
\textblock{{\large\emph{מתני׳}} אחד בור הרבים ובאר הרבים ובאר היחיד עושין להן פסין אבל}
\textblock{לבור היחיד עושין לו מחיצה גבוה י' טפחים דברי ר' עקיבא}
\textblock{ר' יהודה בן בבא אומר אין עושין פסין אלא לבאר הרבים בלבד ולשאר עושין חגורה גבוה עשרה טפחים:}
\newsection{דף כג}
\textblock{{\large\emph{גמ׳}} אמר רב יוסף אמר רב יהודה אמר שמואל הלכה כר' יהודה בן בבא ואמר רב יוסף אמר רב יהודה אמר שמואל לא הותרו פסי ביראות אלא לבאר מים חיים בלבד}
\textblock{וצריכא דאי אשמעינן הלכה כר' יהודה בן בבא הוה אמינא דרבים ואפילו מכונסין}
\textblock{והאי דקתני באר הרבים לאפוקי מדר' עקיבא קא משמע לן דלא הותרו פסי ביראות אלא לבאר מים חיים}
\textblock{ואי אשמעינן באר מים חיים הוה אמינא לא שנא דרבים ולא שנא דיחיד קא משמע לן הלכה כרבי יהודה בן בבא:}
\textblock{{\large\emph{מתני׳}} ועוד אמר ר' יהודה בן בבא הגינה והקרפף שהן שבעים אמה ושיריים על ע' אמה ושיריים המוקפות גדר גבוה עשרה טפחים מטלטלין בתוכה ובלבד שיהא בה שומירה או בית דירה או שתהא סמוכה לעיר}
\textblock{ר' יהודה אומר אפילו אין בה אלא בור ושיח ומערה מטלטלין בתוכה רבי עקיבא אומר אפילו אין בה אחת מכל אלו מטלטלין בתוכה ובלבד שיהא בה שבעים אמה ושיריים על שבעים אמה ושיריים}
\textblock{רבי אליעזר אומר אם היתה ארכה יתר על רחבה אפי' אמה אחת אין מטלטלין בתוכה רבי יוסי אומר אפילו ארכה פי שנים ברחבה מטלטלין בתוכה}
\textblock{אמר רבי אלעאי שמעתי מרבי אלעזר ואפילו היא כבית כור}
\textblock{וכן שמעתי ממנו אנשי חצר ששכח אחד מהן ולא עירב ביתו אסור מלהכניס ולהוציא לו אבל להם מותר}
\textblock{וכן שמעתי ממנו שיוצאין בערקבלין בפסח וחזרתי על כל תלמידיו ובקשתי לי חבר ולא מצאתי:}
\textblock{{\large\emph{גמ׳}} מאי תנא דקתני ועוד}
\textblock{אילימא משום דתנא ליה חדא לחומרא וקתני אחריתי משום הכי קתני ועוד והא רבי יהודה דתנא ליה חדא לחומרא וקתני אחריתי ולא קתני ועוד}
\textblock{התם אפסקוה רבנן הכא לא אפסקוה רבנן}
\textblock{וכל היכא דאפסקוה רבנן לא קתני ועוד והא רבי אליעזר דסוכה דאפסקוה רבנן וקתני ועוד}
\textblock{התם במילתיה הוא דאפסקוה הכא במילתא אחריתי אפסקוה:}
\textblock{רבי עקיבא אומר אפילו אין בה אחד מכל אלו מטלטלין בתוכה:}
\textblock{רבי עקיבא היינו תנא קמא}
\textblock{איכא בינייהו דבר מועט דתניא ר' יהודה אומר דבר מועט יש על שבעים אמה ושירים ולא נתנו חכמים בו שיעור}
\textblock{וכמה שיעור סאתים כחצר המשכן}
\textblock{מנא הני מילי}
\textblock{אמר רב יהודה דאמר קרא (שמות כז, יח) ארך החצר מאה באמה ורחב חמשים בחמשים אמרה תורה טול חמשים וסבב חמשים}
\textblock{פשטיה דקרא במאי כתיב אמר אביי העמד משכן על שפת חמשים כדי שיהא חמשים אמה לפניו ועשרים אמה לכל רוח ורוח:}
\textblock{ר"א אומר אם היתה ארכה כו': והתניא ר"א אומר אם היתה ארכה יתר על פי שנים ברחבה אפי' אמה אחת אין מטלטלין בתוכה}
\textblock{אמר רב ביבי בר אביי כי תנן נמי מתני' [יתר על] פי שנים ברחבה תנן אי הכי היינו ר' יוסי}
\textblock{איכא בינייהו ריבועא דריבעוה רבנן:}
\textblock{ר' יוסי אומר כו': איתמר אמר רב יוסף אמר רב יהודה אמר שמואל הלכה כר' יוסי ורב ביבי אמר רב יהודה אמר שמואל הלכה כר' עקיבא}
\textblock{ותרוייהו לקולא וצריכא דאי אשמעינן הלכה כרבי יוסי הוה אמינא עד דאיכא שומירה או בית דירה קמ"ל הלכה כר' עקיבא}
\textblock{ואי אשמעינן הלכה כר"ע הוה אמינא דאריך וקטין לא קמ"ל הלכה כר' יוסי}
\textblock{קרפף שהוא יותר מבית סאתים שהוקף לדירה נזרע רובו הרי הוא כגינה ואסור}
\textblock{נטע רובו הרי הוא כחצר ומותר}
\textblock{נזרע רובו אמר רב הונא בריה דרב יהושע לא אמרן אלא יותר מבית סאתים אבל בית סאתים מותר}
\textblock{כמאן כרבי שמעון דתנן ר' שמעון אומר אחד גגות ואחד חצירות ואחד קרפיפות רשות אחת הן לכלים ששבתו בתוכן ולא לכלים ששבתו בתוך הבית}
\textblock{לר"ש נמי כיון דנזרע רובו הוי ההוא מעוטא}
\newsection{דף כד}
\textblock{בטיל ליה לגבי רובה והוה ליה קרפף יותר מבית סאתים ואסור}
\textblock{אלא אי איתמר הכי איתמר הא מיעוטא שרי אמר רב הונא בריה דרב יהושע לא אמרן אלא דלא הוי בית סאתים אבל בית סאתים אסור}
\textblock{כמאן כרבנן}
\textblock{ורב ירמיה מדיפתי מתני לקולא הא מיעוטא שרי אמר רב הונא בריה דרב יהושע לא אמרן אלא בית סאתים אבל יותר מבית סאתים אסור כמאן כר"ש:}
\textblock{נטע רובו הרי הוא כחצר ומותר: אמר רב יהודה אמר אבימי והוא שעשויין אצטבלאות ורב נחמן אמר אע"פ שאין עשויין אצטבלאות}
\textblock{מר יהודה אקלע לבי רב הונא בר יהודה חזנהו להנהו דלא עבידי אצטבלאות וקא מטלטלי בגוייהו אמר ליה לא סבר לה מר להא דאבימי אמר ליה אנא כרב נחמן סבירא לי}
\textblock{אמר רב נחמן אמר שמואל קרפף יותר מבית סאתים שלא הוקף לדירה כיצד הוא עושה פורץ בו פירצה יותר מעשר וגודרו ומעמידו על עשר ומותר}
\textblock{איבעיא להו פרץ אמה וגדר אמה [ופרץ אמה וגדרה] עד שהשלימו ליותר מעשר מהו }
\textblock{א"ל לאו היינו דתנן כל כלי בעלי בתים שיעורן כרמונים}
\textblock{ובעי חזקיה ניקב כמוציא זית וסתמו וחזר וניקב כמוציא זית וסתמו עד שהשלימו למוציא רמון מהו}
\textblock{וא"ל רבי יוחנן רבי שניתה לנו סנדל שנפסקה אחת מאזניו ותיקנה טמא מדרס}
\textblock{נפסקה שניה ותיקנה טהור (בה) מן המדרס אבל טמא מגע מדרס}
\textblock{ואמרת עלה מאי שנא ראשונה דהא קיימא שניה שניה נמי הא קיימא ראשונה}
\textblock{ואמרת לן עלה פנים חדשות באו לכאן הכא נמי פנים חדשות באו לכאן}
\textblock{קרי עליה לית דין בר אינש איכא דאמרי כגון דין בר נש}
\textblock{אמר רב כהנא רחבה שאחורי הבתים אין מטלטלין בו אלא בד' אמות}
\textblock{ואמר רב נחמן אם פתח לו פתח מותר לטלטל בכולו פתח מתירו ולא אמרן אלא שפתח ולבסוף הוקף אבל הוקף ולבסוף פתח לא}
\textblock{פתח ולבסוף הוקף פשיטא לא צריכא דאית ביה בי דרי מהו דתימא אדעתא דבי דרי עבדיה קמ"ל}
\textblock{קרפף יותר מבית סאתים שהוקף לדירה ונתמלא מים סבור רבנן למימר כזרעים דמו ואסיר}
\textblock{אמר להו רב אבא (אבוה) דרב בריה דרב משרשיא הכי אמרינן משמיה דרבא מים כנטעים דמו ושרי}
\textblock{אמר אמימר והוא דחזיין לתשמישתא אבל לא חזיין לתשמישתא לא}
\textblock{אמר רב אשי ודחזיין לתשמישתא נמי לא אמרן אלא שאין בעומקו יותר מבית סאתים אבל אם יש בעומקו יותר מבית סאתים אסור}
\textblock{ולאו מילתא היא מידי דהוה אכריא דפירי}
\textblock{ההיא רחבה דהואי בפום נהרא דחד גיסא הוה פתיח למתא וחד גיסא הוה פתיח לשביל של כרמים ושביל של כרמים הוה סליק לגודא דנהרא}
\textblock{אמר אביי היכי נעביד לעביד ליה מחיצה אגודא דנהרא אין עושין מחיצה על גבי מחיצה}
\textblock{ולעביד ליה צורת הפתח אפומא דשביל של כרמים אתו גמלי שדיין ליה}
\textblock{אלא אמר אביי ליעביד לחי אפיתחא דשביל של כרמים דמגו דמהניא לשביל של כרמים מהני נמי לרחבה}
\textblock{אמר ליה רבא יאמרו לחי מועיל לשביל של כרמים דעלמא}
\textblock{אלא אמר רבא עבדינן ליה לחי לפיתחא דמתא דמגו דמהני ליה לחי למתא מהני נמי לרחבה}
\textblock{הלכך טלטולי במתא גופה שרי טלטולי ברחבה גופה שרי ממתא לרחבה ומרחבה למתא פליגי בה רב אחא ורבינא חד אסר וחד שרי}
\newsection{דף כה}
\textblock{מאן דשרי דהא ליכא דיורין ומאן דאסר זימנין דהוי בה דיורין ואתי לטלטולי}
\textblock{קרפף יותר מבית סאתים שלא הוקף לדירה ובא למעטו מיעטו באילנות לא הוי מיעוט}
\textblock{בנה בו עמוד גבוה עשרה ורחב ד' הוי מיעוט פחות מג' לא הוי מיעוט מג' ועד ד' רבה אמר הוי מיעוט ורבא אמר לא הוי מיעוט}
\textblock{רבה אמר הוי מיעוט דהא נפיק ליה מתורת לבוד רבא אמר לא הוי מיעוט כיון דלא הוי מקום ד' לא חשיב}
\textblock{הרחיק מן הכותל ד' ועשה מחיצה הועיל פחות מג' לא הועיל מג' ועד ד' רבה אמר הועיל רבא אמר אינו מועיל}
\textblock{רבה אמר הועיל דהא נפיק ליה מתורת לבוד רבא אמר אינו מועיל כיון דלא הוי מקום ד' לא חשיב}
\textblock{רב שימי מתני לקולא טח בו טיט ויכול לעמוד בפני עצמו הוי מיעוט אינו יכול לעמוד בפני עצמו רבה אמר הוי מיעוט רבא אמר לא הוי מיעוט}
\textblock{רבה אמר הוה מיעוט השתא מיהא קאי רבא אמר לא הוי מיעוט כיון דלא יכול למיקם בפני עצמו לא כלום הוא}
\textblock{הרחיק מן התל ד' ועשה מחיצה הועיל}
\textblock{פחות מג' או על שפת התל רב חסדא ורב המנונא חד אמר הועיל וחד אמר לא הועיל}
\textblock{תסתיים דרב חסדא אמר הועיל דאתמר העושה מחיצה על גבי מחיצה אמר רב חסדא בשבת הועיל}
\textblock{בנכסי הגר לא קנה}
\textblock{ורב ששת אמר אף בשבת נמי לא הועיל תסתיים}
\textblock{אמר רב חסדא ומודה לי רב ששת שאם עשה מחיצה על התל שהועיל}
\textblock{מאי טעמא הואיל ובאויר מחיצות העליונות הוא דר}
\textblock{בעי רבה בר בר חנה נבלעו מחיצות התחתונות והעליונות קיימות מהו}
\textblock{למאי אי לנכסי הגר היינו דירמיה ביראה דאמר ירמיה ביראה אמר רב יהודה האי מאן דשדא ליפתא אפילא דארעא דגר ואתא ישראל אחרינא רפק בה פורתא בתרא קני קמא לא קני}
\textblock{מ"ט בעידנא דשדא לא קא שבח כי קא שבחא ממילא קא משבחא}
\textblock{ואלא לענין שבת הוי מחיצה הנעשה בשבת}
\textblock{ותניא כל מחיצה הנעשה בשבת בין בשוגג בין במזיד שמה מחיצה}
\textblock{לאו איתמר עלה אמר רב נחמן לא שנו אלא לזרוק אבל לטלטל אסור}
\textblock{כי איתמר דרב נחמן אמזיד איתמר}
\textblock{ההיא איתתא דעבדה מחיצה על גבי מחיצה בנכסי הגר אתא ההוא גברא רפק בה פורתא אתא לקמיה דרב נחמן אוקמה בידיה אתת איהי וקא צווחא קמיה אמר לה מאי איעביד לך דלא מחזקת כדמחזקי אינשי}
\textblock{קרפף בית שלש וקירה בו בית סאה רבא אמר אויר קירויו מייתרו ורבי זירא אמר אין אויר קירויו מייתרו}
\textblock{לימא רבא ורבי זירא בפלוגתא דרב ושמואל קא מיפלגי דאיתמר אכסדרה בבקעה רב אמר מותר לטלטל בכולה ושמואל אמר אין מטלטלין אלא בד' אמות}
\textblock{רב אמר מותר לטלטל בכולה אמרינן פי תקרה יורד וסותם ושמואל אמר אין מטלטלין אלא בארבע אמות לא אמרי' פי תקרה יורד וסותם}
\textblock{אי דעבידא כי אכסדרה הכי נמי הב"ע דעבדה כי אורזילא}
\textblock{אמר רבי זירא ומודינא בקרפף שנפרץ במלואו לחצר שאסור מ"ט הואיל ואויר חצר מייתרו}
\textblock{מתקיף לה רב יוסף וכי אויר המותר לו אוסרו}
\textblock{א"ל אביי כמאן כרבי שמעון לרבי שמעון נמי הא איכא אויר מקום מחיצות}
\textblock{דאמר רב חסדא קרפף שנפרץ במלואו לחצר חצר מותרת וקרפף אסור}
\textblock{חצר מאי טעמא דאית ליה גיפופי והא זמנין דמשכחת לה איפכא}
\textblock{אלא משום דאמרינן זה אויר מחיצות מייתרו וזה אין אויר מחיצות מייתרו}
\textblock{ההוא בוסתנא דהוה סמיך לגודא דאפדנא נפל אשיתא ברייתא דאפדנא סבר רב ביבי למימר ליסמוך אגודא גוויאתא}
\textblock{אמר ליה רב פפי משום דאתו ממולאי אמריתו מילי מולייאתא הנך מחיצות לגואי עבידן לבראי לא עבידן}
\textblock{ההיא אבוורנקא דהוה ליה לריש גלותא בבוסתניה א"ל לרב הונא בר חיננא ליעביד מר תקנתא דלמחר נאכול נהמא התם}
\textblock{אזל עבד קנה קנה פחות משלשה אזל רבא}
\newsection{דף כו}
\textblock{שלפינהו אזל רב פפא ורב הונא בריה דרב יהושע נקטינהו מבתריה}
\textblock{למחר איתיביה רבינא לרבא עיר חדשה מודדין לה מישיבתה וישנה מחומתה}
\textblock{איזו היא חדשה ואיזו היא ישנה חדשה שהוקפה ולבסוף ישבה ישנה ישבה ולבסוף הוקפה והאי נמי כהוקפה ולבסוף ישבה דמי}
\textblock{א"ל רב פפא לרבא והאמר רב אסי מחיצות אדרכלין לא שמה מחיצה אלמא כיון דלצניעותא עבידא לה לא הויא מחיצה ה"נ כיון דלצניעותא עבידא לא הויא מחיצה}
\textblock{ואמר רב הונא בריה דרב יהושע לרבא והאמר רב הונא מחיצה העשויה לנחת לא שמה מחיצה}
\textblock{דהא רבה בר אבוה מערב לה לכולה מחוזא ערסייתא ערסייתא משום פירא דבי תורי והא פירא דבי תורי כמחיצה העשויה לנחת דמיא}
\textblock{קרי עלייהו ריש גלותא (ירמיהו ד, כב) חכמים המה להרע ולהיטיב לא ידעו:}
\textblock{א"ר אלעאי שמעתי מר"א ואפי' בית כור: מתניתין דלא כחנניה דתניא חנניה אומר ואפילו היא ארבעים סאה כאסטרטיא של מלך}
\textblock{א"ר יוחנן ושניהם מקרא אחד דרשו שנאמר (מלכים ב כ, ד) ויהי ישעיהו לא יצא (אל) חצר התיכונה כתיב העיר וקרינן חצר מכאן לאיסטרטיא של מלך שהיו כעיירות בינוניות}
\textblock{במאי קמיפלגי מר סבר עיירות בינוניות הויין בית כור ומ"ס מ' סאה הויין}
\textblock{וישעיהו מאי בעי התם אמר רבה בר בר חנה אמר ר' יוחנן מלמד שחלה חזקיה והלך ישעיהו והושיב ישיבה על פתחו}
\textblock{מכאן לת"ח שחלה שמושיבין ישיבה על פתחו ולאו מילתא היא דילמא אתי לאיגרויי ביה שטן:}
\textblock{וכן שמעתי הימנו אנשי חצר ששכח אחד ולא עירב ביתו אסור:}
\textblock{והתנן ביתו אסור להוציא ולהכניס לו ולהן}
\textblock{אמר רב הונא בריה דרב יהושע אמר רב ששת לא קשיא}
\textblock{הא ר"א והא רבנן}
\textblock{כשתימצי לומר לדברי ר' אליעזר המבטל רשות חצירו רשות ביתו ביטל לרבנן המבטל רשות חצירו רשות ביתו לא ביטל}
\textblock{פשיטא}
\textblock{אמר רחבה אנא ורב הונא בר חיננא תרגימנא לא נצרכא אלא לחמשה ששרוין בחצר אחד ושכח אחד מהן ולא עירב}
\textblock{לדברי ר"א כשהוא מבטל רשותו אין צריך לבטל לכל אחד וא'}
\textblock{לרבנן כשהוא מבטל רשותו צריך לבטל לכל אחד ואחד}
\textblock{כמאן אזלא הא דתניא חמשה ששרוין בחצר אחד ושכח אחד מהן ולא עירב כשהוא מבטל רשותו אין צריך לבטל רשות לכל אחד ואחד כמאן כר"א}
\textblock{רב כהנא מתני הכי רב טביומי מתני הכי כמאן אזלא הא דתניא ה' ששרוים בחצר אחד ושכח אחד מהן ולא עירב כשהוא מבטל רשותו אינו צריך לבטל רשות לכל אחד ואחד כמאן אמר רב הונא בר יהודה אמר רב ששת כמאן כר"א}
\textblock{א"ל רב פפא לאביי לר"א אי אמר לא מבטילנא ולרבנן אי אמר מבטילנא מאי}
\textblock{טעמא דר"א משום דקסבר המבטל רשות חצירו רשות ביתו ביטל והאי אמר אנא לא מבטילנא}
\textblock{או דילמא טעמא דר"א משום דבית בלא חצר לא עבידי אינשי דדיירי וכי קאמר לא מבטילנא לאו כל כמיניה אע"ג דאמר דיירנא לאו כלום קאמר}
\textblock{ולרבנן אי אמר מבטילנא מאי טעמא דרבנן משום דקסברי המבטל רשות חצירו רשות ביתו לא ביטל והאי אמר מבטילנא}
\textblock{או דלמא טעמא דרבנן משום דלא עביד איניש דמסלק נפשיה לגמרי מבית וחצר והוי כי אורח לגבייהו והאי כי אמר מבטילנא לאו כל כמיניה (קאמר)}
\textblock{א"ל בין לרבנן בין לר"א כיון דגלי דעתיה גלי:}
\textblock{וכן שמעתי ממנו שיוצאים בערקבלין בפסח: מאי ערקבלין אמר ריש לקיש אצוותא חרוזיאתא:}
\textblock{\par \par {\large\emph{הדרן עלך עושין פסין}}\par \par }
\textblock{}
\textblock{מתני׳ {\large\emph{בכל}} מערבין ומשתתפין חוץ מן המים ומן המלח}
\textblock{והכל ניקח בכסף מעשר חוץ מן המים ומן המלח הנודר מן המזון מותר (במלח ובמים)}
\textblock{מערבין לנזיר ביין ולישראל בתרומה סומכוס אומר בחולין}
\textblock{}
\textblock{ולכהן בבית הפרס ר' יהודה אומר אפי' בין הקברות}
\newchap{פרק \hebrewnumeral{3}\quad בכל מערבין}
\newsection{דף כז}
\textblock{}
\textblock{מפני שיכול לחוץ ולילך ולאכול:}
\textblock{{\large\emph{גמ׳}} א"ר יוחנן אין למידין מן הכללות ואפילו במקום שנאמר בו חוץ}
\textblock{מדקאמר אפי' במקום שנאמר בו חוץ מכלל דלאו הכא קאי היכא קאי}
\textblock{התם קאי כל מצות עשה שהזמן גרמא אנשים חייבין ונשים פטורות ושלא הזמן גרמא אחד נשים ואחד אנשים חייבין}
\textblock{וכללא הוא דכל מצות עשה שהזמן גרמא נשים פטורות הרי מצה שמחה והקהל דמצות עשה שהזמן גרמא הוא ונשים חייבות}
\textblock{וכל מצות עשה שלא הזמן גרמא נשים חייבות הרי תלמוד תורה פריה ורביה ופדיון הבן דמצות עשה שלא הזמן גרמא ונשים פטורות אלא אמר רבי יוחנן אין למידין מן הכללות ואפילו במקום שנאמר בו חוץ}
\textblock{אמר אביי ואיתימא רבי ירמיה אף אנן נמי תנינא עוד כלל אחר אמרו כל שנישא על גבי הזב טמא וכל שהזב נישא עליו טהור חוץ מן הראוי למשכב ומושב והאדם ותו ליכא והא איכא מרכב}
\textblock{האי מרכב היכי דמי אי דיתיב עליה היינו מושב אנן הכי קאמרינן הא איכא גבא דאוכפא דתניא האוכף טמא מושב והתפוס טמא מרכב אלא שמע מינה אין למידין מן הכללות ואפילו במקום שנאמר בו חוץ}
\textblock{אמר רבינא ואיתימא רב נחמן אף אנן נמי תנינא בכל מערבין ומשתתפין חוץ מן המים והמלח ותו ליכא והא איכא כמיהין ופטריות אלא שמע מינה אין למידין מן הכללות ואפי' במקום שנאמר בו חוץ:}
\textblock{הכל ניקח בכסף מעשר כו': ר' אליעזר ור' יוסי בר חנינא חד מתני אעירוב וחד מתני אמעשר}
\textblock{חד מתני אעירוב ל"ש אלא מים בפני עצמו ומלח בפני עצמו דאין מערבין אבל במים ומלח מערבין}
\textblock{וחד מתני אמעשר לא שנו אלא מים בפני עצמו ומלח בפני עצמו דאין ניקחין אבל מים ומלח ניקחין בכסף מעשר}
\textblock{מאן דמתני אמעשר כ"ש אעירוב ומאן דמתני אעירוב אבל אמעשר לא מ"ט פירא בעינן}
\textblock{כי אתא רבי יצחק מתני אמעשר מיתיבי העיד ר' יהודה בן גדיש לפני ר"א של בית אבא היו לוקחין ציר בכסף מעשר אמר לו שמא לא שמעת אלא כשקרבי דגים מעורבין בהן ואפילו רבי יהודה בן גדיש לא קאמר אלא בציר דשומנא דפירא היא אבל מים ומלח לא}
\textblock{אמר רב יוסף}
\textblock{לא נצרכה אלא שנתן לתוכן שמן}
\textblock{אמר ליה אביי ותיפוק ליה משום שמן לא צריכא שנתן דמי מים ומלח בהבלעה}
\textblock{ובהבלעה מי שרי אין והתניא בן בג בג אומר (דברים יד, כו) בבקר מלמד שלוקחין בקר על גב עורו ובצאן מלמד שלוקחין צאן על גב גיזתה וביין מלמד שלוקחין יין על גב קנקנו ובשכר מלמד שלוקחין תמד משהחמיץ}
\textblock{א"ר יוחנן מאן דמתרגם לי בבקר אליבא דבן בג בג מובילנא מאניה אבתריה לבי מסותא}
\textblock{מאי טעמא כולהו צריכי לבר מבבקר דלא צריך מאי צריכי דאי כתב רחמנא בבקר הוה אמינא בקר הוא דמזדבן על גב עורו משום דגופיה הוא אבל צאן על גב גיזתה דלאו גופיה הוא אימא לא}
\textblock{ואי כתב רחמנא בצאן על גב גיזתה הוה אמינא משום דמחובר בה אבל יין ע"ג קנקנו אימא לא}
\textblock{ואי כתב רחמנא ביין הוה אמינא משום דהיינו נטירותיה אבל תמד משהחמיץ דקיוהא בעלמא הוא אימא לא כתב רחמנא שכר}
\textblock{ואי כתב רחמנא בשכר הוה אמינא מאי שכר דבילה קעילית דפירא הוא אבל יין על גב קנקנו אימא לא}
\textblock{ואי כתב רחמנא יין על גב קנקנו דהיינו נטירותיה אבל צאן על גב גיזתה אימא לא כתב רחמנא צאן דאפילו על גב גיזתה}
\textblock{בבקר למה לי וכ"ת אי לא כתב רחמנא בבקר הוה אמינא צאן על גב עורה אין על גב גיזתה לא כתב רחמנא בבקר לאתויי עורו אייתר ליה צאן לאתויי גיזתה}
\textblock{אי לא כתב רחמנא בקר לא הוה אמינא צאן על גב עורה אין על גב גיזתה לא דאם כן לכתוב רחמנא בקר דממילא אייתר ליה צאן}
\textblock{וכיון דכתב רחמנא צאן דאפילו על גב גיזתה בבקר למה לי השתא צאן על גב גיזתה מיזדבנא בקר על גב עורו מיבעיא היינו דקאמר רבי יוחנן מאן דמתרגם לי בבקר אליבא דבן בג בג מובילנא מאניה לבי מסותא}
\textblock{במאי קא מיפלגי רבי יהודה בן גדיש ור"א והני תנאי דלקמן ר' יהודה בן גדיש ור"א דרשי רבויי ומיעוטי והני תנאי דרשי כללי ופרטי}
\textblock{ר' יהודה בן גדיש ור"א דרשי ריבויי ומיעוטי (דברים יד, כו) ונתתה הכסף בכל אשר תאוה נפשך ריבה בבקר ובצאן וביין ובשכר מיעט ובכל אשר תשאלך נפשך חזר וריבה ריבה ומיעט וריבה ריבה הכל מאי רבי רבי כל מילי ומאי מיעט לר"א מיעט ציר לר' יהודה בן גדיש מיעט מים ומלח}
\textblock{והני תנאי דרשי כללי ופרטי דתניא ונתתה הכסף בכל אשר תאוה נפשך כלל בבקר ובצאן וביין ובשכר פרט ובכל אשר תשאלך נפשך חזר וכלל כלל ופרט וכלל אי אתה דן אלא כעין הפרט מה הפרט מפורש פרי מפרי וגידולי קרקע אף כל פרי מפרי וגידולי קרקע}
\textblock{ותניא אידך מה הפרט מפורש ולד ולדות הארץ אף כל ולד ולדות הארץ}
\textblock{מאי בינייהו אמר אביי דגים איכא בינייהו למאן דאמר פרי מפרי וגידולי קרקע הני דגים גידולי קרקע נינהו למאן דאמר ולד ולדות הארץ דגים ממיא איברו}
\textblock{ומי אמר אביי דגים גידולי קרקע נינהו והאמר אביי}
\newsection{דף כח}
\textblock{אכל פוטיתא לוקה ארבע נמלה לוקה חמש צירעה לוקה שש ואם איתא פוטיתא נמי לילקי משום השרץ השורץ על הארץ}
\textblock{אלא אמר רבינא עופות איכא בינייהו למאן דאמר פרי מפרי וגידולי קרקע הני נמי גידולי קרקע נינהו למאן דאמר ולד ולדות הארץ הני עופות מן הרקק נבראו}
\textblock{מאן דמרבי עופות מאי טעמיה ומאן דממעיט עופות מאי טעמיה}
\textblock{מאן דמרבי עופות קסבר כללא בתרא דוקא פרט וכלל נעשה כלל מוסף על הפרט ואיתרבו להו כל מילי ואהני כללא קמא למעוטי כל דלא דמי ליה משני צדדין}
\textblock{ומאן דממעט עופות קסבר כללא קמא דווקא כלל ופרט ואין בכלל אלא מה שבפרט הני אין מידי אחרינא לא ואהני כללא בתרא לרבויי כל דדמי ליה משלשה צדדין:}
\textblock{א"ר יהודה משמיה דרב שמואל בר שילת משמיה דרב מערבין בפעפועין ובחלגלוגות ובגודגדניות אבל לא בחזיז ולא בכפניות}
\textblock{ובגודגדניות מי מערבין והתניא גודגדניות מרובי בנים יאכלו חשוכי בנים לא יאכלו ואם הוקשו לזרע אף מרובי בנים לא יאכלו}
\textblock{תרגמא אשלא הוקשו לזרע ומרובי בנים}
\textblock{ואיבעית אימא לעולם לחשוכי בנים דהא חזו למרובי בנים מי לא תנן מערבין לנזיר ביין ולישראל בתרומה אלמא אע"ג דלא חזי להאי חזי להאי הכא נמי אע"ג דלא חזי להאי חזי להאי}
\textblock{ואיבעית אימא כי קאמר רב בהנדקוקי מדאי}
\textblock{ובחזיז לא והאמר רב יהודה א"ר כשות וחזיז מערבין בהן ומברכין עליהן בורא פרי האדמה}
\textblock{לא קשיא הא מקמי דאתא רב לבבל הא לבתר דאתא רב לבבל}
\textblock{ובבל הויא רובא דעלמא והתניא הפול והשעורה והתילתן שזרען לירק בטלה דעתו אצל כל אדם לפיכך זרען חייב וירקן פטור השחליים והגרגיר שזרען לירק מתעשרין ירק וזרע זרען לזרע מתעשרין זרע וירק}
\textblock{כי קאמר רב}
\textblock{בדגנונייתא}
\textblock{זרע גרגיר למאי חזי א"ר יוחנן שכן ראשונים שלא היה להן פלפלין שוחקין אותו ומטבילין בו את הצלי}
\textblock{ר' זירא כי הוה חליש מגרסיה הוה אזיל ויתיב אפיתחא דרב יהודה בר אמי אמר כי נפקי ועיילי רבנן איקום מקמייהו ואקבל בהו אגרא}
\textblock{נפק אתא ינוקא דבי רב א"ל מאי אגמרך רבך א"ל כשות בורא פרי האדמה חזיז שהכל נהיה בדברו אמר ליה אדרבה איפכא מיסתברא האי מארעא קא מרבי והאי מאוירא קא מרבי}
\textblock{והלכתא כינוקא דבי רב מ"ט האי גמר פירי והאי לאו גמר פירי ומאי דקאמרת האי מארעא קא רבי והאי מאוירא קא רבי לא היא כשות נמי מארעא קא רבי דהא קא חזינן דקטלינן לה להיזמתא ומייתא כשותא}
\textblock{ובכפניות אין מערבין והתניא קור ניקח בכסף מעשר ואין מטמא טומאת אוכלין וכפניות נקחות בכסף מעשר ומטמאות טומאת אוכלים}
\textblock{רבי יהודה אומר קור הרי הוא כעץ לכל דבריו אלא שניקח בכסף מעשר וכפניות הרי הן כפרי לכל דבריהם אלא שפטורות מן המעשר}
\textblock{התם בדניסחני}
\textblock{אי הכי בהא לימא רבי יהודה פטורות מן המעשר והתניא אמר רבי יהודה לא הוזכרו פגי ביתיוני אלא לענין מעשר בלבד פגי ביתיוני ואהיני דטובינא חייבין במעשר}
\textblock{אלא לעולם לאו בניסחני ולענין טומאת אוכלין שאני כדאמר רבי יוחנן הואיל וראוי למתקן ע"י האור הכא נמי הואיל ויכול למתקן ע"י האור}
\textblock{והיכא אתמר דרבי יוחנן אהא דתניא שקדים המרים קטנים חייבין גדולים פטורין מתוקים גדולים חייבין קטנים פטורין רבי שמעון ברבי יוסי אומר משום אביו זה וזה לפטור ואמרי לה זה וזה לחיוב א"ר אילעא הורה רבי חנינא בציפורי כדברי האומר זה וזה לפטור}
\textblock{ולמאן דאמר זה וזה לחיוב למאי חזי א"ר יוחנן הואיל וראוי למתקן ע"י האור}
\textblock{אמר מר רבי יהודה אומר קור הרי הוא כעץ לכל דבריו אלא שניקח בכסף מעשר היינו תנא קמא}
\textblock{אמר אביי שלקו וטגנו איכא בינייהו}
\textblock{מתקיף לה רבא מי איכא למאן דאמר שלקו וטגנו לא והתניא העור והשיליא אין מטמאין טומאת אוכלין עור ששלקו ושיליא שחישב עליה מטמאין טומאת אוכלין}
\textblock{אלא אמר רבא איכא בינייהו ברכה דאתמר קור רב יהודה אמר בורא פרי האדמה ושמואל אמר שהכל נהיה בדברו}
\textblock{רב יהודה אמר בורא פרי האדמה אוכלא הוא ושמואל אמר שהכל נהיה בדברו כיון שסופו להקשות לא מברכינן עילויה בורא פרי האדמה}
\textblock{א"ל שמואל לרב יהודה שיננא כוותיך מסתברא דהא צנון שסופו להקשות ומברכינן עליה בורא פרי האדמה}
\textblock{ולא היא צנון נטעי אינשי אדעתא דפוגלא דיקלא לא נטעי אינשי אדעתא דקורא ואע"ג דקלסיה שמואל לרב יהודה הלכתא כוותיה דשמואל:}
\textblock{גופא אמר רב יהודה אמר רב כשות וחזיז מערבין בהן ומברכין עליהם בורא פרי האדמה כשות בכמה כדאמר רב יחיאל כמלא היד הכא נמי כמלא היד}
\textblock{חזיז בכמה אמר רבה בר טוביה בר יצחק אמר רב כמלא אוזילתא דאיכרי}
\textblock{אמר רב חלקיה בר טוביה מערבין בקליא בקליא ס"ד אלא בירקא דקליא וכמה אמר רב יחיאל כמלא היד}
\textblock{רבי ירמיה נפק לקירייתא בעו מיניה מהו לערב בפולין לחין לא הוה בידיה כי אתא לבי מדרשא אמרו ליה הכי אמר רבי ינאי מערבין בפולין לחין וכמה אמר רב יחיאל כמלא היד}
\textblock{אמר רב המנונא מערבין בתרדין חיין איני והאמר רב חסדא סילקא חייא קטיל גברא חייא}
\newsection{דף כט}
\textblock{ההוא בבשיל ולא בשיל}
\textblock{איכא דאמרי אמר רב המנונא אין מערבין בתרדין חיין דאמר רב חסדא סילקא חייא קטיל גברא חייא והא קא חזינן דקא אכלי ולא מייתי התם בבשיל ולא בשיל}
\textblock{אמר רב חסדא תבשיל של תרדין יפה ללב וטוב לעינים וכל שכן לבני מעיים אמר אביי והוא דיתיב אבי תפי ועביד תוך תוך}
\textblock{אמר רבא הריני כבן עזאי בשוקי טבריא אמר ליה ההוא מרבנן לרבא תפוחים בכמה אמר ליה וכי מערבין בתפוחים}
\textblock{ולא והתנן כל האוכלין מצטרפין לפסול את הגוויה בחצי פרס ובמזון שתי סעודות לעירוב וכביצה לטמא טומאת אוכלים}
\textblock{והאי מאי תיובתא אילימא משום דקתני כל האוכלין והני בני אכילה נינהו והאמר רבי יוחנן אין למדין מן הכללות ואפילו במקום שנאמר בו חוץ}
\textblock{אלא משום דקתני ובמזון שתי סעודות לעירוב וכביצה לטמא טומאת אוכלין והני נמי בני טמויי טומאת אוכלין נינהו}
\textblock{וכמה אמר ר"נ תפוחים בקב}
\textblock{מיתיבי ר"ש בן אלעזר אומר עוכלא תבלין וליטרא ירק ועשרה אגוזין וחמשה אפרסקין ושני רמונים ואתרוג אחד ואמר גורסק בר דרי משמיה דרב מנשיא בר שגובלי משמיה דרב וכן לעירוב והני נמי ליהוו כי אפרסקין}
\textblock{הני חשיבי והני לא חשיבי}
\textblock{אמר רב יוסף שרא ליה מריה לרב מנשיא בר שגובלי אנא אמריתא ניהליה אמתני' והוא אמרה אברייתא דתנן אין פוחתין לעני בגורן מחצי קב חטין וקב שעורין רבי מאיר אומר חצי קב שעורין וקב וחצי כוסמין וקב גרוגרות או מנה דבילה רבי עקיבא אומר פרס וחצי לוג יין ר"ע אומר רביעית ורביעית שמן ר"ע אומר שמינית ושאר כל הפירות אמר אבא שאול כדי שימכרם ויקח בהן מזון שתי סעודות ואמר רב וכן לעירוב}
\textblock{ומאי אולמיה דהאי מהך אילימא משום דקא תני בהך תבלין ותבלין לאו בני אכילה נינהו אטו הכא מי לא קתני חטין ושעורין ולאו בני אכילה נינהו}
\textblock{אלא משום דקתני חצי לוג יין ואמר רב מערבין בשתי רביעיות של יין מדבעינן כולי האי ש"מ כי אמר רב וכן לעירוב אהא מתני' קאמר ש"מ}
\textblock{אמר מר ובמזון שתי סעודות לעירוב סבר רב יוסף למימר עד דאיכא סעודה מהאי וסעודה מהאי אמר ליה רבה אפילו למחצה לשליש ולרביע}
\textblock{גופא אמר רב מערבין בשתי רביעיות של יין ומי בעינן כולי האי והתניא רבי שמעון בן אלעזר אומר יין כדי לאכול בו חומץ כדי לטבל בו זיתים ובצלים כדי לאכול בהן שתי סעודות}
\textblock{התם בחמרא מבשלא}
\textblock{אמר מר חומץ כדי לטבל בו אמר רב גידל אמר רב כדי לטבל בו מזון שתי סעודות של ירק איכא דאמרי אמר רב גידל אמר רב ירק הנאכל בשתי סעודות}
\textblock{אמר מר זיתים ובצלים כדי לאכול בהן מזון שתי סעודות ובבצלים מי מערבין והתניא אמר ר' שמעון בן אלעזר פעם אחת שבת רבי מאיר בערדיסקא ובא אדם אחד לפניו אמר לו ר' עירבתי בבצלים לטיבעין והושיבו ר' מאיר בארבע אמות שלו}
\textblock{לא קשיא הא בעלים הא באימהות דתניא אכל בצל והשכים ומת אין אומרין ממה מת ואמר שמואל לא שנו אלא בעלים אבל באימהות לית לן בה ובעלין נמי לא אמרן אלא}
\textblock{דלא אבציל זירתא אבל אבציל זירתא לית לן בה}
\textblock{אמר רב פפא לא אמרן אלא דלא אישתי שיכרא אבל אישתי שיכרא לית לן בה}
\textblock{תנו רבנן לא יאכל אדם בצל מפני נחש שבו ומעשה ברבי חנינא שאכל חצי בצל וחצי נחש שבו וחלה ונטה למות ובקשו חביריו רחמים עליו וחיה מפני שהשעה צריכה לו:}
\textblock{א"ר זירא אמר שמואל שכר מערבין בו ופוסל את המקוה בשלשת לוגין מתקיף לה רב כהנא פשיטא וכי מה בין זה למי צבע דתנן רבי יוסי אומר מי צבע פוסלין את המקוה בשלשת לוגין אמרי התם מיא דצבעא מיקרי הכא שיכרא איקרי}
\textblock{ובכמה מערבין סבר רב אחא בריה דרב יוסף קמיה דרב יוסף למימר בתרין רבעי שכרא כדתנן המוציא יין כדי מזיגת הכוס ותני עלה כדי מזיגת כוס יפה מאי כוס יפה כוס של ברכה ואמר ר"נ אמר רבה בר אבוה כוס של ברכה צריך שיהא בו רובע רביעית כדי שימזגנו ויעמוד על רביעית וכדרבא דאמר רבא כל חמרא דלא דרי על חד תלת מיא לאו חמרא הוא}
\textblock{וקתני סיפא ושאר כל המשקין ברביעית וכל השופכין ברביעית מדהתם על חד ארבע הכא נמי על חד ארבע}
\textblock{ולא היא התם הוא דבציר מהכי לא חשיב אבל הכא לא דעבידי אינשי דשתו כסא בצפרא וכסא בפניא וסמכי עילויהו}
\textblock{תמרים בכמה אמר רב יוסף תמרים בקב אמר רב יוסף מנא אמינא לה דתניא אכל גרוגרות ושילם תמרים תבא עליו ברכה}
\textblock{היכי דמי אילימא לפי דמים דאכל מיניה בזוזא וקא משלם ליה בזוזא מאי תבא עליו ברכה בזוזא אכל בזוזא קא משלם אלא לאו לפי מדה דאכל מיניה גריוא דגרוגרות דשויא זוזא וקא משלם ליה גריוא דתמרים דשוי ארבעה וקתני תבא עליו ברכה אלמא תמרים עדיפי}
\textblock{אמר ליה אביי לעולם דאכל מיניה בזוזא וקא משלם בזוזא ומאי תבא עליו ברכה דאכל מיניה מידי דלא קפיץ עליה זבינא וקא משלם ליה מידי דקפיץ עליה זבינא}
\textblock{שתיתא אמר רב אחא בר פנחס תרי שרגושי כיסאני אמר אביי תרי בוני דפומבדיתא}
\textblock{אמר אביי אמרה לי אם הני כסאני מעלו לליבא ומבטלי מחשבתא}
\textblock{ואמר אביי אמרה לי אם האי מאן דאית ליה חולשא דליבא לייתי בישרא דאטמא ימינא דדיכרא ולייתי כבויי דרעיתא דניסן ואי ליכא כבויי דרעיתא לייתי סוגייני דערבתא וניכבביה וניכול ונשתי בתריה חמרא מרקא:}
\textblock{אמר רב יהודה אמר שמואל כל שהוא ליפתן כדי לאכול בו כל שאינו ליפתן כדי לאכול הימנו בשר חי כדי לאכול הימנו בשר צלי רבה אמר כדי לאכול בו ורב יוסף אמר כדי לאכול הימנו}
\textblock{אמר רב יוסף מנא אמינא לה דהני פרסאי אכלי טבהקי בלא נהמא אמר ליה אביי ופרסאי הוו רובא דעלמא והתנן בגדי עניים לעניים בגדי עשירים לעשירים}
\newsection{דף ל}
\textblock{אבל בגדי עשירים לעניים לא}
\textblock{וכי תימא הכא לחומרא והכא לחומרא והתניא רבי שמעון בן אלעזר אומר מערבין לחולה ולזקן כדי מזונו ולרעבתן בסעודה בינונית של כל אדם קשיא}
\textblock{ומי אמר ר"ש בן אלעזר הכי והתניא רבי שמעון בן אלעזר אומר עוג מלך הבשן פיתחו כמלואו}
\textblock{ואביי התם היכי ליעביד הדומי נהדמיה [ונפקיה]}
\textblock{איבעיא להו פליגי רבנן עליה דרבי שמעון בן אלעזר או לא ת"ש דאמר רבה בר בר חנה אמר רבי יוחנן עוג מלך הבשן פיתחו בארבעה}
\textblock{התם דאיכא פתחים קטנים טובא ואיכא חד דהוי ארבעה דודאי כי קא מרוח בההוא קא מרוח}
\textblock{אמר רב חייא בר רב אשי אמר רב מערבין בבשר חי אמר רב שימי בר חייא מערבין בביצים חיות וכמה אמר רב נחמן בר יצחק (אחת) סיני אמר שתים:}
\textblock{הנודר מן המזון מותר במים כו': מלח ומים הוא דלא איקרי מזון הא כל מילי איקרי מזון לימא תיהוי תיובתא דרב ושמואל דרב ושמואל דאמרי תרוייהו אין מברכין בורא מיני מזונות אלא על חמשת המינין בלבד}
\textblock{ולא אותביניה חדא זימנא לימא תיהוי תיובתייהו נמי מהא}
\textblock{אמר רב הונא באומר כל הזן עלי מים ומלח הוא דלא זייני הא כל מילי זייני}
\textblock{והאמר רבה בר בר חנה כי הוה אזילנא בתריה דרבי יוחנן למיכל פירי דגינוסר כי הוינן בי מאה הוה מנקטינן לכל חד וחד עשרה עשרה כי הוינן בי עשרה הוה מנקטינן לכל חד וחד מאה מאה וכל מאה מינייהו (לא) הוי מחזיק להו צנא בת תלתא סאוי והוה אכיל להו לכולהון ואמר שבועתא דלא טעים לי זיונא אימא מזונא}
\textblock{אמר רב הונא אמר רב שבועה שלא אוכל ככר זו מערבין לו בה ככר זו עלי אין מערבין לו בה}
\textblock{מיתיבי הנודר מן הככר מערבין לו בה מאי לאו דאמר עלי לא דאמר זו}
\textblock{הכי נמי מסתברא דקתני סיפא אימתי בזמן שאמר שבועה שלא אטעמנה}
\textblock{אבל אמר עלי מאי הכי נמי דאין מערבין לו בה א"ה אדתני ככר זו הקדש אין מערבין לו בה לפי שאין מערבין בהקדשות ליפלוג וליתני בדידה במה דברים אמורים דאמר זו אבל אמר עלי אין מערבין לו בה}
\textblock{אמר לך רב הונא אלא מאי כל היכא דאמר עלי מערבין קשיא רישא}
\textblock{חסורי מיחסרא והכי קתני הנודר מן הככר מערבין לו בה ואפילו אמר עלי נעשה כאומר שבועה שלא אטעמנה}
\textblock{מכל מקום קשיא לרב הונא הוא דאמר כרבי אליעזר דתניא רבי אליעזר אומר שבועה שלא אוכל ככר זו מערבין לו בה ככר זו עלי אין מערבין לו בה}
\textblock{ומי אמר רבי אליעזר הכי והתניא זה הכלל אדם אוסר עצמו באוכל מערבין לו בה אוכל הנאסר לו לאדם אין מערבין לו בה רבי אליעזר אומר ככר זו עלי מערבין לו בה ככר זו הקדש אין מערבין לו בה לפי שאין מערבין לו בהקדשות}
\textblock{תרי תנאי ואליבא דרבי אליעזר:}
\textblock{מערבין לנזיר ביין כו': מתני' דלא כב"ש דתניא ב"ש אומרים אין מערבין לנזיר ביין ולישראל בתרומה ב"ה אומרים מערבין לנזיר ביין ולישראל בתרומה אמרו להן ב"ה לב"ש אי אתם מודים}
\textblock{שמערבין לגדול ביום הכפורים}
\textblock{אמרו להן אבל אמרו להן כשם שמערבין לגדול ביום הכפורים כן מערבין לנזיר ביין ולישראל בתרומה}
\textblock{ובית שמאי התם איכא סעודה הראויה מבעוד יום הכא ליכא סעודה הראויה מבעוד יום}
\textblock{כמאן דלא כחנניה דתניא חנניה אומר כל עצמן של בית שמאי לא היו מודים בעירוב עד שיוציא מטתו וכל כלי תשמישיו לשם}
\textblock{כמאן אזלא הא דתניא עירב בשחורים לא יצא בלבנים בלבנים לא יצא בשחורים כמאן אמר רב נחמן בר יצחק חנניה היא ואליבא דב"ש}
\textblock{ולחנניה בשחורים הוא דלא יצא הא בלבנים יצא האמר עד שיוציא מטתו וכלי תשמישיו לשם הכי קאמר עירב בלבנים והוצרך לשחורים אף בלבנים לא יצא כמאן אמר רב נחמן בר יצחק חנניה היא ואליבא דבית שמאי:}
\textblock{סומכוס אומר בחולין: ואילו לנזיר ביין לא פליג מאי טעמא אפשר דמתשיל אנזירותיה}
\textblock{אי הכי תרומה נמי אפשר דמיתשיל עילויה אי מתשיל עלה הדרא לטיבלא}
\textblock{וליפרוש עלה ממקום אחר לא נחשדו חבירים לתרום שלא מן המוקף}
\textblock{ולפרוש עלה מיניה וביה דלית בה שיעורא}
\textblock{ומאי פסקא אלא סומכוס סבר לה כרבנן דאמרי כל דבר שהוא משום שבות גזרו עליו בין השמשות}
\textblock{כמאן אזלא הא דתנן יש שאמרו הכל לפי מה שהוא אדם מלא קומצו מנחה ומלא חפניו קטרת והשותה מלא לוגמיו ביום הכפורים ובמזון שתי סעודות לעירוב כמאן א"ר זירא סומכוס היא דאמר מאי דחזי ליה בעינן}
\textblock{לימא פליגא אדרבי שמעון בן אלעזר דתניא ר"ש בן אלעזר אומר מערבין לחולה ולזקן כדי מזונו ולרעבתן בסעודה בינונית של כל אדם}
\textblock{תרגומא אחולה וזקן אבל רעבתן בטלה דעתו אצל כל אדם:}
\textblock{ולכהן בבית הפרס: דאמר רב יהודה אמר שמואל מנפח אדם בית הפרס והולך רבי יהודה בר אמי משמיה דרב יהודה אמר בית הפרס שנידש טהור:}
\textblock{רבי יהודה אומר אף בית הקברות: תנא מפני שיכול לחוץ ולילך בשידה תיבה ומגדל קא סבר אהל זרוק שמיה אהל}
\textblock{ובפלוגתא דהני תנאי דתניא הנכנס לארץ העמים בשידה תיבה ומגדל רבי מטמא רבי יוסי ברבי יהודה מטהר}
\textblock{במאי קמיפלגי מ"ס אהל זרוק לאו שמיה אהל ומ"ס אהל זרוק שמיה אהל}
\textblock{והא דתניא רבי יהודה אומר}
\newsection{דף לא}
\textblock{מערבין לכהן טהור בתרומה טהורה בקבר היכי אזיל בשידה תיבה ומגדל}
\textblock{והא כיון דאחתא איטמיא לה בשלא הוכשרה או שנילושה במי פירות}
\textblock{והיכי מייתי לה בפשוטי כלי עץ דלא מקבלי טומאה}
\textblock{והא קא מאהיל דמייתי לה אחוריה}
\textblock{אי הכי מ"ט דרבנן קסברי אסור לקנות בית באיסורי הנאה}
\textblock{מכלל דר' יהודה סבר מותר קסבר מצות לאו ליהנות ניתנו}
\textblock{אלא הא דאמר רבא מצות לאו ליהנות ניתנו לימא כתנאי אמרה לשמעתיה אמר לך רבא אי סבירא להו דאין מערבין אלא לדבר מצוה דכולי עלמא מצות לאו ליהנות ניתנו והכא בהא קמיפלגי מ"ס אין מערבין אלא לדבר מצוה ומ"ס מערבין אפילו לדבר הרשות}
\textblock{אלא הא דאמר רב יוסף אין מערבין אלא לדבר מצוה לימא כתנאי אמרה לשמעתיה}
\textblock{אמר לך רב יוסף דכולי עלמא אין מערבין אלא לדבר מצוה ודכולי עלמא מצות לאו ליהנות ניתנו ובהא קמיפלגי מ"ס כיון דקנה ליה עירוב לא ניחא ליה דמינטרא ומ"ס ניחא ליה דמינטרא דאי איצטריך אכיל ליה}
\textblock{{\large\emph{מתני׳}} מערבין בדמאי ובמעשר ראשון שנטלה תרומתו ובמעשר שני והקדש שנפדו והכהנים בחלה}
\textblock{אבל לא בטבל ולא במעשר ראשון שלא נטלה תרומתו ולא במעשר שני והקדש שלא נפדו:}
\textblock{{\large\emph{גמ׳}} דמאי הא לא חזי ליה מיגו דאי בעי מפקר להו לנכסיה והוי עני וחזו ליה השתא נמי חזי ליה דתנן מאכילין את העניים דמאי}
\textblock{ואת אכסניא דמאי}
\textblock{אמר רב הונא תנא בית שמאי אומרים אין מאכילין את העניים דמאי ובית הלל אומרים מאכילין את העניים דמאי:}
\textblock{ובמעשר ראשון שנטלה כו': פשיטא לא צריכא שהקדימו בשבלין ונטלה ממנו תרומת מעשר ולא נטלה ממנו תרומה גדולה}
\textblock{וכדר' אבהו אמר ריש לקיש דאמר ר' אבהו אמר ריש לקיש מעשר ראשון שהקדימו בשבלין פטור מתרומה גדולה שנאמר (במדבר יח, כו) והרמותם ממנו תרומת ה' מעשר מן המעשר מעשר מן המעשר אמרתי לך ולא תרומה גדולה ותרומת מעשר מן המעשר}
\textblock{א"ל רב פפא לאביי אי הכי אפילו הקדימו בכרי נמי א"ל עליך אמר קרא (במדבר יח, כח) מכל מעשרותיכם תרימו את כל תרומת ה'}
\textblock{ומה ראית האי אידגן והאי לא אידגן:}
\textblock{ובמעשר שני והקדש שנפדו: פשיטא לא צריכא שנתן את הקרן ולא נתן את החומש וקמ"ל דאין החומש מעכב:}
\textblock{אבל לא בטבל: פשיטא לא צריכא בטבל טבול מדרבנן וכגון שזרעו בעציץ שאינו נקוב:}
\textblock{ולא במעשר ראשון שלא נטלה תרומתו: פשיטא לא צריכא שהקדימו בכרי ונטלה ממנו תרומת מעשר ולא נטלה ממנו תרומה גדולה}
\textblock{מהו דתימא כדאמר ליה רב פפא לאביי קמ"ל כדשני ליה:}
\textblock{ולא במעשר שני והקדש שלא נפדו: פשיטא}
\textblock{לא צריכא שפדאן ולא פדאן כהלכתן מעשר שפדאו על גב אסימון ורחמנא אמר (דברים יד, כה) וצרת הכסף כסף שיש עליו צורה}
\textblock{הקדש שחיללו על גב קרקע דרחמנא אמר ונתן הכסף וקם לו:}
\textblock{{\large\emph{מתני׳}} השולח עירובו ביד חרש שוטה וקטן או ביד מי שאינו מודה בעירוב אינו עירוב ואם אמר לאחר לקבלו ממנו הרי זה עירוב:}
\textblock{{\large\emph{גמ׳}} וקטן לא והאמר רב הונא קטן גובה את העירוב לא קשיא כאן בעירובי תחומין כאן בעירובי חצירות:}
\textblock{או ביד מי שאינו מודה בעירוב: מאן אמר רב חסדא כותאי:}
\textblock{ואם אמר לאחר לקבלו הימנו הרי זה עירוב: וליחוש דילמא לא ממטי ליה כדאמר רב חסדא בעומד ורואהו ה"נ בעומד ורואהו}
\textblock{וליחוש דילמא לא שקיל מיניה כדאמר רב יחיאל חזקה שליח עושה שליחותו הכא נמי חזקה שליח עושה שליחותו}
\textblock{והיכא איתמר דרב חסדא ורב יחיאל אהא אתמר דתניא נתנו לפיל והוליכו לקוף והוליכו אין זה עירוב ואם אמר לאחר לקבלו הימנו הרי זה עירוב ודילמא לא ממטי ליה אמר רב חסדא בעומד ורואהו ודילמא לא מקבל ליה מיניה אמר רב יחיאל חזקה שליח עושה שליחותו}
\textblock{אמר רב נחמן בשל תורה אין חזקה שליח עושה שליחותו}
\newsection{דף לב}
\textblock{בשל סופרים חזקה שליח עושה שליחותו ורב ששת אמר אחד זה ואחד זה חזקה שליח עושה שליחותו}
\textblock{א"ר ששת מנא אמינא לה דתנן משקרב העומר הותר החדש מיד}
\textblock{והרחוקים מותרים מחצות היום ואילך והא חדש דאורייתא הוא וקתני הרחוקים מותרין מחצות היום ואילך לאו משום חזקה שליח עושה שליחותו}
\textblock{ורב נחמן התם כדקתני טעמא לפי שיודעין שאין בית דין מתעצלין בו}
\textblock{ואיכא דאמרי אמר רב נחמן מנא אמינא לה דקתני טעמא לפי שיודעין שאין ב"ד מתעצלין בו ב"ד הוא דלא מתעצלין בו הא שליח מתעצל בו}
\textblock{ורב ששת אמר לך ב"ד עד פלגיה דיומא שליח כולי יומא}
\textblock{אמר רב ששת מנא אמינא לה דתניא האשה שיש עליה לידה או זיבה מביאה מעות ונותנת בשופר וטובלת ואוכלת בקדשים לערב מאי טעמא לאו משום דאמרינן חזקה שליח עושה שליחותו}
\textblock{ורב נחמן התם כדרב שמעיה דאמר רב שמעיה חזקה אין ב"ד של כהנים עומדים משם עד שיכלו כל מעות שבשופר}
\textblock{אמר רב ששת מנא אמינא לה דתניא האומר לחבירו צא ולקט לך תאנים מתאנתי אוכל מהן עראי ומעשרן ודאי מלא לך כלכלה זה תאנים מתאנתי אוכל מהן עראי ומעשרן דמאי}
\textblock{במה דברים אמורים בעם הארץ אבל בחבר אוכל ואינו צריך לעשר דברי רבי רבן שמעון בן גמליאל אומר במה דברים אמורים בעם הארץ אבל בחבר אינו אוכל עד שיעשר לפי שלא נחשדו חברים לתרום שלא מן המוקף}
\textblock{אמר רבי נראין דברי מדברי אבא מוטב שיחשדו חברים לתרום שלא מן המוקף ולא יאכילו לעמי הארץ טבלים}
\textblock{עד כאן לא פליגי אלא דמר סבר נחשדו ומר סבר לא נחשדו אבל כולי עלמא חזקה שליח עושה שליחותו}
\textblock{ורב נחמן התם כדרב חנינא חוזאה דאמר רב חנינא חוזאה חזקה הוא על חבר שאינו מוציא דבר שאינו מתוקן מתחת ידו}
\textblock{אמר מר במה דברים אמורים בעם הארץ אבל בחבר אוכל ואינו צריך לעשר דברי רבי}
\textblock{האי עם הארץ דקאמר ליה למאן אילימא דקאמר לעם הארץ חבריה מעשרן דמאי מי ציית אלא בעם הארץ דקאמר ליה לחבר אימא סיפא נראין דברי מדברי אבא מוטב שיחשדו חברים לתרום שלא מן המוקף ואל יאכילו לעמי הארץ טבלין עמי הארץ מאי בעי התם}
\textblock{אמר רבינא רישא בעם הארץ שאמר לחבר סיפא בחבר שאמר לעם הארץ וחבר אחר שומעו רבי}
\textblock{סבר אותו חבר אוכל ואינו צריך לעשר דודאי עישורי מעשר ההוא חבר קמא עילויה ורבן שמעון בן גמליאל אומר לא יאכל עד שיעשר לפי שלא נחשדו חברים לתרום שלא מן המוקף ואמר ליה רבי מוטב שיחשדו חברים לתרום שלא מן המוקף ואל יאכילו עמי הארץ טבלים}
\textblock{במאי קמיפלגי רבי סבר ניחא ליה לחבר דלעביד הוא איסורא קלילא ולא ליעבד עם הארץ איסורא רבה ורבן שמעון בן גמליאל סבר ניחא ליה לחבר דליעבד עם הארץ איסורא רבה ואיהו אפי' איסורא קלילא לא ליעבד:}
\textblock{{\large\emph{מתני׳}} נתנו באילן למעלה מעשרה טפחים אין עירובו עירוב למטה מעשרה טפחים עירובו עירוב נתנו בבור אפילו עמוק מאה אמה עירובו עירוב:}
\textblock{{\large\emph{גמ׳}} יתיב רבי חייא בר אבא ורבי אסי ורבא בר נתן ויתיב רב נחמן גבייהו ויתבי וקאמרי האי אילן דקאי היכא אילימא דקאי ברשות היחיד מה לי למעלה מה לי למטה רשות היחיד עולה עד לרקיע}
\textblock{ואלא דקאי ברשות הרבים דמתכוין לשבות היכא אילימא דנתכוון לשבות למעלה הוא ועירובו במקום אחד הוא אלא נתכוון לשבות למטה והא קא משתמש באילן}
\textblock{לעולם דקאי ברה"ר ונתכוון לשבות למטה ורבי היא דאמר כל דבר שהוא משום שבות לא גזרו עליו בין השמשות}
\textblock{אמר להו רב נחמן ישר וכן אמר שמואל אמרו ליה פתריתו בה כולי האי אינהו נמי הכי קא פתרי בה [אלא הכי] אמרו ליה קבעיתו ליה בגמרא אמר להו אין אתמר נמי אמר רב נחמן אמר שמואל הכא באילן העומד ברשות הרבים עסקינן גבוה עשרה ורחב ארבעה ונתכוון לשבות למטה ורבי היא דאמר כל דבר שהוא משום שבות לא גזרו עליו בין השמשות}
\textblock{אמר רבא ל"ש אלא באילן העומד חוץ לעיבורה של עיר אבל אילן העומד בתוך עיבורה של עיר אפילו למעלה מעשרה הרי זה עירוב דמתא כמאן דמליא דמיא}
\textblock{אי הכי חוץ לעיבורה של עיר נמי כיון דאמר רבא הנותן עירובו יש לו ארבע אמות הויא לה רשות היחיד ורשות היחיד עולה עד לרקיע}
\textblock{אמר רב יצחק בריה דרב משרשיא הכא באילן הנוטה חוץ לארבע אמות עסקינן}
\newsection{דף לג}
\textblock{ונתכוין לשבות בעיקרו ומאי למעלה ומאי למטה דהדר זקיף}
\textblock{והא אי בעי מייתי לה דרך עליו}
\textblock{כשרבים מכתפין עליו וכדעולא דאמר עולא עמוד תשעה ברשות הרבים ורבים מכתפין עליו וזרק ונח על גביו חייב:}
\textblock{מאי רבי ומאי רבנן}
\textblock{דתניא נתנו באילן למעלה מעשרה טפחים אין עירובו עירוב למטה מעשרה טפחים עירובו עירוב ואסור ליטלו בתוך שלשה מותר ליטלו נתנו בכלכלה ותלאו באילן אפילו למעלה מעשרה טפחים עירובו עירוב דברי רבי וחכמים אומרים כל מקום שאסור ליטלו אין עירובו עירוב}
\textblock{וחכ"א אהייא אילימא אסיפא לימא קסברי רבנן צדדין אסורין אלא ארישא}
\textblock{האי אילן היכי דמי אי דלית ביה ארבעה מקום פטור הוא ואי דאית ביה ארבעה כי נתנו בכלכלה מאי הוי}
\textblock{אמר רבינא רישא דאית ביה ארבעה סיפא דלית ביה ארבעה וכלכלה משלימתו לארבעה}
\textblock{ורבי סבר לה כר' מאיר וסבר לה כר' יהודה}
\textblock{סבר לה כרבי מאיר דאמר חוקקין להשלים}
\textblock{וסבר לה כר' יהודה דאמר בעינן עירוב על גבי מקום ארבעה וליכא}
\textblock{מאי רבי יהודה דתניא רבי יהודה אומר נעץ קורה ברשות הרבים והניח עירובו עליה גבוה י' ורחבה ד' עירובו עירוב ואם לאו אין עירובו עירוב}
\textblock{אדרבה הוא ועירובו במקום אחד אלא הכי קאמר גבוה עשרה צריך שיהא בראשה ארבעה אין גבוהה עשרה אין צריך שיהא בראשה ארבעה}
\textblock{כמאן דלא כרבי יוסי ברבי יהודה דתניא רבי יוסי ברבי יהודה אומר נעץ קנה ברשות הרבים והניח בראשו טרסקל וזרק ונח על גביו חייב}
\textblock{אפילו תימא רבי יוסי ברבי יהודה התם הדרן מחיצתא הכא לא הדרן מחיצתא}
\textblock{רבי ירמיה אמר שאני כלכלה הואיל ויכול לנטותה ולהביאה לתוך עשרה}
\textblock{יתיב רב פפא וקא אמר להא שמעתא איתיביה רב בר שבא לרב פפא כיצד הוא עושה מוליכו בראשון ומחשיך עליו ונוטלו ובא לו בשני מחשיך עליו ואוכלו ובא לו}
\newsection{דף לד}
\textblock{אמאי נימא כיון דאי בעי אמטויי מצי ממטי ליה אע"ג דלא אמטייה כמאן דאמטייה דמי}
\textblock{א"ר זירא גזירה משום י"ט שחל להיות אחר שבת}
\textblock{איתיביה נתכוון לשבות ברה"ר והניח עירובו בכותל למטה מעשרה טפחים עירובו עירוב למעלה מי' טפחים אין עירובו עירוב נתכוון לשבות בראש השובך או בראש המגדל למעלה מי' טפחים עירובו עירוב למטה מי' טפחים אין עירובו עירוב}
\textblock{ואמאי הכי נמי נימא הואיל ויכול לנטותו ולהביאו לתוך עשרה א"ר ירמיה הכא במגדל מסומר עסקינן}
\textblock{רבא אמר אפילו תימא במגדל שאינו מסומר והכא במגדל ארוך עסקינן דאי ממטי ליה פורתא אזיל חוץ לארבע אמות}
\textblock{היכי דמי אי דאיכא כוותא ומתנא לייתיה בכוותא ומתנא דלית ליה כוותא ומתנא:}
\textblock{נתנו בבור אפילו עמוק מאה אמה וכו': האי בור דקאי היכא אילימא דקאי ברשות היחיד}
\textblock{פשיטא רה"י עולה עד לרקיע וכי היכי דסלקא לעיל ה"נ דנחתא לתחת ואלא דקאי ברשות הרבים}
\textblock{דנתכוון לשבות היכא אי למעלה הוא במקום אחד ועירובו במקום אחר הוא אי למטה פשיטא הוא ועירובו במקום אחד}
\textblock{לא צריכא דקאי בכרמלית ונתכוון לשבות למעלה ורבי היא דאמר כל דבר שהוא משום שבות לא גזרו עליו בין השמשות:}
\textblock{{\large\emph{מתני׳}} נתנו בראש הקנה או בראש הקונדס בזמן שהוא תלוש ונעוץ אפילו גבוה ק' אמה הרי זה עירוב:}
\textblock{{\large\emph{גמ׳}} רמי ליה רב אדא בר מתנא לרבא תלוש ונעוץ אין לא תלוש ונעוץ לא מני רבנן היא דאמרי כל דבר שהוא משום שבות גזרו עליו בין השמשות והא אמרת רישא רבי רישא רבי וסיפא רבנן}
\textblock{א"ל כבר רמי ליה רמי בר חמא לרב חסדא ושני ליה רישא רבי וסיפא רבנן}
\textblock{רבינא אמר כולה רבי היא וסיפא גזירה שמא יקטום:}
\textblock{ההוא פולמוסא דאתא לנהרדעא אמר להו רב נחמן פוקו עבידו כבושי כבשי באגמא ולמחר ניזיל וניתיב עלויהו}
\textblock{איתיביה רמי בר חמא לרב נחמן ואמרי לה רב עוקבא בר אבא לרב נחמן תלוש ונעוץ אין לא תלוש ולא נעוץ לא}
\textblock{א"ל התם בעוזרדין ומנא תימרא דשני לן בין עוזרדין לשאין עוזרדין דתניא הקנין והאטדין וההגין מין אילן הן ואינן כלאים בכרם ותניא אידך הקנים והקידן והאורבנין מין ירק הן והן כלאים בכרם קשיא אהדדי}
\textblock{אלא ש"מ כאן בעוזרדין כאן בשאין עוזרדין ש"מ}
\textblock{וקידה מין ירק הוא והתנן אין מרכיבין פגם ע"ג קידה לבנה מפני שהוא ירק באילן אמר רב פפא קידה לחוד וקידה לבנה לחוד:}
\textblock{{\large\emph{מתני׳}} נתנו במגדל ואבד המפתח הרי זה עירוב ר"א אומר אם אינו יודע שהמפתח במקומו אינו עירוב:}
\textblock{{\large\emph{גמ׳}} ואמאי הוא במקום אחד ועירובו במקום אחר הוא}
\textblock{רב ושמואל דאמרי תרוייהו הכא במגדל של לבנים עסקינן ור"מ היא דאמר פוחת לכתחילה ונוטל דתנן בית שמילאהו פירות סתום ונפחת נוטל ממקום הפחת ר' מאיר אומר פוחת ונוטל לכתחילה}
\textblock{והאמר רב נחמן בר אדא אמר שמואל באוירא דליבני הכא נמי באוירא דליבני}
\textblock{והא אמר רבי זירא בי"ט אמרו אבל לא בשבת ה"נ בי"ט}
\textblock{אי הכי היינו דקתני עלה ר"א אומר אם בעיר אבד עירובו עירוב ואם בשדה אבד אין עירובו עירוב ואי ביום טוב מה לי עיר מה לי שדה}
\newsection{דף לה}
\textblock{חסורי מיחסרא והכי קתני נתנו במגדל ונעל בפניו ואבד המפתח הרי זה עירוב במה דברים אמורים ביום טוב אבל בשבת אין עירובו עירוב נמצא המפתח בין בעיר בין בשדה אין עירובו עירוב רבי אליעזר אומר בעיר עירובו עירוב בשדה אין עירובו עירוב}
\textblock{בעיר עירובו עירוב כר' שמעון דאמר אחד גגות ואחד חצירות ואחד קרפיפות רשות אחת הן לכלים ששבתו בתוכן בשדה אין עירובו עירוב כרבנן}
\textblock{רבה ורב יוסף דאמרי תרוייהו הכא במגדל של עץ עסקינן דמר סבר כלי הוא ואין בנין בכלים ואין סתירה בכלים ומר סבר אהל הוא}
\textblock{ובפלוגתא דהני תנאי דתנן הקיש על גבי שידה תיבה ומגדל טמאין רבי נחמיה ורבי שמעון מטהרין}
\textblock{מאי לאו בהא קמפלגי מר סבר כלי הוא ומר סבר אהל הוא}
\textblock{אמר אביי ותיסברא והתניא אהל וניסט טמא כלי ואינו ניסט טהור וקתני סיפא ואם היו ניסוטין טמאים זה הכלל ניסט מחמת כחו טמא מחמת רעדה טהור}
\textblock{אלא אמר אביי דכ"ע היסט מחמת כחו טמא מחמת רעדה טהור והכא ברעדה מחמת כחו עסקינן ובהא קא מיפלגי דמר סבר הוי היסט ומר סבר לא הוי היסט}
\textblock{ומתניתין במאי מוקמינן לה אביי ורבא דאמרי תרוייהו במנעול וקטיר במתנא עסקינן ובעי סכינא למיפסקיה}
\textblock{תנא קמא סבר לה כרבי יוסי דאמר כל הכלים ניטלין בשבת חוץ ממסר הגדול ויתד של מחרישה}
\textblock{ורבי אליעזר סבר לה כרבי נחמיה דאמר אפי' טלית אפי' תרווד אין ניטלין אלא לצורך תשמישן:}
\textblock{{\large\emph{מתני׳}} נתגלגל חוץ לתחום נפל עליו גל או נשרף תרומה ונטמאת מבעוד יום אינו עירוב משחשיכה הרי זה עירוב}
\textblock{אם ספק ר"מ ור' יהודה אומרים הרי זה חמר גמל}
\textblock{ר' יוסי ור"ש אומרים ספק עירוב כשר אמר ר' יוסי אבטולמוס העיד משום חמשה זקנים על ספק עירוב שכשר:}
\textblock{{\large\emph{גמ׳}} נתגלגל חוץ לתחום אמר רבא לא שנו אלא שנתגלגל חוץ לארבע אמות אבל לתוך ד' אמות הנותן עירובו יש לו ד' אמות:}
\textblock{נפל עליו גל וכו': קא ס"ד דאי בעי מצי שקיל ליה}
\textblock{לימא מתניתין דלא כרבי דאי כרבי האמר כל דבר שהוא משום שבות לא גזרו עליו בין השמשות}
\textblock{אפילו תימא כרבי לא צריכא דבעי מרא וחצינא}
\textblock{וצריכי דאי תנא נתגלגל משום דליתא גביה אבל נפל עליו גל דאיתיה גביה אימא ליהוי עירוב}
\textblock{ואי תנא נפל עליו גל משום דמיכסי אבל נתגלגל זימנין דאתי זיקא ומייתי ליה אימא ליהוי עירוב צריכא:}
\textblock{או נשרף תרומה ונטמאת: למה לי תנא נשרף}
\textblock{להודיעך כחו דרבי יוסי תנא תרומה ונטמאת להודיעך כחו דרבי מאיר}
\textblock{וסבר ר"מ ספיקא לחומרא והתנן טמא שירד לטבול ספק טבל ספק לא טבל ואפילו טבל ספק טבל בארבעים סאה ספק לא טבל בארבעים סאה וכן שני מקוואות באחת יש בה ארבעים סאה ובאחת אין בה ארבעים סאה וטבל באחת מהן ואינו יודע באיזה מהן טבל ספיקו טמא}
\textblock{במה דברים אמורים בטומאה חמורה}
\textblock{אבל בטומאה קלה כגון שאכל אוכלין טמאין ושתה משקין טמאין והבא ראשו ורובו במים שאובין או שנפלו על ראשו ועל רובו שלשה לוגין מים שאובין וירד לטבול ספק טבל ספק לא טבל ואפילו טבל ספק טבל בארבעים סאה ספק לא טבל בארבעים סאה וכן שני מקוואות באחת יש בה ארבעים סאה ואחת אין בה ארבעים סאה וטבל באחת מהן ואינו יודע באיזה מהן טבל ספיקו טהור}
\textblock{רבי יוסי מטמא}
\textblock{קסבר ר"מ תחומין דאורייתא נינהו}
\textblock{וסבר רבי מאיר תחומין דאורייתא והא תנן אם אין יכול להבליעו בזו אמר רבי דוסתאי בר ינאי משום ר"מ שמעתי שמקדרין בהרים}
\textblock{ואי ס"ד תחומין דאורייתא מי מקדרין והא אמר רב נחמן אמר רבה בר אבוה אין מקדרין לא בערי מקלט ולא בעגלה ערופה מפני שהן של תורה}
\textblock{לא קשיא הא דידיה הא דרביה דיקא נמי דקתני בזו אמר רבי דוסתאי בר ינאי משום רבי מאיר שמעתי שמקדרין בהרים ש"מ}
\textblock{ורמי דאורייתא אדאורייתא לרבי מאיר}
\textblock{דתנן נגע באחד בלילה ואינו יודע אם חי אם מת ולמחר השכים ומצאו מת רבי מאיר מטהר וחכמים מטמאין שכל הטמאות כשעת מציאתן}
\textblock{אמר רבי ירמיה משנתנו שהיה עליה שרץ כל בין השמשות אי הכי בהא לימא רבי יוסי ספק עירוב כשר}
\textblock{רבה ורב יוסף דאמרי תרוייהו הכא בשתי כיתי עדים עסקינן אחת אומרת מבעוד יום נטמאה ואחת אומרת משחשיכה}
\newsection{דף לו}
\textblock{רבא אמר התם תרי חזקי לקולא והכא חדא חזקה לקולא}
\textblock{קשיא דר' יוסי אדר' יוסי}
\textblock{אמר רב הונא בר חיננא שאני טומאה הואיל ויש לה עיקר מן התורה שבת נמי דאורייתא היא קסבר ר' יוסי תחומין דרבנן}
\textblock{ואיבעית אימא הא דידיה הא דרביה דיקא נמי דקתני א"ר יוסי אבטולמוס העיד משום חמשה זקנים שספק עירוב כשר ש"מ}
\textblock{רבא אמר התם היינו טעמא דרבי יוסי העמד טמא על חזקתו ואימא לא טבל}
\textblock{אדרבה העמד מקוה על חזקתו ואימא לא חסר במקוה שלא נמדד}
\textblock{תניא כיצד אמר ר' יוסי ספק עירוב כשר עירב בתרומה ספק מבעוד יום נטמאת ספק משחשיכה נטמאת וכן בפירות ספק מבעוד יום נתקנו ספק משחשיכה נתקנו זה הוא ספק עירוב כשר}
\textblock{אבל עירב בתרומה ספק טהורה ספק טמאה וכן בפירות ספק נתקנו ספק לא נתקנו אין זה ספק עירוב כשר}
\textblock{מאי שנא תרומה דאמר העמד תרומה על חזקתה ואימא טהורה היא פירות נמי העמד טבל על חזקתו ואימא לא נתקנו}
\textblock{לא תימא ספק מבעוד יום נתקנו אלא אימא ספק מבעוד יום נדמעו ספק משחשיכה נדמעו}
\textblock{בעא רב שמואל בר רב יצחק מרב הונא היו לפניו שתי ככרות אחת טמאה ואחת טהורה ואמר עירבו לי בטהורה בכל מקום שהיא מהו}
\textblock{תיבעי לרבי מאיר תיבעי לר' יוסי תיבעי לר"מ עד כאן לא קאמר ר"מ התם דליכא טהורה הכא הא איכא טהורה או דילמא אפילו לרבי יוסי לא קאמר אלא התם דאם איתא דהיא טהורה ידע לה אבל הכא הא לא ידע לה}
\textblock{אמר ליה בין לר' יוסי בין לרבי מאיר בעינן סעודה הראויה מבעוד יום וליכא}
\textblock{בעא מיניה רבא מרב נחמן ככר זו היום חול ולמחר קדש ואמר עירבו לי בזה מהו א"ל עירובו עירוב}
\textblock{היום קדש ולמחר חול ואמר עירבו לי בזה מהו א"ל אין עירובו עירוב מאי שנא}
\textblock{א"ל לכי תיכול עליה כורא דמלח' היום חול ולמחר קדש מספיקא לא נחתא ליה קדושה היום קדש ולמחר חול מספיקא לא פקעא ליה קדושתיה מיניה}
\textblock{תנן התם לגין טבול יום שמלאו מן החבית של מעשר טבל ואמר הרי זה תרומת מעשר לכשתחשך דבריו קיימין}
\textblock{ואם אמר עירבו לי בזה לא אמר כלום אמר רבא זאת אומרת סוף היום קונה עירוב}
\textblock{דאי סלקא דעתך תחילת היום קונה עירוב אי אמר עירבו לי בזה אמאי לא אמר כלום}
\textblock{אמר רב פפא אפילו תימא תחילת היום קונה עירוב בעינן סעודה הראויה מבעוד יום וליכא:}
\textblock{{\large\emph{מתני׳}} מתנה אדם על עירובו ואומר אם באו נכרים מן המזרח עירובי למערב מן המערב עירובי למזרח אם באו לכאן ולכאן למקום שארצה אלך לא באו לא לכאן ולא לכאן הריני כבני עירי}
\textblock{אם בא חכם מן המזרח עירובי למזרח מן המערב עירובי למערב בא לכאן ולכאן למקום שארצה אלך לא לכאן ולא לכאן הריני כבני עירי רבי יהודה אומר אם היה אחד מהן רבו הולך אצל רבו ואם היו שניהן רבותיו למקום שירצה ילך:}
\textblock{{\large\emph{גמ׳}} כי אתא רבי יצחק תני איפכא כולה מתניתין קשיא נכרים אנכרים קשיא חכם אחכם}
\textblock{נכרים אנכרים לא קשיא הא בפרה גבנא הא במרי דמתא}
\textblock{חכם אחכם לא קשיא הא במותיב פירקי הא במקרי שמע:}
\textblock{ר' יהודה אומר אם היה אחד מהן וכו': ורבנן זימנין דניחא ליה בחבריה טפי מרביה}
\textblock{אמר רב ליתא למתניתין מדתני איו דתני איו ר' יהודה אומר אין אדם מתנה על שני דברים כאחד אלא אם (כן) בא חכם למזרח עירובו למזרח ואם בא חכם למערב עירובו למערב אבל לכאן ולכאן לא}
\textblock{מאי שנא לכאן ולכאן דלא דאין ברירה למזרח למערב נמי אין ברירה}
\textblock{אמר רבי יוחנן וכבר בא חכם}
\textblock{אדרבה ליתא לדאיו ממתניתין}
\textblock{לא סלקא דעתך דהא שמעינן ליה לרבי יהודה דלית ליה ברירה דתנן הלוקח יין מבין הכותים}
\newsection{דף לז}
\textblock{אומר שני לוגין שאני עתיד להפריש הרי הן תרומה עשרה מעשר ראשון תשעה מעשר שני ומיחל ושותה מיד דברי ר"מ ר' יהודה ור' יוסי ור' שמעון אוסרין}
\textblock{עולא אמר ליתא לאיו ממתני' ואלא הא דקתני ר' יהודה ורבי יוסי ורבי שמעון אוסרין}
\textblock{עולא זוזי זוזי קתני דברי רבי מאיר ור' יהודה ר' יוסי ור' שמעון אוסרין}
\textblock{וסבר ר' יוסי אין ברירה והתנן רבי יוסי אומר שתי נשים שלקחו את קיניהן בעירוב או שנתנו קיניהן לכהן איזהו שירצה כהן יקריב עולה ולאיזה שירצה יקריב חטאת}
\textblock{אמר רבה התם כשהתנו}
\textblock{אי הכי מאי למימרא קמ"ל כדרב חסדא דאמר רב חסדא אין הקינין מתפרשות}
\textblock{אלא אי בלקיחת בעלים אי בעשיית כהן}
\textblock{ואכתי סבר ר' יוסי אין ברירה והתניא עם הארץ שאמר לחבר קח לי אגודה אחת של ירק או גלוסקא אחת אינו צריך לעשר דברי רבי יוסי}
\textblock{וחכ"א צריך לעשר איפוך}
\textblock{תא שמע האומר מעשר שיש לי בביתי מחולל על סלע שתעלה בידי מן הכיס רבי יוסי אומר מחולל}
\textblock{איפוך אימא ר' יוסי אומר לא חילל ומאי חזית דאפכת תרתי מקמי חדא איפוך חדא מקמי תרתי}
\textblock{הא ודאי איפכא תניא דקתני סיפא ומודה רבי יוסי באומר מעשר שיש לי בתוך ביתי יהא מחולל על סלע חדשה שתעלה בידי מן הכיס שחילל מדקאמר הכא שחילל מכלל דהתם לא חילל}
\textblock{האי סלע חדשה ה"ד אי דאיכא תרתי תלת דיש ברירה היינו קמייתא אלא דליכא אלא חדא מאי תעלה}
\textblock{איידי דתני רישא תעלה תנא סיפא נמי תעלה}
\textblock{א"ל רבא לרב נחמן מאן האי תנא דאפילו בדרבנן לית ליה ברירה דתניא אמר לחמשה הריני מערב על איזה מכם שארצה רציתי ילך לא רציתי לא ילך רצה מבעו"י עירובו עירוב משחשיכה אין עירובו עירוב}
\textblock{אישתיק ולא א"ל ולא מידי ולימא ליה תנא דבי איו הוא לא שמיע ליה}
\textblock{רב יוסף אמר תנאי שקלת מעלמא תנאי היא דתניא הריני מערב לשבתות של כל השנה רציתי אלך לא רציתי לא אלך רצה מבעוד יום עירובו עירוב משחשיכה רבי שמעון אומר עירובו עירוב וחכמים אומרים אין עירובו עירוב}
\textblock{והא שמעינן לרבי שמעון דלית ליה ברירה קשיא דרבי שמעון אדר"ש אלא איפוך}
\textblock{מאי קשיא דילמא כי לית ליה לר' שמעון ברירה בדאורייתא אבל בדרבנן אית ליה}
\textblock{קסבר רב יוסף מאן דאית ליה ברירה ל"ש בדאורייתא ל"ש בדרבנן אית ליה ומאן דלית ליה ברירה ל"ש בדאורייתא ול"ש בדרבנן לית ליה}
\textblock{רבא אמר שאני התם דבעינן ראשית ששיריה ניכרין}
\textblock{א"ל אביי אלא מעתה היו לפניו שני רמונים של טבל ואמר אם ירדו גשמים היום יהא זה תרומה על זה ואם לא ירדו גשמים היום יהא זה תרומה על זה ה"נ בין ירדו בין לא ירדו דאין בדבריו כלום}
\textblock{וכ"ת הכי נמי והתנן תרומת הכרי הזה ומעשרותיו בתוכו ותרומת מעשר זה בתוכו ר"ש אומר קרא השם}
\textblock{שאני התם דאיכא סביביו}
\textblock{ואב"א כדקתני טעמא אמרו לו לר"מ אי אתה מודה שמא יבקע הנוד ונמצא זה שותה טבלים למפרע אמר להן לכשיבקע}
\textblock{ולמאי דסליק אדעתין מעיקרא דבעינן ראשית ששיריה ניכרין מאי קאמרי ליה}
\textblock{הכי קאמרי ליה לדידן בעינן ראשית ששיריה ניכרין לדידך}
\newsection{דף לח}
\textblock{אי אתה מודה שמא יבקע הנוד ונמצא שותה טבלים למפרע אמר להן לכשיבקע:}
\textblock{{\large\emph{מתני׳}} רבי אליעזר אומר יו"ט הסמוך לשבת בין מלפניה ובין מלאחריה מערב אדם שני עירובין ואומר עירובי בראשון למזרח ובשני למערב בראשון למערב ובשני למזרח עירובי בראשון ובשני כבני עירי עירובי בשני ובראשון כבני עירי}
\textblock{וחכ"א או מערב לרוח אחת או אינו מערב כל עיקר או מערב לשני ימים או אינו מערב כל עיקר}
\textblock{כיצד יעשה מוליכו בראשון ומחשיך עליו ונוטלו ובא לו בשני מחשיך עליו ואוכלו ובא לו ונמצא משתכר בהליכתו ומשתכר בעירובו}
\textblock{נאכל בראשון עירובו לראשון ואין עירובו לשני}
\textblock{אמר (להן) ר' אליעזר מודים אתם לי שהן שתי קדושות:}
\textblock{{\large\emph{גמ׳}} לרוח אחת מאי ניהו לשני ימים לשני ימים מאי ניהו לרוח אחת היינו קמייתא}
\textblock{הכי קאמרי ליה רבנן לר' אליעזר אי אתה מודה שאין מערבין ליום אחד חציו לצפון וחציו לדרום אמר להן אבל כשם שאין מערבין ליום אחד חציו לדרום וחציו לצפון כך אין מערבין לשני ימים יום אחד למזרח ויום אחד למערב}
\textblock{ור"א התם קדושה אחת הכא ב' קדושות}
\textblock{אמר להן ר"א אי אתם מודים שאם עירב ברגליו ביום ראשון מערב ברגליו ביום שני נאכל עירובו ביום ראשון אין יוצא עליו ביום שני}
\textblock{אמרו לו אבל הא לאיי ב' קדושות הן ורבנן ספוקי מספקא להו והכא לחומרא והכא לחומרא}
\textblock{אמרו לו לרבי אליעזר אי אתה מודה שאין מערבין בתחילה מיו"ט לשבת אמר להן אבל הא לאיי קדושה אחת היא}
\textblock{ורבי אליעזר התם משום הכנה}
\textblock{ת"ר עירב ברגליו ביום ראשון מערב ברגליו ביום שני נאכל עירובו ביום ראשון אין יוצא עליו ביום שני דברי רבי}
\textblock{ר' יהודה אומר}
\textblock{הרי זה חמר גמל}
\textblock{רשב"ג ור' ישמעאל בנו של ר' יוחנן בן ברוקה אומרים עירב ברגליו בראשון אין מערב ברגליו בשני נאכל עירובו ביום ראשון יוצא עליו בשני}
\textblock{אמר רב הלכה כד' זקנים הללו ואליבא דר"א דאמר ב' קדושות הן ואלו הן ד' זקנים רשב"ג ור' ישמעאל בר' יוחנן בן ברוקה ור"א בר"ש ור' יוסי בר יהודה סתימתאה ואיכא דאמרי חד מינייהו רבי אלעזר ומפיק ר' יוסי בר יהודה סתימתאה}
\textblock{והא רשב"ג ור' ישמעאל בר רבי יוחנן בן ברוקה איפכא שמעינן להו איפוך}
\textblock{אי הכי היינו רבי אימא וכן אמר רשב"ג וכו'}
\textblock{וליחשוב נמי רבי רבי תני לה ולא סבר לה}
\textblock{רבנן נמי תנו לה ולא סברי לה רב גמרא גמיר לה}
\textblock{כי נח נפשיה דרב הונא עייל רב חסדא למירמא דרב אדרב מי אמר רב הלכה כד' זקנים ואליבא דר"א דאמר שתי קדושות הן}
\textblock{והא איתמר שבת ויו"ט רב אמר נולדה בזה אסורה בזה}
\textblock{אמר רבה התם משום הכנה}
\textblock{דתניא (שמות טז, ה) והיה ביום הששי והכינו חול מכין לשבת וחול מכין ליו"ט ואין יו"ט מכין לשבת ואין שבת מכינה ליום טוב}
\textblock{א"ל אביי אלא הא דתנן כיצד הוא עושה מוליכו בראשון ומחשיך עליו ונוטלו ובא לו בשני מחשיך עליו ואוכלו ובא לו הא קא מכין מיו"ט לשבת}
\textblock{א"ל רבה מי סברת סוף היום קונה עירוב תחלת היום קונה עירוב ושבת מכינה לעצמה}
\textblock{אלא מעתה יערבו בלגין}
\textblock{בעינן סעודה הראויה מבעוד יום וליכא}
\textblock{אלא הא דתנן ר"א אומר יו"ט הסמוך לשבת בין מלפניה ובין מלאחריה מערב אדם שני עירובין הא בעינן סעודה הראויה מבעוד יום וליכא}
\textblock{מי סברת דמנח ליה בסוף אלפים אמה לכאן ובסוף אלפים אמה לכאן לא דמנח ליה בסוף אלף אמה לכאן ובסוף אלף אמה לכאן}
\textblock{אלא הא דאמר רב יהודה עירב ברגליו יום ראשון מערב ברגליו יום שני עירב בפת ביום ראשון מערב בפת ביום שני הא קא מכין מיו"ט לשבת}
\textblock{א"ל מי סברת דאזיל ואמר מידי דאזיל ושתיק ויתיב}
\textblock{כמאן כרבי יוחנן בן נורי דאמר חפצי הפקר קונין שביתה}
\textblock{אפילו תימא רבנן עד כאן לא פליגי רבנן עליה דרבי יוחנן בן נורי אלא בישן דלא מצי אמר אבל בניעור דאי בעי למימר מצי אמר אע"ג דלא אמר כמאן דאמר דמי}
\textblock{א"ל רבה בר רב חנין לאביי אי הוה שמיע ליה למר הא דתניא לא יהלך אדם לסוף שדהו לידע מה היא צריכה כיוצא בו}
\newsection{דף לט}
\textblock{לא יטייל אדם על פתח מדינה כדי שיכנס למרחץ מיד הדר ביה}
\textblock{ולא היא שמע ליה ולא הדר ביה התם מוכחא מילתא הכא לאו מוכחא מילתא היא}
\textblock{אי צורבא מרבנן הוא אמרי' שמעתא משכתיה ואי עם הארץ הוא אמרי' חמרא אירכס ליה}
\textblock{גופא אמר רב יהודה עירב ברגליו ביום ראשון מערב ברגליו בשני עירב בפת ביום ראשון מערב בפת ביום שני}
\textblock{עירב בפת בראשון מערב ברגליו בשני עירב ברגליו בראשון אין מערב בפת בשני שאין מערבין בתחלה בפת}
\textblock{עירב בפת ביום ראשון מערב בפת ביום שני אמר שמואל ובאותה הפת אמר רב אשי דיקא נמי מתני' דקתני כיצד הוא עושה מוליכו בראשון ומחשיך עליו ונוטלו ובא לו בשני מחשיך עליו ואוכלו ובא לו}
\textblock{ורבנן דילמא התם עצה טובה קמ"ל:}
\textblock{{\large\emph{מתני׳}} ר' יהודה אומר ראש השנה שהיה ירא שמא תתעבר מערב אדם שני עירובין ואומר עירובי בראשון למזרח ובשני למערב בראשון למערב ובשני למזרח עירובי בראשון ובשני כבני עירי עירובי בשני ובראשון כבני עירי ולא הודו לו חכמים}
\textblock{ועוד אמר ר' יהודה מתנה אדם על הכלכלה ביו"ט ראשון ואוכלה בשני}
\textblock{וכן ביצה שנולדה בראשון תאכל בשני ולא הודו לו חכמים}
\textblock{ר' דוסא בן הרכינס אומר העובר לפני התיבה ביו"ט של ר"ה אומר החליצנו ה' אלהינו את יום ראש החדש הזה אם היום אם למחר ולמחר הוא אומר אם היום אם אמש ולא הודו לו חכמים:}
\textblock{{\large\emph{גמ׳}} מאן לא הודו לו אמר רב ר' יוסי היא דתניא מודים חכמים לר"א בר"ה שהיה ירא שמא תתעבר מערב אדם שני עירובין ואומר עירובי בראשון למזרח ובשני למערב בראשון למערב ובשני למזרח עירובי בראשון ובשני כבני עירי עירובי בשני ובראשון כבני עירי ר' יוסי אוסר}
\textblock{אמר להן ר' יוסי אי אתם מודים שאם באו עדים מן המנחה ולמעלה שנוהגין אותו היום קדש ולמחר קדש}
\textblock{ורבנן התם כי היכי דלא לזלזולי ביה:}
\textblock{ועוד א"ר יהודה וכו':}
\textblock{וצריכא דאי אשמעינן ר"ה בהא קאמר ר' יהודה משום דלא קעביד מידי אבל כלכלה דמיחזי כמתקן טיבלא אימא מודה להו לרבנן}
\textblock{ואי אשמעינן הני תרתי משום דליכא למיגזר עלייהו אבל ביצה דאיכא למיגזר בה משום פירות הנושרין ומשום משקין שזבו אימא מודה להו לרבנן צריכא:}
\textblock{תניא כיצד א"ר יהודה מתנה אדם על הכלכלה ביו"ט ראשון ואוכלה בשני היו לפניו שתי כלכלות של טבל אומר אם היום חול ולמחר קדש תהא זו תרומה על זו ואם היום קדש ולמחר חול אין בדברי כלום וקורא עליה שם ומניחה}
\textblock{ולמחר הוא אומר אם היום חול תהא זו תרומה על זו ואם היום קדש אין בדברי כלום וקורא עליה שם ואוכלה ר' יוסי אוסר וכן היה ר' יוסי אוסר בשני ימים טובים של גליות}
\textblock{ההוא בר טביא דאתא לבי. ריש גלותא דאתציד ביו"ט ראשון של גליות ואשתחיט ביו"ט שני}
\textblock{ר"נ ורב חסדא אכלו רב ששת לא אכל אמר ר"נ מאי אעביד ליה לרב ששת דלא אכיל בישרא דטביא א"ל רב ששת והיכי איכול דתני איסי ואמרי לה איסי תני וכן היה ר' יוסי אוסר שני ימים טובים של גליות}
\textblock{אמר רבא ומאי קושיא דילמא ה"ק וכן היה ר' יוסי אוסר בשני יו"ט של ר"ה בגולה א"ה של גליות בגולה מיבעי ליה}
\textblock{א"ר אסי ומאי קושיא דילמא הכי קאמר וכן היה ר' יוסי עושה איסור שני ימים טובים של גליות כשני ימים טובים של ראש השנה לרבנן דשרו}
\textblock{אשכחיה רב ששת לרבה בר שמואל אמר ליה תני מר מידי בקדושות אמר ליה תנינא מודה רבי יוסי בשני ימים טובים של גליות אמר ליה אי משכחת להו לא תימא להו ולא מידי}
\textblock{אמר רב אשי לדידי אמר לי אמימר ההוא בר טביא לאו איתצודי איתציד}
\newsection{דף מ}
\textblock{אלא מחוץ לתחום אתא מאן דאכל סבר הבא בשביל ישראל זה מותר לישראל אחר}
\textblock{ומאן דלא אכל סבר כל דאתי לבי ריש גלותא אדעתא דכולהו רבנן אתי}
\textblock{והא אשכחיה רב ששת לרבה בר שמואל וא"ל לא היו דברים מעולם}
\textblock{ההוא ליפתא דאתי למחוזא נפק רבא חזיא דכמישא שרא רבא למיזבן מיניה אמר הא ודאי מאיתמול נעקרה}
\textblock{מאי אמרת מחוץ לתחום אתיא הבא בשביל ישראל זה מותר לאכול לישראל אחר וכל שכן האי דאדעתא דנכרים אתא}
\textblock{כיון דחזא דקא מפשי ומייתי להו אסר להו:}
\textblock{הנהו בני גננא דגזו להו אסא בי"ט שני לאורתא שרא להו רבינא לאורוחי ביה לאלתר א"ל רבא בר תחליפא לרבינא ליסר להו מר מפני שאינן בני תורה}
\textblock{מתקיף לה רב שמעיה טעמא דאינן בני תורה הא בני תורה שרי והא בעינן בכדי שיעשו אזלו שיילוה לרבא אמר להו בעינן בכדי שיעשו:}
\textblock{ר' דוסא אומר העובר לפני התיבה כו':}
\textblock{אמר רבה כי הוינן בי רב הונא איבעיא לן מהו להזכיר של ראש חדש בראש השנה כיון דחלוקין במוספין אמרינן או דילמא זכרון אחד עולה לכאן ולכאן}
\textblock{אמר לן תניתוה רבי דוסא אומר העובר לפני התיבה כו' מאי לאו להזכיר}
\textblock{לא להתנות}
\textblock{הכי נמי מסתברא מדקתני בברייתא וכן היה ר' דוסא עושה בראשי חדשים של כל השנה כולה ולא הודו לו}
\textblock{אי אמרת בשלמא להתנות משום הכי לא הודו לו אלא אי אמרת להזכיר אמאי לא הודו לו}
\textblock{ואלא מאי להתנות למה לי לאיפלוגי בתרתי צריכא דאי אשמעינן ר"ה הוה אמינא בהא קאמרי רבנן דלא משום דאתי לזלזולי ביה אבל בראשי חדשים של כל השנה כולה אימא מודו ליה לר' דוסא}
\textblock{ואי אתמר בהא בהא קאמר ר' דוסא אבל בהך אימא מודה להו לרבנן צריכא}
\textblock{מיתיבי ראש השנה שחל להיות בשבת בית שמאי אומרים מתפלל עשר ובית הלל אומרים מתפלל תשע ואם איתא בית שמאי אחת עשרה מבעי ליה}
\textblock{אמר רבי זירא שאני ר"ח מתוך שכולל לשחרית וערבית כולל נמי במוספין}
\textblock{ומי אית להו לב"ש כולל והתניא ר"ח שחל להיות בשבת ב"ש אומרים מתפלל שמנה וב"ה אומרים מתפלל שבע קשיא:}
\textblock{וכולל עצמו תנאי היא דתניא שבת שחל להיות בר"ח או בחולו של מועד ערבית שחרית ומנחה מתפלל כדרכו שבע ואומר מעין המאורע בעבודה ר' אליעזר אומר בהודאה ואם לא אמר מחזירין אותו}
\textblock{ובמוספין מתחיל בשל שבת ומסיים בשל שבת ואומר קדושת היום באמצע}
\textblock{רשב"ג ור' ישמעאל בנו של ר' יוחנן בן ברוקה אומרים כל מקום שזקוק לשבע מתחיל בשל שבת ומסיים בשל שבת ואומר קדושת היום באמצע}
\textblock{מאי הוה עלה א"ר חסדא זכרון אחד עולה לו לכאן ולכאן וכן אמר רבה זכרון אחד עולה לו לכאן ולכאן:}
\textblock{ואמר רבה כי הוינא בי רב הונא איבעיא לן מהו לומר זמן בראש השנה וביום הכפורים כיון דמזמן לזמן אתי אמרינן או דילמא כיון דלא איקרו רגלים לא אמרינן לא הוה בידיה}
\textblock{כי אתאי בי רב יהודה אמר אנא אקרא חדתא נמי אמינא זמן א"ל רשות לא קא מיבעיא לי כי קא מיבעיא לי חובה מאי א"ל רב ושמואל דאמרי תרווייהו אין אומר זמן אלא בשלש רגלים}
\textblock{מיתיבי (קהלת יא, ב) תן חלק לשבעה וגם לשמונה ר' אליעזר אומר שבעה אלו ז' ימי בראשית שמונה אלו ח' ימי מילה ר' יהושע אומר שבעה אלו שבעה ימי פסח שמונה אלו שמונה ימי החג וכשהוא אומר וגם לרבות עצרת ור"ה ויוה"כ}
\textblock{מאי לאו לזמן לא לברכה}
\textblock{הכי נמי מסתברא דאי ס"ד לזמן זמן כל שבעה מי איכא הא לא קשיא דאי לא מברך האידנא מברך למחר וליום אוחרא}
\textblock{מ"מ בעינן כוס לימא מסייע ליה לר"נ דאמר ר"נ זמן אומרו אפילו בשוק הא לא קשיא דאיקלע ליה כוס}
\textblock{התינח עצרת ור"ה יום הכפורים היכי עביד אי מברך עליה ושתי ליה כיון דאמר זמן קבליה עליה ואסר ליה}
\textblock{דהאמר ליה רב ירמיה בר אבא לרב מי בדלת וא"ל אין בדילנא}
\textblock{לברוך עליה ולנחיה המברך צריך שיטעום ליתביה לינוקא לית הלכתא כרב אחא דילמא אתי למסרך}
\textblock{מאי הוי עלה שדרוה רבנן לרב יימר סבא קמיה דרב חסדא במעלי יומא דריש שתא אמרו ליה זיל חזי היכי עביד עובדא תא אימא לן כי חזייה א"ל דלויה לרטיבה רפסא ליה בדוכתיה אייתו ליה כסא דחמרא קדיש ואמר זמן}
\textblock{והלכתא אומר זמן בר"ה וביוה"כ והלכתא זמן אומרו אפילו בשוק:}
\textblock{ואמר רבה כי הוינן בי רב הונא איבעיא לן בר בי רב דיתיב בתעניתא במעלי שבתא מהו לאשלומי לא הוה בידיה אתאי לקמיה דרב יהודה ולא הוה בידיה}
\textblock{אמר רבא נחזייה אנן דתניא ט' באב שחל להיות בשבת}
\newsection{דף מא}
\textblock{וכן ערב תשעה באב שחל להיות בשבת אוכל ושותה כל צרכו ומעלה על שולחנו אפילו כסעודת שלמה בשעתו חל להיות תשעה באב בערב שבת מביאין לו כביצה ואוכל כדי שלא יכנס לשבת כשהוא מעונה}
\textblock{תניא אמר רבי יהודה פעם אחת היינו יושבין לפני ר"ע ותשעה באב שחל להיות בע"ש היה והביאו לו ביצה מגולגלת וגמעה בלא מלח ולא שהיה תאב לה אלא להראות לתלמידים הלכה}
\textblock{ורבי יוסי אומר מתענה ומשלים אמר להן ר' יוסי אי אתם מודים לי בט' באב שחל להיות באחד בשבת שמפסיק מבעוד יום אמרו לו אבל אמר להם מה לי ליכנס בה כשהוא מעונה מה לי לצאת ממנה כשהוא מעונה}
\textblock{אמרו לו אם אמרת לצאת ממנה שהרי אכל ושתה כל היום כולו תאמר ליכנס בה כשהוא מעונה שלא אכל ושתה כל היום כולו}
\textblock{ואמר עולא הלכה כרבי יוסי ומי עבדינן כרבי יוסי ורמינהי אין גוזרין תענית על הציבור בראשי חדשים בחנוכה ובפורים ואם התחילו אין מפסיקין דברי ר"ג אמר ר"מ אף על פי שאמר רבן גמליאל אין מפסיקין מודה היה שאין משלימין וכן בט' באב שחל להיות בע"ש}
\textblock{ותניא לאחר פטירתו של (רשב"ג) נכנס ר' יהושע להפר את דבריו עמד רבי יוחנן בן נורי על רגליו ואמר חזי אנא דבתר רישא גופא אזיל כל ימיו של רבן גמליאל קבענו הלכה כמותו עכשיו אתה מבקש לבטל דבריו יהושע אין שומעין לך שכבר נקבעה הלכה כר"ג ולא היה אדם שערער בדבר כלום}
\textblock{בדורו של רבן גמליאל עבוד כר"ג בדורו של רבי יוסי עבוד כרבי יוסי}
\textblock{ובדורו של ר"ג עבוד כר"ג והתניא א"ר אלעזר (בן) צדוק אני (הייתי) מבני סנאב בן בנימין פעם אחת חל תשעה באב להיות בשבת ודחינוהו לאחר השבת והתענינו בו ולא השלמנוהו מפני שיו"ט שלנו היה טעמא דיו"ט הא ערב יו"ט משלימין}
\textblock{אמר רבינא שאני יו"ט של דבריהם מתוך שמתענין בו שעות משלימין בו ערביות שבת הואיל ואין מתענין בה שעות אין משלימין בה ערביות}
\textblock{אמר רב יוסף לא שמיע לי הא שמעתא אמר ליה אביי את אמרת ניהלן ואהא אמרת ניהלן אין גוזרין תענית על הצבור בראשי חדשים וכו' ואמרינן עלה אמר רב יהודה אמר רב זו דברי רבי מאיר שאמר משום רבן גמליאל אבל חכמים אומרים מתענה ומשלים}
\textblock{מאי לאו אכולהו לא אחנוכה ופורים}
\textblock{הכי נמי מסתברא}
\textblock{דאי סלקא דעתך אכולהו הא בעי מיניה רבה מרב יהודה ולא פשט ליה}
\textblock{ולטעמיך הא דדרש מר זוטרא משמיה דרב הונא הלכה מתענה ומשלים הא בעא מיניה רבה מרב הונא ולא פשט ליה}
\textblock{אלא הא מקמי דשמעה והא לבתר דשמעה הכא נמי הא מקמי דשמעה הא לבתר דשמעה}
\textblock{דרש מר זוטרא משמיה דרב הונא הלכה מתענין ומשלימין:}
\textblock{\par \par {\large\emph{הדרן עלך בכל מערבין}}\par \par }
\textblock{}
\textblock{מתני׳ {\large\emph{מי}} שהוציאוהו נכרים או רוח רעה אין לו אלא ד' אמות}
\textblock{החזירוהו כאילו לא יצא}
\textblock{הוליכוהו לעיר אחרת נתנוהו בדיר או בסהר ר"ג ור' אלעזר בן עזריה אומרים מהלך את כולה רבי יהושע ור"ע אומרים אין לו אלא ד' אמות:}
\textblock{מעשה שבאו מפלנדרסין והפליגה ספינתם בים ר"ג ורבי אלעזר בן עזריה הלכו את כולה ר' יהושע ור"ע לא זזו מד"א שרצו להחמיר על עצמן}
\textblock{פעם אחת לא נכנסו לנמל עד שחשיכה אמרו לו לרבן גמליאל מה אנו לירד}
\textblock{אמר להם מותרים אתם שכבר הייתי מסתכל והיינו בתוך התחום עד שלא חשיכה:}
\textblock{{\large\emph{גמ׳}} ת"ר ג' דברים מעבירין את האדם על דעתו ועל דעת קונו אלו הן עובדי כוכבים ורוח רעה ודקדוקי עניות}
\textblock{למאי נפקא מינה למיבעי רחמי עלייהו}
\textblock{ג' אין רואין פני גיהנם אלו הן דקדוקי עניות וחולי מעיין והרשות ויש אומרים אף מי שיש לו אשה רעה}
\textblock{ואידך אשה רעה מצוה לגרשה}
\textblock{ואידך זימנין דכתובתה מרובה אי נמי אית ליה בנים מינה ולא מצי מגרש לה}
\textblock{למאי נפקא מינה לקבולי מאהבה}
\textblock{שלשה מתין כשהן מספרין ואלו הן חולי מעיין וחיה והדרוקן}
\textblock{למאי נפקא מינה למשמושי בהו זוודתא:}
\textblock{אמר רב נחמן אמר שמואל יצא לדעת אין לו אלא ארבע אמות פשיטא השתא מי שהוציאוהו נכרים אין לו אלא ד' אמות יצא לדעת מיבעיא}
\textblock{אלא אימא חזר לדעת אין לו אלא ד"א}
\textblock{הא נמי תנינא החזירוהו נכרים כאילו לא יצא החזירוהו הוא דכאילו לא יצא אבל הוציאוהו נכרים וחזר לדעת אין לו אלא ד"א}
\textblock{אלא אימא יצא לדעת והחזירוהו נכרים אין לו אלא ד"א}
\textblock{הא נמי תנינא הוציאוהו והחזירוהו כאילו לא יצא הוציאוהו והחזירוהו הוא דכאילו לא יצא אבל יצא לדעת לא}
\textblock{מהו דתימא לצדדין קתני מי שהוציאוהו נכרים וחזר לדעת אין לו אלא ד' אמות אבל יצא לדעת והחזירוהו נכרים כאילו לא יצא קמ"ל}
\textblock{בעו מיניה מרבה הוצרך לנקביו מהו אמר להם גדול כבוד הבריות שדוחה את לא תעשה שבתורה}
\textblock{אמרי נהרדעי אי פיקח הוא עייל לתחומא וכיון דעל על}
\textblock{א"ר פפא פירות שיצאו חוץ לתחום וחזרו אפילו במזיד לא הפסידו את מקומן מ"ט אנוסין נינהו}
\textblock{איתיביה רב יוסף בר שמעיה לרב פפא ר' נחמיה ור' אליעזר בן יעקב אומרים לעולם אסורין עד שיחזרו למקומן שוגגין בשוגג אין במזיד לא}
\textblock{}
\textblock{תנאי היא דתניא פירות שיצאו חוץ לתחום בשוגג יאכלו במזיד לא יאכלו}
\newchap{פרק \hebrewnumeral{4}\quad מי שהוציאוהו}
\newsection{דף מב}
\textblock{}
\textblock{רבי נחמיה אומר במקומן יאכלו שלא במקומן לא יאכלו}
\textblock{מאי במקומן אילימא במקומן במזיד והא קתני בהדיא רבי נחמיה ורבי אליעזר בן יעקב אומרים לעולם אסורין עד שיחזרו למקומן שוגגין בשוגג אין במזיד לא}
\textblock{אלא לאו במקומן בשוגג וחסורי מחסרא והכי קתני פירות שיצאו חוץ לתחום בשוגג יאכלו במזיד לא יאכלו}
\textblock{במה דברים אמורים שלא במקומן אבל במקומן אפילו במזיד יאכלו ואתא ר' נחמיה למימר אפי' במקומן נמי בשוגג אין במזיד לא}
\textblock{לא במזיד במקומן דכולי עלמא לא פליגי דאסור והכא בשוגג שלא במקומן פליגי תנא קמא סבר בשוגג שרי שלא במקומן ורבי נחמיה סבר אפילו שוגג במקומן אין שלא במקומן לא}
\textblock{והא מדקתני סיפא רבי נחמיה ורבי אליעזר בן יעקב אומרים לעולם אסורין עד שיחזרו למקומן שוגגין שוגג אין במזיד לא מכלל דת"ק סבר במזיד נמי שרי שמע מינה:}
\textblock{אמר רב נחמן אמר שמואל היה מהלך ואינו יודע תחום שבת מהלך אלפים פסיעות בינוניות וזו היא תחום שבת}
\textblock{ואמר רב נחמן אמר שמואל שבת בבקעה והקיפוה נכרים מחיצה בשבת מהלך אלפים אמה ומטלטל בכולה על ידי זריקה}
\textblock{ורב הונא אמר מהלך אלפים אמה ומטלטל ד' אמות וניטלטל בכולה על ידי זריקה}
\textblock{שמא ימשך אחר חפצו}
\textblock{באלפים מיהת ליטלטל כי אורחיה}
\textblock{משום דהוי כמחיצה שנפרצה במלואה למקום האסור לה}
\textblock{חייא בר רב אמר מהלך אלפים אמה ומטלטל באלפים אמה כמאן דלא כרב נחמן ולא כרב הונא}
\textblock{אימא מטלטל בארבע אי הכי היינו דרב הונא אימא וכן אמר רבי חייא בר רב}
\textblock{א"ל רב נחמן לרב הונא לא תיפלוג עליה דשמואל דתניא כוותיה דתניא}
\textblock{היה מודד ובא וכלתה מדתו בחצי העיר מותר לטלטל בכל העיר כולה ובלבד שלא יעבור את התחום ברגליו במאי מטלטל לאו על ידי זריקה}
\textblock{אמר רב הונא לא על ידי משיכה}
\textblock{אמר רב הונא היה מודד ובא וכלתה מדתו בחצי חצר אין לו אלא חצי חצר}
\textblock{פשיטא אימא יש לו חצי חצר}
\textblock{האי נמי פשיטא מהו דתימא ליחוש דלמא אתי לטלטולי בכולה קמ"ל}
\textblock{אמר רב נחמן מודה לי הונא היה מודד ובא וכלתה מדתו על שפת תקרה מותר לטלטל בכל הבית}
\textblock{מאי טעמא הואיל ותקרת הבית חובטת}
\textblock{אמר רב הונא בריה דרב נתן כתנאי הוליכוהו לעיר אחרת ונתנוהו בדיר או בסהר רבן גמליאל ורבי אלעזר בן עזריה אומרים מהלך את כולה ורבי יהושע ורבי עקיבא אומרים אין לו אלא ארבע אמות}
\textblock{מאי לאו רבן גמליאל ורבי אלעזר בן עזריה דאמרו מהלך את כולה דלא גזרי הילוך דיר וסהר אטו הילוך בבקעה}
\textblock{ומדהילוך אטו הילוך לא גזרי טלטול אטו הילוך לא גזרי}
\textblock{ור' יהושע ור' עקיבא דאומרים אין לו אלא ארבע אמות דגזרי הילוך דיר וסהר אטו הילוך דבקעה ומדהילוך אטו הילוך גזרי טלטול אטו הילוך נמי גזרי}
\textblock{ממאי דילמא כי לא גזרי רבן גמליאל ורבי אלעזר בן עזריה הילוך סהר ודיר אטו הילוך בקעה הני מילי התם דשני מקומות הן}
\textblock{אבל טלטול אטו הילוך דמקום אחד הוא ה"נ דגזרי גזירה שמא ימשך אחר חפצו}
\textblock{ורבי יהושע ור' עקיבא נמי ממאי דמשום דגזרי הוא דילמא משום דקא סברי כי אמרינן כל הבית כולו כארבע אמות דמי הני מילי היכא דשבת באויר מחיצות מבעוד יום}
\textblock{אבל היכא דלא שבת באויר מחיצות מבעו"י לא}
\textblock{אמר רב הלכתא כרבן גמליאל בדיר וסהר וספינה ושמואל אמר הלכתא כרבן גמליאל בספינה אבל בדיר וסהר לא}
\textblock{דכולי עלמא מיהת הלכה כרבן גמליאל בספינה מאי טעמא}
\textblock{אמר רבה הואיל ושבת באויר מחיצות מבעוד יום}
\textblock{ר' זירא אמר הואיל וספינה נוטלתו מתחילת ארבע ומנחתו בסוף ארבע}
\textblock{מאי בינייהו איכא בינייהו שנפחתו דופני ספינה אי נמי בקופץ מספינה לספינה}
\textblock{ורבי זירא מאי טעמא לא אמר כרבה אמר לך מחיצות}
\newsection{דף מג}
\textblock{להבריח מים עשויות}
\textblock{ורבה מאי טעמא לא אמר כרבי זירא במהלכת כולי עלמא לא פליגי כי פליגי בשעמדה}
\textblock{אמר רב נחמן בר יצחק מתני' נמי דיקא דבמהלכת לא פליגי ממאי מדקתני מעשה שבאו מפלנדרסין והפליגה ספינתם בים רבן גמליאל ורבי אלעזר בן עזריה הלכו את כולה ורבי יהושע ורבי עקיבא לא זזו מארבע אמות שרצו להחמיר על עצמן}
\textblock{אי אמרת בשלמא במהלכת לא פליגי היינו דקתני רצו דילמא עמדה}
\textblock{אלא אי אמרת פליגי האי רצו להחמיר איסורא הוא}
\textblock{אמר רב אשי מתניתין נמי דיקא דקתני ספינה דומיא דדיר וסהר מה דיר וסהר דקביעי אף ספינה נמי דקביעא}
\textblock{אמר ליה רב אחא בריה דרבא לרב אשי הלכתא כרבן גמליאל בספינה הלכתא מכלל דפליגי}
\textblock{אין והתניא חנניא (בן אחי רבי יהושע) אומר כל אותו היום ישבו ודנו בדבר הלכה אמש הכריע אחי אבא הלכה כרבן גמליאל בספינה והלכה כרבי עקיבא בדיר וסהר:}
\textblock{בעי רב חנניא יש תחומין למעלה מעשרה או אין תחומין למעלה מעשרה}
\textblock{עמוד גבוה עשרה ורחב ארבעה לא תיבעי לך דארעא סמיכתא היא}
\textblock{כי תיבעי לך בעמוד גבוה עשרה ואינו רחב ארבעה אי נמי דקאזיל בקפיצה}
\textblock{לישנא אחרינא בספינה מאי}
\textblock{אמר רב הושעיא ת"ש מעשה שבאו מפלנדרסין והפליגה ספינתם בים וכו' אי אמרת בשלמא יש תחומין משום הכי רצו אלא אי אמרת אין תחומין אמאי רצו}
\textblock{כדאמר רבא במהלכת ברקק הכא נמי במהלכת ברקק}
\textblock{תא שמע פעם אחת לא נכנסו לנמל עד שחשיכה וכו' אי אמרת בשלמא יש תחומין שפיר אלא אי אמרת אין תחומין כי לא היינו בתוך התחום מאי הוי}
\textblock{אמר רבא במהלכת ברקק}
\textblock{תא שמע הני שב שמעתא דאיתאמרן בצפר' בשבתא קמיה דרב חסדא בסורא בהדי פניא בשבתא קמיה דרבא בפומבדיתא}
\textblock{מאן אמרינהו לאו אליהו אמרינהו אלמא אין תחומין למעלה מעשרה לא דלמא יוסף שידא אמרינהו}
\textblock{תא שמע הריני נזיר ביום שבן דוד בא מותר לשתות יין בשבתות ובימים טובים}
\textblock{ואסור לשתות יין כל ימות החול}
\textblock{אי אמרת בשלמא יש תחומין היינו דבשבתות ובימים טובים מותר אלא אי אמרת אין תחומין בשבתות ובימים טובים אמאי מותר}
\textblock{שאני התם דאמר קרא (מלאכי ג, כג) הנה אנכי שולח לכם את אליה הנביא וגו' והא לא אתא אליהו מאתמול}
\textblock{אי הכי בחול כל יומא ויומא נמי לישתרי דהא לא אתא אליהו מאתמול אלא אמרינן לבית דין הגדול אתא הכא נמי לימא לבית דין הגדול אתא}
\textblock{כבר מובטח להן לישראל שאין אליהו בא לא בערבי שבתות ולא בערבי ימים טובים מפני הטורח}
\textblock{קא סלקא דעתך מדאליהו לא אתא משיח נמי לא אתי במעלי שבתא לישתרי אליהו לא אתי משיח אתי דכיון דאתי משיחא הכל עבדים הן לישראל}
\textblock{בחד בשבא לישתרי לפשוט מינה דאין תחומין דאי יש תחומין בחד בשבא לישתרי דלא אתא אליהו בשבת}
\textblock{האי תנא ספוקי מספקא ליה אי יש תחומין או אין תחומין ולחומרא}
\textblock{דקאי אימת דקא נדר אילימא דקאי בחול כיון דחל עליה נזירות היכי אתיא שבתא ומפקעא ליה}
\textblock{אלא דקאי בשבתא וקא נדר וביום טוב וקא נדר וההוא יומא דשרי ליה מיכן ואילך אסיר ליה:}
\textblock{פעם אחת לא נכנסו לנמל וכו':}
\textblock{תנא שפופרת היתה לו לרבן גמליאל שהיה מביט וצופה בה אלפים אמה ביבשה וכנגדה אלפים בים}
\textblock{הרוצה לידע כמה עומקו של גיא מביא שפופרת ומביט בה וידע כמה עומקו של גיא}
\textblock{והרוצה לידע כמה גובהו של דקל מודד קומתו וצלו וצל קומתו וידע כמה גובה של דקל}
\textblock{הרוצה שלא תשרה חיה רעה בצל קבר נועץ קנה בד' שעות ביום ויראה להיכן צלו נוטה משפיע ועולה משפיע ויורד}
\textblock{נחמיה בריה דרב חנילאי משכתיה שמעתא ונפק חוץ לתחום אמר ליה רב חסדא לרב נחמן נחמיה תלמידך שרוי בצער}
\textblock{אמר לו עשה לו מחיצה של בני אדם ויכנס}
\textblock{יתיב רב נחמן בר יצחק אחוריה דרבא ויתיב רבא קמיה דרב נחמן א"ל רב נחמן בר יצחק לרבא מאי קא מבעיא ליה לרב חסדא}
\textblock{אילימא בדמלו גברי עסקינן וקא מבעיא ליה הלכתא כרבן גמליאל}
\newsection{דף מד}
\textblock{או אין הלכה כר"ג או דילמא בדלא מלו גברי עסקינן וקא מבעיא ליה הלכה כרבי אליעזר או אין הלכה כר"א}
\textblock{פשיטא בדלא מלו גברי עסקינן דאי סלקא דעתך בדמלו גברי עסקינן מאי תיבעי ליה האמר רב הלכה כר"ג בדיר וסהר וספינה אלא ודאי בדלא מלו גברי עסקינן ודר' אליעזר קמיבעיא ליה}
\textblock{דיקא נמי דקאמר ליה יכנס מאי יכנס לאו בלא מחיצה}
\textblock{איתיביה ר"נ בר יצחק לרבא נפל דופנה לא יעמיד בה אדם בהמה וכלים ולא יזקוף את המטה לפרוס עליה סדין לפי שאין עושין אהל עראי בתחילה ביו"ט ואין צריך לומר בשבת}
\textblock{א"ל את אמרת לי מהא ואנא אמינא לך מהא עושה אדם את חבירו דופן כדי שיאכל וישתה וישן ויזקוף את המטה ויפרוס עליה סדין כדי שלא תפול חמה על המת ועל האוכלין}
\textblock{קשיין אהדדי ל"ק הא ר"א הא רבנן דתנן פקק החלון ר"א אומר בזמן שקשור ותלוי פוקקין בו ואם לאו אין פוקקין בו וחכ"א בין כך ובין כך פוקקין בו}
\textblock{והא איתמר עלה אמר רבה בר בר חנה א"ר יוחנן הכל מודים שאין עושין אהל עראי בתחילה ביום טוב וא"צ לומר בשבת לא נחלקו אלא להוסיף שר' אליעזר אומר אין מוסיפין ביום טוב ואין צריך לומר בשבת וחכמים אומרים מוסיפין בשבת ואין צריך לומר ביום טוב}
\textblock{אלא ל"ק הא כרבי מאיר הא כרבי יהודה דתניא עשאה לבהמה דופן לסוכה רבי מאיר פוסל ור' יהודה מכשיר}
\textblock{רבי מאיר דקא פסיל התם אלמא לא מחיצה היא הכא שרי דלאו מידי קא עביד}
\textblock{ורבי יהודה דקא מכשיר התם אלמא מחיצה היא הכא אסר}
\textblock{ותיסברא אימר דשמעת ליה לרבי מאיר בהמה אדם וכלים מי שמעת ליה}
\textblock{ותו רבי מאיר אליבא דמאן אי אליבא דרבי אליעזר להוסיף נמי אסר}
\textblock{אלא אליבא דרבנן אימר דאמרי רבנן להוסיף לכתחילה מי אמור}
\textblock{אלא הא והא רבנן וכלים אכלים ל"ק הא בדופן שלישית הא בדופן רביעית}
\textblock{דיקא נמי דקתני נפל דופנה שמע מינה}
\textblock{אלא אדם אאדם קשיא}
\textblock{אדם אאדם נמי לא קשיא כאן לדעת כאן שלא מדעת}
\textblock{והא דרבי נחמיה בריה דרבי חנילאי לדעת הוה שלא מדעת הוה}
\textblock{רב חסדא מיהא לדעת הוה רב חסדא שלא מן המנין הוה:}
\textblock{הנהו בני גננא דאעילו מיא במחיצה של בני אדם נגדינהו שמואל אמר אם אמרו שלא מדעת יאמרו לדעת}
\textblock{הנהו זיקי דהוה שדיין בריסתקא דמחוזא בהדי דאתא רבא מפירקיה אעלינהו ניהליה לשבתא אחריתי בעי עיילינהו ואסר להו דהוה ליה כלדעת ואסור}
\textblock{לוי אעילו ליה תיבנא זעירי אספסתא רב שימי בר חייא מיא:}
\textblock{{\large\emph{מתני׳}} מי שיצא ברשות ואמרו לו כבר נעשה מעשה יש לו אלפים אמה לכל רוח}
\textblock{אם היה בתוך התחום כאילו לא יצא כל היוצאים להציל חוזרין למקומן:}
\textblock{{\large\emph{גמ׳}} מאי אם היה בתוך התחום כאילו לא יצא אמר רבה הכי קאמר אם היה בתוך תחום שלו כאילו לא יצא מתוך ביתו דמי}
\textblock{פשיטא מהו דתימא הואיל ועקר עקר קמ"ל}
\textblock{רב שימי בר חייא אמר הכי קאמר אם היו תחומין שנתנו לו חכמים מובלעין בתוך התחום שלו כאילו לא יצא מתחומו}
\textblock{במאי קמיפלגי מר סבר הבלעת תחומין מילתא היא ומר סבר לאו מילתא היא}
\textblock{א"ל אביי לרבה ואת לא תסברא דהבלעת תחומין מילתא היא ומה אילו שבת במערה שבתוכה ארבעת אלפים ועל גגה פחות מארבעת אלפים אמה לא נמצא מהלך את כולה וחוצה לה אלפים אמה}
\textblock{אמר ליה ולא שני לך בין היכא דשבת באויר מחיצות מבעוד יום להיכא דלא שבת באויר מחיצות מבעוד יום}
\textblock{והיכא דלא שבת לא}
\newsection{דף מה}
\textblock{והתנן ר"א אומר שתים יכנס ג' לא יכנס מאי לאו רבי אליעזר לטעמיה דאמר והוא באמצען}
\textblock{וארבע אמות דיהבו ליה רבנן כמאן דמיבלען דמו וקאמר יכנס אלמא הבלעת תחומין מילתא היא}
\textblock{א"ל רבה בר בר חנה לאביי ומדר"א קמותבת ליה למר אמר ליה אין דשמיע לי מיניה דמר עד כאן לא פליגי רבנן עליה דר"א אלא לדבר הרשות אבל לדבר מצוה מודו ליה:}
\textblock{וכל היוצאין להציל חוזרין למקומן: ואפי' טובא והא אמרת רישא אלפים אמה ותו לא}
\textblock{אמר רב יהודה אמר רב שחוזרין בכלי זיין למקומן ומאי קושיא דילמא להציל שאני}
\textblock{אלא אי קשיא הא קשיא דתנן בראשונה לא היו זזין משם כל היום כולו}
\textblock{התקין ר"ג הזקן שיש להן אלפים אמה לכל רוח ולא אלו בלבד אמרו אלא אפי' חכמה הבאה לילד והבא להציל מן הגייס ומן הנהר ומן המפולת ומן הדליקה הרי הן כאנשי העיר ויש להן אלפים אמה לכל רוח}
\textblock{ותו לא והא אמרת כל היוצאין להציל חוזרין למקומן אפילו טובא}
\textblock{אמר רב [יהודה אמר רב] שחוזרין בכלי זיין למקומן כדתניא בראשונה היו מניחין כלי זיינן בבית הסמוך לחומה}
\textblock{פעם אחת הכירו בהן אויבים ורדפו אחריהם ונכנסו ליטול כלי זיינן ונכנסו אויבים אחריהן דחקו זה את זה והרגו זה את זה יותר ממה שהרגו אויבים באותה שעה התקינו שיהו חוזרין למקומן בכלי זיינן}
\textblock{רב נחמן בר יצחק אמר ל"ק כאן שנצחו ישראל את אומות העולם כאן שנצחו אומות העולם את עצמן}
\textblock{אמר רב יהודה אמר רב נכרים שצרו על עיירות ישראל אין יוצאין עליהם בכלי זיינן ואין מחללין עליהן את השבת}
\textblock{תניא נמי הכי נכרים שצרו וכו' במה דברים אמורים כשבאו על עסקי ממון אבל באו על עסקי נפשות יוצאין עליהן בכלי זיינן ומחללין עליהן את השבת}
\textblock{ובעיר הסמוכה לספר אפילו לא באו על עסקי נפשות אלא על עסקי תבן וקש יוצאין עליהן בכלי זיינן ומחללין עליהן את השבת}
\textblock{אמר רב יוסף בר מניומי אמר רב נחמן ובבל כעיר הסמוכה לספר דמיא ותרגומא נהרדעא}
\textblock{דרש רבי דוסתאי דמן בירי מאי דכתיב (שמואל א כג, א) ויגידו לדוד לאמר הנה פלשתים נלחמים בקעילה והמה שוסים את הגרנות}
\textblock{תנא קעילה עיר הסמוכה לספר היתה והם לא באו אלא על עסקי תבן וקש דכתיב והמה שוסים את הגרנות וכתיב (שמואל א כג, ב) וישאל דוד בה' לאמר האלך והכיתי בפלשתים האלה ויאמר ה' אל דוד לך והכית בפלשתים והושעת את קעילה}
\textblock{מאי קמבעיא ליה אילימא אי שרי אי אסור הרי בית דינו של שמואל הרמתי קיים}
\textblock{אלא אי מצלח אי לא מצלח דיקא נמי דכתיב לך והכית בפלשתים והושעת את קעילה ש"מ:}
\textblock{{\large\emph{מתני׳}} מי שישב בדרך ועמד וראה הרי (זה) הוא סמוך לעיר [הואיל] ולא היתה כוונתו לכך לא יכנס דברי רבי מאיר}
\textblock{ר' יהודה אומר יכנס א"ר יהודה מעשה היה ונכנס רבי טרפון בלא מתכוין:}
\textblock{{\large\emph{גמ׳}} תניא א"ר יהודה מעשה ברבי טרפון שהיה מהלך בדרך וחשכה לו ולן חוץ לעיר לשחרית מצאוהו רועי בקר אמרו לו רבי הרי העיר לפניך הכנס נכנס וישב בבית המדרש ודרש כל היום כולו}
\textblock{(אמרו לו) משם ראייה שמא בלבו היתה או בית המדרש מובלע בתוך תחומו היה:}
\textblock{{\large\emph{מתני׳}} מי שישן בדרך ולא ידע שחשיכה יש לו אלפים אמה לכל רוח דברי ר' יוחנן בן נורי}
\textblock{וחכמים אומרים אין לו אלא ארבע אמות ר"א אומר והוא באמצען}
\textblock{ר' יהודה אומר לאיזה רוח שירצה ילך ומודה ר' יהודה שאם בירר לו שאינו יכול לחזור בו}
\textblock{היו שנים מקצת אמותיו של זה בתוך אמותיו של זה מביאין ואוכלין באמצע}
\textblock{ובלבד שלא יוציא זה מתוך שלו לתוך של חברו}
\textblock{היו שלשה והאמצעי מובלע ביניהן הוא מותר עמהן והן מותרין עמו ושנים החיצונים אסורין זה עם זה}
\textblock{אמר רבי שמעון למה הדבר דומה לשלש חצירות הפתוחות זו לזו ופתוחות לרשות הרבים עירבו שתים עם האמצעית היא מותרת עמהן והם מותרות עמה ושתים החיצונות אסורות זו עם זו:}
\textblock{{\large\emph{גמ׳}} בעי רבא מאי קסבר ר' יוחנן בן נורי מסבר קא סבר חפצי הפקר קונין שביתה}
\textblock{ובדין הוא דליפלוג בכלים והא דקמיפלגי באדם להודיעך כוחן דרבנן דאע"ג דאיכא למימר הואיל וניעור קנה ישן נמי קנה קמ"ל דלא}
\textblock{או דילמא קסבר ר' יוחנן בן נורי בעלמא חפצי הפקר אין קונין שביתה והכא היינו טעמא הואיל וניעור קנה ישן נמי קנה}
\textblock{אמר רב יוסף ת"ש גשמים שירדו מעי"ט יש להן אלפים אמה לכל רוח בי"ט הרי הן כרגלי כל אדם}
\textblock{אי אמרת בשלמא קסבר רבי יוחנן בן נורי חפצי הפקר קונין שביתה הא מני ר' יוחנן היא}
\textblock{אלא אי אמרת חפצי הפקר אין קונין שביתה הא מני לא רבי יוחנן ולא רבנן}
\textblock{יתיב אביי וקאמר לה להא שמעתא א"ל רב ספרא לאביי ודילמא בגשמים הסמוכין לעיר עסקינן ואנשי אותה העיר דעתם עילייהו}
\textblock{אמר ליה לא ס"ד דתנן בור של יחיד כרגלי יחיד ושל אותה העיר כרגלי אותה העיר ושל עולי בבל כרגלי הממלא}
\textblock{ותניא בור של שבטים יש להן אלפים אמה לכל רוח קשיין אהדדי}
\textblock{אלא לאו ש"מ הא רבי יוחנן בן נורי הא רבנן}
\textblock{כי אתא לקמיה דרב יוסף א"ל הכי קאמר רב ספרא והכי אהדרי ליה אמר ליה ואמאי לא תימא ליה מגופה אי סלקא דעתך גשמים הסמוכין לעיר עסקינן האי יש להן אלפים אמה לכל רוח}
\textblock{הא כרגלי אנשי אותה העיר מיבעיא ליה:}
\textblock{אמר מר ביום טוב הרי הן כרגלי כל אדם ואמאי ליקני שביתה באוקיינוס}
\textblock{לימא דלא כרבי אליעזר דאי כר' אליעזר הא אמר כל העולם כולו ממי אוקיינוס הוא שותה}
\textblock{אמר ר' יצחק הכא בעבים שנתקשרו מערב יום טוב עסקינן}
\textblock{ודילמא הנך אזלי והנך אחריני נינהו דאית להו סימנא בגוייהו}
\textblock{ואיבעית אימא הוי ספק דדבריהם וספק דדבריהם להקל}
\textblock{וליקני שביתה בעבים תיפשוט מינה דאין תחומין למעלה מי' דאי יש תחומין ליקני שביתה בעבים}
\textblock{לעולם אימא לך יש תחומין ומיא בעיבא מיבלע בליעי}
\newsection{דף מו}
\textblock{כל שכן דהוו להו נולד דאסירי}
\textblock{אלא מיא בעבים מינד ניידי השתא דאתית להכי אוקיינוס נמי לא ליקשו לך מיא באוקיינוס נמי מינד ניידי ותניא נהרות המושכין ומעיינות הנובעין הרי הן כרגלי כל אדם}
\textblock{אמר רבי יעקב בר אידי אמר רבי יהושע בן לוי הלכה כרבי יוחנן בן נורי אמר ליה רבי זירא לרבי יעקב בר אידי בפירוש שמיע לך או מכללא שמיע לך אמר ליה בפירוש שמיע לי}
\textblock{מאי כללא דאמר רבי יהושע בן לוי הלכה כדברי המיקל בעירוב}
\textblock{ותרתי למה לי}
\textblock{אמר רבי זירא צריכי דאי אשמעינן הלכה כר' יוחנן בן נורי הוה אמינא בין לקולא ובין לחומרא קמ"ל הלכה כדברי המיקל בעירוב}
\textblock{ולימא הלכה כדברי המיקל בעירוב הלכה כרבי יוחנן בן נורי למה לי}
\textblock{איצטריך ס"ד אמינא הני מילי יחיד במקום יחיד ורבים במקום רבים אבל יחיד במקום רבים אימא לא}
\textblock{אמר ליה רבא לאביי מכדי עירובין דרבנן מה לי יחיד במקום יחיד ומה לי יחיד במקום רבים}
\textblock{אמר ליה רב פפא לרבא ובדרבנן לא שני לן בין יחיד במקום יחיד ליחיד במקום רבים}
\textblock{והתנן רבי אלעזר אומר כל אשה שעברו עליה שלש עונות דייה שעתה}
\textblock{ותניא מעשה ועשה רבי כר' אלעזר לאחר שנזכר אמר כדי הוא רבי אלעזר לסמוך עליו בשעת הדחק}
\textblock{מאי לאחר שנזכר אילימא לאחר שנזכר דאין הלכה כרבי אלעזר אלא כרבנן בשעת הדחק היכי עביד כוותיה}
\textblock{אלא דלא איתמר הלכתא לא כרבי אלעזר ולא כרבנן לאחר שנזכר דלאו יחיד פליג עליה אלא רבים פליגי עליה אמר כדי הוא רבי אלעזר לסמוך עליו בשעת הדחק}
\textblock{אמר רב משרשיא לרבא ואמרי לה רב נחמן בר יצחק לרבא ובדרבנן לא שני בין יחיד במקום יחיד בין יחיד במקום רבים}
\textblock{והתניא שמועה קרובה נוהגת שבעה ושלשים רחוקה אינה נוהגת אלא יום אחד}
\textblock{ואי זו היא קרובה ואיזו היא רחוקה בתוך שלשים קרובה לאחר שלשים רחוקה דברי רבי עקיבא וחכמים אומרים אחת שמועה קרובה ואחת שמועה רחוקה נוהגת שבעה ושלשים}
\textblock{ואמר רבה בר בר חנה אמר רבי יוחנן כל מקום שאתה מוצא יחיד מיקל ורבים מחמירין הלכה כדברי המחמירין המרובים חוץ מזו שאע"פ שרבי עקיבא מיקל וחכמים מחמירין הלכה כדברי רבי עקיבא}
\textblock{וסבר לה כשמואל דאמר שמואל הלכה כדברי המיקל באבל}
\textblock{באבילות הוא דאקילו בה רבנן אבל בעלמא אפילו בדרבנן שני בין יחיד במקום יחיד בין יחיד במקום רבים}
\textblock{ורב פפא אמר איצטריך ס"ד אמינא הני מילי בעירובי חצירות אבל בעירובי תחומין אימא לא צריכא}
\textblock{ומנא תימרא דשני לן בין עירובי חצירות לעירובי תחומין דתנן א"ר יהודה במה דברים אמורים בעירובי תחומין אבל בעירובי חצירות מערבין בין לדעת ובין שלא לדעת שזכין לאדם שלא בפניו ואין חבין לאדם אלא בפניו}
\textblock{רב אשי אמר איצטריך ס"ד אמינא הני מילי בשיורי עירוב אבל בתחילת עירוב אימא לא}
\textblock{ומנא תימרא דשני לן בין שיורי עירוב לתחילת עירוב דתנן א"ר יוסי במה דברים אמורים בתחילת עירוב אבל בשיורי עירוב אפילו כל שהוא}
\textblock{ולא אמרו לערב חצירות אלא כדי שלא לשכח תורת עירוב מן התינוקות}
\textblock{רבי יעקב ורבי זריקא אמרו הלכה כרבי עקיבא מחבירו וכרבי יוסי מחבריו וכרבי מחבירו}
\textblock{למאי הלכתא רבי אסי אמר הלכה ורבי חייא בר אבא אמר מטין ור' יוסי בר' חנינא אמר נראין}
\textblock{כלשון הזה א"ר יעקב בר אידי אמר ר' יוחנן ר' מאיר ור' יהודה הלכה כרבי יהודה רבי יהודה ורבי יוסי הלכה כרבי יוסי ואצ"ל ר"מ ור' יוסי הלכה כרבי יוסי השתא במקום רבי יהודה ליתא במקום רבי יוסי מיבעיא}
\textblock{אמר רב אסי אף אני לומד רבי יוסי ור' שמעון הלכה כרבי יוסי דאמר רבי אבא אמר רבי יוחנן רבי יהודה ורבי שמעון הלכה כר' יהודה השתא במקום רבי יהודה ליתא במקום רבי יוסי מיבעיא}
\textblock{איבעיא להו ר"מ ור"ש מאי תיקו}
\textblock{אמר רב משרשיא ליתנהו להני כללי מנא ליה לרב משרשיא הא}
\textblock{אילימא מהא דתנן ר"ש אומר למה הדבר דומה לג' חצירות הפתוחות זו לזו ופתוחות לרשות הרבים עירבו שתים החיצונות עם האמצעית היא מותרת עמהן והן מותרות עמה ושתים החיצונות אסורות זו עם זו}
\textblock{ואמר רב חמא בר גוריא אמר רב הלכה כרבי שמעון ומאן פליג עליה רבי יהודה והא אמרת רבי יהודה ורבי שמעון הלכה כרבי יהודה אלא לאו ש"מ ליתנהו}
\textblock{ומאי קושיא דילמא היכא דאיתמר איתמר היכא דלא איתמר לא איתמר}
\textblock{אלא מהא דתנן עיר של יחיד ונעשית של רבים מערבין את כולה של רבים ונעשית של יחיד אין מערבין את כולה אלא אם כן עושה חוצה לה כעיר חדשה שביהודה שיש בה חמשים דיורין דברי רבי יהודה}
\textblock{רבי שמעון אומר}
\newsection{דף מז}
\textblock{שלש חצירות של שני בתים ואמר רב חמא בר גוריא אמר רב הלכה כרבי שמעון ומאן פליג עליה רבי יהודה והא אמרת רבי יהודה ור"ש הלכה כרבי יהודה}
\textblock{ומאי קושיא דילמא הכא נמי היכא דאיתמר איתמר היכא דלא איתמר לא איתמר}
\textblock{אלא מהא דתנן המניח את ביתו והלך לשבות בעיר אחרת אחד נכרי ואחד ישראל אוסר לבני חצירות דברי רבי מאיר}
\textblock{רבי יהודה אומר אינו אוסר רבי יוסי אומר נכרי אוסר ישראל אינו אוסר מפני שאין דרך ישראל לבא בשבת רבי שמעון אומר אפילו הניח את ביתו והלך לשבות אצל בתו באותה העיר אינו אוסר שכבר הסיח דעתו}
\textblock{ואמר רב חמא בר גוריא אמר רב הלכה כר"ש ומאן פליג עליה ר"י והא אמרת רבי יהודה ור"ש הלכה כרבי יהודה}
\textblock{ומאי קושיא דלמא הכא נמי היכא דאיתמר איתמר היכא דלא איתמר לא איתמר}
\textblock{אלא מהא דתנן וזהו שאמרו העני מערב ברגליו רבי מאיר אומר אנו אין לנו אלא עני}
\textblock{רבי יהודה אומר אחד עני ואחד עשיר לא אמרו מערבין בפת אלא להקל על העשיר שלא יצא ויערב ברגליו}
\textblock{ומתני ליה רב חייא בר אשי לחייא בר רב קמיה דרב אחד עני ואחד עשיר ואמר ליה רב סיים בה נמי הלכה כרבי יהודה}
\textblock{תרתי למה לי והא אמרת ר"מ ורבי יהודה הלכה כרבי יהודה}
\textblock{ומאי קושיא דילמא רב לית ליה להני כללי}
\textblock{אלא מהא דתנן היבמה לא תחלוץ ולא תתייבם עד שיהו לה שלשה חדשים}
\textblock{וכן שאר כל הנשים לא ינשאו ולא יתארסו עד שיהו להן שלשה חדשים אחד בתולות ואחד בעולות אחד אלמנות ואחד גרושות אחד ארוסות ואחד נשואות}
\textblock{ר' יהודה אומר נשואות יתארסו}
\textblock{וארוסות ינשאו חוץ מארוסה שביהודה מפני שלבו גס בה}
\textblock{ר' יוסי אומר כל הנשים יתארסו חוץ מן האלמנה מפני האיבול}
\textblock{ואמרינן רבי (אליעזר) לא על לבי מדרשא אשכחיה לרבי אסי דהוה קאים אמר ליה מאי אמור בבי מדרשא אמר ליה הכי אמר רבי יוחנן הלכה כרבי יוסי מכלל דיחידאה פליג עליה}
\textblock{אין והתניא הרי שהיתה רדופה לילך לבית אביה או שהיתה לה כעס עם בעלה או שהיה בעלה זקן או חולה או שהיתה היא חולה עקרה זקנה קטנה ואיילונית ושאינה ראויה לילד או שהיה בעלה חבוש בבית האסורין המפלת לאחר מיתת בעלה כולן צריכין להמתין ג' חדשים דברי ר"מ רבי יוסי מתיר ליארס ולינשא מיד}
\textblock{למה לי והא אמרת ר"מ ורבי יוסי הלכה כר' יוסי}
\textblock{ומאי קושיא דלמא לאפוקי מדרב נחמן אמר שמואל דאמר הלכה כרבי מאיר בגזירותיו}
\textblock{אלא מהא דתניא הולכין ליריד של נכרים ולוקחים מהן בהמה ועבדים ושפחות בתים שדות וכרמים וכותב ומעלה בערכאות שלהן מפני שהוא כמציל מידן}
\textblock{ואם היה כהן מטמא בחוצה לארץ לדון ולערער עמהן וכשם שמטמא בחוצה לארץ כך מטמא בבית הקברות}
\textblock{בית הקברות ס"ד טומאה דאורייתא היא}
\textblock{אלא בבית הפרס דרבנן}
\textblock{ומטמא לישא אשה וללמוד תורה אמר רבי יהודה אימתי בזמן שאין מוצא ללמוד אבל מוצא ללמוד לא יטמא}
\textblock{ר' יוסי אומר אף בזמן שמוצא ללמוד נמי יטמא לפי}
\textblock{שאין מן הכל זוכה אדם ללמוד ואמר ר' יוסי מעשה ביוסף הכהן שהלך אצל רבו לצידן ללמוד תורה}
\textblock{ואמר ר' יוחנן הלכה כרבי יוסי ולמה לי והא אמרת רבי יהודה ור' יוסי הלכה כר' יוסי}
\textblock{אמר אביי איצטריך סד"א הני מילי במתני' אבל בברייתא אימא לא קמ"ל}
\textblock{אלא הכי קאמר הני כללי לאו ד"ה נינהו דהא רב לית ליה הני כללי:}
\textblock{אמר רב יהודה אמר שמואל חפצי נכרי אין קונין שביתה}
\textblock{למאן אילימא לרבנן פשיטא השתא חפצי הפקר דלית להו בעלים אין קונין שביתה חפצי הנכרי דאית להו בעלים מיבעיא}
\textblock{אלא אליבא דר' יוחנן בן נורי וקמ"ל אימר דאמר ר' יוחנן בן נורי קונין שביתה הני מילי חפצי הפקר דלית להו בעלים אבל חפצי הנכרי דאית להו בעלים לא}
\textblock{מיתיבי ר"ש בן אלעזר אומר השואל כלי מן הנכרי ביום טוב וכן המשאיל לו לנכרי כלי מעיו"ט והחזירו לו ביום טוב והכלים והאוצרות ששבתו בתוך התחום יש להן אלפים אמה לכל רוח ונכרי שהביא לו פירות מחוץ לתחום הרי זה לא יזיזם ממקומן}
\textblock{אי אמרת בשלמא קסבר רבי יוחנן בן נורי חפצי נכרי קונין שביתה הא מני ר' יוחנן בן נורי היא}
\textblock{אלא אי אמרת קסבר רבי יוחנן בן נורי חפצי הנכרי אין קונין שביתה הא מני לא ר' יוחנן בן נורי ולא רבנן}
\textblock{לעולם קסבר רבי יוחנן בן נורי חפצי הנכרי קונין שביתה ושמואל דאמר כרבנן ודקאמרת לרבנן פשיטא מהו דתימא גזירה בעלים דנכרי אטו בעלים דישראל קמ"ל}
\textblock{ורב חייא בר אבין אמר רבי יוחנן חפצי נכרי קונין שביתה גזירה בעלים דנכרי אטו בעלים דישראל}
\textblock{הנהו דכרי דאתו למברכתא שרא להו רבא לבני מחוזא למיזבן מינייהו}
\textblock{א"ל רבינא לרבא מאי דעתיך דאמר רב יהודה אמר שמואל חפצי נכרי אין קונין שביתה}
\textblock{והא שמואל ור' יוחנן הלכה כר' יוחנן ואמר רב חייא בר אבין אמר ר' יוחנן חפצי נכרי קונין שביתה גזירה בעלים דנכרי אטו בעלים דישראל}
\textblock{הדר אמר רבא ליזדבנו לבני מברכתא דכולה מברכתא לדידהו כד' אמות דמיא}
\textblock{תני רבי חייא חרם שבין תחומי שבת צריך}
\newsection{דף מח}
\textblock{מחיצה של ברזל להפסיקו מחייך עליה ר' יוסי בר' חנינא}
\textblock{מאי טעמא קא מחייך אילימא משום דתני לה כר' יוחנן בן נורי לחומרא ואיהו סבירא ליה כרבנן לקולא ומשום דסבר לקולא מאן דתני לחומרא מחייך עלה}
\textblock{אלא משום דתניא נהרות המושכין ומעיינות הנובעין הרי הן כרגלי כל אדם}
\textblock{ודלמא במכונסין}
\textblock{אלא משום דקתני צריך מחיצה של ברזל להפסיקו ומאי שנא קנים דלא דעיילי בהו מיא של ברזל נמי עיילי בהו מיא}
\textblock{ודילמא צריך ואין לו תקנה קאמר}
\textblock{אלא משום דקל הוא שהקילו חכמים במים}
\textblock{כדרבי טבלא דבעא מיניה רבי טבלא מרב מחיצה תלויה מהו שתתיר בחורבה}
\textblock{א"ל אין מחיצה תלויה מתרת אלא במים קל הוא שהקילו חכמים במים:}
\textblock{וחכ"א אין לו אלא ארבע וכו': רבי יהודה היינו ת"ק}
\textblock{אמר רבא שמונה על שמונה איכא בינייהו תנ"ה יש לו שמונה על שמונה דברי ר"מ}
\textblock{ואמר רבא מחלוקת להלך אבל לטלטל דברי הכל ארבע אמות אין טפי לא}
\textblock{והני ד' אמות היכא כתיבא}
\textblock{כדתניא (שמות טז, כט) שבו איש תחתיו כתחתיו [וכמה תחתיו] גופו שלש אמות ואמה כדי לפשוט ידיו ורגליו דברי ר' מאיר ר' יהודה אומר גופו שלש אמות ואמה כדי שיטול חפץ מתחת מרגלותיו ומניח תחת מראשותיו}
\textblock{מאי בינייהו איכא בינייהו ארבע אמות מצומצמות}
\textblock{אמר ליה רב משרשיא לבריה כי עיילת לקמיה דרב פפא בעי מיניה ארבע אמות שאמרו באמה דידיה יהבינן ליה או באמה של קדש יהבינן ליה}
\textblock{אם אמר לך אמות של קדש יהבינן ליה עוג מלך הבשן מה תהא עליו ואם אמר לך באמה דידיה יהבינן ליה אימא ליה מאי טעמא לא קתני לה גבי יש שאמרו הכל לפי מה שהוא אדם}
\textblock{כי אתא לקמיה דרב פפא א"ל אי דייקינן כולא האי לא הוי תנינן}
\textblock{לעולם באמה דידיה יהבינן ליה ודקא קשיא לך מאי טעמא לא קתני גבי יש שאמרו דלא פסיקא ליה משום דאיכא ננס באבריו:}
\textblock{היו שנים מקצת אמותיו של זה וכו': למה ליה למימר למה הדבר דומה}
\textblock{הכי קאמר להו רבי שמעון לרבנן מכדי למה הדבר דומה לשלש חצירות הפתוחות זו לזו ופתוחות לר"ה מאי שנא התם דפליגיתו ומ"ש הכא דלא פליגיתו}
\textblock{ורבנן התם אוושי דיורין הכא לא אוושי דיורין:}
\textblock{ושתים החיצונות כו': ואמאי כיון דערבי להו חיצונות בהדי אמצעית הויא להו חדא}
\textblock{אמר רב יהודה כגון שנתנה אמצעית עירובה בזו ועירובה בזו}
\textblock{ורב ששת אמר אפילו תימא שנתנו עירובן באמצעית כגון שנתנוהו}
\textblock{בשני בתים}
\textblock{כמאן כבית שמאי דתניא חמשה שגבו את עירובן ונתנוהו בשני כלים בית שמאי אומרים אין ערובן עירוב ובית הלל אומרים עירובן עירוב}
\textblock{אפילו תימא בית הלל עד כאן לא קאמרי בית הלל התם אלא בשני כלים בבית אחד אבל בשני בתים לא}
\textblock{א"ל רב אחא בריה דרב אויא לרב אשי לרב יהודה קשיא ולרב ששת קשיא לרב יהודה קשיא דאמר כגון שנתנה אמצעית עירובה בזו ועירובה בזו וכיון דעירבה אמצעית בהדי חיצונה הויא ליה חדא וכי הדרה וערבה בהדי אידך שליחותה עבדה}
\textblock{ולרב ששת קשיא תיהוי כחמשה ששרויין בחצר אחת ושכח אחד מהן ולא עירב דאסרי אהדדי}
\textblock{א"ל רב אשי לא לרב יהודה קשיא ולא לרב ששת קשיא לרב יהודה ל"ק כיון דעירבה לה אמצעית בהדי חיצונה ושתים חיצונות בהדי הדדי לא עירבו גליא דעתיה דבהא ניחא ליה ובהא לא ניחא ליה}
\textblock{ולרב ששת לא קשיא אם אמרו דיורין להקל יאמרו דיורין להחמיר}
\textblock{אמר רב יהודה אמר רב זו דברי ר' שמעון אבל חכמים אומרים רשות אחת משמשת לשתי רשויות אבל לא שתי רשויות משמשות לרשות אחת}
\textblock{כי אמריתה קמיה דשמואל אמר לי}
\newsection{דף מט}
\textblock{אף זו דברי ר' שמעון אבל חכמים אומרים שלשתן אסורות}
\textblock{תניא כוותיה דרב יהודה אליבא דשמואל א"ר שמעון למה הדבר דומה לשלש חצירות הפתוחות זו לזו ופתוחות לרה"ר עירבו שתים עם האמצעית זו מביאה מתוך ביתה ואוכלת וזו מביאה מתוך ביתה ואוכלת זו מחזרת מותרה לתוך ביתה וזו מחזרת מותרה לתוך ביתה}
\textblock{אבל חכמים אומרים שלשתן אסורות}
\textblock{ואזדא שמואל לטעמיה דאמר שמואל חצר שבין שני מבואות עירבה עם שניהם אסורה עם שניהם}
\textblock{לא עירבה עם שניהם אוסרת על שניהן}
\textblock{היתה באחד רגילה ובאחד אינה רגילה זה שרגילה בו אסור וזה שאינה רגילה בו מותר}
\textblock{אמר רבה בר רב הונא עירבה עם שאינה רגילה בו הותר רגילה לעצמו}
\textblock{ואמר רבה בר רב הונא אמר שמואל אם עירבה רגילה לעצמו וזה שאינה רגילה בו לא עירב והיא עצמה לא עירבה דוחין אותה אצל שאינה רגילה בו}
\textblock{וכגון זה כופין על מדת סדום}
\textblock{אמר רב יהודה אמר שמואל המקפיד על עירובו אין עירובו עירוב מה שמו עירוב שמו}
\textblock{ר' חנינא אמר עירובו עירוב אלא שנקרא מאנשי ורדינא}
\textblock{אמר רב יהודה אמר שמואל החולק את עירובו אינו עירוב}
\textblock{כמאן כבית שמאי דתניא חמשה שגבו את עירובן ונתנוהו בשני כלים ב"ש אומרים אין זה עירוב וב"ה אומרים הרי זה עירוב}
\textblock{אפילו תימא ב"ה עד כאן לא קאמרי ב"ה התם אלא דמליין למנא ואייתר אבל היכא דפלגיה מיפלג לא}
\textblock{ותרתי למה לי צריכי דאי אשמעינן התם משום דקפיד אבל הכא אימא לא}
\textblock{ואי אשמעינן הכא משום דפלגיה מיפלג אבל התם אימא לא צריכא}
\textblock{אמר ליה ר' אבא לרב יהודה בבי מעצרתא דבי רב זכאי מי אמר שמואל החולק את עירובו אינו עירוב והאמר שמואל בית שמניחין בו עירוב אינו צריך ליתן את הפת מ"ט לאו משום דאמר דכיון דמנח בסלא כמאן דמנח הכא דמי ה"נ כיון דמנח בסלא כמאן דמנח הכא דמי}
\textblock{א"ל התם אע"פ שאין פת מ"ט דכולהו הכא דיירי}
\textblock{אמר שמואל עירוב משום קנין}
\textblock{וא"ת מפני מה אין קונין במעה מפני שאינה מצויה בערבי שבתות}
\textblock{היכא דעירב מיהו לקני}
\textblock{גזירה שמא יאמרו מעה עיקר וזמנין דלא שכיח מעה ולא אתי לאיערובי בפת דאתי עירוב לאיקלקולי}
\textblock{רבה אמר עירוב משום דירה}
\textblock{מאי בינייהו איכא בינייהו כלי}
\textblock{ופחות משוה פרוטה}
\textblock{וקטן}
\textblock{אמר ליה אביי לרבה לדידך קשיא ולשמואל קשיא הא תניא חמשה שגבו את עירובן כשהם מוליכין את עירובן למקום אחר אחד מוליך לכולן הוא ניהו דקא קני ותו לא הוא ניהו דקא דייר ותו לא}
\textblock{אמר ליה לא לדידי קשיא ולא לשמואל קשיא שליחות דכולהו קא עביד}
\textblock{אמר רבה אמר רב חמא בר גוריא אמר רב הלכה כרבי שמעון:}
\textblock{{\large\emph{מתני׳}} מי שבא בדרך וחשכה לו והיה מכיר אילן או גדר ואמר שביתתי תחתיו לא אמר כלום}
\textblock{שביתתי בעיקרו מהלך ממקום רגליו ועד עיקרו אלפים אמה ומעיקרו ועד ביתו אלפים אמה נמצא מהלך משחשיכה ארבעת אלפים אמה}
\textblock{אם אינו מכיר או שאינו בקי בהלכה ואמר שביתתי במקומי זכה לו מקומו אלפים אמה לכל רוח}
\textblock{עגולות דברי ר' חנינא בן אנטיגנוס וחכמים אומרים מרובעות כטבלא מרובעת כדי שיהיה נשכר לזויות}
\textblock{וזו היא שאמרו העני מערב ברגליו אמר ר"מ אנו אין לנו אלא עני רבי יהודה אומר אחד עני ואחד עשיר לא אמרו מערבין בפת אלא להקל על העשיר שלא יצא ויערב ברגליו:}
\textblock{{\large\emph{גמ׳}} מאי לא אמר כלום}
\textblock{אמר רב לא אמר כלום כל עיקר דאפילו לתחתיו של אילן לא מצי אזיל}
\textblock{ושמואל אמר לא אמר כלום לביתו אבל לתחתיו של אילן מצי אזיל}
\textblock{ונעשה תחתיו של אילן חמר גמל}
\textblock{בא למדוד מן הצפון מודדין לו מן הדרום בא למדוד מן הדרום מודדין לו מן הצפון}
\newsection{דף נ}
\textblock{אמר רבה מ"ט דרב משום דלא מסיים אתריה}
\textblock{ואיכא דאמרי אמר רבה מ"ט דרב משום דקסבר כל שאינו בזה אחר זה אפילו בבת אחת אינו}
\textblock{מאי בינייהו איכא בינייהו דאמר ליקנו לי בארבע אמות מגו שמונה}
\textblock{מאן דאמר משום דלא מסיים אתריה הא לא מסיים אתריה}
\textblock{ומאן דאמר משום כל שאינו בזה אחר זה אפילו בבת אחת אינו האי כארבע אמות דמי דהכא ארבע אמות קאמר}
\textblock{גופא אמר רבה כל דבר שאינו בזה אחר זה אפילו בבת אחת אינו איתיביה אביי לרבה המרבה במעשרות פירותיו מתוקנין ומעשרותיו מקולקלין}
\textblock{אמאי לימא כל שאינו בזה אחר זה אפילו בבת אחת אינו}
\textblock{שאני מעשר דאיתיה לחצאין דאי אמר תקדוש פלגא פלגא דחיטתא קדשה}
\textblock{והרי מעשר בהמה דליתיה לחצאין}
\textblock{ואמר (רבה) יצאו שנים בעשירי וקראן עשירי עשירי ואחד עשר מעורבין זה בזה}
\textblock{שאני מעשר בהמה דאיתיה בזה אחר זה בטעות}
\textblock{דתנן קרא לתשיעי עשירי ולעשירי תשיעי ולאחד עשר עשירי שלשתן מקודשין}
\textblock{והרי תודה דליתה בטעות וליתה בזה אחר זה ואיתמר תודה שנשחטה על שמונים חלות חזקיה אמר קדשו עלה מ' מתוך שמונים ר' יוחנן אמר לא קדשו עלה מ' מתוך שמונים}
\textblock{הא איתמר עלה אמר ר' (זירא) הכל מודים היכא דאמר ליקדשו ארבעים מתוך שמונים דקדשי לא יקדשו ארבעים אלא אם כן קדשו שמונים כולי עלמא לא פליגי דלא קדשו}
\textblock{כי פליגי בסתמא מר סבר לאחריות קא מכוין ועל תנאי אייתינהו}
\textblock{ומ"ס לקרבן גדול קא מכוין}
\textblock{אמר אביי לא שנו אלא באילן שתחתיו י"ב אמה אבל באילן שאין תחתיו י"ב אמה הרי מקצת ביתו ניכר}
\textblock{מתקיף לה רב הונא בריה דרב יהושע ממאי דבארבעי מציעתא קא מסיים דלמא בארבעי דהאי גיסא ובארבעי דהאי גיסא קמסיים}
\textblock{אלא אמר רב הונא בריה דרב יהושע לא שנו אלא באילן שתחתיו ח' אמות אבל באילן שתחתיו ז' אמות הרי מקצת ביתו ניכר}
\textblock{תניא כוותיה דרב תניא כוותיה דשמואל}
\textblock{תניא כוותיה דרב מי שבא בדרך וחשכה לו והיה מכיר אילן או גדר ואמר שביתתי תחתיו לא אמר כלום אבל אם אמר שביתתי במקום פלוני מהלך עד שמגיע לאותו מקום הגיע לאותו מקום מהלך את כולו וחוצה לו אלפים אמה}
\textblock{בד"א במקום המסויים כגון ששבת בתל שהוא גבוה י' טפחים והוא מד"א ועד בית סאתים}
\textblock{וכן בקעה שהיא עמוקה י' והיא מד"א ועד בית סאתים אבל במקום שאין מסויים אין לו אלא ד"א}
\textblock{היו שנים אחד מכיר ואחד שאינו מכיר זה שאינו מכיר מוסר שביתתו למכיר והמכיר אומר שביתתי במקום פלוני}
\textblock{בד"א כשסיים ד"א שקבע אבל לא סיים ד"א שקבע לא יזוז ממקומו}
\textblock{לימא תיהוי תיובתיה דשמואל אמר לך שמואל הכא במאי עסקינן כגון דאיכא ממקום רגליו ועד עיקרו תרי אלפי וארבע גרמידי דאי מוקמית ליה באידך גיסא דאילן קם ליה לבר מתחומא}
\textblock{אי סיים ד"א מצי אזיל ואי לא לא מצי אזיל}
\textblock{תניא כוותיה דשמואל טעה ועירב לשתי רוחות כמדומה הוא שמערבין לו לשתי רוחות או שאמר לעבדיו צאו וערבו לי אחד עירב עליו לצפון ואחד עירב עליו לדרום מהלך לצפון כעירובו לדרום ולדרום כעירובו לצפון}
\textblock{ואם מיצעו עליו את התחום לא יזוז ממקומו}
\textblock{לימא תיהוי תיובתיה דרב רב תנא הוא ופליג:}
\textblock{אמר שביתתי בעיקרו מהלך ממקום רגליו ועד עיקרו אלפים אמה ומעיקרו לביתו אלפים אמה נמצא מהלך משחשיכה ד' אלפים אמה:}
\newsection{דף נא}
\textblock{אמר רבא והוא דכי רהיט לעיקרו מטי א"ל אביי והא חשכה לו קתני}
\textblock{חשכה לביתו אבל לעיקרו של אילן מצי אזיל איכא דאמרי אמר רבא חשכה לו כי מסגי קלי קלי אבל רהיט מטי}
\textblock{רבה ורב יוסף הוו קא אזלי באורחא א"ל רבה לרב יוסף תהא שביתתנו תותי דיקלא דסביל אחוה ואמרי לה תותי דיקלא דפריק מריה מכרגא}
\textblock{(ידע ליה מר) א"ל לא ידענא ליה אמר ליה סמוך עלי דתניא ר' יוסי אומר אם היו שנים אחד מכיר ואחד שאינו מכיר זה שאינו מכיר מוסר שביתתו למכיר זה שמכיר אומר תהא שביתתנו במקום פלוני}
\textblock{ולא היא לא תנא ליה כר' יוסי אלא כי היכי דליקבל לה מיניה משום דר' יוסי נימוקו עמו:}
\textblock{אם אינו מכיר או שאינו בקי וכו':}
\textblock{הני אלפים אמה היכן כתיבן דתניא (שמות טז, כט) שבו איש תחתיו אלו ארבע אמות אל יצא איש ממקומו אלו אלפים אמה}
\textblock{מנא לן אמר רב חסדא למדנו מקום ממקום ומקום מניסה וניסה מניסה וניסה מגבול וגבול מגבול וגבול מחוץ וחוץ מחוץ דכתיב (במדבר לה, ה) ומדותם מחוץ לעיר את פאת קדמה אלפים באמה וגו'}
\textblock{ונילף (במדבר לה, ד) מקיר העיר וחוצה אלף אמה דנין חוץ מחוץ ואין דנין חוץ מחוצה}
\textblock{ומאי נפקא מינה הא תנא דבי רבי ישמעאל (ויקרא יד, לט) ושב הכהן (ויקרא יד, מד) ובא הכהן זו היא שיבה זו היא ביאה}
\textblock{הני מילי היכא דליכא מידי דדמי ליה אבל היכא דאיכא מידי דדמי ליה מדמי ליה ילפינן:}
\textblock{אלפים אמה עגולות: ורבי חנינא בן אנטיגנוס מה נפשך אי אית ליה ג"ש פיאות כתיבן אי לית ליה גזירה שוה אלפים אמה מנא ליה}
\textblock{לעולם אית ליה גזירה שוה ושאני הכא דאמר קרא (במדבר לה, ה) זה יהיה להם מגרשי הערים לזה אתה נותן פיאות ואי אתה נותן פיאות לשובתי שבת}
\textblock{ורבנן תני רב חנניה אומר כזה יהו כל שובתי שבת}
\textblock{א"ר אחא בר יעקב המעביר ד"א ברה"ר אינו חייב עד שמעביר הן ואלכסונן}
\textblock{א"ר פפא בדיק לן רבא עמוד ברשות הרבים גבוה י' ורוחב ד' צריך הן ואלכסונן או לא ואמרינן ליה לאו היינו דרב חנניה דתניא רב חנניה אומר כזה יהו כל שובתי שבת:}
\textblock{וזה הוא שאמרו העני מערב ברגליו אמר ר' מאיר אנו אין לנו אלא עני וכו':}
\textblock{אמר רב נחמן מחלוקת במקומי דר"מ סבר עיקר עירוב בפת}
\textblock{עני הוא דאקילו רבנן עילויה אבל עשיר לא}
\textblock{ורבי יהודה סבר עיקר עירוב ברגל אחד עני ואחד עשיר אבל במקום פלוני דברי הכל עני אין עשיר לא}
\textblock{וזו היא שאמרו מאן קתני לה ר"מ ואהייא קאי אאינו מכיר או שאינו בקי בהלכה ולא אמרו מערבין בפת אלא להקל מאן קתני לה רבי יהודה}
\textblock{ורב חסדא אמר מחלוקת במקום פלוני דר' מאיר סבר עני אין עשיר לא ורבי יהודה סבר אחד עני ואחד עשיר אבל במקומי דברי הכל אחד עני ואחד עשיר דעיקר עירוב ברגל}
\textblock{וזו היא שאמרו מאן קתני לה ר"מ ואהייא קאי אהא מי שבא בדרך וחשכה ולא אמרו מערבין בפת אלא להקל מאן קתני לה דברי הכל}
\textblock{תניא כוותיה דרב נחמן אחד עני ואחד עשיר מערבין בפת ולא יצא עשיר חוץ לתחום ויאמר שביתתי במקומי לפי שלא אמרו מערבין ברגל אלא למי שבא בדרך וחשכה דברי רבי מאיר}
\textblock{רבי יהודה אומר אחד עני ואחד עשיר מערבין ברגל ויצא עשיר חוץ לתחום ויאמר תהא שביתתי במקומי וזה הוא עיקרו של עירוב והתירו חכמים לבעל הבית לשלח עירובו ביד עבדו ביד בנו ביד שלוחו בשביל להקל עליו}
\textblock{א"ר יהודה מעשה באנשי בית ממל ובאנשי בית גוריון בארומא שהיו מחלקין גרוגרות וצימוקין לעניים בשני בצורת ובאין עניי כפר שיחין ועניי כפר חנניה ומחשיכין על התחום למחרת משכימין ובאין}
\textblock{אמר רב אשי מתניתין נמי דיקא דקתני מי שיצא לילך לעיר שמערבין לה והחזירו חברו הוא מותר לילך וכל בני העיר אסורין דברי רבי יהודה}
\textblock{והוינן בה מאי שנא איהו ומאי שנא אינהו ואמר רב הונא הכא במאי עסקינן כגון שיש לו שני בתים ושני תחומי שבת ביניהן}
\textblock{איהו כיון דנפקא ליה לאורחא הוה ליה עני}
\textblock{והנך עשירים נינהו אלמא כל במקום פלוני עני אין עשיר לא ש"מ}
\textblock{מתני ליה רב חייא בר אשי לחייא בר רב קמיה דרב אחד עני ואחד עשיר א"ל רב סיים בה נמי הלכה כר' יהודה}
\textblock{רבה בר רב חנן הוה רגיל דאתי מארטיבנא לפומבדיתא}
\newsection{דף נב}
\textblock{אמר תהא שביתתי בצינתא א"ל אביי מאי דעתיך ר"מ ורבי יהודה הלכה כרבי יהודה ואמר רב חסדא מחלוקת במקום פלוני}
\textblock{והא ר"נ ותניא כוותיה א"ל הדרי בי}
\textblock{אמר רמי בר חמא הרי אמרו שבת יש לו ד"א הנותן את עירובו יש לו ד"א או לא}
\textblock{אמר רבא ת"ש לא אמרו מערבין בפת אלא להקל על העשיר שלא יצא ויערב ברגליו ואי אמרת אין לו האי להקל להחמיר הוא}
\textblock{אפילו הכי ניחא ליה כי היכי דלא נטרח וניפוק:}
\textblock{{\large\emph{מתני׳}} מי שיצא לילך בעיר שמערבין בה והחזירו חבירו הוא מותר לילך וכל בני העיר אסורין דברי רבי יהודה}
\textblock{ר"מ אומר כל שהוא יכול לערב ולא עירב הרי זה חמר גמל:}
\textblock{{\large\emph{גמ׳}} מאי שנא איהו ומאי שנא אינהו אמר רב הונא הכא במאי עסקינן כגון שיש לו שני בתים וביניהן שני תחומי שבת}
\textblock{איהו כיון דנפק ליה לאורחא הוה ליה עני והני עשירי נינהו}
\textblock{תניא נמי הכי מי שיש לו שני בתים וביניהן שני תחומי שבת כיון שהחזיק בדרך קנה עירוב דברי ר' יהודה}
\textblock{יתר על כן אמר ר' יוסי בר' יהודה אפילו מצאו חבירו ואמר לו לין פה עת חמה הוא עת צינה הוא למחר משכים והולך}
\textblock{אמר רבה לומר כולי עלמא לא פליגי דצריך כי פליגי להחזיק}
\textblock{ורב יוסף אמר להחזיק דכ"ע לא פליגי דצריך כי פליגי לומר}
\textblock{כמאן אזלא הא דאמר עולא מי שהחזיק בדרך והחזירו חבירו הרי הוא מוחזר ומוחזק}
\textblock{אי מוחזר למה מוחזק ואי מוחזק למה מוחזר}
\textblock{הכי קאמר אע"פ שמוחזר מוחזק כמאן כרב יוסף ואליבא דרבי יוסי ברבי יהודה}
\textblock{רב יהודה בר אישתתא אייתי ליה כלכלה דפירי לרב נתן בר אושעיא כי הוה אזיל שבקיה עד דנחית דרגא אמר ליה בית הכא למחר קדים ואזיל}
\textblock{כמאן כרב יוסף ואליבא דר' יוסי בר יהודה}
\textblock{לא כרבה ואליבא דר' יהודה:}
\textblock{ר"מ אומר כל שיכול לערב כו': הא תנינא חדא זימנא ספק ר"מ ורבי יהודה אומרים הרי זה חמר גמל}
\textblock{אמר רב ששת לא תימא טעמא דר"מ ספק עירב ספק לא עירב הוא דהוי חמר גמל אבל ודאי לא עירב לא הוי חמר גמל}
\textblock{אלא אפילו ודאי לא עירב הוי חמר גמל דהא הכא ודאי לא עירב וקא הוי חמר גמל:}
\textblock{{\large\emph{מתני׳}} מי שיצא חוץ לתחום אפילו אמה אחת לא יכנס ר"א אומר שתים יכנס שלש לא יכנס:}
\textblock{{\large\emph{גמ׳}} א"ר חנינא רגלו אחת בתוך התחום ורגלו אחת חוץ לתחום לא יכנס דכתיב (ישעיהו נח, יג) אם תשיב משבת רגלך רגלך כתיב}
\textblock{והתניא רגלו אחת בתוך התחום ורגלו אחת חוץ לתחום יכנס הא מני אחרים היא דתניא אחרים אומרים למקום שרובו הוא נזקר}
\textblock{איכא דאמרי אמר ר' חנינא רגלו אחת בתוך התחום ורגלו אחת חוץ לתחום יכנס דכתיב אם תשיב משבת רגלך רגליך קרינן}
\textblock{והתניא לא יכנס הוא דאמר כאחרים דתניא למקום שרובו הוא נזקר:}
\textblock{ר"א אומר שתים יכנס שלש לא יכנס: והתניא ר"א אומר אחת יכנס שתים לא יכנס לא קשיא הא דעקר חדא וקם אתרתי הא דעקר תרתי וקם אתלת}
\textblock{והתניא ר"א אומר אפילו אמה אחת לא יכנס כי תניא ההיא למודד דתנן ולמודד שאמרו נותנין לו אלפים אמה אפילו סוף מדתו כלה במערה:}
\textblock{{\large\emph{מתני׳}} מי שהחשיך חוץ לתחום אפילו אמה אחת לא יכנס ר"ש אומר אפילו חמש עשרה אמות יכנס שאין המשוחות ממצין את המדות מפני הטועין:}
\textblock{{\large\emph{גמ׳}} תנא מפני טועי המדה:}
\textblock{\par \par {\large\emph{הדרן עלך מי שהוציוהו}}\par \par }
\textblock{}
\textblock{}
\textblock{מתני׳ {\large\emph{כיצד}} מעברין את הערים בית נכנס בית יוצא פגום נכנס פגום יוצא היו שם גדודיות גבוהות עשרה טפחים}
\newchap{פרק \hebrewnumeral{5}\quad כיצד מעברין}
\newsection{דף נג}
\textblock{}
\textblock{וגשרים ונפשות שיש בהן בית דירה מוציאין את המדה כנגדן ועושין אותה כמין טבלא מרובעת כדי שיהא נשכר את הזויות:}
\textblock{{\large\emph{גמ׳}} רב ושמואל חד תני מעברין וחד תני מאברין}
\textblock{מאן דתני מאברין אבר אבר ומאן דתני מעברין כאשה עוברה (בראשית כג, ט)}
\textblock{מערת המכפלה רב ושמואל חד אמר שני בתים זה לפנים מזה וחד אמר בית ועלייה על גביו}
\textblock{בשלמא למאן דאמר זה על גב זה היינו מכפלה אלא למאן דאמר שני בתים זה לפנים מזה מאי מכפלה}
\textblock{שכפולה בזוגות (בראשית לה, כז) ממרא קרית ארבע א"ר יצחק קרית הארבע זוגות אדם וחוה אברהם ושרה יצחק ורבקה יעקב ולאה (בראשית יד, א)}
\textblock{ויהי בימי אמרפל רב ושמואל חד אמר נמרוד שמו ולמה נקרא שמו אמרפל שאמר והפיל לאברהם אבינו בתוך כבשן האש וחד אמר אמרפל שמו ולמה נקרא שמו נמרוד שהמריד את כל העולם כולו עליו במלכותו (שמות א, ח)}
\textblock{ויקם מלך חדש על מצרים רב ושמואל חד אמר חדש ממש וחד אמר שנתחדשו גזירותיו}
\textblock{מ"ד חדש ממש דכתיב חדש ומאן דאמר שנתחדשו גזירותיו מדלא כתיב וימת וימלוך}
\textblock{ולמאן דאמר שנתחדשו גזירותיו הא כתיב (שמות א, ח) אשר לא ידע את יוסף מאי אשר לא ידע את יוסף דהוה דמי כמאן דלא ידע ליה ליוסף כלל:}
\textblock{(סימן שמונה עשרה ושנים עשר למדנו בדוד ויבן):}
\textblock{א"ר יוחנן י"ח ימים גידלתי אצל רבי אושעיא בריבי ולא למדתי ממנו אלא דבר אחד במשנתינו כיצד מאברין את הערים באלף}
\textblock{איני והאמר רבי יוחנן י"ב תלמידים היו לו לרבי אושעיא בריבי וי"ח ימים גידלתי ביניהן ולמדתי לב כל אחד ואחד וחכמת כל אחד ואחד}
\textblock{לב כל אחד ואחד וחכמת כל אחד ואחד גמר גמרא לא גמר איבעית אימא מנייהו דידהו גמר מיניה דידיה לא גמר ואב"א דבר אחד במשנתינו קאמר}
\textblock{וא"ר יוחנן כשהיינו לומדין תורה אצל ר' אושעיא היינו יושבין ארבעה ארבעה באמה אמר רבי כשהיינו לומדין תורה אצל רבי אלעזר בן שמוע היינו יושבין ששה ששה באמה}
\textblock{א"ר יוחנן רבי אושעיא בריבי בדורו כר' מאיר בדורו מה רבי מאיר בדורו לא יכלו חבריו לעמוד על סוף דעתו אף רבי אושעיא לא יכלו חבריו לעמוד על סוף דעתו}
\textblock{אמר ר' יוחנן לבן של ראשונים כפתחו של אולם ושל אחרונים כפתחו של היכל ואנו כמלא נקב מחט סידקית}
\textblock{ראשונים ר"ע אחרונים ר"א בן שמוע איכא דאמרי ראשונים ר' אלעזר בן שמוע אחרונים ר' אושעיא בריבי ואנו כמלא נקב מחט סידקית}
\textblock{אמר אביי ואנן כי סיכתא בגודא לגמרא אמר רבא ואנן כי אצבעתא בקירא לסברא אמר רב אשי אנן כי אצבעתא בבירא לשכחה}
\textblock{אמר רב יהודה אמר רב בני יהודה שהקפידו על לשונם נתקיימה תורתם בידם בני גליל שלא הקפידו על לשונם לא נתקיימה תורתם בידם}
\textblock{מידי בקפידא תליא מילתא אלא בני יהודה דדייקי לישנא ומתנחי להו סימנא נתקיימה תורתן בידן בני גליל דלא דייקי לישנא ולא מתנחי להו סימנא לא נתקיימה תורתן בידם}
\textblock{בני יהודה גמרו מחד רבה נתקיימה תורתן בידם בני גליל דלא גמרי מחד רבה לא נתקיימה תורתן בידם}
\textblock{רבינא אמר בני יהודה דגלו מסכתא נתקיימה תורתן בידם בני גליל דלא גלו מסכתא לא נתקיימה תורתן בידם}
\textblock{דוד גלי מסכתא שאול לא גלי מסכתא דוד דגלי מסכתא כתיב ביה (תהלים קיט, עד) יראיך יראוני וישמחו שאול דלא גלי מסכתא כתיב ביה ((שמואל א יד, מז) אל כל) אשר יפנה}
\textblock{ירשיע}
\textblock{ואמר ר' יוחנן מניין שמחל לו הקב"ה על אותו עון שנאמר (שמואל א כח, יט) מחר אתה ובניך עמי עמי במחיצתי}
\textblock{א"ר אבא אי איכא דמשאיל להו לבני יהודה דדייקי לשני מאברין תנן או מעברין תנן אכוזו תנן או עכוזו תנן ידעי}
\textblock{שאילינהו ואמרי ליה איכא דתני מאברין ואיכא דתני מעברין איכא דתני אכוזו ואיכא דתני עכוזו}
\textblock{בני יהודה דייקי לישנא מאי היא דההוא בר יהודה דאמר להו טלית יש לי למכור אמרו ליה מאי גוון טליתך אמר להו כתרדין עלי אדמה}
\textblock{בני גליל דלא דייקי לישנא מאי היא (דתניא) דההוא בר גלילא [דהוה קאזיל] ואמר להו אמר למאן אמר למאן אמרו ליה גלילאה שוטה חמר למירכב או חמר למישתי עמר למילבש או אימר לאיתכסאה}
\textblock{ההיא איתתא דבעיא למימר לחברתה תאי דאוכליך חלבא אמרה לה שלוכתי תוכליך לביא}
\textblock{ההיא אתתא דאתיא לקמיה דדיינא אמרה ליה מרי כירי תפלא הוית לי וגנבוך מין וכדו הוות דכד שדרו לך עילויה לא מטי כרעיך אארעא}
\textblock{אמהתא דבי רבי כי הוה משתעיא בלשון חכמה אמרה הכי עלת נקפת בכד ידאון נישריא לקיניהון}
\textblock{וכד הוה בעי דליתבון הוה אמרה להו יעדי בתר חברתה מינה ותתקפי עלת בכד כאילפא דאזלא בימא}
\textblock{רבי יוסי בר אסיין כי הוה משתעי בלשון חכמה אמר עשו לי שור במשפט בטור מסכן}
\textblock{וכד הוה שאיל באושפיזא אמר הכי גבר פום דין חי מה זו טובה יש}
\textblock{רבי אבהו כי הוה משתעי בלשון חכמה הוה אמר הכי אתריגו לפחמין ארקיעו לזהבין ועשו לי שני מגידי בעלטה איכא דאמרי ויעשו לי בהן שני מגידי בעלטה}
\textblock{אמרו ליה רבנן לרבי אבהו הצפיננו היכן רבי אלעאי צפון אמר להן עלץ בנערה אהרונית אחרונית עירנית והנעירתו}
\textblock{אמרי לה אשה}
\textblock{ואמרי לה מסכתא}
\textblock{אמרי ליה לרבי אלעאי הצפיננו הכין רבי אבהו [צפון] אמר להן נתייעץ במכתיר והנגיב למפיבשת}
\textblock{אמר רבי יהושע בן חנניה מימי לא נצחני אדם חוץ מאשה תינוק ותינוקת אשה מאי היא פעם אחת נתארחתי אצל אכסניא אחת עשתה לי פולין ביום ראשון אכלתים ולא שיירתי מהן כלום שנייה ולא שיירתי מהן כלום ביום שלישי הקדיחתן במלח כיון שטעמתי משכתי ידי מהן}
\textblock{אמרה לי רבי מפני מה אינך סועד אמרתי לה כבר סעדתי מבעוד יום אמרה לי היה לך למשוך ידיך מן הפת}
\textblock{אמרה לי רבי שמא לא הנחת פאה בראשונים ולא כך אמרו חכמים אין משיירין פאה באילפס אבל משיירין פאה בקערה}
\textblock{תינוקת מאי היא פעם אחת הייתי מהלך בדרך והיתה דרך עוברת בשדה והייתי מהלך בה אמרה לי תינוקת אחת רבי לא שדה היא זו אמרתי לה לא דרך כבושה היא אמרה לי ליסטים כמותך כבשוה}
\textblock{תינוק מאי היא פעם אחת הייתי מהלך בדרך וראיתי תינוק יושב על פרשת דרכים ואמרתי לו באיזה דרך נלך לעיר אמר לי זו קצרה וארוכה וזו ארוכה וקצרה והלכתי בקצרה וארוכה כיון שהגעתי לעיר מצאתי שמקיפין אותה גנות ופרדיסין}
\textblock{חזרתי לאחורי אמרתי לו בני הלא אמרת לי קצרה אמר לי ולא אמרתי לך ארוכה נשקתיו על ראשו ואמרתי לו אשריכם ישראל שכולכם חכמים גדולים אתם מגדולכם ועד קטנכם:}
\textblock{רבי יוסי הגלילי הוה קא אזיל באורחא אשכחה לברוריה אמר לה באיזו דרך נלך ללוד אמרה ליה גלילי שוטה לא כך אמרו חכמים אל תרבה שיחה עם האשה היה לך לומר באיזה ללוד}
\textblock{ברוריה אשכחתיה לההוא תלמידא דהוה קא גריס בלחישה}
\newsection{דף נד}
\textblock{בטשה ביה אמרה ליה לא כך כתוב (שמואל ב כג, ה) ערוכה בכל ושמורה אם ערוכה ברמ"ח אברים שלך משתמרת ואם לאו אינה משתמרת תנא תלמיד אחד היה לרבי אליעזר שהיה שונה בלחש לאחר ג' שנים שכח תלמודו}
\textblock{תנא תלמיד אחד היה לו לרבי אליעזר שנתחייב בשריפה למקום אמרו הניחו לו אדם גדול שמש}
\textblock{א"ל שמואל לרב יהודה שיננא פתח פומיך קרי פתח פומיך תני כי היכי דתתקיים ביך ותוריך חיי שנאמר (משלי ד, כב) כי חיים הם למצאיהם ולכל בשרו מרפא אל תקרי למצאיהם אלא למוציאיהם בפה}
\textblock{א"ל שמואל לרב יהודה שיננא חטוף ואכול חטוף ואישתי דעלמא דאזלינן מיניה כהלולא דמי}
\textblock{א"ל רב לרב המנונא בני אם יש לך היטב לך שאין בשאול תענוג ואין למות התמהמה ואם תאמר אניח לבני חוק בשאול מי יגיד לך בני האדם דומים לעשבי השדה הללו נוצצין והללו נובלין}
\textblock{א"ר יהושע בן לוי המהלך בדרך ואין עמו לוייה יעסוק בתורה שנאמר (משלי א, ט) כי לוית חן הם}
\textblock{חש בראשו יעסוק בתורה שנאמר כי לוית חן הם לראשך חש בגרונו יעסוק בתורה שנאמר וענקים לגרגרותיך חש במעיו יעסוק בתורה שנאמר רפאות תהי לשרך חש בעצמותיו יעסוק בתורה שנאמר ושקוי לעצמותיך חש בכל גופו יעסוק בתורה שנאמר ולכל בשרו מרפא}
\textblock{אמר רב יהודה בר' חייא בא וראה שלא כמדת הקב"ה מדת בשר ודם מדת בשר ודם אדם נותן סם לחבירו לזה יפה ולזה קשה אבל הקב"ה אינו כן נתן תורה לישראל סם חיים לכל גופו שנאמר ולכל בשרו מרפא}
\textblock{א"ר אמי מ"ד (משלי כב, יח) כי נעים כי תשמרם בבטנך יכונו יחדיו על שפתיך אימתי ד"ת נעי' בזמן שתשמרם בבטנך ואימתי תשמרם בבטנך בזמן שיכונו יחדיו על שפתיך}
\textblock{ר' זירא אמר מהכא (משלי טו, כג) שמחה לאיש במענה פיו ודבר בעתו מה טוב אימתי שמחה לאיש בזמן שמענה בפיו ל"א אימתי שמחה לאיש במענה פיו בזמן שדבר בעתו מה טוב}
\textblock{ר' יצחק אמר מהכא (דברים ל, יד) כי קרוב אליך הדבר מאד בפיך ובלבבך לעשותו אימתי קרוב אליך בזמן שבפיך ובלבבך לעשותו}
\textblock{רבא אמר מהכא (תהלים כא, ג) תאות לבו נתתה לו וארשת שפתיו בל מנעת סלה אימתי תאות לבו נתתה לו בזמן שארשת שפתיו בל מנעת סלה}
\textblock{רבא רמי כתיב תאות לבו נתתה לו וכתיב וארשת שפתיו בל מנעת סלה זכה תאות לבו נתתה לו לא זכה וארשת שפתיו בל מנעת סלה}
\textblock{תנא דבי ר"א בן יעקב כל מקום שנאמר נצח סלה ועד אין לו הפסק עולמית נצח דכתיב (ישעיהו נז, טז) כי לא לעולם אריב ולא לנצח אקצוף}
\textblock{סלה דכתיב (תהלים מח, ט) כאשר שמענו כן ראינו בעיר ה' צבאות בעיר אלהינו אלהים יכוננה עד עולם סלה ועד דכתיב (שמות טו, יח) ה' ימלוך לעולם ועד:}
\textblock{(סימן ענקים לחייו לוחות חרות): א"ר (אליעזר) מאי דכתיב (משלי א, ט) וענקים לגרגרותיך אם משים אדם עצמו כענק זה שרף על הצואר ונראה ואינו נראה תלמודו מתקיים בידו ואם לאו אין תלמודו מתקיים בידו}
\textblock{ואמר ר"א מאי דכתיב (שיר השירים ה, יג) לחיו כערוגת הבשם אם משים אדם עצמו כערוגה זו שהכל דשין בה וכבושם זה שהכל מתבשמין בה תלמודו מתקיים ואם לאו אין תלמודו מתקיים}
\textblock{וא"ר מ"ד (שמות לא, יח) לוחות אבן אם אדם משים עצמו את לחייו כאבן זו שאינה נמחית תלמודו מתקיים בידו ואם לאו אין תלמודו מתקיים בידו}
\textblock{וא"ר (אליעזר) מאי דכתיב (שמות לב, טז) חרות על הלוחות אלמלי לא נשתברו לוחות הראשונות לא נשתכחה תורה מישראל}
\textblock{רב אחא בר יעקב אמר אין כל אומה ולשון שולטת בהן שנאמר חרות אל תיקרי חרות אלא חירות}
\textblock{אמר רב מתנה מאי דכתיב (במדבר כא, יח) וממדבר מתנה אם משים אדם עצמו כמדבר זה שהכל דשין בו תלמודו מתקיים בידו ואם לאו אין תלמודו מתקיים בידו}
\textblock{רבא בריה דרב יוסף בר חמא הוה ליה מלתא לרב יוסף בהדיה כי מטא מעלי יומא דכיפורי אמר איזיל ואפייסיה אזל אשכחיה לשמעיה דקא מזיג ליה כסא אמר ליה הב לי ואימזגיה אנא יהב ליה מזגיה כדטעמיה אמר דמי האי מזיגא למזיגא דרבא בריה דרב יוסף בר חמא א"ל אנא הוא}
\textblock{א"ל לא תתיב אכרעיך עד דמפרשת לי הני קראי מאי דכתיב וממדבר מתנה וממתנה נחליאל ומנחליאל במות ומבמות הגיא}
\textblock{א"ל אם אדם משים עצמו כמדבר זה שהכל דשין בו תורה ניתנה לו במתנה וכיון שניתנה לו במתנה נחלו אל שנאמר וממתנה נחליאל וכיון שנחלו אל עולה לגדולה שנאמר ומנחליאל במות}
\textblock{ואם מגיס לבו הקדוש ברוך הוא משפילו שנאמר ומבמות הגיא ואם חוזר בו הקב"ה מגביהו שנאמר (ישעיהו מ, ד) כל גיא ינשא}
\textblock{אמר רב הונא מ"ד (תהלים סח, יא) חיתך ישבו בה תכין בטובתך לעני אלהים אם אדם משים עצמו כחיה זו שדורסת ואוכלת ואיכא דאמרי שמסרחת ואוכלת תלמודו מתקיים בידו ואם לאו אין תלמודו מתקיים בידו ואם עושה כן הקדוש ברוך הוא עושה לו סעודה בעצמו שנאמר תכין בטובתך לעני אלהים}
\textblock{א"ר חייא בר אבא א"ר יוחנן מאי דכתיב (משלי כז, יח) נוצר תאנה יאכל פריה למה נמשלו דברי תורה כתאנה מה תאנה זו}
\textblock{כל זמן שאדם ממשמש בה מוצא בה תאנים אף דברי תורה כל זמן שאדם הוגה בהן מוצא בהן טעם}
\textblock{א"ר שמואל בר נחמני מאי דכתיב (משלי ה, יט) אילת אהבים ויעלת חן וגו' למה נמשלו דברי תורה לאילת לומר לך מה אילה רחמה צר וחביבה על בועלה כל שעה ושעה כשעה ראשונה אף דברי תורה חביבין על לומדיהן כל שעה ושעה כשעה ראשונה}
\textblock{ויעלת חן שמעלת חן על לומדיה דדיה ירווך בכל עת למה נמשלו דברי תורה כדד מה דד זה כל זמן שהתינוק ממשמש בו מוצא בו חלב אף דברי תורה כל זמן שאדם הוגה בהן מוצא בהן טעם}
\textblock{באהבתה תשגה תמיד כגון רבי (אליעזר) בן פדת אמרו עליו על רבי (אליעזר) שהיה יושב ועוסק בתורה בשוק התחתון של ציפורי וסדינו מוטל בשוק העליון של ציפורי (תניא) א"ר יצחק בן אלעזר פעם אחת בא אדם ליטלו ומצא בו שרף}
\textblock{תנא דבי רב ענן מאי דכתיב (שופטים ה, י) רוכבי אתונות צחורות יושבי על מדין [והולכי על דרך שיחו] רוכבי אתונות אלו תלמידי חכמים שמהלכין מעיר לעיר וממדינה למדינה ללמוד (בו) תורה צחורות שעושין אותה כצהרים יושבי על מדין שדנין דין אמת לאמיתו והולכי אלו בעלי מקרא על דרך אלו בעלי משנה שיחו אלו בעלי תלמוד שכל שיחתן דברי תורה }
\textblock{אמר רב שיזבי משום רבי אלעזר בן עזריה מאי דכתיב (משלי יב, כז) לא יחרוך רמיה צידו לא יחיה ולא יאריך ימים צייד הרמאי}
\textblock{רב ששת אמר צייד הרמאי יחרוך}
\textblock{כי אתא רב דימי אמר משל לצייד שצד צפרים אם ראשון ראשון משבר כנפיו משתמר ואם לאו אין משתמר}
\textblock{אמר (רבה) אמר רב סחורה אמר רב הונא מאי דכתיב (משלי יג, יא) הון מהבל ימעט וקובץ על יד ירבה אם עושה אדם תורתו חבילות חבילות מתמעט ואם לאו קובץ על יד ירבה}
\textblock{אמר (רבה) ידעי רבנן להא מלתא ועברי עלה אמר רב נחמן בר יצחק אנא עבדתה ואיקיים בידאי:}
\textblock{ת"ר כיצד סדר משנה משה למד מפי הגבורה נכנס אהרן ושנה לו משה פירקו נסתלק אהרן וישב לשמאל משה נכנסו בניו ושנה להן משה פירקן נסתלקו בניו אלעזר ישב לימין משה ואיתמר לשמאל אהרן רבי יהודה אומר לעולם אהרן לימין משה חוזר נכנסו זקנים ושנה להן משה פירקן נסתלקו זקנים נכנסו כל העם ושנה להן משה פירקן נמצאו ביד אהרן ארבעה ביד בניו שלשה וביד הזקנים שנים וביד כל העם אחד}
\textblock{נסתלק משה ושנה להן אהרן פירקו נסתלק אהרן שנו להן בניו פירקן נסתלקו בניו שנו להן זקנים פירקן נמצא ביד הכל ארבעה}
\textblock{מכאן א"ר אליעזר חייב אדם לשנות לתלמידו ארבעה פעמים וקל וחומר ומה אהרן שלמד מפי משה ומשה מפי הגבורה כך הדיוט מפי הדיוט על אחת כמה וכמה}
\textblock{ר"ע אומר מניין שחייב אדם לשנות לתלמידו עד שילמדנו שנאמר (דברים לא, יט) ולמדה את בני ישראל ומניין עד שתהא סדורה בפיהם שנאמר שימה בפיהם}
\textblock{ומניין שחייב להראות לו פנים שנאמר (שמות כא, א) ואלה המשפטים אשר תשים לפניהם}
\textblock{וליגמרו כולהו ממשה כדי לחלוק כבוד לאהרן ובניו וכבוד לזקנים}
\textblock{וניעול אהרן וניגמר ממשה וליעיילו בניו וליגמרו מאהרן וליעיילו זקנים ולילפו מבניו וליזלו וליגמרינהו לכולהו ישראל כיון דמשה מפי הגבורה גמר מסתייעא מלתיה}
\textblock{אמר מר רבי יהודה אומר לעולם אהרן לימין משה חוזר כמאן אזלא הא דתניא שלשה שהיו מהלכין בדרך הרב באמצע וגדול בימינו וקטן בשמאלו לימא רבי יהודה היא ולא רבנן}
\textblock{אפילו תימא רבנן משום טירחא דאהרן}
\textblock{רבי פרידא הוה ליה ההוא תלמידא דהוה תני ליה ארבע מאה זימני וגמר יומא חד בעיוה למלתא דמצוה תנא ליה ולא גמר}
\textblock{א"ל האידנא מאי שנא א"ל מדההיא שעתא דא"ל למר איכא מילתא דמצוה אסחאי לדעתאי וכל שעתא אמינא השתא קאי מר השתא קאי מר א"ל הב דעתיך ואתני ליך הדר תנא ליה ד' מאה זימני [אחריני]}
\textblock{נפקא בת קלא וא"ל ניחא ליך דליספו לך ד' מאה שני או דתיזכו את ודרך לעלמא דאתי אמר דניזכו אנא ודריי לעלמא דאתי אמר להן הקב"ה תנו לו זו וזו}
\textblock{אמר רב חסדא אין תורה נקנית אלא בסימנין שנאמר שימה בפיהם אל תקרי שימה אלא סימנה}
\textblock{שמעה רב תחליפא ממערבא אזל אמרה קמיה דר' אבהו אמר אתון מהתם מתניתו לה אנן מהכא מתנינן לה (ירמיהו לא, כא) הציבי לך ציונים שימי לך וגו' עשו ציונים לתורה ומאי משמע דהאי ציון לישנא דסימנא הוא דכתיב (יחזקאל לט, טו) וראה עצם אדם ובנה אצלו ציון}
\textblock{ר' אליעזר אמר מהכא (משלי ז, ד) אמור לחכמה אחותי את ומודע לבינה תקרא עשה מודעים לתורה רבא אמר עשה מועדים לתורה}
\newsection{דף נה}
\textblock{והיינו דאמר אבדימי בר חמא בר דוסא מאי דכתיב (דברים ל, יב) לא בשמים היא ולא מעבר לים היא לא בשמים היא שאם בשמים היא אתה צריך לעלות אחריה ואם מעבר לים היא אתה צריך לעבור אחריה}
\textblock{רבא אמר לא בשמים היא לא תמצא במי שמגביה דעתו עליה כשמים ולא תמצא במי שמרחיב דעתו עליה כים}
\textblock{רבי יוחנן אמר לא בשמים היא לא תמצא בגסי רוח ולא מעבר לים היא לא תמצא לא בסחרנים ולא בתגרים:}
\textblock{תנו רבנן כיצד מעברין את הערים ארוכה כמות שהיא עגולה עושין לה זויות מרובעת אין עושין לה זויות היתה רחבה מצד אחד וקצרה מצד אחר רואין אותה כאילו היא שוה}
\textblock{היה בית אחד יוצא כמין פגום או שני בתים יוצאין כמין שני פגומין רואין אותן כאילו חוט מתוח עליהן ומודד ממנו ולהלן אלפים אמה היתה עשויה כמין קשת או כמין גאם רואין אותה כאילו היא מלאה בתים וחצירות ומודד ממנו ולהלן אלפים אמה}
\textblock{אמר מר ארוכה כמות שהיא פשיטא לא צריכא דאריכא וקטינא מהו דתימא ליתן לה פותיא אאורכה קמ"ל}
\textblock{מרובעת אין עושין לה זויות פשיטא לא צריכא דמרבעא ולא מרבעא בריבוע עולם מהו דתימא לירבעא בריבוע עולם קא משמע לן}
\textblock{היה בית אחד יוצא כמין פגום או שני בתים יוצאין כמין שני פגומין השתא בית אחד אמרת שני בתים מיבעיא}
\textblock{לא צריכא משתי רוחות מהו דתימא מרוח אחת אמרינן משתי רוחות לא אמרינן קא משמע לן}
\textblock{היתה עשויה כמין קשת או כמין גאם רואין אותה כאילו היא מלאה בתים וחצירות ומודד ממנה ולהלן אלפים אמה אמר רב הונא עיר העשויה כקשת אם יש בין שני ראשיה פחות מארבעת אלפים אמה מודדין לה מן היתר ואם לאו מודדין לה מן הקשת}
\textblock{ומי אמר רב הונא הכי והאמר רב הונא חומת העיר שנפרצה במאה וארבעים ואחת ושליש}
\textblock{אמר רבה בר עולא לא קשיא כאן ברוח אחת כאן משתי רוחות}
\textblock{ומאי קמ"ל דנותנין קרפף לזו וקרפף לזו הא אמרה רב הונא חדא זימנא דתנן}
\textblock{נותנין קרפף לעיר דברי רבי מאיר וחכמים אומרים לא אמרו קרפף אלא בין שתי עיירות}
\textblock{ואיתמר רב הונא אמר קרפף לזו וקרפף לזו וחייא בר רב אמר אין נותנין אלא קרפף אחד לשניהם}
\textblock{צריכא דאי אשמעינן הכא משום דהוה ליה צד היתר מעיקרא אבל התם אימא לא}
\textblock{ואי אשמעינן התם משום דדחיקא תשמישתייהו אבל הכא דלא דחיקא תשמישתייהו אימא לא צריכא}
\textblock{וכמה הוי בין יתר לקשת רבה בר רב הונא אמר אלפים אמה רבא בריה דרבה בר רב הונא אמר אפילו יתר מאלפים אמה}
\textblock{אמר אביי כוותיה דרבא בריה דרבה בר רב הונא מסתברא דאי בעי הדר אתי דרך בתים:}
\textblock{היו שם גדודיות גבוהות עשרה טפחים כו': מאי גדודיות אמר רב יהודה שלש מחיצות שאין עליהן תקרה}
\textblock{איבעיא להו שתי מחיצות ויש עליהן תקרה מהו ת"ש אלו שמתעברין עמה נפש שיש בה ארבע אמות על ארבע אמות והגשר והקבר שיש בהן בית דירה ובית הכנסת שיש בה בית דירה לחזן ובית עבודת כוכבים שיש בה בית דירה לכומרים והאורוות והאוצרות שבשדות ויש בהן בית דירה והבורגנין שבתוכה והבית שבים הרי אלו מתעברין עמה}
\textblock{ואלו שאין מתעברין עמה נפש שנפרצה משתי רוחותיה אילך ואילך והגשר והקבר שאין להן בית דירה ובית הכנסת שאין לה בית דירה לחזן ובית עבודת כוכבים שאין לה בית דירה לכומרים והאורוות והאוצרות שבשדות שאין להן בית דירה ובור ושיח ומערה וגדר ושובך שבתוכה והבית שבספינה אין אלו מתעברין עמה}
\textblock{קתני מיהת נפש שנפרצה משתי רוחותיה אילך ואילך מאי לאו דאיכא תקרה לא דליכא תקרה}
\textblock{בית שבים למאי חזי אמר רב פפא בית שעשוי לפנות בו כלים שבספינה}
\textblock{ומערה אין מתעברת עמה והתני רבי חייא מערה מתעברת עמה אמר אביי כשיש בנין על פיה}
\textblock{ותיפוק ליה משום בנין גופיה לא צריכא להשלים}
\textblock{אמר רב הונא יושבי צריפין אין מודדין להן אלא מפתח בתיהן}
\textblock{מתיב רב חסדא (במדבר לג, מט) ויחנו על הירדן מבית הישימות ואמר רבה בר בר חנה אמר רבי יוחנן לדידי חזי לי ההוא אתרא והוי תלתא פרסי על תלתא פרסי}
\textblock{ותניא כשהן נפנין אין נפנין לא לפניהם ולא לצדיהן אלא לאחריהן}
\textblock{אמר ליה רבא דגלי מדבר קאמרת כיון דכתיב בהו (במדבר ט, כ) על פי ה' יחנו ועל פי ה' יסעו כמאן דקביע להו דמי}
\textblock{אמר רב חיננא בר רב כהנא אמר רב אשי אם יש שם שלש חצירות של שני בתים הוקבעו}
\textblock{אמר רב יהודה אמר רב יושבי צריפין והולכי מדברות חייהן אינן חיים ונשיהן ובניהן אינן שלהן}
\textblock{תניא נמי הכי אליעזר איש ביריא אומר יושבי צריפין כיושבי קברים ועל בנותיהם הוא אומר (דברים כז, כא) ארור שוכב עם כל בהמה}
\textblock{מאי טעמא עולא אמר שאין להן מרחצאות ורבי יוחנן אמר מפני שמרגישין זה לזה בטבילה}
\textblock{מאי בינייהו איכא בינייהו נהרא דסמיך לביתא}
\textblock{אמר רב הונא כל עיר שאין בה ירק אין תלמיד חכם רשאי לדור בה למימרא דירק מעליא והתניא שלשה מרבין את הזבל וכופפין את הקומה ונוטלין אחד מחמש מאות ממאור עיניו של אדם ואלו הן}
\newsection{דף נו}
\textblock{פת קיבר ושכר חדש וירק לא קשיא הא בתומי וכרתי הא בשאר ירקי כדתניא שום ירק כרישין חצי ירק נראה צנון נראה סם חיים}
\textblock{והא תניא נראה צנון נראה סם המות לא קשיא כאן בעלין כאן באמהות כאן בימות החמה כאן בימות הגשמים}
\textblock{אמר רב יהודה אמר רב כל עיר שיש בה מעלות ומורדות אדם ובהמה שבה מתים בחצי ימיהן מתים ס"ד אלא אימא מזקינים בחצי ימיהן אמר רב הונא בריה דרב יהושע הני מולייתא דבי בירי ודבי נרש אזקנון}
\textblock{תנו רבנן בא לרבעה מרבעה בריבוע עולם נותן צפונה לצפון עולם ודרומה לדרום עולם וסימניך עגלה בצפון ועקרב בדרום}
\textblock{רבי יוסי אומר אם אינו יודע לרבעה בריבוע של עולם מרבעה כמין התקופה כיצד חמה יוצאה ביום ארוך ושוקעת ביום ארוך זה הוא פני צפון חמה יוצאה ביום קצר ושוקעת ביום קצר זה הוא פני דרום תקופת ניסן ותקופת תשרי חמה יוצאה בחצי מזרח ושוקעת בחצי מערב}
\textblock{שנאמר (קהלת א, ו) הולך אל דרום וסובב אל צפון הולך אל דרום ביום וסובב אל צפון בלילה סובב סובב הולך הרוח אלו פני מזרח ופני מערב פעמים מהלכתן ופעמים מסבבתן}
\textblock{אמר רב משרשיא ליתנהו להני כללי דתניא לא יצאה חמה מעולם מקרן מזרחית צפונית ושקעה בקרן מערבית צפונית ולא יצאה חמה מקרן מזרחית דרומית ושקעה בקרן מערבית דרומית}
\textblock{אמר שמואל אין תקופת ניסן נופלת אלא בארבעה רבעי היום או בתחלת היום או בתחלת הלילה או בחצי היום או בחצי הלילה}
\textblock{ואין תקופת תמוז נופלת אלא או באחת ומחצה או בשבע ומחצה בין ביום ובין בלילה ואין תקופת תשרי נופלת אלא או בשלש שעות או בתשע שעות בין ביום ובין בלילה ואין תקופת טבת נופלת אלא או בארבע ומחצה או בעשר ומחצה בין ביום ובין בלילה}
\textblock{ואין בין תקופה לתקופה אלא תשעים ואחד יום ושבע שעות ומחצה ואין תקופה מושכת מחברתה אלא חצי שעה}
\textblock{ואמר שמואל אין לך תקופת ניסן שנופלת בצדק שאינה משברת את האילנות ואין לך תקופת טבת שנופלת בצדק שאינה מייבשת את הזרעים והוא דאיתליד לבנה או בלבנה או בצדק:}
\textblock{תנו רבנן המרבע את העיר עושה אותה כמין טבלא מרובעת וחוזר ומרבע את התחומין ועושה אותן כמין טבלא מרובעת}
\textblock{וכשהוא מודד לא ימדוד מאמצע הקרן אלפים אמה מפני שהוא מפסיד את הזויות אלא מביא טבלא מרובעת: שהיא אלפים אמה על אלפים אמה ומניחה בקרן באלכסונה}
\textblock{נמצאת העיר משתכרת ארבע מאות אמות לכאן וארבע מאות אמות לכאן נמצאו תחומין משתכרין ח' מאות אמות לכאן ושמונה מאות לכאן נמצאו העיר ותחומין משתכרין אלף ומאתים לכאן ואלף ומאתים לכאן}
\textblock{אמר אביי ומשכחת לה במתא דהויא תרי אלפי אתרי אלפי}
\textblock{תניא אמר רבי אליעזר ברבי יוסי תחום ערי לוים אלפים אמה צא מהן אלף אמה מגרש נמצא מגרש רביע והשאר שדות וכרמים}
\textblock{מנא הני מילי אמר רבא דאמר קרא (במדבר לה, ד) מקיר העיר וחוצה אלף אמה סביב אמרה תורה סבב את העיר באלף נמצא מגרש רביע}
\textblock{רביע פלגא הוי אמר רבא בר אדא משוחאה אסברה לי משכחת לה במתא דהויא תרי אלפי אתרי אלפי תחום כמה הויא שיתסר קרנות כמה הויין שיתסר דל תמניא דתחומין וארבעה דקרנות כמה הוי תריסר}
\textblock{נמצא מגרש רביע טפי מתלתא נינהו}
\textblock{אייתי ארבעה דמתא שדי עלייהו אכתי תילתא הוי}
\textblock{מי סברת בריבועא קאמר בעיגולא קאמר}
\textblock{כמה מרובע יתר על העגול רביע דל רביע מינייהו פשו להו תשעה ותשעה מתלתין ושיתא ריבעא הוי}
\textblock{אביי אמר משכחת לה נמי במתא דהויא אלפא באלפא תחומין כמה הוו תמניא קרנות כמה הוי שיתסר}
\newsection{דף נז}
\textblock{דל ארבע דתחומין וארבע דקרנות כמה הוי תמניא}
\textblock{תילתא הוו מי סברת ברבועא קאמר בעיגולא קאמר כמה מרובע יתר על העגול רביע דל רביע פשו לה שיתא ושיתא מכ"ד ריבעא הוי}
\textblock{רבינא אמר מאי רביע רביע דתחומין}
\textblock{רב אשי אמר מאי רביע רביע דקרנות}
\textblock{אמר ליה רבינא לרב אשי והא סביב כתיב}
\textblock{מאי סביב סביב דקרנות דאי לא תימא הכי גבי עולה דכתיב (ויקרא א, ה) וזרקו (בני אהרן) את הדם על המזבח סביב ה"נ סביב ממש אלא מאי סביב סביב דקרנות ה"נ מאי סביב סביב דקרנות}
\textblock{א"ל רב חביבי מחוזנאה לרב אשי והא איכא מורשא דקרנתא}
\textblock{במתא עיגולתא והא ריבעוה אימור דאמרינן חזינן כמאן דמרבעא רבועי ודאי מי מרבענא}
\textblock{אמר ליה רב חנילאי מחוזנאה לרב אשי מכדי כמה מרובע יתר על העגול רביע הני תמני מאה שית מאה ושיתין ושבע נכי תילתא הוי}
\textblock{אמר ליה ה"מ בעיגולא מגו רבוע אבל באלכסונא בעינא טפי דאמר מר כל אמתא בריבוע אמתא ותרי חומשי באלכסונא:}
\textblock{{\large\emph{מתני׳}} נותנין קרפף לעיר דברי רבי מאיר וח"א לא אמרו קרפף אלא בין שתי עיירות אם יש לזו שבעים אמה ושיריים ולזו שבעים אמה ושיריים עושה קרפף את שתיהן להיות אחד}
\textblock{וכן ג' כפרים המשולשין אם יש בין שנים חיצונים מאה וארבעים ואחת ושליש עשה אמצעי את שלשתן להיות אחד:}
\textblock{{\large\emph{גמ׳}} מנא הני מילי אמר רבא דאמר קרא (במדבר לה, ד) מקיר העיר וחוצה אמרה תורה תן חוצה ואחר כך מדוד:}
\textblock{וחכ"א לא אמרו וכו': איתמר רב הונא אמר נותנין קרפף לזו וקרפף לזו חייא בר רב אמר קרפף [א'] לשתיהן}
\textblock{תנן וחכמים אומרים לא אמרו קרפף אלא בין ב' עיירות תיובתא דרב הונא}
\textblock{אמר לך רב הונא מאי קרפף תורת קרפף ולעולם קרפף לזו וקרפף לזו}
\textblock{ה"נ מסתברא מדקתני סיפא אם יש לזו ע' אמה ושיריים ולזו ע' אמה ושיריים עושה קרפף לשתיהן להיות אחד שמע מינה}
\textblock{לימא תיהוי תיובתיה דחייא בר רב אמר לך חייא בר רב}
\textblock{הא מני ר' מאיר היא}
\textblock{אי רבי מאיר היא הא תני ליה רישא נותנין קרפף לעיר דברי רבי מאיר}
\textblock{צריכא דאי מההיא הוה אמינא חד לחדא וחד לתרתי קמ"ל דלתרתי תרי יהבינן להו}
\textblock{ואי אשמעינן הכא משום דדחיקא תשמישתייהו אבל התם דלא דחיקא תשמישתייהו אימא לא צריכא}
\textblock{תנן וכן שלשה כפרים המשולשין אם יש בין שנים החיצונים מאה וארבעים ואחת אמה ושליש עושה אמצעי את שלשתן להיות אחד טעמא דאיכא אמצעי הא ליכא אמצעי לא תיובתא דרב הונא}
\textblock{אמר לך רב הונא הא אתמר עלה אמר רבה אמר רב אידי א"ר חנינא לא משולשין ממש אלא רואין כל שאילו מטיל אמצעי ביניהן ויהיו משולשין ואין בין זה לזה אלא ק"מ אמה ואחת ושליש עשה אמצעי את שלשתן להיות אחד}
\textblock{א"ל רבא לאביי כמה יהא בין חיצון לאמצעי א"ל אלפים אמה}
\textblock{והא את הוא דאמרת כוותיה דרבא בריה דרבה בר רב הונא מסתברא דאמר יותר מאלפים אמה}
\textblock{הכי השתא התם איכא בתים הכא ליכא בתים}
\textblock{ואמר ליה רבא לאביי כמה יהא בין חיצון לחיצון כמה יהא מאי נפקא לך מינה כל שאילו מכניס אמצעי ביניהן ואין בין זה לזה אלא מאה וארבעים ואחת ושליש}
\textblock{ואפילו ד' אלפים אמה א"ל אין והאמר רב הונא עיר העשויה כקשת אם יש בין ב' ראשיה פחות מארבעת אלפים אמה מודדין לה מן היתר ואם לאו מודדין לה מן הקשת}
\textblock{א"ל התם ליכא למימר מלי הכא איכא למימר מלי}
\textblock{א"ל רב ספרא לרבא הרי בני אקיסטפון דמשחינן להו תחומא מהאי גיסא דארדשיר ובני (תחומא) דארדשיר משחינן להו תחומא מהאי גיסא דאקיסטפון הא איכא דגלת דמפסקא יתר ממאה וארבעים ואחת ושליש}
\textblock{נפק אחוי ליה הנך אטמהתא דשורא דמבלעי בדגלת בע' אמה ושיריים:}
\textblock{{\large\emph{מתני׳}} אין מודדין אלא בחבל של נ' אמה לא פחות ולא יותר ולא ימדוד אלא כנגד לבו}
\textblock{היה מודד והגיע לגיא או לגדר מבליעו וחוזר למדתו הגיע להר מבליעו וחוזר למדתו}
\newsection{דף נח}
\textblock{ובלבד שלא יצא חוץ לתחום}
\textblock{אם אינו יכול להבליעו בזו אמר רבי דוסתאי בר ינאי משום ר' מאיר שמעתי שמקדרין בהרים:}
\textblock{{\large\emph{גמ׳}} מנא הני מילי אמר רב יהודה אמר רב דאמר קרא (שמות כז, יח) ארך החצר מאה באמה ורוחב חמשים בחמשים אמרה תורה בחבל של חמשים אמה מדוד}
\textblock{האי מיבעי ליה ליטול חמשים ולסבב חמשים}
\textblock{א"כ לימא קרא חמשים חמשים מאי חמשים בחמשים שמעת מינה תרתי:}
\textblock{לא פחות ולא יותר: תנא לא פחות מפני שמרבה ולא יותר מפני שממעט}
\textblock{א"ר אסי אין מודדין אלא בחבל של אפסקימא מאי אפסקימא א"ר אבא נרגילא מאי נרגילא א"ר יעקב דיקלא דחד נברא איכא דאמרי מאי אפסקימא רבי אבא אמר נרגילא רבי יעקב אמר דיקלא דחד נברא}
\textblock{תניא אמר רבי יהושע בן חנניא אין לך שיפה למדידה יותר משלשלאות של ברזל אבל מה נעשה שהרי אמרה (זכריה ב, ה) ובידו חבל מדה}
\textblock{והכתיב (יחזקאל מ, ה) וביד האיש קנה המדה ההוא לתרעי}
\textblock{תני רב יוסף שלשה חבלים הם של מגג של נצר ושל פשתן}
\textblock{של מגג לפרה דתנן כפתוה בחבל המגג ונתנוה על גב מערכתה של נצרים לסוטה דתנן ואח"כ מביא חבל המצרי וקושרו למעלה מדדיה של פשתן למדידה:}
\textblock{היה מודד והגיע: מדתני חוזר למידתו מכלל דאם אינו יכול להבליעו הולך למקום שיכול להבליעו ומבליעו וצופה כנגד מידתו וחוזר}
\textblock{תנינא להא דתנו רבנן היה מודד והגיע המידה לגיא אם יכול להבליעו בחבל של חמשים אמה מבליעו ואם לאו הולך למקום שיכול להבליעו ומבליעו וצופה וחוזר למידתו}
\textblock{אם היה גיא מעוקם מקדיר ועולה מקדיר ויורד הגיע לכותל אין אומרים יקוב הכותל אלא אומדו והולך לו}
\textblock{והא אנן תנן מבליעו וחוזר למידתו התם ניחא תשמישתא הכא לא ניחא תשמישתא}
\textblock{אמר רב יהודה אמר שמואל לא שנו אלא שאין חוט המשקולת יורד כנגדו}
\textblock{אבל חוט המשקולת יורד כנגדו מודדו מדידה יפה}
\textblock{וכמה עומקו של גיא אמר רב יוסף אלפים}
\textblock{איתיביה אביי עמוק ק' ורוחב נ' מבליעו ואם לאו אין מבליעו הוא דאמר כאחרים דתניא אחרים אומרים אפילו עמוק אלפים ורוחב נ' מבליעו}
\textblock{איכא דאמרי אמר רב יוסף אפילו יתר מאלפים כמאן דלא כת"ק ולא כאחרים}
\textblock{התם שאין חוט המשקולת יורד כנגדו הכא בחוט המשקולת יורד כנגדו}
\textblock{וכי אין חוט המשקולת יורד כנגדו עד כמה אמר אבימי ד' וכן תני רמי בר יחזקאל ד':}
\textblock{הגיע להר מבליעו וחוזר למידתו: אמר רבא לא שנו אלא בהר המתלקט י' מתוך ד' אבל בהר המתלקט י' מתוך ה' מודדו מדידה יפה}
\textblock{רב הונא בריה דרב נתן מתני לקולא אמר רבא לא שנו אלא בהר המתלקט עשרה מתוך חמש אבל בהר המתלקט עשרה מתוך ד' אומדו והולך לו:}
\textblock{ובלבד שלא יצא חוץ לתחום: מאי טעמא אמר רב כהנא גזירה שמא יאמרו מדת תחומין באה לכאן:}
\textblock{אם אינו יכול להבליעו: תנו רבנן כיצד מקדרין תחתון כנגד לבו עליון כנגד מרגלותיו אמר אביי נקיטינן אין מקדרין אלא בחבל של ארבע אמות}
\textblock{אמר ר"נ אמר רבה בר אבוה נקיטינן אין מקדרין לא בעגלה ערופה ולא בערי מקלט מפני שהן של תורה:}
\textblock{{\large\emph{מתני׳}} אין מודדין אלא מן המומחה ריבה למקום א' ומיעט למקום אחר שומעין למקום שריבה ריבה לאחד ומיעט לאחד שומעין למרובה}
\textblock{ואפילו עבד אפילו שפחה נאמנין לומר עד כאן תחום שבת שלא אמרו חכמים את הדבר להחמיר אלא להקל:}
\newsection{דף נט}
\textblock{{\large\emph{גמ׳}} למקום שריבה אין למקום שמיעט לא אימא אף למקום שריבה:}
\textblock{ריבה לאחד ומיעט לאחד כו': הא תו למה לי היינו הך הכי קאמר ריבה אחד ומיעט אחד שומעין לזה שריבה}
\textblock{אמר אביי ובלבד שלא ירבה יותר ממדת העיר באלכסונא:}
\textblock{שלא אמרו חכמים את הדבר להחמיר אלא להקל: והתניא לא אמרו חכמים את הדבר להקל אלא להחמיר}
\textblock{אמר רבינא לא להקל על דברי תורה אלא להחמיר על דברי תורה ותחומין דרבנן:}
\textblock{{\large\emph{מתני׳}} עיר של יחיד ונעשית של רבים מערבין את כולה}
\textblock{ושל רבים ונעשית של יחיד אין מערבין את כולה אא"כ עשה חוצה לה כעיר חדשה שביהודה שיש בה חמשים דיורין דברי רבי יהודה ר"ש אומר ג' חצירות של שני בתים:}
\textblock{{\large\emph{גמ׳}} היכי דמי עיר של יחיד ונעשית של רבים אמר רב יהודה כגון דאיסקרת' דריש גלותא}
\textblock{א"ל ר"נ מ"ט אילימא משום דשכיחי גבי הרמנא מדכרי אהדדי כולהו ישראל נמי בצפרא דשבתא שכיחי גבי הדדי אלא אמר רב נחמן כגון דיסקרתא דנתזואי}
\textblock{ת"ר עיר של יחיד ונעשית של רבים ורה"ר עוברת בתוכה כיצד מערבין אותה עושה לחי מכאן ולחי מכאן או קורה מכאן וקורה מכאן ונושא ונותן באמצע ואין מערבין אותה לחצאין אלא או כולה או מבוי מבוי בפני עצמו}
\textblock{היתה של רבים והרי היא של רבים}
\textblock{ואין לה אלא פתח אחד מערבין את כולה}
\textblock{מאן תנא דמיערבא רה"ר אמר רב הונא בריה דרב יהושע רבי יהודה היא דתניא יתר על כן א"ר יהודה מי שיש לו שני בתים בשני צידי רה"ר עושה לחי מכאן ולחי מכאן או קורה מכאן וקורה מכאן ונושא ונותן באמצע אמרו לו אין מערבין רה"ר בכך}
\textblock{אמר מר ואין מערבין אותה לחצאין אמר רב פפא לא אמרו אלא לארכה אבל לרחבה מערבין}
\textblock{כמאן דלא כר"ע דאי כר"ע הא אמר רגל המותרת במקומה אוסרת אפי' שלא במקומה}
\textblock{אפי' תימא ר' עקיבא עד כאן לא קאמר רבי עקיבא התם אלא בשתי חצירות זו לפנים מזו דפנימית לית לה פיתחא אחרינא אבל הכא הני נפקי בהאי פיתחא והני נפקי בהאי פיתחא}
\textblock{איכא דאמרי אמר רב פפא לא תימא לארכה הוא דלא מערבין אבל לרחבה מערבין אלא אפילו לרחבה נמי לא מערבין}
\textblock{כמאן כר"ע אפילו תימא רבנן עד כאן לא קאמרי רבנן התם אלא בשתי חצירות זו לפנים מזו דפנימית אחדא לדשא ומשתמשא אבל הכא מי מצו מסלקי רה"ר מהכא}
\textblock{אמר מר או כולה או מבוי מבוי בפני עצמו מ"ש דלחצאין דלא דאסרי אהדדי מבוי מבוי נמי אסרי אהדדי}
\textblock{הב"ע כגון דעבוד דקה וכי הא דאמר רב אידי בר אבין אמר רב חסדא אחד מבני מבוי שעשה דקה לפתחו אינו אוסר על בני מבוי:}
\textblock{היתה של רבים והרי היא כו': רבי זירא ערבה למתא דבי רבי חייא ולא שבק לה שיור א"ל אביי מאי טעמא עבד מר הכי}
\textblock{אמר ליה סבי דידה אמרי לי רב חייא בר אסי מערב כולה ואמינא ש"מ עיר של יחיד ונעשית של רבים היא}
\textblock{א"ל לדידי אמרו לי הנהו סבי ההיא אשפה הוה לה מחד גיסא והשתא דאיפניא לה אשפה הוה לה כשני פתחים ואסיר א"ל לאו אדעתאי}
\textblock{בעי מיניה רב אמי בר אדא הרפנאה מרבה סולם מכאן ופתח מכאן מהו א"ל הכי אמר רב סולם תורת פתח עליו}
\textblock{אמר להו רב נחמן לא תציתו ליה הכי אמר רב אדא אמר רב סולם תורת פתח עליו ותורת מחיצה עליו תורת מחיצה עליו כדאמרן תורת פתח עליו בסולם שבין שתי חצירות רצו אחד מערב רצו שנים מערבין}
\textblock{ומי אמר רב נחמן הכי והאמר רב נחמן אמר שמואל אנשי חצר ואנשי מרפסת ששכחו}
\newsection{דף ס}
\textblock{ולא עירבו אם יש לפניהם דקה ארבעה אינה אוסרת ואם לאו אוסרת}
\textblock{הכא במאי עסקינן בדלא גבוה מרפסת עשרה}
\textblock{ואי לא גבוה מרפסת עשרה כי קא עביד דקה מאי הוי במגופפת עד עשר אמות דכיון דעביד דקה איסתלוקי איסתלוק ליה מהכא}
\textblock{אמר רב יהודה אמר שמואל כותל שרצפה בסולמות אפילו ביתר מעשר תורת מחיצה עליו}
\textblock{רמי ליה רב ברונא לרב יהודה במעצרתא דבי רב חנינא מי אמר שמואל תורת מחיצה עליו והאמר רב נחמן אמר שמואל אנשי מרפסת ואנשי חצר ששכחו ולא עירבו אם יש לפניה דקה ארבעה אינה אוסרת ואם לאו אוסרת}
\textblock{הכא במאי עסקינן דלא גבוה מרפסת עשרה ואי לא גבוה מרפסת עשרה כי עביד דקה מאי הוי במגופפת עד עשר אמות דכיון דעביד דקה איסתלוקי איסתלק מהכא}
\textblock{הנהו בני קקונאי דאתי לקמיה דרב יוסף אמרו ליה הב לן גברא דליערב לן מאתין א"ל לאביי זיל ערב להו וחזי דלא מצווח' עלה בבי מדרשא אזל חזא להנהו בתי דפתיחי לנהרא אמר הני להוי שיור למתא}
\textblock{הדר אמר אין מערבין את כולה תנן [מכלל] דאי בעי לעירובי מצי מערבי אלא איעביד להו כווי דאי בעו לעירובי דרך חלונות מצו מערבי}
\textblock{הדר אמר לא בעי דהא רבה בר אבוה מערב לה לכולה מחוזא ערסייתא ערסייתא משום פירא דבי תורי דכל חד וחד הוי שיור לחבריה ואע"ג דאי בעו לערובי בהדי הדדי לא מצו מערבי}
\textblock{הדר אמר לא דמי התם אי בעי לערובי דרך גגות והני לא מערבי הילכך נעבדן כווי }
\textblock{הדר אמר כווי נמי לא בעי דההוא בי תיבנא דהו"ל למר בר פופידתא מפומבדיתא ושויה שיור לפומבדיתא}
\textblock{אמר היינו דאמר לי מר חזי דלא מצווחת עלה בבי מדרשא:}
\textblock{אלא א"כ עשה חוצה לה כעיר חדשה: תניא א"ר יהודה עיר אחת היתה ביהודה וחדשה שמה והיו בה נ' דיורים אנשים ונשים וטף ובה היו משערים חכמים והיא היתה שיור}
\textblock{איבעיא להו חדשה מהו חדשה כי היכי דאיהי הויא שיור לגדולה גדולה נמי הויא שיור לקטנה}
\textblock{אלא כעין חדשה מהו רב הונא ורב יהודה חד אמר בעיא שיור וחד אמר לא בעיא שיור:}
\textblock{ר"ש אומר ג' חצירות וכו': אמר רב חמא בר גוריא אמר רב הלכה כר"ש רבי יצחק אמר אפי' בית אחד וחצר אחת חצר אחת ס"ד אלא אימא בית אחד בחצר אחת}
\textblock{אמר ליה אביי לרב יוסף הא דרבי יצחק גמרא או סברא אמר ליה מאי נפקא לן מינה אמר ליה גמרא גמור זמורתא תהא:}
\textblock{{\large\emph{מתני׳}} מי שהיה במזרח ואמר לבנו ערב לי במערב במערב ואמר לבנו ערב לי במזרח אם יש הימנו ולביתו אלפים אמה ולעירובו יותר מכאן מותר לביתו ואסור לעירובו}
\textblock{לעירובו אלפים אמה ולביתו יתר מכאן אסור לביתו ומותר לעירובו}
\textblock{הנותן את עירובו בעיבורה של עיר לא עשה ולא כלום}
\textblock{נתנו חוץ לתחום אפילו אמה אחת}
\textblock{מה שנשכר הוא מפסיד:}
\textblock{{\large\emph{גמ׳}} קא סלקא דעתך למזרח למזרח ביתו למערב למערב ביתו}
\textblock{בשלמא הימנו ולביתו אלפים אמה ולעירובו יתר מכאן משכחת לה דמטי לביתיה ולא מטי לעירובו אלא הימנו ולעירובו אלפים אמה ולביתו יתר מכאן היכי משכחת לה}
\textblock{א"ר יצחק מי סברת למזרח למזרח ביתו למערב למערב ביתו לא למזרח למזרח בנו למערב למערב בנו.}
\textblock{רבא בר רב שילא אמר אפי' תימא למזרח למזרח ביתו ולמערב למערב ביתו כגון דקאי ביתיה באלכסונא:}
\textblock{הנותן עירובו בתוך עיבורה וכו': חוץ לתחום ס"ד אלא אימא חוץ לעיבורה:}
\textblock{מה שנשכר הוא מפסיד: מה שנשכר ותו לא והתניא הנותן את עירובו בתוך עיבורה של עיר לא עשה ולא כלום נתנו חוץ לעיבורה של עיר אפילו אמה אחת משתכר אותה אמה ומפסיד את כל העיר כולה מפני שמדת העיר עולה לו במדת התחום}
\textblock{לא קשיא כאן שכלתה מדתו בחצי העיר כאן שכלתה מדתו בסוף העיר}
\textblock{וכדר' אידי דאמר רבי אידי א"ר יהושע בן לוי היה מודד ובא וכלתה מדתו בחצי העיר אין לו אלא חצי העיר כלתה מדתו בסוף העיר נעשית לו העיר כולה כד' אמות ומשלימין לו את השאר}
\textblock{א"ר אידי אין אלו אלא דברי נביאות מה לי כלתה בחצי העיר מה לי כלתה בסוף העיר}
\textblock{אמר רבא תרוייהו תננהי אנשי עיר גדולה מהלכין את כל עיר קטנה}
\newsection{דף סא}
\textblock{ואין אנשי עיר קטנה מהלכין את כל עיר גדולה}
\textblock{מאי טעמא לאו משום דהני כלתה מדתן בחצי העיר והני כלתה מדתן בסוף העיר}
\textblock{ורבי אידי אנשי אנשי תני ומוקים לה בנותן אבל מודד לא תנן}
\textblock{ולא והתנן ולמודד שאמרו נותנין לו אלפים אמה שאפילו סוף מדתו כלה במערה}
\textblock{סוף העיר איצטריכא ליה דלא תנן}
\textblock{אמר רב נחמן מאן דתני אנשי לא משתבש ומאן דתני אין אנשי לא משתבש}
\textblock{מאן דתני אנשי לא משתבש דמוקים לה בנותן ומאן דתני אין אנשי לא משתבש דמוקים לה במודד}
\textblock{וחסורי מחסרא והכי קתני אנשי עיר גדולה מהלכין את כל עיר קטנה ואין אנשי עיר קטנה מהלכין את כל עיר גדולה במה דברים אמורים במודד אבל מי שהיה בעיר גדולה והניח את עירובו בעיר קטנה היה בעיר קטנה והניח את עירובו בעיר גדולה מהלך את כולה וחוצה לה אלפים אמה}
\textblock{אמר רב יוסף אמר רמי בר אבא אמר רב הונא עיר שיושבת על שפת הנחל אם יש לפניה דקה ארבעה מודדים לה משפת הנחל ואם לאו אין מודדין לה אלא מפתח ביתו}
\textblock{אמר ליה אביי דקה ארבע אמות אמרת לן עלה מאי שנא מכל דקי דעלמא דארבעה}
\textblock{א"ל התם לא בעיתא תשמישתא הכא בעיתא תשמישתא}
\textblock{אמר רב יוסף מנא אמינא לה דתניא התיר ר' שיהו בני גדר יורדין לחמתן ואין בני חמתן עולין לגדר מאי טעמא לאו משום דהני עבוד דקה והני לא עבוד דקה}
\textblock{כי אתא רב דימי אמר טטרוגי מטטרגי להו בני גדר לבני חמתן ומאי התיר התקין}
\textblock{ומאי שנא שבת דשכיחא בה שכרות}
\textblock{כי אזלי להתם נמי מטטרגי להו כלבא בלא מתיה שב שנין לא נבח}
\textblock{השתא נמי מטטרגי בני חמתן לבני גדר כולי האי לא כייפי להו}
\textblock{רב ספרא אמר עיר העשויה כקשת הואי}
\textblock{רב דימי בר חיננא אמר אנשי עיר גדולה ואנשי עיר קטנה הואי}
\textblock{רב כהנא מתני הכי רב טביומי מתני הכי רב ספרא ורב דימי בר חיננא חד אמר עיר העשויה כקשת הואי וחד אמר אנשי עיר קטנה ואנשי עיר גדולה הואי:}
\textblock{{\large\emph{מתני׳}} אנשי עיר גדולה מהלכין את כל עיר קטנה ואנשי עיר קטנה מהלכין את כל עיר גדולה כיצד מי שהיה בעיר גדולה ונתן את עירובו בעיר קטנה בעיר קטנה ונתן את עירובו בעיר גדולה מהלך את כולה וחוצה לה אלפים אמה}
\textblock{רבי עקיבא אומר אין לו אלא ממקום עירובו אלפים אמה אמר להן ר"ע אי אתם מודים לי בנותן עירובו במערה שאין לו אלא ממקום עירובו אלפים אמה}
\textblock{אמרו לו אימתי בזמן שאין בה דיורין אבל יש בה דיורין מהלך את כולה וחוצה לה אלפים אמה נמצא קל תוכה מעל גבה}
\textblock{ולמודד שאמרו נותנין אלפים אמה שאפילו סוף מדתו כלה במערה:}
\textblock{{\large\emph{גמ׳}} אמר רב יהודה אמר שמואל שבת בעיר חריבה לרבנן מהלך את כולה וחוצה לה אלפים אמה הניח את עירובו בעיר חריבה אין לו ממקום עירובו אלא אלפים אמה ר"א אומר אחד שבת ואחד הניח מהלך את כולה וחוצה לה אלפים אמה}
\textblock{מיתיבי אמר להן ר"ע אי אתם מודים לי בנותן את עירובו במערה שאין לו ממקום עירובו אלא אלפים אמה אמרו לו אימתי בזמן שאין בה דיורין הא באין בה דיורין מודו ליה}
\textblock{מאי אין בה דיורין אינה ראויה לדירה}
\textblock{תא שמע שבת בעיר אפי' היא גדולה כאנטיוכיא במערה אפילו היא כמערת צדקיהו מלך יהודה מהלך את כולה וחוצה לה אלפים אמה קתני עיר דומיא דמערה מה מערה חריבה אף עיר חריבה ושבת אין אבל הניח לא}
\textblock{מני אילימא רבי עקיבא מאי איריא חריבה אפילו ישיבה נמי אלא לאו רבנן וטעמא דשבת אין אבל הניח לא}
\textblock{לא תימא עיר דומיא דמערה אלא אימא מערה דומיא דעיר מה עיר ישיבה אף מערה ישיבה ורבי עקיבא היא דאמר אין לו ממקום עירובו אלא אלפים אמה ובשבת מודי}
\textblock{והא כמערת צדקיהו קתני כמערת צדקיהו ולא כמערת צדקיהו כמערת צדקיהו גדולה ולא כמערת צדקיהו דאילו התם חריבה והכא ישיבה}
\textblock{מר יהודה אשכחינהו לבני מברכתא דקא מותבי עירובייהו בבי כנישתא דבי אגובר אמר להו גוו ביה טפי כי היכי דלישתרי לכו טפי}
\textblock{אמר ליה רבא פלגאה בעירובין לית דחש להא דרבי עקיבא:}
\textblock{\par \par {\large\emph{הדרן עלך כיצד מעברין}}\par \par }
\textblock{}
\textblock{מתני׳ {\large\emph{הדר}} עם העכו"ם בחצר או עם מי שאינו מודה בעירוב הרי זה אוסר עליו}
\textblock{ר' אליעזר בן יעקב אומר לעולם אינו אוסר עד שיהו שני ישראלים אוסרין זה על זה}
\textblock{אמר ר"ג מעשה בצדוקי אחד שהיה דר עמנו במבוי בירושלים ואמר לנו אבא מהרו והוציאו את הכלים למבוי עד שלא יוציא ויאסר עליכם}
\textblock{}
\textblock{רבי יהודה אומר בלשון אחר מהרו ועשו צרכיכם במבוי עד שלא יוציא ויאסר עליכם:}
\newchap{פרק \hebrewnumeral{6}\quad הדר}
\newsection{דף סב}
\textblock{}
\textblock{{\large\emph{גמ׳}} יתיב אביי בר אבין ורב חיננא בר אבין ויתיב אביי גבייהו ויתבי וקאמרי בשלמא ר"מ קסבר דירת עובד כוכבים שמה דירה ולא שנא חד ולא שנא תרי}
\textblock{אלא ר' אליעזר בן יעקב מאי קסבר אי קסבר דירת עובד כוכבים שמה דירה אפילו חד נמי ניתסר ואי לא שמה דירה אפי' תרי נמי לא ניתסר}
\textblock{אמר להו אביי וסבר רבי מאיר דירת עכו"ם שמה דירה והתניא חצירו של עובד כוכבים הרי הוא כדיר של בהמה}
\textblock{אלא דכ"ע דירת עכו"ם לא שמה דירה והכא בגזירה שמא ילמד ממעשיו קא מיפלגי}
\textblock{ר' אליעזר בן יעקב סבר כיון דעובד כוכבים חשוד אשפיכות דמים תרי דשכיחי דדיירי גזרו בהו חד לא שכיח לא גזרו ביה רבנן}
\textblock{ור"מ סבר זמנין דמקרי ודייר ואמרו רבנן אין עירוב מועיל במקום עכו"ם ואין ביטול רשות מועיל במקום עכו"ם עד שישכיר ועכו"ם לא מוגר}
\textblock{מ"ט אילימא משום דסבר דלמא אתי לאחזוקי ברשותו הניחא למ"ד שכירות בריאה בעינן}
\textblock{אלא למ"ד שכירות רעועה בעינן מאי איכא למימר דאתמר רב חסדא אמר שכירות בריאה ורב ששת אמר שכירות רעועה}
\textblock{מאי רעועה מאי בריאה אילימא בריאה בפרוטה רעועה פחות משוה פרוטה מי איכא למאן דאמר מעכו"ם בפחות משוה פרוטה לא והא שלח רבי יצחק ברבי יעקב בר גיורי משמיה דרבי יוחנן הוו יודעין ששוכרין מן העכו"ם אפילו בפחות משוה פרוטה}
\textblock{ואמר רבי חייא בר אבא אמר רבי יוחנן בן נח נהרג על פחות משוה פרוטה ולא ניתן להשבון}
\textblock{אלא בריאה במוהרקי ואבורגני רעועה בלא מוהרקי ואבורגני הניחא למ"ד שכירות בריאה בעינן}
\textblock{אלא למ"ד שכירות רעועה בעינן מאי איכא למימר אפי' הכי חשיש עכו"ם לכשפים ולא מוגר}
\textblock{גופא חצירו של עובד כוכבים הרי הוא כדיר של בהמה ומותר להכניס ולהוציא מן חצר לבתים ומן בתים לחצר}
\textblock{ואם יש שם ישראל אחד אוסר דברי רבי מאיר}
\textblock{רבי אליעזר בן יעקב אומר לעולם אינו אוסר עד שיהו שני ישראלים אוסרים זה על זה}
\textblock{אמר מר חצירו של עכו"ם הרי הוא כדיר של בהמה והא אנן תנן הדר עם העכו"ם בחצר הרי זה אוסר עליו}
\textblock{לא קשיא הא דאיתיה הא דליתיה}
\textblock{ומאי קסבר אי קסבר דירה בלא בעלים שמה דירה אפי' עכו"ם נמי ניתסר ואי קסבר דירה בלא בעלים לא שמה דירה אפילו ישראל נמי לא ניתסר}
\textblock{לעולם קסבר דירה בלא בעלים לא שמה דירה וישראל דכי איתיה אסר כי ליתיה גזרו ביה רבנן}
\textblock{עכו"ם דכי איתיה גזירה שמא ילמד ממעשיו כי איתיה אסר כי ליתיה לא אסר}
\textblock{וכי ליתיה לא אסר והתנן המניח את ביתו והלך לו לשבות בעיר אחרת אחד נכרי ואחד ישראל אוסר דברי רבי מאיר}
\textblock{התם דאתי ביומיה}
\textblock{אמר רב יהודה אמר שמואל הלכה כרבי אליעזר בן יעקב ורב הונא אמר מנהג כרבי אליעזר בן יעקב ור' יוחנן אמר נהגו העם כר' אליעזר בן יעקב}
\textblock{א"ל אביי לרב יוסף קי"ל משנת רבי אליעזר בן יעקב קב ונקי ואמר רב יהודה אמר שמואל הלכה כרבי אליעזר בן יעקב}
\textblock{מהו לאורויי במקום רבו}
\textblock{א"ל אפילו ביעתא בכותחא בעו מיניה מרב חסדא כל שני דרב הונא ולא אורי}
\textblock{א"ל ר' יעקב בר אבא לאביי כגון מגלת תענית דכתיבא ומנחא מהו לאורויי באתרי דרביה א"ל הכי א"ר יוסף אפי' ביעתא בכותחא בעו מיניה מרב חסדא כל שני דרב הונא ולא אורי}
\textblock{רב חסדא אורי בכפרי בשני דרב הונא}
\newsection{דף סג}
\textblock{רב המנונא אורי בחרתא דארגז בשני דרב חסדא}
\textblock{רבינא סר סכינא בבבל א"ל רב אשי מאי טעמא עבד מר הכי}
\textblock{א"ל והא רב המנונא אורי בחרתא דארגז בשני דרב חסדא אמר ליה לאו אורי אתמר}
\textblock{אמר ליה אתמר אורי ואתמר לא אורי בשני דרב הונא רביה הוא דלא אורי ואורי בשני דרב חסדא דתלמיד חבר דיליה הוה ואנא נמי תלמיד חבר דמר אנא}
\textblock{אמר רבא צורבא מרבנן חזי לנפשיה רבינא איקלע למחוזא אייתי אושפיזכניה סכינא וקא מחוי ליה אמר ליה זיל אמטייה לרבא}
\textblock{אמר ליה לא סבר מר הא דאמר רבא צורבא מרבנן חזי לנפשיה אמר ליה אנא מיזבן זבינא}
\textblock{(סימן זיל"א להני"א מחלי"ף איק"א ויעק"ב)}
\textblock{רבי אלעזר מהגרוניא ורב אבא בר תחליפא איקלעו לבי רב אחא בריה דרב איקא באתריה דרב אחא בר יעקב בעי רב אחא בריה דרב איקא למיעבד להו עיגלא תילתא אייתי סכינא וקא מחוי להו}
\textblock{אמר להו רב אחא בר תחליפא לא ליחוש ליה לסבא אמר להו ר"א מהגרוניא הכי אמר רבא צורבא מרבנן חזי לנפשיה חזי ואיעניש רבי אלעזר מהגרוניא}
\textblock{והאמר רבא צורבא מרבנן חזי לנפשיה שאני התם דאתחילו בכבודו}
\textblock{ואי בעית אימא שאני רב אחא בר יעקב דמופלג}
\textblock{אמר רבא ולאפרושי מאיסורא אפילו בפניו שפיר דמי רבינא הוה יתיב קמיה דרב אשי חזייה לההוא גברא דקא אסר ליה לחמריה בצינתא בשבתא רמא ביה קלא ולא אשגח ביה א"ל ליהוי האי גברא בשמתא}
\textblock{א"ל כי האי גוונא מי מתחזא כאפקרותא אמר ליה (משלי כא, ל) אין חכמה ואין תבונה ואין עצה לנגד ה' כל מקום שיש בו חילול השם אין חולקין כבוד לרב}
\textblock{אמר רבא בפניו אסור וחייב מיתה שלא בפניו אסור ואין חייב מיתה}
\textblock{ושלא בפניו לא והא תניא ר"א אומר לא מתו בני אהרן עד שהורו הלכה בפני משה רבן מאי דרוש (ויקרא א, ז) ונתנו בני אהרן הכהן אש על המזבח אמרו אף על פי שהאש יורדת מן השמים מצוה להביא מן ההדיוט}
\textblock{ותלמיד אחד היה לו לרבי אליעזר שהורה הלכה בפניו אמר רבי אליעזר לאימא שלום אשתו תמיה אני אם יוציא זה שנתו ולא הוציא שנתו}
\textblock{אמרה לו נביא אתה אמר לה לא נביא אנכי ולא בן נביא אנכי אלא כך מקובלני כל המורה הלכה בפני רבו חייב מיתה}
\textblock{ואמר רבה בר בר חנה אמר רבי יוחנן אותו תלמיד יהודה בן גוריא שמו והיה רחוק ממנו שלש פרסאות}
\textblock{בפניו הוה והא רחוק ממנו שלש פרסאות קאמר וליטעמיך שמו ושם אביו למה אלא שלא תאמר משל היה}
\textblock{אמר ר' חייא בר אבא אמר רבי יוחנן כל המורה הלכה בפני רבו ראוי להכישו נחש שנאמר (איוב לב, ו) ויען אליהוא בן ברכאל הבוזי ויאמר צעיר אני לימים וגו' על כן זחלתי וכתיב (דברים לב, כד) עם חמת זוחלי עפר}
\textblock{זעירי אמר רבי חנינא נקרא חוטא שנאמר (תהלים קיט, יא) בלבי צפנתי אמרתך למען לא אחטא לך}
\textblock{רב המנונא רמי כתיב בלבי צפנתי אמרתך וכתיב (תהלים מ, י) בשרתי צדק בקהל רב לא קשיא כאן בזמן שעירא היאירי קיים כאן בזמן שאין עירא היאירי קיים}
\textblock{אמר רבי אבא בר זבדא כל הנותן מתנותיו לכהן אחד מביא רעב לעולם שנאמר (שמואל ב כ, כו) עירא היאירי היה כהן לדוד לדוד הוא דהוה כהן לכו"ע לא אלא שהיה משגר לו מתנותיו וכתיב בתריה ויהי רעב בימי דוד}
\textblock{ר' אליעזר אומר מורידין אותו מגדולתו שנאמר (במדבר לא, כא) ויאמר אלעזר הכהן אל אנשי הצבא וגו' אע"ג דאמר להו לאחי אבא צוה ואותי לא צוה אפ"ה איענש}
\textblock{דכתיב (במדבר כז, כא) ולפני אלעזר הכהן יעמד ולא אשכחן דאיצטריך ליה יהושע}
\textblock{א"ר לוי כל דמותיב מלה קמיה רביה אזיל לשאול בלא ולד שנאמר (במדבר יא, כח) ויען יהושע בן נון משרת משה מבחוריו ויאמר אדוני משה כלאם}
\textblock{וכתיב (דברי הימים א ז, כז) נון בנו יהושע בנו}
\textblock{ופליגא דר' אבא בר פפא דאמר ר' אבא בר פפא לא נענש יהושע אלא בשביל שביטל את ישראל לילה אחת מפריה ורביה}
\textblock{שנאמר (יהושע ה, יג) ויהי בהיות יהושע ביריחו וישא עיניו וירא וגו' וכתיב ויאמר (לו) כי אני שר צבא ה' עתה באתי וגו'}
\textblock{אמר לו אמש ביטלתם תמיד של בין הערבים ועכשיו ביטלתם תלמוד תורה על איזה מהן באת אמר לו עתה באתי}
\textblock{מיד (יהושע ח, יג) וילך יהושע בלילה ההוא בתוך העמק ואמר רבי יוחנן מלמד שהלך בעומקה של הלכה}
\textblock{וגמירי דכל זמן שארון ושכינה שרויין שלא במקומן אסורין בתשמיש המטה}
\textblock{א"ר שמואל בר איניא משמיה דרב גדול תלמוד תורה יותר מהקרבת תמידין דאמר ליה עתה באתי}
\textblock{אמר רב ברונא אמר רב כל הישן בקילעא שאיש ואשתו שרויין בה עליו הכתוב אומר (מיכה ב, ט) נשי עמי תגרשון מבית תענוגיה}
\textblock{ואמר רב יוסף אפי' באשתו נדה}
\textblock{רבא אמר אם אשתו נדה היא תבא עליו ברכה ולא היא דעד האידנא מאן נטריה}
\textblock{ההוא מבואה דהוה דייר בה לחמן בר ריסתק אמרו ליה אוגר לן רשותך לא אוגר להו}
\textblock{אתו אמרו ליה לאביי אמר להו זילו בטילו רשותייכו לגבי חד הוה ליה יחיד במקום נכרי ויחיד במקום נכרי לא אסר}
\textblock{אמרו ליה מידי הוא טעמא אלא דלא שכיח דדיירי והכא הא קדיירי}
\textblock{אמר להו כל בטולי רשותייהו גבי חד מילתא דלא שכיחא היא ומילתא דלא שכיחא לא גזרו בה רבנן}
\textblock{אזל רב הונא בריה דרב יהושע אמרה לשמעתא קמיה דרבא אמר ליה}
\newsection{דף סד}
\textblock{אם כן ביטלת תורת עירוב מאותו מבוי}
\textblock{דמערבי יאמרו עירוב מועיל במקום נכרי דמכרזינן}
\textblock{אכרזתא לדרדקי}
\textblock{אלא אמר רבא ליזיל חד מינייהו ליקרב ליה ולשאול מיניה דוכתא ולינח ביה מידי דהוה ליה כשכירו ולקיטו ואמר רב יהודה אמר שמואל אפילו שכירו ואפילו לקיטו נותן עירובו ודיו}
\textblock{אמר ליה אביי לרב יוסף היו שם חמשה שכירו וה' לקיטו מהו אמר ליה אם אמרו שכירו ולקיטו להקל יאמרו שכירו ולקיטו להחמיר}
\textblock{גופא אמר רב יהודה אמר שמואל אפילו שכירו ואפי' לקיטו נותן עירובו ודיו אמר רב נחמן כמה מעליא הא שמעתא}
\textblock{אמר רב יהודה אמר שמואל שתה רביעית יין אל יורה אמר רב נחמן לא מעליא הא שמעתא דהא אנא כל כמה דלא שתינא רביעתא דחמרא לא צילא דעתאי}
\textblock{אמר ליה רבא מאי טעמא אמר מר הכי האמר ר' אחא בר חנינא מאי דכתיב (משלי כט, ג) ורועה זונות יאבד הון כל האומר שמועה זו נאה וזו אינה נאה מאבד הונה של תורה אמר ליה הדרי בי}
\textblock{אמר רבה בר רב הונא שתוי אל יתפלל ואם התפלל תפלתו תפלה שיכור אל יתפלל ואם התפלל תפלתו תועבה}
\textblock{היכי דמי שתוי והיכי דמי שיכור כי הא דרבי אבא בר שומני ורב מנשיא בר ירמיה מגיפתי הוו קא מפטרי מהדדי אמעברא דנהר יופטי אמרו כל חד מינן לימא מילתא דלא שמיע לחבריה דאמר מרי בר רב הונא לא יפטר אדם מחבירו אלא מתוך דבר הלכה שמתוך כך זוכרו}
\textblock{פתח חד ואמר היכי דמי שתוי והיכי דמי שיכור שתוי כל שיכול לדבר לפני המלך שיכור כל שאינו יכול לדבר לפני המלך}
\textblock{פתח אידך ואמר המחזיק בנכסי הגר מה יעשה ויתקיימו בידו יקח בהן ספר תורה אמר רב ששת: אפילו}
\textblock{בעל בנכסי אשתו}
\textblock{רבא אמר אפילו עבד עיסקא ורווח רב פפא אמר אפי' מצא מציאה אמר רב נחמן בר יצחק אפילו כתב בהו תפילין}
\textblock{ואמר רב חנין ואיתימא ר' חנינא מאי קראה דכתיב (במדבר כא, ב) וידר ישראל נדר וגו'}
\textblock{אמר רמי בר אבא דרך מיל ושינה כל שהוא מפיגין את היין אמר רב נחמן אמר רבה בר אבוה לא שנו אלא ששתה כדי רביעית אבל שתה יותר מרביעית כל שכן שדרך טורדתו ושינה משכרתו}
\textblock{ודרך מיל מפיגה היין והתניא מעשה בר"ג שהיה רוכב על החמור והיה מהלך מעכו לכזיב והיה רבי אילעאי מהלך אחריו מצא גלוסקין בדרך אמר לו אילעאי טול גלוסקין מן הדרך מצא נכרי אחד אמר לו מבגאי טול גלוסקין הללו מאילעאי}
\textblock{ניטפל לו ר' אילעאי אמר לו מהיכן אתה אמר לו מעיירות של בורגנין ומה שמך מבגאי שמני כלום היכירך רבן גמליאל מעולם אמר לו לאו}
\textblock{באותה שעה למדנו שכוון רבן גמליאל ברוח הקודש ושלשה דברים למדנו באותה שעה למדנו שאין מעבירין על האוכלין}
\textblock{ולמדנו שהולכין אחרי רוב עוברי דרכים ולמדנו שחמצו של נכרי אחר הפסח מותר בהנאה}
\textblock{כיון שהגיע לכזיב בא אחד לישאל על נדרו אמר לזה שעמו כלום שתינו רביעית יין האיטלקי אמר לו הן אם כן יטייל אחרינו עד שיפיג יינינו}
\textblock{וטייל אחריהן ג' מילין עד שהגיע לסולמא של צור כיון שהגיע לסולמא דצור ירד ר"ג מן החמור ונתעטף וישב והתיר לו נדרו}
\textblock{והרבה דברים למדנו באותה שעה למדנו שרביעית יין האיטלקי משכר ולמדנו שיכור אל יורה ולמדנו שדרך מפיגה את היין ולמדנו שאין מפירין נדרים לא רכוב ולא מהלך ולא עומד אלא יושב}
\textblock{קתני מיהת שלשה מילין שאני יין האיטלקי דמשכר טפי}
\textblock{והאמר רב נחמן אמר רבה בר אבוה לא שנו אלא ששתה רביעית אבל שתה יותר מרביעית כל שכן דרך טורדתו ושינה משכרתו}
\textblock{רכוב שאני השתא דאתית להכי לרמי בר אבא נמי לא קשיא רכוב שאני}
\textblock{איני והאמר רב נחמן מפירין נדרים בין מהלך בין עומד ובין רכוב}
\textblock{תנאי היא דאיכא למאן דאמר פותחין בחרטה}
\textblock{ואיכא למאן דאמר אין פותחין בחרטה}
\textblock{דאמר רבה בר בר חנה אמר רבי יוחנן מאי פתח ליה רבן גמליאל לההוא גברא (משלי יב, יח) יש בוטה כמדקרות חרב ולשון חכמים מרפא כל הבוטה ראוי לדוקרו בחרב אלא שלשון חכמים מרפא}
\textblock{אמר מר ואין מעבירין על האוכלין אמר רבי יוחנן משום רבי שמעון בן יוחאי לא שנו אלא בדורות הראשונים שאין בנות ישראל פרוצות בכשפים אבל בדורות האחרונים שבנות ישראל פרוצות בכשפים מעבירין}
\textblock{תנא שלימין מעבירין פתיתין אין מעבירין אמר ליה רב אסי לרב אשי ואפתיתין לא עבדן והכתיב (יחזקאל יג, יט) ותחללנה אותי אל עמי בשעלי שעורים ובפתותי לחם דשקלי באגרייהו}
\textblock{אמר רב ששת משום רבי אלעזר בן עזריה}
\newsection{דף סה}
\textblock{יכול אני לפטור את כל העולם כולו מן הדין מיום שחרב בית המקדש ועד עכשיו שנאמר (ישעיהו נא, כא) לכן שמעי נא זאת ענייה ושכורת ולא מיין}
\textblock{מיתיבי שיכור מקחו מקח וממכרו ממכר עבר עבירה שיש בה מיתה ממיתין אותו מלקות מלקין אותו כללו של דבר הרי הוא כפיקח לכל דבריו אלא שפטור מן התפלה}
\textblock{מאי יכולני לפטור דקאמר נמי מדין תפלה}
\textblock{אמר רבי חנינא ל"ש אלא שלא הגיע לשכרותו של לוט אבל הגיע לשכרותו של לוט פטור מכולם:}
\textblock{א"ר חנינא כל המפיק מגן בשעת גאוה סוגרין וחותמין צרות בעדו שנאמר (איוב מא, ז) גאוה אפיקי מגנים סגור חותם צר}
\textblock{מאי משמע דהאי אפיק לישנא דעבורי הוא דכתיב (איוב ו, טו) אחי בגדו כמו נחל כאפיק נחלים יעבורו}
\textblock{ר' יוחנן אמר כל שאינו מפיק אתמר}
\textblock{מאי משמע דהאי מפיק לישנא דגלויי הוא דכתיב (תהלים יח, טז) ויראו אפיקי מים ויגלו מוסדות תבל}
\textblock{מכדי קראי משמע בין למר ובין למר מאי בינייהו איכא בינייהו דרב ששת: דרב ששת מסר שינתיה לשמעיה מר אית ליה דרב ששת ומר לית ליה דרב ששת}
\textblock{אמר רב חייא בר אשי אמר רב כל שאין דעתו מיושבת עליו אל יתפלל משום שנא' בצר אל יורה רבי חנינא ביומא דרתח לא מצלי אמר בצר אל יורה כתיב מר עוקבא ביומא דשותא לא הוה נפיק לבי דינא}
\textblock{אמר רב נחמן בר יצחק הלכתא בעיא צילותא כיומא דאסתנא אמר אביי אי אמרה לי אם קריב כותחא לא תנאי}
\textblock{אמר רבא אי קרצתן כינה לא תנאי מר בריה דרבינא עבדה ליה אמיה ז' מני לז' יומי}
\textblock{אמר רב יהודה לא איברי ליליא אלא לשינתא א"ר שמעון בן לקיש לא איברי סיהרא אלא לגירסא אמרי ליה לר' זירא מחדדן שמעתך אמר להו דיממי נינהו}
\textblock{אמרה ליה ברתיה דרב חסדא לרב חסדא לא בעי מר מינם פורתא אמר לה השתא אתו יומי דאריכי וקטיני ונינום טובא}
\textblock{אמר רב נחמן בר יצחק אנן פועלי דיממי אנן רב אחא בר יעקב יזיף ופרע}
\textblock{אמר ר"א הבא מן הדרך אל יתפלל שלשה ימים שנאמר (עזרא ח, טו) ואקבצם אל הנהר הבא אל אחוא ונחנה שם ימים שלשה ואבינה בעם וגו'}
\textblock{אבוה דשמואל כי אתי באורחא לא מצלי תלתא יומי שמואל לא מצלי בביתא דאית ביה שיכרא רב פפא לא מצלי בביתא דאית ביה הרסנא}
\textblock{א"ר חנינא כל המתפתה ביינו יש בו מדעת קונו שנאמר (בראשית ח, כא) וירח ה' את ריח הניחוח וגו'}
\textblock{אמר ר' חייא כל המתיישב ביינו יש בו דעת ע' זקנים יין ניתן בע' אותיות וסוד ניתן בע' אותיות נכנס יין יצא סוד}
\textblock{א"ר חנין לא נברא יין אלא לנחם אבלים ולשלם שכר לרשעים שנא' (משלי לא, ו) תנו שכר לאובד וגו'}
\textblock{א"ר חנין בר פפא כל שאין יין נשפך בתוך ביתו כמים אינו בכלל ברכה שנא' (שמות כג, כה) וברך את לחמך ואת מימיך מה לחם שניקח בכסף מעשר אף מים שניקח בכסף מעשר ומאי ניהו יין וקא קרי ליה מים}
\textblock{אי נשפך בביתו כמים איכא ברכה ואי לא לא}
\textblock{א"ר אילעאי בשלשה דברים אדם ניכר בכוסו ובכיסו ובכעסו ואמרי ליה אף בשחקו:}
\textblock{אמר רב יהודה אמר רב ישראל ונכרי בפנימית וישראל בחיצונה בא מעשה לפני רבי ואסר ולפני ר' חייא ואסר}
\textblock{יתוב רבה ורב יוסף בשילהי פירקיה דרב ששת ויתיב רב ששת וקאמר כמאן אמרה רב לשמעתיה כר' מאיר כרכיש רבה רישיה}
\textblock{אמר רב יוסף תרי גברי רברבי כרבנן ליטעו בהאי מילתא אי כרבי מאיר למה לי ישראל בחיצונה}
\textblock{וכי תימא מעשה שהיה כך היה והא בעו מיניה מרב פנימי במקומו מהו ואמר להן מותר}
\textblock{ואלא מאי כר"א בן יעקב האמר עד שיהו שני ישראלים אוסרין זה על זה}
\textblock{אלא כר"ע דאמר רגל המותרת במקומה אוסרת שלא במקומה}
\textblock{למה לי נכרי אפילו ישראל נמי}
\textblock{אמר רב הונא בריה דרב יהושע לעולם כר' אליעזר בן יעקב וכרבי עקיבא והכא במאי עסקינן כגון שעירבו וטעמא דאיכא נכרי דאסיר אבל ליכא נכרי לא אסיר}
\textblock{בעא מיניה רבי אליעזר מרב ישראל ונכרי בחיצונה וישראל בפנימית מהו התם טעמא משום דשכיח דדייר דמירתת נכרי וסבר השתא אתי ישראל ואמר לי ישראל דהוה גבך היכא}
\textblock{אבל הכא אמינא ליה נפק אזל ליה}
\textblock{או דילמא ה"נ מירתת דסבר השתא אתי ישראל וחזי לי}
\textblock{א"ל (משלי ט, ט) תן לחכם ויחכם עוד}
\textblock{ר"ל ותלמידי דרבי חנינא איקלעו לההוא פונדק ולא הוה שוכר והוה משכיר}
\textblock{אמרו מהו למיגר מיניה כל היכא דלא מצי מסליק ליה לא תיבעי לך דלא אגרינא כי תיבעי היכא דמצי מסליק ליה}
\textblock{מאי כיון דמצי מסליק אגרינא או דילמא השתא מיהא הא לא סלקיה}
\textblock{אמר להן ריש לקיש נשכור ולכשנגיע אצל רבותינו שבדרום נשאל להן אתו שיילו לר' אפס אמר להן יפה עשיתם ששכרתם}
\textblock{רבי חנינא בר יוסף ור' חייא בר אבא ור' אסי איקלעו לההוא פונדק דאתא נכרי מרי דפונדק בשבתא אמרו מהו למיגר מיניה שוכר כמערב דמי מה מערב מבעוד יום אף שוכר מבעוד יום}
\textblock{או דילמא שוכר כמבטל רשות דמי מה מבטל רשות ואפילו בשבת אף שוכר ואפילו בשבת}
\textblock{רבי חנינא בר יוסף אמר נשכור ור' אסי אמר לא נשכור אמר להו ר' חייא בר אבא נסמוך על דברי זקן ונשכור אתו שיילו ליה לרבי יוחנן אמר להן}
\newsection{דף סו}
\textblock{יפה עשיתם ששכרתם תהו בה נהרדעי ומי א"ר יוחנן הכי והא"ר יוחנן שוכר כמערב דמי מאי לאו מה מערב מבעוד יום אף שוכר מבעוד יום}
\textblock{לא מה מערב ואפילו בפחות משוה פרוטה אף שוכר בפחות משוה פרוטה ומה מערב אפילו שכירו ולקיטו אף שוכר אפילו שכירו ולקיטו}
\textblock{ומה מערב חמשה ששרוין בחצר אחת אחד מערב ע"י כולן שוכר נמי חמשה ששרוין בחצר אחת אחד שוכר ע"י כולן}
\textblock{תהי בה רבי אלעזר אמר רבי זירא מאי תהייא דר"א אמר רב ששת גברא רבה כרבי זירא לא ידע מאי תהייא דר"א קא קשיא ליה דשמואל רביה}
\textblock{דאמר שמואל כל מקום שאוסרין ומערבין מבטלין מערבין ואין אוסרין אוסרין ואין מערבין אין מבטלין}
\textblock{כל מקום שאוסרין ומערבין מבטלין כגון ב' חצירות זו לפנים מזו}
\textblock{מערבין ואין אוסרין אין מבטלין כגון ב' חצירות ופתח א' ביניהן}
\textblock{אוסרין ואין מערבין אין מבטלין לאתויי מאי לאו לאתויי נכרי}
\textblock{ואי דאתא מאתמול לוגר מאתמול}
\textblock{אלא לאו דאתא בשבתא וקתני אוסרין ואין מערבין אין מבטלין שמע מינה}
\textblock{אמר רב יוסף לא שמיע לי הא שמעתא אמר ליה אביי את אמרת ניהלן ואהא אמרת ניהלן דאמר שמואל אין ביטול רשות מחצר לחצר}
\textblock{ואין ביטול רשות בחורבה}
\textblock{ואמרת לן עלה כי אמר שמואל אין ביטול רשות מחצר לחצר לא אמרן אלא שתי חצירות ופתח אחד ביניהן אבל זו לפנים מזו מתוך שאוסרין זה על זה מבטלין}
\textblock{א"ל אנא אמינא משמיה דשמואל הכי והאמר שמואל אין לנו בעירובין אלא כלשון משנתנו אנשי חצר ולא אנשי חצירות}
\textblock{אמר ליה כי אמרת לן אין לנו בעירובין אלא כלשון משנתנו אהא אמרת לן שהמבוי לחצירות כחצר לבתים}
\textblock{גופא אמר שמואל אין ביטול רשות מחצר לחצר ואין ביטול רשות בחורבה ורבי יוחנן אמר יש ביטול רשות מחצר לחצר ויש ביטול רשות בחורבה}
\textblock{וצריכא דאי אשמעינן מחצר לחצר בהא קאמר שמואל משום דהא תשמישתא לחוד והא תשמישתא לחוד אבל חורבה דתשמישתא חדא לתרווייהו אימא מודי ליה לרבי יוחנן}
\textblock{וכי אתמר בהא בהא קאמר רבי יוחנן אבל בהך מודי ליה לשמואל צריכא}
\textblock{אמר אביי הא דאמר שמואל אין ביטול רשות מחצר לחצר לא אמרן אלא בשתי חצירות ופתח אחד ביניהן אבל ב' חצירות זו לפנים מזו מתוך שאוסרין מבטלין}
\textblock{רבא אמר אפילו שתי חצירות זו לפנים מזו פעמים מבטלין ופעמים אין מבטלין כיצד נתנו עירובן בחיצונה ושכח אחד בין מן הפנימית ובין מן החיצונה ולא עירב שתיהן אסורות}
\textblock{נתנו עירובן בפנימית ושכח אחד מן הפנימית ולא עירב שתיהן אסורות}
\textblock{שכח אחד מן החיצונה ולא עירב פנימית מותרת וחיצונה אסורה}
\textblock{נתנו עירובן בחיצונה ושכח אחד בין מן הפנימית ובין מן החיצונה ולא עירב שתיהן אסורות האי בר פנימית למאן ניבטיל ליבטיל לבני פנימית ליתא לערובייהו גבייהו ליבטיל לבני חיצונה אין בטול רשות מחצר לחצר}
\textblock{האי בר חיצונה למאן נבטיל ליבטיל לבני חיצונה איכא פנימית דאסרה עלייהו ליבטיל לבני פנימית אין ביטול רשות מחצר לחצר}
\textblock{נתנו עירובן בפנימית ושכח אחד מן הפנימית ולא עירב שתיהן אסורות האי בר פנימית למאן נבטיל ליבטיל לבני הפנימית איכא חיצונה דאסרה עלייהו ליבטיל לבני חיצונה אין ביטול רשות מחצר לחצר}
\newsection{דף סז}
\textblock{שכח אחד מן החיצונה ולא עירב ודאי פנימית מותרת דאחדא דשא ומשתמשא וחיצונה אסורה}
\textblock{אמר ליה רב הונא בריה דרב יהושע לרבא וכי שכח אחד מן הפנימית ולא עירב אמאי שתיהן אסורות לבטיל בר פנימית לבני פנימית ותיתי חיצונה ותשתרי בהדייהו}
\textblock{כמאן כרבי אליעזר דאמר אינו צריך לבטל רשות לכל אחד ואחד כי קאמינא לרבנן דאמרי צריך לבטל לכל אחד ואחד:}
\textblock{רב חסדא ורב ששת כי פגעי בהדי הדדי רב חסדא מרתען שיפוותיה ממתנייתא דרב ששת ורב ששת מרתע כוליה גופיה מפלפוליה דרב חסדא}
\textblock{בעא מיניה רב חסדא מרב ששת שני בתים משני צידי רשות הרבים ובאו נכרים והקיפום מחיצה בשבת מהו}
\textblock{אליבא דמאן דאמר אין ביטול רשות מחצר לחצר לא תיבעי לך השתא דאי בעו לערובי מאתמול מצו מערבי אמרת אין ביטול רשות מחצר לחצר הכא דאי בעו לערובי מאתמול לא מצו מערבי לא כל שכן}
\textblock{כי תיבעי לך אליבא דמאן דאמר יש ביטול רשות מחצר לחצר התם דאי בעו לערובי מאתמול מצו מערבי בטולי נמי מצי מבטל אבל הכא דלא מצו מערבי מאתמול בטולי נמי לא מצי מבטל}
\textblock{או דילמא לא שנא אמר ליה אין מבטלין}
\textblock{מת נכרי בשבת מהו}
\textblock{אליבא דמאן דאמר שוכרין לא תיבעי לך השתא תרתי עבדינן חדא מיבעיא}
\textblock{אלא כי תיבעי לך אליבא דמ"ד אין שוכרין תרתי הוא דלא עבדינן הא חדא עבדינן או דלמא לא שנא אמר ליה אני אומר מבטלין והמנונא אמר אין מבטלין:}
\textblock{אמר רב יהודה אמר שמואל נכרי שיש לו פתח ארבעה על ארבעה פתוח לבקעה אפילו מכניס ומוציא גמלים וקרונות כל היום כולו דרך מבוי אין אוסר על בני מבוי}
\textblock{מ"ט בפיתחא דמיחד ליה בההוא ניחא ליה}
\textblock{איבעיא להו פתוח לקרפף מהו א"ר נחמן בר אמי משמיה דאולפנא}
\textblock{אפי' פתוח לקרפף}
\textblock{רבה ורב יוסף דאמרי תרוייהו נכרי בית סאתים אוסר יותר מבית סאתים אינו אוסר}
\textblock{וישראל בית סאתים אינו אוסר}
\textblock{יותר מבית סאתים אוסר}
\textblock{בעא מיניה רבא בר חקלאי מרב הונא פתוח לקרפף מהו א"ל הרי אמרו בית סאתים אוסר יותר מבית סאתים אינו אוסר}
\textblock{אמר עולא אמר רבי יוחנן קרפף יותר מבית סאתים שלא הוקף לדירה ואפילו כור ואפילו כוריים הזורק לתוכו חייב מ"ט מחיצה היא אלא שמחוסרת דיורין}
\textblock{מתיב רב הונא בר חיננא סלע שבים גבוה עשרה ורוחב ארבעה אין מטלטלין לא מן תוכו לים ולא מן הים לתוכו פחות מכאן מטלטלין עד כמה עד בית סאתים}
\textblock{אהייא אילימא אסיפא בית סאתים טפי לא והא מכרמלית לכרמלית קא מטלטל}
\textblock{אלא לאו ארישא והכי קאמר סלע שבים גבוה עשרה ורוחב ד' אין מטלטלין לא מתוכו לים ולא מן הים לתוכו ועד כמה עד בית סאתים הא יתר מבית סאתים מטלטלין אלמא כרמלית היא תיובתא דר' יוחנן}
\textblock{אמר רבא מאן דלא ידע תרוצי מתנייתא תיובתא מותיב ליה לרבי יוחנן לעולם ארישא והכי קאמר הא בתוכו מטלטלין ועד כמה עד בית סאתים}
\textblock{רב אשי אמר לעולם ארישא הן אמרו והן אמרו}
\textblock{הן אמרו קרפף יתר מבית סאתים שלא הוקף לדירה אין מטלטלין בו אלא בד' אמות והן אמרו אין מטלטלין מרשות היחיד לכרמלית}
\textblock{בית סאתים דשרי לטלטולי בכוליה אסרי רבנן לטלטולי לא מן הים לתוכו ולא מתוכו לים מ"ט רה"י גמורה היא}
\textblock{יתר מבית סאתים דאסור לטלטולי בכוליה שרו רבנן לטלטולי מתוכו לים ומן הים לתוכו מ"ט דלמא אמרי רה"י גמורה היא ואתי לטלטולי בכוליה}
\textblock{ומאי שנא תוכו שכיח מתוכו לים ומן הים לתוכו לא שכיח:}
\textblock{ההוא ינוקא דאשתפיך חמימיה אמר להו רבה נייתו ליה חמימי מגו ביתאי א"ל אביי והא לא ערבינן}
\textblock{א"ל נסמוך אשיתוף א"ל הא לא שתפינן נימרו ליה לנכרי ליתי ליה}
\textblock{אמר אביי בעי לאותביה למר ולא שבקן רב יוסף דאמר רב [יוסף אמר רב] כהנא כי הוינן בי רב יהודה הוה אמר לן בדאורייתא מותבינן תיובתא והדר עבדינן מעשה בדרבנן עבדינן מעשה והדר מותבינן תיובתא}
\textblock{לבתר הכי אמר ליה מאי בעית לאותביה למר אמר [ליה דתניא] הזאה שבות ואמירה לנכרי שבות}
\newsection{דף סח}
\textblock{מה הזאה שבות ואינה דוחה את השבת אף אמירה לנכרי שבות ואינה דוחה את השבת}
\textblock{א"ל ולא שני לך בין שבות דאית ביה מעשה לשבות דלית ביה מעשה דהא מר לא אמר לנכרי זיל אחים}
\textblock{א"ל רבה בר רב חנן לאביי מבואה דאית ביה תרי גברי רברבי כרבנן לא ליהוי ביה לא עירוב ולא שיתוף א"ל מאי נעביד מר לאו אורחיה אנא טרידנא בגירסאי אינהו לא משגחי}
\textblock{ואי אקני להו פיתא בסלא כיון דאי בעו לה מינאי ולא אפשר ליתבה נהלייהו בטיל שיתוף}
\textblock{דתניא אחד מבני מבוי שביקש יין ושמן ולא נתנו לו בטל השיתוף}
\textblock{ונקני להו מר רביעתא דחלא בחביתא תניא אין משתתפין באוצר}
\textblock{והא תניא משתתפין אמר רב אושעיא ל"ק הא ב"ש הא ב"ה}
\textblock{דתנן המת בבית ולו פתחים הרבה כולן טמאין}
\textblock{נפתח אחד מהן הוא טמא וכולן טהורין חישב להוציאו באחד מהן או בחלון שיש בו ארבעה על ארבעה מציל על כל הפתחים כולן}
\textblock{בית שמאי אומרים והוא שחישב עד שלא ימות המת וב"ה אומרים אף משימות המת}
\textblock{ההוא ינוקא דאישתפוך חמימיה אמר להו רבא נישיילה לאימיה אי צריכא נחים ליה נכרי אגב אימיה}
\textblock{אמר ליה רב משרשיא לרבא אימיה קא אכלה תמרי א"ל אימור תונבא בעלמא הוא דנקט לה}
\textblock{ההוא ינוקא דאישתפוך חמימיה אמר להו רבא פנו לי מאני מבי גברי לבי נשי ואיזיל ואיתיב התם ואיבטיל להו הא חצר}
\textblock{א"ל רבינא לרבא והאמר שמואל אין ביטול רשות מחצר לחצר אמר ליה אנא כר' יוחנן סבירא לי דאמר יש ביטול מחצר לחצר}
\textblock{ואי לא סבר לה מר כשמואל}
\textblock{ניתיב מר בדוכתיה וניבטיל להו לדידהו וניהדרו אינהו וניבטלו ליה למר דהא אמר רב מבטלין וחוזרין ומבטלין}
\textblock{אנא בהא כשמואל סבירא לי דאמר אין מבטלין וחוזרין ומבטלין}
\textblock{ולאו חד טעמא הוא מ"ט אין מבטלין וחוזרין ומבטלין לאו משום דכיון דבטליה לרשותיה אסתלק ליה מהכא לגמרי והוה ליה כבן חצר אחרת ואין ביטול רשות מחצר לחצר מר נמי לא ניבטיל}
\textblock{התם היינו טעמא כי היכי דלא ליהוי מלתא דרבנן כחוכא ואטלולא}
\textblock{גופא רב אמר מבטלין וחוזרין ומבטלין ושמואל אמר אין מבטלין וחוזרין ומבטלין}
\textblock{לימא רב ושמואל בפלוגתא דרבנן ור' אליעזר קא מיפלגי}
\textblock{דרב דאמר כרבנן ושמואל דאמר כר"א}
\textblock{אמר לך רב אנא דאמרי אפי' לרבי אליעזר עד כאן לא קאמר רבי אליעזר התם המבטל רשות חצירו רשות ביתו ביטל משום דבבית בלא חצר לא דיירי אינשי אבל לענין איסתלוקי מי אמר}
\textblock{ושמואל אמר אנא דאמרי אפילו כרבנן עד כאן לא קאמרי רבנן התם אלא מאי דבטיל בטיל ודלא בטיל לא בטיל אבל מאי דבטיל מיהא איסתלק לגמרי}
\textblock{אמר רב אחא בר חנא אמר רב ששת כתנאי מי שנתן רשותו והוציא בין בשוגג בין במזיד אוסר דברי ר"מ רבי יהודה אומר במזיד אוסר בשוגג אינו אוסר}
\textblock{מאי לאו בהא קמיפלגי דמר סבר מבטלין וחוזרין ומבטלין ומר סבר אין מבטלין וחוזרין ומבטלין}
\textblock{אמר רב אחא בר תחליפא משמיה דרבא לא דכ"ע אין מבטלין וחוזרין ומבטלין והכא בקנסו שוגג אטו מזיד קא מיפלגי מ"ס קנסו שוגג אטו מזיד ומר סבר לא קנסו שוגג אטו מזיד}
\textblock{רב אשי אמר רב ושמואל בפלוגתא דר"א ורבנן קא מיפלגי:}
\textblock{אמר רבן גמליאל מעשה בצדוקי אחד שהיה דר עמנו: צדוקי מאן דכר שמיה}
\textblock{חסורי מיחסרא והכי קתני צדוקי הרי הוא כנכרי ורבן גמליאל אומר צדוקי אינו כנכרי ואמר רבן גמליאל מעשה בצדוקי אחד שהיה דר עמנו במבוי בירושלים ואמר לנו אבא מהרו והוציאו את הכלים למבוי עד שלא יוציא ויאסר עליכם}
\textblock{והתניא הדר עם נכרי צדוקי וביתוסי הרי אלו אוסרין עליו (רבן גמליאל אומר צדוקי וביתוסי אינן אוסרין) ומעשה בצדוקי אחד שהיה דר עם רבן גמליאל במבוי בירושלים ואמר להם רבן גמליאל לבניו בני מהרו והוציאו מה שאתם מוציאין והכניסו מה שאתם מכניסין עד שלא יוציא התועב הזה ויאסר עליכם שהרי ביטל רשותו לכם דברי רבי מאיר}
\textblock{רבי יהודה אומר בלשון אחרת מהרו ועשו צורכיכם במבוי עד שלא תחשך ויאסר עליכם}
\textblock{אמר מר הוציאו מה שאתם מוציאין והכניסו מה שאתם מכניסין עד שלא יוציא התועב הזה ויאסר עליכם למימרא דכי מפקי אינהו והדר מפיק איהו לא אסר}
\newsection{דף סט}
\textblock{והתנן מי שנתן רשותו והוציא בין בשוגג בין במזיד אוסר דברי רבי מאיר}
\textblock{אמר רב יוסף אימא אינו אוסר אביי אמר לא קשיא כאן שהחזיקו בני מבוי במבוי כאן שלא החזיקו בני מבוי במבוי}
\textblock{והתניא עד שלא נתן רשותו הוציא בין בשוגג בין במזיד יכול לבטל דברי רבי מאיר רבי יהודה אומר בשוגג יכול לבטל במזיד אינו יכול לבטל}
\textblock{מי שנתן רשותו והוציא בין בשוגג בין במזיד אוסר דברי רבי מאיר רבי יהודה אומר במזיד אוסר בשוגג אינו אוסר במה דברים אמורים בשלא החזיקו בני מבוי במבוי אבל החזיקו בני מבוי במבוי בין בשוגג ובין במזיד אינו אוסר:}
\textblock{אמר מר ר' יהודה אומר בלשון אחרת מהרו ועשו צורכיכם במבוי עד שלא תחשך ויאסר עליכם אלמא נכרי הוא והא אנן עד שלא יוציא תנן}
\textblock{אימא עד שלא יוציא היום ואיבעית אימא לא קשיא כאן במומר לחלל שבתות בצנעא כאן במומר לחלל שבתות בפרהסיא}
\textblock{כמאן אזלא הא דתניא מומר וגילוי פנים הרי זה אינו מבטל רשות גילוי פנים מומר הוי}
\textblock{אלא מומר בגילוי פנים אינו יכול לבטל רשות כמאן כר' יהודה}
\textblock{ההוא דנפק בחומרתא דמדושא כיון דחזייה לר' יהודה נשיאה כסייה אמר כגון זה מבטל רשות לר' יהודה}
\textblock{אמר רב הונא איזהו ישראל מומר זה המחלל שבתות בפרהסיא א"ל רב נחמן כמאן אי כר"מ דאמר חשוד לדבר א' חשוד לכל התורה כולה אפי' בא' מכל איסורין שבתורה נמי}
\textblock{אי כרבנן האמרי חשוד לדבר א' לא הוי חשוד לכל התורה כולה}
\textblock{עד דהוי מומר לעבודת כוכבים}
\textblock{אמר רב נחמן בר יצחק ליתן רשות ולבטל רשות וכדתניא ישראל מומר משמר שבתו בשוק מבטל רשות שאינו משמר שבתו בשוק אינו מבטל רשות}
\textblock{מפני שאמרו ישראל נוטל רשות ונותן רשות ובנכרי עד שישכיר כיצד אומר לו רשותי קנויה לך רשותי מבוטלת לך קנה ואין צריך לזכות}
\textblock{רב אשי אמר האי תנא הוא דחמירא עליה שבת כע"ז}
\textblock{כדתניא (ויקרא א, ב) מכם ולא כולכם פרט למומר מכם בכם חלקתי ולא באומות}
\textblock{מן הבהמה להביא בני אדם הדומין לבהמה מכאן אמרו מקבלין קרבנות מפושעי ישראל כדי שיחזרו בתשובה חוץ מן המומר והמנסך יין והמחלל שבתות בפרהסיא}
\textblock{הא גופא קשיא אמרת מכם ולא כולכם להוציא את המומר והדר תני מקבלין קרבנות מפושעי ישראל הא לא קשיא רישא במומר לכל התורה כולה מציעתא במומר לדבר אחד}
\textblock{אימא סיפא חוץ מן המומר והמנסך יין האי מומר היכי דמי אי מומר לכל התורה היינו רישא אי לדבר אחד קשיא מציעתא}
\textblock{אלא לאו הכי קאמר חוץ מן המומר לנסך ולחלל שבתות בפרהסיא אלמא ע"ז ושבת כי הדדי נינהו שמע מינה:}
\textblock{{\large\emph{מתני׳}} אנשי חצר ששכח אחד מהן ולא עירב ביתו אסור מלהכניס ומלהוציא לו ולהם ושלהם מותרין לו ולהם נתנו לו רשותן הוא מותר והן אסורין}
\textblock{היו שנים אוסרין זה על זה שאחד נותן רשות ונוטל רשות שנים נותנין רשות ואין נוטלין רשות}
\textblock{מאימתי נותנין רשות ב"ש אומרים מבעוד יום וב"ה אומרים משחשיכה מי שנתן רשותו והוציא בין בשוגג בין במזיד ה"ז אוסר דברי ר' מאיר ר' יהודה אומר במזיד אוסר בשוגג אינו אוסר:}
\textblock{{\large\emph{גמ׳}} ביתו הוא דאסור הא חצירו שריא}
\textblock{היכי דמי אי דבטיל ביתו אמאי אסור אי דלא בטיל חצירו אמאי שריא הכא במאי עסקינן כגון שביטל רשות חצירו ולא ביטל רשות ביתו וקא סברי רבנן המבטל רשות חצירו רשות ביתו לא ביטל דדייר איניש בבית בלא חצר}
\textblock{ושלהן מותר לו ולהן מאי טעמא דהוי אורח לגבייהו:}
\textblock{נתנו לו רשותן הוא מותר והן אסורין: ונהוי אינהו לגביה כי אורחין חד לגבי חמשה הוי אורח חמשה לגבי חד לא הוי אורח}
\textblock{ש"מ מבטלין וחוזרין ומבטלין}
\textblock{הכי קאמר נתנו לו רשותן מעיקרא הוא מותר והן אסורין:}
\textblock{היו שנים אוסרין זה על זה פשיטא לא צריכא דהדר חד מינייהו ובטיל ליה לחבריה מהו דתימא לישתרי קמ"ל דכיון דבעידנא דבטיל לא הוה ליה שריותא בהאי חצר:}
\textblock{שאחד נותן רשות הא תו למה לי אי נותן תנינא אי נוטל תנינא}
\textblock{סיפא איצטריכא ליה שנים נותנין רשות הא נמי פשיטא מהו דתימא}
\newsection{דף ע}
\textblock{ליגזר דילמא אתי לבטולי להו קמ"ל:}
\textblock{ואין נוטלין רשות: למה לי לא צריכא אף על גב דאמרי ליה קני על מנת להקנות}
\textblock{בעא מיניה אביי מרבה חמשה ששרויין בחצר אחת ושכח אחד מהן ולא עירב כשהוא מבטל רשותו צריך לבטל לכל אחד ואחד או לא א"ל צריך לבטל לכל אחד ואחד}
\textblock{איתיביה אחד שלא עירב נותן רשותו לאחד שעירב שנים שעירבו נותנין רשותן לאחד שלא עירב ושנים שלא עירבו נותנין רשותן לשנים שעירבו או לאחד שלא עירב}
\textblock{אבל לא אחד שעירב נותן רשותו לאחד שלא עירב ואין שנים שעירבו נותנין רשותן לשנים שלא עירבו ואין שנים שלא עירבו נותנין רשותן לשנים שלא עירבו}
\textblock{קתני מיהת רישא אחד שלא עירב נותן רשותו לאחד שעירב ה"ד אי דליכא אחרינא בהדיה בהדי מאן עירב}
\textblock{אלא פשיטא דאיכא אחרינא בהדיה וקתני לאחד שעירב}
\textblock{ורבה הכא במאי עסקינן דהוה ומית}
\textblock{אי דהוה ומית אימא סיפא אבל אין אחד שעירב נותן רשותו לאחד שלא עירב ואי דהוה ומית אמאי לא}
\textblock{אלא פשיטא דאיתיה ומדסיפא איתיה רישא נמי איתיה}
\textblock{מידי איריא הא כדאיתא והא כדאיתא}
\textblock{תדע דקתני סיפא דרישא ושנים שלא עירבו נותנין רשותן לשנים שעירבו לשנים אין לאחד לא}
\textblock{ואביי אמר מאי לב' לאחד מב' אי הכי ליתני לאחד שעירב או לאחד שלא עירב קשיא}
\textblock{אחד שלא עירב נותן רשותו לאחד שעירב לאביי דאיתיה וקמ"ל דאין צריך לבטל רשות לכל אחד ואחד לרבה דהוה ומית ולא גזור זימנין דאיתיה}
\textblock{ושנים שעירבו נותנין רשותן לאחד שלא עירב פשיטא מהו דתימא כיון דלא עירב ליקנסיה קמ"ל}
\textblock{וב' שלא עירבו נותנין רשותן לשנים שעירבו לרבה תנא סיפא לגלויי רישא לאביי ב' שלא עירבו איצטריכא ליה סד"א לגזר דלמא אתי לבטולי להו קמ"ל}
\textblock{או לאחד שלא עירב למה לי מהו דתימא הני מילי היכא דמקצתן עירבו ומקצתן לא עירבו אבל היכא דכולן לא עירבו ליקנסינהו כדי שלא תשתכח תורת עירוב קמ"ל}
\textblock{אבל אין אחד שעירב נותן רשותו לאחד שלא עירב לאביי תנא סיפא לגלויי רישא לרבה איידי דתנא רישא תנא נמי סיפא}
\textblock{ואין שנים שעירבו נותנין רשותן לשנים שלא עירבו הא תו למה לי לא צריכא דבטיל ליה חד מינייהו לחבריה מהו דתימא לשתרי ליה קמ"ל כיון דבעידנא דבטיל לא הוו ליה שריותא בהא חצר לא}
\textblock{ואין שנים שלא עירבו נותנין רשותן לשנים שלא עירבו הא תו למה לי לא צריכא דאמרי קני על מנת להקנות}
\textblock{בעא מיניה רבא מרב נחמן יורש מהו שיבטל רשות}
\textblock{היכא דאי בעי לערובי מאתמול מצי מערב בטולי נמי מצי מבטל אבל האי כיון דאי בעי לערובי מאתמול לא מצי מערב לא מצי מבטל}
\textblock{או דלמא יורש כרעיה דאבוה הוא}
\textblock{א"ל אני אומר מבטל והני דבי שמואל תנו אין מבטל איתיביה זה הכלל כל שמותר למקצת שבת הותר לכל השבת וכל שנאסר למקצת שבת נאסר לכל השבת חוץ ממבטל רשות}
\textblock{כל שהותר למקצת שבת מותר לכל השבת כגון עירב דרך הפתח ונסתם הפתח עירב דרך חלון ונסתם חלון}
\textblock{זה הכלל לאתויי מבוי שניטלו קורותיו או לחייו}
\textblock{כל שנאסר למקצת שבת נאסר לכל השבת כולה כגון שני בתים בשני צידי רה"ר והקיפום נכרים מחיצה בשבת}
\textblock{זה הכלל לאתויי מאי לאתויי מת נכרי בשבת}
\textblock{וקתני חוץ ממבטל רשות איהו אין יורש לא}
\textblock{אימא חוץ מתורת ביטול רשות}
\textblock{איתיביה אחד מבני חצר שמת והניח רשותו לאחד מן השוק מבעוד יום אוסר משחשיכה אינו אוסר}
\textblock{ואחד מן השוק שמת והניח רשותו לאחד מבני חצר מבעוד יום אינו אוסר משחשיכה אוסר}
\textblock{אמאי אוסר ניבטיל מאי אוסר נמי דקתני עד שיבטל}
\textblock{תא שמע ישראל וגר שרויין במגורה אחת ומת גר מבעוד יום}
\newsection{דף עא}
\textblock{אע"פ שהחזיק ישראל אחר בנכסיו אוסר משחשיכה אע"פ שלא החזיק ישראל אחר אינו אוסר}
\textblock{הא גופא קשיא אמרת מבעוד יום אע"פ שהחזיק ולא מיבעיא כי לא החזיק אדרבה כי לא החזיק לא אסר}
\textblock{אמר רב פפא אימא אע"פ שלא החזיק והא אע"פ שהחזיק קתני}
\textblock{ה"ק אע"פ שלא החזיק מבעוד יום אלא משחשיכה כיון דהוה ליה להחזיק מבעוד יום אוסר משחשיכה אע"פ שלא החזיק ישראל אחר אינו אוסר}
\textblock{אע"פ שלא החזיק ישראל אחר ולא מיבעיא כי החזיק אדרבה כי החזיק אסר}
\textblock{אמר רב פפא אימא אע"פ שהחזיק והא אע"פ שלא החזיק קתני ה"ק אע"פ שהחזיק משחשיכה כיון דלא הוה ליה להחזיק מבעוד יום אינו אוסר}
\textblock{קתני מיהת רישא אוסר אמאי אוסר ניבטל}
\textblock{מאי אוסר דקתני עד שיבטל}
\textblock{ר' יוחנן אמר מתני' מני ב"ש היא דאמרי אין ביטול רשות בשבת דתנן מאימתי נותנין רשות ב"ש אומרים מבעוד יום וב"ה אומרים משתחשך}
\textblock{אמר עולא מ"ט דב"ה נעשה כאומר כלך אצל יפות}
\textblock{אמר אביי מת נכרי בשבת מאי כלך אצל יפות איכא}
\textblock{אלא הכא בהא קמיפלגי דב"ש סברי ביטול רשות מיקנא רשותא הוא ומיקנא רשותא בשבת אסור וב"ה סברי אסתלוקי רשותא בעלמא הוא ואסתלוקי רשותא בשבת שפיר דמי:}
\textblock{{\large\emph{מתני׳}} בעל הבית שהיה שותף לשכניו לזה ביין ולזה ביין אינן צריכין לערב}
\textblock{לזה ביין ולזה בשמן צריכין לערב ר"ש אומר אחד זה ואחד זה אינן צריכין לערב:}
\textblock{{\large\emph{גמ׳}} אמר רב ובכלי אחד אמר רבא דיקא נמי דקתני לזה ביין ולזה בשמן צריכין לערב אי אמרת בשלמא רישא בכלי אחד וסיפא בשני כלים שפיר אלא אי אמרת רישא בשני כלים וסיפא בשני כלים מה לי יין ויין מה לי יין ושמן}
\textblock{א"ל אביי יין ויין ראוי לערב יין ושמן אין ראוי לערב:}
\textblock{ר"ש אומר אחד זה ואחד זה אין צריכין לערב: ואפילו לזה ביין ולזה בשמן אמר רבה הכא במאי עסקינן בחצר שבין שני מבואות ור"ש לטעמיה}
\textblock{דתנן אמר ר"ש למה הדבר דומה לשלש חצירות הפתוחות זו לזו ופתוחות לרה"ר עירבו שתים החיצונות עם האמצעית היא מותרת עמהן והן מותרות עמה ושתים החיצונות אסורות זו עם זו}
\textblock{א"ל אביי מי דמי התם קתני שתים החיצונות אסורות הכא קתני אין צריכין לערב כלל}
\textblock{מאי אין צריכין לערב שכנים בהדי בעל הבית אבל שכנים בהדי הדדי צריכין לערב}
\textblock{ורב יוסף אמר רבי שמעון ורבנן בפלוגתא דרבי יוחנן בן נורי ורבנן קא מיפלגי דתנן שמן שצף על גבי יין ונגע טבול יום בשמן לא פסל אלא שמן בלבד ורבי יוחנן בן נורי אומר שניהן חיבורין זה לזה}
\textblock{רבנן כרבנן ורבי שמעון כר"י בן נורי}
\textblock{תניא ר"א בן תדאי אומר אחד זה ואחד זה צריכין לערב ואפילו לזה ביין ולזה ביין}
\textblock{אמר רבה זה בא בלגינו ושפך וזה בא בלגינו ושפך כולי עלמא לא פליגי דהוי עירוב}
\textblock{כי פליגי כגון שלקחו חבית של יין בשותפות ר"א בן תדאי סבר אין ברירה ורבנן סברי יש ברירה}
\textblock{רב יוסף אמר ר"א בן תדאי ורבנן בסומכין על שיתוף במקום עירוב קמיפלגי}
\textblock{דמר סבר אין סומכין ומר סבר סומכין}
\textblock{אמר רב יוסף מנא אמינא לה דאמר רב יהודה אמר רב הלכה כר"מ ואמר רב ברונא אמר רב הלכה כר"א בן תדאי מ"ט לאו משום דחד טעמא הוא}
\textblock{א"ל אביי ואי חד טעמא תרתי הילכתא למה לי הא קמ"ל דלא עבדינן כתרי חומרי בעירובין}
\textblock{מאי ר"מ ומאי רבנן דתניא מערבין בחצירות בפת ואם רצו לערב ביין אין מערבין משתתפין במבוי ביין ואם רצו להשתתף בפת משתתפין}
\textblock{מערבין בחצירות ומשתתפין במבוי שלא לשכח תורת עירוב מן התינוקות שיאמרו אבותינו לא עירבו דברי ר"מ וחכ"א או מערבין או משתתפין}
\textblock{פליגי בה ר' נחומי ורבה חד אמר בפת דכולי עלמא לא פליגי דבחדא סגי כי פליגי ביין}
\newsection{דף עב}
\textblock{וחד אמר ביין דכו"ע לא פליגי דבעינן תרתי כי פליגי בפת}
\textblock{מיתיבי וחכמים אומרים או מערבין או משתתפין מאי לאו או מערבין בחצר בפת או משתתפין במבוי ביין}
\textblock{אמר רב גידל אמר רב ה"ק או מערבין בחצר בפת ומותרין כאן וכאן או משתתפין במבוי בפת ומותרין כאן וכאן}
\textblock{אמר רב יהודה אמר רב הלכה כרבי מאיר ורב הונא אמר מנהג כר"מ ורבי יוחנן אמר נהגו העם כרבי מאיר:}
\textblock{{\large\emph{מתני׳}} חמשה חבורות ששבתו בטרקלין אחד ב"ש אומרים עירוב לכל חבורה וחבורה וב"ה אומרים עירוב אחד לכולן}
\textblock{ומודים בזמן שמקצתן שרויין בחדרים או בעליות שהן צריכין עירוב לכל חבורה וחבורה:}
\textblock{{\large\emph{גמ׳}} אמר רב נחמן מחלוקת במסיפס אבל במחיצה עשרה דברי הכל עירוב לכל חבורה וחבורה איכא דאמרי אמר ר"נ אף במסיפס מחלוקת}
\textblock{פליגי בה רבי חייא ורבי שמעון ברבי חד אמר מחלוקת במחיצות המגיעות לתקרה אבל מחיצות שאין מגיעות לתקרה דברי הכל עירוב אחד לכולן וחד אמר מחלוקת. במחיצות שאין מגיעות לתקרה אבל מחיצות המגיעות לתקרה דברי הכל צריכין עירוב לכל חבורה וחבורה}
\textblock{מיתיבי אמר רבי יהודה הסבר לא נחלקו ב"ש וב"ה על מחיצות המגיעות לתקרה שצריכין עירוב לכל חבורה וחבורה על מה נחלקו על מחיצות שאין מגיעות לתקרה שבית שמאי אומרים עירוב לכל חבורה וחבורה ובית הלל אומרים עירוב א' לכולן}
\textblock{למאן דאמר במחיצות המגיעות לתקרה מחלוקת תיובתא ולמאן דאמר במחיצות שאין מגיעות לתקרה מחלוקת סייעתא להך לישנא דאמר רב נחמן מחלוקת במסיפס תיובתא}
\textblock{להך לישנא דאמר רב נחמן אף במסיפס מחלוקת לימא תהוי תיובתא}
\textblock{אמר לך רב נחמן פליגי במחיצה והוא הדין במסיפס והאי דקא מיפלגי במחיצה להודיעך כחן דבית הלל}
\textblock{וליפלגי במסיפס להודיעך כחן דב"ש כח דהיתרא עדיף}
\textblock{אמר רב נחמן אמר רב הלכה כרבי יהודה הסבר}
\textblock{אמר רב נחמן בר יצחק מתניתין נמי דיקא דקתני ומודים בזמן שמקצתן שרויין בחדרים ובעליות שצריכין עירוב לכל חבורה וחבורה מאי חדרים ומאי עליות אילימא חדרים חדרים ממש ועליות עליות ממש פשיטא אלא לאו כעין חדרים כעין עליות ומאי ניהו מחיצות המגיעות לתקרה שמע מינה}
\textblock{תנא במה דברים אמורים כשמוליכין את עירובן למקום אחר אבל אם היה עירובן בא אצלן דברי הכל עירוב אחד לכולן}
\textblock{כמאן אזלא הא דתניא חמשה שגבו את עירובן כשמוליכין את עירובן למקום אחר עירוב אחד לכולן כמאן כבית הלל}
\textblock{ואיכא דאמרי במה דברים אמורים כשהיה עירוב בא אצלן אבל אם היו מוליכין את עירובן למקום אחר דברי הכל צריכין עירוב לכל חבורה וחבורה}
\textblock{כמאן אזלא הא דתניא חמשה שגבו את עירובן כשמוליכין את עירובן למקום אחר עירוב אחד לכולן כמאן דלא כחד:}
\textblock{{\large\emph{מתני׳}} האחין שהיו אוכלין על שלחן אביהם וישנים בבתיהם צריכין עירוב לכל אחד ואחד לפיכך אם שכח אחד מהם ולא עירב מבטל את רשותו}
\textblock{אימתי בזמן שמוליכין עירובן במקום אחר אבל אם היה עירוב בא אצלן או שאין עמהן דיורין בחצר אינן צריכין לערב:}
\textblock{{\large\emph{גמ׳}} ש"מ מקום לינה גורם אמר רב יהודה אמר רב במקבלי פרס שנו}
\textblock{ת"ר מי שיש לו בית שער אכסדרה ומרפסת בחצר חבירו הרי זה אין אוסר עליו (את) בית התבן (ואת) בית הבקר בית העצים ובית האוצרות הרי זה אוסר עליו רבי יהודה אומר אינו אוסר אלא מקום דירה בלבד}
\textblock{אמר רבי יהודה מעשה בבן נפחא שהיו לו חמש חצרות באושא ובא מעשה לפני חכמים ואמרו אינו אוסר אלא בית דירה בלבד}
\textblock{בית דירה סלקא דעתך אלא אימא מקום דירה}
\textblock{מאי מקום דירה רב אמר}
\newsection{דף עג}
\textblock{מקום פיתא ושמואל אמר מקום לינה}
\textblock{מיתיבי הרועים והקייצין והבורגנין ושומרי פירות בזמן שדרכן ללין בעיר הרי הן כאנשי העיר בזמן שדרכן ללין בשדה יש להם אלפים לכל רוח}
\textblock{התם אנן סהדי דאי ממטו להו ריפתא התם טפי ניחא להו}
\textblock{אמר רב יוסף לא שמיע לי הא שמעתא אמר ליה אביי את אמרת ניהלן ואהא אמרת ניהלן האחין שהיו אוכלין על שלחן אביהן וישנים בבתיהן צריכין עירוב לכל אחד ואחד ואמרינן לך שמע מינה מקום לינה גורם ואמרת לן עלה אמר רב יהודה אמר רב במקבלי פרס שנו}
\textblock{תנו רבנן מי שיש לו חמש נשים מקבלות פרס מבעליהן וחמשה עבדים מקבלין פרס מרביהן רבי יהודה בן בתירה מתיר בנשים ואוסר בעבדים}
\textblock{רבי יהודה בן בבא מתיר בעבדים ואוסר בנשים}
\textblock{אמר רב מאי טעמא דרבי יהודה בן בבא דכתיב (דניאל ב, מט) ודניאל בתרע מלכא}
\textblock{פשיטא בן אצל אביו כדאמרן אשה אצל בעלה ועבד אצל רבו פלוגתא דרבי יהודה בן בתירה ורבי יהודה בן בבא תלמיד אצל רבו מאי}
\textblock{ת"ש דרב בי רבי חייא אמר אין אנו צריכין לערב שהרי אנו סומכין על שולחנו של רבי חייא ורבי חייא בי רבי אמר אין אנו צריכין לערב שהרי אנו סומכין על שולחנו של רבי}
\textblock{בעא מיניה אביי מרבה חמשה שגבו את עירובן כשמוליכין את עירובן למקום אחר עירוב אחד לכולן או צריכין עירוב לכל אחד ואחד אמר ליה עירוב אחד לכולן}
\textblock{והא אחין דכי גבו דמו וקתני צריכין עירוב לכל אחד ואחד הכא במאי עסקינן כגון דאיכא דיורין בהדייהו דמגו דהני אסרי הני נמי אסרי}
\textblock{הכי נמי מסתברא דקתני אימתי בזמן שמוליכין את עירובן במקום אחר אבל אם היה עירובן בא אצלם או שאין דיורין עמהן בחצר אין צריכין לערב שמע מינה}
\textblock{בעא מיניה רב חייא בר אבין מרב ששת בני בי רב דאכלי נהמא בבאגא ואתו ובייתי בבי רב כי משחינן להו תחומא מבי רב משחינן להו או מבאגא משחינן להו אמר ליה משחינן מבי רב}
\textblock{והרי נותן את עירובו בתוך אלפים אמה ואתי וביית בביתיה דמשחינן ליה תחומא מעירוביה}
\textblock{בההוא אנן סהדי ובהדא אנן סהדי בההוא אנן סהדי דאי מיתדר ליה התם ניחא ליה ובהדא אנן סהדי דאי מייתו להו ריפתא לבי רב ניחא להו טפי}
\textblock{בעי רמי בר חמא מרב חסדא אב ובנו הרב ותלמידו כרבים דמו או כיחידים דמו צריכין עירוב או אין צריכין עירוב מבוי שלהן ניתר בלחי וקורה או אין ניתר בלחי וקורה}
\textblock{א"ל תניתוה אב ובנו הרב ותלמידו בזמן שאין עמהן דיורין הרי הן כיחידים ואין צריכין לערב ומבוי שלהן ניתר בלחי וקורה:}
\textblock{{\large\emph{מתני׳}} חמש חצירות פתוחות זו לזו ופתוחות למבוי עירבו בחצירות ולא נשתתפו במבוי מותרין בחצירות ואסורין במבוי}
\textblock{ואם נשתתפו במבוי מותרין כאן וכאן}
\textblock{עירבו בחצירות ונשתתפו במבוי ושכח אחד מבני חצר ולא עירב מותרין כאן וכאן}
\textblock{מבני מבוי ולא נשתתף מותרין בחצירות ואסורין במבוי שהמבוי לחצירות כחצר לבתים:}
\textblock{{\large\emph{גמ׳}} מני רבי מאיר היא דאמר בעינן עירוב ובעינן שיתוף}
\textblock{אימא מציעתא ואם נשתתפו במבוי מותרין כאן וכאן אתאן לרבנן דאמרי בחדא סגיא}
\textblock{הא ל"ק ואם נשתתפו נמי קאמר}
\textblock{אימא סיפא עירבו בחצירות ונשתתפו במבוי ושכח אחד מבני חצר ולא עירב מותרים כאן וכאן היכי דמי אי דלא בטיל אמאי מותרים אלא פשיטא דבטיל אימא סיפא שכח אחד מבני מבוי ולא נשתתפו מותרין בחצירות ואסורין במבוי ואי דבטיל אמאי אסורין במבוי}
\textblock{וכי תימא קסבר רבי מאיר אין ביטול רשות במבוי והא תניא שהרי ביטל לכם רשותו דברי רבי מאיר}
\textblock{אלא פשיטא דלא בטיל ומדסיפא דלא בטיל רישא נמי דלא בטיל רישא וסיפא רבי מאיר מציעתא רבנן}
\textblock{כולה רבי מאיר היא וטעמא מאי אמר רבי מאיר בעינן עירוב ובעינן שיתוף שלא לשכח תורת עירוב מן התינוקות והכא כיון דרובה עירבו לא משתכחא}
\textblock{אמר רב יהודה רב לא תני פתוחות זו לזו וכן אמר רב כהנא רב לא תני פתוחות זו לזו איכא דאמרי רב כהנא גופיה לא תני פתוחות זו לזו}
\textblock{אמר ליה אביי לרב יוסף מאי טעמא דלא תני פתוחות זו לזו קסבר כל שיתוף שאין מכניסו ומוציאו דרך פתחים במבוי לאו שמיה שיתוף}
\textblock{איתיביה בעל הבית שהיה שותף לשכניו לזה ביין ולזה ביין אין צריכין לערב התם דאפקיה ועייליה}
\textblock{(איתיביה) כיצד משתתפין במבוי וכו' התם נמי דאפקיה ועייליה}
\textblock{מתקיף לה רבה בר חנן אלא מעתה הקנה לו פת בסלו ה"נ דלא הוי שיתוף וכי תימא הכי נמי והא אמר רב יהודה אמר רב בני חבורה שהיו מסובין וקדש עליהן היום הפת שעל שלחן סומכים עליה משום עירוב ואמרי לה משום שיתוף}
\textblock{ואמר רבה לא פליגי כאן במסובין בבית כאן במסובין בחצר}
\textblock{אלא טעמא דרב דקא סבר אין מבוי ניתר בלחי וקורה עד שיהו בתים וחצירות פתוחים לתוכו}
\textblock{גופא אמר רב אין מבוי ניתר בלחי וקורה}
\newsection{דף עד}
\textblock{עד שיהו בתים וחצירות פתוחין לתוכו ושמואל אמר אפילו בית אחד וחצר אחת ורבי יוחנן אמר אפילו חורבה}
\textblock{אמר ליה אביי לרב יוסף אמר רבי יוחנן אפילו בשביל של כרמים אמר ליה לא אמר רבי יוחנן אלא בחורבה דחזי לדירה אבל שביל של כרמים דלא חזי לדירה לא}
\textblock{אמר רב הונא בר חיננא ואזדא רבי יוחנן לטעמיה דתנן (אמר ר"ש) אחד גגות ואחד קרפיפות ואחד חצרות רשות אחת הן לכלים ששבתו לתוכן ולא לכלים ששבתו בתוך הבית}
\textblock{ואמר רב הלכה כר"ש והוא שלא עירבו אבל עירבו גזרינן דילמא אתי לאפוקי מאני דבתים לחצר}
\textblock{ושמואל אמר בין עירבו ובין לא עירבו וכן א"ר יוחנן הלכה כר"ש בין עירבו ובין לא עירבו אלמא לא גזרינן דילמא אתי לאפוקי מאני דבתים לחצר הכא נמי לא גזרינן דילמא אתי לאפוקי מאני דחצר לחורבה}
\textblock{יתיב רב ברונא וקאמר להא שמעתא א"ל ר"א בר בי רב אמר שמואל הכי א"ל אין א"ל אחוי לי אושפיזיה אחוי ליה אתא לקמיה דשמואל אמר ליה אמר מר הכי אמר ליה אין}
\textblock{והא מר הוא דאמר אין לנו בעירובין אלא כלשון משנתינו שהמבוי לחצירות כחצר לבתים אישתיק}
\textblock{קבלה מיניה או לא קבלה מיניה ת"ש דההוא מבואה דהוה דייר ביה איבות בר איהי עבד ליה לחייא ושרא ליה שמואל}
\textblock{אתא רב ענן שדיה אמר מבואה דדיירנא ביה ואתינא משמיה דמר שמואל ניתי רב ענן בר רב נישדייה מן שמע מינה לא קיבלה מיניה}
\textblock{לעולם אימא לך קיבלה מינה והכא חזנא הוא דהוה אכיל נהמא בביתיה ואתי ביית בבי כנישתא}
\textblock{ואיבות בר איהי סבר מקום פיתא גרים ושמואל לטעמיה דאמר מקום לינה גרים:}
\textblock{אמר רב יהודה אמר רב מבוי שצידו אחד עובד כוכבים וצידו אחד ישראל אין מערבין אותו דרך חלונות להתירו דרך פתחים למבוי}
\textblock{א"ל אביי לרב יוסף אמר רב אפילו בחצר אמר ליה אין דאי לא אמר מאי}
\textblock{הוה אמינא טעמא דרב משום דקסבר אין מבוי ניתר בלחי וקורה עד שיהו בתים וחצירות פתוחין לתוכו}
\textblock{ותרתי למה לי צריכא דאי מההיא}
\newsection{דף עה}
\textblock{הוה אמינא דירת עובד כוכבים שמה דירה קא משמע לן דדירת עובד כוכבים לא שמה דירה ואי מהכא הוה אמינא לא ידענא בתים כמה קא משמע לן בתים תרין}
\textblock{השתא דאמר רב אפילו חצר טעמא דרב דקא סבר אסור לעשות יחיד במקום עובד כוכבים}
\textblock{אמר רב יוסף אי הכי היינו דשמענא ליה לרבי טבלא דאמר עובד כוכבים עובד כוכבים תרי זימני ולא ידענא מאי אמר:}
\textblock{{\large\emph{מתני׳}} שתי חצירות זו לפנים מזו עירבה הפנימית ולא עירבה החיצונה הפנימית מותרת והחיצונה אסורה}
\textblock{החיצונה ולא הפנימית שתיהן אסורות עירבה זו לעצמה וזו לעצמה זו מותרת בפני עצמה וזו מותרת בפני עצמה}
\textblock{רבי עקיבא אוסר החיצונה שדריסת הרגל אוסרתה וחכ"א אין דריסת הרגל אוסרתה}
\textblock{שכח אחד מן החיצונה ולא עירב הפנימית מותרת והחיצונה אסורה מן הפנימית ולא עירב שתיהן אסורות}
\textblock{נתנו עירובן במקום אחד ושכח אחד בין מן הפנימית בין מן החיצונה ולא עירב שתיהן אסורות ואם היו של יחידים אינן צריכין לערב:}
\textblock{{\large\emph{גמ׳}} כי אתא רב דימי א"ר ינאי זו דברי רבי עקיבא דאמר אפילו רגל המותרת במקומה אוסרת שלא במקומה אבל חכמים אומרים כשם שרגל המותרת אינה אוסרת כך רגל האסורה אינה אוסרת}
\textblock{תנן עירבה חיצונה ולא פנימית שתיהן אסורות מני אילימא רבי עקיבא מאי איריא רגל אסורה אפילו רגל מותרת נמי אלא לאו רבנן}
\textblock{לעולם ר"ע ולא זו אף זו קתני}
\textblock{תנן עירבה זו לעצמה וזו לעצמה זו מותרת בפני עצמה וזו מותרת בפני עצמה טעמא דעירבה הא לא עירבה שתיהן אסורות}
\textblock{והא האי תנא דאמר רגל המותרת אינה אוסרת רגל האסורה אוסרת מני הא אילימא ר"ע היא אפילו רגל המותרת נמי אלא לאו רבנן היא ועוד מדסיפא רבי עקיבא רישא לאו ר"ע}
\textblock{כולה רבי עקיבא היא וחסורי מיחסרא והכי קתני עירבה זו לעצמה וזו לעצמה זו מותרת בפני עצמה וזו מותרת בפני עצמה בד"א שעשתה דקה אבל לא עשתה דקה חיצונה אסורה דברי ר"ע שר"ע אוסר את החיצונה מפני שדריסת הרגל אוסרת וחכ"א אין דריסת הרגל אוסרת}
\textblock{מתיב רב ביבי בר אביי ואם היו של יחידים אין צריכין לערב הא של רבים צריכין לערב אלמא רגל המותרת במקומה אינה אוסרת רגל האסורה אוסרת}
\textblock{ועוד מתיב רבינא שכח אחד מן החיצונה ולא עירב הפנימית מותרת וחיצונה אסורה שכח אחד מן הפנימית ולא עירב שתיהן אסורות טעמא דשכח הא לא שכח שתיהן מותרות אלמא רגל המותרת אינה אוסרת רגל האסורה אוסרת}
\textblock{אלא כי אתא רבין א"ר ינאי ג' מחלוקות בדבר ת"ק סבר רגל המותרת אינה אוסרת רגל האסורה אוסרת ר"ע סבר אפילו רגל המותרת אוסרת ורבנן בתראי סברי כשם שרגל מותרת אינה אוסרת כך רגל האסורה אינה אוסרת:}
\textblock{נתנו עירובן במקום אחד ושכח אחד בין מן הפנימית וכו': מאי מקום אחד}
\textblock{(סימן חיצונה עצמה בבית יחידאה רבינא דלא משכח בפנים)}
\textblock{אמר רב יהודה אמר רב חיצונה ומאי קרו לה מקום אחד מקום המיוחד לשתיהן}
\textblock{תניא נמי הכי נתנו עירובן בחיצונה ושכח אחד בין מן החיצונה ובין מן הפנימית ולא עירב שתיהן אסורות נתנו עירובן בפנימית ושכח אחד מן הפנימית ולא עירב שתיהן אסורות מן החיצונה ולא עירב שתיהן אסורות דברי רבי עקיבא וחכמים אומרים בזו פנימית מותרת וחיצונה אסורה}
\textblock{אמר ליה רבה בר חנן לאביי מאי שנא לרבנן דאמרי פנימית מותרת משום דאחדא דשא ומשתמש' לרבי עקיבא נמי תיחד דשא ותשמש אמר ליה עירוב מרגילה}
\textblock{לרבנן נמי עירוב מרגילה דאמרה לתקוני שיתפתיך ולא לעוותי}
\textblock{לר"ע נמי תימא לתקוני שיתפתיך ולא לעוותי דאמרה לה מבטלינן לך רשותי ורבנן אין ביטול רשות מחצר לחצר}
\textblock{לימא שמואל ורבי יוחנן בפלוגתא דרבנן ור"ע קא מיפלגי דשמואל אמר כרבנן ורבי יוחנן דאמר כר"ע}
\textblock{אמר לך שמואל אנא דאמרי אפי' לרבי עקיבא ע"כ לא קאמר ר"ע הכא אלא בשתי חצירות זו לפנים מזו דאסרן אהדדי אבל התם מי קא אסרן אהדדי}
\textblock{ורבי יוחנן אמר אנא דאמרי אפילו לרבנן ע"כ לא קאמרי רבנן הכא אלא דאמרה לה אדמבטלת לי קא אסרת עלאי אבל התם מי קאסרת עלה:}
\textblock{ואם היו של יחידים וכו': אמר רב יוסף תני רבי היו ג' אסורין}
\textblock{אמר להו רב ביבי לא תציתו ליה אנא אמריתה ניהלה ומשמיה דרב אדא בר אהבה אמריתה ניהלה הואיל ואני קורא בהן רבים בחיצונה אמר רב יוסף מריה דאברהם רבים ברבי איחלף לי}
\textblock{ושמואל אמר לעולם מותרות עד שיהו שנים בפנימית ואחד בחיצונה}
\textblock{אמר רבי אלעזר ונכרי הרי הוא כרבים מאי שנא ישראל דלא אסר דמאן דידע ידע ומאן דלא ידע סבר עירובי עירב נכרי נמי אמרינן דידע ידע דלא ידע סבר אגירי אוגר}
\textblock{סתם נכרי אי איתא דאוגר מיפעא פעי}
\textblock{אמר רב יהודה אמר שמואל י' בתים זה לפנים מזה פנימי נותן את עירובו ודיו}
\textblock{ור' יוחנן אמר אפילו חיצון חיצון בית שער הוא חיצון של פנימי}
\textblock{במאי קמיפלגי מר סבר בית שער דיחיד שמיה בית שער ומ"ס לא שמי' בית שער}
\textblock{א"ר נחמן אמר רבה בר אבוה אמר רב ב' חצירות וג' בתים ביניהן זה בא דרך זה ונותן עירובו בזה וזה בא דרך זה ונותן עירובו בזה}
\newsection{דף עו}
\textblock{זה נעשה בית שער לזה וזה נעשה בית שער לזה אמצעי הוה ליה בית שמניחין בו עירוב ואין צריך ליתן את הפת}
\textblock{בדיק להו רחבה לרבנן ב' חצרות וב' בתים ביניהם זה בא דרך זה ונתן עירובו בזה וזה בא דרך זה ונתן עירובו בזה קנו עירוב או לא מי משוי' להו לגבי דהאי בית ולגבי דהאי בית שער [ולגבי דהאי בית שער ולגבי דהאי בית]}
\textblock{אמרו ליה שניהן לא קנו עירוב מה נפשך אי בית שער משוית ליה הנותן את עירובו בבית שער אכסדרה ומרפסת אינו עירוב אי בית משוית ליה קא מטלטל לבית דלא מערב ליה}
\textblock{ומאי שנא מדרבא דאמר רבא אמרו לו שנים צא וערב עלינו לאחד עירב עליו מבעוד יום ולא' עירב עליו בין השמשות זה שעירב עליו מבעוד יום נאכל עירובו בין השמשות וזה שעירב עליו בין השמשות נאכל עירובו משתחשך שניהם קנו עירוב}
\textblock{הכי השתא התם ספק יממא ספק ליליא לא מינכרא מילתא אבל הכא אי דלגבי דהאי בית לגבי דהאי בית אי לגבי דהאי בית שער לגבי דהאי נמי בית שער:}
\textblock{\par \par {\large\emph{הדרן עלך הדר}}\par \par }
\textblock{}
\textblock{מתני׳ {\large\emph{חלון}} שבין ב' חצירות ד' על ד' בתוך עשרה מערבין שנים ואם רצו מערבין א'}
\textblock{פחות מד' על ד' או למעלה מי' מערבין שנים ואין מערבין אחד:}
\textblock{{\large\emph{גמ׳}} לימא תנן סתמא כר"ש בן גמליאל דאמר כל פחות מד' כלבוד דמי}
\textblock{אפי' תימא כרבנן עד כאן לא פליגי רבנן עליה דרשב"ג אלא לענין לבודין אבל לענין פתחא אפי' רבנן מודו דאי איכא ד' על ד' חשיב ואי לא לא חשיב:}
\textblock{פחות מד' וכו': פשיטא כיון דאמר ד' על ד' בתוך עשרה ממילא אנא ידענא דפחות מד' ולמעלה מי' לא}
\textblock{הא קמ"ל טעמא דכוליה למעלה מי' אבל מקצתו בתוך י' מערבין שנים ואם רצו מערבין אחד}
\textblock{תנינא להא דת"ר כולו למעלה מי' ומקצתו בתוך עשרה כולו בתוך י' ומקצתו למעלה מי' מערבין שנים ואם רצו מערבין אחד}
\textblock{השתא כולו למעלה מי' ומקצתו בתוך י' אמרת מערבין שנים ואם רצו מערבין א' כולו בתוך י' ומקצתו למעלה מי' מיבעיא}
\textblock{זו ואצ"ל זו קתני}
\textblock{א"ר יוחנן חלון עגול צריך שיהא בהיקפו עשרים וארבעה טפחים ושנים ומשהו מהן בתוך י' שאם ירבענו נמצא משהו בתוך י'}
\textblock{}
\textblock{מכדי כל שיש בהיקפו שלשה טפחים יש בו ברוחבו טפח בתריסר סגיא}
\newchap{פרק \hebrewnumeral{7}\quad חלון}
\textblock{}
\textblock{הני מילי בעיגולא אבל בריבועא בעינן טפי}
\textblock{מכדי כמה מרובע יתר על העגול רביע בשיתסר סגיא}
\textblock{ה"מ עיגולא דנפיק מגו ריבועא אבל ריבועא דנפיק מגו עיגולא בעינן טפי מ"ט משום מורשא דקרנתא}
\textblock{מכדי כל אמתא בריבוע אמתא ותרי חומשי באלכסונא בשיבסר נכי חומשא סגיא}
\textblock{רבי יוחנן אמר כי דייני דקיסרי ואמרי לה כרבנן דקיסרי דאמרי עיגולא מגו ריבועא ריבעא ריבועא מגו עיגולא פלגא:}
\textblock{פחות מד' על ד' וכו': אמר רב נחמן לא שנו אלא חלון שבין ב' חצירות אבל חלון שבין ב' בתים אפילו למעלה מעשרה נמי אם רצו לערב מערבין אחד מ"ט ביתא כמאן דמלי דמי}
\textblock{איתיביה רבא לרב נחמן אחד לי חלון שבין ב' חצירות ואחד לי חלון שבין ב' בתים ואחד לי חלון שבין ב' עליות ואחד לי חלון שבין ב' גגין ואחד לי חלון שבין ב' חדרים כולן ד' על ד' בתוך עשרה}
\textblock{תרגומא אחצירות והא אחד לי קתני תרגומא אד' על ד'}
\textblock{בעא מיניה ר' אבא מרב נחמן לול הפתוח מן בית לעלייה צריך סולם קבוע להתירו או אין צריך סולם קבוע להתירו}
\textblock{כי אמרינן ביתא כמאן דמלי דמי הני מילי מן הצד אבל באמצע לא או דילמא לא שנא}
\textblock{אמר ליה אינו צריך סבור מינה סולם קבוע הוא דאינו צריך הא סולם עראי צריך איתמר אמר רב יוסף בר מניומי אמר רב נחמן אחד סולם קבוע ואחד סולם עראי אינו צריך:}
\textblock{{\large\emph{מתני׳}} כותל שבין ב' חצירות גבוה עשרה ורוחב ארבעה מערבין שנים ואין מערבין אחד}
\textblock{היו בראשו פירות אלו עולין מכאן ואוכלין ואלו עולין מכאן ואוכלין ובלבד שלא יורידו למטן}
\textblock{נפרצה הכותל עד עשר אמות מערבין שנים ואם רצו מערבין אחד מפני שהוא כפתח יותר מכאן מערבין אחד ואין מערבין שנים:}
\textblock{{\large\emph{גמ׳}} אין בו ארבעה מאי אמר רב אויר שתי רשויות שולטת בו לא יזיז בו אפילו מלא נימא}
\newsection{דף עז}
\textblock{ורבי יוחנן אמר אלו מעלין מכאן ואוכלין ואלו מעלין מכאן ואוכלין}
\textblock{תנן אלו עולין מכאן ואוכלין ואלו עולין מכאן ואוכלין עולין אין מעלין לא}
\textblock{הכי קאמר יש בו ארבעה על ארבעה עולין אין מעלין לא אין בו ארבעה על ארבעה מעלין נמי}
\textblock{ואזדא רבי יוחנן לטעמיה דכי אתא רב דימי אמר רבי יוחנן מקום שאין בו ארבעה על ארבעה מותר לבני רשות הרבים ולבני רשות היחיד לכתף עליו ובלבד שלא יחליפו}
\textblock{ורב לית ליה דרב דימי אי ברשויות דאורייתא הכי נמי}
\textblock{הכא במאי עסקינן ברשויות דרבנן וחכמים עשו חיזוק לדבריהם יותר משל תורה}
\textblock{אמר רבה (אמר) רב הונא אמר רב נחמן כותל שבין שתי חצירות צידו אחד גבוה עשרה טפחים וצידו אחד שוה לארץ נותנין אותו לזה ששוה לארץ}
\textblock{משום דהוה לזה תשמישו בנחת ולזה תשמישו בקשה וכל לזה בנחת ולזה בקשה נותנין אותו לזה שתשמישו בנחת}
\textblock{אמר רב שיזבי אמר רב נחמן חריץ שבין שתי חצירות צידו אחד עמוק עשרה וצידו אחד שוה לארץ נותנין אותו לזה ששוה לארץ משום דהוה ליה לזה תשמישו בנחת ולזה תשמישו בקשה וכו'}
\textblock{וצריכי דאי אשמעינן כותל משום דבגובהא משתמשי אינשי אבל חריץ בעומקא לא משתמשי אינשי אימא לא}
\textblock{ואי אשמעינן בחריץ משום דלא בעיתא תשמישתא אבל כותל דבעיתא תשמישתא אימא לא צריכא}
\textblock{בא למעטו אם יש במיעוטו ארבעה מותר להשתמש בכל הכותל כולו ואם לאו אין משתמש אלא כנגד המיעוט}
\textblock{מה נפשך אי אהני מעוטא בכוליה כותל לישתמש אי לא אהני אפילו כנגד המיעוט נמי לא אמר רבינא כגון שעקר חוליא מראשו}
\textblock{אמר רב יחיאל כפה ספל ממעט}
\textblock{ואמאי דבר הניטל בשבת הוא ודבר הניטל בשבת אינו ממעט לא צריכא דחבריה בארעא}
\textblock{וכי חבריה בארעא מאי הוי והא תניא פגה שהטמינה בתבן וחררה שהטמינה בגחלים אם מגולה מקצתה נטלת בשבת}
\textblock{הכא במאי עסקינן דאית ליה אוגניים}
\textblock{וכי אית ליה אוגניים מאי הוי והתנן הטומן לפת וצנון תחת הגפן בזמן}
\textblock{שמקצת עלין מגולין אינו חושש לא משום כלאים ולא משום מעשר ולא משום שביעית וניטלין בשבת}
\textblock{לא צריכא דבעי מרא וחצינא:}
\textblock{סולם המצרי אינו ממעט והצורי ממעט: היכי דמי סולם המצרי אמרי דבי רבי ינאי כל שאין לו ארבעה חווקים}
\textblock{אמר ליה רב אחא בריה דרבא לרב אשי מאי טעמא דסולם המצרי דלא ממעט אמר ליה לא שמיע לך הא דאמר רב אחא בר אדא אמר רב המנונא אמר רב משום דהוה ליה דבר שניטל בשבת וכל דבר שניטל בשבת אינו ממעט}
\textblock{אי הכי אפי' צורי נמי התם כובדו קובעו}
\textblock{אמר אביי כותל שבין שתי חצירות גבוה עשרה טפחים והניח סולם רחב ארבעה מכאן וסולם רחב ארבעה מכאן ואין בין זה לזה שלשה טפחים ממעט שלשה אינו ממעט}
\textblock{ולא אמרן אלא דלא הוי כותל ארבעה אבל הוי כותל ארבעה אפילו מופלג טובא נמי}
\textblock{אמר רב ביבי בר אביי בנה איצטבא על גב איצטבא אם יש באיצטבא התחתונה ארבעה ממעט אי נמי אין בתחתונה ארבעה ויש בעליונה ארבעה ואין בין זה לזה שלשה ממעט}
\textblock{ואמר רב נחמן אמר רבה בר אבוה סולם ששליבותיו פורחות אם יש בשליבה התחתונה ארבעה ממעט אי נמי אין בשליבה התחתונ' ארבע' ויש בשליבה העליונה ארבעה ואין בין זה לזה שלשה ממעט}
\textblock{ואמר רב נחמן אמר רבה בר אבוה}
\newsection{דף עח}
\textblock{זיז היוצא מן הכותל ד' על ד' והניח עליו סולם כל שהוא מיעטו}
\textblock{ולא אמרן אלא דאותביה עליה אבל אותביה בהדי' ארווחי ארוחיה}
\textblock{ואמר רב נחמן אמר רבה בר אבוה כותל תשעה עשר צריך זיז אחד להתירו}
\textblock{כותל עשרים צריך שני זיזים להתירו אמר רב חסדא והוא שהעמידן זה שלא כנגד זה}
\textblock{אמר רב הונא עמוד ברה"ר גבוה עשרה ורחב ארבעה ונעץ בו יתד כל שהוא מיעטו}
\textblock{אמר רב אדא בר אהבה ובגבוה שלשה אביי ורבא דאמרי תרוייהו אף על פי שאין גבוה שלשה}
\textblock{מאי טעמא לא משתמש ליה}
\textblock{רב אשי אמר אפילו שגבוה שלשה מאי טעמא אפשר דתלי' ביה מידי}
\textblock{אמר ליה רב אחא בריה דרבא לרב אשי מלאו כולו ביתדות מהו}
\textblock{א"ל לא שמיע לך הא דאמר רבי יוחנן בור וחולייתה מצטרף לעשרה}
\textblock{ואמאי הא לא משתמש ליה אלא מאי אית לך למימר דמנח מידי ומשתמש הכא נמי דמנח מידי ומשתמש}
\textblock{אמר רב יהודה אמר שמואל כותל עשרה צריך סולם ארבעה עשר להתירו רב יוסף אמר אפילו שלשה עשר ומשהו}
\textblock{אביי אמר אפילו אחד עשר ומשהו}
\textblock{רב הונא בריה דרב יהושע אמר אפילו שבעה ומשהו}
\textblock{אמר רב סולם זקוף ממעט גמרא ולא ידענא מ"ט}
\textblock{אמר שמואל ולא ידע אבא טעמא דהא מלתא מידי דהוה אאיצטבא על גבי איצטבא}
\textblock{אמר רבה אמר רבי חייא דקלים שבבבל אינן צריכין קבע מאי טעמא כבידן קובעתן}
\textblock{ורב יוסף אמר רבי אושעיא סולמות שבבבל אינן צריכין קבע מ"ט כבידן קובעתן}
\textblock{מאן דאמר סולמות כל שכן דקלים ומאן דאמר דקלים אבל סולמות לא}
\textblock{בעא מיניה רב יוסף מרבה סולם מכאן וסולם מכאן וקשין באמצע מהו}
\textblock{אמר ליה אין כף הרגל עולה בהן}
\textblock{קשין מכאן וקשין מכאן וסולם באמצע מהו אמר ליה הרי כף הרגל עולה בהן}
\textblock{חקק להשלים בכותל בכמה א"ל בעשרה}
\textblock{א"ל חקקו כולו בכותל בכמה א"ל מלא קומתו ומאי שנא א"ל התם מסתלק ליה הכא לא מסתלק ליה}
\textblock{בעא מיניה רב יוסף מרבה עשאו לאילן סולם מהו}
\textblock{תיבעי לרבי תיבעי לרבנן}
\textblock{תיבעי לרבי עד כאן לא קאמר רבי התם כל דבר שהוא משום שבות לא גזרו עליו ה"מ בין השמשות אבל כולי יומא לא}
\textblock{או דילמא אפילו לרבנן פיתחא הוא ואריא הוא דרביע עליה}
\textblock{עשאו לאשירה סולם מהו תיבעי לרבי יהודה תיבעי לרבנן}
\textblock{תיבעי לרבי יהודה ע"כ לא קאמר רבי יהודה התם דמותר לקנות בית באיסורי הנאה אלא התם דבתר דקנה ליה עירוב לא ניחא ליה דלינטר}
\textblock{או דילמא אפילו לרבנן פיתחא הוא ואריא דרביע עליה}
\textblock{א"ל אילן מותר ואשירה אסורה מתקיף לה רב חסדא אדרבה אילן שאיסור שבת גורם לו ניתסר}
\textblock{אשירה שאיסור דבר אחר גורם לו לא ניתסר}
\textblock{איתמר נמי כי אתא רבין א"ר אלעזר ואמרי לה א"ר אבהו א"ר יוחנן כל שאיסור שבת גרם לו אסור כל שאיסור דבר אחר גרם לו מותר}
\textblock{ר"נ בר יצחק מתני הכי אילן פלוגתא דרבי ורבנן אשירה פלוגתא דרבי יהודה ורבנן:}
\textblock{{\large\emph{מתני׳}} חריץ שבין ב' חצירות עמוק י' ורוחב ד' מערבין שנים ואין מערבין אחד אפילו מלא קש או תבן מלא עפר או צרורות מערבין אחד ואין מערבין שנים}
\textblock{נתן עליו נסר שרחב ארבעה טפחים וכן ב' גזוזטראות זו כנגד זו מערבין שנים ואם רצו מערבין אחד פחות מכאן מערבין שנים ואין מערבין אחד:}
\textblock{{\large\emph{גמ׳}} ותבן לא חייץ והא אנן תנן מתבן שבין שתי חצירות גבוה עשרה טפחים מערבין שנים ואין מערבין אחד}
\textblock{אמר אביי לענין מחיצה כולי עלמא לא פליגי דהויא מחיצה אבל לענין חציצה אי בטלי' חייץ ואי לא בטלי' לא חייץ:}
\textblock{מלא עפר: ואפילו בסתמא והתנן בית שמילאהו תבן או צרורות וביטלו בטל}
\textblock{ביטלו אין}
\newsection{דף עט}
\textblock{לא ביטלו לא אמר רב הונא מאן תנא אהלות רבי יוסי היא}
\textblock{אי רבי יוסי איפכא שמעינן ליה דתניא רבי יוסי אומר תבן ואין עתיד לפנותו הרי הוא כסתם עפר ובטל עפר ועתיד לפנותו הרי הוא כסתם תבן ולא בטיל}
\textblock{אלא אמר רב אסי מאן תנא עירובין רבי יוסי היא}
\textblock{רב הונא בריה דרב יהושע אמר טומאה אשבת קרמית הנח איסור שבת דאפילו ארנקי נמי מבטל איניש}
\textblock{רב אשי אמר בית אחריץ קא רמית בשלמא חריץ למיטיימיה קאי אלא בית למיטיימיה קאי:}
\textblock{נתן עליו נסר שרוחב ד': אמר רבא לא שנו אלא שנתן לרחבו אבל לארכו אפילו כל שהוא נמי שהרי מיעטו מד':}
\textblock{וכן שתי גזוזטראות זו כנגד זו: אמר רבא הא דאמרת זו כנגד זו אין זו שלא כנגד זו לא וזו למעלה מזו נמי לא אמרן אלא שיש בין זה לזה שלשה טפחים אבל אין בין זה לזה שלשה גזוזטרא עקומה היא:}
\textblock{{\large\emph{מתני׳}} מתבן שבין שתי חצירות גבוה עשרה טפחים מערבין שנים ואין מערבין אחד אלו מאכילין מכאן ואלו מאכילין מכאן נתמעט התבן מעשרה טפחים מערבין אחד ואין מערבין שנים:}
\textblock{{\large\emph{גמ׳}} אמר רב הונא ובלבד שלא יתן לתוך קופתו ויאכיל}
\textblock{ולאוקמי שרי והאמר רב הונא אמר רבי חנינא מעמיד אדם את בהמתו על גבי עשבים בשבת ואין מעמיד אדם את בהמתו על גבי מוקצה בשבת}
\textblock{דקאים לה באפה ואזלה ואכלה}
\textblock{ולא יתן לתוך קופתו תבן והתניא בית שבין שתי חצירות ומילאהו תבן מערבין שנים ואין מערבין אחד זה נותן לתוך קופתו ויאכיל וזה נותן לתוך קופתו ויאכיל נתמעט התבן מי' טפחים שניהם אסורים}
\textblock{כיצד הוא עושה נועל את ביתו ומבטל את רשותו הוא אסור וחבירו מותר}
\textblock{וכן אתה אומר בגוב של תבן שבין ב' תחומי שבת קתני מיהת זה נותן לתוך קופתו ויאכיל וזה נותן לתוך קופתו ויאכיל}
\textblock{אמרי בית כיון דאיכא (מחיצות ו) תקרה כי מיפחית מינכרא ליה מלתא הכא לא מינכרא ליה מלתא:}
\textblock{נתמעט התבן מעשרה טפחים שניהן אסורין: הא עשרה שרי ואע"ג דמידליא תקרה טובא שמע מינה מחיצות שאין מגיעות לתקרה שמן מחיצות}
\textblock{אמר אביי הכא בבית שלשה עשר חסר משהו עסקינן ותבן עשרה}
\textblock{ורב הונא בריה דרב יהושע אמר אפילו תימא בבית עשרה}
\textblock{ותבן שבעה ומשהו דכל פחות משלשה כלבוד דמי}
\textblock{בשלמא לאביי היינו דקתני מעשרה אלא לרב הונא בריה דרב יהושע מאי מעשרה}
\textblock{מתורת עשרה:}
\textblock{שניהן אסורין: שמע מינה דיורין הבאין בשבת אסורין}
\textblock{דלמא דאימעט מאתמול:}
\textblock{כיצד הוא עושה נועל את ביתו ומבטל רשותו: תרתי הכי קאמר או נועל את ביתו או מבטל את רשותו}
\textblock{ואיבעית אימא לעולם תרתי כיון דדש ביה אתי לטלטולי:}
\textblock{הוא אסור וחבירו מותר: פשיטא לא צריכא דהדר אידך ובטיל ליה לחבריה והא קמ"ל דאין מבטלין וחוזרין ומבטלין:}
\textblock{וכן אתה אומר בגוב של תבן שבין שני תחומי שבת: פשיטא לא צריכא לרע"ק דאמר תחומין דאורייתא מהו דתימא ליגזור דלמא אתי לאיחלופי קמ"ל:}
\textblock{{\large\emph{מתני׳}} כיצד משתתפין במבוי מניח את החבית ואומר הרי זו לכל בני מבוי ומזכה להן על ידי בנו ובתו הגדולים וע"י עבדו ושפחתו העברים וע"י אשתו}
\textblock{אבל אינו מזכה לא ע"י בנו ובתו הקטנים ולא ע"י עבדו ושפחתו הכנענים מפני שידן כידו:}
\textblock{{\large\emph{גמ׳}} אמר רב יהודה חבית של שיתופי מבואות צריך להגביה מן הקרקע טפח}
\textblock{אמר רבא הני תרתי מילי סבי דפומבדיתא אמרינהו חדא הא אידך המקדש אם טעם מלא לוגמיו יצא ואם לאו לא יצא}
\textblock{אמר רב חביבא הא נמי סבי דפומבדיתא אמרינהו דאמר רב יהודה אמר שמואל עושין מדורה לחיה בשבת}
\textblock{סבור מינה לחיה אין לחולה לא בימות הגשמים אין בימות החמה לא}
\textblock{איתמר אמר רב חייא בר אבין אמר שמואל הקיז דם ונצטנן עושין לו מדורה בשבת ואפילו בתקופת תמוז}
\textblock{אמר אמימר הא נמי סבי דפומבדיתא אמרינהו דאיתמר איזו היא אשירה סתם}
\textblock{אמר רב כל שמשרתי עו"ג שומרין אותה}
\newsection{דף פ}
\textblock{ואין טועמין מפירותיה}
\textblock{ושמואל אמר כגון דאמרי הני תמרי לשיכרא דבי נצרפי דשתו ליה ביום חגם (אמר אמימר) ואמרו לי סבי דפומבדיתא הלכתא כוותיה דשמואל}
\textblock{מיתיבי כיצד משתתפין במבוי מביאים חבית של יין ושל שמן ושל תמרים ושל גרוגרות ושל שאר מיני פירות}
\textblock{אם משלו צריך לזכות ואם משלהן צריך להודיע ומגביה מן הקרקע משהו מאי משהו נמי דקאמר טפח}
\textblock{איתמר שיתופי מבואות רב אמר אין צריך לזכות ושמואל אמר צריך לזכות עירובי תחומין רב אמר צריך לזכות ושמואל אמר אין צריך לזכות}
\textblock{בשלמא לשמואל הכא תנן והכא לא תנן אלא לרב מאי טעמא}
\textblock{תנאי היא דאמר רב יהודה אמר רב מעשה בכלתו של רבי אושעיא שהלכה לבית המרחץ וחשכה לה ועירבה לה חמותה}
\textblock{ובא מעשה לפני רבי חייא ואסר אמר לו רבי ישמעאל ברבי יוסי בבלאי כל כך אתה מחמיר בעירובין כך אמר אבא כל שיש לך להקל בעירובין הקל}
\textblock{ואבעיא להו משל חמותה עירבה לה ומשום דלא זיכתה לה או דילמא משלה עירבה לה ומשום דשלא מדעתה}
\textblock{אמר להן ההוא מרבנן ורבי יעקב שמיה לדידי מיפרשא לה מיניה דרבי יוחנן משל חמותה עירבה ומשום דלא זיכתה לה}
\textblock{אמר ליה רבי זירא לרבי יעקב (בריה) דבת יעקב כי מטית התם אקיף וזיל לסולמא דצור ובעי' מיניה מרב יעקב בר אידי}
\textblock{בעא מיניה משל חמותה עירבה ומשום דלא זיכתה לה או דלמא משלה עירבה ומשום דשלא מדעתה}
\textblock{אמר ליה משל חמותה עירבה לה ומשום דלא זיכתה לה}
\textblock{אמר רב נחמן נקטינן אחד עירובי תחומין ואחד עירובי חצירות אחד שיתופי מבואות צריך לזכות בעי רב נחמן עירובי תבשילין צריך לזכות או אין צריך לזכות}
\textblock{אמר רב יוסף ומאי תיבעי ליה לא שמיע ליה הא דאמר ר"נ בר רב אדא אמר שמואל עירובי תבשילין צריך לזכות אמר ליה אביי פשיטא דלא שמיע ליה דאי שמיע ליה מאי תיבעי ליה}
\textblock{אמר ליה אטו עירובי תחומין מי לא אמר שמואל אין צריך לזכות ואמר איהו צריך לזכות}
\textblock{הכי השתא בשלמא התם פליגי רב ושמואל וקא משמע לן כחומרין דמר וכי חומרין דמר אבל הכא אי איתא דשמיע ליה מי איכא דמאן דפליג}
\textblock{ההוא טורזינא דהוה בשיבבותיה דרבי זירא א"ל אוגיר לן רשותך לא אוגיר להו אתו לקמיה דר' זירא אמרו ליה מהו למיגר מדביתהו}
\textblock{אמר להו הכי אמר ר"ל משמיה דגברא רבה ומנו רבי חנינא אשתו של אדם מערבת שלא מדעתו}
\textblock{ההוא טורזינא דהוה בשיבבותיה דרב יהודה בר אושעיא אמרי ליה אוגר לן רשותך לא אוגר להו אתו לקמיה דרב יהודה בר אושעיא אמרי ליה מהו למיגר מדביתהו לא הוה בידיה אתו לקמיה דרב מתנה לא הוה בידיה אתו לקמיה דרב יהודה אמר להו הכי אמר שמואל אשתו של אדם מערבת שלא מדעתו}
\textblock{מיתיבי נשים שעירבו ונשתתפו שלא מדעת בעליהן אין עירובן עירוב ואין שיתופן שיתוף}
\textblock{לא קשיא הא דאסר הא דלא אסר}
\textblock{הכי נמי מסתברא דא"כ קשיא דשמואל אדשמואל דאמר שמואל אחד מבני מבוי שרגיל להשתתף עם בני מבוי ולא נשתתף בני המבוי נכנסין לתוך ביתו ונוטלין שיתופן ממנו בעל כרחו}
\textblock{רגיל אין שאין רגיל לא שמע מינה}
\textblock{לימא מסייע ליה כופין אותו לעשות לחי וקורה למבוי}
\textblock{שאני התם דליכא מחיצות}
\textblock{ל"א מצד שאני}
\textblock{אתמר רב חייא בר אשי אמר עושין לחי אשירה ור' שמעון בן לקיש אמר עושין קורה אשירה}
\textblock{מאן דאמר קורה כל שכן לחי ומאן דאמר לחי אבל קורה לא כתותי מכתת שיעוריה:}
\textblock{{\large\emph{מתני׳}} נתמעט האוכל מוסיף ומזכה ואין צריך להודיע נתוספו עליהן מוסיף ומזכה וצריך להודיע}
\textblock{כמה הוא שיעורן בזמן שהן מרובין מזון שתי סעודות לכולם בזמן שהן מועטין כגרוגרת לכל אחד ואחד}
\textblock{אמר ר' יוסי במה דברים אמורים בתחילת עירוב אבל בשירי עירוב כל שהוא}
\textblock{ולא אמרו לערב בחצירות אלא כדי שלא לשכח את התינוקות:}
\textblock{{\large\emph{גמ׳}} במאי עסקינן אילימא במין אחד מאי איריא נתמעט אפילו כלה נמי}
\textblock{אלא בשני מינין אפילו נתמעט נמי לא דתניא כלה האוכל ממין אחד אין צריך להודיע מב' מינים צריך להודיע}
\textblock{איבעית אימא ממין אחד ואיבעית אימא משני מינין איבעית אימא ממין אחד מאי נתמעט נתמטמט}
\textblock{ואיבעית אימא משני מינין כלה שאני:}
\textblock{ניתוספו עליהן מוסיף ומזכה וכו': אמר רב שיזבי אמר רב חסדא זאת אומרת חלוקין עליו חביריו על ר' יהודה}
\textblock{דתנן אמר רבי יהודה במה דברים אמורים בעירובי תחומין אבל בעירובי חצירות מערבין בין לדעת ובין שלא לדעת פשיטא דחלוקין}
\textblock{מהו דתימא הני מילי בחצר שבין שני מבואות אבל חצר של מבוי אחד אימא לא קמשמע לן: כמה הוא שיעורו וכו':}
\textblock{כמה הוא מרובין אמר רב יהודה אמר שמואל שמונה עשרה בני אדם שמונה עשרה ותו לא אימא משמונה עשרה ואילך}
\textblock{ומאי שמונה עשרה דנקט אמר רב יצחק בריה דרב יהודה לדידי מיפרשא לי מניה דאבא כל שאילו מחלקו למזון שתי סעודות ביניהן ואין מגעת גרוגרת לכל אחד ואחד הן הן מרובין וסגי במזון שתי סעודות ואי לא (הן הן) מועטין נינהו}
\textblock{ואגב אורחיה קא משמע לן דשתי סעודות הויין שמונה עשרה גרוגרות:}
\textblock{{\large\emph{מתני׳}} בכל מערבין ומשתתפין חוץ מן המים ומן המלח דברי ר"א רבי יהושע אומר ככר הוא עירוב אפילו מאפה סאה והוא פרוסה אין מערבין בה ככר כאיסר והוא שלם מערבין בו:}
\newsection{דף פא}
\textblock{{\large\emph{גמ׳}} תנינא חדא זימנא בכל מערבין ומשתתפין חוץ מן המים והמלח}
\textblock{אמר רבה לאפוקי מדרבי יהושע דאמר ככר אין מידי אחרינא לא קמשמע לן בכל}
\textblock{איתיביה אביי בכל מערבין עירובי חצירות ובכל משתתפין שיתופי מבואות ולא אמרו לערב בפת אלא בחצר בלבד מאן שמעת ליה דאמר פת אין מידי אחרינא לא רבי יהושע וקתני בכל}
\textblock{אלא אמר רבה בר בר חנה לאפוקי מדרבי יהושע דאמר שלימה אין פרוסה לא קמשמע לן בכל}
\textblock{ופרוסה מאי טעמא לא אמר רבי יוסי בן שאול אמר רבי משום איבה}
\textblock{אמר ליה רב אחא בריה דרבא לרב אשי עירבו כולן בפרוסות מהו א"ל שמא יחזור דבר לקלקולו}
\textblock{א"ר יוחנן בן שאול ניטלה הימנה כדי חלתה וכדי דימועה מערבין לו בה}
\textblock{והתניא כדי דימועה מערבין לו בה כדי חלתה אין מערבין לו בה}
\textblock{לא קשיא הא בחלת נחתום הא בחלת בעל הבית}
\textblock{דתנן שיעור חלה אחד מעשרים וארבעה העושה עיסה לעצמו ועיסה למשתה בנו אחד מעשרים וארבעה נחתום שהוא עושה למכור בשוק וכן האשה שעשתה למכור בשוק אחד מארבעים ושמונה}
\textblock{אמר רב חסדא תפרה בקיסם מערבין לו בה והא תניא אין מערבין לו בה לא קשיא הא דידיע תיפרה הא דלא ידיע תיפרה}
\textblock{א"ר זירא אמר שמואל מערבין בפת אורז ובפת דוחן אמר מר עוקבא לדידי מיפרשא לי מיניה דמר שמואל בפת אורז מערבין ובפת דוחן אין מערבין}
\textblock{אמר רב חייא בר אבין אמר רב מערבין בפת עדשים איני והא ההיא דהואי בשני דמר שמואל ושדייה לכלביה ולא אכלה}
\textblock{ההיא דשאר מינים הויא דכתיב (יחזקאל ד, ט) ואתה קח לך חטין ושעורים ופול ועדשים ודוחן וכוסמים וגו'}
\textblock{רב פפא אמר ההיא צלויה בצואת האדם הואי דכתיב (יחזקאל ד, יב) והיא בגללי צאת האדם תעגנה לעיניהם}
\textblock{מאי ועוגת שעורים תאכלנה אמר רב חסדא לשיעורים רב פפא אמר עריבתה כעריבת שעורים ולא כעריבת חטים:}
\textblock{{\large\emph{מתני׳}} נותן אדם מעה לחנוני ולנחתום כדי שיזכה לו עירוב דברי רבי אליעזר}
\textblock{וחכמים אומרים לא זכו לו מעותיו}
\textblock{ומודים בשאר כל האדם שזכו לו מעותיו שאין מערבין לאדם אלא מדעתו}
\textblock{אמר רבי יהודה במה דברים אמורים בעירובי תחומין אבל בעירובי חצירות מערבין לדעתו ושלא לדעתו לפי שזכין לאדם שלא בפניו ואין חבין לאדם שלא בפניו:}
\textblock{{\large\emph{גמ׳}} מ"ט דר' אליעזר הא לא משך}
\textblock{אמר רב נחמן אמר רבה בר אבהו עשאו ר"א כד' פרקים בשנה דתנן בד' פרקים אלו משחיטין את הטבח בעל כרחו אפילו שור שוה אלף דינר ואין ללוקח אלא דינר אחד כופין אותו לשחוט}
\textblock{לפיכך אם מת מת ללוקח מת ללוקח הא לא משך אמר רב הונא בשמשך}
\textblock{אי הכי אימא סיפא בשאר ימות השנה אינו כן לפיכך אם מת מת למוכר אמאי הא משך}
\textblock{א"ר שמואל בר יצחק לעולם בשלא משך הכא במאי עסקינן בשזיכה לו על ידי אחר}
\textblock{בד' פרקים אלו דזכות הוא לו זכין לו שלא בפניו בשאר ימות השנה דחוב הוא לו אין חבין לו אלא בפניו}
\textblock{ורב אילא אמר ר' יוחנן בד' פרקים אלו העמידו חכמים דבריהן על דברי תורה דאמר רבי יוחנן דבר תורה מעות קונות}
\textblock{ומפני מה אמרו משיכה קונה גזירה שמא יאמר לו נשרפו חיטיך בעלייה:}
\textblock{ומודים בשאר כל האדם כו': מאן שאר כל אדם אמר רב בעל הבית}
\textblock{וכן אמר שמואל בעה"ב דאמר שמואל ל"ש אלא נחתום אבל בעה"ב קונה ואמר שמואל ל"ש אלא מעה אבל כלי קונה}
\textblock{ואמר שמואל לא שנו אלא דאמר לו זכה לי אבל אמר ערב לי שליח שויה וקני:}
\textblock{אמר רבי יהודה בד"א וכו': אמר רב יהודה אמר שמואל הלכה כרבי יהודה ולא עוד אלא כל מקום ששנה רבי יהודה בעירובין הלכה כמותו}
\textblock{אמר ליה רב חנא בגדתאה לרב יהודה אמר שמואל אפילו במבוי שניטלו קורותיו או לחייו}
\textblock{אמר ליה בעירובין אמרתי לך ולא במחיצות}
\textblock{אמר ליה רב אחא בריה דרבא לרב אשי הלכה מכלל דפליגי והאמר ר' יהושע בן לוי כל מקום שאמר ר' יהודה אימתי ובמה במשנתנו אינו אלא לפרש דברי חכמים}
\textblock{ולא פליגי והא אנן תנן נתוספו עליהן מוסיף ומזכה וצריך להודיע}
\textblock{התם בחצר שבין שני מבואות}
\textblock{והאמר רב שיזבי אמר רב חסדא זאת אומרת חלוקין עליו חביריו על רבי יהודה}
\textblock{אלא}
\newsection{דף פב}
\textblock{גברא אגברא קא רמית מר סבר פליגי ומר סבר לא פליגי}
\textblock{גופא א"ר יהושע בן לוי כל מקום שאמר ר' יהודה אימתי ובמה במשנתינו אינו אלא לפרש דברי חכמים ור' יוחנן אמר אימתי לפרש ובמה לחלוק}
\textblock{ואימתי לפרש הוא והא תנן ואלו הן הפסולים המשחק בקוביא ומלוה בריבית ומפריחי יונים וסוחרי שביעית}
\textblock{אמר ר' יהודה אימתי בזמן שאין לו אומנות אלא היא אבל יש לו אומנות שלא היא הרי זה כשר}
\textblock{ותני עלה בברייתא וחכ"א בין שאין לו אומנות אלא היא ובין שיש לו אומנות שלא היא הרי זה פסול}
\textblock{ההיא דרבי יהודה א"ר טרפון היא}
\textblock{דתניא א"ר יהודה משום ר"ט לעולם אין אחד מהן נזיר לפי שאין נזירות אלא להפלאה}
\textblock{אלמא כיון דמספקא ליה אי נזיר אי לא נזיר הוא לא משעביד נפשיה הכא נמי כיון דלא ידע אי קני אי לא קני לא גמר ומקנה:}
\textblock{\par \par {\large\emph{הדרן עלך חלון}}\par \par }
\textblock{}
\textblock{מתני׳ {\large\emph{כיצד}} משתתפין בתחומין מניח את החבית ואומר הרי זה לכל בני עירי לכל מי שילך לבית האבל או לבית המשתה וכל שקיבל עליו מבע"י מותר משתחשך אסור שאין מערבין משתחשך:}
\textblock{{\large\emph{גמ׳}} אמר רב יוסף שאין מערבין אלא לדבר מצוה מאי קמ"ל תנינא לכל מי שילך לבית האבל או לבית המשתה}
\textblock{מהו דתימא אורחא דמלתא קתני קמ"ל}
\textblock{וכל שקיבל עליו מבע"י שמעת מינה אין ברירה דאי יש ברירה תיגלי מילתא למפרע דמבעוד יום הוה ניחא ליה}
\textblock{אמר רב אשי הודיעוהו ולא הודיעוהו קתני}
\textblock{אמר רב אסי קטן בן שש יוצא בעירוב אמו מיתיבי קטן שצריך לאמו יוצא בעירוב אמו ושאין צריך לאמו אין יוצא בעירוב אמו}
\textblock{ותנן נמי גבי סוכה כי האי גוונא קטן שאין צריך לאמו חייב בסוכה}
\textblock{והוינן בה ואיזהו קטן שאין צריך לאמו אמרי דבי ר' ינאי כל שנפנה ואין אמו מקנחתו}
\textblock{ר"ש בן לקיש אמר כל שניעור ואינו קורא אימא אימא ס"ד גדולים נמי קרו אלא אימא כל שניעור משנתו ואינו קורא אימא אימא}
\textblock{}
\textblock{וכמה כבר ארבע כבר חמש}
\newchap{פרק \hebrewnumeral{8}\quad כיצד משתתפין}
\textblock{}
\textblock{אמר רב יהושע בריה דרב אידי כי קאמר רב אסי כגון שעירב עליו אביו לצפון ואמו לדרום דאפילו בר שש נמי בצוותא דאמיה ניחא ליה}
\textblock{מיתיבי קטן שצריך לאמו יוצא בעירוב אמו עד בן שש תיובתא דרב יהושע בר רב אידי תיובתא}
\textblock{לימא תיהוי תיובתיה דרב אסי אמר לך רב אסי עד ועד בכלל}
\textblock{לימא תיהוי תיובתיה דרבי ינאי ור"ל ל"ק הא דאיתיה אבוהי במתא הא דלא איתיה אבוהי במתא}
\textblock{ת"ר מערב אדם על ידי בנו ובתו הקטנים על ידי עבדו ושפחתו הכנענים בין לדעתן בין שלא לדעתן אבל אינו מערב לא ע"י עבדו ושפחתו העברים [ולא] על ידי בנו ובתו הגדולים ולא על ידי אשתו אלא מדעתם}
\textblock{תניא אידך לא יערב אדם על ידי בנו ובתו הגדולים וע"י עבדו ושפחתו העברים ולא ע"י אשתו אלא מדעתן אבל מערב הוא ע"י עבדו ושפחתו הכנענים ועל ידי בנו ובתו הקטנים בין לדעתן ובין שלא לדעתן מפני שידן כידו}
\textblock{וכולן שעירבו ועירב עליהם רבן יוצאין בשל רבן חוץ מן האשה מפני שיכולה למחות}
\textblock{אשה מאי שנא אמר רבה אשה וכל דדמי לה}
\textblock{אמר מר חוץ מן האשה מפני שיכולה למחות טעמא דמחי הא סתמא נפקא בדבעלה הא קתני רישא אלא מדעתם מאי לאו דאמרי אין}
\textblock{לא מאי אלא מדעתם דאשתיקו לאפוקי היכא דאמרי לא}
\textblock{הא וכולן שעירבו ועירב עליהן רבן יוצאין בשל רבן וסתמא הוא וקתני חוץ מן האשה דלא נפקי}
\textblock{אמר רבא כיון שעירבו אין לך מיחוי גדול מזה:}
\textblock{{\large\emph{מתני׳}} כמה הוא שיעורו מזון שתי סעודות לכל אחד ואחד מזונו לחול ולא לשבת דברי רבי מאיר רבי יהודה אומר לשבת ולא לחול וזה וזה מתכוונים להקל}
\textblock{ר' יוחנן בן ברוקה אומר מככר בפונדיון מד' סאין בסלע ר"ש אומר שתי ידות לככר משלש לקב}
\textblock{חצייה לבית המנוגע וחצי חצייה לפסול את הגוייה:}
\textblock{{\large\emph{גמ׳}} וכמה מזון שתי סעודות א"ר יהודה אמר רב תרתי ריפתא איכרייתא רב אדא בר אהבה אמר תרתי ריפתא נהר פפיתא}
\textblock{א"ל רב יוסף לרב יוסף בריה דרבא אבוך כמאן ס"ל כר' מאיר ס"ל אנא נמי כר"מ ס"ל דאי כרבי יהודה קשיא הא דאמרי אינשי רווחא לבסימא שכיח:}
\textblock{רבי יוחנן בן ברוקה אומר: תנא וקרובים דבריהן להיות שוין מי דמי דר"י ארבע סעודתא לקבא דר"ש תשע סעודתא לקבא}
\textblock{א"ר חסדא צא מהן שליש לחנוני}
\textblock{ואכתי למר תשעה ולמר שית}
\textblock{[אלא כאידך] דרב חסדא דאמר צא מהן מחצה לחנוני}
\textblock{ואכתי למר תשע ולמר תמני היינו דקאמר וקרובים דבריהם להיות שוין}
\textblock{קשיא דרב חסדא אדרב חסדא ל"ק הא דקא יהיב בעל הבית ציבי הא דלא יהיב בעל הבית ציבי:}
\textblock{חצייה לבית המנוגע וחצי חצייה לפסול את הגוייה}
\newsection{דף פג}
\textblock{תנא וחצי חצי חציה לטמא טומאת אוכלין ותנא דידן מ"ט לא תני טומאת אוכלין משום דלא שוו שיעורייהו להדדי}
\textblock{דתניא כמה שיעור חצי פרס ב' ביצים חסר קימעא דברי רבי יהודה ר' יוסי אומר ב' ביצים שוחקות שיער רבי ב' ביצים ועוד כמה ועוד אחד מעשרים בביצה}
\textblock{ואילו גבי טומאת אוכלין תניא רבי נתן ורבי דוסא אמרו כביצה שאמרו כמוה וכקליפתה וחכ"א כמוה בלא קליפתה}
\textblock{אמר רפרם בר פפא אמר רב חסדא זו דברי רבי יהודה ורבי יוסי אבל חכ"א כביצה ומחצה שוחקות ומאן חכמים רבי יוחנן בן ברוקה}
\textblock{פשיטא שוחקות אתא לאשמועינן}
\textblock{כי אתא רב דימי אמר שיגר בוניוס לרבי מודיא דקונדיס דמן נאוסא ושיער רבי מאתן ושבע עשרה ביעין}
\textblock{הא סאה דהיכא אי דמדברית קמ"ד הויא}
\textblock{ואי דירושלמית קע"ג הויא}
\textblock{ואי דציפורית ר"ז הויין}
\textblock{לעולם דציפורית אייתי חלתא שדי עלייהו}
\textblock{חלתא כמה הויין תמני אכתי בצר ליה}
\textblock{אלא אייתי ועודות דרבי שדי עלייהו}
\textblock{אי הכי הוי ליה טפי כיון דלא הוי כביצה לא חשיב ליה}
\textblock{ת"ר סאה ירושלמית יתירה על מדברית שתות ושל ציפורית יתירה על ירושלמית שתות נמצאת של ציפורית יתירה על מדברית שליש}
\textblock{שליש דמאן אילימא שליש דמדברית מכדי שליש דמדברית כמה הוי ארבעין ותמניא ואילו עודפא שיתין ותלת}
\textblock{ואלא שליש דירושלמית שליש דידה כמה הוי חמשין ותמניא נכי תילתא ואילו עודפא שתין ותלת ואלא דציפורי שליש דידה כמה הוי שבעין נכי חדא ואילו עודפא ס"ג}
\textblock{אלא א"ר ירמיה ה"ק נמצאת סאה של ציפורי יתירה על מדברית קרוב לשליש שלה ושליש שלה קרוב למחצה דמדברית}
\textblock{מתקיף לה רבינא מידי קרוב קרוב קתני אלא אמר רבינא ה"ק נמצאת שליש של ציפורי בועודיות של רבי יתירה על מחצה של מדברית שליש ביצה}
\textblock{תנו רבנן (במדבר טו, כ) ראשית עריסותיכם}
\textblock{כדי עיסותיכם וכמה עיסותיכם כדי עיסת המדבר וכמה עיסת המדבר}
\textblock{דכתיב (שמות טז, לו) והעומר עשירית האיפה הוא מכאן אמרו ז' רבעים קמח ועוד חייבת בחלה שהן ו' של ירושלמית שהן ה' של ציפורי}
\textblock{מכאן אמרו האוכל כמדה זו ה"ז בריא ומבורך יתר על כן רעבתן פחות מכאן מקולקל במעיו:}
\textblock{{\large\emph{מתני׳}} אנשי חצר ואנשי מרפסת ששכחו ולא עירבו כל שגבוה י' טפחים למרפסת פחות מכאן לחצר}
\textblock{חוליית הבור והסלע גבוהים עשרה טפחים למרפסת פחות מכאן לחצר}
\textblock{בד"א בסמוכה אבל במופלגת אפילו גבוה י' טפחים לחצר ואיזו היא סמוכה כל שאינה רחוקה ארבעה טפחים:}
\textblock{{\large\emph{גמ׳}} פשיטא לזה בפתח ולזה בפתח היינו חלון שבין שתי חצירות}
\textblock{לזה בזריקה ולזה בזריקה היינו כותל שבין ב' חצירות לזה בשלשול ולזה בשלשול היינו חריץ שבין ב' חצירות}
\textblock{לזה בפתח ולזה בזריקה היינו דרבה בר רב הונא אמר רב נחמן לזה בפתח ולזה בשלשול היינו דרב שיזבי אמר רב נחמן}
\textblock{לזה בשלשול ולזה בזריקה מאי}
\textblock{אמר רב שניהן אסורין ושמואל אמר נותנין אותו לזה שבשלשול שלזה תשמישו בנחת ולזה תשמישו בקשה וכל דבר שתשמישו לזה בנחת ולזה בקשה נותנים אותו לזה שתשמישו בנחת}
\textblock{תנן אנשי חצר ואנשי מרפסת ששכחו ולא עירבו כל שגבוה עשרה טפחים למרפסת פחות מכאן לחצר}
\textblock{קא סלקא דעתך מאי מרפסת}
\newsection{דף פד}
\textblock{בני עלייה ומאי קרו לה מרפסת דקסלקי במרפסת אלמא כל לזה בשלשול ולזה בזריקה נותנין אותו לזה שבשלשול}
\textblock{כדאמר רב הונא לאותן הדרים במרפסת ה"נ לאותן הדרין במרפסת}
\textblock{אי הכי אימא סיפא פחות מכאן לחצר אמאי לזה בפתח ולזה בפתח הוא}
\textblock{מאי לחצר אף לחצר ושניהן אסורין}
\textblock{ה"נ מסתברא מדקתני סיפא בד"א בסמוכה אבל במופלגת אפילו גבוה י' טפחים לחצר מאי לחצר אילימא לחצר ושרי אמאי רשותא דתרוייהו הוא}
\textblock{אלא מאי לחצר אף לחצר ושניהן אסורין ה"נ מאי לחצר אף לחצר ושניהן אסורין ש"מ}
\textblock{תנן חוליית הבור והסלע שהן גבוהין עשרה למרפסת פחות מכאן לחצר אמר רב הונא לאותן הדרים במרפסת}
\textblock{תינח סלע בור מאי איכא למימר}
\textblock{אמר רב יצחק בריה דרב יהודה הכא בבור מלאה מים עסקינן והא חסרא}
\textblock{כיון דכי מליא שריא כי חסרא נמי שריא אדרבה כיון דכי חסרא אסירא כי מליא נמי אסירא}
\textblock{אלא אמר אביי הכא בבור מליאה פירות עסקינן והא חסרי}
\textblock{בטיבלא}
\textblock{דיקא נמי דקתני דומיא דסלע ש"מ}
\textblock{ול"ל למיתנא בור ול"ל למיתנא סלע צריכא דאי אשמעינן סלע דליכא למיגזר אבל בור ליגזור זמנין דמליא פירות מתוקנין צריכא}
\textblock{תא שמע אנשי חצר ואנשי עלייה ששכחו ולא עירבו אנשי חצר משתמשין בעשרה התחתונים ואנשי עלייה משתמשין בעשרה העליונים כיצד זיז יוצא מן הכותל למטה מעשרה לחצר למעלה מעשרה לעלייה}
\textblock{הא דביני ביני אסור}
\textblock{אמר רב נחמן הכא בכותל תשעה עשר עסקינן וזיז יוצא ממנו למטה מעשרה לזה בפתח ולזה בשלשול למעלה מעשרה לזה בפתח ולזה בזריקה}
\textblock{תא שמע שתי גזוזטראות זו למעלה מזו עשו לעליונה ולא עשו לתחתונה שתיהן אסורות עד שיערבו}
\textblock{אמר רב אדא בר אהבה בבאין בני תחתונה דרך עליונה למלאות}
\textblock{אביי אמר כגון דקיימין בתוך עשרה דהדדי ולא מיבעיא קאמר לא מיבעיא עשו לתחתונה ולא עשו לעליונה דאסירי דכיון דבגו י' דהדדי קיימין אסרן אהדדי}
\textblock{אלא אפילו עשו לעליונה ולא עשו לתחתונה סד"א כיון דלזה בנחת ולזה בקשה ליתביה לזה שתשמישו בנחת קמ"ל כיון דבגו עשרה קיימין אסרן אהדדי}
\textblock{כי הא דאמר רב נחמן אמר שמואל גג הסמוך לרה"ר צריך סולם קבוע להתירו סולם קבוע אין סולם עראי לא מ"ט לאו משום דכיון דבתוך עשרה דהדדי קיימי אסרן אהדדי}
\textblock{מתקיף לה רב פפא ודילמא כשרבים מכתפין עליו בכומתא וסודרא}
\textblock{אמר רב יהודה אמר שמואל}
\newsection{דף פה}
\textblock{בור שבין שתי חצירות מופלגת מכותל זה ארבעה ומכותל זה ארבעה זה מוציא זיז כל שהוא וממלא וזה מוציא זיז כל שהוא וממלא ורב יהודה דידיה אמר אפילו קניא}
\textblock{אמר ליה אביי לרב יוסף הא דרב יהודה דשמואל היא דאי דרב הא אמר אין אדם אוסר על חבירו דרך אויר}
\textblock{ודשמואל מהיכא אילימא מהא דאמר רב נחמן אמר שמואל גג הסמוך לרשות הרבים צריך סולם קבוע להתירו דילמא כדרב פפא}
\textblock{אלא מהא זה מוציא זיז כל שהוא וממלא וזה מוציא זיז כל שהוא וממלא טעמא דאפיק הא לא אפיק אמרינן אדם אוסר על חבירו דרך אויר}
\textblock{ודרב מהיכא אילימא מהא שתי גזוזטראות זו למעלה מזו עשו מחיצה לעליונה ולא עשו מחיצה לתחתונה שתיהן אסורות עד שיערבו}
\textblock{ואמר רב הונא אמר רב לא שנו אלא בסמוכה אבל במופלגת ארבעה עליונה מותרת ותחתונה אסורה}
\textblock{דלמא שאני הכא דכיון דלזה בזריקה ושלשול ולזה בשלשול לחודיה כלזה בזריקה ולזה בפתח דמי}
\textblock{אלא מהא דאמר רב נחמן אמר רבה בר אבוה אמר רב שני בתים ושלש חורבות ביניהם זה משתמש בסמוך שלו על ידי זריקה וזה משתמש בסמוך שלו על ידי זריקה}
\textblock{והאמצעי אסור}
\textblock{יתיב רב ברונא וקאמר להא שמעתא א"ל רבי אליעזר בר בי רב אמר רב הכי א"ל אין אחוי לי אושפיזיה אחוי ליה אתא לקמיה דרב א"ל אמר מר הכי א"ל אין}
\textblock{א"ל והא מר הוא דאמר לזה בשלשול ולזה בזריקה שניהן אסורין}
\textblock{א"ל מי סברת דקיימי כשורה לא דקיימי כחצובה }
\textblock{א"ל רב פפא לרבא לימא שמואל לית ליה דרב דימי דכי אתא רב דימי א"ר יוחנן מקום שאין בו ד' על ד' מותר לבני רה"ר ולבני רה"י לכתף עליו ובלבד שלא יחליפו}
\textblock{התם רשויות דאורייתא הכא רשויות דרבנן וחכמים עשו חיזוק לדבריהם יותר משל תורה}
\textblock{אמר ליה רבינא לרבא מי אמר רב הכי והא איתמר שני בתים משני צידי רשות הרבים רבה בר רב הונא אמר רב אסור לזרוק מזה לזה ושמואל אמר מותר לזרוק מזה לזה}
\textblock{א"ל לאו מי אוקימנא דמדלי חד ומתתי חד זימנין דמגנדר ונפיל ואתי לאיתויי:}
\textblock{{\large\emph{מתני׳}} הנותן את עירובו בבית שער אכסדרה ומרפסת אינו עירוב והדר שם אינו אוסר עליו}
\textblock{בית התבן ובית הבקר ובית העצים ובית האוצרות הרי זה עירוב והדר שם אוסר רבי יהודה אומר אם יש שם תפיסת יד של בעל הבית אינו אוסר:}
\textblock{{\large\emph{גמ׳}} אמר רב יהודה בריה דרב שמואל בר שילת כל מקום שאמרו הדר שם אינו אוסר הנותן את עירובו אינו עירוב חוץ מבית שער דיחיד וכל מקום שאמרו חכמים אין מניחין בו עירוב מניחין בו שיתוף חוץ מאויר מבוי}
\textblock{מאי קמ"ל תנינא הנותן את עירובו בבית שער אכסדרה ומרפסת אינו עירוב עירוב הוא דלא הוי הא שיתוף הוי}
\textblock{בית שער דיחיד ואויר דמבוי איצטריכא ליה דלא תנן תניא נמי הכי הנותן את עירובו בבית שער אכסדרה ומרפסת ובחצר ובמבוי ה"ז עירוב והתנן אין זה עירוב אימא ה"ז שיתוף}
\textblock{שיתוף במבוי לא מינטר אימא בחצר שבמבוי}
\textblock{אמר רב יהודה אמר שמואל בני חבורה שהיו מסובין וקדש עליהן היום פת שעל השלחן סומכין עליהן משום עירוב ואמרי לה משום שיתוף}
\textblock{אמר רבה ולא פליגי כאן במסובין בבית כאן במסובין בחצר}
\textblock{אמר ליה אביי לרבה תניא דמסייע לך עירובי חצירות בחצר ושיתופי מבואות במבוי והוינן בה עירובי חצירות בחצר והתנן הנותן את עירובו בבית שער אכסדרה ומרפסת אינו עירוב אימא עירובי חצירות בבית שבחצר שיתופי מבואות בחצר שבמבוי:}
\textblock{רבי יהודה אומר אם יש שם תפיסת יד וכו': היכי דמי תפיסת יד כגון חצירו של בונייס}
\textblock{בן בונייס אתא לקמיה דרבי אמר להו פנו מקום לבן מאה מנה אתא איניש אחרינא אמר להו}
\newsection{דף פו}
\textblock{פנו מקום לבן מאתים מנה אמר לפניו רבי ישמעאל ברבי יוסי רבי אביו של זה יש לו אלף ספינות בים וכנגדן אלף עיירות ביבשה אמר לו לכשתגיע אצל אביו אמור לו אל תשגרהו בכלים הללו לפני}
\textblock{רבי מכבד עשירים ר' עקיבא מכבד עשירים כדדרש רבא בר מרי (תהלים סא, ח) ישב עולם לפני אלהים חסד ואמת מן ינצרוהו אימתי ישב עולם לפני אלהים בזמן שחסד ואמת מן ינצרוהו}
\textblock{רבה בר בר חנה אמר כגון יתד של מחרישה}
\textblock{אמר רב נחמן תנא דבי שמואל דבר הניטל בשבת אוסר דבר שאינו ניטל בשבת אינו אוסר}
\textblock{תניא נמי הכי יש לו טבל יש לו עששית וכל דבר שאינו ניטל בשבת אינו אוסר:}
\textblock{{\large\emph{מתני׳}} המניח ביתו והלך לשבות בעיר אחרת אחד נכרי ואחד ישראל הרי זה אוסר דברי ר' מאיר ר' יהודה אומר אינו אוסר}
\textblock{רבי יוסי אומר נכרי אוסר ישראל אינו אוסר שאין דרך ישראל לבא בשבת}
\textblock{רבי שמעון אומר אפילו הניח ביתו והלך לשבות אצל בתו באותה העיר אינו אוסר שכבר הסיע מלבו:}
\textblock{{\large\emph{גמ׳}} אמר רב הלכה כר"ש ודוקא בתו אבל בנו לא דאמרי אינשי נבח בך כלבא עול נבח בך גורייתא פוק:}
\textblock{{\large\emph{מתני׳}} בור שבין שתי חצירות אין ממלאין ממנו בשבת אלא אם כן עשו לו מחיצה גבוה עשרה טפחים בין מלמטה בין מתוך אוגנו}
\textblock{רבן שמעון בן גמליאל אומר בית שמאי אומרים מלמטה ובית הלל אומרים מלמעלה א"ר יהודה לא תהא מחיצה גדולה מן הכותל שביניהם:}
\textblock{{\large\emph{גמ׳}} אמר רב הונא למטה למטה ממש למעלה למעלה ממש וזה וזה בבור ורב יהודה אמר למטה למטה מן המים למעלה למעלה מן המים}
\textblock{א"ל רבה בר רב חנן לאביי הא דאמר רב יהודה למטה למטה מן המים מאי שנא למטה ממש דלא דעריבי מיא למטה מן המים נמי הא עריבי מיא}
\textblock{א"ל לא שמיע לך הא דאמר רב יהודה אמר רב ומטו בה משום רבי חייא צריך שיראו ראשן של קנים למעלה מן המים טפח}
\textblock{ותו הא דאמר רב יהודה למעלה למעלה מן המים מאי שנא למעלה ממש דלא דעריבי מיא למעלה מן המים נמי הא עריבי מיא אמר ליה לא שמיע לך הא דתני יעקב קרחינאה צריך שישקע ראשי קנים במים טפח}
\textblock{ואלא הא דאמר רב יהודה קורה ארבע מתרת בחורבה ורב נחמן אמר רבה בר אבוה}
\textblock{קורה ארבעה מתרת במים}
\textblock{הא קא אזיל דלי לאידך גיסא ומייתי קים להו לרבנן דאין דלי מהלך יותר מארבעה טפחים }
\textblock{תחת קורה מיהא הא עריבי מיא אלא משום דקל הוא שהקילו חכמים במים כדבעא מיניה רבי טבלא מרב מחיצה תלויה מהו שתתיר בחורבה א"ל אין מחיצה תלויה מתרת אלא במים קל הוא שהקילו חכמים במים:}
\textblock{א"ר יהודה לא תהא מחיצה: אמר רבה בר בר חנה אמר רבי יוחנן ר' יהודה בשיטת ר' יוסי אמרה דאמר מחיצה תלויה מתרת אפילו ביבשה}
\textblock{דתנן המשלשל דפנות מלמעלה למטה בזמן שגבוהות מן הארץ ג' טפחים פסולה ממטה למעלה אם גבוהות י' טפחים כשירה}
\textblock{רבי יוסי אומר כשם שמלמטה למעלה עשרה כך מלמעלה למטה עשרה}
\textblock{ולא היא לא רבי יהודה סבר לה כרבי יוסי ולא ר' יוסי סבר לה כרבי יהודה}
\textblock{רבי יהודה לא סבר לה כרבי יוסי עד כאן לא קאמר רבי יהודה אלא בעירובי חצירות דרבנן אבל סוכה דאורייתא לא}
\textblock{ולא ר' יוסי סבר לה כר' יהודה עד כאן לא קאמר ר' יוסי אלא בסוכה דאיסור עשה הוא אבל שבת דאיסור סקילה הוא לא אמר}
\textblock{ואם תאמר אותו מעשה שנעשה בציפורי על פי מי נעשה לא על פי רבי יוסי אלא על פי רבי ישמעאל בר' יוסי נעשה}
\textblock{דכי אתא רב דימי אמר פעם אחת שכחו ולא הביאו ספר תורה מבעוד יום למחר פרסו סדין על העמודים והביאו ספר תורה וקראו בו}
\textblock{פרסו לכתחילה מי שרי והא הכל מודים שאין עושין אהל עראי בשבת}
\textblock{אלא מצאו סדינין פרוסין על העמודים והביאו ספר תורה וקראו בו}
\textblock{אמר רבה ר' יהודה ור' חנניא בן עקביא אמרו דבר אחד ר' יהודה הא דאמרן ר' חנניא בן עקביא (דתנן) ר' חנניא בן עקביא אומר גזוזטרא שיש בה ארבע אמות על ארבע אמות}
\newsection{דף פז}
\textblock{חוקק בה ד' על ד' וממלא}
\textblock{א"ל אביי ודילמא לא היא עד כאן לא קאמר רבי יהודה התם אלא דאמר גוד אחית מחיצתא אבל כוף וגוד לא}
\textblock{ועד כאן לא קאמר רבי חנניא בן עקביא התם אלא בימה של טבריא הואיל ויש לה אוגנים ועיירות וקרפיפות מקיפות אותה אבל בשאר מימות לא}
\textblock{אמר אביי ולדברי ר' חנניא בן עקביא היתה סמוכה לכותל בפחות מג' טפחים צריך שיהא אורכה ד' אמות ורוחבה אחד עשר ומשהו}
\textblock{היתה זקופה צריך שיהא גובהה עשרה טפחים ורוחבה ו' טפחים ושני משהויין}
\textblock{אמר רב הונא בריה דרב יהושע היתה עומדת בקרן זוית צריך שיהא גובהה י' טפחים ורוחבה ב' טפחים ושני משהויין}
\textblock{ואלא הא דתניא ר' חנניא בן עקביא אומר גזוזטרא שיש בה ד' אמות על ד"א חוקק בה ד' על ד' וממלא היכי משכחת לה}
\textblock{דעבידא כי אסיתא:}
\textblock{{\large\emph{מתני׳}} אמת המים שהיא עוברת בחצר אין ממלאין הימנה בשבת אלא אם כן עשו לה מחיצה גבוה י' טפחים בכניסה וביציאה ר' יהודה אומר כותל שעל גבה תידון משום מחיצה}
\textblock{אמר רבי יהודה מעשה באמה של אבל שהיו ממלאין ממנה על פי זקנים בשבת אמרו לו מפני שלא היה בה כשיעור:}
\textblock{{\large\emph{גמ׳}} תנו רבנן עשו לה בכניסה ולא עשו לה ביציאה עשו לה ביציאה ולא עשו לה בכניסה אין ממלאין הימנה בשבת אא"כ עשו לה מחיצה י' טפחים ביציאה ובכניסה ר' יהודה אומר כותל שעל גבה תידון משום מחיצה}
\textblock{אמר רבי יהודה מעשה באמת המים שהיתה באה מאבל לצפורי והיו ממלאין הימנה בשבת על פי הזקנים}
\textblock{אמרו לו משם ראייה מפני שלא היתה עמוקה י' טפחים ורוחבה ד'}
\textblock{תניא אידך אמת המים העוברת בין החלונות פחות מג' משלשל דלי וממלא ג' אין משלשל דלי וממלא רשב"ג אומר פחות מד' משלשל דלי וממלא ד' אין משלשל דלי וממלא}
\textblock{במאי עסקינן אילימא באמת המים גופה ואלא הא דכי אתא רב דימי אמר ר' יוחנן אין כרמלית פחותה מד'}
\textblock{לימא כתנאי אמרה לשמעתיה}
\textblock{אלא באגפיה ולהחליף}
\textblock{והא כי אתא רב דימי אמר רבי יוחנן מקום שאין בו ארבעה על ארבעה מותר לבני רשות היחיד ולבני רשות הרבים לכתף עליו ובלבד שלא יחליפו}
\textblock{התם רשויות דאורייתא}
\textblock{הכא רשויות דרבנן}
\textblock{והא ר' יוחנן ברשויות דרבנן נמי אמר (דתניא) כותל שבין ב' חצירות גבוה י' טפחים ורוחב ארבעה מערבין שנים ואין מערבין אחד}
\textblock{היו בראשו פירות אלו עולין מכאן ואוכלין ואלו עולין מכאן ואוכלין}
\textblock{נפרץ הכותל עד עשר אמות מערבין שנים ואם רצו מערבין אחד מפני שהוא כפתח יותר מכאן מערבין אחד ואין מערבין שנים}
\textblock{והוינן בה אין בו ארבעה מאי אמר רב אויר ב' רשויות שולטת בו ולא יזיז בו מלא נימא}
\textblock{ורבי יוחנן אמר אלו מעלין מכאן ואוכלין ואלו מעלין מכאן ואוכלין}
\textblock{ואזדא רבי יוחנן לטעמיה דכי אתא רב דימי אמר רבי יוחנן מקום שאין בו ארבעה על ארבעה מותר לבני רה"ר ולבני רה"י לכתף עליו ובלבד שלא יחליפו}
\textblock{ההיא זעירי אמרה ולזעירי קשיא הא}
\textblock{זעירי מוקים לה באמת המים גופה ורב דימי תנאי היא}
\textblock{ותיהוי כי חורי כרמלית}
\textblock{אביי בר אבין ורב חנינא בר אבין דאמרי תרוייהו אין חורין לכרמלית}
\textblock{רב אשי אמר אפילו תימא יש חורין לכרמלית ה"מ בסמוכה הכא במופלגת}
\textblock{רבינא אמר כגון דעבד לה ניפקי אפומה}
\textblock{ואזדו רבנן לטעמייהו ור' שמעון בן גמליאל לטעמיה:}
\textblock{{\large\emph{מתני׳}} גזוזטרא שהיא למעלה מן המים אין ממלאין הימנה בשבת אלא אם כן עשו לה מחיצה גבוהה עשרה טפחים בין מלמעלה בין מלמטה}
\textblock{וכן שתי גזוזטראות זו למעלה מזו עשו לעליונה ולא עשו לתחתונה שתיהן אסורות עד שיערבו:}
\textblock{{\large\emph{גמ׳}} מתניתין דלא כחנניא בן עקביא דתניא חנניא בן עקביא אומר גזוזטרא שיש בה ד' על ד' אמות חוקק בה ד' על ד' וממלא}
\textblock{אמר ר' יוחנן משום רבי יוסי בן זימרא לא התיר רבי חנניא בן עקביא אלא בימה של טבריא הואיל ויש לה אוגנים ועיירות וקרפיפות מקיפות אותה אבל בשאר מימות לא}
\textblock{ת"ר ג' דברים התיר רבי חנניא בן עקביא לאנשי טבריא ממלאין מים מגזוזטרא בשבת וטומנין בעצה ומסתפגין באלונטית}
\textblock{ממלאין מים מגזוזטרא בשבת הא דאמרן וטומנין בעצה מאי היא דתניא השכים להביא פסולת אם בשביל שיש עליו טל הרי הוא בכי יותן}
\textblock{ואם בשביל שלא יבטל ממלאכתו אינו בכי יותן וסתם}
\newsection{דף פח}
\textblock{אנשי טבריא כמי שלא יבטל ממלאכתו דמי}
\textblock{ומסתפגין באלונטית מאי היא דתניא מסתפג אדם באלונטית ומניחה בחלון ולא ימסרנה לאוליירין מפני שחשודין על אותו דבר ר"ש אומר אף מביאה בידו לתוך ביתו}
\textblock{אמר רבה בר רב הונא לא שנו אלא למלאות אבל לשפוך אסור}
\textblock{מתקיף לה רב שיזבי וכי מה בין זה לעוקה}
\textblock{הני תיימי והני לא תיימי}
\textblock{איכא דאמרי אמר רבה בר רב הונא לא תימא למלאות הוא דשרי לשפוך אסור אלא לשפוך נמי שרי אמר רב שיזבי פשיטא היינו עוקה מהו דתימא הני תיימי והני לא תיימי קמ"ל:}
\textblock{וכן שתי גזוזטראות זו וכו': אמר רב הונא אמר רב לא שנו אלא בסמוכה אבל במופלגת עליונה מותרת}
\textblock{ורב לטעמיה דאמר רב אין אדם אוסר על חבירו דרך אויר}
\textblock{אמר רבה א"ר חייא ורב יוסף אמר רבי אושעיא יש גזל בשבת וחורבה מחזיר לבעלים}
\textblock{הא גופא קשיא אמרת יש גזל בשבת אלמא קניא וחורבה מחזיר לבעלים אלמא לא קניא}
\textblock{הכי קאמר יש דין גזל בשבת כיצד דחורבה מחזיר לבעלים}
\textblock{אמר רבה ומותבינן אשמעתין וכן שתי גזוזטראות זו למעלה מזו וכו' ואי אמרת יש דין גזל בשבת אמאי אסורות}
\textblock{אמר רב ששת הכא במאי עסקינן כגון שעשו מחיצה בשותפות}
\textblock{אי הכי כי עשו לתחתונה נמי}
\textblock{כיון דעשו לתחתונה גלוי גלי דעתה דאנא בהדך לא ניחא לי:}
\textblock{{\large\emph{מתני׳}} חצר שהיא פחותה מארבע אמות אין שופכין בתוכה מים בשבת אא"כ עשו לה עוקה מחזקת סאתים מן הנקב ולמטה}
\textblock{בין מבחוץ בין מבפנים אלא שמבחוץ צריך לקמור מבפנים אין צריך לקמור}
\textblock{ר"א בן יעקב אומר ביב שהוא קמור ארבע אמות ברה"ר שופכים לתוכו מים בשבת וחכ"א אפילו גג או חצר מאה אמה לא ישפוך על פי הביב אבל שופך הוא לגג והמים יורדין לביב}
\textblock{החצר והאכסדרה מצטרפין לארבע אמות וכן שתי דיוטאות זו כנגד זו מקצתן עשו עוקה ומקצתן לא עשו עוקה את שעשו עוקה מותרין את שלא עשו עוקה אסורין:}
\textblock{{\large\emph{גמ׳}} מ"ט אמר רבה מפני שאדם עשוי להסתפק סאתים מים בכל יום בארבע אמות אדם רוצה לזלפן}
\textblock{פחות מד' שופכן אי דעביד עוקה שרי אי לא אסור}
\textblock{ר' זירא אמר ד' אמות תיימי פחות מד' אמות לא תיימי}
\textblock{מאי בינייהו אמר אביי אריך וקטין איכא בינייהו}
\textblock{תנן חצר ואכסדרה מצטרפין לד' אמות בשלמא לרבי זירא ניחא אלא לרבה קשיא}
\textblock{תרגמא רבי זירא אליבא דרבה באכסדרה מהלכת על פני כל החצר כולה}
\textblock{ת"ש חצר שאין בה ד' אמות על ד' אמות אין שופכין לתוכה מים בשבת בשלמא לרבה ניחא אלא לר' זירא קשיא}
\textblock{אמר לך ר' זירא הא מני רבנן היא ומתני' ר"א בן יעקב היא}
\textblock{ומאי דוחקיה דר' זירא לאוקמה למתני' כר"א בן יעקב אמר רבא מתניתין קשיתיה מאי איריא דתני חצר שהיא פחותה ליתני חצר שאין בה ד' אמות על ד' אמות}
\textblock{אלא לאו ש"מ דר"א בן יעקב היא ש"מ}
\textblock{והא מדסיפא ר' אליעזר בן יעקב רישא לאו ר"א בן יעקב}
\textblock{כולה ר"א בן יעקב היא וחסורי מיחסרא והכי קתני חצר שהיא פחותה מד' אמות אין שופכין לתוכה מים בשבת הא ד' אמות שופכין שר"א בן יעקב אומר ביב הקמור ד' אמות ברה"ר שופכין לתוכו מים בשבת:}
\textblock{ר"א בן יעקב אומר ביב הקמור:}
\textblock{מתני' דלא כחנניא דתניא חנניא אומר אפילו גג מאה אמה לא ישפוך לפי שאין הגג עשוי לבלוע אלא לקלח}
\textblock{תנא במה דברים אמורים בימות החמה אבל בימות הגשמים שופך ושונה ואינו נמנע מ"ט אמר רבא אדם רוצה שיבלעו מים במקומן}
\textblock{אמר ליה אביי והרי שופכין דאדם רוצה שיבלעו וקתני לא ישפוך}
\textblock{א"ל התם למאי ניחוש לה אי משום קלקול חצירו הא מיקלקלא וקיימא ואי משום גזירה שמא יאמרו צנורו של פלוני מקלח מים סתם צנורות מקלחים הם}
\textblock{אמר רב נחמן בימות הגשמים עוקה מחזיק סאתים נותנין לו סאתים מחזיק סאה נותנין לו סאה בימות החמה מחזיק סאתים נותנין לו סאתים סאה אין נותנין לו כל עיקר}
\textblock{בימות החמה נמי מחזיק סאה ניתיב ליה סאה גזרה דלמא אתי ליתן ליה סאתים א"ה בימות הגשמים נמי ליגזור}
\textblock{התם מאי ניחוש לה אי משום קילקול הא מיקלקלא וקיימא אי משום גזירה שמא יאמרו צנורו של פלוני מקלח מים סתם צנורות מקלחין הן}
\textblock{אמר אביי הלכך אפילו כור ואפילו כוריים:}
\textblock{וכן שתי דיוטאות זו כנגד זו: אמר רבא אפילו עירבו}
\textblock{אמר (ליה) אביי מאי טעמא אילימא משום נפישא דמיא והתניא אחת לי עוקה ואחת לי גיסטרא בריכה ועריבה אף על פי שנתמלאו מים מערב שבת שופכין לתוכן מים בשבת}
\textblock{אלא אי איתמר הכי איתמר אמר רבא}
\newsection{דף פט}
\textblock{לא שנו אלא שלא עירבו אבל עירבו מותרין}
\textblock{וכי לא עירבו מאי טעמא לא אמר רב אשי גזירה דילמא אתי לאפוקי ממאני דבתים להתם:}
\textblock{\par \par {\large\emph{הדרן עלך כיצד משתתפין}}\par \par }
\textblock{}
\textblock{מתני׳ {\large\emph{כל}} גגות העיר רשות אחת ובלבד שלא יהא גג גבוה י' או נמוך י' דברי ר"מ וחכ"א כל אחד ואחד רשות בפני עצמו}
\textblock{רבי שמעון אומר אחד גגות ואחד חצירות ואחד קרפיפות רשות אחת הן לכלים ששבתו לתוכן ולא לכלים ששבתו בתוך הבית:}
\textblock{{\large\emph{גמ׳}} יתיב אביי בר אבין ורבי חנינא בר אבין ויתיב אביי גבייהו ויתבי וקאמרי בשלמא רבנן סברי כשם שדיורין חלוקין למטה כך דיורין חלוקין למעלה}
\textblock{אלא רבי מאיר מאי קסבר אי קסבר כשם שדיורין חלוקין למטה כך דיורין חלוקין למעלה אמאי רשות אחת הן ואי קסבר אין חלוקין דכל למעלה מי' רשות אחת היא אפילו גג גבוה עשרה ונמוך י' נמי}
\textblock{אמר להו אביי לא שמיע לכו הא דאמר רב יצחק בר אבדימי אומר היה רבי מאיר כל מקום שאתה מוצא שתי רשויות והן רשות אחת כגון עמוד ברה"י גבוה עשרה ורחב ד' אסור לכתף עליו גזירה משום תל ברה"ר ה"נ גזירה משום תל ברה"ר}
\textblock{סבור מינה אפילו מכתשת ואפילו גיגית}
\textblock{אמר להו אביי הכי אמר מר לא אמר ר"מ אלא עמוד ואמת הריחים הואיל ואדם קובע להן מקום}
\textblock{והרי כותל שבין ב' חצירות דקבוע ואמר רב יהודה כשתימצי לומר לדברי ר"מ גגין רשות לעצמן חצירות רשות לעצמן קרפיפות רשות לעצמן}
\textblock{מאי לאו דשרי לטלטולי דרך כותל}
\textblock{אמר רב הונא בר יהודה אמר רב ששת לא להכניס ולהוציא דרך פתחים:}
\textblock{וחכ"א כל אחד ואחד רשות בפני עצמו: איתמר רב אמר אין מטלטלין בו אלא בד' אמות ושמואל אמר מותר לטלטל בכולו}
\textblock{במחיצות הניכרות דכולי עלמא לא פליגי כי פליגי במחיצות שאינן ניכרות}
\textblock{רב אמר אין מטלטלין בו אלא בד' אמו' לא אמר גוד אסיק מחיצתא ושמואל אמר מותר לטלטל בכולו דאמר גוד אסיק מחיצתא}
\textblock{}
\textblock{תנן וחכמים אומרים כל אחד ואחד}
\newchap{פרק \hebrewnumeral{9}\quad כל גגות}
\textblock{}
\textblock{רשות לעצמו בשלמא לשמואל ניחא אלא לרב קשיא}
\textblock{אמרי בי רב משמיה דרב שלא יטלטל ב' אמות בגג זה וב' אמות בגג זה}
\textblock{והא א"ר אלעזר כי הוינן בבבל הוה אמרינן בי רב משמיה דרב אמרו אין מטלטלין בו אלא בד' אמות והני דבי שמואל תנו אין להן אלא גגן}
\textblock{מאי אין להן אלא גגן לאו דשרו לטלטולי בכוליה ומי אלימא ממתני' דאוקימנא שלא יטלטל שתי אמות בגג זה ושתי אמות בגג זה ה"נ ב' אמות בגג זה וב' אמות בגג זה}
\textblock{אמר רב יוסף לא שמיע לי הא שמעתא א"ל אביי את אמרת ניהלן ואהא אמרת ניהלן גג גדול הסמוך לקטן הגדול מותר והקטן אסור}
\textblock{ואמרת לן עלה אמר רב יהודה אמר שמואל לא שנו אלא שיש דיורין על זה ודיורין על זה דהויא לה הא דקטן מחיצה נדרסת}
\textblock{אבל אין דיורין על זה ועל זה שניהן מותרין}
\textblock{א"ל אנא הכי אמרי לכו ל"ש אלא שיש מחיצה על זה ומחיצה על זה דגדול מישתרי בגיפופי וקטן נפרץ במלואו אבל אין מחיצה לא על זה ולא על זה שניהן אסורין}
\textblock{והא דיורין אמרת לן אי אמרי לכו דיורין הכי אמרי לכו ל"ש אלא שיש מחיצה ראויה לדירה על זה ומחיצה ראויה לדירה על זה דגדול מישתרי בגיפופי וקטן נפרץ במלואו}
\textblock{אבל יש מחיצה ראויה לדירה על הגדול ואין ראויה לדירה על הקטן אפילו קטן שרי לבני גדול מאי טעמא כיון דלא עבוד מחיצה סלוקי סליקו נפשייהו מהכא}
\textblock{כהא דאמר רב נחמן עשה סולם קבוע לגגו הותר בכל הגגין כולן}
\textblock{אמר אביי בנה עלייה על גבי ביתו ועשה לפניה דקה ארבע הותר בכל הגגין כולן}
\textblock{אמר רבא פעמים שהדקה לאיסור היכי דמי דעבידא להדי תרביצא דביתיה דאמר}
\newsection{דף צ}
\textblock{לנטורי תרביצא הוא דעבידא}
\textblock{בעי רמי בר חמא שתי אמות בגג ושתי אמות בעמוד מהו אמר רבה מאי קא מיבעיא ליה כרמלית ורה"י קא מיבעיא ליה}
\textblock{ורמי בר חמא אגב חורפיה לא עיין בה אלא הכי קמיבעיא ליה ב' אמות בגג וב' אמות באכסדרה מהו}
\textblock{מי אמרינן כיון דלא האי חזי לדירה ולא האי חזי לדירה חדא רשותא היא או דילמא כיון דמגג לגג אסיר מגג לאכסדרה נמי אסיר}
\textblock{בעי רב ביבי בר אביי ב' אמות בגג וב' אמות בחורבה מהו}
\textblock{אמר רב כהנא לאו היינו דרמי בר חמא אמר רב ביבי בר אביי וכי מאחר אתאי ונצאי אכסדרה לא חזיא לדירה וחורבה חזיא לדירה}
\textblock{וכי מאחר דחזיא לדירה מאי קמיבעיא ליה אם תימצי לומר קאמר אם תימצי לומר אכסדרה לא חזיא לדירה חורבה חזיא לדירה או דילמא השתא מיהא לית בה דיורין תיקו}
\textblock{גגין השוין לר"מ וגג יחידי לרבנן רב אמר מותר לטלטל בכולו ושמואל אמר אין מטלטלין בו אלא בד'}
\textblock{רב אמר מותר לטלטל בכולו קשיא דרב אדרב התם לא מינכרא מחיצתא הכא מינכרא מחיצתא}
\textblock{ושמואל אמר אין מטלטלין בו אלא בד' אמות קשיא דשמואל אדשמואל התם לא הוי יותר מבית סאתים הכא הוי יותר מבית סאתים והני מחיצות למטה עבידן למעלה לא עבידן והוה כקרפף יתר מבית סאתים שלא הוקף לדירה וכל קרפף יותר מבית סאתים שלא הוקף לדירה אין מטלטלין בו אלא בד'}
\textblock{איתמר ספינה רב אמר מותר לטלטל בכולה ושמואל אמר אין מטלטלין בה אלא בד' רב אמר מותר לטלטל בכולה}
\textblock{דהא איכא מחיצתא ושמואל אמר אין מטלטלין בה אלא בארבע אמות מחיצות להבריח מים עשויות}
\textblock{אמר ליה רב חייא בר יוסף לשמואל הילכתא כוותך או הילכתא כרב אמר ליה הילכתא כרב}
\textblock{אמר רב גידל אמר רב חייא בר יוסף ומודה רב שאם כפאה על פיה שאין מטלטלין בה אלא בארבע אמות כפאה למאי אילימא לדור תחתיה מאי שנא מגג יחידי}
\textblock{אלא שכפאה לזופתה}
\textblock{רב אשי מתני לה אספינה ורב אחא בריה דרבא מתני לה אאכסדרא דאיתמר אכסדרה בבקעה רב אמר מותר לטלטל בכולה ושמואל אמר אין מטלטלין בה אלא בארבע}
\textblock{רב אמר מותר לטלטל בכולה אמרינן פי תקרה יורד וסותם ושמואל אמר אין מטלטלין בה אלא בארבע לא אמרינן פי תקרה יורד וסותם}
\textblock{ורב אליבא דר"מ ליטלטלי מגג לחצר גזירה משום דרב יצחק בר אבדימי}
\textblock{ושמואל אליבא דרבנן ניטלטל מגג לקרפף אמר רבא בר עולא גזירה שמא יפחת הגג}
\textblock{א"ה מקרפף לקרפף נמי לא יטלטל דילמא מיפחית ואתי לטלטולי התם אי מיפחית מינכרא ליה מילתא הכא אי מיפחית לא מינכרא מילתא}
\textblock{אמר רב יהודה כשתמצא לומר לדברי רבי מאיר גגין רשות לעצמן חצירות רשות לעצמן}
\newsection{דף צא}
\textblock{קרפיפות רשות לעצמן לדברי חכמים גגין וחצירות רשות אחת קרפיפות רשות אחת הן לדברי רבי שמעון כולן רשות אחת הן}
\textblock{תניא כוותיה דרב תניא כוותיה דרב יהודה תניא כוותיה דרב כל גגות העיר רשות אחת הן ואסור להעלות ולהוריד מן הגגין לחצר ומן החצר לגגין וכלים ששבתו בחצר מותר לטלטלן בחצר בגגין מותר לטלטלן בגגין ובלבד שלא יהא גג גבוה י' או נמוך י' דברי ר"מ וחכ"א כל אחד ואחד רשות לעצמו ואין מטלטלין בו אלא בד'}
\textblock{תניא כוותיה דרב יהודה אמר רבי כשהיינו לומדים תורה אצל ר"ש בתקוע היינו מעלין שמן ואלונטית מגג לגג ומגג לחצר ומחצר לחצר ומחצר לקרפף ומקרפף לקרפף אחר עד שהיינו מגיעין אצל המעיין שהיינו רוחצין בו}
\textblock{אמר רבי יהודה מעשה בשעת הסכנה והיינו מעלין ס"ת מחצר לגג ומגג לחצר ומחצר לקרפף לקרות בו}
\textblock{אמרו לו אין שעת הסכנה ראיה:}
\textblock{ר"ש אומר אחד גגין וכו':}
\textblock{אמר רב הלכה כר"ש והוא שלא עירבו אבל עירבו לא דגזרינן דילמא אתי לאפוקי מאני דבתים לחצר}
\textblock{ושמואל אמר בין עירבו בין שלא עירבו וכן אמר ר' יוחנן מי לחשך בין עירבו ובין שלא עירבו}
\textblock{מתקיף לה רב חסדא לשמואל ולרבי יוחנן יאמרו שני כלים בחצר אחת זה מותר וזה אסור}
\textblock{ר"ש לטעמיה דלא גזר דתנן א"ר שמעון למה הדבר דומה לשלש חצירות הפתוחות זו לזו ופתוחות לרה"ר ועירבו שתי החיצונות עם האמצעית היא מותרת עמהן והן מותרות עמה וב' החיצונות אסורין זו עם זו}
\textblock{ולא גזר דילמא אתי לאפוקי מאני דהא חצר להא חצר ה"נ לא גזרינן דילמא אתי לאפוקי מאני דבתים לחצר}
\textblock{מתיב רב ששת ר"ש אומר אחד גגות אחד חצירות ואחד קרפיפות רשות אחת הן לכלים ששבתו בתוכן ולא לכלים ששבתו בתוך הבית אי אמרת בשלמא דעירבו היינו דמשכחת לה מאני דבתים בחצר}
\textblock{אלא אי אמרת בשלא עירבו היכי משכחת לה מאני דבתים בחצר הוא מותיב לה והוא מפרק לה בכומתא וסודרא}
\textblock{ת"ש אנשי חצר ואנשי מרפסת ששכחו ולא עירבו כל שגבוה י' טפחים למרפסת פחות מכאן לחצר בד"א שהיו אלו של רבים ואלו של רבים ועירבו אלו לעצמן ואלו לעצמן או של יחידים שאין צריכין לערב}
\textblock{אבל היו של רבים ושכחו ולא עירבו גג וחצר ואכסדרה ומרפסת כולן רשות אחת הן}
\textblock{טעמא דלא עירבו הא עירבו לא הא מני רבנן היא}
\textblock{דיקא נמי דלא קתני קרפף ומבוי ש"מ}
\textblock{ת"ש חמש חצירות הפתוחות זו לזו ופתוחות למבוי ושכחו כולם ולא עירבו אסור להכניס ולהוציא מחצר למבוי ומן המבוי לחצר וכלים ששבתו בחצר מותר לטלטלן בחצר ובמבוי אסור}
\textblock{ור"ש מתיר שהיה ר' שמעון אומר כל זמן שהן של רבים ושכחו ולא עירבו גג וחצר ואכסדרה ומרפסת וקרפף ומבוי כולן רשות אחת הן}
\textblock{טעמא דלא עירבו הא עירבו לא מאי לא עירבו לא עירבו חצירות בהדי הדדי הא חצר ובתים עירבו}
\textblock{והא לא עירבו קתני מאי לא עירבו לא נשתתפו}
\textblock{ואבע"א ר"ש לדבריהם דרבנן קאמר להו לדידי לא שנא עירבו ולא שנא לא עירבו אלא לדידכו אודו לי מיהת דהיכא דלא עירבו רשות אחת היא}
\textblock{ואמרו ליה רבנן לא שתי רשויות הן}
\textblock{אמר מר ובמבוי אסור לימא מסייע ליה לרבי זירא אמר רב דאמר רבי זירא אמר רב מבוי שלא נשתתפו בו אין מטלטלין אלא בד"א אימא ולמבוי אסור}
\textblock{היינו רישא משנה יתירא איצטריכא ליה מהו דתימא כי פליגי רבנן עליה דרבי שמעון הני מילי היכא דעירבו אבל היכא דלא עירבו מודו ליה קמ"ל}
\textblock{אמר ליה רבינא לרב אשי}
\newsection{דף צב}
\textblock{מי אמר ר' יוחנן הכי והא אמר רבי יוחנן הלכה כסתם משנה ותנן כותל שבין שתי חצירות גבוה עשרה ורוחב ארבעה מערבין שנים ואין מערבין אחד היו בראשו פירות אלו עולין מכאן ואוכלים ואלו עולין מכאן ואוכלים ובלבד שלא יורידו למטה}
\textblock{מאי למטה למטה לבתים והא תני רבי חייא ובלבד שלא יהא זה עומד במקומו ואוכל וזה עומד במקומו ואוכל}
\textblock{אמר ליה וכי רבי לא שנאה ר' חייא מנין לו}
\textblock{אתמר שתי חצירות וחורבה אחת ביניהם אחת עירבה ואחת לא עירבה אמר רב הונא נותנין אותה לזו שלא עירבה אבל לשעירבה לא דילמא אתי לאפוקי מאני דבתים לחורבה}
\textblock{וחייא בר רב אמר אף לשעירבה ושתיהן אסורות וא"ת שתיהן מותרות מפני מה אין נותנין חצר שלא עירבה לחצר שעירבה}
\textblock{התם כיון דמנטרי מאני דבתים בחצר אתי לאפוקי הכא בחורבה כיון דלא מנטרי מאני דחצר בחורבה לא אתי לאפוקי}
\textblock{איכא דאמרי חייא בר רב אמר אף לשעירבה ושתיהן מותרות ואם תאמר שתיהן אסורות לפי שאין נותנים חצר שלא עירבה לחצר שעירבה התם כיון דמנטרי מאני דבתים בחצר לא שרו בהו רבנן דאתי לאפוקי אבל בחורבה לא מנטרי:}
\textblock{{\large\emph{מתני׳}} גג גדול סמוך לקטן הגדול מותר והקטן אסור חצר גדולה שנפרצה לקטנה גדולה מותרת וקטנה אסורה מפני שהיא כפתחה של גדולה:}
\textblock{{\large\emph{גמ׳}} למה לי' למיתני תרתי}
\textblock{לרב קתני גג דומיא דחצר מה חצר מנכרא מחיצתא אף גג נמי מנכרא מחיצתא}
\textblock{ולשמואל גג דומיא דחצר מה חצר דקא דרסי לה רבים אף גג נמי דקא דרסי ליה רבים}
\textblock{יתיב רבה ורבי זירא ורבה בר רב חנן ויתיב אביי גבייהו ויתבי וקאמרי שמע מינה ממתניתין דיורי גדולה בקטנה ואין דיורי קטנה בגדולה}
\textblock{כיצד גפנים בגדולה אסור לזרוע את הקטנה ואם זרע זרעין אסורין}
\textblock{גפנים מותרין גפנים בקטנה מותר לזרוע את הגדולה}
\textblock{אשה בגדולה וגט בקטנה מתגרשת אשה בקטנה וגט בגדולה אינה מתגרשת}
\textblock{צבור בגדולה ושליח צבור בקטנה יוצאין ידי חובתן ציבור בקטנה ושליח ציבור בגדולה אין יוצאין ידי חובתן}
\textblock{תשעה בגדולה ויחיד בקטנה מצטרפין תשעה בקטנה ואחד בגדולה אין מצטרפין}
\textblock{צואה בגדולה אסור לקרות קריאת שמע בקטנה צואה בקטנה מותר לקרות קריאת שמע בגדולה}
\textblock{אמר להו אביי א"כ מצינו מחיצה לאיסור שאילמלי אין מחיצה מרחיק ד"א וזורע ואילו השתא אסורה}
\textblock{א"ל רבי זירא לאביי ולא מצינו מחיצה לאיסור והא תנן חצר גדולה שנפרצה לקטנה גדולה מותרת וקטנה אסורה מפני שהיא כפתחה של גדולה}
\textblock{ואילו השוה את גיפופיה גדולה נמי אסורה}
\textblock{א"ל התם סילוק מחיצות הוא}
\textblock{אמר ליה רבא לאביי ולא מצינו מחיצה לאיסור והא אתמר}
\newsection{דף צג}
\textblock{סיכך על גבי אכסדרה שיש לה פצימין כשירה ואילו השוה פצימיה פסולה}
\textblock{א"ל אביי לדידי כשירה לדידך סילוק מחיצות היא}
\textblock{א"ל רבה בר רב חנן לאביי ולא מצינו מחיצה לאיסור והתניא בית שחציו מקורה וחציו אינו מקורה גפנים כאן מותר לזרוע כאן}
\textblock{ואילו השוה את קרויו אסור א"ל התם סילוק מחיצות הוא}
\textblock{שלח ליה רבא לאביי ביד רב שמעיה בר זעירא ולא מצינו מחיצה לאיסור והתניא יש במחיצות הכרם להקל ולהחמיר כיצד כרם הנטוע עד עיקר מחיצה זורע מעיקר מחיצה ואילך שאילו אין שם מחיצה מרחיק ד"א וזורע וזה הוא מחיצות הכרם להקל}
\textblock{ולהחמיר כיצד היה משוך מן הכותל י"א אמה לא יביא זרע לשם שאילמלי אין מחיצה מרחיק ד"א וזורע וזוהי מחיצות הכרם להחמיר}
\textblock{א"ל וליטעמיך אותבן ממתניתין דתנן קרחת הכרם ב"ש אומרים כ"ד אמות וב"ה אומרים ט"ז אמה מחול הכרם ב"ש אומרים ט"ז אמה וב"ה אומרים י"ב אמה}
\textblock{ואיזו היא קרחת הכרם כרם שחרב אמצעיתו אם אין שם ט"ז אמה לא יביא זרע לשם היו שם ט"ז אמה נותן לו כדי עבודתו וזורע את המותר}
\textblock{אי זו היא מחול הכרם בין הכרם לגדר שאם אין שם י"ב אמה לא יביא זרע לשם היו שם י"ב אמה נותן לו כדי עבודתו וזורע את המותר}
\textblock{אלא התם לאו היינו טעמא דכל ד"א לגבי כרם עבודת הכרם לגבי גדר כיון דלא מזדרען אפקורי מפקר להו דביני ביני אי איכא ד' חשיבן ואי לא לא חשיבן}
\textblock{אמר רב יהודה ג' קרפיפות זה בצד זה ושנים החיצונים מגופפים והאמצעי אינו מגופף ויחיד בזה ויחיד בזה נעשה כשיירא ונותנין להן כל צורכן ודאי}
\textblock{אמצעי מגופף ושנים החיצונים אינן מגופפין ויחיד בזה ויחיד זה [ויחיד בזה] אין נותנין להם אלא בית שש}
\textblock{איבעיא להו אחד בזה ואחד בזה ושנים באמצעי מהו אי להכא נפקי תלתא הוו ואי להכא נפקי תלתא הוו}
\textblock{או דילמא חד להכא נפיק וחד להכא נפיק}
\textblock{ואם תימצי לומר חד להכא נפיק וחד להכא נפיק שנים בזה ושנים בזה ואחד באמצעי מהו הכא ודאי אי להכא נפיק תלתא הוו ואי להכא נפיק תלתא הוו או דילמא אימר להכא נפיק ואימר להכא נפיק}
\textblock{והלכתא בעיין לקולא}
\textblock{אמר רב חסדא}
\textblock{גידוד חמשה ומחיצה חמשה אין מצטרפין עד שיהא או כולו בגידוד או כולו במחיצה}
\textblock{מיתיבי שתי חצירות זו למעלה מזו ועליונה גבוהה מן התחתונה עשרה טפחים או שיש בה גידוד חמשה ומחיצה חמשה מערבין שנים ואין מערבין אחד פחות מכאן מערבין אחד ואין מערבין שנים}
\textblock{אמר (רב) מודה רב חסדא בתחתונה הואיל ורואה פני עשרה אי הכי תחתונה תערב שנים ולא תערב אחד עליונה לא תערב לא אחד ולא שנים}
\textblock{אמר רבה בר עולא כגון שהיתה עליונה מגופפת עד עשר אמות}
\textblock{אי הכי אימא סיפא פחות מכאן מערבין אחד ואין מערבין שנים אי בעיא חד תערב אי בעיא תרי תערב}
\textblock{אמר רבה בריה דרבא כגון שנפרצה התחתונה במלואה לעליונה}
\textblock{אי הכי תחתונה חד תערב תרי לא תערב עליונה אי בעיא תרי תערב אי בעיא חד תערב}
\textblock{אין הכי נמי וכי קתני פחות מכאן מערבין אחד ואין מערבין שנים אתחתונה}
\textblock{דרש מרימר גידוד חמשה ומחיצה חמשה מצטרפין אשכחיה רבינא לרב אחא בריה דרבא אמר ליה תני מר מידי במחיצה אמר ליה לא והלכתא גידוד חמשה ומחיצה חמשה מצטרפין}
\textblock{בעי רב הושעיא דיורין הבאין בשבת מהו שיאסרו}
\textblock{אמר רב חסדא תא שמע חצר גדולה שנפרצה לקטנה הגדולה מותרת והקטנה אסורה מפני שהיא כפתחה של גדולה אמר רבה אימר מבעוד יום נפרצה}
\textblock{אמר ליה אביי לא תימא מר אימר אלא ודאי מבעוד יום נפרצה דהא מר הוא דאמר בעי מיניה מרב הונא ובעי מיניה מרב יהודה עירב דרך הפתח ונסתם הפתח עירב דרך חלון ונסתם החלון מהו ואמר לי שבת כיון שהותרה הותרה}
\textblock{אתמר כותל שבין שתי חצירות שנפל רב אמר אין מטלטלין בו אלא בד"א}
\textblock{ושמואל אמר}
\newsection{דף צד}
\textblock{זה מטלטל עד עיקר מחיצה וזה מטלטל עד עיקר מחיצה}
\textblock{והא דרב לאו בפירוש אתמר אלא מכללא אתמר דרב ושמואל הוו יתבי בההוא חצר נפל גודא דביני ביני אמר להו שמואל שקולו גלימא נגידו בה}
\textblock{אהדרינהו רב לאפיה אמר להו שמואל אי קפיד אבא וקטרו בה}
\textblock{ולשמואל למה לי הא הא אמר זה מטלטל עד עיקר מחיצה וזה מטלטל עד עיקר מחיצה}
\textblock{שמואל עביד לצניעותא בעלמא}
\textblock{ורב אי סבירא ליה דאסיר לימא ליה אתריה דשמואל הוה}
\textblock{אי הכי מאי טעמא אהדרינהו לאפיה דלא נימרו כשמואל סבירא ליה (והדר ביה משמעתיה):}
\textblock{{\large\emph{מתני׳}} חצר שנפרצה לרשות הרבים המכניס מתוכה לרה"י או מרה"י לתוכה חייב דברי רבי אליעזר}
\textblock{וחכמים אומרים מתוכה לרשות הרבים או מרשות הרבים לתוכה פטור מפני שהיא ככרמלית:}
\textblock{{\large\emph{גמ׳}} ורבי אליעזר משום דנפרצה לרשות הרבים הויא לה רשות הרבים אין רבי אליעזר לטעמיה}
\textblock{דתניא רבי יהודה אומר משום רבי אליעזר רבים שבררו דרך לעצמן מה שבררו בררו}
\textblock{איני והאמר רב גידל אמר רב והוא שאבדה להן דרך באותו שדה}
\textblock{וכי תימא הכא נמי כגון שאבדה לה דרך באותה חצר והאמר רבי חנינא עד מקום מחיצה מחלוקת}
\textblock{אימא על מקום מחיצה מחלוקת}
\textblock{ואיבעית אימא בצידי רשות הרבים קמיפלגי דרבי אליעזר סבר צידי רשות הרבים כרשות הרבים דמו ורבנן סברי צידי רשות הרבים לאו כרשות הרבים דמו}
\textblock{וליפלוג בצידי רשות הרבים בעלמא אי איפליגו בצידי רשות הרבים בעלמא הוה אמרינן כי פליגי רבנן עליה דרבי אליעזר הני מילי היכא דאיכא חיפופי אבל היכא דליכא חיפופי אימא מודו ליה קא משמע לן}
\textblock{והא מתוכה קאמר}
\textblock{איידי דאמור רבנן מתוכה אמר איהו נמי מתוכה}
\textblock{ורבנן אמר רבי אליעזר צידי רשות הרבים ומהדרו ליה אינהו מתוכה}
\textblock{הכי קאמרי ליה רבנן לרבי אליעזר מי לא קא מודית לן היכא דטילטל מתוכה לרשות הרבים ומרשות הרבים לתוכה דפטור מפני שהיא כרמלית צידי נמי לא שנא}
\textblock{ורבי אליעזר התם לא קא דרסי לה רבים הכא קא דרסי לה רבים:}
\textblock{{\large\emph{מתני׳}} חצר שנפרצה לרה"ר משתי רוחותיה וכן בית שנפרץ משתי רוחותיו וכן מבוי שנטלו קורותיו או לחייו מותרים באותו שבת ואסורים לעתיד לבא דברי רבי יהודה}
\textblock{רבי יוסי אומר אם מותרין לאותו שבת מותרין לעתיד לבא ואם אסורין לעתיד לבא אסורין לאותו שבת:}
\textblock{{\large\emph{גמ׳}} במאי עסקינן אילימא בעשר מאי שנא מרוח אחת דאמר פיתחא הוא משתי רוחות נמי פיתחא הוא אלא ביתר מעשר א"ה אפילו מרוח אחת נמי}
\textblock{אמר רב לעולם בעשר}
\textblock{וכגון שנפרצה בקרן זוית דפיתחא בקרן זוית לא עבדי אינשי:}
\textblock{וכן בית שנפרץ משתי רוחותיו: מאי שנא מרוח אחת דאמרינן פי תקרה יורד וסותם משתי רוחות נמי לימא פי תקרה יורד וסותם}
\textblock{אמרי דבי רב משמיה דרב כגון שנפרץ בקרן זוית וקירויו באלכסון דליכא למימר פי תקרה יורד וסותם}
\textblock{ושמואל אמר אפילו ביתר מעשר אי הכי מרוח אחת נמי}
\textblock{משום בית}
\textblock{ובית גופיה תקשי מאי שנא מרוח אחת דאמרי' פי תקרה יורד וסותם מב' רוחות נמי נימא פי תקרה יורד וסותם}
\textblock{ותו מי אית ליה לשמואל פי תקרה יורד וסותם והא אתמר אכסדרה בבקעה רב אמר מותר לטלטל בכולה ושמואל אמר אין מטלטלין בה אלא בד' אמות}
\textblock{הא לא קשיא כי לית ליה בד' אבל בשלש אית ליה}
\textblock{מ"מ קשיא}
\textblock{כדאמרי בי רב משמיה דרב כגון שנפרץ בקרן זוית וקירויו באלכסון הכא נמי כגון שנפרץ בקרן זוית וקירויו בארבע}
\textblock{שמואל לא אמר כרב אלכסון לא קתני ורב לא אמר כשמואל אם כן הויא ליה אכסדרה ורב לטעמיה דאמר אכסדרה מותר לטלטל בכולה}
\textblock{דאיתמר אכסדרה בבקעה רב אמר מותר לטלטל בכולה ושמואל אמר אין מטלטלין בה אלא בארבע אמות}
\textblock{רב אמר מותר לטלטל בכולה אמרינן פי תקרה יורד וסותם ושמואל אמר אין מטלטלין בה אלא בארבע אמות לא אמרינן פי תקרה יורד וסותם}
\textblock{בעשר כולי עלמא לא פליגי כי פליגי ביתר מעשר}
\textblock{ואיכא דאמרי ביתר כולי עלמא לא פליגי כי פליגי בעשר}
\textblock{והא דאמר רב יהודה}
\newsection{דף צה}
\textblock{קורה ד' מתיר בחורבה ורב נחמן אמר רבה בר אבוה קורה ד' מתיר במים מני}
\textblock{להך לישנא דאמרת בעשר לא פליגי בעשר ודברי הכל להך לישנא דאמרת בעשר פליגי כרב}
\textblock{לימא אביי ורבא בפלוגתא דרב ושמואל קמיפלגי דאיתמר סיכך על גבי אכסדרה שיש לה פצימין כשירה אין לה פצימין אביי אמר כשירה ורבא אמר פסולה}
\textblock{אביי אמר כשירה אמר פי תקרה יורד וסותם ורבא אמר פסולה לא אמר פי תקרה יורד וסותם לימא אביי כרב ורבא כשמואל}
\textblock{אליבא דשמואל כולי עלמא לא פליגי כי פליגי אליבא דרב אביי כרב ורבא עד כאן לא קאמר רב התם אלא דהני מחיצות לאכסדרה עבידי אבל הכא דהני מחיצות לאו לסוכה עבידי לא:}
\textblock{רבי יוסי אומר אם מותרין: איבעיא להו רבי יוסי לאסור או להתיר}
\textblock{אמר רב ששת לאסור וכן אמר רבי יוחנן לאסור תניא נמי הכי אמר רבי יוסי כשם שאסורין לעתיד לבא כך אסורין לאותו שבת}
\textblock{איתמר רב חייא בר יוסף אמר הלכה כרבי יוסי ושמואל אמר הלכה כרבי יהודה}
\textblock{ומי אמר שמואל הכי והתנן א"ר יהודה בד"א בעירובי תחומין אבל בעירובי חצירות מערבין בין לדעת בין שלא לדעת לפי שזכין לאדם שלא בפניו ואין חבין שלא בפניו}
\textblock{ואמר רב יהודה אמר שמואל הלכה כרבי יהודה ולא עוד אלא כל מקום ששנה רבי יהודה בעירובין הלכה כמותו}
\textblock{וא"ל רב חנא בגדתאה לרב יהודה אמר שמואל אפילו במבוי שניטל קורתו או לחייו וא"ל בעירובין אמרתי לך ולא במחיצות}
\textblock{אמר רב ענן לדידי מיפרשא לי מיניה דשמואל כאן שנפרצה לכרמלית כאן שנפרצה לרשות הרבים:}
\textblock{{\large\emph{מתני׳}} הבונה עלייה על גבי שני בתים וכן גשרים המפולשים מטלטלין תחתיהן בשבת דברי רבי יהודה וחכמים אוסרין}
\textblock{ועוד א"ר יהודה מערבין למבוי המפולש וחכמים אוסרין:}
\textblock{{\large\emph{גמ׳}} אמר רבה לא תימא היינו טעמא דרבי יהודה משום דקא סבר ב' מחיצות דאורייתא אלא משום דקסבר פי תקרה יורד וסותם}
\textblock{איתיביה אביי יתר על כן א"ר יהודה מי שיש לו שני בתים משני צידי רה"ר עושה לחי מכאן ולחי מכאן או קורה מכאן וקורה מכאן ונושא ונותן באמצע אמרו לו אין מערבין רה"ר בכך}
\textblock{א"ל מההיא אין מהא ליכא למשמע מינה}
\textblock{אמר רב אשי מתניתין נמי דיקא מדקתני ועוד א"ר יהודה מערבין במבוי המפולש וחכמים אוסרין}
\textblock{אי אמרת בשלמא משום דקא סבר פי תקרה יורד וסותם היינו דקתני ועוד}
\textblock{אלא אי אמרת משום דקא סבר שתי מחיצות דאורייתא מאי ועוד שמע מינה:}
\textblock{\par \par {\large\emph{הדרן עלך כל גגות}}\par \par }
\textblock{}
\textblock{מתני׳ {\large\emph{המוצא}} תפילין מכניסן זוג זוג ר"ג אומר שנים שנים בד"א בישנות אבל בחדשות פטור}
\textblock{}
\textblock{מצאן צבתים או כריכות מחשיך עליהן ומביאן}
\newchap{פרק \hebrewnumeral{10}\quad המוצא תפילין}
\textblock{}
\textblock{ובסכנה מכסן והולך לו}
\textblock{רבי שמעון אומר נותנן לחבירו וחבירו לחבירו עד שמגיע לחצר החיצונה}
\textblock{וכן בנו נותנו לחבירו וחבירו לחבירו אפילו מאה רבי יהודה אומר נותן אדם חבית לחבירו וחבירו לחבירו אפילו חוץ לתחום אמרו לו לא תהלך זו יותר מרגלי בעליה:}
\textblock{{\large\emph{גמ׳}} זוג אחד אין טפי לא לימא תנן סתמא דלא כרבי מאיר}
\textblock{דאי כרבי מאיר האמר לובש כל מה שיכול ללבוש ועוטף כל מה שיכול לעטוף דתנן ולשם מוציא כל כלי תשמישו ולובש כל מה שיכול ללבוש ועוטף כל מה שיכול לעטוף}
\textblock{וההיא סתמא ממאי דרבי מאיר היא דקתני עלה לובש ומוציא ופושט ולובש ומוציא ופושט אפילו כל היום כולו דברי ר"מ}
\textblock{אמר רבא אפילו תימא רבי מאיר התם דרך מלבושו כחול שוויה רבנן והכא דרך מלבושו כחול שוויה רבנן}
\textblock{התם דבחול כמה דבעי לביש לענין הצלה נמי שרו ליה רבנן הכא דבחול נמי זוג אחד אין טפי לא לענין הצלה נמי זוג אחד אין טפי לא:}
\textblock{רבן גמליאל אומר שנים שנים: מאי קסבר אי קסבר שבת זמן תפילין הוא זוג אחד אין טפי לא}
\textblock{ואי קסבר שבת לאו זמן תפילין הוא ומשום הצלה דרך מלבוש שרו ליה רבנן אפילו טפי נמי}
\textblock{לעולם קסבר שבת לאו זמן תפילין הוא וכי שרו רבנן לענין הצלה דרך מלבוש במקום תפילין}
\textblock{אי הכי זוג אחד נמי אין טפי לא אמר רב שמואל בר רב יצחק מקום יש בראש להניח בו שתי תפילין}
\textblock{הניחא דראש דיד מאי איכא למימר}
\textblock{כדרב הונא דאמר רב הונא פעמים שאדם בא מן השדה וחבילתו על ראשו ומסלקן מראשו וקושרן בזרועו}
\textblock{אימר דאמר רב הונא שלא ינהג בהן דרך בזיון ראוי מי אמר}
\textblock{אלא כדאמר רב שמואל בר רב יצחק מקום יש בראש שראוי להניח בו שתי תפילין הכא נמי מקום יש ביד שראוי להניח בו שתי תפילין}
\textblock{תנא דבי מנשה (דברים ו, ח) על ידך זו קיבורת בין עיניך זו קדקד היכא אמרי דבי רבי ינאי מקום שמוחו של תינוק רופס}
\textblock{לימא בדרב שמואל בר רב יצחק קמיפלגי דת"ק לית ליה דרב שמואל בר רב יצחק ורבן גמליאל אית ליה דרב שמואל בר רב יצחק}
\textblock{לא דכולי עלמא אית להו דרב שמואל בר רב יצחק והכא בשבת זמן תפילין קמיפלגי דת"ק סבר שבת זמן תפילין הוא}
\textblock{ורבן גמליאל סבר שבת לאו זמן תפילין הוא}
\textblock{ואיבעית אימא דכ"ע שבת זמן תפילין הוא והכא במצות צריכות כוונה קמיפלגי ת"ק סבר לצאת בעי כוונה}
\textblock{ורבן גמליאל סבר לא בעי כוונה}
\newsection{דף צו}
\textblock{ואיבעית אימא דכ"ע לצאת לא בעי כוונה והכא לעבור משום בל תוסיף קמיפלגי דתנא קמא סבר לעבור משום בל תוסיף לא בעי כוונה ורבן גמליאל סבר לעבור משום בל תוסיף בעי כוונה}
\textblock{ואיבעית אימא אי דסבירא לן דשבת זמן תפילין דכ"ע לא לעבור בעי כוונה ולא לצאת בעי כוונה}
\textblock{והכא בלעבור שלא בזמנו קמיפלגי תנא קמא סבר לא בעי כוונה ורבן גמליאל סבר לעבור שלא בזמנו בעי כוונה}
\textblock{אי הכי לרבי מאיר זוג אחד נמי לא}
\textblock{ועוד הישן בשמיני בסוכה ילקה אלא מחוורתא כדשנינן מעיקרא}
\textblock{ומאן שמעת ליה שבת זמן תפילין ר' עקיבא דתניא (שמות יג, י) ושמרת את החקה הזאת למועדה מימים ימימה ימים ולא לילות מימים ולא כל ימים פרט לשבתות וימים טובים דברי רבי יוסי הגלילי}
\textblock{ר' עקיבא אומר לא נאמר חוקה זו אלא לענין פסח בלבד}
\textblock{ואלא הא דתנן הפסח והמילה מצות עשה לימא דלא כרבי עקיבא דאי ר"ע כיון דמוקי לה בפסח לאו נמי איכא כדרבי אבין א"ר אילעאי דאמר רבי אבין אמר רבי אילעאי כל מקום שנאמר השמר פן ואל אינו אלא בלא תעשה}
\textblock{אפילו תימא רבי עקיבא השמר דלאו לאו השמר דעשה עשה}
\textblock{וסבר רבי עקיבא שבת זמן תפילין הוא והתניא ר"ע אומר יכול יניח אדם תפילין בשבתות וימים טובים ת"ל (שמות יג, ט) והיה לך לאות על ידך מי שצריכין אות יצאו אלו שהן גופן אות}
\textblock{אלא האי תנא הוא דתניא הניעור בלילה רצה חולץ רצה מניח דברי רבי נתן יונתן הקיטוני אומר אין מניחין תפילין בלילה מדלילה לתנא קמא זמן תפילין שבת נמי זמן תפילין}
\textblock{דילמא ס"ל לילה זמן תפילין הוא שבת לאו זמן תפילין הוא דהא שמעינן ליה לרבי עקיבא דאמר לילה זמן תפילין הוא שבת לאו זמן תפילין הוא}
\textblock{אלא האי תנא הוא דתניא מיכל בת כושי היתה מנחת תפילין ולא מיחו בה חכמים ואשתו של יונה היתה עולה לרגל ולא מיחו בה חכמים מדלא מיחו בה חכמים אלמא קסברי מצות עשה שלא הזמן גרמא היא}
\textblock{ודילמא סבר לה}
\textblock{כרבי יוסי דאמר נשים סומכות רשות}
\textblock{דאי לא תימא הכי אשתו של יונה היתה עולה לרגל ולא מיחו בה מי איכא למ"ד רגל לאו מצות עשה שהזמן גרמא הוא אלא קסבר רשות הכא נמי רשות}
\textblock{אלא האי תנא היא דתניא המוצא תפילין מכניסן זוג זוג אחד האיש ואחד האשה אחד חדשות ואחד ישנות דברי ר"מ ר' יהודה אוסר בחדשות ומתיר בישנות}
\textblock{ע"כ לא פליגי אלא בחדשות וישנות אבל באשה לא פליגי שמע מינה מצות עשה שלא הזמן גרמא הוא וכל מצות עשה שאין הזמן גרמא נשים חייבות}
\textblock{ודילמא סבר לה כר' יוסי דאמר נשים סומכות רשות לא ס"ד דלא רבי מאיר סבר לה כרבי יוסי ולא רבי יהודה סבר לה כרבי יוסי}
\textblock{לא רבי מאיר סבר לה כרבי יוסי דתנן אין מעכבין את התינוקות מלתקוע הא נשים מעכבין וסתם מתני' רבי מאיר}
\textblock{ולא ר' יהודה סבר לה כר' יוסי דתניא (ויקרא א, ב) דבר אל בני ישראל וסמך בני ישראל סומכין ואין בנות ישראל סומכות רבי יוסי ור"ש אומרים נשים סומכות רשות}
\textblock{וסתם סיפרא מני ר' יהודה}
\textblock{א"ר אלעזר המוצא תכלת בשוק לשונות פסולות חוטין כשרין}
\textblock{מאי שנא לשונות דאמר אדעתא דגלימא צבעינהו חוטין נמי נימא אדעתא דגלימא טוינהי בשזורים}
\textblock{שזורים נמי נימא אדעתא דשיפתא דגלימא עייפינהו במופסקין דכולי האי ודאי לא טרחי אינשי}
\textblock{אמר רבא וכי אדם טורח לעשות קמיע כמין תפילין דתנן במה דברים אמורים בישנות אבל בחדשות פטור}
\textblock{א"ר זירא לאהבה בריה פוק תני להו המוצא תכלת בשוק לשונות פסולין חוטין מופסקין כשירין לפי שאין אדם טורח}
\textblock{אמר רבא ומשום דתני לה אהבה בריה דרבי זירא כיפי תלא לה והתנן במה דברים אמורים בישנות אבל בחדשות פטור}
\textblock{אלא אמר רבא טרח ולא טרח תנאי היא}
\textblock{דתניא המוצא תפילין מכניסן זוג זוג אחד האיש ואחד האשה}
\newsection{דף צז}
\textblock{אחד חדשות ואחד ישנות דברי רבי מאיר רבי יהודה אוסר בחדשות ומתיר בישנות אלמא מר סבר טרח איניש ומר סבר לא טרח איניש:}
\textblock{(שיצ"י עצב"י סימן): והשתא דתני אבוה דשמואל בר רב יצחק אלו הן ישנות כל שיש בהן רצועות ומקושרות חדשות יש בהן רצועות ולא מקושרות דכולי עלמא לא טרח איניש}
\textblock{וליענבינהו מיענב אמר רב חסדא זאת אומרת עניבה פסולה בתפילין}
\textblock{אביי אמר רבי יהודה לטעמיה דאמר עניבה קשירה מעלייתא היא}
\textblock{טעמא דעניבה קשירה מעלייתא היא הא לאו הכי עניב להו והאמר רב יהודה בריה דרב שמואל בר שילת משמיה דרב קשר של תפילין הלכה למשה מסיני הוא ואמר רב נחמן ונוייהן לבר}
\textblock{דעניב להו כעין קשירה דידהו}
\textblock{אמר רב חסדא אמר רב הלוקח תפילין ממי שאינו מומחה בודק שתים של יד ואחת של ראש או שתים של ראש ואחת של יד}
\textblock{מה נפשך אי מחד גברא קא זבין לבדוק או שלש של יד או שלש של ראש}
\textblock{אי מתרי תלתא גברי זבין כל חד וחד ליבעי בדיקה לעולם מחד גברא זבין ובעינן דמיתמחי בשל יד ובשל ראש}
\textblock{איני והא תני רבה בר שמואל בתפילין בודק שלש של יד ושל ראש מאי לאו או שלש של יד או שלש של ראש לא שלש מהן של יד מהן של ראש}
\textblock{והתני רב כהנא בתפילין בודק שתים של יד ושל ראש הא מני רבי היא דאמר בתרי זימני הוי חזקה}
\textblock{אי רבי אימא סיפא וכן בצבת השני וכן בצבת השלישי ואי רבי שלישי מי אית ליה}
\textblock{מודה רבי בצבתים דמתרי תלתא גברי זבין אי הכי אפילו רביעי נמי ואפילו חמישי נמי}
\textblock{אין הכי נמי והאי דקתני שלישי לאפוקי מחזקיה ולעולם אפילו רביעי וחמישי נמי:}
\textblock{מצאן צבתים או כריכות וכו': מאי צבתים ומאי כריכות אמר רב יהודה אמר רב הן הן צבתים הן הן כריכות צבתים זווי זווי כריכות דכריכן טובא:}
\textblock{מחשיך עליהן ומביאן: ואמאי לעיילינהו זוג זוג אמר רב יצחק בריה דרב יהודה לדידי מיפרשא ליה מיניה דאבא כל שאילו מכניסן זוג זוג וכלות קודם שקיעת החמה מכניסן זוג זוג ואי לא מחשיך עליהן ומביאן:}
\textblock{ובסכנה מכסן והולך: והתניא ובסכנה מוליכן פחות פחות מארבע אמות אמר רב לא קשיא הא בסכנת נכרי הא בסכנת לסטים}
\textblock{א"ל אביי במאי אוקימתא למתניתין בסכנת עו"ג אימא סיפא ר' שמעון אומר נותנן לחבירו וחבירו לחבירו כל שכן דאוושא מילתא}
\textblock{חסורי מיחסרא והכי קתני במה דברים אמורים בסכנת עו"ג אבל בסכנת ליסטים מוליכן פחות פחות מד' אמות:}
\textblock{רבי שמעון אומר נותנן לחבירו וכו': במאי קמיפלגי מר סבר פחות מארבע אמות עדיף דאי אמרת נותנן לחבירו וחבירו לחבירו אוושא מלתא דשבת}
\textblock{ומר סבר נותנן לחבירו עדיף דאי אמרת מוליכן פחות מארבע אמות זימנין דלאו אדעתיה ואתי לאתויינהו ארבע אמות ברה"ר:}
\textblock{וכן בנו: בנו מאי בעי התם דבי מנשה תנא בשילדתו אמו בשדה}
\textblock{ומאי אפילו הן מאה דאע"ג דקשיא ליה ידא אפילו הכי הא עדיפא:}
\textblock{רבי יהודה אומר נותן אדם חבית: ולית ליה לרבי יהודה הא דתנן הבהמה והכלים כרגלי הבעלים}
\textblock{אמר ריש לקיש משום לוי סבא הכא במאי עסקינן במערן מחבית לחבית ורבי יהודה לטעמיה דאמר מים אין בהם ממש}
\textblock{דתנן רבי יהודה פוטר במים מפני שאין בהן ממש}
\textblock{ומאי לא תהלך זו לא יהלך מה שבזו יותר מרגלי הבעלים}
\textblock{אימר דשמעת ליה לר' יהודה היכא דבליען בעיסה היכא דאיתנהו בעינייהו מי שמעת ליה השתא בקדירה אמר רבי יהודה לא בטלי בעינייהו בטלי דתניא רבי יהודה אומר מים ומלח בטלין בעיסה ואין בטלין בקדירה מפני רוטבה}
\textblock{אלא אמר רבא הכא בחבית שקנתה שביתה ומים שלא קנו שביתה עסקינן דבטלה חבית לגבי מים}
\textblock{כדתנן המוציא החי במטה פטור אף על המטה מפני שהמטה טפילה לו}
\textblock{המוציא אוכלין פחות מכשיעור בכלי פטור אף על הכלי מפני שהכלי טפל לו}
\textblock{מתיב רב יוסף ר' יהודה אומר בשיירא נותן אדם חבית לחבירו וחבירו לחבירו בשיירא אין שלא בשיירא לא אלא אמר רב יוסף כי תנן נמי במתניתין בשיירא תנן}
\textblock{אביי אמר בשיירא אפי' חבית שקנתה שביתה ומים שקנו שביתה שלא בשיירא חבית שקנתה שביתה ומים שלא קנו שביתה}
\textblock{רב אשי אמר הכא בחבית דהפקר עסקינן ומים דהפקר עסקינן ומאן אמרו לו רבי יוחנן בן נורי היא דאמר חפצי הפקר קונין שביתה ומאי לא תהלך זו יותר מרגלי הבעלים לא יהלכו אלו יותר מכלים שיש להם בעלים:}
\textblock{{\large\emph{מתני׳}} היה קורא בספר על האיסקופה ונתגלגל הספר מידו גוללו אצלו}
\textblock{היה קורא בראש הגג ונתגלגל הספר מידו עד שלא הגיע לעשרה טפחים גוללו אצלו משהגיע לעשרה טפחים הופכו על הכתב}
\textblock{רבי יהודה אומר אפילו אין מסולק מן הארץ אלא כמלא מחט גוללו אצלו ר' שמעון אומר אפילו בארץ עצמו גוללו אצלו שאין לך דבר משום שבות עומד בפני כתבי הקודש:}
\textblock{{\large\emph{גמ׳}} האי איסקופה ה"ד אילימא איסקופה רשות היחיד וקמה רשות הרבים ולא גזרינן דילמא נפיל ואתי לאתויי}
\newsection{דף צח}
\textblock{מני ר"ש היא דאמר כל דבר שהוא משום שבות אינו עומד בפני כתבי הקודש אימא סיפא רבי יהודה אומר אפילו אין מסולק מן הארץ אלא מלא החוט גוללו אצלו רבי שמעון אומר אפילו בארץ עצמה גוללו אצלו}
\textblock{רישא וסיפא רבי שמעון מציעתא רבי יהודה אמר רב יהודה אין רישא וסיפא ר"ש מציעתא רבי יהודה}
\textblock{רבה אמר הכא באיסקופה הנדרסת עסקינן ומשום בזיון כתבי הקדש שרו רבנן}
\textblock{איתיביה אביי תוך ד' אמות גוללו אצלו חוץ לד' הופכו על הכתב ואי אמרת באיסקופה נדרסת עסקינן מה לי תוך ד' אמות מה לי חוץ לארבע אמות}
\textblock{אלא אמר אביי הכא באיסקופה כרמלית עסקינן ורשות הרבים עוברת לפניה}
\textblock{תוך ד' אמות דאי נפיל ומייתי ליה לא אתי לידי חיוב חטאת שרו ליה רבנן}
\textblock{חוץ לארבע אמות דאי מייתי ליה אתי לידי חיוב חטאת לא שרו ליה רבנן}
\textblock{אי הכי תוך ד' אמות נמי נגזר דילמא מעייל מרה"ר לרה"י וכי תימא כיון דמפסקת כרמלית לית לן בה והאמר רבא המעביר חפץ מתחלת ארבע לסוף ארבע והעבירו דרך עליו חייב}
\textblock{הכא במאי עסקינן באיסקופה ארוכה אדהכי והכי מידכר}
\textblock{ואיבעית אימא לעולם באסקופה שאינה ארוכה וסתם כתבי הקדש עיוני מעיין בהו ומנח להו וליחוש דילמא מעיין בהו ברה"ר ועייל להו בהדיא לרה"י}
\textblock{הא מני בן עזאי היא דאמר מהלך כעומד דמי ודילמא זריק להו מזרק דאמר רבי יוחנן מודה בן עזאי בזורק}
\textblock{אמר רב אחא בר אהבה זאת אומרת אין מזרקין כתבי הקודש:}
\textblock{היה קורא בראש הגג וכו': ומי שרי והתניא כותבי ספרים תפילין ומזוזות לא התירו להן להפך יריעה על פניה אלא פורס עליה את הבגד}
\textblock{התם אפשר הכא לא אפשר ואי לא אפיך איכא בזיון כתבי הקודש טפי:}
\textblock{הופכו על הכתב והא לא נח [אמר רבא] בכותל משופע}
\textblock{[אמר ליה אביי] במאי אוקימתא למתניתין בכותל משופע אימא סיפא רבי יהודה אומר אפילו אינו מסולק מן הארץ אלא מלא החוט גוללו אצלו והא נח ליה}
\textblock{חסורי מיחסרא והכי קתני במה דברים אמורים בכותל משופע אבל}
\textblock{בכותל שאינו משופע למעלה משלשה גוללו אצלו למטה משלשה הופכו על הכתב:}
\textblock{רבי יהודה אומר אפילו אינו מסולק מן הארץ וכו': דבעינן הנחה על גבי משהו}
\textblock{ואלא הא דאמר רבא תוך שלשה לרבנן צריך הנחה לימא כתנאי אמרה לשמעתיה}
\textblock{אלא כולה רבי יהודה היא וחסורי מיחסרא והכי קתני במה דברים אמורים בכותל משופע אבל בכותל שאינו משופע אפילו פחות משלשה טפחים גוללו אצלו שרבי יהודה אומר אפילו אינו מסולק מן הארץ אלא מלא החוט גוללו אצלו}
\textblock{מאי טעמא דבעינן הנחה על גבי משהו:}
\textblock{{\large\emph{מתני׳}} זיז שלפני חלון נותנין עליו ונוטלין ממנו בשבת:}
\textblock{{\large\emph{גמ׳}} האי זיז דמפיק להיכא אילימא דמפיק לרשות הרבים ליחוש דילמא נפיל ואתי לאיתויי אלא דמפיק לרשות היחיד פשיטא}
\textblock{אמר אביי לעולם דמפיק לרשות הרבים ומאי נותנין עליו דקתני כלים הנשברים:}
\textblock{תניא נמי הכי זיז שלפני החלון היוצא לרשות הרבים נותנין עליו קערות וכוסות קיתוניות וצלוחיות}
\textblock{ומשתמש בכל הכותל עד עשרה התחתונים ואם יש זיז אחד למטה ממנו משתמש בו ובעליון אין משתמש בו אלא כנגד חלונו}
\textblock{האי זיז היכי דמי אי דלית ביה ארבעה מקום פטור הוא ואפילו כנגד חלונו נמי לא ישתמש ואי אית ביה ארבעה בכולי הכותל לישתמש}
\textblock{אמר אביי תחתון דאית ביה ארבעה ועליון לית ביה ארבעה וחלון משלימתו לד' כנגד חלון משתמש דחורי חלון הוא דהאי גיסא ודהאי גיסא אסור:}
\textblock{{\large\emph{מתני׳}} עומד אדם ברשות היחיד ומטלטל ברשות הרבים ברשות הרבים ומטלטל ברה"י ובלבד שלא יוציא חוץ מארבע אמות:}
\textblock{לא יעמוד אדם ברשות היחיד וישתין ברה"ר ברה"ר וישתין ברה"י וכן לא ירוק}
\textblock{רבי יהודה אומר אף משנתלש רוקו בפיו לא יהלך ארבע אמות עד שירוק:}
\textblock{{\large\emph{גמ׳}} מתני ליה רב חיננא בר שלמיא לחייא בר רב קמיה דרב לא יעמוד אדם ברה"י ומטלטל ברה"ר אמר ליה שבקת רבנן ועבדת כר"מ}
\newsection{דף צט}
\textblock{הוא סבר מדסיפא ר"מ רישא נמי ר"מ ולא היא סיפא ר"מ ורישא רבנן:}
\textblock{ובלבד שלא יוציא חוץ: הא הוציא חייב חטאת לימא מסייע ליה לרבא דאמר רבא המעביר חפץ מתחילת ארבע לסוף ארבע והעבירו דרך עליו חייב}
\textblock{מי קתני אם הוציא חייב חטאת דילמא אם הוציא פטור אבל אסור}
\textblock{איכא דאמרי הא הוציא פטור אבל אסור לימא תיהוי תיובתיה דרבא דאמר רבא המעביר מתחילת ד' לסוף ד' והעבירו דרך עליו חייב מי קתני הוציא פטור אבל אסור דילמא אם הוציא חייב חטאת:}
\textblock{לא יעמוד אדם ברה"י וכו': אמר רב יוסף השתין ורק חייב חטאת}
\textblock{והא בעינן עקירה והנחה מעל גבי מקום ד' וליכא}
\textblock{מחשבתו משויא ליה מקום דאי לא תימא הכי הא דאמר רבא זרק ונח בפי הכלב או בפי הכבשן חייב חטאת והא בעינן הנחה ע"ג מקום ד' וליכא}
\textblock{אלא מחשבתו משויא ליה מקום ה"נ מחשבה משויא לה מקום}
\textblock{בעי רבא הוא ברה"י ופי אמה ברה"ר מהו בתר עקירה אזלינן או בתר יציאה אזלינן תיקו:}
\textblock{וכן לא ירוק רבי יהודה אומר וכו': אף על גב דלא הפיך בה}
\textblock{והתנן היה אוכל דבילה בידים מסואבות והכניס ידו לתוך פיו ליטול צרור רבי מאיר מטמא ורבי יוסי מטהר}
\textblock{רבי יהודה אומר היפך בה טמא לא היפך בה טהור}
\textblock{אמר רבי יוחנן מוחלפת השיטה}
\textblock{ריש לקיש אמר לעולם לא תחליף והכא במאי עסקינן בכיחו}
\textblock{והתניא רבי יהודה אומר כיחו ונתלש מאי לאו רוק ונתלש לא כיחו ונתלש והא תניא ר' יהודה אומר כיחו שנתלש וכן רוקו שנתלש לא יהלך ד' אמות עד שירוק אלא מחוורתא כדשנינן מעיקרא:}
\textblock{אמר ר"ל כיח בפני רבו חייב מיתה שנאמר (משלי ח, לו) כל משנאי אהבו מות אל תקרי למשנאי אלא למשניאי}
\textblock{והא מינס אניס כיח ורק קאמרינן:}
\textblock{{\large\emph{מתני׳}} לא יעמוד אדם ברה"י וישתה ברה"ר ברה"ר וישתה ברה"י אלא אם כן הכניס ראשו ורובו למקום שהוא שותה וכן בגת:}
\textblock{{\large\emph{גמ׳}} רישא רבנן וסיפא רבי מאיר}
\textblock{אמר רב יוסף בחפיצין שצריכין לו ודברי הכל}
\textblock{איבעיא להו כרמלית מאי אמר אביי היא היא אמר רבא היא גופא גזירה ואנן ניקום ונגזור גזירה לגזירה}
\textblock{אמר אביי מנא אמינא לה מדקתני}
\textblock{וכן בגת}
\textblock{ורבא אמר לענין מעשר וכן אמר רב ששת וכן בגת לענין מעשר}
\textblock{דתנן שותין על הגת בין בחמין ובין בצונן ופטור דברי רבי מאיר רבי אליעזר בר צדוק מחייב}
\textblock{וחכ"א על החמין חייב ועל הצונן פטור מפני שמחזיר את המותר:}
\textblock{{\large\emph{מתני׳}} קולט אדם מן המזחילה למטה מי' טפחים ומן הצינור מ"מ שותה:}
\textblock{{\large\emph{גמ׳}} קולט אין אבל מצרף לא מ"ט אמר רב נחמן הכא במזחילה פחות מג' סמוך לגג עסקינן דכל פחות מג' סמוך לגג כגג דמי}
\textblock{תניא נמי הכי עומד אדם ברה"י ומגביה ידו למעלה מעשרה טפחים לפחות משלשה סמוך לגג וקולט ובלבד שלא יצרף}
\textblock{תניא אידך לא יעמוד אדם ברה"י ויגביה ידו למעלה מעשרה טפחים לפחות מג' סמוך לגג ויצרף אבל קולט הוא ושותה:}
\textblock{מן הצינור מ"מ שותה: תנא אם יש בצינור ד' על ד' אסור מפני שהוא כמוציא מרשות לרשות:}
\textblock{{\large\emph{מתני׳}} בור ברה"ר וחולייתו גבוה י' טפחים חלון שעל גביו ממלאין הימנו בשבת אשפה}
\textblock{ברה"ר גבוה י' טפחים חלון שעל גביו שופכין לתוכה מים בשבת:}
\textblock{{\large\emph{גמ׳}} במאי עסקינן אילימא בסמוכה למה לי חוליא י'}
\textblock{אמר רב הונא הכא במאי עסקינן במופלגת מן הכותל ארבעה}
\textblock{וטעמא דאיכא חוליא עשרה הא ליכא חוליא עשרה קא מטלטל מרה"י לרה"י דרך רה"ר}
\textblock{ורבי יוחנן אמר אפילו תימא בסמוכה הא קמ"ל דבור וחולייתו מצטרפין לעשרה:}
\textblock{אשפה ברה"ר וכו': ולא חיישינן שמא תנטל אשפה}
\textblock{והא רבין בר רב אדא אמר רבי יצחק מעשה במבוי אחד שצידו אחד כלה לים וצידו אחד כלה לאשפה ובא מעשה לפני רבי ולא אמר בו לא איסור ולא היתר}
\textblock{היתר לא אמר בו דחיישינן שמא תנטל אשפה ויעלה הים שירטון}
\textblock{איסור לא אמר בו דהא קיימין מחיצות}
\textblock{לא קשיא הא דיחיד הא דרבים:}
\textblock{{\large\emph{מתני׳}} אילן שהיה מיסך על הארץ אם אין נופו גבוה מן הארץ שלשה טפחים מטלטלים תחתיו}
\textblock{שרשיו גבוהים מן הארץ ג' טפחים לא ישב עליהן:}
\textblock{{\large\emph{גמ׳}} א"ר הונא בריה דרב יהושע אין מטלטלין בו יתר מבית סאתים מ"ט}
\newsection{דף ק}
\textblock{משום דהוי דירה שתשמישה לאויר וכל דירה שתשמישה לאויר אין מטלטלין בה יתר מבית סאתים:}
\textblock{שרשיו גבוהין מן הארץ וכו': איתמר שרשי אילן הבאין מלמעלה משלשה לתוך שלשה רבה אמר מותר להשתמש בהן רב ששת אמר אסור להשתמש בהן}
\textblock{רבה אמר מותר להשתמש בהן דכל פחות מג' דארעא ארעא היא רב ששת אמר אסור להשתמש בהן דכיון דמכח איסור קאתי אסורין}
\textblock{דדמו כמשוניתא דסלקין לעילא אסורין דנחתין לתתאי שרו לצדדין פלוגתא דרבה ורב ששת}
\textblock{וכן אניגרא וכן בקרן זוית}
\textblock{ההוא דיקלא דהוה לאביי והוה סליק באיפומא אתא לקמיה דרב יוסף ושרא ליה}
\textblock{אמר רב אחא בר תחליפא דשרא לך כרבה שרא לך}
\textblock{פשיטא מהו דתימא אפילו לרב ששת ביתא כמאן דמלי דמי ולישתמש בפחות מג' סמוך לגג קמ"ל}
\textblock{תנן שרשיו גבוהין מן הארץ ג' טפחים לא ישב עליהם היכי דמי אי דלא הדרי כיפי פשיטא אלא לאו אע"ג דהדרי כיפי}
\textblock{לא לעולם דלא הדרי כיפי והא קמ"ל אע"ג דצידו אחד שוה לארץ:}
\textblock{ת"ר שרשי אילן שגבוהין מן הארץ ג' טפחים או שיש חלל תחתיהן ג' טפחים אע"פ שצידו אחד שוה לארץ ה"ז לא ישב עליהן לפי שאין עולין באילן ואין נתלין באילן ואין נשענין באילן}
\textblock{ולא יעלה באילן מבעוד יום וישב שם כל היום כולו אחד אילן ואחד כל הבהמה אבל בור שיח ומערה וגדר מטפס ועולה מטפס ויורד ואפילו הן מאה אמה}
\textblock{תני חדא אם עלה מותר לירד ותני חדא אסור לירד לא קשיא כאן מבעוד יום כאן משחשיכה}
\textblock{ואיבעית אימא הא והא משחשיכה ול"ק כאן בשוגג כאן במזיד}
\textblock{ואיבעית אימא הא והא בשוגג והכא בקנסו שוגג אטו מזיד קמיפלגי מר סבר קנסינן ומר סבר לא קנסינן}
\textblock{אמר רב הונא בריה דרב יהושע כתנאי הניתנין במתנה אחת שנתערבו בניתנין מתנה אחת ינתנו במתנה אחת מתן ד' במתן ד' ינתנו במתן ד'}
\textblock{מתן ארבע במתן אחת ר"א אומר ינתנו במתן ד' ור' יהושע אומר ינתנו במתנה אחת}
\textblock{אמר לו ר"א הרי הוא עובר על בל תגרע אמר לו ר' יהושע הרי הוא עובר בבל תוסיף}
\textblock{א"ר אליעזר לא אמרו אלא כשהוא בעצמו אמר לו ר' יהושע לא נאמר בל תגרע אלא כשהוא בעצמו}
\textblock{ועוד א"ר יהושע כשנתת עברת על בל תוסיף ועשית מעשה בידך כשלא נתת עברת על בל תגרע ולא עשית מעשה בידך}
\textblock{לר"א דאמר התם קום עשה עדיף ה"נ ירד לר' יהושע דאמר התם שב ואל תעשה עדיף ה"נ לא ירד}
\textblock{דילמא לא היא ע"כ לא קאמר ר"א התם קום עשה עדיף אלא דקא עביד מצוה אבל הכא דלא עביד מצוה ה"נ לא ירד}
\textblock{ואי נמי ע"כ לא קאמר ר' יהושע התם שב ואל תעשה עדיף אלא}
\textblock{דלא קא עביד איסורא אבל הכא דקא עביד איסורא הכי נמי דירד}
\textblock{תני חדא אחד אילן לח ואחד אילן יבש ותניא אידך בד"א בלח אבל ביבש מותר}
\textblock{אמר רב יהודה ל"ק כאן בשגזעו מחליף כאן בשאין גזעו מחליף}
\textblock{גזעו מחליף יבש קרית ליה אלא לא קשיא כאן בימות החמה כאן בימות הגשמים}
\textblock{בימות החמה הא נתרי פירי בדליכא פירי והא קא נתרי קינסי בגדודא}
\textblock{איני והא רב איקלע לאפסטיא ואסר בגדודא רב בקעה מצא וגדר בה גדר:}
\textblock{אמר רמי בר אבא אמר רב אסי אסור לאדם שיהלך על גבי עשבים בשבת משום שנאמר (משלי יט, ב) ואץ ברגלים חוטא}
\textblock{תני חדא מותר לילך ע"ג עשבים בשבת ותניא אידך אסור ל"ק הא בלחים הא ביבשים}
\textblock{ואי בעית אימא הא והא בלחים ולא קשיא כאן בימות החמה כאן בימות הגשמים}
\textblock{ואיבעית אימא הא והא בימות החמה ול"ק הא דסיים מסאניה הא דלא סיים מסאניה}
\textblock{ואיבעית אימא הא והא דסיים מסאניה ול"ק הא דאית ליה עוקצי הא דלית ליה עוקצי}
\textblock{ואיבעית אימא הא והא דאית ליה עוקצי הא דאית ליה שרכא הא דלית ליה שרכא}
\textblock{והאידנא דקיימא לן כר"ש כולהו שרי:}
\textblock{ואמר רמי בר חמא אמר רב אסי אסור לאדם שיכוף אשתו לדבר מצוה שנאמר ואץ ברגלים חוטא}
\textblock{וא"ר יהושע בן לוי כל הכופה אשתו לדבר מצוה הווין לו בנים שאינן מהוגנין אמר רב איקא בר חיננא מאי קראה (משלי יט, ב) גם בלא דעת נפש לא טוב}
\textblock{תניא נמי הכי גם בלא דעת נפש לא טוב זה הכופה אשתו לדבר מצוה ואץ ברגלים חוטא זה הבועל ושונה}
\textblock{איני והאמר רבא הרוצה לעשות כל בניו זכרים יבעול וישנה ל"ק כאן לדעת כאן שלא לדעת:}
\textblock{א"ר שמואל בר נחמני א"ר יוחנן כל אשה שתובעת בעלה לדבר מצוה הווין לה בנים שאפילו בדורו של משה לא היו כמותן דאילו בדורו של משה כתיב (דברים א, יג) הבו לכם אנשים חכמים ונבונים וידועים לשבטיכם וכתיב ואקח את ראשי שבטיכם אנשים חכמים וידועים ואילו נבונים לא אשכח}
\textblock{ואילו גבי לאה כתיב (בראשית ל, טז) ותצא לאה לקראתו ותאמר אלי תבוא כי שכר שכרתיך וכתיב (דברי הימים א יב, לג) ומבני יששכר יודעי בינה לעתים לדעת מה יעשה ישראל ראשיהם מאתים וכל אחיהם על פיהם}
\textblock{איני והאמר רב יצחק בר אבדימי עשר קללות נתקללה חוה דכתיב}
\textblock{(בראשית ג, טז) אל האשה אמר הרבה ארבה אלו שני טפי דמים אחת דם נדה ואחת דם בתולים עצבונך זה צער גידול בנים והרונך זה צער העיבור בעצב תלדי בנים כמשמעו}
\textblock{ואל אישך תשוקתך מלמד שהאשה משתוקקת על בעלה בשעה שיוצא לדרך והוא ימשל בך מלמד שהאשה תובעת בלב והאיש תובע בפה זו היא מדה טובה בנשים}
\textblock{כי קאמרינן דמרציא ארצויי קמיה}
\textblock{הני שבע הווין כי אתא רב דימי אמר עטופה כאבל ומנודה מכל אדם וחבושה בבית האסורין}
\textblock{מאי מנודה מכל אדם אילימא משום דאסיר לה ייחוד איהו נמי אסיר ליה ייחוד אלא דאסירא לבי תרי}
\textblock{במתניתא תנא מגדלת שער כלילית ויושבת ומשתנת מים כבהמה ונעשית כר לבעלה}
\textblock{ואידך הני שבח הוא לה}
\textblock{דא"ר חייא מאי דכתיב (איוב לה, יא) מלפנו מבהמות ארץ ומעוף השמים יחכמנו מלפנו מבהמות זו פרידה שכורעת ומשתנת מים ומעוף השמים יחכמנו זה תרנגול שמפייס ואחר כך בועל}
\textblock{אמר רבי יוחנן אילמלא לא ניתנה תורה היינו למידין צניעות מחתול וגזל מנמלה ועריות מיונה דרך ארץ מתרנגול שמפייס ואחר כך בועל}
\textblock{ומאי מפייס לה אמר רב יהודה אמר רב הכי קאמר לה זביננא ליך זיגא דמטו ליך עד כרעיך לבתר הכי אמר לה לישמטתיה לכרבלתיה דההוא תרנגולא אי אית ליה ולא זביננא ליך:}
\newsection{דף קא}
\textblock{{\large\emph{מתני׳}} הדלת שבמוקצה וחדקים שבפרצה ומחצלות אין נועלין בהן אלא אם כן גבוהים מן הארץ:}
\textblock{{\large\emph{גמ׳}} ורמינהו דלת הנגררת ומחצלת הנגררת וקנקן הנגרר בזמן שקשורין ותלויין נועלין בהן בשבת ואין צריך לומר ביום טוב}
\textblock{אמר אביי בשיש להם ציר רבא אמר בשהיה להן ציר}
\textblock{מיתיבי דלת הנגררת ומחצלת הנגררת וקנקן הנגרר בזמן שקשורין ותלויין וגבוהים מן הארץ אפילו מלא נימא נועלין בהן ואם לאו אין נועלין בהן}
\textblock{אביי מתרץ לטעמיה ורבא מתרץ לטעמיה אביי מתרץ לטעמיה או שיש להן ציר או שגבוהין מן הארץ רבא מתרץ לטעמיה כשהיה להן ציר או שגבוהין מן הארץ}
\textblock{ת"ר סוכי קוצים וחבילין שהתקינן לפירצה שבחצר בזמן שקשורין ותלויין נועלין בהן בשבת וא"צ לומר ביו"ט}
\textblock{תני ר' חייא דלת אלמנה הנגררת אין נועלין בה היכי דמי דלת אלמנה איכא דאמרי דחד שיפא ואיכא דאמרי דלית ליה גשמה}
\textblock{אמר רב יהודה האי מדורתא ממעלה למטה שרי ממטה למעלה אסיר}
\textblock{וכן ביעתא וכן קידרא וכן פוריא וכן חביתא}
\textblock{א"ל ההוא צדוקי לרבי יהושע בן חנניה חדקאה דכתיב בכו (מיכה ז, ד) טובם כחדק אמר ליה שטיא שפיל לסיפיה דקרא דכתיב ישר ממסוכה ואלא מאי טובם כחדק כשם שחדקים הללו מגינין על הפירצה כך טובים שבנו מגינים עלינו דבר אחר טובם כחדק שמהדקין את הרשעים לגיהנם שנאמר (מיכה ד, יג) קומי ודושי בת ציון כי קרנך אשים ברזל ופרסותיך אשים נחושה והדיקות עמים רבים וגו':}
\textblock{{\large\emph{מתני׳}} לא יעמוד אדם ברשות היחיד ויפתח ברשות הרבים ברשות הרבים ויפתח ברשות היחיד אא"כ עשה מחיצה גבוה עשרה טפחים דברי ר' מאיר}
\textblock{אמרו לו מעשה בשוק של פטמים שהיה בירושלים שהיו נועלין ומניחין את המפתח בחלון שעל גבי הפתח רבי יוסי אומר שוק של צמרים הוה:}
\textblock{{\large\emph{גמ׳}} ורבנן אמר רבי מאיר רשות הרבים ומהדרו אינהו כרמלית דאמר רבה בר בר חנה אמר רבי יוחנן ירושלים אלמלא דלתותיה ננעלות בלילה חייבין עליה משום רשות הרבים}
\textblock{אמר רב פפא כאן קודם שנפרצו בה פרצות כאן לאחר שנפרצו בה פרצות}
\textblock{רבא אמר סיפא אתאן לשערי גינה והכי קאמר וכן לא יעמוד ברשות היחיד ויפתח בכרמלית בכרמלית ויפתח ברשות היחיד}
\textblock{אלא אם כן עשה מחיצה גבוהה עשרה טפחים דברי רבי מאיר אמרו לו מעשה בשוק של פטמים שהיה בירושלים שהיו נועלין ומניחין את המפתח בחלון שעל גבי הפתח רבי יוסי אומר שוק של צמרים היה:}
\textblock{תנו רבנן פתחי שערי גינה בזמן שיש להן בית שער מבפנים פותח ונועל מבפנים מבחוץ פותח ונועל מבחוץ מכאן ומכאן פותח ונועל כאן וכאן אין להן לא לכאן ולא לכאן אסורין כאן וכאן}
\textblock{וכן חנויות הפתוחות לרה"ר בזמן שהמנעול למטה מעשרה מביא מפתח מערב שבת ומניחו באיסקופה למחר פותח ונועל ומחזירו לאיסקופה}
\textblock{ובזמן שהמנעול למעלה מעשרה מביא מפתח מערב שבת ומניחו במנעול למחר פותח ונועל ומחזירו למקומו דברי רבי מאיר}
\textblock{וחכמים אומרים אף בזמן שהמנעול למעלה מעשרה טפחים מביא מפתח מערב שבת ומניחו באיסקופה למחר פותח ונועל ומחזירו למקומו או בחלון שעל גבי הפתח}
\textblock{אם יש בחלון ארבעה על ארבעה אסור מפני שהוא כמוציא מרשות לרשות}
\textblock{מדקאמר וכן חנויות מכלל דבאיסקופת כרמלית עסקינן האי מנעול היכי דמי אי דלית ביה ארבעה מקום פטור הוא}
\textblock{ואי אית ביה ארבעה בהא לימא רבנן אף בזמן שהמנעול למעלה מי' מביא מפתח מע"ש ומניחו באיסקופה למחר פותח ונועל בו ומחזירו לאיסקופה או לחלון שעל גבי הפתח והא קא מטלטל מכרמלית לרה"י}
\textblock{אמר אביי לעולם דאין בו ארבעה ויש בו לחוק ולהשלימו לד'}
\textblock{ובהא פליגי דר"מ סבר חוקקין להשלים ורבנן סברי אין חוקקין להשלים}
\textblock{אמר רב ביבי בר אביי ש"מ מהא מתניתא תלת ש"מ חוקקין להשלים וש"מ הדר ביה ר"מ משערי גינה}
\textblock{ושמעת מינה מדרבנן איתא לדרב דימי דכי אתא רב דימי א"ר יוחנן מקום שאין בו ארבעה על ארבעה מותר לבני רשות הרבים ולבני רשות היחיד לכתף עליו ובלבד שלא יחליפו:}
\textblock{{\large\emph{מתני׳}} נגר שיש בראשו גלוסטרא רבי אליעזר אוסר ורבי יוסי מתיר}
\textblock{אמר רבי אליעזר מעשה בכנסת שבטבריא שהיו נוהגין בו היתר עד שבא רבן גמליאל והזקנים ואסרו להן ר' יוסי אומר איסור נהגו בו ובא רבן גמליאל והזקנים והתירו להן:}
\textblock{{\large\emph{גמ׳}} בניטל באגדו כ"ע לא פליגי כי פליגי}
\newsection{דף קב}
\textblock{בשאינו ניטל באיגדו דמר סבר כיון דיש בראשו גלוסטרא תורת כלי עליו ומר סבר כיון דאינו ניטל באיגדו לא:}
\textblock{{\large\emph{מתני׳}} נגר הנגרר נועלין בו במקדש אבל לא במדינה}
\textblock{והמונח כאן וכאן אסור ר' יהודה אומר המונח מותר במקדש והנגרר במדינה:}
\textblock{{\large\emph{גמ׳}} תנו רבנן איזהו נגר הנגרר שנועלין במקדש אבל לא במדינה כל שקשור ותלוי וראשו אחד מגיע לארץ רבי יהודה אומר זה אף במדינה מותר אלא איזהו נגר שנועלין במקדש אבל לא במדינה כל שאינו לא קשור ולא תלוי ושומטו ומניחו בקרן זוית}
\textblock{אמר רב יהודה אמר שמואל הלכה כרבי יהודה בנגרר}
\textblock{אמר רבא והוא שקשור בדלת איני והא רבי טבלא איקלע למחוזא וחזא לההוא דהוה תלי בעיברא דדשא ולא אמר להו ולא מידי ההוא ניטל באיגדו הוה}
\textblock{רב אויא איקלע לנהרדעא חזייה לההוא גברא דהוה קא קטיר בגמי אמר דין לא נטרוק}
\textblock{בעי רבי זירא נקמז מהו אמר רב יוסף מאי תיבעי ליה לא שמיע ליה הא דתניא נשמט אסור נקמז מותר ורבי יהודה אמר נקמז אף על פי שאינו נשמט אסור}
\textblock{ואמר רב יהודה אמר שמואל הלכה כרבי יהודה בנקמז וטעמא מאי אמר אביי משום דמיחזי כבונה}
\textblock{בעא מיניה רב נחומי בר זכריה מאביי עשה לו בית יד מהו אמר ליה בוכנא קאמרת איתמר אמר רב נחומי בר אדא עשה לו בית יד מותר}
\textblock{ההוא שריתא דהוה בי רבי פדת דהוה מדלו לה בי עשרה ושדו לה אדשא ולא אמר להו ולא מידי אמר תורת כלי עליה}
\textblock{ההיא אסיתא דהות בי מר שמואל דהוה מחזקת אדריבא שרא מר שמואל למישדייה אדשא אמר תורת כלי עליה:}
\textblock{שלח ליה רמי בר יחזקאל לרב עמרם נימא לן מר מהלין מילי מעלייתא דאמרת לן משמיה דרב אסי בכיפי דארבא שלח ליה הכי אמר רב אסי הני כיפי דארבא בזמן שיש בהן טפח אי נמי אין בהן טפח ואין בין זה לזה שלשה למחר מביא מחצלת ופורס עליהן}
\textblock{מאי טעמא מוסיף על אהל עראי הוא ושפיר דמי}
\textblock{הנהו דכרי דהוו ליה לרב הונא דביממא בעו טולא ובליליא בעו אוירא אתא לקמיה דרב}
\textblock{א"ל זיל כרוך בודייא ושייר בה טפח למחר פשטה ומוסיף על אהל עראי הוא ושפיר דמי}
\textblock{אמר רב משום רבי חייא וילון מותר לנטותו ומותר לפורקו בשבת}
\textblock{כילת חתנים מותר לפורקה ולנטותה בשבת אמר רב ששת בריה דרב אידי לא אמרן אלא שאין בגגה טפח אבל יש בגגה טפח אסור}
\textblock{וכי אין בגגה טפח לא אמרן אלא שאין בפחות משלשה סמוך לגג טפח אבל יש בפחות משלשה סמוך לגג טפח אסור}
\textblock{וכי אין בפחות משלשה סמוך לגג טפח נמי לא אמרן אלא שאין}
\textblock{בשיפועה טפח אבל יש בשיפועה טפח שיפועי אהלים כאהלים דמו}
\textblock{ואמר רב שישא בריה דרב אידי סיינא שרי והתניא אסור לא קשיא הא דאית ביה טפח הא דלית ביה טפח}
\textblock{אלא מעתה שרביב בגלימיה טפח הכי נמי דאסור}
\textblock{אלא לא קשיא הא דמיהדק הא דלא מיהדק:}
\textblock{{\large\emph{מתני׳}} מחזירין ציר התחתון במקדש אבל לא במדינה והעליון כאן וכאן אסור רבי יהודה אומר העליון במקדש והתחתון במדינה:}
\textblock{{\large\emph{גמ׳}} תנו רבנן ציר דלת שידה תיבה ומגדל במקדש מחזירין במדינה דוחקין והעליון כאן וכאן לא יחזיר גזרה שמא יתקע ואם תקע חייב חטאת}
\textblock{של בור ושל דות ושל יציע לא יחזיר ואם החזיר חייב חטאת:}
\textblock{{\large\emph{מתני׳}} מחזירין רטיה במקדש אבל לא במדינה אם בתחילה כאן וכאן אסור:}
\textblock{{\large\emph{גמ׳}} תנו רבנן רטיה שפרשה מעל גבי מכה מחזירין בשבת ר' יהודה אומר הוחלקה למטה דוחקה למעלה למעלה דוחקה למטה ומגלה מקצת הרטיה ומקנח פי המכה וחוזר ומגלה מקצת רטיה ומקנח פי המכה}
\textblock{ורטיה עצמה לא יקנח מפני שהוא ממרח ואם מירח חייב חטאת}
\textblock{אמר רב יהודה אמר שמואל הלכה כר' יהודה אמר רב חסדא לא שנו אלא שפירשה על גבי כלי אבל פירשה על גבי קרקע דברי הכל אסור}
\textblock{אמר מר בר רב אשי הוה קאימנא קמיה אבא נפלה ליה אבי סדיא וקא מהדר ליה אמינא ליה לא סבר לה מר להא דאמר רב חסדא מחלוקת שפירשה על גבי כלי אבל פירשה על גבי קרקע אסור ואמר שמואל הלכה כרבי יהודה}
\textblock{אמר לי לא שמיע לי כלומר לא סבירא לי:}
\textblock{{\large\emph{מתני׳}} קושרין נימא במקדש אבל לא במדינה ואם בתחילה כאן וכאן אסור:}
\textblock{{\large\emph{גמ׳}} ורמינהו נימת כנור שנפסקה לא היה קושרה אלא עונבה לא קשיא הא רבנן והא ר' אליעזר}
\textblock{לרבי אליעזר דאמר מכשירי מצוה דוחין את השבת קושרה לרבנן דאמרי אין דוחין עונבה}
\textblock{אי רבי אליעזר אפילו לכתחילה נמי}
\textblock{אלא לא קשיא הא רבי יהודה והא רבנן}
\textblock{ורבי יהודה אליבא דמאן}
\newsection{דף קג}
\textblock{אי אליבא דר' אליעזר קאמר אפי' לכתחילה נמי}
\textblock{אלא לא קשיא הא ר"ש הא רבנן דתניא בן לוי שנפסקה לו נימא בכנור קושרה ר' שמעון אומר עונבה}
\textblock{רבי שמעון בן אלעזר אומר אף היא אינה משמעת את הקול אלא משלשל מלמטה וכורך מלמעלה או משלשל מלמעלה וכורך מלמטה}
\textblock{ואיבעית אימא הא והא רבנן ולא קשיא כאן באמצע כאן מן הצד}
\textblock{ואיבע"א הא והא באמצע מר סבר גזרינן ומר סבר לא גזרינן:}
\textblock{{\large\emph{מתני׳}} חותכין יבלת במקדש אבל לא במדינה ואם בכלי כאן וכאן אסור:}
\textblock{{\large\emph{גמ׳}} ורמינהו הרכיבו והבאתו מחוץ לתחום וחתיכת יבלתו אין דוחין רבי אליעזר אומר דוחין}
\textblock{רבי אלעזר ורבי יוסי) (בן) חנינא חד אמר הא והא בלחה ולא קשיא כאן ביד כאן בכלי}
\textblock{וחד אמר הא והא ביד ולא קשיא הא בלחה הא ביבישה}
\textblock{ולמאן דאמר הא ביד הא בכלי מאי טעמא לא אמר הא בלחה הא ביבישה אמר לך יבישה אפילו בכלי נמי שרי מ"ט איפרוכי איפרכא}
\textblock{ולמ"ד הא בלחה והא ביבישה מאי טעמא לא אמר הא ביד הא בכלי אמר לך בכלי הא תנן אם בכלי כאן וכאן אסור}
\textblock{ואידך הא דקתני התם משום דקא בעי איפלוגי ר' אליעזר ורבנן}
\textblock{ואידך דומיא דהרכיבו והבאתו מחוץ לתחום קתני דרבנן}
\textblock{ואידך הרכיבו דלא כרבי נתן דאמר החי נושא את עצמו הבאתו מחוץ לתחום כר"ע דאמר תחומין דאורייתא}
\textblock{מתיב רב יוסף אמר רבי אליעזר ק"ו ומה שחיטה שהיא משום מלאכה דוחה את השבת אלו שמשום שבות אינו דין שידחו את השבת}
\textblock{אלא אמר רב יוסף הא והא ביד ושבות מקדש במקדש התירו שבות דמקדש במדינה לא התירו}
\textblock{יתיב אביי וקאמר להא שמעתא איתיביה רב ספרא לאביי היה קורא בספר על האסקופה ונתגלגל הספר מידו גוללו אצלו והא הכא דשבות דמקדש במדינה הוא ולא גזרי' דילמא נפיל ואתי לאיתויי}
\textblock{ולא אוקימנא באסקופה כרמלית ורה"ר עוברת לפניה דכיון דאיגדו בידו אפילו שבות נמי ליכא}
\textblock{איתיביה משלשלין את הפסח לתנור עם חשיכה והא הכא דשבות דמקדש במדינה ולא גזרינן שמא יחתה בגחלים}
\textblock{אישתיק כי אתא לקמיה דרב יוסף א"ל הכי אמר לי רב ספרא אמר ליה מאי טעמא לא תשני ליה בני חבורה זריזין הן}
\textblock{ואביי כהנים זריזין הן אמרינן בני חבורה זריזין הן לא אמרינן}
\textblock{רבא אמר רבי אליעזר היא דאמר מכשירי מצוה דוחין את השבת ומודה ר' אליעזר דכמה דאפשר לשנויי משנינן}
\textblock{מאי היא דתניא כהן שעלתה בו יבלת חבירו חותכה לו בשיניו בשיניו אין בכלי לא חבירו אין איהו לא}
\textblock{מני אילימא רבנן ובמקדש כיון דאמרי רבנן בעלמא משום שבות הכא מה לי הוא מה לי חבירו}
\textblock{אלא לאו ר"א דאמר בעלמא חייב חטאת והכא אע"ג דמכשירי מצוה דוחין את השבת כמה דאפשר לשנויי משנינן}
\textblock{לא לעולם רבנן ואי עלתה בכריסו הכי נמי}
\textblock{הכא במאי עסקינן כגון שעלתה לו נשיכה בגבו ובאצילי ידיו דאיהו לא מצי שקיל לה}
\textblock{ואי רבנן נשקליה ניהליה ביד ותפשוט דרבי אלעזר דאמר רבי אלעזר מחלוקת ביד אבל בכלי דברי הכל חייב}
\textblock{וליטעמיך לרבי אליעזר נמי לישקליה ניהליה ביד האי מאי אי אמרת בשלמא רבי אליעזר היינו דגזר יד אטו כלי אלא אי אמרת רבנן היא נשקליה ניהליה ביד ותו לא מידי:}
\textblock{{\large\emph{מתני׳}} כהן שלקה באצבעו כורך עליה גמי במקדש אבל לא במדינה אם להוציא דם כאן וכאן אסור:}
\textblock{{\large\emph{גמ׳}} אמר רב יהודה בריה דרבי חייא לא שנו אלא גמי אבל צלצול קטן הוי יתור בגדים}
\textblock{ורבי יוחנן אמר לא אמרו יתור בגדים אלא במקום בגדים אבל שלא במקום בגדים לא הוי יתור בגדים}
\textblock{ותיפוק ליה משום חציצה בשמאל}
\textblock{אי נמי בימין ושלא במקום עבודה}
\textblock{ופליגא דרבא דאמר רבא אמר רב חסדא במקום בגדים אפילו נימא אחת חוצצת שלא במקום בגדים שלש על שלש חוצצות פחות משלש על שלש אינן חוצצות}
\textblock{אדרבי יוחנן ודאי פליגא אדרב יהודה בריה דרבי חייא מי נימא פליגא}
\textblock{שאני צלצול קטן דחשיב}
\textblock{לישנא אחרינא אמרי לה אמר רב יהודה בריה דרבי חייא לא שנו אלא גמי אבל צלצול קטן חוצץ}
\textblock{ורבי יוחנן אמר לא אמרו חציצה בפחות משלש על שלש אלא במקום בגדים אבל שלא במקום בגדים}
\newsection{דף קד}
\textblock{שלש על שלש חוצצות פחות משלש אין חוצצות והיינו דרבא אמר רב חסדא}
\textblock{לימא פליגא דרב יהודה בריה דרבי חייא שאני צלצול קטן דחשיב}
\textblock{ולרבי יוחנן אדאשמעינן גמי לישמעינן צלצול קטן}
\textblock{מילתא אגב אורחיה קמ"ל דגמי מסי:}
\textblock{{\large\emph{מתני׳}} בוזקין מלח על גבי כבש בשביל שלא יחליקו וממלאין מבור הגולה ומבור הגדול בגלגל בשבת ומבאר הקר ביום טוב:}
\textblock{{\large\emph{גמ׳}} רמי ליה רב איקא מפשרוניא לרבא תנן בוזקין מלח על גבי הכבש בשביל שלא יחליקו במקדש אין במדינה לא ורמינהי חצר שנתקלקלה במימי גשמים מביא תבן ומרדה בה}
\textblock{שאני תבן דלא מבטיל ליה}
\textblock{א"ל רב אחא בריה דרבא לרב אשי האי מלח ה"ד אי דמבטליה קא מוסיף אבנין (וכתיב (דברי הימים א כח, יט) הכל בכתב מיד ה' עלי השכיל)}
\textblock{ואי דלא קא מבטליה קא הויא חציצה}
\textblock{בהולכת אברים לכבש דלאו עבודה היא}
\textblock{ולא והא כתיב (ויקרא א, יג) והקריב הכהן את הכל והקטיר המזבחה ואמר מר זו הולכת אברים לכבש אלא אימא בהולכת עצים למערכה דלאו עבודה היא}
\textblock{דרש רבא חצר שנתקלקלה במימי גשמים מביא תבן ומרדה בה אמר ליה רב פפא לרבא והתניא כשהוא מרדה אינו מרדה לא בסל ולא בקופה אלא בשולי קופה}
\textblock{הדר אוקים רבא אמורא עליה ודרש דברים שאמרתי לפניכם טעות הן בידי ברם כך אמרו משום רבי אליעזר וכשהוא מרדה אינו מרדה לא בסל ולא בקופה אלא בשולי קופה:}
\textblock{ממלאין מבור הגולה: עולא איקלע לבי רב מנשה אתא ההוא גברא טרף אבבא אמר מאן האי ליתחל גופיה דקא מחיל ליה לשבתא}
\textblock{א"ל רבה לא אסרו אלא קול של שיר איתיביה אביי מעלין בדיופי ומטיפין מיארק לחולה בשבת}
\textblock{לחולה אין לבריא לא ה"ד לאו דנים וקא בעי דליתער ש"מ אולודי קלא אסיר}
\textblock{לא דתיר וקא בעי דלינים דמשתמע כי קלא דזמזומי}
\textblock{איתיביה המשמר פירותיו מפני העופות ודלעיו מפני החיה משמר כדרכו בשבת ובלבד שלא יספק ולא יטפח ולא ירקד כדרך שהן עושין בחול}
\textblock{מאי טעמא לאו דקמוליד קלא וכל אולודי קלא אסיר אמר רב אחא בר יעקב גזירה שמא יטול צרור}
\textblock{ואלא הא דאמר רב יהודה אמר רב נשים המשחקות באגוזים אסור מאי טעמא לאו דקא מוליד קלא וכל אולודי קלא אסיר}
\textblock{לא דלמא אתי לאשוויי גומות}
\textblock{דאי לא תימא הכי הא דאמר רב יהודה נשים משחקות בתפוחים אסור התם מאי אולודי קלא איכא אלא דילמא אתי לאשוויי גומות}
\textblock{תנן ממלאין מבור הגולה ומבור הגדול בגלגל בשבת במקדש אין במדינה לא מאי טעמא לאו משום דאולודי קלא ואסיר}
\textblock{לא גזירה שמא ימלא לגינתו ולחורבתו}
\textblock{אמימר שרא למימלא בגילגלא במחוזא אמר מאי טעמא גזרו רבנן שמא ימלא לגינתו ולחורבתו הכא לא גינה איכא ולא חורבה איכא}
\textblock{כיון דקא חזא דקא}
\textblock{תרו בה כיתנא אסר להו:}
\textblock{ומבאר הקר: מאי באר הקר אמר שמואל בור שהקרו עליה דברים והתירוה}
\textblock{מיתיבי לא כל הבורות הקרות התירו אלא זו בלבד ואי אמרת שהקרו דברים עליה מאי זו בלבד}
\textblock{אלא אמר רב נחמן בר יצחק באר מים חיים שנאמר (ירמיהו ו, ז) כהקיר ביר מימיה וגו'}
\textblock{גופא לא כל הבורות הקרות התירו אלא זו בלבד וכשעלו בני הגולה חנו עליה ונביאים שביניהן התירו להן ולא נביאים שביניהן אלא מנהג אבותם בידיהם:}
\textblock{{\large\emph{מתני׳}} שרץ שנמצא במקדש כהן מוציאו בהמיינו שלא לשהות את הטומאה דברי רבי יוחנן בן ברוקה רבי יהודה אומר בצבת של עץ שלא לרבות את הטומאה}
\textblock{מהיכן מוציאין אותו מן ההיכל ומן האולם ומבין האולם ולמזבח דברי ר"ש בן ננס}
\textblock{רבי עקיבא אומר כל מקום שחייבין על זדונו כרת ועל שגגתו חטאת משם מוציאין אותו ושאר כל המקומות כופין עליו פסכתר}
\textblock{רבי שמעון אומר מקום שהתירו לך חכמים משלך נתנו לך שלא התירו לך אלא משום שבות:}
\textblock{{\large\emph{גמ׳}} אמר רב טבי בר קיסנא אמר שמואל המכניס טמא שרץ למקדש חייב שרץ עצמו פטור מאי טעמא אמר קרא (במדבר ה, ג) מזכר ועד נקבה תשלחו מי שיש לו טהרה במקוה יצא שרץ שאין לו טהרה}
\textblock{לימא מסייע ליה מזכר עד נקבה תשלחו פרט לכלי חרש דברי רבי יוסי הגלילי מאי טעמא לאו משום דלית ליה טהרה במקוה}
\textblock{לא מי שנעשה אב הטומאה יצא כלי חרס שאינו נעשה אב הטומאה}
\textblock{לימא כתנאי שרץ שנמצא במקדש כהן מוציאו בהמיינו שלא לשהות את הטומאה דברי רבי יוחנן בן ברוקה רבי יהודה אומר בצבת של עץ מוציאו שלא לרבות את הטומאה}
\textblock{מאי לאו בהא קא מיפלגי דמאן דאמר שלא לשהות קסבר המכניס שרץ למקדש חייב ומ"ד שלא לרבות קסבר המכניס שרץ למקדש פטור}
\textblock{לא דכולי עלמא חייב והכא בהא קא מיפלגי מר סבר שהויי טומאה עדיף ומר סבר אפושי טומאה עדיף}
\textblock{אלא כהני תנאי דתנן מהיכן מוציאין אותו כו'}
\textblock{מאי לאו בהא קא מיפלגי דמאן דאמר מעזרה לא קסבר המכניס שרץ למקדש פטור ומאן דאמר מכולה עזרה קסבר חייב}
\newsection{דף קה}
\textblock{אמר רבי יוחנן ושניהם מקרא אחד דרשו (דברי הימים ב כט, טז) ויבאו הכהנים לפנימה בית ה' לטהר ויוציאו את כל הטומאה אשר מצאו בהיכל ה' לחצר בית ה' ויקבלו הלוים להוציא לנחל קדרון חוצה}
\textblock{מר סבר מדאשתני בעזרה בלוים טומאה בעזרה ליכא}
\textblock{ומר סבר עד היכא דלא אפשר בלוים מפקי כהנים השתא דאפשר בלוים תו לא מטמאי כהנים}
\textblock{תנו רבנן הכל נכנסין בהיכל לבנות לתקן ולהוציא את הטומאה ומצוה בכהנים אם אין שם כהנים נכנסין לוים אין שם לוים נכנסין ישראלים ואידי ואידי טהורין אין טמאין לא}
\textblock{אמר רב הונא רב כהנא מסייע כהני דתני רב כהנא מתוך שנאמר (ויקרא כא, כג) אך אל הפרוכת לא יבא יכול לא יהו כהנים בעלי מומין נכנסין בין האולם ולמזבח לעשות ריקועי פחים}
\textblock{תלמוד לומר אך חלק מצוה בתמימים אין שם תמימים נכנסין בעלי מומין מצוה בטהורין אין שם טהורין נכנסין טמאין אידי ואידי כהנים אין ישראלים לא}
\textblock{איבעיא להו טמא ובעל מום איזו מהן נכנס רב חייא בר אשי אמר רב טמא נכנס דהא אישתרי בעבודת ציבור ר' אלעזר אומר בעל מום נכנס דהא אישתרי באכילת קדשים:}
\textblock{רבי שמעון אומר וכו': רבי שמעון היכא קאי התם קאי דתנן מי שהחשיך חוץ לתחום אפילו אמה אחת לא יכנס ר' שמעון אומר אפי' חמש עשרה אמה יכנס לפי שאין המשוחות ממצעין את המדות מפני הטועים}
\textblock{דקאמר תנא קמא לא יכנס ואמר ליה רבי שמעון יכנס:}
\textblock{שלא התירו לך אלא משום שבות: היכא קאי התם קאי דקאמר תנא קמא קושרה ואמר ליה רבי שמעון}
\textblock{עונבה עניבה דלא אתי לידי חיוב חטאת שרו ליה רבנן קשירה דאתי לידי חיוב חטאת לא שרו ליה רבנן:}
\textblock{\par \par {\large\emph{הדרן עלך המוצא תפילין וסליקא לה מסכת עירובין}}\par \par }

\end{document}
