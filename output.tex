\documentclass[12pt, openany]{book}
\usepackage[
paperheight=9in,
paperwidth=6in,
top=0.5in,
bottom=0.5in,
inner=0.7in,
outer=0.5in,
marginparsep=0.1in,
headsep=16pt
]{geometry}

\newcommand{\texttitle}{חפץ חיים}\usepackage{titlesec}
\renewcommand{\partname}[1]{}
\usepackage{resources/unnumberedtotoc}

\usepackage{fancyhdr}
\pagestyle{fancy}
\fancyhf{}
\fancyhead[LO,RE]{\thepage}
\fancyhead[CO]{\chapname}
\fancyhead[CE]{\texttitle}

\usepackage{paracol}
\usepackage{anyfontsize}
\usepackage{ragged2e}
\usepackage{polyglossia}
\usepackage{multicol}
\usepackage{hyperref}
\usepackage[marginal]{footmisc}
\usepackage[titles]{tocloft}
\usepackage{xifthen}
\usepackage{graphicx}
%twocolfootnote

\setdefaultlanguage{hebrew}
\setotherlanguage{english}
\usepackage{fontspec}
\setmainfont{Times New Roman}
\newfontfamily\englishfont{Times New Roman}
\setsansfont{Aharoni}

\newcommand{\sethebfont}{
\fontsize{11pt}{13.8pt} \selectfont
}

\newcommand{\LTRmark}{‎}

\newcommand{\hebeng}[2]{
	{\sethebfont #1}
	
	%\vspace{0.5\baselineskip}
	{\beginL\englishfont{\raggedright #2 \hfill} \LTRmark\endL}
	
	\vspace{\baselineskip}
}

\newcommand{\twocol}[1]{
	{\sethebfont \begin{multicols}{2}
			#1
	\end{multicols}}	
}

\newcommand{\textblock}[1]{
{\sethebfont #1\\}	
}

\setlength{\parskip}{6pt}
\setlength\parindent{0in}

\newcommand{\chapname}{}
\newcommand{\sectname}{}

\newcommand{\newchap}[1]{
	\addcontentsline{toc}{chapter}{#1}
	\renewcommand{\chapname}{#1}
		\begin{center}
			\textbf{%
\fontsize{16pt}{16pt}\selectfont
				#1}
		\end{center}
}

\let\footnoterule\relax
\setlength\premulticols{10\baselineskip}
\setlength{\columnsep}{0.25in}

\newcommand{\newsection}[1]{
	%\addcontentsline{toc}{section}{#1}
	\renewcommand{\sectname}{#1}	
	\vspace{-\baselineskip}
	\begin{center}
		\textbf{%
\fontsize{16pt}{16pt}\selectfont
			#1}
	\end{center}
	\vspace{-\baselineskip}
	\nopagebreak
}

\newcommand{\footnotecomment}[1]{
	\renewcommand\thefootnote{}
	\footnote{\textsf{#1}}}

\newcommand{\parencomment}[1]{\footnotesize (#1)}

\newcommand{\blockcomment}[2]{ 
\vspace{\baselineskip}
\newsection{#1}
\sethebfont	\textsf{#2}
\vspace{\baselineskip}}

\newcommand{\commenta}[1]{\footnotecomment{#1}\hspace{0em}}

\newcommand{\vsnum}[1]{(\hebrewnumeral{#1})\space}
\newcommand{\vsnumeng}[1]{(#1)\space}

\begin{document}
\frontmatter
\pagenumbering{roman}

\newcommand{\oneline}[1]{%
	\newdimen{\namewidth}%
	\setlength{\namewidth}{\widthof{#1}}%
	\ifthenelse{\lengthtest{\namewidth < \textwidth}}%
	{#1}% do nothing if shorter than text width
	{\resizebox{\textwidth}{!}{#1}}% scale down
}

\title{\oneline{\hspace*{0.5in}\texttitle\hspace*{0.5in}}}

\author{}

\date{}

\maketitle

\begin{minipage}[b][\textheight][b]{\textwidth}\englishfont\footnotesize
	\begin{english}
		\vfill
		The following book includes:
\begin{itemize}
\item[$\bullet$] Pesach Haggadah
\begin{itemize}
\item[$\bullet$] License: Public Domain
\item[$\bullet$] Source: \url{http://www.daat.ac.il/daat/shabat/pesach/hagada-2.htm}
\end{itemize}
\item[$\bullet$] Barukh Sheamar on Haggadah, Tel Aviv 1968
\begin{itemize}
\item[$\bullet$] License: CC-BY
\item[$\bullet$] Source: \url{http://merhav.nli.org.il/primo-explore/fulldisplay?docid=NNL_ALEPH001741667&context=L&vid=NLI&search_scope=Local&tab=default_tab&lang=iw_IL}
\end{itemize}
\item[$\bullet$] Enhanced edition, 2023
\begin{itemize}
\item[$\bullet$] License: CC-BY
\item[$\bullet$] Source: \url{https://www.sefaria.org/texts}
\end{itemize}
\item[$\bullet$] Sefaria Edition
\begin{itemize}
\item[$\bullet$] License: CC0
\item[$\bullet$] Source: \url{http://www.sefaria.org/sefaria-edition}
\end{itemize}
\end{itemize}
		It was retrieved from Sefaria on \today\space \texthebrew{(\Hebrewtoday)}.  It was typeset and formatted by Ktavi.
		\clearpage
		
	\end{english}
\end{minipage}

\titleformat{\chapter}[hang]{\huge\bfseries}{\thechapter.}{1em}{}
\titlespacing*{\chapter}{0pt}{-3em}{1.1\parskip}
\titlelabel{\thetitle\quad}
%\addtocontents{toc}{\protect\vspace{-\baselineskip}}
\addtocontents{toc}{\protect\begin{multicols}{2}}
%\vspace*{-5\baselineskip}
{\small \tableofcontents}


\clearpage
\mainmatter
\pagenumbering{arabic}

\newchap{קדש}
\hebeng{{\small מוזגים כוס ראשון. המצּות מכוסות. } }{{\small \textit{We pour the first cup. The matsot are covered}} }
\hebeng{\textbf{קַדֵּשׁ} }{Make Kiddush}
\hebeng{{\small בְּשַׁבָּת מַתְחִילִין } }{{\small \textit{On Shabbat, begin here:}} }
\hebeng{וַיְהִי עֶרֶב וַיְהִי בֹקֶר יוֹם הַשִּׁשִּׁי. וַיְכֻלּוּ הַשָּׁמַיִם וְהָאָרֶץ וְכָל־צְבָאָם. וַיְכַל אֱלֹהִים בַּיּוֹם הַשְּׁבִיעִי מְלַאכְתּוֹ אֲשֶׁר עָשָׂה וַיִּשְׁבֹּת בַּיּוֹם הַשְּׁבִיעִי מִכָּל מְלַאכְתּוֹ אֲשֶׁר עָשָׂה. וַיְבָרֵךְ אֱלֹהִים אֶת יוֹם הַשְּׁבִיעִי וַיְקַדֵּשׁ אוֹתוֹ כִּי בוֹ שָׁבַת מִכָּל־מְלַאכְתּוֹ אֲשֶׁר בָּרָא אֱלֹהִים לַעֲשׂוֹת.}{And there was evening and there was morning, the sixth day. And the heaven and the earth were finished, and all their host. And on the seventh day God finished His work which He had done; and He rested on the seventh day from all His work which He had done. And God blessed the seventh day, and sanctified it; because He rested on it from all of His work which God created in doing (Genesis 1:31-2:3).}
\hebeng{{\small בחול מתחילין: } }{{\small \textit{On weekdays, begin here:}} }
\hebeng{סַבְרִי מָרָנָן וְרַבָּנָן וְרַבּוֹתַי. בָּרוּךְ אַתָּה ה׳, אֱלֹהֵינוּ מֶלֶךְ הָעוֹלָם בּוֹרֵא פְּרִי הַגָּפֶן. }{Blessed are You, Lord our God, King of the universe, who creates the fruit of the vine.}
\hebeng{בָּרוּךְ אַתָּה ה׳, אֱלהֵינוּ מֶלֶךְ הָעוֹלָם אֲשֶׁר בָּחַר בָּנוּ מִכָּל־עָם וְרוֹמְמָנוּ מִכָּל־לָשׁוֹן וְקִדְּשָׁנוּ בְּמִצְוֹתָיו. וַתִּתֶּן לָנוּ ה׳ אֱלֹהֵינוּ בְּאַהֲבָה ({\small לשבת:} שַׁבָּתוֹת לִמְנוּחָה וּ) מוֹעֲדִים לְשִׂמְחָה, חַגִּים וּזְמַנִּים לְשָׂשוֹן, ({\small לשבת:} אֶת יוֹם הַשַּׁבָּת הַזֶּה וְ) אֶת יוֹם חַג הַמַּצּוֹת הַזֶּה זְמַן חֵרוּתֵנוּ, ({\small לשבת: } בְּאַהֲבָה) מִקְרָא קֹדֶשׁ זֵכֶר לִיצִיאַת מִצְרָיִם. כִּי בָנוּ בָחַרְתָּ וְאוֹתָנוּ קִדַּשְׁתָּ מִכָּל הָעַמִּים, ({\small לשבת:} וְשַׁבָּת) וּמוֹעֲדֵי קָדְשֶׁךָ ({\small לשבת:} בְּאַהֲבָה וּבְרָצוֹן) בְּשִׂמְחָה וּבְשָׂשוֹן הִנְחַלְתָּנוּ. }{Blessed are You, Lord our God, King of the universe, who has chosen us from all peoples and has raised us above all tongues and has sanctified us with His commandments. And You have given us, Lord our God, {[Sabbaths for rest]}, appointed times for happiness, holidays and special times for joy, {[this Sabbath day, and]} this Festival of Matsot, our season of freedom {[in love]} a holy convocation in memory of the Exodus from Egypt. For You have chosen us and sanctified us above all peoples. In Your gracious love, You granted us Your {[holy Sabbath, and]} special times for happiness and joy. }
\hebeng{בָּרוּךְ אַתָּה ה׳, מְקַדֵּשׁ ({\small לשבת:} הַשַׁבָּת וְ) יִשְׂרָאֵל וְהַזְּמַנִּים. }{Blessed are You, O Lord, who sanctifies {[the Sabbath,]} Israel, and the appointed times.}
\hebeng{{\small בּמוצאי שבת מוסיפים: } }{{\small \textit{On Saturday night add the following two paragraphs:}} }
\hebeng{בָּרוּךְ אַתָּה ה׳, אֱלֹהֵינוּ מֶלֶךְ הָעוֹלָם, בּוֹרֵא מְאוֹרֵי הָאֵשׁ. בָּרוּךְ אַתָּה ה׳, אֱלֹהֵינוּ מֶלֶךְ הָעוֹלָם הַמַּבְדִיל בֵּין קֹדֶשׁ לְחֹל, בֵּין אוֹר לְחשֶׁךְ, בֵּין יִשְׂרָאֵל לָעַמִּים, בֵּין יוֹם הַשְּׁבִיעִי לְשֵׁשֶׁת יְמֵי הַמַּעֲשֶׂה. בֵּין קְדֻשַּׁת שַׁבָּת לִקְדֻשַּׁת יוֹם טוֹב הִבְדַּלְתָּ, וְאֶת־יוֹם הַשְּׁבִיעִי מִשֵּׁשֶׁת יְמֵי הַמַּעֲשֶׂה קִדַּשְׁתָּ. הִבְדַּלְתָּ וְקִדַּשְׁתָּ אֶת־עַמְּךָ יִשְׂרָאֵל בִּקְדֻשָּׁתֶךָ. }{Blessed are You, Lord our God, King of the universe, who creates the light of the fire. Blessed are You, Lord our God, King of the universe, who distinguishes between the holy and the profane, between light and darkness, between Israel and the nations, between the seventh day and the six working days. You have distinguished between the holiness of the Sabbath and the holiness of the Festival, and You have sanctified the seventh day above the six working days. You have distinguished and sanctified Your people Israel with Your holiness. }
\hebeng{בָּרוּךְ אַתָּה ה׳, הַמַּבְדִיל בֵּין קֹדֶשׁ לְקֹדֶשׁ. }{Blessed are You, O Lord, who distinguishes between the holy and the holy.}
\hebeng{בָּרוּךְ אַתָּה ה׳, אֱלֹהֵינוּ מֶלֶךְ הָעוֹלָם, שֶׁהֶחֱיָנוּ וְקִיְּמָנוּ וְהִגִּיעָנוּ לַזְּמַן הַזֶּה. }{Blessed are You, Lord our God, King of the universe, who has granted us life and sustenance and permitted us to reach this season.}
\hebeng{{\small שותה בהסיבת שמאל ואינו מברך ברכה אחרונה.} }{{\small \textit{Drink while reclining to the left and do not recite a blessing after drinking.}} }
\newchap{ורחץ}
\hebeng{\textbf{וּרְחַץ} }{And Wash}
\hebeng{{\small נוטלים את הידים ואין מברכים ״עַל נְטִילַת יָדַיִּם״} }{{\small \textit{Wash your hands but do not say the blessing "on the washing of the hands."}} }
\newchap{כרפס}
\hebeng{\textbf{כַּרְפַּס} }{Greens.}
\hebeng{{\small לוקח מן הכרפס פחות מכזית – כדי שלא יתחייב בברכה אחרונה – טובל במי מלח, מברך ״בורא פרי האדמה״, ומכווין לפטור בברכה גם את המרור. אוכל בלא הסבה. } }{{\small Take from the greens less than a \textit{kazayit} - so that you will not need to say the blessing after eating it; dip it into the salt water; say the blessing "who creates the fruit of the earth;" and have in mind that this blessing will also be for the bitter herbs. Eat without reclining.} }
\hebeng{בָּרוּךְ אַתָּה ה׳, אֱלֹהֵינוּ מֶלֶךְ הָעוֹלָם, בּוֹרֵא פְּרִי הָאֲדָמָה. }{Blessed are you, Lord our God, King of the universe, who creates the fruit of the earth.}
\newchap{יחץ}
\hebeng{\textbf{יַחַץ} }{Break}
\hebeng{{\small חותך את המצה האמצעית לשתים, ומצפין את הנתח הגדול לאפיקומן} }{{\small Split the middle matsah in two, and conceal the larger piece to use it for the afikoman.} }
\newchap{מגיד}
\newsection{הא לחמא עניא}
\hebeng{\textbf{מַגִּיד} }{The Recitation {[of the exodus story]}}
\hebeng{{\small מגלה את המצות, מגביה את הקערה ואומר בקול רם: } }{{\small The leader uncovers the matsot, raises the Seder plate, and says out loud:} }
\hebeng{הָא לַחְמָא עַנְיָא דִּי אֲכָלוּ אַבְהָתָנָא בְאַרְעָא דְמִצְרָיִם. כָּל דִכְפִין יֵיתֵי וְיֵיכֹל, כָּל דִצְרִיךְ יֵיתֵי וְיִפְסַח. הָשַּׁתָּא הָכָא, לְשָׁנָה הַבָּאָה בְּאַרְעָא דְיִשְׂרָאֵל. הָשַּׁתָּא עַבְדֵי, לְשָׁנָה הַבָּאָה בְּנֵי חוֹרִין. }{This is the bread of destitution that our ancestors ate in the land of Egypt. Anyone who is famished should come and eat, anyone who is in need should come and partake of the Pesach sacrifice. Now we are here, next year we will be in the land of Israel; this year we are slaves, next year we will be free people.}%
\commenta{\textrm{\textbf{הא לחמא עניא די אכלו אבהתנא בארעא דמצרים}} הלשון בארעא דמצרים לאו דוקא במצרים ממש. שהרי זה אכלו בצאתם משם, והסבה לזה, כי במצרים לא הספיק בצקם להחמיץ, וגם כי במצרים לא אכלו כלל לחם עוני, שהרי גם דגים ובשר הי׳ להם למדי, כמבואר בפ׳ בהעלתך, ומכש״כ לחם, אך הכונה בארעא דמצרים — סמוך למצרים בצאתם משם.\textrm{\textbf{כל דכפין ייתי וייכול, כל דצריך ייתי ויפסח}} לכאורה שני אלה המאמרים הם רק כפל לשון בענין אחד, שמי שאין לו מה לאכול יבא ויאכל. אך באמת אפשר לומר שהן שתי קריאות משני ענינים, האחד — כל דכפין, כלומר, הרעב יבא ויאכל, והשני — אף כי לא רעב, אך חסרים לו צרכי פסח (צרכי יו״ט) כמו יין ובשר ועוד, יבא ויעשה פסח, כלומר, ימצא הכל לפניו, והלשון ויפסח הוא כמו ויחוג חג הפסח, ומן השם פסח נעשה פעל יפסח. ומצאתי און לי בבאור זה בנוסח ההגדה ברמב״ם סוף הלכות חמץ מצה, כי הגירסא שם כל דצריך לפסח, וזה קרוב לבאורנו. ואמנם זה ודאי, שאין הכונה במלת ויפסח מענין קרבן פסח, שהרי שוברו בצדו, כי סוף הלשון השתא הכא (בחוץ לארץ), ובחוץ לארץ אין קרבן פסח. ודע דיש הבדל בין הלשון כל דכפין ובין כל דצריך, כי הכפן שענינו רעב, כל מי שיאמר כי רעב הוא ייתי ויכול. ואין בודקין אם אמנם רעב הוא ואין לו מה לאכול יען דקי״ל אין בודקין למזונות. ולא כן הלשון כל דצריך שענינו כל מי שבא לקבל צרכי פסח, יין ובשר ועוד עליו דרוש להראות כי אמנם דרוש הוא לקבל ואין לו משלו. וטעם הדבר הוא עפ״י המבואר במס׳ תענית דף כ׳ סע״ב, במנהגו של רב הונא, שבכל ערב שבת הי׳ קונה כל הירק הנשאר ממכירת היום ומשליכם לנהר ולא נתנם לעניים, ומפרש הטעם, כדי שלא יסמכו עניים על זה ולא יכינו לשבת, ואפשר לצרף טעם זה גם לכאן. ויש להעיר, על שלא קבעו קריאה זו קודם קידוש, שהרי אם מזמנים לאורח לסעודה דרוש שיהיה גם בשעת קידוש. ואפשר לומר עפ״י מה שאמרו בפסחים (ק״ו א׳) על הפסוק זכור את יום השבת (פ' יתרו) זכרהו בכניסתו והיינו מיד כשנכנס שבת חובה להזכירו, ומבואר שם, דהזכירה היא בקידוש על היין. והנה גם בפסח כתיב (פ׳ בא, י״ג ב׳) זכור את היום הזה אשר יצאתם ממצרים, ואם כן גם בפסח מיד כשנכנס החג, חלה חובת זכירה בקידוש על היין, ויותר מזה בפסח חלה עוד מיד בכניסתו חובת ספור יציאת מצרים, וכן אמרו (ברכות נ״ב א׳) עיולי יומי כל כמה דמקדמינן עדיף. וכבר כתבנו בתחלת הפתיחה דכל מצוה התלוי׳ ביום גם הלילה בכלל, עיי״ש, וא״כ מיד כשנכנס הלילה חלה חובת קידוש, ומתבאר, דאין לדחות הקידוש עד אשר יבאו עניים, שאין לזה שעה מוגבלת, וגם אולי לא יבאו כלל, אך צריך לקדש מיד, ואם יבאו עניים יקדשו לעצמן.\textrm{\textbf{כל דכפין יתי ויכול... השתא הכא, לשנה הבאה בארעא דישראל, השתא עבדי, לשנה הבאה בני חורין.}} לא נתבאר יחש המאמרים כל דכפין ייתי וייכול וכו׳ לזה דהשתא הכא לשנה הבאה בארעא דישראל, השתא עבדי, לשנה הבאה בני חורין. ואפשר לפרש דמכוין למה שאמרו במס׳ ברכות (ל״ד ב׳) ועוד, אין בין עולם הזה לימות המשיח אלא שעבוד מלכיות בלבד, שנאמר (פ׳ ראה) כי לא יחדל אביון מקרב הארץ, ע״כ. וכלומר, כי העולם כמנהגו ינהוג, יהיו עשירים ויהיו עניים, אך ההבדל יהי׳ בזה כי לא יהי׳ עוד שעבוד מלכיות. וזהו שאמר כאן, כל דכפין ייתי וייכול כל דצריך ייתי ויפסח, השתא הכא לשנה הבאה בארעא דישראל, כלומר, גם בארעא דישראל כל דכפין ייתי וייכול, יען כי לא יחדל אביון מקרב הארץ, ואם תאמר, אם כן, מה ההבדל בין האידנא בגלות ובין ימות המשיח (לעת שנהי׳ בארץ ישראל) — ההבדל בזה הוא רק כי השתא עבדי (בשעבוד מלכיות), לשנה הבאה נהי׳ בני חורין (משעבוד מלכיות). וגם אפשר לפרש בפשיטות יחש המאמרים מן כל דכפין להשתא הכא ולשנה הבאה בארעא דישראל עפ״י מה שאמרו בב״ב (י׳ א׳) גדולה צדקה שמקרבת את הגאולה, ויליף זה מקרא דישעיה (נ״ו א׳) שמרו משפט ועשו צדקה כי קרובה ישועתי לבא, ומכוין לומר, כי עבור הצדקה שמזמנים לאכול כל דכפין (שענינו כל הרעב) נזכה לשנה הבאה להיות בא״י ולגאולת בני חורין. ולכאורה יש להעיר בהמאמר הנזכר, כי הוא מסיים במה שלא פתח, כי פתח בצדקה לבד, גדולה צדקה ומביא ראי׳ מן שמרו משפט ועשו צדקה כי קרובה ישועתי לבא, הרי דצריך לזה שתי זכיות, משפט וצדקה. אך הבאור הוא, כי באמת ענין משפט כשהוא לבדו לא יחשב למעלה, יען כי הפכו עול וגזל, ולא כן צדקה שבמניעתה הוא רק מניעת מצוה, ונמצא שמעלת הצדקה דיה גם לבדה, אך בתנאי שתהי׳ במשפט, כלומר, לא בהעדר משפט, והיינו שלא יתן צדקה ברכוש גזל ועול.}%endcomment%
\commentb{\textrm{\textbf{תשובות לשערים א'-ד'.}}\textrm{\textbf{"הא לחמא עניא די אכלא אבהתנא" וכו'.}}ההגדה הזאת היא במכילתא, וכבר זכרתי בשערים השאלה שיש במאמר הראשון, למה סידרו חז"ל המאמר הזה בלשון ארמי ושאר מאמרי ההגדה כולם בלשון הקודש? ונתנו החכמים ז"ל טעמים על דרך הדרוש, מהם אמרו מפני שלא יבינו המזיקים בלשון הקודש ויבואו ללכלך הסעודה אחר שניתן להם רשות באומרו "כל דכפין ייתי ויכול".אבל הדעה הזאת היא ספק ספיקא בכמה מדרגות. ראשונה במציאות השדים, ושנית שהם יבינו לשון הקודש ולא לשון ארמי, ושלישית אף אם נודה במציאות השדים שהוא קשה ומשא כבד להאמין, ושלא ילכלכו הסעודה כי אם ברשות בעל הבית כבן יכבד אב ועבד אדוניו, הנה כבר השרישו בנו חז"ל האמונה ששלוחי מצווה אינם ניזוקין (קידושין ל"ט, ב'), ואמרו ליל שימורים הוא לילה המשומר מן המזיקין.ויש אומרים שנתחבר המאמר הזה בלשון ארמי כדי שלא יבינו אותו מלאכי השרת ולא יתחילו לשורר להקב"ה בלילה הזה שהוא ליל יציאת מצרים שלא אמרו בו שירה שנאמר "ולא קרב זה אל זה כל הלילה", אך קשה גם לדעתם שהיה ראוי שיאמרו כן בהלל שהוא השיר המיוחס ללילה הזה ולא המאמר הראשון מההגדה שאין בו שירה. ועם היות שבליל יציאת מצרים לא אמרו שירה מפני שנטבעו המצריים בים, אבל בלילי פסח הבאים הם אומרים שירה ואין לחוש שיקנאו בישראל ומדוע לא יאמרו "הא לחמא עניא" בלשון הקודש שיבינו המלאכים? ובפרט שכבר אמרו חז"ל (סוטה ל"ג, א') כי מלאכי השרת שומעים כל הלשונות של שבעים אומות כנגד שבעים שרים המקיפים כסא כבוד ה', וכי המלאך גבריאל למד את יוסף שבעים לשון, ולפי זה יבינו גם לשון ארמי.עוד אחרים אמרו בטעם שמפני שהיו אבותינו בבבל כאשר סדרו ההגדה, הסכימו לומר המאמר הזה בלשון ארמי כדי שיבינו הנשים והטף והם יפרסמו  את המצווה, אבל קשה לדעה הזאת למה סידרו המאמר הראשון דווקא בלשון ארמי ושאר המאמרים בהגדה כולם בלשון הקודש, ובפרט מאמר "מה נשתנה" שהוא יותר כנגד הנשים והטף היה ראוי להיות בלשון ארמי כדי שיבינו אותו.והיותר נראה בזה הוא כי מפני ההכרזה אשר בזה המאמר "כל דכפין ייתי ויכול כל דצריך ייתי ויפסח" תקנו אותו בלשון ארמי כי היו אז בבבל, ובבוא החג עשו זכר לחגיגת הפסח בירושלים לסמוך על שולחנם העניים והאביונים, וסדרו שגם פה בגלות יתחייב כל בעל הבית להרבות מתנתו, כמו שנאמר "וּשְׂמַחְתֶּם לִפְנֵי ה' אֱלֹהֵיכֶם אַתֶּם וּבְנֵיכֶם וּבְנֹתֵיכֶם וְעַבְדֵיכֶם וְאַמְהֹתֵיכֶם וְהַלֵּוִי" וגו' (דברים י"ב, י"ב). ולכן כשישב על שולחנו ירים קולו אל העניים אשר בפתח הבית לקרוא לכולם בשם ה' "כל דכפין ייתי ויכול כל דצריך ייתי ויפסח" רצה לומר שיבואו לסמוך על שולחנו. ולפי שהעניים לא יבינו לשון הקודש תקנו שיעשה ההכרזה הזאת בלשון ארמי כדי שיבינו אותו ויכנסו לביתו. וכבר ציווה הנביא על הצדקה הזאת באמרו "הֲלוֹא פָרֹס לָרָעֵב לַחְמֶךָ וַעֲנִיִּים מְרוּדִים תָּבִיא בָיִת" וגו' "אָז יִבָּקַע כַּשַּׁחַר אוֹרֶךָ וַאֲרֻכָתְךָ מְהֵרָה תִצְמָח וְהָלַךְ לְפָנֶיךָ צִדְקֶךָ כְּבוֹד ה' יַאַסְפֶךָ" (ישעיהו נ"ח ז'-ח').אמנם למה נקראה המצה לחם עוני? כבר נמצאו לחכמינו ששה טעמים חדשים וגם ישנים,  כי בפרק "ערבי פסחים" אמר שמואל "לחם עוני שעונין עליו דברים", אבל זה בלתי מספיק, כי גם על הפסח והמרור יש עניית דברים, וכמו שאמר רבן גמליאל "כל מי שלא אמר שלושה דברים אלו בפסח לא יצא ידי חובתו פסח מצה ומרור" וכו', ועוד שהרי הכתוב קרא את המצה לחם עוני ולא היה זה מפני ההגדה כי עדיין לא נתקנה.עוד אמרו שם טעם אחר לחם עוני, "מה דרכו של עני בפרוסה אף כאן בפרוסה" (פסחים דף קט"ו עמוד ב'), כלומר שבלילה הזה יפרוס כל בעל הבית את המצה לזכרון שהיו ישראל במצרים עניים, אבל קשה שהתורה עשתה המצה זכרון לגאולה כמו שנאמר "שבעת ימים תאכל עליו מצות לחם עוני כי בחפזון יצאת מארץ מצרים למען תזכור את יום צאתך" וגו', ומזה משתמע כי המצה היא זכר לגאולה לא לעבדות ולא לעניות.הטעם השלישי שאמרו שנקרא לחם עוני לפני שמפני עבדותם במצרים לא היו מניחים אותם המצריים ללוש עיסתם ולאוכלה כראוי, בטל גם כן על פי שאמרנו כי לא באה המצה לזכרון הגלות והעבדות כי אם לזכרון הגאולה, והיה ראוי לקרותו לחם גאולה.הטעם הרביעי שנקרא לחם עוני לפי שבמצרים אכלו בני ישראל מצה עם קורבן פסח כמו שנאמר "על מצות ומרורים יאכלוהו", ולכך נאמר "די אכלו אבהתנא בארעא דמצרים" וקאי על אכילת אותו הלילה. אך גם זה אינו שווה לי, כי היה ראוי לומר "די אכלו אבהתנא ביציאתם ממצרים" ולא "בארעא דמצרים" שמורה על ההרגל וההתמדה כל זמן ישיבתם שם.הטעם החמישי אמרו שקראו לחמא עניא מפני העיסה שתקנו שתהיה מעשירית האיפה כמו קורבן העני, שנאמר "וְאִם לֹא תַשִּׂיג יָדוֹ לִשְׁתֵּי תֹרִים אוֹ לִשְׁנֵי בְנֵי יוֹנָה וְהֵבִיא אֶת קָרְבָּנוֹ אֲשֶׁר חָטָא עֲשִׂירִת הָאֵפָה" (ויקרא ה', י"א), וכן הייתה עיסת המן במדבר שהיה כל אחד מלקט ממנה עשירית האיפה. אבל רחוק הוא שיכוון המגיד לזה, שאם כן היה ראוי שיאמר "הא קורבנא דעניא" או  "עיסא לעניא", אם נקראת כן על שם כמות העיסה, לא ראוי שיאמר "הא לחמא עניא", המורה היות העוני בטבע הלחם ומהותו ולא בכמותו ומקריו.הטעם השישי הוא שהביאו הרמב"ן בפירושו בסדר ראה, וזה לשונו: "זכר במצה שתהיה לחם עוני להגיד שציווה בה לזכור שיצאו ממצרים בחיפזון, והוא עוני זכר כי היו במצרים בלחם צר ומים לחץ, והנה תרמוז לשני העניינים, וכן אמרו 'הא לחמא עניא די אכלו אבהתנא בארעא דמצרים', או יאמר שתהיה עוד עשויה כלחם עוני ולא שתהיה מצה עשירה כמו שהזכירו חז"ל" עד כאן. הנך רואה שנעתק הרב מעניין לעניין לפי שלא נתיישב אצלו מה שאמר בזה, כי הנה השיגוהו הספקות שזכרתי לשאר הדעות.והנכון אצלי בזה הוא שהמצה נקראת בכתוב "לחם עוני" לשתי סבות, האחת מפני טבעה לפי שהלחם בהיותו עיסה טרם יחמץ הוא קטן הכמות ואחר החמוץ יגדל הבצק ויעלה למעלה, והמצה להיעדר החמוץ לא תגדל ולא תעלה אלא תשאר שפלה ונמוכה, ובבחינה הזאת נקראת לחם עוני, לשפלותה ודלותה בעצמה. והסיבה השנית היא לפי שהמצה קשה להתעכל באצטומכא ותשאר בה זמן רב ולכן תספיק מעט מהמצה לאוכליה ועניים יאכלוה. וכבר כתב יצחק הישראלי7רבי יצחק בן שלמה הישראלי היה רופא, פילוסוף ופרשן מקרא מתקופת הגאונים, בן המאה העשירית. ר' יצחק שימש כרופא בחצרות מלכים באפריקה, ביניהם בחצרו של מלך מצרים. ספריו התפרסמו באירופה לאחר שתורגמו מערבית ללטינית בסוף המאה ה-11. במסעדים אשר לו כי פת המצה קשה מאוד להתעכל באצטומכא ומתארכת לצאת משם ומולידה רוח ועצירות, ושהוא מאכל ניאות לעמלים יותר מן הלחם החמץ, והמצריים לשנאתם את בני ישראל וגדי שיספיק להם לחם מעט היו נותנים להם לחם מצה כשהיו עוסקים בבנייני המלך ובמעשי הלבנים. ועל פי שתי הבחינות האלה נאמר כאן "הא לחמא עניא", כנגד הראשונה מפאת טבעה כמו שזכרתי, וכנגד הסיבה השנית אמר "די אכלא אבהתנא בארעא דמצרים". ועם היות שנקראה מצה לחם עוני על פי הבחינות האלה, לא נאמר מפני זה הטעם שצוותה התורה מצוות המצה ואיסור החמץ, כי אם לזכרון מהירות גאולתם כמו שנאמר "כִּי בְחִפָּזוֹן יָצָאתָ מֵאֶרֶץ מִצְרַיִם" (דברים ט"ז, ג'). ולפי זה המאמר "הא לחמא עניא" אינו סותר דרשת רבן גמליאל במצה, לפי שכאן לא אמר המגיד טעם מצוות המצה אלא זכר עניינה שהיא "לחמא עניא" בטבעה ושכן היו אוכלים אותה במצרים. אחר כך פירש רבן גמליאל הסבה וטעם המצווה שאנו אוכלים מצה בלילה הזה, שהוא לזכרון מהירות הגאולה והיציאה ממצרים, שלא הספיק בצקם להחמיץ. ויהיה פירוש הכתוב שבעת ימים תאכל עליו מצות, שהמצות הוא לכם עוני נמוך ושפל ובלתי גדול לפי שבחפזון יצאת מארץ מצרים, עד שלא הספיק בצקם להתגדל ולעלות ונשאר נמוך ועני, וכמו שאפרש עוד במקומו. אמנם למה התחילה ההגדה בעניין המצה ולא בעניין הפסח שהוא הקודם במעלה, או למה לא שכר שלשתם פסח, מצה ומרור? ומה עניין אומרם הא שתא הכא והכפל אשר בא בו? שהן השאלות שבאו בשער השלישי והרביעי.כבר אמרו איזה חכמים שמה שאנו אומרים כאן "כל דכפין" וכו' הם דברי אנינות על גלותנו, לומר כי בעוונותינו גלינו מארצנו ואין אנו יכולים להקריב קרבן פסח חוצה לארץ ולכן אין צריכין טהרה וכל מי שיביא יאכל מזה הפת כטמא וכטהור יחדיו. ולכן אנו מתפללים כי לשנה הבאה נעלה לירושלים ושבו בנים לגבולם, והאיש אשר הוא טהור יעשה כחוקת הפסח. ואחרים אמרו שהתחיל במצה מפני שאנחנו חייבים לתת לכל אחד די מחסורו בזמן הזה, ולכן אמר כל דכפין ייתי ויכול, ומאמר "לשנה הבאה בירושלים" נאמר לזכרון ירושלים, שנאמר "אם לא אעלה את ירושלים על ראש שמחתי", כאילו יאמר שבזכות הנדיבות והצדקה הזאת להאכיל לעניים יגאלנו השם יתברך. וראיתי מי שפירש המאמר הזה כולו כמו סיפור מה שעשוה אבותינו במצרים והדברים שאמרו באותו ליל שימורים, שמפני חפזונם היו מחלקים המצה והיו אומרים אלו לאלו כל דכפין ייתי ויכול כל דכפין ייתי ויפסח, כי עתה אנו עבדים במצרים ולשנה הבאה נשב בארעא דישראל בני חורין. וכי עתה כאשר אנחנו מחלקים את המצה הננו מודיעים לנשים ולתינוקות שכן עשו אבותינו במצרים והיו אומרים ככה ואנחנו עושים כדוגמתם, והתקינו לומר לשנה הבאה בני חורין בלשון הקודש ולא בלשון ארמי כשאר מילות המאמר כדי שלא יבינו אותו הגויים (בני בבל) ויחשבו את ישראל למורדים במלכות כי אומרים שיעשו עצמם בני חורין. אך גם בפירוש זה לא נמלט מהספקות.והנכון אצלי הוא שחז"ל תקנו שיהיה האדם בליל פסח קודם ההגדה בוצע מצה, וישים חציה תחת המפה ויעקור הקערה מלפניו ויתננה בצד השולחן כאילו כבר אכלו, כדי שיהיה מקום לשואל לשאול למה עושין כן ועדיין לא אכלו? כמו שביארו בפרק ערבי פסחים. ותקנו שיקרא את העניים שיבואו לאכול עמו מפתו ולסמוך על שולחנו, ולכן דיבר מהמצה, לא שהייתה כוונתו לבאר טעמה והוראתה כאן, כי אין זה המקום ויבוא להלן, אבל הוא לעניין ההכרזה כאילו יאמר העניים והאביונים מבקשים להם ואין, לכו לחמו בלחמי כי הוא לחם עוני ונאות אליכם, ובלילה זה כולנו שווים, ואף על פי שאתם עניים אל תתביישו כי כן היו אבותינו במצרים, ולפי זה יבואו דבריו מתוך ענוונותו לדבר על לב העניים ועל כן קרא לחמו לחם עוני, ואפשר שקראו "לחם" לכל הסעודה הכוללת פת ותבשיל וגם בשר, כמו שנאמר "הִנְנִי מַמְטִיר לָכֶם לֶחֶם מִן הַשָּׁמָיִם" (שמות ט"ז, ד') ואמר משה לישראל "בְּתֵת ה' לָכֶם בָּעֶרֶב בָּשָׂר לֶאֱכֹל וְלֶחֶם בַּבֹּקֶר לִשְׂבֹּעַ" (שמות ט"ז, ח'), הרי שלחם כולל גם בשר . ותועיל ההכרזה הזאת גם כן שלמה שכוונו לעורר את הנשים והטף לשאול מה נשתנה, אחרי שאתה שמת לפניך לחם ובצעת אותו והקדשת קרואיך ואמרת כל דכפין ייתי ויכול, מה זה שנתחרטת ממה שהתחלת לעשות ועקרת את הקערה? והשאלה הזאת תביא התשובה בסיפור יציאת מצרים.הנה התבאר מזה למה התחיל המגיד מאמרו מן הלחם, ולכן לא הזכיר הפסח כי אינו נוהג בחוצה לארץ, ולא הזכיר המרור לפי שאין בו אכילה מספקת לעניים, כי אם המצה באשר היא לחם שלבב אנוש יסעד, ועל בדומה לזה אמר עזרא "וַיֹּאמֶר לָהֶם לְכוּ אִכְלוּ מַשְׁמַנִּים וּשְׁתוּ מַמְתַקִּים וְשִׁלְחוּ מָנוֹת לְאֵין נָכוֹן לוֹ כִּי קָדוֹשׁ הַיּוֹם לַאֲדֹנֵינוּ" (נחמיה ח', י'). ולפי שאנחנו עושים בלילה הזה זכר לפסח מאותו אפיקומן שאוכלים על השובע במקומו, לכן יאמר "כל דכפין ייתי ויכול" כנגד המצה והמרור ו"כל דצריך ייתי ויפסח" כנגד האפיקומן שהוא זכר לפסח. ולכן זכרו באחרונה כאילו יאמר אחרי שיאכל מהסעודה יעשה זכר לפסח, או שכפל הדבר במילות שונות "ייתי ויכול" "ייתי ויפסח", רצה לומר לחוג עמנו.  אמנם מה שאמר עוד "הא שתא הכא" וכו' בא להתיר ספק שאפשר שיסופק נגדו, והוא כי הייתה חובה לחוג את חג הפסח בירושלים, כמו שנאמר " לֹא תוּכַל לִזְבֹּחַ אֶת הַפָּסַח בְּאַחַד שְׁעָרֶיךָ וגו' כִּי אִם אֶל הַמָּקוֹם אֲשֶׁר יִבְחַר ה'" (דברים ט"ז, ה'-ו') וגו' ואיך יאמר אם כן יבוא ויפסח כאילו יעשה בחו"ל חג הפסח? וכדי להשיב על זה אמר "הא שתא הכא לשנה הבאה בארעא דישראל", רצה לומר השנה הזאת נעשה כאן חג ושנה הבאה יהי רצון מלפני ה' שנעשה אותו בארעא דישראל, ונתן הסיבה הזאת באומרו "הא שתא עבדי" רצה לומר הנה נעשה בשנה הזאת החג כאן מפני שאנחנו עבדים בגלות, ולא נוכל לעלות לחוג בירושלים, אולם לשנה הבאה כאשר יגאלנו ה' נוכל לעשותו בארעא דישראל לפי שנהיה אז בני חורין ונעשה המצווה כהלכתה. ולפי זה המאמר הראשון שאמר "הא שתא הכא לשנה הבאה בארעא דישראל" הוא הודעה והנחה, ומה שהוסיף לומר עוד "הא שתה הכא עבדי" וכו' היא נתינת הסבה בדבר, ולא בא זה בכאן על דרך תפילה ובקשה כמו מה שנאמר בסוף ההגדה "כן ה' אלוהינו יגיענו למועדים אחרים" וכו', כי בכאן בא רק להתיר הספק כמו שזכרתי.הנה נתבאר זה המאמר הראשון והותרו השאלות והספקות אשר בארבעת השערים הראשונים.}%endcomment
\newsection{מה נשתנה}
\hebeng{{\small מסיר את הקערה מעל השולחן. מוזגין כוס שני. הבן שואל:} }{{\small He removes the plate from the table. We pour a second cup of wine. The son then asks:} }
\hebeng{מַה נִּשְׁתַּנָּה הַלַּיְלָה הַזֶּה מִכָּל הַלֵּילוֹת? שֶׁבְּכָל הַלֵּילוֹת אָנוּ אוֹכְלִין חָמֵץ וּמַצָּה, הַלַּיְלָה הַזֶּה – כֻּלּוֹ מַצָּה. שֶׁבְּכָל הַלֵּילוֹת אָנוּ אוֹכְלִין שְׁאָר יְרָקוֹת – הַלַּיְלָה הַזֶּה (כֻּלּוֹ) מָרוֹר. שֶׁבְּכָל הַלֵּילוֹת אֵין אָנוּ מַטְבִּילִין אֲפִילוּ פַּעַם אֶחָת – הַלַּיְלָה הַזֶּה שְׁתֵּי פְעָמִים. שֶׁבְּכָל הַלֵּילוֹת אָנוּ אוֹכְלִין בֵּין יוֹשְׁבִין וּבֵין מְסֻבִּין – הַלַּיְלָה הַזֶּה כֻּלָּנוּ מְסֻבִּין.}{What differentiates this night from all {[other]} nights? On all {[other]} nights we eat \textit{chamets} and matsa; this night, only matsa? On all {[other]} nights we eat other vegetables; tonight (only) \textit{marror}. On all {[other]} nights, we don't dip {[our food]}, even one time; tonight {[we dip it]} twice. On {[all]} other nights, we eat either sitting or reclining; tonight we all recline.}%
\commenta{\textrm{\textbf{מה נשנתנה הלילה הזה}} הנה הגר״א בבאורו כאן כותב, כי שם ״לילה״ הוא שם ממין נקבה (ומה  דכתיב כאן הלילה הזה הוא יוצא מן הכלל), וראי׳ לזה (כך דבריו), כי כשבא שם זה במספר רבים נחתם בסימן נקבה, בוא״ו תיו (לילות), ולא ביו״ד ומ״ם שהוא סימן זכר, כמו ימים, עכ״ד. ולא אוכל להאמין שיצאו הדברים מהגר״א, יען כי בכל מקום במקרא  כשנזכר שם לילה בא בסימן זכר, כמו בס״פ וישב, ונחלמה בלילה אחד, ובתהלים (קי״ט) ולילה כיום יאיר (ולא תאיר), ושם (ק״ד) תשת חושך ויהי לילה, ובאיוב (ג) והלילה אמר, ובנחמיה (ד׳) והיה לנו הלילה משמר, ופעמים רבות מאוד הלשון בלילה ההוא, ואין אף מקום אחד בכל המקרא שיבא שם לילה  בסימן נקבה. והראי׳ שהביא מחתימת השם במספר רבים בסימן נקבה, בוא״ו ת״יו (לילות) ולא בסימן זכר, ביו״ד מ״מ — ראי׳ זו אינה מכרחת כלל, שהרי כן מצינו הרבה שמות זכרים שנחתמים בוא״ו ותיו, שזה סימן נקבה, כמו, אבות, אריות, בכורות, לבבות, מטות, מקומות, שמות, ועוד, והם כולם ממין זכר. וכנגד זה יש הרבה שמות ממין נקבה, ונחתמים במספר רבים ביו״ד מ״ם, שהם סימן החתימה למין זכרים, כמו נשים, פלגשים, דבורים, נמלים, שנים, תאנים, ועוד הרבה. והוא שאמרתי, כי קשה עלי מאוד להאמין שיצאו הדברים מהגר״א, ואעפ״י שכבר העירותי מזה בחבור אעפ״י כן אמרתי לשנות הדברים כאן, מקום מקור דבריו אלה, וגם כי אולי ימצא מי שיעמיד דבריו, והיה׳ זה לי לנוחם נפש, ולו — לצדקה ולזכות. ושוב ראיתי כי בשל״ה הלכות פסחים ד״ה מעשה בר״א, כתב גם הוא, כי לילה לשון נקבה, והנה לא אדע מה אידון בזה. ולבי יגיד לי, כי עמדו שניהם על נוסח התפלה בברכת השבת שאומרים בלילה וינוחו בה וביום — וינוחו בו, והנה לוא גם מסדר התפלה כיון זה, אך איך יודחו לשונות כמה וכמה פסוקים המורים ההיפך. ועל האמת אין מנוסח התפלה כל סתירה ללשונות הכתובים, דלילה שם זכר, וכונת הלשון וינוחו בה מיוסד על דברי המכילתא בפ׳ תשא על הפסוק כל העושה מלאכה ביום השבת אין לי אלא ביום, בלילה מניין תלמוד לומר מחלליה מות יומת, ע״כ. והבאור הוא דהלשון מחלליה מוסב על שם שבת, ושם  ״שבת״ כולל כל המעל״ע מערב עד ערב, וגם לילה בכלל, ומכיון ששביתת לילה נלמד מלשון מחלליה, שם נקבה, לכן אומרים לענין מנוחה בלילה וינוחו בה, אבל אין בזה שום הוראה ליחש המין בשם לילה. ואמנם כלל דברי המכילתא אין לי אלא ביום בלילה מניין, צריך באור, כי מהיכי תיתא צריך למוד מיוחד ללילה, והלא ידוע, כי בכל המצות וכל הענינים שהלילה הולך אחר היום שלאחריו (לבד מקדשים, שהלילה הולך אחר היום שלפניו, עיין בחולין פ״ג א׳), וכמש״כ בראשית הפתיחה, וא״כ למה צריך לענין ליל שבת לימוד מיוחד. ואפשר לומר, משום דבסדר הבריאה אנו למדין דהלילה הולך אחר היום מדכתיב בכולם ויהי ערב ויהי בקר יום אחד משמע דאחדות היום תבנה מן יום ולילה, אבל בשבת לא כתיב ויהי ערב ויהי בקר יום השביעי יום אחד אפשר שאין הלילה נמשך אחר היום, ולכן צריך למוד. וגם יש סברא להוציא איסור מלאכת שבת מן ליל שבת, משום דכידוע כל מלאכות שבת ילפינן ממשכן ואלה שלא היו במשכן אין אסורות (ע' שבת צ״ו ב׳), והנה מלאכת המשכן לא נעשתה בלילה (ע׳ שבועות ט״ו ב׳) אם כן יש סברא להוציא איסור מלאכה בלילה גם בשבת, לכן צריך למוד מיוחד לאסור מלאכה בשבת גם בלילה.}%endcomment%
\commentb{\textrm{\textbf{תשובות לשערים ה' – ו'}}\textrm{\textbf{"מה נשתנה הלילה הזה מכל הלילות" וכו',}}בפרק ערבי פסחים אמר רבא אטו חיובא לדרדקי? אלא אמר רבא "הכי קתני אין אנו מטבילין" (פסחים קט"ז, א'). הרצון שהגירסא שבספרים שלנו "אין אנו חייבים לטבול" אינה אמיתית כי אם "אין אנו מטבילין", ונראה לי שאין השאלה הזאת מה נשתנה מה שאמרו בפסחים שיעשו בעקירת הקערה כדי שישאלו התינוקות, כי אותה השאלה לא סידרו אותה חז"ל בכתב, ודי שמפי עוללים ויונקים נשמע "מה זאת" ואז אב לבנים יודיע עניין יציאת מצרים. אמנם השאלה הזאת "מה נשתנה" כוללת לכל אדם, שאם אין נערים ותינוקות לשאול אז נער וזקן נושאין ונותנין בדבר, וכמו שאמרו חז"ל תלמיד חכם בניו שואלים אותו ואם לא אשתו שואלת, ואם לאו הוא שואל לעצמו.ועניין השאלה הזאת הוא שהשואל רואה בלילה הזה שאנו עושים דברים המורים על היותנו בני חורין בני מלכים שרים ויועצי ארץ, ונעשה דברים אחרים המורים בהפך על היותנו עבדים נכנעים בזויים ושפלים, כי הנה בכל הלילות אין אנו חייבים לטבול אפילו פעם אחת והלילה הזה אנו מטבילי בחובה שתי פעמים, או אין אנו נותנין לטבול קודם הסעודה והלילה הזה שתי פעמים, וזה מורה על היותנו בני חורין ושרים ונדיבי עם כיוון שאנו אוכלין המאכל עם הקוני הטבולים כי זהו דרך השרים האוכלי מעדנים, ומצד אחר יראה ההפך שבכל הלילות אנו אוכלין חמץ או מצה כל אדם כפי רצונו מבלי חיוב, והלילה הזה כולו מצה בחיוב, וזה בלי ספק מורה על העבדות, שהמצה היא לחם עוני ומאכל העבדים העמלים כמו שזכרתי, ובזה הדרך עצמו אנו אוכלים בכל הלילות שאר ירקות בתבשיל מרוקח ובפת והלילה הזה כולו מרור, רצה לומר מרור כמו שהוא חי ולא מבושל ובלי פת וזה גם כן סימן עבדות ועניות. עוד נעשה בהפך זה שבכל הלילות אנו אוכלין בין יושבים ובין מסובין והלילה הזה כולנו מסובין, וזה מורה על היותנו בני חורין כיוון שאנחנו כקטן כגדול אוכלים בהסבה בכבוד גדול. לפי זה יש לנו בלילה הזה שני דברים המורים על החירות שהם הטיבול וההסבה, והוא דבר זר שנעשה בלילה אחת דברים הפכיים בהוראותם, ולפי שעל פי שנים עדים יקום דבר הביא על העבדות שני העדים ההם ועל החירות השנים האחרים, ולא זכר אכילת הפסח, לפי שהיא אינה מורה לא על החירות ולא על השיעבוד, גם חיוב ארבע כוסות אינו מורה על החירות כי גם העבדים והמשועבדים לפעמים ירבו בשתייה כמאמר שלמה "תְּנוּ שֵׁכָר לְאוֹבֵד וְיַיִן לְמָרֵי נָפֶשׁ" (משלי ל"א, ו'). הנה התבאר המאמר הזה והותרו הספקות הנופלות עליו בשער החמישי והשישי.}%endcomment
\newsection{עבדים היינו}
\hebeng{{\small מחזיר את הקערה אל השולחן. המצות תִהיינה מגלות בִשעת אמירת ההגדה. } }{{\small He puts the plate back on the table. The matsot should be uncovered during the saying of the Haggadah} .}
\hebeng{עֲבָדִים הָיִינוּ לְפַרְעֹה בְּמִצְרָיִם, וַיּוֹצִיאֵנוּ ה׳ אֱלֹהֵינוּ מִשָּׁם בְּיָד חֲזָקָה וּבִזְרֹעַ נְטוּיָה. וְאִלּוּ לֹא הוֹצִיא הַקָּדוֹשׁ בָּרוּךְ הוּא אֶת אֲבוֹתֵינוּ מִמִּצְרָיִם, הֲרֵי אָנוּ וּבָנֵינוּ וּבְנֵי בָנֵינוּ מְשֻׁעְבָּדִים הָיִינוּ לְפַרְעֹה בְּמִצְרָיִם. וַאֲפִילוּ כֻּלָּנוּ חֲכָמִים כֻּלָּנוּ נְבוֹנִים כֻּלָּנוּ זְקֵנִים כֻּלָּנוּ יוֹדְעִים אֶת הַתּוֹרָה מִצְוָה עָלֵינוּ לְסַפֵּר בִּיצִיאַת מִצְרָיִם. וְכָל הַמַּרְבֶּה לְסַפֵּר בִּיצִיאַת מִצְרַיִם הֲרֵי זֶה מְשֻׁבָּח. }{We were slaves to Pharaoh in the land of Egypt. And the Lord, our God, took us out from there with a strong hand and an outstretched forearm. And if the Holy One, blessed be He, had not taken our ancestors from Egypt, behold we and our children and our children's children would {[all]} be enslaved to Pharaoh in Egypt. And even if we were all sages, all discerning, all elders, all knowledgeable about the Torah, it would be a commandment upon us to tell the story of the exodus from Egypt. And anyone who adds {[and spends extra time]} in telling the story of the exodus from Egypt, behold he is praiseworthy. }%
\commenta{\textrm{\textbf{עבדים היינו לפרעה במצרים}} הנה סגנון כל נוסח ההגדה, שהמגיד פתח הענין בלשונו שלו, ומקיים אותו בלשון התורה, בלשון ״שנאמר״, וכן סובב הולך כל הספר. ולכן לפלא הוא, שכאן בתחלת הספר שינה סגנונו באופן שכל המאמר אינו דבוק, כי הן הלשון ״עבדים היינו וכו׳״ אינו שלו, אך כולו מלשון התורה בפרשה ואתחנן (ו' כ״א). ולפי סגנונו בכל ההגדה הי׳ צריך לומר עבדים היו אבותינו לפרעה במצרים ויוציאם ה׳, שנאמר עבדים היינו וכו׳, ועל זה מכוון הלשון ואלו לא הוציא הקב״ה את אבותינו, אבל כשפותח בלשון עבדים היינו הי׳ צריך לומר ואלו לא הוציאנו. ויותר מזה, כי את לשון התורה אי אפשר לצרף לכאן, יען כי שם אמר זה משה לאותו הדור שיצא ממצרים, ושם הענין והלשון עבדים היינו מכוון, ולא כן כאן. ואמנם כי בגמרא פסחים (קט״ז א׳) נזכר הלשון עבדים היינו, אבל הגמרא נסמכה על לשון בעל ההגדה, וצע״ג. וכעין זה אנו מוצאים בההגדה שנוי לשון מאשר הוא במקורו. אבל שם השינוי מוכרח לבא, והוא מה שאומרים במאמר רבן גמליאל, כל שלא אמר שלשה דברים הללו בפסח וכו׳, ומפרש, פסח שהיו אבותינו אוכלים וכו, בעוד שמקור מאמר זה שבמשנה פסחים (קט״ז ב׳) הלשון פסח שאנו אוכלים, ואמנם שם השינוי מוכרח ומובן, יען כי בעל המאמר הזה, רבן גמליאל (הוא הידוע בשם רבן גמליאל הזקן מפני שהיו אחריו עוד שני חכמים בשם זה) הי׳ חי בזמן הבית, ואז אמנם אכלו מן הפסח. אבל סגנון הלשון שלפנינו, עבדים היינו, לא אדע ליישבו כפי שבארנו.\textrm{\textbf{ואלו לא הוציא הקב״ה את אבותינו ממצרים, הרי אנו ובנינו ובני בנינו משועבדים היינו לפרעה במצרים.}} לא נתברר מניין לו וודאות זאת. כי היינו עד עולם משועבדים במצרים, כי הלא המקרים לחליפות עקירות והנחות מעמים שונים וארצות שונות רבים ושונים הם, ואי אפשר להחליט שעבוד נצחי לעם ידוע במקום אחד וביותר בנוגע לעם ישראל. ואפשר לומר, משום דעל הרוב יקרה סבת יציאת עם ישראל ממקומם שהם שם עפ״י זה שבני המדינה דוחקים אותם לצאת. בהמציאם להם עילות ועלילות שונות ששוללים מהם פרנסתם ואומנתם, או עילות שאינם אוהבי העם והמדינה שהם בה וכדומה עלילות משונות ובדויות, ולכן רוצים להוציאם ממקומם. אבל ענין ישיבת ישראל ושעבודם במצרים הי׳ במגמת הפוכה, והיינו שלא רצו המצרים לשלחם, כפי המבואר בתורה מעקשות לב פרעה שלא לשלחם, ולכן לא הי׳ כל תקוה גם לדורות לצאת משם בלא סיוע מן השמים.\textrm{\textbf{ואלו לא הוציא הקב״ה את אבותינו ממצרים הרי אנו ובנינו ובני בנינו משועבדים היינו לפרעה במצרים}} יש להעיר לפי הנחה זו למה לי הדרשה דלהלן בכל דור ודור חייב אדם לראות את עצמו כאלו הוא יצא ממצרים, וללמוד זה מדיוק פסוק אחד, והלא פשיטא הוא, דאחרי שלולא הוציאנו ה׳ משם היינו עד כה משועבדים במצרים בודאי דרוש לכל אחד בכל דור לראות כאלו הוא יצא, כיון דלעולם לא היו נפטרים מזה. וצריך לאמר, דאלו בשביל זה היינו מחויבים לראות עצמנו רק כמשועבדים, אבל לא כיוצאים לחירות, והדרשא תוסיף, כי מחויבים לראות עצמם גם כיוצאים לחירות.\textrm{\textbf{ואפילו כולנו חכמים... מצוה עלינו לספר ביציאת מצרים וכל המרבה לספר הרי זה משובח}} לכאורה לפי המבואר במשנה פסחים (קט״ז ב׳) שכל שלא אמר שלושה דברים אלו בפסח לא יצא ידי חובתו, פסח מצה ומרור, הם ופירושם, או טעמיהם, כמו פסח*בנוסח ההגדה — שהיו אבותינו אוכלים בזמן שביהמ״ק היה קיים. וטעם חסרון לשון זה, שהיו אבותינו אוכלים, במשנה פשוט, משום דבעל המאמר הזה, כל שלא אמר וכו׳, הוא רבן גמליאל הזקן, והוא היה עוד בזמן המקדש, שאז הקריבו ואפלו מקרבן פסח, ולכן אמרו סתם פסח, וזה כמו שהי׳ אומר פסח שאנו אוכלים.  על שום שפסח הקב״ה על בתי אבותינו במצרים, וכן מצה על שום שלא הספיק בצקם של אבותינו להחמיץ, ומרור על שום שמררו המצרים את חיי אבותינו במצרים וכו׳. ומתבאר, דבמאמר שלשת דברים אלה עם פירושיהם יוצאים חובת ספור יצ״מ, ובאמת מצינו סמוכים לזה בתורה בתשובות האבות להבנים על אודות  עניני יצ״מ, השיבו בקצרה בלשון פסוקים אחדים בתורה, כמו ואמרתם זבח פסח הוא לה׳ אשר פסח על בתי בני ישראל במצרים וכו׳ (פ׳ בא, י״ב כ״ז), וכן והגדת לבנך ביום ההוא לאמר בעבור זה עשה ה׳ לי בצאתי ממצרים (שם י״ג ח׳), וכן ואמרת לבנך עבדים היינו לפרעה במצרים... ויתר ה׳ אותות ומופתים... ואותנו הוציא משם (פ׳ ואתחנן, ו׳ כ״א) — והנה באלה הלשונות הגבילה התורה את ספור יצ״מ. וכנגד זה אמרו במעלת תאר למוד התורה לא ימוש ספר התורה הזה מפיך והגית בו יומם ולילה (יהושע א׳:ח׳), ורק מפני קיום העולם הותר לנהוג גם עניני דרך ארץ (ראה ברכות ל״ה ב׳). ועוד אמרו בזה, שאמר הקב״ה לדוד, טוב לי יום אחד שאתה עוסק בתורה מאלף עולות שעתיד שלמה בנך להקריב לפני (שבת ל׳ א׳), והעוסק בתורה פטור מן התפילין (מכילתא, בא), ולענין הדברים שאין להם שיעור אמרו ותלמוד תורה כבגד כולם (כלומר, עולה על כולם), ועוד כמה מאמרים המפליגים בשבח תלמוד תורה וקראו להם חיי עולם (שבת י׳ א׳). אם כן הלא הי׳ מן הדין ומן הסברא, שהמון עם שלא ידעו לעסוק בתורה, להם ראוי להאריך ולהרבות בספור יצ״מ, אבל בחכמים ונבונים וזקנים (ואין זקן אלא מי שקנה חכמה (קדושין ל״ב ב׳), להם הי׳ מהראוי לספר ביצ״מ כפי הלשונות שהגבילה התורה כפי שחשבנו הפסוקים בהוראה זו, ויתר הזמן יתעסקו בתורה. כך הי׳ באפשר לחשוב, על זה אמרו, ואפילו כולנו הכמים וכו׳ שאפשר לנו לעסוק בתורה — אפילו הכי מצוה עלינו לספר ביצ״מ בארוכה וכל המרבה הרי זה משובח, כלומר, דגם רבוי הספור הוא בגלל המצוה. ואמנם גם אחרי הפלגת דברים בזה לא שמענו טעם בדבר, ובפרט לפי מה שהערנו בזה שמצוה זו דוחה מצות או חיוב ת״ת — בודאי טעמא בעי. ואפשר לומר עפ״י כלל גדול שהגבילו חז״ל בחיוב מצות, כי מצוה שנאמרה קודם מת״ת גדולה במעלתה עד שדוחה מצוה שלאחר מת״ת (יבמות ה׳ ב׳), ואשר על כן מצוות מילה דוחה שבת אעפ״י שאסור לעשות חבורה בשבת. ולכן מצות ספור יצ״מ שהוא קודם מת״ת דוחה מצות ת״ת שהיא לאחר  מת״ת, כמש״כ ודברת בם, ועוד, ומכיון שהיא מצוה ממילא כל המרבה משובח, כמו בכל מצוה. ומה שהעוסק בתורה פטור מן התפילין, כפי שהבאנו, אעפ״י שמצות תפילין היא קודם מת״ת (בס״פ בא, וקשרתם, והיו לטוטפות וכו׳) זה הוא משום דכפי המבואר בתורה שם כל עיקר מצות תפילין הוא ״למען תהיה תורת ה׳ בפיך״ (שם י״ג ט׳), וכיון שהוא עוסק בתורה הרי השיג המטרה והתכלית שבסבתם באה מצות תפילין, ולכן פטור מהם. וטעם יתרון מצוה שקודם מתן תורה על מצוה שאחר מת״ת אפשר לומר משום דמצינו שהקב״ה מוקיר ומכבד את ערך הזקנה, אשר על כן צוה על הידור פני זקן, ומצוה שקודם מת״ת היא בערך קודם בזמן למצוה שאחר מת״ת. ואמנם התוס׳ בקדושין (ריש דף ל״ה א׳) כתבו בשם הירושלמי להיפך מדעת הבבלי, והיינו דמצוה שאחר מתן תורה גדולה במעלה ממצוה שקודם מת״ת, דלא ביארו הטעם. ואולי הטעם מפני שנתנה בהמון לכל ישראל, וברוב עם הדרת מלך (משלי י״ד ט״ו) והדרת כבוד חופפת עליה. וידוע ממעלת כבוד הצבור. ולדידי׳ (לדעת הירושלמי) צריך לאמר יתרון מצות ספור ביציאת מצרים עפ״י המבואר במס׳ ב״מ (ס״א ב׳) דכמה מצות תלתה התורה ביציאת מצרים, כמו שחשיב שם הרבה, ויש עוד הרבה כאלה (ובטור או״ח ריש הלכות סוכה (סימן תרכ״ח) האריך בזה, והסביר טעם הדבר). ולכן נחשב ענינה כיתד שכל מצות התורה תלויות בו, ואין להאריך עוד. ודע, כי עפ״י חילוק הדעות מבבלי וירושלמי בענין קדמות ואחרות מצוה  שקודם ואחר מת״ת — עפ״י זה נראה לפרש יפה ענין אחד בגמרא שעמלו בו הרבה מהראשונים וכל אחד דחה דברי זולתו, ולבסוף לא באו לידי באור בזה שיעלה ישר ויפה ולא יסתר מן איזה מקום שהוא. והוא ענין קצר וישר בקדושין (כ״ט ב׳), ת״ר, ללמוד תורה ולישא אשה (כשיש על האדם שני חיובים אלה) ילמוד תורה ואח״כ ישא אשה, ורב יהודה אמר שמואל, נושא אשה ואח״כ ילמד תורה (וטעמיהם בזה, משום דלשני החיובים יש יתרון לאחד מה שאין לשני, כי להתורה יש יתרון שאין עליו עול חיי משפחה ויוכל ללמוד במנוחה, ולנשואי אשה יתרון כי ינצל מהרהורים וילמוד בטהרה), וסיימו בגמרא, ולא פליגי, הא לן והא להו, ע״כ, ולא בארו הדברים בגמרא. ורש״י ותוס׳ ור״ן ועוד מפרשים פירשו כל אחד כפי שחשב בדעתו, אחר שדחה פירוש זולתו, ולבסוף נדחה גם פירושו שלו וכתב פירוש אחר, וגם הוא נדחה, וכה לא נתבאר הדבר. על כן נראה הבאור עפ״י המבואר לפנינו למעלה בנגוד הדעות בין תלמוד בבלי ובין תלמוד ירושלמי, לאיזו מצוה יש יתרון קדימה, אם לזו שקודם מת״ת או לזו שאחר מת״ת, כי לדעת הבבלי יתרון קדימה מצוה שקודם מת״ת ולדעת הירושלמי להיפך, שיתרון לזו שאחר מת״ת, כפי שבארנו. והנה ידוע, שמצות פו״ר היא מצוה שקודם מת״ת. ועלי׳ נצטוו עוד אדה״ר ונח ובניו, וכנגד זה מצות ת״ת היא מצוה שאחר הדבור, שסמכו אותה על הלשון ודברת בם, ולמדתם, ולא ימוש ספר התורה הזה מפיך. ולכן להבבלים מצות נשואי אשה קודמין ולהירושלמים (לבני א״י) מצות ת״ת קודמת, וזהו באור הלשון הא לן (להבבלים) והא להו (לבני א״י).\textrm{\textbf{וכל המרבה לספר ביציאת מצרים הרי זה משובח}} הנה זה פשוט, דכלל ענין ספור יציאת מצרים הוא לספר ביותר מגבורת ה׳ ומנפלאותיו, אבל לכאורה מצינו הוראת חז״ל למנוע מרבוי דברים בזה, שכן מתבאר במס׳ ברכות (ל״ג ב׳) ההוא (שליח צבור) התפלל ואמר בשבח ה', האל הגדול הגבור והנורא והאדיר והחזק והאמיץ והעזוז והודאי והנכבד, ואמר לא חכם אחד בקפידא, האם סיימתינהו לכולהו שבחי דמרי? ומבואר מזה שאין ראוי להרבות בשבח יתר על הקבוע, משום דזה מורה שכל מה שיש לו נחשב ויותר אין, וזה פגם בכבודו. ובבית יוסף לטור או״ח סימן קי״ג כתב בזה דרק בנוסח תפלה אין ראוי להרבות אבל בשירות ותשבחות אין קפידא, והאריך הרבה בה, אבל הכל עפ״י הסברא לבד, בלא כל יסוד ומקור. ואנחנו בארנו ענין זה עפ״י מקור נאמן בחז״ל, וקבענו הדברים למעלה בתפלת שחרית בברכת ישתבח שמך, ומפני האריכות טורח לי להעתיקם לכאן, וראוי לעיין שם.}%endcomment%
\commentb{\textrm{\textbf{תשובות לשערים ז' - י"ב}}\textrm{\textbf{עבדים היינו לפרעה במצריים וכו'}}, זוהי תשובת השאלה "מה נשתנה", על מה ששאל השואל למה בלילה הזה נעשה דברים מורים על העבדות ודברים מורים על החירות, היתה תשובתו כי בלילה הזה בתחילתו עבדים היינו לפרעה במצרים לכן נעשה אותם דברים המורים על העבדות, ובלילה הזה בסופו הוציאנו ה' אלוהינו משם והיינו בני חורין לכן נעשה אותם הדברים המורים על החירות. ולפי שהלילה הזה משונה בדברים הפכיים מכל שאר הלילות לכך אנו עושים דברים הפכיים לזכרון העבדות והחירות.והתשובה הזו מספקת וכוללת לכל חלקי השאלה והותר בזה הספק אשר בשער השביעי. ולכן לא נתן כאן טעם המרור משום "וַיְמָרְרוּ אֶת חַיֵּיהֶם" (שמות א', י"ד) כי לא בא לתת טעם למצוות מצה או מרור כי אם להפכיות הפעולות כמו שכתבתי.  ובאומרו "עבדים היינו לפרעה" רמז אל שתי מניעות עצומות שהיו בעניין היציאה, האחת המלך הקשה ורע מעללים פרעה אשר לרשעתו נקרא כן מלשון פרעו אהרון, כאילו הוא שם מורכב משתי מילות "פועל-רע", הלוא תראה קושי ערפו שחייב לעצמו כל המכות אשר קיבל, ולרמוז על רשעו ופשעו אמר המגיד "עבדים היינו לפרעה". והמניעה השניה היא מצד ארץ מצרים עצמה, כי להיותה מקור הצרות נקראת מצרים מלשון מֵיצַרִים. וכבר זכרו חז"ל שהייתה מערכת ארץ מצרים מחייבת שלא יצא עבד משם לעולם, עד שאמרו שהיו עבדים נסגרים שם בכ"ד מפתחות שלא היו יכולים לצאת משם, ולכן נזכרה מצרים חמישים פעמים בתורה, שהם כנגד חמישים שערי בינה שנמסרו למשה ובהם פתח המסגרות והוציאם ממצרים.  ולזה כיוון באומרו "ויוציאנו ה' אלוהינו משם ביד חזקה ובזרוע נטויה", רוצה לומר ששדד המערכות העליונות ושינה הטבעיות בידו החזקה על השמים ובזרוע הנטויה על השרים העליונים המניעים אותם, וכמו שאמר "וּבְכָל אֱלֹהֵי מִצְרַיִם אֶעֱשֶׂה שְׁפָטִים" (שם י"ב, י"ב), וכן כתב הראב"ע כי לפי המערכה העליונה לא היו ישראל יוצאים ממצרים,והנה מה שאמר: ואילו לא הוציא הקב"ה את אבותינו ממצרים' וכו' כיוון בזה שתי כוונות, האחת שאם יאמר אדם הנה הפעל הזה לאכול ולשתות בהסבה יאות לאותם שיצאו מעבדות לחירות, אבל אנחנו העומדים בגלות מה לנו המעשה הזה? והוא משיב על זה: לומר אילו אבותינו לא היו יוצאים אז גם אנחנו היינו שם עתה עבדים, ולכן הוא כאילו אנחנו יצאנו משם, ומוכרחים אנו לעשות כמעשיהם. והכוונה השניה היא שרצה המגיד להודיע מעלת היציאה ממצרים לפי שהיציאה משם לא היתה אפשר כי אם באחד משלושה דברים: א') אם בכוחם וזרועם של ישראל שיתגברו על המצריים, וכמו שהם עצמם חששו לזה באומרם 'הבה נתחכמה לו פן ירבה והיה כי תקראנה מלחמה ונוסף גם הוא על שונאינו ועלה מן הארץ', ב') או שיצאו בלשון פרעה ורשותו כמו שעשה כורש ליהודים אשר בבבל, ג') או בכוח הקב"ה שהוא כל יכול על המערכות כולן ועל מלכי האדמה באדמה.  ואמר המגיד כי אילו לא הוציא הקב"ה את אבותינו ממצרים שהיא היתה הסיבה האמיתית אז לא היה אפשר שיצאו באחת משתי הסיבות האחרות, אם מגבורת העם או מרצון פרעה, כי העם בני ישראל אף על פי שהיו כמות מופלג עד אין חקר לא היו יכולים להשתחרר עצמם, כי לא היה להם לב אמיץ כי נולדו בתוך השיעבוד והעינוי, לכן אמר "עדיין אנו ובנינו ובני בנינו משועבדים היינו לפרעה", רצה לומר לא די לשש מאות אלף רגלי העם אשר לא יכלו לצאת, אבל גם הם ובניהם ובני בניהם עד עולם לא היו יכולים לצאת בגבורתם, ותמיד היו עבדים נרצעים. כי היוכלו אלף אלפי אלפים עדרי צאן להתגבר בבוא עליהם אריה? וגם בסיבה השניה שהוא רצון פרעה אין ספק שלא היו יוצאים, כי אנו יודעים את רשעתו וגם הארץ לא הייתה מוכנה לכך כפי מערכה השמיימית, ועל זה אמר "עדיין אנו ובנינו ובני בנינו משועבדים היינו לפרעה במצרים". לא נשאר לפי זה אלא צד ואופן אחר ליציאתם ממצרים כי אם היכולת האלוהי ורצונו. ולכן לא בא המאמר הזה להגיד שחייב אדם להראות עצמו כאילו הוא יצא ממצרים, כי אין זה מקומו כי אם בסוף ההגדה, אבל כאן בא להודיע גבורות השם יתברך וחסדו ביציאת מצרים. ובזה האופן המאמר הזה הוא מענין התשובה, ויצדק מאמרו "עדיין אנו ובנינו ובני בנינו" וכו'. והותרו בזה הספקות אשר בשערים שמיני ותשיעי וי"א. והנה לקח המגיד הטענה הנכבדת הזאת משה שאמר הקב"ה למשה בתחילת שליחותו, "וַאֲנִי יָדַעְתִּי כִּי לֹא יִתֵּן אֶתְכֶם מֶלֶךְ מִצְרַיִם לַהֲלֹךְ וְלֹא בְּיָד חֲזָקָה, וְשָׁלַחְתִּי אֶת יָדִי וְהִכֵּיתִי אֶת מִצְרַיִם בְּכֹל נִפְלְאֹתַי אֲשֶׁר אֶעֱשֶׂה בְּקִרְבּוֹ וְאַחֲרֵי כֵן יְשַׁלַּח אֶתְכֶם" (שמות ג', י"ט – כ'). וכבר נתקשה הפסוק הזה על המפרשים, לפי שישראל יצאו ממצרים באמצעות מכת בכורות שנקראת יד חזקה ואיך יאמר הכתוב "ולא ביד חזקה"?וכתב רש"י כל עוד שאין אני מודיעם ידי החזקה לא יתן אתכם להלוך, דבר אחר ולא בשביל שידו של פרעה חזקה. אבל אמיתת הכתוב נראה לי שבא לומר אני ידעתי שלא יתן אתכם מלך מצרים להלוך ברצונו וחפצו ורשותו, וגם לא תהיה היציאה משם ביד חזקה וגבורה שלכם שתחשבו שתתגברו על המצרים ותעלו מן הארץ, אבל תהיה היציאה בידי העליונה ובכוח ה' שאשלח את ידי והכיתי את מצרים, רצה לומר את הארץ ואת אלוהיה, לא בדרך הטבע כי אם בכל נפלאותי אשר אעשה בקרבו ואחרי כן ישלח אתכם.אמנם אם נשאל מה הרווחנו ביציאת מצרים ובגאולה משם אם אנחנו כעת בגלות? הנה בלי ספק קנינו בזה שלושה מעלות עליונות שאנחנו גם היום זוכים בהן.האחת, שנודעה לכל באי עולם מעלת האומה לפני ה' שהשחית וביטל הסדר הטבעי וכוחות השמיימים והשדים העליונים בעבור ישראל אשר לא עשה כן לכל גוי.והשנית, הכבוד הגדול שרנו אבותינו ביציאתם ממצרים ביד רמה ובירושת הארץ, והמעלה והכבוד אשר רכשו בה בישיבתם עליה ימים רבים, ומעלת האבות וכבודם היא ירושה רבה לבאים אחריהם.ושלישית שאם לא יצאנו ממצרים לא באנו לפני הר סיני ולא קיבלנו תורה ומצוות ולא שרתה שכינה ביננו ולא היינו עם סגולה לאלוהים, ורק על ידי היציאה משם זכינו לכל אלה והשגחת ה' דבקה עמנו שהיא התכלית האחרון ושלמות עליון אשר לנו, והשארת הנפש ושלמות הרוחני הנמשך אחריו.ואף על פי שאנחנו בגלות, אנו זוכים בהם, וכנגדם אמר "אפילו כולנו חכמים כולנו נבונים כולנו יודעים את התורה", רמז ב"חכמים" היודעים דרכי הטבע ומערכות העליונות, ורמז ב"נבונים" המבינים ענייני המעלות והכבוד, וכנגד שלמות התורנית הנפשית אמר "כולנו יודעים את התורה". לכן על אלה ואלה מצווה רבה לספר ביציאת מצרים, שהיא מקור הטובות והתחלה וסיבה לכל המעלות והשלמיות. ולכן הזהירה התורה הרבה פעמים לספר ביציאת מצרים, ונאמר בשאלת הבנים והגדה וגו'. והנוסחא האמיתית היא "וכל המרבה לספר ביציאת מצרים הרי זה משובח", כי הסיפור הוא מצווה והמרבה בו הרי זה משובח.והותרו בזה שאמרתי הספקות אשר נזכרו בשער העשירי והי"ב.}%endcomment
\newsection{מעשה שהיה בבני ברק}
\hebeng{מַעֲשֶׂה בְּרַבִּי אֱלִיעֶזֶר וְרַבִּי יְהוֹשֻׁעַ וְרַבִּי אֶלְעָזָר בֶּן־עֲזַרְיָה וְרַבִּי עֲקִיבָא וְרַבִּי טַרְפוֹן שֶׁהָיוּ מְסֻבִּין בִּבְנֵי־בְרַק וְהָיוּ מְסַפְּרִים בִּיצִיאַת מִצְרַיִם כָּל־אוֹתוֹ הַלַּיְלָה, עַד שֶׁבָּאוּ תַלְמִידֵיהֶם וְאָמְרוּ לָהֶם רַבּוֹתֵינוּ הִגִּיעַ זְמַן קְרִיאַת שְׁמַע שֶׁל שַׁחֲרִית. }{It happened once {[on Pesach]} that Rabbi Eliezer, Rabbi Yehoshua, Rabbi Elazar ben Azariah, Rabbi Akiva and Rabbi Tarfon were reclining in Bnei Brak and were telling the story of the exodus from Egypt that whole night, until their students came and said to them, "The time of {[reciting]} the morning Shema has arrived."}%
\commenta{\textrm{\textbf{מעשה ברבי אליעזר וכו׳}} הנה המשך כל האגדה מיוסד על לשונות התורה ודרשות חז״ל, ולא על סיפורי מעשיות, וצריך באור לאיזו כונה בא סיפור זה. אך הבאור פשוט, דסמוך בזה למעלה בא חידוש שני דברים, האחד, כי גם חכמים היודעים ובקיאים בכל ענין יציאת מצרים אעפ״י כן מצוה שיספרו עתה בלילה הזה, והשני — כי לא די בספור הקבוע, אך כל חכם המבין לחדש מדעתו בענין זה מצוה להוסיף ולחדש. ועל זה מביא לראי׳ או למקור מקבוצת חכמים גדולים, שאעפ״י שלכל אחד מהם הי׳ ידוע כל ענין יצ״מ, אעפ״י כך נתקבצו כולם לתכלית הסיפור בלילה הזה, ועוד זאת, כי בהתעסקם בענין זה כל הלילה עד שפסקום התלמידים — בודאי המציאו מה לחדש ולדרוש בענין זה, ומתבאר, וכל המרבה לספר משובח.}%endcomment%
\commentb{\textrm{\textbf{תשובה לשער י"ג}}\textrm{\textbf{מעשה ברבי אליעזר ורבי יהושע וכו'.}}המגיד הביא ראיה כי המרבה לספר ביציאת מצרים הוא המשובח, אפילו לחכמים ולנבונים וליודעים את התורה כולה, מאותה עובדא שאירע לחמישה גדולי הדור וחכמי ישראל, שבליל פסח היו מסובין בבני ברק ועם כל חכמתם היו מספרין ביציאת מצרים כל אותו הלילה. וזה היה אחר האכילה, דאי קודם האכילה הא אמרינן חוטפין מצה בליל פסח בשביל שלא יישנו בתינוקות, אלא אחר האכילה היו מספרין ביציאת מצרים עד שעלה עמוד השחר וראום תלמידיהם לקריאת שמע. כי היו כל כך מתבודדים בסיפוריהם ובמשא ומתן שהיה בעיניהם שגם באור הבוקר לא היו מרגישים, עד שתלמידיהם העירום לעניין קריאת שמע שמצוותה עם הנץ החמה.אמנם קשה להבין הלשון "שהיו מסובין בבני ברק", יש מפרשים שהיו נסמכים לאכול את הפסח עם תלמידיהם שנקראו "בנים" בעיר אחת ושמה ברק. אבל זה בלתי מתיישב אצלי, כי היה ראוי לומר שהיו נסמכין בבתי בניהם או תלמידיהם בברק ולא שיאמר מסובין בבני ברק. ועוד מפרשים כי חמשת החכמים האלה היו מתאכסנים עם בני אותה העיר לאכול בימי חג הפסח, ואם כדבריהם היו החכמים בין בני ברק, או התלמידים בברק בבתים שונים ולא היו כולם באגודה אחת מספרין ביציאת מצרים כל אותו הלילה לשיצטרכו התלמידים לומר אליהם יחד הגיע זמן קריאת שמע. ועוד נאמר שהיו אוכלים בבתי תלמידיהם היאך יאמר עד \textrm{\textbf{שבאו}} תלמידיהם, הלוא היו החכמים בבתי התלמידים ומאין באו? ועוד קשה שרבי אלעזר בן עזריה היה עשיר וגדול הגדור ונשיא בישראל ואיך יהיה נסמך על שולחן תלמידיו לאכול הפסח? וכל זה מוכח שבני ברק אינם אנשי או תלמידי העיר ברק, אבל הם כלים יפים שהיו מסובין עליהם בליל הפסח בדת ובהלכה, ולפי שאמר למעלה "והלילה הזה כולנו מסובין" סיפר כאן שחמישה גאוני עולם הללו היו אוכלים יחד בליל הנשיא רבי אלעזר בן עזריה, ושם שמו לפניהם לכבוד ולתפארת הכלים ההם שנקראו "בני ברק" כי בנים יאמר על כל דבר בניין ומעשה, כמו "בין לילה היה ובין לילה אבד", עד שהסעיף יקרא בן, כמו "ועל בן אימצת לך" (תהלים פ', ט"ז), "בן פורת יוסף" שפירשו המדקדקים לשון סעיף פורה וגידול, והברק יאמר על מראה המשי המבריק לעיניים, כי על הכלים האלה היו מסובין. יהיה מה שיהיה. הנה למדנו מזה שהיו מספרין ביציאת מצרים כל אותו הלילה, וזו ראיה שהמרבה בסיפורו הרי זה משובח.האמנם מה ראו השלמים האלה לספר בזה כל הלילה ולהדיר שינה מעיניהם? הנה הוא לפי שהלילה הזה ליל שימורים הוא לה', וישראל לא היו ישנים כלל באותו הלילה כשיצאו ממצרים, כי חלק הראשון מהלילה התעסקו ביציאה ולכן לא נתנו כל הלילה תנומה לעפעפיהם. ולפי שחייב אדם להראות את עצמו כאילו הוא יצא ממצרים, לכן ראו הקדושים האלה לעשות כמעשיהם שמיד בתחילת הלילה התעסקו במצוות המצה והמרור וזכר הפסח כמו שעשו אבותיהם במצרים, ואחר כך בכל שאר הלילה היו מספרים ביציאה, ובזה הראו את עצמם כאילו הם יצאו.והותר בזה הספק אשר בשער י"ג.}%endcomment
\hebeng{אָמַר רַבִּי אֶלְעָזָר בֶּן־עֲזַרְיָה הֲרֵי אֲנִי כְּבֶן שִׁבְעִים שָׁנָה וְלֹא זָכִיתִי שֶׁתֵּאָמֵר יְצִיאַת מִצְרַיִם בַּלֵּילוֹת עַד שֶׁדְּרָשָׁהּ בֶּן זוֹמָא, שֶׁנֶּאֱמַר, לְמַעַן תִּזְכֹּר אֶת יוֹם צֵאתְךָ מֵאֶרֶץ מִצְרַיִם כֹּל יְמֵי חַיֶּיךָ. יְמֵי חַיֶּיךָ הַיָּמִים. כֹּל יְמֵי חַיֶּיךָ הַלֵּילוֹת. וַחֲכָמִים אוֹמְרִים יְמֵי חַיֶּיךָ הָעוֹלָם הַזֶּה. כֹּל יְמֵי חַיֶּיךָ לְהָבִיא לִימוֹת הַמָּשִׁיחַ:}{Rabbi Elazar ben Azariah said, "Behold I am like a man of seventy years and I have not merited {[to understand why]} the exodus from Egypt should be said at night until Ben Zoma explicated it, as it is stated (Deuteronomy 16:3), 'In order that you remember the day of your going out from the land of Egypt all the days of your life;' 'the days of your life' {[indicates that the remembrance be invoked during]} the days, '\textit{all} the days of your life' {[indicates that the remembrance be invoked also during]} the nights." But the Sages say, "'the days of your life' {[indicates that the remembrance be invoked in]} this world, '\textit{all} the days of your life' {[indicates that the remembrance be invoked also]} in the days of the Messiah."}%
\commenta{\textrm{\textbf{ולא זכיתי שתאמר יציאת מצרים בלילות עד שדרשה בן זומא וכו'}} הלשון לא זכיתי שתאמר אינו מבואר ברחבה, כי מה שייך זכות לזה, ומי עיכב בעדו לספר ביציאת מצרים בלילות. ולכן קרוב לומר, דקיצור לשון בדבריו, ועיקר כוונתו לאמר ״לא זכיתי למצוא מקור בתורה לזה שתאמר יציאת מצרים בלילות עד שזכה לזה בן זומא״, כדמפרש. וכלל כונת דבריו בזה להורות זכות מציאות דבר נעלם בתורה, ואמר ברגש, הן אני כבן שבעים שנה וכבר הספקתי לקנות הרבה תורה. ואעפ״י כן לא זכיתי למצוא מקור לענין זה, עד שזכה לזה איש צעיר לימים, אשר גם לא נודע בשמו, ונקרא על שם אביו, והנה זה זכות יתירה. ובבאורנו לפרקי אבות פרק ד׳ משנה א׳ נבאר עוד ענין מאמר זה של רבי אלעזר בן עזריה, וכן יתבאר שם טעם קריאת שם בן זומא על שם אביו ולא בשמו העצמי, עי״ש.\textrm{\textbf{עד שדרשה בן זומא, שנאמר, למען תזכור את יום צאתך מארץ מצרים כל ימי חייך, ימי חייך הימים כל ימי חייך הלילות}} לפלא, כי כל כך בקל קיבל ראב״ע דרשה זו דבן זומא, בעוד שיחד עם  דרשה זו במשנה מבואר, דחכמים פליגי על דרשה זו ומבארים כל ימי חייך ימי חייך העולם הזה, כל ימי חייך להביא לימות המשיח, כלומר, כי גם בימות המשיח יזכרו ענין יציאת מצרים (הריבותא בזה יתבאר בסמוך), ואם כן מה אולמי׳ דבן זומא בדרשתו שלו. ויתר מזה, כי לפי המתבאר בגמרא הודה בן זומא לדרשת החכמים, כי זה לשון הגמרא שם. תניא, אמר להם בן זומא לחכמים, וכי מזכירין יציאת מצרים לימות המשיח, והלא כבר נאמר (ירמיהו כ״ג:ז׳) הנה ימים באים נאום ה׳ ולא יאמרו עוד חי ה׳ אשר העלה את בני ישראל מארץ מצרים כי אם חי ה׳ אשר העלה ואשר הביא את זרע בית ישראל מארץ צפונה ומכל הארצות וכו׳, אמרו לו, לא שתעקר יציאת מצרים ממקומה אלא שתהא שעבוד מלכיות עיקר ויציאת מצרים טפל לו, כיוצא בו אתה אומר, לא יקרא שמך עוד יעקב. כי אם ישראל יהיה שמך, (פ׳ וישלח, ל״ה י׳) לא שיעקר שם יעקב ממקומו אלא ישראל עיקר ויעקב טפל לו, ע״כ. והנה לא מצינו שהשיב בן זומא על זה, ומדשתיק אודיי אודי, וא״כ בודאי אין לבכר דרשתו שלו על דרשת חכמים, ואיך זה קיבל אותה ראב״ע כל כך בפשטות ובודאות. אך באמת פשטות לשון הפסוק שהביא בן זומא לראי׳ לדבריו כי אם אשר העלה... מארץ צפונה וכו', לשון זה ״כי אם״ מורה על עקירה מוחלטת מהלשון או מן הענין הקודם, ואפילו משמוש טפלי כמו בפרשה וישב (ל״ט ו׳) ולא ידע אתו מאומה כי אם הלחם, ושם (ט׳) ולא חשך ממני כי אם אותך, ובשמואל א׳ (כ״א ה׳) אין לחם חול אל תחת ידי כי אם לחם קודש, ובמלכים ב׳ (ה׳ י״ח) כי לא יעשה עוד עבדך עולה וזבח לאלהים אחרים כי אם לה׳, בכולם המובן מן ״כי אם״ אך כן ולא אחרת, אף לא כטפל. והנה אם כן מוכרחים הדברים שאמר בן זומא לחכמים כי לימות המשיח לא יזכרו עוד יציאת מצרים, כהלשון שבפסוק ״ולא יאמרו עוד חי ה׳ אשר העלה את בני ישראל מארץ מצרים כי אם חי ה׳ אשר העלה ואשר הביא את זרע בית ישראל מארץ צפונה״, והלשון ״כי אם״ מורה אף לא כטפל. וככל הלשונות ״כי אם״ שהבאנו. והראי׳ שהביאו חכמים משם יעקב, כי אעפ״י שאמר לו ה׳ כי אם ישראל יהיה שמך ואעפ״י כן לא נעקר כולו שם יעקב אך נעשה כטפל לשם ישראל — ראי׳ זו מופרכת בגמרא ברכות שם בענין אחר, כי סמוך לדברי חכמים ובן זומא שם איתא מאמר בענין אחר, תני בר קפרא, כל הקורא לאברהם אברם עובר בעשה, שנאמר והיה שמך אברהם (פ׳ לך, י״ז), ופריך, אלא מעתה הקורא ליעקב יעקב הכי נמי (אחרי דכתיב לא יקרא שמך עוד יעקב כי אם ישראל יהיה שמך), ומשני, שאני התם, דהדר אהדדי׳ קרא, דכתיב, ויאמר אלהים לישראל במראית הלילה ויאמר יעקב יעקב, ע״כ. והנה אנו אין לנו לחקור אחר דרכי ה׳ למה חזר לקרותו יעקב אחר הודעתו ששמו יהי׳ ישראל, אבל עכ״פ אין מזה ראי׳ לדעלמא, שהלשון ״כי אם״ מורה רק על הויה עקרית ולא על הטפל כמו שרצו חכמים לפרש, יען כי בענין זה רק לה׳ פתרונים. ויותר מזה כי בכל המקומות שבא לשון ״כי אם״ כמו שהבאנו בכולם מורה על עקירה מוחלטת, וגם על עקירה טפלה, כמו שבארנו. ועל כל זה לא הי׳ לבן זומא להשיב על הראי׳ שהביאו חכמים משם יעקב אחרי שהגמרא עצמה פירשה ענין זה, ואשר לפיה אין כל ראי׳ להשגתם על הנחת בן זומא לדעתו. ומתוך כל זה קיים רבי אלעזר בן עזריה את דרשתו של בן זומא בשויון נפש ומצאה רצויה ונוחה למה שרצה לדעת (על חיוב ספור יציאת מצרים בלילות).}%endcomment%
\commentb{\textrm{\textbf{תשובות לשערים י"ד - י"ח}}\textrm{\textbf{אמר רבי אלעזר בן עזריה הרי אני כבן שבעים שנה.}}זו הראיה השניה שהביא המגיד להוכיח שמצווה לספר ביציאת מצרים. מאותה משנה שאמר רבי אלעזר בן עזריה באותו היום שמנוהו נשיא, ועם היותו בחור בשנים י"ח שנה כדאיתא בירושלמי דברכות ואמר" הרי אני כבן שבעים שנה", להגיד שעם היות שגדלה חוכמתי עד שהייתה בבקיאותו והקפתו בדרכים התוריים כאילו היה בן שבעים שנה, עם כל זה לא הגיעה מעלתו למצוא ראיה לנצח חבריו ולא הוכיח שתאמר יציאת מצרים בלילות עד שדרשה בן זומא והוא שמעון זן זומא, וכבר ידעת שבן זומא זה היה אומר כשהיה רואה אוכלסי ישראל, "ברוך שברא כל אלה לשמשני", להגיש שזהו התכלית שבמין האנושי, אבל לא היה המאמר הזה ממנו דרך נאה וגאון ודרך רע חלילה, כי אם לעורר ליבות התלמידים ולהלהיבם היה אומר כן. והנה בן זומא היה קטן בשנים ולכן לא נסמך ונקרא המיר על שם אביו בן זומא, וכן היה עניין בן עזאי. ולפי זה היה מאמר רבי אלעזר בן עזריה שעם היותו בחור בשנים, הנה בחמלת ה' עליו נעשה כבן שבעים שנה בשלמות תורתו מפני כבוד הנשיאות, והוא היה מתלונן שלא זכה למה שזכה בן זומא שהיה נער ממנו ולא נעשה לו אותו נס, ולכן לא קראו בשמו שמעון אלא בן זומא להורות שהיה ילד מסכן וחכם. אמנם אין ספק שכוונת דרשתו היא להוכיח שתאמר יציאת מצרים בלילות רצה לומר בכל הלילות תמיד, וכבר הוכיח שם בגמרא שהזכרון הזיה יהיה בזכרון פרשת ציצית שיש בה זכר ליציאת מצרים. ואף על פי שציצית אין מצוותה בלילה שנאמר ""וראיתם אותו", מכל מקום הוכיח בן זומא שראוי לומר אותה פרשה בקריאת שמע של לילה מפני שיש בה זכר ליציאת מצרים, והחסד הגדול ההוא נעשה בלילה, ולכן ראוי שיזכר בכל לילה, והוכיח מזה הפסוק "למען תזכור את יום צאתך מארץ מצרים כל ימי חייך", ואף כי הפסוק הזה נאמר בפרשת הפסח הנה ראה בן זומא שהוא בלתי אפשר שיפורש על מצוות הפסח, לפי שנאמר  \textrm{\textbf{"כל ימי חייך"}}, ואי אפשר לומר כן כי אם על דבר אשר יזכר בכל יום ויום, ולא על מצוות הפסח שהיא מדי שנה בשנה ואיך יאמר שבכל ימות השנה ידרוש ויזכור הלכות הפסח, הלוא אמרו חז"ל ש"שואלים ודורשים הלכות הפסח בפסח והלכות החג בחג" וכו' שלושים יום מקודם ולא בכל יום. לכן דרש בן זומא אותו פסוק על זכרון יציאת מצרים בלילות בקריאת שמע לפי שהוא דבר תמידי בכל יום. ומה שאמר "כי בחיפזון יצאת ממצרי" הוא טעם למצוות הפסח והמצה אשר הזכיר, ואמר עוד "למען תזכור את יום צאתך מארץ מצרים כל ימי חייך" כאמור הלוא ציוויתיך והזהרתיך על זכרון יציאת מצרים לא לבד בזמן הפסח אלא גם בכל יום ויום ביחד עם קריאת שמע.אמנם אם היה כתוב ימי חייך די היה ללמד שנזכור אותה מצווה ביום כי כן נאמר "חוקות עולם לדורותיכם" ולא נזכר שם "כל" ומפני שנאמר כאן \textrm{\textbf{"כל ימי חייך"}} יש בכלל הזה היום וגם הלילה שמחוייבים אנו לזכור יציאת מצרים. אבל החכמים לא קיבלו הדרשה הזאת, לפי שראו שהלילה הוא נמשך ליום שנאמר "ויהי ערב ויהי בוקר יום אחד", ואם כן הלילה מכלל היום הוא, וכל כ"ד שעות מעת לעת נקרא יום, ולכן דרשו "ימי חייך העולם הזה" שהוא עולם הגלות ו"כל" בא לרבות ימות המשיח, שגם בימים ההם לא ימוש סיפור יציאת מצרים מפינו.ועל זה אמר להם בן זומא לחכמים וכי מזכירים יציאת מצרים לימות המשיח? והלא כבר נאמר "הנה ימים באים נאום ה' לא יאמרו עוד חי ה' אשר העלה את בני ישראל מארץ מצרים כי אם חי ה' אשר העלה ואשר הביא את זרע בית ישראל מארץ צפון ומכל הארצות אשר הדיחם שם" (ירמיהו כ"ג, ז')? אמרו לו חכמים לא שתעקר יציאת מצרים ממקומה אלא שתהא שיעבוד מלכויות עיקר ויציאת מצרים טפלה לו. רצו בזה לומר שלא בא הכתוב להגיד שיתבטלו מועדי ה' ולא שיתחדשו מצוות אחרות, כי תורת ה' ומצוותיו הם נצחיות עדי עד ולא יקבלו תוספת ולא חסרון, ואמרו אלה המצוות מלמד שאין נביא ראשי לחדש דבר ומקרא מגילה לאו תוספת הוא. אלא שהייתה כוונת הנביא באומרו "לא יאמרו עוד" וגו' שלא תהיה אז יציאת מצרים עיקר הגאולות או היותר גדולה והרשומה שבהן, כמו שהיא אצלנו היום, כי אז יהיה קיבוץ גלויות יותר גדול ותהיה הגאולה שלמה ועיקרית בין הגאולות כולן, עד שיציאת מצרים תהיה טפלה לה בפי בריות בהיותה קטנה ממנה בכמות ובמעלה.והנה אף כי התשובה שהשיבו חכמים לבן זומא היא אמיתית מכל צד, בכל זאת נוכל אנחנו להשיב לזה בדרך אחר והוא שהנביא באומר "לא יאמרו עוד חי ה' אשר העלה" רצה לומר שבני אדם בדברם זה עם זה ובשבועיהם לקיים דבר שרגילים לומר "חי ה' אשר העלה", לפי שזו השבועה הגדולה אצלנו מכל השבועות, אבל לימות המשיח יאמרו "חי ה' אשר העלה ואשר הביא" וגו' לפי שהגאולה העתידה תהיה פליאה יותר גדולה ועצומה מיציאת מצרים. ואין זה סותר למה שאמרו שיזכרו יציאת מצרים לימות המשיח בקראנו פרשת ציצית ביום ובלילה, להיות המצווה הזאת קבועה ועומדת בתורה ולא תשתנה לעולם. והנביא מדבר רק משבועת העם ודבריהם המורגלים ביניהם שמזכירים יציאת מצרים ולעתיד יזכרו הגאולה האחרונה שהיא יותר פליאה.ועם היות מאמר רבי אלעזר בן עזריה נאמר על זכרון פרשת ציצית, בכל זאת הביא המגיד לראיה על סיפור יציאת מצרים בליל פסח כדי לבאר גודל מעלת המצווה הזאת, מדברי רבי אלעזר בן עזריה שהשתדל להוכיח המצווה לזכור יציאת מצרים בכל לילות השנה על פי דרשת בן זומא. ולדעת החכמים שאמרו שם לימות המשיח נתחייב בזכרון יציאת מצרים, כל שכן שאנו חייבים בכל הלילות בזמן הזה, ויותר ראוי שנספר יציאת מצרים בליל פסח שהוא הזמן המיוחד אליו. ולכן כל המרבה לספר ביציאת מצרים הרי זה משובח, והנה לא הוכיח זה רבי אלעזר בן עזריה מפסוק "ליל שימורים הוא לה'" לפי שאותו פסוק הוא בליל פסח, ור' אלעזר בן עזריה לא היה מדבר בו כי אם בכל הלילות במצוות ציצית כמו שכתבתי.ובזה הותרו הספקות שבאו בשער י"ד, ט"ו, ט"ז, י"ז ובשער י"ח דוק והשבח.}%endcomment
\newsection{כנגד ארבעה בנים}
\hebeng{בָּרוּךְ הַמָּקוֹם, בָּרוּךְ הוּא, בָּרוּךְ שֶׁנָּתַן תּוֹרָה לְעַמּוֹ יִשְׂרָאֵל, בָּרוּךְ הוּא. כְּנֶגֶד אַרְבָּעָה בָנִים דִּבְּרָה תוֹרָה: אֶחָד חָכָם, וְאֶחָד רָשָׁע, וְאֶחָד תָּם, וְאֶחָד שֶׁאֵינוֹ יוֹדֵעַ לִשְׁאוֹל. }{Blessed be the Place {[of all]}, Blessed be He; Blessed be the One who Gave the Torah to His people Israel, Blessed be He. Corresponding to four sons did the Torah speak; one {[who is]} wise, one {[who is]} evil, one who is innocent and one who doesn't know to ask.}%
\commenta{\textrm{\textbf{כנגד ארבעה בנים דברה תורה}} ודברי כולם נסמכים על לשונות התורה דברי החכם מבואר בפרשה ואתחנן (ו׳ כ׳), ודברי הרשע — בפ׳ בא (י״ב כ״ו), ודברי התם — שם (י״ג י״ד), ושל אינו יודע לשאול — שם (שם ח׳) מבלי שאלה. ועפ״י זה, שכל השאלות והתשובות סובבות הולכות בסגנון שאלה ותשובה מבנים לאבות ומאבות לבנים — עפ״י זה אפשר לכוין כונת הדברים ברש״י פרשה בא (י״ב כ״ז) בפסוק ויקד העם וישתחוו, ופירש״י (והוא ממכילתא) שהקידה והשתחויה היו על בשורת הבנים שיהיו להם, עכ״ל. ואין מבואר ענין בשורה זו ויחושה לענין מעשה הפסוק, שהפסוק הנזכר חותם כל ענין פרשת הפסח. ואפשר לומר, שהיו ישראל מצטערים ודואגים על ערכם הרוחני של בניהם, כי נראה, אשר בין ארבעה בנים נמצא רק אחד חכם, ויתרם רשע ותם ושאינו יודע לשאול, ואיזה דורות איפה אפשר עוד לצאת מהם. אך הנה קיי״ל, דאסור לאב להעביר נחלה מברא בישא לברא טבא, משום דאפשר שיצא ממנו (מן הברא בישא) בנים טובים (כתובות נ״ג א׳), ועפ״י רעיון זה ניחם אותם משה שלא ידאגו ובישר אותם שעוד יהיו להם בנים טובים. וזהו ענין בשורת בנים שכתב רש״י. ובמכילתא איתא בלשון הנחמה שיהיו להם בנים ובני בנים טובים, ונוסח זה מאמץ באורנו עפ״י הגמרא דכתובות שהבאנו. וחס מאוד על כי מלים אלה (ובני בנים) נשמטו בפירש״י. ואמנם כל זה הוא על דרך הדרש, ופשוטו של ענין הקידה והשתחויה זו היתה לאות קבלתם את דברי ה׳ ע״י משה ברגשי הכנעה ותודה. כמו ויקדו וישתחוו (ס״פ מקץ) ביחש אחי יוסף עם יוסף.\textrm{\textbf{אחד חכם ואחד רשע}} הנה לא מצינו לנגוד לשם רשע — שם חכם, וההיפך משם רשע הוא צדיק. אך אמנם מצינו, אם כי במספר מצומצם הוראת שם חכם במובן צדיק, כמו בתהלים (קי״ט צ״ח) מאויבי תחכמני מצותיך, שהמובן הוא שאחכים לקיים המצוות, וזה מדת ורצון צדיק. ובמשלי (ה׳ ל״ג) שמעו מוסר וחכמו, שהמובן הוא — וצדקו. ומה שבחר המגיד בשם חכם תחת שם צדיק, הוא מפני שבכלל, האיש הצדיק אין דרכו לשאול שאלות, הולך לתומו ועושה על דרך הלשון נעשה ונשמע. אך יש צדיק שגם בצדקתו אוהב לחקור ולבחון כל מה שרואה שומע, ובענין זה הוא גם חכם וצדיק, ויען כי כל ענין הדברים כאן לספר ולהבין בענינים המתיחשים ליציאת מצרים קבע השם חכם במובן צדיק והפכו רשע.\textrm{\textbf{ואחד תם}} בירושלמי פסחים פרק י׳ הלכה ד׳, הנוסח תחת ואחד תם — ואחד טיפש. ובאמת נראה נוסח זה ליותר מכוון, יען כי מסגנון הדברים נראה שקורא לו תם במובן גנאי, ובאמת מצינו לשם זה במובן שבח ומעלה, שהוא בעל נקיון נפש מכל פשע ועון והרהור רע, ואף גם הוא נלוה עם השם ״יושר״, תום ויושר (תהלים כ״ח י,), ועם שם ״צדק״, בצדקי ותומתי (שם ז׳ ט׳), יעקב אבינו מוכתר בשם איש תם (ר״פ תולדות), ומדרך התם שלא לשאול ולא לחקור, כמש״כ (פ׳ ראה) תמים תהיה עם ה׳, ובמשלי (כ״ח י׳) הולך בתומו. אבל הטיפש שענינו בעל לב כהה וכבד להבין דבר מעצמו, על דרך הלשון טפש כחלב לבם (תהלים קי״ט ע׳), ובישעיה (ו׳ י׳) השמן לב העם תרגומו טפש לבי׳, וזה הוא מפני כי השמנונית שעל הלב מטמטמת התפתחות רגשי השכל — ולכן דרכו רק לשאול, וניחא נוסח הירושלמי.}%endcomment%
\commentb{\textrm{\textbf{תשובות לשערים י"ט – כ"ג}}\textrm{\textbf{ברוך שנתן תורה לעמו ישראל ברוך הוא כנגד ארבעה בנים וכו'}}, המאמר הקצר הזה נבאר ענינו וקישור דבריו בפנים שונים: הפן הראשון, שהמגיד רצה להוכיח מה שהציע בראשונה שמצוה לספר ביציאת מצרים ושכל המרבה לספר בה הרי זה משובח, ולכן אחרי שהביא עליו ראיה מפעולת חכמים וגם ממאמר ר' אלעזר בן עזריה כמו שכתבתי, הביא עוד ראיה לזה מן התורה, שבענין יציאת מצרים דברה כנגד ארבעה בנים, ועשה בה שאלות ותשובות באופנים מתחלפים, כנגד החכם וכנגד הרשע וכנגד התם ואף כנגד מי שאינו יודע לשאול צוותה התורה לסדר לפניו הרבדים וידיעת הסיפורים, וכמו שנאמר "את פתח לו". ולפי זה כל המאמרים עד כאן ודרשות הארבע בנים הן כולן להוכיח הדבר הזה. וכשהשלים להוכיח על סיפור יציאת מצרים התחיל סגנון ההגדה בגנאי באומרו "מתחילה עובדי עבודה זרה" וכו' ואחריו השבח בדרשת "ארמי אובד אבי" וכו', ובזה האופן יבואו הדברים כולם כדרך ראוי ומסודר.והפן השני הוא שבפרק "ערבי פסחים" אמרו שיתחיל בגנות ויסיים בשבח, ונשאל שם מאי גנות? רב אמר "מתחילה עובדי עבודה זרה היו אבותינו" וכו' ושמואל אמר "עבדים היינו לפרעה במצרים" וכו', כי הייתה דעת שמואל שכל ענין היום הוא יציאת מצרים ולכן אין לנו להזכיר גנות אחר כי אם העבדות, רב סבר שאין גנות רק עבודה זרה שהיו אבותינו עובדים מתחילה, והשבח הוא שקירבם המקום לעבודתו והוציאם ממצרים ונתן להם תורתו שהיא תכלית השבח (פסחים קט"ז, א'). ואנחנו עבדינן כתרויהו, ולפי דרכם זה ראוי היה שהמגיד יעשה ראשונה בדרך שמואל שהתחיל בגנות המאמר "עבדים היינו לפרעה במצרים", שהגנות היא השעבוד, והשבח הוא הגאולה והיציאה משם, והנה באותו המאמר "עבדים היינו לפרעה" הזכיר השעבוד והגנאי, וסמך אליו מעשה החכמים ומאמר ר' אלעזר בן עזריה לתשלום ביאור מה שבא במאמר "עבדים היינו", ואחר כך התחיל בעניין השבח ממאמר "ברוך המקום שנתן תורה לעמו ישראל ברוך הוא" וכו', וזכר דרשות בארבעה בנים שבאו כולן בעניין היציאה והגאולה, והשלים עם זה דרך ההגדה לדעת שמואל. אחר כך נעתק המגיד לעשותה כדרך רב והתחיל בגנות "מתחילה עובדי עבודה זרה" ושאר המאמרים, וזכר השבח בדרשת ודוי הבכורים שהביא, כמו שיתבאר. ואנו שעושין כתרוייהו מסדרין מאמרי ההגדה מתחילה כדעת שמואל ואחר כך כדעת רב.הפן השלישי הוא שהמגיד הוכיח ראשונה שמצוה להרבות בסיפור יציאת מצרים מכוח אותן הראיות שהזכיר, ואחר כך ראה לבאר איך היה הסיפור הזה, אם כפי הפשט בלבד, ואם בביאור מצוות המצה והמרור ואכילת הפסח והחגיגה, ואמר שאין הכוונה בסיפור יציאת מצרים כפי פשט הכתובים ולא לענין הדין וביאור המצוות, כי אם בדרך הגדה, לפי שכנגד ארבעה בנים דיברה תורה, להודיע שהסיפור יהיה אם לחכם ויחכם עוד, ואם לרשע משום ענה כסיל כאוולתו, ואם לתם מתהלך לתומו צדיק ואם לשאינו יודע לשאול, וכולם בדרך הגדה. ולפי שבבן הרביעי הוא מסיים "את פתח לו שנאמר והגדת לבנך" וכו' שהכוונה שיפתח לו הדברים מתחילתם ועד סופם, לכן פתח סיפורו מ"תחילה עובדי עבודה זרה" וכו'.הרי לך כפי כל אחד משלושת אופני הדרכים האלה טעם סדר מאמרי ההגדה וכוונתם וקישורם כפי הראוי.העניין השני שראוי לבאר בזה, הוא למה אמר המגיד בדרשת הבנים "ברוך המקום שנתן תורה לישראל ברוך הוא" ולא אמר כן בשאר הדרשות ממאמרי ההגדה? ונוכל לתת טעמים שונים:הטעם הראשון -  לפי שאמר המגיד שהתורה האלוהית כוונה לתת לימוד לכל מדרגות בני אדם בעניין יציאת מצרים, אם חכם עיניו בראשו ואם רשע כרשעתו אשר טמן ואם תם יושב אהלים ואם מי שאינו יודע לשאול שהוא החלק ההמוני הכולל בעם ישראל. לכן נתן שבח והודאה להשם יתברך שעיניו על כל דרכי איש והוא הנותן ידיעה ושלמות לכל בשר, ומפני זה אמר המגיד בדרשה זו "ברוך המקום שנתן תורה לישראל ברוך הוא" מפני שכנגד ארבעה בנים דיברה תורה, כי בהיות לימודו כולל לכל כיתות בני האדם ולעשות כרצון איש ואיש יהיה ברוך ומבורך ומשובח ומפואר בלשון כל עבדיו, והוא על דרך המשורר "ברוך אתה ה' למדני חוקיך" (תהלים קי"ט, ב'), שברכו ושיבחו על למדו אותו את חוקיו.הטעם השני -  שהמגיד מצא בעניין יציאת מצרים בארבע מקומות בתורה שנזכר שם "בן", אם בפרשה "משכו וקחו לכם" שנאמר "והיה כי יאמרו בניכם" (שמות י"ב, כ"ו), ואחר כך בפרשת "קדש לי כל בכור" אמר "והגדת לבנך" (שמות י"ג, ח'), ובפרשת "והיה כי יביאך" אמר "כי ישאלך בנך מחר לאמר" (שם שם, י"ד) ובסוף סדר "ואתחנן" אמר "כי ישאלך בנך מחר לאמר" (דברים ו', כ'). ולפי זה נזכר שם "בן" בארבע הפרשיות האלה לא במקרה ועל דרך ההזדמן, כי אם בכוונה מכוונת כדי לדבר בעניין זה בארבעה מיני דיבורים מתחלפים, כפי הנושא המקבל אותו, והוא המורה על שלמות התורה ואלוהותה שנסדרו דברים באופן כך, שכללו כל האופנים שאפשר שידובר בהם יציאת מצרים. וקרה בזה לדברי האלוהיים על האנושיים כמו שקרה לדבר הטבעי על המלאכותי, והוא שהמלאכותי יכוון לתכלית אחד בלבד אשר אליו כונן הפועל אותו. אבל הדברים הטבעיים יש בהם תכליות ופעולות הרבה מתחלפות, מהם כפי החומד ומהם כפי הצורה, מהם כפי המורכב ומהם סגולות וטבעים נעלמים אשר לא נדע סיבתם ואופן המשכם כי אם על ידי הניסיון. וכן הדברים האנושיים כוננו לתכלית אחד מכוון, אמנם הדברים האלוהיים עם היותם מורים לכוונה אחת, כבר נכללו בה כוונות אחרות, וזה עניין אמרם "שבעים פנים לתורה", ואמרו בדעות המתחלפות "אלו ואלו דברי אלוהים חיים", כי יש בכולם כוונה של החכמה האלוהית. ולפי שזוהי שלמות עצומה בתורת ה', לכן אמר המגיד בדרשה הזאת "ברוך המקום שנתן תורה לישראל", רצה לומר שנתנה להם באופן כך מן השלמות שתכלול במאמר אחד כוונות מתחלפות. ועם היותה תורה אחת ומצווה אחת למדנו כפי סגנון המאמרים שכנגד ארבעה בנים דיברה תורה במצווה הזאת. והברכה שזכר בכאן, ענינה השבח וההלול, ולכן אמר "ברוך הוא", רצה לומר הוא אינו מצטרך לברכה ולשבח שנברכהו אנחנו כי הוא ברוך ומבורך מעצמו.והעניין השלישי שראוי לבאר בכאן הוא, מהו כוונת אמרו "ברוך \textrm{\textbf{המקום}}"? האם נאמר על השם יתברך ולמה זה תארו בשם "מקום", או אם אמרו על מקום אחד מיוחד? כי יש מי שפירשו על הר סיני שהוא היה מבחר המקומות ומשם קיבלו ישראל את התורה וכמו שאמרו במשנה "משה קיבל תורה מסיני" וכן אמרו בּיְלָמְדֵנוּ8ילמדנו הוא מדרש קדמון שנזכר כמה פעמים ב"ערוך וב"ילקוט". נחלקו חכמי הדור האם הוא מדרש תנחומא הנדפס או מדרש שאבד מאיתנו. התקבלה הדעה כי מדובר במדרש קדמון שאבד ומדרש תנחומא הוא קיצור שלו. בשעה שבא הקב"ה ליתן תורה שמעו תבור וכרמל והניחו מקומם ובאו להם, והקב"ה אמר להם "למה תרצדון הרים גבנונים", למה אתם רצים ומדיינים? בעלי מומים אתם! כעניין שנאמר "או גבן או דק" אבל ההר חמד אלהים לשבתו זה סיני!וכל זה רחוק מהכוונה האמתית במאמר הזה, שכנגד הקב"ה אמר "ברוך המקום", שנאמר "והניף ידו אל המקום ואסף המצורע" (מלכים ב', ה', י"א) וכן אמרו חז"ל בבראשית רבה "מעונה אלהי קדם, הוא מקומו של עולם ואין העולם מקומו, הסוס טפל לרוכב ואין הרוכב טפל לסוס". האמנם למה תארו את השם יתברך בשם "מקום"? הוא על דרך ההמשל, לפי שראו בדרך המקום סגולות יאותו על דרך הדמיון לאלוהינו יתברך; האחת, שהמתקומם ינוח במקומו הטבעי כשיגיע אליו ויתנועע ממגמתו כשיהיה נפרד ממנו. השנית, שהמקום הוא מקיף למקומם וכולל אותו. השלישי, שהמקום שווה למקומם. הרביעי, שנבדל בטבעו ממקומם. החמישי, שממנו יבחן בצד המעלה וממנו יבחן בצד המטה. וכמו שנתבאר כל זה בחכמה הטבעית. והחכמים הראשונים בחכמתם הכוללת העתיקו כל זה על דרך ההמשל והדמיון בהשם יתברך כי הוא \textrm{\textbf{המקום}} המשלים אותנו, עד שקראוהו חכמים נשמת העולם וצורתו, להיות סיבה בקיומו ושמירתו בהיותנו דבקים בו יתברך וחוסים תחת כנפיו חיינו במנוחה ושלמות; א) כי הוא באמת המנוחה והנחלה, וכשתפרד הנפש מדבקותו תבקש לשוב אל בית אביה ולהדבק בו אם לא יעכבנה מונע. ב) הוא אלוהינו יתברך המקיף בנו, כי הוא בוחן ליבות וכליות בידיעה פרטית מקפת, וכל דבר לא יכחד מן המלך. ג) הוא אלוהינו יתברך בהשגחתו הפרטית השוה אלינו, לפי שהשגחה הנפלאה ההיא לא תעבור מהאומה לזולתה, ולא יעלם ממנה דבר לכל פרטי האומה, ובזה האופן היתה השגחתו שוה אלינו, לא תחסר מלהשגיח דבר ולא תוסיף ולא תעדיף על אומה אחרת מאותה ההשגחה הפרטית. ד) הוא אלוהינו נבדל ממנו תכלית ההבדל, עד שמציאות יאמר עליו ועל כל זולתו בשיתוף השם הגמור, לא בענין ויחס כלל. ה) וכמו שבמקום ימצאו המעלה והמטה, כך בהשגחת השם יתברך תמצא המעלה והמטה, כי הוא בשמים ממעל ועל הארץ מתחת, המגביהי לשבת המשפילי לראות, וכדברי המשורר "כי רם ה' ושפל יראה" (תהלים קל"ח, ו'), ואמר "אם אסק שמים שם אתה ואציעה שאול הנך" (תהלים קל"ט, ח'). הנה בכל הבחינות האלה על דרך הדמוי הממשליי יתואר השם יתברך בשם "\textrm{\textbf{מקום}}". מצורף לזה מה שכבר ידעת, שמקום יאמר גם כן על המעלה כמו שכתב המורה פרק ה' מחלק ראשון, ולפי שמעלת מציאותו יתברך בלתי מושג ולא נודע לשום נמצא, עד שמלאכי השרת היו שואלים "איה מקום כבודו להעריצו" רצה לומר איזו מעלה היא מכבודו יתברך כדי שנעריצהו בה. לכן אמר המגיד, לבלתי הכשל בדבריו "ברוך המקום", רצה לומר ברוך המעלה המוחלטת אשר אין אנחנו משיגים אותה.העניין הרביעי, למה עשה המגיד חלוקת הבנים בזה הדרך אחד חכם ואחד רשע ואחד תם ואחד שאינו יודע לשאול? ומה טבע החומר שחייב והכריח החלוקה הזאת? כי הנה החכם אינו מקביל לההפכי הרשע ולא התם למי שאינו יודע לשאול? כמו שבא בשער הספקות. וכבר השתדלו המפרשים להסכים אלה ארבעה בנים עם ארבעה היסודות, ואחרים רצו להסכימם כנגד האבות אברהם יצחק ויעקב ודוד. ובא החכם על פי מידת החסד וכנגדו אברם, ורשע לפי מידת הדין וכנגדו יצחק ולכן יצא ממנו עשו, והתם כנגד יעקב שהיה איש תם, ושאינו יודע לשאול כנגד דוד שאמר "כי עני ואביון אני". אבל הרבדים האלה רחוקים ממני, וכפי הסברא הישרה נראה לי לתת בזה שלושה טעמים:האחד, שהמגיד עשה בזה חלוקה הכרחית, כי מי מאלה ארבעה בנים שזכרה התורה לא ימלט בכל אחד מהם אם שיהיה שואל או בלתי שואל, כי אין בין החיוב והשלילה אמצעי, ואם הוא שואל לא ימלט מאחד מן שלושה תכליות בשאלתו, אם להראות הבטתו וידיעתו, ואם להיות מספק ומערער בדבר מפני שנכנס רוח מינות ואפיקורסות כמקנטר, ואם שישאל לחסרון ידיעתו ויחפוץ לדעת חכמה ומוסר ולהבין אמרי בינה. והם החכם והרשע והתם. כי החכם עם היותו יודע אמיתת העניין בכל זאת ישאל שאלות כדי להראות העמים והשרים את עושר כבוד תורתו וחכמתו, והרשע אשר בקרב לכו לא יאמין במצוות לכן יקנטר עליהן, והתם ישאל כדי שישיבוהו ויודיעוהו דרך ישכון אור. שלושת אלה שואלים על פי שלושת תכליותם, והחלק הרביעי הוא בלתי שואל ולזה לא עשה בו תואר אחד כי אם בשלילה מוחלטת וקראו "ושאנו יודע לשאול", לפי שאם היה בלתי שואל מפני רוח רעה היה נכלל ברשע, וכמו שיתבאר אחר כך. ולפי זאת הבחינה נכנסו הארבעה בנים בחלוקה הכרחית מחוייבת.הטעם השני הוא שידיעת הדבר והכרחתו בשלמות תהיה בכל ארבעת סיבותיו: החומרית והצוריית והפועלת והתכליתית, וכבר יקרה שהספיק אחת מהנה או שתים או שלוש להשיב אל כתות השואלים בפי שלמות דעת השואל או קצורו. המשל בזה, אם ישאלך איש שוגה ופתי איך נעשה זה הדביר? ואתה תשיבהו על ידי עצים ואבנים שהיא הסיבה החומרית, והנה נתפייסה דעתו הגסה ויחשוב אפשרות מציאות הדביר למציאות חומרו בחיוב. אמנם מי שלא יתפייס בזה ויאמר הנני רואה הרבה פעמים עצים ואבנים ולא דביר, אז תשיבהו עם הסיבה הפועלת לפי שהיה שם פלוני הנגר שבנאו, וכפי קוצר דעתו יספיקו שתי הסיבות החומרית והפועלת לחיוב מציאות הדביר, אבל אם המצא ימצא אדם אחד יותר פיקח שגם בזה לא תתישב דעתו, אז תצטרך לומר לו צורך הבנין של הדביר שהוא למחסה ולמסתור, וכשיתחברו אצלו שלושת הסיבות: חומרית, פועלת ותכליתית, תנוח דעתו. אך איש לבוב לא ישקוט ולא ינוח גם באלה, כי יאמר עם כל זה שאמרת קשה לי, למה נעשה דביר ולא בנין אחר? ואז תצטרך לבאר לו גם הסבה הצוריית, ועל ידי התחברות ארבעת הסיבות האלה תהיה הידיעה נשלמת, ולפי שיש כתות בני אדם בזה המספר ארבעה, החכימה התורה האלוהית לדבר מהפלאים שנעשו ביציאת מצרים וסידרה אותם עם ארבע מדרגות של בני אדם, כי בפרשת החכם ביאר הארבע הסבות כולן ולכן נזכר בראשונה ובפרשת הרשע נזכרו רק שלושת הסיבות תכליתי חומרית ופועלת, ובפרשת התם נזכרו שתי סבות בלבד, חומרית ופועלת, ולפי שאינו יודע לשאול הספיקה התורה בלימוד הסיבה החומרית לבדה, וכמו שאוכיח כל זה בפרט בדרשת כל בן ובן מהם. הנה נתבאר מזה טעם חלוקת ארבעה בנים וסדר זכרונם בקדימה ואיחור שהיא לפי מספר הסיבות שנאמרו בהם.הטעם השלישי הוא שהייתה חלוקת הבנים הכרחית כל אחד עם זוגו, וזכר ראשונה החכם והרשע ואחריהם זכר הפכיהם, כי התם הוא הפך הרשע, שהרשע הוא המרשיע בדעתו וכוונתו והתם הוא נקי מהתאוות ומשולל מכל רשע, וכבר הרגילו חז"ל לקרוא תם לשור שאינו מזיק, ועל דרך "ויעקב איש תם", ושלמה אמר "תומת ישרים תנחם וסלף בוגדים ישרם". וכן החכם ושאינו יודע לשאול הפכיים, כי עניין החכם הוא להראות את חכמתו ולנתח שאלתו לנתחיה, כמו שאמר "מה העדות והחוקים והמשפטים אשר ציווה ה' אלוהינו אתכם", והפכו למי שאינו יודע לשאול, כי לחסרון דעתו וקוצר הבנתו לא ידע לשאול דבר, וכבר אמרו הפילוסופים שהשאלה היא חצי החכמה, לפי שהיא הגש הסתירה או הספק אשר בדבר. ולפי זה הארבעה בנים הם שני מינים הפכיים, החכם עם מי שאינו יודע לשאול והרשע עם התם. ונזכרו כדי מעלתם בחכמה, ראשונה החכם לכבוד תורתו, ואחריו הרשע שגם הוא חכם להרע, ואחריו התם לקוצר ידיעתו, ושאינו יודע לשאול באחרונה להיותו משולל בידיעה בהחלט. ובזה נתבארו הספקות אשר בשער י"ט, כ', כ"א, כ"ב ובשער כ"ג.}%endcomment
\hebeng{חָכָם מָה הוּא אוֹמֵר? מָה הָעֵדוֹת וְהַחֻקִּים וְהַמִּשְׁפָּטִים אֲשֶׁר צִוָּה ה׳ אֱלֹהֵינוּ אֶתְכֶם. וְאַף אַתָּה אֱמוֹר לוֹ כְּהִלְכוֹת הַפֶּסַח: אֵין מַפְטִירִין אַחַר הַפֶּסַח אֲפִיקוֹמָן:}{What does the wise {[son]} say? "'What are these testimonies, statutes and judgments that the Lord our God commanded you?' (Deuteronomy 6:20)" And accordingly you will say to him, as per the laws of the Pesach sacrifice, "We may not eat an afikoman {[a dessert or other foods eaten after the meal]} after {[we are finished eating]} the Pesach sacrifice (Mishnah Pesachim 10:8)." }%
\commenta{\textrm{\textbf{ואף אתה אמור לו כהלכות הפסח אין מפטירין אחר הפסח אפיקומן}} הרבה מפרשים טרחו לפרש ענין תשובה זו על שאלת החכם. ולמה תפס  המגיד דוקא פרט זה מענין אפיקומן מכל דיני הסדר. ועל דעתי הדבר פשוט, כי בכל המשניות מפרק עשירי ממס׳ פסחים באו כל דיני סדר הלילה ההוא (ליל חמשה עשר בניסן), והדין האחרון בהמשניות הוא זה, שאין מפטירין אחר הפסח אפיקומן. ואמרו בזה בסגנון עצה לאבי הבן החכם השואל, כי אחרי שהוא, הבן, רוצה ללמוד כל דיני סדר הפסח ילמד אותו אביו על סדר המשניות את כל הדינים אשר בהן, וילמד עד אחר הדין האחרון שהוא אין מפטירין אחר הפסח אפיקומן״ ואחר הלשון כהלכות הפסח חסרה המלה ״עד״. וכסמך לזה הוא סוף לשון התשובה להחכם (פ׳ ואתחנן, ו׳ כ״ד) ויצונו ה׳ לעשות את כל החקים האלה. והנה הוא לומד אתו בזה כל החוקים. ובכלל באור הלשון ״אין מפטירין אחר הפסח אפיקומן״ טרחו כמה מפרשים ולא העלו באור נכון, לא בהמשך הלשון ולא בבאור הענין. ולי נראה, דכונת הלשון לחלק הענין לשני זמנים, לזמן שהמקדש קיים ולזמן הזה, ואמר, שאין לגמור הסעודה באכילה ושתי׳ לאחר שאוכלים הפסח בזמן שביהמ״ק קיים, ולאחר שאוכלים האפיקומן בזה״ז הבא במקום פסח, אך שניהם גומרים הסעודה, כל אחד בזמנו, כפי המבואר, ובא הלשון מקוצר. ובשלמותו יהי׳ ״אין מפטירין אחר הפסח (בזמן הקרבת הפסח) ואחר אפיקומן (בזה״ז)״, כמבואר. והלשון מפטירין הוא מענין גמר, כמו שקורין למת ״נפטר״, והיינו שגמר ענינו עם החיים. ובאור זה קרוב לענין הנרצה שאסור לאכול ולשתות אחר האפיקומן. ואמנם מה שכתבו המפרשים והפוסקים בטעם איסור טעימת כל דבר אכילה ושתי׳ לאחר אכילת האפיקומן כדי שלא יפוג טעמו מפה ומגרון — טעם זה קשה מאוד, שהרי כידוע, לאחר האפיקומן שותים כוס שלישי (של ברהמ״ז) וכוס רביעי, ואת זה האחרון שותים כולו, למען שתחול עליו ברכת הגפן בשלמותה, ואם כן, כבר פג טעם האפיקומן ע״י שתי כוסות אלה. ולכן הטעם הנכון ממניעת טעימה לאחר האפיקומן הוא מפני שהוא בא תחת הפסח, וצריך להיות בדומה לו בדיניו, ושני הכוסות האחרונים אין נחשבים להפסק כיון שבאים למצוה לזכרון לשונות הגאולה, והוצאתי, והצלתי, וגאלתי, ולקחתי, כידוע.}%endcomment%
\commentb{\textrm{\textbf{תשובות לשערים כ"ה – כ"ו}}\textrm{\textbf{חכם מה הוא אומר מה העדות והחוקים והמשפטים וכו'.}}כבר אמרתי שעם היות פרשת החכם אחרונה בספר משנה תורה בסוף סדר ואתחנן, בכל זאת הזכיר אות ראשונה מפני מעלת חכמתו כי כבוד חכמים ינחלו, וגם מפני שלמות תשובתו שנזכרו בה כל ארבע הסיבות כמו שיתבאר להלן. ואין ספק שפשט המאמר "מה העדות והחוקים והמשפטים אשר ציוה ה' אלוהינו אתכם", שעל כל מצוות התורה ועדותיה ומשפטיה וחוקותיה הייתה השאלה. וכבר פירשתי אותה פרשה כפי פשוטה ב"מרכבת המשנה" אשר לי והוא פירוש ספר משנה תורה אשר עשיתי. אמנם המגיד חשב שהיה ראוי לדרוש אותה שאלה והפרשה כולה על עניין מצוות הפסח בפרט, לפי שתוכן השאלה וכוונתה נראה מדברי התשובה, וכיוון שהתורה השיבה שמה לאותה שאלה "ואמרת לבנך עבדים היינו לפרעה במצרים", מורה שהשאלה על מצות הפסח בפרט היתה. ומצא עוד ראיה אחרת לזה ממה שאמר בסוף התשובה "וצדקה תהיה לנו כי נשמור לעשות את כל המצווה הזאת לפני ה' אלוהינו כאשר ציוונו". כי מאשר אמר בלשון יחיד "את כל המצווה הזאת" מורה שהיא מצווה פרטית ושלא היה שואל על כל המצוות כי אם על מצוות הפסח בלבד. אמנם להיותו בן חכם רצה להראות את חכמתו באמרו שיש במצווה הזאת חלק שהוא עדות, כמו ענין המצוה שהיא זכר למהירות הגאולה והיציאה בחיפזון ממצרים, ועניין הפסח שהוא זכר למזל טלה שנלקח בלילה הזה, ועשה בו ה' שפטים ואת בתינו הציל. וגם עניין המרור שהוא זכרון שמררו המצריים את חייהם של ישראל בעבודה קשה, וכל זה הוא מחלק העדות. ויש במצווה הזאת גם כן מחלק החוקים שמה שציווה שבבית אחד יאכל "לא תוציאו מן הבשר" וגו', "ועצם לא תשברו בו" וגו' שכל זה הוא מהחוקים שאין טעמם נודע אלינו. וגם יש בזאת המצווה מהמשפטים והוא "כל עדת ישראל יעשו אותו", ו"כי יגור איתך גר ועשה פסח" וגו', "כל ערל לא יאכל בו" וגו', שכל זה משפט ישר, לפי שלא נעשה הנס כי אם לבני בריתו של אברהם אבינו והבנים בניו. ולפי זה הייתה שאלת הבן החכם על מצוות הפסח בכל חלקיה וכפי צורתה העצמית, והוא המורה על חכמתו בדיני התורה והמצוות. ובאמת ראה המגיד שבננו זה הוא חכם וירא וסר מרע, לפי שתחילת דבריו היא "אשר ציווה ה' אלוהינו אתכם" רצה לומר שהייתה שאלתו אחרי ההודאה שהמצווה היא אלוהית ומפי הגבורה נתנה לאבות, ולכן שאל על טעם חלקי, למה יש חלק ממנה מסוג העדות וחלק מסוג החוקים וחלק מסוג המשפטים. וההנה באה תשובה מכוונת לכל חלקי השאלה, כי על מה ששאל טעם חלק העדות השיבו "עבדים היינו לפרעה במצרים ויוציאנו ה' אלוהינו ממצרים ביד חזקה ויתן אותות ומופתים", רצה לומר שהיה המרור זכר לעבדות, והמצה זכר לחפזון הגאולה, והפסח רמז לאותות והמופתים הגדולים והרעים שנתן ה' לפרעה ובכל מצרים ובכל ארצו שהיא מכת בכורות, ולזכרון ועדות האמתיות האלה היו העדות אשר במצוה הזאת. ולעניין חלק המשפטים השיבו "ואותנו הוציא משם", רצה לומר ראוי הוא ומשפט ישר שכל בן נכר וכל ערל לא יאכל בו ושתהיה המצוה פרטית לכל עדת ישראל בעלי ברית אברהם, לפי שאותנו הוציא משם ולא אומה אחרת. ולעניין החוקים, השיבו "ויצונו ה' לעשות אל כל החוקים האלה" שהם הדברים שבאו במצוה הזאת שלא ידענו טעמיהם, אבל תועלתם מבואר שהם לטוב לנו כל הימים, שהוא הטוב הנצחי המיוחס לנפש, להיותנו ביום הזה באורך החיים הגשמיים. וכדי שלא יחשוב שהשכר כולו הוא בשמירת החוקים ולא בשמירת העדות והמשפטים, עוד אמר "וּצְדָקָה תִּהְיֶה לָּנוּ כִּי נִשְׁמֹר לַעֲשׂוֹת אֶת כָּל הַמִּצְוָה הַזֹּאת לִפְנֵי ה' אֱלֹהֵינוּ כַּאֲשֶׁר צִוָּנוּ" (דברים ו', כ"ה) כלומר שהשכר יקובל על כללות המצוה בכל חלקיה, כי המורכב הוא יותר שלם על ידי חיבור כל אחד מפשוטיו. ובאופן זה ביארה התורה את המצוה בכל ארבעת סיבותיה: אם החומר באומרו "את כל המצוה הזאת", אם הצורה במה שפירט עדותיה ומשפטיה וחוקותיה, אם הפועל באומרו "ויצוונו ה'", ואם התכלית באומרו "לְטוֹב לָנוּ כָּל הַיָּמִים לְחַיֹּתֵנוּ כְּהַיּוֹם הַזֶּה" (דברים ו', כ"ד).אמנם מאמר המגיד "אף אתה אמור לו כהלכות הפסח" וכו' אין עניינו שלא יתן לבן החכם אותה התשובה המספקת שנתנה התורה, אלא יאמר כי בעבור שהבן החכם הזה התגאה בחכמתו לעשות חלוקים במצוה הזאת, לכן מלבד מה שהשיבתו התורה לבל יהיה חכם בעיניו, אף אתה אמור לו כהלכות הפסח, רצה לומר מלבד מה שכתוב בפרשה אמור לו אתה עוד שאר הלכות הפסח, עד הדבר האחרון שבו שהוא "אין מפטירין אחר הפסח אפיקומן", והיא משנה בערבי פסחים, ועניינה שאחרי אכילת הפסח, לא ישאלו עוד ולא יוציאו ויתנו לפניהם מיני מאכלים אחרים אלא תהי אכילתו באחרונה כדי שישאר טעמו בפה. והכל כדי שלא תשתכח יציאת מצרים אלא תקבע בלב לעולם. ומלת "מפטירין" מלשון יפטירו שהוא לשון דיבור. והכלל העולה מזה ידע וישכיל השואל עם כל חכמתו, שיש עוד לאלוה מלין ולמצוה הזאת כפלים לתושיה יותר משאלתו, על דרך "תן לחכם ויחכם עוד". ומה שפירשתי בזה יתיר הספקות אשר בשער כ"ה וכ"ו.}%endcomment
\hebeng{רָשָׁע מָה הוּא אוֹמֵר? מָה הָעֲבוֹדָה הַזּאֹת לָכֶם. לָכֶם – וְלֹא לוֹ. וּלְפִי שֶׁהוֹצִיא אֶת עַצְמוֹ מִן הַכְּלָל כָּפַר בְּעִקָּר. וְאַף אַתָּה הַקְהֵה אֶת שִׁנָּיו וֶאֱמוֹר לוֹ: ״בַּעֲבוּר זֶה עָשָׂה ה׳ לִי בְּצֵאתִי מִמִּצְרָיִם״. לִי וְלֹא־לוֹ. אִלּוּ הָיָה שָׁם, לֹא הָיָה נִגְאָל: }{What does the evil {[son]} say? "'What is this worship to you?' (Exodus 12:26)" 'To you' and not 'to him.' And since he excluded himself from the collective, he denied a principle {[of the Jewish faith]}. And accordingly, you will blunt his teeth and say to him, "'For the sake of this, did the Lord do {[this]} for \textit{me} in my going out of Egypt' (Exodus 13:8)." 'For me' and not 'for him.' If he had been there, he would not have been saved. }%
\commenta{\textrm{\textbf{רשע מה הוא אומר מה העבודה הזאת לכם}} לכאורה איז הבדל ממשי בין שאלת החכם לשאלת הרשע. כי החכם שאל גם כן מה העדות והחוקים והמשפטים אשר צוה ה׳ אתכם, והלשון ״אתכם״  קרוב ללשון ״לכם״ שבשאלת הרשע, ואם כן, מה רעה כל כך ראה דמגיד בשאלת הרשע. אך הבאור הוא, כי החכם לבד שזכר שם ה׳ וכלל עצמו עם כל ישראל בקבלת מלכות שמים בלשון ״ה׳ אלהינו״ (אשר צוה ה׳ אלהינו) — לבד זה הדגיש בלשון השאלה מה העדות... אשר צוה ה׳, והן הצוי הי׳ באמת רק להאבות, והם, הבנים לא שמעו אותו, כי היו עוד קטנים, ומכוון לשון השאלה אשר צוה ה׳ אלהינו אתכם, כמו שהי׳ בפועל. אבל הרשע. לא רק שלא זכר שם ה׳. אך גם הוא שואל על כלל העבודה על מה היא באה, וכלל העבודה בפועל שייד כמו להאבות כמו להבנים, ובכל זאת הדגיש לאמר לכם, ולא לנו, הרי הוציא את עצמו מן הכלל. ועיין עוד המשך לזה במאמר הבא. ודבר פלא הוא, כי במכילתא פ׳ בא בהעתק שאלת החכם הועתק לשון השאלה מה העדות והחקים והמשפטים אשר צוה ה׳ אלהינו אותנו תחת הלשון אתכם, והנה לפי זה הי׳ שאלת החכם מכוונת מאוד, אבל הן בפסוק כתוב אתכם, והגר״א הגיה וקבע ״אתכם״, אבל הן גם בירושלמי פסחים פ״י ה״ד הגירסא ״אותנו״, ודרוש להגיה גם שם, והמפרשים לא עמדו על זה.\textrm{\textbf{הקהה את שיניו}} הוראת הפעל קהה הוא קלקול והשחתה, כמו אם קהה הברזל (קהלת ה׳ י׳), ובירמיה (ל״א כ״ט) האוכל בוסר (פירות בלתי מבושלים די צרכם בגידולם) תקהינה שיניו, ובמשל קדמוני — אבות אכלו בוסר ושיני בנים תקהינה (יחזקאל י״ח ב׳), ומשל הוא לבנים שנענשים בעון אבותיהם. והנה לא נתבאר ענין לשון זה לתשובה על שאלת הרשע, מה יחש השינים וקלקולם לזה. ואפשר לומר, משום דכוונת הדברים בשאלתו ״מה העבודה הזאת לכם״ הוא, כי לדעתו, אחרי שכל תכליתו של סדר הלילה בזה וכל עניניו הוא לספר ביציאת מצרים, ואם כן, הלא אפשר לקרוא כל הענין בספר ובזה ישיגו את התכלית הנרצה, אבל למה כל המעשים וכל הענינים בפועל, למה פסח ולמה מצה ומרור וכל הפרטים המעשיים. ועל זה משיבים לו במשל, כי אם די לפטור כל מעשה בפועל בקריאה בספר, למה לך לטחון מאכלך בשיניך ולקלקלם בטחינתם (והשינים נקראו טוחנות, קהלת (י״ב נ׳) ובטלו הטוחנות) — אחרי שתוכל לקרוא בספר שיעור מהרכבת ומהפרדת כל מין אוכל (כימי), ובזה תפטר מלטחונם בשיניך. אך אתה לא תסתפק בזה, אך תרצה לאכול בפועל ממש, ובזה הלא תודה, כי לא די בקריאה בספר בלבד, כי לא ישביעו את הנפש, ואך דרוש להוציא את הדברים בפועל ממש, ובזה תוכל לדון בזה תשובה לשאלתך. וזהו כונת הלשון הקהה את שיניו, הביאה לו תשובה בסגנון מליצי במשל מקלקול השינים. וסגנון התשובה בכלל הוא על דרך הלשון ענה כסיל כאולתו (משלי כ״ז ה׳), כלומר, בסגנון דבריו שלו, וכמו שאמרו במס׳ ביצה (כ׳ ב׳), האי מאן דאמר לי׳ חבריה מילתא (בקפידא) יהדר לי׳ ממאי דאמר לי׳ הא, כלומר, בלשונו ובענינו. עיי״ש. וכעין סגנון תשובה כזה מצינו בתורה בריש פרשת קרח שפנה קרח למשה ולאהרן בלשון רב לכם (ט״ז ג׳) השיב לו משה ג״כ בלשון זה. רב לכם (שם פסוק ז׳). ובמס׳ סוטה (י׳ ב׳) כעין הערה זו, יהודה בהכר בישר (הכר נא הכתונת בנך היא) בהכר בישרוהו (הכר נא למי החותמת). ובב״ב (ט״ז א׳) איוב בסערה דיבר (שאמר אשר בסערה ישופנו (ט׳ י״ז) ובסערה השיבוהו (ויען ה׳ את איוב מן הסערה (ל״ח ח׳).\textrm{\textbf{אלו היה שם לא היה נגאל}} לכאורה זה פלא, דהא ביציאת ישראל ממצרים כתיב וגם ערב רב עלה אתם (פ׳ בא, י״ב ל״ח), ואיך אמר לא הי׳ נגאל. וצריך לומר, דהערב רב לא היו רשעים, אך ערבוב גרים מאומות, כמש״כ רש״י שם, ויצאו כדי להסתפח על אומת ישראל, אבל הרשעים מתו בשלשת ימי אפלה (במכת חושך), כמבואר במדרשים וברש״י פ׳ בא (י׳ כ״א). ודע דבענין שאלת הבן הרשע והתשובה לו יש לי שאלה משולשת ולדעתי כולן נמרצות. ראשית, כי בפסוק שבענין שאלה זו כתיב והיה כי יאמרו אליכם בניכם (פ׳ בא, י״ב כ״ו) בלשון רבים, והמגיד תפס בזה לשון יחיד, רשע מה הוא אומר. שנית, כי בפסוק באה התשובה לשאלה זו (מה העבודה הזאת לכם) ואמרתם זבח פסח הוא, והמגיד שם בפי האב התשובה בעבור זה עשה ה׳ לי, בעוד שבתורה באה תשובה זו לשאלת התם (שם י״ג ח׳). ושלישית, אם כן הוא, שלשון השאלה ״מה העבודה הזאת לכם״ היא אמנם שאלת רשע, למה באמת קבעה התורה שאלה זו בלשון רבים, כי יאמרו אליכם בניכם. כאלו מודיע שיקומו הרבה בנים רשעים לשאול שאלה רשעית, בעוד שבשאלות שלושת הבנים כתובות כולן בלשון יחיד (פ׳ בא י״ג ח' וי״ד) דפ׳ ואתחנן (ו׳ כ׳). ולמרבה הפלא אפשר לומר, כי אמנם בלשון השאלה מה העבודה הזאת לכם אין ברור שבה ודאי טמונה כפירה, יען כי אפשר שאשיגרא דלישנא הוא ולא דייק כל כך בלשונו, ועל זה רמזה התורה בלשון והיה כי יאמרו אליכם בניכם, בלשון רבים, להורות, כי אין כונת כפירה בלשון זה, אך חסרון דקדוק לשון, יען כי לא נחשדו ישראל על כפירה מפי בנים רבים בפרהסיא, ומפני זה באה גם התשובה רצויה, ואמרתם זבח פסח הוא. אך המגיד, ברצותו לכלול ארבע דעות שונות בארבעה בנים עפ״י שאלות שונות — תפס הלשון ״לכם״ במובן רשע, לכם ולא לו, ולכן שינה מלשון רבים, כי יאמרו אליכם בניכם ללשון יחיד, רשע מה הוא אומר, וזה הוא כדי שלא לעשות סתם בנים לרשעים, וממילא תפס בהתשובה לשון מקביל ומכוון לרשע, הקהה את שיניו, ושימש בלשון התשובה את הלשון בתורה בעבור זה עשה ה׳ לי, כדי לדייק לי ולא לו וכו׳. וראוי לדעת, כי סגנון זה מחליפות לשונות ומהעתקתם ממקום למקום למען מטרת דרשות — הוא מדרך הפייטנים וגם עורך ההגדה בכלל.}%endcomment%
\commentb{\textrm{\textbf{תשובות לשערים כ"ד, כ"ז – כ"ט}}\textrm{\textbf{"רשע מה הוא אומר מה העבודה הזאת לכם" וכו'.}} אחרי שהביא דרשת החכם הביא דרשת הרשע, לפי שגם הוא חכם להרע, כמו שאמרתי. וכדי שנעמוד בשלמות על כוונת המגיד ראוי להטיב העיון ראשונה בייחוס ענינו בפרשת "משכו וקחו לכם" כפי פשוטה. והנה ראיתי בה ספקות:הראשון שאתה תמצא בפרשת החדש שצווה ה' למשה בשלמות על פסח מצרים אם בלקיחתו ואם בשחיטתו, ואם באכילתו צלי ראשו על כרעיו ועל קרבו, ואכילתו על מצות ומרורים ובהיות מתניהם חגורים, ועם זה נתינת הדם על המשקוף ועל שתי המזוזות, ואיסור חמץ בכל שבעה ומצות המצה ושבעת ימי החג. אמנם כשבא משה ללמד לעם לא אמרה כולה כמו ששמע אותה מפי הגבורה, אבל צווה לזקנים על אכילת הפסח ושחיטתו ונתינת הדם על הדלת והמשקוף, ולא הגיד להם דבר אחד ממצה ומרורים ואיסור חמץ ושבעת ימי החג, וגם לא מאכילת הפסח כי אם מלקיחתו לבד ושחיטתו ונתינת הדם. והוא דבר מתמיה מאוד, למה לא זכר המצווה בשלמותה ובכל חלקיה וזכר קצתה והשמיט קצתה? והנה כתב הרמב"ן שקצר הכתוב כי בידוע שאמר להם משה הכל בפרט וכו', אבל לא נתן הרב הסבה למה היה כן.ושנית, מה ראה משה ללמד המצווה הזאת לזקנים ולא לכל ישראל, הלא ה' יתברך צווהו לדבר אל כל עדת בני ישראל, וראוי היה שיעשה כן כי לכל העם חובת המצווה ולמה דיברה ביחוד לזקנים? ומה הענין שאמר "מִשְׁכוּ וּקְחוּ לָכֶם"? (שמות י"ב, כ"א) וכתב הרב הנ"ל שקרא משה לזקנים והם אספו אליו כל העם ואז אמר משה לכל עדת ישראל "משכו וקחו לכם", כי לא היה הדיבור ההוא מיוחד לזקנים, והיא דעת ר' יאשיהו במכילתא, ור' יונתן אמר שהיה הדיבור ההוא לזקנים בלבד, ושהם ידברו אל העדה. והנה אתה רואה שלדעת שניהם העיקר חסר מן הספר.והשלישית, מה ענין "וַיֵּלְכוּ וַיַּעֲשׂוּ בְּנֵי יִשְׂרָאֵל כַּאֲשֶׁר צִוָּה ה' אֶת מֹשֶׁה" (שם שם, כ"ח), אם לזקנים אמר המצווה ראוי היה שיאמר "וילכו ויעשו הזקנים כאשר ציוה משה" ולא כל ישראל כאשר ציווה את משה ואת אהרון כי הם לא שמעו את המצווה לא מפי משה ולא מפי אהרון.והנראה לי בזה הוא שה' יתברך ביום ראש חודש ניסן ציווה למשה ולאהרון על חג הפסח ככל חוקיו ומשפטיו שהיא המצווה הראשונה שנצטוו בה ישראל, וציווה שיעשו אותה במצרים, והודיע להם גם כן שיהיה החג הזה להם חוקת עולם לדורותם, כמו שמפורש בפרשה. ואין ספק שמשה אמר מיד לכל עדת ישראל המצווה הזאת כמו שציווהו ה' ולא השמיט דבר ממנה ואין להתפלא על אשר לא סיפר הכתוב שאמר משה לישראל את המצוה הזאת, כי הנה בכל מצוות התורה תמצא "וידבר ה' אל משה דבר אל בני ישראל", ולא הגיד הכתוב שמשה סיפר המצוה לעם כמו ששמע. כי בידוע שלא יעבור משה על דברי ה' ובודאי ידבר לעם את אשר ציווהו עליו. ולכן נראה לי שלא באה פרשת "משכו וקחו לכם" לספר להם את המצווה שנאמר לו, אלא ראה משה רבינו עליו השלום לצוות לזקנים שהם יזדרזו במצווה זו, והם יתחילו ראשונה בלקיחת הפסח, וכן בשחיטתו ונתינת הדם על הפתח שהיו מהדברים היותר מסוכנים, ולכן חשש שיהיה הדבר בעיני העדה קשה לעשותו מפני הסכנה, כמו שאמר "הן נזבח את תועבת מצרים לעיניהם ולא יסקלונו".והיה זה לפי שכבר נתפרסם להם שמזל טלה השפיע עליהם יקר וגדולה באמצעות הכבוד והעילוי שהיו נוהגים במקנה ההוא שנקרא שמו עליו והוא בדמותו ובצלמו, והיו חושבים שהנוגע באחד אחוז מן הכשבים או מן העיזים כנוגע בבת עינו של אלוהיהם ואחת דתו להמית. ועתה כי יראו עולם הפוך ברגע קטן שהיהודים האומללים ישלחו ידיהם במקנה הנכבד והנעבד והמקבל השפע העליון, וכי עם בזוי ושסוי כזה ירימו יד בעדר אשר בחרו להם המצריים לאלוהים, ולוקח איש שה לבית אבות, ובכרעי מטתו יאסור אותו לעיניהם ויתנו במשמר שלושת ימים ואין לאל ידם להסירו משם, ויבוא יומו ושחט אותו לפניו בעוד השמש בצהריים, ודמיו צועקים על כל דלת ועל כל מזוזה ואין מושיע, וכדי שכל רואיו יכירוהו אינו נאכל אלא צלי, גלוי לכל בצורתו, ראשו על כרעיו ועל קרבו, ובדרך עראי וביזיון יכרו עליו חברים לאכול את בשרו ועצמותיו עד בלי השאיר מן הבשר מאומה עד אור הבוקר, והנותר שרוף ישרפו, ויש בכל אלה סכנה עצומה מהמצריים, וכדי שלא יפחדו המון בני ישראל מהמצריים בעשיית מצווה זו, התחכם אדון הנביאים לצוות לזקנים: "משכו וקחו לכם", רצה לומר אתם ראשי בני ישראל, התחילו בדבר! ולזה היה מאמרו אל הזקנים בלבד, כאומר עליכן המצווה הזאת בראשונה, אתם הגדולים והנשיאים, משכו וקחו \textrm{\textbf{לכם}}, רצה לומר לא על ידי שליח, אלא בעצמכם ובידכם תקחו לעיני השמש איש שה לבית אבות שה לבית לכל משפחותיכם, וביום הארבעה עשר ההיו אתם הראשונים לשחוט הפסח, ותקחו אגודת אזוב וטבלתם בדם והגעתם אל המשקוף וגו'.ולכן לא הזכיר דבר מהאכילה ולא מהמצוות ולא איסור החמץ או דבר אחר, כי כבר הגיד להם המצווה בשלמותה, ולא הוצרך עתה כי אם לזרז הזקנים הרודים בעם שהם יתחילו לעשות הדברים שיש בהם סכנה מפאת המצריים, וכדי שהמון העם יקחו לב ומהם יראו וכן יעשו, כי העם תמיד ילך אחרי הגדולים. והזהירם עוד שאף שצריכים להיות בוטח בה' לבלתי יאים מהמצרים בלקיחת השה ושחיטתו ונתינת הדם בכל זאת צריכים שלא יצאו מפתח ביתם בלילה ההוא עד הבוקר ללכת בתוך העיר, ונתן טעם לזה באומרו "ועבר ה' לנגוף את מצרים...וראה את הדם...ולא יתן המשחית" (שמות י"ב, כ"ג) וגו' רצה לומר שבאותה שעה שהם יאכלו את הפסח יכה ה' את בכורי מצרים, ובני ישראל בזכות אותה מצווה ינצלו ולא יהיה נגף בבתיהם. אמנם האזהרה שלא יצאו מפתח ביתם אפשר שהיה כדי שיהיו עסוקים בעשיית מצוות הפסח והלכתו ולא יפנו לדבר אחר.וחז"ל למדו מזה כי משניתן רשות למשחית לחבל אינו מבחין בין טוב לרע (מכילתא, פסחא י"א)  או בין צדיק לרשע, רצה לומר שיצא הקצף מלפני ה' החל הנגף ואין ראוי שיתערבו החיים עם המתים פן יספו בחטאיהם. ואולי כיוון גם כן בזה שלא יצאו מפתח ביתם בעבור המצרים שלא יראו אותם בעת קלקוליהם במכת בכורות ויחרחרו ריב עמהם לומר אתם המיתם את עמנו, וכדי להרחיק מן הכיעור ומן הדומה לו ציווה אותם: "וְאַתֶּם לֹא תֵצְאוּ אִישׁ מִפֶּתַח בֵּיתוֹ עַד בֹּקֶר" (שמות י"ב, כ"ב). ואם תאמרו אף על פי שלא נצא מבתינו מי הוא המונע שלא יבוא המצרי במרת נפשו על בנו אל בית היהודי להורגו? לזה אמר ולא יתן המשחית לבוא אל בתיכם לנגוף, והמשחית הוא אנשי מצרים שהבטיח ה' לא יבואו אל בתיהם של ישראל להרוג ולאבד אותם על דבר קורבן הפסח ולא על מכת בכורות, כי בלילה ההוא יהיו משומרים מכל המזיקין, שהם יתברך ישמור על פתח ישראל למלטו מן הנגף והמכה ולא יתן יכולת להמצרי לבוא אל בתיהם לנגוף. ולפי שהדברים המסוכנים אשר יחרפו אנשים את נפשם עליהם בגבורה ובמלחמה הם נזכרים ונשמרים לכבוד ולתפארת העושים אותם ראשונה, כמו שאמר דוד כשבא על ירושלים ללוכדה כל מכה יבוסי בראשונה יהיה לראש ולשר, לכן אמר משה לזקנים "וּשְׁמַרְתֶּם אֶת הַדָּבָר הַזֶּה לְחָק לְךָ וּלְבָנֶיךָ עַד עוֹלָם" (שמות י"ב, כ"ד), והודיעם בזה שתהיה המצווה הזאת לדורות ולכן ראוי שיתחילו בה הגדולים מתוך העם.ובסוף הפרשה הזכיר הכתוב שאף שהיה משה מצווה לזקנים בלבד, הנה העם כולו שהיו שמה היו כורעים ומשתחווים לפניו כאילו אמרו: "אדונינו! נחנו נעבור חלוצים וכוללנו מוכנים לקיים את המצווה, ואין ראוי שיתחילו בה הזקנים כי אם כל העם יחד". וזה העניין שנאמר "ויקוד העם וישתחוו וילכו ויעשו בני ישראל כאשר ציווה ה' את משה ואת אהרון כן עשו", רצה לומר, שעם היותו מדבר לזקנים, הנה כל ישראל אשר שמעו דבריו השתחוו לפניו כתלמיד הנפטר מלפני רבו, והלכו לעשות המצווה בכל חלקיה כמו שנצטוו בה משה ואהרון, ולא חיכו למעשה הזקנים וקדימתם, כי כל העם מקצה בלי מורא ופחד הלכו לקיים את המצווה. ובדרך הזאת מבוארת הפרשה כפי פשוטה והותרו הספקות אשר העידותי בה.האמנם במה שנאמר בפרשה "והיה כי יאמרו אליכם בניכם מה העבודה הזאת לכם" ראה המגיד שהמאמר הזה היה מאויב מחרף ה' "רָשָׁע כְּגֹבַהּ אַפּוֹ בַּל יִדְרֹשׁ אֵין אֱלֹהִים כָּל מְזִמּוֹתָיו" (תהלים י', ד'), והנה שלושה דברים הביאו את המגיד לייחס את המאמר הזה אל הרשע:הראשון, שלא אמר הכתוב בזה הלשון שאלה כמו שאמר במאמר החכם "כי ישאלך בנך מחר" וגו', ובהם נאמר כי ישאלך בנך, אבל בזה המאמר לא נזכר לשון שאלה כי אם אמירה "והיה כי יאמרו אליכם בניכם", רצה לומר כאשר בניכם, שהיו ראויים שישאלו וילמדו מאבותיהם ליראה את ה', הם בנים סוררים ולא ידברו בדרך שאלה אלא בדרך כפירה וערעור, "וְהָיָה כִּי \textrm{\textbf{יֹאמְרוּ}} אֲלֵיכֶם בְּנֵיכֶם מָה הָעֲבֹדָה הַזֹּאת לָכֶם" (שמות י"ב, כ"ו), וכבר השתמש הכתוב בלשון אמירה לדבר מגונה, כמו שמצינו בבני גד ובני ראובן בבנותם המזבח בעבר הירדן באומרם מדאגה מדבר עשינו זאת לאמור "מחר \textrm{\textbf{יאמרו}} בניכם לבנינו מה לכם ולה' אלוהי ישראל" (יהושע כ"ב, כ"ד) וגו'. וכן אמר הנביא "הוי \textrm{\textbf{האומרים}} לרע טוב ולטוב רע" (ישעיהו ה', כ',), "האומרים מהרה יחישה מעשהו" (ישעיהו ה', י"ט). והחכם שלמה אמר "אומר לרשע צדיק אתה יקבוהו עמים" (משלי כ"ד, כ"ד), כי תמיד יבוא הלשון אמירה על הקינטור והמאמר המגונה, ולא על השאלה בכוונה טובה.והסיבה השניה לפי שלא אמר "מה העבודה הזאת אשר ציווה ה' אלוהינו אתכם" (דברים ו', כ') כמו שאמר הבן החכם, שבתחילת דבריו הודה היות המצווה ההיא אלוהית, ואם הבנים האלה היו אומרים "מה העבודה הזאת אשר ציווה ה' אלוהינו לכם" לא היינו חוששין למילת \textrm{\textbf{לכם}} כמו שלא חששנו על מילת "אתכם" שאמר החכם, אבל הרשע הזה באמת כפר שהמצווה היא אלוהית ולכן לא אמר "אשר ציווה אלוהינו", והסיבה השלישית מפני שהבן ההוא לא קרא את המצווה הזאת עדות או חוקים או משפטים, לפי שלא האמין היות בה עדות יציאת מצרים, וגם כן כפר היותה משפט וצדק, גם לא האמין שהיה חוק וגזירת המלך, אלא אמר "מה \textrm{\textbf{העבודה}} הזאת לכם", וכיוון בזה המאמר שני דברים ושתי כפירות, האחת שלא הייתה מצווה אלוהית ולא ציווה אותה ה', אלא הם מעצמם סידרו אותה, וזהו שאמר "מה העבודה הזאת \textrm{\textbf{לכם}}" שהוא כפירה בפועל, וכפירה שניה בעניין התכלית, כאילו אמר אין זו עבודה לגבוה ולא זכר לחסדי ה', ואינה עבודה רוחנית בתפילה ועיון התורה, אלא היא עבודה עשויה לתועלתם שתאכלו כבש בן שנתו רך וטוב צלי אש ותשתו ארבע כוסות עליו, הלוא זאת העבודה היא לכם ולא לשמיים, ולפי שיש בזה שתי כפירות לכן אמר בלשון רבים "והיה כי יאמרו עליכם \textrm{\textbf{בניכם}}", כאילו היה בכאן בנים סוררים בנים משחיתים בכפירות שונות, ולזה בפרשה הזאת נזכרו בנים בלשון רבים ובשאר הפרשיות באו כולם בלשון יחיד. טעם אחד כי שאר הפרשיות באו כולן בלשון יחיד "קדש לי כל בכור", "והיה כי הביאך ה' ועבדת את העבודה", "שבעת ימים תאכל מצות", כל הפרשה באה בלשון יחיד, ופרשת החכם גם כן באה בלשון יחיד "ואמרת לבנך", אבל פרשת "משכו וקחו לכם" היתה לזקנים ובאה כולה בלשון רבים, ומפני זה נזכרו בה הבנים גם כן בלשון רבים, "והיה כי יאמרו עליכם בניכם".הנה כפי כל אחת משתי הבחינות האלה הותר הספק אשר בשער כ"ד.ולפי שהכפירות הן שתיים במאמר הרשע "מה העבודה הזאת לכם", לכן השיבה התורה על שתיהן, אם בעניין הפועל והמצווה אמר "ואמרתם זבח פסח הוא לה' " כי הוא ציווה בה ולא אנחנו, וכן בעניין התכלית שחשב הרשע שנעשה זאת המצווה לאכול לשובעה  ולמכסה עתיק (ישעיהו כ"ג, י"ח), אינו כן, ולכן ציווה ה' יתברך שלא יאכלו את הפסח כי אם על השובעה, כדי שלא תמצא באכילתו שום הנאה גשמית כי אם קיום המצווה מפאת עצמה ותכליתה, שהוא לזכר שפסח ה' על בתי בני ישראל בנגפו את מצרים ואת בתינו הציל. ובזה ביארה התורה לרשע כי מלבד חומר המצווה אשר לפניו יש פועל ותכלית אמיתי בה, ומבוטלים הם ערעוריו וכפירותיו. והנה המגיד לפי שראה שכיוון הרשע במילות "לכם" שתי הכפירות האלה אמר אף אני במילת "לכם" אדרוש דרשה שלישית והיא שפי כסיל מחתה לו (משלי י"ח, ז'), ואמר "מה העבודה הזאת \textrm{\textbf{לכם}}" להוציא את עצמו מן הכלל שלא יבוא בקהל ה', כי המצווה נתנה לטובים ולישרים בלבותם ולא לרשע. לכן אמר "לכם ולא לו" והוא מדברי המגיד, ולזה אמר "לפי שהוציא את עצמו מן הכלל", ולא היתה היציאה מן הכלל בקיום המצווה כי אם באמונתו, וכמו שאמר "כפר בעיקר" שהעיקר הוא היות המצווה אלוהית ולא אנושית. ולכן אמר המגיד "אף אתה הקהה את שניו" כלומר מלבד מה שהשיבתו התורה תשובה אמיתית על כפירותיו, אף אתה השומע כפירה מפיו ענה כסיל כאולתו, הקהה את שניו, והשיבתו כפי הדרשה השלישית, "ואמור לו בעבור זה עשה ה' לי- לי ולא לו", רצה לומר אמור לו כי בעבור קיום האמונה שהיה לי בה' ובעבור עבודת המצווה הזאת אשר עשיתי במצרים הלילה ההוא עשה ה' לי חסד גדול כי הוציאנו ממצרים והצילני מנגף הבכורות. ואתה אויב מחרף ה' "אילו היה שם לא היה נגאל" לא היה זוכה להיגאל, כי היה מת באותם ימי החושך שמתו בהם פושעי ישראל לבל יראו בגאולת ה' ותשועתו. ולמדנו מזה בעניין השגחת ה', כי לא לבד שהוא ה' יתברך מבחין בין עם לעם כמו שאמר הכתוב "בנגפו את מצרים ואת בתינו הציל", אלא שהוא מבחין ומבדיל גם כן בין איש לאיש מהאומה עצמה לתת לאיש כדרכיו וכפרי מעלליו, שפסוק "בעבור זה עשה ה' לי בצאתי ממצרים" בה בפרשת שאינו יודע לשאול ולקחו המגיד בהלוואה משם כדי לחרף ולגדף בו את הרשע כדי רשעתו, ובשאינו יודע לשאול נדרש מהכתוב רק מילות "בעבור זה" ושאר חלקי הכתוב שהם "עשה ה' לי בצאתם ממצרים"  דרש ברשע, וכאילו אמר אם אין לו עניין אצל שאינו יודע לשאול תנהו לענין רשע. וכבר אמרתי שאין הדרשה הזאת ביאור הכתוב באשר הוא שם, לכן אין להקשות עליו איך הביאו מפרשה אחרת, כי המגיד רצה להקהות שני הרשע ומתלעות אויל על נרפו בה', ולא חשש אם לקח הפסוק מפרשה אחרת.ובזה נתבאר המאמר והותרו הספקות אשר בשער כ"ז, כ"ח, ושער עשרים ותשעה.}%endcomment
\hebeng{תָּם מָה הוּא אוֹמֵר? מַה זּאֹת? וְאָמַרְתָּ אֵלָיו ״בְּחוֹזֶק יָד הוֹצִיאָנוּ ה׳ מִמִּצְרַיִם מִבֵּית עֲבָדִים״. }{What does the innocent {[son]} say? "'What is this?' (Exodus 13:14)" And you will say to him, "'With the strength of {[His]} hand did the Lord take us out from Egypt, from the house of slaves' (Exodus 13:14).'"}
\hebeng{וְשֶׁאֵינוֹ יוֹדֵעַ לִשְׁאוֹל – אַתְּ פְּתַח לוֹ, שֶׁנֶּאֱמַר, וְהִגַּדְתָּ לְבִנְךָ בַּיּוֹם הַהוּא לֵאמֹר, בַּעֲבוּר זֶה עָשָׂה ה׳ לִי בְּצֵאתִי מִמִּצְרָיִם.}{And {[regarding]} the one who doesn't know to ask, you will open {[the conversation]} for him. As it is stated (Exodus 13:8), "And you will speak to your son on that day saying, for the sake of this, did the Lord do {[this]} for me in my going out of Egypt."}%
\commenta{\textrm{\textbf{ושאינו יודע לשאול את פתח לו}} באיזה מקומות במקרא בא הלשון אתה בכנוי נוכח לנקבה, את, והם בפ׳ בהעלתך (י״א ט״ו) ואם ככה את עושה לי, ובפ׳ ואתחנן (ה׳ כ״ד) ואת תדבר אלינו. ובקהלת (ז׳ כ״ב) גם את קללת, וביחזקאל (כ״ח י״ד) את כרוב ממשח, ובכולם נדרש שינוי זה, וכאן לא נתבאר על מה הכנוי לנוכח זכר בכנוי לנוכח נקבה, את תחת אתה. ושמעתי רמז לזה, שבעל ההגדה מרמז שלא תפטרנו לזה הבן בדברי הפסוק הזה בעבור זה עשה ה׳ לי בצאתי ממצרים כי הם לא יספיקו להגיד את כל הקורות בענין זה, אך תלמדהו והבינהו מכל ענין יציאת מצרים מתחלתו ועד סופו, וזה מרומז במלת ״א׳ ת׳״, כדאמרי אינשי, מאל״ף ועד תי״ו.\textrm{\textbf{את פתח לו שנאמר והגדת לבנך וגו׳}} לא נתבאר איפה מרומז בפסוק זה דאיירי בבן שאינו יודע לשאול, ואיזה הכרח לזה. אך הבאור הוא, כי הדברים לשלשת הבנים הראשונים באו בסגנון תשובה  לשאלותיהם, כמו בשאלת החכם, כי ישאלך בנך וכו׳ (פ׳ ואתחנן, ו׳ ה׳) ובשאלת הרשע, והיה כי יאמרו אליכם בניכם וכו׳ (פ׳ בא, י״ג כ״ו), ובשאלת התם והיה כי ישאלך בנך וכו׳ (שם י״ג י״ד), אבל בענין שאינו יודע לשאול לא כתיב שישאל, אך ישר מדברי האב, והגדת לבנך (שם שם ח׳), וזה הוא מפני שהבן אינו יודע לשאול.\textrm{\textbf{בעבור זה עשה ה׳ לי בצאתי ממצרים}} הנה תלה הסבה במסובב, כי הן הכונה הוא לבאר סבת מניעת אכילת חמץ וחיוב אכילת מצה שבפסוק הקודם, שזה בעבור הגם שעשה ה׳ לי בצאתי ממצרים, א״כ הי׳ לו לומר כי זה הוא בעבור (הנס) שעשה ה׳ לי בצאתי ממצרים. אך אמנם מצינו לפעמים קרובות לשונות הפוכות במקרא, ובלשון חז״ל נקראים מקראות מסורסים, כמו בסוטה (ל״ח א׳) מקרא זה מסורס הוא, ובב״ב (קי״ט ב׳) סרס המקרא ודרשהו, וכמ״ד פ׳ אחרי, ריש לקיש מסרס קראי ודרש, ויתבאר זה להלן בדרשה על הפסוק וירד מצרימה ויגר שם, ונביא שם הרבה משלים לזה, ואם כן גם כאן אפשר לפרש זה בסגנון הפוך כמו שכתבנו. אכן אפשר גם לקיים הלשון ככתיבתו, והענין יתבאר ישר וטוב. וזה יתבאר עפ״י מה שדקדקתי בהרבה מקומות במקרא, שהמלה ״זה״ או ״זו״ באות תחת מלת ההוראה ״אשר״, כמו בתהלים (ק״ד) אל מקום זה יסדת — תחת אל מקום אשר יסדת, ובמשלי (כ״ז כ״ב) שמע לאביך זה ילדך — תחת אשר ילדך, ובאיוב (י״ט י״ט) וזה אהבתי נהפכו בי — תחת ואשר אהבתי. וכן המלה ״זו״ בתהלים (י״ט) ברשת זו טמנו — תחו אשר טמנו (עיי״ש כראב״ע), ושם (י׳ ב׳) במזמות זו חשבו — תחת אשר חשבו, ועוד שם (ל״ב ח׳) אשכילך ואורך בדרך זו תלך — תחת בדרך אשר תלך, ושם (ס״ח כ״ט) זו פעלת לנו — תחת אשר פעלת, ובישעיה (מ״ב כ״ד) ה׳ זו חטאנו לו — תחת אשר חטאנו לו. ובזה יתבאר בפרשת שירת מצרים נחית בחסדך עם זו גאלת. עד יעבור עמך ה׳ עם זו קנית, ובישעיה (מ״ג כ״א) עם זו יצרתי, וידוע, דשם ״עם״ הוא שם ממין זכר (עם אחד, והרבה כהנה), ואיך ילווה לו ההוראה ״זו״. אך לפי המבואר, דהמלה ״זו״ באה תחת מלת ההוראה ״אשר״ יתפרש עם אשר גאלת, עם אשר קנית, עם אשר יצרתי. והבקי במקרא ימצא הרבה כאלה. והנה גם כאן בפסוק שלפנינו יתבאר, והגדת לבנך לאמר (כלומר, שנסמך על הפסוק הקודם מצות יאכל ולא יראה לך חמץ) הוא בעבור אשר עשה ה׳ לי בצאתי ממצרים. ומתבאר יפה וטוב ורצוי.}%endcomment
\newsection{יכול מראש חודש}
\hebeng{יָכוֹל מֵראשׁ חֹדֶשׁ? תַּלְמוּד לוֹמַר בַּיּוֹם הַהוּא. אִי בַּיּוֹם הַהוּא יָכוֹל מִבְּעוֹד יוֹם? תַּלְמוּד לוֹמַר בַּעֲבוּר זֶה – בַּעֲבוּר זֶה לֹא אָמַרְתִּי, אֶלָּא בְּשָׁעָה שֶׁיֵּשׁ מַצָּה וּמָרוֹר מֻנָּחִים לְפָנֶיךָ. }{It could be from Rosh Chodesh {[that one would have to discuss the Exodus. However]} we learn {[otherwise, since]} it is stated, "on that day." If it is {[written]} "on that day," it could be from while it is still day {[before the night of the fifteenth of Nissan. However]} we learn {[otherwise, since]} it is stated, "for the sake of this." I didn't say 'for the sake of \textit{this}' except {[that it be observed]} when {[\textit{this}]} matsa and \textit{maror} are resting in front of you {[meaning, on the night of the fifteenth]}.}%
\commenta{\textrm{\textbf{יכול מראש חודש... לא אמרתי אלא בשעה שיש מצה ומרור מונחים לפניך.}} הסברא שהי׳ אפשר לחייב מצות והגדת לבנך מר״ח, י״ל משום דתחלת הפסח היתה בר״ח, כמבואר בפרשה בא (י״ב ב׳) החודש הזה לכם ראש חדשים, ומבואר במנחות (כ״ט א׳) דזה היה בר״ח, שהראה ד׳ להם (למשה ולאהרן) מולד הלבנה, ואם כן אפשר שחל החיוב מזמן הצווי. וזהו שאמר, יכול מר״ח, כפי אפשרות הסברא, ת״ל בעבור זה. — לא אמרתי אלא בשעה שיש מצה ומרור מונחים לפניך, וזה הוא בליל ט״ו בניסן. ואמנם צריך באור, מה ההפסד בזה אם יספרו להבנים מענין יציאת מצרים מראש חודש או מבעוד יום, ועל מה בא עכוב לזה. אך אפשר להסביר זה עפ״י מה שבארנו במשלי (כ׳) את הפסוק גם במעלליו יתנכר נער אם זך ואם ישר פעלו *קרוב לומר כי סגנון הלשון מהופך במקצת, ושיעורו ידי׳, גם נער במעלליו יתנכר אם זך ואם ישר פעלו, כי עיקר החידוש בא להורות כשרון הנער, וענין היפוך לשון במקרא הוא חזון נפרץ, וגם חז״ל עמדו על סגנון זה ואמרו ״מקרא מסורס הוא״ (סוטה ל״ח א׳), או ״סרס המקרא ודרשהו״ (ב״ב קי״ט ב׳) ועוד כהנה. ועוד יבא מזה בהרחב באור להלן בפיסקא וירד מצרימה.), ולא אמר אם הוא סר מרע, וזה הוא, כי אחרי שרוצה הכתוב לומר, כי הוא יתנכר במעלליו, והדבר מתבלט אך ורק במעשה בפועל ולא בשב ואל תעשה. וכן משמע במס׳ שבת (ס״ט ב׳) במי שתעה במדבר ושכח איזה יום שבת מונה ששה ימים ושובת יום אחד, ופריך, וההוא יומא במאי מינכר לי׳ — בקידושא והבדלתא. — הרי דלא די בשביתה בלבד, בשם ואל תעשה וצריך לזה איזו פעולה, ומשני דמינכר ליה בקידוש והבדלה. וכן כאן, היכר סיפור הדברים הוא רק בשעה שמצה ומרור מונחים על השולחן. וזה כעין מעשה בפועל ומתוך כך יתקבלון וישתמרון הדברים שמספרים בענין. ואמנם נראה, כי כל זה הוא רק לענין לצאת חובת סיפור, כי בזמן אחר אין יוצאין, אבל לענין רשות, באמת רשאים לספר בכל זמן שהוא, וע״ע במאמר הבא.\textrm{\textbf{בשעה שיש מצה ומרור מונחים לפניך}} המלה ״שיש״ נראית כמו מיותרת, כי לא הי׳ חסר הלשון והענין אם הי׳ אומר בשעה שמצה ומרור מונחים לפניך. ואפשר לומר שבא להוציא שלא נאמר שגם באיזו שעה שהיא ובכל זמן שהוא, שמונחים לפניך מצה ומרור, אף שלא בזמנם הקבוע, גם אז מצוה לספר ביצ״מ — על זה אומר, בשעה ״שיש״, כלומר, בשעה שיש חובה ומצוה להניח מצה ומרור לפניך, והוא בערב יום חמשה עשר בניסן, שאז תחלת י״ט. וראוי להעיר, כי במכילתא פרשה בא נתוסף ללשון מונחים לפניך המלים על ״השלחן״, וזה יטעים דברינו ביותר, דהכונה — בעת שיגיע שעתם להיות מונחים על השלחן, כי מדרך לערוך שלחן בזמן שהשעה מחייבת.\textrm{\textbf{בשעה שיש מצה ומרור מונחים לפניך.}} ידוע הוא, כי בכל מקום שבאים השמות מצה ומרור מקדים מצה למרור, ולכאורה יש להעיר, כי אחרי שמצה היא זכר לחרות, כמבואר בתורה (ס״פ בא) ומרור זכר לשעבוד, כמש״כ (ר״פ שמות) וימררו את חייהם בעבודה קשה וכו׳ והשעבוד קודם לחירות. אם כן הי׳ דרוש להקדים בכל מקום מרור למצה, כסדר הענינים. אך אפשר לבאר זה עפ״י הידוע בגמרא ומדרשים, כי יען שרצה הקב״ה להחיש גאולתם קודם הזמן המוגבל (תחת ארבע מאות אך מאתים ועשר שנים) אך לא היה באפשר לבטל הגזירה ממספר השנים סבב כבדות העבודה, וחשב זה תחת זה, קושי עבודה תחת אריכות הזמן, ועבור זה החיש זמן החירות — ואם כן, היתה כונת הגאולה באמת קודמת למרירות העבודה, ולכן מצה קודמת.}%endcomment%
\commentb{\textrm{\textbf{תשובות לשערים ל' – ל"ד}}\textrm{\textbf{"תם מה הוא אומר מה זאת ואמרת אליו בחוזק יד הוציאנו" וכו',}} הדרשות האלה שדרש המגיד על התם ושאינו יודע לשאול תמצאם בסוף סדר "בא אל פרעה" במצוות "קדש לי כל בכור פטר כל רחם באדם ובבהמה לי הוא, ויאמר משה אל העם זכור את היום הזה אשר יצאתם ממצרים" וגו' "ולא יאכל חמץ" וגו' "שבעת ימים תאכל מצות" וגו' "והגדת לבנך ביום ההוא לאמור בעבור זה עשה לי ה' בצאתי ממצרים, והיה לך לאות על ידך" וגו'. והפרשה הזאת דרש המגיד על ושאינו יודע לשאול, ואחריה באה פרשה אחרת "והיה כי יביאך ה' אל ארץ הכנעני" וגו' "והעברת כל פטר" וגו' "והיה כי ישאלך בנך מחר לאמור מה זאת" וגו', "וְהָיָה לְאוֹת עַל יָדְכָה" וגו' (שמות פרק י"ג א' – ט"ז), והפרשה הזאת פירשה המגיד על התם. ויש לעיין בפרשיות האלה:ראשונה אם אמנם באה הדיבור למשה לאמור "קדש לי כל בכור" וגו' למה כאשר בא משה רבינו ע"ה להגיד המצווה הזאת לישראל לא זכרה כמו ששמע שהיא קידוש הבכורות, והזכיר במקומה מצוות מצה ואיסור חמץ שנאמר "ויאמר משה אל העם זכור את היום הזה וגו' ולא יאכל חמץ וגו' שבעת ימים תאכל מצה" וגו'. וקשה אם ציווה ה' אותו על קידוש הבכורות לעם ציווה הוא לעם מצוות המצה?שנית למה הזכיר גם כן מצוות קרבן הפסח ומשפטו והמרור? ומה ראה משה רבינו עליו השלום על ככה שפעמים יצווה על הפסח ולא על המצות ופעמים יצווה על המצות ולא על הפסח?השלישית מה ראה לצוות מצוות התפילין עם המצה בפסח?והנראה לי בעניין הפרשיות האלה אבאר פה ומזה תצא דרשת המגיד בתם ובשאינו יודע לשאול. ואומר שהתורה ספרה כי הקב"ה ציווה למשה על קידוש הבכורות ולא היה עניינו שיקדש בכורות ישראל מיד במדבר להיותם לעבודת הקודש כמו שכתב הרמב"ן, אלא עניינו שיקדש כל בכור פטר רחם באדם ובבהמה אחרי בואם לארץ כנען, כי גם במצוות הפסח לא נתחייבו לדורות כי אם בארץ. ופסח מדבר נצטוו באותו פרק, ולפי שהיה קידוש הבכורות זכר למכת בכורות וליציאת מצרים, לכן ראה משה רבינו עליו השלום להזהיר את ישראל על המצווה הכוללת ובכל חלקיה שהם פסח, מצה ומרור וקידוש הבכורות, שכולם לתכלית אחד ולזיכרון אחד נצטוו, וזהו "ויאמר משה אל העם זכור את היום הזה אשר יצאתם ממצרים". כי התחיל לימודו להתם מהטעם העיקרי והסיבה האמיתית, ולכן ציווה אותם שיזכרו תמיד את היום הזה, והזיכרון יהיה שבחוזק יד הוציא ה' אותם ממצרים. והרושם אשר יעשה בהם הזיכרון הזה ראשונה הוא על ידי מצוות מצה ואיסור חמץ. ולפי שהיה זה תלוי בארץ, כי במדבר לא היה להם חיטים לאת פת חמץ ולא מצה, לכן אמר "והיה כי יביאך ה' אל ארץ הכנעני" וגו' "ועבדת את העבודה הזאת בחדש הזה", ולא נקרא עבודה עניין המצה כי אם קרבן הפסח שנקרא כן, כמו שאמר במצוותו "ושמרתם את העבודה", וכן פירש רש"י ז"ל.ולפי זה באומרו "ועבדת את העבודה הזאת" משמעו הפסח והמרור, וקיצר בעניינו לפי שכבר פירשו למעלה באר היטב. ואחר הפסח הזכיר עניין המצה, ואמר "שבעת ימים תאכל מצות" וגו' ועל זה אמר ולא יראה לך חמץ והגדת לבנך ביום ההוא לאמור. ומאשר לא הזכירה בו התורה עניין שאלה ולא אמירה אלא אמר לבד והגדת לבנך, גזר המגיד שהבן הזה לא היה יודע לשאול, ושהתורה ציוותה אביו שיפתח פיו לאלם ויגיד לו הדברים כמו שעברו.ואם תשאל ומי אמר למגיד שהבן הזה נמנע מלשאול מחוסר ידיעתו אולי הוא בלתי מאמין בדבר ומאס בתורת ה' ואמר לא אזכרהו ולא אדבר עוד בשמו? הנה התורה גילתה אמיתת זה במה שאמרה מיד "והיה לאות על ידך ולזיכרון בין עיניך למען תהיה תורת ה' בפיך, ושמרת את החוקה הזאת במועדה מימים ימימה", וידוע שהדברים האלה לא יאמרו כי אם על איש ירא ה' במצוותיו חפץ מאוד, שלא יעדר ממנה כי אם הידיעה, ולחסרון ידיעתו אמר "למען תהיה תורת ה' בפיך", רצה לומר שילמד הדברים כפי פשוטם מצוות אנשים מלומדה, ויהיו על פיו אף על פי שלא יעמיק בשכלו, ולכן נסתפק להודיעו חומר המצווה בלבד, כמו שאמר "בעבור זה ה' לי בצאתי ממצרים", רצה לומר "בעבור זה" היינו המצה והמרור כדי שאבוא לעשות המצווה הזאת עשה ה' לי להוציאני ממצרים. וזהו מה שדרש המגיד "ושאינו יודע לשאול את פתח לו". והוצרך לדייק בו יכול מראש חודש? יכול מבעוד יום? לא אמרתי אלא בשעה שמצה ומרור מונחים לפניך. כי הבן הנער והסכל הזה יאות לו לדעת רק הסיבה החומרית במצווה, ולכן יצטרך האב להראות לו באצבע על מצה ומרור המונחים לפניו, שהם הדברים המוחשיים מחומר המצווה, ולזאת נתייחד לימודו החומרי בלילה הזה, ולא לימוד הפסח שהוא עניין רוחני.יש מפרשים שאמר "והגדת לבנך יכול מראש חודש" וכו' אינו מהתשובה, אלא שמחבר ההגדה אחרי שהשלים זכרון הארבעה בנים וחשיבותם בא להזכיר הדרשה הזאת בפסוק והגדת לבנך ואמר יכול מראש חודש, תלמוד לומר משנכנס ניסן; אי ביום ההוא יכול מבעוד יום רצה לומר שתאמר ההגדה ביום ששוחטים פסחיהם תלמוד לומר בעבור זה. אבל מה שכתבתי ראשונה הוא הנכון שעל מי שאינו יודע לשאול נדרש זה ולא על שאר הבנים, לפי שלא נתייחד לימודם לאותו זמן להיותו מושכל ובלתי חומרי, ואצל שאינו יודע לשאול הוא כולו חומרי. והנה דרש בשאינו יודע לשאול פסוק "בעבור זה" לפי שהוא מעניינו, אף כי מילות "עשה ה' לי" כבר דרש אותם ברשע, אמנם בשאינו יודע לשאול דרשו בייחוד, וברשע דרש בדרך אסמכתא ועל צד הצחות לקוח בהלואה ממקום אחר.אמנם בבן התם כלל כל העניינים, כי בפרשה "והיה כי יביאך ה'" ציווה משה רבינו על זכרון היום ההוא בעבודת הפסח והמרור הנכלל עמו, ומצוות המצב הזהיר אחר כך בפרשה הזאת איך יעשו הזיכרון ההוא, וגם כן בקידוש הבכורות. ולהיות המצוות הללו כולם קשורות ומצרניות הן זו עם זו כי תכליתן אחת ומכוונות ובאות לזיכרון אותו היום, ונמשכות מאותו דבור ראשון. וזהו אמרו "והיה כי יביאך ה' אל ארץ הכנעני" וגו' שצווה מצוות הבכורות באותו לשון עצמו שציווה מצוות הפסח והמצה. ולכן אמר עליו "והיה כי ישאלך בנך מחר לאמור מה זאת?" כי בראותו את חומר המצווה ישאל על עניינה והיתה התשובה: "וְאָמַרְתָּ אֵלָיו בְּחֹזֶק יָד הוֹצִיאָנוּ ה' מִמִּצְרַיִם מִבֵּית עֲבָדִים" (שמות י"ג, י"ד).התורה מורה כי הבן התם שאל על הזיכרון בכללותיו אם בפסח אם במצה ואם בקידוש הבכורות, כי ראה שהיתה המצוה מתאחדת בכולם ולכן שאל "מה זאת". והיתה התשובה שהפועל כולו הוא פועל אלוהי, והוא צווהו, ולזה אמר "בחוזק יד הוציאנו ה' מבית עבדים" שמוסב כנגד הפסח והמצה שהזכיר, וכנגד קידוש הבכורות אמר עוד "ויהי כי הקשה פרעה לשלחנו ויהרוג ה' כל בכור בארץ מצרים" וגו' "על כן אני זובח לה' כל פטר רחם", הרי משמע כי הבן התם שאל על כללות העניין שהוא הפסח, המצה והבכורות, אף כי בתום לבבו וקוצר חכמתו לא ידע להבחין ביניהם ואמר "מה זאת?". והוכיחה התורה שהיה זה בתום לבבו ולא נטה למינות או שיש איזה טינה וקנטור בליבו, והראיה שהוסיפה התורה לומר עליו "והיה לאות על ידך ולטוטפות בין עיניך כי בחוזק יד הוציאנו ה'  ממצרים", ולא יפול מאמר זה כי אם על התאב למצוות ה' ולא נמנע מהן כי אם מפאת תמימותו וקוצר הבנתו. ולפי זה אף שבאה שאלת התם בפרשת בכורות, הייתה שאלתו בייחוד על יציאת מצרים, כמו שנראה מדברי התשובה, ובאה בתום לבבו כפי מה שהשיבה לו התורה. ושאינו יודע לשאול דרש המגיד מהפסוק "בעבור זה" כפי עניינו, והסב הגדתו שתהא בליל פסח, לפי שהכין השאלה עם הסיבה החומרית בלבד. ולפי ששניהם התם ושאינו יודע לשאול לא נמצא להם עוון כי אם קוצר ידיעה, לא בהתפארות החכמה כמו החכם ולא ברֵשַע ופשע כמו הרשע, לכן הסתפק המגיד להשיב עליהם מה שהשיבה התורה ולא הוסיף לומר עליהם "אף אתה..." כמו שעשה בבן החכם והרשע.והותרו עם זה הספקות אשר בשער שלושים, ל"א, ל"ב ול"ג ושער ל"ד.}%endcomment
\newsection{מתחילה עובדי עבודה זרה היו אבותינו}
\hebeng{מִתְּחִלָּה עוֹבְדֵי עֲבוֹדָה זָרָה הָיוּ אֲבוֹתֵינוּ, וְעַכְשָׁיו קֵרְבָנוּ הַמָּקוֹם לַעֲבדָתוֹ, שֶׁנֶּאֱמַר: וַיֹאמֶר יְהוֹשֻעַ אֶל־כָּל־הָעָם, כֹּה אָמַר ה׳ אֱלֹהֵי יִשְׂרָאֵל: בְּעֵבֶר הַנָּהָר יָשְׁבוּ אֲבוֹתֵיכֶם מֵעוֹלָם, תֶּרַח אֲבִי אַבְרָהָם וַאֲבִי נָחוֹר, וַיַּעַבְדוּ אֱלֹהִים אֲחֵרִים. }{From the beginning, our ancestors were idol worshipers. And now, the Place {[of all]} has brought us close to His worship, as it is stated (Joshua 24:2-4), "Yehoshua said to the whole people, so said the Lord, God of Israel, 'Over the river did your ancestors dwell from always, Terach the father of Avraham and the father of Nachor, and they worshiped other gods. }%
\commenta{\textrm{\textbf{מתחילה עובדי ע״ז היו אבותינו}} מאמר זה בא במשנה וגמרא פסחים (קט״ז א׳), וצריך באור לאיזו כונה ומטרה בא כולו בעניני ההגדה, אשר אין לו כל יחש לענין יציאת מצרים. ואולי אפשר שרצה להקדים הסבר כללי על מה זה נגזר על ישראל בכלל כל ענין שעבוד מצרים וכל הקורות אותם שם. והנה ידוע, דמחלה הבאה לאדם בתולדה מירושת אבות, שגם הם היו נגועים בה — מחלה כזו קשה יותר לרפאות מאשר מחלה הבאה לאדם ממקור גופו שלו עפ״י סבה שונה, ולמחלה כזו דיה רפואה קלה בערך. והנה לוא היו ישראל עובדי ע״ז רק מרגשות נפשם במקרה ככל המקרים הי׳ באפשר לרפאם בקל ע״י עונש קל, ושבו לתעודתם ולקדושתם. אבל מכיון שהם היו שקועים בע״ז עפ״י מסורת אבותיהם נשרש חטא זה בדמם ובנפשם והיו צריכים לרפואה עקרית וארוכה ונכבדה, כדי לעקר חטא זה מיסודו ושרשו ולזכך נשמתם, ורפואה זו הי׳ שעבוד מצרים וכל הנסתעף מזה, עד מתן תורה והכניסה לא״י. זהו שאמר בהתחלת הספור משעבוד ישראל במצרים, כי סבות כל קושי זה באה לרגלי שקועם של ישראל בע״ז עפ״י מסורת מאבותיהם עוד מזמנו של תרח, והי׳ חטאם קשה ורפואתם קשה, כפי שנתבאר.\textrm{\textbf{מתחלה עובדי עבודה זרה היו אבותינו}} לכאורה לפי מה שאמרו בגמרא ברכות (ט״ז ב׳) אין קורין אבות (סתם) אלא לשלושה ופירש״י, אברהם יצחק ויעקב, אם כן יוצא, שהלשון הזה חלילה מוסב עליהם. אך הבאור הוא, דהא דאין קורין אבות אלא לשלושה הוא רק מזמנם של אברהם יצחק ויעקב, וכמו שפירש״י על הלשון אין קורין אבות — ״אבות ישראל״, אבל האבות שעד זמנם היו אבות כל אומות העולם וכמש״כ בבני נח, מאלה נפרדו הגוים (נח, י׳ ל״ב), וסמך זה על הפסוק דיהושע, משום דשם בא בקיצור כל מה שרצה בעל ההגדה להגיד בזה.}%endcomment%
\commentb{\textrm{\textbf{תשובות לשערים ל"ה – ל"ז}}\textrm{\textbf{"מתחילה עובדי עבודה זרה היו אבותינו ועכשיו קרבנו המקום לעבודתו" וכו'.}} אחר שהזכיר המגיד דרשת הבנים ע"פ הדברים שביארתי, מתחיל עתה בענין ההגדה, ותחילתו בגנות כדי לסיים בשבח. והנה בתחילת דבריו אומר "מתחילה עובדי עבודה זרה היו אבותינו ועכשיו קירבנו המקום לעבודתו ומזכיר קריבת ה' שהוא יתר שהכל תלוי בו ומקור הטובות והחסדים כולם, ומביא ראיה מדברי יהושע על שלושה חסדים גדולים שעשה הקדוש ברוך הוא עם אבותינו עד להפליא, והם שאברהם בתחילתו לא היתה לו ירושה ונחלת ארצות, וגם כן לא היתה לו אמונה אלהית, ולא היה לו זרע ותולדות בנים כלל, כי הוא היה נעדר מכל אלה, עד אשר בחמלת ה' עליו השלימו בכולם. וזהו שאמר "בעבר הירדן ישבו אבותיכם מעולם" (יהושע כ"ד, ב') , ר"ל אתם בני ישראל דעו נא וראו שארץ כנען אשר ירשתם והתנחלתם בה לא היתה לכם נחלת אבות, כי אבותיכם היו יושבים בעבר הנהר ולא בארץ כנען, כי שם בן נוח שיצא מן התיבה הוא וכל תולדותיו עד אברהם נתישבו בעבר הנהר בבבל. ולזה הזכיר תרח אבי אברהם, לפי שממנו נפרד אב המון גויים, וכל זה להודיע לאבותיהם היו נעדרים מירושת הארץ. עוד הזכיר ההעדר השני שהיתה להם מהאמונה האלהית, באמרו "ויעבדו אלהים אחרים", רמז לתרח ולנחור שהזכיר, כי האבות הקודמים להם היו צדיקים ורק תרח אבי אברהם ונחור אחיו היו עובדים אלהים אחרים, ואברהם היה ראוי ללמוד מהם ולהדמות אליהם באמונתם. והזכיר אברהם בלבד ולא אמר אברהם ובניו כמו ביעקב, רמז שהיה יחידי ועקר מבלי בנים.ועל שלושת ההעדרים האלה שהיו לאברהם: מנחלת הארץ, משלמות האמונה, מקנין הבנים. אך הקדוש ברוך הוא בחסדו הגדול חמל עליו והבדילו מהם, לכן הזכירו בראשונה מפני מעלתו, ואמר "ואקח את אביכם את אברהם" (שם שם , ג'), רצה לומר הלא אחים היו אברהם ונחור, וזהו החסד העליון באמונה, ועליו כיון בתחילת המאמר באמרו "ועכשיו קירבנו המקום לעבודתו", כי הקריבה היתה בשלקח את אברהם ובחר בו והגישו לעבודתו והוא מאמר אמיתי בלי ספק, כי האדם לא יקרב לבוראו כי אם אשר יקרבה ויקחהו ה' לו. וכמאמר המשורר "אשרי תבחר ותקרב ישכון חצריך" (תהלים ס"ה, ה'), ואמר הנביא "והקרבתיו ונגש אלי כי מי אשר ערב לבו לגשת אלי" (ירמיה ל', כ"א).עוד הזכיר החסד השני אשר עשה עמו במתנת הארץ, ועליו אמר "ואוליך אותו בכל ארץ כנען", רצה לומר הולכתיו בה לתת לו חזקה עליה, וכמו שאמר "קום התהלך בארץ לארכה ולרחבה כי לך אתננה" (בראשית י"ג, י"ז).עוד הזכיר החסד השלישי, והוא שעם היות אברהם ושרה בטבעם עקרים בכל זאת הרבה את זרעם, רוצה לומר שנתן לאברהם זרע ונתרבה הזרע ההוא בזמן מועט הפלא ופלא, כמו שנאמר "וארבה את זרעו". ואמר עוד "ואתן לו את יצחק", כלומר שלא נאמרה הברכה "וארבה את זרעו" על ישמעאל ובני קטורה שהיו זרעו של אברהם, שמהם יצאו הישמעאלים ובני רדן אשורים ולטושים ולאומים ויתר האומות שהם רוב חלקי הישוב, כי לא עליהם אמר "וארבה את זרעו", כי כולם נחשבו כקליפה להגזע והלב שבו, לכן הזכיר את העיקר והלב באמרו "ואתן לו את יצחק", כי בו בחר ה' ולא בשאר האחים. כמו שאמר "כי ביצחק יקרא לך זרע".  והזכיר עוד שכמו שהיו לתרח שני בנים אברהם ונחור ובחר ה' באחד מהם אברהם ועזב את נחור, וכן לאברהם היו בנים רבים ובחר ביצחק ועזב ישמעאל ובני קטורה, וכן היה הענין ביצחק שעם היותו עקר ואשתו רבקה עקרה נפקדה בדרך נס ונולדו להם יעקב ועשו, ועם היותם אחים בחר ביעקב ולא בעשו, וכמאמר הנביא "הלא אח עשו ליעקב נאום ה' ואוהב את יעקב ואת עשו שנאתי" (מלאכי א' ב'-ג').ולכן אף כי שניהם בני האבות אף על פי כן לא נשתוו ביעודי אבותיהם, כי הוא יתברך נתן לעשו הר שעיר לרשת אותו, רצה לומר שנתן לו זה מתנה בעד זכותו כדי להרחיקו על יעקב, באופן שיעקב ובניו ירשו בנחלת אברהם ויצחק בירושת הארץ הנבחרת צבי הוא לכל הארצות, שזו היתה המנוחה והנחלה האמיתית הנפשית והגופנית, וכדי שיזכו בה יעקב ובניו לבדם נתן ה' לעשו תמורתה את הר שעיר. והתבונן מה שאמר המגיד "ויעקב ובניו" שהודיע לנו יהושע שר צבא ה' בזה שיעקב היתה מטתו שלמה כי הנה תרח ואברהם בנו ויצחק בן בנו בכולם היתה בתוך הבנים קליפה סביב הלב שהוא העיקר,  לכם נזכר כאן נחור להודיע שהיה אחיו של אברהם והפכו ולא נבחר עמו, וכן בבני אברהם נבחר יצחק ונעזבו האחרים ומבני יצחק נבחר יעקב ונעזב עשו, אבל יעקב לא היה כן, כי אף שהיו לו שנים עשר בנים היו כולם בחורי ה', כלם היו לב מבלי קליפה, כולם היו יסודות האומה ועמודיה, וזהו שאמר "ויעקב \textrm{\textbf{ובניו}}" ומה שאמר "ירדו מצרימה" יבואר אחר כך. והותרו בזה הספקות שרשמתי שער ל"ה ול"ו ושער ל"ז.}%endcomment
\hebeng{וָאֶקַּח אֶת־אֲבִיכֶם אֶת־אַבְרָהָם מֵעֵבֶר הַנָּהָר וָאוֹלֵךְ אוֹתוֹ בְּכָל־אֶרֶץ כְּנָעַן, וָאַרְבֶּה אֶת־זַרְעוֹ וָאֶתֵּן לוֹ אֶת־יִצְחָק, וָאֶתֵּן לְיִצְחָק אֶת־יַעֲקֹב וְאֶת־עֵשָׂו. וָאֶתֵּן לְעֵשָׂו אֶת־הַר שֵּׂעִיר לָרֶשֶׁת אתוֹ, וְיַעֲקֹב וּבָנָיו יָרְדוּ מִצְרָיִם. }{And I took your father, Avraham, from over the river and I made him walk in all the land of Canaan and I increased his seed and I gave him Yitschak. And I gave to Yitschak, Ya'akov and Esav; and I gave to Esav, Mount Seir {[in order that he]} inherit it; and Yaakov and his sons went down to Egypt.'"}
\hebeng{בָּרוּךְ שׁוֹמֵר הַבְטָחָתוֹ לְיִשְׂרָאֵל, בָּרוּךְ הוּא. שֶׁהַקָּדוֹשׁ בָּרוּךְ הוּא חִשַּׁב אֶת־הַקֵּץ, לַעֲשׂוֹת כְּמוֹ שֶּׁאָמַר לְאַבְרָהָם אָבִינוּ בִּבְרִית בֵּין הַבְּתָרִים, שֶׁנֶּאֱמַר: וַיֹּאמֶר לְאַבְרָם, יָדֹעַ תֵּדַע כִּי־גֵר יִהְיֶה זַרְעֲךָ בְּאֶרֶץ לֹא לָהֶם, וַעֲבָדוּם וְעִנּוּ אֹתָם אַרְבַּע מֵאוֹת שָׁנָה. וְגַם אֶת־הַגּוֹי אֲשֶׁר יַעֲבֹדוּ דָּן אָנֹכִי וְאַחֲרֵי־כֵן יֵצְאוּ בִּרְכֻשׁ גָּדוֹל. }{Blessed be the One who keeps His promise to Israel, blessed be He; since the Holy One, blessed be He, calculated the end {[of the exile,]} to do as He said to Avraham, our father, in the Covenant between the Pieces, as it is stated (Genesis 15:13-14), "And He said to Avram, 'you should surely know that your seed will be a stranger in a land that is not theirs, and they will enslave them and afflict them four hundred years. And also that nation for which they shall toil will I judge, and afterwards they will go out with much property.'"}%
\commenta{\textrm{\textbf{ברוך שומר הבטחתו לישראל}} צריך באור מה ריבותא היא לגבי ה׳ ששמר הבטחתו, והלא גם ממדת בשר ודם נאמן רוח לשמור מה שמבטיח. ואפשר לומר הכונה, כי בשעה שאמר ה׳ לאברהם ענין ההבטחה מירושת הארץ בפעם הראשונה ״הארץ אשר אתה רואה לך אתננה ולזרעך״ (פרשה לך, י״ג ט״ו) באותה שעה לא היו לו עוד זרע, ואח״כ נולד ישמעאל, ואם כן הי׳ אפשר שהבטחה זו תתייחש אליו ויזכה הוא בנחלת הארץ. אך במקום אחר פירש הקב״ה הבטחתו זו לאברהם (אתננה לזרעך) בזה שאמר לו כי ביצחק יקרא לך זרע (פ׳ וירא, כ״א י״ב), וא״כ בשעה שאמר לראשונה אתננה לזרעך כיון למי שהוא קרוי זרעך, ליצחק, ועל זה אנו מודים לו שקיים שם זרעך ביצחק. ודע, כי כעין דיוק זה שדייקנו כאן מה ריבותא לגבי הקב״ה שהוא שומר הבטחתו, בעוד שזה גם ממדת בו״ד נאמן רוח — כעין זה דייקנו בברכת הודאה ברוך שאמר, שאומרים שם ברוך גוזר ומקיים, והנה גם בפרט זה אין כל ריבותא לגבי׳, כי הלא גם מלך בו״ד מקיים גזירתו, כי אם למשל שופטים הבאים מכחו דנים אחד לעונש יקיימו בו תוצאת הדין כפי גזירת המלך. אך הביאור הוא, כי לפעמים אין בכח המלך בשום אופן לקיים גזירתו, ולא רק הוא לבדו, אך גם אם יסייעוהו לזה כל מלכי תבל לא יעלה זה בידם, וזה יצוייר, אם למשל גזר על אחד לישב במאסר במשך שלש שנים, וגם נאסר, אך לאחר שנה לשבתו מת שם — אז אין כל אפשרות לקיים הגזירה שישב במאסר שלש שנים, כי לא יחזיקו שם גוף מת. אבל בכח הקב״ה, אם הוא גוזר על איש להיות מעונה במשך שלש שנים (וכן להיפך למצוא טוב במשך זמן כזה) — אז בטוח הוא האיש שיוציא שנותיו אלה בחיים, וגזירת ה׳ תתקיים במלואה, ולכן רק עליו יתברך יונח התואר גוזר ומקיים, והיא תהלתו ותקפו. ועוד דיוק כזה שבכאן ובברוך שאמר דייקנו בתפלת שמו״ע בנוסח אל מלך רופא נאמן ורחמן אתה, ודייקנו כי הלא גם ברופאים אנשים יקרה בעל מעלות אלו. רופא נאמן ורחמן, ומה ריבותא הוא לגבי ה׳, ושם בארנו זה היטב, וטורח להעתיק. עיי״ש.}%endcomment%
\commentb{\textrm{\textbf{תשובות לשערים ל"ח – מ"א}}\textrm{\textbf{ברוך שומר הבטחתו לישראל ברוך הוא שהקדוש ברוך הוא מחשב את הקץ וכו'.}} פירוש המאמר הזה אצלי הוא שבעבור שחשב למעלה את הייעודים אשר ייעד ה' לאברהם בריבוי הזרע וירושת הארץ להיות לו לעם, ושכל אותן ההבטחות נתקיימו ביעקב ובניו ולא בישמעאל ובני קטורה, עם היות כלם יוצאי ירך אברהם, נתן בעבור זה שבח והודיה להשם יתברך באומרו "ברוך שומר הבטחתו לישראל", רצה לומר ברוך הוא ומשובח האל יתברך ששמר אותה ההבטחה שהבטיח, שמר אותה לתתה לישראל שהוא יזבח בה ויהנה ממנה, ולא עשו וזרעו ושאר בני אברהם. וזהו הדבר בעצם שאמר משה רבנו עליו השלום "וְהָיָה עֵקֶב תִּשְׁמְעוּן אֵת הַמִּשְׁפָּטִים הָאֵלֶּה וּשְׁמַרְתֶּם וַעֲשִׂיתֶם אֹתָם וְשָׁמַר יְהוָה אֱלֹהֶיךָ לְךָ אֶת הַבְּרִית וְאֶת הַחֶסֶד אֲשֶׁר נִשְׁבַּע לַאֲבֹתֶיךָ" (דברים ז', י"ב). ואין הכוונה בשמירות האלה שהשם יתברך יקיים את דברו ולא יחללו, כי כל משפטיו צדק ואמת ואין מי שיסתפק בהבטחתו, אבל הכוונה שאותו הברית והבטחת החסד שייעד ונשבע לאבות לא נתנה לישמעאל וזרעו ולא לעשו וזרעו, אלא שמר אותה והפרישה לתתה לישראל שיזכו בה ולא אחרים. ובדרך זה נוכל לפרש מה שיאמר אחר זה "והיא שעמדה לאבותינו ולנו", רצה לומר אותה ההבטחה עמדה ערבה בכל ושמורה לאבותינו השבטים בני יעקב ולנו, ובעבורה ובזכותה אנו ניצולים מכל צרה ופגע, אף כי בכל דור ודור עומדים עלינו לכלותינו. ולפי זה "ברוך שומר הבטחתו לישראל", ענינו ששמר הבטחת אברהם לתתה לישראל ולא לשאר הבנים, והזכיר רצון ה' וחפצו לזכותם בהבטחה ההיא, בהיותו מחשב את הקץ יום ליום ולילה ללילה מתי יבוא זמן הקץ כדי לעשות ולקיים מה שאמר לאברהם אבינו בברית בין הבתרים, רצה לומר באותו מראה שבתר העגלה המשולשת והאיל המשולש שאז אמר לו: "יָדֹעַ תֵּדַע כִּי גֵר יִהְיֶה זַרְעֲךָ בְּאֶרֶץ לֹא לָהֶם וַעֲבָדוּם וְעִנּוּ אֹתָם אַרְבַּע מֵאוֹת שָׁנָה" (בראשית ט"ו, י"ג).אמר יצחק: יען וביען המגיד קיצר כאן בזכרון הגלות ובסבתו, וממראה בין הבתרים הביא סופה ולא תחילה, הביא תשובת ה' לאברהם "ידוע תדע וגו'" ולא הביא שאלתו "במה אדע כי אירשנה", לכן משום שלמות המלאכה, וכדי להבין הדרשה על בוריה ראיתי להרחיב פה הדיבור בדרוש זה ובכתובים שבאו בהם המראה בברית בין הבתרים, כדי שנעמוד על סבת גלות כפי האמת, וממנה נעמוד על חסדי ה' בגאולתנו ופדות נפשנו ועם זה נבין ענין יציאת מצרים בסיבותיה ותהלוכותיה. וכדי להיישיר לפני הדרך בדרוש הזה אעיר על הספקות אשר יתחייבו במראה בין הבתרים המתיחסות לעניננו זה.ראשונה, במה ששאל אברהם "במה אדע כי אירשנה", כי למה לא שאל אות על העיקר שהוא הזרע שהיה מיעדו עליו ושאלו על ירושת הארץ הנמשכת אחריו? והנה הרמב"ן והרלב"ג והר"ן שלושתם הסכימו שייעוד זה מהזרע היה הטבה לאברהם לבדו ולכן לא שאל עליו, כי ידע בעצמו שלא ימנעהו חטאו, וגם כי ייעוד הטוב אינו חוזר, כמו שקיבלו חז"ל כל ייעוד שיצא מפי הקדוש ברוך הוא לטובה אפילו על תנאי אינו חוזר, אבל בירושת הארץ אף שהיא טובה לישראל היא גם כן עונש ורעה לעממים, ואולי ישובו בתשובה כאנשי נינוה. ולזה הוצרך אברהם לשאול אות ברית על קיומו כדי שלא תמנע ירושת הארץ לזכות הכנעניים ולא בשביל חטאת בני ישראל. ופלא הוא כי האנשים השלמים האלה הסכימו פה אחד לדעה מוטעה כזו, כי באמת הייעוד עם הברית כמו בלתו לא ימנע ההחזרה מהרע בזכות המקבלים, כי איך אפשר לומר שבעבור הברית זה שעשה הקדוש ברוך הוא עם אברהם ינעל מאנשי הארץ דלתי התשובה? "חלילה לאל מרשע ושדי מעול!" (איוב ל"ד, י') .ושנית למה שאל אברהם במראה הזאת אות על ירושת הארץ ולא שאלו קודם לזה בשאר הייעודים שהבטיח השם יתברך עליהם? וכבר התחכם הר"י בן ג'יקטיליא ז"ל להשיב על זה באמרו שקודם הייעוד הזה לא ייעד הקדוש ברוך הוא לאברהם כי אם על נתינת הארץ, ובכאן שאל על ירשתה, שהירושה היא התמדת בניו בה לנצח ולא יסחו ממנה בסבת עבירות שיעשו, וזהו שאמר "במה אדע כי אירשנה" רצה לומד באיזה זכות יתמידו בירושתה? והשיבו השם יתברך שבזכות הקורבנות יתמידו כי בהם יתכפרו עונותיהם. עד כאן לשונו. ונראה שדייק זה מדברי רש"י ז"ל שכתב הודיעני נא באיזה זכות יתקיימו בני בארץ, ואמר לו הקדוש ברוך הוא שבזכות הקורבנות. אבל הדעה הזאת היא אצלי דבר בטל, לפי ששם ירושה לא יורה על הנצחיות וההתמדה כי אם שיירשנה היורש מיד אחרי מות המוריש מבלי התבוננות אם יתמיד בירושה או לא. וכמוהו "ותורישני עונות נעורי" (שם י"ג, כ"ו), "אשר יורישך כמוש אלהיך אותו תירש" (שופטים י"א, כ"ד), ובאמת יורה יותר לשון נתינה על הנצחיות מלשון ירושה. כי הדבר הניתן לאדם במתנה הוא לעד הנותן לא השאיר לו שום זכות במתנתו. ועוד כי קודם לזה כבר נאמר לאברהם אחרי הפרד לוט מעמו על ירושת הארץ: "לך אתננה ולזרעך עד עולם" (בראשית י"ג, ט"ו) ובאומרו עד עולם מורה על הנצחיות כמו "מעולם ועד עולם אתה אל" (תהלים צ', ב'), וכן פירש הרמב"ן. ועוד שאם נאמר כי במראה הזאת היה הברית על ירושת הארץ ולא על נתינה, אם כן איך אמר בסוף המראה "ביום ההוא כרת ה' ברית את אברם לאמר  לזרעך נתתי". והיה ראוי שיזכיר שם לשון ירושה או לשון נתינה. ועוד שאם אברהם שאל על נצחיות הארץ והתמדת ירושתה ועליו בא הברית בזכות הקרבנות, מדוע לא נתקיים הברית כל זמן שהקרבנות לא נתבטלו? ואיך בא צר ואויב בשערי ירושלים וגלתה יהודה מעוני ומרוב עבודה? הלא מעולם לא שמענו שגלו יהודה וישראל על ביטול קרבנות כי אם על עבודה זרה, גילוי עריות ושפיכות דמים.השאלה השלישית שראוי לשאול על אמרו "במה אדע כי אירשנה", כי אם כיון לשאול אות היה לו לומר כמו חזקיהו המלך "מה אות כי אעלה בית ה'" (ישעיה ל"ח, כ"ב), וגם לא מצינו כן שנתן לו האל יתברך אות כלל. ואם לא ביקש אות כי אם חיזוק וקיום ברית שיירשנה ולא יגרום חטא זרעו או תשובת גויי הארץ כדברי המפרשים, אם כן לא היה ראוי שיאמר "במה אדע כי אירשנה" כי אם "השבעה לי כיום" או "כרות נא לי ברית", כי לשון "במה אדע" אינו מורה על החיזוק והקיום כי אם על חידוש ידיעה אשר לא נתחדשה שמה.השאלה הרביעית, באומרו עגלה משולשת ועז משולשת ואיל משולש ותור וגוזל, קשה לי למה לא היו התור והגוזל משולשים כבהמות, כי אם פירוש משולש שלשה מכל מין כדברי רש"י, ורלב"ג ורמב"ן ז"ל, היה ראוי שיבואו גם מהתור והגוזל שהם אפרוחי היונים שלשה מכל אחד מהן. ואם הפירוש משולש הוא בן שלוש שנים או שלשה שבועות או ימי, אם השלוש מורה על הזמן כדברי הרמב"ן והר"ן מדוע לא יהיו גם כן התור והגוזל משולשים?השאלה החמישית באומרו "ויבתר אותם בתוך ואת הציפור לא בתר" אם היה הברית ההוא הבתירה כמו שנאמר העגל אשר כרתו לשנים ויעברו בין בתריו, היה צריך שיבתר כל בעלי החיים שבאו לתכלית הברית?וראוי שתדע כי במשל המראה הזאת באו שתי דעות, ועל שניהן העיר רש"י ז"ל בפירושו, האחת רמז לקרבנות שיעשו ישראל מהמינים האלה שבזכותם יירשו את הארץ וינחלוה, ולדעה זו נמשכו המפרשים כולם, ועליו אמרו בבראשית רבה שהראה לו הקדוש ברוך הוא לאברהם שמבדילין (בסימנים) בעולם הבהמה ואין מבדילין בעוף. והדעה הזו אינה מתישבת אצלי כפי פשט הכתובים, לפי שלא זכו ישראל לירש את הארץ בזכות הקרבנות, ולא נזכר זה בכתוב כלל. אבל נזכר שזכו בה מפני שבועת האבות, והקרבנות נצטוו עליהם על כונה אחרת, כמאמר הנביא "לא דברתי עם אבותיכם ולא צויתים ביום הוציאי אותם מארץ מצרים על דבר עולה וזבח, כי אם את הדבר הזה צויתי אותם לאמר שמעו בקולי" (ירמיהו ז', כ"ב) וגו', ואסף המשורר אמר "לֹא עַל זְבָחֶיךָ אוֹכִיחֶךָ", "לא אקח מביתך פר" וגו', "זבח לאלוהים תודה" וגו' "וקראני ביום צרה" וגו' (תהלים נ', ח'). שכל זה מורה שלא היו הקרבנות כי אם על הכוונה השניה, וכמו שפירש הרב המורה בחלק השלישי פרק ל"ד. ואם כן איך נאמר שבמעמד בין הבתרים יודיעהו ה' שבזכות הקרבנות ירשו בניו הארץ, וכל שכן שלא יתכן שיבא לו פרטי הדינים שאין הבדלה בעוף. הדעה השניה שהביא רש"י ז"ל הוא מדברי המדרש שהעגלה רמז לאומות בכללם וסבבוני פרים רבות, והאיל רמז למלך פרס כמו שנאמר "האיל אשר ראית בעל הקרנים" (דניאל ח', כ'), והעז רמז למלכי יון והוא צפיר העזים שראה דניאל, וישראל נמשלו ליונה שנאמר "יונתי בחגוי הסלע" ולפיכך בתר את הבהמות לרמז שהיו העמים כלים והולכים, והציפור לא בתר שיהיה ישראל קיים לעולם. וגם הדעה הזאת לפי סגנונה בלתי נאותה כפי הכתובים, כי אם בעגלה נרמזו האויבים והאומות כולן, לא היה צורך לרמז האיל על פרס והעז על יון אחרי היותן נכללות באומות? ואם רצה לרמוז ביחד על אלו לפי שהרעו לישראל, קשה למה לא עשה רמז גם כן על מלך בבל שהחריב בית ראשון ועל אדום שהחריב בית שני, ויהיו החיות במספר ארבע כמו שראה דניאל, וכמו שדרשו על אימה חשכה גדולה נופלת עליו שירמוז להם. וקשה עוד לדעה זאת, למה הביא בכתוב הזה העז קודם לאיל בחיות שפרס הנרמז באיל היה קודם יון הנרמז בעז? וגם קשה על מה רמז התור אחרי שבן היונה הוא הרומז על ישראל, ועוד מה הרווחנו אם אברהם לא בתר את התור והגוזל אחרי שהמית אותם והרי הם כַּלִים כמו הבהמות? כי מדברי המפרשים כלם משמע שהמית התור והגוזל, וכן משמע במדרש שהראה לו הקדוש ברוך הוא שמבדילים בבהמה ואין מבדילים בעוף, ולא יפול המאמר הזה כי אם בשהמיתם. סוף דבר שתי הדרשות לא יתכנו כפי סדר הכתובים והוראתם.השאלה השישית, באמרו "וישב אותם אברם", רשה להבין אם זה חוזר לעיט אשר הזכיר היה ראוי לומר "וישב אותו אברם" בלשון יחיד ולא בלשון רבים, גם קש/ה מה הועיל אברם בזה שנשב העיט מעל הפגרים אחרי שנשארו אותם הפגרים ומיד כאשר ילך לביתו יהיו מאכל לעוף השמים ולחית הארץ?השאלה השביעית אודות הגלות שנאמר "ידוע תדע כי גר יהיה זרעך" ומובא במסכת נדרים (סוף פרק ארבעה נדרים) שלשה דעות, הראשון לר' אבהו שאמר מפני מה נענש אברהם אבינו ונשתעבדו בניו מאתים ועשר שנים מפני שעשה אנגריא בתלמידי חכמים שנאמר "וירק את חניכיו ילידי ביתו", השני לשמואל שאמר מפני שהפריז על מידותיו של הקדוש ברוך הוא שנאמר "במה אדע כי אירשנה", והשלישי לר' יוחנן שאמר מפני שהפריז על מידותיו של הקדוש ברוך הוא  מלהביא תחת כנפי השכינה שנאמר "תן לי הנפש והרכוש קח לך". והנה הדעות הללו כולן מלבד שהן חלישות, עוד יכללו ספק עצום, והוא שאברהם החוטא לא נענש כלל, וכמו שהבטיח לו ,"ואתה תבוא אל אבותיך בשלום תקבר בשיבה טובה" (בראשית ט"ו, ט"ו), ונענשו עליו דור השלישי והרביעי ליוצאי חלציו על לא חמס בכפם, ועל יצא בזה נאמר "אבות יאכלו בוסר ושיני בנים תקהינה" (יחזקאל י"ח, ב'). וקשה גם כן למה היה הגלות במצרים יותר מבשאר הארצות? וכבר נתנו החכמים על זה טעמים אך הם חלושים מאוד. אמנם הרמב"ן הביא בזה דעת רביעי והוא שהיה חוטא אברהם כשהביא אשתו הצדקת למצרים כנסיון באומרו אחותי היא מפחד שיהרגוהו, וחטא גם כן בצאתו מפני הרעב מארץ כנען אשר בחר בה ה' ואשר הוריש לזרעו אחריו, והיה לו לבטוח בה' שברעב יפדהו ממות, ועל זה החטא נגזר על זרעו הגלות במצרים, מקום הרשע שמה המשפט. אבל לפי זה נעשנו בני אברהם בעון אביהם ולא נענש החוטא כלל רק הנקיים. ומלבד זה כבר הכה על קדקודו הר"ן באמרו שהדבר הזה היה אחד מהנסיונות שנתנסה בהם אברהם ועמד בכולם ואיך יחשב לו זאת לעון? ואם באמת היה חוטא בדבר הזה איך שגה באולתו בבואו אל ארץ אבימלך וחזר לומר שם גם כן על שרה אשתו "אחותי היא"? והנה הביא הר"ן דעת חמישי שאמר שלא היתה הגלות על החטא כלל כי אם להכניע ליבותם של ישראל כפי שיהיו ראויים לקבל את התורה, ושהיה זה ככלל יסורין של אהבה, וגם זה איננו שווה לי, כי יסוד הדעה הזאת מסופק מאוד, וכבר אמרו חז"ל "אין מיתה בלא חטא ואין יסורין בלי עון" (שבת נ"ה), וכתב הרב המורה בחלק ג' פרק כ"ד9מורה נבוכים, הרב המורה הוא הרמב"ם., שזו היא דעה אמתית. וגם הרמב"ן בשער הגמול שלו מרחיק מאוד שיבואו יסורין בלא עון. והנסיון שהביא הרב לראיה אינה טענה, וכבודו במקומו מונח, כי הוא מונח בכלל הניסיון בשלא יצאו הרע והיסורין לפועל אחרי שהיו מעותדים לבוא בענין העקדה. ועוד כי אף שנודה באיש יחיד שלפעמים יבואו עליו יסורין להיטיב לו באחריתו, אבל איך אפשר שאמר זה באומה בכללה המושגחת מהשם יתברך שתבוא בגזרת ה' לידי גלות בלי עון קודם?השאלה השמינית באמרו "ויהי השמש לבוא ותרדמה נפלה" וגו'. קשה כי לא מצינו בנבואות אברהם שיזכיר הכתוב העת מהיום אשר חל עליו שפע הנבואה, ואין בהגבלת הזמן תועלת כלל. וכתב הר"ן שהגביל הכתוב העת מהיום מצד שראה הקדוש ברוך הוא להעביד לפיד אש בין הגזרים אחרי שיחשוך לגמרי, לפי שאור האש יורגש בחשכת הלילה יותר מאשר יורגש ביום, כמו שאמרו חז"ל "שרגא בטיהרא למאי אהני"10חולין, ס' ב'. בעברית: "נר בצהריים מה מועיל?". ועם היום זה טעם נאות בפסוק "ויהי השמש באה ועלטה היה והנה תנור עשן ולפיד אש אשר עבר", שהזכיר שם שהיתה העברת האש בלילה להיותה יותר מורגשת, הנה לא יצדק באומרו כאן ויהי השמש לבא, לפי שלא נזכר כאן העברת הלפיד אש, רק אמר לאברהם ידוע תדע כי גר יהיה זרעך וגו'. ולהודעה הזאת לא היה צריך להגביל העת מאותה סבה שכתב הרב.השאלה התשיעית באומרו "ועבדום וענו אותם ארבע מאות שנה", וידוע בדברי חז"ל שהיתה העבדות במצרים רק מאתים ועשר שנה, והעינוי היה 87 שנה בלבד. ובסדר "בא אל פרעה" נאמר "ומושב בני ישראל אשר ישבו במצרים שלושים שנה וארבע מאות שנה" ונראה מזה גם כן שלא היה ה' מחשב את הקץ לקרבו אלא שהוסיף על הגלות. ואם היו שנות הגלות קצובות מאתו יתברך איך אמר הכתוב בסדר ואלה שמות "ויאנחו בני ישראל מן העבודה ויזעקו ותעל שועתם אל האלהים מן העבודה" (שמות ב, כ"ג) משמע שמפני זעקתם נגאלו ולא מפני שנשלם זמן הקץ. והנה רש"י ז"ל כתב שארבע מאות שנה מהשעבוד התחילו מעת שנולד יצחק, ומדברי בעל סדר עולם11סדר עולם המכונה גם סדר עולם רבה הוא חיבור העוסק בכרונולוגיה יהודית, מימי אדם הראשון, עד חורבן בית שני ומרד בר כוכבא. הכרונולוגיה שבספר מונה את השנים השלמות שקדמו למועד הנזכר. הספר מבוסס על מסורות ודרשות. על פי המסורת, כתיבת הספר החלה בתקופת התנאים. לפי המסורת התלמודית הוא מיוחס לתנא רבי יוסי בן חלפתא הספר בצורתו הנוכחית נערך ונכתב במשך הדורות. לפי השערת החוקר יום-טוב ליפמן צונץ, עריכתו המאוחרת של הספר היא משנת 806. הוא, והדעה הזאת בנויה על מה שנאמר שאחר שיצא אברהם מחרן חזר וישב שם חמש שנים, ועל הפעם האחרונה נאמר "ואברהם בן חמש שנים ושבעים שנה בצאתו מחרן" (בראשית י"ב, ד'), אך הדבר הזה לא נזכר בכתוב. והרלב"ג כתב שארבע מאות שנה התחילו משנולד יעקב והקדוש ברוך הוא מהר את הקץ וגאלם קודם זמנו אשר יעד, והוא באמת דעה זרה מאוד ומתנגדת לכתובים.השאלה העשירית, באמרו "ואתה תבוא אל אבותיך בשלום", וקשה כי אבות אברהם היו עובדים עבודה זרה ואיך יאמר שיבוא אליהם? הלא גופו לא נקבר עם גופם ונפשו חלילה שתהיה במחיצתם. ורש"י כתב להנצל מזה הספק, למדך שעשה תרח תשובה, אבל זה לא נזכר בכתוב. ולא אמר "ואתה תבוא אל אביך בשלום, כי אם על "אבותיך", ואף אם נודה שתרח עשה תשובה מי יאמר שגם נחור אביו של תרח ושרוג אביו של נחור עשו גם כן תשובה? והנה ביעקב כתיב "ושכבתי עם אבותי" וכן ברור לפי שהיו צדיקים, מה שאין ראוי לומר באברהם שאבותיו היו נכרים.השאלה האחד עשר, באמרו "ודור רביעי ישובו הנה" וקשה אם יקרא בן לכל דור הנולד, הנה לא בא הדור הרביעי לירש אץ הארץ, כי צא וחשוב יצחק, יעקב, יהודה, פרץ, חצרון וכלב שבא אל הארץ הרי דור ששי? ורש"י ז"ל כתב לאחר שגלו למצרים יהיו שם שלושה דורות והדור הרביעי ישובו לארץ הזאת, הרי שבארץ כנען היה מדבר עמו וכרת אתו הברית, דכתיב "לתת לך את הארץ \textrm{\textbf{הזאת}} לרשתה" (שם ט"ו, ז'), וכן היה כי יעקב גלה למצרים, צא וחשוב דורותיו יהודה, פרץ, חצרון וכלב מבאי הארץ היה וכו'. וכתב עליו הרמב"ן שאינו נכון כלל ולא ביאר למה, ואחשוב כי לפי שלא נזכר בכתוב גלות מצרים שימנה בדורות ממנו כי אם מן גרות הזרע, וגם אם ימנה הדורות מיורדי מצרים היה ראוי שיהיה יעקב בכללם והוא הדור הראשון וכלב היה הדור החמישי שבאו לארץ ולא דור הרביעי? ודעת הרמב"ן שהדור הרביעי הוא לאמורי היושב בארץ, והבל הביא גם הוא, כי לא נזכר בכתוב הכנעני שיאמר עליו ודור הרביעי, גם מה שאמר "ישובו הנה" לא יסבול פירושו בשום צד.השאלה השנים עשר, באמרו "כי לא שלם עוון האמורי עד הנה", ויראה מזה המאמר שלא נגזר הגלות מארבע מאות שנה להיותם ראויים שליו כפי הדין להיענש כל אותו הזמן כי אם מטעם לפי שעדיין לא נשלם עוון האמורי שינערו מן הארץ, ולכן צריכים בני ישראל להתמהמה במצרים עד שיושלם עוון האמורי, והוא דבר זר וקשה להאמין בחוק היושר האלהי, שיענשו ישראל בגלות ארוך ללא סיבה עד אשר יהיו הכנענים ראוים ליענש. ויותר טוב היה שיהיו ישראל עומדים בארץ כנען גרים ותושבה בשובה ונחת כל אותו הזמן. ועוד קשה שיאמר כי לא שלם עוון האמרי ואם כי לא היו עדיין חטאים, ואיך נגזרה עליהם הגזרה קודם החטא, ותהיה לפי זה גזרת ישראל בגלות וגזרת הכנענים בגירוש שתיהן בלא קדימת חטא אשר לא כדת?אלה הן השאלות אשר יפלו בדרוש הזה, ואתה רואה שהן מעצם סיפור הגלות ויציאת ישראל ממצרים, ושצריכים אנחנו לדעת הדברים האלה כולם על אמיתתם כדי שמהם נעמוד על חסדי ה' שעשה עמנו ביציאת מצרים.ואומר שדעתי בזה היא, שאברהם אבינו לא שאל אות ולא ברית על ירושת הארץ, כי כבר יעדו השם יתברך עליו פעמים רבות ולא היה מסַפֵק בדבור האל חלילה, אמנם הספק שהיה אצלו בדבר הזה הוא אם ירושת הארץ תהיה בימיו ומידו תשאר לבניו, או אם לא יירשנה הוא בעצמו כי אם זרעו אחריו ולא ידע איזה דור מזרעו ירש את הארץ, אם בנו אן בן בנו או אם בא היעוד על שילשים ועל ריבעים, או אם יהיה ה' עושה החסד לאלפים מהדורות באז יתקיים הייעוד. ונפל אצלו הספק הזה לפי שבראשונה אמר לו ה' "לזרעך אתן את הארץ הזאת" ולא פירש מהו הזרע והדור אשר יזכה לכך, ואחרי הפרד לוט מעמו אמר לו "כי את כל הארץ אשר אתה רואה לך אתננה ולזרעך" שמורה שבימי אברהם תהיה ירושת הארץ מידו ינחלוה בניו וזרעו. עוד אמר לו "קום התהלך בארץ לארכה ולרחבה כי לך אתננה" הורה גם כן שאליו ביחוד יתננה. וכן אמר לו בזה המחזה: "אני ה' אלוהיך אשר הוצאתיך מאור כשדים לתת לך את הארץ לרשתה". לכן נסתפק אברהם ביעודים השונים האלה אם הוא עצמו ירש הארץ או זרעו אחריו, ואם זרעו איזה דור ממנו יהיה? וכאשר ראה נצחונו במלכים נדמה לו שעת לעשות לה' ואולי זו היא ההתחלה שיגבור על הארץ אשר הוא עליה ויירשנה. ועל הספק הזה באה השאלה "במה אידע כי אירשנה?" וידוע שמילת "כי" תורה ותשמש על הזמן כמו "כי תבואו אל הארץ", "כי ימצא בקרבך", ואמר – ה' אלוהים במה אדע הזמן והעת אשר אז אירשנה. או יאמר במה אדע אם אירשנה אני בעצמי או אם תהיה הירושה לזרעי אחרי, שבאו בייעוד הזה מאמרים מורים על שאני אירשנה ומאמרים המורים שזרעי יירשוהַ.והראיה המוכחת על היות זאת שאלת אברהם באמת ולא קיום וחיזוק הברית על הירושה, היא התשובה שהשיבו השם יתברך שהודיעו שיַגלה את זרעו ארבע מאות שנה ודור הרביעי ישובו הנה. ואם היתה השאלה ההיא על חיזוק וברית לא היה מקום לתשובה כזו, אבל מאשר שאל על זמן הירושה באה לו עליה התשובה. ועם הדברים האלה תפתרנה שלושת השאלות הראשונות אשר העירותי בכתובים. האמנם מראה הבהמות והעופות שבאו על זה היה מיוחס לענין התשובה, לפי שכיון בו השם יתברך שתי כוונות, אחת שהיו העגלה והעז והאיל משל לשלשת האבות, ושרשי היחס אברהם יצחק ויעקב המשילם בשלושה אלה לפי שרוב הבהמות הטהורות הטובות והביתיות הן מאלה, ולהיותם שורש הריבוי נמשלו לאבות שיצא מהם זרע רב והמונים עצומים, וראוי היה להִמַשֵל אברהם בעגלת בקר כמו שנאמר בו "ואל הבקר רץ אברהם, ויקח בן בקר" וגו', ויצחק נמשל לאיל לפי שנעקד במקומו כמו שנאמר "ויקח את האיל ויעלהו לעולה תחת בנו", והעז נמשל ליעקב מפני שהלך לגנוב את ברכות אביו בשעירות ידיו מעורות גדיי עיזים, כי כן אמרה לו אמו "וקח לי משם שני גדיי עיזים", ואין להפליא שלא נזכרו על הסדר כי באיל שרמז ליצחק בא באחרונה, כי לפעמים תמצא גם בכתוב שלא נזכרו האבות בסדר, כמו שנאמר "וזכרתי את בריתי יעקב  ואף את בריתי יצחק ואף את בריתי אברהם אזכור והארץ אזכור", הרי לך זכרם בחילוף תולדותם, והנביא אמר "כי אברהם לא ידענו וישראל לא יכירנו" (ישעיהו ס"ג, ט"ז), שסמך ישראל לאברהם ולא סמך ליצחק, וכן נזכר כאן יעקב בין אברהם ויצחק.ולהיות העגלה והעז והאיל רומזים אל האבות אמר בהם משולשת ומשולש, לא להיותם בן שלוש שנים או שלשת מכל מין כמו שחשבו המפרשים כי אם להיותם משובחים מאותו שלוש משרשי היחס המשלשים קדושה, ובא זה להודיע לאברהם שירושת הארץ לא תהיה בימי העגלה ולא בימי האיל ולא בימי העז, כי כולם ימותו ולא יזכו בה, והוא ענין "ויבתר אותם בתוך". וצוה שיקח עוד תור וגוזל והוא רמז לשני מנהיגים אחרים שיבואו אחרי האבות זה אחר זה והם משה רבנו ויהושע בן נון ע"ה ועליהם אמר "ואת הציפור לא בתר", רצה לומר שאברהם לא הרג אותם ולא בתרם כי נשארו חיים, לרמוז שהמה יירשו את הארץ, כי הנה משה התחיל בירושתה במלחמת סיחון ועוג ויהושע השלים הירושה והכיבוש, לכן אמר וירד העיט על הפגרים שהם העגלה והעז והאיל, שבאו העופות ואכלום, רמז למיתתם. אבל על תור ואיל נאמר "וישב אותם אברם", ולא נאמר זה על העיט כי אם על התור והגוזל, שאברהם בתר הבהמות ואכלום עוף השמים ואת הציפור שהוא שם כולל לתור וגוזל לא המית ולא בתר אותם, אלא הפריחם באויר חיים על פני השדה, רומז למשה ויהושע הנמשלים בהם כי המה יעברו ארחות ימים כציפור לעוף והמה יירשו ארץ, ולכן המשילם בהם. זהו הפן הראשון מהדימוי והמשל במראה הזאת.והפן השני הוא שרמז השם יתברך בעגלה בעז ובאיל  לשלשת הנביאים אחים ואחות רועי ישראל אשר הוציאם ממצרים, והם מרים ואהרן ומשה, וכפי ימיהם ותולדותם נזכרו העגלה ראשונה שהיא מרים והעז רומז לאהרון והאיל רומז למשה, כי נמצאו בענייניהם דברים אשר יתדמו לתכונת הבהמות האלה, ולזה אמר בהן משולשת ומשולש, והיה חוט משולש מהם, כאילו אמר העז שהוא מאותו שילוש והאיל שהוא מאותו שילוש וכן בעגל, לפי ששלושתם היו עתידים למות במדבר ושלא יכנסו לארץ, לכן אמר שבתר אותם בתוך, שירד העיט על הפגרים, שהוא משל למלאך המוות המושל על כל חי. ואמנם התור והגוזל שהם רמז ליהושע בן נון ואלעזר הכהן שיבואו ויכנסו לארץ, אחד בתור מנהיג העם ושוטר ומושל ואחר כהן לאל עליון, ולכן לא בתר אותם ולא המיתם; "וישב אותם אברם" שהפריחם באויר כציפור שמים, ומיד כשראה זאת אברהם נבהל מן המראה ועמד מתבודד ומחשב כל היום, מה המה אלה? ויהי לעת ערב בבוא השמש קרוב ללילה ותרדמה נפלה על אברם וראה חשכה גדולה נופלת עליו, כאילו ליבו אומר אליו שעת רעה היא וצרה באה עליו, ואז הגיחה עליו הנבואה והודיעה לו מה זה ועל מה זה ופתרון החשכה בכללות, באמרו "ידוע תדע כי גר יהיה זרעך" וגו', רצה לומר החשיכה והאימה אשר ראית הוא רמז על גלות קשה שנגזר על זרעך, ומזה גלה לו זמן ירושת הארץ שתהיה אחר שעבוד ארבע מאות שנה, ושימותו האבות או המנהיגים ודור רביעי ישובו הנה. ובזה הותרו הספקות והשאלות ד', ה' וז' אשר העירותי בכתובים.אמנם בסיבת הגלות חשבתי דרכי אשיבה רגלי אל כל אחד משלושה דרכי ה' אשר צדיקים ילכו בם. \textrm{\textbf{הדרך הראשון}} הוא שבא הגלות על צד העונש לפי שבני יעקב חטאו חטאה גדולה בשנאתם את יוסף אחיהם שנאת חינם, ובמה שהשליכוהו בבור ומכרו אותו למצרים. ועם היות שראובן לא היה במכירה, היה שונא אותו ונתן עצה רעה להשליכו בבור. ובאמת נחשב להם הדבר לעוון פלילי, והם עצמם הודו בחטאם באמרם "אבל אשמים אנחנו על אחינו אשר ראינו צרת נפשו בהתחננו אלינו ולא שמענו על כן באה לנו הצרה הזאת" (בראשית מ"ב, כ"א). ולפי שהם חטאו היה משורת הדין שיקבלו עונשם ולפי שבמצרים חטאו היה ראוי ששם ילקו ויענשו הם ובניהם וזרעם, כמו שגלה יוסף ובניו שמה בסבתם, ולפי שהשליכו אותו אל הבור היה עונשם גזרת פרעה שכל הבן הילוד היאורה תשליכוהו, ולפי שהם בידיהם סבבו ירידת יוסף למצרים ונשאר שמה הוא ובניו וזרעו היה גם כן ראוי שעל ידו יבוא עונשם וירדו הם לגלות מצרים עם זרעם, ולפי שהם עשו המעשה המגונה הזה בהיותם בצאן ובלכת יוסף לראות את שלום הצאן, היה מהמשפט האלהי של ידי הצאן יבואו למצרים כמו שנאמר "כי אין מרעה לצאן" וגו'.ויען וביען שהיה יוסף חוטא גם כן בגאותו על אחיו ובהביאו את דיבתם רעה אל אביהם, וכן חטא אבינו הזקן יעקב בצד מה בשלוח מדנים בין אחים בעשותו ליוסף כתונת פסים ובאהבתו אותו מכל אחיו שבזה הטיל קנאה ביניהם, לכן נענשו יעקב ויוסף גם כן בגלות. אבל מאשר היה חטאם בקלות ולא בכוונה רעה חלילה, ולהיותם מן הנעלבים ואינם עולבים, זכו שלא נקברו במצרים, כי יעקב העלו אותו אל מערת המכפלה, ונאמר "ויקח משה את עצמות יוסף עמו ויקברו אותם בשכם" ובמאמרם ז"ל משכם יצא ולשכם חזר12סוטה י"ג, ב' וכן מכילתא פסחא, י"אוזר.אבל שאר השבטים נקברו במצרים כי נידונו על הגלות בחייהם ובמותם לא נפרדו ממנו. ואף על פי שחז"ל דרשו על "והעליתם את עצמותי מזה אתכם" שאף עצמות שאר השבטים העלו עמהם, הוא דרך דרש, ופשט הכתוב לא יסבלהו, שנאמר "ויקח משה את עצמות יוסף כי השבע השביע" וגו', ובסוף ספר יהושע כתיב "ואת עצמות יוסף אשר העלו בני ישראל ממצרים קברו בשכם", ואם היו מעלים עצמות שאר השבטים לא היה הכתוב שותק מלהזכירם, אבל האמת הוא שיעקב ירד מצרימה בעבור יוסף בנו ולזה אמר לו השם יתברך "אל תירא מרדה מצרימה אנוכי ארד עמך מצרימה ואנכי אעלך גם עלה" ולא נזכר שיעלה אחד מבניו כי אם יעקב בלבד, לפי שהוא היה נקי מאותו עוון, ובניו נדונו בעוונם. אמנם בנימין נענש להיות נטפל עם אחיו ולא בעונו, כמו שאמרו שמת בעטיו של נחש, רצה לומר שלא נענש בשביל חטאו כי אם בעוון אחיהם לי שהיה בתוכם, כי העולם נדון אחר רובו, ועל כיוצא באלה אמרו "משנתנה רשות למשחית לחבל אינו מבחין בין צדיק לרשע" (בבא קמא ס', א').ובהיות שהשבטים הוגלו מצרים בעוונם הנה בניהם אשר קמו אחריהם וזרעם נשארו באותה גלות, וכמאמר הנביא "אבותינו חטאו ואינם ואנחנו עוונותיהם סבלנו" (איכה ה', ז'). לכן כתב הרלב"ג בענין גלותנו זה שאנשי בית ראשון נענשו בעצם כפי חטאתם ובניהם וזרעם נענשו במקרה לפי שנולדו בעונש האבות ונטבעו בו עד תום זמן הגזרה או עד שיזכו להיגאל מפני זכותם, והסתכל שלפי שהיו העונשים אלהיים על בנים ועל בני בנים על שילשים ועל רבעים, לכן נאמר בגלות הזה "ודור רביעי ישובו הנה" כלומר שהדור הרביעי מהגולים ישובו לירש הארץ. ולפי זה הסִבּה הזאת לגלות מצרים יותר אמיתית ומתיישבת מכל מה ששערו הראשונים.ולהיות מעשיו של הקדוש ברוך הוא מידה כנגד מידה, תמצא שחטאו השבטים בהיותם עם הצאן, וכאשר הלך יוסף לראות את שלום הצאן, מכרו אותו אחיו לעובדי הצאן, וכדי לרמות אביהם שחטו שעיר עזים מן הצאן, והקדש ברוך הוא השליט את יוסף על עובדי הצאן, וכאשר נרצה עוונם ציוה עליהם מצוות הפסח הבא מן הצאן.ואם כן בצאן חטאו ובעובדי הצאן לקו ובצאן נתכפרו. ואין להתפלא כי השם יתברך ייעד העונש קודם החטא, כי זה היה להודיע לאברהם זמן ירושת הארץ אשר ביקש ולמה יתעכבו בניו בירושתה. ולכן אמר לו כמגלה סוד מה שיהיה באחרית הימים: "ידוע תדע כי גר יהיה זרעך" וגו', ובפרשת "כי תוליד בנים" ובפרשת "והיה כי יבואו עליך" ובפרשת "הנך שוכב עם אבותיך", וכמה אחרות, ייחד ה' בחטא ובעונש העתיד לבוא לאלפים מן השנים.ורש"י ז"ל כתב בזה בפרשת משפטים בפסוק "הנה אנוכי שולח מלאך לפניך", וזה לשונו: "כאן נתבשרו שעתידין לחטוא ושכינה אומרת להם כי לא אעלה בקרבך" וגו', וכמו ששם בישר אותם בעונש מבלי שיבאר החטא כן בכאן הודיע ענשם שיתחייבו מחמת חטאם ואשר בשביל זה תתעכב ירושתם את הארץ ולא ביאר החטא. אמנם עניין הארבע מאות שנה ראוי שיפורש על דרך שיכלול ענייני גרות ועבדות ועינוי, ששלושת הלשונות הללו נזכרו בכתוב והיו בזמנים מתחלפים, כי באמרו "ידוע תדע כי גר יהיה זרעך בארץ לא להם" גילה לו שלא תהיה ירושת הארץ בימי בנו יצחק ולא בימי בן בנו יעקב, כי הם יהיו גרים שמה תחת אדונים קשים וזהו שאמר "בארץ לא להם". ואמר עוד "ועבדום" כלומר עוד יבוא זמן שיהיו מזרעך לא לבד גרים אלא עבדים,  והיה זה בימי השבטים שנשתעבדו במצרים מסבת חטאם כמו שכתבתי.ואמר עוד "וענו אותם" רצה לומר גם יבוא זמן אחר שמלבד הגרות והעבדות יהיו עוד בעינוי גדול, והיה זה אחר שמתו השבטים משנולדה מרים בקבלתם ז"ל, וכל זה היינו הגרות וזמן העבדות וזמן העינוי יעלו כולם לארבע מאות שנה. אמנם מה שאמר "וגם את הגוי אשר יעבודו דן אנכי" ענינו שזמן הגרות שסבלו בו יצחק ויעקב לא יהיה לאומות עונש על זה, לפי שכך יצאה הגזרה מאיתו יתברך, אבל העבדות אשר ישתעבדו בהם המצריים בעבדים נמכרים והעינוי אשר יענו אותם במסכת אכזרי, על זה ישפוט ה' וידון דינו. והסתכל באמרו "וגם את הגוי" שכולל גם את זרע אברהם, ורצה לומר כמו שאהיה דן את השבטים על אשר עשו מעשה עבדות ליוסף, ככה אהיה דן את הגוי אשר יעבודו בכם, שאדין את המצריים על אשר שעבדו את ישראל וענו אותם. והיה זה לפי שהשם יתברך לא גזר על השבטים כי אם הגרות בארץ לא להם, והמצריים ברשעתם אף כי באו השבטים בתל קורתם כאוהבים בבטחון ובשלוה נהפכו להם לאויבים, ובקשר ובמעל ובמרד בגדו בם וענו אותם, ולכן יהיה דינם שבעד העבדות יפרעו שכר עבודתם, וזהו שאמר "ואחרי כן יצאו ברכוש גדול" ועל העינוי יתענו במכות ובאותות ובמופתים. ואין הכוונה שמייד אחרי הארבע מאות שנה יצאו ברכוש גדול, כי אם שאחר כך יהיו ראויים לגאולה ויגאלו ויצאו משם אם לא יהיה להם מונע מצידם, כי היעודים האלהיים הם בתנאי הכנת המקבלים, ואין ספק שאחרי הארבע מאות שנה שנשלמה הגזרה היה הקדוש ברוך הוא מוציאם ממצרים, אבל הם הוסיפו על חטאתם פשע ויעבדו אלהים אחרים ולכן נתארך גלותם עוד שלושים שנה לא מכח הגזרה הראשונה כי אם בעבור חטאם שנתחדש. וכבר ביאר זה הנביא יחזקאל ע"ה באמרו: "ביום בחרי בישראל ואשא ידי לזרע בית יעקב ואודע להם בארץ מצרים ואשא ידי להם לאמור אני ה' אלהיכם, ביום ההוא נשאתי להם להוציאם מארץ מצרים אל הארץ אשר תרתי להם" וגו', "ואומר אליהם איש שיקוצי עיניו השליכו ובגלולי מצרים אל תטמאו" וגו', "וימרו בי בית ישראל" וגו', ואומר לשפוך חמתי עליהם לכלות אפי בהם בתוך ארץ מצרים ואעש למען שמי" וגו'. (יחזקאל פרק כ')הנה באר הנביא שכאשר כלו הארבע מאות שנה שלח הקדוש ברוך הוא נביא למצרים להתרות בישראל ולהזהירם שיעזבו גילולי מצרים ופסיליהם כדי להוציאם מיד ולהביאם אל הארץ אשר נשבע לאבותם מפני שנשלם זמן הגזרה, ואמרו חז"ל שאהרן ניבא להם הדבר הזה במצרים, והזכיר הנביא שמפני היותם שטופים בחטא עבודה זרה במצרים לפיכך נעדרה מהם הגאולה. ובלי ספק היו אותם שלושים שנה שנתווספו להזמן הקצוב, והיה הגלות נמשך יותר ויותר מזה אם לא ששבו בתשובה "ויצעקו אל ה' בצר להם ויאנחו מן העבודה ותעל שועתם אל האלהים", ועל ידי תשובתם זכר את בריתו אשר כרת את אברהם שהוא זמן הגזרה, וידע אלוהים ויחמול עליהם.ועל זה נאמר ומושב בני ישראל שישו בגלות בשאר הארצות ובמצרים בין מפאת גזרת הגלות ובין מבית עון עבודה זרה שנתחדש ביניהם היו כולם 430 שנה. והביטה וראה נפלאות תמים דעים שהנה ישבו בני ישראל בגרותם וגלותם ארבע מאות ושלושים שנה, וכן תמצא שישבו בארץ אחרי ירושתה קודם בנין בית המקדש ארבע מאות וארבעים שנה, ותמצא גם כן שעמד הבית הראשון בבנינו ארבע מאות ועשרים ושבע  שנה, ועמד הבית השני בבנינו 428 שנה. ראה גם ראה שכל הדברים נמשכו כפי זה המספר מן השנים בין רב למעט, והוא דבר מתמיה!ונחזור לעניננו, שהודיע הקדוש ברוך הוא לאברהם כי הארבע מאות שנה מהגרות יתחיל מזרעו, כי בימיו לא יהיה גרות ולא עבדות ולא עינוי, כי הוא לא היה כגר בארץ כי אם כשר וגדול בה, ולכן כאשר הוא בענוותנותו אמר באזני בני חת: "גר ותושב אנוכי עמכם" השיבוהו: "לא אדני נשיא אלהים אתה בתוכנו" (בראשית כ"ג, ו') ומלכי ארץ היו כורתים עמו ברית לפי מעלתו ויכולתו, ועל זה אמר לו השם יתברך "ואתה תבוא אל אבותיך בשלום". ואחשוב שהאבות שנזכרו בכתוב הזה הם אדם וחוה אבות לכל נוצר, שנקברו במערת המכפלה כפי קבלתם ז"ל, ויבשרו הקדוש ברוך הוא שיקבר עמהם שמה, ועל זה נאמר "ואתה תבוא אל אבותיך בשלום". ולפי שייעדו שידמה להם בקבורה הודיעו עוד שלא יקבר בשיבה רעה כמו שנקברו אדם וחוה שהיתה מיתתם וקבורתם ביגון ואנחה, כי ראו בחייהם הגירוש מגן עדן וקין הורג את הבל אחיו ונגזר על קין להיות נע ונד בארץ, אבל אברהם לא יראה דבר רע וזהו שהבטיחו "תקבר בשיבה טובה", כי ישתתף אל אבותיו אדם וחוה בקבורה ולא בשיבה. ומה שנאמר "ודור רביעי ישובו הנה" הוא באמת על דורות הנענשים בחטא מכירת יוסף, כי יפקוד אלוהים עוונם עליהם ועל בניהם עד הדור הרביעי כמו שכתבתי, ואחר זה יהיה עושה חסד לאלפים בהיותם אוהביו ושומרי מצוותיו.ואתה רואה שיהודה היה בין השבטים בחטא ההוא ואחריו פרץ וחצרון וכלב היה הדור הרביעי שבא אל הארץ. אולם מה שאמר עוד "כי לא שלם עוון האמורי עד הנה" בא להתיר ספק אחר אשר אפשר לעורר כנגדו, והוא כי אם אמת הוא שאברהם לא חטא ולא יצחק בנו ולא יעקב בן בנו אם כן למה לא ירשו שלושת האבות הקדושים האלה את הארץ? ואם אחרי כן יחטאו בניהם יהיו גולם ממנה, כי לא יומתו אבות על בנים. וכדי להשיב על זה אמר "ודור רביעי ישובו הנה", רצה לומר הסיבה אשר בעבורה לא תהיה הירושה עתה מיד היא לפי שלא שלם עוון האמורי יושבי הארת עדיין ולא נתמלאה סאתם שינערו רשעים ממנה, ולכן לא יכלו אברהם ויצחק ויעקב לרשת הארץ, ובניהם לא ירשו אותה אחר כך בסיבת חטאם עד הדור הרביעי, ואחר כך ישובו הנה, כלומר יצאו מגלותם וישובו אל הארץ, כי תשתלם רשעת הגויים ולא תמנע חטאת ישראל. מצורף לזה שירושת הארץ היתה ראויה שתהיה בריבוי אוכלוסין מבני ישראל, כי איך יוכל איש אחד או שנים לירש ולהתנחל בארץ של שבעה עממין? וכבר ידעת שהמאמרים האלה הם הגדת עתידות ושהידיעה האלוהית בעתיד לא תשתנה טבע האפשרי ולא תבטל הבחירה האנושית, ולכן אמר השם יתברך לאברהם ידע תדע במגיד מה שיהיה.ולפי הדרך הזה נתבארו כתובי הפרשה על בוריים והותרו השאלות ז', ח', ט' י' וי"א שהעירותי בפסוקים ולפי הרעיון הזה שגלות מצרים היה עונש על חטא השבטים, ראינו לפרש המאמר ויעקב ובניו ירדו מצרימה, כי הם בחטאם ירדו שמה ולא גזר ה' עליהם הירידה כי אם עונותיהם, ועם כל זה היה השם יתברך מחשב את הקץ לעשות כמו שאמר לאברהם אבינו בין הבתרים שהודיעו גזרת הגלות וזמנה וייעוד הגאולה ודין המצריים, ולפי שישראל לא היו ראויים לגאולה בעבור עוונותיהם ולא שישמור להם הבטחת אברהם אביהם, לכן אמר המגיד "ברוך שומר הבטחתו לישראל" כי ראוי לברך ולשבח ולהלל את ה' על זה. זהו הדרך הראשון בסיבת גלות מצרים והוא דרך העונש.\textrm{\textbf{הדרך השני}} הוא דרך החסד, רצה לומר, שהקדוש ברוך הוא הביא את זרע אברהם בגלות כדי להיטיב להם לגמלם ולנשאם, ונבאר זה לפי בידוע ששלמות העולם בכללו ותכלית מציאותו הוא בהכרת בוראו ולשבחו ולהודות באלוהותו, וכמו שאמר הנביא "כל הנקרא בשמי ולבכורי בראתיו" (ישעיהו מ"ג, ז'). והנה בימי אנוש נתרבה עבודת האלילית בין בני האדם בכל העמים, זולת יהירים מהם שהיו נמשכים אחרי עבודת השרים העליונים, וכמו שכתב הרב המורה חלק א' פס"ג שבכל הדורות הראשונים לא היו מרגישים ומשערים במציאות הסיבה הראשונה, ואם נמצאו איזה אנשים שהודו במציאותה היו מסלקים השגחתה מהעולם השפל, ויש שחשבו שכח הסיבה הראשונה מוגבל ויכולתו מסודר והוא בעל תכלית בשאר השרים העליונים, ועיקר הכפירות והדעות המשובשות האלה היו במצרים. וכאשר זרח אורו של אברהם בעולם והתחיל ללמד בני אדם אמיתת אמונת האל יתברך רצה הקדוש ברוך הוא לזכות בני האדם ולפרסם אמיתת נבואותיו וראה שלא היה מקום לזה כי אם על ידי אותות ומופתים שיעשה בשמים ובארץ כנגד המנהג הטבעי וכנגד המערכות השמימיות, וכי דרך הנאות לזה היה להביא עם אחר לארץ מצרים מקור הכפירות והאמונות הנפסדות ואחר כך ישלח נביאו להוציאם משם, ויקשה את לב מלך מצרים שלא לתתם להלוך למען רבות מופתיו בקרבו, והעם הזה שיוציא במסות באותות ובמופתים ובמוראים גדולים, יקבל תורתו ויפרסם אמונתו לבני אדם, והעם ההוא יהיה עם סגולתו מיוחד להנהגתו, ובעבור זה ירכיבם על במתי ארץ הנבחרת המיוחדת גם כן להנהגתו, והוא יהיה להם לאלוהים. ובעבור שהיה אברהם ראש המאמינים והתחיל ללמד דעת ויראת ה', לכם בחר בזרעו אחריו להיות כלי אמצעי שיעשה באמצעותו הדבר הגדול הזה. וכל זה עשה על דרך החסד, והודיע לאברהם שיקח את זרעו לעם סגולתו וינחילם את הארץ, וכדי שיבואו לקנין השלמות בגוף ונפש, הודיעו במראה בין הבתרים שיביא את זרעו בגלות ויהיו עבדים בארץ לא להם כדי שעל ידיהם יתגלו נפלאותיו ויתודע שמו של הקדוש ברוך הוא ויכירו כל יושבי תבל מציאות אלהותו והשגחתו ויכולתו, כמו שהוכיח זה במכות מצרים וכו שנתבאר להלן, והודיע שעם היות שיסבלו בניו וזרעו צער ועינוי הכל יהיה לטובתם, וישארו על ידי זה עם קדוש לה' ויוציאם בכסף וזהב ויורישם ארצות גוים ויקנו כל השלמות, ולפי זה לא נחשב הגלות לעונש כי אם לחסד על האומה.ויש להדעה הזאת שלוש ראיות:א – מפאת הסברא, לפי שראינו אברהם בשמעו הגזרה החרותה הזאת לא ביקש רחמים על זרעו ולא התחנן לאלוהיו שינחם על הרעה, ואיך יעלה על הדעת שאב המון גוים לא יחוש על זרעו ואת בניו לא ידע? ואם על גזרת סדום ועמורה הרבה בקשה והפציר בתפילה על אנשים נכרים, איך שם יד לפה על גלות בניו וצרתם? אם לא שהיה גלוי וידוע לפניו שהצער שיסבלו הוא לטובתם והמצוקה תהיה להם צדקה, וכי זה היה חסד ולא רעה ולא עונש, ולכן קבל והאמין בה' ויחשבה לו צדקה, ונאמר בסוף המראה: "ביום ההוא כרת ה' את אברהם ברית לאמור לזרעך נתתי את הארץ הזאת" וגו', כי בהודעת הגלות נתקיים הברית והחסד והשבועה.ב – ומורה גם כן על זה דברי המשורר במזמור ק"ח שאמר כי היה ענין הגלות והירידה למצרים ומכירת יוסף ויתר העניינים מסודרים ונגזרים ממנו יתברך לתכלית כוונתו לעשות טובה וחסד עם ישראל, לכן התחיל "הודו לה' קראו בשמו וגו' אשר כרת את אברהם וגו' לאמור לך אתן את ארץ כנען חבל נחלתכם בהיותכם מתי מספר כמעט וגרים בה וגו'" והזכיר ענין האבות והליכתם ממקום למקום באמרו "ויתהלכו מגוי אל גוי" ואחר זה אמר ויקרא רעב על הארץ שלה לפניהם איש לעבד נמכר יוסף היתה על פי הדיבור, וכן "ויבוא ישראל למצרים וגו' ויפר את עמו וגו' הפך לבם לשנוא עמו וגו' שלח משה עבדו וגו' שמו בם דברי אותותם וגו', ובסוף המזמור נתן הסיבה בכל הסדור וההתגלגלות באמרו כי זכר את דבר קדשו את אברהם עבדו ויוציא עמו בששון ברינה את בחיריו ויתן להם ארצות גוים ועמל לאומים יירשו בעבור ישמרו חקיו ותודותיו ינצורו הללויה (דברי הימים א', ט"ז). הנה ביאר שהיה כל זה בדרך חסד ושראוי מפני זה להודות לה' ולקרוא בשמו ולהללו.ג – עוד ראיה לדעה הזאת מדברי חז"ל בבראשית רבה: "אמר ר' יהודה ראוי היה יעקב אבינו לירד למצרים בשלשלאות של ברזל ובקולרין, ועשה הקדוש ברוך הוא כמה עלילות כדי שירד שם שנאמר 'לעבד נמכר יוסף ויקרא רעב על הארץ', וכל כך למה – ויבא ישראל למצרים. אמר ר' פנחס משל לפרה שהיו רוצים למשוך אותה למקולין שלה ולא היתה נמשכת, מה עשו משכו את בנה תחילה והיא רצתה אחריו, כך עשה הקדוש ברוך הוא שימכר יוסף למצרין כדי שירד יעקב אביו אחריו, שנאמר 'בחבלי אדם אמשכם בעבותות אהבה" (הושע י"א, ד'). עד כאן לשונו. רצו בזה לומר כי מהשקפת התועליות העצומות ובבחינת השלמיות המקווה לעתיד הנה לא היה ראוי יעקב שירד בעצמו למצרים אלא שיורידוהו שמה בחבלי עוני וברזל למען ירד עמו כבודו כי אז טוב לו, ויותר שיצטער יותר ישתלם לעתיד. אמנם נתחסד עימו הקדוש ברוך הוא להורידו בסיבות יותר נאותות ובעלילות מצערי הבנים, וייחס כל זה ליעקב לפי שהוא עצמו ירד למצרין והוא היה הסיבה הקרובה להם מהאבות.ואמר ר' פנחס שענינו בזה דומה לענין הפרה שתכלית מציאותה אצל אדונה להביאה אל מקולין שלה והוא בית השחיטה, ועם כל זה מקוצר רעתה והרגשתה לא תביט אל הצער הגשמי ההווה, ולכן לא תמשך שם ברצונה, כי אם בתחבולת בנה. ואין ספק שאדון הפרה גדלה ופטמה לתועלתו, ולכן תכליתה היותר משובח הוא להוליכה לבית מקולין שלה, כמו שאמרו סוף בהמה לשחיטה, ובזה תקבל שלמות גדול, כי על ידי התיקונים והבישולים שיעשה בה ויאכלו ממנה לבוא להיות חלק אבר אנושי, ובמקום שהיתה בהמה גסה תקבל צורת בשר אדם, והיא השלמה גדולה אשר אפשר לזה לקבל. וכן הוא ענין האומה, כי אברהם אבינו היה עקר ואשתו שרה עקרה והשם יתברך נתן לו ריבוי זרע וגדלו כדי שיעבוד עבודתו ויקל צער ומיתה על קדוש שמו, ועם היות שבירידת יעקב ובניו למצרים קבלו צער והרג ואבדן בגלותם אשר תקבל הפרה במקולין שלה, אך על ידי זה יקבלו תכליתם האמתי כדי שתתפרסם אמונת השם יתברך על ידיהם וגם בניהם יקבלו השלמות המכוון בבראה אשר זכו אליו במעמד הר סיני. והנה לא נמשכו לבוא שמה ברצונם, להעדר ידיעתם בתכלית הדבר כעניין הפרה, כי יאבה האדם לבוא בנקל תחת עול התורה והאמת, ועל זה אמר "בחבלי אדם אמשכם", והחבלים האלה שהיו בתחילה חבלי יולדה לבסוף נעשו חבלי אהבה, באהבת ה' אותם וברצונו לשמור את הברית אשר נשבע לאברהם.ואף כי הסיבות הראשונות שהיו בירידה מכתונת הפסים ושנאת האחים ליוסף היו חבלי אדם דקות ושדופות קדים, לא היה בהם כח להמשיך את הפרה את מקולין שלה. אמנם הוא יתברך בחכמתו עשה מהחבלים שרשרות גבלות מעשה עבות שלא ינתקו עד אשר יוציא מחשבתו לפעולה, וכמו שאמר יוסף לאחיו "וְעַתָּה לֹא אַתֶּם שְׁלַחְתֶּם אֹתִי הֵנָּה כִּי הָאֱלֹהִים" (בראשית מ"ה, ח'). ונתבאר מזה שבאה גלות מצרים על צד החסד ולתכלית ההטבה לאומה הישראלית, והוא הדרך השני, ולפי זה יהיה מאמר המגיד "ויעקב ובניו ירדו מצרימה", מחסדי ה' שנתן לעשיו את הר שעיר ויעקב עם בניו ירדו למצרים ונתגלה אליהם הכבוד האלוהי וזכו לכל מה שזכו בעבור זה, ולכן אמר "ברוך שומר הבטחתו לישראל" שהיא הודאה על הטובה.\textrm{\textbf{הדרך השלישי}} הוא דרך הבחירה, וענינו  שלא ירדו השבטים למצרים בשביל העונש, ולא היו מוכרחים לבוא שמה מצד הגזרה העליונה לתכלית טובתם, אלא שהיו המעשים כולם בחיריים, הינו כי בבחירתם ורצונם החופשי מכרו את יוסף, וזו היא הסיבה הראשונה מהסיבות הפועלות לרדתם שמה, כמו שאמרו חז"ל שעל ידי זה נתגלגלו הדברים וירדו למצרים, וזה טעם  "ארמי אובד אבי וירד מצרימה" (דברים כ"ו, ה') שהוא מעצמו ירד שם, וכן שלח משה לומר למלך אדום "וירדו אבותינו מצרימה" (במדבר כ', ט"ו), ואמר גם כן "בשבעים נפש ירדו אבותיך מצרימה" ולא אמר שהוליכם ה' שמה כמו שאמר שהביאם אל הארץ, וכן אמר יהושע (יהושע כ"ד, ד'), ויעקב ובניו ירדו מצרימה, לא שהם מעצמם ובחירתם ירדו שמה, ומה שאמר הקדוש ברוך הוא ליעקב, אל תרא מרדה מצרימה, היה רשות שנתן לו לא שצוהו לרדת בשום פנים. ולפי זה הדרך מה שאמר "ידוע תדע כי גר יהיה זרעך בארץ לא להם" ארבע מאות שנה, היא הודעה מה שיהיה בזמנים העתידיים, ולזה כי גר \textrm{\textbf{יהיה}} זרעך ולא אמר אתן את זרעם או אשים אום, כי הם ברצונם ובבחירתם עקרו שור וירדו למצרים ונטבעו בגלות, והשם יתברך ברוב חסדיו שמר להם הבטחת אברהם לנקום נקמתם מאויביהם ולהוציאם מעבדות לחירות ולתת להם את ארץ כנען לנחלה. ולפיכך אמר המגיד "ברוך שומר הבטחתו לישראל", שהקדוש ברוך הוא מחשב את הקץ רצה לומר שהיה הגלות מפאת ישראל ובחירתם והגאולה ממנו יתעלה.הנה בכל אחת משלושת דרכים האלה התבארו הספקות אשר בשער  ל"ח, ל"ט, מ' ובשער מ"א.}%endcomment
\hebeng{{\small מכסה המצה ומגביה את הכוס בידו, ואומר: } }{{\small He covers the matsa and lifts up the cup and says:} }
\hebeng{וְהִיא שֶׁעָמְדָה לַאֲבוֹתֵינוּ וְלָנוּ. שֶׁלֹּא אֶחָד בִּלְבָד עָמַד עָלֵינוּ לְכַלּוֹתֵנוּ, אֶלָּא שֶׁבְּכָל דּוֹר וָדוֹר עוֹמְדִים עָלֵינוּ לְכַלוֹתֵנוּ, וְהַקָּדוֹשׁ בָּרוּךְ הוּא מַצִּילֵנוּ מִיָּדָם. }{And it is this that has stood for our ancestors and for us; since it is not {[only]} one {[person or nation]} that has stood {[against]} us to destroy us, but rather in each generation, they stand {[against]} us to destroy us, but the Holy One, blessed be He, rescues us from their hand. }%
\commentb{\textrm{\textbf{תשובות לשערים מ"ב – מ"ו}}\textrm{\textbf{היא שעמדה לאבותינו ולנו שלא אחד בלבד עמד עלינו אלא שבכל דור ודור עומדים עלינו וכו'.}}אחרי שהמגיד נתן ברכה והודאה להשם יתברך על אשר שמר הבטחתו לישראל בענין יציאתם ממצרים, אמר עוד ששמירת אותה הבטחה לא לבד הועילה להם לענין היציאה מהגלות, כי גם בצרות אחרות שהיו אבותינו או אנחנו מאז ועד היום הועילה ההבטחה ההיא ועמדו להם ולנו לפליטה גדלה, וכמו שהוכיח מענין לבן.והסתכל והתבונן מה שאמר שלא אחד בלבד עמד עלינו ולא אמר המגיד בזה מילת "לכלותינו" כי אם עמד עלינו בלבד, לפי שהאחד ההוא שמזכיר כיון בו אל פרעה מלך מצרים, וידוע שלא היתה כוונתו לכלותם כי אם לשעבדם ולכן היה מחזיק בהם לבלתי שלחם, ולכן אמר המגיד שזה האחד עמד עלינו לצר ואויב וכן בכל דור ודור עד היום עומדים עלינו, ולא רק בדרך פרעה לתכלית השעבוד כי אם גם לכלותנו מעל פני האדמה.}%endcomment
\newsection{ארמי אבד אבי}
\hebeng{{\small יניח הכוס מידו ויגלה אֶת הַמצות. } }{{\small He puts down the cup from his hand and uncovers the matsa.} }
\hebeng{צֵא וּלְמַד מַה בִּקֵּשׁ לָבָן הָאֲרַמִּי לַעֲשׂוֹת לְיַעֲקֹב אָבִינוּ: שֶׁפַּרְעֹה לֹא גָזַר אֶלָּא עַל הַזְּכָרִים, וְלָבָן בִּקֵּשׁ לַעֲקֹר אֶת־הַכֹּל. שֶׁנֶּאֱמַר: אֲרַמִּי אֹבֵד אָבִי, וַיֵּרֶד מִצְרַיְמָה וַיָּגָר שָׁם בִּמְתֵי מְעָט, וַיְהִי שָׁם לְגוֹי גָּדוֹל, עָצוּם וָרָב. }{Go out and learn what Lavan the Aramean sought to do to Ya'akov, our father; since Pharaoh only decreed {[the death sentence]} on the males but Lavan sought to uproot the whole {[people]}. As it is stated (Deuteronomy 26:5), "An Aramean was destroying my father and he went down to Egypt, and he resided there with a small number and he became there a nation, great, powerful and numerous."}%
\commenta{\textrm{\textbf{צא ולמד מה ביקש לבן הארמי לעשות ליעקב אבינו וכו׳, שנאמר, ארמי אובד אבי}} הלשון צא ולמד אינו מבואר ברחבה, ולא מצינו כמותו, וצריך באור. ואפשר לפרשו עפ״י המבואר בשבת (ל״א א׳) בגוי אחד שבא לפני הלל ואמר לו גיירני על מנת שתלמדני כל התורה על רגל אחת, גיירי׳, אמר לו, ואהבת לרעך כמוך, מה דעלך סני לחברך לא תעביד (כגון גניבה, גזילה, רציחה, עריות ורוב המצוות (רש״י) היא כל התורה, ואידך פירושא היא, זיל גמור, ע״כ. ועפ״י זה יש לכוין הענין כאן, כי כמבואר במדרשים ובאגדות רבו מאוד הרעות והרמאות שעשה לבן ליעקב, ורמז מזה ניכר בדברי יעקב שאמר והחליף את משכורתי עשרת מונים ולא נתנו אלהים להרע עמדי (פ׳ ויצא ל״א ז׳). אכן דברי האגדות האלה רבים הם ואין השעה עתה בשעת הסדר לאספם ולפרטם ולפרשם, ואמר על זה, צא ולמד לעת הפנאי את כל הנאמר מכל הרעות שעשה לבן ליעקב, ואני אומר לך רק כלל הדברים, והוא מה שנאמר ארמי (לבן) אובד אבי, הוא רצה לאבד את אבי, זה הוא יסוד כל מעשיו וכללם ואידך, פרטיהם, צא ולמד, וזה מכוון לסגנון האגדה דשבת שהבאנו. ומדי זכרתי את הסיפור דשבת הנזכר, אעיר על הלשון שאמר הלל להגוי ״זיל גמור״, זה קשה, שהרי הגר רצה ללמוד הכל רק על רגל אחת, ואיך אמר לו זיל גמור. ואולי אפשר להגיה קצת בלשונו, תחת הלשון זיל גמור — זיל גייר, כלומר, אנכי למדתיך כל התורה על רגל אחת, כי זה שלמדתיך הוא יסוד כל התורה ועתה עליך לקיים הבטחתך להתגייר, והלשון גיירי' צריך להיות לבסוף.\textrm{\textbf{שנאמר ארמי אובד אבי}} לשון זה הוא פסוק בריש פרשה תבוא (כ״ו ה׳), והוא התחלת תפלת ההודאה על הבאת בכורים, ומספר בקצרה את קורות עם ישראל, ופירש״י, דשם ארמי מוסב על לבן הארמי, שרצה לעקור את כל אשר ליעקב בעת שרדף אחריו, אך ה׳ מנעהו מזה, כמבואר בס״פ ויצא, ונחשבה לו מחשבתו זאת כמעשה עכ״ל. אבל יש מפרשים (רשב״ם ועוד) דשם ארמי מוסב על אברהם שהי׳ מארץ ארם, והי׳ אובד וגולה מארצו, כדכתיב לך לך מארצך, עכ״ל. ויהי׳ לפי זה השם ״אובד״ נמשך לשם ״ארמי״, כלומר, ארמי אובד אבי — ארמי שהי׳ אובד הוא אבי, אברהם. וע׳ בסמוך. ועוד יש מפרשים (ספורני ועוד), דשם ארמי מוסב על יעקב, שהי׳ זמן ארוך כמו אובד בחיים, ולא הי' לו בית מושב ומנוחה, ולא יפלא שקוראים לו ארמי (כן הוא דברי המפרשים) שמצינו דוגמתו בשם יתר הישמעאלי (דה״י, ב׳ י״ז) אעפ״י שהי׳ ישראל, אך מפני שהי׳ גר בארץ ישמעאל, עכ״ל. והנה אין גבול ואין ערך לכל הדוחק בפירושים אלה, ולמותר להאריך בזה, וגם כי אין קץ לפלא לכנות עתה בתפילת הודאה בשעת הבאת בכורים (אשר על זה יומשך הלשון ארמי) לכנות את אברהם ואת יעקב בשם ארמי, ולא שמענו מעולם דוגמא לזה. ולבד שפירש״י שהבאנו נוח ורצוי, ועוד יסכים לו אונקלוס שמתרגם ארמי אובד אבי — לבן ארמאה בעא לאבדא ית אבא — עוד יוצדק עפ״י מה שכתב הראש בבאורו לנדרים (ל״ז ב׳), כי הנגינות ופסקי הטעמים של המלים יורו פשט הכתוב, וכאן השם ארמי נטעם בפשטא, שהוא טעם מפסיק המלה בנטי׳ לעצמה מבלי התחברה עם המלה הסמוכה, וכנגד זה, המלה ״אובד״ מוטעמת במונח, המורה להמשכה הלאה לשם ״אבי״ ויוצא לפי זה המובן מהמשך המלים, ארמי — הוא אובד אבי, וזה כפירש״י ותרגום אונקלוס. אבל לדעת המפרשים הנזכרים ובאורם הי׳ דרוש להטעים המלים ארמי אובד בנגינה מחברת, כהוראת שם ותואר. ודברי הרא״ש הם קלורין לעינים בכמה וכמה מקומות בפסוקים שטרחו מפרשים בבאורם. וכן נראה שהבין הלשון כאן כרש״י ות״א בעל הפזמון שלהלן ״ובכן ויהי בחצי הלילה״ שכתב, הפחדת ארמי באישון לילה״. ומוסב על לשון הפסוק בס״פ ויצא וירא אלהים אל לבן הארמי בחלום הלילה, והזהיר אותו על הנהגתו עם יעקב, הרי דמכנה את לבן בשם סתמי ארמי, ולמד זה מלשון התורה כאן.}%endcomment%
\commentb{עוד אמר "צא ולמד מה ביקש לבן הארמי", רצה לומר צא מענין מצרים שאתה עוסק בו ולמד מסיפורי יעקב ולבן, מה ביקש לבן הארמי לעשות ליעקב אבינו? כלומר עם היות שלבן לא עשה עמו רעה בפועל, צא ולמד מדבריו מה היתה כוונתו ומחשבתו לעשות ליעקב אבינו, ודבר זה תלמד מתוך דבריו ולא ממעשיו, לפי שהוא עצמו הודה ואמר יש לאל ידי לעשות עמכם רע ואלהי אביכם אמש אמר אלי לאמור השמר לך מדבר עם יעקב מטוב ועד רע" (בראשית ל"א, כ"ט).וזה מורה שמחשבתו היתה להרע עמו. והנה מה שאמר שפרעה לא גזר אלא על הזכרים מובן זה בעוד שהיו ישראל במצרים, כי שם באמת לא גזר אלא על הזכרים כמו שנאמר "כל הבן הילוד" וגו', אמנם אחרי יציאתם משם כשרדף אחריהם אמר "ארדוף אשיג אחלק שלל" וגו', רצה לומר שהיה בדעתו לרדוף אחרי בני ישראל ולהשיגם, ומזכיר ברדיפתו תכליות מחלפות, שיקח את כל אשר להם ויחלק השלל ההוא לבעליהם אשר השאילום והשאר לאנשי המלחמה, ועל זה אמר "אחלק שלל, עוד יאמר שיקח הבחורים והנערות להיותם לו לעבדים כרצונו ואות נפשו, וזהו שאמר "תמלאמו נפשי", כלומר תמיד הייתי חפץ בהם שיעבדוני ועתה תמלא תאוות נפשי מהם לשעבדם בכל אופן.עוד אמר ששאר העם והזקנים ואת משה ואהרן יהרגו להנקם מהם, ועל אלה אמר "אריק חרבי תורישמו ידי". אם כן, לא היתה דעתו של פרעה לעקור את הכל, אבל לבן היתה שאלתו ובקשתו לעקור את הכל ולהכות אם על בנים, והא ראיה שאמר ליעקב "הבנות בנותי והבנים בני והצאן צאני וכל אשר אתה רואה לי הוא" (בראשית ל"א, מ"ג), רצה לומר לי היה ראוי להיות וזו בקשתי לולא אלהים אשר מנע זאת ממני. ומזה למד המגיד שלבן ביקש לעקור את הכל, רצה לומר נשים ובנים וצאן וכל אשר ליעקב, וסמך זה באמרו "ארמי אובד אבי" שעשה אובד יוצא מהקל. ומזכיר רק אבדת האב שהוא השורש ובכללו גם העפים, כי האבדון היא תכלית המוחלטת.ועם מה שפרשתי בזה הותרו הספקות אשר בשער מ"ב ומ"ג.ואמר המגיד שההבטחה ההיא שנאמרה לאברהם היא עמדה להציל את זרעו מאויביו, ולכן הזכיר את פרעה ואת לבן ולא הזכיר את עשו כי הוא היה גם כן מזרע אברהם ולא ביקש לכלות את זרעו, אבל שאר האויבים שלא היו מיוצאי ירך אברהם אשר קמו על זרעו היתה כוונתם לכלות אותם אם לא שניצולו מהם כדי שתתקיים ההבטחה שנאמרה לאברהם אבינו. והנה יעקב עם היותו צדיק בדינו עם לבן לא היה צדיק במעשיו בעניין המקלות, אף על פי שהשתדל להעלים הדבר במה שאמר אחר כך לנשיו וללבן. אמנם מה שעמדה לו היתה הבטחת אברהם, והמגיד הזכיר ענין לבן כדי להתחיל בדבריו מפרשת וידוי הביכורים שנזכרה בסדר והיה כי תבא שנאמר "וענית ואמרת לפני ה' אלהיך ארמי אובד אבי וירד מצרימה ויגר שם" וגו'. ולפי שבלילה הזה אנו חייבים להודות להשם יתברך על כל אשר גמלנו, ולהתחיל בגנות ולסיים בשבח, לכן ראה המגיד שלא יוכל אדם לעשות הודאה יותר הגונה ומתיחסת לעניין הפסח ויציאת מצרים כי אם אותה הגדה ווידוי שהיה קורא המביא את ביכורים, שגם הוא מתחיל בגנות ומסיים בשבח ומזכיר גם כן עניין היציאה והמכות והאותות והמופתים, ויעשה אדם עצמו בליל פסח עם אותם הדברים אשר לפניו בקערה כאילו הוא מביא הביכורים, ולכן נותן ההודאה ואומר הוידוי כמו המביא ביכורים. ומפני זה התחיל הסיפור בלבן ולא הזכיר דבר מעשו, לפי שלא בא ענין לבן מצד עצמו אלא מפאת שנזכר באותה פרשה אשר בוידוי הבכורים שהתחיל בה.אמנם איך יצדק מה שאמר שאותה ההבטחה עמדה לו עד היום הזה קרוב לאלפים שנה אחר שכבר נתקיימה ביציאת מצרים? התשובה הזאת נלמד מפשט הכתובים או מדברי חז"ל. אם מן הכתובים, כי בסוף מראה בין הבתרים נאמר "ביום ההוא כרת ה' את אברם לאמר לזרעך נתתי את הארץ הזאת מנהר מצרים ועד הנהר הגדול נהר פרת, את הקני ואת הקניזי ואת הקדמוני ואת החתי ואת הפריזי ואת הרפאים ואת האמורי ואת הכנעני ואת הגרגשי ואת היבוסי" (בראשית ט"ו, י"ח – כ"א), הרי הם עשרה עממים וידוע שישראל לא נחלו בארץ כי אם שבעת העממים בלבד, לא ארץ הקני והקניזי והקדמוני שהם אדום ועמון ומואב שלא נכבשו לארץ ישראל, ועתידים להכבש בגאולה העתידה שנאמר "אדום ומואב משלוח ידם ובני עמון משמעתם" (ישעיהו י"א, י"ד). והיא אותה ההבטחה שנאמרה לאברהם בברית בין הבתרים, שהוא חלק לעתיד לבוא בגאולה האחרונה. ולכן בצדק אמר המגיד "היא שעמדה לאבותינו ולנו" וכו'. וכן מדבריהם ז"ל מבואר כי המאמר בברית בין הבתרים ראה אברהם לא לבד גלות מצרים כי אם גלויות שאר המלכיות, וכמו שדרשו על והנה אימה חשכה, אימה זו בבל, חשיכה זו מדי שהחשיכה פניהם של ישראל בצום ובתענית, גדולה זו יוון, נופלת עליו זו אדום. וכמו שאמר הרמב"ן שהרגיש הנביא כאילו הצרה האחרונה היא נופלת במשא כבד, וכאילו הודיע הקדוש ברוך הוא לאברהם כמשייר במתנתו שעוד ארבע מלכיות אחרות ישעבדום וימשלו בארצם אם יחטאו לפניו, וכן דרשו את הגוי אשר יעבודו דן אנכי לרבות ארבע מלכיות שעתיד הקדוש ברוך הוא לדון. ועל הבחינות האלה אמר המגיד שאותה ההבטחה עמדה לנו עד היום הזה. והתבאר בזה היתר הספקות אשר בשער מ"ד, מ"ה, ושער מ"ו.}%endcomment
\hebeng{וַיֵּרֶד מִצְרַיְמָה – אָנוּס עַל פִּי הַדִּבּוּר. וַיָּגָר שָׁם. מְלַמֵּד שֶׁלֹא יָרַד יַעֲקֹב אָבִינוּ לְהִשְׁתַּקֵּעַ בְּמִצְרַיִם אֶלָּא לָגוּר שָׁם, שֶׁנֶּאֱמַר: וַיֹּאמְרוּ אֶל־פַּרְעֹה, לָגוּר בָּאָרֶץ בָּאנוּ, כִּי אֵין מִרְעֶה לַצֹּאן אֲשֶׁר לַעֲבָדֶיךָ, כִּי כָבֵד הָרָעָב בְּאֶרֶץ כְּנָעַן. וְעַתָּה יֵשְׁבוּ־נָא עֲבָדֶיךָ בְּאֶרֶץ גֹּשֶן. }{"And he went down to Egypt" - helpless on account of the word {[in which God told Avraham that his descendants would have to go into exile]}. "And he resided there" - {[this]} teaches that Ya'akov, our father, didn't go down to settle in Egypt, but rather {[only]} to reside there, as it is stated (Genesis 47:4), "And they said to Pharaoh, 'To reside in the land have we come, since there is not enough pasture for your servant's flocks, since the famine is heavy in the land of Canaan, and now please grant that your servants should dwell in the Land of Goshen.'"}%
\commenta{\textrm{\textbf{וירד מצרימה אנוס עפ״י הדיבור}} לא נתבאר מאין למד זה. כי אין לפרש שמדייק הלשון ״וירד״ שהילוך זה הי׳ לו לרגש ירידה, אבל הן מצינו בכמה מקומות, שההליכה מארץ ישראל למצרים תתבטא בלשון ירידה מפני שארץ מצרים היא במורד כלפי עמדתה של א״י, וכה מציגו באברהם וירד מצרימה (פ׳ לך) וביוסף הורד מצרימה (פ׳ וישב), ירד ירדנו (פ׳ מקץ), רדה אלי (פ׳ ויגש), והרבה כאלה. אך אפשר לומר, כי כפי המתבאר בס״פ ויחי הי׳ יעקב דואג הרבה שלא להקבר במצרים, ובעת שנתבשר מחיי יוסף אמר אלך ואראנו בטרם אמות, הרי שחשב כי במעט קרב יומו, ומפני זה בודאי לא ברצונו הלך למצרים כי דאג על ביאת יומו ועל אפשרות קבורתו שם — והלך אנוס עפ״י הדבור. אך צריך באור, מה ענין האונס בזה ולמה זה. וצריך לומר, שהוא כדי לקיים הגזירה כי גר יהיה זרעך וכו׳ (פ׳ לך ט״ו י״ג). ואמנם לא נתבאר, למה נפל הגורל בזה על יעקב ולא על אברהם ויצחק. וצריך לומר, משום דכידוע, כל עיקר שעבוד מצרים הי׳ לתכלית התוצאות לכשיצאו משם, תוצאת החירות ומתן תורה והכניסה לא״י וכל המסתעף מזה, ואם הי׳ מקיים זה באברהם ויצחק היו נכללים בזה גם ישמעאל ועשו, והרצון בזה הי׳ לזכת בכל אלה את זרע יעקב שכולו קודש, ואגלאי מילתא למפרע, דמה שאמר כי גר יהיה זרעך וכו׳ כיון לזרע יעקב.\textrm{\textbf{וירד מצרימה ויגר שם במתי מעט ויהי שם לגוי גדול.}} קרוב לומר, שהלשון מהופך קצת, וצריך לומר וירד מצרימה במתי מעם ויגר שם ויהי שם לגוי גדול. יען כי מתי מעט מוסב רק על זמן הירידה למצרים, אבל הגירות היתה בהמון. ודבר סירוס מקראות הוא חזון נפרץ, כמו בפרשה לך (י״ד י״ב) ויקחו את לוט ואת רכושו בן אחי אברם — תחת את לוט בן אחי אברהם ואת רכושו. ובפרשת שמות (ד׳ ל״א) ויאמן העם וישמעו כי פקד ה׳ — תחת וישמעו כי פקד ה׳ את בני ישראל ויאמן העם, יען כי האמונה באה לאחרי השמועה, מאמינים אל מה ששומעים. ובפרשה בא (י״ב ל״ב) במאמר פרעה למשה ואהרן קומו צאו כאשר דברתם וברכתם גם אותי — תחת וגם ברכתם אותי, כי לפי הכתוב משמע שברכו את מי שהוא שיאמר שיברכו גם אותו, ולא נמצא שברכו את מי. וכן נראה בפ׳ בשלח (י״ד י״ג) כי אשר ראיתם את מצרים היום לא תוסיפו לראותם עוד עד עולם, וזה במשמע שלא יראו עוד בכלל איש מצרי, וזה לא יתכן, אך הלשון מהופך, ושיעורו, כי את מצרים אשר ראיתם היום לא תוסיפו לראותם. כי כולם יטבעו. ובר״פ ויקרא, אדם כי יקריב מכם, ולפי סגנון הלשון צריך לאמר אדם מכם כי יקריב. ושם (א׳ ט״ז) והקטיר המזבחה ונמצה דמו — תחת ונמצה דמו והקטיר המזבחה. כי מצוי הדם הוא קודם להקטרה אותו (וע׳ רש״י). ובפרשה אמור (כ״ב ב׳) דבר אל אהרן ואל בניו וינזרו מקדשי בני ישראל ולא יחללו את שם קדשי אשר הם מקדישים לי — תחת וינזרו מקדשי בני ישראל אשר הם מקדישים לי ולא יחללו את שם קדשי (וע׳ רש״י). ובפרשה חקת (י״ט ז׳) ורחץ בשרו במים ואחר יבוא אל המחנה וטמא עד הערב — תחת וטמא עד הערב ואחר יבוא אל המחנה, כי הכניסה למחנה הותרה לאחר הערב שמש (וע׳ רש״י). ובירמיה (י״ז נ׳) כל אוצרותיך לבז אתן במותיך בחטאת — תחת בחטאת במותיך. ובהושע (ח׳ ב׳) לי יזעקו אלהי ידענוך ישראל — תחת לי יזעקו ישראל אלהי ידענוך. ושם (י״ד ג׳) אמרו אליו כל תשא עון — תחת כל עון תשא. ובחבקוק (ג׳ ב׳) אלוה מתימן יבא וקדוש מהר פארן סלה — תחת וקדוש סלה מהר פארן, כי השם קדוש סלה הוא כמו חדוש נצחי, מעין הלשון מלך הכבוד סלה (תהילים כ״ד:י׳) והוא תואר נצחי. ובתהלים (פ׳ ו׳) ותשקמו בדמעות שליש — תחת ותשקמו דמעות בשליש (מין מדה). ועוד שם (קי״ט ח׳) את חקיך אשמור אל תעזבני עד מאוד — תחת את חקיך אשמור עד מאוד אל תעזבני. והרבה כהנה. וכן נשתמשו בסגנון זה בחז״ל. ואמרו בסוטה (ל״ח א׳) מקרא זה מסורס הוא, ובב״ב (קי״ט ב׳) סרס המקרא ודרשהו, ומ״ד פ׳ אחרי ריש לקיש מסרס קראי ודריש. ובקדושין (ע״ח רע״ב) פירשו הפסוק בשמואל א׳ (נ׳) ונר אלהים טרם יכבה ושמואל שוכב בהיכל ה׳ — ופירשו, ונר אלהים טרם יכבה בהיכל ושמואל שוכב — במקומו. וכן נמצא ענין סירוס באותיות, עפ״י העתקת אותיות במלה אחת ממוקדם למאוחר וממאוחר למוקדם, וזה יתבאר להלן בדרשה ובמורא גדול זו גילוי שכינה. מבואו שם כמה וכמה דוגמאות.}%endcomment%
\commentb{\textrm{\textbf{תשובות לשערים מ"ז – מ"ח}}\textrm{\textbf{וירד מצרימה אנוס על פי הדבור וכו'.}}המאמר הזה ראיתי בו נוסחאות שונות, כי יש ספרים אשר כתוב בהם הגירסא הזאת אשר כתבתי, ובספרים אחרים לא נמצא כלל דרש על וירד מצרימה כי אם על ויגר שם. והרמב"ם ז"ל בספר זמנים (ביד החזקה) כשהביא טופס ההגדה כך הביאה: "וירד מצרימה ויגר שם מלמד שלא ירד להשתקע" וכו' ולא הביא אנוס על פי הדבור, ואין ספק שהנוסחאות האל נמשכו מדרכי הדעות שכתבתי למעלה בסבות גלות מצרים. כי לפי הדרך הראשון שהיה הגלות בסבת חטא מכירת יוסף או לפי הדרך השלישי כי בא בבחירתם הרע, לא יתכן לדרוש וירד מצרימה אנוס על פי הדבור, כי יעקב ובניו בחיריים היו ולא אנוסים. אמנם לפי הדרך השני שהיתה הירידה נמשכת אחר הגזירה העליונה יש לדרוש וירד מצרימה אנוס על פי הדבור, שיעקב היה ירא לרדת מצרימה והשם יתברך הכריחו לבוא שמה, והוא על דרך שדרשו וישלחו מעמק חברון מעצה עמוקה של צדיק הקבור בחברון, לקיים מה שנאמר "כי גר יהיה זרעך", וכאילו סבת הסבות הוא יתברך העיד את רוח יעקב לרדת שמה וזהו אמרו אנוס על פי הדבור.וראוי לישב כפי הדעה הזאת כתובי התורה: "ויאמר ישראל רב! עוד יוסף בני חי, אלכה ואראנו בטרם אמות, ויסע ישראל וכל אשר לו ויבא בארה שבע ויזבח זבחים לאלהי אביו יצחק, ויאמר אלהים לישראל במראות הלילה, ויאמר יעקב יעקב! ויאמר הנני. ויאמר אנכי האל אלה אביך אל תירא מרדה מצרימה כי לגוי גדול אשימך שם. אנכי ארד עמך מצרימה ואנכי אעלך גם עלה, ויוסף ישית ידו על עיניך" (בראשית מ"ה, כ"ח – פרק מ"ו, ד') עד כאן.והספק בפסוקים האלה הוא שנראה כי יעקב התעורר ללכת למצרים קודם שאמר לו ה' אל תרא מרדה מצרימה, כי בידוע שלא היה ירא, כיוון שכבר היה הולך לשם.ועוד קשה למה זבח זבחים לאלהי אביו יצחק ולא לאלהי אבי אביו אברהם? ואין ראוי שנתפייס עם דברי רש"י שחייב בכבוד אביו יותר מכבוד זקנו, כי כבר כתב עליו הרמב"ן שהיה ראוי שיאמר לאלהי אבותיו, וכמו שאמר האלהים אשר התהלכו אבותי לפניו, ויזכיר אביו ראשונה ואחר כך אבי אביו שהיה ראש היחס ואבי המשפחה ולא היה ראוי לשוכחו. והנה דרך הרמב"ן על פי הקבלה ואני אפרש על פי הפשט. אך מתחילה אוסיף להקשות כי תמצא שהשיבו השם תברך אנכי אלהי אביך ולא אלהי אבותיך.ועוד שאלה שלישית כי אם נתיירא על שהוזקק לצאת חוצה לארץ מה הנחמה שנחמו ה' כי לגוי גדול אשימך שם? אם יהיו בניו בגלות מה יוסיף ומה יתן גדולתו ורבויו, אדרבא ראוי לפתור מזה ולהביא בלבו מורא על מורא, אם ישימהו במצרים לגוי גדול, כי אולי לא יוכלו לצאת משם.ועוד רביעית אם בא השם יתברך להבטיחו טובה למה לא הבטיחו בגאולה, כי באמרו לגוי גדול אשימך שם מודה שיתמידו בניו במצרים וישארו שם בגלות אלא שיתרבו מאוד.ועוד חמישית מה ענין אמרו אנכי ארד מך מצרימה ואני אעלך גם עלה, כי אם רצה להבטיחו שיהיה נקבר בארץ כדברי רש"י ז"ל היה לו לומר ואתה תבוא אל אבותיך בשלום ותקבר עמהם כמו שנאמר לאברהם ומה לו לרדת עמו ולעלות עמו?והנראה לי בזה הוא שיעקב לא עלה על לבו לרדת למצרים ולדור שם בקביעות אבל היתה דעתו לרדת שמה לראות את יוסף ולשוב מיד לארץ כנען, כי כל ישעו וחפצו היה לשבת בארץ הנבחרת הוא וזרעו, כי היה מקווה לרשת אותה כמו שנשבע ה' לאבותיו, אך בשמעו מבניו שבאו ממצרים כי עוד יוסף חי וכי הוא מושל בכל ארץ מצרים, והיו מספרים לו את כבודו ואף יקר תפארת גדולתו, כמו שציוה יוסף אותם, "והגדתם לאבי את כל כבודי במצרים" וגו', לכן השיבם הזקן רב עוד יוסף בני חי, רצה לומר אינו חושש לממשלתו וגדולתו כאשר אמרתם כי אם אל היותו חי ולכן אלכה ואראנו בטרם אמות כלומר שתהא הליכתי לבד לראותו ולשוב מיד. ולפי שהיה יעקב מסופק בהליכתו אולי לא יישר בעיני האלהים לצאת מהארץ הנבחרה להכנס בכור הברזל במצרים, בפרט בראותו שהקדוש ברוך הוא מנע מיצחק אביו ההליכה שמה גם בשנת רעבון ואמר לו "אל תרד מצרימה גור בארץ הזאת ואהיה עמך ואברכך כי לך ולזרעך אתן את כל הארצות האל" (בראשית כ"ו, ב-ג). גם היה ירא יעקב אולי ישיגהו המוות בהיותו במצרים, ומה יעשה לעת פקודה? ומפני הפחד והמורא והספק שהיה בלבו בא לבאר שבע לא שבא שם בדרך מסעו למצרים כי אם להתחנן שם לפני אלהיו שיודיעהו מה יעשה, להיות המקום ההוא מקום תפילת אבותיו. לכן זבח זבחים לאלהי אביו וייחד בזה את אביו יצחק לפי שממנו נמנעה ההליכה למצרים, וכאילו ביקש ושאל מה' שיודיעהו אם יהיה ענינו כענין יצחק אביו אם לא. והשם יתברך השיבו במראות הלילה, שבאה אליו הנבואה בחלום וקראה אליו יעקב יעקב, מפני שהייעוד של הנבואה ההיא בא לו לעצמו בתור איש פרטי ששמו יעקב וגם בא הייעוד בכלל לזרעו אחריו שנקראו גם כן בשם יעקב, ולהיות המאמר הזה כולל שתי הודאות אלה אחת לעצמו לאחת לזרעו, לכן הזכיר שמו שתי פעמים יעקב יעקב! אל תירא מרדה מצרימה, רצה לומר עם היות שאני מנעתי לאביך הירידה מצרימה אני מתיר לך אותה אבל אל תחשוב שיש שגוי לפני הלילה יען אנכי אלהי שמנעתי ממנו ההליכה ואנכי האומר לך אל תירא מרדה מצרימה כי אני ה' לא שגיתי, והשגוי הוא לפי תכונת המקבלים, כי יצחק לא היה ראוי לבוא בגלות מפני שהיה עולה תמימה על העקדה ולכן נמנעה ממנו הירידה, אבל יעקב עתיד היה לבוא שם ולכן לא נמנעה הירידה מיעקב.האמנם לפי שהיה יעקב ירא מארבעה דברים לפיכך הבטיח לו ה' בהם:א – שמא במצרים יתמעט זרעו,  כי אולי יהרגו המצריים את בניו ויענו את בנותיו כי ארץ חמס היא, וכנגד זה הבטיחו "כי לגוי גדול אשימך שם", כלומר שלא יתמעט שם זרעך אלא יתרבה במאוד מאוד.ב – שהיה מפחד אולי להיותו יוצא מארץ הקדושה תיפרד ממנו ההשגחה האלוהית הדבקה בה, וכגד זה הבטיחו "אנכי ארך עמך מצרימה" רצה לומר לא תהיה שם בהסתר פנים כי גם שם תדבק בך השגחתי.ג - היה מפחד שמא ימות במצרים ויקבר שם ולא יזכה להיקבר במערת המכפלה עם אבותיו, וכנגד זה אמר לו "ואנכי אעלך גם עלה" רצה לומר שהשם יתברך יסבב סבות שיביאהו להיקבר בארץ עם אבותיו בכבוד ובמעלה רמה.ד – שהיה מפחד אולי ימות יוסף בחייו ויהיה לאבל כינורו ושמחתו לקול בוכים, וכנגד זה אמר "ויוסף ישית ידו על עיניך" שהוא בשעת המיתה. וכל זה אמר כנגד יעקב הפרטי, אמנם כנגד יעקב הכולל כל האומה הישראלית אמר גם כן "אל תירא מרדה מצרימה כי לגוי גדול אשימך שם", רצה לומר אל תירא ואל תפחד כי הם לא ישנו את שמם ואת לשונם וחותם ברית קודש על בשרם ויראת אלהים בקרבם. וכאילו אמר מה לך לירא מן הדבר ההכרחי שלא תוכל להימלט ממנו בשום צד. גם כן אמר אל האומה "אנכי ארד עמך מצרים", להגיד שאף על פי שארץ מצרים מלאה גילולים ובלתי מובנת שתדבק בה ההשגחה והדבקות העליונה הנה עם זרע קודש הזה בפרט תדבק בו תמיד ההשגחה, וכמו שאמרו חז"ל גלו למצרים שכינה עמהם (מכילתא על שמות י"ב, מ'). עוד אמר גם כן להאומה "ואנכי אעלך גם עלה", שרמז בזה לגאולת מצרים ויציאת האומה משם בכוח אלהי במסות באותות ובמופתים, ועל זה אמר "ואנכי אעלך גם עלה". ומה טוב אמרו "גם עלה" שרמז שתהיה עלייתם משם ביד רמה בכבוד ובמעלה. אמנם מה שאמר עוד "ויוסף ישית ידו על עיניך", זה בעניין האומה, נאמר כנגד משה רבינו שיוסיף עליהם כבוד ומעלה, כי הוא בתורתו ישית ידו בין עיניהם להאיר להם.  הנה התבאר מזה שדברי המראה בעצמם נאמרו בפרט על יעקב אבינו ונאמרו בכלל על כל בית יעקב, והתבארו הכתובים על בוריים והותרו הספקות אשר העירותי עליהן. ולמדנו מזה שיעקב ירד למצרים אנוס על פי הדבור, כי הוא לא הסכים בדעתו לרדת עד שבא לבאר שבע ושם הודיעו השם יתברך שכך יצאה הגזרה מלפניו. ולפי זה נכונה הגירסא אנוס על פי הדבור, כפי הדרך השני שכתבתי למעלה בסיבת גלות מצרים, ולא בא עליו פסוק אחד כי הוא יוצא מפסוקים רבים ומכללות הפרשה, לכן לא הביא עליו המגיד פסוק לראיה, כי הפרשיות כולם ודברי המשורר שהזכרתי בדרך השני הם לראיה עליו. יש מפרשים אנוס על פי הדבור בשתי טעמים, אנוס לחוד ועל פי הדבור לחוד, ופירוש אנוס כי ירידת יעקב למצרים אונס היה לו או מפני מכירת יוסף או מפני הרעב, ועל פי הדבור עננו שנתן לו רשות לרדת. אבל מה שכתבתי ראשונה הוא האמת וזו כוונת המגיד בלי ספק, והותרו עם זה הספקות אשר בשער מ"ז ומ"ח.\textrm{\textbf{תשובות לשערים מ"ט – נ'}}\textrm{\textbf{ויגר שם מלמד שלא ירד להשתקע אלא לגור שם" וכו'.}}הנה לא ראה המגיד לדרוש "ויגר שם" על הישיבה במצרים בלשון הכתוב שנאמר "ויעקב גר בארץ חם" (תהלים ק"ה, כ"ג), לשתי סיבות: האחת, לפי שהיה ראוי לומר וישב שם, כיון שנתעכבו שם ימים רבים והיו תושבים בארץ, ולמה אם כן יזכיר בהם לשון גרות, אם לא שהחליט בהם הלשון ההוא כפי מחשבתם וכוונתם בעת שהלכו שמה. ובזה הדרך ראוי לפרש ויבא ישראל מצרימה ויעקב גר בארץ חם על כוונת ביאתו. והסיבה השניה ויגר שם במתי מעט מורה שלא אמר לשון הגרות על הישיבה שם בתמידות, כי זה היה בתחילת עניינם כשהיו מתי מעט שבעים נפש בעת שהלכו למצרים לא בעת שכבר ישבו בה, ולכן דרש בו הסברא הישרה ויגר שם במתי מעט על מחשבתם בעת שהלכו לגור שמה, ועל התמדת הישיבה דרש אחר כך "ויהי שם לגוי גדול ועצום". והנה הביא ראיה לזה מפסוק "לגור בארץ באנו" שהוא בסדר ויגש, וראה המגיד כפי אמיתת הסיפור שיוסף לא אמר להביא את אביו למצרים כי אם לשבת שם בשנות  הרעב כמו שאמר "וכלכלתי אותך שם כי עוד חמש שנים רעב פן תורש אתה וביתך וכל אשר לך" (בראשית מ"ה, י"א). והיה בדעתו שאחרי עבור שני הרעב ישובו לארץ כנען וכאשר באו מצרימה התחכם יוסף בתחבולה כדי שפרעה בעצמו יושיבם בארץ גושן, וציוה לאחיו שכאשר יציגם לפני פרעה והוא ישאלם מה מעשיכם יאמרו לו "אנשי מקנה היו עבדיך מנעורינו ועד עתה גם אנחנו גם אבותינו", וביאר להם הסבה למה יאמרו כן, והיא "בעבור תשבו בארץ גושן כי תועבת מצרים כל רועה צאן", ושמא לא יאבו להניחם לשבת בתוך ארץ מצרים. ועם היות העצה היעוצה הזאת היתה לתועלתם, אין ספק שאמת יהגה חִכּו כי רועה צאן היו הם ואבותיהם, אבל אמר יוסף כל זה כדי שלא יחשוד פרעה שבאו אחי יוסף לאכול את כל אשר לו ויוסף יתן להם מפתבג המלך ומנכסיו. והנה אחי יוסף באו לפני פרעה והוא שאל אותם מה מעשיכם? והם השיבוהו מה שציוה אותם יוסף: רועי צאן היו עבדיך גם אנחנו גם אבותינו" והיה די בזה, ולמה הוסיפו לומר עוד  "ויאמרו לגור בארץ באנו" וגו', ולמה תבעו בפיהם מפרעה "ועתה ישבו נא עבדיך בארץ גושן", הלא כוונת יוסף הייתה שהם יספרו לפניו במסיחים לפי תומם שהם רועי צאן, ופרעה בעצמו יתן להם רשות לשבת בארץ גושן מבלי שישאלו זאת בפיהם. אבל נראה כי אמיתת העניין הוא שהם אמרו רועי צאן עבדיך גם אנחנו גם אבותינו, ושתקו ועמדו ולא ענו עוד וחשבו שפרעה מעצמו ישיבם אם כן אפוא תשבו בארץ גושן, אבל פרעה ירד לסוף דעתם כאשר שמע מפיהם זה המאמר הראשון ולא השיבם כלל,  וכאשר ראו השבטים כן הוצרכו הם לדבר שנית ויאמרו לגור בארץ באנו, והייתה זו אמירה מחודשת שהודיעוהו אמיתת כוונתם שלא היתה לדור בארצו דירה מתמדת, כי אם לגור בארץ דרך עראי ולצורך השעה מהכרח המקנה כי כבד הרעב בארץ כנען, ובקשו ממנו בביאור שיתן להם רשות מפני מקניהם לשבת בארץ גושן אותם הימים המעטים שיהיו גרים בארצו. אך גם על המאמר השני לא השיב פרעה אליהם כלל, אך חזר פניו כנגד יוסף ואמר לו אביך ואחיך באו אליך, רצה לומר כל הדברים האלה אצלי הם הונאת דברים, כי הנה אביך ואחיך לא באו בסבת המקנה כי עיניך ראות שלא באו רועי צאן אחרים מכל ארץ כנען והאמת הוא שהמה באו \textrm{\textbf{אליך}} כלומר בעבור שאתה הוא שוטר ומושל בארצי, וכדי שתכלכל אותם והם רוצים לשבת בארץ גושן, ואני חפץ שתיטיב עמהם, וכיוון שכן הוא, הנה כל הארץ לפניך במיטב הארץ הושב" וגו'. על כל פנים נתברר מזה שמה שאמרו לגור בארץ באנו לא אמרו בדרך מרמה כי אם באמת ובתמים כי כן היתה דעתם בעזבם את ארץ כנען. ולכן אמר המגיד "ויגר שם מלמד שלא ירד להשתקע" וכו', והותרו בזה הספקות אשר בשערים מ"ט וחמישים.}%endcomment
\hebeng{בִּמְתֵי מְעָט. כְּמָה שֶּׁנֶּאֱמַר: בְּשִׁבְעִים נֶפֶשׁ יָרְדוּ אֲבוֹתֶיךָ מִצְרָיְמָה, וְעַתָּה שָׂמְךָ ה׳ אֱלֹהֶיךָ כְּכוֹכְבֵי הַשָּׁמַיִם לָרֹב. }{"As a small number" - as it is stated (Deuteronomy 10:22), "With seventy souls did your ancestors come down to Egypt, and now the Lord your God has made you as numerous as the stars of the sky."}%
\commentb{\textrm{\textbf{תשובות לשערים נ"א – נ"ב}}\textrm{\textbf{במתי מעט כמו שנאמר בשבעים נפש ירדו אבותיך מצריה וכו'.}}כבר הודעתיך שלהיות המצוה בלילה הזה לספר ביציאת מצרים, לכן בחר לו לאותו סיפור פרשת וידוי הבכורים שבאה בסדר והיה כי תבא, כדי שכל אשר בשם ישראל יכונה יקרא אותה בלילה הזה בעת שקערת החג לפניו, כמו שהיה קורא המביא בכורים לפני ה'. ולכן עשה עיקר הדרשה מאותה פרשה, כי מצא בה עדות דבר, רצה לומר הגנות בתחילתה והוא יסיים בשבח, ויש בה עניני הירידה למצרים הגלות והגאולה. אולם ראה להביא על כל מלה ומלה מאותה פרשה פסוקים אחרים מסיפור המאורע, לא לאמת ולהביא ראיה על הדבר, כי אלו ואלו דברי אלהים חיים, אבל הביא אותם הפסוקים האחרים בדרך הביאור  כאילו יבאר בהם אותם המלות שבאו בוידוי הבכורים ולגלות ענינם, לפי שבאו באותו וידוי בדרך קצרה, ויפרש אותם לפי מה שנזכר במאורע. ולפיכך לא בא פסוק על דרך ראיה כי אם בדרך ביאור, והמשל בזה בא בוידוי הבכורים: וירד מצרימה, ולא ידענו אם היתה הירידה ההיא בחיריית אם לא – ביאר המגיד שהיה אנוס על פי הדבור, וכן ויגר שם ביאר שלא נאמר זה על הזמן אשר ישבו במצרים, כיון שסמך במתי מעט ואמר בו לשון גרות, אלא שנאמר כפי הכוונה שהיה להם באשר הלכו שמה. וכן ביאר ויגר שם במתי מעט, שהמיעוט ההוא היה שבעים נפש שירדו עם יעקב למצרים, ולמד זה ממה שאמר משה רבינו עליו השלום בפרשת עקב, בשבעים נפש ירדו אבותיך מצרימה. ותחילת הדברים שם "הוא תהלתך והוא אלהיך אשר ראו עיניך... בשבעים נפש ירדו אבותיך מצרימה ועתה שמך ה' אלהיך ככוכבי השמים לרוב" (דברים י', כ"א-כ"ב), שגלה להם שכאשר ירדו למצרים היו מעטי הכמות ושפלי האיכות, ואמר "בשבעים נפש" שהוא הכמות, "ירדו אבותיך מצרימה" שהוא האיכות, שהיתה להם ההליכה שמה ירידה ושפלות גדולה להשתעבד למצרים ולהכנע להם, והנה במשך מאתיים שנה נתרבה כמותם ברבוי מופלג חוץ ממנהג הטבעי וכן עלו באיכות ומעלה עליונה, וזהו אמרו "עתה שמך ה' אלהיך ככוכבי השמים" רמז למעלתם וכבודם. ובאומרו "לרוב" רמז לכמות המופלג, ועליהם אמר "הוא תהלתך והוא אלהיך אשר עשה אתך את הנוראות ואת הגדולות " ורמז בגדולות רבוי העם והפלגתו משבעים נפש לשש מאות אלף רגלי מלבד הנשים והטף, וכמו שנאמר "וראו כל עמי הארץ כי שם ה' נקרא עליך ויראו ממך" (שם כ"ח, י'). ואולי שגם כן כיון לזה באמרו "ויגר שם במתי מעט" כי "מתי" מורה על האיכות שהיו במתים נעדרי היכולת והכחות ו"מעט" נאמר על הכמות. הנה התבאר מזה שהביא פסוקים על פסוקים לצורך ביאור הביאור, ועשה עיקר מפרשת וידוי הבכורים לפי שהיא הנאות יותר לסיפור יציאת מצרים, והותרו בז הספקות אשר בשערים נ"א ונ"ב.ויש ללמוד גם כן שהיו בזה האומה התחלות שונות מחלופי מספרים, כי תמצא להם התחלה אחת שהיא אברהם וכמו שאמר "אחד היה אברהם וירש את ארץ" (יחזקאל ל"ג, כ"ד), ותמצא התחלת שלשה אברהם יצחק ויעקב, ונמצא להם התחלה של י"ב שבטים, והתחלה של שבעת הרועים שהם שלשת אבות אברהם יצחק ויעקב ושני נביאים משה ואהרן ושני מלכים דוד ושלמה, ונמצאו להם התחלתה של שבעים נפש אשר ירדו למצרים. פקח עיניך וראה כי כל בחינות ההתחלות האלה יורו על ענינים גדולים מטבע המציאות, אם האחדות שהוא האב הראשון אברהם כנגד הממציא הראשון שהוא השם יתברך, ושלשה אבות שהם כנגד שלשה עולמות הרוחני והשמיימי והחומרי, וי"ב שבטים כנגד י"ב מזלות, להורות כי אותם בעולם השפל הם חשובים כמו אלה בעולם השמיימי. והיו גם כן שבעה רועים כנגד שבעה כוכבי לכת, וכמו השמש היא באמצע הרקיע ולמעלה ממנו שבתאי צדק ומאדים ולמטה ממני נוגה כוכב לבנה, כך משה רבינו עליו השלום שהוא השמש המאיר לארץ ולדרים עליה היה באמצע הרועים והיו עליו ראשונים ממנו אברהם יצחק ויעקב, ואחרונים אליו אהרן דוד ושלמה. והיתה גם כן התחלת השבעים כנגד שבעים שרים שבמרום הממונים על שבעים אומות, להגיד שכל איש מיורדי מצרים היה שקול כנגד כל אחת מהאומות, ועל זה אמר "בהנחל עליון גויים בהפרידו בני אדם יצב גבולות עמים למספר בני ישראל" (דברים ל"ב, ח'), רצה לומר שהיו האומות וגבולות העמים במספר בני ישראל שהם שבעים יורדי מצרים.}%endcomment
\hebeng{וַיְהִי שָׁם לְגוֹי. מְלַמֵד שֶׁהָיוּ יִשְׂרָאֵל מְצֻיָּנִים שָׁם. גָּדוֹל עָצוּם – כְּמָה שֶּׁנֶּאֱמַר: וּבְנֵי יִשְׂרָאֵל פָּרוּ וַיִּשְׁרְצוּ וַיִּרְבּוּ וַיַּעַצְמוּ בִּמְאֹד מְאֹד, וַתִּמָּלֵא הָאָרֶץ אֹתָם. }{"And he became there a nation" - {[this]} teaches that Israel {[became]} distinguishable there. "Great, powerful" - as it is stated (Exodus 1:7), "And the Children of Israel multiplied and swarmed and grew numerous and strong, most exceedingly and the land became full of them." }%
\commenta{\textrm{\textbf{ויהי שם לגוי גדול מלמד שהיו ישראל מצוינים שם}} הנה לשם ״גדול״ שתי הוראות, האחת, גדול בכמות, במדה ובמספר, כמו המאור הגדול (פ׳ בראשית), ברכוש גדול (פ׳ לך), הים הגדול, ועוד כהנה. וההוראה השניה — גדול באיכות ובמעלה, כמו איננו גדול בבית הזה ממני (פ׳ וישב, ל״ט ט׳), והכהן הגדול מאחיו (פ׳ אמור), מטעם המלך וגדוליו (יונה ד׳), גדול שמי בגוים (מלאכי א׳ י״א) ועוד כהנה. והנה כאן אין הכרע, לאיזו הוראה יתיחש כאן השם גוי גדול, אם יתפרש גדול בכמות ובמספר או גדול באיכות ובמעלה וכבוד, ולכן צריך באור מאין למד המגיד, כי הביאור כאן משם גוי גדול הוא מענין גדולה וכבוד, כמו שאמר, שהיו מצוינים שם, וענין ציון יונח על מי שנהדר במעלת הכבוד וההידור, וכמ״ש במדרש שה״ש על הפסוק פתחי לי אחותי רעיתי יונתי, (ה׳ ב׳) מהו יונתי — כיונה זו שמצטיינת מכל הצפרים בתמימותה ובענותה כך ישראל מצוינים מכל העמים במעלת התורה והמצות וכו׳. והנה בזה חלוק שם ״גוי״ משם ״עם״, כי ״עם״ מורה על המון עם, כמו העם ההולכים בחושך (ישעיה מ״ט א׳), ובירמיה (ה׳ ח׳) מדוע שובבה העם הזה, וכמ״ד  פ׳ בלק (פ׳ כ׳) כל מקום שנאמר עם הוא לשון גנאי, וכן ברש״י פ׳ בהעלותך (י״א א׳) בפסוק ויהי העם כמתאוננים. ולהיפך השם ״גוי״ מורה על גדולי העם ומכובדים וכמו ישראל גוי אחד בארץ (ש״ב ז׳ כ״ג), גוי וקהל גוים יהיו ממך (פ׳ וישלח ל״ה), והלשון שני גויים בבטנך (ר״פ תולדות) פירשו לרמז לשני גדולי העם שיצאו מיעקב ומעשו, והם רבי ואנטונינוס המלך (ברכות נ״ז ב׳), ועל כן משותף שם זה עם שם צדיק, הגוי גם צדיק תהרוג (פ׳ וירא). ולכן מדכתיב כאן ויהי שם לגוי, ולא כתיב שם לעם, דריש שהיו ישראל מצוינים שם, כלומר, מצוינים בגדולה וכבוד.\textrm{\textbf{גדול עצום כמה שנאמר ובני ישראל פרו וישרצו וירבו ויעצמו}} כדי לחזק דבריו ״גדול עצום״ שמוסב על גודל הכמות ועצמות האיכות הביא ארבעה לשונות המורים מעלות מתכונות המתייחשות לתולדה. ואמר, ובני ישראל פרו, פשוט שהיו שלמים בכחות ההולדה ולא היו בהם עקרים. וישרצו — רמז להמבואר באגדות שהיו יולדות ששה בכרס אחד, ושם שרץ מורה על הפלגת רבוי התולדה, וכמש״כ בפ׳ נח (ט׳ י׳) שרצו בארץ. וירבו — הוא לשון גידול, כמו רבתה גוריה (יחזקאל י״ט ב׳), ובאיכה (ב׳ כ״ב) טפחתי ורביתי, והכונה, שגדלו ילדיהם בנחת ובקלות. ויעצמו — אשמעינן ריבותא אעפ״י שהנולדים במספר מרובה הם בטבעם חלושים ורפויי כח, ושם נולדו ששה בכרס אחד, כמבואר למעלה, והי׳ מצד הטבע שיהיו חלושים — אמר ויעצמו שהיו בריאים וחזקים, כלשון הפסוק בישעיה (מ׳ כ״ט) ולאין אונים עצמה ירבה.}%endcomment%
\commentb{\textrm{\textbf{תשובות לשערים נ"ג – נ"ט}}\textrm{\textbf{ויהי שם לגוי גדול מלמד שהיו ישראל מצויינין שם וכו'.}}הציון הלז נדרש על מילת גוי בלבד, ועוד דרש המגיד על מלת "גדול ועצום" מה שאמר הכתוב "ובני ישראל פרו וישרצו" וגו' ועל מלת "ורב" דרש "רבבה כצמח השדה נתתיך" (יחזקאל ט"ז, ז'), ולפי זה באה דרשה על כל מלה ומלה בפני עצמה, שהמגיד מצא בכתוב הזה אשר בוידוי הבכורים תארים "גוי גדול עצום ורב", שהם ארבעה שמות התאר, ולכן דרש שיורו על ארבעה ענינים נפלאים שנמצאו בבני ישראל בהיותם בגלות.א' -  היה גוי רצה לומר שתמיד היתה האומה נבדלת ונפרשת מהמצריים ולא נתערבו בתוכם בשנות מאתיים ועשר שישבו ביניהם  כמו שיקרה לעמים כולם כי בהתישב גוי בקרב גוי יתערבו אלה באלה והיו לגוי אחד ודת אחד לכולם, וכמו שאמר חמור ושכם בנו לאנשי עירם וגם בני יעקב אמרו להם ונתנו את בנותינו לכם ואת בנותיכם נקח לנו וישבנו אתכם והיינו לעם אחד, אבל ישראל במצרים לא היו כן, שלא שינו את שמם ולא שנו את לשונם ודתם ומלבושיהם בכל השנים הרבות ההן אשר ישבו שם. באופן שתמיד היו גוי נבדל מהמצריים, וכמו שאמר משה אדוננו "או הנסה אלהים לבא לקחת לו גוי מקרב גוי" שהיו בני ישראל והמצריים נבדלים זה מזה, ועל זה אמר המגיד ודרש במלת "גוי, מלמד שהיו ישראל מצויינין, והציון הוא הסימן אשר בו ניכר ורשום האדם, וכן היו ישראל מצויינין ונרשמים להיותם נבדלים ומפורשים מהמצריים בכל מנהגיהם ועניניהם.ב - עוד דרש במלת "גדול" רבוי העם וכמותו, וענין זה שונה מדרכי הטבע שהרבוי אשר יפול באחת האומות לרוב הוא כשתתחבר אליה אומה אחרת ותתערב בה והיו לעם אחד, וכמו שאמר ורבים מעמי הארץ מתיהדים" (אסתר ח', י"ז), אבל ישראל לא היה כן, כי אף בהיותם מצויינין נתרבו בכמות מופלג על דרך הפלא. וכבר העיר על הפליאה הזו בלעם בדברי נבואתו באמרו "כי מראש צורים אראנו ומגבעות אשורנו הן עם לבדד ישכון ובגויים לא תחשב, מי מנה עפר יעקב" (במדבר כ"ג, ט' – י'), ופירושו אצלי שאמר הנה כל האומות יתרבו בהתערב בהן אומות אחרות, אבל העם הזה אינו כן, כי אני מביט ורואה אותם מתחילת שורשיהם והם הצורים האבות, מגבעות אשורנו שהן האמהות, וראיתי שהם מסתעפים בהסתעפות והשתלשלות ישר זה מזה, ושהם תמיד עם לבדד ישכון ולא עבר זר ביניהם, וזהו ובגויים לא יתחשב, רצה לומר בגויים אחרים לא יתערב ולא יתחשב בתוכם, ועם כל זה נתרבה לאין שיעור, והוא שאמר מי מנה עפר יעקב לרוב ריבויים, וכל זה כלל המגיד בדרשת "גדול".ג' – עוד עשה דרשה אחרת במלת "עצום" שנאמר על הכוח והעוצם, וכבר דרשו חז"ל על וירבו ויעצמו במאד מאד שהיו פרים ורבים שלא הפילו נשותיהם ולדותיהן, וישרצו שהיו יולדות ששה בכרס אחד, שנאמר פרו וישרצו וירבו, כל אחד ואחד שנים הרי שישה. והנה אף שהיו תאומים והם לרוב חלושי כוח לפי שהטבע לא הספיק לתת לכל אחד כח ועצמה לפי שנחלק ביניהם, ולכן ילדי התאומים לרוב קטני הקומה דקי הפרצוף וחלושי הכוח, אבל ילדי ישראל בחמלת ה' עליהם היו גדולים בגופם והקיפים וחזקים באבריהם.ולפי זה בפסוק שהביא התבארו שלשת הדרשות, כי הראשונה שאמר ויהי שם לגוי מלמד שהיו ישראל מצויינין למד מאמרו "ובני ישראל" כי להיותם מצויינין ונרשמים וניכרים לגוי אחד בארץ נקראו תמיד כן בשמם, והדרשה השנית שדרש מן גדול שכוונו בו על הריבוי התבארה מאמרו "פרו וישרצו" שנאמר על הריבוי כמו שכתבתי, והדרשה השלישית שעשה מן עצום שהוא על העצמה ובכוח התבארה באמרו "וירבו ויעצמו במאד מאד" שרמז בלשון וירבו על גודל הקומה ויעצמו על הכוח.ד' - אמנם מה שאמר עוד בכתוב "ורב" עשה בו דרשה רביעית, וענינה כי מדרך הטבע הילדים בינקותם הם עלולים לסכנות ורובם מתים מחולאים כי להיותם רכים לא יסבלו פשיעה בגידולם, וכמו שאמר יעקב לעשו "אדוני יודע כי הילדים רכים וגו' ודפקום יום אחד וגו" (בראשית ל"ג, י"ג). לכן אמר שבהיותם במצרים לא היו כן, כי מתוקפן של צרות הגלות לא היו חוששין על גידול הילדים כראוי, אך כיד ה' הטובה עליהם היו גדלים מעצמם כצמח השדה. ולכן הביא לבאר זה מהפסוק יחזקאל "רבבה כצמח השדה נתתיך ותגדלי" וגו' רצה לומר שהם היו גדלים כמי הדשאים שיצאו מבטן האדמה בצבעים יפים ושונים מבלי שום בעל מלאכה ולא נמצא אומן צובע שיוכל לעשות כמתכונתם. ככה היו בני ישראל גדלים ביפים ובמידותיהם מבלי חינוך ולימוד, כאילו היו נולדים בטבע באותה שלמות. וזהו שאמר "ותרבי ותגדלי ותבואי בעדי עדיים שדים נכונו ושערך צימח" (יחזקאל ט"ז, ז'), שכולו נאמר על גודל הגוף וגודלו ותקון המידות שהוא עדי עדיים, אף כי היו ערום ועריה מלימוד התורה והמוסר שהוא המשלים את האדם במעלותיו, וכבר פירשו הקדושים ז"ל שבנות ישראל במצרים בבואם ללדת, מיראת המצריים שיקחו את בניהם להשליכם ליאור היו יוצאות השדה ויולדות תחת האילנות כדי שלא ישמעו את קולם, ועל זה אמר החכם "תחת התפוח עוררתיך שמה חבלתך אמך" וגו' (שיר השירים ח', ה'), והיו מניחין שם הבנים וחוזרין להניקם והם היו גדלים מעצמם בתוך העשבים, ועל זה אמר "רבבה כצמח השדה נתתיך". ואמר שדים נכונים על משה אהרן שהיו שתי דדים שזימן להן הקדוש ברוך הוא לישראל, ושערך צמח, דרשו אותו על השבטים שכבר היו ראויים להיגמל, ואת עירום ועריה שהיו ערומים מן המצוות, והעניין כולו שדמה אותם ככלה והקדוש ברוך הוא כחתן.והותרו עם זה הספקות אשר בשערים נ"ג, נ"ד, נ"ה ונ"ו.אמנם לא אמר בכל הדרשות האלה "מלמד" אלא בדרשת ויגר שם ובדרשת ויהי שם לגוי לפי שהדרשות האלה הוציא אותן בדרך הסברא מסגנון הפרשה ופירוש הכתובים ולא היה בהם הדבר מבואר כשאר הדרשות.}%endcomment
\hebeng{וָרָב. כְּמָה שֶּׁנֶּאֱמַר: רְבָבָה כְּצֶמַח הַשָּׂדֶה נְתַתִּיךְ, וַתִּרְבִּי וַתִּגְדְּלִי וַתָּבֹאִי בַּעֲדִי עֲדָיִים, שָׁדַיִם נָכֹנוּ וּשְׂעָרֵךְ צִמֵּחַ, וְאַתְּ עֵרֹם וְעֶרְיָה. וָאֶעֱבֹר עָלַיִךְ וָאֶרְאֵךְ מִתְבּוֹסֶסֶת בְּדָמָיִךְ, וָאֹמַר לָךְ בְּדָמַיִךְ חֲיִי, וָאֹמַר לָךְ בְּדָמַיִךְ חֲיִי.}{"And numerous" - as it is stated (Ezekiel 16:7), "I have given you to be numerous as the vegetation of the field, and you increased and grew and became highly ornamented, your breasts were set and your hair grew, but you were naked and barren." "And when I passed by thee, and saw thee weltering in thy blood, I said to thee, In thy blood live! yea, I said to thee, In thy blood live!" (Ezekiel 16:6).}
\hebeng{וַיָּרֵעוּ אֹתָנוּ הַמִּצְרִים וַיְעַנּוּנוּ, וַיִתְּנוּ עָלֵינוּ עֲבֹדָה קָשָׁה. וַיָּרֵעוּ אֹתָנוּ הַמִּצְרִים – כְּמָה שֶּׁנֶּאֱמַר: הָבָה נִתְחַכְּמָה לוֹ פֶּן יִרְבֶּה, וְהָיָה כִּי תִקְרֶאנָה מִלְחָמָה וְנוֹסַף גַּם הוּא עַל שֹׂנְאֵינוּ וְנִלְחַם־בָּנוּ, וְעָלָה מִן־הָאָרֶץ. }{"And the Egyptians did bad to us" (Deuteronomy 26:6) - as it is stated (Exodus 1:10), "Let us be wise towards him, lest he multiply and it will be that when war is called, he too will join with our enemies and fight against us and go up from the land."}%
\commenta{\textrm{\textbf{וירעו אותנו המצרים ויענונו}} בכל מקום שבא הפועל ״הרע״ מושך אחריו את השם המתיחש להפעל בקשר אות למ״ד, לי, לנו, להם, לכם, וכדומה, ולא בקשר ״את״, כמו בפרשה מקץ (מ״ג ו') למה הרעותם לי, ולא אותי, ובס״פ שמות (ה׳ כ״ב) למה הרעות  לעם הזה, ושם (פסוק כ״ג) הרע לעם הזה, ובפרשה בהעלתך (י״א י״א) למה הרעות לעבדך, ובתהלים (ק״ו ל״ב) וירע למשה בעבורם, וביהושע (כ״ד כ׳) ושב והרע לכם, ונעמי אמרה ושדי הרע לי (רות כ״א), ועוד כהנה, ולפי זה הי׳ צריך לומר כאן וירעו לנו המצרים, ולא אותנו. ואפשר לבאר עפ״י המבואר באגדות, שהיו המצרים בושים מפני עמי תבל על אשר יענו על לא דבר את עם ישראל, עם שקט ושאנן — ולכן, המציאו להם הצטדקות בזה שהיו בודים עילות ואשמות על ישראל ועשו אותם בעיני עולם לאנשים רעים וחטאים, ובזה מצאו להם הצטדקות על שעבודם ועל ענותם אותם. ולפי זה יתפרש הלשון וירעו אותנו המצרים, שעשו אותנו לפני עמי תבל לרעים וחטאים. ועל כן — ויענונו — כלומר, על כן מצאו להם עילה להתנצלות על מעשיהם אתנו לענות אותנו, ומדויק המשך הלשון וירעו אותנו — ויענונו. וראוי להעיר, כי אמנם פעם אחת, ובענין זה ממש, בענוי מצרים אותנו, כתיב הלשון וירעו לנו, והוא בפרשה חקת, בשלוח משה מלאכים אל מלך אדום, ולתכלית בקשתו ממנו (כפי שיבא בסמוך) הרצה לפניו מעט מקורות עם ישראל ומתלאותיו, ובהמשך הדברים אמר, ונשב במצרים וירעו לנו מצרים (כ׳ ט״ו), והנה שינה מלשון התורה, מן וירעו אותנו ל״לנו״. אך על האמת אפשר לומר כמו קרוב לודאי, כי במכוון הדגיש משה כאן הלשון וירעו לנו, ולא ״אותנו״, יען כי כפי המתבאר בפרשה, היתה תכלית דבריו אל מלך אדום שירשה לישראל לעבור דרך ארצו, והבטיחו, כי ״לא נעבור בשדה ובכרם ולא נשתה מי באר ולא נטה ימין ושמאל (שם פסוק י״ז), כלומר, שלא יזיקו ולא יפסידו לו במאומה בעברם דרך ארצו. והי׳ חושש, שאם יאמר לו לשון וירעו אותנו מצרים, לשון שאפשר להבין בו שעשו אותנו המצרים לרעים וחטאים, כמו שבארנו — אפשר שיחשוב מלך אדום, כי אולי אמנם כן הוא, כפי שהעריכו אותם המצרים, שהם רעים וחטאים, ואז בטח לא יתן להם לעבור דרך ארצו, כאנשים חשובים. וכן שינה בדבריו מלשון ״אותנו״ ללשון ״לנו״.}%endcomment%
\commentb{\textrm{\textbf{וירעו אותנו המצריים כמו שנאמר הבה נתחכמה לו פן ירבה וכו'.}}  גם זה מוידוי הבכורים, והמגיד מצא בכתוב הזה שלושה חלקים: א – וירעו אותנו המצרים. ב – ויענונו. ג' – ויתנו עלינו עבודה קשה.והוא דורש בו שלוש דרשות מתייחסות זו לזו, שפירש וירעו אותנו המצריים שחשדו את בני ישראל לאנשים רעים וחטאים כי עם היותם זרע ברך ה' ועולה לא נמצא בשפתיו, בכל זאת נתנו אותם כמרגלים את הארץ, בדרך הקושרים והמורדים המתפרצים באדוניהם, והראיה על פירוש זה שלא אמר וירעו לנו כי אם וירעו אותנו, לפי שאין הכוונה שעשו עמהם רעה כי אם שחשדום כאנשי רשע, והוכיח זה מהפסוק "הבה נתחכמה לו" שהיא הייתה עצת המצריים להתעולל עלילות כנגד ישראל ולהתחכם בתחבולות כדי שלא יתרבו ולא יתחזקו בארץ, כמו שנאמר "והיה כי תקרינה מלחמה ונוסף גם הוא על שונאינו ונלחם בנו ועלה מן הארץ". שחשבו כי בני ישראל ברשעתם יעשו זה אם יתרבו, ומפני שחשדום לרעים ופושעים באו להתחכם ולהתייעץ על זה. ומהרעה הזאת נמשכה בדרך סיבוב רעה אחרת, והיא מה שאמר "ויענונו", רצה לומר שעלה בעצתם שיענו אותם בכל מיני עינויים כדי לתשש כחם כנקבה ולא יעצרו בה לעשות דבר ממה שדמו. אמנם מה היה העינוי הזה? הנה התורה ספרה אותו שנאמר "וישימו עליו שרי מסים למען ענותו בסבלותם, ויבן ערי מסכנות לפרעה" כי העינוי ההוא היה במלאכת החומר והלבנים ובנין הערים למלך. ואתה תבין מזה מה ראה המגיד לדרוש וירעו אותנו המצריים על עניין הבה נתחכמה לו, כי הביאוהו לזה מה שאמר הכתוב אחרי וירעו אותנו המצרים "ויענונו", והיה אחרי כן העינוי מסובב ונמשך מן "וירעו", והנה מצא המגיד שהעינוי היה בחומר ובלבנים ובבנין הערים כמו שנאמר "וישימו עליו שרי מסים למען ענותו", ולמד ענוי מענוי, ואותו הענוי הנזכר שם נמשך מעצת "הבה נתחכמה לו", והוא שיוכיח כי מה שנאמר בוידוי הבכורים "וירעו אותנו המצרים" היתה אותה עצה שסבבה העינוי. עוד הזכיר שנמשכה מאותה העצה רעה אחרת והתחכמות משחת, שמלבד העינוי ההוא שהיו עובדים למלך בבנין הערים נתנו עליהם "עבודה קשה", והיא שכל מצרי מן העם היה משעבד את בני ישראל כאילו היו עבדיו ושפחותיו. ולפי זה "ויענונו" נאמר על עבודת המלך ונקראת ענוי כפי איכותה ועמלה הרב, "ויתנו עלינו עבודה קשה" נאמר על השתעבדות של כל אחד מן המצריים בבני ישראל, כמו שביאר מפסוק "ויעבידו מצרים את בני ישראל בפרך", שלמד עבודה מעבודה, ולפי שהיה השעבוד הזה בפרך וזדון ואכזריות יתירה נאמר עליו, "עבודה קשה", וגם כי עבודת המלך היתה מסודרת בפלס ומאזני משפט כי מלך במשפט יעמיד ארץ, אך עבודת ההמון המצריים הייתה בלי סדר ומשטר, כי כל איש שורר לעצמו והשתרר על הישראל לכן נקראת עבודה זו קשה ועבודת פרך, וכמאמר המשורר "מִכָּל פְּשָׁעַי הַצִּילֵנִי חֶרְפַּת נָבָל אַל תְּשִׂימֵנִי" (תהלים ל"ט, ט'). וחז"ל דרשו הבה נתחכמה לו למושיעם של ישראל, בטעם גוי ואלוהיו, וכוונו בזה מה שהתחתונים קשורים בעליונים, כמו שנאמר "וְהָיָה בַּיּוֹם הַהוּא יִפְקֹד יְהוָה עַל צְבָא הַמָּרוֹם בַּמָּרוֹם וְעַל מַלְכֵי הָאֲדָמָה עַל הָאֲדָמָה" (ישעיהו כ"ד, כ"א), וכאילו התחכמו אל ה' כמו שהתחכמו על עמו ועל חסידיו.והותרו בזה הספקות אר בשערים נ"ז, נ"ח, נ"ט.}%endcomment
\hebeng{וַיְעַנּוּנוּ. כְּמָה שֶּׁנֶּאֱמַר: וַיָּשִׂימוּ עָלָיו שָׂרֵי מִסִּים לְמַעַן עַנֹּתוֹ בְּסִבְלֹתָם. וַיִּבֶן עָרֵי מִסְכְּנוֹת לְפַרְעֹה. אֶת־פִּתֹם וְאֶת־רַעַמְסֵס.}{"And afflicted us" - as is is stated (Exodus 1:11); "And they placed upon him leaders over the work-tax in order to afflict them with their burdens; and they built storage cities, Pithom and Ra'amses." }%
\commenta{\textrm{\textbf{ויענונו כמה שנאמר וישימו עליו שרי מסים למען ענותו בסבלותם}} לכאורה אין בפסוק שמביא לראי׳ יותר משמעות לענוי מאשר הלשון ויענונו, כי שניהם מוכנים בתכלית ענינם, בהוראת ענוי, ומדרך הלשון ״במה שנאמר״ הוא לחדש דבר והוראה בהלשון הקודם, מה שזולת הראי׳ הי׳ המובן באותו הלשון חסר או רפוי, וכאן לא נתבאר איזה תוספת חידוש והוראה בלשון שמביא לראיה. ואפשר לומר שבהפסוק שמביא לראי׳ מכוין להודיע רשעתם של המצרים שלא עבדו בישראל מפני שהיו צריכים לתוצאת העבודה. אך הכונה למען ענותם. זה רשעות כפולה, כובד העבודה ומטרתה, וזה כלול בתוספת הראי׳ מן וישימו עליו שרי מסים, לא למען גוף העבודה אך למען ענותו בסבלותם, אבל מן ויענונו לבד אין זה מוכח, דאפשר עינו אותנו בעבודה מפני שהיו צריכים לגוף העבודה. כמבואר.}%endcomment
\hebeng{וַיִתְּנוּ עָלֵינוּ עֲבֹדָה קָשָׁה. כְּמָה שֶֹׁנֶּאֱמַר: וַיַּעֲבִדוּ מִצְרַיִם אֶת־בְּנֵי יִשְׂרָאֵל בְּפָרֶךְ.}{"And put upon us hard work" - as it is stated (Exodus 1:11), "And they enslaved the children of Israel with breaking work."}
\hebeng{}{}
\hebeng{וַנִּצְעַק אֶל־ה׳ אֱלֹהֵי אֲבֹתֵינוּ, וַיִּשְׁמַע ה׳ אֶת־קֹלֵנוּ, וַיַּרְא אֶת־עָנְיֵנוּ וְאֶת עֲמָלֵנוּ וְאֶת לַחֲצֵנוּ. }{"And we we cried out to the Lord, the God of our ancestors, and the Lord heard our voice, and He saw our affliction, and our toil and our duress" (Deuteronomy 26:7).}%
\commenta{\textrm{\textbf{ונצעק אל ה׳ אלהי אבותינו וישמע ה׳ את קולנו}} לפי סגנון הלשון היה די לאמר וישמע את קולנו, ושם ה׳ אינו מוכרח לכאן, כיון דהשמיעה מוסבת אל הלשון הקודם, ונצעק אל ה׳ אלהי אבותינו. אך יש לומר, דבלשון זה בא לרמז מה שאמרו בברכות (י׳ ב׳) כל התולה (תפלתו ובקשתו) בזכות עצמו תולין לו בזכות אחרים, וכל התולה בזכות אחרים תולין לו בזכות עצמו, מהו שאמר כאן, כי ישראל במצרים לא מצאו און להם לבקש בזכות עצמן ובקשו בזכות אבות, אבל ה׳ שמע להם בזכות שלהם, וזהו באור הלשון ונצעק אל ה׳ אלהי אבותינו — בזכות אבותינו, וישמע ה׳ את קולנו — בזכות שלנו.}%endcomment
\hebeng{וַנִּצְעַק אֶל־ה׳ אֱלֹהֵי אֲבֹתֵינוּ – כְּמָה שֶּׁנֶּאֱמַר: וַיְהִי בַיָּמִים הָרַבִּים הָהֵם וַיָּמָת מֶלֶךְ מִצְרַיִם, וַיֵּאָנְחוּ בְנֵי־יִשְׂרָאֵל מִ־הָעֲבוֹדָה וַיִּזְעָקוּ, וַתַּעַל שַׁוְעָתָם אֶל־הָאֱלֹהִים מִן הָעֲבֹדָה. }{"And we cried out to the Lord, the God of our ancestors" - as it is stated (Exodus 2:23); "And it was in those great days that the king of Egypt died and the Children of Israel sighed from the work and yelled out, and their supplication went up to God from the work."}%
\commentb{\textrm{\textbf{תשובות לשער ס'}}\textrm{\textbf{ונצעק אל ה' אל ה' אלהינו כמו שנאמר "ויהי בימים הרבים ההם"}} (שמות ב', כ"ג) \textrm{\textbf{וכו'.}}גם זה מוידוי הבכורים, ואף כי הכתובים מבוארים בעצמם, בכל זאת כיון המגיד לבאר בהם שני ענינים יקרים: האחד שעם היות שאמרה תורה "ויהי בימים הרבים ההם וימת מלך מצרים ויאנחו בני ישראל מן העבודה ויצעקו", שלא יחשוב אדם שהיתה צעקתם כצועק חמס או מתכעס וקורא תגר, אלא היתה תשובה גמורה ובקשה אל ה' אלהי אבותיכם שבזכותם יחמול עליהם ויושיעם.ואפשר כי במות פרעה יצאו כל העם המצריים ובני ישראל לעשות עליו הספד ויללה וצעקה ברחוב העיר כמו שהוא המנהג במות המלכים, והוא שאמר הכתוב "וימת מלך מצרים ויאנחו בני ישראל ויצעקו", כי בא לעשות צעקה והספד על המלך, אך בלבם ובדעתם היו מכוונים בצעקתם על צרתם ועל העבודה הקשה, לא על מיתת המלך. וזהו שנאמר "ויאנחו בני ישראל מן העבודה ותעל שועתם אל האלהים מן העבודה", כי היודע תעלומות לב הבין מחשבותם וכוונתם באותה צעקה, אולם יען שלא אמרה התורה שצעקו אל האלהים רק סתם "ויאנחו בני ישראל מן העבודה ויצעקו", לכן בא המגיד לבאר שמה שנאמר שם אודות האנחה והצעקה לא היה על מות פרעה כי אם שצעקו אל ה' אלהיהם על עבודתם, וכמו שאמר כאן "ונצעק אל ה' אלהי אבותינו", שגילה הכתוב בוידוי הבכורים מה שקצר שם בסיפור המאורע. והביאור השני שעשה הוא בפסוק "וישמע אלהים את קולנו", שלא לבד היתה סיבת גאולתם בעבור ששמע ה' את קולם וצעקתם כי אם שנחברה לזו סיבה שניה, שזכר את הברית שכרת את האבות, ונתחברו לפי זה בגאולה תשובת העם וצעקתם אל ה' וברית האבות, ולזה הביא לבאר "וישמע ה' את קולנו" מה שנאמר שם בספור המאורע "וישמע ה' את נאקתם ויזכור אלהים את בריתו את אברהם ואת יצחק ואת יעקב", כי תשובתם מבלי הברית לא היתה מספקת לגאולתם. והנה אף כי באמת נכרת הברית לאברהם לבדו כמו שנאמר "ביום ההוא כרת ה' את אברם ברית" (בראשית ט"ו, י"ח), אך לפי שאותו ברית זכרו גם כן ליצחק וליעקב וקיימו בידם לכן תלה הכתוב ענין הברית בשלשתם. ואולי ידמה לעגלה משולשת וגו' שנזכרו ברית בין הבתרים, שהיה משל לשלשת האבות כמו שפירשתי למעלה, לכן אמר כאן את בריתי אברהם, את יצחק ואת יעקב, לפי שהברית ההוא כנגד כולם נכרת ושלשתם דמו בו. הנה נתבאר בזה הספק אשר בשער ששים.}%endcomment
\hebeng{וַיִּשְׁמַע ה׳ אֶת קלֵנוּ. כְּמָה שֶּׁנֶּאֱמַר: וַיִּשְׁמַע אֱלֹהִים אֶת־נַאֲקָתָם, וַיִּזְכֹּר אֱלֹהִים אֶת־בְּרִיתוֹ אֶת־אַבְרָהָם, אֶת־יִצְחָק וְאֶת־יַעֲקֹב. }{"And the Lord heard our voice" - as it is stated (Exodus 2:24); "And God heard their groans and God remembered His covenant with Avraham and with Yitschak and with Ya'akov."}
\hebeng{וַיַּרְא אֶת־עָנְיֵנוּ. זוֹ פְּרִישׁוּת דֶּרֶךְ אֶרֶץ, כְּמָה שֶּׁנֶּאֱמַר: וַיַּרְא אֱלֹהִים אֶת בְּנֵי־יִשְׂרָאֵל וַיֵּדַע אֱלֹהִים. }{"And He saw our affliction" - this {[refers to]} the separation from the way of the world, as it is stated (Exodus 2:25); "And God saw the Children of Israel and God knew."}%
\commenta{\textrm{\textbf{וירא את ענינו זו פרישות דרך ארץ, כמו שנאמר וירא אלהים את בני ישראל וידע אלהים}} לא נתבאר מה יתרון להפסוק וירא אלהים על הפסוק וירא את ענינו עד שמביא ממנו ראי׳ לבאור הפסוק הקודם, והלא לכאורה אדרבה, הלשון ענינו לבדו מורה על זה, כמו שאמרו ביומא (ע״ז ב׳) על הפסוק בסוף פרשה ויצא, אם תענה את בנותי, שמכוין לענין חיי דרך ארץ, ועוד שם לשונות אלה ממובן זה. אך הבאור הוא, כי שרש שם ענינו (ענה) אפשר להבין גם ממובן דכאות ומצוקה, כמו אלמנה ויתום לא תענון (פ׳ משפטים), עניתי בצום נפשי (תהילים ל״ה:י״ג), ענה בדרך כחי (שם ק״ב), ועוד. וברצות המגיד לפרש כאן המובן מן וירא את ענינו לפרישות דרך ארץ — מביא לראי׳ מן הלשון וירא אלהים את בני ישראל וידע אלהים. וכונת הראיה מזה הפסוק הוא, משום דבכל מקום דכתיב הפעל ראי׳ — סמוך לו שם הדבר הנראה, מה ראה, כמו וירא אלהים את האור כי טוב, (פ׳ בראשית) והרבה כהנה, וכאן כתיב וירא אלהים את בני ישראל ולא סיים מה ראה בבני ישראל. ולכן מפרש, דהראי׳ מוסבת על פרישות דרך ארץ, ואין מן הנמוס לייחש להקב״ה ענין גשמי זה, ולכן כתיב רק וירא ולא סיים מה ראה. ויתפרש לפי זה הלשון וירא ממובן דעת ובינה, כמו בקהלת (א׳ ט״ז) ולבי ראה הרבה חכמה, שבאורו ולבי ידע והבין, וכאן הבין כביכול הקב״ה את תכונת דאגתם לסבת פרישות דרך ארץ, וגם לא כתיב וישמע, מפני כי מתכונת הכבוד לא הוציאו ענין זה מפורש בפיהם בזעקתם ותפלתם. וכעין מידה זו שלא לייחש ענין גשמי הבלתי מכובד כלפי ה׳ מצינו במס׳ מו״ק (כ״ח רע״א) בסמיכות למה שמבואר שם, שכל מי שנאמר בו שמת על פי ה׳ היתה מיתתו בנשיקה, — ואמרו, דגם מרים מתה בנשיקה ומפני מה לא נאמר בה עפ״י ה׳, מפני שאין זה מדרך הכבוד. ובמס׳ ב״ב (ע״ד ב׳) לענין דאיירי שם ביחש פרט אחד שאינו מנומס בהסבה להקב״ה, אמרו ״לאו אורח ארעא״, עיי״ש, ועוד יתבאר מזה בסמוך. וגם מצינו שהקפיד הקב״ה על שמיחסים לו מעשה שאינו מכובד, כמו שאמרו במס׳ ברכות (ס״ב ב׳) על מה שאמר דוד לשאול אם ה׳ הסיתך בי (ש״א כ״ו י״ט) הקפיד הקב״ה על לשון קשה זה, ואמר לו, מסית אתה קורא לי? ונענש על זה, עיי״ש *ובירושלמי סוטה (פרק ז׳ הלכה א׳) אמרו, לא מצינו שידבר הקב״ה עם אשה אלא עם שרה בלבד, שנאמר ויאמר, לא כי צדקת (פ׳ וירא, י״ח ט״ו), ומפרש שם, כי אעפ״י דכתיב אל האשה אמר (פ׳ בראשית׳ ג׳ ט״ז), וברבקה — ויאמר ה׳ לה (פ׳ תולדות, כ״ה כ״ג) זה הי׳ ע״י מתורגמן..והנה מכל האמור בזה מתבאר, דהלשון וירא את ענינו מוסב על פרישות דרך ארץ, כמו וירא אלהים את בני ישראל, וכפי שנתבאר. ורבי אברהם אבן עזרא בריש פרשה שמות (א׳ י״א) מפרש הלשון שם, וכאשר יענו אותו כן ירבה — יענו אותו, כדי ליבש זרע הזכרים, וכפי הנראה ראה לפרש כן, משום דענין רבוי תולדה אינו מתיחש לענין עבודת מלאכה, לאמר, שלרגלי הענוי רבה התולדה — אך מפני שכיונו בזה ליגעם להחליש זריעתם, מפרש הפסוק, כי אעפ״י כן רבתה תולדתם, ומפרש המובן מן יענו ענין דרך ארץ, כפי שנתבאר. ואמנם גם לבד הדרש בהוראת מלת ענינו עפ״י הסמיכות מן וירא אלהים, כפי שבארנו — גם לבד זה אי אפשר לפרש כפשוטו שהמובן מן ענינו הוא לחץ ודחק — יען דזה הלא כבר זכר המגיד מפסוק ותעל שועתם מן העבודה אל האלהים וישמע אלהים את נאקתם, ועוד, ומה הוסיף בלשון וירא את ענינו. ולכן הוציאו לדרשה, כמבואר. הנה הבטחתי למעלה לשוב עוד אל הענין המדובר שם על דבר הרחקה מלשון בלתי מנומס כלפי ה', ואעיר, כי עפ״י זה יתבארו דברי המכילתא בפרשה בא על הפסוק וראיתי את הדם ופסחתי עליכם (י״ב י״ג) ״אל תקרא ופסחתי אלא ופסעתי״ (בהחלף חי״ת לעי״ן). והנה אעפ״י דהחי״ת עם העי״ן רגילין להתחלף, מפני שהם ממוצא אחד (אהח"ע מן הגרון), וידוע, שאותיות ממוצא אחד דרכן להתחלף זו בזו, וכן מצינו חילופיהן בפרשה תצוה, בלשון ולא יזח החושן מעל האפוד, שמתפרש כמו ולא יזע (היינו שיתדבק החושן אל האפוד בחזקה שלא ינוע, מלשון ולא קם ולא זע (אסתר ה׳ ט׳), ובקהלת (י״ב ג׳) ביום שיזועו שומרי הבית — ובמלכים א׳ (ה׳ ל״א) ויסיעו אבנים גדולות, ובקהלת (י׳ ט׳) מסיע אבנים — שענינם עקירה והעברה ממקומם, היינו הך דלא יזח החושן שהבאנו, ומזה הלשון בפרשה בשלח (ט״ו כ״ב) ויסע משה את ישראל, שבאורו כמו ויסח, שהעתיקם ממקומם *וקרוב לומר, כי מזה המקור למה שקוראים ליציאה לדרך בשם ״נסיעה״ שהוראתו בעיקר העתק ממקומו, ופעם באו שני הפעלים, נסיעה ויציאה, יחד, בירמיה (ד׳ ז׳) נסע יצא ממקומו, כלומר, נעתק ממקומו ויצא לדרכו.), וכן יתפרש ביואל (ד׳ י״א) עושו ובאו — תחת חושו, ורמב״ן בפרשה דברים (ב׳ כ״ג) כותב, כי הלשון והעוים (שם אומה) הוא כמו והחוים, והביא משלים לזה.  הנה אחר כל אלה, לא נתבאר מה ראתה המכילתא לפרש ופסחתי כמו ופסעתי, אם לא כמו שבארנו, ולשון שאינו מכובד אין נכון לייחס להקב״ה ואף כי הוא המבטא אותו, והלשון פסחתי הוראתו בעלמא חיגר, בעל מום, כמו בפ׳ אמור איש עור או פסח, או ממובן תועה בדרכיו, כמו פוסחים על שתי הסעיפים (מ״א י״ח כ״א), ולכן דרשה כמו ופסעתי, ממובן צעד ופסיעה. אבל לפי זה קשים הלשונות ופסח ה׳, ואמרתם זבח פסח הוא לה׳ אשר פסח על בתי ישראל, האם גם שם תשתנה הקריאה. ואמנם אפשר לומר דאין כונת המכילתא לשנות הקריאה, דזה לא יתכן מכמה טעמים, אך הכונה בלשון אל תקרא כמו אל תבין, כלומר, שבלשון ופסחתי תבין ההוראה כמו ופסעתי, ואם כן גם ביתר המקומות כן המובן. ומצינו הפעל ראה במובן הבנה, במו בקהלת (א׳ ט״ז) ולבי ראה הרבה חכמה, שבאורו ולבי הבין, ובשמואל א׳ (כ״ה י״ז) דעי וראי מה תעשי וכלשון זה במלאכים א׳ (ב׳ כ״ב), ובירמיה (ב׳ ל״א) ראו דבר ה׳, וביאורם הבין, הבינו. וכעין זה כתב הגר״י יעבץ בהגהותיו למס׳ ערכין (ט״ו א׳) בלשון הגמרא על הפסוק בפ׳ שלח כי חזק הוא ממנו, שאמרו בגמרא אל תקרא ממנו (ביחש שאליו) אלא ממנו (ביחס נסתר), וכתב, שאין הכונה לשנות הקריאה כי אם להבין כמו בשינוי קריאה, ויש עוד כאלה. וכאן אדבר משפטים עם הראב״ע על דבריו בקהלת (י״ב ה׳) על הפסוק ויסתבל החגב אשר כל תעודת הפסוק לתאר את רפיון חושי האדם בערבי חייו, כותב בזה הלשון ״ואחרים אומרים, כי החי״ת מן החגב בא במקום עי״ן, וזה הבל, כי לא תתחלף אות באות חוץ מאותיות אהו״י״. והנה כונת החילוף מן חגב לעגב הוא למשל על תאות בשרים, מעין הלשונות מאסו בך עוגבים (ירמיה ד׳ ל׳), ותעגב על מאהביה (יחזקאל כ״ג ח׳) שירי עגבים (שם שם ל״ב), ומשותף לשון זה ויסתבל החגב (העגב) עם הלשון הקודם ותפר האביונה, כלומר, כי יתופר תאות בשרים, ובאור הלשון ויסתבל — שבשר התאוה יהי׳ עליו (על הזקן) למשא. ואין קץ לפלא, איך קרא לחילוף חי״ת עם עין הבל, בעוד שהוא חזון נפרץ במקרא ובחז״ל, מפני שבאים ממוצא אחד, אהח"ע מן הגרון, וכבר הבאנו משלים לזה, וכן כאן במכילתא, ופסחתי תחת ופסעתי ובסנהדרין (ק״ג א׳) פירשו ויעתר יצחק כמו ויחתר, וע׳ במ״ר ר״פ תולדות, ובמ״ר איכה פירשו הלשון איכה יעיב כמו איכה יחיב, וכן פירש״י כאן בשם חגב כמו עגב. וכן נראה באור הלשונות בתורה, בפ׳ תבא (כ״ח ס״ג) ונסחתם מעל האדמה, ובתהלים (נ״ב ז׳) ויסחך מאהל, ובמשלי (ב׳ כ״ב) ובוגדים יסחו ממנו, ושם (ט״ו כ״ה) בית נאים יסח ה׳ — בכולם באים החי״תין תחת העי״נין, ונסעתם, ויסעו, יסעו, יסע, ממובן העתק ממקום, כמש״כ למעלה בבאור הפעל נסע. וכן נראה באור השם שרוע (איש חרום או שרוע (בפ׳ אמור, כ״א י״ח), ושם  (כ״ב כ״ג) ושור ושה שרוע וקלוט, ופירש״י, שרגלו אחת ארוכה מחבירתה (וכן יתר האברים), ולא נודע שרש שם שרוע, ונראה שבא תחת שם סרוח, כמו וסרח העודף (פ׳ תרומה) ובחילוף סמ״ך בשי״ן שמאלי. וכן שוים החי״ת והעי״ן בהברה בשיחות פרטיות, כמו שהי׳ קורא רבי לרבי חייא — עייא (מו״ק ט״ז רע״ב), ובמ״ר פ׳ תולדות בערביא קורין לחתרתא עתרתא, ועוד. }%endcomment%
\commentb{\textrm{\textbf{תשובות לשערים ס"א – ס"ה}}\textrm{\textbf{וירא את עניינו זו פרישות דרך ארץ וכו'.}}בתחילת המחשבה יראה מהמשפט שידרוש המגיד עניינו עמלנו ולחצנו על עניין אחר כשמות נרדפים שהם מתחלפים בעצמם ומורים על דרך אחד, וגם שידרוש "וירא את עניינו" כמו שדש "ויענונו" וידרוש "עמלנו ואת לחצנו" מענין עבודה קשה. אמנם המגיד ראה בטיב שכלו שאם היה עניינו עמלנו ולחצנו מאותו המין שדרש עליו "וירעו אותנו המצריים ויענונו יתנו עלינו עבודה קשה", אם כן יהיה אחד מן הכתובים לבטלה, והיה ראוי שיאמר וירעו אותנו המצריים ויענונו ויתנו עלינו עבודה קשה ונצעק אל ה' וגו' וישמע ה' וגו' וירא את העשוי לנו במצרים ויוציאנו וגו', ולמה לו לכפול הדברים בעצמם עניינו עמלנו ולחצנו, לכן בא המגיד לדרוש כל אחד מהכתובים בפני עצמו, כי בפסוק הראשון הזכיר מה שעשו להם המצריים בפרהסיא והוא הענוי בחומר ובלבנים בבנין ערי המלך, והעבודה הקשה היא העבודה שהיו המצריים משעבדים בה את בני ישראל, ואמר אחר כך "וירא את עניינו ואת עמלנו ואת לחצנו" להגיד שמלבד אותם הצרות והרעות המפורסמות שעשו להם המצריים בנגלה, היו עושים המצריים דברים רעים נסתרים אשר ראה אלהים היודע מחשבות ובוחן לבות, עם היותם בלתי מפורסמות, ולכן דרש "וירא את עניינו" על פרישות דרך ארץ, שהיו רבים פורשים מנשיהם כדי שלא יגעו לריק ולא ילדו לבהלה, וכבר מצאו חז"ל שפרישות מתשמיש המטה נקרא ענוי כמו שנאמר "אם תענה את בנותי" (בראשית ל"א, נ'), ועל זה נאמר "וירא אלהים את בני ישראל וידע אלהים" (שמות ב, כ"ה), כי ראה פרישותם מנשיהם ונתן להם נחת רוח ורצון להזדקק להם כדי שיתרבו וכמו שדרשו על "תחת התפוח עוררתיך" (שיר השירים ח', ה'), ששם בשדה היו נזקקים להן ויהיה "\textrm{\textbf{וידע"}} מלשון "והאדם ידע את חוה אשתו" (בראשית ד', א') או "ואיש לא ידעה" (בראשית כ"ד, ט"ז), ועשו בדרך אסמכתא המלה יוצאת מהקל, ועשו גזרה שוה מן וירא אל וירא, נאמר כאן "וירא את עניינו" ונאמר להלן "וירא אלהים את בני ישראל וידע אלהים", מה להלן שנתן להם חפץ לדעת נשיהן אף כאן "וירא את עניינו" כך ענינו. ושם דרשו "וידע אלהים" על תשמיש המטה כי אין אדם יודע מה שבינו  לבינה רק הקדוש ברוך הוא יודע נסתרות.וכן דרש המגיד ואת עמלנו אלו הבנים שהוא גם כן צרה נסתרת שגזר פרעה וצוה למילדות כי "כל בן הילוד היאורה תשליכוהו". וכתב הרמב"ן שצוה זה בסתר למילדות שיעשו הדבר הזה מעצמם כי חרפת היא לו לעשות על זה דת מפורסמת, אך בוחן לב וכליות ראה הדבר הרע הזה שנעשה להם אף על פי שלא היה מפורסם , וכאילו אמר וירא אלהים מה נעשה מבנינו עשר עמלו בתולדותם וגידולם שהיו משליכין אותם ליאור. ואפשר גם כן לפרש ואת עמלנו אלו הבנים שהיו מגדילים אותם בעמל רב ובסתר המדרגה שלא ידע אדם מהם, ולפי שהיו עושים זה בסתר אמר בלשון "וירא", ומה שהביא על זה הפסוק "כל הבן הילוד" היה לבאר עמל ושמירת הבנים והסתרם שהיה מפני גזרת המלך וגזר "כל הבן הילוד", אך הפירוש הראשון הוא יותר נכון.ודרש המגיד ואת לחצנו זה הדחק, ורצה בזה כי מלת לחץ תאמר פעם על הכאה גופנית כמו "ותלחץ את רגל בלעם" (במדבר כ"ב, כ"ה), שענינו למעכו, ופעם תאמר על ההכרח והאונס שהנוגש עושה לעובדים לפניו כדי שיעשו איזה דבר בלחץ גדול, וזה השימוש מהלחץ נקרא בדבריהם ז"ל "דוחק". ולכן אמר המגיד שמה שנאמר בכתוב הזה ואת לחצנו אין ראוי שיפורש על המעיכה והכאה גופנית כי זה כבר נכלל במה שנאמר "ויתנו עלינו עבודה קשה",  אלא נאמר לחצנו על הדחק, שלא די שהיו מעבידים אותם לא היו נותנים להם מוחה בעבודה ההיא כי היו דוחקים אותם כמו שנאמר "והנוגשים אצים" (שמות ה', י"ג), וגם זה לא היה דבר מסודר ולא מפורסם ולכן הזכירו בפני עצמו. ואולי היה הלחץ והדוחק שהיו עושים המצריים לישראל כדי שיעבדו את אלוהיהם, לכן הביא לבאר את הפסוק "וגם ראיתי את הלחץ אשר מצרים לוחצים אותם ואזכור את בריתי" (שמות ג', ט'), רצה לומר ברית האלהות וקיומו שהיה בלבם ולא זזה מהם עם כל הלחץ. ויש מפרשים הדחק תוכן הלבנים שהיו נותנו דבר יום ביומו, והנכון אצלי ששלושת אלה שנזכרו בפסוק "וירא את עניינו ואת עמלנו ואת לחצנו" (דברים כ"ו, ז'), הם צרות נפשיות, כי אחרי שדרש הפסוק הראשון "ויענונו ויתנו עלינו עבודה קשה" על הענוי הגופני והעבודה הגשמית, אמר הכתוב שהשם יתברך שמע נאקתם וצעקתם מן העבודה הגופנית ההיא, ומלבד זה ראה תוגת לבבם בפרישות דרך ארץ ובעמל שמירת הבנים והסתרם, ובדוחק חייהם ולחיצת לבבם מפני חמת המציק, כי התוגות והצרות הנפשיות הם יותר קשות מהגופניות, ולכן שמע השם יתברך נאקתם הגופניות וראה והתבונן על צרת לבבם הנפשיות. ובזה הדרך לא יפול הכפל והמותר במאמר הזה כי כולו בא לצורך. והותרו בזה הספקות אשר בשערים ס', ס"א, ס"ב, ס"ג, ס"ד וס"ה.}%endcomment
\hebeng{וְאֶת־עֲמָלֵנוּ. אֵלּוּ הַבָּנִים. כְּמָה שֶּׁנֶּאֱמַר: כָּל־הַבֵּן הַיִּלּוֹד הַיְאֹרָה תַּשְׁלִיכֻהוּ וְכָל־הַבַּת תְּחַיּוּן.}{"And our toil" - this {[refers to the killing of the]} sons, as it is stated (Exodus 1:22); "Every boy that is born, throw him into the Nile and every girl you shall keep alive." }%
\commenta{\textrm{\textbf{ואת עמלנו אלו הבנים, כמה שנאמר כל הבן הילוד היאורה תשליכוהו וכל הבת תחיון..}} הנה גם אחרי הראי׳ מפסוק כל הבן הילוד וכו׳ אין מבואר איפה מרומז ענין זה בלשון עמלנו. ואפשר לומר, משום דמצינו לשם ״עמל״ שמורה על יגיעה לריק ולהבל, כמו באיוב (ד׳ ח׳) וזורעי עמל יקצרוהו, שהבאור שהקצירה מזריעה זו תהי׳ ריקה ושוממה, ושם (ט״ו ל״ה) הרה עמל וילד און, כלומר, לא כלום, ועוד שם (ט״ז ב׳) מנחמי עמל כולכם, כלומר, מנחמים לריק (כענין הבל ינחמון, זכריה (י׳ ב׳) ובתלמוד (כתובות י׳ ב׳) תנחומין של הבל), ועוד בקהלת (ד׳ ו׳) עמל ורעות רוח. ואמר בזה, ואת עמלנו אלו הבנים שילדנו במצרים, והוא על דרך הלשון ולא ילדו לבהלה (ישעיה ס״ה כ״ג), ומפרש סבת הדבר — היא הגזירה כל הבן הילוד היאורה תשליכוהו. וא״כ כל התולדה עמל וריק. ויש להעיר למה מביא עוד סוף הפסוק וכל הבת תחיון, דהא לזה אין ענין לגזירת הבנים. וצריך לומר, דגם תחית הבנות בעת ההיא לרגלי גזירת הבנים, היא ג״כ גזירה, יען כי אם לא יהיו גברים שתנשאנה להם הנשים תצאנה מזה קלקלות שונות, להיות דבר זה משבר בחיי העולם. ודע דמלשון זה אלו הבנים סמך ראי׳ דבשם בנים יובן רק זכרים ולא נקבות, משום דאם לא כן, הי׳ לו לומר אלו בנים זכרים, וכן משמע במס׳ נזיר (י״ב א׳) במשנה, האומר הריני נזיר לכשתלד אשתי בן וילדה בת אינו נזיר, וכן בב״ב (קמ״ג ב׳) ההוא דאמר נכסי לבני (בלשון רבים) והי׳ לו בן ובת, ופסקו חכמים דאין הבת בכלל, ומה שקרה לבנו האחד בשם רבים בני, זה הוא מפני שכן דרך הלשון, כמו בתורה בפ׳ ויגש, ובני דן חושים, ובפ׳ פנחס ובני פלוא אליאב, אעפ״י שבשניהם חשיב רק בן אחד, והי׳ לו לומר ובן דן, ובן פלוא, ולא ובני — אך כן הוא מדרך הלשון.  ויש להעיר בגמרא שם, דרב יוסף מביא עוד פסוק כזה מדהי״א (ב׳) ובני איתן עוזיה, ולא נתבאר מה ראה רב יוסף להוסיף על פסוקי התורה המובאים מקודם עוד פסוק אחד מדה״י. ואפשר לומר, דמדייק שאין ראי׳ מלשון התורה ללשון בני אדם, כמו שאומרים בעלמא לשון תורה לחוד ולשון בני אדם לחוד, ואם כן אפשר שבלשון בני אדם אין קורין לבן אחד בני, בלשון רבים. ועל זה מביא מדה״י. וזה עפ״י מש״כ התוס׳ בנדרים (נ״ה א׳) ד״ה ובפרוץ, דלשון דברי הימים קרוב ללשון בני אדם, וא״כ מהפסוק דרה״י ראי׳ מכררחת דגם בלשון בני אדם קורין לבן בני. ועל מה שכתבנו דבשם בנים יובן רק זכרים ולא בנות — לכאורה יש להעיר על זה מפסוק בעצב תלדי בנים (פ׳ ראשית). ושם ודאי גם בנות בכלל. אך באמת משם אין הכרח דבסתם שם בנים גם בנות בכלל, ושם הבנות בכלל מטעם אחר, והוא עפ״י מה שאמרו ביבמות (ע״ח א׳) על הפסוק בפ׳ תצא בענין מצרי ואדומי דכתיב שם בנים אשר יולדו להם דור שלישי יבוא להם בקהל ה׳, ופירשו, בנים ואפילו בנות, ומפרש מאי טעמא, דכתיב אשר יולדו — בלידה תלי רחמנא, כלומר, כיון דבלידה תלאן, הלא גם בנות נולדות הן. ולפי זה, באותו פ׳ דפרשה בראשית שהבאנו כתוב בעצב תלדי, ממילא גם בנות בכלל, דגם הן נולדות. ומה שמצינו בתורה כמה פעמים במקרא הלשון ויולד בנים ובנות, אעפ״י דאחרי דכתיב זה סמוך ללשון הולדה הי׳ די לכתוב בנים לבד והיו גם בנות בכלל — זה הוא, מפני שבספור דברים רגילים הכתובים לפרט במלים גם דברים שאפשר להסתפק בלעדם, כמו שמצינו כזה בכמה מקראות, כמו: בפרשה בראשית (ב׳ י״ז) ומעץ הדעת טוב ורע לא תאכל ממנו, ואחרי שאמר ומעץ הדעת הי׳ סיפוק בלשון ובענין בלא מילת ״ממנו״. בפרשה וירא (כ״ב ב׳) את בנך את יחידך אשר אהבת את יצחק ציונים רבים. בפרשה ויצא (כ״ט י״ח) אעבדך ברחל בתך הקטנה, וציון הקטנה אינו מוכרח אחרי שקראה בשמה. בפרשה וישלח (ל״ב י״ב) הצילני נא מיד אחי מיד עשו, ודי הי׳ לומר מיד אחי, כי לא היו לו עוד אחים זולת עשו. בפרשה מקץ (מ״א (מ״א ל״ב) ועל השנות החלום אל פרעה פעמים, היינו השנות היינו פעמים. בפרשה שמות (ב׳ ו׳) ותפתח ותראהו את הילד, ודי הי׳ לומר ותראהו לבד, דמוסב על הילד. בפרשה תרומה (כ״ה ב׳) ויקחו לי תרומה מאת כל איש אשר ידבנו לבו תקחו את תרומתי, והי׳ הענין מובן וברור גם בלא המלים תקחו את תרומתי. בפרשה ויקהל (ל״ה ה׳) קחו מאתכם תרומה לה׳ כל נדיב לב יביאיה את תרומת ה׳, והי׳ הענין מובן ישר גם בלא המלים את תרומת ה׳. בפרשה האזינו (ל״ב מ״ט ונ״ב) וראה את ארץ כנען אשר אני נותן להם לבני ישראל, כי כנגד תראה את הארץ ושמה לא תבוא ״אל הארץ אשר אני נותן לבני ישראל״, והכפל והאריכות נראים לעין. ביהושע (א׳ ב׳) קום עבור את הירדן הזה אתה וכל העם הזה אל הארץ אשר אני נותן להם לבני ישראל, ואחר הלשון ״להם״ המוסב אל ״כל העם הזה״ נראה הלשון לבני ישראל כפול ואינו מוכרח. בשמואל ב׳ (ז׳ ה׳ וז׳) אל עבדי אל דוד — את עמי את ישראל, שמות כפולים, ושם (ז׳ כ״ג) ומי כעמך כישראל. ועוד הרבה כאלה, וכן כמה מלים בודדות מהוראה אחת באו כפולות, כמו בישעיה (ה׳ י״ג) צחה צמא, ושם (י׳ כ״ה) מעט מזער, ושם (מ״א ד׳) פעל ועשה, ועוד שם (ס״ג ט׳) וינטלם וינשאם, וביואל (ד׳ ד׳) קל מהרה, וביונה (ב׳ ט׳) הבלי שוא, ובתהלים (ט״ז ה׳) מנת חלקי, ובאיוב (י׳ כ׳) יחדל ישית ממני, ושם (כ׳ י״ז) נהרי נחלי. ועוד הרבה כאלה. ויותר מזה, לא רק מלים בודדות ומבטאים משותפים רגילים לבא כפולים במהלך ספורים, אך גם פרשיות שלמות דרכן להשנות כמו שהן, כנראה מספור עבד אברהם לפני לבן ובתואל מכל הקורות אותו בענין זה, אעפ״י שהי׳ באפשר להציע הדברים במלים אחדות. לאמר, ויספר העבד את כל אשר עבר עליו בזה (וע׳ במ״ר כאן). וכן מספור פרעה לפני יוסף את חלומו הי׳ באפשר שיאמר הפסוק ויספר פרעה לפני יוסף את חלומו, והנה בשני אלה הענינים, מעבד אברהם ומחלום פרעה לא חסה התורה לקבוע פרשיות שלמות בהעתקה שלמה. ועוד יש כמה ספורים משנים בתורה, במקום שהי׳ באפשר לשנותם בקצרה, והנה כן דרכי הספורים, ואולי כן גם מסגנון השפה במכוון ליפותה, לאדרה ולהדרה. ואמנם בנוגע להשנות ספור עבד אברהם לפני לבן ובתואל וחלום פרעה לפני יוסף — אפשר וגם קרוב לומר, כי בכונה מיוחדת באה תכונה זו מההרצאות  השלמות בפרטיהן, וזה הוא, מפני השנוים השונים שבין ספור העבד ובין מהלך עניניו בפועל, וכן בין ספור פרעה את החלום לפני יוסף ובין חזיונו את החלום במקורו. והשנוים האלה לא על חנם באו, כי אלה שבספור העבד נותנים כבוד לרבקה על כי למרות גידולה בבית לבן ובתואל למדה לעצמה צניעות במהלך החיים, בדבור ובמעשה, כפי שיתבאר. והשנוים אשר בספור החלום מפרעה ליוסף ובין חזיונו במקורו — יורו לעין השגחה פרטית לחלצו ליוסף מן המיצר ולהביאו לגדולת מלכים. ואבארם על הסדר: הנה בספר העבד לפני לבן ובתואל את הסימן שעשה לפגישת רבקה אמר, שאמרה שתה וגם גמליך אשקה וגם הגמלים השקתה (פ׳ חיי, כ״ד מ״ו), ובאמת לא אמרה וגם לגמליך אשקה, אך אשאב, וכן במעשה כתוב ותשאב לכל גמליו, ולא ותשקה, וזה הוא, מפני כי כפי המתבאר בכתובות (ס״א ב׳) אין ממדת הצניעות לאשה להשקות לבעלי חיים זכרים, ולכן אמרה רק אשאב, והם ישקו מעצמם (ואף הוא לאחר ששמע לשונה אמר גם הוא משמה וגם לגמליך אשאב (פסוק מ״ד). וכן שינתה ללשון כבוד מלשון שאמר הוא הגמיאיני מעט מים (פסוק י״ז), דלשון גמיאה הוא לשון גס, וקרוב ללשון לעיטה. כמש״כ רש״י בריש פ׳ תולדות בפ׳ הלעיטני, וזה מורה על שפיכה לפיו, והיא העמידה הכד ואמרה שתה — מעצמך. וכן שינה העבד בספורו מלשון שאמר לו אברהם ולקחת אשה לבני ליצחק (פסוק ד׳), והוא בספרו זה לפני לבן ובתואל אמר רק הלשון ״ולקחת אשה לבני״ והשמיט ״ליצחק״, וזה אפשר להסביר כי אברהם באמרו ״ליצחק״ כיון שתהא האשה צנועה וצדקת כפי הראוי ליצחק, אך העבד, בהכרתו בחיי לבן ובתואל שרחוקים הם מתומת צדקתם של אברהם ויצחק חשש שמא לא יאבו שתהי׳ רבקה מוקדשת לחיי פרישות באורח צדיקים, ולא יתנו אותה ללכת אתו — לכן השמיט הלשון ליצחק. ויותר מזה ניכר הכרח השנות חלום פרעה לפגי יוסף בכל פרטיו, כפי שנבאר. ולזה דרוש הקדמה קצרה להבין את הלשון הוודאי שאמר פרעה ליוסף אחרי ששמע פתרונו אמר לו ״אחרי הודיע אלהים אותך את כל זה״, והתמי׳ בולטת, כי מלשון זה נראה, שהי׳ דבר הפתרון אצלו כמו נראה ונרגש בחוש כי הוא כיון לפתרונו הנאמן בלא כל צל ספק, והשאלה בולטת מאין הי׳ בטוח בזה, עד שהכתיר אותו בלשון רם ומופלא מאוד כי ״ה׳ הודיעו כל זה״ והן תוצאת הפתרון התחילה רק לאחר זמן ארוך, ומאין ידע מיד כשמעו הפתרון באמונת הבטחון. אך הדבר יתגלה ויתאמת עפ״י זה, כי במהלך עניני החלום בשעתו עם  סגנון הרצאתו לפני יוסף נראים ונרגשים שנוים שונים בלשון ובנין. ונבארם על הסדר: כי בחלומו ראה בהמות ״יפות מראה״ (פ׳ מקץ, מ״א ד׳), ובהרצאתו אמר ״יפות תואר״, ובפרשה ויצא בפסוק ורחל היתה יפת תואר ויפת מראה (כ״ט י״ז) כתב רש״י, דיפת תואר ויפת מראה הם שני ענינים, דיפת תואר מורה על צורת הפרצוף בחתוך האברים, ויפת מראה מורה על זיו הקלסתר. ושוב ראה בחלום השני שבלים ״בריאות וטובות״ (פסוק ה׳), ובהרצאתו אמר, שראה שבלים מלאות וטובות (פסוק כ״ב), ויש הבדל בזה, כי בריאות אפשר להיות אף כי לא מלאות, כפי שנראה בכמה ברואים, (וגם באנשים) שיש בריאים וחזקים וגופן איננו ממולא. ועוד ראה בחלומו שבלים שדופות קדים (פסוק ו׳), ובספרו ליוסף הגיד שראה שבלים צנומות דקות (פסוק כ״ג) ויש הבדל ביניהן, כי שדופות קדים ענינן שבלים נחרות מלהט רוח קדים, וצנומות דקות ענינן קשות ויבשות. וקרוב לודאי ששכח פרטי המראות כמו שהן בדיוק, ועמד רק על כללי המראות. והנה איתא במדרש רבי תנחומא (השלם), כי כאשר סיפר פרעה ליוסף את החלומות לפרטיהם העיר אותו יוסף על פרטי המראות ואמר לו, לא בהמות יפות תואר ראית כי אם יפות מראה, ובהזכירו שראה שבלים מלאות וטובות, אמר לו, לא כך ראית כי אם בריאות וטובות ובספרו שראה שבלים צנומות דקות, העיר אותו יוסף והזכירו, כי לא שבלים צנומות דקות ראה, כי אם שדופות קדים. וכאשר הזכירו יוסף נזכר גם הוא כי אמנם ראה כמו שהעירו (כך מבואר במדרש תנחומא הנזכר), וזה אמנם מנוסה בטבע, וכמ״ש במס׳ נדה (כ״ד ב׳) דכי מדכרו לי׳ מדכר (כשמזכירין לאדם נזכר שאמנם כן הוא). ומתוך כל זה הי׳ פרעה בטוח כי אמנם מן השמים הודיעו לו את כל אלה וממילא גם פתרונו, יען כי מאין ידע כל פרטי השינוים. וזו היא כונת הלשון אחרי הודיע אלהים אותך את כל אלה. ואמנם יש להעיר ולהתבונן למה זה, כנגד האריכות בספורים, קמצה התורה בהלכות ודינים, עד שבכמה מקומות אפשר גם לכנותם רק רמזים, וכמו שלפעמים קרובות אנו למדים דינים והלכות מיתור או מחסרון מלה אחת או אות אחת, כנודע מזה לכל בעל תלמוד. ואולי מכוון סגנון זה למה שיעץ החכם ״תן לחכם ויחכם עוד״ (משלי ט׳:ט׳), והבאור הוא, תן לחכם דברי תורה וחכמה במדה מצומצמת כזו, עד שישאר לו מקום למצוא בהם דעה וחכמה משלו, מדעתו ומשכלו, כי אז תהי׳ התורה ופרי השגתו בה מדעתו חביבה עליו, כטבע כל דבר שמשיגים בידיעה, וכמ״ש באבות  דרבי נתן פ״ג נוח לו לאדם דבר אחד הבא לו ביגיעה ועמל ממאה דברים הבאים לו בריוח, כלומר, בקל, עכ״ד. ובמס׳ ביצה (ל״ח רע״ב) וכי יאכל הלה וחדי, ופרש״י וכי ישמח בדבר שלא עמל בו, ומתקיים זה בדברי הקב״ה עצמו, שאמר ליונה בתמי׳, אתה חסת. על הקקיון אשר לא עמלת בו ולא גדלתו וכו׳ (יונה ד׳ י׳) ומתבאר דמטבע האדם לחוס ולהוקיר ביותר דבר שטרח ועמל בו, — וזו היא כונת הגמרא אם יאמר לך אדם מצאתי (תורה וחכמה) ולא יגעתי אל תאמין (מגילה ו׳ ב׳), וזה הוא טעם אהבת הורים לבניהם מפני שמטפלים הרבה בגדולם. וזהו באור הלשון הרגיל בגמרא ״מילתא דאתיא מדרשא חביבא לי׳״, כלומר חביבא לי׳ יותר מדבר שמוצא מוכן לפניו, וזה הוא מפני שבדבר הבא מדרשא יגע את מוחו ורעיונו, כמבואר. ובזה ניחא מה שלפי מהלך הענין בגמרא הי׳ ראוי לומר מילתא דאתיא מדרשא חשיבא לי׳ (ע׳ יבמות ב׳ ב׳), אך תפס את היסוד הראשי, הוא החביבות, ומתוך החביבות באה גם החשיבות. ויש הרבה להאריך בענין זה, אבל אין זה המקום, ולא באתי אלא להעיר, וחכם לב יוסיף דעת. ואשוב אל המקום אשר עמדנו בו בתחלה בענין אם שם בנים כולל גם בנות, ואעיר רק את זה, כי החקירה בזה אפשר שתצטמצם רק בשם בנים, אבל בשם זרע בודאי כולל גם בנות. וטעם הדבר יש להסביר עפ״י מה שכתוב במדרש רבי תנחומא פרשת בראשית, למה נקרא שמו של זכר ״בן״ מפני שהוא בונה את העולם, ע״כ. והבאור הוא, כי לשם ״בנה״ שתי הוראות, האחת בן תולדי, והשנית מענין בנין, ודרך העולם שזכרים בונים את העולם והאשה יושבת בית וצופה הליכות ביתה, ולכן יש סברא, כי בשם ״בנים״ לא יהי׳ כלולה בת — אבל בשם זרע בודאי כלולות גם בנות, כי תבואת האדם כללית היא מה שמגיע מכחו. ועפ״י דברי המדרש תנחומא הנזכר בבאור השם בן אפשר לפרש היטב מה שאמרו בירושלמי פסחים פ״ב ה״א על הפסוק דריש פרשה תזריע, אשה כי תזריע וילדה זכר — אמרו, זכר — לרבות המת, כלומר, דגם אם ילדה ולד מת גם כן חייבת לנהוג בכל דיני הפרשה ההיא. והמפרשים לא ביארו מאומה איפה מרומז זה בלשון וילדה זכר. אבל יתבאר זה בפשיטות עפ״י המדרש הנ״ל דזכר נקרא גם בן, מפני שהוא  בונה את העולם, ופשוט הדבר, דרק חי בונה ולא מת. ולכן אם הי׳ כתוב וילדה בן הי׳ במשמע רק חי, זה שבונה את העולם, אבל מדכתיב וילדה זכר, שזה מורה רק על המין הנולד, וגם מת בכלל מין, ופשוט.}%endcomment
\hebeng{וְאֶת לַחָצֵנוּ. זֶו הַדְּחַק, כְּמָה שֶּׁנֶּאֱמַר: וְגַם־רָאִיתִי אֶת־הַלַּחַץ אֲשֶׁר מִצְרַיִם לֹחֲצִים אֹתָם. }{"And our duress" - this {[refers to]} the pressure, as it is stated (Exodus 3:9); "And I also saw the duress that the Egyptians are applying on them." }
\hebeng{וַיּוֹצִאֵנוּ ה׳ מִמִצְרַיִם בְּיָד חֲזָקָה, וּבִזְרֹעַ נְטוּיָה, וּבְמֹרָא גָּדֹל, וּבְאֹתוֹת וּבְמֹפְתִים. }{"And the Lord took us out of Egypt with a strong hand and with an outstretched forearm and with great awe and with signs and with wonders" (Deuteronomy 26:8).}
\hebeng{וַיּוֹצִאֵנוּ ה׳ מִמִּצְרַיִם. לֹא עַל־יְדֵי מַלְאָךְ, וְלֹא עַל־יְדֵי שָׂרָף, וְלֹא עַל־יְדֵי שָׁלִיחַ, אֶלָּא הַקָּדוֹשׁ בָּרוּךְ הוּא בִּכְבוֹדוֹ וּבְעַצְמוֹ. שֶׁנֶּאֱמַר: וְעָבַרְתִּי בְאֶרֶץ מִצְרַיִם בַּלַּיְלָה הַזֶּה, וְהִכֵּיתִי כָּל־בְּכוֹר בְּאֶרֶץ מִצְרַיִם מֵאָדָם וְעַד בְּהֵמָה, וּבְכָל אֱלֹהֵי מִצְרַיִם אֶעֱשֶׂה שְׁפָטִים. אֲנִי ה׳. }{"And the Lord took us out of Egypt" - not through an angel and not through a seraph and not through a messenger, but {[directly by]} the Holy One, blessed be He, Himself, as it is stated (Exodus 12:12); "And I will pass through the Land of Egypt on that night and I will smite every firstborn in the Land of Egypt, from men to animals; and with all the gods of Egypt, I will make judgments, I am the Lord."}%
\commentb{\textrm{\textbf{תשובות לשערים ס"ו – ע"א}}\textrm{\textbf{ויוציאנו ה' ממצרים לא על ידי מלאך ולא על ידי שרף וכו' אני ולא שליח אני ה' אני הוא ולא אחר.}}כבר כתבתי בשערים שיפלו במאמר הזה שש ספקות:א' – איך דרש על היציאה שלילת המלאך והשרף והשליח? והראיה שהביא על זה לא תדבר מהיציאה כי אם מהעברה במצרים ומכת בכורות ומשפט האלהיות, ואף שנודה שהדברים אלה נעשו על ידו תברך מבלי אמצעי, לא יתחייב מזה שהיציאה ממצרים שנמשכה אחרי זה יהיה כמו כן, שאפשר שהסבות יהיו ממנו יתברך והדברים שנמשכו מהן בו על ידי אמצעים.ב' – גם בהדרשה על "ועברתי בארץ מצרים" לא שלל המלאך והשרף והשליח שלשתם בדבר אחד, אבל ייחס שלילת המלאך בענין ההעברה במצרים וייחס שלילת השרף במכת בכורות וייחס שלילת השליח במשפט האלהיות, ואיך חשב מכאן המגיד לחבר שלשתם יחד בענין היציאה בלבד, כי דרש "ויוציאנו ה' ממצרים לא על ידי מלאך ולא על ידי שרף ולא על ידי שליח"?ג' – איך דרש על מלת "ועברתי" לחוד ועל מלת "והכתי" לחוד הלא אין שום דבר בהעברה זולת מכת בכורות ומשפט האלהיות, ושני הענינים האלה דרש בפירוש ממה שנאמר והכיתי כל בכור אני ולא שרף ובכל אלהי מצרים אעשה שפטים אני ולא שליח, ואיך עשה דרשה שלישית נבדלת ממלת "ועברתי" כי אין לנו בזה רק שני נסים ולא שלשה.ד' – איך דרש ויוציאנו ה' ממצרים לא על ידי מלאך: הלא פסוק מלא הוא בהפך זה שנאמר "וישלח מלאך ויוציאנו ממצרים" (במדבר כ', ט"ז) המורה שהיציאה היתה על ידי מלאך?ה'– איך דרש שלא היתה ההבאה על ידי שרף, הלא הכתוב אומר "ולא יתן המשחית לבא אל בתיכם לנגוף" (שמות י"ב, כ"ג), המורה שעל ידי משחית הוכו והוא השרף או השליח? וכתב הרב רבינו נסים בדרשותיו שעם היות ה' יתעלה עבר במצרים והכה הבכורות בעצמו לא ימנע שיעברו עמו משחיתים עושי דברו ובזה הדרך אין סתירה מהכתוב אל הדרש, אבל זה בלתי מספיק, כי אם היו שם משחיתים בהכרח היו מלאכים או שרפים או שליחים באותה ההבאה שיעשו, ואם כן לא הכה קדוש ברוך הוא כל בכור בעצמו כי אם קצתם.ו' – למה בחר השם יתברך שלא להכות בכורי מצרים על ידי מלאך. ולמה לא יהיה זה בענין מפלת מחנה סנחריב שנאמר שם "ויצא מלאך ה' ויך במחנה אשור מאה ושמונים וחמשה אלף בלילה אחד" (מלכים ב' י"ט, ל"ה)? והנה חז"ל אמרו שבא להבדיל בין טיפה לטפה בין בן פשוט לבן בכור, ובספר הזהר כתוב שלא יכיר המלאך בבני המצריות להיותן שטופות בזימה אם הוא בכור או לא, ולכן הוצרך הקדוש ברוך הוא בעצמו שהוא המותן נפש בעובר והוא היודע. אבל הדעה הזאת בלתי מתישבת על השכל ומתחייבת ביטולים שאינם נאותים במקום הזה.והנראה לי תדע אחרי שאקדים לך הקדמה קצרה אודות האמצעיים והוא, שכל מה שיפעל הקדוש ברוך הוא בעולם המורגש הזה על ידי אמצעים, הנה האמצעים ההם יהיו משני מינים, המין האחד אמצעים בעלי שכל בחירה ורצון ואלה יפעלו פעולותיהם כפי הסדר בדעת ובחירה ורצון, והמין השני אמצעים שאים בעלי שכל ובחירה אבל הם במדרגת הכלים שיפעלו פעולותיהם במבע מבלי רצון ולא בבחירה מה שיעשו, ומזה המין הם היסודות והכתות והדברים הטבעיים כולם, ועליהם אמר "עושה מלאכיו רוחות משרתיו אש לוהט" (תהלים ק"ד, ד') ואלו ראויים שיקראו אמצעים כליים. ואין ספק שכאשר בא בדבריהם ז"ל דבר מן הדברים שעשאו הקדוש ברוך הוא בכבודו ובעצמו לא על ידי מלאך ולא על ידי שרף ולא על ידי שליח, שרצונם בזה שלא יהיה הפעל ההוא טבעי מסודר מהמערכות השמימיות ומניעיהם שהם מלאכים ושלוחים מהשם יתברך בהנהגת העולם השפל, אלא שהוא מהשגחה הפרטית כפי הרצון האלהי למעלה מן המנהג הטבעי והנהגת העליונים, וזה אמרם בכבודו ובעצמו, אבל לא ימלט שיהיה הפעל הזה על ידי כלים, כי הרוחני לא יפעל במורגש במישוש כי אם על ידי כלים. והנה חז"ל ראו שהחכמה האלהית השליטה פרנסת העולם על ידי שתי ההנהגות, האחת טבעית והיא מסודרת מהמערכה השמיימית על ידי התנועות הגלגליות, וממנה גם כן השפעות השרים העליונים שיפעלו בעולם כפי המסודר להם אבל בדעת ובחירה. וההנהגה השנית היא השגחיית המסודרת מהשם יתברך מרצונו הפשוט באשר תגזר חכמתו, ואיננה כפי הסידור השמיימי, וגם לא על ידי המלאכים והשרים העליונים, אלא השם יתברך יעשה אותה בעצמו מבלי היות שם מלאך או אמצעי אחר בעל דעת ובחירה. בכל זאת לא ימלא שתהיה פעולתו ההשגחית והרצונית ההיא בדברים החומרים באמצעות כלים, כאילו תאמר למשל שהמלך יהרוג הרוצחים פעם על ידי שופטים ושומרים הממונים בארצות אשר תחת ידו, וגם כן על ידי שר משריו שישלח ויצוה אותו להשמיד להרוג ולאבד, ופעם יעשה המלך בעצמו ובידו ההרג מבלי אמצעי, אבל לא ימלט שיצטרך כלי לעשות ההרג כמו החרב או דבר אחר שיאות לו.והנה ההשגחה האישית הרצונית הזאת הוא יותר משובחת לאין תכלית מהסדור השמיימי ומפעל העליונים, ועל זה אמר המשורר "כי גדול מעל שמים חסדך ועד שחקים אֳמִתֶךָ" (שם ק"ח, ה'), ושיערו חז"ל שיציאת מצרים לא היתה מסודרת ולא מחוייבת כפי ההנהגה הטבעית ולא היתה מפעל השרים העליונים, וראית לזה שלוש בחינות: האחת, לפי שהמערכה השמיימית המורה על ארץ מצרים היתה מונעת ומעכבת יציאת העבדים משם, וכמו שכתב הראב"ע. שנית, מורה על זה רצון פרעה ועבדיו שבשום צד לא נתרצו ולא נתפייסו שיצאו ישראל מעבדותם, שהיה הדבר חוץ מטבעם, וכנגד הכחות העליונים המשפיעים עליהם. ושלישית, שהמצריים היו עובדים למזל טלה והיה משפיע עליהם מעלה ובכבוד במדה מיוחדת, והיה המזל שורר שם כאיש בביתו בגבורתו ובמעלתו.ומפני הסבות האלה ראה הקדוש ברוך הוא לעשות הגאולה הזאת בעצמו כי מי ישדד המערכות השמימיות ומי ישנה הטבעים המוחזקים ומי יבטל כוחות השרים העליונים והשפעותיהם זולת השם יתברך אשר יצרם, כ רק הבורא הכללי יוכל לשנות טבעיהם, והוא יתברך עשה את יציאת מצרים בשינוי הטבעים התחתונים, והעליונים מעין הבריאה הראשונה הכללית. ולכן הוצרכה הבריאה שתהיה על ידו יתברך כי יוצר הכל הוא. ומפני שהיתה יציאת מצרים מעין הבריאה הראשונה ומורה עליה מעשית זכר לכל המצוות כולן, ואל שלושת הבחנות הנזכרות כיון באמרו "ועברתי בארץ מצרים" וגו' ומורה במלת "ועברתי" שעם היות המערכה העליונה מחייבת שלא יצאו ישראל ממצרים, בכל זאת \textrm{\textbf{תעבור}} ההנהגה ההשגחית, ותשדד המערכה ויצאו לחירות.ולכן דרשו חז"ל על הבחינה הראשונה ענין ההעברה, "אני ולא מלאך", לפי שהמכוון בה הוא לשדד בה המערכה כי אי אפשר לה לעמוד כי אם ברצונו הפשוט וביכולתו יתברך, וכנגד הבחינה השניה מרשעת פרעה ועבדיו והפצרם שלא להוציא את ישראל, אמר "והכיתי כל בכור בארץ מצרים", רצה לומר אם הם יחזיק בבני בכורי ישאל הנה אני אכה את בכוריהם באופן שפרעה וכל עבדיו יבואו לחלות פניהם ויאמרו להם קומו צאו מתוך עמי, ולמי יאות שנוי הרצון של פרעה ועבדיו כי אם אליו יתברך ובמאמר הנביא "עקוב הלב מכל ואנוש הוא מי ידענו אני ה'" (ירמיהו י"ז, י"ט) וגו'.ולפי שהכאת הבכורות לא היתה בדרך טבעי בענין הקדחת השורפה והמגפות המתחדשות בעולם, לכן דרשו על זה אני ולא שרף, והוא תאר לחום הזר הממית במחלות בסדר טבעי כפי המערכה, כי הבכורות האלה כפי טבעם היו בגדר הבריאות אבל ה' הדפם.וכנגד הבחינה השלישית מאלהיות מצרים שהיו מגינים בעירם ומשפיעים עליהם טובות, אמר "ובכל מצרים אעשה שפטים, ולא אמר זה על הכוחות השמיימות כי אם על השרים המניעים אותם, וענינו שאף על פי שהשרים העליונים יחייבו מעלה וכבוד וחיים וחסד למצרים, הנה הוא יתברך ישחית השפעתם ויסיר גבורתם באופן שלא יעמדו להם למושיע, כי אלו האלהיות והבכורות יקבלו מכה אחת ברגע אחד בחצות הלילה. ויען כי היכולת לבטל הכוחות של השרים העליונים הוא דבר מיוחד להשם יתברך ולא לזולתו, לכן דרש עליו אני ולא אחר, ואמר בסוף הדברים אני ה' וגו' ודרש אני ה' אני הוא ולא אחר, לפי שהפעל הזה לא יאות כי אם אליו יתברך בלבד.ובמאמר עטרת זקנים אשר חברתי בבחרותי פירשתי "ועברתי בארץ מצרים אני ולא מלאך" באופן אחר, והוא שעשה הקדוש ברוך בלילה הזה שלוש פליאות, האחת והיא קודמת בסבה לכולן, היא להדבק ההשגחה האלהית הפרטית באומה מבלי אמצעי בהיותם במצרים, עם היות טבע הארץ מונע גדול לזה, וכמו שאמרו חז"ל שהיתה מצרים מלאה גילולים, ולכן לא היה משה רבינו עליו השלום מתפלל בתוכה שנאמר "כצאתי את העיר אפרוש כפי אל ה'" (שמות ט', כ"ט), ועל הפלא הזה דרש ועברתי בארץ מצרים אני ולא מלאך, רצה לומר שתדבק השגחתי המצרים על בני ישראל מבלי אמצעי. והפלא השני במכת בכורות והשלישי במשפט האלהיות כפי מה שכתבתי.הנה התבאר מזה כלו שהיציאה ממצרים היתה בביטול הדברים המונעים אותה, ומפני זה ראה המגיד לדרוש בפסוק ויוציאנו ה' ממצרים לא על ידי אד משלשתם היינו מלאך שרף שליח, אל שהיציאה היתה על ידי עצמו יתברך, כיוון שהוא שידד המערכה העליונה של המלאכים שהיתה מונעת היציאה, וכן מכת בכורות שהיא היתה הסבה ביציאתם נעשתה ברצונו הפשוט ולא על ידי מפעל טבעי שהוא שרף, וגם כן לא על ידי שליח לפי שאלהי מצרים היו מגינים ומשפיעים עליהם עשה בהם שפטים תחילה והסיר יכלתם והשחית השפעתם. הרי לך שכל מה שדרשו חז"ל בפסוק "ועברתי בארץ מצרים" כולו נכלל בפסוק "ויוציאנו ה' ממצרים", כי שלשת הסיבות נכללו ביציאה. ושיערו חז"ל שלא היה ראוי לכתוב השם הנכבד אצל "ויוציאנו" לפי שכבר נאמר למעלה בכתוב "וישמע ה' את קולנו". אלא שרצה להגיד שם עניני היציאה שעשאם בעצמו.והנה המלאך והשרף והשליח שנאמר בדרשה זו הם כולם שמות לאמצעיים הבחירים, אמנם מה ששלח משה לאדום לאמור "וישלח מלאך ויוציאנו מארץ מצרים" אין ספק שמלאך ההוא נאמר על הכלי והוא משה רבינו עליו השלום שהיה כלי ה' באותה הוצאה, וכבר ביארתי שלא שללו בדרשה כי אם האמצעיים הבחיריים שהם מנהיגי העולם, אבל האמצעיים הכליים לא ימלטו בפעולות ההשגחיות. ואפשר עוד לפרש כי מה שאמר משה "וישלח מלאך ויוצאנו מארץ מצרים" שלא הייתה היציאה על ידי אותו מלאך רק \textrm{\textbf{השליחות}} לבד, כי השם יתברך שלח לפניו מלאך והוא משה להתרות בפרעה, ואחרי התראתו הוציאנו השם יתברך בעצמו ממצרים ולא המלאך. אולם מה שאמרו במכילתא "משנתנה רשות למשחית לחבל אינו מבחין בין צדיק לרשע" והקדוש ברוך הוא אמר "ולא יתן המשחית לבוא אל בתיכם", אינו סותר לזה שדרשו על "ועברתי בארץ מצרים" לפי ששם שללו האמצעיים שהם המנהיגים הבחיריים לומר שלא היתה מכת בכורות והצלת ישראל מסודרת מהם, וקראום מלאך שרף ושליח שהם שמות למנהיגים הבחיריים, אבל המשחית הוא הכלי שעשה ה' לשעתו לפעול בו כרצונו. וכבר ביארתי שההנהגה ההשגחיית לא תמלט מהאמצעים הכליים שאין להם בחירה ורצון ולא הנהגה כי אם הוצאת הרצון האלהי הפרטי לפעול כחרב אשר ביד האדם אשר בו יעשה מה שירצה, והאדם הוא הפועל ולא החרב.ולהיות המשחית הזה שהזכיר הכתוב וחז"ל גם כן במדרגת הכלי, כי השליט עליהם פתאום אוויר מעופש שהמית את הבכורות או סבה אחרת המחייבת את המוח, ולכן אמרו אין מבחין בין צדיק לרשע מפני שהסיבה או הכלי אין לו בחינה ובחירה, מאחר שישולח הכלי לפעול הרי הוא כמו האבן שישליכה אדם מידו אשר לא תחדל פעולתה מפני זכות המקבלים. וכבר פירשתי למעלה "ולא יתן המשחית לבא אל בתיכם לנגוף" על המצריים, כי בראותם בני ישראל יקחו אלהיהם ויצלהו ויאכלו אותו בכל פה והכלבים המצרים עזי הנפש לא יוכלו לנבוח ואין לאל ידם לבא אל בתי ישראל להנקם מהם, כי לא יתן השם יתברך אותם לבוא אל בתיהם לנגוף לישראל על עבודתם. והגאון רבי דוד אבודרהם13רבי דוד אבודרהם היה פרשן תפילות נודע בן המאה ה-14. חי ופעל בסיביליה.  כתב שהמשחית הנזכר בכתוב הוא כינוי להשם יתברך או שהוא שם הנאמר על ההשחתה רצה לומר שלא יתן ההשחתה לבוא אל בתיהם.וצריכים אנו לתרץ למה לא נעשה הנס הזה על ידי מלאך כמו שנעשה במחנה מלך אשור? הסיבה בו מבוארת כי להיות בכלל המכה הזאת שודדת המערכה השמיימית והסרת השפעה משרי מעלה וגבורתם לא היה אפשר עי אם בעצמו יתברך, שהוא גבוה מעל גבוה ופוקד על צבא המרום במרום. אולם במחנה אשר לא נלקו העליונים כי אם התחתונים בלבד, ולכן היה אפשר שיהיה על ידי מלאך, ואפשר לומר שגם המלאך ההוא שהכה במחנה אשור היה גם כן כלי משחיתו של הקדוש ברוך הוא ונקרא מלאך לפי שאף הכלים נקראו כן, ובאמת גם שם היה המכה אותם הקדוש ברוך הוא בעצמו, אך הכתוב הזכיר ענין הכלי בפעולתו. ובפרק חלק בסנהדרין (סנהדרין צ"ד) נתנו בזה טעם אחר "בי האנא משום ר' יהושע בן קרחא פרעה שחרף בעצמו להקדוש ברוך הוא שאמר מי ה' אשר אשמע בקולו נפרע ממנו הקדוש ברוך הוא בעצמו שנאמר "וינער ה' את מצרים" ואמר "דרכת בים סוסיך", סנחריב שחרף על ידי שליח שנאמר ביד מלאכיך חרפת" ה' נפרע ממנו על ידי שליח שנאמר "ויצא מלאך ויך המחנה אשור". רצו בזה שפרעה היה מכחיש מציאות ה' לכן הוצרך השם יתברך להראות לו מציאותו ליכולתו והשגחתו כמו שיתבאר אחר כך, אמנם סנחריב לא היה מכחיש כל זה אלא היה עושה עצמו אלוה באחד משרי מעלה לכן ציוה ה' בחורבנו על ידי אחד ממשרתיו. והרמב"ן ביאר הדרשה הזו באופן אחר מהמלאך והשרף והשליח אך לא אזכרהו כי לא יישר בעיני, ודרך המקובלים בזה באופן אחר, שהם פירשו לא על ידי מלאך על מלאך הרחמים ממלאכי עליונים המיוחסים על ממשלחת המים והרחמים הוא מיכאל, ולא על ידי שרף, הוא ממדת הדין המיוחס לממשלת האש, וראש הכת הזאת הוא גבריאל, ושליח הוא כולל באוויר ובארץ שהם אמצעים בין המים והאש והוא שר הפנים אשר קראו אותו מטטרון ששמו כשם רבו הכולל כל המעלות והוא הממונה על התחתונים אלא הקדוש ברוך הוא בכבודו ובעצמו נקרא כבוד ה' ושמו המפורש, בענין שאמרו וה' הוא ובית דינו. ואני שמעתי ולא אבין הדברים האלה, ומה שכתבתי בראשונה הוא הנכון לפי הפשט. והותרו עם זה ששת הספקות הנזכרות בשערים ס"ו, ס"ז, ס"ח, ס"ט, ע' וע"א.}%endcomment
\hebeng{וְעָבַרְתִּי בְאֶרֶץ מִצְרַיִם בַּלַּיְלָה הַזֶּה – אֲנִי וְלֹא מַלְאָךְ; וְהִכֵּיתִי כָל בְּכוֹר בְּאֶרֶץ־מִצְרַים. אֲנִי וְלֹא שָׂרָף; וּבְכָל־אֱלֹהֵי מִצְרַיִם אֶעֱשֶׂה שְׁפָטִים. אֲנִי וְלֹא הַשָּׁלִיחַ; אֲנִי ה׳. אֲנִי הוּא וְלֹא אַחֵר.}{"And I will pass through the Land of Egypt" - I and not an angel. "And I will smite every firstborn" - I and not a seraph. "And with all the gods of Egypt, I will make judgments" - I and not a messenger. "I am the Lord" - I am He and there is no other.}%
\commenta{\textrm{\textbf{ועברתי בארץ מצרים אני ולא מלאך.}} צריך באור לפי זה הלשון בתורה (פ׳ בא, י״ב כ״ג) ולא יתן המשחית לבוא אל בתיכם לנגוף, וזה מוסב על מלאך המות, ולמה זה להודיע, אחרי שאמר אני ולא מלאך. וצריך לומר, דמכוין גם על סתם מיתות הבאות בטבע בהמשך הלילה בתוך המון עם גדול מישראל שהיו אז כשש מאות אלף רגלי הגברים מבן עשרים שנה ומעלה, לבד מנשים וטף (פ׳ בא, י״ב ל״ז), ועיי״ש ברש״י ושפ״ח), ומבטיח, שגם אפשרות מיתות טבעיות כאלה לא יהי׳, יען כי בכלל לא יהי׳ שם מעבר למלאך המות בלילה ההוא.\textrm{\textbf{אני ולא אחר.}} טעם דרש זה והקודם הוא משום דהלשון הסמוך ועברתי — הוא לבדו מורה על סימן הגוף שהוא (היינו אני) יעבור, ומדהוסיף לזה עוד מלת הגוף ״אני״ מכוין בזה לומר אני ולא אחר. ובזה תתבאר הדרשה במס׳ קדושין (מ״א ב׳) על הפסוק דס״פ קרח בענין הפרשת תרומה, תרימו אתם, ודרשו, אתם ולא שלוחכם, וזה הוא משום דגם הלשון ״תרימו״ לבד מדבר לנוכחים רבים, ובא תוספת המלה ״אתם״ לחזק הפעולה — אתם ולא שלוחכם (ולבסוף ילפינן שם שליחות בתרומה מרבוי הלשון גם אתם). ועל פי המבואר יש להעיר במשנה ריש מס׳ ב״מ, שנים אוחזין בטלית, זה אומר אני מצאתיה וזה אומר אני מצאתיה, זה אומר כולה שלי וזה אומר כולה שלי וכו׳. (ואם) זה אומר כולה שלי וזה אומר חציה שלי וכו׳. והנה לפי המבואר — הנה גם הפעל מצאתיה לבד (בלא תוספת ״אני״) מוסב למדבר בעדו שהוא מצאה, ושוב מוסיף מלת הגוף ״אני״ (מצאתיה) בא בזה לחזק הפעולה, שרק הוא לבדו מצאה ולא אחר, ולפי זה קשה, איך אמר זה אח״כ חציה שלי, אחרי דמתחילת לשונו אני מצאתיה מכוין לומר אני לבדי מצאתיה ולא אחר וכמו כאן אני ה׳ ולא אחר, וממילא הוא מקיים שכולה שלו, ובאמרו חציה שלי הרי הוא תוך כדי דבור סותר דברי עצמו.  אך אפשר לחלק, דמבקום שמלת הגוף ״אני״ בא לראשונה, כמו במשנה, אני מצאתיה, אינו בהכרח שימעט כולו זולתו, יען כי כן דרך לשון בני אדם לצרף מלת הגוף לסימן הגוף ולכן באמרו אני מצאתיה אינו ממעט מציאתה גם באיש אחר, ולכן אפשר שיתקיימו דבריו חציה שלי. כיון דמודה באפשרות מציאתה גם בשני. אבל כאן שלאחר שאמר ״ועברתי״ מוסיף ביחוד ״אני״, אשר איננו מוכרח — באופן כזה בא להדגיש ולהחליט כי רק אני ולא אחר. ומכבר כתבתי לפרש עפ״י מה שבארנו בכונת תוספת מלת הגוף על מלת סימן הגוף את הפסוק במלכים ב׳ (ד׳ א') ואשה אחת מנשי הנביאים צעקה אל אלישע, עבדך אישי מת ואתה ידעת כי עבדך היה יירא ה׳ והנושה בא לקחת את שני ילדי וכו׳. והנה בעל האשה הזאת, כפי המבואר בתרגום ובמדרשים הי׳ עובדי׳ (וכן פירש״י), ולפי המבואר במקרא, אעפ״י שהי׳ עבד לאחאב הרשע, אעפ״י כן הי׳ נודע לכל בצדקתו, והוא החביא את הנביאים מפני קנאת איזבל אשת אחאב, כמבואר במלכים א׳ (י״ח, י״ב וי״ג). ולפי זה קשה הלשון ואתה ידעת, דלפי המבואר, תוספת מלת הגוף על סימן הגוף מורה שרק הוא, המדובר בזה, מתיחש בהענין ולא אחר, אם כן הלשון ואתה ידעת משמע דרק הוא יודע בצדקת בעלה שהי׳ יירא ה׳, ולא אחר — בעוד שבאמת הי׳ נודע בצדקתו לכל, ודי הי׳ לה לאמר וידעת, או — ועבדך הלא היה יירא ה׳, ולהשמיט המלה הנרגשת ״ואתה״. אך הבאור הוא, כי באמת הי׳ מקום להעולם להרהר על צדקת עובדיה, לפי הכתוב במשלי (כ״ט י״ב) מושל מקשיב על דבר שקר כל משרתיו רשעים, ואחרי שעובדיה שרת אצל אחאב הרשע, הלא הי׳ גם הוא כמוהו. והנה אמרו במס׳ חולין (ד׳ ב׳) בסמיכות על הפסוק במשלי הנזכר, מדמושל מקשיב על דבר שקר כל משרתיו רשעים — הא מושל מקשיב על דבר אמת כל משרתיו צדיקים. אבל על זה יש להעיר, כי הלא אלישע (זה שהאשה קראה לו אתה) הי׳ נביא וצדיק, ולו הי׳ משרת גחזי, שהי׳ נודע לרשע (ברכות י׳ ב׳). וצריך לומר דאמנם כן הוא הכלל, אך כידוע אין כלל בלא פרט יוצא וכאן היה עובדי׳ יוצא מן הכלל מושל מקשיב על דבר שקר כל משרתיו רשעים, שהוא הי׳ צדיק, וגחזי יוצא מן הכלל מושל מקשיב על דבר אמת כל משרתיו צדיקים, שהוא הי׳ רשע. ולפי זה — לזה כיונה אשת עובדיה לאמר, הן אם כל העולם יכולים להסתפק בצדקת בעלי ששרת אצל אחאב עפ״י הכלל מושל מקשיב על דבר שקר כל משרתיו רשעים — אבל אתה ידעת, כי הכלל ההפוך מושל מקשיב על דבר אמת כל משרתיו צדיקים איננו כלל מוחלט, שהרי אתה צדיק ועבדך גחזי רשע — אם כן אפשר להיות כיוצא מן הכלל גם זה שעובדיה בעלי אף שעבד אצל אחאב היה צדיק. וזהו שהדגישה לאמר ואתה ידעת כי עבדך הי׳ יירא ה׳, כלומר, רק אתה תוכל לדעת ולקיים זה עפ״י זה שיש יוצא מן הכלל בדמיון מעבדך, שגם הוא בענינו שלו הי׳ יוצא מן הכלל.}%endcomment
\hebeng{בְּיָד חֲזָקָה. זוֹ הַדֶּבֶר, כְּמָה שֶּׁנֶּאֱמַר: הִנֵּה יַד־ה׳ הוֹיָה בְּמִקְנְךָ אֲשֶׁר בַּשָּׂדֶה, בַּסּוּסִים, בַּחֲמֹרִים, בַּגְּמַלִים, בַּבָּקָר וּבַצֹּאן, דֶּבֶר כָּבֵד מְאֹד. }{"With a strong hand" - this {[refers to]} the pestilence, as it is stated (Exodus 9:3); "Behold the hand of the Lord is upon your herds that are in the field, upon the horses, upon the donkeys, upon the camels, upon the cattle and upon the flocks, {[there will be]} a very heavy pestilence." }%
\commenta{\textrm{\textbf{ביד חזקה זו הדבר, במה שנאמר הנה יד ה׳ הויה במקנך.}} הנה גם אחר שהביא סמך מן הפסוק עדיין לא נתבאר למה מתיחשת דוקא מכת דבר אל יד ה׳, בעוד אשר כידוע בהמשך ההגדה נלקו כל המכות באצבע, וכמש״כ בתורה אצבע אלהים הוא, ולהלן בהגדה, כמה לקו באצבע. ואפשר לומר, משום דכפי סדר המכות היתה מכת דבר — המכה החמישית, ואם כן בהלקותו כל מכה שעד מכה זו ולכל אחת הוסיף אצבע אחת, אם כן כשבא ללקות המכה החמישית, שהיא מכת דבר, צירף האצבע החמישית ואתה יחד כל אצבעות היד, ויצאה ההכאה ביד שלמה. ויש להעיר על טעם הדבר שדוקא מכה זו, מכת דבר, מכונה בתואר יד חזקה, מה החזקה בה יותר מכל המכות. ואפשר לפרש עפ״י המסופר בשמואל ב׳ (כ״ד), כי לאחר שחטא דוד במנותו את ישראל (כי אסור למנות את ישראל, יומא כ״ב ב׳) אמר לו הנביא גד בשם ה׳, כי יברר לו לעונש אחת משלש אלה, או כי יהיה דבר בארצו שלשת ימים, או רעב, או כי ימות במלחמה בחרב. ועל זה ענה דוד ואמר, נפלה נא ביד ה׳, ופירשו באגדות, שכונתו בזה הלשון היתה לברור מכת דבר, כי אמר, אם אני בורר רעב יאמרו שאני בוטח בעשרי להשיג אוכל, ואם אברור חרב, יאמרו שאני בוטח בגבורים שלי שלא יתנו להרגני, אך אברור מכת דבר שהכל שוים בה. ומבואר מזה, דמכת דבר הכל מודים שאי אפשר להמלט ממנה, ולכן מכנה כאן מכת דבר פעולה בשם יד חזקה, מפני כי גם לפי השגת בני אדם היא חזקה על כל ואין מוצא ממנה. וכמו שאמר דוד המלך.}%endcomment%
\commentb{\textrm{\textbf{תשובות לשערים ע"ב – ע"ט}}\textrm{\textbf{ביד חזקה זה הדבר כמו שנאמר "הנה יד ה' הויה במקנך" וכו'.}}כבר אמרתי בשערים כי בדרשה הזו יפלו שמונה ספקות:א' – למה דרש ביד חזקה על דבר המקנה ולא דרש אותו על מכת בכורות שנקראת בכתוב "יד חזקה" ויצדק יותר אמרו ויוציאנו ה' ממצרים ביד חזקה כי במכת בכורות יצאו משם.ב' – מה שדרש ובזרוע נטויה על החרב לא מצינו במכות מצרים מחת חרב, והפסוק שהביא לראיה ידבר על ירושלים ולא על מצרים.ג' – מה שדרש ובמורא גדול זה גילוי שכינה, לא מצינו כי נגלה השכינה במצרים כי אם אחר כך במעמד הר סיני.ד' – הפסוק שהביא עליו הוא כולל כל נסי מצרים.ה' – מה שדרש ובאותות על המטה, קשה כי אין המטות אותות אלא הוא כלי להם והכתוב שהביא מורה עליו כן.ו' – מה שדרש ובמופתים זה הדם, קשה כי אם היות הדם מכלל המופתים אין ראוי לומר שהדם הוא המופתים כי הוא מופת אחד לא מופתים רבים.ז' – הפסוק שהביא על זה מדבר מהגאולה העתידה.ח' – לה מכל מכות מצרים לא הזכיר בפרט בדרשה זו כי אם שנים הדבר והדם והשמיט יתר המכות? ויותר ראוי היה לדרוש ובאותות ובמופתים על כל המכות מלדרוש אותן על הדבר והדם בלבד.והנראה לי בזה הוא שהמגיד מצא בכתוב הזה "ויוציאנו ה' ממצרים" וגו' דברים שהזכיר בלשון יחיד כמו יד חזקה זרוע נטויה ומורא גדול, ודברים בלשון רבים כמו אותות ומופתים, ולכן דרש כל דבר כפי ענינו והוראת המילה בדרך גזירה שווה ממקומות אחרים, כי הוא דרך יד חזקה על הדבר לפי שהוא נקרא יד ה' כמו שהביא "הנה יד ה' הויה במקנך אשר בשדה", והנה לא דרש זה על מכת בכורות לפי שידרוש עליו "ובזרוע נטויה", וסבר שכמו שאצבע אלהים הוא דבר קטן לעומת יד ה' שהוא דבר חזק ממנו, כי היד גדולה מהאצבע, בכוח זרוע נטויה מורה על חוזק המכה יותר מיד ה' לפי שהזרוע כולה היא גדולה וחזקה מהיד, ולכן בהיות יד ה' נדרש על דבר מקנה ידרוש זרוע נטויה על מכת בכורות להיותה יותר עצומה, ועליה אמר זו החרב, וירמוז למגפת מכת בכורות על ידי חרב מלאך המות שהכה בהם, ועשה גזרה שוה מנטויה לנטויה, נאמר כאן "ובזרוע נטויה" ונאמר להלן במגפה שבאה בימי דוד "וחרבו שלופה בידו נטויה על ירושלים" (דברי הימים א' כ"א, ט"ז) (דברי הימים א' כ"א, ט"ז), מה להלן חרב מלאך המות שהוא שם למגפה הנטויה בכוח מה שנאמר כאן "זרוע נטויה" נאמר על מגפת בכורות, וחז"ל דרשו החרב על חרב הבכורות שהרגו את אביהם, והוא דרך דרש. ולפי שאלהיות נלקו עם מכת בכורות כמו שנאמר "ובכל ארץ מצרים אעשה שפטים", לכן דרש ובמורא גדול זה גילוי שכינה. ובא על דבקות השגחת ה' במצרים האות שהכה הבכורות והאלהיות, ועל זה הביא הפסוק "או הניסה אלהים לבא לקחת לו גוי מקרב גוי" שמדבר ביציאת מצרים ונאמר בסוף הפסוק "וביד חזקה ובזרוע נטויה ובמוראים גדולים ככל אשר עשה לכם ה' אלהיכם במצרים לעיניך", ומזה הוכיח המגיד שהדברים כולם נעשו במצרים עם יד חזקה שהיא יד הדבר וזרוע נטויה שהיא מכת בכורות ובמוראים גדולים שהוא משפט האלהיות, ולהיותו בנמצאים הרוחנים קראו מוראים גדולים. ולא היה זה בהר סיני כי אם במצרים כיוון שהכתוב אומר "במצרים לעיני". ואמר "במוראים" בלשון רבים לפי שכל מכה ומכה מהבכורות בהיותה ממנו יתברך ומבלי אמצעי היה מורא גדול והמכות כולם היו מוראים גדולים. והמתרגם אונקלוס נראה שעשה מורא מלשון מראה וירמוז למראה ה' שראו במצרים.ואמר המגיד ובאותות זה המטה חשבו המפרשים שאמר זה על נס שנהפך המטה לנחש, ואינו כן כי הנס הזה לא נמנה בכלל המכות כמו שיתבאר. ויש מן החכמים שאמרו שהיה כתוב על המטה שם בן ע"ב אותיות שבו נעשו האותות ולפי זה נכללו במטה כל העשר מכות, אבל יקשה לדעה הזאת למה פרט הדבר והדם? והנראה לי בדרשה הזרה הזאת הוא לתרץ על פי שני דרכים:האחד, שהמגיד דרש "ובאותות" על האותות אשר נעשו באמצעות המטה שהם חמישה, היינו הפך המים לדם, העלה הצפרדע, תולדות הכינים, מכת הברד ומכת הארבה, כי בכל אלה נאמר שנעשו בתנועת המטה, ועליהם כיוון באומרו "זה המטה", רצה לומר האותות הם אלה שנעשו עם המטה, ועל זה הביא הפסוק "וְאֶת הַמַּטֶּה הַזֶּה תִּקַּח בְּיָדֶךָ אֲשֶׁר תַּעֲשֶׂה בּוֹ אֶת הָאֹתֹת" (שמות ד, י"ז), שאותן המכות נקראו אותות אשר נעשו על ידי המטה. אם כן יצא לנו מן הדרשה הזאת זכרון שבע מכות, שהן מכת הדבר ומכת בכורות וחמישה האותות שנעשו עם המטה, ונשארו שלוש מכות שהן הערוב והשחין והחשך ואותן קרא המגיד בשם "דם", לפי שהערוב היה מבעלי חיים הטורפים שהיו נכנסים בבתים והורגים הנערים וטורפים כל מה שיוכלו, ולהיותם שופכי דם תוארם בשם "דם". גם השחין תואר בשם "דם" לפי שהיה ענינו עיפוש הדם והפסדו פתאום, גם החושך תואר בשם "דם" לפי שהיה אור השמש אדמדם ונחשך וכל זה לקח מדברי הנביא יואל שנאמר: "וְנָתַתִּי מוֹפְתִים בַּשָּׁמַיִם וּבָאָרֶץ דָּם וָאֵשׁ וְתִימֲרוֹת עָשָׁן הַשֶּׁמֶשׁ יֵהָפֵךְ לְחֹשֶׁךְ וְהַיָּרֵחַ לְדָם לִפְנֵי בּוֹא יוֹם ה' " וגו' (יואל ג', ג' –ד'). הנה ביאר שלא אמר דם על ההפך המים אשר בנהרות לדם כמו שהיה במצרים כי המופתים שייעד הזכיר שיהיו בשמים ובארץ לא במים, ולכן אמר "דם ואש ותמרות עשן", כי הדם רומז למכת החיות הרעות על דרך "וְשֶׁן בְּהֵמוֹת אֲשַׁלַּח בָּם" (דברים ל"ב, כ"ד), והאש רמז לשחין שהוא באש שורף ברתיחת הדם, ותמרות עשן הוא רמז לחשך, ולפי שגם החשך יתאר בשם דם, לכן אמר מיד "השמש יהפך לחשך והירח לדם" כי הדם הוא האדמימות החזק הנוטה לחושך. ולזה אמר המגיד "ובמופתים זה הדם" רצה לומר המופתים הם השלושה המתוארים בשם "דם" שעליהם אמר הנביא "ונתתי מופתים". ועם היות שהכתוב ההוא נאמר על הגאולה העתידה עשה המגיד גזרה שווה ממופתים למופתים, נאמר כאן "ובמופתים" ונאמר להלן "ונתתי מופתים" מה להלן שלשה אלה אך כאן שלשה אלה, מה להלן נקרא "דם" ההרג והחשך אף כאן כן, ובזה הדרך נזכרו כל העשר מכות, והותרו הספקות כולם, זהו הדרך הראשון.והדרך השני הוא כאשר נדע על מה יורה שם אות ועל מה יורה שם מופת, וכבר חשבו המפרשים ששניהם יורו תמיד על הדברים היוצאים ממנהג הטבעי ואינו כן, כי לפעמים נאמרו שניהם על דברים אנושיים ופעמים נאמרים על דברים אלהיים הנסיים. כי הנה האות יאמר פעמים על הסימן שיעשה אדם לזכרון דבר מה כמו שנאמר "וְהָיָה לְךָ לְאוֹת עַל יָדְךָ וּלְזִכָּרוֹן בֵּין עֵינֶיךָ" (שמות י"ג, י"ט), וצורות הדגלים היו נקראים אותות "אִישׁ עַל דִּגְלוֹ בְאֹתֹת" (במדבר ב', ב') עד שגם חלקי הכתיבה נקראים אותיות. וכן המופת יאמר גם כן בדברים אנושיים על הסימן שהוא יותר חזק מרושם האות, ומפני זה הראיות החזקות הפילוסופיות הם מופתים, וישעיה אמר: "הִנֵּה אָנֹכִי וְהַיְלָדִים אֲשֶׁר נָתַן לִי ה' לְאֹתוֹת וּלְמוֹפְתִים בְּיִשְׂרָאֵל" (ישעיהו ח', י"ח), ולא היו נולדים בדרך נס לא שהיו שמותיהן לסימן ולזיכרון שקראם מהר שלל חש בז, וכן כתיב "וְהָיָה יְחֶזְקֵאל לָכֶם לְמוֹפֵת" (יחזקאל כ"ד, כ"ד), והוא אמר על עצמו "אני מופתיכם" (שם י"ב, י"א), לפי שהיו ענייניו ראיה חזקה על מה שיקרה אותם בעתיד. ולפי זה יאמר אות או מופת בדברים האנושיים על הסימן אשר יעשו בני האדם, ורק שהאות אינו חזק הרושם כל כך כמו המופת. וכזה בעצמו תמצא בעניינים האלהיים היינו במעשה הנפלאות, שיאמר אות על נס קטן הזרות לפי שהוא אות וראיה על אמיתת דברי הנביא, ויאמר על מופת שהוא נס עצום וחזק ממנו לפי שהוא יותר מאמת דברי הנביא. ולכן נאמר חמשה בתחילת שליחותו "וְהָיָה אִם לֹא יַאֲמִינוּ לָךְ וְלֹא יִשְׁמְעוּ לְקֹל הָאֹת הָרִאשׁוֹן וְהֶאֱמִינוּ לְקֹל הָאֹת הָאַחֲרוֹן" (שמות ד', ח'), שקרא אותו מה שנהפך המטה לנחש והיד המצורעה. ולהיות המופת שם הנס החזר אמר פרעה "תְּנוּ לָכֶם מוֹפֵת" (שם ז', ט') להגיד שלא יחפוץ כי אם בנס חזק ונגלה ונאמן עם היות שהשם יתברך לא רצה שיעשה בראשונה כי אם אות המטה כי ההתחלה לעולם בדבר קל, והנביא יואל כדי ליעד על חוזק הניסים אמר "ונתתי מופתים בשמים ובארץ". ולפי זה ענין הניסים והנפלאות הוא כי יאמר אות על הנס מעט מהזרות והמופת על הפלא העצום ורב הזרות, ואחרי הודיע אלהים אותך כל זאת תוכל לומר שהמגיד עשה בכל המכות חלוקה אחת שמהם אותות חלושים כמו שהיה ענין המטה והדומים אליו, ומהם היו מופתים חזקים מאות כמו ההפך המים לדם שהוא מופת חזק כמו שיתבאר אחר כך והדומים אליו, וזה אמר בראשי מילים כמעיר על זה: "ובאותות זה המטה ובמופתים זה הדם", שהביא משל נס המטה למין האותות והביא משל נס הדם לענין המופתים, והוא הדין לשאר המופתים שהם ממין מופת הדם. והוכיח שאמר על הדם מופת שהוא שם הנס החזק מהפסוק: "ונתתי מופתים בשמים ובארץ" (יואל ג', ג'). זהו הנראה לי בזה, ומי שיתן עליו טעם אחר טוב מאלה ישא ברכה מאת ה'.והותרו עם מה שאמרתי הספקות אשר בשערים ע"ב, ע"ג, ע"ד, ע"ה, ע"ו, ע"ז, ע"ח וע"ט.}%endcomment
\hebeng{וּבִזְרֹעַ נְטוּיָה. זוֹ הַחֶרֶב, כְּמָה שֶּׁנֶּאֱמַר: וְחַרְבּוֹ שְׁלוּפָה בְּיָדוֹ, נְטוּיָה עַל־יְרוּשָלָיִם. }{"And with an outstretched forearm" - this {[refers to]} the sword, as it is stated (I Chronicles 21:16); "And his sword was drawn in his hand, leaning over Jerusalem."}
\hebeng{וּבְמוֹרָא גָּדֹל. זוֹ גִּלּוּי שְׁכִינָה. כְּמָה שֶּׁנֶּאֱמַר, אוֹ הֲנִסָּה אֱלֹהִים לָבוֹא לָקַחַת לוֹ גּוֹי מִקֶּרֶב גּוֹי בְּמַסֹּת בְּאֹתֹת וּבְמוֹפְתִים וּבְמִלְחָמָה וּבְיָד חֲזָקָה וּבִזְרוֹעַ נְטוּיָה וּבְמוֹרָאִים גְּדוֹלִים כְּכֹל אֲשֶׁר־עָשָׂה לָכֶם ה׳ אֱלֹהֵיכֶם בְּמִצְרַיִם לְעֵינֶיךָ. }{"And with great awe" - this {[refers to the revelation of]} the Divine Presence, as it is stated (Deuteronomy 4:34), "Or did God try to take for Himself a nation from within a nation with enigmas, with signs and with wonders and with war and with a strong hand and with an outstretched forearm and with great and awesome acts, like all that the Lord, your God, did for you in Egypt in front of your eyes?" }%
\commenta{\textrm{\textbf{ובמורא גדול זו גלוי שכינה.}} לא נתבאר איך יובן בלשון מורא גלוי שכינה, מה יחש מורא לאורה. ואפשר לומר, כי מצינו בתלמוד ומדרשים בסגנון דרשות לכמה מלים עפ״י העתק אותיות בהמלה ממוקדם למאוחר וממאוחר למוקדם, ועפ״י זה יצא ענין הדרש לפי הרצון. כה מצינו בברכות (ל׳ ב׳) דרשו הפסוק בתהלים (כ״ט ב׳) השתחוו לה׳ בהדרת כמו בחרדת (ובחלוף ה׳ בחי״ת, כאותיות אהח״ע שמתחלפין), והועתק הרי״ש להה״א. ובשבת (פ״ה א׳) פירשו השם ״חרי״, הוא שם אומה, בסוף פרשה וישלח (ל״ו כ׳, שעיר החרי) ודרשו חרי כמו ריח, בהפוך האותיות, ופירשו, שהיו מריחין את הארץ וידעו את כל חלקת אדמה לאיזה מין גידולים מוכשרת, ואמרו זו לזית וזו לגפנים וכו׳. ובעירובין (נ״ד סע״ב) פירשו הפסוק במשלי (ז׳ ד׳) ומודע לבינה תקרא, ופירשו כמו ומועד לבינה, שיקבעו עתים לתורה. ובפסחים (פ״ז ב׳) פירשו הלשון בשופטים (ה׳ י״א) צדקת פרזונו בישראל, כמו פזרונו, עי״ש. וביומא (ע״ה ב׳) פירשו הפסוק דפ׳ בהעלתך (י״א ל״ב) וישטחו כמו וישחטו, עיי״ש. ובמו״ק (ט׳ ב׳) פירשו הפסוק בתהלים (מ״ט י״ב) קרבם בתימו כמו קברם, עיי״ש. ובתמורה (ט״ז א׳) פירשו השם עכסה (בת כלב) ממובן כעס, עיי״ש. ועוד שם פירשו השם יעבץ (דהי״א ד׳ ט׳) שיעץ וריבה תורה בישראל, ופירשו כן עפ״י העתק אות מן יעבץ ליעצ״ב*ואמנם העתק זה מפורש בפסוק גופי׳, כי לשונו שם, ואמו קראה שמו יעבץ לומר, כי בעצב ילדתיו. והנה לפי זה הי׳ לה לקרוא לו יעצב בהקבלה לשם עצב, וזה פלא. ואמנם מצינו באור שמות עפ״י העתק מלים, קין — כי קניתי (הנו״ן קודם היו״ד). שמואל על שם כי מה׳ שאלתיו (ש״א א כ׳).). ובירושלמי נזיר (פ״ז ה״ב) פירשו הלשון בפרשת חקת (י״ט ט״ז) או בקבר, כמו או ברקב, לומר שגם בשר רקב של מת מטמא. ובמדרשים פירשו השם לבן הארמי כמו הרמאי, ועוד חלופים כאלה. והנה כאן מפרשים המלה ובמורא כמו ובמאור, בהעתק האותיות, ומבארים, מה ענינו של מאור גדול — זה גלוי שכינה שאורה מבהיק.  ויש עוד סמך נאמן דבלשון מורא כאן מכוין מענין מאור, כי את הלשון ובמוראים גדולים (פ׳ ואתחנן, ד׳ ל״ד) תרגם אונקלוס ובחזונין רברבין, ושרש הזה משותף עם שם ראי', כמו ותחזינה עינינו, ובאיוב (ט״ו י״ז) את זה חזיתי ואספרה, ועוד הרבה, וכאן המובן ראיה חזקה עפ״י מאור גדול, ואיך אפשר להגביל גדולת האורה במדה היותר גדולה — זה גלוי שכינה. ואף כי מצינו שם מאור גדול בכנוי להשמש (פ׳ בראשית), אך שם אין הכרח לפרש דמוסב על יתרון גדולת האורה, כי אפשר שאינה גדולה בערך אור גדול, אך נקרא גדול רק לפי ערך חבירו הקטן ממנו (הלבנה), וכמו שני אגוזים אחד גדול ואחד קטן, והנה גם הגדול קטן הוא, אך נקרא גדול לפי ערך חבירו אשר הוא עוד יותר קטן ממנו, ולא כן כאן דכתיב מאור גדול סתם, ונערך בערך גדלות מופלגת, והיינו גלוי שכינה.}%endcomment
\hebeng{וּבְאֹתוֹת. זֶה הַמַּטֶּה, כְּמָה שֶּׁנֶּאֱמַר: וְאֶת הַמַּטֶּה הַזֶּה תִּקַּח בְּיָדְךָ, אֲשֶׁר תַּעֲשֶׂה־בּוֹ אֶת הָאֹתוֹת. }{"And with signs" - this {[refers to]} the staff, as it is stated (Exodus 4:17); "And this staff you shall take in your hand, that with it you will perform signs."}
\hebeng{וּבְמֹפְתִים. זֶה הַדָּם, כְּמָה שֶּׁנֶּאֱמַר: וְנָתַתִּי מוֹפְתִים בַּשָּׁמַיִם וּבָאָרֶץ. }{"And with wonders" - this {[refers to]} the blood, as it is stated (Joel 3:3); "And I will place my wonders in the skies and in the earth:}
\newsection{עשר המכות}
\hebeng{{\small כשאומר דם ואש ותימרות עשן, עשר המכות ודצ״ך עד״ש באח״ב – ישפוך מן הכוס מעט יין:} }{{\small And when he says, "blood and fire and pillars of smoke" and the ten plagues and "detsakh," "adash" and "ba'achab," he should pour out a little wine from his cup.} }
\hebeng{דָּם וָאֵשׁ וְתִימְרוֹת עָשָׁן.}{blood and fire and pillars of smoke."}
\hebeng{}{}
\hebeng{דָבָר אַחֵר: בְּיָד חֲזָקָה שְׁתַּיִם, וּבִזְרֹעַ נְטוּיָה שְׁתַּיִם, וּבְמֹרָא גָּדֹל – שְׁתַּיִם, וּבְאֹתוֹת – שְׁתַּיִם, וּבְמֹפְתִים – שְׁתַּיִם. }{Another {[explanation]}: "With a strong hand" {[corresponds to]} two {[plagues]}; "and with an outstretched forearm" {[corresponds to]} two {[plagues]}; "and with great awe" {[corresponds to]} two {[plagues]}; "and with signs" {[corresponds to]} two {[plagues]}; "and with wonders" {[corresponds to]} two {[plagues]}.}
\hebeng{אֵלּוּ עֶשֶׂר מַכּוֹת שֶׁהֵבִיא הַקָּדוֹשׁ בָּרוּךְ הוּא עַל־הַמִּצְרִים בְּמִצְרַיִם, וְאֵלוּ הֵן:}{These are {[the]} ten plagues that the Holy One, blessed be He, brought on the Egyptians in Egypt and they are:}
\hebeng{-----}{-----}
\hebeng{דָּם}{Blood}
\hebeng{צְפַרְדֵּעַ}{Frogs}
\hebeng{כִּנִּים}{Lice}
\hebeng{עָרוֹב}{{[The]} Mixture {[of Wild Animals]}}
\hebeng{דֶּבֶר}{Pestilence}
\hebeng{שְׁחִין}{Boils}
\hebeng{בָּרָד}{Hail}
\hebeng{אַרְבֶּה}{Locusts}
\hebeng{חשֶׁךְ}{Darkness}
\hebeng{מַכַּת בְּכוֹרוֹת}{Slaying of {[the]} Firstborn}%
\commenta{\textrm{\textbf{מכת בכורות.}} צריך באור למה כל המכות נקראו רק בשמן לבד, דם, צפרדע וכו', ורק במכת בכורות הוסיף המלה ״מכת״ ולא בשמה העצמי לבד, ״בכורות״, כסגנון כולן. וגם אפשר לומר פשוט הטעם על שלמכת בכורות נוספה המלה ״מכת״ ולא ״בכורות״ לבד, כמו כל המכות שנקראו בשמן לבד, דם, צפרדע וכו׳ בלא תוספת המלה ״מכת״, יען כי כל המכות שמותיהם לבד יעידון כי מכות הן, כי בודאי דם, צפרדע, כנים, ערוב וכו׳ אינן כי אם מכות, ולא כן השם בכורות הוא שם סתמי שאין בו כל רמז שענינו מכות, לכן צריך להוסיף עליו המלה ״מכת״. — ואפשר לפרש עפ״י המבואר במדרש שוחר טוב (הוא מדרש תהלים), (פרשה קל״ו), כי אחר התראת משה ממכת בכורים רצו הבכורים לשלוח את ישראל למען ינצלו מעונש מיתה, אך אבותיהם לא שמעו להם ולא רצו לשלוח את ישראל, ואז הרגו הבכורים את אבותיהם, ועל כן נקראה מכה זו מכת בכורות, כי מה שהבנים הרגו את אבותיהם זו מכה לעצמה, ואל זה מרמז הלשון המיוחד, ״מכת״ בכורות. ודע, כי עפ״י המדרש הנזכר מענין הריגת המצרים ע״י בניהם הבכורים, יתבאר היטב הלשון בתהלים (קל״ו) למכה מצרים בבכוריהם, ואין זה לשון רגיל. והי׳ צריך לומר בכורים במצרים, אך לפי המדרש הנזכר הבאור פשוט, למכה מצרים בבכוריהם, ע״י בכוריהם, וכמבואר, זו מכה לעצמה.}%endcomment
\hebeng{רַבִּי יְהוּדָה הָיָה נוֹתֵן בָּהֶם סִמָּנִים: דְּצַ״ךְ עַדַ״שׁ בְּאַחַ״ב.}{Rabbi Yehuda was accustomed to giving {[the plagues]} mnemonics: \textit{Detsakh} {[the Hebrew initials of the first three plagues]}, \textit{Adash} {[the Hebrew initials of the second three plagues]}, \textit{Beachav} {[the Hebrew initials of the last four plagues]}.}%
\commenta{\textrm{\textbf{רבי יהודה הי׳ נותן בהם סימנים, דצ״ך עד״ש באח״ב.}} כלל ענין סימנים אלה מופלא מאוד, כי הן רגילים אנחנו להבין בהמלה שמביאים לסימן איזה משמעות, מובן או הוראה, ולא רק חבור אותיות בלבד, ובאמת כן הסברה מחייבת, יען כי בלא זה אין כל יסוד לזכור הסימן עצמו, ״וסימנך סימנא צריך״, ואם כן מה ענינו ותכליתו של הסימן דצ״ך אד״ש באח״ב אחרי שהמלים האלה אין מגידים מאומה. אך הבאור הוא, כי מצינו לרבי יהודה, בעל הסימנים האלה, שמבאר דבריו אלה במקום אחר, ואם לא מפורט אך מכללא, והוא בספרי פרשה האזינו שבא לשון זה, רבי יהודה אומר, לעולם יהא אדם כונס דברי תורה כללים, שאם כונסם פרטים מייגעין אותו, ע״כ, ורוצה לומר, כי זכירת כולם לפרטיהם מייגעין אותו. ולפי זה יתבאר, שכאן מרמז לדעתו בספרי שם, שראוי לקבץ פרטים רבים לכוללם במלים אחדות, כדי שיהי׳ נקל ביותר לזכור אותן, יען כי נקל לזכור שלש מלים מאשר עשר, ואף כי אין בהן משמעות. וגם במשנה מנחות (צ״ו א׳) נתן רבי יהודה סימנים לששה פרטים בשתי מלים, וג״כ אין בהן משמעות, ונראה שגם שם כיון לדעתו בספרי שהבאנו. וכן צריך לומר הכונה בהסימן יע״ל קג״ם הרגיל בגמרא, והלכתה כותי׳ דאביי ביע״ל קג״ם, עיין קדושין נ״ב א׳, וברש״י שם מבואר פרטי השמות, וכונת הסימן הזה ג״כ משום דנוח לזכור שתי מלים מאשר שש. ודע. כי עפ״י מה שגלינו דעתו דרבי יהודה כאן ובמנחות עפ״י דבריו המפורשים בספרי האזינו — עפ״י זה יכוננו מאוד דברי הירושלמי במס׳ רה״ש פ״ג ה״ה בזה הלשון, תני בשם רבי נחמיה. היתה כאניות סוחר ממרחק תביא לחמה (פסוק הוא במשלי (ל״א) המדבר בשבח האשת חיל, ובדרך השאלה לענין התורה) דברי תורה עניים במקום זה ועשירים במקום אחר, ע״כ, כלומר יש דברי תורה שאין מובנים ואין מבוארים במקומם (והמשל לזה — עניות) והם מפורשים ומבוארים יפה במקום אחר (והמשל לזה — עשירות) וצריד להביאם ממרחק. וכך ענינם של סתימת הסימנים כאן ובמנחות ע״י רבי יהודה ובאורם בספרי על ידו — ובענין זה קשה לי ברמב״ם הלכות קדוש החודש פרק י׳ הלכה א׳ שכתב בזה הלשון, ״תוספת שנת החמה על שנת הלבנה עשרה ימים וכ״א שעות וקכ״א חלקים ומ״א רגעים, סימן לזה ״יכא קכ״א מ״א״ עכ״ל, (הם חזרת המספרים הקודמים ממלים לאותיות), ועוד שם בהלכה ב׳, ״בין תקופה לתקופה צ״א יום וז׳ שעות ותקי״ט חלקים ול״א רגעים, סימן להם ״צא״ז תקי״ט ל״א״, עכ״ל. והפליאה בולטת, מה יתרון לסימנים על מספרים במלים שלמות באותיות בודדות המקבילות להן ואין בהן כל מובן ומשמעות, ולכאן לא שייך מש״כ למעלה בענין הסימנים דרבי יהודה, וצע״ג. —}%endcomment%
\commentb{\textrm{\textbf{תשובות לשערים פ' – פ"ג}}\textrm{\textbf{דבר אחר ביד חזקה שתים ובזרוע נטויה שתים וכו'.}}אין ספק שהדרשה הזו חששה למנין המכות בלבד, ולפי שהיו המכות עשר ומצא בכתוב חמשה שמות: יד חזקה, זרוע נטויה, מורא גדול, אותות ומופתים, חשב המגיד כיון שאלו חמשה לשונות נאמרו על כל עשר המכות יתחייב מזה שכל אחד מאלה החמשה יורה על שתים מכות ובזה הדרך יוכללו כל עשר מכות באלה החמשה לשונות. ומצא סמך לזה לפי שיד חזקה הם שתי מילות, ולכן ירמזו לשתי מכות כי מילת יד ירמוז למכת דבר ומילת חזקה ירמוז למכת בכורות, וכן בזרוע נטויה מצא שתי מילות שהן זרוע – נטויה וירמוזו לשתי מכות, ואולי היו הברד והארבה שבאה בהם התראה רכה ביד רמה וזרוע נטויה, גם במורא גדול הם שתי מילות מורא וגדול לרמוז על שתי מכות אחרות, ואולי היו הערוב והחשך לפי שלקחו מהם המצריים מורא רבה. אמנם אותות עם היותו מילה חדא הוא לשון רבים ומיעוט רבים שנים, ואולי נאמר על הדם והצפרדע כפי זאת הדרשה, וכן ובמופתים הוא לשון רבים ומורה על שתי מכות ויהיו כינים ושחין. אחרים פירשו ביד חזקה דם וצפרדע ובזרוע נטויה כינים וערוב ובמורא גדול דבר ושחין ובאותות ברד וארבה ובמופתים חושך ומכת בכורות, יהיו מה שיהיו הנה באלה החמשה לשונות יוכללו עשרת המכות כולן, ולפי שלא כיוון המגיד להזכיר כי אם המכות שהוכו המצריים במצרין לכן לא מנה בכללן הפך המטה לנחש, כי אף שהיה בכלל הניסים לא נכלל בהמכות וכן הפך יד אדון הנביאים להיות מצורעת כשלג לא נעשו לפני פרעה כי אם לפני בני ישראל ולא היתה מכה למצרים ולכן לא מנה אותם, וכמו כן לא נזכר דבר מנסי הים לפי שלא כיון המגיד כי אם לסיפור יציאת מצרים המיוחס ללילה הזה, לכן מנה זולת העשר מכות שהוכו בהן המצריים בהיותם במצרים קודם היציאה.אמנם למה עשה בהם רבי יהודה הסימנים האלה דצ"ך עד"ש באח"ב? יש מי שפירשו בעבור שדוד המלך בספר תהלים סדר המכות באופן אחר ממה שנכתבו בתורה, לכן עשה בהם רבי יהודה זה הסימן כדי שיזכרו אותם כפי שנכתבו בתורה ולא כפי שנכתבו בספר תהלים. ויש אומרים שדם וצפרדע היו בהתראה וכינים בלא התראה, וכן ערוב ודבר היו בהתראה ושחים בלא התראה וכן ברד וארבה היו בהתראה וחשך בלא התראה ולהיותם בזה הדרך שתים בהתראה ואחת בלא התראה אחריהם, לכן עשה רבי יהודה הסימנים ההם, והעשירית שהיא מכת בכורות בהתראה היתה, אבל להיותה מכה יחידה לא עשה בה סימן ונסמכה לשלש האחרונות. ויש אומרים כי דצ"ך הוא סימן לנסים שנעשו על יד אהרן ועד"ש הוא סימן לנסים שנעשו על ידי הקדוש ברוך הוא מבלי משה ואהרן ובאח"ב  הוא סימן לנסים שנעשו על ידי משה. אבל אין דעתי נוטה עם הדעה הזאת כי היות השלשה מכות הראשונות דצ"ך נעשו על ידי אהרן היה זה מחכמת השם יתברך לפי שהיה יודע שהחרטומים יערערו עליהם כי הם עשו הדם והצפרדעים גם כן, ובכינים השתדלו לעשות ולא עלה בידם, ולא רצה הקדוש ברוך הוא שהנסים הנעשים על ידי משה רבינו עליו השלום יעשו החרטומים כמותם או ישתדלו לעשותם, ולכן הניח לעשות שלש המכות האלה על ידי אהרן. אבל במכת שחין שיחסו לה' מצינו שזרק משה פיח הכבשן השמימה והיה שחין, ומכת בכורות הקדוש ברוך הוא עשאה בעצמו ולא משה.אמנם הנכון בעיני הוא שמצרים היתה עיר מלאה גילולי הכפירות בשרשי אלהותו יתברך כמו שהזכרתי למעלה, ויש מהם שלא היו מאמינים במציאות סבה ראשונה עילת העילות, ומהם האמינו במציאותו ובל לא האמינו בהשגחתו בעולם השפל, ומהם היו חושבים שהיה יכלתו יתברך מוגבל והוא בעל תכלית כשאר שרי מעלה, ושלא היה כחו בלתי בעל תכלית, ולכן תמצא שמשה רבינו עליו השלום בתחילת שליחותו הזכיר לפרעה שלשת העיקרים האלה, באמרו "כֹּה אָמַר ה' אֱלֹהֵי יִשְׂרָאֵל שַׁלַּח אֶת עַמִּי וְיָחֹגּוּ לִי בַּמִּדְבָּר" (שמות ה', א'), כי באומרו "כה אמר ה'" ביאר לו מציאות סבת הסבות ועילת העילות יתברך, ובאומרו אלוהי ישראל ביאר לו שזה המחוייב המציאות הוא משגיח על ישראל בפרט כי בחר בהם בבחינתו בין הנמצאים עליונים ושפלים, ובאמרו "שלח את עמי" ביאר יכלתו יתברך הבלתי תכלית, לא בלבד על אומה אחת הוא שר כשאר שרי האומות כי אם לעל יושבי תבל ושוכני ארץ, ולכן היתה מצוה עליו אף על פי שלא נחשב בין בני ישראל. והנה לא היתה תשובת פרעה בתחילת דברי פיהו סכלות כי אם כפירה מוחלטה, באומרו "מִי ה' אֲשֶׁר אֶשְׁמַע בְּקֹלוֹ לְשַׁלַּח אֶת יִשְׂרָאֵל לֹא יָדַעְתִּי אֶת ה' וְגַם אֶת יִשְׂרָאֵל לֹא אֲשַׁלֵּחַ" (שם שם, ב') כי באומרו "מי ה'" כפר בעיקר רצה לומר במציאותו, ובאומרו "אשר אשמע בקולו" כפר בהשגחתו, כלומר אין לו קול ולא השגחה וצווי בדברים השפלים, ובאומרו עוד "לא ידעתי את ה' וגם את ישראל לא אשלח" כפר בפנת היכולת המוחלטת, כאילו אמר אף שנודה שהוא נמצא אין ראוי שנאמין שהוא משגיח בדברים אשר בכאן בעולם השפל, ואם נאמין בהשגחתו הרי הוא כאחד השרים העליונים שיכולתו רק על אומה מיוחדת ואין לו יכולת על שאר האומות, ולכן אם הוא אלוהי ישראל אינו אלוה שלי ואין יכולתו עלי, וזהו "לא ידעתי את ה' וגם את ישראל לא אשלח".  ומפני ששלושת הכפירות נפלאו ממנו לכן באו המכות מכוונות לאַמֵת הפינות האלהיות האלה, ולכן תמצא כי בתחילה אמר משה "בְּזֹאת תֵּדַע כִּי אֲנִי ה'" (שם ז', י"ז) שהוא מציאות הסיבה הראשונה ועשה לאמת הפינה הזאת שלוש מכות דם צפרדע כינים, לפי שעל פי שלושה עדים יקום דבר, ומפני זה עשה ר' יהודה באלה שלוש מכות סימן בפני עצמו דצ"ך, להיותן מכוונות שלשתן לאמת הפינה הזאת. ובתחילת המכה הרביעית שהיא ערוב אמר "לְמַעַן תֵּדַע כִּי אֲנִי ה' בְּקֶרֶב הָאָרֶץ" (שם ח, י"ח) רצה לומר יודע ומשגיח על דרכי בני אדם, ולאמת הפינה השניה הזאת באו שלוש מכות אחרות שהן ערוב דבר שחין, ולהיותן שלושתן מכוונות לתכלית המיוחד הזה עשה רבי יהודה בהן סימן בפני עצמו עד"ש, אחר כך בתחילת המכה השביעית שהיא ברד אמר "בַּעֲבוּר תֵּדַע כִּי אֵין כָּמֹנִי בְּכָל הָאָרֶץ" (שם ט', י"ט) שהוא פינת היכולת, ועליה באו ארבע מכות האחרונות ברד ארבה חשך מכת בכורות, ולהיותן מכוונות לתכלית אמיתת העיקר עשה רבי יהודה בהן סימן אחר באח"ב.הנה התבאר שרבי יהודה בחכמה יסד הסימנים האלה להעיד על שלשת הפינות שהיו התכליות במעשה המכות כפי חלוקתן זאת, ולפי שהיה הדבר מרגלתא בפומיה דר' יהודה היה נותן בהן סימנים.והתבארו עם זה הספקות אשר בשערים פ', פ"א ופ"ב.אמנם למה היו המכות במצרים אלו העשר שנזכרו בכתובים ולא היו אחרות, על זה ראוי לומר דבר מספיק המתישב על הלב. כי אן ספק שלא היה במקרה ובהזדמן, ולפי שהפעולות האלוהיות נהירו ושכלתנו וחכמה יתירא אשתבחת בהון, הן בכמותן והן באיכותן. ולכן מהראוי לבקש תכלית וסיבה שהיו המכות האלו ולא אחרות. וכבר התעוררו לזה חז"ל אבל נתנו לזה טעמים חלושים, כי אמרו שבאה מכת הדם מפני שלא הניחו נשי ישראל להטהר מדמיהן, ומכת צפרדע לפי שלא היו יכולות הנשים להרים קול בשעת לידתן מפחד המצריים לכן שלח להם הקדוש ברוך הוא צפרדעים שהיו נותנים עליהם קולות בבתיהם ובמשכבותם, ומכת כינים מפני שגזלו מישראל את אדמתן והיו כנים על פני הארץ כדי לבטלם ממלאכתם, ומכת ערוב מפני שהיו ישראל רועים מקניהם של מצרים ונשיהם מניקות את בני המצריים והיה בא הערוב אל בית היהודי ולוקח משם את בן המצרי ומוליכו ואוכלו, ומכת דבר מפני שהיו חורשים המצריים על גבי היהודים שנאמר "עַל גַּבִּי חָרְשׁוּ חֹרְשִׁים" (תהלים קכ"ט, ג') כדי שלא יטרחו מקניהם בא הדבר והמית המקנה שלהם, ומכת שחין כדי שיהיו בני ישראל מתקבצים במרחצאות המצריים לרווחה הוכו המצריים בשחין, שאדם שיש לו שחים לא יכנס למרחץ.  ומכת חשך כדי שבאותם ימי החשך ימותו פושעי ישראל ויקברו אותם ולא יראו המצריים שלא יאמרו כמו שתבע בנו כל תבע בהם, ועוד כדי שבימי החשך יחפשו בני ישראל בבתי המצריים ויראו את אשר להם וישאלו מהם אחר כך ולא יוכלו לומר שאין להם מה להשאיל, ומכת בכורות כי באותה מכה התחיל להתרות בהם שנאמר "כה אמר ה' בני בכורי ישראל ואומר אליך שלח את עמי ויעבדוני ותמאן לשלחו הנה אנכי הורג את בנך בכורך" (שמות ד' כ"ב-כ"ג), ובה כלה שנאמר "ויהי בחצות הלילה וה' הכה" וגו'. זוהי דרכם ז"ל.ומה שנראה לי בזה הוא שבענין מצרים עשה הקדוש ברוך הוא נסים ונפלאות לתכליות מחולפים, כי עשה ראשונה נס של הסנה שראה משה בפעל ובהקיץ כי הסנה בוער באש ואיננו אוכל, והנס הזה היה למשה בלבד כדי שלא יירא ולא יחת מלכת אל פרעה ולהתרות בו, פעמים לחרפו ולגדפו ולהכותו מכות רעות ונאמנות, כי השם יתברך שלח מלאכיו לסגור פום אריותא פרעה וחבריו שלא יחבלוהו. ולכן הראה לו הסנה שהוא עץ שפל ונבזה משל למשה והראהו בוער באש ואיננו אוכל, כך יהיה ענינו של משה שבהיותו במצרים לעשות המכות לא יאכלהו אש חמת המלך ורשעתו, ומפני שראה השם יתברך כי אדוננו משה משתומם על המראה, צוהו של נעלך מעל רגלך, לפי שהמנעלים נעשו כדי להגן על הרגל ולהצילו מהפגעים והנזקין, וכאילו אמר אל תרא ואל תחת פן תגוף באבן רגלך, של נעליך כי אינך צריך אליהם לשמירה, אף שעל שחל ופתן תדרוך תרמוס כפיר ותנין, ולפי זה נעשה הנס הזה למשה בלבד כדי להבטיחו על ענינו. עוד נעשו ניסים אחרים כנגד העם, כמו שאמר "מה זה בידך ויאמר מטה" וגו'. והזכיר שלושה נסים: א' הפך המטה לנחש, ב' צרעת היד ורפואתה, ג' הפך המי לדם. ועליהם אמר "והיה אם לא יאמינו לך ולא ישמעון לקול האות הראשון והאמינו לקול האות האחרון, והיה אם לא יאמינו גם לשני האותות האלה ולא ישמעון לקולך ולקחת ממימי היאור ושפכת היבשה והיו המים אשר תקח מן היאור והיו לדם ביבשת" (שם שם ח'-ט'). וראוי לדעת אם היו שלושת הנסים האלה כנגד העם להבטיחם ולאמת שליחות משה, למה יוחדו שלשת אלה ולא זולתם, ולמה אמר "אם לא יאמינו לקול האות הראשון והאמינו לקול האות האחרון", מה היפוי כוח שמצא בנס השני יותר מהראשון שיאמר עליו כן? ומה שראוי לומר בזה הוא שהיציאה ממצרים והגאולה אוי שתקשה בעיני בני ישראל משלוש בחינות: האחת מפאת פרעה לרשעתו ורוע לבבו שהיה כמו פתן חרש יאטם אזנו (תהלים נ"ח, ה') ולא שמע לקול משה ואהרן, והבחינה השניה מפאת ישראל שהיו שקועים בגלות כל כך כאילו כבר התיאשו מהגאולה והתשועה, לפי שהיה חולי חזק מאוד ובלתי מקבל רפואה. והבחינה השלישית מצד עם מצרים שהיו מחזיקים בישראל לעדים ומתגברים עליהם בחזקת היד. והודיעם השם יתברך  שהשלושה המה שנפלאו מהם נקל בעיני ה' להחזירם לרצונו, ולכן בא ראשונה גם המטה כנגד פרעה שנקרא בדברי הנביא "נחש בריח", וכבר כתב הראב"ע שמלשון "ויקם מלך חדש על מצרים" (שם א', ח') נראה שלא היה פרעה מזרע המלוכה, אלא הוא קם מעצמו כמו כי הקים בני עבדי, ולכן המשילו במטה שהיה עץ יבש וכשהשליכו ארצה היה לנחש, כן פרעה בראשונה היה עץ יבש והומלך בחוזקה ונעשה נחש בריח, וינס משה מפניו ובאמת ברח מפני פרעה אל מדין. וצוה ה' למשה שעם היותו מלך אל יירא מפניו מלהכותו, וזהו "שלח ידך ואחוז בזנבו" (שמות ד', ד') רצה לומר שישוב לשפלותו, וכמו שנאמר "ויהי למטה בכפו", להורות על שפלותו על ידי כף משה. והנה אמר בנס הזה "למען יאמינו כי נראה אליך ה' אלוהי אבותם" (שמות ד', ה') להגיד כי הוא יפקוד על מלכי האדמה באדמה בזכות אבותיהם. ובא הנס השני מצרעת היד בבחינת האומה, כי כמו שהיו בני חורין וטובים עם ה' ועם אנשים אשר באו מצרימה ורוע המקום עשאם עבדים וכנענים אנושים ונחלשים, כך היתה ידו של משה בריאה וטובה ואך בהביאה אל חיקו שהוא משל למצרים נעשתה שמה מצורעת כשלג שהוא החולי הבלתי מקבל רפואה והוא משל להתיאשם בגלות.והנה צוהו ה' "השב ידך אל חיקך" להודיעו שכמו שהיד באותו מקום שנצטרעה נתרפאה ככה בני ישראל במצרים נשתעבדו ובמצרים יהיו בני חורין, כי ה' יתברך מחץ מכתו ירפא, וכמו שאמר "מחצתי ואני ארפא" (דברים ל"ב, ל"ט), ולפי שבריאותם תהיה יחד עם היציאה משם, לכן אמר "ויוציאה מחיקו והנה שבה כבשרו", כי עם היציאה ממצרים תהיה רפואתם בשלמות ושבו בנים לגבולם. אמנם בעבור היות העם בלתי ראוי ומוכן לקבל רפואת הגאולה היתה סיבה יותר חזקה למניעת פרעה מרשעתו, לכן אמר "והיה אם לא יאמינו לקול האות הראשון" וגו' לפי שתרופת האומה וגאולתה היא הוראה יותר חזקה מהכנעת פרעה. ולפי שעדיין ישאר הספק מצד המצריים והשפעת אלהיהם מהפכת המים לדם אשר הם בוטחים בם, צוהו בנס השלישי שייעד שירבה שפיכת הדם בכל ארץ מצרים במכת בכורות כמים לים מכסים, ומתוך כך ישלחו אותם. הנה אם כן באו שלשת הנסים האלה כנגד העם להסיר כל ספק מליבם, ולכן אמר "ויעש האותות לעיני העם ויאמן העם וישמעו כי פקד ה'" (שמות ד', ל"א) וגו', כי ענין האותות והוראתם הוא לאמת הדבר ההוא בלבם. וראיתי מי שכתב שהיו שלושת הנסים האלה זה גדול מזה, לפי שהדבר אשר יאמר עליו שהוא נסיי באחד משלושה פנים הגדול מהם הוא יהיה נמנע אצל הטבע בהחלט, כמו שיהפוך הברזל גזת הצמר או אפרוח לדבר אחר, או שיהיה נמנע אצל הטבע בזמן מה ולא יהיה נמנע בהחלט כמו שיהפך קב תמים לאפרוח שהוא נמנע אצל הטבע להיות פתאום אבל אפשר לאחר זמן שיהיו החטים מאכל לעוף השמים ותתהוה מהם הטפח הזרעית ויולד ביצה וממנה אפרוח, או שיהיה הנס מורכב משני ענינים יחד בהתהוות מביצת התרנגול חצי תרנגול וחצי צפרדע שהוא נמנע מצד כללותו. וכבר נתבאר בחכמת הטבע שלא כל דבר שיזדמן יתהוה מאיזה דבר שיזדמן, אבל יפול בזה צד אפשרות מצד קצתו. והנה מבואר כי אות המטה שנהפך לנחש היה המנעותו מצד הזמן, כי המטה כשיותך לעפר ונחש עפר לחמו אפשר שיתהפך זה לזה באורך הזמן, והיה הנס בו שנעשה כן בזולת זה הזמן.והמופת השני היה קצתו נס מוחלט וקצתו לא והוא שנעשה האדם בריא וחולה בזמן אחד, כי בדרך הטבע אי אפשר להיות הדם ידו מצורעת רק בשיהיה הוא עצמו מצורע בהמצא בו הפסד כוח המשנה בכלל, ואם שיוכר זה בקצת חלקיו וראשי אבריו לא יאמר מפני זה שידו או רגלו או חוטמו מצורעת והוא בריא, כי אם הוא בריא וכוחו שלם בכל פעולותיו ואבר אחד מצורע הוא כמו חצי תרנגול וחצי צפרדע, ויהיה נמנע אצל הטבע יותר מהראשון. אולם לפי שכבר אפשר שתפול הצרעת ביד בצד מן הצדדין כמו שאמרתי, לכן בא האות השלישי שהוא מהמין הראשון שנמנע להיות בהחלט בשום זמן, כי ההפך המים שהוא יסוד פשוט לדם הוא כמו ההפך הברזל לאפרוח, כי אף שגוף האדם יצטרך לשתיית המים אינו להזנה כי אם להעביר המאכל במעברים הצרים, כי היסוד הפשוט אי אפשר שיהפך מזון וכל שכן שיתהוה ממנו מיטב הלחות שבניזון. וכל שכן שיתהוה הדם, חוץ מן הכבד שהוא מקום הולדתו הטבעי, ודומה להאפשרות שיעשה מהבן לב או כבד או מוח נפרדים מכלל הגוף שהוא בתכלית הנמנעות ולהיות שלושה המופתים מודרגים מן הקל אל הקשה לכן אמר "אם לא יאמינו" וגו'. ולפי זה הניסים האלה היו כנגד העם לאמת נבואתו של משה. ונעשו עוד ניסים אחרים כדי להכות בהם את המצריים, ולכן נקראו מכות, והן אותן העשר הכתובות בתורה. ועניינן אצלי שבאו כפי ארבעת היסודות שמהם הורכבו הדברים השפלים ומהם נתקיימו. שנים מהם כבדים – הארץ והמים, ושנים מהם קלים האוויר והאש, ולכן נאמר ראשונה שהכה את המים ונהפכו לדם, ובזה היה צער גדול אליהם כי שתיית המים היא הכרחית להשקיט תגבורת החום אשר מבפנים וכדי לדקדק את המזון ולהעבירו במעברים הצרים, ובהתהפכם לדם לא יכלו מצרים לשתות מים. וגם הבעלי חיים הנולדים במים שהיו אוכלים מהם – מתו, כמו שנאמר "והדגה אשר ביאור מתה" (שמות ז', כ"א). ולא די שמתו הבעלי חיים הימיים המועלים להם, אלא שגברו חיל ועצמו הבעלי חיים המזיקים אשר במים שיצאו להרבה לצערם בקולם וריבוים, והיא היתה מכת הצפרדעים. ואחר שהכה יסוד המים בשתי המכות האלה הכה יסוד הארץ בשלושה אותות אחרים, אם בכינים שכל עפר הארץ היה כנים, אם בערוב ואם בדבר לפי שהביא הקדוש ברוך הוא מהבעלי חיים הנולדים בארץ והם כלל הבעלי חיים המזיקים מעורבים עם אלו שרצים וחיות טורפות ובהמות רעות וכל מיני מזיקים והם הערוב. והבעלי חיים הביתיים הטובים והמועילים הנולדים בארץ המית הדבר. הרי לך שלשת המכות האלה כנגד יסוד הארץ בעצמותה ובעלי חיים, הנולדים בה. ולפי זה חמשת המכות דם צפרדע כנים ערב דבר הם כנגד שני היסודות הכבדים המים והארץ. עוד הביא בשני יסודות הקלים אויר ואש חמש מכות אחרות  כי השחין נעשית באויר שנאמר "ויזרוק אותו משה ויהי שחין אבעבועות" (שם ט', י'), והברד היה ביסוד האש כמו שנאמר "ויט משה את מטהו על השמים וה' נתן קולות וברד ותהלך אש ארצה" (שם שם, כ"ג) וגו', רצה לומד שהיסודות הקלים המתנועעים כפי טבעם למעלה הביאם למטה שהוא תנועה הפך טבעם. והביא גם כן הארבה שהוא מבעלי חיין המעופפים באויר כדי שיוזקו בהם. וכן החשך שהיתה מכה באויר כי הוא זך בטבעו ונתחדש בו עובי כפול ומכופל באופן שלא יכול לעבור בו ניצוץ השמש על הארץ. וכן מכת בכורות ששלח בגופים אש שורף שהתיך רוחם וכחם פתע פתאום. הרי לך חמשת המכות האחרונות האלה שהיו בשני היסודות הקלים האויר והאש, ובזה האופן נתחייבו עשרת המכות לפי ארבעת היסודות הנלקים. וראוי היה שיהיו המכות עשרה לפי שזה המספר קדוש ומקודש, והיו טבעי הדברים השפלים כנגדם והגלגלים עם גלגל השכל עשרה, והשכלים הנבדלים עם סבתם השם יתברך הם עשרה או עשרה מהמאות או מהאלפים כמו שאמר "אֶלֶף  אַלְפִין יְשַׁמְּשׁוּנֵּהּ " (דניאל ז', י'), ולכן נברא העולם בעשרה מאמרות, ועשרה דורות מנח עד אברהם, ונתנסה אברהם בעשרה נסיונות, ושאר העשיריות שהזכירו חז"ל. וזהו הדרך הראשון בביאור הדרוש הזה.והדרך השני הוא שנאמר שבאו המכות מדה כנגד מדיה, כפי הצרות והרעות שקיבלו ישראל ממצרים שהם עשרה מיני רעות רבות וצרות:א' – מיתת בניהם במימי היאור כמו שנאמר "כל הבן הילוד היאורה תשליכוהו", וכאילו היה יאור מצרים מלא דם בניהם ועל הגמול הרעה הזאת באה מכת הדם, והדגה אשר ביאור מתה.ב' – צעקת בנות ישראל ודמעתן על לחיין והספדן תמיד על היאור שהיה מכלה פרי בטניהן ולהגמול הרעה הזאת באה מכת הצפרדעים, כאילו הם יצערו וילילו על ילילת בנות ישראל, ויצאו מן היאור כי משם יצאה היללה והצעקה.ג' – שמררו את חייהם בעבודה קשה בחומר ובלבנים, לפי שנתנו עמלם ויגיעם בעפר האדמה באה מכת הכנים כאשר הכה משה במטהו בעפר הארץ, כאילו הכה ויקלל אותו העפר שהיה לבני ישראל עמל וכעס, ולכן נהפך למצרים לכנים ולצנינים בצידם.ד' – לא לבד היו עובדים בעבודת המלך אלא גם כל אחד מן העם היה עושה בהם כרצונו, ולתגמול זה באה מכת הערוב שבעלי החיים כקטן וכגדול עלו על הארץ ונכנסו בבתים והיו טורפים ומחזיקים בהם כמעשה המצריים בבני ישראל.ה' – כי ישראל מפני עבדות המצריים ועמלם לא היו רועים את צאנם ובקרם מפני טורח העבדות, וגם היו המצריים גוזלים ממקנה ישראל כרצונם, ובעבור זה באה מכת הדבר במקנה מצרים, וממקנה ישראל לא מת אחד.ו' – זו פרישות דרך ארץ שבסבת עבודתם היו נפרשים בני ישראל מנשותיהם כאילו היו מוכי שחין, שהאשה נפרשת מבעלה שהוא מוכה שחין, על כן באה על המצריים מכת השחין שנעשו כמצורעים עם אבעבועות ושחין פורח שלא היו יכולים להזדקק אל נשיהם.ז – לפי שהיו המצריים בחוץ וברחוב מכים בישראל באגרוף וזורקים בהם אבנים, והיו לועגים להם בגידופים ונותנים עליהם בקולם, לכן הביא עליהן מכת הברד שהיו אבנים מן השמים להומם ולאבדם, ואל הכבוד הרעים עליהם בקולות וברקים.ח' – לפי שהיו מישראל עובדי אדמה למצוא אוכל לנפשם והמצריים היו גוזלים תבואותיהם אשר בשדה לכן הביא עליהם מכת הארבה שהיה אוכל כל תבואותיהם אשר להם בשדה.ט' – לפי שהיו ישראל תמיד בחשך הגלות, וכמו שאמר הנביא על צרת הגלות "בְּמַחֲשַׁכִּים הוֹשִׁיבַנִי כְּמֵתֵי עוֹלָם" (איכה ג', ו'), ובהפך תמיד יתאר הגאולה וההצלחה כאור כמו שנאמר "קוּמִי אוֹרִי כִּי בָא אוֹרֵךְ" (ישעיהו ס', א'), "לַיְּהוּדִים הָיְתָה אוֹרָה וְשִׂמְחָה" (אסתר ח', ז'), לכן באה עליהם מכת החשך, להעיד שהנה החשך יכסה ארצם לכל בני ישראל יהי אור במושבותם הפך חשך הגלות שהיו בו ונהפך בשונאיהם.י' – שהיו עם אלהי אברהם בגלות ונפרדו מעל שולחן אביהם ואכזרים ימשלו בם, וכנגד זה באה מכת בכורות, וכמו שהמצריים השמידו בניו של הקדוש ברוך הוא כן הוא יתברך הרג בכוריהם ואת בני ישראל לקח לו לעם. וגם בזה הדרך תדע ותשכיל שהיו המכות גמול אלהים ובאו מסודרות ומכוונות כפי בצרות והרעות אשר קבלו ישראל ממצרים.ובמה שפירשתי הותר הספק אשר העירותי עליו בשער פ"ג.}%endcomment
\hebeng{רַבִּי יוֹסֵי הַגְּלִילִי אוֹמֵר: מִנַּיִן אַתָּה אוֹמֵר שֶׁלָּקוּ הַמִּצְרִים בְּמִצְרַיִם עֶשֶׂר מַכּוֹת וְעַל הַיָּם לָקוּ חֲמִשִּׁים מַכּוֹת? בְּמִצְרַיִם מַה הוּא אוֹמֵר? וַיֹּאמְרוּ הַחַרְטֻמִּם אֶל פַּרְעֹה: אֶצְבַּע אֱלֹהִים הִוא, וְעַל הַיָּם מָה הוּא אוֹמֵר? וַיַּרְא יִשְׂרָאֵל אֶת־הַיָּד הַגְּדֹלָה אֲשֶׁר עָשָׂה ה׳ בְּמִצְרַיִם, וַיִּירְאוּ הָעָם אֶת־ה׳, וַיַּאֲמִינוּ בַּיי וּבְמשֶׁה עַבְדוֹ. כַּמָה לָקוּ בְאֶצְבַּע? עֶשֶׂר מַכּוֹת. אֱמוֹר מֵעַתָּה: בְּמִצְרַים לָקוּ עֶשֶׂר מַכּוֹת וְעַל הַיָּם לָקוּ חֲמִשִּׁים מַכּוֹת. }{Rabbi Yose Hagelili says, "From where can you {[derive]} that the Egyptians were struck with ten plagues in Egypt and struck with fifty plagues at the Sea? In Egypt, what does it state? 'Then the magicians said unto Pharaoh: ‘This is the \textit{finger} of God' (Exodus 8:15). And at the Sea, what does it state? 'And Israel saw the Lord's great \textit{hand} that he used upon the Egyptians, and the people feared the Lord; and they believed in the Lord, and in Moshe, His servant' (Exodus 14:31). How many were they struck with with the finger? Ten plagues. You can say from here that in Egypt, they were struck with ten plagues and at the Sea, they were struck with fifty plagues."}%
\commentb{\textrm{\textbf{תשובות לשערים פ"ד – פ"ו}}\textrm{\textbf{ר' יוסי הגלילי אומר מנין אתה אומר שלקו המצריים במצרים עשר מכות וכו'.}}בפרק ערבי פסחים הזכירו חז"ל ההגדה ויציאת מצרים אשר חייב על כל איש לספר בלילה הזה, והביאו פרשת וידוי הבכורים כלה בדרשותיהם עד רבי יהודה היה נותן בהם סימנים דצ"ך עד"ש באח"ב, והזכירו מיד אחרי זה רבן גמליאל אומר כל מי שלא אמר שלשה דברים אלו בפסח וכו' ואחריו ההלל וברכת אשר גאלנו, ולא הביאו שמה דבר מדרשת ר' יוסי הגלילי ורבי אליעזר ורבי עקיבא שעשו יסוד ועיקר כל דרשותיהם על ניסי הים.אמנם החכמים מסדרי המכילתא נוכחו לאות כי ראוי להכניס המאמרים האלה הערבים בדרשות עניני מצרים ומפליגין במעשה ה' כי נורא הוא, וכל שכן בדברי השלם רבי עקיבא. ובמאמר כמה מעלות טובות נכללו החסדים כולם אשר יתחברו ביציאת מצרים ושנמשכו אחריה, ומפני שעניניהם ונפלאותיהם הם קשורים עם מופתי מצרים ראויים הם שיעלה זכרונם על הספר, ולכן הוסיפו המאמרים האלה בטופס ההגדה, וגם אנחנו היום אומרים אותם, ונזכרים אחרי זיכרון מכות מצרים, כי מהם עשו התחלה ראיה וטענה על מופתי הים, ובאחרונה בא מאמר רבן שמעון בן גמליאל שהוא היותר הכרחי כמו שיתבאר להלן.והנה אחד שמנה המגיד העשר מכות הביא עליו דברי רבי יוסי הגלילי "מנין אחת אתה אומר" וכו', ושיעור מאמרו כך הוא: אתה אומר שלקו המצריים במצרים עשר מכות מנין לנו שעל הים לקו חמישים מכות? כי לא ישאל רבי יוסי ולא ידרוש להוכיח שלקו במצרים עשר מכות כי זהו דבר ידוע לכל, אלא בא להוכיח על מכות הים שהיו חמישים ומביא ראיה מפסוק "אצבע אלוהים היא", ואף שאותו פסוק הוא במכת כינים לדעתו לא אמרו אותו החרטומים על הכינים בלבד אלא על כל המכות שנמנו וגמרו שהן אצבע אלהים, כי באות המטה ומכת הדם ומכת הצפרדעים התאמצו החרטומים לעשות כמותם כדי להראות שהיו דברים מלאכותיים הנעשים בתחבולה כדרכם ובעלילותם ולא היו נפלאות אלהיות, אבל במכת הכינים שלא יכלו לעשות כמתכונתה, החליטו שכל המעשים שעשו משה ואהרן מתחילתם ועד סופם הם מעשה אלהים ואמרו אצבע אלהים היא, ולכן לא השתדלו עוד החרטומים לעשות כמותם ולא שאל אותם פרעה עוד לעשות גם הם בלטיהם, כי כבר קיימו וקבלו שהן פעולות אלהיות כולן הראשונות והאחרונות. ומזה הוכיח רבי יוסי הגלילי מה שאמרו אצבע אלהים היא על עשר המכות, ולפיכך דרש אם כל מופתי מכות מצרים מתוארות באצבע להורות על מעלת הפעול יתברך כאילו כולן נעשו באצבע קטנה מאצבעותיו, אם כן במופתי הים שנאמר "וירא ישראל את \textrm{\textbf{היד}} הגדולה" נתחייב לומר שהיה ערכם לערך מופתי מצרים כמו היד לעומת האצבע, ואם באצבע הכח עשר מכות ראוי לומר שהכח ביד שיש בה חמש אצבעות חמישים מכות כי חמש פעמים עשר הם חמישים. ואתה תראה בענין לוחות הברית שהיו כתובים בהם עשרת הדברות ונאמר בהם כתובים באצבע אלהים ועל הגרמים השמיימים אמר המשורר "כי אראה שמיך מעשה אצבעותיך" (תהלים ח', ד') וגו', וכבר ראיה דברי הרב המורה שהיו כדורים חמישים, והסכמות הידיעות האלה מורה על התקשרות הדברים העליונים עם התחתונים, והנה הרב דוד אבודרהם הקשה על זה המאמר כי במכת דבר נאמר גם כן "יד ה'", הרי שעשה ביד ולא באצבע, והשיב עליו שהפסוקים אינם אלא אסמכתא בעלמא והקבלה היא העיקר. ואני אומר שרבי יוסי הגלילי סבר שמילת יד היא משותפת בהוראותיה כי נאמר על יד על המקום כמו "יד הירדן" "ויד תהיה לך על אזנך" ונאמר על המגפה כמו "ושלחתי את ידי והכיתי את מצרים" ונאמר על היד הגשמית הכוללת חמש אצבעות. ונראה לפרש "הנה יד ה' הויה במקנך| על המגפה לפי שהכתוב קראה דבר אבל מה שנאמר על הים "וירא ישראל את היד הגדולה" מלה גדולה מורה שנאמר יד הנחתה הגשמית ולכן היה מקום לדרשתו.והותרו בזה הספקות בשערים פ"ד, פ"ה, פ"ו.אמנם קשה לנו מאוד במאמרי האנשים ההם השלמים אתנו האומרים שעל הים לקו חמישים מכות לדעת ר' יוסי הגלילי ומאתיים מכות לדעת רבי אליעזר ומאתיים וחמישים מכות לדעת רבי עקיבא, כי איך ציירו אותו המספר הגדול מהמכות שעל הים? ומשנה מפורשת היא באבות פרק ח': "עשרה ניסים נעשו לאבותינו במצרים \textrm{\textbf{ועשרה על הים}}", ומה יעשו בדרשותיהם הסותרות משנה זו? גם אשאלם ויודיעוני במה ישתנו המכות כולם אלו מאלו, ואם היו מתיחסות במספרם עם מכות מצרים האם היו גם כן מתיחסות אליהן באיכותן או שהיו מכות הים ממין אחר מתחלף ממכות מצרים?ומה שנראה בזה הוא שהחכמים שלשתם הבינו אותה משנה באבות על הניסים שנעשו \textrm{\textbf{לישראל}} בהצלתם במצרים או על הים ולא על המכות שלקו \textrm{\textbf{המצריים}} לא במצרים ולא על הים, וכן פירש הרמב"ם שם במשנה. אמנם העשרה ניסים שנעשו לישראל המצרים הם מה שנצולו מן העשר מכות, והיות כל מכה ומכה מהן מיוחדת למצריים ולא לישראל, והם ניסים בלא ספק. וכן משמע מלשון התורה בכל מכה ומכה מלבד מכת הכנים שלא נתברר בה אבל כן מקובל באומה. כי נאמר בפירוש בדם "ולא יכלו מצרים לשתות מים מן היאור" להורות שהנס היה לישראל שלא הורע להם. ונאמר בצפרדע "ובאו בביתך ובחדרי משכבך", ונאמר בערוב "והפלתי ביום ההוא את ארץ גושן", ונאמר בדבר "וממקנה בני ישראל לא מת אחד", ונאמר בשחין "כי היה השחין בחרטומים ובכל מצרים", ונאמר בברד "רק בארץ גושן אשר שם בני ישראל לא היה ברד", ונאמר בארבה "ויעל הארבה על כל ארץ מצרים", ונאמר בחושך "ולכל בני ישראל היה אור במושבותם". וכל זה מורה שלא היו המכות בישראל. אמנם העשה ניסים שנעשו לישראל על הים הם קבלה, ואלו הם:א – הובקעו המים כפשוטו בפסוק "ויבקעו המים" (שמות י"ד, כ"א).ב – אחר שבאו נעשו כקובה ועליו כדמות גג לא מקופח ולא משופע, כי היה כל הדרך בים כאילו היה בו נקב וחלול והמין מימין ומשמאל וממעל, וזהו מאמר חבקוק: "נָקַבְתָּ בְמַטָּיו רֹאשׁ  פְּרָזָיו" (חבקוק, ג', י"ד).ג – קרקע הים נתקשתה ונקפה להם שנאמר "ובני ישראל הלכו ביבשה בתוך הים" ולא נשאר בקרקעיתו שום חומר וטיט כבשאר נהרות.ד – כי דרכי מצרים היו בחומר מדובק והוא מה שנאמר "חֹמֶר מַיִם רַבִּים" (שם שם, ט"ו).ה – שנבקע הים לדרכים רבים כמספר השבטים כעין קשה ועגול כמו שנאמר "לְגֹזֵ֣ר יַם־ס֭וּף לִגְזָרִ֑ים" (תהלים קל"ו, י"ג).ו – שנקפאו המים ונתקשו כאבנים שנאמר "שִׁבַּ֖רְתָּ רָאשֵׁ֥י תַ֝נִּינִ֗ים עַל־הַמָּֽיִם" (תהלים ע"ד, י"ג) רצה לומר שנתקשו המים עד שנעשו חזקים שאופן שיברו הראשים עליהם.ז – שלא נקפאו כקפיאת שאר המים הנקפאים רצה לומר להיות חתיכה אחת אלא היו חתיכות רבות כאילו היו אבנים מסודרים קצתם על קצתם והוא מה שאמר "אַתָּ֤ה פוֹרַ֣רְתָּ בְעׇזְּךָ֣ יָ֑ם" (שם.).ח – שנקפאו כזכוכית והיו בהירים כשוהם כדי שיראו קצהם את קצהם בעת עברם בים כמו שנאמר "חֶשְׁכַת מַיִם עָבֵי שְׁחָקִים" (תהלים י"ח, י"ב), רצה לומר קיבוץ המים היה כעצם השמים לטוהר שהוא בהיר.ט – שהיו נוזלים מהם מים מתוקים והיו שותים אותם.י – שהיו נקפאים בעת שהיו נוזלים אחר שלקחו מהם מה ששתו בטרם ירדו לארץ וכמו שנאמר "נצבו כמו נד נוזלים", רצה לומר הדבר הנוזל היה נקפא בלב הים.ומצאנו בקבלה גם כן שבאו על המצריים מכות בים יותר ממכות מצרים, אבל כולן היו מעשרה המינים אשר לקו במצרים ונתחלקו למינים רבים על הים ועל זה רמז באומרו אלה הם האלהים המכים את מצרים בכל מכה במדבר, היינו במדבר ים סוף, עד כאן לשון הרב.ולמדנו מדבריו שהמשנה ההיא היה ענינה הניסים שנעשו עם ישראל בהצלתם, אבל ענין המכות שנלקו בהן המצריים במצרים ועל הים הוא דבר אחר ועליו נאמר במשנה "עשר מכות נלקו המצריים". ומורה שהעשר מכות הן כנגד המצריים והעשרה ניסים הם בבחינת ישראל וערכם, ובדרך הזה לא תפול סתירה מהמשנה לדברי השלמים האלה, עוד ביאר שהמכות שנלקו בהם המצריים כל הים היו ממין עשר מכות, והביאו לזה לפי שראה שכל החכמים האלה עשו יחס וערך גדול בין מכות מצרים למכות הים, באמרם שבעד כל אחת ואחת ממכות מצרים נעשו חמש על הים, כי היעלה על הדעת שלא יעשו היחס והערך הזה כי אם במספר שהוא דבר מקרי בהן ושלא יהיה שום הדמות ויחס בעינינם ועצמם? האם נאמר ששני מיני האדם הם כנגד שני אגוזים שנמכרים בשוק? אלא אין ספק שלא באו החכמים להעריך וליחס מכות מצרים ומכות הים כי אם להיותם בכמותן ואיכותם ומבען. ולכן היתה הדרשה "כמה לקו באצבע עשר מכות אמור מעתה" וכו', אצה למר שנעשו על הים חמישה פעמים כנגד האצבע שנעשה במצרים כי יש ביד חמש אצבעות. ומזה הוכיח שהיו מכות היום ממין מכות מצרים כיון שכולם היו אצבעות. ואולי יסמוך לזה גם כן הפסוק שאמרו הפלישתים כשבא ארון האלהים אל המחנה בימי עלי שאמרו: "אוֹי לָנוּ, מִי יַצִּילֵנוּ, מִיַּד הָאֱלֹהִים הָאַדִּירִים הָאֵלֶּה; אֵלֶּה הֵם הָאֱלֹהִים, הַמַּכִּים אֶת־מִצְרַיִם בְּכָל־מַכָּה בַּמִּדְבָּר" (שמואל א', ד' ח'). ובאמרם "כל מכה" רמזו לעשר המכות שבמצרים ובאמרם "במדבר" הכוונה על מדבר ים סוף כמו שהביאו הרב. וראוי שנפרט זה ונראה אם יצדק בכל המכות, כי הדם שנעשה מהיאור במצרים גם כן נעשה ממנו הרבה בים סוף כי נשפכו בתוכו דמים רבים מהמצריים. וכן אם במצרים עלו עליהם הצפרדעים כל שכן שהיה זה על הים מקום היותם. אם במצרים נעשו הכינים בעפר האדמה גם בבוא פרעה לרדוף אחרי בני ישראל העלו עפר על ראשם בדרך והיו לכנים. אם במצרים באו בעלי חיות מעורבים להזיק בהם גם בים באו יותר ויותר כי "שם רמש ואין מספר חיות קטנות עם גדולות". אם המצרים באה מכת השחין פורח אבעבועות על המצרים כל שכן שהיה להם שחין במים שמתים שמה נפוחים ומלאים אבעבועות. אם במצרים בא הברד והקולות גם בים המים נעשו כאבנים ונופלים עליהם כמו שאמרו חז"ל "שברת ראשי תנינים על המים" ששם היו המצריים. אם במצרים אם במצרים בא הארבה והם העופות היורדים לאכול כל שכן שהיה זה בים בהיותם מתים שם פגרים היו יורדים עליהם העופות וכמו שנאמר "וירד העיט על הפגרים". אם במצרים היתה מכת החושך כל שכן שהיה חושך למצריים בתוך הים וכמו שנאמר "ויהי הענן והחושך". אם במצרים היתה מכת בכורות הנה על הים נלקו הבכור כבכורתו והצעיר כצעירותו, נער וזקן קטון וגדול כולם ספו המו מן הבלהות.ונתבאר מזה שמכות מצרים נחשבו כאצבע אחת, ומכות הים היו חמש אצבעות, היינו תוספת מיני המכות ההם בעצמן על אחת כמה וכמה כפולה ומכופלת, ובכל זאת היו מאותו המין. והוא מה שרציתי לבאר ולכן אמרו במשנה שהוכו במצרים עשר מכות ועל הים עשר מכות רצה לומר עשרה מינים שהם סוגי המכות אך נכפלו כמה פעמים.}%endcomment
\hebeng{רַבִּי אֱלִיעֶזֲר אוֹמֵר: מִנַּיִן שֶׁכָּל־מַכָּה וּמַכָּה שֶׁהֵבִיא הַקָּדוֹשׁ בָּרוּךְ הוּא עַל הַמִּצְרִים בְּמִצְרַיִם הָיְתָה שֶׁל אַרְבַּע מַכּוֹת? שֶׁנֶּאֱמַר: יְשַׁלַּח־בָּם חֲרוֹן אַפּוֹ, עֶבְרָה וָזַעַם וְצָרָה, מִשְׁלַחַת מַלְאֲכֵי רָעִים. עֶבְרָה – אַחַת, וָזַעַם – שְׁתַּיִם, וְצָרָה – שָׁלשׁ, מִשְׁלַחַת מַלְאֲכֵי רָעִים – אַרְבַּע. אֱמוֹר מֵעַתָּה: בְּמִצְרַיִם לָקוּ אַרְבָּעִים מַכּוֹת וְעַל הַיָּם לָקוּ מָאתַיִם מַכּוֹת. }{Rabbi Eliezer says, "From where {[can you derive]} that every plague that the Holy One, blessed be He, brought upon the Egyptians in Egypt was {[composed]} of four plagues? As it is stated (Psalms 78:49): 'He sent upon them the fierceness of His anger, wrath, and fury, and trouble, a sending of messengers of evil.' 'Wrath' {[corresponds to]} one; 'and fury' {[brings it to]} two; 'and trouble' {[brings it to]} three; 'a sending of messengers of evil' {[brings it to]} four. You can say from here that in Egypt, they were struck with forty plagues and at the Sea, they were struck with two hundred plagues."}%
\commentb{\textrm{\textbf{תשובות לשערים פ"ז – צ'}}\textrm{\textbf{רבי אליעזר אומר מנין שכל מכה ומכה שהביא הקדוש ברוך הוא על המצריים במצרים היתה של ארבע מכות וכו'.}} כך היא הגירסא האמיתית בזו ההגדה כי באמת לא בא רבי אליעזר לחלוק על רבי יוסי, וגם רבי עקיבא לא חלק על רבי אליעזר אלא כל אחד ואחד מהם קיבל דעת חברו והוסיף עליו. כי המגיד הראשון ביאר שהיו מכות מצרים עשר ובא רבי יוסי הגלילי וקיבל זה ועשה ממנו הוכחה שיהיו מכות היום חמישים בחשבון האצבעות כמו שביארתי.וההקדמות האלה קיבל אותן ר' אליעזר והוסיף עליהם, שאם היו מכות מצרים עשר כדעת המגיד הראשון ומכות הים היו חמש פעמים שהם כדעת רבי יוסי הגלילי, ובא רבי אליעזר וביאר שבכל מכה ומכה בין במצרים ובין על הים נכללו ארבע מכות ויהיו אם כן מכות מצרין ארבעים ומכות הים מאתים, כי בהיות מופתי הים חמש פעמים יתירים ממופתי מצרים שמספרם ארבעים יתחייב שיהיו המכות על הים מאתים, ולכן לא אמר רבי אליעזר מנין את ה אומר שהיו מכות מצרים עשר לפי שזה כבר מנה המגיד  ורבי יהודה בסימניו, ולא אמר גם כן מנין שהיו מופתי הים חמישה נגד מופתי מצרים לפי שזה כבר ביאר רבי יוסי, אלא הוכיח רק שהיו ארבע מכות בכל מכה ומכה.וכשתתחבר הקדמתו עם הקדמות החכמים המונחות יצא מהם שהיו מכות מצרים ארבעים ומכות היום מאתים. אמנם מהו ענין ארבע מכות שהיו בכל מכה ומכה? ונוכל לומר לפי שבכל מכה היו מורכבים הארבע היסודות אם בהפסדם ואם בהתהוותם, ואחרי שלא נעשה הדבר בדרך הטבע כי אם בדרך נס יתחייב שכל אחד מהיסודות הוכו באותה מכה, ובזה האופן היתה כל מכה של ארבע מכות: אחת שנעשה ביסוד האש של אותו מורכב, ואחת שנעשה ביסוד האויר של אותו מורכב, ואחת שנעשה ביסוד הארץ שלו, ואחת ביסוד המים. כי בכל מכה ומכה נשתנה כל אחד מהיסודות אשר באותו מורכב על דרך נס. והנה הוכיח המגיד היות בכל מכה ומכה ארבע מפסוק "ישלח בם חרון אפו" וגו', והוא בספר תהלים במזמור ע"ז המתחיל "האזינה עמי תורתי",  וראה המגיד לדורשו על זה מפני שבכתובים שמה התחיל לספור המכות, באומרו "אשר שם במצרים אותותיו" וגו' "ויהפוך לדם יאוריהם" וגו' "ישלח בם ערוב ויאכל וצפרדע ותשחיתם", ואחר זה הזכיר הארבעה ויתן לחסיל יבולם ויגיעם לארבה, ואח"כ "יהרג בברד גפנם" ואחריו הדבר "ומקניהם לרשפים".וכיון שאמר אחרי זכרון המכות "ישלח בם חרון אפו" הוכיח ר' אליעזר שלא אמר על המצריים כי אם על המכות שהיו נכללים בכל מכה ארבע מכות שהם א' – חרון אפו, ב – עברה, ג – זעם, ד – צרה משלחת מלאכי רעים, כי הצרה ההיא היא משלחת מלאכי הרעים, ואין ספק שהיה ראוי לומר צרת משלחת מלאכי רעים, וכן היה ראוי לומר מלאכים רעים, אבל כהר יבוא המוכרת במקום הסמוך כמו "מי המרים", והסמוך במקום מוכרת כמו "אַל תִּתֵּן לְחַיַּת נֶפֶשׁ תּוֹרֶךָ" (תהלים ע"ד, י"ט). ויש ספרים שגורסים הדרשה הזאת באופן אחר: עברה אחת, זעם שתים, צרה שלוש, משלחת מלאכי רעין ארבע, שלא מנה רבי אליעזר חרון אפו באחת מהן לפי שסבר שהוא כולל לכל העשר מכות לגופן איצטריך, אבל הגירסא הראשונה היא יותר אמיתית אצלי.והותרו במה שאמרתי הספקות אשר בשערים פ"ז, פ"ח, פ"ט ו-צ'.}%endcomment
\hebeng{רַבִּי עֲקִיבָא אוֹמֵר: מִנַּיִן שֶׁכָּל־מַכָּה וּמַכָּה שֶהֵבִיא הַקָּדוֹשׁ בָּרוּךְ הוּא עַל הַמִּצְרִים בְּמִצְרַיִם הָיְתָה שֶׁל חָמֵשׁ מַכּוֹת? שֶׁנֶּאֱמַר: יְִשַׁלַּח־בָּם חֲרוֹן אַפּוֹ, עֶבְרָה וָזַעַם וְצַרָה, מִשְׁלַחַת מַלְאֲכֵי רָעִים. חֲרוֹן אַפּוֹ – אַחַת, עֶבְרָה – שְׁתָּיִם, וָזַעַם – שָׁלוֹשׁ, וְצָרָה – אַרְבַּע, מִשְׁלַחַת מַלְאֲכֵי רָעִים – חָמֵשׁ. אֱמוֹר מֵעַתָּה: בְּמִצְרַיִם לָקוּ חֲמִשִּׁים מַכּות וְעַל הַיָּם לָקוּ חֲמִשִּׁים וּמָאתַיִם מַכּוֹת. }{Rabbi Akiva says, says, "From where {[can you derive]} that every plague that the Holy One, blessed be He, brought upon the Egyptians in Egypt was {[composed]} of five plagues? As it is stated (Psalms 78:49): 'He sent upon them the fierceness of His anger, wrath, and fury, and trouble, a sending of messengers of evil.' 'The fierceness of His anger' {[corresponds to]} one; 'wrath' {[brings it to]} two; 'and fury' {[brings it to]} three; 'and trouble' {[brings it to]} four; 'a sending of messengers of evil' {[brings it to]} five. You can say from here that in Egypt, they were struck with fifty plagues and at the Sea, they were struck with two hundred and fifty plagues."}%
\commentb{\textrm{\textbf{תשובה לשער צ"א}}\textrm{\textbf{רבי עקיבא אומר מנין שכל מכה ומכה שהביא הקב"ה על המצריים במצרים היתה של חמש מכות}}  וכו'. כבר ביארתי שגם דעת רבי עקיבא לא בא לחלוק על רבי אליעזר ולא על הראשונים, כי אם להוסיף עליהם בחינה אחרת מפאת המורכב, לפי שכבר התבאר בחכמה שהמורכב יש לו כוח נוסף על כל אחד מהפשוטים אשר בו, ולכן סבר ר' עקיבא שהיו בכל מכה ומכה חמש מכות שהיו ארבע כנגד ארבע יסודות, וחמישי כפי המורכב שהוא זולת מכל אחד מהפשוטים, כיון שבהוויה או בהפסד נבחין הווית המורכב או השתנותו והפסדו. ונבחן גם כן השינוי בכל אחד מהארבעה פשוטים, והוכיח היותם חמישה כפי המילים באותו פסוק שדרש רבי אליעזר, כי הוא דרש הכתוב כולו מראש ועד סוף.  חרון אפו אחת, עברה שתים, זעם שלוש, ודרש צרה בפני עצמה שהוא ארבע, ומשלחת מלאכי רעים דרש בפני עצמו שהוא חמש. ובצדק כל אמרי פיו של רבי עקיבא כי חרון אפו וכן עברה וזעם וצרה הם שמות מורים על הייחוד, ולכן ראוי שידרשו כל אחד מהם על כל אחד מהפשוטים הארבעה, אולם משלחת מלאכי רעין היה ראוי שידרש על המורכב לפי שהוא קיבוץ מלאכים רעים שהם היסודות ההפכיים בתורותיהם ואיכותיהם, ולכן אמר על קיבוצם והרכבתם משלחת מלאכי רעים, ומזה הוכיח שבהיות חמש מכות בכל מכה ממכות מצרים שיתחייב שהיו במצרים חמישים מכות ועל הים מאתים וחמישים מכות. ובעל המלמד14בעל המלמד הוא רבי יעקב בן רבי אבא מרי אנטולי, חותנו של ר'  שמואל אבן תיבון. נקרא על שם ספרו "מלמד התלמידים" חי ופעל בתחילת המאה ה-13. כתב שהיו בחמש מכות ארבע כנגד ארבע היסודות והחמישית כנגד הגלגל שהוא גשם חמישי, ואליו נמשכו המפרשים, אך מה שכתבתי הוא יותר נכון, והותר בזה הספק אשר בשער צ"א.}%endcomment
\newsection{דיינו}
\hebeng{כַּמָה מַעֲלוֹת טוֹבוֹת לַמָּקוֹם עָלֵינוּ!}{How many degrees of good did the Place {[of all bestow]} upon us!}%
\commentb{\textrm{\textbf{תשובות לשערים צ"ב – צ"ד}}\textrm{\textbf{כמה מעלות טובות למקום עלינו, אילו הוציאנו ממצרים ולא עשה בהם שפטים}} וכו'.השלם האלהי רבי עקיבא במאמרו היקר הזה רצה להזכיר כל החסדים והטובות שקיבלנו מה' ביציאת מצרים ובסבת היציאה, והזכיר עוד שרובם היו על צד היותר טוב ולא בדרך הכרח, כי גם בזולתם היה מתפייס הדור היוצא והשם יתברך עשה עמהם יותר מהצורך בדרך ותרנית. ולכן אמר "כמה מעלות טובות למקום עלינו", רצה לומר כמה מהחסדים והמעלות והכבודות שלא על צד ההכרח אלא על צד היותר טוב שהוא הנרצה, כי הוא גמלנו טובות ברחמיו וברוב חסדיו לא בדברים ההכרחיים כי אם על דרך ההטבה. ולכן המעלות הטובות האלה הם להקב"ה תמיד עלינו כשטר חוב שיש לאדם על בעל חובו שלא פרעו, וכמאמר המשורר "מה אשיב לה' כל תגמולוהי עלי" (תהלים קט"ז, י"ב).}%endcomment
\hebeng{אִלּוּ הוֹצִיאָנוּ מִמִצְרַיִם וְלֹא עָשָׂה בָהֶם שְׁפָטִים, דַּיֵּנוּ. }{If He had taken us out of Egypt and not made judgements on them; {[it would have been]} enough for us.}%
\commentb{והזכיר ראשונה היציאה ממצרים כי זאת היתה להם מעלה הכרחית ואמר שאילו הוציאנו הקדוש ברוך הוא ממצרים ולא עשה במצריים שפטים והם המכות שנלקו דיינו, כלומר די היה לנו בזה, לפי שבין שנאמר שהיה גלות מצרים בחטא השפטים או שהיה בגזרת ה' לתכלית חכמתו או שבני ישראל בבחירתם ירדו שמה, הנה אין ספק שבאחד משלושת הדרכים האלה שהם סבת הגלות כמו שכתבתי לעיל היה די להם שיצאו משם, אף אם לא ינקמו מאויביהם, ולפי זה משפט המצריים ומכותיהם היה לישראל על צד היותר טוב.עוד עשה בחינה שניה והוא שכיון שעשה בהם שפטים להיותם רשעים במה שהעבידו את בני ישראל בפרך, אף על פי שהקדוש ברוך הוא לא ישדד המערכות העליונות ולא יסיר גבורת השרים היושבים ראשונה במלכות דיינו, כלומר די היה לנו בזה, אבל לשדד המערכות ולבטל כוחות העליונים מפני התחתונים זה באמת היתה מעלה רמה לישראל ועל צד היותר טוב.}%endcomment
\hebeng{אִלּוּ עָשָׂה בָהֶם שְׁפָטִים, וְלֹא עָשָׂה בֵאלֹהֵיהֶם, דַּיֵּנוּ. }{If He had made judgments on them and had not made {[them]} on their gods; {[it would have been]} enough for us.}%
\commentb{עוד עשה בחינה שלישית והיא "אילו עשה השפטים באלוהיהם ולא הרג בכוריהם דיינו", רצה לומר כי משפט המצריים ומכותיהם היה כפי הצדק לפי שהם הרשיעו לישראל והנפש החוטאת היא תמות, אבל הבכורות לא היו ראויים לעונש כי לא יומתו בנים על אבות. לכן אמר שמלבד מכת בכורות היה די לנו במה שנעשה במצרים עצמם עם היציאה מהגלות, ומהבחינה הזאת באה מכת בכורות למעלה בפני עצמה ולא נכללה בכלל השפטים שעשה בהם. והנה נזכרה מכת בכורות אחרי משפט האלוהיות לפי שנמשכה ממנה, כי אחרי הכות העליונים נלקו התחתונים המתיחסים אליהם, ונזכרה הסיבה הראשונה ואחריה המסובב ממנה.}%endcomment
\hebeng{אִלּוּ עָשָׂה בֵאלֹהֵיהֶם, וְלֹא הָרַג אֶת־בְּכוֹרֵיהֶם, דַּיֵּנוּ. }{If He had made {[them]} on their gods and had not killed their firstborn; {[it would have been]} enough for us.}
\hebeng{אִלּוּ הָרַג אֶת־בְּכוֹרֵיהֶם וְלֹא נָתַן לָנוּ אֶת־מָמוֹנָם, דַּיֵּנוּ. }{If He had killed their firstborn and had not given us their money; {[it would have been]} enough for us.}%
\commentb{עוד עשה בחינה רביעית והיא "אילו הרג את בכוריהם ולא נתן לנו את ממונם דיינו", ואמר זה בעבור שמן הדין אין ראוי שיענש אדם בשני עונשין כי קטן וגדול קים ליה בדרבה מיניה, ואם המצריים נענשו בגופם היה די בזה ולא יענשו גם כן בממונם, אבל היה על צד היותר טוב שנתן ה' את חן העם בעיני מצרים, שעם כל הרעות שקיבלו בעבורם השאילום את כל אשר להם, וכבר היו ישראל מתעצלים מלשאול מהם, והוצרך הקדוש ברוך הוא לומר למשה שיחלה פניהם בזה, שנאמר "דבר נא באזני העם" (שמות י"א, ב'). ותמהתי על השואלים איך אמר רבי עקיבא "ולא עשה בהם שפטים דיינו, ולא נתן לנו את ממונם דיינו", הלוא השם יתברך אמר לאברהם "וגם את הגוי אשר יעבודו דן אנוכי ואחרי כן יצאו ברכוש גדול (בראשית ט"ו, י"ד)"? ומפני זה באו לפרש דברי השלם רבי עקיבא אשר לא כדת ואשר לא עלה על ליבו, כי באמת הוא מזכיר חסדי ה' עמנו אם באופן הגזרה שאמר לאברהם ואם במעשה היציאה ממצרים, ולכן אותם הדברים עצמם שהבטיח לאברהם הם הטובות אשר השפיע עלינו.}%endcomment
\hebeng{אִלּוּ נָתַן לָנוּ אֶת־מָמוֹנָם וְלֹא קָרַע לָנוּ אֶת־הַיָּם, דַּיֵּנוּ. }{If He had given us their money and had not split the Sea for us; {[it would have been]} enough for us.}%
\commentb{עוד עשה בחינה חמישית באומרו "אילו נתן לנו את ממונם ולא קרע את הים דיינו", רצה לומר שעשה ה' נס גדול בקריעת ים סוף כי שינה הסדר הטבעי המוטבע מששת ימי בראשית מבלי הכרח כי אם על צד היותר טוב, לפי שכבר היה הקדוש ברוך הוא יכול להצילם מיד מצרים אחרי שהוציאם משם מבלי שיקרע את הים, אם לא היה מקשה את לב פרעה לרדוף אחריהם, כי הוא לא רדף כי אם בעבור שהקשה ה' את רוחו ובני ישראל עשו עלילות כמו שאמר "וישובו ויחנו לפני פי החירות" (שמות י"ד, ב') ואמר פרעה לבני ישראל "נבוכים הם בארץ", או שישים בלב פרעה לשוב מצרימה כי "לֶב מֶלֶךְ בְּיַד ה' עַל כָּל אֲשֶׁר יַחְפֹּץ יַטֶּנּו" (משלי כ"א, א'), או ישליך על המצריים אבנים מן השמים או באופן אחר מהאופנים כי רבים הם לפני המקום לעשות כרצונו. ולזה אמר רבי עקיבא שהיה די לנו בכל הניסים שנעשו במצרים גם מבלי מופתי הים שלא נעשו על צד ההכרח אלא על צד היותר טוב כדי ש"יִתַּמּוּ חַטָּאִים מִן הָאָרֶץ וּרְשָׁעִים עוֹד אֵינָם" (תהלים ק"ד, ל"ה) .}%endcomment
\hebeng{אִלּוּ קָרַע לָנוּ אֶת־הַיָּם וְלֹא הֶעֱבִירָנוּ בְּתוֹכוֹ בֶּחָרָבָה, דַּיֵּנוּ. }{If He had split the Sea for us and had not taken us through it on dry land; {[it would have been]} enough for us.}%
\commenta{\textrm{\textbf{אלו קרע לנו את הים ולא העבירנו בתוכו בחרבה דיינו.}} אינו מבואר איך הי׳ באפשר באופן אחר, ואולי יכוון הדבר עפ״י הכתוב בפרשה תבא (כ״ח ס״ח) והשיבך ה׳ מצרים באניות, אם כן הי׳ באפשר להעבירם באופן זה, אך העבירם בים בחרבה, וזה נוח יותר. וראי׳ לזה, שהרי ההשבה באניות בתוכחה, ועל העברה קלה בחרבה בים עלינו להודות ביחוד. או יש לומר שעיקר ההודאה היא על ההעברה בחרבה, דאפשר הי׳ להעביר בקרקע הים ברפש וטיט כדרך המים ששוטפים ועוברים ונשארה הקרקע לחה ורטובה. —}%endcomment%
\commentb{עוד עשה בחינה שישית והיא "אילו קרע לנו את הים ולא העבירנו בתוכו בחרבה". וענין זה שישראל לא נכנסו לים לעבור בו משפתו על שפתו כי באותו צד שנכנסו עצמו יצאו כי היתה הבקיעה כעין קשת עגול, ומה הועילו אם כן בכניסתם בלב הים אם לא לאחד משני דרכים או לשניהם יחד: אם להגדיל את ה' שיעשה עמהם פלא, או אם כדי שיכנסו המצריים אחריהם ויהיו נשקעים בתוכו. אף על פי כן היה די בפלא שנקרע להם הים ולא צללו המצריים כעופרת במים אדירים, וזהו על צד היותר טוב. גם היטיב ה' עמהם לעשות פלאים אחרים באותה העברה במה שיבשה הארץ בתוך הים ולא היה שם רפש וטיט כדי להקל מהם הטורח והעמל באותה ההעברה, וכל שכן לנשים וטף, והשם יתברך החריב את קרע הים והוביש אותו כארץ פשוטה וישרה וכאילו לא היה שם טיט מעולם. ולזה באה רוח קדים עזה כל הלילה להחריבו וליבשו. גם היה קרקע הים מלא גבאות ועמקים ותלולים הרבה שלא היה באפשרות לאדם שילך בו, והשם יתברך שם "הֶעָקֹב לְמִישׁוֹר וְהָרְכָסִים לְבִקְעָה" (ישעיהו מ', ג'). וזה כולו על צד היותר טוב ועליו אמר השלם רבי עקיבא אילו קרע לנו את הים ולא העבירנו בתוכו בחרבה דיינו. כי באומרו קרע לנו את הים כבר הודה הנס מעיקרו שנכנסו בים ולא הוטבעו, אלא שהעבירם בתוכו ביבשה גמורה, זו הטבה כפולה.}%endcomment
\hebeng{אִלּוּ הֶעֱבִירָנוּ בְּתוֹכוֹ בֶּחָרָבָה וְלֹא שִׁקַּע צָרֵנוּ בְּתוֹכוֹ דַּיֵּנוּ.}{If He had taken us through it on dry land and had not pushed down our enemies in {[the Sea]}; {[it would have been]} enough for us.}%
\commentb{עוד עשה בחינה שביעית באומרו "אילו העבירנו בתוכו בחרבה ולא שיקע צרינו בתוכו דיינו", כלומר אילו העבירנו ה' בתוך הים ביבשה כמו שהגדיל ה' לעשות והמצריים היו רואים הפלא העצום ההוא והיו פוחדים ובורחים למצרים, בזה היה די לנו מהתשועה וההצלה, אבל השם יתברך לא רצה כי אם להקשות רוחם ולעצום את עיניהם כדי שיכנסו בים וישקעו בו ולא ישאר עוד במצרים אויב וצורר. וכבר נתתי את ליבי פעמים הרבה לדעת איך נכנסו כל המצריים בים ולא כחדו את ה' ואת טובו ונסיו על ישראל, הלא ידעו הלא יבינו כי בהכנסם לים ימותו, ואף שראו את ישראל נכנסים בתוכו, הלא להם לדעת כי אצבע אלוהים היא, וגם הבדל הבדיל ה' ארץ גושן בכל מכות מצרים, ואיך אם כן נשתגעו המצריים להיכנס בתוך הים? אבל האמת בזה הוא שהם היו רודפים אחרי בני ישראל דרך ישר ביבשה על שפת הים, לא שהיו נכנסים לתוכו, וזהו שאמר הכתוב "וַיִּסַּע מַלְאַךְ הָאֱלֹהִים הַהֹלֵךְ לִפְנֵי מַחֲנֵה יִשְׂרָאֵל וַיֵּלֶךְ מֵאַחֲרֵיהֶם" (שמות י"ד, ט"ו) וגו' "וַיָּבֹא בֵּין מַחֲנֵה מִצְרַיִם וּבֵין מַחֲנֵה יִשְׂרָאֵל וַיְהִי הֶעָנָן וְהַחֹשֶׁךְ" (שמות י"ד, כ') וכמו שפירש רש"י ז"ל ויהי הענן והחושך למצרים. אמנם באשמורת הבוקר אחרי שהיו כולם נכנסים בתוך המים נאמר "וַיְהִי בְּאַשְׁמֹרֶת הַבֹּקֶר וַיַּשְׁקֵף יְהוָה אֶל מַחֲנֵה מִצְרַיִם בְּעַמּוּד אֵשׁ וְעָנָן" (שם שם, י"ד), ואז עם אותו אור ראו המצריים שהיו בתוך הים וראו חומת הים ואז נבהלו נחפזו, והוא מה שאמר " וַיָּהָם אֵת מַחֲנֵה מִצְרָיִם" (שם שם כ"ד) כי נפלה עליהם בהלה ומהומה רבה. ואף על פי שקודם זה היו הולכים בסדר ואופן נאה בדרך החיילים, מיד נתבללו ראותם את הים, וזהו "וַיָּסַר אֵת אֹפַן מַרְכְּבֹתָיו וַיְנַהֲגֵהוּ בִּכְבֵדֻת" (שם שם כ"ה), רצה לומר שלא היו רוצים להלוך והם אומרים אלו לאלו "אנוסה מפני ישראל", כלומר נחזר לאחור וננוס, כי זהו מכלל הנפלאות שנעשו במצרים, והוא אמרו "כי ה' נלחם להם במצרים". ואז "וַיֹּאמֶר ה' אֶל מֹשֶׁה נְטֵה אֶת יָדְךָ עַל הַיָּם וְיָשֻׁבוּ הַמַּיִם עַל מִצְרַיִם" (שם שם כ"ו), רצה לומר כבר הם מרגישים בנס קודם שינוסו, ואז "תהומות יכסיומו ירדו במצולות כמו אבן". ולכן אמר "ולא שיקע צרינו בתוכו דיינו" כי השקיעה הייתה טובה כפולה ומכופלת.}%endcomment
\hebeng{אִלּוּ שִׁקַּע צָרֵנוּ בְּתוֹכוֹ וְלֹא סִפֵּק צָרְכֵּנוּ בַּמִדְבָּר אַרְבָּעִים שָׁנָה דַּיֵּנוּ.}{If He had pushed down our enemies in {[the Sea]} and had not supplied our needs in the wilderness for forty years; {[it would have been]} enough for us.}%
\commentb{עוד עשה בחינה שמינית באומרו "אילו שקע צרינו בתוכו ולא סיפק צרכנו במדבר ארבעים שנה דיינו", רצה לומר לא די שהקדוש ברוך הוא עשה עמהם התשועה והגאולה עד להפליא במצרים ועל הים כי גם אחרי כן לא עזבם ולא נטשם וסיפק צרכם במדבר ארץ לא זרועה מקום שרף עקרב וצמאון. וזהו בדרך הטבה לפי שאפשר היה להוליכם מצד ארץ פלישתים כי קרוב הוא והיתה ארץ נושבת למצוא אוכל לנפשם, אבל כדי להשלימם עוד בתורתו ובמצוותיו ולאר הכרחיות שראתה חכמתו העליונה הוליכם במדבר השמם ולא נתן להם צרכיהם בצמצום כי אם בסיפוק גדול ולא יום אחד ולא חודש ימים ולא שנתים אלא ארבעים שנה לא חסרו דבר, וידוע שכאשר יצאו מן הים אילו היו שואלים להם הצריכים אתם עוד דבר יותר מזה בודאי היו אומרים דיינו במה שנעשה לנו מהניסים במצרים ועל הים, ומכאן ואילך די לנו שינהג עמנו בדרך הטבע, אמנם השם יתברך על צד היותר טוב סיפק צרכם במדבר ארבעים שנה בכל הדברים הנצרכים להם בדרך הפלא.}%endcomment
\hebeng{אִלּוּ סִפֵּק צָרְכֵּנוּ בְּמִדְבָּר אַרְבָּעִים שָׁנָה וְלֹא הֶאֱכִילָנוּ אֶת־הַמָּן דַּיֵּנוּ.}{If He had supplied our needs in the wilderness for forty years and had not fed us the manna; {[it would have been]} enough for us.}%
\commenta{\textrm{\textbf{אילו סיפק צרכינו במדבר ארבעים שנה.}} לא נתבאר על איזה צרכים מכוין כאן, והלא באמת לא הי׳ באפשר במדבר להשיג כל הדרוש בחיים. ואולי מכוין למה שכתוב בסוף פרשה תבא, ואולך אתכם ארבעים שנה במדבר לא בלו שלמותיכם מעליכם ונעלך לא בלתה מעל רגליך, ואמרו בברכות (ס׳ ב׳) בין יתר ברכות השחר, כד סיים מסאני׳ (כשנועל מנעליו בשחרית) מברך שעשה לי כל צרכי, ולא נתבאר מה יחש לשון ברכה לנעילת מנעלים. ואפשר לומר עפ״י הגמרא דשבת (קכ״ט א׳) לעולם ימכור אדם כל מה שיש לו ויקח מנעלים לרגליו (כך גורס הרשב״ם בפסחים (קי״ב א׳), וטעם הדבר ביאר הרשב״ם שם מפני שגנאי הוא לאדם לילך יחף, עכ״ל. אבל טעם זה אינו מספיק לתוצאת הדבר ממכירת כל מה שיש לו, ואולי אפשר להטעים הדבר עפ״י הכתוב במשלי (כ״ב ח׳) צנים פחים (קור וחום) בדרך עקש שומר נפשו ירחק מהם, וידוע דמקור יסוד הצטננות הגוף הוא ברגלים, כי אחר הצטננות הרגלים נמשכה ההצטננות לכל הגוף, ולכן נעילת מנעלים היא יסוד לשמירת הבריאות של כל הגוף, וסמך לזה בירמיה (ב׳ כ״ה) מנעי רגליך מיחף, דהכונה שבזה תשמור כל גופך מהצטננות, והמלה יחף הוא שם ולא תואר, דאם הי׳ תואר הי׳ צ״ל יחפה, (דרגל שם נקבה), ואמר בדרך הפלגה כי ראוי לאדם למכור כל מה שיש לו ויקח מנעלים לרגליו ולא יצננן ומתוך כך יבריא כל הגוף וכל החיים.  ועפ״י כל המתבאר יכוון יחש הברכה בשחרית כשנועל מנעליו ״שעשה לי כל צרכי״ כי אחרי שיש לו מנעלים אינו דרוש עוד למכור כל מה שיש לו, ואם כן יש לו הכל, וזהו כל צרכי. ועתה נבא לבאור הלשון בהגדה כאן. אלו סיפק צרכינו במדבר, ומוסב על לשון הפסוק שהבאנו ונעלך לא בלתה מעל רגליך, ואם כן הי׳ להם כל מה שהי׳ להם בשלמות, כי לא הוכרחו למכור הכל למען קנין מנעלים, ובזה סיפק צרכנו במדבר (ועוד יצטרף זה שלמות הבגדים, כמש״כ לא בלו שלמותיכם). ויש להעיר, כי בפרשה עקב (ח׳ ד׳) כתיב בענין זה שמלתך לא בלתה, בלשון יחיד, וכאן כתיב לא בלו שמלותיכם בלשון רבים, ואפשר לומר כי בשניהם לריבותא, כאן הריבותא שגם ברבוי בגדים אפילו אחד לא בלה, ושם הרבותא שאפילו בגד אחד שלובשים אותו תמיד ומחמת זה נוטה יותר להבלות גם כן לא בלה. ומש״כ נעלך בלשון יחיד, ושמלות בלשון רבים אפשר לומר, דשני נעלים כחד חשיבא, שאין דרך לנעול מנעל אחד, משא״כ במלבושים אין אחד משועבד לחבירו, כי אפשר ללבוש כל אחד לעצמו. ועפ״י זה אפשר לבאר בפ׳ בא (י״ב י״א) ונעליכם ברגליכם ולא נעלך ברגלך — יען כי שם נעשה הכל בחפזון, הזהיר שיהיו זהירים לנעול שני המנעלים מקודם, כדי שבשעתו לא יטעו מחמת החפזון לנעול רק אחד.\textrm{\textbf{ולא האכילנו את המן.}} צריך באור, מה הנחה היא זה, הלא במדבר לא הי׳ להם כל דבר מאכל זולת המן, כי הי׳ ״לא מקום זרע״ (לשון הפסוק בפ׳ חקת, כ׳ ה׳), והי׳ זה כביכול מחובת הקב״ה לתת להם לאכול. אך י״ל, דהכונה שאמרו במס׳ יומא (ע״ה ב׳) שהיו טועמין במן כל מיני מאכל, וזה היתה ההנחה מצדו. ואמנם כי על אגדה זו דגמרא יומא אפשר להשיב מה שטענו במדבר מי יאכילנו בשר, אם כן מבואר שלא חשו טעם זה במן. ואפשר לומר, דאמנם מניעת טעם בשר במן הי׳ כדבר יוצא מל הכלל עפ״י סבה אחרת, והוא עפ״י הכתוב בפרשה בשלח (ט״ז ל״א) שטעם המן היה כצפיחית בדבש ובפרשה בהעלתך (י״א ח׳) כתיב שטעמו הי׳ כלשד השמן, ומבואר בגמרא ע״ז (ל״ט ב׳) דבשר המטוגן בדבש ושמן נותן טעם לפגם ומעורר תיעוב וגועל נפש (וכ״ה ביו״ד סימן ק״ג ועיי״ש בב״י לטור), — ואם כן לא היו יכולים לטעום במן טעם בשר מפני התערובת טעמו מדבש ושמן.}%endcomment%
\commentb{עוד עשה בחינה תשיעית, באומרו "אילו סיפק צרכנו במדבר ארבעים שנה ולא האכילנו את המן דיינו", כלומר אפשר היה לספק צרכנו ולתת לחם בר ולחם ומזון אנושי אף על פי שהיה קניינו בדרכי ניסיים, אבל האל יתברך על הצד היותר טוב נתן להם את המן שהיה מזון אלוהי מוגשם באוויר והיה בו כוח עליון, כמו שדרשו חז"ל "לֶחֶם אַבִּירִים אָכַל אִישׁ" (תהלים ע"ח, כ"ה), לחם שאכלים מלאכי השרת , עד שלהיותו דבר יוצא מן הטבע לא ידעו מה הוא והיו מרגישים בו כמה טעמים, לקטנים היה כלשד השמן, הבחורים עושים אותו עוגות, לחולים היה טעמו כצפיחית בדבש, לזקנים ובשלו בפרוד, לאומות העולם כזרע גד והיה יכול לספק צרכם בעודות שהוציאו ממצרים כמו שספק מלבושים.וכבר נזכרו שמונה ניסים נפלאים:א – היותו באויר הפשוט ונעשה ממנו מזון ראוי להזנה.ב – שהתמיד הנס הזה ארבעים שנה ובכל הנסים שנזכרו בכתוב לא נמצא נס מתמיד זמן ארוך כזה.ג – וימודו בעומר ולא העדיף המרבה והממעיט לא החסיר.ד – ויותירו ממנו עד בוקר וירום תולעים ויבאש כאשר דיבר להם משה.ה – שלא היה כן ביום השבת כמו שנאמר "וַיַּנִּיחוּ אֹתוֹ עַד הַבֹּקֶר כַּאֲשֶׁר צִוָּה מֹשֶׁה וְלֹא הִבְאִישׁ וְרִמָּה לֹא הָיְתָה בּוֹ  (שמות ט"ז, כ"ד)."ו – שלא היה יורד בשבת שנאמר "וַיְהִי בַּיּוֹם הַשְּׁבִיעִי יָצְאוּ מִן הָעָם לִלְקֹט וְלֹא מָצָאוּ" (שם שם כ"ז).ז – שפסק בבואם אך ארץ נושבת שנאמר "את המן אכלו עד בואם אל ארץ נושבת" (שם שם ל"ה).ח – קיום המן בצנצנת לדורות אף כי מטבעו נמס שנאמר "וחם השמש ונמס". (שם שם כ"א)וכל זה מורה על מעלת המזון האלהי הזה.}%endcomment
\hebeng{אִלּוּ הֶאֱכִילָנוּ אֶת־הַמָּן וְלֹא נָתַן לָנוּ אֶת־הַשַׁבָּת, דַּיֵּנוּ. }{If He had fed us the manna and had not given us the Shabbat; {[it would have been]} enough for us.}%
\commentb{עוד עשה בחינה עשירית באמרו "אילו האכילנו את המן ולא נתן לנו את השבת דיינו", לפי שהשבת ניתן במרה קודם בואם אל הר סיני כאשר ירד המן, ומעלת מצוות השבת רבה היא, כי תורה על חידוש העולם ועל המנוחה הנפשית לעולם שכולו שבת, וכמו שנאמר בתפילת היום "ולא נתתו ה' אלוהינו לגויי הארצות ולא הנחלתו מלכנו לעובדי פסילים גם במנוחתו לא ישכנו ערלים כי לעמך ישראל נתתו באהבה לזרע יעקב אשר בם בחרת", כי באומרו "ולא נתתו מלכנו לעובדי פסילים" רמז לטעם חידוש העולם שהוריש את האומה זכרונו ולא לעובדי פסילים, ובאומרו "גם במנוחתו לא ישכנו ערלים" רמז למנוחה הנפשית לעולם שכולו שבת.}%endcomment
\hebeng{אִלּוּ נָתַן לָנוּ אֶת־הַשַׁבָּת, וְלֹא קֵרְבָנוּ לִפְנֵי הַר סִינַי, דַּיֵּנוּ. }{If He had given us the Shabbat and had not brought us close to Mount Sinai; {[it would have been]} enough for us.}%
\commentb{עוד עשה בחינה י"א באומרו "אילו נתן לנו את השבת ולא קרבנו לפני הר סיני דיינו" רצה לומר מלבד מה שהשלימם בשלמות הגופני עוד נתן להם התורה, כי זה מה שקנו בקרבם לפני הר סיני ובקבלת התורה שפסקה זוהמתן. וידוע שלא עשה כן לכל גוי, וישראל לא שאלו אותה כי לא עלה על ליבם שיזכו לחסד העליון ההוא, ולזה אמר בו דיינו.}%endcomment
\hebeng{אִלּוּ קֵרְבָנוּ לִפְנֵי הַר סִינַי, וְלא נָתַן לָנוּ אֶת־הַתּוֹרָה. דַּיֵּנוּ. }{If He had brought us close to Mount Sinai and had not given us the Torah; {[it would have been]} enough for us.}%
\commenta{\textrm{\textbf{אלו קרבנו לפני הר סיני ולא נתן לנו את התורה דיינו.}} לכאורה מה חסד הוא לקרוב להר סיני בלא קבלת התורה. וצריך לומר, דקיצור לשון הוא, ועיקר הענין מן אלו קרבנו לפני הר סיני לראות במחזה אלהים כידוע בענין זה בפרשה ובאגדות. וגם אפשר לפרש הכונה עפ״י מה שאמרו במס׳ ביצה (כ״ה ב׳) מפני מה נתנה תורה לישראל מפני שעזין הם, (ומביא שם על זה מאמר ״עזין שבאומות — ישראל״, עיי״ש) ופירש״י, שהתורה מכניע לבם, ואיתא במס׳ סוטה (ה׳ א׳) לעולם ילמוד אדם מדעת קונו, כלומר, ללמוד מדת ענוה, שהרי הניח הקב״ה את כל הרים וגבעות גדולים ונתן את התורה על הר סיני הנמוך *ואמנם אין ללמוד מענין זה על ביטול כל ערך הכבוד שראוי לת״ח לנהוג, יען דאם כן הי׳ דרוש ליתן תורה על הארץ, אלא ודאי דבמקצת כבוד דרוש לת״ח לאחוז, מפני כבוד התורה, ועל כן נתן על הר נמוך. ולפי זה, אחרי דמטרת נתינת התורה היתה, כמבואר, להכניע לבם — הנה גם בהתקרבות ובהתגלות ה׳ על אותו ההר בלבד היתה להם תועלת שלמדו ממנו מדת הכנעה וענוה, וגם זה מדה נכונה.}%endcomment%
\commentb{עוד עשה בו בחינה י"ב באמרו "אילו קרבנו לפני הר סיני ולא נתן לנו את התורה דיינו" ועניין זה הוא שנתן להם הקדוש ברוך הוא את התורה בכבודו ובעצמו, כי הנה קבלת התורה והמצוות כבר נכלל בהקרבה לפני הר סיני, אבל אמר שנתינת התורה היתה על ידי השם יתברך ומפי הגבורה שמעו עשרת הדברות. וידוע שלא היו ישראל ראויים לאותה מדרגה, והיה די לתת להם התורה על ידי משה רבינו עליו השלום, אבל על צד היותר טוב דיבר ה' אל כל קהלם. יש מפרשים ולא נתן לנו את התורה בכללותיה אמונותיה ומצוותיה והיה מסתפק במצוות מעטות כמו שנתן לאדם הראשון או לנוח או לאברהם.}%endcomment
\hebeng{אִלּוּ נָתַן לָנוּ אֶת־הַתּוֹרָה וְלֹא הִכְנִיסָנוּ לְאֶרֶץ יִשְׂרָאֵל, דַּיֵּנוּ. }{If He had given us the Torah and had not brought us into the land of Israel; {[it would have been]} enough for us.}%
\commenta{\textrm{\textbf{אלו נתן לנו את התורה ולא הכניסנו לא״י דיינו.}} ולא אמר עוד להיפך, כסגנון כל המאמרים, אלו הכניסנו לארץ ישראל ולא נתן לנו את התורה דיינו, יען כי באופן כזה, באמת —לא דיינו, מפני כי בלא תורה אין יתרון לארץ ישראל על כל הארצות, וכמו באומת ישראל, שזולת התורה אין להם יתרון על שאר האומות, ואשר על כן עודם במצרים, טרם שקבלו התורה, הי׳ המצרי נקרא ״רעהו״, של ישראל, וכמש״כ בפ׳ בא (י״א ב׳) וישאלו איש מאת רעהו, ומוסב על המצרי, ובפרשה משפטים (כ״א ל״ד.) כתיב וכי יגוף שור איש את שור רעהו, ודרשו (ב״ק ל״ז ב׳) רעהו ולא עכו״ם (מבואר בגמרא שם דמוסב על עובדי אלילים בימים הקדמונים שאין מקיימין שבע מצות הנחוצות לקיום העולם). וזה הוא מפני שבמצרים היו עוד קודם קבלת התורה, והענינים שבפרשה משפטים לאחר מתן תורה, אשר אז נבדלו מכל העמים, כמבואר בפרשה הקודמת לפ׳ משפטים, פ׳ יתרו. ועל מה שכתבנו במעלת ארץ ישראל, כי מעלתה היא רק ביחש דבוק עם התורה — אל זה יש סמך בגמרא ערכין (י׳ ב׳) שאמרו, דמעלת א״י החלה רק משנכנסו ישראל לתוכה (ומובן עם התורה) ואז הוקדשה ונתעלה, ועד אז הי׳ ערכה כערך כל הארצות. וכן מה שכתבנו דבלתי התורה אין יתרון לעם ישראל על שאר האומות, עפ״י זה יתבאר במס׳ יבמות (מ״ז ב׳) בדברי רות לנעמי עמך עמי ואלהיך אלהי, ולכאורה גם הלשון עמך עמי די להלות על עם ישראל, אך רמזה בזה, שלא תסתפק בלאומיות לבד בלא תורה ודת (כמו שיש חושבים היום) אך תקבל עלי׳ גם אלהות ובכללה גם התורה ומעשה המצוות. ועפ״י הנחת הגמרא דמעלת א״י החלה רק משנכנסו ישראל לתוכה — עפ״י זה, יובן מה שאמרו בטעם הדבר שאין אומרים הלל בפורים, מפני שאין אומרים שירה על נס שבחוץ לארץ (והלל היינו שירה, ועיין בסמוך בטעם הדבר). והנה קשה איך אמרו ישראל שירה על הים והיו אז בחו״ל, אך לפי המבואר דקדושת א״י החלה רק משנכנסו ישראל לא״י, ובעת קריעת הים עדיין לא נכנסו לארץ, ולכן הי׳ אפשר להגיד שירה בכל ארץ ומדינה שהיא, מה שאין כן גם פורים שכבר נתקדשה א״י על כל הארצות, וכולן חולין לגבי קדושתה, לכן אין אומרים בהן שירה. וזולת זה אפשר לתת טעם על מה שאמרו ישראל שירה אעפ״י שאין אומרים שירה על נס שבחו״ל, עפ״י מש״כ מהרש״א בערכין שם בטעם דבר זה שאין אומרים שירה על נס שבחו״ל משום דנסים שבא״י נעשים ע״י הקב״ה בעצמו וראוי מאוד לומר עליו שירה והלל, ולא כן בחו״ל נעשים ע״י שליח או מלאך, ואחרי שידוע כי הנסים שבמצרים נעשים כולם ע״י הקב״ה בעצמו, כמ״ש אני ולא שליח אני ולא מלאך, וגם מבואר במדרשים שעל הים ראו כולם במאור השכינה, וזה מכוון בלשון שאמרו זה אלי, וכמו שהראו באצבע. לפי כל זה הי׳ אז השעה והמקום קדושים לשיר שירה, אף כי המעמד הי׳ בחוץ־לארץ.}%endcomment%
\commentb{עוד עשה בחינה י"ג באומרו "אילו נתן לנו את התורה ולא הכניסנו לארץ ישראל דיינו", ואין הכוונה שישארו במדבר כערביים ולא יירשו ארץ, אלא עניינו שנתן להם ארץ ישראל שהוא הקו האמצעי מן היישוב וממנו הושתת העולם, ואין מזל ולא שר שולט עליהם כי אם שכינת ה', ולפי שצבי היא לכל הארצות בשלמות הגופניים והרוחניים, לכן אמר שהיה די להם להושיבם בארץ אחרת, אלא שעל צד היותר טוב נתן להם את הארץ אשר ה' תמיד דורש אותה. יש מפרשים ולא הכניסנו לארץ ישראל שהיה די שיתן את הארץ לבניהם או לבני בניהם שנולדו במדבר ולא לאותם שיצאו ממצרים, שהיו בהם הרבה פחותים מעשרים שנה שיצאו ממצרים ונכנסו הם בארץ והיה די להם בניסי היציאה שראו.}%endcomment
\hebeng{אִלּוּ הִכְנִיסָנוּ לְאֶרֶץ יִשְׂרָאֵל וְלֹא בָנָה לָנוּ אֶת־בֵּית הַבְּחִירָה דַּיֵּנוּ.}{If He had brought us into the land of Israel and had not built us the 'Chosen House' {[the Temple; it would have been]} enough for us.}%
\commentb{עוד עשה בחינה י"ד באומרו "אילו הכניסנו לארץ ישראל ולא בנה בית המקדש דיינו", לפי שבו היו מתכפרים עונותיהם של ישראל והקדים ה' רפואה למכתם שיתרפאו מחולי הנפש, לפי שאין איש בארץ אשר יעשה טוב ולא יחטא (קהלת ז', כ'). גם כי בבית המקדש נעשו בו נסים הרבה, וכל המעלות שהזכיר ניסים הם.}%endcomment
\hebeng{עַל אַחַת, כַּמָה וְכַּמָה, טוֹבָה כְפוּלָה וּמְכֻפֶּלֶת לַמָּקוֹם עָלֵינוּ: שֶׁהוֹצִיאָנוּ מִמִּצְרַיִם, וְעָשָׂה בָהֶם שְׁפָטִים, וְעָשָׂה בֵאלֹהֵיהֶם, וְהָרַג אֶת־בְּכוֹרֵיהֶם, וְנָתַן לָנוּ אֶת־מָמוֹנָם, וְקָרַע לָנוּ אֶת־הַיָּם, וְהֶעֱבִירָנוּ בְּתוֹכוֹ בֶּחָרָבָה, וְשִׁקַּע צָרֵנוּ בְּתוֹכוֹ, וְסִפֵּק צָרְכֵּנוּ בַּמִדְבָּר אַרְבָּעִים שָׁנָה, וְהֶאֱכִילָנוּ אֶת־הַמָּן, וְנָתַן לָנוּ אֶת־הַשַּׁבָּת, וְקֵרְבָנוּ לִפְנֵי הַר סִינַי, וְנָתַן לָנוּ אֶת־הַתּוֹרָה, וְהִכְנִיסָנוּ לְאֶרֶץ יִשְׂרָאֵל, וּבָנָה לָנוּ אֶת־בֵּית הַבְּחִירָה לְכַפֵּר עַל־כָּל־עֲוֹנוֹתֵינוּ. }{How much more so is the good that is doubled and quadrupled that the Place {[of all bestowed]} upon us {[enough for us]}; since he took us out of Egypt, and made judgments with them, and made {[them]} with their gods, and killed their firstborn, and gave us their money, and split the Sea for us, and brought us through it on dry land, and pushed down our enemies in {[the Sea]}, and supplied our needs in the wilderness for forty years, and fed us the manna, and gave us the Shabbat, and brought us close to Mount Sinai, and gave us the Torah, and brought us into the land of Israel and built us the 'Chosen House' {[the Temple]} to atone upon all of our sins.}%
\commentb{אחר כך כלל רבי עקיבא כל הטובות והמעלות האלה ואמר כי בהתחברן כולן יחד על אחת כמה וכמה שאנו חייבים להיות מכירי טובה כפולה ומכופלת בזו. והנה לא הזכיר רבי עקיבא נס השלו והבאר וענני הכבוד ושאר הניסים שנעשו לאבותינו במדבר לפי שנכללו באומרו כי סיפק צרכנו במדבר ארבעים שנה. גם לא הזכיר המלחמה עם סיחון ועוד ומלכי כנען שנעשה בהם כמה ניסים לפי שכללם באומרו הכניסנו לארץ ישראל.בסך הכל הזכיר רבי עקיבא ט"ו מעלות עם מעלת יציאת מצרים, וכבר אמרו שהיו ט"ו מעלות האלה כנגד ט"ו שיר המעלות שאמר דוד, וכנגד ט"ו מעלות שעשה הקדוש ברוך הוא במצרים עם ישראל כמו שהזכירם יחזקאל באומרו "וָאֶעֱבֹר עָלַיִךְ וָאֶרְאֵךְ מִתְבּוֹסֶסֶת בְּדָמָיִךְ" (יחזקאל ט"ז, ו) "וָאֶעֱבֹר עָלַיִךְ וָאֶרְאֵךְ וְהִנֵּה עִתֵּךְ עֵת דֹּדִים" (שם שם, ח') וגו', וכנגדם היו במקדש חמש עשרה מעלות כשהיו עולים מעזרת הנשים לעזרת ישראל, ולכן קראם רבי עקיבא מעלות ולא חסרים להעיר על הרמזים האלה.והותרו עם זה הספקות אשר בשערים צ"ב, צ"ג ו-צ"ד.}%endcomment
\newsection{פסח מצה ומרור}
\hebeng{רַבָּן גַּמְלִיאֵל הָיָה אוֹמֵר: כָּל שֶׁלֹּא אָמַר שְׁלשָׁה דְּבָרִים אֵלּוּ בַּפֶּסַח, לא יָצָא יְדֵי חוֹבָתוֹ, וְאֵלּוּ הֵן: פֶּסַח, מַצָּה, וּמָרוֹר. }{Rabban Gamliel was accustomed to say, Anyone who has not said these three things on Pesach has not fulfilled his obligation, and these are them: the Pesach sacrifice, matsa and \textit{marror}.}%
\commenta{\textrm{\textbf{רבן גמליאל הי׳ אומר, כל שלא אמר שלשה דברים אלו בפסח לא יצא ידי חובתו.}} ההרגש בזה, מה נשתנו שלש מצות אלו מכל המצות שהקפיד לפרש ענינם, אבל אפשר לפרש, דסבירא לי׳ לרבן גמליאל כמו שפסקו בברכות (י״ג ב׳) דמצות צריכות כונה, וענין כונה הוא לכוין ענינן, ומפרש כאן מה ענינן של אלה המצות, ומציע אותן על הסדר, ולפי זה אין מצות אלו יוצאות מכלל המצות, אך מפני דס״ל, דמצות צריך לכוין ענינן ולכן מפרש אותן. ומה שאמר כל שלא אמר, דמשמע שצריך להוציא בפה ובאמת צריך כונת הלב, כמבואר בברכות שם, וכן מורה הלשון ׳״כונה״ וכונה היא רק בלב, יען כי באמירה אפשר שמבטא בשפתיו ולבו בל עמו — אפשר לומר, דבאמת מצינו הפעל אמירה שמוסב על מחשבה בלב ובנפש, כמו בפרשה שמות (ב׳ י״ד) הלהרגני אתה אומר, כלומר, תחשב בלבך, ובתהלים (ד׳ ח׳) אמרו בלבבכם, ושם (י׳ י׳) אמר בלבו, ובס״פ תולדות ויאמר עשו בלבו, ובשמואל א׳ (ב׳ ד׳) מה תאמר נפשך, והרבה כהנה, והכונה כאן כל שלא אמר — כל שלא חשב ולא כיון *וכמובן זה בלשון אמירה יתבאר בפ׳ בראשית ויאמר קין אל הבל אחיו, ודייקו מפרשים שלא פירש מה אמר, אך לפי שבארנו יתבאר ויאמר — שחשב בלבו עליו להרגו, והמובן ״אל הבל״ כמו ״על הבל״ (בחילוף אהח״ע) על אודות הבל, כמו ויאמר אברהם אל שרה אשתו אחותי היא (פ וירא כ׳ כ״א) והמובן על שרה, וביחזקאל (ו׳ י׳) לא אל חנם דברתי, דהמובן לא על חנם, ובתהלים (ב׳) ספרה אל חק — תחת על חק, ועוד.).}%endcomment
\hebeng{פֶּסַח שֶׁהָיוּ אֲבוֹתֵינוּ אוֹכְלִים בִּזְמַן שֶׁבֵּית הַמִּקְדָּשׁ הָיָה קַיָּם, עַל שׁוּם מָה? עַל שׁוּם שֶׁפָּסַח הַקָּדוֹשׁ בָּרוּךְ הוּא עַל בָּתֵּי אֲבוֹתֵינוּ בְּמִצְרַיִם, שֶׁנֶּאֱמַר: וַאֲמַרְתֶּם זֶבַח פֶּסַח הוּא לַיי, אֲשֶׁר פָּסַח עַל בָּתֵּי בְנֵי יִשְׂרָאֵל בְּמִצְרַיִם בְּנָגְפּוֹ אֶת־מִצְרַיִם, וְאֶת־בָּתֵּינוּ הִצִּיל וַיִּקֹּד הָעָם וַיִּשְׁתַּחווּ. }{The Pesach {[passover]} sacrifice that our ancestors were accustomed to eating when the Temple existed, for the sake of what {[was it]}? For the sake {[to commemorate]} that the Holy One, blessed be He, passed over the homes of our ancestors in Egypt, as it is stated (Exodus 12:27); "And you shall say: 'It is the passover sacrifice to the Lord, for that He passed over the homes of the Children of Israel in Egypt, when He smote the Egyptians, and our homes he saved.’ And the people bowed the head and bowed."}%
\commenta{\textrm{\textbf{פסח שהיו אבותינו אוכלים.}} ובמצה ובמרור אמר שאנו אוכלים, וזה פשוט, מפני שמצה ומרור נוהגים גם בזמן הזה, מה שאין כן פסח נוהג רק בזמן שבית המקדש קים. ובזה יתבאר מה שבנוסח זה שבמשנה (פסחים קט״ז ב׳) חסר הלשון ״שאנו אוכלים״. וזה הוא מפני שבעל המאמר הזה (כל שלא אמר שלשה דברים אלו  בפסח) הוא רבן גמליאל הזקן *כי שלשה חכמים מחכמי המשנה והתלמוד, וכולם נקראו בשם לוויי מיוחד, רבן גמליאל בעל המאמר הזה נקרא ״הזקן״ להיותו זקן מכולם, והשני בנו של הקודם נקרא רבן גמליאל דיבנה, על שם שהי׳ נשיא הסנהדרין ביבנה, לאחר החורבן, והשלישי, בנו של רבי (רבי יהודה הנשיא) נקרא רבן גמליאל ברבי (עיין כתובות ק״ג ב׳).) והוא היה עוד בזמן המקדש, ואכל מפסח, והרי זה כמו שהי׳ אומר שאנו אוכלים.\textrm{\textbf{פסח שהיו אבותינו אוכלים בזמן שבית המקדש הי׳ קיים וכו׳.}} יש להעיר על מניעת אכילת פסח בזמן הזה בחוץ לארץ, והלא את הפסח הראשון אכלו בלילה עודם במצרים טרם שיצאו משם, כמבואר בהמשך הפרשה (פ׳ בא, י״ב), ולמה לא נעשה לזה זכר בחו״ל. אך אפשר לומר, משום דעיקר המטרה מאכילת פסח הוא קרבן תודה לה׳ על אשר פסח על הבתים, ומקום הקרבנות מכוון רק בירושלים מקום השראת השכינה, ולכן אין לו מקום בחו״ל, ומה שאכלו אז בחו״ל הוא מפני שבאותו הלילה עבר ה׳ בכבודו על ארץ מצרים, כמש״כ ועברתי בארץ מצרים... אני ולא מלאך אני ולא אחר, אם כן היתה באותו הלילה במצרים השראת השכינה כמו בארץ ישראל, ולכן נחשב להם המקום אז כקדושת א״י, ולכן אכלו אז הפסח, ואין ראיה מן אותו הלילה להעתידות בגלות. — וע׳ ברש״י פ׳ בהעלותך (ט׳ א׳). ומה שכתבנו, דמקום הקרבת הקרבנות הוא רק בירושלים ולא בחו״ל — לכאורה יש להעיר ממה שהקריבו נח והאבות בחו״ל, אך הבאור הוא, דארץ ישראל נתקדשה רק משנכנסו ישראל לתוכה וקדשוה, אבל עד אז הי׳ ערכה ודיניה ככל הארצות, ועיין במס׳ ערכין (י׳ ב׳).\textrm{\textbf{פסח שהיו אבותינו אוכלים בזמן שביהמ״ק הי׳ קיים על שום שפסח הקב״ה על בתי בני ישראל במצרים.}} לכאורה היה לו לומר סתם שהיו אבותינו אוכלים ולהשמיט הלשון ״בזמן שביהמ״ק הי׳ קיים״, אחרי כי את הפסח הראשון אכלו עודם במצרים. והי׳ השאלה והתשובה מכוונת. אך הבאור הוא, כי על הפסח שאכלו במצרים נצטוו עוד טרם ההודעה מפסיחת הקב״ה על בתי ישראל, כמבואר בהמשך הפרשה בפרשה בא, וטעם צוי זה לא נודע לנו, ויתכן שטעמו כדי לבטל הע״ז של המצרים שהיו עובדים לטלה, כידוע, וכמבואר במכילתא בא על הפסוק משכו וקחו לכם (י״ב כ״א) משכו מע״ז, ולכן אחרי שתלה המגיד את חיוב אכילתו על שום שפסח, בהכרח הי׳ צריך להוסיף בזמן שביהמ״ק הי׳ קיים, שעל אותו הזמן מכוון זכרון הטעם אשר פסח וכו׳, ולא על זה שאכלו במצרים, כמבואר.\textrm{\textbf{שנאמר ואמרתם זבח פסח הוא לה׳ אשר פסח על בתי בני ישראל במצרים בנגפו את מצרים ואת בתינו הציל.}} לכאורה הוא כפל לשון, כי אחרי ״אשר פסח על בתי בני ישראל״ ממילא מתבאר כי נצולו בתינו. אך אפשר לבאר עפ״י מה דאיתא במדרשים, שהיו בבכורי מצרים כאלה שהטמינו עצמם בבתי ישראל, בחשבם להנצל ע״י זה, אך לא הועיל להם זה ונגפו בתוך הבתים. והנה אם כן הי׳ המות גם בהבתים, ובכל זאת נצולו, ועל זה אמר, כי נס כפול הי׳ שם, האחד — שפסח על בתי בני ישראל, והשני — כי גם בעת שהי׳ המות שורר בהבתים (לרגלי הבכורים שנטמנו שם) גם כן נצולו הבתים.\textrm{\textbf{ויקד העם וישתחוו.}} עיין ברש״י במקום פסוק זה (פ׳ בא, י״ב כ״ז) כי ההודאה היתה על בשורת בנים, וזה פלא, מה יחש בשורה זו לענין הפרשה. וראה מה שכתבנו בזה למעלה במאמר כנגד ארבעה בנים דברה תורה, וצרף לכאן.}%endcomment
\hebeng{{\small אוחז המצה בידו ומראה אותה למסובין: } }{{\small He holds the matsa in his hand and shows it to the others there.} }
\hebeng{מַצָּה זוֹ שֶׁאָנוֹ אוֹכְלִים, עַל שׁוּם מַה? עַל שׁוּם שֶׁלֹּא הִסְפִּיק בְּצֵקָם שֶׁל אֲבוֹתֵינוּ לְהַחֲמִיץ עַד שֶׁנִּגְלָה עֲלֵיהֶם מֶלֶךְ מַלְכֵי הַמְּלָכִים, הַקָּדוֹשׁ בָּרוּךְ הוּא, וּגְאָלָם, שֶׁנֶּאֱמַר: וַיֹּאפוּ אֶת־הַבָּצֵק אֲשֶׁר הוֹצִיאוּ מִמִּצְרַיִם עֻגֹת מַצּוֹּת, כִּי לֹא חָמֵץ, כִּי גֹרְשׁוּ מִמִּצְרַיִם וְלֹא יָכְלוּ לְהִתְמַהְמֵהַּ, וְגַם צֵדָה לֹא עָשׂוּ לָהֶם. }{This matsa that we are eating, for the sake of what {[is it]}? For the sake {[to commemorate]} that our ancestors' dough was not yet able to rise, before the King of the kings of kings, the Holy One, blessed be He, revealed {[Himself]} to them and redeemed them, as it is stated (Exodus 12:39); "And they baked the dough which they brought out of Egypt into matsa cakes, since it did not rise; because they were expelled from Egypt, and could not tarry, neither had they made for themselves provisions."}%
\commenta{\textrm{\textbf{מצה זו שאנו אוכלים על שום מה על שום שלא הספיק בצקם של אבותינו להחמיץ וכו׳.}} לכאורה לשון זה אין מכוון לענין, שהרי על מצות מצה נצטוו עוד בתחלת החודש בעוד היותם במצרים, כפי שמתבאר בהמשך כל הפרשה (י׳ ב׳), אז לא הי׳ כל הרהור וכל דבור על דבר ספוק בצק להחמיץ.  אך הבאור הוא עפ״י המבואר בפסחים (ק״ב א׳) דאכילת מצה בכל ימי הפסח אינו אלא רשות, כלומר, דעיקר הקפידא רק שלא לאכול חמץ, אבל אם רוצה לאכול כל מיני אוכלין שאין בהם חשש חמץ כמו דגים ובשר בלא מצה רשאי, אך בליל ראשון של פסח, החובה לאכול מצה, ויליף זה שם מקרא, ולכן מוסב זה על ליל ראשון של פסח, שאז אומרים זה, ואז באמת היתה סיבת המצוה על שום שלא הספיק בצקם להחמיץ, כי בלילה ההוא יצאו ממצרים.}%endcomment%
\commentb{\textrm{\textbf{תשובה לשער צ"ו}}\textrm{\textbf{מצה זו שאנו אוכלים על שום מה, על שום שלא הספיק בצקם של אבותינו להחמיץ}} וכו'.כבר כתבתי בשערים הספק אשר יפול בטעם שנתו רבן גמליאל במצה בהיות שנצטוו עליה ישראל במצרים קודם יציאתם משם, ואיך יאמר שהיה טעמה לזכרון מהירות היציאה שלא הספיק בצקם להחמיץ? והר"ן ז"ל בחידושיו על הרב אלפסי בפרק "ערבי פסחים" כתב על זה וזה לשונו: "מצה על שם שנגאלו שנאמר ויאפו את הבצק ולא יכלו להתמהמה שאילו יכלו להתמהמה היו מחמיצין אותו דפסח מצרים לא נהנו אלא לילה ויום בפסח שני ולמחר היו מותרים במלאכה ובחמץ, לפיכך אילו יכלו להתמהמה החמיצו עיסותיהם שלא הוזהרו בבל יראה, אבל מתוך שלא היה להם פנאי אפוהו מצה, וזכר לאותה גאולה נצטוו באכילת המצה", עד כאן. ונראה שהרב מבין מצוות מצרים מפסוק "על מצות ומרורים יאכלוהו" ויאמר כי זה ידבר רק בפסח מצרים, ומה שנאמר אחר כך "והיה היום הזה לכם לזיכרון וחגותם אותו" וגו' "שבעת ימים תאכל מצות" וגו' הוא פסח לדורות. אחרי בואם אל הארץ ששם יהיה איסור החמת כל שבעה, כי לדעתו לא היה פסח מצרים כי אם בלילה הראשונה, וכמו שצוותה תורה בפסח שני גם לדורות. אמנם במדבר לא היה להם חיטה ושעורה לעשות מצה ולכן לא נצטוו עליה כי אם בשנה השנית על פי הדיבור.אולם הדעה הזאת רחוקה מאוד מפשט הכתובים בפרשת החודש, ועדיין הקושיא במקומה עומדת, למה אמר רבן גמליאל שהסיבה שאנו אוכלין המצה הוא לפי שלא הספיק בצקם של אבותינו להחמיץ, הלוא קודם ליציאתם ממצרים כבר נצטוו במצוות מצה ואיסור החמת כל שבעה לדורות, ומפני אותה מצוה שנצטוינו בה באותה שעה אנו אוכלין את המצה, לא מפני שלא הספיק בצקם להחמיץ. והנה ראיתי מה שפירש שהמצווה הזו נצטוו בה ישראל בראשונה במצרים לבד, אך באשר נעשה להם נס של מהירות היציאה שלא הספיק בצקם להחמיץ קיימו וקיבלו עליהם לדורות את אשר החלו לעשות, ויהי אם כן טעם רבן גמליאל מצורף אל המצוה.אמנם מה שראוי לומר בהיתר הספק הזה הוא באחד משלושה דרכים, והצד השווה שבהם שהמצווה נצטוו בה במצרים בענין המצה ואיסור חמץ כל שבעה כמו שנצטוו עליה לדורות, ושעל פסח מצרים אמרה תורה "וְהָיָה הַיּוֹם הַזֶּה לָכֶם לְזִכָּרוֹן וְחַגֹּתֶם אֹתוֹ חַג לַה'" (שמות י"ב, י"ד) "שִׁבְעַת יָמִים תֹּאכַל מַצֹּת" (שם י", ו') וגו', רצה לומר שיעשו את החג לדורות כמו שיעשו אותו במצרים בפעם הראשונה, ועל זה נאמר "שבעת ימים תאכל מצות" ושאר הפסוקים. וכן כתב הרמב"ן כי טעם "וַיֹּאפוּ אֶת הַבָּצֵק... עֻגֹת מַצּוֹת" (שמות י"ב, ל"ט) מפני המצווה שנצטוו בה "שִׁבְעַת יָמִים שְׂאֹר לֹא יִמָּצֵא בְּבָתֵּיכֶם כִּי כָּל אֹכֵל מַחְמֶצֶת" (שם שם, י"ט) וגו', והיתה המצוה ההיא לזכור מהירות גאולתם כדי שלא יוכלו לאפות את הבצק במצרים. ואף כי היו עם רב בכל זאת בחיפזון עד שלא הספיק בצקם להחמיץ כאשר הגיעו למקום שמצאו תנורים לאפות וראו כי בדרך נס נעשו מהירותם, כי הלכו מרעמסס לסוכות שהוא מהלך יום אחד בשעה קלה כמו שהזכירו חז"ל כדי שלא יחמץ בצקם. וכבר יורה על זה דברי משה אדוננו בתורה "שִׁבְעַת יָמִים תֹּאכַל עָלָיו מַצּוֹת לֶחֶם עֹנִי כִּי בְחִפָּזוֹן יָצָאתָ מֵאֶרֶץ מִצְרַיִם" (דברים ט"ז, ג'), כי לזכרון החיפזון נצטוו במצה. שעם היות שנצטוו עליה קודם היציאה ממצרים אינו מבטל הענין הזה כי השם יתברך היודע העתידות ציוה אותם לעשות המצה לזכרון אותו פלא שהיה עתיד לעשות להם. הלוא תראה שציוה בפסח מצרים משום "וּפָסַח יְהוָה עַל הַפֶּתַח" (שמות י"ב, כ"ג) ובאה המצוה קודם מעשה הנס, וכן היה בענין המצה מפני חיפזון היציאה נצטוו בה והיה הציווי קודם מפני התחברותה למצוות הפסח, זהו הדרך הראשון.הדרך השני שהמצה נצטוו עליה מטעם מהירות הגאולה והיציאה כמו שהזכרתי, אולם אם היתה המצווה נתונה להם אחרי צאתם אפשר שלא היו מרגישים במהירות הגאולה ולא ידעו ולא הבינו טעם המצווה לפי המצווה לפי אמיתתה, לכן התחכם הקדוש ברוך הוא לצוות המצוה בעודם במצרים, ולהיותה המצוה הראשונה שנצטוו בה היו זריזין עליה מאוד ולשו עסותיהם בחשבם שיהיה להם פנאי לאפות המצות המצרים, אך קודם האפיה בא פרעה ועבדיו לילה לאמור קומו צאו מתוך עמי, ויצאו בחיפזון גדול וישאו את בצקם צרורות בשמלותם על שכמם כי לא יכלו להתמהמה, והיו בני ישראל ונשותיהם חרדים ומצטערים על עיסותיהם שמא יחמץ ויחטאו לאלוהים במצווה הראשונה אשר ציווה אותם, וכאשר הגיעו לסוכות או למקום אחר שיכלו לאפות את בצקם ומצאו עוגות מצות כי לא חמץ ואז הכירו וידעו שבחיפזון גדול יצאו וכי הגדיל ה' לעשות עמהם נס ופלא, וההיכרא הזאת הגיעה אליהם בעבור שנצטוו קודם היציאה באותה מצווה. לכן נתן רבן גמליאל טעם המצווה שלא הספיק וכו' שהוא טעם המצווה באמת, מפני שנצטוו עליה לאותה סיבה שכתבתי, זהו הדרך השני.והדרך השלישי והוא היותר נכון אצלי הוא שראה רבן גמליאל בסגנון הכתובים שני טעמים במצוות מצות, האחד מורה על הגלות להיותו לחם עוני שהוא כחוש וקשה ונאות לעמלים ועבדים וכמו שהיו אוכלים בארץ מצרים, ולכן נאמת במצוות הפסח "על מצות ומרורים יאכלוהו" שהפסח יבוא אחרי המצות והמרורים, לפי שכן באה מכת בכורות אחרי הגלות ומרירות העוני. וטעם שני הורה עליו משה רבינו עליו השלום במשנה תורה באומרו "שִׁבְעַת יָמִים תֹּאכַל עָלָיו מַצּוֹת לֶחֶם עֹנִי כִּי בְחִפָּזוֹן יָצָאתָ מֵאֶרֶץ מִצְרַיִם לְמַעַן תִּזְכֹּר אֶת יוֹם צֵאתְךָ מֵאֶרֶץ מִצְרַיִם כֹּל יְמֵי חַיֶּיךָ" (דברים ט"ז, ג') שביאר היות המצווה זכר ליציאה המהירה ולא לגלות. וכן נאמר בפרשת קדש לי כל בכור – "וַיֹּאמֶר מֹשֶׁה אֶל הָעָם זָכוֹר אֶת הַיּוֹם הַזֶּה אֲשֶׁר יְצָאתֶם מִמִּצְרַיִם" וגו' "וְלֹא יֵאָכֵל חָמֵץ" (שמות י"ג, ג') וגו' "שִׁבְעַת יָמִים תֹּאכַל מַצֹּת" (שם שם, ו') וגו'. ומאשר מצא רבן גמליאל הסתירה הזאת בטעם המצווה באומרו שהמצווה שנצטוו עליה במצרים לא היתה מאותו טעם המצווה שנצטוו עליה לדורות, כי המצווה עם הפסח במצרים היתה זכר לגלות ולעינוי השעבוד, ואותה מצווה שציווה לדורות היתה זכרון לחיפזון ומהירות הגאולה, ולזה אמר רבן גמליאל מצה זו שאנו אוכלים בדור הזה על שום מה, כי בענין הפסח אמר פסח שהיו אבותינו אוכלין בזמן שבית המקדש קיים, וביאר בו שפסח דורות היה מאותו טעם שהיה פסח מצרים לפי שפסח ה' על בתי ישראל וכו'. אל המצה שאנו אוכלין, רצה לומר שנצטווינו בה לדורות לא היתה מאותו טעם שנצטוו בה ישראל בפסח מצרים. היינו מפני העינוי כמו המרורים, אלא המצה שאנו אוכלין על שום מה שלא הספיק בצקם של אבותינו להחמיץ וכו' ולזכר מהירות הגאולה נעשה המצווה הזאת. ומפני כן בתחילת ההגדה אמרו "הא לחמא עניא די אכלו אבהתנא בארעא דמצרים", שאותה הכרזה עשינו בה בחינה מהגלות והעינוי כמו שפירשתי שם. וכאן נתן רבן גמליאל טעם המצווה להורות על שום שלא הספיק וכו'.אמנם אין ראוי לחשוב כי מה שאמר "עד \textrm{\textbf{שנגלה}} עליהם מלך מלכי המלכים וגאלם מיד" מורה על אותו הגבול שלא הספיק בצקם להחמיץ עד שנגלה עליהם הקדוש ברוך הוא במכת בכורות, כי שם היתה התגלות ה' כמו שנאמר "ועברתי בארץ מצרים" וגו', לפי שלא היה הנס של הבצק שלא נחמץ כי אם אחר היציאה והגאולה עד שבאו לאפות את בצקם לא עד מכת בכורות, כי אז לשו את הבצק ולא היה לו עדיין זמן להחמיץ, ועוד שהכתוב שמביא לראיה אינו מדבר בכלל בהתגלות אלהות.אבל האמת בוא שמילת "עד" במקום הזה אינו מורה על גבול כמו "עַד בֹּא הַשֶּׁמֶשׁ" (שם כ"ב, כ"ה), אלא נאמר על המשך הזמנים וכאילו אמר שלא הספיק בצקם להחמיץ \textrm{\textbf{בעוד}} שנגלה עליהם מלך לכי המלכים, שהתגלות היתה במכת בכורות ופסח על הבתים, ובעוד שהיתה התגלות הזאות יצאו ממצרים, ובכל הזמן הזה לא הספיק בצקם להחמיץ. והביא ראיה לזה מפסוק "וַיֹּאפוּ אֶת הַבָּצֵק אֲשֶׁר הוֹצִיאוּ מִמִּצְרַיִם עֻגֹת מַצּוֹת כִּי לֹא חָמֵץ" (שמות י"ב, ל"ט). ואם תאמר למה לא אפו אותם במצרים? זהו לפי שלא יכלו להתמהמה וגם צדה לא עשו להם, רצה לומר מזון אחר מלבד הלחם כמו בשר ודגים שהיו ראוי להם לעשות צדה לדרכם לא עשו למהירות יציאתם, כי גם הם בעצמם לא היו מאמינים שתהיה יציאתם כל כלך בחיפזון.ואפשר לומר עוד בזה שהתגלות האלהות שהזכיר רבן גמליאל אינו במכת בכורות כי אף שהיה ה' עובר במצאים לא נגלה כבודו לעיני העם, אבל היה עמוד הענן. וידוע שזה בא לישראל בהגיעם לסוכות, שנאמר "ויסעו מסוכות" וגו' "וה' הולך לפניהם יומם", ולכן אמר שעד אותה ההתגלות על ידי עמוד הענן שבא אליהם בסוכות לא הספיק בצקם להחמיץ ושם אפו אותו.והותר בזה הספק אשר בשער צ"ו.}%endcomment
\hebeng{{\small אוחז המרור בידו ומראה אותו למסובין: } }{{\small He holds the \textit{marror} in his hand and shows it to the others there.} }
\hebeng{מָרוֹר זֶה שֶׁאָנוּ אוֹכְלִים, עַל שׁוּם מַה? עַל שׁוּם שֶׁמֵּרְרוּ הַמִּצְרִים אֶת־חַיֵּי אֲבוֹתֵינוּ בְּמִצְרַיִם, שֶׁנֶּאֱמַר: וַיְמָרְרוּ אֶת חַיֵּיהם בַּעֲבֹדָה קָשָה, בְּחֹמֶר וּבִלְבֵנִים וּבְכָל־עֲבֹדָה בַּשָּׂדֶה אֶת כָּל עֲבֹדָתָם אֲשֶׁר עָבְדוּ בָהֶם בְּפָרֶךְ. }{This \textit{marror} {[bitter greens]} that we are eating, for the sake of what {[is it]}? For the sake {[to commemorate]} that the Egyptians embittered the lives of our ancestors in Egypt, as it is stated (Exodus 1:14); "And they made their lives bitter with hard service, in mortar and in brick, and in all manner of service in the field; in all their service, wherein they made them serve with rigor." }%
\commenta{\textrm{\textbf{מרור זה שאנו אוכלים על שום שמררו המצרים את חיי אבותינו במצרים.}} סבת אכילת מרור אינה דומה לסבות אכילת פסח ומצה, שהם עניני ישועה, פסח על שום שפסח הקב״ה על הבתים בשעת נגף הבכורים ומצה על מהירות היציאה, ושייך להודות עליהם, אבל באכילת מרור אין רמז כל ישועה, ולא עוד אלא שמזכיר דאגה וצער ממרירות החיים, ומה יחש זכירה והודאה על זה. אך על האמת גם בזה יש יחש רגשי הודאה, כי לפי הידוע בתורה בפרשה לך (ט״ו י״ג) הוגבל זמן השעבוד ארבע מאות שנה, ומבואר במדרשים, דקושי השעבוד ומרירות החיים גרמו לחירותם ק״צ שנה קודם זמנם (ראה החשבון ברש״י פרשה בא (י״ב מ״א), ואם כן שייך הודאה גם על מרור על שמיהר השעבוד. ולדעת המדרשים שהבאנו, דקושי השעבוד ומרירות החיים גרמו לצאתם בקדמות הזמן, לדעה זו אפשר למצוא סמך בתורה בהקדם מש״כ הרא״ש, בבאורו לנדרים (ל״ז ב׳) דטעמי הנגינות שעל המלים מורים כונות מיוחדות, וכאן על הלשון וימררו את חייהם (ר״פ שמות) הנגינה ״קדמא ואזלא״ ובא לרמז שמסבת מרירות החיים ״קדמו ואזלו״.}%endcomment%
\commentb{\textrm{\textbf{תשובה לשער צ"ז}}\textrm{\textbf{מרור זה שאנו אוכלים על שום מה על שום שמררו המצריים את חיי אבותינו וכו'.}}זהו הדבר השלישי אשר חובה עלינו לומר בליל הסדר, ועל טעם המרור אמר שהוא זכרון השעבוד ומרירות חיי אבותינו. כי לא אמר רבן גמליאל שהיה המרור זכר לעבדות כי אם למרירות חייהם בעבודתם הקשה. והנה התורה צוותה בפסח מצרים "על מצות ומרורים יאכלוהו", לפי שהמצה תרמוז אל העבדות הגופני והמרור ירמוז על המרירות הנפשי, ועליהם היה נאכל הפסח לרמוז שבעבור עבודתם ומרירות חייהם אשר עשו להם המצריים בפרך באה עליהם מכת בכורות ופסח ה' על בתי בני ישראל.ואף כי בפרשת "משכו וקחו לכם" לא נזכר ענין המרור כבר ביארתי סיבתו שהפרשה ההיא לא באה כי אם להזהיר הזקנים שיהיו מקדימין לעניין לקיחת הפסח ושחיטתו ולא דיבר בפרשה כלל מאכילתו. גם בפרשת "זאת חוקת התורה" לא נזכר המרור לפי שלא באה אותה פרשה כי אם לפרט מי ומי האוכלים את הפסח, וכמו שאמר "זֹאת חֻקַּת הַפָּסַח כָּל בֶּן נֵכָר לֹא יֹאכַל בּוֹ (שמות י"ב, מ"ג)" אמנם בשנה השנית צווה ה' יתעלה שיעשו את הפסח, ואמר שם בֵּין הָעֲרְבַּיִם תַּעֲשׂוּ אֹתוֹ בְּמוֹעֲדוֹ כְּכָל חֻקֹּתָיו וּכְכָל מִשְׁפָּטָיו תַּעֲשׂוּ אֹתוֹ. (במדבר ט', ג')"ובזה נכלל עניין המרור שהוא מכלל המשפטים שציווה בו. ולכן כשציווה על הטמאים לנפש אדם שיעשו פסח שני אמר "בַּחֹדֶשׁ הַשֵּׁנִי בְּאַרְבָּעָה עָשָׂר יוֹם בֵּין הָעַרְבַּיִם יַעֲשׂוּ אֹתוֹ עַל מַצּוֹת וּמְרֹרִים יֹאכְלֻהוּ לֹא יַשְׁאִירוּ מִמֶּנּוּ עַד בֹּקֶר וְעֶצֶם לֹא יִשְׁבְּרוּ בוֹ כְּכָל חֻקַּת הַפֶּסַח יַעֲשׂוּ אֹתוֹ" (שם שם י"א - י"ב). וידוע כי פסח שני לא היה בו הבדל כלל מפסח ראשון כי אם בהיותו בחודש השני. וכיון שנזכר בו עניין המרור ואמר עליו "ככל חוקת הפסח" ידענו ששלושת הדברים היו חובה לפסח הדורות והם פסח מצה ומרור. ולכן עשה רבן גמליאל חובה לזכרון שלושתם בלילה הזה בדרך שאלה "על שום מה" בכל אחד מהם ובתשובת טעמו.וחז"ל אמרו ששלושה שמות נקראו לו, מרור חסא חזרת, מרור על שום שמררו המצריים את חיי אבותינו, ומפני שהמרור מתוק ואחריתו מרה כלענה, וכן היתה ארץ מצרים לישראל מתוקה בימי יוסף ואחר כך נעשה להם מרה;  חסה מפני שחס הקדוש ברוך הוא עליהם שנאמר ויפן עליהם ברחמים ויחונם; וחזרת לפי שהיו ישראל חוזרין על הפתחין. ולמדנו מזה המאמר שכל אחד משלושה אלה יש להם שני רמזים ושתי הוראות הפכיות זו מזו, כי הנה הפסח מורה על מכת בכורות ולקוי מזל טלה שהיה מושל עליהם, שהיא מכת חרב והרג ואובדן, ויורה גם כן על החמלה והרחמים כמו שאמר "אֲשֶׁר פָּסַח עַל בָּתֵּי בְנֵי יִשְׂרָאֵל בְּמִצְרַיִם בְּנָגְפּוֹ אֶת מִצְרַיִם וְאֶת בָּתֵּינוּ הִצִּיל" (שמות י"ב, כ"ז). וכן בעניין המצה שהיא תרמוז לעניים ולחצם ועמלם, וכמו שנאמר "הא לחמא עניא די אכלו אבהתנא בארעא דמצרים". ותורה גם כן על מהירות גאולתם ויציאתם משם שלא הספיק בצקם להחמיץ, וכן המרור היה רמז למרירות חייהם בגלות. ורמז השם חסא שחס עליהם הקדוש ברוך הוא. הרי לך אחד משלושתם המורה על טוב ורע. ויורה גם על הוראה אחרת באשר אזכיר אחר כך.והותר בזה הספק אשר בשער צ"ז.}%endcomment
\hebeng{בְּכָל־דּוֹר וָדוֹר חַיָּב אָדָם לִרְאוֹת אֶת־עַצְמוֹ כְּאִלּוּ הוּא יָצָא מִמִּצְרַיִם, שֶׁנֶּאֱמַר: וְהִגַּדְתָּ לְבִנְךָ בַּיּוֹם הַהוּא לֵאמֹר, בַּעֲבוּר זֶה עָשָׂה ה׳ לִי בְּצֵאתִי מִמִּצְרַיִם. לֹא אֶת־אֲבוֹתֵינוּ בִּלְבָד גָּאַל הַקָּדוֹשׁ בָּרוּךְ הוּא, אֶלָּא אַף אוֹתָנוּ גָּאַל עִמָּהֶם, שֶׁנֶּאֱמַר: וְאוֹתָנוּ הוֹצִיא מִשָּׁם, לְמַעַן הָבִיא אוֹתָנוּ, לָתֶת לָנוּ אֶת־הָאָרֶץ אֲשֶׁר נִשָׁבַּע לַאֲבֹתֵינוּ. }{In each and every generation, a person is obligated to see himself as if he left Egypt, as it is stated (Exodus 13:8); "And you shall explain to your son on that day: For the sake of this, did the Lord do {[this]} for me in \textit{my} going out of Egypt." Not only our ancestors did the Holy One, blessed be He, redeem, but rather also us {[together]} with them did He redeem, as it is stated (Deuteronomy 6:23); "And He took us out from there, in order to bring us in, to give us the land which He swore unto our fathers."}%
\commenta{\textrm{\textbf{בכל דור ודור חייב אדם לראות את עצמו כאלו הוא יצא ממצרים, שנאמר, והגדת לבנך ביום ההוא לאמר בעבור זה עשה ה׳ לי בצאתי ממצרים.}} לא נתבאר איפה מרומז כאן צוי לדורות, והלא לפי המשך וענין הפרשה איירי זה בדור המדבר.  ואפשר לבאר עפ״י זה, שלשון זה נמשך ונוסד על הפסוק הקודם (פ׳ בא׳ י״ג ה׳) והיה כי יבאך ה׳ אל ארץ הכנעני... ועבדת את העבודה הזאת והגדת לבנך, וידוע, שדור המדבר לא נכנס לארץ ישראל, כמבואר בפ׳ שלח ובפ׳ דברים. ורק בניהם נכנסו, ואם כן מוסב זה הלשון והגדת לבנך על הדור המאוחר, דור שנכנס לא״י, והן הוא לא יצא, ואם כן הי׳ לו לומר בעבור זה עשה ה׳ לאבותי בצאתם ממצרים, ומדאמר הכל בסגנון מדבר בעדו, עשה לי, בצאתי. מתבאר, דבכל דור ודור חייב אדם לראות את עצמו כאלו הוא יצא ממצרים, ועל כן תולה הדברים בו. ועוד אפשר לפרש דיוק חיוב זה מפסוק והגדת לבנך וכו׳ עפ״י הידוע בגמרא שהלשון לאמר, שנראה לפעמים כאינו מוכרח לענין, מתפרש לפעמים כמו "לאמר לאחרים״ (ועיין יומא ד׳ ב׳), וכאן כתיב והגדת לבנך לאמר בעבור זה עשה ה׳ לי והיינו שעוד יאמר הבן לאחרים שיאמרו זה גם הם, וממילא מתבאר, דחיוב האמירה לדורות לכל דור ודור, ובאמור כל אחד את הלשון בעבור זה עשה ה׳ לי בצאתי ממצרים, מתבאר דבכל דור ודור חייב אדם לראות את עצמו כאלו הוא יצא, כדמפרש.}%endcomment%
\commentb{\textrm{\textbf{תשובה לשער צ"ח}}\textrm{\textbf{בכל דור ודור חייב אדם להראות את עצמו כאילו הוא יצא ממצרים}} וכו'.כבר כתבתי בשערים מה שיש במאמר הזה מן הספק, אם במה שציווה שיראה האדם את עצמו כאילו הוא יצא ממצרים, כי זה יצדק ביורשי הארץ שאילו לא הוציא הקדוש ברוך הוא את אבותינו ממצרים לא היו הם בני חורין שרים ונכבדים בעלי קרקעות ומלכים מהם יצאו, אבל אנחנו בגלותנו שלא ירשנו ארץ ולא זכינו לבוא עליה ונולדנו בגלות, מהו החיוב אשר לנו להראות את עצמנו כאילו יצאנו ממרים מכח אותה ראיה ואותנו הוציא משם? ואם הדבר הזה לא יצדק בנו, איך הוכיח משם "לפיכך אנו חייבין להודות"? אחרי שאין אנו באותו לחיוב מזה הפסוק?ומה שראוי לומר בזה הוא שהמגיד אמר ש"בכל דור ודור חייב אדם להראות את עצמו כאילו הוא יצא ממצרים", והיה זה לפי שביציאה משם קנינו שרשי האמונה ופינותיה במציאות הסיבה הראשונה יתעלה והשגחתו באמונתנו בכלל ובפרט, ויכולתו הבלתי בעל תכלית לשנות הטבעיים ולבטל המערכות העליונות, ונתאמתה גם כן אצלינו שם פינת חידוש העולם, לפי שמעשה הניסים והנפלאות מורות על פעל עליון רצוני המשנה הטבעיים ברצונו, מה שהוא בלתי אפשר אם לא היה בורא אותם. ונתאמתה גם כן פינת הידיעה האלהית בפרטי בני אדם והשגחתו בשכר ועונש. ונתאמתה גם כן אצלנו פינת הנבואה שמה שראינו מייעודי משה והתראתו לפרעה בכל המכות. ונמשכה אחרי היציאה ממצרים קבלת התורה והמצות האלהיות אשר הם חיינו ואורך ימינו, וכמו שנאמר "בְּהוֹצִיאֲךָ אֶת הָעָם מִמִּצְרַיִם תַּעַבְדוּן אֶת הָאֱלֹהִים עַל הָהָר הַזֶּה" (שמות ג', י"ב).ונמשכה עוד אחרי יציאת מצרים ירושת הארץ הנבחרת ומעלת האומה בה מלכיה שריה וכהניה וכל עם הארץ, וקדושת בית המקדש והשריית השכינה בתוכנו ומציאות הנבואה בנביאינו, שכל זה משתלשל ונמשך ומתחייב מיציאת מצרים. וכיוון שהקדוש ברוך הוא לא יעשה ניסים ונפלאות בכל דור ודור להיותם בלתי ראויים וזכאים אליו, יעצה חכמתו העליונה שנעשה תמיד זכר ליציאת מצרים ולהזכיר המסות הגדולות אשר ראו עינינו באופן שיעתק זכרונו מן האבות אל הבנים ויתמידו האמונות האמיתיות ולא ימושו מפינו עד עולם. וכמו שאמר "לְמַעַן תִּזְכֹּר אֶת יוֹם צֵאתְךָ מֵאֶרֶץ מִצְרַיִם כֹּל יְמֵי חַיֶּיךָ" (דברים ט"ז, ג'). לפי שהאמונה היא שלמות הנפש ומעלתה וסיבת השארתה. ולהיות זה שורש האמונות כולן וערותם לכן באו ברוב המצוות זכר ליציאת מצרים, ויהיה זה כאשר יתפעל בזכרון הניסים ההם ויתבונן בהם ויאמין בפינות אשר יורו עליהם, ויגיד כאילו הוא ראה אותם בעיניו והוא היה מיוצאי מצרים.והנה הביא ראיה לזה ממה שנאמר בסדר ואתחנן בפרשת הבן החכם, ודע שלא כיוון לבד לפסוק הנזכר כאן "וְאוֹתָנוּ הוֹצִיא מִשָּׁם לְמַעַן הָבִיא אֹתָנוּ לָתֶת לָנוּ אֶת הָאָרֶץ אֲשֶׁר נִשְׁבַּע לַאֲבֹתֵינוּ" (דברים ו', כ"ג), אלא גם כן לאותם הפסוקים הנמשכים אחריו, ולכן אמר כאן וגו' רצה לומר גמור שאר הפסוקים שבאו אחרי זה שהם "וַיְצַוֵּנוּ ה' לַעֲשׂוֹת אֶת כָּל הַחֻקִּים הָאֵלֶּה לְיִרְאָה אֶת ה' אֱלֹהֵינוּ לְטוֹב לָנוּ כָּל הַיָּמִים לְחַיֹּתֵנוּ כְּהַיּוֹם הַזֶּה וּצְדָקָה תִּהְיֶה לָּנוּ כִּי נִשְׁמֹר לַעֲשׂוֹת אֶת כָּל הַמִּצְוָה הַזֹּאת לִפְנֵי ה' אֱלֹהֵינוּ כַּאֲשֶׁר צִוָּנוּ" (דברים ו', כ"ד – כ"ה), רצה לומר שהמעשים האלה שנעשה בלילה הזה באיזה זמן שיהיה הלא הם יביאונו אל החיים הנפשיים הנצחיים ולחיים הגופיים, אם כן תהיה מצווה זו צדקה לפני ה', כלומר במחיצה העליונה בעולם הרוחני לפניו. ועל כן אמרה תורה "וְזָכַרְתָּ כִּי עֶבֶד הָיִיתָ בְּאֶרֶץ מִצְרַיִם" (דברים ה', י"ד), כאילו אתה בעצמך היית שם עבד ונפדית, ומה טוב דייקו חז"ל הפסוק שאמרו "ואותנו הוציא משם", שכיוון שכבר נאמר למעלה "ויוציאנו ה' ממצרים" לא היה צורך לומר ואותנו הוציא משם, כי אם להגיד שחייב אדם להראות את עצמו כאילו הוא יצא ממצרים. הנה התבאר לפי זה שלא כיוון המגיד לפי הפסוקים האלה אל ירושת הארץ כי אם לשאר השלמות של הנפש והגוף. ולכן הוציא מזה שאנו חייבין להודות להלל לשבח ולפאר וכו'.והנה יש במאמר "לפיכך" שבעה לשונות של שבח, וכפי דרך חז"ל כנגד שבעה רקיעים, ויש אומרים שהם כנגד שבעת הרועים, כי אמר "להודות" כנגד אברהם אבינו שנתן הודאה לה', ואמר "להלל" כנגד יצחק אבינו, ואמר "לשבח" כנגד יעקב אבינו, ואמר "לפאר" על אהרן שפארו הקדוש ברוך הוא בכהונה, "ולרומם" על משה רבינו שרוממו בנבואה, "להדר" על דוד מלך ישראל, "ולקלס" על שלמה שקלס את הקדוש ברוך הוא.ומה שנראה לי בזה הוא שרומז על שבעה אמונות גדולות שלמדנו ביציאת מצרים:א – מציאות הסיבה הראשונה כמו שאמר "אָנֹכִי ה' אֱלֹהֶיךָ אֲשֶׁר הוֹצֵאתִיךָ מֵאֶרֶץ מִצְרַיִם" (שמות כ', ב'), ואמר " בְּזֹאת תֵּדַע כִּי אֲנִי ה'" (שם ז', י"ז) וגו'. כי אך שיחידי האומות שלמים היו באמונתם אבל רוב העם לא היה עובד עיונם מהמורגש, כמו שכתב הרב המורה.ב – פינת אחדותו כמו שאמר במכת בכורות "אני ה' אני הוא ולא אחר" (מתוך ההגדה).ג' – ידיעת הפרטית כמו שאמר "וַיֹּאמֶר ה' רָאֹה רָאִיתִי אֶת עֳנִי עַמִּי אֲשֶׁר בְּמִצְרָיִם וְאֶת צַעֲקָתָם שָׁמַעְתִּי מִפְּנֵי נֹגְשָׂיו כִּי יָדַעְתִּי אֶת מַכְאֹבָיו" (שמות ג', ז').ד' – השגחתו בתתו שכר ועונש לראויים להם. וכמו שאמר  לְמַעַן תֵּדַע כִּי אֲנִי ה' בְּקֶרֶב הָאָרֶץ" (שמות ח', י"ח).ה' – יכולתו הבלתי בעל תכלית המוחלט בשנותו הטבעיים כרצונו, וכמו שאמר "בַּעֲבוּר תֵּדַע כִּי אֵין כָּמֹנִי בְּכָל הָאָרֶץ" (שמות ט', י"ד).ו' – חידוש העולם, לפי שהאותות והמופתים הם עדים נאמנים על הבריאה הראשונה הרצונית הכוללת, ועליו אמר הנביא "ה' אֱלֹהַי אַתָּה אֲרוֹמִמְךָ אוֹדֶה שִׁמְךָ כִּי עָשִׂיתָ פֶּלֶא עֵצוֹת מֵרָחוֹק אֱמוּנָה אֹמֶן" (ישעיהו כ"ה, א'), רצה לומר שעם מעשה הפלאים יקנו אמונה קיימת בעצות הרחוקות שהוא סיפור בריאת העולם וחידושו.ז' – פינת הנבואה אשר נתאמתה בייעודי משה בכל מכה ומכה אשר באו.ומפני שבע האמונות האלה אשר קנינו ביציאת מצרים לשלמות נפשותינו היו שבעה מיני ההודעה הנזכרים.ואפשר לומר בזה עוד שהזכיר שבעה לשונות של שבח כנגד שבעה דברים ומעלות טובות שקנו ביציאת מצרים והן:א – חירות, ב – נקמה מאויביהם, ג – ממון ושלל, ד – כבוד ומעלה, ה – אמנה שלמה, ו – התורה האלהית, ז' – ירושת הארץ הנבחרת. ושבעת המעלות הטובות שחשב רבי עקיבא נכללות באלו.ולפי שהמעלות האלה קנו באמצעות הניסים לכן אמר למי שעשה לאבותינו ולנו את כל הניסים האלה, וחזר וכלל את כל המעולות האלה וכל הטובות בחמישה מינים: א – הוציאנו מעבדות לחירות, ב – משעבוד לגאולה, ג – מיגון לשמחה, ד – מאבל ליום טוב, ח' – מאפלה לאור גדול.לפי שהיו להם בעבודתם חמש רעות גדולות:הראשונה היותם נכנעים תחת אחרים, כי ההכנעה איזו שתהיה היא רעה רבה, וכנגד זה אמר "הוציאנו מעבדות לחירות", היינו להיות בני חורין ולא נכנעים לשום אדם.השניה שהיו משתעבדים בהן בעבודת פרך כאילו בני ישראל היו להם עבדים נרצעים מקנה כספם, ועל זה אמר "משעבוד לגאולה".השלישית כי ישראל בהיותם שני מעלה וכבוד לא היו סובלים עבודתם ברצון כאשר יעשו הכושים והעבדים שלא טעמו טעם חירות, אבל תמיד היתה עבודתם ביגון ואנחה לפי שלא נסו באלה, על כן אמר כנגד זה "ומיגון לשמחה".הרביעית היותם עבדים שמה לא היה להם יום מנוחה ושביתה לא חודש ולא שבת ולא יום טוב, עד שמפני זה נאמר במצוות השבת "וְזָכַרְתָּ כִּי עֶבֶד הָיִיתָ בְּאֶרֶץ מִצְרַיִם" (דברים ה', י"ד) וגו', כי במצרים לא היו יכולים לשבות. וכנגד זה אמר "ומאבל ליום טוב" כי שם בעבודתם היו תמיד באבלות וכאשר באה גאולתם מיד ניתן להם חג הפסח שהוא יום טוב.החמישית בהיותם המצרים היו משוללים מהאמונה ותהי האמת נעדרת אצלם ובבוא גאולתם יצאו מאפלת הסכלות לאור גדול אור האמונה והתורה אשר קבלו. ובעבור זה היה ראוי לומר לפניו הללויה, להלל ולהודות לשמו שגמלם כרחמיו וכרוב חסדיו.ואפשר עוד לומר שכיוונו בזה המאמר "חייב אדם להראות את עצמו כאילו הוא יצא ממצרים", לבאר ענין אמיתי, והוא שכל אחד ואחד מישראל ימצאוהו בגלות הזה משעבוד מלכיות בפרטיותו מה שקרה לאומה בכללה במצרים, כי יש מהם שיתן ה' אותם לרחמים לפני שוביהם, אבל אף שלא יתקפוהו צרות לא ימלא מהיות נכנע ועבד והאומה המושלת עליו, ויש מהם שתכבד העבודה עליו מן האויבים העובדים בו בפרך, ויש מהם שסבבוהו כמים בלהות יגונות ואנחות שניא דא מם דא, ויש מהם שיהיה פעמים אסור באזיקים וסוכל סכנות עצומות, ומהם שלא יוכלו לשמור את יום השבת ומועדי ה' מפני חמת המציק, והקדוש ברוך הוא מקים להם מושיע ורב בדרכים נפלאים, אם בהכות אויביהם ואם בדרכים אחרים, וכמו שכתב הרמב"ן כי כל מעשה ה' עמנו בגלות הזה הם ניסים נסתרים. ולזה תקנו לומר שכל אדם יראה עצמו כאילו הוא יצא ממצרים, לפי שאי אפשר מבלתי שיעברו עליו בגלותינו צרות ממיני צרות, יש בגופן, ומהם בממונם ומהם בבניהם, ומהם בחילול שבתות ויום טוב ממצוקת האויבים, וכיוון שהשם יתברך מציל אותנו בגלות בכל יום על ידי מושיע שגואל במסות באותות ובמופתים לפיכך ראוי הוא שיראה כל איש את עצמו כאילו הוא יצא ממצרים, שלא את אבותינו בלבד גאל באותה הגאולה הכוללת אלא אף אותנו הוא פודה ומציל בכל יום מצרות שונות כאשר עשה להם, ועל כן הכתוב אומר "ואותנו הוציא משם" לא אמר ואותם כי אם אותנו, לפי שכל אחד ואחד ממנו נגאל פעמים רבות בימיו בגלותו. ולפיכך אנו חייבים להודות לשבח וכו'. והזכיר שבעה מיני הודאה כנגד שבעה מיני פורעניות שבאין לעולם ונזכרים במשנה אבות, וכל שכן שהגיעו לנו הפורעניות בגלותינו, וההודאות הן לה' על הכלל ועל הפרט, על הראשונות ועל האחרונות, על גאולת האבות ועל גאולתנו אנחנו מצרות הפרטיות ליחידים הבאות עלינו. לכן תקנו לומר "למי שעשה לאבותינו ולנו את כל הניסים האלה", שגם לנו עשה נס גאולה ותשועה ופורקן כהיום הזה.ואחשוב גם כן שאמר "הוציאנו מעבדות לחירות" כנגד גלות מצרים, "ומשעבוד לגאולה" כנגד גלות בבל, "ומיגון לשמחה" כנגד גלות פרס ומדי, "ומאבל ליום טוב" מצרות יוון, "ומאפלה לאור גדול" בכל יום ויום בגלות אדום. ובאו חמישה המאמרים האלה אחד על מצרים וארבעה על ארבע מלכיות, שבכולן היתה לנו תשועה וגאולה אם לכלל ואם ליחידים. ומפני זה תקנו שאחרי קריאת שני פרקי ההלל נברך "אשר גאלנו וגאל את אבותינו ממצרים", היינו גאולתנו בפרטיות בגלותנו וגאולת אבותינו ממצרים גאולה כוללת. ולכן יאמר "והגיענו הלילה הזה", רצה לומר כי מרוב הצרות כמעט כלונו בארץ וחסדי ה' כי לא תמנו, והוא ברחמיו הגיענו הלילה הזה לאכול בו מצה ומרור וכו'. ולא הזכיר בזה את הפסח, לפי שבחוצה לארץ לא נוכל לזבוח את הפסח. וראוי היה לתת הודאה ושבח לפניו יתברך על כל זה מפני שמי שנהנה מן העולם הזה בלא ברכה כאילו מעל וגזל את הקדוש ברוך הוא, וכמאמר הנביא המתרגם על זה "יָדַע שׁוֹר קֹנֵהוּ" (ישעיהו א', ג'), ולכן אדוננו משה צווה לבל נהיה כפויי טובה שנאמר " וְאָכַלְתָּ וְשָׂבָעְתָּ וּבֵרַכְתָּ אֶת ה' אֱלֹהֶיךָ" (דברים ח', י').וראוי שתדע גם כן עם זה שזכרון גאולת מצרים אצלנו הוא תקווה רבה והודאה גדולה על הגאולה העתידה, וכמו שאמר הנביא "כִּימֵי צֵאתְךָ מֵאֶרֶץ מִצְרָיִם אַרְאֶנּוּ נִפְלָאוֹת" (מיכה ז', ט"ו), ואמר "הן גאלתי אתכם אחרית כראשית"15תוספת למוסף של שבת ויום טוב כל פי תקנת הסבוראים אחרי חורבן בית שני., להגיד שהיו שתי הגאולות קשורות זו בזו, וכמאמר הנביא "יוֹסִיף ה' שֵׁנִית יָדוֹ לִקְנוֹת אֶת שְׁאָר עַמּוֹ" (ישעיהו י"א, י"א) וגו', עד שמפני זה אמרו חז"ל שנאמר למשה במראה הסנה "אהיה אשר אהיה", לומר אהיה עמהם בצרה זו אהיה עמהם בצרה אחרת של שעבוד מלכיות, כי סמך זו לזו להיות מצרניות, ועוד דרשו על "שְׁלַח נָא בְּיַד תִּשְׁלָח" (שמות ד', י"ג) שרמז על מלך המשיח, אותו שאתה עתיד לשלח בעולם, ואף על פי שאלה הם דברי אגדה אין בהם נפתל ועקש, ולכן תקנו שנעשה בזריזות בכל חוקת הפסח ומשפטיו, לפי שגם זה הוא לנו תמיד עדות ברורה על הגאולה העתידה, ובעבוד זה ראוי לומר בברכת הגאולה: "כן ה' יגיענו למועדים ולרגלים אחרים הבאים לקראתנו לשלום, ויהיו בגאולה כוללת שלמה שמחים בבנין עירך ושמחים בעבודתך", רצה לומר עתה בגלות לא נוכל לקיים מצוות "ושמחת בחגך" לפי שאמור לאדם למלא שחוק פיו, אמנם בשוב ה' את שיבת ציון אז נגילה ונשמחה. ואף על פי שעתה אין לנו חג כי אם במצה ומרור, הנה אם נאכל מן הזבחים שהם שלמי חגיגה ומן הפסחים שהיו נאכלים על השובע, כשיגיע דמם על קיר מזבחך לרצון, ואז נודה לך שיר חדש מלבד הלל על יציאת מצרים, ואותו שיר חדש יהיה על גאולתנו ועל פדות נפשנו משעבוד מלכיות. וחותם "גאל ישראל", רצה לומר הגואל והמושיע במצרים את ישראל, כי על זה באה הברכה הזאת בלבד.הנה נתבארו המאמרים האלה וברכת הגאולה והותר הספק אשר בשער צ"ח.}%endcomment
\newsection{חצי הלל}
\hebeng{{\small יאחז הכוס בידו ויכסה המצות ויאמר: } }{{\small He holds the cup in his hand and and he covers the matsa and says:} }
\hebeng{לְפִיכָךְ אֲנַחְנוּ חַיָּבִים לְהוֹדוֹת, לְהַלֵּל, לְשַׁבֵּחַ, לְפָאֵר, לְרוֹמֵם, לְהַדֵּר, לְבָרֵךְ, לְעַלֵּה וּלְקַלֵּס לְמִי שֶׁעָשָׂה לַאֲבוֹתֵינוּ וְלָנוּ אֶת־כָּל־הַנִסִּים הָאֵלּוּ: הוֹצִיאָנוּ מֵעַבְדוּת לְחֵרוּת מִיָּגוֹן לְשִׂמְחָה, וּמֵאֵבֶל לְיוֹם טוֹב, וּמֵאֲפֵלָה לְאוֹר גָּדוֹל, וּמִשִּׁעְבּוּד לִגְאֻלָּה. וְנֹאמַר לְפָנָיו שִׁירָה חֲדָשָׁה: הַלְלוּיָהּ. }{Therefore we are obligated to thank, praise, laud, glorify, exalt, lavish, bless, raise high, and acclaim He who made all these miracles for our ancestors and for us: He brought us out from slavery to freedom, from sorrow to joy, from mourning to {[celebration of]} a festival, from darkness to great light, and from servitude to redemption. And let us say a new song before Him, Halleluyah!}%
\commenta{\textrm{\textbf{לפיכך אנחנו חייבים להודות ולהלל... לעלה ולקלס.}} וכלשון זה בא בתפלת שחרית לשבת בפיוט ובמקהלות רבבות עמך, וכתבנו שם, שקשה מאוד לסבול קביעות המלה ״ולקלס״, אשר בכל המקרא, בתנ״ך באה להורות לעג וחרפה. בוז והתעוללות, כמו תשימנו לעג וקלס לסביבותינו (תהילים מ״ד:י״ד), לחרפה ולקלס כל היום (ירמיהו כ׳:ח׳), חרפה בגוים וקלסה לכל הארצות (יחזקאל כ״ב:ד׳), והוא במלכים יתקלס (חבקוק א׳:י׳), ועוד. וכן הוקבעה מלה זו בתפלת תחנון, הבט משמים וראה כי היינו לעג וקלס בגוים, ואיך זה באה מלה בזויה ומתועבת זו לכאן בסיום לכל המעלות והשבחים וההלולים שקדמוה, והרי זה ככוס רעל הבא לאחר מיני מעדנים. ואמת, כי בלשון התלמוד מצינו מלה זו בהוראת כבוד, כמו בברכות (נ׳ א׳), וביבמות (צ״ב סע״ב) או לאו דקלסך גברא רבה, ולפי זה אפשר לומר כמ״ש במס׳  מכות (ח׳ א׳), דכל מלה שיש לה שתי הוראות צריך להבין בכל אחת לפי הענין, וכן אמרו ביבמות (ק״ב ב׳) ביחש הוראת מלה אחת, משמע הכי ומשמע הכי. אך להצטדקות זאת יש מקום רק אם המלה בשתי ההוראות נובעת ממקור שפה אחת, אם משפת עברית או מלשון התלמוד וכדומה, אבל כאן ובתפלת שחרית לשבת שהבאנו באה כולה בתוך הרבה מלים מקוריים מלשה״ק ובלשה״ק, בכל מקום בואה תורה אך גנאי וחרפה, ואיך זה מלים תכופות מהוראות משונות זו מזו ישתמשו כאחד, וזה פלא. ועל דעתי ראוי להשמיטה כאן ומתפלת שבת שהבאנו ונצדק קודש וכבוד שמים. וסמך להשמטה זו, כי בנוסח תפלה זו ברב אלפס ליתא למלה זו, ובירושלמי פסחים פ״י ה״ד וברמב״ם בנוסח ההגדה באה תחת מלה זו המלה ״ולנצח״. והנה עפ״י שלשה עדים נאמנים אלה תתקיים ישרת השמטה מלה טרודה וכאבה זו, ודי בלעדה.\textrm{\textbf{הוציאנו מעבדות לחירות מיגון לשמחה ומאבל ליו״ט ומאפלה לאור גדול ומשעבוד לגאולה.}} בנוסח ההגדה ברמב״ם כתוב מעבדות לחרות ומשעבוד לגאולה ואח״כ מיגון לשמחה וכו׳, ולכאורה זו נוסחה ישרה, יען כי עבדות וחרות עם שעבוד וגאולה מתיחשים אלה עם אלה, אלא שערוכים בסגנון ״לא זו אף זו״, כלומר, לא רק הוצאנו מקושי עבדות אך גם מסתם שעבוד בלא עבודה. אבל יש לקיים הנוסח שלפנינו עפ״י מ״ש במס׳ נדה (ס״ט א׳) בשנוי לשון קצת, תחלתו שהוא טוב וסופו שהוא טוב, וכן כאן, תחלתו חופש (מעבדות לחירות) וסופו חופש (משעבוד לגאולה), וסמך חופש לחופש.\textrm{\textbf{הוציאנו מעבדות לחירות מיגון לשמחה ומאבל ליו״ט ומאפלה לאור גדול ומשעבוד לגאולה.}} מדרך התלמוד ומדרשים, כי כל דבר וענין הבא במספר לתלות ערך המספר כנגד עוד דבר הבא במספר כזה (ואולי הוא כדי לסייע לזכרון המספר ואתו יחד הענינים). כה מצינו למשל מספר ארבע כוסות בפסח כנגד ארבע לשונות של גאולה (בר״פ וארא) והוצאתי, והצלתי, וגאלתי ולקחתי (ירושלמי פסחים פ״י ה״א) ובברכות (כ״ט א׳), הני שבע ברכות בתפלה של שבת כנגד שבעה קולות על המים שאמר דוד (תהלים כ״ט), הני תשע תפלות שברה״ש כנגד תשע אזכרות שאמרה חנה בתפלתה (כלומר, כנגד תשע פעמים שזכרה חנה שם ה׳ במאמר אחד, שמואל א׳, ב׳ א׳—י׳) ועוד שם כאלה. וברה״ש (ל״ב א׳) הני עשרה מלכיות שבמוסף רה״ש (כלומר, עשרה פסוקים שנזכר בהם מלכיות) כנגד עשרת הדברות וכנגד עשרה מאמרות שבהם נברא העולם. ובירושלמי ברכות (פ״ד ה״ג), הני י״ח ברכות שבתפלת חול כנגד שמונה עשרה צווין שבפרשת המשכן (כאשר צוה ה׳ את משה, בפ׳ פקודי) ועוד הרבה כאלה. ומעתה אף אנו נאמר, הני חמשה לשונות של טיב הגאולה, מעבדות לחירות, מיגון לשמחה מאבל ליו״ט, מאפלה לאור גדול ומשעבוד לגאולה, כנגד חמש זכיות שהי׳ לישראל במצרים, כמבואר במדרשים, שלא שינו את שמם ואת לשונם ואת בגדיהם ושלא היו בהם דלטורים (מלשינים) וגם היו עתידים לקבלת התורה (מ״ר שמות), והיו אלה כנגד אלה.}%endcomment
\hebeng{הַלְלוּיָהּ הַלְלוּ עַבְדֵי ה׳, הַלְלוּ אֶת־שֵׁם ה׳. יְהִי שֵׁם ה׳ מְבֹרָךְ מֵעַתָּה וְעַד עוֹלָם. מִמִּזְרַח שֶׁמֶשׁ עַד מְבוֹאוֹ מְהֻלָּל שֵׁם ה׳. רָם עַל־כָּל־גּוֹיִם ה׳, עַל הַשָּׁמַיִם כְּבוֹדוֹ. מִי כַּיי אֱלֹהֵינוּ הַמַּגְבִּיהִי לָשָׁבֶת, הַמַּשְׁפִּילִי לִרְאוֹת בַּשָּׁמַיִם וּבָאָרֶץ? מְקִימִי מֵעָפָר דָּל, מֵאַשְׁפֹּת יָרִים אֶבְיוֹן, לְהוֹשִׁיבִי עִם־נְדִיבִים, עִם נְדִיבֵי עַמּוֹ. מוֹשִׁיבִי עֲקֶרֶת הַבַּיִת, אֵם הַבָּנִים שְׂמֵחָה. הַלְלוּיָהּ. }{Halleluyah! Praise, servants of the Lord, praise the name of the Lord. May the Name of the Lord be blessed from now and forever. From the rising of the sun in the East to its setting, the name of the Lord is praised. Above all nations is the Lord, His honor is above the heavens. Who is like the Lord, our God, Who sits on high; Who looks down upon the heavens and the earth? He brings up the poor out of the dirt; from the refuse piles, He raises the destitute. To seat him with the nobles, with the nobles of his people. He seats a barren woman in a home, a happy mother of children. Halleluyah! (Psalms 113)}%
\commenta{באורנו להלל תמצא בחבורי על תפלות השנה}%endcomment%
\commentb{\textrm{\textbf{תשובה לשער צ"ט}}\textrm{\textbf{הללויה הללו עבדי ה' ההלו את שם ה' יהי שם ה' מבורך מעתה ועד עולם וכו'.}}כבר כתבתי בשערים על הספק למה תקנו להפסיק בקריאת ההלל ושיהיה נקרא בדילוג לפרקים ממנו קודם הסעודה עד "למעינו מים", וחותם בברכת הגאולה, ושם ינוחו ויאכלו וישתו, ואחר הסעודה יגמרו שאר פרקי ההלל ויברכו ברכת השיר. ולמה לא תהיה קריאתו מדובקת אם קודם הסעודה או לאחריה מבלי דילוג והפסקה? ולמה לא נאמר ברכת הגאולה על גמירתו?ומה שנראה לי בזה הוא שחז"ל שיערו בהלל הזה שני חלקים, החלק הראשון מתחילתו עד "למעינו מים" ידבר מיציאת מצרים וקריעת ים סוף, ולכן קראוהו "הלל המצרי" לפי שקבלה בידם שנאמר על יציאת מצרים. ובמדרש אמרו "הַלְלוּיָהּ הַלְלוּ עַבְדֵי ה'", זהו שאמר המשורר "אֶזְכְּרָה נְגִינָתִי בַּלָּיְלָה" (תהלים ע"ז, ז'), רבי יהודה בן רבי סימון אומר אמרה כנסת ישאל לפני הקדוש ברוך הוא בכל שנה ושנה אזכרה הניסים שעשית עמי בלילה במצרים והייתי מנגן לך על אותם הניסים ואמרתי הלל ושירים שנאמר "הַשִּׁיר יִהְיֶה לָכֶם כְּלֵיל הִתְקַדֶּשׁ חָג" (ישעיהו ל', כ"ט), ואימתי כשהרגת בכורי מצרים נאמר "ויהי בחצות הלילה" באותו לילה נגאלנו ויצאנו מעבדות לחרות, באותו לילה לא היינו עבדים לפרעה ונעשינו עבדים להקדוש ברוך הוא שנאמר "הַלְלוּיָהּ הַלְלוּ עַבְדֵי ה'" (תהלים קי"ג).ובפרק ערבי פסחים (פסחים קי"ז, א') "אמר רב יהודה אמר שמואל שיר שבתורה משה וישראל אמרוהו בשעה ששעלו מן הים, הללויה מי אמרו? נביאים שביניהם תקנו אותו לישראל שהיו אומרים אותו על גאולת מצרים ועל כל צרה שלא תבוא עליהם, וכשיגאלו יאמרו אותו על גאולתם". ושם רבו הדעות בהלל זה מי אמרו, יש מי שסבר כי משה רבינו עליו השלום תקנו לישראל , ויש מה שסבר כי יהושע וישראל תקנוהו כשעמדו עליהם מלכי כנען, ור' אליעזר המודעי אמר שדבורה וברק אמרוהו כשעמד עליהם סיסרא, ר' אלעזר בן עזריה סבר שחזקיהו אמרו על מפלת סנחריב, ורבי עקיבא אמר שחנניה מישאל ועזריה אמרוהו, וחכמים הסכימו עם דעת הראשון שזקנים באותו הדור תקנו להם לישראל.עוד אמרו שם וכי מאחר דאיכא הלל הגדול (הודו) מאי טעמא אמרינן האי הלולא? אמר ר' יוחנן מפני שיש בה חמשה דברים: יציאת מצרים וקריעת הים שנאמר "הים ראה וינס", מתן תורה דכתיב "ההרים רקדו כאלים", תחיית המתים דכתיב "אתהלך לפני ה' בארצות החיים", חבלו של משיח שנאמר "לא לנו ה' לא לנו", איכא לאמרי תנא רבי יוחנן לא לנו ה' לא לנו על מלחמת גוג ומגוג. (פסחים קי"ח, א'). הרי לך מבואר ונגלה מדבריהם ז"ל כי שני הפרקים הראשונים מהלל עד "למעינו מים נאמרו על יציאת מצרים, ולכן עשו בו הפסקה, ותקנו אחריו ברכת הגאולה שענינה ההודאה על גאולת מצרים. ותקנו שיעשו אחריו מצוות המצה והמרור ויאכלו הסעודה. ואחריו יבוא החלק השני מן ההלל וענינו לעתיד לבוא לזמן גליות, וכמו שאמר ר' יוחנן באלה הפרקים האחרונים על תחיית המתים וחבלו של משיח ומלחמת גוג ומגוג. ומפני זה לא חברו אותו עם החלק הראשון אלא הפסיקו בסעודה ביניהם, ולפי שהוא עניין בפני עצמו נבדל מהראשון. ובטעם ארבע כוסות אוסיף עוד הביאור על זה, ודי בזה להתר הספק אשר בשער צ"ט.ובִילַמדנו אומרו כל הניסים שנעשו לישראל בזכות אברהם נעשו, יציאת מצרים בזכותו שנאמר "כִּי זָכַר אֶת דְּבַר קָדְשׁוֹ אֶת אַבְרָהָם עַבְדּוֹ וַיּוֹצִא עַמּוֹ בְשָׂשׂוֹן" (תהלים ק"ה, מ"ב – מ"ג) וגו'; קריעת ים סוף בזכותו שנאמר "לְגֹזֵר יַם סוּף לִגְזָרִים" (שם קל"ו, י"ג)  וכתיב " אֲשֶׁר עָבַר בֵּין הַגְּזָרִים הָאֵלֶּה" (בראשית ט"ו, י"ז); קריעת ירדן בזכותו שנאמר "וַיַּעַמְדוּ הַמַּיִם הַיֹּרְדִים מִלְמַעְלָה קָמוּ נֵד אֶחָד הַרְחֵק מְאֹד  מֵאָדָם הָעִיר אֲשֶׁר מִצַּד צָרְתָן" (יהושע ג', ט"ז), שמעת מימיך עיר שנקראת אדם? אלא זה אברהם האדם הגדול בענקים; מתן תורה בזכותו שנאמר "עָלִיתָ לַמָּרוֹם שָׁבִיתָ שֶּׁבִי לָקַחְתָּ מַתָּנוֹת בָּאָדָם" (תהלים ס"ח, י"ט) (והוא אברהם כנ"ל). ובעבור שבזכות אברהם זכו ישראל להגאל ממצרים נזכר בספר תהלים "הַלְלוּ יָהּ אַשְׁרֵי אִישׁ יָרֵא אֶת ה'" (שם קי"ב, א') קודם הלל המצרי זה, לפי שהמזמור ההוא נאמר על אברהם אבינו כי הוא היה האיש המאושר ירא ה' באמת, כמו שאמר המלאך "עַתָּה יָדַעְתִּי כִּי יְרֵא אֱלֹהִים אַתָּה" (בראשית כ"ב, י"ב), ואמר "בְּמִצְו‍ֹתָיו חָפֵץ מְאֹד" (תהלים קי"ב, א') כי כן העיד עליו השם יתברך שנאמר " עֵקֶב אֲשֶׁר שָׁמַע אַבְרָהָם בְּקֹלִי וַיִּשְׁמֹר מִשְׁמַרְתִּי מִצְו‍ֹתַי חֻקּוֹתַי וְתוֹרֹתָי" (בראשית כ"ו, ה'), ועליו אמר גבור בארץ יהיה זרעו דור רשעים יבורך, לפי שבלעם בברכתו אמר על זרעו "הֶן עָם כְּלָבִיא יָקוּם וְכַאֲרִי יִתְנַשָּׂא" (במדבר כ"ג, כ"ד) ואמר " תָּמֹת נַפְשִׁי מוֹת יְשָׁרִים וּתְהִי אַחֲרִיתִי כָּמֹהוּ" (שם שם, י'). הנה שתאר אותם כגיבורים וישרים, ולכן אמר "יבורך" לעמוד על ברכת בלעם שברך אותם בזה. עוד אמר על אברהם  הוֹן וָעֹשֶׁר בְּבֵיתוֹ" (תהלים קי"ב, ג') להעיד על עשרו וכבודו, ולפי שמעלת העושר אינה שלמות מפאת עצמה אלא בעבוד הצדקה שיעשה אדם ממנו, לכן אמר  וְצִדְקָתוֹ עֹמֶדֶת לָעַד" (תהלים קי"ב, ג'), כי הצדקה היא הנשארת לעד אחר המוות. עוד אמר זָרַח בַּחֹשֶׁךְ אוֹר לַיְשָׁרִים" (שם שם, ד') לפי שאברהם האיר לבני אדם בלמוד האמונה האלהית כמו שאמר "וְאֶת הַנֶּפֶשׁ אֲשֶׁר עָשׂוּ בְחָרָן" (בראשית י"ב, ה'). וזכר שהיו לו כל המדות הטובות לפי תכונת המקבלים כי הוא היה חנון ורחום למי שהיה ראוי לחון עליו, לצדיק למי שהיה צריך לצדק, ולאחרים היה חונן ומלוה באופן שהיה מכלכל דבריו במשפט. ולפי ששמו וזכרו היה נצחי בעולם אמר "כִּי לְעוֹלָם לֹא יִמּוֹט לְזֵכֶר עוֹלָם יִהְיֶה צַדִּיק" (תהלים קי"ב, ו'). ובשביל שעצר כח להלחם במלכים כדי להציל את לוט בן אחיו כשבאה השמועה כי נשבה, לכן אמר "מִשְּׁמוּעָה רָעָה לֹא יִירָא... עַד אֲשֶׁר יִרְאֶה בְצָרָיו" (שם שם, ז' – ח'), רצה לומר שניצח אותם ולקח כל אשר להם. ולפי שאברהם השיב את מלך אדום כל הרכוש לכן אמר "פִּזַּר נָתַן לָאֶבְיוֹנִים" (שם שם, ט') ולפי שהשם יתברך שבחו על זה כמו שאמר "שְׂכָרְךָ הַרְבֵּה מְאֹד" (בראשית ט"ו, א'), על כן אמר "צִדְקָתוֹ עֹמֶדֶת לָעַד קַרְנוֹ תָּרוּם בְּכָבוֹד" (תהלים קי"ב, ט') מהניצחון, גם אמר קרנו תרום בכבוד כנגד המלכות והממשלה שניתנה לזרעו שנקרא תמיד עם אלהי אברהם. ולפי שפרעה הרשע השתדל להכניע את בניו ולהסיר מהם קרן הכבוד ולא עלה בידו, לכן אמר "רָשָׁע יִרְאֶה וְכָעָס שִׁנָּיו יַחֲרֹק" מרוב כעסו "וְנָמָס" ליבו בקרבו ו"תַּאֲוַת רְשָׁעִים" שהם המצריים לאבד את ישראל "תֹּאבֵד" (תהלים קי"ב, י').ולהיות המזמור הזה כולו נאמר על אברהם וזרעו וכנגד פרעה והמצריים אויביהם, לכן נסמך אליו הלל המצרי "הַלְלוּיָהּ הַלְלוּ עַבְדֵי ה'" (תהלים קי"ג), רצה לומר אתם בני ישראל הללו את ה' ושבחו לשמו על אשר עשה עמכם להפליא ביציאת מצרים. והנה אמר שם יה, וכבר ידעת אמרם שבחצי השם ברא ה' את עולמו שנאמר "כִּי בְּיָהּ ה' צוּר עוֹלָמִים" (ישעיהו כ"ו, ד'), ואחשוב שכוונו לומר בזה שאנחנו נבחון בשם ה' אלהיו שתי בחינות, האחת בחינת מהותו ועצמותו, ועל זה לא יפול ההלל והשבח, כי אין אנחנו משיגים ממנו דבר, וכמו שיתבאר עוד בברכת ה'. והבחינה השניה מצד פעולותיו במה שהוא בורא העולם ופועלו ושומר אותו. לכן שני השמות באחד יה ה', שהראשון הוא שם המפורש הנבדל, ולא ישתתף בו אחרף והשני הוא שם פעולתו כי הוא מלשון הויה, לפי שהוא יתעלה ברא העולם ומתהווה אותו. ומשני השמות האלה אין ספק שהראשון לא יסמך ולא יתייחס לדבר מה כי לא יפול יחס כלל ולא צורך בין ה' יתעלה ודבר אחר. ורק השם השני שהוא כפי פעולותיו וסמיכתו, כמו שנאמר "ה' צבאות", "ה' אלהי השמים", ה' אלהי ישראל". לפי שהסמיכות הוא בבחינת היותו פועל הדברים ושומר אותם. והעיר על זה עצמו אמרו שברא העולם בחצי השם, רצה לומר בחינת הפעולה, ועליה אמר "כִּי יָד עַל כֵּס יָהּ" (שמות י"ז, ט"ז), רצה לומר שמגיע השמים שהוא הכסא הוא יה כלומר מהבחינה השניה. ולרמוז על זה נאמר כאן "הללויה", רצה לומר שההילול והשבח אליו יתברך יהיה כפי חצי השם שהוא בבחינת פעולותיו לא כפי עצמו. ולכן אמרו במדרש "אמר רבי ירמיה אין העולם כדאי להלל בכל השם אלא בחציו שנאמר כל הנשמה תהלל יה הללויה" (מדרש תהלים קי"ג).והנה כאן ביאר בעניין ההלול והשבח ארבעה דברים:א – מי הוא זה ואיזה הוא הראוי והלל ולשבח.ב – על מי נהלל ונשבח.ג – איזה הוא הזמן הראוי לשבח בו.ד – באיזה מקום או ארץ יאות ההלל והשבח.וכנגד הראשון שהוא מצד המשבח ביאר ואמר "הַלְלוּ עַבְדֵי ה'" (תהלים קי"ג, א'), רצה לומר אל תחשבו שכל פה וכל לשון וכל בני אדם רשאים להלל את השם הנכבד, אינו כן כי אם עבדי ה' הדבקים בו – להם יאות ההלול והשבח, וכמו שנאמר במקום אחר "הִנֵּה בָּרְכוּ אֶת ה' כָּל עַבְדֵי ה' הָעֹמְדִים בְּבֵית ה' בַּלֵּילוֹת" (תהלים קל"ד, א') כי להיותם עבדיו ושוקדים על דלתותיו וחצות לילה יקומו להודות לפיו יאות עליהם השבח וההלול. ומפני זה הבדיל ה' את שבט לוי לשבחו ולהללו, כמו שנאמר "רננו צדיקים בה' לישרים נאוה תהילה" כי לישרים בליבותם היא נאוותו. וכמו שחקנו בתפילת שבת של שחרית "בפי ישרים תתרומם ובלשון חסידים תתקדש ובדברי צדיקם תתברך ובקרב קדושים תתהדר" זהו הנכלל באומרו "הללויה הללו עבדי ה'".אמנם בענין השני שהוא בבחינת המשובח אמר " הַלְלוּ אֶת שֵׁם ה'" (תהלים קי"ג, א') ורצה בזה שלא יחשוב שיוכל בפיו ובלשונו להלל ולשבח האלוה יתברך כפי שלמות מעלתו ועצמותו, כי אנחנו לא נשיג מהותו ושלמותו היא עצמותו, ואין לו תארים עצמיים וכל שכן שאין בו תארים מקריים, ואיך נוכל להללו?  לכן אמר "הללו את שם הַלְלוּ אֶת \textrm{\textbf{שֵׁם ה'}}" כי אין אנו משיגים ממנו רק השם בלבד אשר למדנו מפי הנביאים, ואותו השם תהללו, כי לא תוכלו להשיג יותר מזה. ולמדנו זה ראשונה ממשה אדוננו באומרו "כִּי שֵׁם ה' אֶקְרָא הָבוּ גֹדֶל לֵאלֹהֵינוּ" (דברים ל"ב, ג'). ולפי שמעלת שלמותו יתברך בלתי מושגת אף למלאכי השרת, ואמרו חז"ל כי קילוסם להקדוש ברוך הוא הוא "ברוך \textrm{\textbf{שם}} כבוד מלכותו לעולם ועד".וכנגד העניין השלישי והוא מהו הזמן אשר בו יאות להלל את השם יתברך, אמר "יְהִי שֵׁם יְהוָה מְבֹרָךְ מֵעַתָּה וְעַד עוֹלָם" (תהלים קי"ג, ב'), וענינו מה שאמרו במדרש, את מוצא כ"ו דורות מעת שנברא העולם עד שיצאו ישראל ממצרים ובכל אותם הדורות לא אמרו הלל ושירה, וכשבאה מכת בכרות אמרו "הללויה הללו עבדי ה'", רצו בזה לומר שבאותם הדורות הראשונים לא הכירו בני אדם שנוי הטבעיים וביטול המערכות העליונות על דרך פלא, ורק במכת בכורות שפקד ה' על צבא המרום במרום ועל בכורי האדמה באדמה אז הכירו וידעו כל יושבי תבל כי לה' המלוכה והוא שודד כוחות השרים העליונים כרצונו, וזהו עניין ההלל שלא אמרו עד מכת בכורות, ומשם ואילך היה מהולל שם ה' לפי שנתפרסמה השגחתו ויכלתו. ועל זה אמר "יהי שם ה' מבורך מעתה ועד עולם", רצה לומר מעתה שראו ניסי מצרים ואותותיו ונפלאותיו ראוי הוא כי מעתה ולנצח נצחים יהי מבורך ומהולל שם ה'.אמנם כנגד הענין הרביעי שהוא המקום הנאות להללו אמר "מִמִּזְרַח שֶׁמֶשׁ עַד מְבוֹאוֹ מְהֻלָּל שֵׁם ה'" (שם שם, ג') ושאר הפסוקים ועניינם שאין ספק שבכל חלקי יישוב בני ממזרח השמש ועד מערבו השם יתברך הוא מהולל להיותו הסיבה הראשונה לכל הדברים, וזה שגור בפי כל האומות, וכמו שנאמר "כִּי מִמִּזְרַח שֶׁמֶשׁ וְעַד מְבוֹאוֹ גָּדוֹל שְׁמִי בַּגּוֹיִם וּבְכָל מָקוֹם מֻקְטָר מֻגָּשׁ לִשְׁמִי" (מלאכי א', י"א), ואמרו חז"ל "דקרו ליה אלהא דאלהיא" (מנחות ק"י, א'), רצה לומר שכולם מודים במציאות סיבה ראשונה לכל הדברים, אבל ענין אותן האומות הוא ליחס אל האל יתברך רוממות ומעלה, ולכן יסלקו ממנו הידיעה וההשגחה בדברים השפלים, וזהו שנאמר "רָם עַל כָּל גּוֹיִם ה' עַל הַשָּׁמַיִם כְּבוֹדוֹ" (תהלים קי"ג, ד') שהם מיחסים אליו יתברך רוממות ההנהגה ותנועה העליונה, ויחשבו שמפני רוממותו לא ישגיח בדברים אשר בכאן, כמו שנאמר "אֲשֶׁ֣ר יֹ֭מְרוּךָ לִמְזִמָּ֑ה" (תהלים קל"ט, כ'), שיחסו אליו יתברך הרוממות לתכלית הרע, והוא לסלק ממנו ההשגחה בשפלים. אבל אנחנו בני ישראל לא נאמין בזה, אלא שעם כל רוממותו ומעלתו הוא משגיח בשפלים, וה0וא מה שאמר "מִ֭י כה' אֱלֹהֵ֑ינוּ הַֽמַּגְבִּיהִ֥י לָשָֽׁבֶת" (תהלים קי"ד, ה'), ואמר בו ישיבה להורות על ה נצחיות של הקיום, ובכל זאת הוא "משפילי לראות", רצה לומר שהוא רם ונישא וגם רואה ומשגיח בדברים השפלים, ואמר "בשמים ובארץ" כנגד מה שאמר "המגביהי לשבת המשפילי לראות", כאילו אמר המגביהי לשבת בשמים ומשפילי לראות בארץ, ולעוצם השגחתו פעמים רבות הוא מקים מעפר העם או האדם הדל או העני, ומאשפות ירים האביון שאין לו מאומה והוא בתכלית השפלות, כמו שהיה ישראל במצרים, ומאותו שפלות המופלג, יעלהו בתכלית המעלה. ואף על פי שמערכות השמיימיות יחייבו את האיש או את העם הזה להיות נבזה וחדל אישים בזוי ושסוי, הנה הקדוש ברוך הוא יבטל כח מזלו וירימהו משפלותו, והוא אמרו "לְהוֹשִׁיבִ֥י עִם נְדִיבִ֑ים" (תהלים קי"ג, ח'). ואף על פי שאין נביא בעירו ומי שהיה שפל ונבזה לא יחשב במדינתו לאיש גדול המעלה, הנה ברצות ה' יעלה הנבזה ההוא כל כך עד שיושיבהו עם נדיבי עמו בענין העושר והכבוד והמעלה. ואחר שהזכיר נפלאותיו בעושר ובכבוד הוסיף לו בנים ואמר "מֽוֹשִׁיבִי עֲקֶרֶת הַבַּיִת אֵם הַבָּנִים שְׂמֵחָה" (שם שם, ט') שהקדוש ברוך הוא על ידי נפלאותיו ישנה את הטבע והמערכה השמיימית כדי להושיב בהרווחה את האישה שהיתה עקרה ודואגת ועצבת לב להעדר הבנים בתוך ביתה, כי יתן לה בנים רבים ושלמים שתשמח בהם. וכיוון שכך הוא, לכן אתם עבדי ה' הללויה ושבחוהו מפאת גבורותיו על העליונים ועל התחתונים.והנה אודות ה-י' של המשפילי ומקימי, להושיבי ומושיבי, כתבו המדקדקים שהם נוספות ושיבואו ליפוי הלשון, אך אפשר לומר שבא ה-י' במקומות אלה לרמוז שהוא יתברך עשה אותות ומופתים כנגד המערכות שהם תשעה הגלגלים וכנגד הטבע שהם י' פועלים מסודרים, והוא יתברך גבוה עליהם לבטל פעולותיהם כרצונו. ויש מפרשים חמישה ה-י' האלה הנוספות הם כנגד חמישים מכות שלקו המצריים על הים, ואחרי שהניח זכרון ה' ויכולתו, הביא ראיה עליו מיציאת מצרים, כי שם ראו בחוש כל הגבורות והנפלאות הנזכרות, והוא אמרו "בְּצֵאת יִשְׂרָאֵל מִמִּצְרָיִם בֵּית יַעֲקֹב מֵעַם לֹעֵז" (תהלים קי"ד, א') לפי שכל מי שאינו מדבר בלשון הקודש יקרא לועז, ואולי קרא למצריים עם לועז מפני עזותם, ואמר שכאשר יצאו משם "הָיְתָה יְהוּדָה לְקָדְשׁוֹ יִשְׂרָאֵל מַמְשְׁלוֹתָיו" (שם שם, ב'), שהפריש והבדיל הקדוש ברוך הוא לעם יהודה להיות לו עם קדוש, ולקח את ישראל תחת ממשלתו. והזכיר יהודה וישראל לפי שתמיד היו בישראל שתי כתות, שבט יהודה בפני עצמו ושאר השבטים שנקראו יעקב ובית ישראל כת בפני עצמה, ולא היה זה רק כאשר נחלקה המלוכה בימי רחבעם וירבעם, אלא גם מעת שיצאו ממצרים היה יהודה לראש ונוסע ראשונה כברכת יעקב אבינו, וגם בימי שאול תמצא כשמנה את העם שאמר "וַיִּפְקְדֵם בְּבָזֶק" נאמר "וַיִּפְקְדֵם בְּבָזֶק וַיִּהְיוּ בְנֵי יִשְׂרָאֵל שְׁלֹשׁ מֵאוֹת אֶלֶף וְאִישׁ יְהוּדָה שְׁלֹשִׁים אָלֶף" (שמואל א', י"א, ח'), וכן בימי דוד נזכר שמלך בחברון על איש יהודה שבע שנים ואחר כך מלך על ישראל שלשים ושלוש שנים. וכאשר ציווה את יואב לפקוד את העם אמר "לֵךְ מְנֵה אֶת יִשְׂרָאֵל וְאֶת יְהוּדָה" (שמואל ב' כ"ד, א'), ובמניין נאמר "וַתְּהִי יִשְׂרָאֵל שְׁמֹנֶה מֵאוֹת אֶלֶף אִישׁ חַיִל שֹׁלֵף חֶרֶב וְאִישׁ יְהוּדָה חֲמֵשׁ מֵאוֹת אֶלֶף אִישׁ" (שמואל ב' כ"ד, ט'), הרי לך שתמיד עם היות לשבטים כולם ראש אחד ומלך אחד היה שבט יהודה כת בפני עצמה ושאר השבטים כולם כת אחרת, ועל זה הדרך נאמר כאן "הייתה יהודה לקדשו ישראל ממשלותיו".אמנם אמרו חז"ל במדרש כי שבט יהודה נכנס בים סוף ראשונה, לפי שהיו ישראל יראים להכנס בים ועמד נחשון נשיא שבט יהודה וקפץ בים תחילה ועמו כל שבט יהודה ואחריו כל ישראל, על כן נאמר "היתה יהודה לקדשו" לפי שיהודה קדש את ה' בקרב ישראל, ואמר המשורר שלהיות יהודה וישראל קודש לה' הים ראה וינס, שים סוף נקרע מפניהם כאדם הנס ובורח מגיבור ממנו, ושכן קרה לירדן שנקרע גם כן ויסב לאחור בבוא יהושע וכל ישראל אליו, ויקרעו מי הירדם וביבשה עברו ישראל ומה שנאמר "הֶהָרִים רָקְדוּ כְאֵילִים" (תהלים קי"ד, ד') אפשר לפרש על מלכי האומות, כאש שמעו קריעת ים סוף וקריעת הירדן, כמו שנאמר "שָׁמְעוּ עַמִּים יִרְגָּזוּן חִיל אָחַז יֹשְׁבֵי פְּלָשֶׁת" (שמות ט"ו, י"ד). ויותר נכון לפרש על ההרים והגבעות אשר בתוך הים שרעשו גם הם ונִמוֹחו כדי ליישר הדרך אל העם. ואולי אמר ההרים רקדו כאלים על מתן תורה, שההרים נמסו מלפני ה' והר סיני היה מזדעזע מפני אשר ירד עליו ה' באש, וכאילו סדרך מליצה ישאל לים "מַה לְּךָ הַיָּם כִּי תָנוּס הַיַּרְדֵּן תִּסֹּב לְאָחוֹר" (תהלים קי"ד, ה') ומה לכם ההרים שתרקדו כאלים והגבעות כבני צאן? מה היה השנוי הזה בכם כי אין זה דרככם וטבעכם? והם ישיבו שכן הוא באמת שלא היה להם זה כפי הסדר הטבעי כי אם על דרך הפלא וביכולת הבורא, וזה שאמר "מִלִּפְנֵי אָדוֹן חוּלִי אָרֶץ מִלִּפְנֵי אֱלוֹהַּ יַעֲקֹב" (שם שם, ז') שהם שתי סיבות הפועלת והתכלית. רצה לומר בקריעה היתה בכוח האדון שברא העולם כולו ויסד ארץ על ארבע יסודות, כמו שבא בפסוק "בראשית ברא אלהים את השמים ואת הארץ" כאילו אמר שהאדון אשר ברא את הטבע הוא יכול לשנותו כחפצו. ו-י' במילת "חוּלי" כתבו המדקדקים שהוא במקום י' הכפל, כאילו אמר חולל הארץ ויוצרה, וכזה העיד לעניין פועל הנס, אמנם לעניין התכלית אמר "מפני אלוה יעקב", רצה לומר שלהיותו אלוה יעקב ביטל הסדר הטבעי והמערכה העליונה כדי להצילו, ואין לתמוה על זה, אנו ראינו שעשה הקדוש ברוך הוא מהמים ארץ, כי גם כן יעשה בהיפך מהארץ מים. וזהו "ההופכי הצור אגם מים", ולא היה זה באבנים הספוגיים בלבד, כי אם גם החלמיש שהוא האבן הקשה והחזק יהפוך אותו למעינו מים, כי האבנים והצורים החזקים שהם בטבע הארץ, הנה בכוחות הגדול ובזרועו הנטויה יהפכו לאגמים ומעיינות, כי הוא פועל ההפכים, וכן בים סוף עדה מהים מהמים אבנים וארץ. ובמדבר עשה מאבנים מים.ובזה נשלם החלק הראשון מההלל שידבר מקריעת ים סוף ויציאת מצרים, ולכן נעשה בו הפסקה, כמו שכתבתי. ובאה עליו ברכת הגאולה, לפי שנתקנה על גאולת מצרים. וכבר פירשתי למעלה את הברכה הזאת. וסדר המצוות הנעשות בלילה הזה אני אסדר אחרי תשלום ביאור ההלל כדי שלא להפסיק בפירושו.}%endcomment
\hebeng{בְּצֵאת יִשְׂרָאֵל מִמִצְרַיִם, בֵּית יַעֲקֹב מֵעַם לֹעֵז, הָיְתָה יְהוּדָה לְקָדְשׁוֹ, יִשְׂרָאֵל מַמְשְׁלוֹתָיו. הַיָּם רָאָה וַיַּנֹס, הַיַּרְדֵּן יִסֹּב לְאָחוֹר. הֶהָרִים רָקְדוּ כְאֵילִים, גְּבַעוֹת כִּבְנֵי צֹאן. מַה לְּךָ הַיָּם כִּי תָנוּס, הַיַּרְדֵּן – תִּסֹּב לְאָחוֹר, הֶהָרִים – תִּרְקְדוּ כְאֵילִים, גְּבַעוֹת כִּבְנֵי־צֹאן. מִלְּפְנֵי אָדוֹן חוּלִי אָרֶץ, מִלְּפְנֵי אֱלוֹהַ יַעֲקֹב. הַהֹפְכִי הַצּוּר אֲגַם־מָיִם, חַלָּמִיש לְמַעְיְנוֹ־מָיִם. }{In Israel's going out from Egypt, the house of Ya'akov from a people of foreign speech. Yehudah became His -holy one, Israel, His dominion. The Sea saw and fled, the Jordan turned to the rear. The mountains danced like rams, the hills like young sheep. What is happening to you, O Sea, that you are fleeing, O Jordan that you turn to the rear; O mountains that you dance like rams, O hills like young sheep? From before the Master, tremble O earth, from before the Lord of Ya'akov. He who turns the boulder into a pond of water, the flint into a spring of water. (Psalms 114)}
\newsection{כוס שניה}
\hebeng{{\small מגביהים את הכוס עד גאל ישראל. } }{{\small We raise the cup until we reach "who redeemed Israel"} }
\hebeng{בָּרוּךְ אַתָּה ה׳ אֱלֹהֵינוּ מֶלֶךְ הָעוֹלָם, אֲשֶׁר גְּאָלָנוּ וְגָאַל אֶת־אֲבוֹתֵינוּ מִמִּצְרַיִם, וְהִגִּיעָנוּ הַלַּיְלָה הַזֶּה לֶאֱכָל־בּוֹ מַצָּה וּמָרוֹר. כֵּן ה׳ אֱלֹהֵינוּ וֵאלֹהֵי אֲבוֹתֵינוּ יַגִּיעֵנוּ לְמוֹעֲדִים וְלִרְגָלִים אֲחֵרִים הַבָּאִים לִקְרָאתֵנוּ לְשָׁלוֹם, שְׂמֵחִים בְּבִנְיַן עִירֶךְ וְשָׂשִׂים בַּעֲבוֹדָתֶךָ. וְנֹאכַל שָׁם מִן הַזְּבָחִים וּמִן הַפְּסָחִים אֲשֶׁר יַגִּיעַ דָּמָם עַל קִיר מִזְבַּחֲךָ לְרָצון, וְנוֹדֶה לְךָ שִׁיר חָדָש עַל גְּאֻלָּתֵנוּ וְעַל פְּדוּת נַפְשֵׁנוּ. בָּרוּךְ אַתָּה ה׳, גָּאַל יִשְׂרָאֵל.}{Blessed are You, Lord our God, King of the universe, who redeemed us and redeemed our ancestors from Egypt, and brought us on this night to eat matsa and \textit{marror}; so too, Lord our God, and God of our ancestors, bring us to other appointed times and holidays that will come to greet us in peace, joyful in the building of Your city and happy in Your worship; that we shall eat there from the offerings and from the Pesach sacrifices, the blood of which shall reach the wall of Your altar for favor, and we shall thank You with a new song upon our redemption and upon the restoration of our souls. Blessed are you, Lord, who redeemed Israel. }%
\commenta{\textrm{\textbf{אשר גאלנו וגאל את אבותינו.}} ולמעלה בנוסח ההודאה לפיכך אומרים להודות להלל... למי שעשה נסים לאבותינו ולנו, שם הקדים להמאורע אבות לבנים וכאן מקדימין גאולתנו שלנו לגאולת אבותינו, וצריך טעם. ואפשר לבאר, משום דבמעשה הנסים בודאי האבות קודמים, שהם קודמים לנו בזמן, אבל לענין הגאולה הנה ידוע, כי ענין גאולה הוא כמו השבת אבידה. יען כי על מניעת גאולה נאמר ואבדתם בגויים (פ׳ בחקותי, כ״ז ל״ח), ובענין אבידה מפורש בגמרא ב״מ (ל״ג א׳) כי החזרה על חיפוש אבידה שלו ואבידת אביו אבידתו קודמת, ויליף זה שם מקרא, ולכן מקדימין כאן גאולתנו לגאולת אבותינו. ומה שאנו אומרים בתפלה אלהינו ואלהי אבותינו, אעפ״י דבהכרת אלהות קדמו אבותינו — אפשר לומר משום שאם יקדימו אלהות שקבלו אבותינו זה יחליש באיזה ערך את הכרתנו שלנו באלהות, מפני שמטבע הבנים ללכת בדרכי אבותיהם, ועל דרך שאמרו מנהג אבותיהם בידיהם (תענית כ״ח ב׳), מעשה אבותיהם בידיהם (חולין י״ג ב׳), שותא דינוקא דאבוה או דאימיה (סוכה נ״ו ב׳), ואם כן אפשר להבין. כי רק מתוך רגילתנו שלנו במעשה אבותינו נחזיק בתומת אמונת אלהות, ולא מתוך הכרה עצמית, ולכן מקדימין לומר אלהינו להורות כי קבלתנו שלנו באלהות הוא מהכרה עצמית, ולא רק מתוך אחיזתנו בחיי אבותינו ובהכרתם שלהם.\textrm{\textbf{שמחים בבנין עירך וששים בעבודתך.}} חוקרי הלשון חקרו בתכונת השמות שמחה וששון, וקבעו, כי שמחה יונח על הרגש בתחלת הדבר הנכון לבוא. וששון — בסוף התוצאה מהדבר הנרצה, ולפי זה יתבאר כאן שמחים בבנין עירך, שהוא רק כהקדמה להעבודה בביהמ״ק, וששים בעבודתך — ביהמ״ק, שזה עיקר ותכלית קדושת ירושלים. ולהנחה זו של המדקדקים הנזכרים יסכים גם הלשון בפזמון אל אדון (לשחרית של שבת), כי צבא השמים (הם השמש והירח וכוכבים וכו׳) ״שמחים בצאתם  וששים בבואם״, כלומר, שמחים בצאתם למלא תפקידם, וששים בבואם — לאחר שהוציאו פעולתם. וע׳ מש״כ שם בכלל ענין שמחתם בזה. וכן יסכים להנחה זו הלשון בתפלה לרה״ש ויוהכ״פ ״שמחה לארצך וששון לעירך״, כי הקיבוץ להארץ היא התחלת הגאולה והכניסה לירושלים היא עיקר גאולת הארץ. אך על הנחה זו יש להעיר מישעיה (ל״ה י׳) ששון ושמחה ישיגו. ולפי המבואר, הלא היה להקדים שמחה לששון כסדר הענינים, כמו שבארנו. אבל על האמת מכוון הלשון שם, כי הן שם איירי בגמר קץ הגאולה כמש״כ שם, ופדויי ה׳ ישובון ובאו ציון ברנה ושמחת עולם על ראשם ששון ושמחה ישיגו — ושם באמת תכונת ששון קודמת, דתבונה זו באה בקץ תכלית הדבר, ואחר הששון המופלא — שמחה רגילית. ובירמיה (ז׳ ל״ד) והשבתי מערי יהודה ומחוצות ירושלים קול ששון וקול שמחה — יתבאר על דרך ״לא זו אף זו״, לא רק קול ששון גדול ומופלא אך גם שמחה רגילית תשבת, כמבואר, ועיין עוד במס׳ סוכה (מ״ח ב׳), ואין להאריך עוד.\textrm{\textbf{ונאכל שם מן הזבחים ומן הפסחים.}} ורשום במקומו במוסגר בזה הלשון (״במוצאי שבת מן הפסחים ומן הזבחים״). ופירשו המפרשים בטעם הלשון ההפוך, משום דבשם ״זבחים״ יובן קרבן חגיגה הבאה עם הפסח ואוכלים מקודם את החגיגה ואח״כ הפסח, משום דדין הפסח להאכל על השובע, כדי שיהי׳ נאכל רק לשם מצוה ולא לשם רעבון. ואמנם זהו רק כשחל פסח בימי חול, אבל כשחל פסח בשבת לא תבוא אז חגיגה יחד עם הפסח, מפני שמעשה החגיגה אינה דוחה את השבת כמו שדוחה הפסח, ובאה רק למחרת השבת, ולפי זה יוצא שמקודם אוכלים את הפסח (בשבת) ואח״כ החגיגה ביום ראשון, ולכן כשחל פסח במוצאי שבת אומרים ונאכל שם מן הפסחים ומן הזבחים, כסדר אכילתם, הפסח בשבת והחגיגה למחרתו. זה הוא טעם הנוסח במוצאי שבת מן הפסחים ומן הזבחים. אבל באור זה קשה, יען כי לפי זה הי׳ צריך להקדים פסחים לזבחים בחג הפסח שלפני אותה שנה שבה יחול פסח במוצאי שבת, כיון דהתפלה מוסבת על חג הפסח העתיד בשנה הבאה, כהוראת הלשון יגענו למועדים ולרגלים אחרים הבאים לקראתנו ונאכל שם, ואם כן צריך לחשב בפסח זה הבא בחול מתי יחול פסח בשנה הבאה, ואם יחול בשבת יקדימו אז פסחים לזבחים, אבל להקדים זה בפסח דהשתא אין כל טעם ויסוד.  ולבד זה קשה מאוד כלל ענין זה, כי הנה כל דיני הסדר בליל פסח נתבאר במשנה ובגמרא (פ״י מפסחים) ובפוסקים ראשונים, הרי״ף ורמב״ם ורא״ש וכל בו, ובסדור רב עמרם (המיוחד כולו להעמיד נוסחאות אמתיות), וכן בספרי אחרונים כל הפרטים ופרטי פרטים וכל המנהגים המקובלים עד שלא יחסר כל בהם, ומכל ענין זה להקדים במוצ״ש פסחים וזבחים אין אף רמז קל. והראשון בפוסקים אחרונים שחידש זה הוא הט״ז באו״ח סימן תע״ג ס״ק ט׳ בשם אחד האחרונים (מהרי״ז), וזולתו לא נמצא דבר ורמז, וכמה הוא מן הפלא, ועיין בתוס׳ פסחים (קט״ז ב׳), ולענין מוצאי שבת אין דבר שם. וכתבתי על זה בספרי מקור ברוך, כי בכלל כל חידוש זה (להקדים במוצ״ש פסחים לזבחים) בא לרגלי טעות קלה ע״י אחד הסופרים, ובהתגלות הטעות יתברר הדבר. והבאור הוא, כי מקור ברכת הודאה זו (אשר גאלנו וכו׳) היא במשנה פסחים, משנה ז׳ פרק י׳, ושם הגירסא מן הזבחים ומן הפסחים, ובאותה המשנה שבגמרא (דף קט״ז ע״ב) הגירסא מן הפסחים ומן הזבחים, ורשום שם במסרת הש״ס ״במשנה שבמשניות מן הזבחים ומן הפסחים״. וזה חזון מצוי שגירסות במשנה שבמשניות ובגמרא שונות הן. והיתה הגירסא העיקרית בההגדה כמו שהיא במשנה שבגמרא, מן הזבחים ומן הפסחים, ואחד המעתיקים לו את ההגדה מכתיבת יד סופר (טרם גלות מעשה הדפוס) רשם לו בצדו בראשי תיבות ״במ״ש מן הפסחים ומן הזבחים״, וכיון בזה לומר, כי במשנה שבגמרא (הנוסח) מן הפסחים ומן הזבחים, באשר כן הי׳ דרך הסופרים אז לכתוב בראשי תיבות משונים כדי לקצר מלאכתם. ומעתיק אחד שהעתיק מהגדה זו, חשב, כי הר״ת מן ״במ״ש״ הוא במוצאי שבת״, ועפ״י הסברא כי אז נאכל הפסח קודם החגיגה, כמו שבארנו, מפני שהחגיגה אינה דוחה שבת ובאה למחר השבת — עפ״י סברא זו העתיק לו מפורש ״במוצאי שבת מן הפסחים ומן הזבחים״, ויען כי סברא זו קרובה להתקבל, לכן בא נוסח בהעתקות הבאות ומהן לדפוס ומדפוס לדפוס עד היום הזה. ולהלכה בודאי העיקר בגירסת המשנה שבגמרא, מן הזבחים ומן הפסחים, וכן הגירסא ברי״ף ורמב״ם ורא״ש וכל הראשונים שזכרו ענין זה, וטעמם, משום דהפסח נאכל על השובע לאחר החגיגה, כמש״כ למעלה, ואם לפעמים יקרה שאכילת הפסח תוקדם לאכילת החגיגה, כגון בערב פסח שחל בשבת שוב לא פלוג בהנוסח,  יען כי עכ״פ בשבת יחול לזמן רחוק מאוד וגם כי האמירה בכלל אינה מעכבת, ולכן הנוסח הזה תמידי. והנה את כל זה כתבתי להעמיד הנוסח שבתלמוד ופוסקים בענין זה אבל אני לדעתי נראה טעם פשוט ומיוחד להקדים זבחים לפסחים בכל שנה, ואפילו בשנה שחל ערב פסח בשבת, ולא רק משום ״לא פלוג״ אך מיסוד הדבר, וכל חקירה בזה תעבור ותשתקע. וטעמי בזה, כי הן באותה הבקשה מבקשים שיגיענו ה׳ למועדים ולרגלים אחרים הבאים לקראתנו, ומבואר מפורש, שלא רק על חג הפסח מוסבת תפלה זו אך גם על רגלים אחרים הבאים לאחר הפסח בסדר מהלך השנה, והם שבועות וסכות, וידוע, כי באלה הרגלים מביאים שלמי חגיגה שנקראו זבחים, כמו החגיגה שבפסח. וכה מבקשים שיגיענו לשבועות ולסוכות ונאכל בהם מן הזבחים (שלמי חגיגה ביו״ט), ועוד יגיענו לפסח הבא ונאכל שם מן הפסחים, ואין צריך להזכיר גם החגיגה כי היא כלולה במצות הפסח. וכה הנוסח מן הזבחים ומן הפסחים קיים ונאמן בכל שנה ושנה. והנה ידוע בגמרא שטעם אכילת חגיגה קודם אכילת הפסח היא מפני שמצוה לאכול הפסח על השובע. וטעם הדבר לא נתבאר. ויש אומרים, דהוא כדי שיאכל הפסח לשם מצוה ולא לשם תאות אכילה. אבל זה קשה, דהא כידוע סעודת שבת היא מצוה (עיין שבת קי״ג א׳) ובכל זאת עיקר מצות סעודת שבת הוא שיאכל לתיאבון, ומהאי טעמא יש נוהגים להתענות בערב שבת כדי שיאכל בשבת לתיאבון, והתענית נקרא תענית צדיקים (ירושלמי פסחים פרק י׳ הלכה א׳).  ואפשר לומר בטעם אכילת הפסח על השובע עפ״י מה שאמרו ביומא (ע״ד סע״ב) דיהא אדם רגיל לאכול סעודתי׳ רק ביום, מפני שהאכילה בחושך לא תשביע הגוף. יען כי הנאת הגוף באה יחד עם מראה עינים על המאכל. והנה לא מצינו שיהא חיוב בפסח הנאת האורה. וידוע דמצות אכילת הפסח הוא רק בלילה, כמבואר בתורה ואכלו את הבשר בלילה הזה ולכן אם יאכל הפסח קודם החגיגה בלילה לא יהי׳ שבע, ולכן אוכלים מקודם החגיגה והחגיגה אפשר לאכול מבעוד יום, ולכשיאכל אחרי׳ את הפסח יאכלנו רק לשם מצוה, יען כי לבשר אינו רעב עוד. אבל אי אפשר לאכול מקודם הפסח ואח״כ החגיגה, יען כי לבד החשש מאכילת הפסח לתיאבון עוד תפוג בשר החגיגה את טעם בשר הפסח, זה אסור, כמבואר בפסחים (קי״ט ב׳). ומה שבליל שבת לא חיישינן לזה שאכילה בחושך לא תשביע יען כי בשבת חובה ומצוה להרבות בנרות כמבואר בשבת כ״ה ב׳. —}%endcomment
\hebeng{{\small שותים את הכוס בהסבת שמאל.} }{{\small We say the blessing below and drink the cup while reclining to the left} }
\hebeng{בָּרוּךְ אַתָּה ה׳, אֱלֹהֵינוּ מֶלֶךְ הָעוֹלָם בּוֹרֵא פְּרִי הַגָּפֶן. }{Blessed are You, Lord our God, who creates the fruit of the vine. }
\newchap{רחצה}
\hebeng{\textbf{רָחְצָה} }{Washing}
\hebeng{{\small נוטלים את הידים ומברכים: } }{{\small We wash the hands and make the blessing.} }
\hebeng{בָּרוּךְ אַתָּה ה׳, אֱלֹהֵינוּ מֶלֶךְ הָעוֹלָם, אֲשֶׁר קִדְּשָׁנוּ בְּמִצְוֹתָיו וְצִוָּנוּ עַל נְטִילַת יָדַיִם. }{Blessed are You, Lord our God, King of the Universe, who has sanctified us with His commandments and has commanded us on the washing of the hands.}
\newchap{מוציא מצה}
\hebeng{\textbf{מוֹצִיא מַצָּה} }{\textit{Motsi} Matsa}
\hebeng{{\small יקח המצות בסדר שהניחן, הפרוסה בין שתי השלמות, יאחז שלשתן בידו ויברך ״המוציא״ בכוונה עַל העליונה, ו״על אכילת מַצָּה״ בכוונה על הפרוסה. אחר כך יבצע כזית מן העליונה השלמה וכזית שני מן הפרוסה, ויטבלם במלח, ויאכל בהסבה שני הזיתים: } }{{\small He takes out the matsa in the order that he placed them, the broken one between the two whole ones; he holds the three of them in his hand and blesses "\textit{ha-motsi}" with the intention to take from the top one and "on eating matsa" with the intention of eating from the broken one. Afterwards, he breaks off a \textit{kazayit} from the top whole one and a second \textit{kazayit} from the broken one and he dips them into salt and eats both while reclining.} }
\hebeng{בָּרוּךְ אַתָּה ה׳, אֱלֹהֵינוּ מֶלֶךְ הָעוֹלָם הַמּוֹצִיא לֶחֶם מִן הָאָרֶץ. }{Blessed are You, Lord our God, King of the Universe, who brings forth bread from the ground.}
\hebeng{בָּרוּךְ אַתָּה ה׳, אֱלֹהֵינוּ מֶלֶךְ הָעוֹלָם, אֲשֶׁר קִדְּשָׁנוּ בְּמִצְוֹתָיו וְצִוָּנוּ עַל אֲכִילַת מַצָּה. }{Blessed are You, Lord our God, King of the Universe, who has sanctified us with His commandments and has commanded us on the eating of matsa.}
\newchap{מרור}
\hebeng{\textbf{מָרוֹר} }{\textit{Marror}}
\hebeng{{\small כל אחד מהמסבִים לוקח כזית מרור, ּמטבִלו בַחרוסת, ּמנער החרוסת, מברך ואוכל בלי הסבה. } }{{\small All present should take a \textit{kazayit} of \textit{marror}, dip into the \textit{haroset}, shake off the \textit{haroset}, make the blessing and eat without reclining.} }
\hebeng{בָּרוּךְ אַתָּה ה׳, אֱלֹהֵינוּ מֶלֶךְ הָעוֹלָם, אֲשֶׁר קִדְּשָנוּ בְּמִצְוֹתָיו וְצִוָּנוּ עַל אֲכִילַת מָרוֹר. }{Blessed are You, Lord our God, King of the Universe, who has sanctified us with His commandments and has commanded us on the eating of \textit{marror}.}
\newchap{כורך}
\hebeng{\textbf{כּוֹרֵךְ} }{Wrap}
\hebeng{{\small כל אחד מהמסבים לוקח כזית מן המצה השְלישית עם כזית מרור, כורכים יחד, אוכלים בהסבה ובלי ברכה. לפני אכלו אומר. } }{{\small All present should take a \textit{kazayit} from the third whole matsa with a \textit{kazayit} of \textit{marror}, wrap them together and eat them while reclining and without saying a blessing. Before he eats it, he should say:} }
\hebeng{זֵכֶר לְמִקְדָּשׁ כְּהִלֵּל. כֵּן עָשָׂה הִלֵּל בִּזְמַן שֶׁבֵּית הַמִּקְדָּשׁ הָיָה קַיָּם:}{In memory of the Temple according to Hillel. This is what Hillel would do when the Temple existed:}
\hebeng{הָיָה כּוֹרֵךְ מַצָּה וּמָרוֹר וְאוֹכֵל בְּיַחַד, לְקַיֵּם מַה שֶּׁנֶּאֱמַר: עַל מַצּוֹת וּמְרוׂרִים יֹאכְלֻהוּ.}{He would wrap the matsa and \textit{marror} and eat them together, in order to fulfill what is stated, (Numbers 9:11): "You should eat it upon matsot and \textit{marrorim}."}
\newchap{שולחן עורך}
\hebeng{\textbf{שֻׁלְחָן עוֹרֵךְ} }{The Set Table}
\hebeng{{\small אוכלים ושותים.} }{{\small We eat and drink.} }
\newchap{צפון}
\hebeng{\textbf{צָפוּן} }{The Concealed {[Matsa]}}
\hebeng{{\small אחר גמר הסעודה לוקח כל אחד מהמסבים כזית מהמצה שהייתה צפונה לאפיקומן ואוכל ממנה כזית בהסבה. וצריך לאוכלה קודם חצות הלילה. } }{{\small After the end of the meal, all those present take a \textit{kazayit} from the matsa, that was concealed for the afikoman, and eat a \textit{kazayit} from it while reclining.} }
\hebeng{{\small לפני אכילת האפיקומן יאמר:} זֵכֶר לְקָרְבָּן פֶּסַח הָנֶאֱכַל עַל הָשוֹׁבַע.}{{\small Before eating the afikoman, he should say:} "In memory of the Pesach sacrifice that was eaten upon being satiated."}
\newchap{ברך}
\newsection{ברכת המזון}
\hebeng{\textbf{בָּרֵךְ} }{Bless}
\hebeng{{\small מוזגים כוס שלישִי ומבָרכים בִרכַת המזון.} }{{\small We pour the third cup and recite the Grace over the Food} }
\hebeng{שִׁיר הַמַּעֲלוֹת, בְּשוּב ה׳ אֶת שִׁיבַת צִיּוֹן הָיִינוּ כְּחֹלְמִים. אָז יִמָּלֵא שְׂחוֹק פִּינוּ וּלְשׁוֹנֵנוּ רִנָּה. אָז יֹאמְרוּ בַגּוֹיִם: הִגְדִּיל ה׳ לַעֲשׂוֹת עִם אֵלֶּה. הִגְדִּיל ה׳ לַעֲשׂוֹת עִמָּנוּ, הָיִינוּ שְׂמֵחִים. שׁוּבָה ה׳ אֶת שְׁבִיתֵנוּ כַּאֲפִיקִים בַּנֶּגֶב. הַזֹּרְעִים בְּדִמְעָה, בְּרִנָּה יִקְצֹרוּ. הָלוֹךְ יֵלֵךְ וּבָכֹה נֹשֵׂא מֶשֶךְ הַזָּרַע, בֹּא יָבֹא בְרִנָּה נֹשֵׂא אֲלֻמֹּתָיו. }{A Song of Ascents; When the Lord will bring back the captivity of Zion, we will be like dreamers. Then our mouth will be full of mirth and our tongue joyful melody; then they will say among the nations; "The Lord has done greatly with these." The Lord has done great things with us; we are happy. Lord, return our captivity like streams in the desert. Those that sow with tears will reap with joyful song. He who surely goes and cries, he carries the measure of seed, he will surely come in joyful song and carry his sheaves.(Psalms 126)}
\hebeng{{\small שלשה שֶאכלו כאחד חיבים לזמן והמזַמן פותח: } }{{\small Three that ate together are obligated to introduce the blessing and the leader of the introduction opens as follows:} }
\hebeng{רַבּוֹתַי נְבָרֵךְ:}{My masters, let us bless:}
\hebeng{{\small המסבים עונים: } }{{\small All those present answer:} }
\hebeng{יְהִי שֵׁם ה׳ מְבֹרָךְ מֵעַתָּה וְעַד עוֹלָם.}{May the Name of the Lord be blessed from now and forever. (Psalms 113:2)}
\hebeng{{\small הַמְזַמֵן אומֵר: } }{{\small The leader says:} }
\hebeng{בִּרְשׁוּת מָרָנָן וְרַבָּנָן וְרַבּוֹתַי, נְבָרֵךְ [אֱלֹהֵינוּ] שֶׁאָכַלְנוּ מִשֶּׁלוֹ. }{With the permission of our gentlemen and our teachers and my masters, let us bless {[our God]} from whom we have eaten. }
\hebeng{{\small המסבים עונים: } }{{\small Those present answer:} }
\hebeng{בָּרוּךְ [אֱלֹהֵינוּ] שֶׁאָכַלְנוּ מִשֶּׁלוֹ וּבְטוּבוֹ חָיִינוּ}{Blessed is {[our God]} from whom we have eaten and from whose goodness we live.}
\hebeng{{\small המזמן חוזר ואומר: } }{{\small The leader repeats and says:} }
\hebeng{בָּרוּךְ [אֱלֹהֵינוּ] שֶׁאָכַלְנוּ מִשֶּׁלוֹ וּבְטוּבוֹ חָיִינוּ}{Blessed is {[our God]} from whom we have eaten and from whose goodness we live.}
\hebeng{{\small כלם אומרים: } }{{\small They all say:} }
\hebeng{בָּרוּךְ אַתָּה ה׳, אֱלֹהֵינוּ מֶלֶךְ הָעוֹלָם, הַזָּן אֶת הָעוֹלָם כֻּלּוֹ בְּטוּבוֹ בְּחֵן בְּחֶסֶד וּבְרַחֲמִים, הוּא נוֹתֵן לֶחֶם לְכָל בָּשָׂר כִּי לְעוֹלָם חַסְדוֹ. וּבְטוּבוֹ הַגָּדוֹל תָּמִיד לֹא חָסַר לָנוּ, וְאַל יֶחְסַר לָנוּ מָזוֹן לְעוֹלָם וָעֶד. בַּעֲבוּר שְׁמוֹ הַגָּדוֹל, כִּי הוּא אֵל זָן וּמְפַרְנֵס לַכֹּל וּמֵטִיב לַכֹּל, וּמֵכִין מָזוֹן לְכָל בְּרִיּוֹתָיו אֲשֶׁר בָּרָא. בָּרוּךְ אַתָּה ה׳, הַזָּן אֶת הַכֹּל. }{Blessed are You, Lord our God, King of the Universe, who nourishes the entire world in His goodness, in grace, in kindness and in mercy; He gives bread to all flesh since His kindness is forever. And in His great goodness, we always have not lacked, and may we not lack nourishment forever and always, because of His great name. Since He is a Power that feeds and provides for all and does good to all and prepares nourishment for all of his creatures that he created. Blessed are You, Lord, who sustains all. }
\hebeng{נוֹדֶה לְךָ ה׳ אֱלֹהֵינוּ עַל שֶׁהִנְחַלְתָּ לַאֲבוֹתֵינוּ אֶרֶץ חֶמְדָה טוֹבָה וּרְחָבָה, וְעַל שֶׁהוֹצֵאתָנוּ ה׳ אֱלֹהֵינוּ מֵאֶרֶץ מִצְרַיִם, וּפְדִיתָנוּ מִבֵּית עֲבָדִים, וְעַל בְּרִיתְךָ שֶׁחָתַמְתָּ בְּבְשָׂרֵנוּ, וְעַל תּוֹרָתְךָ שֶׁלִּמַּדְתָּנוּ, וְעַל חֻקֶּיךָ שֶׁהוֹדַעְתָּנוּ, וְעַל חַיִּים חֵן וָחֶסֶד שֶׁחוֹנַנְתָּנוּ, וְעַל אֲכִילַת מָזוֹן שָׁאַתָּה זָן וּמְפַרְנֵס אוֹתָנוּ תָּמִיד, בְּכָל יוֹם וּבְכָל עֵת וּבְכָל שָׁעָה: }{We thank you, Lord our God, that you have given as an inheritance to our ancestors a lovely, good and broad land, and that You took us out, Lord our God, from the land of Egypt and that You redeemed us from a house of slaves, and for Your covenant which You have sealed in our flesh, and for Your Torah that You have taught us, and for Your statutes which You have made known to us, and for life, grace and kindness that You have granted us and for the eating of nourishment that You feed and provide for us always, on all days, and at all times and in every hour.}
\hebeng{וְעַל הַכּל ה׳ אֱלֹהֵינוּ, אֲנַחְנוּ מוֹדִים לָךְ וּמְבָרְכִים אוֹתָךְ, יִתְבָּרַךְ שִׁמְךָ בְּפִי כָּל חַי תָּמִיד לְעוֹלָם וָעֶד. כַּכָּתוּב: וְאָכַלְתָּ וְשָׂבַעְתָּ וּבֵרַכְתָּ אֶת ה׳ אֱלֹהֵיךָ עַל הָאָרֶץ הַטּוֹבָה אֲשֶּׁר נָתַן לָךְ. בָּרוּךְ אַתָּה ה׳, עַל הָאָרֶץ וְעַל הַמָּזוֹן: }{And for everything, Lord our God, we thank You and bless You; may Your name be blessed by the mouth of all life, constantly forever and always, as it is written (Deuteronomy 8:10); "And you shall eat and you shall be satiated and you shall bless the Lord your God for the good land that He has given you." Blessed are You, Lord, for the land and for the nourishment.}
\hebeng{רַחֵם נָא ה׳ אֱלֹהֵינוּ עַל יִשְׂרָאַל עַמֶּךָ וְעַל יְרוּשָׁלַיִם עִירֶךָ וְעַל צִיּוֹן מִשְׁכַּן כְּבוֹדֶךָ וְעַל מַלְכוּת בֵּית דָּוִד מְשִׁיחֶךָ וְעַל הַבַּיִת הַגָּדוֹל וְהַקָּדוֹשׁ שֶׁנִּקְרָא שִׁמְךָ עָלָיו: אֱלֹהֵינוּ אָבִינוּ, רְעֵנוּ זוּנֵנוּ פַרְנְסֵנוּ וְכַלְכְּלֵנוּ וְהַרְוִיחֵנוּ, וְהַרְוַח לָנוּ ה׳ אֱלֹהֵינוּ מְהֵרָה מִכָּל צָרוֹתֵינוּ. וְנָא אַל תַּצְרִיכֵנוּ ה׳ אֱלֹהֵינוּ, לֹא לִידֵי מַתְּנַת בָּשָׂר וָדָם וְלֹא לִידֵי הַלְוָאתָם, כִּי אִם לְיָדְךָ הַמְּלֵאָה הַפְּתוּחָה הַקְּדוֹשָׁה וְהָרְחָבָה, שֶׁלֹא נֵבוֹשׁ וְלֹא נִכָּלֵם לְעוֹלָם וָעֶד. }{Please have mercy, Lord our God, upon Israel, Your people; and upon Jerusalem, Your city; and upon Zion, the dwelling place of Your Glory; and upon the monarchy of the House of David, Your appointed one; and upon the great and holy house that Your name is called upon. Our God, our Father, tend us, sustain us, provide for us, relieve us and give us quick relief, Lord our God, from all of our troubles. And please do not make us needy, Lord our God, not for the gifts of flesh and blood, and not for their loans, but rather from Your full, open, holy and broad hand, so that we not be embarrassed and we not be ashamed forever and always. }
\hebeng{{\small בשבת מוסיפין: } }{{\small On Shabbat, we add the following paragraph} }
\hebeng{רְצֵה וְהַחֲלִיצֵנוּ ה׳ אֱלֹהֵינוּ בְּמִצְוֹתֶיךָ וּבְמִצְוַת יוֹם הַשְּׁבִיעִי הַשַּׁבָּת הַגָּדול וְהַקָּדוֹשׂ הַזֶּה. כִּי יוֹם זֶה גָּדוֹל וְקָדוֹשׁ הוּא לְפָנֶיךָ לִשְׁבָּת בּוֹ וְלָנוּחַ בּוֹ בְּאַהֲבָה כְּמִצְוַת רְצוֹנֶךָ. וּבִרְצוֹנְךָ הָנִיחַ לָנוּ ה׳ אֱלֹהֵינוּ שֶׁלֹּא תְהֵא צָרָה וְיָגוֹן וַאֲנָחָה בְּיוֹם מְנוּחָתֵנוּ. וְהַרְאֵנוּ ה׳ אֱלֹהֵינוּ בְּנֶחָמַת צִיּוֹן עִירֶךָ וּבְבִנְיַן יְרוּשָׁלַיִם עִיר קָדְשֶׁךָ כִּי אַתָּה הוּא בַּעַל הַיְשׁוּעוֹת וּבַעַל הַנֶּחָמוֹת.}{May You be pleased to embolden us, Lord our God, in your commandments and in the command of the seventh day, of this great and holy Shabbat, since this day is great and holy before You, to cease work upon it and to rest upon it, with love, according to the commandment of Your will. And with Your will, allow us, Lord our God, that we should not have trouble, and grief and sighing on the day of our rest. And may You show us, Lord our God, the consolation of Zion, Your city; and the building of Jerusalem, Your holy city; since You are the Master of salvations and the Master of consolations.}
\hebeng{אֱלֹהֵינוּ וֵאלֹהֵי אֲבוֹתֵינוּ, יַעֲלֶה וְיָבֹא וְיַגִּיעַ וְיֵרָאֶה וְיֵרָצֶה וְיִשָּׁמַע וְיִפָּקֵד וְיִזָּכֵר זִכְרוֹנֵנוּ וּפִקְדּוֹנֵנוּ, וְזִכְרוֹן אֲבוֹתֵינוּ, וְזִכְרוֹן מָשִׁיחַ בֶּן דָּוִד עַבְדֶּךָ, וְזִכְרוֹן יְרוּשָׁלַיִם עִיר קָדְשֶׁךָ, וְזִכְרוֹן כָּל עַמְּךָ בֵּית יִשְׂרָאַל לְפָנֶיךָ, לִפְלֵיטָה לְטוֹבָה לְחֵן וּלְחֶסֶד וּלְרַחֲמִים, לְחַיִּים וּלְשָׁלוֹם בְּיוֹם חַג הַמַּצּוֹת הַזֶּה זָכְרֵנוּ ה׳ אֱלֹהֵינוּ בּוֹ לְטוֹבָה וּפָקְדֵנוּ בוֹ לִבְרָכָה וְהושִׁיעֵנוּ בוֹ לְחַיִּים. וּבִדְבַר יְשׁוּעָה וְרַחֲמִים חוּס וְחָנֵּנוּ וְרַחֵם עָלֵינוּ וְהוֹשִׁיעֵנוּ, כִּי אֵלֶיךָ עֵינֵינוּ, כִּי אֵל מֶלֶךְ חַנּוּן וְרַחוּם אָתָּה. וּבְנֵה יְרוּשָׁלַיִם עִיר הַקֹּדֶשׁ בִּמְהֵרָה בְיָמֵינוּ. בָּרוּךְ אַתָּה ה׳, בּוֹנֶה בְרַחֲמָיו יְרוּשָׁלַיִם. אָמֵן. }{God and God of our ancestors, may there ascend and come and reach and be seen and be acceptable and be heard and be recalled and be remembered - our remembrance and our recollection; and the remembrance of our ancestors; and the remembrance of the messiah, the son of David, Your servant; and the remembrance of Jerusalem, Your holy city; and the remembrance of all Your people, the house of Israel - in front of You, for survival, for good, for grace, and for kindness, and for mercy, for life and for peace on this day of the Festival of Matsot. Remember us, Lord our God, on it for good and recall us on it for survival and save us on it for life, and by the word of salvation and mercy, pity and grace us and have mercy on us and save us, since our eyes are upon You, since You are a graceful and merciful Power. And may You build Jerusalem, the holy city, quickly and in our days. Blessed are You, Lord, who builds Jerusalem in His mercy. Amen.}
\hebeng{בָּרוּךְ אַתָּה ה׳, אֱלֹהֵינוּ מֶלֶךְ הָעוֹלָם, הָאֵל אָבִינוּ מַלְכֵּנוּ אַדִירֵנוּ בּוֹרְאֵנוּ גּוֹאֲלֵנוּ יוֹצְרֵנוּ קְדוֹשֵׁנוּ קְדוֹשׁ יַעֲקֹב רוֹעֵנוּ רוֹעֵה יִשְׂרָאַל הַמֶּלֶךְ הַטּוֹב וְהַמֵּטִיב לַכּל שֶׁבְּכָל יוֹם וָיוֹם הוּא הֵטִיב, הוּא מֵטִיב, הוּא יֵיטִיב לָנוּ. הוּא גְמָלָנוּ הוּא גוֹמְלֵנוּ הוּא יִגְמְלֵנוּ לָעַד, לְחֵן וּלְחֶסֶד וּלְרַחֲמִים וּלְרֶוַח הַצָּלָה וְהַצְלָחָה, בְּרָכָה וִישׁוּעָה נֶחָמָה פַּרְנָסָה וְכַלְכָּלָה וְרַחֲמִים וְחַיִּים וְשָׁלוֹם וְכָל טוֹב, וּמִכָּל טוּב לְעוֹלָם עַל יְחַסְּרֵנוּ. }{Blessed are You, Lord our God, King of the Universe, the Power, our Father, our King, our Mighty One, our Creator, our Redeemer, our Shaper, our Holy One, the Holy One of Ya'akov, our Shepherd, the Shepherd of Israel, the good King, who does good to all, since on every single day He has done good, He does good, He will do good, to us; He has granted us, He grants us, He will grant us forever - in grace and in kindness, and in mercy, and in relief - rescue and success, blessing and salvation, consolation, provision and relief and mercy and life and peace and all good; and may we not lack any good ever. }
\hebeng{הָרַחֲמָן הוּא יִמְלוֹךְ עָלֵינוּ לְעוֹלָם וָעֶד. הָרַחֲמָן הוּא יִתְבָּרַךְ בַּשָּׁמַיִם וּבָאָרֶץ. הָרַחֲמָן הוּא יִשְׁתַּבַּח לְדוֹר דּוֹרִים, וְיִתְפָּאַר בָּנוּ לָעַד וּלְנֵצַח נְצָחִים, וְיִתְהַדַּר בָּנוּ לָעַד וּלְעוֹלְמֵי עוֹלָמִים. הָרַחֲמָן הוּא יְפַרְנְסֵנוּ בְּכָבוֹד. הָרַחֲמָן הוּא יִשְׁבּוֹר עֻלֵּנוּ מֵעַל צַּוָּארֵנוּ, וְהוּא יוֹלִיכֵנוּ קוֹמְמִיּוּת לְאַרְצֵנוּ. הָרַחֲמָן הוּא יִשְׁלַח לָנוּ בְּרָכָה מְרֻבָּה בַּבַּיִת הַזֶּה, וְעַל שֻׁלְחָן זֶה שֶׁאָכַלְנוּ עָלָיו. הָרַחֲמָן הוּא יִשְׁלַח לָנוּ אֶת אֵלִיָּהוּ הַנָּבִיא זָכוּר לַטּוֹב, וִיבַשֶּׂר לָנוּ בְּשׂוֹרוֹת טוֹבוֹת יְשׁוּעוֹת וְנֶחָמוֹת. הָרַחֲמָן הוּא יְבָרֵךְ אֶת בַּעֲלִי / אִשְתִּי. הָרַחֲמָן הוּא יְבָרֵךְ אֶת [אָבִי מוֹרִי] בַּעַל הַבַּיִת הַזֶּה. וְאֶת [אִמִּי מוֹרָתִי] בַּעֲלַת הַבַּיִת הַזֶּה, אוֹתָם וְאֶת בֵּיתָם וְאֶת זַרְעָם וְאֶת כָּל אֲשֶׁר לָהֶם. אוֹתָנוּ וְאֶת כָּל אֲשֶׁר לָנוּ, כְּמוֹ שֶׁנִּתְבָּרְכוּ אֲבוֹתֵינוּ אַבְרָהָם יִצְחָק וְיַעֲקֹב בַּכֹּל מִכֹּל כֹּל, כֵּן יְבָרֵךְ אוֹתָנוּ כֻּלָּנוּ יַחַד בִּבְרָכָה שְׁלֵמָה, וְנֹאמַר, אָמֵן. בַּמָּרוֹם יְלַמְּדוּ עֲלֵיהֶם וְעָלֵינוּ זְכוּת שֶׁתְּהֵא לְמִשְׁמֶרֶת שָׁלוֹם. וְנִשָּׂא בְרָכָה מֵאֵת ה׳, וּצְדָקָה מֵאלֹהֵי יִשְׁעֵנוּ, וְנִמְצָא חֵן וְשֵׂכֶל טוֹב בְּעֵינֵי אֱלֹהִים וְאָדָם. בשבת: הָרַחֲמָן הוּא יַנְחִילֵנוּ יוֹם שֶׁכֻּלּוֹ שַׁבָּת וּמְנוּחָה לְחַיֵּי הָעוֹלָמִים. הָרַחֲמָן הוּא יַנְחִילֵנוּ יוֹם שֶׁכֻּלוֹ טוֹב.[יוֹם שֶׁכֻּלוֹ אָרוּךְ. יוֹם שֶׁצַּדִּיקִים יוֹשְׁבִים וְעַטְרוֹתֵיהֶם בְּרָאשֵׁיהֶם וְנֶהֱנִים מִזִּיו הַשְּׁכִינָה וִיהִי חֶלְקֵינוּ עִמָּהֶם]. הָרַחֲמָן הוּא יְזַכֵּנוּ לִימוֹת הַמָּשִׁיחַ וּלְחַיֵּי הָעוֹלָם הַבָּא. מִגְדּוֹל יְשׁוּעוֹת מַלְכּוֹ וְעֹשֶׂה חֶסֶד לִמְשִׁיחוֹ לְדָוִד וּלְזַרְעוֹ עַד עוֹלָם. עשֶׂה שָׁלוֹם בִּמְרוֹמָיו, הוּא יַעֲשֶׂה שָׁלוֹם עָלֵינוּ וְעַל כָּל יִשְׂרָאַל וְאִמְרוּ, אָמֵן. יִרְאוּ אֶת ה׳ קְדֹשָׁיו, כִּי אֵין מַחְסוֹר לִירֵאָיו. כְּפִירִים רָשׁוּ וְרָעֵבוּ, וְדֹרְשֵׁי ה׳ לֹא יַחְסְרוּ כָל טוֹב. הוֹדוּ לַיי כִּי טוֹב כִּי לְעוֹלָם חַסְדּוֹ. פּוֹתֵחַ אֶת יָדֶךָ, וּמַשְׂבִּיעַ לְכָל חַי רָצוֹן. בָּרוּךְ הַגֶּבֶר אֲשֶׁר יִבְטַח בַּיי, וְהָיָה ה׳ מִבְטַחוֹ. נַעַר הָיִיתִי גַם זָקַנְתִּי, וְלֹא רָאִיתִי צַדִּיק נֶעֱזָב, וְזַרְעוֹ מְבַקֶּשׁ לָחֶם. יי עֹז לְעַמּוֹ יִתֵּן, ה׳ יְבָרֵךְ אֶת עַמּוֹ בַשָּׁלוֹם.}{May the Merciful One reign over us forever and always. May the Merciful One be blessed in the heavens and in the earth. May the Merciful One be praised for all generations, and exalted among us forever and ever, and glorified among us always and infinitely for all infinities. May the Merciful One sustain us honorably. May the Merciful One break our yoke from upon our necks and bring us upright to our land. May the Merciful One send us multiple blessing, to this home and upon this table upon which we have eaten. May the Merciful One send us Eliyahu the prophet - may he be remembered for good - and he shall announce to us tidings of good, of salvation and of consolation. May the Merciful One bless my husband/my wife. May the Merciful One bless {[my father, my teacher,]} the master of this home and {[my mother, my teacher,]} the mistress of this home, they and their home and their offspring and everything that is theirs. Us and all that is ours; as were blessed Avraham, Yitschak and Ya'akov, in everything, from everything, with everything, so too should He bless us, all of us together, with a complete blessing and we shall say, Amen. From above, may they advocate upon them and upon us merit, that should protect us in peace; and may we carry a blessing from the Lord and charity from the God of our salvation; and find grace and good understanding in the eyes of God and man. {[On Shabbat, we say: May the Merciful One give us to inherit the day that will be completely Shabbat and rest in everlasting life.]} May the Merciful One give us to inherit the day that will be all good. {[The day that is all long, the day that the righteous will sit and their crowns will be on their heads and they will enjoy the radiance of the Divine presence and my our share be with them.]} May the Merciful One give us merit for the times of the messiah and for life in the world to come. A tower of salvations is our King; may He do kindness with his messiah, with David and his offspring, forever (II Samuel 22:51). The One who makes peace above, may He make peace upon us and upon all of Israel; and say, Amen. Fear the Lord, His holy ones, since there is no lacking for those that fear Him. Young lions may go without and hunger, but those that seek the Lord will not lack any good thing (Psalms 34:10-11). Thank the Lord, since He is good, since His kindness is forever (Psalms 118:1). You open Your hand and satisfy the will of all living things (Psalms 146:16). Blessed is the man that trusts in the Lord and the Lord is his security (Jeremiah 17:7). I was a youth and I have also aged and I have not seen a righteous man forsaken and his offspring seeking bread (Psalms 37:25). The Lord will give courage to His people. The Lord will bless His people with peace (Psalms 29:11). }
\newsection{כוס שלישית}
\hebeng{בָּרוּךְ אַתָּה ה׳, אֱלהֵינוּ מֶלֶךְ הָעוֹלָם בּוֹרֵא פְּרִי הַגָּפֶן. }{Blessed are You, Lord our God, King of the universe, who creates the fruit of the vine.}
\hebeng{{\small ושותים בהסיבה ואינו מברך ברכה אחרונה.} }{{\small We drink while reclining and do not say a blessing afterwards.} }
\newsection{שפוך חמתך}
\hebeng{{\small מוזגים כוס של אליהו ופותחים את הדלת:} }{{\small We pour the cup of Eliyahu and open the door.} }
\hebeng{שְׁפֹךְ חֲמָתְךָ אֶל־הַגּוֹיִם אֲשֶׁר לֹא יְדָעוּךָ וְעַל־מַמְלָכוֹת אֲשֶׁר בְּשִׁמְךָ לֹא קָרָאוּ. כִּי אָכַל אֶת־יַעֲקֹב וְאֶת־נָוֵהוּ הֵשַׁמּוּ. שְׁפָךְ־עֲלֵיהֶם זַעֲמֶךָ וַחֲרוֹן אַפְּךָ יַשִּׂיגֵם. תִּרְדֹף בְּאַף וְתַשְׁמִידֵם מִתַּחַת שְׁמֵי ה׳.}{Pour your wrath upon the nations that did not know You and upon the kingdoms that did not call upon Your Name! Since they have consumed Ya'akov and laid waste his habitation (Psalms 79:6-7). Pour out Your fury upon them and the fierceness of Your anger shall reach them (Psalms 69:25)! You shall pursue them with anger and eradicate them from under the skies of the Lord (Lamentations 3:66).}%
\commenta{\textrm{\textbf{שפוך חמתך על הגוים אשר לא ידעוך..}} הלשון אשר לא ידעוך אינו ברור למדי, כי הן סתם גוים לא ידעו את ה'. שהרי הם היו עובדים את אלהיהם, והלשון ״אשר״ משמע רק אלה שלא ידעו, וטוב הי׳ לומר ״כי״ לא ידעוך, והי׳ זה בנתינת טעם, יען כי לא ידעוך, וכמו בפרשה מקץ (מ״ה כ״ו) ויפג לבו כי לא האמין להם, שהמובן הוא, יען כי לא האמין להם, ובתהלים (מ״ה) לא אירא רע כי אתה עמדי והמובן הוא יען כי אתה עמדי. אך אפשר שהלשון ״אשר״ מקוצר, ובשלמותו יהי׳ ״על אשר״, או ״יעז אשר״, והוא בנתינת טעם על הדבר, וכמו בפרשה ויצא (ל׳ י״ח) נתן אלהים שכרי אשר נתתי שפחתי... תחת ״על אשר״ או ״יען אשר״ נתתי, ובמלכים א׳ (ח׳ ל״ג) בהנגף עמך ישראל לפני אויב אשר יחטאו לך, דהמובן הוא ״על אשר״ יחטאו לך, ועוד שם (ט״ו י״ג) ויסרה מגבירה אשר עשתה מפלצה — תחת על אשר עשתה, ועוד כהנה. — וכעין זה פרשתי במ״ר סוף פרשה בשלח, כי בשעה שבא עמלק ללחום בישראל, אמר משה ליהושע ״צא הלחם בעמלק״ (לשון הפסוק שם), ודייקו במדרש, למה אמר זה דוקא ליהושע, והלא היו אז עמו אהרן וחור (שם י״ז י׳), ופירשו, שכיון לומר ליהושע, זקנך יוסף (כי יהושע בא מאפרים, כמבואר בפ׳ שלח) אמר את האלהים אני ירא (פ׳ מקץ, מ״ב י״ח), ובזה (בעמלק) נאמר ולא ירא אלוהים (ס״פ תצא), ובא בן בנו של זה שאמר את האלהים אני ירא ויפרע ממי שנאמר עליו ולא ירא אלהים, ע״כ. וכל אגדה זו צריכה באור, כי האמנם יראת יוסף את ה׳ אנו צריכים ללמוד מדבריו שלו, והלא הוא ידוע בצדקתו ויראתו את ה', ואין אנו צריכים לראי׳. ואף גם זאת, לפלא, כי האם אם בכל אחיו לא נאמר שהיו יראי אלהים אנו מסתפקין ביראתם את ה׳? ומה ריבותא דיוסף בזה. ועוד בה שלשיה לפלא, בקריאתו של משה לחטא על עמלק שנאמר בו ״ולא ירא אלהים״, כי הן התואר ״ירא אלהים״ הוא מצוין ומכובד מאוד גם בישראל, ולא כל אחד זוכה לזה, ומי לנו גדול וצדיק מאברהם אבינו, והנה אך לאחר העקידה זכה לתואר זה, וכמאמר המלאך לאחר העקידה עתה, רק עתה, לאחר נסיון העקידה, ידעתי כי ירא אלהים אתה (ס״פ וירא, כ״ב י״ב), וכן מתאר הכתוב בשם זה כמו שם מצוין ונפלא את עובדיהו (מ״א י״ח) ואת איוב (איוב א׳:ח׳) — והנה כאן יחשב לחטא לעמלק שאינו ירא אלהים, ואיך ידרש ממנו תואר אשר אך מתי מספר מישראל זכו לו. ואם מצינו במאמר הקודם שפוך חמתך על הגוים אשר לא ידעוך — הנה המשך הלשון מבאר סבת חמה זו כי אכל את יעקב ואת נוהו השמו, ובשמך לא קראו, והיינו שלא הודו ששעבודם בישראל הי׳ בגזירה מאת ה׳ כמו פרעת במצרים ואשור שבט אפו אך תלו הכל ברצונם, כפי שיתבאר במאמר הבא. ולכן נראה דהחטא של עמלק שלא הי׳ ירא אלהים אמנם בולט ומתקיים אך ורק מיוסף, יען כי בעת שאמר יוסף לאחיו את האלהים אני ירא הי׳ מתנכר לאחיו והכיר עצמו למצרי, ובכל זאת תיאר עצמו לירא אלהים, כמו שאמר את האלהים אני ירא, הרי מבואר מזה, דגם איש מאומות העולם אפשר שיהי׳ ירא אלהים, וביותר, שאם לא כן, הלא הי׳ באפשר שיפול חשד בלב אחיו כי עברי הוא — ולכן ראוי לעמלק להענש על כי הוא מחוסר יראת אלהים. וכה מדויק הלשון שאמר משה ליהושע, זקנך יוסף אמר את האלהים אני ירא, וממנו מוכח, שגם באומות העולם יש יראי אלהים — לכן צא והלחם בעמלק על כי הוא מחוסר מעלה זו, ומכוון גם כן מה שפנה משה בדבריו אלה רק ליהושע ולא לאהרן וחור, מפני תולדתו מיוסף וזכותו בזה תגן עליו. —\textrm{\textbf{ועל ממלכות אשר בשמך לא קראו כי אכל את יעקב ואת נוהו השמו.}} דרוש באור הלשון אשר בשמך לא קראו, מה המכוון בזה, ואפשר לפרש, משום דיש ממלכות שעליהם נגזר שישתעבדו בישראל או ילחמו בהם, כמו פרעה בשעבוד מצרים, שעוד בימי אברהם נגזר על שעבוד מצרים (פ׳ לך, ט״ו י״ג), וכן על מלך אשור נאמר אשור שבט אפי (ישעיה י׳ ה׳), והכוונה, כי כשיחטאו ישראל ישלח עליהם את אשור, וכן בסנחריב, ועוד מבואר מזה בסנהדרין (צ״ז ב׳). אך הן, הממלכות האלה לא יקראו בשם ה׳, לומר, כי רק שלוחי המקום הם, אך תולים הכל ברצונם ובכחם, ומתוך זה אכלו את יעקב ושממו את נוהו (רמז לביהמ״ק. כמש״כ בש״ב (ט״ו כ״ה והראני אותו ואת נוהו, ומכוין להבית שהי׳ שם הארון, ותרגומו ואפלח קדמוהי בבית מקדשי'), ועל כן שפך עליהם זעמך.}%endcomment%
\commentb{\textrm{\textbf{תשובה לשער ק'}}\textrm{\textbf{שפוך חמתך אל הגוים אשר לא ידעוך ועל הממלכות אשר בשמך לא קראו וכו'.}} המזמור הזה עם שאר המזמורים אשר באו אחריו עד "אשרי תמימי דרך" הוא החלק השני מההלל. והמפרשים וחז"ל גם הם במדרש פרשו קצתם על דוד וקצתם על ישי אביו ועל אחיו ועל לימוד החכמות, ומהם דרשו על קיבוץ הגליות. ולדעתי כל החלק הזה מההלל נאמר כנגד הגאולה העתידה, ואל הכוונה הזאת תקנו לאומרו בליל פסח, על הסעודה לפי שהגאולה העתידה היא מתקשרת עם יציאת מצרים כמו שכתבתי למעלה, ולכן באו הדברים בחלק הזה בלשון עתיד כמו שבאו הכתובים אשר בחלק הראשון בלשון עבר. והנני מפרש אותם אחד אל אחד שכולם מסכימים ומכוונים לעניין הגאולה האחרונה כאשר עם לבבי. ולפי שהפסוק "לֹא לָנוּ ה' לֹא לָנוּ", אינו נאות לראשית והתחלת הדברים כי הוא חוזר על מאמר קודם אליו, לכן תקנו לשים בתחילתו שני פסוקים ממזמור אחר , והם "שְׁפֹךְ חֲמָתְךָ" וגו' "כִּי אָכַל אֶת יַעֲקֹב" וגו' שהם מספר תהלים במזמור ע"ט "אֱ‍לֹהִים בָּאוּ גוֹיִם בְּנַחֲלָתֶךָ" וגו' והם גם כן בירמיהו בשנוי מעט. ויאמר: ה' אלהינו אחרי שהגדלת לעשות עם יוצאי מצרים ונקמת נקמתם מאויביהם והושעת וגאלת את עמך, והיה זה לשתי סיבות אם להודיע כוחך הגדול במצרים על פרעה ועל עבדיו שאמר " מִי ה' אֲשֶׁר אֶשְׁמַע בְּקֹלוֹ... ל לֹא יָדַעְתִּי אֶת ה'" (שמות ה', ב'), וגם כן להענישם על מה שהרעו לישראל. גם עתה שפוך חמתך אל הגוים, והם אומות העולם כי הם לא ידעוך ובשמך לא קראו, כמו פרעה שלא היה יודע אותך ולא שמך, ואם שפכת חמתך על המצריים בעבור מה שהרעו לישראל גם האומות האלה אכלו את ישראל, ועוד הוסיפו צרה יותר מהמצריים, כי הם החריבו בית המקדש פעמים מה שלא עשו המצריים, והוא אמרו "וְאֶת נָוֵהוּ הֵשַׁמּוּ" (תהלים ע"ט, ז') ואף על פי שאין אנחנו ראויים שתעשה עמנו הגאולה והתשועה הזאת, עשה למען שמך הגדול המחולל בקרב הגוים, והוא אמרו "לא לנו ה' לא לנו כי לשמך תן כבוד", רצה לומר לא בעבורנו תעשה החסד הזה כי אם למען שמך הגדול. ואמר "על חסדך ועל אמיתך", כנגד השתי סיבות שהזכיר, רצה לומר על חסדך תעשה לעמך ישראל ועל אמתך כדי שיכירו וידעו כל יושבי תבל כי יש אלהים בישראל, ואמר על חסדך כנגד "כי אכל את יעקב" ועל אמיתך כנגד "אשר לא ידעוך".והתבאר מזה ששני הפסוקים שהונחו בתחילת המזמור באמת מתייחסים לעניינו והותר בזה הספק אשר בשער מאה.ואתה תראה שעל שתי הקוטבים האלה, "אשר לא ידעוך" ו"כי אכל את יעקב" סובב המזמור כולו. כי הנה ביאר בראשונה איך האומות לא ידעו את אלהי ישראל באמרו "לָמָּה יֹאמְרוּ הַגּוֹיִם אַיֵּה נָא אֱלֹהֵיהֶם" (תהלים קט"ו, ב') רצה לומר למה תרצה שיאמרו הגויים שאומרים שעשה עמהם ניסים להפליא בימים הראשונים, שמא אַפָס כוחו ואין יכולת בידו להושיעם כבראשונה? למה יאמרו כדברים האלה שהם בהיפך ממה שאנו מאמינים שאלוהינו בשמים כל אשר חפץ עשה, כי פוקד על צבא מרום במרום וכל שכן שיוכל לפקוד על מלכי האדמה באדמה, ואין כן עניין הגויים והממלכות ההם כי עצביהם כסף וזהב והם הפסילים והצלמים שעובדים אליהם, וקראם "עצבים" לפי שהם עצב ויגון להקוראים אליהם ולא יענום, וגם יקראו עצבים מלשון עמל כמו "וַעֲצָבֶיךָ בְּבֵית נָכְרִי" (משלי ה', י'), והכוונה שהם עמלים בעשייתם ותכלית הצלחתם הוא שיהיו מכסף וזהב, שהם דמיון הלבנה והשמש, לפי שידעו כי המאורות האלה פועלים יותר בזה העולם השפל, כי הירח מניע יסוד המים והשמש מביא זמני הקור והחום וקיץ וחורף, לכן יעשו להם צלמין מהמתכות המיוחדות ומיוחסות אליהם ומתכוונים לעשותם בשעות ידועות להוריד אליהם כוח המאורות ההם.והנה באומרו "פֶּה לָהֶם וְלֹא יְדַבֵּרוּ" (תהלים קט"ו, ה') וגו' קשה לי מאוד מה צורך לספר בפסילים שיש להם פה ולא ידברו וכן שאר האיברים, כי זה דבר מבואר וידוע וגם עובדי הפסילים בעצמם מודים בזה, וכולם אומרים ביאור שאינם עובדים לאותם הפסילים שאין להם החושים ורוח אין בקרבן כי אם לכוחות העליונים והם משפיעים באמצעות אותם הכלים על העובדים אותם, ומה המענה שטען בזה המשורר? וקשה לי גם כן אמרו "פה להם ולא ידברו" ואמר אחר כל "ולא יהגו בגרונם" (שם שם, ז') כי שני המאמרים האלה עניינם אחד והוא העדר הדיבור.ולכן הנראה לי בזה הוא אחד משני פירושים אם שידבר המשורר מהפסילים והצלמים כמו שכתבתי, ואומר שכל עמל האומות הוא לעשותם ממתכות יותר נכבד וזה "עצביהם כסף וזהב", והפסילים הם דברים מלאכותיים מעשה ידי אדם, ויתפלא עליהם כי אף שיהיו אותם הפסילים כלים להוריד הרוחניות על בני אדם כענין הטַלִסְמַאוֹת16מתוך הערות הרב קפאח למורה נבוכים בחלק 1 פרק ס"ג – ""טלסם" כתב או שרטוט, או צורה, או עקרין, שמדמים ההוזים והשוטים שיש להן השפעה למעלה מן הטבע. ובר"ש "הטלסמאות", פירוש צורות המדברות, והבל המה מעשה תעתועים. למה יעשו בהם דמיון בלי החושים כולם אם לא יפעלו כלל הכלים האלה להשפיע הרוחניות המיוחסות להם. והיה ראוי שיעשו אותם באיזה צורה שיהיה ולא שישימו להם עיניים כיון שאין רואים, ולא אוזניים כיוון שאין שומעים, ולא אף כיוון שלא יריחו, ולא ידיים כיוון שאין בהם מישוש, ולא רגליים כיוון שלא יתנועעו, ולא גרון כיוון שלא ידברו. כי הנה הכלים ההם לא עשה אותם הטבע כי אם לפעול בהם פעולת החמישה חושים החיצוניים. ואם הפסילים האלה לא ישתמשו בהם נחשבו לפועלים במלים, וכיוון שהעובדים עצמם מודים בזה יש לתמוה מאוד עליהם איך לא ישחקו מהם ולא יאמרו לחרשים ולאמנים: אל תעשו צורות הכלים האלה כיוון שהם למותר ודברים במילים, וזה שאמר "פה להם ולא ידברו" כראוי, "עיניים להם ולא יראו אזנים להם ולא ישמעו" וגו', ו"לא יהגו בגרונם" נאמר על הפסילים והוא סוף הדברים, "כְּמוֹהֶם יִהְיוּ עֹשֵׂיהֶם כֹּל אֲשֶׁר בֹּטֵחַ בָּהֶם" (תהלים קט"ו, ח') שיכלו באפם תקוה ולא יהיה להם השארת הנפש, אבל ישראל בוטח בה' וכן בית אהרן כפי משפחותם, ויראי ה' השרידים אשר ה' קורא, יהיה ה' בעזרם ומגנם, כי הוא בהפך הפסילים שהם כלים בלי כח ופעולה, והשם יתברך הוא בעל הכוחות והפעולות ואינו כלי גשמי. זהו הדרך הראשון בפירוש הפסוקים.והדרך השני הוא שלא בא המשורר לספר פה מענייני הפסילים כי אם מרשעת הגויים שלא ידעו את ה' ולא קראו בשמו, והמה הומים אחרי הבלי העולם וישימו תכליתם בקניין העושר והכבוד, ולא חששו להשלים נפשם באמונה האמיתית. והוא אמרו "עצבי הגויים כסף וזהב", רצה לומר מחשבות הגויים וכל עמלם היא לקנות כסף וזהב בחשבם כי בהשתדלותם קונים אותו, וזהו "מעשה ידי אדם" שיאמרו כחי ועוצם ידי עשה לי את החיל הזה. "פה להם ולא ידברו" רצה לומר לאותם הגויים והממלכות ברא להם הטבע בחכמת בוראו הפה לספר תהילות ה' ולא ידברו, עיניים להם לראות נפלאותיו ולא יראון, אזנים להם לשמוע דברי הנביאים והחסידים ולא ישמעו, אף להם ולא יריחון ריח קטורת ולא ישתדלו, או אמר "יריחון" על הרגש הדברים כפי אמיתתם, כמו שאמר "וַהֲרִיחוֹ בְּיִרְאַת ה'" (ישעיהו י"א, ג'), ידיהם ולא ימישון שלא יעשו המצוות התלויות בידם, וכן רגליהם ולא יהלכו למעשה המצוות ועבודת ה', ולא יהגו בגרונם שלא יתפללו אליו יתברך ולא קראו בשמו, ולכן אמר בדרך תפלה אחרי שאלה הגויים והממלכות ישימו זהב בסלם וכסף במהונם, יהי רצון מלפני ה', שכמו הכסף והזהב שהם דוממים מבלי הרגש ככה יהיו עושיהם  כל אשר בוטח בהם, רצה לומר אותם השמים כל השתדלותם עליהם, אבל בית ישראל לא ישימו בטחונם בזהב ובכסף כי אם בה' צור עולמים שלו הכסף ולו הזהב. והוא אמרו "ישראל בטח בה'" כי הוא עזרם ומגנם,  והזכיר פעמים רבות בזה ההלל שלש מדרגות והם ישראל בית אהרן ויראי ה', לפי שבית ישראל הוא שם להמון העם כי להיות ישראל אביהם והם זרע אברהם בחר ה' בהם והמה בוטחים בו, וכן הזכיר בית אהרן שהם הלויים והכהנים לפי שהם גם כן מפאת שבטם ומשפחתם יוחדו להשם יתברך ולעבודתו בהבדל מיוחד מן המיוחד, ואחר כך הזכיר מדרגת האנשים שאינם נבחרים מפאת משפחתם כי אם מפאת עצמם ושלמותם יהיו מאיזה משפחה שיהיו, כי הם יראי ה'. ואפשר גם כן שאמר ישראל על האומה בכללה, ובית אהרן על שבט לוי בכלל ובפרט על הכהנים, ויראי ה' הוא מיוחד מן המיוחד על הכהנים והלויים הנגשים אל ה' וחסידים ואנשי מעשה, ואמר "בטח בה'" אפשר לפרשו מלשון ציווי שיצווה אותם שיבטחו בו כי הוא עזרם ומגנם לא הכסף והזהב. ויש לפרשו מלשון תאר, יאמרו ישראל או בית אהרן ויראי ה' שהוא בוטח בה' באמת ומפני בטחונו בה' הוא עזרם ומגנם תמיד.הנה לפי זה ביאר הסיבה הראשונה בגאולה ממה שאמר "שפוך חמתך אל הגויים אשר לא ידעוך ועל ממלכות אשר בשמך לא קראו", אמנם כנגד הסיבה השניה שהזכיר באמרו "כי אכל את יעקב" ואמר "על חסדך ועל אמתך" כמו שפירשתי שהוא מפאת עמו ונחלתו שראוי למחול עליו, על זה אמר "ה' זכרנו יברך", רצה לומר  ה' אשר זכר אותנו בהיותנו בגלות מצרים הוא אלהינו גם עתה בגלותנו זה והוא יברך אותנו, ומפרט מי ומי המבורכים באומרו "יברך את בית ישראל" שהוא עם ישראל בכלל שהוא בגלות, ו"יברך את בית אהרן" שהם הלויים והכהנים קדושיו, ו"יברך יראי ה'" שהם החסידים שבכל דור ודור. ולפי שהנערים הקטנים אין להם זכות בפני עצמם כי אם בהצטרך אל אבותיהם, לכן אמר "הקטנים עם הגדולים", שהקטנים בזכות הגדולים יתברכו גם כן. ולפי שצפה המשורר ברוח הקודש שישראל באורך הגלות יהיו הולכים ומתמעטים אמר נגדם בדרך הבטחה אל תפחדו ואל תיראו מהטמעתכם ואל תחשבו שמפני זה לא תהיו נושאים לגאולה כי "יוסף ה' עליכם ועל בניכם". ובזה הבטיחם שיתרבו במאד מאוד. ואמר "ברוכים אתם לה'" לתת ראיה על זה כאילו אמרו הביטו וראו שההשפעות הן כפי מדרגת המשפיעים, והאומות הן מושפעות משרי מעלה ולכן תהיה ברכתן ומעלתן בהדרגה וסדר טבעי אבל אתם אינכם כן כי "ברוכים אתם לה' עושה שמים וארץ". ובהיות הסיבה הראשונה השם יתברך והוא המשפיע ומברך אתכם אין מעצור בידו להושיע ברב או במעט, כי הוא עושה שמים וארץ גם מבלי הכנה. ואם תאמרו שגויי הארץ וממלכות האדמה מושלים בכל הארץ ומוחזקים בה ואין מי שימחה בידם, גם זה אינו כן כי "הארץ נתן לבני אדם", רצה לומר ממנו יתברך יא ולה' הארץ ומלואה, ולמאן דיַצִיבָא יתננה17יַצִיבָא = אזרח בארמית, כל שכן שהאומות והממלכות אפילו בחייהן קרויים מים לפי שאין להם השארה נפשית והן מתים בהחלט, ולא יהללו את ה' שנתן להם את הארץ. והוא אמרו "לא המתים יהללו יה ולא כל יורדי דומה", שהן האומות היורדין בכללותן אל הקבר שהוא "דומה" ודומם ונפשותם בלתי נשארת. אמנם אנחנו עם מרעיתו נברך יה לפי שיש לנו היכרא מגודל נפלאותיו, ואמר "מעתה ועד עולם" לרמוז אל זמן הגאולה, יהיה מתי שיהיה.}%endcomment
\newchap{הלל}
\newsection{מסיימים את ההלל}
\hebeng{\textbf{הַלֵּל} \par {\small מוזגין כוס רביעי וגומרין עליו את ההלל} }{\textbf{Hallel} {\small We pour the fourth cup and complete the Hallel} }
\hebeng{לֹא לָנוּ, ה׳, לֹא לָנוּ, כִּי לְשִׁמְךָ תֵּן כָּבוֹד, עַל חַסְדְּךָ עַל אֲמִתֶּךָ. לָמָּה יֹאמְרוּ הַגּוֹיִם אַיֵּה נָא אֱלֹהֵיהֶם. וְאֱלֹהֵינוּ בַּשָּׁמַיִם, כֹּל אֲשֶׁר חָפֵץ עָשָׂה. עֲצַבֵּיהֶם כֶּסֶף וְזָהָב מַעֲשֵׂה יְדֵי אָדָם. פֶּה לָהֶם וְלֹא יְדַבֵּרוּ, עֵינַיִם לָהֶם וְלֹא יִרְאוּ. אָזְנָיִם לָהֶם וְלֹא יִשְׁמָעוּ, אַף לָהֶם וְלֹא יְרִיחוּן. יְדֵיהֶם וְלֹא יְמִישׁוּן, רַגְלֵיהֶם וְלֹא יְהַלֵּכוּ, לׁא יֶהְגּוּ בִּגְרוֹנָם. כְּמוֹהֶם יִהְיוּ עֹשֵׂיהֶם, כֹּל אֲשֶׁר בֹּטֵחַ בָּהֶם. יִשְׂרָאֵל בְּטַח בַּיי, עֶזְרָם וּמָגִנָּם הוּא. בֵּית אַהֲרֹן בִּטְחוּ בַיי, עֶזְרָם וּמָגִנָּם הוּא. יִרְאֵי ה׳ בִּטְחוּ בַיי, עֶזְרָם וּמָגִנָּם הוּא. יי זְכָרָנוּ יְבָרֵךְ. יְבָרֵךְ אֶת בֵּית יִשְׂרָאֵל, יְבָרֵךְ אֶת בֵּית אַהֲרֹן, יְבָרֵךְ יִרְאֵי ה׳, הַקְּטַנִים עִם הַגְּדֹלִים. יֹסֵף ה׳ עֲלֵיכֶם, עֲלֵיכֶם וְעַל בְּנֵיכֶם. בְּרוּכִים אַתֶּם לַיי, עֹשֵׂה שָׁמַיִם וָאָרֶץ. הַשָּׁמַיִם שָׁמַיִם לַיי וְהָאָרֶץ נָתַן לִבְנֵי אָדָם. לֹא הַמֵּתִים יְהַלְלוּ יָהּ וְלֹא כָּל יֹרְדֵי דוּמָה. וַאֲנַחְנוּ נְבָרֵךְ יָהּ מֵעַתָּה וְעַד עוֹלָם. הַלְלוּיָהּ. }{Not to us, not to us, but rather to Your name, give glory for your kindness and for your truth. Why should the nations say, "Say, where is their God?" But our God is in the heavens, all that He wanted, He has done. Their idols are silver and gold, the work of men's hands. They have a mouth but do not speak; they have eyes but do not see. They have ears but do not hear; they have a nose but do not smell. Hands, but they do not feel; feet, but do not walk; they do not make a peep from their throat. Like them will be their makers, all those that trust in them. Israel, trust in the Lord; their help and shield is He. House of Aharon, trust in the Lord; their help and shield is He. Those that fear the Lord, trust in the Lord; their help and shield is He. The Lord who remembers us, will bless; He will bless the House of Israel; He will bless the House of Aharon. He will bless those that fear the Lord, the small ones with the great ones. May the Lord bring increase to you, to you and to your children. Blessed are you to the Lord, the maker of the heavens and the earth. The heavens, are the Lord's heavens, but the earth He has given to the children of man. It is not the dead that will praise the Lord, and not those that go down to silence. But we will bless the Lord from now and forever. Halleluyah! (Psalms 115)}
\hebeng{אָהַבְתִּי כִּי יִשְׁמַע ה׳ אֶת קוֹלִי תַּחֲנוּנָי. כִּי הִטָּה אָזְנוֹ לִי וּבְיָמַי אֶקְרָא. אֲפָפוּנִי חֶבְלֵי מָוֶת וּמְצָרֵי שְׁאוֹל מְצָאוּנִי, צָרָה וְיָגוֹן אֶמְצָא. וּבְשֵׁם ה׳ אֶקְרָא: אָנָּא ה׳ מַלְּטָה נַפְשִׁי. חַנוּן ה׳ וְצַדִּיק, וֵאֱלֹהֵינוּ מְרַחֵם. שֹׁמֵר פְּתָאִים ה׳, דַּלוֹתִי וְלִי יְהושִׁיעַ. שׁוּבִי נַפְשִׁי לִמְנוּחָיְכִי, כִּי ה׳ גָּמַל עָלָיְכִי. כִּי חִלַּצְתָּ נַפְשִׁי מִמָּוֶת, אֶת עֵינִי מִן דִּמְעָה, אֶת רַגְלִי מִדֶּחִי. אֶתְהַלֵךְ לִפְנֵי ה׳ בְּאַרְצוֹת הַחַיִּים. הֶאֱמַנְתִּי כִּי אֲדַבֵּר, אֲנִי עָנִיתִי מְאֹד. אֲנִי אָמַרְתִּי בְחָפְזִי כָּל הָאָדָם כּזֵֹב. }{I have loved the Lord - since He hears my voice, my supplications. Since He inclined His ear to me - and in my days, I will call out. The pangs of death have encircled me and the straits of the Pit have found me and I found grief. And in the name of the Lord I called, "Please Lord, Spare my soul." Gracious is the Lord and righteous, and our God acts mercifully. The Lord watches over the silly; I was poor and He has saved me. Return, my soul to your tranquility, since the Lord has favored you. Since You have rescued my soul from death, my eyes from tears, my feet from stumbling. I will walk before the Lord in the lands of the living. I have trusted, when I speak - I am very afflicted. I said in my haste, all men are hypocritical. (Psalms 116:1-11)}%
\commentb{\textrm{\textbf{אָהַבְתִּי כִּי יִשְׁמַע יְהוָה אֶת קוֹלִי תַּחֲנוּנָי, כִּי הִטָּה אָזְנוֹ לִי וּבְיָמַי אֶקְרָא וגו'}} (תהלים קט"ז, א' – ב').לפי שצפה המשורר ברוח הקודש כי באורך הגלות יאמרו האומות לבני ישראל תפילותיכם תפלות שווא לבלי הועיל, לכן אמר במזמור הזה "אהבת כי ישמע ה' את קולי וגו'. והמפרשים פירשו "אהבתי" את ה' לפי שישמע את קולי ותחנוני. ולי נראה שמילת "אהבתי" חוזרת לתחנוני. כאילו אמר "אהבתי את תחנוני כי ישמע ה' את קולי, ועניינו אני אהבתי תפילתי ובקשתי ולא אעזבם לפי שאני מאמין אמונה קיימת כי ישמע ה' את קולי ויגאלני. והראיה לזה "כִּי הִטָּה אָזְנוֹ לִי" (תהלים קט"ז, ב'), רצה לומר בשהייתי בגלות מצרים הטה אזנו לי ולצעקתי וגאלני, ולכן אף על פי שיארכו לי הימים בגלות הזה בכל ימי אקרא ולא אחדל מהתפלל, כי הנה בגלות הארוך הזה אפפוני וסבבוני חבלי מוות בכמה שמדות, וכמה הריגות שנגזרו על קהילות רבות שמתו על קדושת ה', ושאר הרעות והביזות שסבבוני בגלות, ואף על פ שצרה ויגון מצא בשם ה' אקרא ואתפלל לפניו ואומר אנא ה' מלטה נפשי, אל תעש עמנו כלה  בגלות הזה, כי אתה ה' חנון ורחום. אמנם צדיק על כל הבא עלינו, אבל עם כל תוקף הדין והמשפט הצודק אלהינו מרחם. וכן בזה הגלות אומר לנפשי "שׁוּבִי נַפְשִׁי לִמְנוּחָיְכִי" (שם שם, ז'), רצה לומר "שׁוּבָה יִשְׂרָאֵל עַד ה' אֱלֹהֶיךָ" (הושע י"ד, ב'), כי אם השיבו אליו קרובה ישועתו לבוא, והביא להם ראיה מגלות מצרים באמרו "כִּי ה' גָּמַל עָלָיְכִי" (תהלים קט"ז, ז') שבהיות המצריים מסכימים בהריגת בני ישראל כמו שנאמר "כל הבן הילוד היאורה תשליכוהו", אתה ה' אלהים שמה "חלצת נפשי ממוות את עיני מדמעה" ומבכי, ואת "רגלי מדחי" כשהוצאתני משם והבאתני לארצך ונחלתך, ולכן גם עתה אבטח בך ש"אתהלך לפני ה' בארצות החיים" שהיא ארץ ישראל. כמו שאמר יחזקיה "אמרתי לא אראה יה בארץ החיים" (ישעיהו ל"ח, י"א), כי כן נקראת ארץ ישראל לפי שהיא נותנת חיים ליושביה וצדיקים במיתתם נקראים חיים. וגם כפי קבלתם ז"ל מתים יחיו ויקימו ראשונה לזמן התחיה בארץ ישראל.ולפי שלאורך הגלות רבים יתייאשו מן התשועה באמרם "יָבְשׁוּ עַצְמוֹתֵינוּ ... נִגְזַרְנוּ לָנוּ" (יחזקאל ל"ז, י"א), לכן אמר המשורר בשם האומה, "הֶאֱמַנְתִּי כִּי אֲדַבֵּר אֲנִי עָנִיתִי מְאֹד" (תהלים קט"ז, י') , רצה לומר תמיד האמנתי אמונה קיימת שכמו שאנחנו עתה בליל פסח מדברים ומספרים ביציאת מצרים ומגידים בעבדות והעינוי אשר סבלנו שם, ככה בזמן גאולה האמנתי שאדבר ואספרה אז איך הייתי בגלות הזה עני ודל. וזהו "אני אדבר אני עניתי מאוד", שאדבר ואספר הגלות והעינוי אשר סבלנו. וגם כן אדבר איך "אני אמרתי בחפזי" שהוא בצרת הגלות ומצוקותיו, מלשון "ויהי דוד נחפז ללכת מפני שאול" (שמואל א', כ"ג, כ"ו), או "רקים ופוחזים" (שופטים ט', ד'), שהוא לשון המורה על השפלות והצרה, אז בזמן הגאולה שאהיה חפשי אדבר ואספר איך הייתי אומר בימים ההם "כל האדם כוזב". רצה לומר כל הנביאים שהתנבאו על גאולתי ותשועתי כולם כוזבים לפי שעבר קציר כלה קיץ ואנחנו לא נושענו, כי כזב משה רבינו עליו השלום בייעודיו, כזב ישעיה בנחמותיו, כזבו ירמיה ויחזקאל בנבואותיהם, וכן שאר הנביאים כולם – כל האדם כוזב. כי יבוא זמן בגאולה העתידה שתזכור האומה ותדבר כל הדברים המתיאשים שהיו אומרים בזמן הגלות. ואפשר עוד לפרש "אני אמרתי בחפזי כל האדם כוזב" שישראל בגלות הזה עם כל חפזו וצרותיו תמיד אומרים אל האומות המושלות עליהם, לענין אמונותיהם ודתיהם, "כל האדם כוזב", לא מרובכם מכל העמים ואנחנו מתי מעט תחשבו שהאמת אתכם וכי רבים יחכמו, ואינו כן כי כל האדם כוזב באמונותיהם ואמת ה' לעולם אתנו.ולפי שבזמן התשועה יגדיל ה' לעשות עמנו לכן אמר "מָה אָשִׁיב לַה' כָּל תַּגְמוּלוֹהִי עָלָי" (תהלים קט"ז, י"ב), רצה לומר מה גמול אשיב לו, במה אקדם פניו על כל הטובה אשר עשה עמי כל תגמולוהי עלי, והם ממנו יתברך על צד החסד לא על צד הדין. וכל תגמולוהי עלי ואני בעל חובו בכולם, ומה לי לעשות כי אם ככל אשר אנחנו עושים פה היום שמהללים אותו בכוס של ברכות, וכן בזמן ההוא "כּוֹס יְשׁוּעוֹת אֶשָּׂא" (שם שם, י"ג), רצה לומר לעתיד אני נושא כוס ישועות אחת שהיא ישועת מצרים, אבל בקיבוץ הגלויות כוס ישועות רבות אשא ואקרא בשמו ואפרסם בכל בני אדם גבורתו, ואז בבוא תשועתו "נדרי לה' אשלם" (שם שם, י"ד) לא בבית כמעשה ליל הסדר כי אם בבית המקדש "נגדה נא לכל עמו". ואם תאמר שזה יצדק בזוכים לחיים בזמן הגאולה אבל במקהלות ורבבות עם ישראל שמתו בגלות בנפש מרה ולא ראו בטובה מה יהיה ענינם? על זה "יקר בעיני ה' המותה לחסידיו", רצה לומר יקר וגדולה וכבוד ושלמות הנפש יקנו חסידי ה' במותם בגלות, כי יהיו נפשותם צרורות בצרור החיים את ה'.ואחרי שהזכיר כל זאת בדרך בטחון ואמונה שהוא מאמין שכך יהיה, התפלל אל ה' בשם האומה "אנא ה' כי אני עבדך", רצה לומר זכור נא כי אני עבדך ולא קנין כספך, אלא יליד בית כבן האמה שאמרה התורה "הָאִשָּׁה וִילָדֶיהָ תִּהְיֶה לַאדֹנֶיהָ" (שמות כ"א, ד'), ולכן אמר שתי פעמים מילת "עבדך" שהוא רמז לעם העומד בגלות שהוא מאותה אומה שנתיחדה לעבודתו מכל האומות. וזהו בן אמתך, והיתה טענתו "פתחת למוסרי", רצה לומר הלא אתה ה' אלהינו כשהיינו במצרים אסורים בכור הברזל פתחת למוסרי והוצאתני משעבוד לגאולה, לכן יהי רצון מלפניך שגם עתה תעשה עמדי כן, ואם שם עשו ישראל זבח חג הפסח ואמרו שירה, גם עתה לך אזבח תודה במקום אותו זבח, ובשם ה' אקרא במקום אותה שירה, ויהיה לנו גם בזה יתרון גדול כי הם אמרו השיר וההודאה במדבר כל קהלם יחד, ואנחנו "נדרי לה' אשלם נגדה נא לכל עמו" כמו שהם עשו שירה בקהל גדול, ומוסף עליהם שאנחנו נעשה אותה ההודעה "בחצרות בית ה' בתוככי ירושלים" ובאה ב"ה" מלת "נגדה" כמו נגד ה לפי שהיה שם חמישה חלקים, עזרת נשים, ועזרת ישראל, אולם, דביר והיכל, שכאשר יהיו כל בני ישראל מקובצים תוך חצרות בית ה' שהוא בתוך גבול ירושלים אז ישולמו הנדרים האלה. ותוספת ה"י" בתוככי הוא לרמוז עשרה ניסים שנעשו לאבותינו בבית המקדש, ולכן היה סוף דבריו "הללויה", שיאמר לכל העם: הללו את ה' על הניסים ועל הפורקן ועל הגבורות ועל התשועות שעשה עמנו.}%endcomment
\hebeng{מָה אָשִׁיב לַיי כֹּל תַּגְמוּלוֹהִי עָלָי. כּוֹס יְשׁוּעוֹת אֶשָּׂא וּבְשֵׁם ה׳ אֶקְרָא. נְדָרַי לַיי אֲשַׁלֵּם נֶגְדָה נָּא לְכָל עַמּוֹ. יָקָר בְּעֵינֵי ה׳ הַמָּוְתָה לַחֲסִידָיו. אָנָּה ה׳ כִּי אֲנִי עַבְדֶּךָ, אֲנִי עַבְדְּךָ בֶּן אֲמָתֶךָ, פִּתַּחְתָּ לְמוֹסֵרָי. לְךָ אֶזְבַּח זֶבַח תּוֹדָה וּבְשֵׁם ה׳ אֶקְרָא. נְדָרַי לַיי אֲשַׁלֵּם נֶגְדָה נָּא לְכָל עַמּוֹ. בְּחַצְרוֹת בֵּית ה׳, בְּתוֹכֵכִי יְרוּשָלַיִם. הַלְלוּיָהּ. }{What can I give back to the Lord for all that He has favored me? A cup of salvations I will raise up and I will call out in the name of the Lord. My vows to the Lord I will pay, now in front of His entire people. Precious in the eyes of the Lord is the death of His pious ones. Please Lord, since I am Your servant, the son of Your maidservant; You have opened my chains. To You will I offer a thanksgiving offering and I will call out in the name of the Lord. My vows to the Lord I will pay, now in front of His entire people. In the courtyards of the house of the Lord, in your midst, Jerusalem. Halleluyah! (Psalms 116:12-19)}
\hebeng{הַלְלוּ אֶת ה׳ כָּל גּוֹיִם, שַׁבְּחוּהוּ כָּל הָאֻמִּים. כִּי גָבַר עָלֵינוּ חַסְדּוֹ, וֶאֱמֶת ה׳ לְעוֹלָם. הַלְלוּיָהּ. הוֹדוּ לַיי כִּי טוֹב כִּי לְעוֹלָם חַסְדּוֹ. יֹאמַר נָא יִשְׂרָאֵל כִּי לְעוֹלָם חַסְדּוֹ. יֹאמְרוּ נָא בֵית אַהֲרֹן כִּי לְעוֹלָם חַסְדּוֹ. יֹאמְרוּ נָא יִרְאֵי ה׳ כִּי לְעוֹלָם חַסְדּוֹ. }{Praise the name of the Lord, all nations; extol Him all peoples. Since His kindness has overwhelmed us and the truth of the Lord is forever. Halleluyah! Thank the Lord, since He is good, since His kindness is forever. Let Israel now say, "Thank the Lord, since He is good, since His kindness is forever." Let the House of Aharon now say, "Thank the Lord, since He is good, since His kindness is forever." Let those that fear the Lord now say, "Thank the Lord, since He is good, since His kindness is forever." (Psalms 117-118:4)}%
\commentb{\textrm{\textbf{הללו את ה' כל גוים שבחוהו כל לאומים וגו'}} (תהלים קי"ז). לפי שביציאת מצרים קיבלו בני ישראל אמונת ה' והללו גבורתו, ושאר האומות לא קבלו אמונתו ולא הללוהו, ולכן אמר שלא יהיה כן בגאולה העתידה כי הגויים והאומות כלם הללו וישבחו את ה', מפני שתי הסיבות אשר הזכיר למעל באומרו "על חסדך ועל אמתך", ועליהם אמר "כִּי גָבַר עָלֵינוּ חַסְדּוֹ וֶאֱמֶת ה' לְעוֹלָם" (שם שם, ב'), רצה לומר בעבור חסדו שגבר עלינו בתשועה ובעבוד אמתת אמונתו שנתפרסמה בכל העולם. והוא אמרו "ואמת ה' לעולם הללויה".ואף שגם האומות והלאומים יודו שמו ואמתתו, הנה עם בני ישראל ראוי שיעשה שבח מיוחד בפני עצמם, והוא אמרו "הודו לה' כי טוב כי לעולם חסדו", רצה לומר שהוא מהולל ומשובח משתי בחינות, בבחינת מעלתו ושלמותו כי הוא הטוב המוחלט, ובחינה שניה מצד חסדיו שעושה עם העולם בשמירתו ותשועתו. ולכן אמר שההלול הזה "יאמר נא ישראל", ומילת "נא" מלשון עתה חיינו בגלות. ואם יאמר עתה ישראל כי לעולם חסדו הוא רק על דרך ההעברה כי חסדיו לא הגיעונו בהיותנו בגלות, אבל בזמן הגאולה אז יהיה זמן ראוי שיאמר ישראל כי לעולם חסדו כי אז יקבלו החסד האמתי. והסתכל כי לא אמר יאמר נא ישראל הודו ה' כי טוב כי לעולם חסדו, אלא רק "כי לעולם חסדו", וענינו כי באמרו הודו לה' כי טוב שהוא כפי שלמותו בעצמו כבר יהיו כל הזמנים שווים אליו ובכל עת ובכל רגע יאות לומר הודו לה' כי טוב כפי הבחינה השני, ולכן אמר  יאמר נא ישראל כי לעולם חסדו לפי שיצאו מעבדות לחירות, והוא החסד האמתי. גם בית אהרן לוים וכהני ה' שעתה בגלות אין להם עבודה ולא קדושה לא מעשרות ולא מתנות וכאילו כמלה מהם כהונה ולויה, אבל בזמן הגאולה ישובו למעלתם באמת, ובזמן ההוא יאות לומר "כי לעולם חסדו" כפי החסדים שיקבלו בגאולת ישראל. וכן יראי ה' שלא היה להם בגלות פנאי והכנה להדבק באלהים מפחד האויבים וצרתם, גם להם יאות ההלול לומר בזמן ההוא כי לעולם חסדו. ואפשר עוד לומר שכיון במאמר כי לעולם חסדו, כלומר אל תחשבו שאחרי הגאולה הזו נשוב לגלות אחר כמו אחרי יציאת מצרים באנו בגלות, זה אינו כן כי אותו החסד שעשה עמהם ביציאת מצרים היה זמני, והחסד שיעשה עמהם בתשועה הגדולה הבאה תהיה להם לעולם ועד, וזהו אמרו כי לעולם חסדו, רצה לומר לנצח נצחים יהיה החסד ההוא, על דרך מאמר הנביא "וְחַסְדִּי מֵאִתֵּךְ לֹא יָמוּשׁ וּבְרִית שְׁלוֹמִי לֹא תָמוּט" (ישעיהו נ"ד, י').אמנם אמר עוד "מן המצר קראתי יה", שאין ראוי שנתיאש מפני תוקף צרות הגלות, כי הנה כבר ראינו בגלות מצרים, כי מן המצר קראתי יה כמו שנאמר  "ויצעקו בני ישראל אל ה'", ו"ענני במרחב יה", רצה לומר הוצאני מצרה לרוחה, ולכן אני בטוח שכן יהיה בעתיד, והוא אמרו "ה' לי לא אירא", ולכן בא פסוק "מן המצר" בלשון עבר לפי שמדובר מגלות מצרים שהביא לראיה, ושאר הפסוקים באו בלשון עתיד, וכיון בזה לומר עוד כי הגלות וצרותיו הם באמת בעבור שישראל אינם קוראים אל ה' ואינם שבים אליו בכונה רצויה, כי אם היו דבקים בה' לא היו יראים מהעמים. והביא לראיה ענין מצרים באמרו מן המצר קראתי יה ענני, כלומר כבר הייתי בצרות רבות אחרות וכאשר קראתי בכל לב אל ה' ענני והוציאני למרחה, כן גם עתה אם היה ה' לי והייתי קרוב אליו מה יעשה לי אדם? כי הוא יתברך יהפוך הצרים לעוזרים והאויבים לאוהבים, וכמאמר שלמה "בִּרְצוֹת ה' דַּרְכֵי אִישׁ גַּם אוֹיְבָיו יַשְׁלִם אִתּוֹ" (משלי ט"ז, ז'), וזהו "ה' לי בעוזרי ואני אראה בשונאי", שהמה יעזרוני, או יאמר "ואני אראה בשונאי" רצה לומר אראה ברעתם. ולכן בהיותנו בגלות "טוב לחסות בה'" בתפילות ומעשים טובים "מבטוח בנדיבים", ואמר באדם ובנדיבים לפי שבגלות הזה פעמים יבואו הצרות מפאת העמים וצריך לשוחדם ולעובדם ולהתרצות אל המון, ופעמים יבואו הצרות מפאת המלכים והשרים וצריכים אנו להשתדל לפניהם ולרצותם, לכן אמר שטוב לחסות בה' ולהדבק בו מלבטוח באדם שהוא ההמון או לבטוח בנדיבים שהם מלכיהם ושריהם. והזכירם פה מאשר העמים כולם ומלכיהם אכלונו והממונו והם היו בעוכרינו.ואמר בזה כל גוים סבבוני גם סבבוני כדבורים, שהם ארבע לשונות מסבוב, ואמרו המפרשים שהוא לחוזק המליצה, ואני אחשוב שאמר כל גויים סבבוני על סנחריב ובני אשור שהחריבו שומרון ובנותיה והגלו השבטים ובאו עד ירושלים לשחתה, אך היתה בהם מפלה רבה מאין תקומה על חטאם נגד ישראל, וזהו שאמר "בשם ה' כי אמילם" לפי שבסיבתי נכרתו מן העולם. ואמר "סבוני גם סבבוני" שניבא על נבוכדנצר מלך בבל שבא על ירושלים פעמים רבות בימי יהויקים ובימי יהויכין ובימי צדקיהו והחריבה, ולכן אמר על הבבליים סבוני גם סבבוני, ובעון מה שעשו לישראל ספו תמו מן בלהות וכמו שייעד הנביא עליהם, וזהו אמרו "בשם ה' כי אמילם".ואמר עוד סבוני כדבורים כיון אל פרס ומדי ועל היוונים בזמן בית שני, ואמר "דועכו כאש קוצים בשם ה' כי אמילם", לפי שבני חשמונאי השמידום. אמנם על גלות אדום הארוך הזה אמר "דחה דחיתני לנפול", רצה לומר לא היתה לי דחייה קרובה לנפילה ולכליה מוחלטת כי אם בגלות רומי, אם לא שה' עזרני והצילני מידם.עָזִּי וְזִמְרָת יָהּ וַיְהִי לִי לִישׁוּעָה (תהלים קי"ח, י"ד), רצה לומר העוז אשר היה לי שלא באתי לכליה בגלות אדום שעליו אזמר לה' אלהי ישראל, הוא יה בורא העולם, וכמו שהשיב החכם למאמר האומר "גדולה הכבשה שעמדה בין זאבים", והוא השיבו "גדול הרועה שמצילה"18רבי יהושע בן חנניה לאדריאנוס קיסר – מתוך מדרש תנחומא על תולדות ה', ובישועתו והצלתו יהיה קול רנה וישועה באהלי צדיקים שהם ישראל ויאמרו ברינתם "ימין ה' עושה חיל", לא אומר כחי ועוצם ידי עשה לי את החיל הזה כי אם ימין ה' הוא עושה חיל לא כחי. והזכיר הסיבה לה ייחס אליו יתברך החיל באומרו "ימין ה' רוממה ימין ה' עושה חיל". והסתכל שהדברים האלה "עזי וזמרת יה", "ימין ה' רוממה" הם מדברי שירת הים, כאילו אמר כמו שעתה אנחנו גומרים את ההלל ומשוררים ומשבחים על גאולת מצרים, כן בזמן התשועה העתידה יהיה קול רינה וישועה באהלי צדיקים, ששירות ותושבחות יאמרו שמה, וגם מאותם הדברים עצמם שאמרו בשירת הים. ולפי שראה המשורר ברוח קדשו שיפול ספק גדול בלבבות האנשים אם ישראל יתמו בגלות מתוך צרותיהם לכן הוצרך לומר כמבטיח ומדבר בשם האומה: "לא אמות כי אחיה ואספר מעשה יה", רצה לומר לא אמות ולא אכלה בגלות אבל תמיד אחיה וצפה תשועת ה' כדי לספר מעשה יה כי נורא הוא. ואף על פי שיתקפוני צרות לא תהיה כליה גמורה כי יסור יסרני יה ולמות לא נתנני. וכאילו היו פתחי התהלה סגורים עד זמן התשועה לכן אמר: "פתחו לי שערי צדק" כי אני ישראל  "אבוא בם אודה יה". וכאילו השערים ישובוהו "זה השער לה' צדיקים יבואו בו", רצה לומר ראה אתה ישראל אם שבת בתשובה שלמה כי השער הזה יהיה סגור עד אשר תשוב אל ה', לפי שצדיקים יבואו בו לא אנשים הפושעים. ואז ישראל יתחיל בגנות ויסיים בשבח ויאמר: "אודך כי עניתני" בגלות המר והנמהר הזה, כיון שבאחרונה היתה לי לישועה, באופן שהאבן שמאסו הבונים שהוא עם ישראל הנקרא אבן שנאמר "מִשָּׁ֥ם רֹעֶ֖ה אֶ֥בֶן יִשְׂרָאֵֽל" (בראשית מ"ט, כ"ד), אותו אבן שהבונים והם האומות מואסים בו נעשה עתה לראש פנה, כי בנין העולם ובית ה' יהיה עליו, והתשועה הגדולה הזאת תהיה לא בחיל ולא בכח אנושי כי אם בכחו יתברך, וזהו "מאת ה' היתה זאת היא נפלאת בעינינו|\ שתהיה הגאולה פליאה רבה בעיני כל אדם.ואפשר לפרש עוד שיהיה זה מאמר הבונים ותשובתם ויאמרו: אמת הוא שהיינו מואסים האבן הזאת, אבל זאת התשועה מאת ה' היא ולכן היא נפלאת בעינינו, וזה היום שהוא זמן הגאולה כיון שעשאו ה' נגילה ונשמחה בו, רצה לומר לא נתעצב ולא נדאג על מעלתכם כיון שהוא מתת אלהים, אבל גם אנחנו האומות נגילה ונשמחה בו, הוא מה שנאמר למעלה "הללו את ה' כל גוים שבחוהו כל האומים" וגו'.}%endcomment
\hebeng{מִן הַמֵּצַר קָרָאתִי יָּהּ, עָנָנִי בַמֶּרְחַב יָהּ. ה׳ לִי, לֹא אִירָא – מַה יַּעֲשֶׂה לִי אָדָם, ה׳ לִי בְּעֹזְרָי וַאֲנִי אֶרְאֶה בְּשׂנְאָי. טוֹב לַחֲסוֹת בַּיי מִבְּטֹחַ בָּאָדָם. טוֹב לַחֲסוֹת בַּיי מִבְּטֹחַ בִּנְדִיבִים. כָּל גּוֹיִם סְבָבוּנִי, בְּשֵׁם ה׳ כִּי אֲמִילַם. סַבּוּנִי גַם סְבָבוּנִי, בְּשֵׁם ה׳ כִּי אֲמִילַם. סַבּוּנִי כִדְּבֹרִים, דֹּעֲכוּ כְּאֵשׁ קוֹצִים, בְּשֵׁם ה׳ כִּי אֲמִילַם. דָּחֹה דְּחִיתַנִי לִנְפֹּל, וַיי עֲזָרָנִי. עָזִּי וְזִמְרָת יָהּ וַיְהִי לִי לִישׁוּעָה. קוֹל רִנָּה וִישׁוּעָה בְּאָהֳלֵי צַדִּיקִים: יְמִין ה׳ עֹשָׂה חָיִל, יְמִין ה׳ רוֹמֵמָה, יְמִין ה׳ עֹשָׂה חָיִל. לֹא אָמוּת כִּי אֶחְיֶה, וַאֲסַפֵּר מַעֲשֵׂי יָהּ. יַסֹּר יִסְּרַנִי יָּהּ, וְלַמָּוֶת לֹא נְתָנָנִי. פִּתְחוּ לִי שַׁעֲרֵי צֶדֶק, אָבֹא בָם, אוֹדֶה יָהּ. זֶה הַשַּׁעַר לַיי, צַדִּיקִים יָבֹאוּ בוֹ. }{From the strait I have called, Lord; He answered me from the wide space, the Lord. The Lord is for me, I will not fear, what will man do to me? The Lord is for me with my helpers, and I shall glare at those that hate me. It is better to take refuge with the Lord than to trust in man. It is better to take refuge with the Lord than to trust in nobles. All the nations surrounded me - in the name of the Lord, as I will chop them off. They surrounded me, they also encircled me - in the name of the Lord, as I will chop them off. They surrounded me like bees, they were extinguished like a fire of thorns - in the name of the Lord, as I will chop them off. You have surely pushed me to fall, but the Lord helped me. My boldness and song is the Lord, and He has become my salvation. The sound of happy song and salvation is in the tents of the righteous, the right hand of the Lord acts powerfully. I will not die but rather I will live and tell over the acts of the Lord. The Lord has surely chastised me, but He has not given me over to death. Open up for me the gates of righteousness; I will enter them, thank the Lord. This is the gate of the Lord, the righteous will enter it. (Psalms 118:5-20)}
\hebeng{אוֹדְךָ כִּי עֲנִיתָנִי וַתְּהִי לִי לִישׁוּעָה. אוֹדְךָ כִּי עֲנִיתָנִי וַתְּהִי לִי לִישׁוּעָה. אֶבֶן מָאֲסוּ הַבּוֹנִים הָיְתָה לְראשׁ פִּנָּה. אֶבֶן מָאֲסוּ הַבּוֹנִים הָיְתָה לְראשׁ פִּנָּה. מֵאֵת ה׳ הָיְתָה זֹּאת הִיא נִפְלָאת בְּעֵינֵינוּ. מֵאֵת ה׳ הָיְתָה זֹּאת הִיא נִפְלָאת בְּעֵינֵינוּ. זֶה הַיּוֹם עָשָׂה ה׳. נָגִילָה וְנִשְׂמְחָה בוֹ. זֶה הַיּוֹם עָשָׂה ה׳. נָגִילָה וְנִשְׂמְחָה בוֹ.}{I will thank You, since You answered me and You have become my salvation. The stone that was left by the builders has become the main cornerstone. From the Lord was this, it is wondrous in our eyes. This is the day of the Lord, let us exult and rejoice upon it. (Psalms 118:21-24)}
\hebeng{אָנָּא ה׳, הוֹשִיעָה נָּא. אָנָּא ה׳, הוֹשִיעָה נָּא. אָנָּא ה׳, הַצְלִיחָה נָא. אָנָּא ה׳, הַצְלִיחָה נָא. }{Please, Lord, save us now; please, Lord, give us success now! (Psalms 118:25)}%
\commentb{\textrm{\textbf{"אָנָּא ה' הוֹשִׁיעָה נָּא אָנָּא ה' הַצְלִיחָה נָּא" וגו'}} (תהלים קי"ח, כ"ה). עתה יתפלל המשורר שיקים ה' את דברו הטוב, והוא אמרו "אנא ה' הושיעה נא", ולפי שהתשועה תלויה בתשובת ישראל אמר "אנא ה' הצליחה נא", שיצליח בידיהם לעשות תשובה ומעשים טובים. ומה שאמר "בָּרוּךְ הַבָּא בְּשֵׁם ה'", רצה לומר ברוך ומבורך יהיה מבית ה' כיוון שהוא בא בשמו להתקרב לעבודתו, ולפי שבא ליטהר מסייעים אותו אמר: "אֵל ה' וַיָּאֶר לָנוּ" כי בהיותו מתיישרים נגדו הוא יאיר לפנינו דרכו ואף על פי שהגוף הנגוף ויצר לב האדם הרע יעכבוהו מזה, אתן יראי ה' וחושבי שמו "אִסְרוּ חַג בַּעֲבֹתִים", רצה לומר אסרו הכבש הזה שהוא רמז לגוף וליצר הבהמי, אסרו אותו עד שיגיע ויבוא אל קרנות מזבח ה' שהוא תכליתו האמתי, וכאשר יהיה זה אז תשבחו ותאמרו "אֵלִי אַתָּה וְאוֹדֶךָּ אֱלֹהַי אֲרוֹמְמֶךָּ הוֹדוּ לַיהוָה כִּי טוֹב כִּי לְעוֹלָם חַסְדּוֹ".גם אפשר לפרש הפסוקים האלה על עניין האומות שהזכיר, ואמר "אנא ה' הושיעה נא" לישראל, אנא ה' הצליחה נא" לאומות הבאות לחסות תחת כנפיך, שהוא מלשון "ותצלח עליו רוח ה'", לכן יאמר כנגדם ברוך הבא בשם ה', שאתם באים להתגייר ואנחנו בני ישראל ברכנוכם בשם ה', שתהיה ברכתו עליכם. אמנם אור הנבואי והשכינה לא תחול כי אם בעם סגולתו, וזהו שאמר "אל ה' ויאר לנו". עוד ילמדו לאומות מה שראוי להם לעשות כדי להתקרב אל האלהים, והוא אמרו "אסרו חג בעבותים עד קרנות המזבח", רצה לומר הביאו הקרבן בשמחה ובשירים ובעבותות האהבה עד שיגיע לקרנות המזבח, כי כל קרבן יקרא חג, כמו "חַגִּים יִנְקֹפוּ" (ישעיהו כ"ט, א'), ואמר זה לפי שמוטל על האומות הנכנסות תחת כנפי השכינה להביא קרבן, אבל עם ישראל על עצמו ביחוד הוא אומר "אלי אתה ואודך" וגו', הודו לה' כי טוב" וגו'.ואפשר לפרש עוד "אנא ה' הושיעה נא" על תשועת ישראל וגאולתם, ו"אנא ה' הצליחה נא" על מעשיהם וקנינם. ואמר "ברוך הבא בשם ה'" על מלך המשיח כי כאשר השם יתברך יושיע ויצליח לעמו יבוא משיחו בשם ה', ויאמר לישראל "ברכנוכם מבית ה'", רצה לומר ברכתכם והשפעתכם תהיה מבית ה' שהוא מכסא כבודו ומרום קדשו. "אל ה' ויאר לנו" שיאיר פניו אלינו ויצווה את בני ישראל לעשות חג לה' בשמחה ובשירים. והוא אמרו "אסרו חג בעבותים עד קרנות המזבח", רצה לומר לא תעשו חג בבתיכם כמו בגלות אלא בבית ה' הביאו החג אסור בעבותות אהבה עד קרנות המזבח, ואז תהללו ותאמרו לפניו "אלי אתה ואודך אלהי ארוממך, הודו לה' כי טוב כי לעולם חסדו".}%endcomment
\hebeng{בָּרוּךְ הַבָּא בְּשֵׁם ה׳, בֵּרַכְנוּכֶם מִבֵּית ה׳. בָּרוּךְ הַבָּא בְּשֵׁם ה׳, בֵּרַכְנוּכֶם מִבֵּית ה׳. אֵל ה׳ וַיָּאֶר לָנוּ. אִסְרוּ חַג בַּעֲבֹתִים עַד קַרְנוֹת הַמִּזְבֵּחַ. אֵל ה׳ וַיָּאֶר לָנוּ. אִסְרוּ חַג בַּעֲבֹתִים עַד קַרְנוֹת הַמִּזְבֵּחַ. אֵלִי אַתָּה וְאוֹדֶךָּ, אֱלֹהַי – אֲרוֹמְמֶךָּ. אֵלִי אַתָּה וְאוֹדֶךָּ, אֱלֹהַי – אֲרוֹמְמֶךָּ. הוֹדוּ לַיי כִּי טוֹב, כִּי לְעוֹלָם חַסְדּוֹ. הוֹדוּ לַיי כִּי טוֹב, כִּי לְעוֹלָם חַסְדּוֹ. }{Blessed be the one who comes in the name of the Lord, we have blessed you from the house of the Lord. God is the Lord, and He has illuminated us; tie up the festival offering with ropes until it reaches the corners of the altar. You are my Power and I will Thank You; my God and I will exalt You. Thank the Lord, since He is good, since His kindness is forever.(Psalms 118:26-29)}
\hebeng{יְהַלְלוּךָ ה׳ אֱלֹהֵינוּ כָּל מַעֲשֶׂיךָ, וַחֲסִידֶיךָ צַדִּיקִים עוֹשֵׂי רְצוֹנֶךָ, וְכָל עַמְּךָ בֵּית יִשְׂרָאֵל בְּרִנָה יוֹדוּ וִיבָרְכוּ, וִישַׁבְּחוּ וִיפָאֲרוּ, וִירוֹמְמוּ וְיַעֲרִיצוּ, וְיַקְדִּישׁוּ וְיַמְלִיכוּ אֶת שִׁמְךָ, מַלְכֵּנוּ. כִּי לְךָ טוֹב לְהוֹדותֹ וּלְשִׁמְךָ נָאֶה לְזַמֵּר, כִּי מֵעוֹלָם וְעַד עוֹלָם אַתָּה אֵל. }{All of your works shall praise You, Lord our God, and your pious ones, the righteous ones who do Your will; and all of Your people, the House of Israel will thank and bless in joyful song: and extol and glorify, and exalt and acclaim, and sanctify and coronate Your name, our King. Since, You it is good to thank, and to Your name it is pleasant to sing, since from always and forever are you the Power.}%
\commentb{\textrm{\textbf{יהללוך ה' אלהינו כל מעשיך וחסידיך וחסידים וצדיקים עושי רצונך וגו'.}} זו היא ברכת השיר שהזכירו חז"ל בגמרא19פסחים קי"ח, א', לשיטת רב יהודה שנפסקה להלכה על ידי רב סעדיה גאון והרמב"ם. ועניינה במקום הזה הוא אצלי לשתי כוונות:האחת להעיר על דרוש נכבד וחקירה אודות ההודאות והברכות שאנחנו מברכים ומשבחים אותו יתברך. ונראה שראוי להודות ולהלל להשם יתברך, שאם לא היינו עושים כן היינו כפויי טובה לפניו או בלתי מכירים את גודל ניסיו וחסדיו שעושה עמנו, ואין דרך להכיר גודל החסד האלהי כי אם בתת לו הודאה. ויראה גם כן שההלול הוא דבר נאות לפניו ממה שאמרו חז"ל "לעולם יסדר אדם שבחו של מקום תחילה ואחר כך יתפלל" (ברכות ל"ב, ב') שכיוונו בזה שהשבחים וההודאות יישירו כוונות המתפלל אל מה שראוי לכוון אליו בתפילתו, בציורו שהוא יתברך אל עליון קונה שמים וארץ, והוא משגיח בפרטי בני אדם לתת לאיש כדרכיו וכפרי מעלליו, והוא הכל יכול בעל בלתי תכלית, וכל הדברים בידו בחומר ביד היוצר, וכאשר יחשוב כל זה בלבבו יפיל תחנתו לפני אדון האדונים באמרו היפלא מה' דבר!וכבר נמצאו ראיות מן התורה מן הנביאים ומן הכתובים שההלול הוא דבר נאות וראוי אליו יתברך. מן התורה דכתיב "שִׁירוּ לה' כִּי גָאֹה גָּאָה סוּס וְרֹכְבוֹ רָמָה בַיָּם" (שמות ט"ו, כ"א), רצה לומר אף על פי שהקדוש ברוך הוא גאה גאה ואין אנו יכולין לצייר גדולתו, עם כל זה בדרך שיר והרחבת המאמר אמר שירו לפניו כיון שסוס ורכבו רמה בים. ומן הנביאים שמה שהתרעם הנביא ישעיה שאמרו "תְּכַבְּדֵנִי חַיַּת הַשָּׂדֶה תַּנִּים וּבְנוֹת יַעֲנָה כִּי נָתַתִּי בַמִּדְבָּר מַיִם נְהָרוֹת בִּישִׁימֹן לְהַשְׁקוֹת עַמִּי בְחִירִי, עַם זוּ יָצַרְתִּי לִי תְּהִלָּתִי יְסַפֵּרוּ וְלֹא אֹתִי קָרָאתָ יַעֲקֹב" וגו' (ישעיהו מ"ג כ' – כ"ב). רצה לומר הנה חית השדה עם היות שאין בהן דעת ולא לימוד וכן תנים ובנות יענה עם ארסיותם ורשעתם יכבודני ויהללוני לפי שנתתני במדבר מין ונהרות בארץ ישימון, אף שלא נבראו המים האלה בעבורם כי אם להשקות עמי בחירי, וקל וחומר הוא מה  אם הבעלי חיים הבלתי מדברים האלה יכבדוני על הטובה אשר עשיתי, אף על פי שלא עשיתי אותה לכבודם, כל שכן שעמי יצרתי לי שהוא עם חכם ונבון ויצרתיו מיוחד לעבודתי, שהם ראויים שתהלתי יספרו, ולא עשו כן, עי לא אותי קראת יעקב, רצה לומר שלא נתתם לי תהלה וכבוד. ומן הכתובים מבואר הוא במקומות אין מספר, ובדרך כלל אמר "כֹּל הַנְּשָׁמָה תְּהַלֵּל יָהּ הַלְלוּ-יָהּ" (תהלים ק"נ, ו'). ונראה מכל זה שהשבח וההלול אליו יתברך הוא דבר שראוי לעשותו.אמנם מצד אחר נראה שאין השבח וההלול דבר הגון כביכול שהוא יתברך יתכבד ויושלם עם השבח וההלול אשר יתנו לו, כי הוא יתברך לא ישתלם עם דבר חוץ ממנו בעצמו, ומה יועיל בחוקו השבח וההלול, וכמו שנאמר "אִם צָדַקְתָּ מַה תִּתֶּן לוֹ?" (איוב ל"ה, ז') וכן התוארים שנשבחהו בהם למה הם לו, בין שהם עצמיים או מקריים, לפי שהוא אחד פשוט מכל צד, ואיך נתארהו שאופנים שונים, אש גם עצמותו ומהותו בלתי מושג לזולתו? והוא יתברך אינו נושא מקרים כלל, ואם אין אנו משיגים משלמותו דבר איך נשבחהו? וכבר הפליג המשורר ללמד על זה באמרו " לְךָ דֻמִיָּה תְהִלָּה אֱלֹהִים בְּצִיּוֹן" (תהלים ס"ה, ב') וגו', רצה לומר השתיקה היא התהלה הראויה בחוקך. ולכן ההוא דנחית קמיה דרבי חנינא והיה מרבה בתוארים בשבח האל גער בו השלם ההוא ואמר לו סיימתינהו לכולהי שבחי דמארך? וכו' (ברכות ל"ג, ב') (ברכות פ"א). וכבר הרבה הרב המורה הטענות על זה.אמנם מה שראוי לומר באמיתת הדרוש הזה הוא שהשבח וההלול יש לו שתי בחינות אם לצורך המשבח ואם לצורך המשובח. וידוע שבבחינתו יתברך היה השבח וההלול דבר בטל ובלתי הגון, כי השם יתברך אינו צריך להלול ושבח בני אדם, ולא נוכל לשבחו ולהללו כפי שלמותו. אבל כפי הבחינה השניה השבח וההלול נאה והגון לצורך המתפלל כי מורה על כוונת לבבו ואהבתו את ה' והכרת חסדיו וטובו, ולכן היטיבו אשר דברו שהתוארים שיתואר בהם השם יתברך הם תוארים רק לפעולותיו ומעשיו כי הפעולות האלהיות מושגות אצלנו, ועליהם ראוי לשבחו ולברך אותו, וכאילו מעשיו עצמם המה ישבחוהו ויהללוהו. וכבר בא העניין הזה בתפילת השבת באמרנו "ואילו פינו מלא שירה כים ולשוננו רנה כהמון גליו" וכו' "אין אנו מספיקין להודות לך ה' אלהינו" רצה לומר אף על פי שקבצנו רבוי התהלות כים וגליו וכמרחבי רקיע, ויהיו עינינו מאירות כגרמי השמיימים וכפינו פרושות השמים כנשרים ורגלינו קלות כאיילות למצוות, לא היינו מספיקין להודות לך ה' כפי שלמותך, ואין די בהם לברך את שמך מלכנו על חלק קטן מהחסדים שעשית. וזהו שאמר "על אחת מאלף אלפים" וכו' "נסים ונפלאות שעשית עמנו". ולכן חתם המתפלל דבריו באמרו שאין ההלול והשבח כפי שלמותו ולא תגמול חסדיו, אלא אברי האדם עצמם ישבחוהו להיותם פעולותיו. וזהו אמרו "על כן אברים שפילגת בנו ורוח ונשמה שנפחת באפינו ולשון אשר שמת בפינו הן הם יודו ויברכו" וכו', רצה לומר הם מעצמם ישבחוהו להיותם פעולותו. וכל שכן שראוי זה לצדיקים ולישרים שהם ישראל, והוא אמרו וכתיב רננו צדיקים בה' בפי ישרים תתרומם ובדברי צדיקים תתברך וכו'. וביאר מה הם הצדיקים והקדושים באמרו במקהלות ברבבות עמך בית ישראל, וכמו שאמר עוד שכן חובת כל היצורים לפניך ה' אלהינו להודות להלל לשבח לפאר וכו'.וזה עצמו כיון בברכת השיר באמרו "יהללוך ה' אלהינו על כל מעשיך וחסידיך וצדיקים עושי רצונך", רצה לומר שיהללוהו פעולותיו ויתארוהו במעשיו הנפלאים ויאות זה בפרט לצדיקים ולחסידים שהם עמו בית ישראל כמו שהזכיר. זו היא התכוונה הראשונה בזו הברכה.והכוונה השניה היא שהנביאים ייעדו כולם כי בזמן הגאולה העתידה יכנסו בחברת האלהים ואמתת אמונתו לא לבד האומה הישראלית אלא גם כל שאר האומות, וכמו שניבא ישעיה עליו השלום "וְהָיָה בְּאַחֲרִית הַיָּמִים נָכוֹן יִהְיֶה הַר בֵּית ה' בְּרֹאשׁ הֶהָרִים וְנִשָּׂא מִגְּבָעוֹת וְנָהֲרוּ אֵלָיו כָּל הַגּוֹיִם וְהָלְכוּ עַמִּים רַבִּים וְאָמְרוּ לְכוּ וְנַעֲלֶה אֶל הַר ה' אֶל בֵּית אֱלֹהֵי יַעֲקֹב וְיֹרֵנוּ מִדְּרָכָיו וְנֵלְכָה בְּאֹרְחֹתָיו כִּי מִצִּיּוֹן תֵּצֵא תוֹרָה וּדְבַר ה' מִירוּשָׁלָ‍ִם" (ישעיהו ב', ב'  ג'). וצפניה אמר "כִּי אָז אֶהְפֹּךְ אֶל עַמִּים שָׂפָה בְרוּרָה לִקְרֹא כֻלָּם בְּשֵׁם יְהוָה לְעָבְדוֹ שְׁכֶם אֶחָד" וגו', "מֵעֵבֶר לְנַהֲרֵי כוּשׁ עֲתָרַי בַּת פּוּצַי יוֹבִלוּן מִנְחָתִי" (צפניה ג', ט' – י'). וכמו שדרשו חז"ל "מָה אַשְׁוֶה לָּךְ וַאֲנַחֲמֵךְ" (איוב ב', י"ג), ולכשאשוה לך אנחמך20רבי יעקב דכפר חנן, מתוך תורה תמימה על איכה ב', י"ג. רבי יעקב דכפר חנן היה אמורא בן הדור השלישי, ככל הנראה יליד כפר חנניה, כ- 8 ק"מ דרום מערבית לצפת., והסיבה בזה שכל הקטטות והשנאה אשר לאומות עמנו הם מסיבת חלוף האמונה, כמו שאמרו במסכת שבת "מאי הר סיני הר סימנאי מבעי ליה?" (שבת פ"ט, א') ופירשו שם שעליו ירדה שנאה לאומות העולם עם ישראל, וכן מה שנאמר "אִם תִּמְצְאוּ אֶת דּוֹדִי מַה תַּגִּידוּ לוֹ שֶׁחוֹלַת אַהֲבָה אָנִי" (שיר השירים ה', ח'), רצה לומר  הגידו לו שכל החולאים שאומות העולם מביאין עלי אינם אלא בשביל שאני אוהב אותו, והטעם משום ריחוק האמונות, ולכן יאמר ה' "לכשאשוה לך אנחמך", ועל זה נאמר בתפילת ראש השנה "וייראוך כל המעשים וישתחוו לפניך כל הברואים ויעשו כולם אגודה אחת לעשות רצונך בלבב שלם כמו שידענו ה' אלהינו" וכו', ונאמר בתפילת מוסף "מלוך על כל העולם כולו בכבודך והנשא על כל הארץ ביקרך והופע בהדר גאון עוזך על כל יושבי תבל ארצך וידע כל פעול כי אתה פעלתו ויבין כל יצור כי אתה יצרתו ויאמר כל אשר נשמה באפו ה' אלהי ישראל מלך ומלכותו בכל משלה". ובזה ביארו שבזמן הגאולה העתידה כל הנשמה תהלל יה, וכגדי להעיר על זה נאמר כאן בברכת השיר אחרי שנזכר בהלל כל מה שנזכר מהגאולה העתידה: "יהללוך ה' אלהינו כל מעשיך", רצה לומר שבזמן ההוא יהללו וישבחו אותו כל האומות שהם מעשה ידי ה'  כמו שיורוהו הצדיקים והחסידים עושי רצונו שהם עמו בית ישראל, וכולם יחד יודו ויברכו וישבחו ויפארו לזכר כבוד מלכותו.והנה תקנו בזה ארבע לשונות של הודאה יודו ויברכו וישבחו ויפארו לרמוז על ארבע מדרגות הנמצאות שיהללו את ה': א – מדרגת הרוחניים, ב – מדרגת השמיימים,          ג – מדרגת ישראל, ד – מדרגת האומות. ולפי שכל ברואי מעלה ומטה יעירון וינירון אחדותו ומלכותו, לכן אמר עליהם יודו ויברכו וישבחו ויפארו לשם קדשך הוד והדר זכר למלכותך. ורצו "בשם קדשך" שלמותו ומעלתו כפי מה שהיא שלא ידענו ממנו כי אם השם בלבד, ורמזו ב"זכר מלכותך" הנהגתו את העולם ושמירתו, וזה אמר גם כן "כי לך טוב להודות ולשמך נעים לזמר", לפי שבענין התשועה יכירו וידעו כל יושבי תבל ושוכני ארץ כי לה' המלוכה ומושל בגוים ולו נאה לזמר, וההלול והשבח ההוא הכולל לא יפסק ולא יחדל כל ימי הארץ, והוא אמרו "ומעולם ועד עולם אתה אל". והיתה החתימה "מלך מהולל בתשבחות", לפי שבזמן התשועה יהיה ה' למלך על כל הארץ ויהיה מהולל בתשבחות בפי כל בשר. ולפי זה התבאר שהברכה הזאת מיוחסת לזמן הגאולה העתידה, ולכן נקראת ברכת השיר הכולל שישוררו וישבחו אליו כל באי עולם, ולכן לא נעשה בה זכר ליציאת מצרים, לפי שהיא בערך הגאולה העתידה ויציאת מצרים יוחדה לכוס השלישי כמו שיתבאר להלן.אך למה לא נזכרה הברכה בתחילת ההלל? וכבר כתבו הפוסקים והר"ן בחשודיו לפי שכבר בירך בבית הכנסת "לגמור את ההלל" ולא הוצרך לברך שנית על השולחן, כי היה להם מנהג לקרוא את ההלל בבית הכנסת בליל הפסח. ויתחייב מזה שאם לא בירך בבית הכנסת יצטרך לברך על השולחן לפני ההלל כמו שמברכין לאחריו. אך רב עמרם גאון ובעל הלכות גדולות ורב צמח גאון ורב האי גאון כתבו שאין מברכין עליו לפי שאינו נאמר על השולחן בתודת קריאת ההלל אלא בתודת שיר והודאה לבד, ולכן יפסיק האדם באמצעיתו עם הסעודה ומברך ברכות הרבה וברכת יהללוך היא ברכת הודאה כברכת גשמים ולכן אינה פותחת בברוך. ומפני שיש בדבר הזה דעות חלוקות וההלכה רופפת ילך כל אדם אחרי מנהג עירו. ורבי יוחנן סבר שברכת השיר אומר נשמת כל חי וחותם בישתבח, וכתב הרב אלפסי שהמנהג הוא כרב יהודה, ורשב"ם כתב כיוון דלא איתמר הלכתא לא כמר ולא כמר עבדינן כתרוויהו. ואף כי רב יהודה ור' יוחנן חולקים בברכת השיר מה היא, הנה העניין המכון בשניהם אחד הוא, ולכן תהיה החתימה בין למר ובין למר "מלך מהולל בתשבחות".}%endcomment
\newsection{מזמורי הודיה}
\hebeng{הוֹדוּ לַיי כִּי טוֹב כִּי לְעוֹלָם חַסְדּוֹ. הוֹדוּ לֵאלהֵי הָאֱלהִים כִּי לְעוֹלָם חַסְדּוֹ. הוֹדוּ לָאֲדֹנֵי הָאֲדֹנִים כִּי לְעוֹלָם חַסְדּוֹ. לְעֹשֵׂה נִפְלָאוֹת גְדֹלוֹת לְבַדּוֹ כִּי לְעוֹלָם חַסְדּוֹ. לְעֹשֵׂה הַשָּׁמַיִם בִּתְבוּנָה כִּי לְעוֹלָם חַסְדּוֹ. לְרוֹקַע הָאָרֶץ עַל הַמָּיִם כִּי לְעוֹלָם חַסְדּוֹ. לְעֹשֵׂה אוֹרִים גְּדֹלִים כִּי לְעוֹלָם חַסְדּוֹ. אֶת הַשֶּׁמֶשׁ לְמֶמְשֶׁלֶת בַּיּוֹם כִּי לְעוֹלָם חַסְדּוֹ. אֶת הַיָּרֵחַ וְכוֹכָבִים לְמֶמְשְׁלוֹת בַּלַּיְלָה כִּי לְעוֹלָם חַסְדּוֹ. לְמַכֵּה מִצְרַיִם בִּבְכוֹרֵיהֶם כִּי לְעוֹלָם חַסְדּוֹ. וַיוֹצֵא יִשְׂרָאֵל מִתּוֹכָם כִּי לְעוֹלָם חַסְדּוֹ. בְּיָד חֲזָקָה וּבִזְרוֹעַ נְטוּיָה כִּי לְעוֹלָם חַסְדּוֹ. לְגֹזֵר יַם סוּף לִגְזָרִים כִּי לְעוֹלָם חַסְדּוֹ. וְהֶֶעֱבִיר יִשְׂרָאֵל בְּתוֹכוֹ כִּי לְעוֹלָם חַסְדּוֹ. וְנִעֵר פַּרְעֹה וְחֵילוֹ בְיַם סוּף כִּי לְעוֹלָם חַסְדּוֹ. לְמוֹלִיךְ עַמּוֹ בַּמִּדְבָּר כִּי לְעוֹלָם חַסְדּוֹ. לְמַכֵּה מְלָכִים גְּדֹלִים כִּי לְעוֹלָם חַסְדּוֹ. וַיַּהֲרֹג מְלָכִים אַדִּירִים כִּי לְעוֹלָם חַסְדּוֹ. לְסִיחוֹן מֶלֶךְ הָאֱמֹרִי כִּי לְעוֹלָם חַסְדּוֹ. וּלְעוֹג מֶלֶךְ הַבָּשָׁן כִּי לְעוֹלָם חַסְדּוֹ. וָנָתַן אַרְצָם לְנַחֲלָה כִּי לְעוֹלָם חַסְדּוֹ. נַחֲלָה לְיִשְׂרָאֵל עַבְדוּ כִּי לְעוֹלָם חַסְדּוֹ. שֶׁבְּשִׁפְלֵנוּ זָכַר לָנוּ כִּי לְעוֹלָם חַסְדּוֹ. וַיִפְרְקֵנוּ מִצָּרֵינוּ כִּי לְעוֹלָם חַסְדּוֹ. נֹתֵן לֶחֶם לְכָל בָּשָׂר כִּי לְעוֹלָם חַסְדּוֹ. הוֹדוּ לְאֵל הַשָּׁמַיִם כִּי לְעוֹלָם חַסְדּוֹ.}{Thank the Lord, since He is good, since His kindness is forever. Thank the Power of powers since His kindness is forever. To the Master of masters, since His kindness is forever. To the One who alone does wondrously great deeds, since His kindness is forever. To the one who made the Heavens with discernment, since His kindness is forever. To the One who spread the earth over the waters, since His kindness is forever. To the One who made great lights, since His kindness is forever. The sun to rule in the day, since His kindness is forever. The moon and the stars to rule in the night, since His kindness is forever. To the One that smote Egypt through their firstborn, since His kindness is forever. And He took Israel out from among them, since His kindness is forever. With a strong hand and an outstretched forearm, since His kindness is forever. To the One who cut up the Reed Sea into strips, since His kindness is forever. And He made Israel to pass through it, since His kindness is forever. And He jolted Pharaoh and his troop in the Reed Sea, since His kindness is forever. To the One who led his people in the wilderness, since His kindness is forever. To the One who smote great kings, since His kindness is forever. And he killed mighty kings, since His kindness is forever. Sichon, king of the Amorite, since His kindness is forever. And Og, king of the Bashan, since His kindness is forever. And he gave their land as an inheritance, since His kindness is forever. An inheritance for Israel, His servant, since His kindness is forever. That in our lowliness, He remembered us, since His kindness is forever. And he delivered us from our adversaries, since His kindness is forever. He gives bread to all flesh, since His kindness is forever. Thank the Power of the heavens, since His kindness is forever. (Psalms 136)}
\hebeng{נִשְׁמַת כָּל חַי תְּבַרֵךְ אֶת שִׁמְךָ, ה׳ אֱלֹהֵינוּ, וְרוּחַ כָּל בָּשָׂר תְּפָאֵר וּתְרוֹמֵם זִכְרְךָ, מַלְכֵּנוּ, תָמִיד. מִן הָעוֹלָם וְעַד הָעוֹלָם אַתָּה אֵל, וּמִבַּלְעָדֶיךָ אֵין לָנוּ מֶלֶךְ גּוֹאֵל וּמוֹשִיעַ, פּוֹדֶה וּמַצִּיל וּמְפַרְנֵס וּמְרַחֵם בְּכָל עֵת צָרָה וְצוּקָה. אֵין לָנוּ מֶלֶךְ אֶלָּא אַתָּה. אֱלהֵי הָרִאשׁוֹנִים וְהָאַחֲרוֹנִים, אֱלוֹהַּ כָּל בְּרִיּוֹת, אֲדוׁן כָּל תּוֹלָדוֹת, הַמְּהֻלָּל בְּרֹב הַתִּשְׁבָּחוֹת, הַמְנַהֵג עוֹלָמוֹ בְּחֶסֶד וּבְרִיּוֹתָיו בְּרַחֲמִים. וַיי לֹא יָנוּם וְלא יִישָׁן – הַמְּעוֹרֵר יְשֵׁנִים וְהַמֵּקִיץ נִרְדָּמִים, וְהַמֵּשִׂיחַ אִלְּמִים וְהַמַּתִּיר אֲסוּרִים וְהַסּוֹמֵךְ נוֹפְלִים וְהַזּוֹקֵף כְּפוּפִים. לְךָ לְבַדְּךָ אֲנַחְנוּ מוֹדִים. }{The soul of every living being shall bless Your Name, Lord our God; the spirit of all flesh shall glorify and exalt Your remembrance always, our King. From the world and until the world, You are the Power, and other than You we have no king, redeemer, or savior, restorer, rescuer, provider, and merciful one in every time of distress and anguish; we have no king, besides You! God of the first ones and the last ones, God of all creatures, Master of all Generations, Who is praised through a multitude of praises, Who guides His world with kindness and His creatures with mercy. The Lord neither slumbers nor sleeps. He who rouses the sleepers and awakens the dozers; He who makes the mute speak, and frees the captives, and supports the falling, and straightens the bent. We thank You alone.}
\hebeng{אִלּוּ פִינוּ מָלֵא שִׁירָה כַיָּם, וּלְשׁוֹנֵנוּ רִנָּה כֲּהַמוֹן גַּלָּיו, וְשִׂפְתוֹתֵינוּ שֶׁבַח כְּמֶרְחֲבֵי רָקִיעַ, וְעֵינֵינוּ מְאִירוֹת כַּשֶּׁמֶשׁ וְכַיָּרֵחַ, וְיָדֵינוּ פְרוּשׂות כְּנִשְׂרֵי שָׁמַיִם, וְרַגְלֵינוּ קַלּוֹת כָּאַיָּלוֹת – אֵין אֲנַחְנוּ מַסְפִּיקִים לְהוֹדוֹת לְךָ, ה׳ אֱלהֵינוּ וֵאלֹהֵי אֲבוֹתֵינוּ, וּלְבָרֵךְ אֶת שִׁמְךָ עַל אַחַת מֵאֶלֶף, אַלְפֵי אֲלָפִים וְרִבֵּי רְבָבוֹת פְּעָמִים הַטּוֹבוֹת שֶׁעָשִׂיתָ עִם אֲבוֹתֵינוּ וְעִמָּנוּ. מִמִּצְרַים גְּאַלְתָּנוּ, ה׳ אֱלהֵינוּ, וּמִבֵּית עֲבָדִים פְּדִיתָנוּ, בְּרָעָב זַנְתָּנוּ וּבְשָׂבָע כִּלְכַּלְתָּנוּ, מֵחֶרֶב הִצַּלְתָּנוּ וּמִדֶּבֶר מִלַּטְתָּנוּ, וּמֵחָלָיִם רָעִים וְנֶאֱמָנִים דִּלִּיתָנוּ. }{Were our mouth as full of song as the sea, and our tongue as full of joyous song as its multitude of waves, and our lips as full of praise as the breadth of the heavens, and our eyes as sparkling as the sun and the moon, and our hands as outspread as the eagles of the sky and our feet as swift as deers - we still could not thank You sufficiently, Lord our God and God of our ancestors, and to bless Your Name for one thousandth of the thousand of thousands of thousands, and myriad myriads, of goodnesses that You performed for our ancestors and for us. From Egypt, Lord our God, did you redeem us and from the house of slaves you restored us. In famine You nourished us, and in plenty you sustained us. From the sword you saved us, and from plague you spared us; and from severe and enduring diseases you delivered us. }
\hebeng{עַד הֵנָּה עֲזָרוּנוּ רַחֲמֶיךָ וְלֹא עֲזָבוּנוּ חֲסָדֶיךָ, וְאַל תִּטְּשֵׁנוּ, ה׳ אֱלהֵינוּ, לָנֶצַח. עַל כֵּן אֵבָרִים שֶׁפִּלַּגְתָּ בָּנוּ וְרוּחַ וּנְשָׁמָה שֶׁנָּפַחְתָּ בְּאַפֵּינוּ וְלָשׁוֹן אֲשֶׁר שַׂמְתָּ בְּפִינוּ – הֵן הֵם יוֹדוּ וִיבָרְכוּ וִישַׁבְּחוּ וִיפָאֲרוּ וִירוֹמְמוּ וְיַעֲרִיצוּ וְיַקְדִּישׁוּ וְיַמְלִיכוּ אֶת שִׁמְךָ מַלְכֵּנוּ. כִּי כָל פֶּה לְךָ יוֹדֶה, וְכָל לָשׁוֹן לְךָ תִּשָּׁבַע, וְכָל בֶּרֶךְ לְךָ תִכְרַע, וְכָל קוֹמָה לְפָנֶיךָ תִשְׁתַּחֲוֶה, וְכָל לְבָבוֹת יִירָאוּךָ, וְכָל קֶרֶב וּכְלָיּוֹת יְזַמֵּרוּ לִשְמֶךָ. כַּדָּבָר שֶׁכָּתוּב, כָּל עַצְמֹתַי תֹּאמַרְנָה, ה׳ מִי כָמּוֹךָ מַצִּיל עָנִי מֵחָזָק מִמֶּנוּ וְעָנִי וְאֶבְיוֹן מִגּזְלוֹ. מִי יִדְמֶה לָּךְ וּמִי יִשְׁוֶה לָּךְ וּמִי יַעֲרֹךְ לָךְ הָאֵל הַגָּדוֹל, הַגִּבּוֹר וְהַנּוֹרָא, אֵל עֶלְיוֹן, קנֵה שָׁמַיִם וָאָרֶץ. נְהַלֶּלְךָ וּנְשַׁבֵּחֲךָ וּנְפָאֶרְךָ וּנְבָרֵךְ אֶת שֵׁם קָדְשֶׁךָ, כָּאָמוּר: לְדָוִד, בָּרְכִי נַפְשִׁי אֶת ה׳ וְכָל קְרָבַי אֶת שֵׁם קָדְשׁוֹ. הָאֵל בְּתַעֲצֻמוֹת עֻזֶּךָ, הַגָּדוֹל בִּכְבוֹד שְׁמֶךָ, הַגִּבּוֹר לָנֶצַח וְהַנּוֹרָא בְּנוֹרְאוֹתֶיךָ, הַמֶּלֶךְ הַיּוׁשֵׁב עַל כִּסֵּא רָם וְנִשִֹּא. שׁוֹכֵן עַד מָּרוֹם וְקָּדוֹשׁ שְׁמּוֹ. וְכָתוּב: רַנְּנוּ צַדִּיקִים בַּיי, לַיְשָׁרִים נָאוָה תְהִלָּה. בְּפִי יְשָׁרִים תִּתְהַלָּל, וּבְדִבְרֵי צַדִּיקִים תִּתְבָּרַךְ, וּבִלְשׁוֹן חֲסִידִים תִּתְרוֹמָם, וּבְקֶרֶב קְדושִׁים תִּתְקַדָּשׁ. }{Until now Your mercy has helped us, and Your kindness has not forsaken us; and do not abandon us, Lord our God, forever. Therefore, the limbs that You set within us and the spirit and soul that You breathed into our nostrils, and the tongue that You placed in our mouth - verily, they shall thank and bless and praise and glorify, and exalt and revere, and sanctify and coronate Your name, our King. For every mouth shall offer thanks to You; and every tongue shall swear allegiance to You; and every knee shall bend to You; and every upright one shall prostrate himself before You; all hearts shall fear You; and all innermost feelings and thoughts shall sing praises to Your name, as the matter is written (Psalms 35:10), "All my bones shall say, ‘Lord, who is like You? You save the poor man from one who is stronger than he, the poor and destitute from the one who would rob him.'" Who is similar to You and who is equal to You and who can be compared to You, O great, strong and awesome Power, O highest Power, Creator of the heavens and the earth. We shall praise and extol and glorify and bless Your holy name, as it is stated (Psalms 103:1), " {[A Psalm]} of David. Bless the Lord, O my soul; and all that is within me, His holy name." The Power, in Your powerful boldness; the Great, in the glory of Your name; the Strong One forever; the King who sits on His high and elevated throne. He who dwells always; lofty and holy is His name. And as it is written (Psalms 33:10), "Sing joyfully to the Lord, righteous ones, praise is beautiful from the upright." By the mouth of the upright You shall be praised; By the lips of the righteous shall You be blessed; By the tongue of the devout shall You be exalted; And among the holy shall You be sanctified.}
\hebeng{וּבְמַקְהֲלוֹת רִבְבוֹת עַמְּךָ בֵּית יִשְׂרָאֵל בְּרִנָּה יִתְפָּאֵר שִׁמְךָ, מַלְכֵּנוּ, בְּכָל דּוֹר וָדוֹר, שֶׁכֵּן חוֹבַת כָּל הַיְצוּרִים לְפָנֶיךָ, ה׳ אֱלהֵינוּ וֵאלֹהֵי אֲבוֹתֵינוּ, לְהוֹדוֹת לְהַלֵּל לְשַׁבֵּחַ, לְפָאֵר לְרוֹמֵם לְהַדֵּר לְבָרֵךְ, לְעַלֵּה וּלְקַלֵּס עַל כָּל דִּבְרֵי שִׁירוֹת וְתִשְׁבְּחוֹת דּוִד בֶּן יִשַׁי עַבְדְּךָ מְשִׁיחֶךָ.}{And in the assemblies of the myriads of Your people, the House of Israel, in joyous song will Your name be glorified, our King, in each and every generation; as it is the duty of all creatures, before You, Lord our God, and God of our ancestors, to thank, to praise, to extol, to glorify, to exalt, to lavish, to bless, to raise high and to acclaim - beyond the words of the songs and praises of David, the son of Yishai, Your servant, Your anointed one.}
\hebeng{יִשְׁתַּבַּח שִׁמְךָ לעַד מַלְכֵּנוּ, הָאֵל הַמֶלֶךְ הַגָּדוֹל וְהַקָּדוֹשׁ בַּשָּׁמַיִם וּבָאָרֶץ, כִּי לְךָ נָאֶה, ה׳ אֱלֹהֵינוּ וֵאלֹהֵי אֲבוֹתֵינוּ, שִׁיר וּשְׁבָחָה, הַלֵּל וְזִמְרָה, עֹז וּמֶמְשָׁלָה, נֶצַח, גְּדֻלָּה וּגְבוּרָה, תְּהִלָּה וְתִפְאֶרֶת, קְדֻשָּׁה וּמַלְכוּת, בְּרָכוֹת וְהוֹדָאוֹת מֵעַתָּה וְעַד עוֹלָם. בָּרוּךְ אַתָּה ה׳, אֵל מֶלֶךְ גָּדוֹל בַּתִּשְׁבָּחוֹת, אֵל הַהוֹדָאוֹת, אֲדוֹן הַנִפְלָאוֹת, הַבּוֹחֵר בְּשִׁירֵי זִמְרָה, מֶלֶךְ אֵל חֵי הָעוֹלָמִים.}{May Your name be praised forever, our King, the Power, the Great and holy King - in the heavens and in the earth. Since for You it is pleasant - O Lord our God and God of our ancestors - song and lauding, praise and hymn, boldness and dominion, triumph, greatness and strength, psalm and splendor, holiness and kingship, blessings and thanksgivings, from now and forever. Blessed are You Lord, Power, King exalted through laudings, Power of thanksgivings, Master of Wonders, who chooses the songs of hymn - King, Power of the life of the worlds.}
\newsection{כוס רביעית}
\hebeng{בָּרוּךְ אַתָּה ה׳, אֱלהֵינוּ מֶלֶךְ הָעוֹלָם בּוֹרֵא פְּרִי הַגָּפֶן. }{Blessed are You, Lord our God, King of the universe, who creates the fruit of the vine.}
\hebeng{{\small וְשׁותה בהסיבת שמאל. } }{{\small We drink while reclining to the left} }
\hebeng{בָּרוּך אַתָּה ה׳ אֱלֹהֵינוּ מֶלֶךְ הָעוֹלָם, עַל הַגֶּפֶן וְעַל פְּרִי הַגֶּפֶן, עַל תְּנוּבַת הַשָּׂדֶה וְעַל אֶרֶץ חֶמְדָּה טוֹבָה וּרְחָבָה שֶׁרָצִיתָ וְהִנְחַלְתָּ לַאֲבוֹתֵינוּ לֶאֱכוֹל מִפִּרְיָהּ וְלִשְׂבֹּעַ מִטּוּבָהּ. רַחֶם נָא ה׳ אֱלֹהֵינוּ עַל יִשְׂרָאֵל עַמֶּךָ וְעַל יְרוּשָׁלַיִם עִירֶךָ וְעַל צִיּוֹן מִשְׁכַּן כְּבוֹדֶךָ וְעַל מִזְבְּחֶךָ וְעַל הֵיכָלֶךָ וּבְנֵה יְרוּשָׁלַיִם עִיר הַקֹּדֶשׁ בִּמְהֵרָה בְיָמֵינוּ וְהַעֲלֵנוּ לְתוֹכָהּ וְשַׂמְּחֵנוּ בְּבִנְיָנָהּ וְנֹאכַל מִפִּרְיָהּ וְנִשְׂבַּע מִטּוּבָהּ וּנְבָרֶכְךָ עָלֶיהָ בִּקְדֻשָׁה וּבְטָהֳרָה [{\small בשבת:} וּרְצֵה וְהַחֲלִיצֵנוּ בְּיוֹם הַשַׁבָּת הַזֶּה] וְשַׂמְּחֵנוּ בְּיוֹם חַג הַמַּצּוֹת הַזֶּה, כִּי אַתָּה ה׳ טוֹב וּמֵטִיב לַכֹּל, וְנוֹדֶה לְּךָ עַל הָאָרֶץ וְעַל פְּרִי הַגָּפֶן. }{Blessed are You, Lord our God, King of the universe, for the vine and for the fruit of the vine; and for the bounty of the field; and for a desirable, good and broad land, which You wanted to give to our fathers, to eat from its fruit and to be satiated from its goodness. Please have mercy, Lord our God upon Israel Your people; and upon Jerusalem, Your city: and upon Zion, the dwelling place of Your glory; and upon Your altar; and upon Your sanctuary; and build Jerusalem Your holy city quickly in our days, and bring us up into it and gladden us in its building; and we shall eat from its fruit, and be satiated from its goodness, and bless You in holiness and purity. {[{\small On Shabbat:} And may you be pleased to embolden us on this Shabbat day]} and gladden us on this day of the Festival of Matsot. Since You, Lord, are good and do good to all, we thank You for the land and for the fruit of the vine. }
\hebeng{בָּרוּךְ אַתָּה ה׳, עַל הָאָרֶץ וְעַל פְּרִי הַגָּפֶן.}{Blessed are You, Lord, for the land and for the fruit of the vine}
\newchap{נרצה}
\newsection{חסל סידור פסח}
\hebeng{\textbf{נִרְצָה} }{Accepted}
\hebeng{חֲסַל סִדּוּר פֶּסַח כְּהִלְכָתוֹ, כְּכָל מִשְׁפָּטוֹ וְחֻקָּתוֹ. כַּאֲשֶׁר זָכִינוּ לְסַדֵּר אוֹתוֹ כֵּן נִזְכֶּה לַעֲשׂוֹתוֹ. זָךְ שׁוֹכֵן מְעוֹנָה, קוֹמֵם קְהַל עֲדַת מִי מָנָה. בְּקָרוֹב נַהֵל נִטְעֵי כַנָּה פְּדוּיִם לְצִיּוֹן בְּרִנָּה. }{Completed is the Seder of Pesach according to its law, according to all its judgement and statute. Just as we have merited to arrange it, so too, may we merit to do {[its sacrifice]}. Pure One who dwells in the habitation, raise up the congregation of the community, which whom can count. Bring close, lead the plantings of the sapling, redeemed, to Zion in joy.}%
\commentb{\textrm{\textbf{סדר הדברים בליל שמורים}}אחרי אשר כבר פירשתי מאמרי ההגדה וההלל, ראוי שנזכיר בכאן בקצרה סדר הדברים אשר יעשה אותם האדם, וישראל קרויים אדם, בלילה הזה ליל שימורים לה'. לא מענין הפסח והחגיגה שהיו אבותינו עושים בזמן שבית המקדש קיים, כי עוונותינו ועוונות אבותינו החריבו ביתנו, ואין לנו עתה לא אכילת פסח ולא בשר חגיגה. ולכן אין לנו להתעסק בהלכותיהם וברכותיהם, והרמב"ם הביאם בספר זמנים בהלכות חמץ ומצה פ"ח. אבל נזכור רק הדברים המוטלים עלינו אנשי הגלות לעשותם וזה בדרך קצרה, לפי שחכמי ישראל כבר האריכו בזה בשרשיהם כפי הדין. ויש דעות חלוקות בהרבה מהדברים, ואנחנו גלות ירושלים אשר בספרד נוהגים כדעת הרא"ש ז"ל. ואזכיר ענינו דרך כלל. ואחר כך אעשה כללים בכל מצוות הלילה הזה ואבאר טעמיהם בזה.ליל שימורים לכל בני ישראל לדורותםימזגו לכל אשר בשם ישראל יכונה, אנשים ונשים וטף, כוס של רביעית יין חי שנאמר "חַי חַי הוּא יוֹדֶךָ" (ישעיהו ל"ח, י"ט). ויקדש מיד קודם הנטילה ויברך בורא פרי הגפן  וקדוש היום וברכת שהחיינו, ואינו מברך שעשה לנו ניסים לפי שעתיד לאמרו בהגדה, ואם שבת הוא אומר תחילה ויכולו, ואם חל במוצאי שבת אומר יקנה"ז, יין קידוש נר הבדלה זמן, וישתה רביעית ההין בהסבת שמאל. וכבר נתנו החכמים טעם להסבה שהוא רמז לחירות, עד שאמרו בפרק ערבי פסחים, "יין איתמר משמיה דרב נחמן לא צריך ואיתמר משמה דרב נחמן צריך, ולא פליגי בא בתרי כסי קמאי והא בתרי כסי בתראי, אמרי ליה להאי גיסא ואמרי ליה להאי גיסא, תרי כסי בתראי צריכי הסבה דהוה לן חירות, תרי כסי קמאי לא צריכי הסבה לאכתי לא הוה לן חירות. ואמרי לה להאי גיסא תרי כסי קמאי בעינן דההיא שעתא הויא לן חירות תרי כסי בתראי לא בעי הסבה דמאי דהוה הוה והשתא דאיתמר הכי  ואיתמר הכי כולהו צריכי הסבה", כך כתב הרב אלפסי21הרי"ף על פסחים י"ח, א'.אמנם למה היתה ההסבה על צד שמאל ולא צד ימין ולא על גבו ולא על פניו? פירש רש"י ז"ל כדי שלא יקדים קנה לושת, והרשב"ם כתב מפני שצריך לאכול ולשתות ביד ימין, ואם יהיה פרקדן או על פניו לא יקבל המזון כראוי, ולפי זה אם היה אטר יד ימינו צריך הסבה של ימין.אחר כך מביאין לפניו קערה שיהיו בה שלושה מצות ומרור וחרוסת וחומץ בפני עצמו ושאר ירקות מאיזה מין שירצה, ושני תבשילין אחד זכר פסח ואחד זכר לחגיגה, ונהגו לשים שם במקומם זרוע אחת וביתה אחת מבושלים, ואם חל פסח במוצאי שבת לא ישימו אלא תבשיל אחד ולא את השני שהוא כנגד החגיגה לפי שאז לא היתה חגיגה באה עם הפסח לפי שאינה דוחה את השבת. אמנם הפסקנים כתבו שכיון שאין זה אלא לזכר בעלמא אל ישנה אדם.ונוטל ידיו לצורך טיבול ראשון שכל שטיבולו במשקה צריך נטילה, ומברך על נטילת ידיים\ ולוקח מן הירקות ויברך בורא פרי האדמה ומטבל בחומץ ולא בחרוסת, אף שהרמב"ם כתב שטבולו בחרוסת, אבל שאר הפוסקים אמרו שהחרוסת מצוותו עם המרור ואין לאכול ממנו טרם מצוותו לכן יטביל בחומץ, וכן נהג הרא"ש22רבנו אשר בן יחיאל, המכונה "הרא"ש", היה מגדולי פרשני התלמוד והפוסקים ובעל השפעה מכרעת על עיצוב ההלכה. נולד בקלן בגרמניה בשנת 1250, נפטר בטולדו בספרד בשנת 1327 (ט' בחשון הפ"ח)  ז"ל.אמנם למה הוצרכו לאכול שאר ירקות בהיות מצוות התורה במרור בלבד? כבר העירו עליו חז"ל שהוא כדי שישאלו התינוקות. ונראה לי שלזאת הסיבה עצמה לא הייתה טבילת הירקות בחרוסת, ומפני זה תקנו גם כן נטילת הידיים הראשונה ואכילת הירקות בחומץ כמו שהוא הנוהג בתחלת הסעודה ויראו בציעת המצה לחצאין המורה גם כן על התחלת הסעודה, וכאשר יראו אחרי זה שעקרו הקערה והפסק האכילה ישאלה מה ראו על ככה ומה הגיע אליהם? שהוא התכלית בכל המעשה הזה קודם ההגדה. ואולי מפני זה כתב רבי מאיר מרוטנברג23ר' מאיר בן ר' ברוך מרוטנברג, נודע גם בשם המהר"ם מרוטנברג. חי ופעל בנסיכויות גרמניה במאה ה-13. ר' מאיר מרוטנברג היה מגדולי אשכנז בימי הביניים ומאחרוני בעלי התוספות. נולד בנסיכות וורמס (היא וורמייזא) בשנת 1220 לערך, נפטר בנסיכות שוואביה בשנת 1293 (י"ט באייר הנ"ג). שאין לברך על נטילת ידיים אל טיבול הראשון וכן כתב בעל העיטור, לפי שקם להם דהאידנא אין צריך נטילה לדבר שטיבולו במשקה, וכאילו גלו בזה שהנטילה הצריכה לסעודה היא אחר ההגדה סמוכה לברכת המוציא, אמנם הנטילה הראשונה ומאכל הירקות טבולין בחומץ לא היה מעשה מכוון לעצמו כי אם כדי שטפי עוללים ויונקים יסדו עוז ספור יציאת מצרים. ולכן לא נזכר במשנה, כי אם "הביאו לפנינו מצה ומרור וחזרת וחרוסת" (משנה פסחים י', ג'), אף על פי שאין חרוסת מצווה וכו', ולא הזכירו שאר ירקות. ונהגו להביא במקום הירקות כרפס לפי ששלשה התיבות הראשונות ההפוכות יש בהם פרך ואות ס' הוא רמז לשישים ריבוא  שהיו עובדים בפרך.ויקח מצה האמצעית ויבצענה לשתים, ויתן חצי המצה לשומרה לאפיקומן זכר למשארותם צרורות בשמלתם, וחציה השנית ישים בין שתי השלמות כדי שיפגע בשלמה תחילה כשיבוא לברך ברכת המוציא שממנה יש לבצוע, אף שנראה מדבי הרמב"ם שצריך לברך ברכת המוציא מן הפרוסה משום לחם עוני מה דרכו של עני בפרוסה אף כאן בפרוסה. וענין פריסתה כדי שיאמר  "הא לחמא עניא" על הפרוסה כדרך העניים. וקודם שיברך המוציא ויאכל ממנה כלל יתחיל "הא לחמא עניא" עד "מה נשתנה", ואז צווה להסיר הקערה מן השולחן כאילו כבר אכלו כדי שישאל התינוק למה מסירין הפת ועדיין לא אכלנו. ומוזגין מיד כוס שני כדי שגם כן ישאלו התינוקות למה יביאו כוס אחד קודם הסעודה. ואז מחזירין הקערה לפני בעל הבית ואומר ההגדה ואחריה שני פרקי ההלל עד למעינו מים, ויברך על זה ברכת הגאולה, מהטעם שכתבתי למעלה וישתה כוס השני.והקשה הרי"ף24ר' יצחק בן יעקב אלפסי, נודע בשם הרי"ף, חי ופעל בצפון אפריקה ובספרד במאה ה-11 ולמעשה חתם את תקופת הגאונים. הרי"ף היה מגדולי פוסקי ההלכה וכתביו היוו בסיס בין היתר לשולחן ערוך, יחד עם בעל הטורים הרא"ש. נפטר באליסנה שבספרד בשנת 1103. למה אין מברכין על קריאת ההגדה כמו שמברכין על מקרא מגילה, והלא מצוות עשה היא שנאמר "והגדת לבנך" וגו'? ותירץ כי במה שאמר בקידוש היום זכר ליציאת מצרים יצא. והרשב"א25ר' שלמה בן אברהם אבן אדרת, נודע בשם הרשב"א. היה מגדולי חכמי התורה בספרד בתקופת הגאונים וראש חכמי ספרד בדורו. חי ופעל בברצלונה בין אמצע המאה ה-13 לתחילת המאה ה-14. כתב מפני שהיא מצווה שאין לה קצבה ידועה ואפילו בדיבור בעלמא שידבר מיציאת מצרין יצא ידי חובתו, אלא המרבה הרי זה משובח.ויטול ידיו שנית שהרי הסיח דעתו מהנטילה הראשונה בשעת קראת ההגדה, ויבצע מהמצה השלמה ויברך עליה ברכה המוציא ולא יאכל ממנה עד שיקח מן הפרוסה ויברך עליה על אכילת מצה, ואז יאכל משתיהן יחד כזית מכל אחת מהן בהסבה ואם לא הסב לא יצא. וכתב הרמב"ם שיטבול המצה בחרוסת, ולא הסכימו בזה שאר הפוסקים, דאם משום מצה דלית בה מלח וצריך בשש ליטבל במלחא, וכן הוא לשון הירושלמי. ועוד הקשה בעל המנהיג  שהמצה היא זכר לחירות והחרוסת היא זכר לטיט ואיך יתחברו זה עם זה? לכן הסכימו שאין לטבל את המצה.אחר כך יקח כזית מן המרור וישקענו בחרוסת ולא ישהנו בתוכו כדי שלא יתבטל טעם מרירותו. וראוי שתדע בזה שלשה דברים:ראשונה שהמרור בזמן הזה הוא מדרבנן ולא מדאורייתא , כי מצוותו עם הפסח ורבנן תקנו שאף מבלי הפסח נעשה אותו.והשני שהמרור לא נאכל בהסבה ולכן אמרו בערבי פסחים "איתמר מצה צריכה הסבה מרור אין צריך הסבה" (פסחים ק"ח, א'), והטעם בזה הוא לפי שהמרור הוא רמז לעבדות ומרירות חיי אבותינו, וההסיבה היא זכר לחרות ואיך יהיו שני הפכים בנושא אחד בזמן אחד? ולא תקנו לברך על המרור בורא פרי האדמה לפי שבא אחר ברכת המוציא ובא גם כן מחמת הסעודה והפת פוטרתו.ושלישית שהחרוסת היא מצווה מדרבנן. וכבר פירש הרמב"ם בפירוש המשנה ענין החרוסת והרכבתו, וזה לשונו: "החרוסת הוא תערובת שיש בו קהוי ודמות תבן, וזה זכר לטיט. ואנחנו עושין אותו כך, שורין תאנים או תמרים ומבשלין אותן, ודכין אותן עד שירטבו ולשין הכל בחומץ, ונותנין בו שבולת נרד או איזוב וכיוצא בו בלי שחוקים. ולדעת רבי צדוק הוא מצוה וחייב אדם לברך אשר קדשנו במצותיו וציוונו על אכילת חרוסת ואין הלכה כן עד כאן לשונו26פירוש הרמב"ם על משנה פסחים י', ג'. והרא"ש כתב שאין מברכין על החרוסת אף על פי שהוא מצווה לפי שהוא טפל למרור. והענין אצלי שהוראת החרוסת קרובה להוראת המרור, כי המרור הוא זכר ל"וימררו את חייהם בעבודה קשה בחומר ובלבנים", והחרוסת מיוחדת למרור ולא לדבר אחר וספיקה ברכה אחת לשניהם, ונתיחדה הברכה אל המרור לפי שהוא עיקר המצווה מדאורייתא. והנה הצריכו חז"ל לצוות על החרוסת לדעת ר' יוחנן לפי שהוא זכר לטיט שהוא חומר השעבוד, והוא כמו המרור שהוא זכר למרירות חייהם ועצבון נפשם בעבודתם.ואחר כך נוטל מצה שלישת ובוצע אותה וכורכה עם מרור ואוכל ביחד זכר למקדש כהלל, לפי שהלל היה כורך פסח מצה ומרור ואוכל אותם בבת אחת שנאמר "על מצות ומרורים  יאכלוהו". אמר ר' יוחנן חלוקים עליו חבריו על הלל דתניא יכול לא יצא אדם ידי חובתו אלא אם כן כורכן בבת אחת ואוכלן בדרך שהלל אוכלן? תלמוד לומר על מצות ומרורים יאכלוהו אפילו זה בעצמו. (פסחים קט"ו, א') והשתא דלא איתמר הלכאה לא כהלל ולא כרבנן נעשה המצווה לדעת שניהם, כי ראשונה נאכל מצה בפני עצמה ומרור בפני עצמו ואחר כך נאכל המצה והמרור כרוכים יחד כמו שהיה הלל עושה במקדש כשהיו אוכלים את הפסח. ולפי שראו דעת רבנן יותר מסתבר נעשה המצווה לדעת רבנן בברכה ונעשה המצווה לדעת הלל בלא ברכה. וכתב בעל "אבי העזרי"27ר' אליעזר בן יואל הלוי, נקרא על שם ספרו – "אבי העזרי". ר' יואל היה בן תקופת הראשונים – בן המאה ה-12 – תחילת המאה ה-13. יליד בון שבגרמניה, צאצא למשפחת רבנים ופוסקים מבעלי התוספות. ספרו "אבי העזרי" כולל לקט נרחב של פסקים ופירושים לפי סדר מסכתות התלמוד. שאין לטבל הכריכה בחרוסת, וכן נראה מדברי הרמב"ם, אבל הראש, כתב שצריך לטובלו בחרוסת, וכן כתב רבינו שמעיה בשם רש"י שכל היה הלל עושה, אוכל פסח מצה ומרור וטובל בחרוסת. וענין זה אצלי שהם קיבלו שבלילה הזה מטבילין שתי פעמים ושאין טבול הירקות בחומץ מכללם לפי שנתקנה  בשביל התינוקות בלבד, ולכן סובר הרמב"ם ששני הטבולין הם אחד במצה ואחד במרור, ושאר הפוסקים לא קיבלו שיהא טבול למצה בחרוסת, כי החרוסת היא למרור בלבד, ושני הטבולין הם אחד למרור בפני צמו ואחד למרור עם כריכת המצה.ואחר זה יאכל סעודתו, ואחר הסעודה יאכל מהמצה השמורה תחת המפה והוא האפיקומן הנאכל באחרונה זכר לפסח שהיה נאכל על השובע, ויאכלו בהסיבה ולא יברך עליו, ויהיה זהיר לאוכלו קודם חצות. ומוזגין כוס שלישי ומברך עליו ברכת המזון וישתה בהסיבה. ולא ישתו בינו ובין כוס רביעי, אבל בין כוס שני לכוס שלישי יוכל לשתות. ומוזגין כוס רביעי וגומרין עליו ההלל ומברך עליו ברכת השיר ושותה בהסיבה.ואחר כוס הרביעי יברך בורא פרי הגפן ואחר כך ברכה אחרונה על הגפן ועל פרי הגפן, מה שאין כן בשאר הכוסות, שמברך רק בורא פרי הדפן ולא ברכה אחרונה, זהו דעת רב נטרונאי ורב עמרם ורב אלפסי והרמב"ם, אבל הרא"ש כתב שאין צריך לברך בורא פרי הגפן אלא על כוס ראשון שהוא על קידוש היום ועל כוס שלישי שהוא לברכת המזון, ולא יברך על כוס שני ורביעי בורא פרי הגפן. וגם בברכה האחרונה על הגפן יש דעות שונות, והרא"ש הסכים לדעת האומרים כ אין לאומרה רק אחר כוס הרביעי, כמו שכתבתי.אחר אכילת האפיקומן אין לאכול שום דבר דתנן "אין מפטירן לאחר הפסח אפיקומן" (משנה פסחים י', ח'). אבל אם רוצה לעשות כוס חמישי יש מהפוסקים שכתבו שהוא רשות, ויש מהם שאמרו שאסור לשתות אחר ארבע כוסות, וכן כתב רב אלפס שלא יטעום אחר כוס רביעי כלום. ואף שאין זה מן המשנה דאין מפטירין אחר הפסח א]יקומן היינו שלא לאכול אבל למשתי שריף אבל המנהג אצלנו בספרד הוא שלא לשתות כדי שלא להוסיף על ארבע כוסות.\textrm{\textbf{כלל שתים וארבע}}הכלל היוצא לנו שהמצוות התלויות בלילה הזה הן אלו:\textrm{\textbf{שתים הן נטילת ידיים:}} אחת אחרי קידוש היום קודם אכילת הירקות, והשניה אחרי ההגדה קודם אכילת המצה. ומלבד זה הרגיל במים אחרונים יעשה כמנהגו.\textrm{\textbf{שני מיני ירקות}} יש בקערה: אחד מאיזה ירק (כרפס) שיטבול בחומץ ויברך עליו ברכת בורא פרי האדמה קודם ההגדה, ואחד (חזרת) אחר ההגדה שאוכל עם המרור ומברך על אכילת מרור.\textrm{\textbf{שני תבשילין}} יש בקערה: אחד (זרוע) זכר לפסח שהיה נאכל בזמן בית המקדש, ואחד (ביצה) זכר לבשר חגיגה שהיה נאכל שם, ואין מברכין עליהם.\textrm{\textbf{שתי פעמים יאכלו מרור:}} אחד בפני עצמו בברכתו, ואחד יכרוך עם המצה זכר למקדש כהלל בלא ברכה.\textrm{\textbf{שני דברים צריכים הסיבה:}} אכילת המצה ויין של ארבע כוסות, כי המרור אינו צריך הסיבה כמו שכתבתי למעלה.\textrm{\textbf{ארבע פעמים יאכלו מצת מצווה}}  בליל בסדר: אחת לברכת המוציא ואחת לברכת אכילת מצה, אחת עם המרור ואחת לאפיקומן.\textrm{\textbf{ארבע מיני אכילות}}  יש בקערה: אחת ירק, ואחת מרור בפני עצמו, ואחת מצה בפני עצמה, ואחת מרור כרוך עם המצה זכר למקדש כהלל.\textrm{\textbf{ארבע ברכות}}  יברך על הקערה: בורא פרי האדמה, המוציא לחם מן הארץ, על אכילת מצה, על אכילת מרור.\textrm{\textbf{ארבע כוסות}} הם: אחד לקידוש, אחד לברכת הגאולה, אחד לברכת המזון ואחד לברכת השיר.וראוי שנדע למה תקנו ארבע כוסות בזה המספר לא פחות ולא יותר, וחששו שלא יוסיף על הכוסות, ומה ענין כוס השני על הגאולה? ונראה שזו הברכה היא כברכת היום שנאמר בה אשר בחר בנו וזמן חירותנו וזכרון יציאת מצרים, ולמה לא יספיק בכוס אחד בין שלישי לרביעי? כי אף שאמרו מפני שלא יישן ולא ישתכר ויחדל מלגמור את ההלל, אין ספר שגם כן כוונו בו דבר אחר.והנראה לי בזה מדרך הסברא שבא הכוס הראשון להודות לה' על אשר בחר בנו להיות לו לעם סגולה, והבחירה הזאת היתה בימי אברהם אבינו במצוות המילה, כמו שנאמר "וַהֲקִמֹתִי אֶת בְּרִיתִי בֵּינִי וּבֵינֶךָ וּבֵין זַרְעֲךָ אַחֲרֶיךָ לְדֹרֹתָם לִבְרִית עוֹלָם לִהְיוֹת לְךָ לֵאלֹהִים וּלְזַרְעֲךָ אַחֲרֶיךָ, וְנָתַתִּי לְךָ וּלְזַרְעֲךָ אַחֲרֶיךָ אֵת אֶרֶץ מְגֻרֶיךָ אֵת כָּל אֶרֶץ כְּנַעַן לַאֲחֻזַּת עוֹלָם וְהָיִיתִי לָהֶם לֵאלֹהִים" (בראשית י"ז, ז' – ח'). ולהיות זה החסד הראשון שקיבלנו ממנו יתברך והקודם במעלה וסיבה ליתר הטובות כולם, לכן תקנו עליו הכוס הראשון, ותהיה הברכה "אשר בחר בנו מכל עם" שהיה זה התחלתו בימי אברהם, ומזה נמשך "שרוממנו מכל לשונות האדמה", ונמשך מזה גם כן שקדשנו במצוותיו ונתן לנו חוקים ותורת אמת אשר לא עשה כן לכל גוי. עד שמזה נמשך "שנתן לנו באהבה רבה מועדים לשמחה חגים וזמנים לששון", כי מאהבת ה' אותנו נתן לנו מועדים וחגים לשמוח ולהתעדן בהם, שמכללת חג המצות הזה שהוא מקרא קודש רצה לומר יום טוב אצלנו וזמן חירותנו, כי יצאנו מעבדות לחירות. וכנגד היום טוב מקרא קודש שאמר "באהבה רבה זכר ליציאת מצרים", רצה לומר שאין בזה עול מצות ולא טורח כי אם מצוה אלהית שניתנה מאהבה מסותרת (אהבה טבעית) כדי שנזכור יציאת מצרים. וחזר ואמר "כי בנו בחרת ואותנו קידשת מכל העמים" וחתם "מקדש ישראל והזמנים". הרי לך מבואר מטבע הברכה שהכוס הראשון בא לשבח את ה' על אותה בחירה ראשונה שבחר בנו בימי אברהם אבינו.והכוס השני באה להודות להשם יתברך על גאולת מצרים שאחרי שבחר בנו פדה וגאל אותנו מידי אדונים קשים, ולכן היה מטבע הברכה של הכוס הזה "אשר גאלנו וגאל את אבותינו ממצרים והגיענו הלילה הזה לאכל בו מצה ומרור". ובזמן שבית המקדש קיים היו אומרים "לאכול בו פסח מצה ומרור" ולא אומרים עוד דבר אלא היו חותמים מיד "ברוך אתה ה' גאל ישראל". אך חכמי הגלות הוסיפו בברכה "כן ה' אלהינו יגיענו למועדים ולרגלים אחרים" וכו'.הכוס השלישי בא להודות להק על אשר שמרנו והצילנו בגלות הזה מתוך צרינו ואויבינו, והוא שומר אותנו ומצילנו מכמה זאבים, ונתן לנו  לחם לאכול ובגד ללבוש, וכנגד זה באה ברכת המזון שנודה ומשבח לשמו על אשר לא חסר לנו מזון, ועל הארץ ושאר הטובות שקיבלנו. ומיד נתפלל על גלותנו רחם ה' אלהינו עלינו ועל ישראל עמך ועל ירושלים עירך ועל ציון משכן כבודך ועל הבית הגדול וכו',. ובעוד שאנחנו בגלות הוא רוענו זוננו פרנסנו וכו', ולכן נתפלל לפניו שירויח לנו מכל צרותינו ואל יצריכנו לידי מתנות בשר ודם, שכל זה הוא מחמת הגלות, ונבקש שהוא יתברך יבנה עיר ציון בחיינו וכו', ולפי שכל העניין של כוס שלישי מורה שהוא כנגד בגלות וחורבן בית המקדש, וכוס שני הוא כנגד גאולת מצרים, ונפרדים הענינים זה מזה, לכן אמרו שמותר לשתות כוס אחד בין שני לשלישי.אבל הכוס הרביעי שהוא כנגד הגאולה עתידה וענינו מדובק לענין הגלות הזה, לכן אין להפסיק ביניהם בכוס אחרר, ותקנו לומר על הכוס הרביעי ברכת השיר שאז יושר השיר הזה בארץ יהודה, ונאמר בו "יהללוך ה' אלהינו כל מעשיך", לפי שבגאולה העתידה יהפוך אל עמים שפה ברורה לקרוא כולם בשם ה' כמו שפירשתי. ולפי שההילול והשבח ההווה לא יפסק ולא יחדל, לכן היתה החתימה "מעולם ועד עולם אתה אל ומלך מהולל בתשבחות. ואולי מפני זה יאמרו הפוסקים שאין לברך על הגפן ועל פרי הגפן כי אם על הכוס הרביעי הזה, לפי שבזמן התשועה אז תתברך הגפן ותנובת ארץ ישראל ותהיה ארץ חמדה טובה ורחבה תחת אשר בעת החורבן היתה שממה כמהפכת זרים.ולכן נאמר בברכה הזאת רחם על עמך ועל עירך לפי שהיא מתיחסת לגאולה. וכבר ייעדו הנביאים שבזמן התשועה וקיבוץ גלויות ירבו התבואות והגפן תתן פריה וכל עצי השדה ימחאו כף. ועל זה ניבא הנביא יחזקאל "וְאַתֶּם הָרֵי יִשְׂרָאֵל עַנְפְּכֶם תִּתֵּנוּ וּפֶרְיְכֶם תִּשְׂאוּ לְעַמִּי יִשְׂרָאֵל כִּי קֵרְבוּ לָבוֹא, כִּי הִנְנִי אֲלֵיכֶם וּפָנִיתִי אֲלֵיכֶם וְנֶעֱבַדְתֶּם וְנִזְרַעְתֶּם" (יחזקאל ל"ו, ח' – ט'). והושע הנביא אמר "וְהָיָה בַּיּוֹם הַהוּא אֶעֱנֶה נְאֻם ה' אֶעֱנֶה אֶת הַשָּׁמָיִם וְהֵם יַעֲנוּ אֶת הָאָרֶץ, וְהָאָרֶץ תַּעֲנֶה אֶת הַדָּגָן וְאֶת הַתִּירוֹשׁ וְאֶת הַיִּצְהָר וְהֵם יַעֲנוּ אֶת יִזְרְעֶאל" (הושע ב', כ"ג – כ"ד). וזכריה הנביא אמר "וְעַתָּה לֹא כַיָּמִים הָרִאשֹׁנִים אֲנִי לִשְׁאֵרִית הָעָם הַזֶּה נְאֻם יְהוָה צְבָאוֹת כִּי זֶרַע הַשָּׁלוֹם הַגֶּפֶן תִּתֵּן פִּרְיָהּ וְהָאָרֶץ תִּתֵּן אֶת יְבוּלָהּ וְהַשָּׁמַיִם יִתְּנוּ טַלָּם וְהִנְחַלְתִּי אֶת שְׁאֵרִית הָעָם הַזֶּה אֶת כָּל אֵלֶּה. וְהָיָה כַּאֲשֶׁר הֱיִיתֶם קְלָלָה בַּגּוֹיִם בֵּית יְהוּדָה וּבֵית יִשְׂרָאֵל כֵּן אוֹשִׁיעַ אֶתְכֶם וִהְיִיתֶם בְּרָכָה אַל תִּירָאוּ תֶּחֱזַקְנָה יְדֵיכֶם" (זכריה ח', י' – י"ג).\textrm{\textbf{חתימת המאמר}}כבר כתבו חכמי האמת שהמצוות האלהיות מלבד מה שיש בהם בעשייתם מהעדות והמשפטים והטעמים אשר זכרה התורה בהן, עוד יש בענינן רמזים והערות לענינם עליונים ולמדעים יקרים מטבע המציאות והַיישַרת בני אדן להשגת שלמותם, כי זה לעומת זה עשה האלהים במצוותיו. לכן ראוי לך המעיין אם בעל נפש אתה שתאמר בלבבך למה זה בחר האל יתברך להוציא את עמו ממצרים בחודש האביב ולא באחת משאר תקופות השנה? ועל מה אסר החמץ ועשה באיסורו חומרא רבה שנאמר "כי כל מחמצת לא תאכלו ונכרתה הנפש ההיא" (שמות י"ב) ועל זה ציווה ביום הארבע עשר לבדוק החמץ לאור הנר, וכל הבדיקה והגעלת הכלים שהזהירו חז"ל? ומדוע היה איסור החמץ וימי החג שבעת ימים ולא היה יום אחד כיום מתן תורה? ולמה היום הראשון והיון האחרון אסורין במלאכה מלבד אוכל נפש, ושאר הימים אינם אסורין בה? ולמה צווה בעשיית המצה כל אותה שמירה ונקיות הטחינה הלישה ואפייה  ושיעור המצה? ועל מה יורה המרור והפסח ומדוע לא נאכל הפסח בגבולין כמו המצה כי אם בבית ה'? והחי יתן אל ליבו. ולמה היה ראשו של פסח על כרעיו ועל קרבו ולא נאכל אלא בלילה ואינו נאכל אלא עד חצות, ואינו נאכל אלא למנוייו ואינו נאכל אלא צלי, ואינו נאכל על השובע, ולא חייבו דבר זה המצה? גם עניין ארבע כוסות ושאר הדברים אין ספק שמלבד הטעמים שנתנה התורה בהם ודרשות חז"ל בדבריהם היקרים שהם כולם אמת ואין נפתל ועקש בהם.עוד לאלוה מילין בחכמות עליונות ורמזים נפלאים בדברים האלה, וראוי לנו לחקור עליהם פה משום שלמות המלאכה, ולבקש עליהם דרך ישר אמרי אמת, ובאשר הם סוף כל האדם יהיה זכרונם בסוף המאמר. ואמר ששורש הרבדים האלה כולם ראוי לבעלי השכל השלמים שיבלו ימיהם בטוב ושנותם בנעימים, ויפקידו על זמנן שלא להוציא לבשת ולחרפה אלא להשלמת נפשם, כי כל שנות האדם וחייו ישתדל ללכת בדרך אשר ילכו בו השלמים, וכמו שאמר במשנה אבות, "יפה שעה אחת בתשובה ומעשים טובים בעולם הזה מכל חיי" וכו' (משנה אבות ד', ז'). ואמר רבי מאיר הוי ממעט בעסק ועסוק בתורה" (משנה אבות ד', י'), כלומר שלא יוציא זמנו בדברים בטלים כי אם בהתקרבות אלהים, כמו שנאמר "והגית בו יומם ולילה" (יהושע א', ח'). וכדי שלא ימוש אבדת זמנו מעיניו ראתה החכמה האלהית להעיר את האדם במצוותיו על קוצר ימיו ומספר שנותיו כדי שיהיה זה תמיד לנגד עיניו. ובמאמר רבי טרפון "היום קצר והמלאכה מרובה והפועלים עצלים", ולפקוח עינינו ברא ה' את עולמו במספר שבעת ימים אשר בהם עשה את השמים ואת הארת וביום השביעי שבת וינפש, כדי שהאדם ידע וישכיל שימי שנותיו שבעים שנה, שלא יוציא אותם לריק ולבהלה אלא במעשים נבחרים במעשה בוראם, וביום השביעי ינוח וישבות בעולם הנשמות.וכדי ללמד ולהדריך את האדם באמתת ענינו, כשרצה הקדוש ברוך הוא לזכות את ישראל לקח אותם לעם סגולתו, עשה ביציאת מצרים וצווה אותם על מצוות אשר יעירו את כל אחד על עניין בריאותו וימיו וחייו ויום מותו. כי הנה העם הנבחר עם היותו זרע ברך ה' מפאת טבעם ומולדתם, אך בהיותם במצרים היו שם במצור ובמצוק בחושך ואפלה כימי עולם, דמיון העובר היושב במעי אמו כי עם היות נפשו חצובה מתחת כסא הכבוד הוא שם נעדר השלמות מתבוסס בדמיו, וכמו שעשה על זה השלם רבי שמלאי מאמר גדול החכמה ויקר המליצה מאד באמרו, למה הילד דומה במעי אמו – מקופל ומונח כפנקס ושתי ידיו על צדעיו ושתי אציליו על שתי ארכבותיו, ושתי עקביו על שתי ענבותיו וראשו מונח לו בין ברכיו, פיו סתום וטבורו פתוח, ונר דלוק על ראשו וכו' (מסכת נדה פרק המפלת). וכמו שהולד לא יצא לאויר העולם מבלי שיתחדשו במקום שהוא יושב שם כאבים חזקים וחבלי יולדה, ככה כשהוציא הקדוש ברוך הוא את ישראל ממצרים הכה במצרים מכות רבות, וכמו שהולד ביציאתו לאור העולם לא יורגל מיד בלקיחת המאכלים הגסים כי אם חמאה ודבש יאכל עד הגיעו לגבול התחזקות לאכול לשבעה, וגם יצטרך לחינוך וגידול וללמוד דעת כדי שיתיישר במעשיו לאביו שבשמים, ככה בצאת ישראל ממצרים האכילם ה' את המן ארבעים שנה והביאם במדבר בחינוך ולימוד התורה כדי שאחרי כן ידעו את ה' ויתדבקו בעבודתו.וכבר כתבו הטבעיים שארבעת תקופות השנה הן כנגד הארבעה זמנים ששיערו בחיי האדם: הזמן הראשון שהוא האביב הוא כנגד הילדות והשחרות וימי הנערות, והזמן השני הקיץ הוא כנגד זמן הבחרות שבו יתנגב הלחות וירבה החום, והזמן השלישי החורף הוא כנגד זמן הזקנה עם השארת הכוח, ולכן אף שבוא הבוקר  והערב קרים שעות הצהריים הן חמים, וכן באדם עדיין נשארו בו כחות מהבחרות. והזמן הרביעי הסתו הוא כנגד הישישות וחולת הכוח ואפיסתו, ולכן הוא כולו קר ומוכן למות. ומהבחינה הזו בחר ה' שתהיה היציאה ממצרים בחודש האביב שהוא דוגמת צאתנו לאוויר העולם, וימי התחלתנו להבין ולהשכיל. ולכן היה החדש הזה עלינו ראש חדשים לרמוז על ראשית ימינו ושנותיו.  אמנם איסור החמץ מורה על הרחקת התאוות החומריות, כי החמץ והשאור הוא שמן הוא יצר הרע והוא אשר קראוהו חז"ל "שאור שבעיסה", ולכן ציווה ה' יתברך שבכל הקרבנות לא יקריבו כל שאור וכל דבש, לפי שהשאור רומז ליצר הרע והדבש ותיקות החיים הבהמיים כרוך עמו. ואמר "כי כל אוכל מחמצת ונכרתה הנפש ההיא", לפי שהחמת והיצר הרע הוא משחית ומכרית הנפש ממחצבה. והיה שביתתו ביום ארבעה עשר להעיר שעד ארבע עשרה שנה הראשונות מחיי האדם אינו בר עונשין לא בבית דין של מעלה ולא בבית דין של מטה. אך משם ואילך יקבל עליו עול מצוות וישבית ויבטל היצר הרע והשאור שבעיסה. ולזה כוונו באמרם "אור לארבעה עשר בודקין את החמת לאור הנר" (פסחים א', א'), כי האור הראוי לבדוק בו החמץ אשר בדרכי האדם ובמשכיות לבו הוא הנתון בידו והוא השכל המתעורר בו בסוף שתי שביעיות השנים הראשונים, וכמו שנאמר "נֵר יְהוָה נִשְׁמַת אָדָם חֹפֵשׂ כָּל חַדְרֵי בָטֶן" (משלי כ', ז'). ואמר המליץ ממולח טהר קודש רבי ידעיה הספרדי: "לבי לבי  התורה והאדם נר אלהים בארץ, גוו פתילה נפתלת" וכו'28רבי ידעיה הפניני, מתוך ספרו "בחינת עולם". ר' ידעיה הפניני נולד בפרובאנס בצרפת בשנת 1270 ועבר בצעירותו לברצלונה בספרד. ספרו "בחינת עולם – הוא ספר מוסר בן 37 פרקים הכתוב בלשון מליצית והיה פופולרי במהלך הדורות. הספר יצא לאור בשנת 1484.. וכמה צריך ליזהר בנר הזה בבוא עתו לבדוק בו חורין וסדקין אשר בכל חדרי בטן, שתהא האורה ההיא משוערת כראוי לא חזקה כל כך שתשרוף את הבית ותחריב את הגוף בהרחקת המאכל והמשתה ובדברים הגשמיים בכללותם, ולא חלושה שלא תספיק לגלות סוד החטאים.ובהגעלת הכלים המתחלפים וצורך טהרתם רומז גם כן  על חלוקת כתות בני אדם והכשרם. וראיתי במסכת יומא ששאל רבי מתיא בן חרש לרבי אליעזר ברומי שמעת ארבעה חלוקי כפרה שהיה רבי ישמעאל דורש? אמר ליה שלשה הם ותשובה על כל אחת מהן, עבר על מצוות עשה ועשה תשובה לא זז משם עד שמוחלין לו, ועל זה נאמר שובו בנים שובבים אפר משובותיכם. עבר על מצוות לא תעשה ועשה תשובה, תשובה תולה ויום הכיפורים מכפר ועל זה נאמר כי ביום הזה יכפר עליכם לטהר אתכן. עבר על חייבי כריתות וארבע מיתות בית דין ועשה תשובה, תשובה ויום הכיפורים תולים ויסורין ממרקין, ועל זה נאמר ופקדתי בשבט פשעם ובנגעים עוונם, אבל אם יש בהם חילול השם אין כח בתשובה לתלות ולא ביום הכיפורים לכפר ולא ביסורין למרק, אלא כולם תולין ומיתה ממרקה, ועל זה נאמר ונגלה באזני ה' צבאות אם יכופר העוון הזה לכן עד תמותון" (יומא פ"א, א').והנה התשובה לפי הדעה השלמה היא הטהרה וההגעלה לשביתת החמץ והשאור בגוף, וידוע כי הכלים שנשתמש בהם חמץ יש ד' דינים חלוקים:א – הכלים שהשתמש בהם בצונן שרי להם בשטיפה לבד, וכן יש מבני אדם מי שלא פשע לעשות אחת מתאוות האדם אשר לא תעשינה, אבל לא נזדרז לעשות מצוות עשה המזככות את הנפש, זולת חטא בשב ואל תעשה והחטא הזה היותר קל שבחטאים, ועל זה יהיה קל המחילה עם החרטה והתשובה, והוא אמרו אינו זז משם עד שמוחלין לו שנאמר "שׁוּבוּ בָּנִים שׁוֹבָבִים אֶרְפָּה מְשׁוּבֹתֵיכֶם" (ירמיהו ג', כ"ב).ב – הכלים שנשתמשו בהם בחמין טעונים שפשוף והגעלה, וכן מי שעבר על מצוות העשה אף על פי שאינן חמורות מכל מקום כבר נתלכלכה נפשו לכלוך שאין תקנה בתשובה כמין הראשון אלא עם יום הכפורים.ג – הכלים שהשתמשו בהם על ידי האור שלא יספיק להם עד שיבואו באש ממש, וכן מי שעבר על כריתות וארבע מיתות בית דין שחטא בעבירות חמורות, האיש הזה לא יספיק לנקיונו התשובה ויום הכיפורים אם לא על ידי יסורין.ד – כלי החרס שהשתמשו בהם בחמין בכל דבר איסור לא יקבלו שום תיקום וטעונים שבירה וגניזה, וכן החוטא בחילול השם אין לו תקנה כי אם המוות שהיא מדרגה יותר קשה מכולם. ועל זה אמר הנביא "בְּנֵי צִיּוֹן הַיְקָרִים הַמְסֻלָּאִים בַּפָּז אֵיכָה נֶחְשְׁבוּ לְנִבְלֵי חֶרֶשׂ מַעֲשֵׂה יְדֵי יוֹצֵר" (איכה ד', ב'), רצה לומר שאין להם תקנה אלא בשבירה, ואמר "הַעֶצֶב נִבְזֶה נָפוּץ הָאִישׁ הַזֶּה כָּנְיָהוּ אִם כְּלִי אֵין חֵפֶץ בּוֹ" (ירמיהו כ"ב, כ"ח).הנה התבאר מזה למה נאסרו החמץ והשאור ולמה צריך בדיקה והגעלת הכלים שהשתמשו בהם חמץ. והתחלת האיסור ביום י"ד לרמוז על שנות האדם שמהם התחיל השמירה ממנו. וציווה באיסורו שבעת ימין לפי שזה המספר מן הימים יעיד על ימי שנות האדם שהם שבעים שנה. ולרמוז בזה על שביעיות הרבה במצוות, ובהם ימי החופה שבעת ימים כדי שיזכור כמה סבובים מאלה יעברו עליו בימי חייו ויתן אל ליבו שימי שנותיו שבעים הן, ואחרי החופה יתחייב אדם בשמירת הגדה שבעת ימין ללמדך שכל ימיו יזהר מהטומאה ויבקש הנקיות. אחר זה כשיוליד בן זכר נצטווה כי אחרי עבור עליו שבעת הימין ימול בשר ערלתו, להעיד על זה שימול גם בשר ערלת לבבו ויצרו. ולכן בא בזה המספר שבעה גם כן בטהרת המצורע ובטומאת המת, ושבעת ימי המילואים לכהני ה', והזהיר על שמיטת השנים ויובלים כדי שתמיד יהיה לנגד עיניו מנין שנותיו. ומה טוב אמרו ביובל "וְסָפַרְתָּ לְךָ שֶׁבַע שַׁבְּתֹת שָׁנִים שֶׁבַע שָׁנִים שֶׁבַע פְּעָמִים" וגו' (ויקרא כ"ה, ח') ורצה לומר בספרת \textrm{\textbf{לך}} דווקא, של עצמך אתה מונה השנים, להיותם ימי השמירה, מלבד ימי הנערות שאינו חייב בבית דין של מעלה. ולכן גם בשאר העניינים נזכר תמיד זה המספר. ואמר "כִּי לְיָמִים עוֹד שִׁבְעָה אָנֹכִי מַמְטִיר" וגו' (בראשית ז', ד') "וַיָּחֶל עוֹד שִׁבְעַת יָמִים" (בראשית ח', י'),  "וַיִּמָּלֵא שִׁבְעַת יָמִים אַחֲרֵי הַכּוֹת יְהוָה אֶת הַיְאֹר" (שמות ז', כ"ה), "שִׁבְעַת יָמִים תּוֹחֵל עַד בּוֹאִי אֵלֶיךָ" (שמואל א' י', י'), "בְשִׁבְעָה דְרָכִים יָנוּסוּ לְפָנֶיךָ" (דברים כ"ח, ז'), "שֶׁבַע יִפּוֹל צַדִּיק וָקָם" (משלי כ"ד, ט"ז), "וּבְשֶׁבַע לֹא יִגַּע בְּךָ רָע" (איוב ה', י"ט), "כִּי אֶת שֶׁבַע כְּבָשֹׂת תִּקַּח מִיָּדִי" (בראשית כ"א, ל'), "שֶׁבַע עַל חַטֹּאתֵיכֶם" (ויקרא כ"ו, י"ח), ולכן הם שבע ימי אבלות. ושביעיות רבות אחרות באו בספרי הקודש, וכולם להזכיר האדם את ימי חלדו כדי שישתדל ללכת אל מסעיו בחכמה ובתבונה ובדעת ולא יוציא ימיו לבטחה. ובעבור זה עצמו ימי איסור החמץ שבעה לרמוז שימי שנותיו שבעים ועלינו לבטל כל חמץ וכל שאור בליבנו. ולכן היום הראשון והאחרון אסורים במלאכה לפי שהאדם בילדותו בתחילת בואו לעולם אין לו מעשה ולא מלאכה וגם בישישותו וגבורת ימיו הוא כמת ואינו ראוי למלאכת העולם, אבל בימים ההם יעשה רק אשר יאכל לכל נפש, לפי שלא יאות לנערים בילדותם העמל והמלאכה כי אם לחנכן בתורה ובמצוות שזה הוא מאכל לנפשם, וכן הזקן בסוף שנותיו אין הוא ראוי שיתעסק בדבר אחר כי אם שהכין צידה לדרכו והשלמת נפשו והוא אשר יאכל לכל נפש, אמנם הימים האמצעים הם ימי המעשה.והנה צווה במצה לרמוז אל הנהגת האדם בחייו שתהיה בתמימות וטהרה מבלי יצר הרע ושאור שבעיסה, ולכן הזהירו הקדושים שישמר האדם בעניין המצה משעת הטחינה שהוא הזמן שבו מתחיל האדם לסבוב ולטחון אחרי ענייני העולם הזה כי מאז צריך שימור. והזהירו גם כן על שיעור העיסה שימדוד האדם עיסתו והנהגתו בשימושי גופו וחומרו במדה במשקל ובמשורה. וצוותה התורה שיאכל עם המצה מרור שהוא רמז לכבישת היצר והכנעת בכוחות הגשמיים, כי בזה ירגיש האדם מרירות וצער גשמי, כי אותו המרירות אשר ירגיש גופו בכבישת יצרו ימתק לנפש.אמנם עניין הפסח ירמוז להפסד הגוף והפרדת הנפש ממנו, ולכן יאכל על מצות ומרורים, שכאשר יבוא המוות על ההנהגה הטובה ומרירות כבישות היצר יהיה הפסח קרבן לה' ותהיה שחיטתו אצל מזבח ה', ואכילתו והפסדו במקום קדוש, ויהיה נאכל בלילה שהוא רמז לעת המוות, ועד חצות שהוא חצי המורכב, שחצי האחד הוא מיוחס לנפש ולא ימצא בו דבר גשמי. ולפי שהיה קרבן רמז לגוף האדם במותו והפסדו, לכן היה על כרעיו ועל קרבו דוגמת האדם בעת מותו עם כל אבריו, ויאכל הפסח למנוייו לרמוז אל חברת קרוביו ואוהביו בחייו, ואינו נאכל אלא צלי, להיותו ריחו נודף, רמז שיהיה לו שם טוב ביום מותו, וכמו שנאמר "לְרֵיחַ שְׁמָנֶיךָ טוֹבִים" (שיר השירים א', ג'). וגם שיהיה הקורבן ההוא ריח נחוח לה' ויעל עשנו השמימה, והרוח תשוב אל האלהים אשר נתנה. ולפי שהפסח מורה על המוות והפסד הגוף צווה שתהיה אכילתו באחרונה על השובע ולא יאכלו אחריו כלום, באשר הוא סוף כל האדם ואחרית הדרך. ומפני שחיי האדם נכללים בארבעה זמנין: נערות בחרות זקנה ושיבה כמו שאמרתי לעיל, לכן תקנו לשתות ארבעה כוסות, כי ארבעה אלה ישתה האדם בחייו.הנה נתבאר מכל זה שחג המצות ירמוז אל בריאת האדם וחייו ומספר שנותיו, וטהרת הנהגתו וכבישת יצרו, וצער גופו ושלמות נפשו, וסופו המשובח לפני ה', ולכן היה זה חג ה' לדורות. ואמרו הקדושים חייב אדם להראות את עצמו כאילו הוא יצא ממצרים, לפי שמה שיורה וירמוז עליו היציאה והחג בכל חוקותיו ומשפטיו יכלול לכל אדם בכל דור ודור. ולכן היתה יציאת מצרים שורש לכל המצוות ולשאר מועדי ה', כי הוא למוד והוראה לכל ימי חלדנו. ולכן נצטוינו להמשיך מיד מספר שבעת השביעיות שהם ממש פרקי האדם וגבולי זמנו, כמו שאמר "וּסְפַרְתֶּם לָכֶם מִמָּחֳרַת הַשַּׁבָּת ... שֶׁבַע שַׁבָּתוֹת תְּמִימֹת תִּהְיֶינָה" (ויקרא כ"ג, ט"ו), לפי שהמספר ההוא מורה על ימי חיינו שיהיו כולם להגיענו אל השלמות הרוחני שקיבלנו במתן תורה, והיה יום אחד להורות על אחרות הנותן יתברך והמתנה. והרמז הזה בעצמו מימות האדם ושנותיו בא גם כן בחג הסוכות הנקרא חג האסיף על אסיפת האדם מהעולם הזה, כמו שאמר "בְּאָסְפְּךָ אֶת מַעֲשֶׂיךָ" (שמות  כ"ג, ט"ז), והקפות שבעת ימי החג ושבע הקפות ביום השביעי, הכל בא להזכיר ימי עולם שנות דור ודור, כדי שישים אדם עיניו והחי יתן אל ליבו על ענייניו העצמיים לזכור ימי חיי הבלו ולדקדק עם נפשו בחשבון זמנו וידע כי קרב קיצו, ומיעוט פרי החג מורה עליו גם כן.הנה נתתי בכל המצוות האלה רמז נכבד ואמיתי שענייניו, והוא דרך ישר לפני איש להתהלך לפני ה' בארצות החיים. ופה נשלם מה שרציתי לומר בזה. והתהילה לאל אשר היישיר לפני דרכו וכבודו עלי זרח. והיתה השלמתו בעיר מאנופולי29עיר נמל על שפת הים האדריאטי במחוז פוליה באיטליה. ממחוז הפולייא אשר במלכות נאפוליש, ביום ארבעה עשר ערב חג פסח שנת רנ"ו ליעקב שמחה. אמן ואמן30שנת 1496.תם ונשלם ספר \textrm{\textbf{זבח פסח}} בשנים ועשרים לחדש סיון שנת חמשת אלפים ושלוש מאות וחמישה31שנת 1545, בבית בן משק האדון מארקו אנטוניא יושטיניאן32מרקו אנטוניו יוסטיניאן היה הומניסט איטלקי ובעל בית דפוס עברי בונציה במאותת ה-15 וה-16.. פה ויניציאה.}%endcomment
\newsection{לשנה הבאה}
\hebeng{לְשָׁנָה הַבָּאָה בִּירוּשָלָיִם הַבְּנוּיָה.}{Next year, let us be in the built Jerusalem! }
\newsection{ויהי בחצי הלילה}
\hebeng{{\small בליל רִאשון אומרים:} }{{\small On the first night we say:} }
\hebeng{וּבְכֵן וַיְהִי בַּחֲצִי הַלַּיְלָה.}{And so, it was in the middle of the night. }
\hebeng{אָז רוֹב נִסִּים הִפְלֵאתָ בַּלַּיְלָה, בְּרֹאשׁ אַשְׁמוֹרֶת זֶה הַלַּיְלָה.}{Then, most of the miracles did You wondrously do at night, at the first of the watches this night.}
\hebeng{גֵר צֶדֶק נִצַּחְתּוֹ כְּנֶחֶלַק לוֹ לַיְלָה, וַיְהִי בַּחֲצִי הַלַּיְלָה.}{A righteous convert did you make victorious when it was divided for him at night {[referring to Avraham in his war against the four kings - Genesis 14:15]}, and it was in the middle of the night. }
\hebeng{דַּנְתָּ מֶלֶךְ גְּרָר בַּחֲלוֹם הַלַּיְלָה, הִפְחַדְתָּ אֲרַמִּי בְּאֶמֶשׁ לַיְלָה.}{You judged the king of Gerrar {[Avimelekh]} in a dream of the night; you frightened an Aramean {[Lavan]} in the dark of the night; }
\hebeng{וַיָּשַׂר יִשְׂרָאֵל לְמַלְאָךְ וַיּוּכַל לוֹ לַיְלָה, וַיְהִי בַּחֲצִי הַלַּיְלָה. }{and Yisrael dominated an angel and was able to withstand Him at night {[Genesis 32:25-30]}, and it was in the middle of the night. }
\hebeng{זֶרַע בְּכוֹרֵי פַתְרוֹס מָחַצְתָּ בַּחֲצִי הַלַּיְלָה, חֵילָם לֹא מָצְאוּ בְּקוּמָם בַּלַּיְלָה, טִיסַת נְגִיד חֲרֹשֶׁת סִלִּיתָ בְּכוֹכְבֵי לַיְלָה, וַיְהִי בַּחֲצִי הַלַּיְלָה. }{You crushed the firstborn of Patros {[Pharaoh, as per Ezekiel 30:14]} in the middle of the night, their wealth they did not find when they got up at night; the attack of the leader Charoshet {[Sisera]} did you sweep away by the stars of the night {[Judges 5:20]}, and it was in the middle of the night. }
\hebeng{יָעַץ מְחָרֵף לְנוֹפֵף אִוּוּי, הוֹבַשְׁתָּ פְגָרָיו בַּלַּיְלָה, כָּרַע בֵּל וּמַצָּבוֹ בְּאִישׁוֹן לַיְלָה, לְאִישׁ חֲמוּדוֹת נִגְלָה רָז חֲזוֹת לַיְלָה, וַיְהִי בַּחֲצִי הַלַּיְלָה. }{The blasphemer {[Sancheriv whose servants blasphemed when trying to discourage the inhabitants of Jerusalem]} counseled to wave off the desired ones, You made him wear his corpses on his head at night {[II Kings 19:35]}; Bel and his pedestal were bent in the pitch of night {[in Nevuchadnezar's dream in Daniel 2]}; to the man of delight {[Daniel]} was revealed the secret visions at night, and it was in the middle of the night. }
\hebeng{מִשְׁתַּכֵּר בִּכְלֵי קֹדֶשׁ נֶהֱרַג בּוֹ בַלַּיְלָה, נוֹשַׁע מִבּוֹר אֲרָיוֹת פּוֹתֵר בִּעֲתוּתֵי לַיְלָה, שִׂנְאָה נָטַר אֲגָגִי וְכָתַב סְפָרִים בַּלַּיְלָה, וַיְהִי בַּחֲצִי הַלַּיְלָה. }{The one who got drunk {[Balshatsar]} from the holy vessels was killed on that night {[Daniel 5:30]}, the one saved from the pit of lions {[Daniel]} interpreted the scary visions of the night; hatred was preserved by the Agagite {[Haman]} and he wrote books at night, and it was in the middle of the night. }
\hebeng{עוֹרַרְתָּ נִצְחֲךָ עָלָיו בְּנֶדֶד שְׁנַת לַיְלָה. פּוּרָה תִדְרוֹךְ לְשׁוֹמֵר מַה מִּלַיְלָה, צָרַח כַּשּׁוֹמֵר וְשָׂח אָתָא בֹקֶר וְגַם לַיְלָה, וַיְהִי בַּחֲצִי הַלַּיְלָה. }{You aroused your victory upon him by disturbing the sleep of night {[of Achashverosh]}, You will stomp the wine press for the one who guards from anything at night {[Esav/Seir as per Isaiah 21:11]}; He yelled like a guard and spoke, "the morning has come and also the night," and it was in the middle of the night. }
\hebeng{קָרֵב יוֹם אֲשֶׁר הוּא לֹא יוֹם וְלֹא לַיְלָה, רָם הוֹדַע כִּי לְךָ הַיּוֹם אַף לְךָ הַלַּיְלָה, שׁוֹמְרִים הַפְקֵד לְעִירְךָ כָּל הַיּוֹם וְכָל הַלַּיְלָה, תָּאִיר כְּאוֹר יוֹם חֶשְׁכַּת לַיְלָה, וַיְהִי בַּחֲצִי הַלַּיְלָה.}{Bring close the day which is not day and not night {[referring to the end of days - Zechariah 14:7]}, High One, make known that Yours is the day and also Yours is the night, guards appoint for Your city all the day and all the night, illuminate like the light of the day, the darkness of the night, and it was in the middle of the night.}
\newsection{זבח פסח}
\hebeng{{\small בְליל שני בחו״ל:} וּבְכֵן וַאֲמַרְתֶּם זֶבַח פֶּסַח.}{{\small On the second night, outside of Israel:} And so "And you shall say, 'it is the Pesach sacrifice'"(Exodus 12:27). }
\hebeng{אֹמֶץ גְּבוּרוֹתֶיךָ הִפְלֵאתָ בַּפֶּסַח, בְּרֹאשׁ כָּל מוֹעֲדוֹת נִשֵּׂאתָ פֶּסַח. גִּלִיתָ לְאֶזְרָחִי חֲצוֹת לֵיל פֶּסַח, וַאֲמַרְתֶּם זֶבַח פֶּסַח.}{The boldness of Your strong deeds did you wondrously show at Pesach; at the head of all the holidays did You raise Pesach; You revealed to the Ezrachite {[Avraham]}, midnight of the night of Pesach. "And you shall say, 'it is the Pesach sacrifice.'" }
\hebeng{דְּלָתָיו דָּפַקְתָּ כְּחֹם הַיּוֹם בַּפֶּסַח, הִסְעִיד נוֹצְצִים עֻגּוֹת מַצּוֹת בַּפֶּסַח, וְאֵל הַבָּקָר רָץ זֵכֶר לְשׁוֹר עֵרֶךְ פֶּסַח, וַאֲמַרְתֶּם זֶבַח פֶּסַח.}{Upon his doors did You knock at the heat of the day on Pesach {[Genesis 18:1]}; he sustained shining ones {[angels]} with cakes of matsa on Pesach; and to the cattle he ran, in commemoration of the bull that was set up for Pesach. "And you shall say, 'it is the Pesach sacrifice.'" }
\hebeng{זוֹעֲמוּ סְדוֹמִים וְלוֹׁהֲטוּ בָּאֵשׁ בַּפֶּסַח, חֻלַּץ לוֹט מֵהֶם וּמַצּוֹת אָפָה בְּקֵץ פֶּסַח, טִאטֵאתָ אַדְמַת מוֹף וְנוֹף בְּעָבְרְךָ בַּפֶּסַח. וַאֲמַרְתֶּם זֶבַח פֶּסַח.}{The Sodomites caused Him indignation and He set them on fire on Pesach; Lot was rescued from them and matsot did he bake at the end of Pesach; He swept the land of Mof and Nof {[cities in Egypt]} on Pesach. "And you shall say, 'it is the Pesach sacrifice.'"  }
\hebeng{יָהּ רֹאשׁ כָּל הוֹן מָחַצְתָּ בְּלֵיל שִׁמּוּר פֶּסַח, כַּבִּיר, עַל בֵּן בְּכוֹר פָּסַחְתָּ בְּדַם פֶּסַח, לְבִלְתִּי תֵּת מַשְׁחִית לָבֹא בִּפְתָחַי בַּפֶּסַח, וַאֲמַרְתֶּם זֶבַח פֶּסַח.}{The head of every firstborn did You crush on the guarded night of Pesach; Powerful One, over the firstborn son did You pass over with the blood on Pesach; so as to not let the destroyer come into my gates on Pesach. "And you shall say, 'it is the Pesach sacrifice.'" }
\hebeng{מְסֻגֶּרֶת סֻגָּרָה בְּעִתּוֹתֵי פֶּסַח, נִשְׁמְדָה מִדְיָן בִּצְלִיל שְׂעוֹרֵי עֹמֶר פֶּסַח, שׂוֹרָפוּ מִשְׁמַנֵּי פּוּל וְלוּד בִּיקַד יְקוֹד פֶּסַח, וַאֲמַרְתֶּם זֶבַח פֶּסַח.}{The enclosed one {[Jericho]} was enclosed in the season of Pesach; Midian was destroyed with a portion of the \textit{omer}-barley on Pesach {[via Gideon as per Judges 7]}; from the fat of Pul and Lud {[Assyrian soldiers of Sancheriv]} was burnt in pyres on Pesach. "And you shall say, 'it is the Pesach sacrifice'" }
\hebeng{עוֹד הַיּוֹם בְּנֹב לַעֲמוֹׁד עַד גָּעָה עוֹנַת פֶּסַח, פַּס יַד כָּתְבָה לְקַעֲקֵעַ צוּל בַּפֶּסַח, צָפֹה הַצָּפִית עֲרוֹךְ הַשֻּׁלְחָן בַּפֶּסַח, וַאֲמַרְתֶּם זֶבַח פֶּסַח. }{Still today {[Sancheriv will go no further than]} to stand in Nov {[Isaiah 10:32]}, until he cried at the time of Pesach; a palm of the hand wrote {[Daniel 5:5]} to rip up the deep one {[ the Bayblonian one - Balshatsar]} on Pesach; set up the watch, set the table {[referring to Balshatsar, based on Psalms 21:5]} on Pesach. "And you shall say, 'it is the Pesach sacrifice'"}
\hebeng{קָהָל כִּנְּסָה הֲדַּסָּה לְשַׁלֵּשׁ צוֹם בַּפֶּסַח, רֹאשׁ מִבֵּית רָשָׁע מָחַצְתָּ בְּעֵץ חֲמִשִּׁים בַּפֶּסַח, שְׁתֵּי אֵלֶּה רֶגַע תָּבִיא לְעוּצִית בַּפֶּסַח, תָּעֹז יָדְךָ תָּרוּם יְמִינְךָ כְּלֵיל הִתְקַדֵּשׁ חַג פֶּסַח, וַאֲמַרְתֶּם זֶבַח פֶּסַח. }{The congregation did Hadassah {[Esther]} bring in to triple a fast on Pesach; the head of the house of evil {[Haman]} did you crush on a tree of fifty {[\textit{amot}]} on Pesach; these two {[plagues as per Isaiah 47:9]} will you bring in an instant to the Utsi {[Esav]} on Pesach; embolden Your hand, raise Your right hand, as on the night You were sanctified on the festival of Pesach. "And you shall say, 'it is the Pesach sacrifice'" }
\newsection{אדיר במלוכה}
\hebeng{כִּי לוֹ נָאֶה, כִּי לוֹ יָאֶה.}{Since for Him it is pleasant, for Him it is suited. }
\hebeng{אַדִּיר בִּמְלוּכָה, בָּחוּר כַּהֲלָכָה, גְּדוּדָיו יֹאמְרוּ לוֹ: לְךָ וּלְךָ, לְךָ כִּי לְךָ, לְךָ אַף לְךָ, לְךָ ה׳ הַמַּמְלָכָה, כִּי לוֹ נָאֵה, כִּי לוֹ יָאֶה.}{Mighty in rulership, properly chosen, his troops shall say to Him, "Yours and Yours, Yours since it is Yours, Yours and even Yours, Yours, Lord is the kingdom; since for Him it is pleasant, for Him it is suited." }%
\commenta{\textrm{\textbf{בפזמון אדיר במלוכה. גדודיו יאמרו לו לך ולך, לך כי לך, לך אף לך, לך ה׳ הממלכה.}} השם גדודיו כנוי לצבא השמים, ובלשון הכתוב באיוב (כ״ה ג׳) היש מספר לגדודיו (וע׳ בסמוך). ומאמרם המליצי לך ולך וכו׳ נוסד על לשון הפסוק בדהי״א (כ״ט י״א) לך ה' הגדולה והגבורה והנצח וההוד כי כל בשמים ובארץ לך ה׳ הממלכה, ועוד על פסוק בתהלים (פ״ט י״ב) לך שמים אף לך ארץ, ומשניהם יחד יסודר לשון מליצי זה בסגנון זה: לך ולך — כנגד הלשון לך ה׳ הגדולה והגבורה וכו׳, ויתפרש הלשון לך ה׳ הגדולה ולך ה׳ הגבורה וכו׳ *וגם אפשר לומר, כי בלשון לך לך רומז ללשון הפסוק בתהלים (ס״ח ב) לך אלהים דומיה תהלה ולך ישולם גדר (מהנודרים בעת צרה ואתה תושיעם).) לך כי לך — כנגד הלשון כי כל בשמים ובארץ לך הוא. לך אף לך — כנגד הלשון לך השמים אף לך ארץ, ומסיים בלשון הפסוק בדה״י שם, לך ה׳ הממלכה, מהו מענין שירת המלאכים. והפיטן סוגר מליצה זו בלשון כי לו נאה ולו יאה על שם הלשון כי לך נאה ה׳ אלהינו שיר ושבחה, ועל שם הפסוק בירמיה (י׳) מי לא יראך מלך הגוים כי לך יאתה. והלשון גדודיו יאמרו לו בא בסמוך בלשון סביביו יאמרו לו, ובאמת גדודיו וסביביו ענין אחד הוא, אלה אשר קרובים לו ועומדים במחיצתו, וכן יתר התוארים כאן, ותיקין, למודיו, טפסריו וכו׳. והריב״א בזה, כי במלך בשר ודם פחד המלך יותר על הרחוקים ממנו, מאשר הקרובים אליו, יען כי מטבע הענין, שהקירוב הרגילי בתמידות כמו מחליש באיזה ערך את הפחד, ואמר בזה, כי הקב״ה הוא להיפך, שדוקא גדודיו וסביביו ושארי בעלי התואר הסמוכים לו — דוקא הם מוטלים תחת אימתו עד שברגש יאמרו לו לך ולך וכו׳ ככל הנוסח, כפי שנתבאר.}%endcomment
\hebeng{דָּגוּל בִּמְלוּכָה, הָדוּר כַּהֲלָכָה, וָתִיקָיו יֹאמְרוּ לוֹ: לְךָ וּלְךָ, לְךָ כִּי לְךָ, לְךָ אַף לְךָ, לְךָ ה׳ הַמַּמְלָכָה, כִּי לוֹ נָאֵה, כִּי לוֹ יָאֶה.}{Noted in rulership, properly splendid, His distinguished ones will say to him, "Yours and Yours, Yours since it is Yours, Yours and even Yours, Yours, Lord is the kingdom; since for Him it is pleasant, for Him it is suited." }
\hebeng{זַכַּאי בִּמְלוּכָה, חָסִין כַּהֲלָכָה טַפְסְרָיו יֹאמְרוּ לוֹ: לְךָ וּלְךָ, לְךָ כִּי לְךָ, לְךָ אַף לְךָ, לְךָ ה׳ הַמַּמְלָכָה, כִּי לוֹ נָאֵה, כִּי לוֹ יָאֶה.}{Meritorious in rulership, properly robust, His scribes shall say to him, "Yours and Yours, Yours since it is Yours, Yours and even Yours, Yours, Lord is the kingdom; since for Him it is pleasant, for Him it is suited." }
\hebeng{יָחִיד בִּמְלוּכָה, כַּבִּיר כַּהֲלָכָה לִמּוּדָיו יֹאמְרוּ לוֹ: לְךָ וּלְךָ, לְךָ כִּי לְךָ, לְךָ אַף לְךָ, לְךָ ה׳ הַמַּמְלָכָה, כִּי לוֹ נָאֶה, כִּי לוֹ יָאֶה.}{Unique in rulership, properly powerful, His wise ones say to Him, "Yours and Yours, Yours since it is Yours, Yours and even Yours, Yours, Lord is the kingdom; since for Him it is pleasant, for Him it is suited." }
\hebeng{מוֹשֵׁל בִּמְלוּכָה, נוֹרָא כַּהֲלָכָה סְבִיבָיו יֹאמְרוּ לוֹ: לְךָ וּלְךָ, לְךָ כִּי לְךָ, לְךָ אַף לְךָ, לְךָ ה׳ הַמַּמְלָכָה, כִּי לוֹ נָאֵה, כִּי לוֹ יָאֶה.}{Reigning in rulership, properly awesome, those around Him say to Him, "Yours and Yours, Yours since it is Yours, Yours and even Yours, Yours, Lord is the kingdom; since for Him it is pleasant, for Him it is suited." }
\hebeng{עָנָיו בִּמְלוּכָה, פּוֹדֶה כַּהֲלָכָה, צַדִּיקָיו יֹאמְרוּ לוֹ: לְךָ וּלְךָ, לְךָ כִּי לְךָ, לְךָ אַף לְךָ, לְךָ ה׳ הַמַּמְלָכָה, כִּי לוֹ נָאֵה, כִּי לוֹ יָאֶה.}{Humble in rulership, properly restoring, His righteous ones say to Him, "Yours and Yours, Yours since it is Yours, Yours and even Yours, Yours, Lord is the kingdom; since for Him it is pleasant, for Him it is suited." }
\hebeng{קָּדּוֹשׁ בִּמְלוּכָה, רַחוּם כַּהֲלָכָה שִׁנְאַנָּיו יֹאמְרוּ לוֹ: לְךָ וּלְךָ, לְךָ כִּי לְךָ, לְךָ אַף לְךָ, לְךָ ה׳ הַמַּמְלָכָה, כִּי לוֹ נָאֵה, כִּי לוֹ יָאֶה.}{Holy in rulership, properly merciful, His angels say to Him, "Yours and Yours, Yours since it is Yours, Yours and even Yours, Yours, Lord is the kingdom; since for Him it is pleasant, for Him it is suited." }
\hebeng{תַּקִיף בִּמְלוּכָה, תּוֹמֵךְ כַּהֲלָכָה תְּמִימָיו יֹאמְרוּ לוֹ: לְךָ וּלְךָ, לְךָ כִּי לְךָ, לְךָ אַף לְךָ, לְךָ ה׳ הַמַּמְלָכָה, כִּי לוֹ נָאֵה, כִּי לוֹ יָאֶה. }{Dynamic in rulership, properly supportive, His innocent ones say to Him, "Yours and Yours, Yours since it is Yours, Yours and even Yours, Yours, Lord is the kingdom; since for Him it is pleasant, for Him it is suited."}
\newsection{אדיר הוא}
\hebeng{אַדִּיר הוּא יִבְנֶה בֵּיתוֹ בְּקָרוֹב. בִּמְהֵרָה, בִּמְהֵרָה, בְּיָמֵינוּ בְּקָרוֹב. אֵל בְּנֵה, אֵל בְּנֵה, בְּנֵה בֵּיתְךָ בְּקָרוֹב.}{Mighty is He, may He build His house soon. Quickly, quickly, in our days, soon. God build, God build, build Your house soon. }%
\commenta{\textrm{\textbf{בפזמון אדיר הוא. אדיר הוא יבנה ביתו בקרוב במהרה במהרה בימינו בקרוב.}} כפילת הלשון במהרה ועוד תוספתו ״בימינו בקרוב״, כפי הנראה באו לאיזו כונה, יען דאם לא כן, למה כל הציונים האלה, בקרוב, במהרה במהרה, בימינו, בקרוב, ואין זה סגנון רגיל. וגם ידוע, כי כפילת הלשון בא לרמז על תוקף הדבר, שנעשה ברגש ובזירוז, כמו הלך הלכת, נכסף נכספת (פ׳ ויצא), עזוב תעזוב (פ׳ משפטים), הוכח תוכיח (פ׳ קדושים), נתן תתן, פתוח תפתח, הענק תעניק (פ׳ ראה), הקם תקים (פ׳ תצא), והרבה כהנה, כולם מורים על תוקף הדבר באיזה ערך וענין, אם בתכונת הדבר או בזמנו וערכו, (ומבוארים אצלנו בתו״ת). ולפי זה גם כאן, כפי הנראה, באו לשונות הכפולים שזכרנו לרמז איזו הוראה. והנה כבר כתבנו בנוסח הקדיש שלפני ברכו, בלשון ״בעגלא ובזמן קריב״, ובארנו, דלשון בעגלא הוראתו ארמית כמו בעברית ״מהרה״, וכן מפורש בתלמוד ברכות (י״ח סע״ב) פסחים (ע״ה סע״א), ובסנהדרין (נ״ב סע״א), עיי״ש בכל המקומות ברש״י שפירש שהוא לשון מהירות,*ובזה יתבאר ברש״י בהוריות (י״ב א׳) שכתב על איזה מיני גידולי קרקע ״שהם גדלי לעגל טפי משאר ירקות״ עכ״ל, וכונת דבריו שממהרים לגדל.) ופירשנו, דכפל הלשון במהרה (הוא בעגלא) ועוד לזה הלשון ״ובזמן קריב״ לא באו לחנם ובלא כונה במקרה לבד, אך מורים על איזה דבר. ובארנו, דכל אלה הלשונות באים לרמז על מה שאמרו במס׳ גיטין (פ״ח רע״ב) כי ״מהרה״ דמרי עלמא, (מה שהקב״ה כביכול מבטא שם מהרה) שיעורו תמניא מאה וחמשין ותרתין שנים (עיי״ש החשבון), וכלומר, דגם מספר שנים כזה יחשב ל״מהרה״ — ולכן — כך כתבנו שם — כדי להוציא משיעור זה מוסיפים עוד הלשון ״בזמן קריב״ והוא יפרש הלשון מהרה כפי הוראתו אצלנו, מהרה ממש. והנה גם כאן אפשר לומר, שבא כפילת הלשון במהרה והציונים ״בימינו מקרוב״ להוציא משיעור דמרי עלמא (תתנ״ב שנה), אך כפי משמעותו אצל בני אדם מהרה ממש, וכהוראת תוספת הלשון ״בימינו״.}%endcomment
\hebeng{בָּחוּר הוּא, גָּדוֹל הוּא, דָּגוּל הוּא יִבְנֶה בֵּיתוֹ בְּקָרוֹב. בִּמְהֵרָה, בִּמְהֵרָה, בְּיָמֵינוּ בְּקָרוֹב. אֵל בְּנֵה, אֵל בְּנֵה, בְּנֵה בֵּיתְךָ בְּקָרוֹב.}{Chosen is He, great is He, noted is He. Quickly, quickly, in our days, soon. God build, God build, build Your house soon. }
\hebeng{הָדוּר הוּא, וָתִיק הוּא, זַכַּאי הוּא יִבְנֶה בֵּיתוֹ בְּקָרוֹב. בִּמְהֵרָה, בִּמְהֵרָה, בְּיָמֵינוּ בְּקָרוֹב. אֵל בְּנֵה, אֵל בְּנֵה, בְּנֵה בֵּיתְךָ בְּקָרוֹב.}{Splendid is He, distinguished is He, meritorious is He. Quickly, quickly, in our days, soon. God build, God build, build Your house soon. }
\hebeng{חָסִיד הוּא, טָהוֹר הוּא, יָחִיד הוּא יִבְנֶה בֵּיתוֹ בְּקָרוֹב. בִּמְהֵרָה, בִּמְהֵרָה, בְּיָמֵינוּ בְּקָרוֹב. אֵל בְּנֵה, אֵל בְּנֵה, בְּנֵה בֵּיתְךָ בְּקָרוֹב.}{Pious is He, pure is He, unique is He. Quickly, quickly, in our days, soon. God build, God build, build Your house soon. }
\hebeng{כַּבִּיר הוּא, לָמוּד הוּא, מֶלֶךְ הוּא יִבְנֶה בֵּיתוֹ בְּקָרוֹב. בִּמְהֵרָה, בִּמְהֵרָה, בְּיָמֵינוּ בְּקָרוֹב. אֵל בְּנֵה, אֵל בְּנֵה, בְּנֵה בֵּיתְךָ בְּקָרוֹב.}{Powerful is He, wise is He, A king is He. Quickly, quickly, in our days, soon. God build, God build, build Your house soon. }%
\commenta{\textrm{\textbf{למוד הוא.}} יש מפרשים ממדגישי הלשון, שלא רוחם לתואר זה (למוד הוא) כלפי ה', כי לשון זה מורה שהשיג כביכול את עניניו ע״י למוד. אבל באמת אין כל חשש בזה' יען כי מצינו כמה וכמה שמות משם תואר ובאים במשקל פעול, כמו בישעיהו (כ״ו ג׳) כי בך בטוח — תחת בוטח. ובירמיה (ט׳ ג׳) חץ שחוט לשונם — תחת שוחט (מוסב על כח האסון שמביא בעל לשה״ר) ומבואר זה במס׳ ערכין (ט״ז ב׳), ועוד בירמיה (כ״ב ג׳) והצילו גזול מיד עשוק — תחת מיד עושק, ובהושע (י״ג ה׳) כדוב שכול תחת שוכל, ובתלמוד ב״מ (י׳ ב׳) רכוב ומנהיג — תחת רוכב, וכן בקדושין (ל״ג א׳) רכוב כמהלך, והשמות ארוס, נשוי, מסור, תחת אורס, נושא, מוסר. ובשבת (כ״ב ב׳) הי׳ תפוש נר חנוכה — תחת הי׳ תופש, ושם (קנ״ב א׳) שטוף בזמה פירש״י שטוף כמו שוטף, ובע״ז (ה׳ א׳) כפויי טובה תחת כופי טובה, ורמב״ן פ׳ קדושים פירש הלשון לא תלך רכיל כמו רוכל, וכן השמות רחום וחנון תחת רוחם (או מרחם) וחונן. וכן השם אגור בן יקה (משלי ל׳ א׳) כנוי לשלמה אשר אגר ואסף בינה הרבה והקיאה, (רומז לזה שנשיו הטו לבבו), וראוי הי' להיות שמו אוגר. ועוד הרבה כאלה. ולא יפלא, אם בא כאן הלשון למוד תחת לומד שזה משמותיו של ה׳, כמו ומלמד לאנוש בינה, למדני חקיך (תהלים ועוד). ועפ״י תמורה זו, עפ״י העתק אותיות אפשר לבאר במשלי פסוק אחד אשי עמלו בו כמה מפרשים ולא הצליחו לפרשו בקל. והוא הפסוק במשלי (י״ד כ״ח) ברוב עם הדרת מלך ובאפס לאום מחתת רזון, ואין קץ לכל הדחוקים שנאמרו בבאורו. אבל לפי שבארנו קרוב לומר, שבא השם בסגנון הפוך, ותחת מחתת רזון צ״ל מחתת רוזן, והוא שם משותף עם שם מלך, כמו בתהלים (ג׳) מלכי ארץ ורוזנים נוסדו יחד, ובשופטים (ה׳) שמעו מלכים האזינו רוזנים, ולפי זה מחציתו השניה של הפסוק הוא מענין מחציתו הראשונה, מענין היפך הלשון הדרת — שהוא מחתת רוזן, שבאפס לאום הוא עוצב ודואג על מניעת כבודו ותקפו. וקבע הכתוב שני שמות, מלך ורוזן, אעפ״י שהם בכלל ענין אחד, שררות וממשלה, מפני שכן דרך הלשון בעת שדרוש לבא שני שמות או שתי מלים או שני פעלים דומים בענינים תכופים, רגילים לכתוב בשמות נפרדים, וזה לתפארת הלשון, וכעין זה כתבו התוס׳ בב״מ (ס׳ ב׳ ד״ה למה) בזה הלשון. כיון שהוצרך שני לאוין אורחא דקרא לכתוב לשון משונה שהוא נאה יותר, עכ״ל. והארכנו מזה במקום אחר. ועיין בספרנו תוספת ברכה סוף פרשה ויגש.*הערה פרטית: ראיתי להעיר, כי מאמר זה בא גם בבאורנו לפרקי אבות (פ״ב משנה י״ד) ואעפ״י שהי׳ אפשר לציין שם לעיין לכאן, או להיפך לציין כאן לעיין לשם — אך קבעתיו גם כאן גם שם, וטעמי בזה, מפני שאחשב ברצות ה׳ דרכי להוציא לאור באורי להגדה ולפרקי אבות בתוצאות מיוחדות עם גופי נוסחאות ההגדה והפרקים, מזה לבדו ומזה לבדו — לכן אמרתי לקבוע באור זה פה ושם, למען יהי׳ מוכן למגידי ההגדה וללומדי הפרקים על מקומם.}%endcomment
\hebeng{נוֹרָא הוּא, סַגִּיב הוּא, עִזּוּז הוּא יִבְנֶה בֵּיתוֹ בְּקָרוֹב. בִּמְהֵרָה, בִּמְהֵרָה, בְּיָמֵינוּ בְּקָרוֹב. אֵל בְּנֵה, אֵל בְּנֵה, בְּנֵה בֵּיתְךָ בְּקָרוֹב.}{Awesome is He, exalted is He, heroic is He. Quickly, quickly, in our days, soon. God build, God build, build Your house soon. }
\hebeng{פּוֹדֶה הוּא, צַדִּיק הוּא, קָּדוֹשׁ הוּא יִבְנֶה בֵּיתוֹ בְּקָרוֹב. בִּמְהֵרָה, בִּמְהֵרָה, בְּיָמֵינוּ בְּקָרוֹב. אֵל בְּנֵה, אֵל בְּנֵה, בְּנֵה בֵּיתְךָ בְּקָרוֹב.}{A restorer is He, righteous is He, holy is He. Quickly, quickly, in our days, soon. God build, God build, build Your house soon. }
\hebeng{רַחוּם הוּא, שַׁדַּי הוּא, תַּקִּיף הוּא יִבְנֶה בֵּיתוֹ בְּקָרוֹב. בִּמְהֵרָה, בִּמְהֵרָה, בְּיָמֵינוּ בְּקָרוֹב. אֵל בְּנֵה, אֵל בְּנֵה, בְּנֵה בֵּיתְךָ בְּקָרוֹב. }{Merciful is He, the Omnipotent is He, dynamic is He. Quickly, quickly, in our days, soon. God build, God build, build Your house soon. }
\newsection{ספירת העומר}
\hebeng{{\small ספירת העמר בחוץ לארץ, בליל שני של פסח:} }{{\small The counting of the \textit{omer} outside of Israel on the second night of Pesach:} }
\hebeng{בָּרוּךְ אַתָּה ה׳, אֱלֹהֵינוּ מֶלֶךְ הָעוֹלָם, אֲשֶׁר קִדְּשָׁנוּ בְּמִצְוֹֹּתָיו וְצִוָּנוּ עַל סְפִירַת הָעֹמֶר. הַיּוֹם יוֹם אֶחָד בָּעֹמֶר. }{Blessed are You, Lord our God, King of the Universe, who has sanctified us with His commandments and has commanded us on the counting of the \textit{omer}. Today is the first day of the \textit{omer}.}
\newsection{אחד מי יודע}
\hebeng{אֶחָד מִי יוֹדֵעַ? אֶחָד אֲנִי יוֹדֵעַ: אֶחָד אֱלֹהֵינוּ שֶׁבַּשָּׁמַיִם וּבָאָרֶץ.\par שְׁנַיִם מִי יוֹדֵעַ? שְׁנַיִם אֲנִי יוֹדֵעַ: שְׁנֵי לֻחוֹת הַבְּרִית. אֶחָד אֱלֹהֵינוּ שֶׁבַּשָּׁמַיִם וּבָאָרֶץ.\par שְׁלֹשָׁה מִי יוֹדֵעַ? שְׁלֹשָׁה אֲנִי יוֹדֵעַ: שְׁלֹשָׁה אָבוֹת, שְׁנֵי לֻחוֹת הַבְּרִית, אֶחָד אֱלֹהֵינוּ שֶׁבַּשָּׁמַיִם וּבָאָרֶץ.\par אַרְבַּע מִי יוֹדֵעַ? אַרְבַּע אֲנִי יוֹדֵעַ: אַרְבַּע אִמָּהוֹת, שְׁלשָׁה אָבוֹת, שְׁנֵי לֻחוֹת הַבְּרִית, אֶחָד אֱלֹהֵינוּ שֶׁבַּשָּׁמַיִם וּבָאָרֶץ.\par חֲמִשָּׁה מִי יוֹדֵעַ? חֲמִשָּׁה אֲנִי יוֹדֵעַ: חֲמִשָּׁה חוּמְשֵׁי תוֹרָה, אַרְבַּע אִמָּהוֹת, שְׁלשָׁה אָבוֹת, שְׁנֵי לֻחוֹת הַבְּרִית, אֶחָד אֱלֹהֵינוּ שֶׁבַּשָּׁמַיִם וּבָאָרֶץ.\par שִׁשָּׂה מִי יוֹדֵעַ? שִׁשָּׂה אֲנִי יוֹדֵעַ: שִׁשָּׁה סִדְרֵי מִשְׁנָה, חֲמִשָּׁה חוּמְשֵׁי תוֹרָה, אַרְבַּע אִמָּהוֹת, שְׁלֹשָׁה אָבוֹת, שְׁנֵי לֻחוֹת הַבְּרִית, אֶחָד אֱלֹהֵינוּ שֶׁבַּשָּׁמַיִם וּבָאָרֶץ.\par שִׁבְעָה מִי יוֹדֵעַ? שִׁבְעָה אֲנִי יוֹדֵעַ: שִׁבְעָה יְמֵי שַׁבָּתָא, שִׁשָּׁה סִדְרֵי מִשְׁנָה, חֲמִשָּׁה חוּמְשֵׁי תוֹרָה, אַרְבַּע אִמָּהוֹת, שְׁלשָׁה אָבוֹת, שְׁנֵי לֻחוֹת הַבְּרִית, אֶחָד אֱלֹהֵינוּ שֶׁבַּשָּׁמַיִם וּבָאָרֶץ.\par שְׁמוֹנָה מִי יוֹדֵעַ? שְׁמוֹנָה אֲנִי יוֹדֵעַ: שְׁמוֹנָה יְמֵי מִילָה, שִׁבְעָה יְמֵי שַׁבָּתָא, שִׁשָּׁה סִדְרֵי מִשְׁנָה, חֲמִשָּׁה חוּמְשֵׁי תוֹרָה, אַרְבַּע אִמָּהוֹת, שְׁלשָׁה אָבוֹת, שְׁנֵי לֻחוֹת הַבְּרִית, אֶחָד אֱלֹהֵינוּ שֶׁבַּשָּׁמַיִם וּבָאָרֶץ.\par תִּשְׁעָה מִי יוֹדֵעַ? תִּשְׁעָה אֲנִי יוֹדֵעַ: תִּשְׁעָה יַרְחֵי לֵדָה, שְׁמוֹנָה יְמֵי מִילָה, שִׁבְעָה יְמֵי שַׁבָּתָא, שִׁשָּׁה סִדְרֵי מִשְׁנָה, חֲמִשָּׁה חוּמְשֵׁי תוֹרָה, אַרְבַּע אִמָּהוֹת, שְׁלֹשָׁה אָבוֹת, שְׁנֵי לֻחוֹת הַבְּרִית, אֶחָד אֱלֹהֵינוּ שֶׁבַּשָּׁמַיִם וּבָאָרֶץ.\par עֲשָֹרָה מִי יוֹדֵעַ? עֲשָֹרָה אֲנִי יוֹדֵעַ: עֲשָׂרָה דִבְּרַיָא, תִּשְׁעָה יַרְחֵי לֵדָה, שְׁמוֹנָה יְמֵי מִילָה, שִׁבְעָה יְמֵי שַׁבָּתָא, שִׁשָּׁה סִדְרֵי מִשְׁנָה, חֲמִשָּׁה חוּמְשֵׁי תוֹרָה, אַרְבַּע אִמָּהוֹת, שְׁלשָׁה אָבוֹת, שְׁנֵי לֻחוֹת הַבְּרִית, אֶחָד אֱלֹהֵינוּ שֶׁבַּשָּׁמַיִם וּבָאָרֶץ. עֲשָֹרָה אַחַד עָשָׂר מִי יוֹדֵעַ? אַחַד עָשָׂר אֲנִי יוֹדֵעַ: אַחַד עָשָׂר כּוֹכְבַיָּא, עֲשָׂרָה דִבְּרַיָא, תִּשְׁעָה יַרְחֵי לֵדָה, שְׁמוֹנָה יְמֵי מִילָה, שִׁבְעָה יְמֵי שַׁבָּתָא, שִׁשָּׁה סִדְרֵי מִשְׁנָה, חֲמִשָּׁה חוּמְשֵׁי תוֹרָה, אַרְבַּע אִמָּהוֹת, שְׁלשָׁה אָבוֹת, שְׁנֵי לֻחוֹת הַבְּרִית, אֶחָד אֱלֹהֵינוּ שֶׁבַּשָּׁמַיִם וּבָאָרֶץ.\par שְׁנֵים עָשָׂר מִי יוֹדֵעַ? שְׁנֵים עָשָׂר אֲנִי יוֹדֵעַ: שְׁנֵים עָשָׂר שִׁבְטַיָּא, אַחַד עָשָׂר כּוֹכְבַיָּא, עֲשָׂרָה דִבְּרַיָא, תִּשְׁעָה יַרְחֵי לֵדָה, שְׁמוֹנָה יְמֵי מִילָה, שִׁבְעָה יְמֵי שַׁבָּתָא, שִׁשָּׁה סִדְרֵי מִשְׁנָה, חֲמִשָּׁה חוּמְשֵׁי תוֹרָה, אַרְבַּע אִמָּהוֹת, שְׁלשָׁה אָבוֹת, שְׁנֵי לֻחוֹת הַבְּרִית, אֶחָד אֱלֹהֵינוּ שֶׁבַּשָּׁמַיִם וּבָאָרֶץ.\par שְׁלשָׁה עֶשָׂר מִי יוֹדֵעַ? שְׁלשָׁה עָשָׂר אֲנִי יוֹדֵעַ: שְׁלשָׁה עָשָׂר מִדַּיָּא. שְׁנֵים עָשָׂר שִׁבְטַיָּא, אַחַד עָשָׂר כּוֹכְבַיָּא, עֲשָׂרָה דִבְּרַיָא, תִּשְׁעָה יַרְחֵי לֵדָה, שְׁמוֹנָה יְמֵי מִילָה, שִׁבְעָה יְמֵי שַׁבָּתָא, שִׁשָּׁה סִדְרֵי מִשְׁנָה, חֲמִשָּׁה חוּמְשֵׁי תוֹרָה, אַרְבַּע אִמָּהוֹת, שְׁלשָׁה אָבוֹת, שְׁנֵי לֻחוֹת הַבְּרִית, אֶחָד אֱלֹהֵינוּ שֶׁבַּשָּׁמַיִם וּבָאָרֶץ. }{Who knows one? I know one: One is our God in the heavens and the earth. Who knows two? I know two: two are the tablets of the covenant, One is our God in the heavens and the earth. Who knows three? I know three: three are the fathers, two are the tablets of the covenant, One is our God in the heavens and the earth. Who knows four? I know four: four are the mothers, three are the fathers, two are the tablets of the covenant, One is our God in the heavens and the earth. Who knows five? I know five: five are the books of the Torah, four are the mothers, three are the fathers, two are the tablets of the covenant, One is our God in the heavens and the earth. Who knows six? I know six: six are the orders of the Mishnah, five are the books of the Torah, four are the mothers, three are the fathers, two are the tablets of the covenant, One is our God in the heavens and the earth. Who knows seven? I know seven: seven are the days of the week, six are the orders of the Mishnah, five are the books of the Torah, four are the mothers, three are the fathers, two are the tablets of the covenant, One is our God in the heavens and the earth. Who knows eight? I know eight: eight are the days of circumcision, seven are the days of the week, six are the orders of the Mishnah, five are the books of the Torah, four are the mothers, three are the fathers, two are the tablets of the covenant, One is our God in the heavens and the earth. Who knows nine? I know nine: nine are the months of birth, eight are the days of circumcision, seven are the days of the week, six are the orders of the Mishnah, five are the books of the Torah, four are the mothers, three are the fathers, two are the tablets of the covenant, One is our God in the heavens and the earth. Who knows ten? I know ten: ten are the statements, nine are the months of birth, eight are the days of circumcision, seven are the days of the week, six are the orders of the Mishnah, five are the books of the Torah, four are the mothers, three are the fathers, two are the tablets of the covenant, One is our God in the heavens and the earth. Who knows eleven? I know eleven: eleven are the stars, ten are the statements, nine are the months of birth, eight are the days of circumcision, seven are the days of the week, six are the orders of the Mishnah, five are the books of the Torah, four are the mothers, three are the fathers, two are the tablets of the covenant, One is our God in the heavens and the earth. Who knows twelve? I know twelve: twelve are the tribes, eleven are the stars, ten are the statements, nine are the months of birth, eight are the days of circumcision, seven are the days of the week, six are the orders of the Mishnah, five are the books of the Torah, four are the mothers, three are the fathers, two are the tablets of the covenant, One is our God in the heavens and the earth. Who knows thirteen? I know thirteen: thirteen are the characteristics, twelve are the tribes, eleven are the stars, ten are the statements, nine are the months of birth, eight are the days of circumcision, seven are the days of the week, six are the orders of the Mishnah, five are the books of the Torah, four are the mothers, three are the fathers, two are the tablets of the covenant, One is our God in the heavens and the earth. }%
\commenta{\textrm{\textbf{בפזמון אחד מי יודע. שבעה מי יודע, שבעה אני יודע, שבעה ימי שבתא.}} הנה בכל הסעיפים שבהפזמון תפס ענינים המתיחשים רק לישראל, וזה מסתבר לרגלי רגשות הנפש בשעה אצילית זו, אבל ענין המספר שבעה ימי שבתא הוא ענין כללי לכל העולם שמונים ימי השבוע שבעת ימים. ואפשר לפרש הכונה עפ״י המבואר במכילתא פרשה יתרו בפסוק זכור את יום השבת לקדשו, מהו לקדשו — שלא תהא מונה כדרך שאחרים מונים אלא תהא מונה לשם שבת, ע״כ. והבאור הוא, כי העולם קורא לכל יום בשם מיוחד, אבל בישראל קוראים ביחש מספרו ליום השבת, אחד בשבת, שני בשבת וכו', וזהו לכבוד השבת *ואמנם לסבת אורך גלותנו המר והכבד נתפסנו במנהג כללי לקרוא בשם מיוחד את ימי השבוע, ורק בכתבים דתיים, בכתובות וגיטין ושטרות נשאר לנו מנהג מקורי שלנו, וכותבים באחד בשבת, בשני בשבת וכו'.. ולפי זה הכוונה כאן להזכיר בכבוד את השבת, וזה באור הלשון שבעה ימי שבתא, שבעת ימים שמתיחשים כולם לשם שבת, כמש״כ. ואמנם אחר כל זה קשה, למה לא תפס מספר שבעה ישר בלשונו ובענינו בקודש, כמו שבעת ימי החג, שבעה קני מנורה, שבע הזאות ביוהכ״פ ועוד, וצ״ע.\textrm{\textbf{תשעה מי יודע, תשעה אני יודע, תשעה ירחי לידה.}} גם במספר זה קשה כמו שהערנו במאמר הקודם, שבכל המספרים תפס ענינים כאלה שמתיחשים רק לישראל, וכאן קרא בשם פרט כזה שענינו כללי לכל העולם. ואולי רומז למה שמסופר בגמרא יומא (פ״ב ב׳) באשה מעוברת אחת שהריחה ביוהכ״פ מאכל אחד ותתעורר בה תאוה לטעום אותו, ואתו לקמי דרבי ושאלו אותו איך להתנהג בה, כי אפשר שתסתכן אם לא יתמלאו רגשי תאותה, ומזה אפשר שיסתכן הולד, שמקור התאוה ממנו (רש״י), אמר, לחשו לה באזנה שהיום יוהכ״פ, לחשו לה ואילחשא, פסקה מהתאוות, קרי רבי עלי׳ בטרם אצרך בבטן ידעתיך, נפק מינה רבי יוחנן, הרי דגם לעובר במעי אמו יש רגש קודש. ואמנם צריך לאמר, דהא דלא דיבר עם אשה הוא רק על עניניהן הפרטיים המתיחשים רק להן, אבל בענינים הנוגעים לכל האומה דיבר, וראי׳ לזה, שכן מצינו הרבה נשים נביאות, כמו שחשיב במס׳ מגילה (י״ד א׳) שרה, מרים, דבורה, חנה, אביגיל, חולדה ואסתר, והנבואה באה מרוח ה׳ ומדבריו שלו.}%endcomment
\newsection{חד גדיא}
\hebeng{חַד גַּדְיָא, חַד גַּדְיָא דְּזַבִּין אַבָּא בִּתְרֵי זוּזֵי, חַד גַּדְיָא, חַד גַּדְיָא.}{One kid, one kid that my father bought for two \textit{zuz}, one kid, one kid.}
\hebeng{וְאָתָא שׁוּנְרָא וְאַָכְלָה לְגַדְיָא, דְּזַבִּין אַבָּא בִּתְרֵי זוּזֵי. חַד גַּדְיָא, חַד גַּדְיָא.}{Then came a cat and ate the kid that my father bought for two \textit{zuz}, one kid, one kid.}
\hebeng{וְאָתָא כַלְבָּא וְנָשַׁךְ לְשׁוּנְרָא, דְּאַָכְלָה לְגַדְיָא, דְּזַבִּין אַבָּא בִּתְרֵי זוּזֵי. חַד גַּדְיָא, חַד גַּדְיָא.}{Then came a dog and bit the cat, that ate the kid that my father bought for two \textit{zuz}, one kid, one kid. }
\hebeng{וְאָתָא חוּטְרָא והִכָּה לְכַלְבָּא, דְּנָשַׁךְ לְשׁוּנְרָא, דְּאַָכְלָה לְגַדְיָא, דְּזַבִּין אַבָּא בִּתְרֵי זוּזֵי. חַד גַּדְיָא, חַד גַּדְיָא.}{Then came a stick and hit the dog, that bit the cat, that ate the kid that my father bought for two \textit{zuz}, one kid, one kid.}
\hebeng{וְאָתָא נוּרָא וְשָׂרַף לְחוּטְרָא, דְּהִכָּה לְכַלְבָּא, דְּנָשַׁךְ לְשׁוּנְרָא, דְּאַָכְלָה לְגַדְיָא, דְּזַבִּין אַבָּא בִּתְרֵי זוּזֵי. חַד גַּדְיָא, חַד גַּדְיָא.}{Then came fire and burnt the stick, that hit the dog, that bit the cat, that ate the kid that my father bought for two \textit{zuz}, one kid, one kid.}
\hebeng{וְאָתָא מַיָא וְכָבָה לְנוּרָא, דְּשָׂרַף לְחוּטְרָא, דְּהִכָּה לְכַלְבָּא, דְּנָשַׁךְ לְשׁוּנְרָא, דְּאַָכְלָה לְגַדְיָא, דְּזַבִּין אַבָּא בִּתְרֵי זוּזֵי. חַד גַּדְיָא, חַד גַּדְיָא.}{Then came water and extinguished the fire, that burnt the stick, that hit the dog, that bit the cat, that ate the kid that my father bought for two \textit{zuz}, one kid, one kid.}
\hebeng{וְאָתָא תורָא וְשָׁתָה לְמַיָא, דְּכָבָה לְנוּרָא, דְּשָׂרַף לְחוּטְרָא, דְּהִכָּה לְכַלְבָּא, דְּנָשַׁךְ לְשׁוּנְרָא, דְּאַָכְלָה לְגַדְיָא, דְּזַבִּין אַבָּא בִּתְרֵי זוּזֵי. חַד גַּדְיָא, חַד גַּדְיָא.}{Then came a bull and drank the water, that extinguished the fire, that burnt the stick, that hit the dog, that bit the cat, that ate the kid that my father bought for two \textit{zuz}, one kid, one kid.}
\hebeng{וְאָתָא הַשׁוחֵט וְשָׁחַט לְתורָא, דְּשָּׁתָה לְמַיָא, דְּכָבָה לְנוּרָא, דְּשָׂרַף לְחוּטְרָא, דְּהִכָּה לְכַלְבָּא, דְּנָשַׁךְ לְשׁוּנְרָא, דְּאַָכְלָה לְגַדְיָא, דְּזַבִּין אַבָּא בִּתְרֵי זוּזֵי. חַד גַּדְיָא, חַד גַּדְיָא.}{Then came the \textit{schochet} and slaughtered the bull, that drank the water, that extinguished the fire, that burnt the stick, that hit the dog, that bit the cat, that ate the kid that my father bought for two \textit{zuz}, one kid, one kid.}
\hebeng{וְאָתָא מַלְאָךְ הַמָּוֶת וְשָׁחַט לְשׁוחֵט, דְּשָׁחַט לְתורָא, דְּשָּׁתָה לְמַיָא, דְּכָבָה לְנוּרָא, דְּשָׂרַף לְחוּטְרָא, דְּהִכָּה לְכַלְבָּא, דְּנָשַׁךְ לְשׁוּנְרָא, דְּאַָכְלָה לְגַדְיָא, דְּזַבִּין אַבָּא בִּתְרֵי זוּזֵי. חַד גַּדְיָא, חַד גַּדְיָא.}{Then came the angel of death and slaughtered the \textit{schochet}, who slaughtered the bull, that drank the water, that extinguished the fire, that burnt the stick, that hit the dog, that bit the cat, that ate the kid that my father bought for two \textit{zuz}, one kid, one kid.}
\hebeng{וְאָתָא הַקָּדושׁ בָּרוּךְ הוּא וְשָׁחַט לְמַלְאַךְ הַמָּוֶת, דְּשָׁחַט לְשׁוחֵט, דְּשָׁחַט לְתורָא, דְּשָּׁתָה לְמַיָא, דְּכָבָה לְנוּרָא, דְּשָׂרַף לְחוּטְרָא, דְּהִכָּה לְכַלְבָּא, דְּנָשַׁךְ לְשׁוּנְרָא, דְּאַָכְלָה לְגַדְיָא, דְּזַבִּין אַבָּא בִּתְרֵי זוּזֵי. חַד גַּדְיָא, חַד גַּדְיָא.}{Then came the Holy One, blessed be He and slaughtered the angel of death, who slaughtered the \textit{schochet}, who slaughtered the bull, that drank the water, that extinguished the fire, that burnt the stick, that hit the dog, that bit the cat, that ate the kid that my father bought for two \textit{zuz}, one kid, one kid.}

\addtocontents{toc}{\protect\end{multicols}}
\end{document}
