\documentclass[12pt, openany]{book}
\usepackage[paperheight=9in,paperwidth=6in,top=.5in,bottom=.5in, inner=.7in, outer=.7in, marginparsep=.1in, headsep=16pt]{geometry}

\newcommand{\texttitle}{משנה ברכות}%title_here
\usepackage{titlesec}
\usepackage{resources/unnumberedtotoc}

\usepackage{fancyhdr}
\pagestyle{fancy}
\fancyhf{}
\fancyfoot[C]{\thepage}
\fancyhead[C]{\texttitle \space\textendash\space \leftmark}

\usepackage{paracol}
\usepackage{anyfontsize}
\usepackage{ragged2e}
\usepackage{polyglossia}

\setdefaultlanguage{hebrew}
\setotherlanguage{english}
\usepackage{fontspec}
\setmainfont{FrankRuehl}
%\newfontfamily\englishfont{Arial}

\newcommand{\textblock}[2]{
	{\fontsize{16pt}{20pt}\selectfont #1\\}
	
	\begin{english}
		#2
	\end{english}
	\clearpage
}


\begin{document}
\frontmatter
\pagenumbering{roman}

\title{משנה ברכות}

\author{}

\date{}

\maketitle

\tableofcontents

\clearpage
\mainmatter
\pagenumbering{arabic}

\addchap{פרק \hebrewnumeral{1}}
\addsec{משנה \hebrewnumeral{1}}
\textblock{מאימתי קורין את שמע בערבית. משעה שהכהנים נכנסים לאכול בתרומתן. עד סוף האשמורה הראשונה דברי ר' אליעזר. וחכמים אומרים עד חצות. רבן גמליאל אומר עד שיעלה עמוד השחר. מעשה שבאו בניו מבית המשתה אמרו לו לא קרינו את שמע. אמר להם אם לא עלה עמוד השחר חייבין אתם לקרות ולא זו בלבד אלא כל מה שאמרו חכמים עד חצות מצותן עד שיעלה עמוד השחר. הקטר חלבים ואברים מצותן עד שיעלה עמוד השחר וכל הנאכלין ליום אחד מצותן עד שיעלה עמוד השחר. אם כן למה אמרו חכמים עד חצות כדי להרחיק אדם מן העבירה: }{From what time may one recite the Shema in the evening? From the time that the priests enter {[their houses]} in order to eat their terumah until the end of the first watch, the words of Rabbi Eliezer. The sages say: until midnight. Rabban Gamaliel says: until dawn. Once it happened that his sons came home {[late]} from a wedding feast and they said to him: we have not yet recited the {[evening]} Shema. He said to them: if it is not yet dawn you are still obligated to recite. And not in respect to this alone did they so decide, but wherever the sages say “until midnight,” the mitzvah may be performed until dawn. The burning of the fat and the pieces may be performed till dawn. Similarly, all {[the offerings]} that are to be eaten within one day may be eaten till dawn. Why then did the sages say “until midnight”?  In order to keep a man far from transgression.}
\addsec{משנה \hebrewnumeral{2}}
\textblock{מאימתי קורין את שמע בשחרית משיכיר בין תכלת ללבן רבי אליעזר אומר בין תכלת לכרתי וגומרה עד הנץ החמה. רבי יהושע אומר עד שלש שעות. שכן דרך בני מלכים לעמוד בשלש שעות הקורא מכאן ואילך לא הפסיד כאדם הקורא בתורה: }{From what time may one recite the Shema in the morning? From the time that one can distinguish between blue and white. Rabbi Eliezer says: between blue and green. And he must finish it by sunrise. Rabbi Joshua says: until the third hour of the day, for such is the custom of the children of kings, to rise at the third hour. If one recites the Shema later he loses nothing, like one who reads in the Torah.}
\addsec{משנה \hebrewnumeral{3}}
\textblock{בית שמאי אומרים בערב כל אדם יטו ויקראו ובבוקר יעמדו שנאמר (דברים ו, ז) ובשכבך ובקומך ובית הלל אומרים כל אדם קורא כדרכו שנאמר (שם) ובלכתך בדרך אם כן למה נאמר ובשכבך ובקומך בשעה שבני אדם שוכבים ובשעה שבני אדם עומדים אמר ר' טרפון אני הייתי בא בדרך והטיתי לקרות כדברי בית שמאי וסכנתי בעצמי מפני הלסטים אמרו לו כדי היית לחוב בעצמך שעברת על דברי בית הלל: }{Bet Shammai say: in the evening every man should recline and recite {[the Shema]}, and in the morning he should stand, as it says, “And when you lie down and when you get up” (Deuteronomy 6:7). Bet Hillel say that every man should recite in his own way, as it says, “And when you walk by the way” (ibid).  Why then is it said, “And when you lies down and when you get up?” At the time when people lie down and at the time when people rise up. Rabbi Tarfon said: I was once walking by the way and I reclined to recite the Shema according to the words of Bet Shammai, and I incurred danger from robbers. They said to him: you deserved to come to harm, because you acted against the words of Bet Hillel.}
\addsec{משנה \hebrewnumeral{4}}
\textblock{בשחר מברך שתים לפניה ואחת לאחריה ובערב שתי' לפניה ושתי' לאחריה אחת ארוכה ואחת קצרה מקום שאמרו להאריך אינו רשאי לקצר לקצר אינו רשאי להאריך. לחתום אינו רשאי שלא לחתום. ושלא לחתום אינו רשאי לחתום: }{In the morning he recites two blessings before it and one after it; in the evening two before it and two after it, one long and one short. Where they {[the sages]} said that a long one should be said, he may not say a short one; where they said a short one he may not say a long one {[Where they said]} to conclude {[with a blessing]} he is not permitted to not conclude; where they said to not conclude {[with a blessing]}, he may not conclude.}
\addsec{משנה \hebrewnumeral{5}}
\textblock{מזכירין יציאת מצרים בלילות. אמר ר' אלעזר בן עזריה הרי אני כבן שבעים שנה ולא זכיתי שתאמר יציאת מצרים בלילות עד שדרשה בן זומא שנאמר (דברים טז, ג) למען תזכור את יום צאתך מארץ מצרים כל ימי חייך ימי חייך הימים כל ימי חייך הלילות. וחכמים אומרים ימי חייך העולם הזה כל ימי חייך להביא לימות המשיח: }{They mention the Exodus from Egypt at night. Rabbi Elazar ben Azaryah said: "Behold, I am almost a seventy-year old man and I have not succeeded in {[understanding why]} the Exodus from Egypt should be mentioned at night, until Ben Zoma explained it from a verse (Deuteronomy 16:3): ‘In order that you may remember the day you left Egypt all the days of your life.’ ‘The days of your life’ refers to the days.  ‘All the days of your life’ refers to the nights. And the sages say: ‘the days of your life’ refers to this world.  ‘All the days of your life’ includes the days of the Messiah.}
\addchap{פרק \hebrewnumeral{2}}
\addsec{משנה \hebrewnumeral{1}}
\textblock{היה קורא בתורה והגיע זמן המקרא אם כיון לבו יצא ואם לאו לא יצא. בפרקים שואל מפני הכבוד ומשיב. ובאמצע שואל מפני היראה ומשיב דברי ר' מאיר. ר' יהודה אומר באמצע שואל מפני היראה ומשיב מפני הכבוד בפרקים שואל מפני הכבוד ומשיב שלום לכל אדם: }{If one was reading in the Torah {[the section of the Shema]} and the time for its recital arrived, if he directed his heart {[to fulfill the mitzvah]} he has fulfilled his obligation, but if not he has not fulfilled his obligation. In the breaks {[between sections]} one may give greeting out of respect and return greeting; in the middle {[of a section]} one may give greeting out of fear and return it, the words of Rabbi Meir. Rabbi Judah says: in the middle one may give greeting out of fear and return it out of respect, in the breaks one may give greeting out of respect and return greeting to anyone.}
\addsec{משנה \hebrewnumeral{2}}
\textblock{אלו הן בין הפרקים בין ברכה ראשונה לשניה בין שניה לשמע ובין שמע לוהיה אם שמוע בין והיה אם שמוע לויאמר בין ויאמר לאמת ויציב. רבי יהודה אומר בין ויאמר לאמת ויציב לא יפסיק. אמר ר' יהושע בן קרחה למה קדמה שמע לוהיה אם שמוע אלא כדי שיקבל עליו עול מלכות שמים תחלה ואחר כך יקבל עליו עול מצות. והיה אם שמוע לויאמר שוהיה אם שמוע נוהג ביום ובלילה ויאמר אינו נוהג אלא ביום: }{These are the breaks between the sections: between the first blessing and the second,   between the second and “Shema,” between “Shema” and “And it shall come to pass if you listen” between “And it shall come to pass if you listen” and “And the Lord said” and between “And the Lord said” and “Emet veYatziv” (true and firm). Rabbi Judah says: between “And the Lord said” and “Emet veYatziv” one should not interrupt. Rabbi Joshua ben Korhah said: Why was the section of “Shema” placed before that of “And it shall come to pass if you listen”? So that one should first accept upon himself the yoke of the Kingdom of Heaven and then take upon himself the yoke of the commandments. Why does the section of “And it shall come to pass if you listen” come before that of “And the Lord said”? Because “And it shall come to pass if you listen” is customary during both day and night,   whereas {[the section]} “And the Lord said” is customary only during the day.}
\addsec{משנה \hebrewnumeral{3}}
\textblock{הקורא את שמע ולא השמיע לאזנו יצא. רבי יוסי אומר לא יצא. קרא ולא דקדק באותיותיה רבי יוסי אומר יצא. ר' יהודה אומר לא יצא. הקורא למפרע לא יצא. קרא וטעה יחזור למקום שטעה: }{One who recites the Shema without causing it to be heard by his own ear, he has fulfilled his obligation. Rabbi Yose says: he has not fulfilled his obligation. If he recited it without pronouncing the letters succinctly, Rabbi Yose says he has fulfilled his obligation. Rabbi Judah says: he has not fulfilled his obligation. If he recites it out of order, he has not fulfilled his obligation. If he recites it and makes a mistake he goes back to the place where he made the mistake.}
\addsec{משנה \hebrewnumeral{4}}
\textblock{האומנין קורין בראש האילן או בראש הנדבך מה שאינן רשאין לעשות כן בתפלה: }{Workers may recite {[the Shema]} on the top of a tree or the top of a scaffolding, that which they are not allowed to do in the case of the Tefillah.}
\addsec{משנה \hebrewnumeral{5}}
\textblock{חתן פטור מקריאת שמע בלילה הראשון עד מוצאי שבת אם לא עשה מעשה. מעשה ברבן גמליאל שקרא בלילה הראשון שנשא אמרו לו תלמידיו לא למדתנו רבינו שחתן פטור מקריאת שמע בלילה הראשון אמר להם איני שומע לכם לבטל ממני מלכות שמים אפילו שעה אחת: }{A bridegroom is exempt from reciting the Shema on the first night until the end of the Shabbat, if he has not performed the act. It happened with Rabban Gamaliel who recited the Shema on the first night after he had married.  His students said to him: Our master, have you not taught us that a bridegroom is exempt from reciting the Shema. He replied to them: I will not listen to you to remove from myself the Kingship of Heaven even for a moment.}
\addsec{משנה \hebrewnumeral{6}}
\textblock{רחץ לילה הראשון שמתה אשתו אמרו לו תלמידיו לא למדתנו רבינו שאבל אסור לרחוץ אמר להם איני כשאר כל אדם אסטניס אני: }{{[Rabban Gamaliel]} bathed on the first night after the death of his wife. His disciples said to him: Master, have you not taught us, that a mourner is forbidden to bathe. He replied to them: I am not like other men, I am very delicate.}
\addsec{משנה \hebrewnumeral{7}}
\textblock{וכשמת טבי עבדו קבל עליו תנחומין אמרו לו תלמידיו לא למדתנו רבינו שאין מקבלין תנחומין על העבדים אמר להם אין טבי עבדי כשאר כל העבדים כשר היה: }{When Tabi his {[Rabban Gamaliel’s]} slave died he accepted condolences for him. His disciples said to him: Master, have you not taught us that one does not accept condolences for slaves? He replied to them: My slave Tabi was not like other slaves: he was a fit man.}
\addsec{משנה \hebrewnumeral{8}}
\textblock{חתן אם רצה לקרות קריאת שמע לילה הראשון קורא רבן שמעון בן גמליאל אומר לא כל הרוצה ליטול את השם יטול: }{If a bridegroom wants to recite the Shema on the first night {[of his marriage]}, he may do so. Rabban Shimon ben Gamaliel says: not everyone who desires to take up the name of God may do so.}
\addchap{פרק \hebrewnumeral{3}}
\addsec{משנה \hebrewnumeral{1}}
\textblock{מי שמתו מוטל לפניו פטור מקריאת שמע מן התפלה ומן התפילין נושאי המטה וחלופיהן וחלופי חלופיהן את שלפני המטה ואת שלאחר המטה את שלמטה צורך בהן פטורין ואת שאין למטה צורך בהן חייבין. אלו ואלו פטורים מן התפלה: }{One whose dead {[relative]} lies before him is exempt from the recital of the Shema and from the tefillah and from tefillin. The bearers of the bier and their replacements, and their replacements’ replacement, both those in front of the bier and those behind the bier those needed to carry the bier, are exempt; but those not needed to carry the bier are obligated. Both, however, are exempt from {[saying]} the tefillah.}
\addsec{משנה \hebrewnumeral{2}}
\textblock{קברו את המת וחזרו אם יכולין להתחיל ולגמור עד שלא יגיעו לשורה יתחילו ואם לאו לא יתחילו העומדים בשורה הפנימים פטורים והחיצונים חייבין: }{When they have buried the dead and returned {[from the grave]}, if they have time to begin and finish {[the Shema]} before they get to the row, they should begin, but if not they should not begin. Those who stand in the row, those on the inside are exempt, but those on the outside are obligated.}
\addsec{משנה \hebrewnumeral{3}}
\textblock{נשים ועבדים וקטנים פטורין מקריאת שמע ומן התפילין וחייבין בתפלה ובמזוזה ובברכת המזון: }{Women, slaves and minors are exempt from reciting the Shema and putting on tefillin, but are obligated for tefillah, mezuzah, and Birkat Hamazon (the blessing after meals).}
\addsec{משנה \hebrewnumeral{4}}
\textblock{בעל קרי מהרהר בלבו ואינו מברך לא לפניה ולא לאחריה. ועל המזון מברך לאחריו ואינו מברך לפניו. ר' יהודה אומר מברך לפניהם ולאחריהם: }{One who has had a seminal emission utters the words {[of the Shema]} in his heart and he doesn’t say a blessing, neither before nor after. Over food he says a blessing afterwards, but not the blessing before. Rabbi Judah says: he blesses both before them and after them.}
\addsec{משנה \hebrewnumeral{5}}
\textblock{היה עומד בתפלה ונזכר שהוא בעל קרי לא יפסיק אלא יקצר. ירד לטבול אם יכול לעלות ולהתכסות ולקרות עד שלא תנץ החמה יעלה ויתכסה ויקרא ואם לאו יתכסה במים ויקרא. אבל לא יתכסה לא במים הרעים ולא במי המשרה עד שיטיל לתוכן מים. וכמה ירחיק מהם ומן הצואה ארבע אמות: }{If a man was standing saying the tefillah and he remembers that he is one who has had a seminal emission, he should not stop but he should abbreviate {[the blessings]}. If he went down to immerse, if he is able to come up and cover himself and recite the Shema before the rising of the sun, he should go up and cover himself and recite, but if not he should cover himself with the water and recite. He should not cover himself either with foul water or with steeping water until he pours fresh water into it. How far should he remove himself from it and from excrement? Four cubits.}
\addsec{משנה \hebrewnumeral{6}}
\textblock{זב שראה קרי. ונדה שפלטה שכבת זרע. והמשמשת שראתה נדה. צריכין טבילה. ורבי יהודה פוטר: }{A zav who has had a seminal emission and a niddah from whom semen escapes and a woman who becomes niddah during intercourse require a mikveh. Rabbi Judah exempts them.}
\addchap{פרק \hebrewnumeral{4}}
\addsec{משנה \hebrewnumeral{1}}
\textblock{תפלת השחר עד חצות. רבי יהודה אומר עד ארבע שעות תפלת המנחה עד הערב. רבי יהודה אומר עד פלג המנחה. תפלת הערב אין לה קבע ושל מוספין כל היום רבי יהודה אומר עד שבע שעות: }{The morning Tefillah (Shacharit) is until midday. Rabbi Judah says until the fourth hour. The afternoon Tefillah (Minhah) until evening. Rabbi Judah says: until the middle of the afternoon. The evening prayer has no fixed time. The time for the additional prayers (musaf) is the whole day. Rabbi Judah says:  until the seventh hour.}
\addsec{משנה \hebrewnumeral{2}}
\textblock{רבי נחוניה בן הקנה היה מתפלל בכניסתו לבית המדרש וביציאתו תפלה קצרה אמרו לו מה מקום לתפלה זו. אמר להם בכניסתי אני מתפלל שלא תארע תקלה על ידי וביציאתי אני נותן הודיה על חלקי: }{Rabbi Nehunia ben Hakaneh used to pray as he entered the Bet Hamidrash and as he left it a short prayer. They said to him: what is the reason for this prayer? He replied: When I enter I pray that that no mishap should occur through me, and when I leave I express thanks for my portion.}
\addsec{משנה \hebrewnumeral{3}}
\textblock{רבן גמליאל אומר בכל יום מתפלל אדם שמונה עשרה. רבי יהושע אומר מעין שמונה עשרה. ר' עקיבא אומר אם שגורה תפלתו בפיו יתפלל שמונה עשרה ואם לאו מעין שמונה עשרה: }{Rabban Gamaliel says: every day a man should pray the eighteen {[blessings]}. Rabbi Joshua says: an abstract of the eighteen. Rabbi Akiva says: if he knows it fluently he prays the eighteen, and if not an abstract of the eighteen.}
\addsec{משנה \hebrewnumeral{4}}
\textblock{רבי אליעזר אומר העושה תפלתו קבע אין תפלתו תחנונים רבי יהושע אומר המהלך במקום סכנה מתפלל תפלה קצרה אומר הושע השם את עמך את שארית ישראל בכל פרשת העבור יהיו צרכיהם לפניך ברוך אתה ה' שומע תפלה: }{Rabbi Eliezer says: if a man makes his prayers fixed, it is not {[true]} supplication. Rabbi Joshua says: if one is traveling in a dangerous place, he says a short prayer, saying: Save, O Lord, Your people the remnant of Israel.  In every time of crisis may their needs be before You. Blessed are You, O Lord, who hears prayer.}
\addsec{משנה \hebrewnumeral{5}}
\textblock{היה רוכב על החמור ירד. ואם אינו יכול לירד יחזיר את פניו. ואם אינו יכול להחזיר את פניו יכוין את לבו כנגד בית קדש הקדשים: }{If he is riding on a donkey, he gets down {[and prays.]} If he is unable to get down he should turn his face {[towards Jerusalem]}, and if he cannot turn his face, he should direct his heart to the Holy of Holies.}
\addsec{משנה \hebrewnumeral{6}}
\textblock{היה יושב בספינה או בקרון או באסדא יכוין את לבו כנגד בית קדש הקדשים: }{If he is traveling in a ship, on a wagon or on a raft, he should  direct his heart toward the Holy of Holies.}
\addsec{משנה \hebrewnumeral{7}}
\textblock{רבי אלעזר בן עזריה אומר אין תפלת המוספין אלא בחבר עיר וחכמים אומרים בחבר עיר ושלא בחבר עיר רבי יהודה אומר משמו כל מקום שיש חבר עיר היחיד פטור מתפלת המוספין: }{Rabbi Elazar ben Azaryah says: The musaf prayer is said only with the local congregation. The sages say: whether with or with out the congregation. Rabbi Judah said in his name: wherever there is a congregation, an individual is exempt from saying the musaf prayer.}
\addchap{פרק \hebrewnumeral{5}}
\addsec{משנה \hebrewnumeral{1}}
\textblock{אין עומדין להתפלל אלא מתוך כובד ראש. חסידים הראשונים היו שוהים שעה אחת ומתפללים כדי שיכונו את לבם למקום. אפילו המלך שואל בשלומו לא ישיבנו. ואפילו נחש כרוך על עקבו לא יפסיק: }{One should not stand up to say Tefillah except in a reverent state of mind. The pious men of old used to wait an hour before praying in order that they might direct their thoughts to God. Even if a king greets him {[while praying]} he should not answer him: even if a snake is wound round his heel he should not stop.}
\addsec{משנה \hebrewnumeral{2}}
\textblock{מזכירין גבורות גשמים בתחיית המתים. ושואלין הגשמים בברכת השנים. והבדלה בחונן הדעת. ר' עקיבא אומר אומרה ברכה רביעית בפני עצמה רבי אליעזר אומר בהודאה: }{They mention {[God’s]} power to bring rain in the blessing for the resurrection of the dead. And they ask for rain in the blessing for {[fruitful]} years. And havdalah in “Who grant knowledge.” Rabbi Akiva says: he says it as a fourth blessing by itself. Rabbi Eliezer says: in the thanksgiving blessing.}
\addsec{משנה \hebrewnumeral{3}}
\textblock{האומר על קן צפור יגיעו רחמיך ועל טוב יזכר שמך. מודים מודים משתקין אותו. העובר לפני התיבה וטעה יעבר אחר תחתיו ולא יהא סרבן באותה שעה מנין הוא מתחיל מתחלת הברכה שטעה בה: }{The one who says, “On a bird’s nest may Your mercy be extended,” {[or]} “For good may Your name be blessed” or “We give thanks, we give thanks,” they silence him. One who was passing before the ark and made a mistake, another should pass in his place, and he should not be as one who refuses at that moment. Where does he begin? At the beginning of the blessing in which the other made a mistake.}
\addsec{משנה \hebrewnumeral{4}}
\textblock{העובר לפני התיבה לא יענה אחר הכהנים אמן מפני הטירוף ואם אין שם כהן אלא הוא לא ישא את כפיו. ואם הבטחתו שהוא נושא את כפיו וחוזר לתפלתו רשאי: }{The one who passes before the ark should not respond Amen after {[the blessings of]} the priests because this might confuse him. If there is no priest there except himself, he should not raise his hands {[to recite the priestly blessing]}, but if he is confident that he can raise his hands and go back to his place in his prayer, he is permitted to do so.}
\addsec{משנה \hebrewnumeral{5}}
\textblock{המתפלל וטעה סימן רע לו. ואם שליח צבור הוא סימן רע לשולחיו. מפני ששלוחו של אדם כמותו. אמרו עליו על רבי חנינא בן דוסא כשהיה מתפלל על החולים ואומר זה חי וזה מת. אמרו לו מנין אתה יודע אמר להם אם שגורה תפלתי בפי יודע אני שהוא מקובל ואם לאו יודע אני שהוא מטורף: }{One who is praying and makes a mistake, it is a bad sign for him. And if he is the messenger of the congregation (the prayer leader) it is a bad sign for those who have sent him, because one’s messenger is equivalent to one’s self. They said about Rabbi Hanina ben Dosa that he used to pray for the sick and say, “This one will die, this one will live.” They said to him: “How do you know?” He replied: “If my prayer comes out fluently, I know that he is accepted, but if not, then I know that he is rejected.”}
\addchap{פרק \hebrewnumeral{6}}
\addsec{משנה \hebrewnumeral{1}}
\textblock{כיצד מברכין על הפירות. על פירות האילן אומר בורא פרי העץ. חוץ מן היין. שעל היין אומר בורא פרי הגפן. ועל פירות הארץ אומר בורא פרי האדמה חוץ מן הפת שעל הפת הוא אומר המוציא לחם מן הארץ. ועל הירקות אומר בורא פרי האדמה. רבי יהודה אומר בורא מיני דשאים: }{How do they bless over produce? Over fruit of the tree one says, “Who creates the fruit of the tree,” except for wine, over which one says, “Who creates the fruit of the vine.” Over produce from the ground one says: “Who creates the fruit of the ground,” except over bread, over which one says, “Who brings forth bread from the earth.” Over vegetables one says, “Who creates the fruit of the ground.” Rabbi Judah says: “Who creates diverse species of herbs.”}
\addsec{משנה \hebrewnumeral{2}}
\textblock{ברך על פירות האילן בורא פרי האדמה יצא. ועל פירות הארץ בורא פרי העץ לא יצא. על כולם אם אמר שהכל נהיה יצא: }{If one blessed over fruit of the tree the blessing, “Who creates the fruit of the ground,” he has fulfilled his obligation. But if he said over produce from the ground, “Who creates the fruit of the tree,” he has not fulfilled his obligation. If over anything he says “By Whose word all things exist”, he has fulfilled his obligation.}
\addsec{משנה \hebrewnumeral{3}}
\textblock{על דבר שאין גדולו מן הארץ אומר שהכל. על החומץ ועל הנובלות ועל הגובאי אומר שהכל. על החלב ועל הגבינה ועל הביצים אומר שהכל. ר' יהודה אומר כל שהוא מין קללה אין מברכין עליו: }{Over anything which does not grow from the earth one says: “By Whose word all things exist.” Over vinegar, fallen unripe fruit and locusts one says, “By Whose word all things exist.” Over milk and cheese and eggs one says, “By Whose word all things exist.”  R. Judah says: over anything which is cursed they do not bless at all.}
\addsec{משנה \hebrewnumeral{4}}
\textblock{היו לפניו מינים הרבה. ר' יהודה אומר אם יש ביניהם ממין שבעה מברך עליו. וחכמים אומרים מברך על איזה מהם שירצה: }{There were several kinds of food before him: Rabbi Judah says that if there is among them one of the seven species, he blesses over that. But the sages say: he may bless over which ever one he wants.}
\addsec{משנה \hebrewnumeral{5}}
\textblock{ברך על היין שלפני המזון פטר את היין שלאחר המזון ברך על הפרפרת שלפני המזון פטר את הפרפרת שלאחר המזון. ברך על הפת פטר את הפרפרת. על הפרפרת לא פטר את הפת. בית שמאי אומרים אף לא מעשה קדרה: }{If he blessed over the wine before the meal he has exempted the wine after the meal. If he blessed over the appetizer (parperet) before the meal, he has exempted the dessert (parperet) after the meal. If he blessed over the bread he has exempted the appetizer/dessert (parperet), but if he blessed over the appetizer/dessert (parperet) he has not exempted the bread. Bet Shammai say: {[he has not even exempted]} a cooked {[grain]} dish.}
\addsec{משנה \hebrewnumeral{6}}
\textblock{היו יושבין לאכול כל אחד ואחד מברך לעצמו. הסיבו אחד מברך לכולן. בא להם יין בתוך המזון כל אחד ואחד מברך לעצמו. לאחר המזון אחד מברך לכולם. והוא אומר על המוגמר אף על פי שאין מביאין את המוגמר אלא לאחר הסעודה: }{If {[those at the table]} are sitting upright, each one blesses for himself. If they are reclining, one blesses for them all. If wine came during the meal, each one says a blessing for himself. If after the meal, one blesses for them all. The same one says {[the blessing]} over the incense, even though the incense is not brought until after the meal.}
\addsec{משנה \hebrewnumeral{7}}
\textblock{הביאו לפניו מליח בתחלה ופת עמו מברך על המליח ופוטר את הפת שהפת טפלה לו זה הכלל כל שהוא עיקר ועמו טפלה. מברך על העיקר ופוטר את הטפלה: }{If they brought in front of him salted food at the beginning of the meal and bread with it, he blesses over the salted food and thereby exempts the bread, since the bread is ancillary to it. This is the general principle: whenever there is one kind of food that is the main {[food]} and another that is ancillary, he blesses over the main food and thereby exempts the ancillary.}
\addsec{משנה \hebrewnumeral{8}}
\textblock{אכל תאנים וענבים ורמונים מברך אחריהן שלש ברכות דברי רבן גמליאל. וחכמים אומרים ברכה אחת מעין שלש רבי עקיבא אומר אפילו אכל שלק והוא מזונו מברך אחריו שלש ברכות. השותה מים לצמאו אומר שהכל נהיה בדברו. ר' טרפון אומר בורא נפשות רבות: }{If one has eaten grapes, figs or pomegranates he blesses after them three blessings, the words of Rabban Gamaliel. The sages say: one blessing which includes three. Rabbi Akiva says: even if one ate only boiled vegetables and that is his meal, he says after it the three blessings. If one drinks water to quench his thirst, he says “By Whose word all things exist.” Rabbi Tarfon says: “Who creates many living things and their requirements.”}
\addchap{פרק \hebrewnumeral{7}}
\addsec{משנה \hebrewnumeral{1}}
\textblock{שלשה שאכלו כאחת חייבין לזמן אכל דמאי ומעשר ראשון שנטלה תרומתו ומעשר שני והקדש שנפדו והשמש שאכל כזית והכותי מזמנין עליהם. אבל אכל טבל ומעשר ראשון שלא נטלה תרומתו ומעשר שני והקדש שלא נפדו והשמש שאכל פחות מכזית והנכרי אין מזמנין עליהם: }{Three that have eaten together, it is their duty to invite {[one another to say Birkat Hamazon]}. One who ate demai, or first tithe whose terumah has been separated, or second tithe or sanctified property which have been redeemed, or an attendant who has eaten as much as an olive’s worth of food, or a Samaritan may be included {[in the three]}. But one who ate untithed produce, or first tithe whose terumah has not been separated, or second tithe or sanctified property which have not been redeemed, or an attendant who has eaten less than the quantity of an olive or a Gentile may not be counted.}
\addsec{משנה \hebrewnumeral{2}}
\textblock{נשים ועבדים וקטנים אין מזמנין עליהם. עד כמה מזמנין עד כזית ר' יהודה אומר עד כביצה: }{Women, children and slaves they do not recite an invitation over them. How much {[must one have eaten]} in order for them to recite an invitation? As much as an olive. Rabbi Judah says:  as much as an egg.}
\addsec{משנה \hebrewnumeral{3}}
\textblock{כיצד מזמנין בשלשה אומר נברך. בשלשה והוא אומר ברכו. בעשרה אומר נברך לאלהינו. בעשרה והוא אומר ברכו. אחד עשרה ואחד עשרה רבוא. במאה אומר נברך לה' אלהינו. במאה והוא אומר ברכו. באלף אומר נברך לה' אלהינו אלהי ישראל. באלף והוא אומר ברכו. ברבוא אומר נברך לה' אלהינו אלהי ישראל אלהי הצבאות יושב הכרובים על המזון שאכלנו. ברבוא והוא אומר ברכו. כענין שהוא מברך כך עונין אחריו ברוך ה' אלהינו אלהי ישראל אלהי הצבאות יושב הכרובים על המזון שאכלנו. ר' יוסי הגלילי אומר לפי רוב הקהל הן מברכין שנאמר (תהלים סח, כז) במקהלות ברכו אלהים ה' ממקור ישראל. אמר רבי עקיבא מה מצינו בבית הכנסת אחד מרובין ואחד מועטין אומר ברכו את ה'. רבי ישמעאל אומר ברכו את ה' המבורך: }{How do they invite {[one another to recite the Birkat Hamazon]}?If there are three, he {[the one saying Birkat Hamazon]} says, “Let us bless {[Him of whose food we have eaten]}.” If there are three and him he says, “Bless {[Him of whose food we have eaten]}” If there are ten, he says, “Let us bless our God {[of whose food we have eaten]}.” If there are ten and he says, “Bless.” It is the same whether there are ten or ten myriads (ten ten thousands). If there are a hundred he says, “Let us bless the Lord our God {[of whose food we have eaten]}. If there are a hundred and him he says, “Bless.” If there are a thousand he says “Let us bless the Lord our God, the God of Israel {[of whose food we have eaten]}.” If there are a thousand and him he says “Bless.” If there are ten thousand he says, “Let us bless the Lord our God, the God of Israel, the God of hosts, who dwells among the cherubim, for the food which we have eaten.” If there are ten thousand and him he says, “Bless.” Corresponding to his blessing the others answer after him, “Blessed be the Lord our God the God of Israel, the God of hosts, who dwells among the cherubim, for the food which we have eaten.” Rabbi Yose the Galilean says: According to the number of the congregation, they bless, as it says, “In assemblies bless God, the Lord, O you who are from the fountain of Israel.” Rabbi Akiba said: What do we find in the synagogue? Whether there are many or few the he says, “Bless the Lord your God.” Rabbi Ishmael says: “Bless the Lord your God who is blessed.”}
\addsec{משנה \hebrewnumeral{4}}
\textblock{שלשה שאכלו כאחד אינן רשאין ליחלק וכן ארבעה וכן חמשה. ששה נחלקין עד עשרה ועשרה אינן נחלקין עד שיהיו עשרים: }{Three persons who have eaten together may not separate {[to recite Birkat Hamazon]}. Similarly four and similarly five. Six may separate, up until ten. And ten may not separate until there are twenty.}
\addsec{משנה \hebrewnumeral{5}}
\textblock{שתי חבורות שהיו אוכלות בבית אחד בזמן שמקצתן רואין אלו את אלו הרי אלו מצטרפים לזמון ואם לאו אלו מזמנין לעצמן ואלו מזמנין לעצמן אין מברכין על היין עד שיתן לתוכו מים דברי רבי אליעזר. וחכמים אומרים מברכין: }{Two eating companies that were eating in the same room: When some of them can see some of the other they combine {[for a zimun]}, but if not each group makes a zimun for itself. They do not bless over the wine until they put water into it, the words of Rabbi Eliezer. The sages say they bless.}
\addchap{פרק \hebrewnumeral{8}}
\addsec{משנה \hebrewnumeral{1}}
\textblock{אלו דברים שבין בית שמאי ובית הלל בסעודה. בית שמאי אומרים מברך על היום ואחר כך מברך על היין ובית הלל אומרים מברך על היין ואחר כך מברך על היום: }{These are the points {[of difference]} between Bet Shammai and Bet Hillel in regard to a meal. Bet Shammai says:  first he blesses over the day and then over the wine. Bet Hillel says:  first he blesses over the wine and then over the day.}
\addsec{משנה \hebrewnumeral{2}}
\textblock{בית שמאי אומרים נוטלין לידים ואחר כך מוזגין את הכוס ובית הלל אומרים מוזגין את הכוס ואחר כך נוטלין לידים: }{Bet Shammai says:  they wash their hands and then they pour the cup {[of wine]}. Bet Hillel says: they pour the cup {[of wine]} and then they wash their hands.}
\addsec{משנה \hebrewnumeral{3}}
\textblock{בית שמאי אומרים מקנח ידיו במפה ומניחה על השלחן ובית הלל אומרים על הכסת: }{Bet Shammai says: he wipes his hand with a towel and then places it on the table. Bet Hillel says: on the cushion.}
\addsec{משנה \hebrewnumeral{4}}
\textblock{בית שמאי אומרים מכבדין את הבית ואחר כך נוטלין לידים ובית הלל אומרים נוטלין לידים ואחר כך מכבדין את הבית: }{Bet Shammai says: {[after the meal]} they sweep the floor and then they wash their hands. But Bet Hillel says: they wash their hands and then they sweep the floor.}
\addsec{משנה \hebrewnumeral{5}}
\textblock{בית שמאי אומרים נר ומזון ובשמים והבדלה. ובית הלל אומרים נר ובשמים ומזון והבדלה. בית שמאי אומרים שברא מאור האש ובית הלל אומרים בורא מאורי האש: }{Bet Shammai says: {[the proper order is]} candle, {[Birkat Ha]}Mazon, spices, and Havdalah. But Bet Hillel says: candle, spices, {[Birkat Ha]}Mazon, and Havdalah. Bet Shammai says {[the blessing over the candle concludes with the words]}, “Who created the light of the fire.” But Bet Hillel says: “Who creates the lights of the fire.”}
\addsec{משנה \hebrewnumeral{6}}
\textblock{אין מברכין לא על הנר ולא על הבשמים של עובדי כוכבים ולא על הנר ולא על הבשמים של מתים ולא על הנר ולא על הבשמים שלפני עבודת כוכבים. אין מברכין על הנר עד שיאותו לאורו: }{They do not bless over the candles or the spices of non-Jews; Or over the candles or the spices of the dead; Or over the candles or the spices of idolatry; And a blessing is not said over the light until they benefit from its light.}
\addsec{משנה \hebrewnumeral{7}}
\textblock{מי שאכל ושכח ולא ברך בית שמאי אומרים יחזור למקומו ויברך ובית הלל אומרים יברך במקום שנזכר. עד אימתי הוא מברך עד כדי שיתעכל המזון שבמעיו: }{One who has eaten and forgotten to bless {[Birkat Hamazon]}: Bet Shammai says: he must return to the place where he ate and bless. But Bet Hillel says:  he should say it in the place where he remembered. Until when can he bless? Until sufficient time has passed for the food in his stomach to be digested.}
\addsec{משנה \hebrewnumeral{8}}
\textblock{בא להם יין לאחר המזון ואין שם אלא אותו הכוס. בית שמאי אומרים מברך על היין ואחר כך מברך על המזון. ובית הלל אומרים מברך על המזון ואחר כך מברך על היין. עונין אמן אחר ישראל המברך ואין עונין אמן אחר הכותי המברך עד שישמע כל הברכה: }{If wine comes to them after the food, and there is only that cup: Bet Shammai says: he blesses over the wine and then he blesses over the food; But Bet Hillel says: he blesses over the food and then he blesses over the wine. They answer amen after a blessing said by an Israelite but they do not answer amen after a blessing said by a Samaritan, until he hears the whole blessing.}
\addchap{פרק \hebrewnumeral{9}}
\addsec{משנה \hebrewnumeral{1}}
\textblock{הרואה מקום שנעשו בו ניסים לישראל אומר ברוך שעשה ניסים לאבותינו במקום הזה. מקום שנעקרה ממנו עבודת כוכבים אומר ברוך שעקר עבודת כוכבים מארצנו: }{If one sees a place where miracles have been done for Israel, he says, “Blessed be the One who made miracles for our ancestors in this place.” {[If one sees]} a place from which idolatry has been uprooted, he should say, “Blessed be the One who removed idolatry from our land.”}
\addsec{משנה \hebrewnumeral{2}}
\textblock{על הזיקין ועל הזועות ועל הברקים ועל הרעמים ועל הרוחות אומר ברוך שכחו וגבורתו מלא עולם על ההרים ועל הגבעות ועל הימים ועל הנהרות ועל המדברות אומר ברוך עושה מעשה בראשית. ר' יהודה אומר הרואה את הים הגדול אומר ברוך שעשה את הים הגדול בזמן שרואה אותו לפרקים. על הגשמים ועל הבשורות הטובות אומר ברוך הטוב והמטיב ועל שמועות רעות אומר ברוך דיין האמת: }{{[On witnessing]} comets, earthquakes, thunder, or windy storms one says, “Blessed be He whose strength and might fill the world.” {[On seeing]} mountains, hills, seas, rivers or deserts one says, “Blessed be He who made creation.” Rabbi Judah says: one who sees the Great Sea should say, “Blessed be He who made the Great Sea,” if he sees it at intervals. For rain and for good news one says, “Blessed be He that is good and grants good.” For bad news one says, “Blessed be the true judge.”}
\addsec{משנה \hebrewnumeral{3}}
\textblock{בנה בית חדש וקנה כלים חדשים אומר ברוך שהחיינו. מברך על הרעה מעין הטובה ועל הטובה מעין הרעה. הצועק לשעבר הרי זו תפלת שוא כיצד היתה אשתו מעוברת ואמר יהי רצון שתלד אשתי זכר הרי זו תפלת שוא. היה בא בדרך ושמע קול צוחה בעיר ואמר יהי רצון שלא יהיו אלו בני ביתי הרי זו תפלת שוא: }{One who has built a new house or bought new vessels says, “Blessed be He who has kept us alive {[and preserved us and brought us to this season.]}” One who blesses over the evil as he blesses over the good or over the good as he blesses over evil; one who cries over the past, behold this is a vain prayer. How so?  If his wife was pregnant and he says, “May it be his will that my wife bear a male child,” this is a vain prayer. If he is coming home from a journey and he hears a cry of distress in the town and says, “May it be his will that this is not be those of my house,” this is a vain prayer.}
\addsec{משנה \hebrewnumeral{4}}
\textblock{הנכנס לכרך מתפלל שתים. אחת בכניסתו. ואחת ביציאתו. בן עזאי אומר ארבע. שתים בכניסתו. ושתים ביציאתו. ונותן הודאה לשעבר. וצועק לעתיד לבא: }{One who enters into a large city should say two prayers, one on entering and one on leaving. Ben Azzai says: four two on entering and two on leaving, he gives thanks for the past and cries out for the future.}
\addsec{משנה \hebrewnumeral{5}}
\textblock{חייב אדם לברך על הרעה כשם שהוא מברך על הטובה שנאמר (דברים ו, ה) ואהבת את ה' אלהיך בכל לבבך ובכל נפשך ובכל מאדך. בכל לבבך בשני יצריך ביצר טוב וביצר רע. ובכל נפשך אפילו הוא נוטל את נפשך. ובכל מאדך בכל ממונך. דבר אחר בכל מאדך בכל מדה ומדה שהוא מודד לך הוי מודה לו במאד מאד. לא יקל אדם את ראשו כנגד שער המזרח שהוא מכוון כנגד בית קדשי הקדשים. לא יכנס להר הבית במקלו ובמנעלו ובפונדתו ובאבק שעל רגליו ולא יעשנו קפנדריא ורקיקה מקל וחומר כל חותמי ברכות שהיו במקדש היו אומרים מן העולם משקלקלו המינין ואמרו אין עולם אלא אחד. התקינו שיהו אומרים מן העולם ועד העולם. והתקינו שיהא אדם שואל את שלום חברו בשם שנאמר (רות ב, ד) והנה בעז בא מבית לחם ויאמר לקוצרים ה' עמכם ויאמרו לו יברכך ה' ואומר (שופטים ו, יב) ה' עמך גבור החיל. ואומר (משלי כג, כב) אל תבוז כי זקנה אמך ואומר (תהלים קיט, קכו) עת לעשות לה' הפרו תורתך. ר' נתן אומר הפרו תורתך עת לעשות לה': }{One must bless {[God]} for the evil in the same way as one blesses for the good, as it says, “And you shall love the Lord your God with all your heart, with all your soul and with all your might” (Deuteronomy 6:5). “With all your heart,” with your two impulses, the evil impulse as well as the good impulse. “With all your soul” even though he takes your soul {[life]} away from you. “With all your might” with all your money. Another explanation, “With all your might”  whatever treatment he metes out to you. One should not show disrespect to the Eastern Gate, because it is in a direct line with the Holy of Holies. One should not enter the Temple Mount with a staff, or with shoes on, or with a wallet, or with dusty feet; nor should one make it a short cut, all the more spitting {[is forbidden]}. All the conclusions of blessings that were in the Temple they would say, “forever {[lit. as long as the world is]}.” When the sectarians perverted their ways and said that there was only one world, they decreed that they should say, “for ever and ever {[lit. from the end of the world to the end of the world]}. They also decreed that a person should greet his fellow in God’s name, as it says, “And behold Boaz came from Bethlehem and said to the reapers, ‘May the Lord be with you.’ And they answered him, “May the Lord bless you’” (Ruth 2:. And it also says, “The Lord is with you, you valiant warrior” (Judges 6:12). And it also says, “And do not despise your mother when she grows old” (Proverbs 23:22). And it also says, “It is time to act on behalf of the Lord, for they have violated Your teaching” (Psalms 119:126). Rabbi Natan says: {[this means]} “They have violated your teaching It is time to act on behalf of the Lord.”}

\end{document}
