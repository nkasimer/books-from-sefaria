\documentclass[12pt, openany]{book}
\usepackage[
paperheight=11in,
paperwidth=8.5in,
top=0.5in,
bottom=0.5in,
inner=0.7in,
outer=0.5in,
marginparsep=0.1in,
headsep=16pt
]{geometry}

\newcommand{\texttitle}{חידושי רמב"ן על נדה}%title_here
\usepackage{titlesec}
\usepackage{resources/unnumberedtotoc}

%For header, odd pages have page number on left, chapter title and daf centered; even pages have page number on right, book name centered
\usepackage{fancyhdr}
\pagestyle{fancy}
\fancyhf{}
%\fancyfoot[C]{\thepage}
%\fancyhead[C]{\texttitle \space\textendash\space \leftmark}
\fancyhead[LO, RE]{\thepage}
\fancyhead[CE]{\texttitle}
\fancyhead[CO]{\chapname}

\usepackage{paracol}
\usepackage{anyfontsize}
\usepackage{ragged2e}
\usepackage{polyglossia}
\usepackage{multicol}

\setdefaultlanguage{hebrew}
\setotherlanguage{english}
\usepackage{fontspec}
\setmainfont{Frank Ruehl CLM}
\newfontfamily\englishfont{EB Garamond}

\newcommand{\sethebfont}{
\fontsize{10.5pt}{21.0pt} \selectfont
}

\newcommand{\hebeng}[2]{
	{\sethebfont #1\\}
	
	\begin{english}
		#2
	\end{english}
	\clearpage
}

\newcommand{\twocol}[1]{
	{\sethebfont \begin{multicols}{2}
			#1
	\end{multicols}}	
}

\newcommand{\textblock}[1]{
{\sethebfont #1\\}	
}

\setlength{\parskip}{8pt}

\newcommand{\chapname}{}

\newcommand{\newchap}[1]{
	\addcontentsline{toc}{chapter}{#1}
	\renewcommand{\chapname}{#1}
	\begin{LARGE}
		\begin{center}
			\textbf{#1}
		\end{center}
	\end{LARGE}
}

\begin{document}
\frontmatter
\pagenumbering{roman}

\title{\texttitle}

\author{}

\date{}

\maketitle

\begin{minipage}[b][\textheight][b]{\textwidth}\englishfont	
	\begin{english}
		\vfill
		The following text is used under the terms of 
		%license info
		It was retrieved from Sefaria on \today\space \texthebrew{(\Hebrewtoday)}.  It was typeset and formatted by Ktavi, using \LaTeX . The text is as follows:
		%text attributions here
		\clearpage
		
	\end{english}
\end{minipage}


\tableofcontents

\clearpage
\mainmatter
\pagenumbering{arabic}

\newchap{דף \hebrewnumeral{2}}
\twocol{מתניתין \textbf{שמאי אומר כל הנשים דיין שעתן.}  פי' לשון דיין בכל מקום לומר דיין בכך. ואע"פ שיש עדיין להחמיר יותר אין מחמירין עליו כאותה שאמרו בפ' הזהב (דף נג ע"ב) דיו שיאמר הוא וחומשו מחולל על מעות הראשונות ואתמר עליה אא"ב אין בחומשו היינו דקתני דיו אלא למ"ד אין בו מאי דיו קשיא. ובמסכת סוטה בפ' היה מביא (דף יד ע"ב) מגישה בקרן דרומית מערבית כנגד חודה של קרן ודיו והוינן בה מאי ודיו ואמר רב אשי אצטריך סד"א תיבעי הגשת מנחה גופה קמ"ל אף כאן סד"א תיבעי סייג כדברי הלל וחכמים קמ"ל דיין שעתן, אבל לא דייקינן בשום דוכתין בלשון דיין אם יכולין להקל יותר אלא כיון שהיה בדין להחמיר ואין מחמירין היינו לשון דיין.\par וי"מ דיין שתחמיר עליהם לטמאן שעתן למפרע (מהו) [מיהא] שיעור וסת שהכל מודים שטמאה ואף לשרוף כדלקמן (דף יב, א) ואינו נכון. }
\twocol{\textbf{הלל אומר מפקידה לפקידה ואפילו לימים הרבה.}  פי' אפילו מפקידה לפקידה חוששין להם דמפקידה לפקידה חששא דרבנן היא וספק טומאה גזרו עליו לתלות בתרומה וקדשים כדאיתא בגמרא ואמרינן נמי לקמן בדבר שיש בו דעת לישאל. וממילא נמי שמעינן דדוקא ברשות היחיד אבל ברשות הרבים טהור דלא גזרו אלא דלהוי כספק טומאה דברשות היחיד ובדבר שאין בו דעת לישאל ואפילו ברה"י ספקו טהור וברה"י ודבר שיש בו דעת לישאל נמי לא גזרו אלא לתלות אבל לא טומאה ודאי דחששא בעלמא הוא ולחומרא כטעמא דמפרש בגמרא.\par וי"מ דמעת לעת שבנדה נוהג בין ברה"י בין ברה"ר גזרו לתלות בתרומה וקדשים וכן אמרו בירוש' דמעת לעת שבנדה נוהג בין ברשות היחיד בין ברשות הרבים והיינו נמי דאקשינן הכא בשמעתין להלל קשיא טומאה ודאי דאלו מעת לעת שבנדה תולין ולא אקשינן נמי ברשות הרבים טהור. ואלו הכא קתני בין ברה"י בין ברה"ר טמא והטעם לזה שלא הלכו בגזרה דלמפרע על דרך טומאה דסוטה. דטומאת סוטה מכאן ולהבא היא הילכך השוו רשויות לתלות ולא לשרוף. ואין זה מחוור כלל. }
\twocol{גמרא \textbf{מאי טעמא דשמאי.}  פרושי קא מפרש מתניתין ואזיל דאלו טעמיה דשמאי דכולי עלמא אית להו אלא שהחמירו לתלות בתרומה וקדשים כדפרישית, ואוקים משום דהעמד אשה על חזקתה ובחזקת טהורה עומדת שהרי טבלה לנדתה ובדוקה היא משעה שפסקה לנדתה הראשון.\par  והלל כי אמרי העמד דבר על חזקתו ואפילו לתרומה וקדשים היכא דלית ליה ריעותא מגופיה הך אשה כיון דמגופה קא חזיא לא אמרי' אוקמה אחזקה אלא חיישינן הילכך בחולין אף על גב דאיכא למיחש במילתא כיון דלא מכרעא מילתא דטומאה מוקמינן טהרות אחזקתייהו ג) ד"א מאפישי טומאה לא מפשינן וגבי תרומה וקדשים עבד בהו רבנן מעלה וכיון דליכא חזקה גמורה תולין.\par וי"מ דאף על גב דליכא חזקה מיהו ה"ל ספק טומאה ומסוטה גמרינן מה התם מכאן ולהבא ולא למפרע אף כל ספק טומאה לא מטמינן למפרע אלא משום מעלה דקדשים דתולין וברשות הרבים טהור לגמרי דגמרינן מסוטה בק"ו. }
\twocol{ואקשינן \textbf{מ"ש ממקוה לשמאי קשיא למפרע ולהלל קשיא טומאת ודאי דאלו מעת לעת שבנדה תולין וכו'.}  פי' וכיון דתולין אלמא ספיקא בעלמא הוא וה"ה דקשיא ברשות הרבים ודבר שאין בו דעת לישאל נמי אלא חדא מספיקא נקט משום דתניא לקמן בהדיא וזה וזה תולין.\par  ואי קשיא לך מ"ש אשה מהא דתנן לקמן נגע בא' בלילה וכו' שחכמים מטמאים טומאה ודאי שכל הטומאות כשעת מציאתן וכאן נמי הרי דם לפניך. לא קשיא דשאני אשה דבחזקת טהרה עומדת שהרי בדוקה היא ואף על גב דשכיחי בה דמים מכל מקום כל שהפסיקה וטהרה בחזקתה זו היא אבל אדם זה אינו עומד בחזקת חי תדע דתניא בתוספתא ומודים חכמים לר"מ כשראוהו חי אלמא דבכה"ג בחזקת חי הוא לא מפקינן ליה מחזקתיה אף על פי שנמצא מת בשעת מציאתן ובמקום מציאתן אבל א"ל גבי אשה כיון דספק ביאה הוא ספק הוה ספק לא הוה תולין להקל. ולאו מילתא היא דהא קופה באותה זוית עצמה טהרות הראשונו' (טהורות) [טמאות] לדברי הכל. ואע"פ שהוא דומה לאשה בזה דספק הוה ספק לא הוה הוא אלא משום חזקה ראשונה היא דליתא בקופה וכדבעינא למימר קמן. מ"ש ממקוה לשמאי קשיא למפרע ולהלל קשיא טומאת ודאי דאלו מעת לעת שבנדה תולין וכו'. פי' וכיון דתולין אלמא ספיקא בעלמא הוא וה"ה דקשיא ברשות הרבים ודבר שאין בו דעת לישאל נמי אלא חדא מספיקא נקט משום דתניא לקמן בהדיא וזה וזה תולין.\par  ואי קשיא לך מ"ש אשה מהא דתנן לקמן נגע בא' בלילה וכו' שחכמים מטמאים טומאה ודאי שכל הטומאות כשעת מציאתן וכאן נמי הרי דם לפניך. לא קשיא דשאני אשה דבחזקת טהרה עומדת שהרי בדוקה היא ואף על גב דשכיחי בה דמים מכל מקום כל שהפסיקה וטהרה בחזקתה זו היא אבל אדם זה אינו עומד בחזקת חי תדע דתניא בתוספתא ומודים חכמים לר"מ כשראוהו חי אלמא דבכה"ג בחזקת חי הוא לא מפקינן ליה מחזקתיה אף על פי שנמצא מת בשעת מציאתן ובמקום מציאתן אבל א"ל גבי אשה כיון דספק ביאה הוא ספק הוה ספק לא הוה תולין להקל. ולאו מילתא היא דהא קופה באותה זוית עצמה טהרות הראשונו' (טהורות) [טמאות] לדברי הכל. ואע"פ שהוא דומה לאשה בזה דספק הוה ספק לא הוה הוא אלא משום חזקה ראשונה היא דליתא בקופה וכדבעינא למימר קמן. }
\newchap{דף \hebrewnumeral{3}}
\twocol{\textbf{ושניהם לא למדו אלא מסוטה.}  פי' רבינו תם ז"ל לא למדוה אלא ממה שהן מחלקין בין זו לסוטה דלתנא קמא התם ברשות היחיד ובשיש בו דעת לישאול והכא אפילו ברשות הרבים דאיכא תרתי לריעותא כודאי טומאה הוא ולר"ש התם איכא רגלים הכא ליכא רגלים ואפילו ברשות היחיד תולין כלומר רבנן מחלקים להחמיר ור"ש להקל אבל למגמר מסוטה ממש ליכא דגמר דהא לא דמו ויפה פירש.\par  ועדיין קשה לעיקר שמועה עצמה אי סוטה משום רגלים לדבר הוא כל ספק טומאה ברשות היחיד דטמא נימא שאני סוטה דאיכא רגלים לדבר ובתוס' אמרו דלר"ש כל ספק טומאה ברשות היחיד ספיקא משוי ליה ותולין, ואינו כלום דאטו נימא כולי תנויין דלא כר"ש בכל מקום ספק טומאה ברשות היחיד ספיקו טמא לגמרי קאמרינן, ושמא יש לומר דלר"ש כל שקדמה בדיקה לספק טהור דאמרינן בדיקה ראשונה מהניא עד שעת מציאה דספק. אבל ספקות דעלמא ספק נגע ספק לא נגע ספק הוה ספק לא הוה ספיקו טמא ולישנא דגמרא דקאמר רגלים לדבר לומר דאף על גב דאשה כמי שקדם לה בדיק' היא שאין אשה מזנה והיא אומרת טהורה אני והיה לנו להאמין אותה כמו שמאמינן אותה בטבילה לנדתה וכן כיוצא בהן אפילו הכי משום רגלים לדבר טמאה הכתוב וכל שכן בשאר ספיקות אבל במקום שקדם בדיקה אין מטמאין אלא ברגלים לדבר. }
\twocol{והא דאמרינן בטעמיה דר"ש \textbf{דגמר סוף טומאה מתחלת טומאה}  ולא גמר רשות היחיד דסוף טומאה מתחלת טומאה משום דסבירא ליה דרשות היחיד גמרינן מסוטה בתחלת טומאה וגזרת הכתוב הוא רשות הרבים דינא הוא וגמרינן מיניה דלא אורועי ולטמויי מידי מספיקא. }
\twocol{\textbf{ואב"א היינו טעמיה דשמאי הואיל ומרגשת בעצמה.}  הקשו בתוספות רבותינו הצרפתים ז"ל הואיל אם בדקה עצמה עכשיו ומצאה טמא היאך תאמר מרגשת היתה והלא לא הרגישה עכשיו. ואם תאמר נתלה להקל ונאמר בבדיקה אירע לה אורח וכסבורה הרגשת עד הוא וכפירש"י דא"כ אם שמשה מאתמול נמי נימא הא דלא ארגישה מאתמול כסבורה הרגשי שמש הוא כדאיתמר נמי התם.\par  והם פי' דטעמא דשמאי משום דכיון דרוב פעמים מרגש' לא חיישינן למיעוטא ולא גזרו חכמים מחמתן כלל דהא ללישנ' בתרא דאמרי' משום דאם איתא דהוה דם מעיקרא אתא שוכבת במטה ולא נתהפכה מא"ל אלא משום דבר שאין דרכו בכך לא החמירו לאסור מעת לעת ומיהו שוטה ומשמשת במוך כיון דלעולם כך דינן של אלו לחוש להן לפי שאין בהם הרגשה לעולם לפיכך שמאי מודה בהן ופי' טעמא דשמאי באשה מרגשת לומר דהואיל ואשה מרגשת בעצמה לכך לא חשש שמאי כלל ולא עשה סיג לדבריו. והלל אומר הרגשת מי רגלים היא אלא העמד האשה על ספיק' ודינה לתלות כדפרישי' ללישנא קמא. }
\twocol{\textbf{והאיכא שוטה מודה שמאי בשוטה.}  פירש למאן מודה להלל דהוא בר פלוגתיה והיינו לתלות בתרומה וקדשים ואטעמא דלישנא קמא סמכינן בדהלל דהא ליכא למימר דטעמא דהלל משום דהרגשת מי רגלים חששא היא בעלמא ולפיכך אמר תולין דאם כן אף הלל מודה בשוטה דשורפין ואפילו ברשות הרבים וכן ללישנא בתרא במוך ואנן לא אשכחן במעת לעת אלא תולין, אלא הני לישני בטעמא דשמאי נינהו אבל הלל טעמיה משום דגריעי חזקה דאשה וכיון דדם לפניך עבוד רבנן מעלה בקדשים כדפרישית לעיל, וכן הא דאמרינן נימא תנן כתמים דלא כשמאי, אי הוו להו כתמים כמעת לעת דלא כהלל נמי הוו דהא כתמים לבעלה ולחולין ומעת לעת דוקא לקדשים אלא חזקה באשה הוא דמטהר מעת לעת בחולין להלל כלישנא קמא דגמרא והא דאמרינן נמי להך לישנא ליכא למירמי חבית ומקוה ומבוי הכי נמי פירושי' לשמאי ליכא לאקשויי למפרע אבל להלל בטומאת ודאי לא קושיא היא שהדין נותן כיון דאיכא תרתי לריעותא טומאה ודאי במקום דאיכא חדא ריעותא תולין. וכן שוטה ומשמשת במוך לשמאי אינן אלא תולות כשאר נשים להלל ולא קשיא להו טומאה ודאי דחבית ומקוה ומבוי משום האי טעמא דפרישית ולמה שפירש לעיל דכל ספק טומאה למפרע טהור מסוטה איכא לפרושי דבין שמאי ובין להלל בהאי טעמא בלחוד פליגי מר סבר אשה מרגשת לעצמה ואפילו לתלות אין תולין, ומר סבר אימר הרגשת מי רגלים הוא. והוה ליה ספיקא וכל ספק למפרע טהור מן התורה ומדבריהם החמירו בקדשים לתלות והחמירו בכתמים אפילו לחולין. }
\twocol{ הא דאמרינן \textbf{מודה שמאי בשוטה.}  הוקשה בתוספות דבר שאין בו דעת לישאל היא ואין במעת לעת טומאה כשאין בו דעת לישאל ואומרים שיש לומר שאם נגע בה אדם תולין בו דהא יש בו דעת לישאל בנוגע אעפ"י שאין דעת לישאל בטומאת'.\par עי"ל דאיכא שוטה שיודעת לישאל אם נגעה אם לאו ואין לה הרגש' בראית דמים ואנן שוטה סתם פרכינן לעשותה כשאר הטמאות כשיש בענין דעת לישאל וכשאין בו. }
\twocol{\textbf{והאיכא כתמים לימא תנן כתמים דלא כשמאי.}  פירש"י ז"ל האיכא כתמים דקי"ל דמטמא' למפרע וכו' ולאו דוקא פירכא משום למפרע דהאיכא ר"ש בן אלעזר דאמר בסוף בא סימן דכתם אינו מטמא למפרע כלל שלא יהא כתמה חמור מראיתה אלא פירכא משום מכאן ולהבא היא דלא אשכחן תנא דמטהר בכתמים להבא ולשמאי דאמר אשה מרגש' ולא מחמירין עליה אפילו בדרבנן טהורה היא בכתמי' לגמרי דלא מגופה הוא הואיל ולא ארגישה ומפרקינן מודה שמאי בכתמים אפילו בלמפרע כרבנן מאי טעמא וכו'. }
\twocol{\textbf{משמשת במוך מא"ל.}  פירש"י ז"ל ג' נשים ולא דוקא דהא ארבע נשים דיין שעתן במתניתין אלא כל אשה שמשמשת במוך. }
\twocol{\textbf{מאי איכא בין הנך לישנא להנך לישני.}  איכא למידק והאיכא שוטה ומשמשת במוך ללישנא קמא כולן לשמאי דיין שעתן ולא ללשונות הללו. וא"ל אה"נ אלא האי טעמא עדיפא ליה.\par  ואפשר לפרש מאי איכא בין האי לישנא להנך לישני כדאמר מ"ט משנינן הני לישני בתראי ולא תפסינן לישנא קמא דאתא במתני' כפשטא כל הנשים כולן דיין שעתן ואפילו שוטה ומוך ופריק משום דאיכא למירמי ללישנא קמא חבית מקוה ומבוי ולהני לישנא ליכא למירמי ותו בעי ומאן דתני האי לישנא בתרא מ"ט לא אמר לישנא קמא ופריק משום דאיכא מוך שהדין נותן דשמאי מודה במוך. }
\twocol{\textbf{אי אתה מודה בקופה שנשתמשו בה טהרות וכו'.}  איכא למידק להלל גופיה קשיא טומאה ודאי דאלו קופה טומאה ודאי דקתני טמאות וכדמוכחא נמי שמעתין לקמן ואלו מעל"ע תולין. א"ל להלל גופיה זו יש לה שולים וזו אין לה שולים אלא ה"ק ליה כיון דכשיש לה שולי' טמאות ודאי דין הוא לתלות באשה מפני שהיא כמי שיש לה אוגנים. ואפילו לחזקיה דאמר התם טהורות שאני פירי דלא שרקי וקפיד עלייהו.\par וכן אתה מפרש ללשון שאמרו בקופה שאינה בדוקה א"נ שהיא מכוסה שלא בא הלל להשוות אשה לשאינה בדוקה ולשאילה מכוסה אלא שמאחר שבאלו טמאות ודאי באשה היה לנו לתלות מפני שהיא דומה במקצת לשאינה בדוקה ואינה מכוסה משום דשכיחי בה דמים. }
\newchap{דף \hebrewnumeral{4}}
\twocol{והא דאמרינן \textbf{ואב"א כי מודו שמאי והלל בזויות דקופה.}  ה"פ: לעולם בשאינה בדוקה מודו ובזויות קופה דאיכא תרתי לריעותא אבל בבדוקה אע"פ שבזויות קופה נמצא אינן טמאות דה"ל כאותה שאמרו בתוספתא ומודים חכמים לר"מ בשראוהו חי ועד כאן נמי לא מטמינן בשרץ שנמצא במבוי אלא משום דאיכא שרצים דיליה ושרצים דאתו ליה מעלמא דה"ל כתרתי לריעותא הא לאו הכי לא מטמינן ביה למפרע כלל ל"ש אותה זויות ול"ש זויות אחרת ובקופה לא שכיחי ביה שרצים כלל אלא ודאי בקופה שאינה בדוקה מודו שמאי והלל לכולהו לישני דאי לא א) אמרינן אוקמה אחזקיה ואפי' לתלות ב) וכ"ש דלא מודו בטומאה ודאי וה"נ מוכחא רישא דשמעתין. }
\twocol{\textbf{שאני אומר אדם טהור נכנס לשם ונטלה.}  איכא למידק וליחוש נמי לאדם טמא כדאמרינן בפסחים פ"ק קרדום שאבד בבית או שהניחו בזויות זו ונמצא בזויות אחרת הבית טמא שאני אומר אדם טמא נכנס לשם ונטלו וכ"ש הכא דאיכא למיתלי בתרתי לריעותא באדם טמא ונפלה על גבי מדף. ובפ"ק דשחיטת חולין נמי אמרינן צלוחי' שהניחה מגולה ובא ומצאה מכוסה טמאה שאני אומר אדם טמא נכנס לשם וכסה.\par  ויש מתרצים דכיון דאיכא טומאת מדף תחתיה נראין הדברים שאין אדם נוטלה מלמעלה והניחה למטה אלא כדי שלא תפול ותטמא עשה כן ואלמלא שהוא טהור איך הוא מטמא בידים, ואין זה לשון מחוור.\par אבל ר"ת ז"ל פירש בספר הישר דהתם כלים נינהו וספק כלים הנמצאים טמאים כדאמרינן בפ"ק דשבת על ששה ספקות שורפין את התרומה על ספק כלים הנמצאים דשוינהו רבנן לרובא דעלמא טמאים לגבי כלים הנמצאים אבל ספק אוכלים לא קא חשיב אלמא לא גזרו בהו והוו להו רובא דעלמא טהורים לגבייהו הילכך ליכא למיחש גבי מדף אלא לאדם טהור ולנפילה והוה לו ספק טומאה בדבר שאין בו דעת לישאל וספיקו טהור.\par  ואיני יודע למה גזרו על כלים ולא גזרו על אוכלים הנמצאים שאם נאמר מפני שחששו לפסידתן מפני שאין להם טהרה במקוה אף כלי חרס כגון צלוחית דחולין אין להם טהרה במקוה, ושמא שכלים נמצאים בדרך נפילה ובכל מקום ואין אוכלים נמצאים בדרכים לפי שהן נמאסין מ"ה גזרו על הכלים הנופלים אפילו בבית כל שלא ידענו דרך הנחתו וטימאוה אבל אוכלין שאין מצויין אלא בבית ורוב מציאה דבית דרך הנחה היא ובטהורין הוי לא גזרו עליהם שאלו פירות הנמצאין בשוק דרך נפילה לא הוצרכו לגזור עליהן דרובן נפסלין הן מתורת אוכלים. }
\newchap{דף \hebrewnumeral{5}}
\twocol{\textbf{מתוך שמהומה לביתה אין מכניסתו לחורין ולסדקין.}  מכאן למד הראב"ד ז"ל שכל לבעלה אינו צריך בדיקת חורין וסדקין שהרי בדיקה זו אינה מועילה לטהרות מפני שאין בה בדיקת חורין וסדקין ואע"פ כן מועלת לגבי בעלה. ועוד הביא ראיה ממה שאמרו ג) דתביעה הרי היא כבדיקה לגבי בעלה.\par  והוא ז"ל כתב שיש מי שחולק ואמר דבדיקת זבה בין ביום ההפסקה בין בבדיקת השבעה בעינן בדיקה מעולה כשל טהרות ולא הקלו לגבי בעל אלא בבדיקת המעלות כגון זו שאשה העסוקה בטהרות הצריכו בדיקה זו אף לבעלה מתוך חומר שהחמרת עליה בטהרות אבל בדיקה מעולה הצריכה להן מחמת הבעל עצמו צריכה להיות בדיקה מעולה ועוד שיש להחמיר אפילו בבדיקת מעלות ומה שאמר כאן אינה מכניסתו לא שלא הוצרכה אלא מתוך שמהומה לביתה חוששין לטהרות שמא לא עשתה כהוגן ולא הכניסתן לחורין ולסדקין והוא כעין מה שאמרו בסמוך ניחוש שמא תראה טפת דם ותחפנו שכבת זרע וכו' וכן נראה מדברי רש"י ז"ל דכל בדיקה היינו לחורין וסדקין כשילהי פירקין.\par ולפי הסברא יש להכריע שבדיקת ההפסקה שהיא מעלה אותה מטומאה לטהרה ומוציא אותה מחזקה לחזקה צריכה להיות בדיקה מעולה שאין אחריה עליו ספק, אבל בדיקת השבעה כיון שכבר פסקה להעמידה בחזקתה בבדיקה כל דהו סגיא דהא אפילו לא בדקה כלל אלא בשעת ההפסקה ובסוף שבעה טהורה לקמן בפרק בתרא הילכך לקולא כדברי הראב"ד ז"ל ובעל נפש לא יקל בכך. }
\twocol{\textbf{השתא מעת לעת ממעטה מפקידה לפקידה מיבעיא.}  פי' לדברי חכמים לעולם מפקידה לפקידה זמנה מועט מעת לעת שכבר מיעט מעל"ע ע"י הפקידה וכיון שעד זה ממעט על יד מעת לעת אם לא בדקה מאתמול כ"ש שדינו למעט על יד הפקידה אם בדקה עצמה היום בשעה ראשונה ולרביעית שמשה דכיון שהזמן ביניהם מועט אין לחוש כ"כ בשעת פקידה לומר עם סלוק ידיה ראתה. ופריק סד"א עד זה לא תמעט אלא על יד מעל"ע אע"פ שהחששא שלו יותר קרובה מפני הפסד טהרות הקל. אבל על יד פקידה לא ימעט דליחוש שמא תחפנו שכבת זרע קמ"ל. }
\twocol{\textbf{טעמא אמטו דדיה שעתה דמטה לא מיטמיא הא מעת לעת מטה נמי מיטמיא.}  פי' מיטמיא לעשות אב הטומאה כדין משכב הנדה לטמא אדם ולטמא בגדים והיינו מסייעי ליה לזעירי דאי ס"ד טעמא אמטו דיה שעת' טהורה לגמרי הא מעת לעת טמאה כדין מגעה פשיטא היינו טהרות למה לי למיתנא מטה. וכ"ת הא קמ"ל דאפילו אדם וכלים נמי מיטמו במעת לעת ולא תימא אוכלין ומשקיל בלחוד הוא דמטמא, א"כ ליתני היתה עסוקה בטהרות ונוגעות בכלים מטה למה לי ש"מ לרבותא לדין משכב נדה הוא דקתני, ועוד דהא לא ס"ד דאוכלין ומשקין מטמי' ולא אדם וכלי דהשתא הני דהפסידן מרובה דלית להו טהרה במקוה גזרו בהו רבנן הני דאין הפסד להן מיבעיא אלא ודאי מתני' מסייעי ליה לזעירי.\par ופי' לטמא אדם לטמא בגדים היינו בגדים שהוא לבוש או שהוא תפוס בהן בשעה שהוא נוגע לטומאה דתניא בת"כ מניין לעשות שאר כלים כבגדים ת"ל טמא יכול יטמא אדם וכלי חרס ת"ל בגד בגד הוא מטמא ולא אדם ולא כלי חרס מדקא ממעטינן כלי חרס ומרבינן שאר כלים כבגדים ש"מ דאפילו מה שאינו לבוש בהן הוא מטמא.\par  ומיהו דוקא שהוא נוגע בהן בשעת נגיעתו לטומאה דדומיא דבגדים ריבה אותן הכתוב אבל לאחר שפירש מן הטומאה אם יגע בבגדים ואפילו לבשן אינן מטמאן והיינו דאמרינן בפרק קמא דבתרא טומאה בחבורן שאני ותנן באהלות פרק קמא אדם ובגדים מטמאי בזב חומר באדם מבגדים ובגדים מבאדם שהאדם הנוגע בזב מטמא בגדים ואין בגדים הנוגעין בזב מטמאין וכל זה דוקא בחבורין כדפרישית. }
\twocol{\textbf{מכדי האי מטה דבר שאין בו דעת לישאל הוא.}  פירש מדקא מקשינן הכי אלמא ברשות היחיד בלחוד הוא דגזור רבנן במעת לעת בספק טומאה שאלו ברשות הרבים אינו חלוק בין בדבר שיש בו דעת לשאין בו דעת דלעולם ספיקו טהור וכמו שפירשתי במשנה.\par  ולפיכך הקשו חכמי הצרפתים היכי תרגימנא בשחברותיה נושאו' אותה אם כן הויין לה אינהו תרתי ואיהי חדא הא תלתא ה"ל רשות הרבים וספיקו טהור דהכי אמרינן בגמרא בריש פרק שני נזירים.\par  ויש מי שתירץ אין רשות הרבים אלא בשלשה אנשים אבל נשים אפילו מאה נמי כאחד דמיין ורשות היחיד הוא מאי טעמא דגמרינן מסוטה מה סוטה אין סתירתה אלא באיש אחד אבל ב' אנשים והיא לא סתירה היא שהרי אשה אחת מתייחדת עם ב' אנשים א) אף רשות היחיד בלא שלשה אנשים אבל נשים אפילו עשר נשים אין אדם מתיחד הילכך הויא לה סתירה ורשות היחיד היא, וזה אינו כלום.\par  ואחרים העמידוה לזו כשהיא ישנה בכילה במטה וחלקה רשות לעצמה ולי נראה שזו היא טמאה ודאי ואין הספק בה אלא שהיא מטמאה אינה נחשבת בכלל המנין אלא הרי היא כגוף השרץ. }
\twocol{\textbf{היה מתעטף בטליתו וטהרות וטמאות בצדו טהרות וטמאות למעלה מראשו.}  יש מפרשים כגון שהוא וטליתו טהורים וטמאות בצדן שראויין לטמא בגדים כגון משכב ומושב ושאר אבות הטומאות ספק נגע טליתו בטמאות ונטמא ונגע בטהרות ונטמאו או ספק לא נגע הטלית לא בזה ולא בזה ספיקו טהור בין בטהרות שהן שתי ספיקות בין בטלית שאינן אלא ספיקא חד. ומסקנא ברה"י ספיקו טמא בשתיהן שהרי שנינו כל שאתה יכול לרבות ספיקו' וספק ספקו' ברה"י ספקו טמא אבל ברה"ר ספקו טהור אפילו הטלית שאין אלא ספק אחד.\par  ואחרים פירשו דאו או קתני היה הוא טהור וטמאות בצדו ולמעלה מראשו או שהיה הוא טמא וטהרות בצדו ולמעלה מראשו וכך פי' רש"י ז"ל. }
\twocol{והא דקתני \textbf{ואם א"א לו אלא א"כ נגע טמא.}  לאו דוקא א"א שא"כ האיך אמר רשב"ג אומרים לו שנה והלא א"א וא"ת רשב"ג ארישא פליג, א"כ ה"ל רבנן לקולא ואיהו לחומרא ואנן איפכא אמרינן לקמן במכילתין דכי אמרי רבנן אין שונין בטהרות לחומרא אבל לקולא שונין אלא ודאי רשב"ג אסיפא פליג דה"ל רבנן לחומרא ולפיכך אמרו אין שונין וא"א לאו דוקא אלא שהדבר קרוב הרבה ליגע ורחוק שלא ליגע ובכיוצא בזה א"א דלאו דוקא לגמרי בפרק כיצד העדים.\par  ומה שכתב רש"י ז"ל אין שונין חוששין שמא עכשו נגע ובתחלה לא או חלוף לאו דוקא דא"כ אפילו לחומר' אלא חוששין שמא עכשיו לא נגע ובתחלה נגע ולא חלוף. }
\twocol{\textbf{ומה כלי חרס המוקף צמיד פתיל וכו'.}  הקשו בתוספות ונימא דיו לבא מן הדין להיות כנדון מהיכא מייתית ליה מכלי חרס מה כלי חרס אינו מטמא אדם לטמא בגדים אף משכב ומושב לא יטמא אדם לטמא בגדים. ולאו קושיא היא דאנן הכי קאמרינן ומה כלי חרס שטומאתו מועט' שהוא ניצל באה' המת גזרו על מעת לעת שלו כנד' עצמה משכבות ומושבות שטומאתן מרובה לכ"ש שנעשו מעת לעת שבנדה.\par  ועוד הקשו דנימא פכים קטנים יוכיח שטמאים במת ואין מטמאים במעל"ע שבנדה כדאמרינן בבבא קמא ופירש רש"י ז"ל שהוא של חרס ואי אפשר ליגע בתוכן ואעפ"י שאפשר בהיסט להכי אפקיה רחמנא להיסט בלשון נגיעה לומר שכל שאי אפשר להטמאות בנגיעה אינו מטמא בהיסט, גם זו אינה קושיא דמה לפכין קטנים שהן טהורין בנדה עצמה תאמר במשכבו' ומושבות דכיון שמטמאין בנדה עצמה עשו מעת לעת כמוה דאשכח' בכלי חרס מוקף צמיד פתיל כ"ש לדעת הגאונים שהן מפרשים פכין קטנים שאינן ראויין לישיבה וטהורין במדרס הזב אבל מן ההיסט אין לך ניצל מהן ולא ממגע תוך כגון בשערו רוקו ומשקה הזב והזבה. }
\newchap{דף \hebrewnumeral{6}}
\twocol{הא דאקשינן \textbf{א"ה ליתנייה גבי מעלות.}  ומפרקינן כי קתני היכא דאית ליה דררא דטומאה היכא דלית ליה דררא דטומאה לא קתני. קשיא עלה והא קתני התם דלית בה דררא דטומאה כדאמר התם בגמרא פרק חומר בקודש (דף כ"א ע"ב) חמש קמייתא דאית להו דררא דטומאה דאורייתא גזרו בהו רבנן בין לקדש בין לחולין שנעשו על טהרות קודש חמש בתרייתא דלית בהו דררא דטומאה מדאורייתא לקדש גזרו בהו רבנן לחולין שנעשו על טהרות הקדש לא גזרו בהו רבנן.\par  וי"ל דהתם דאורייתא לית להו אבל דררא דטומאה דרבנן אית להו הכא אפילו דררא דעלמא מדרבנן ליכא דכל שהוא חששא בעלמא למפרע לאו דררא היא כלל אלא כענין קנסא משום דלא בדקה הפסידוה עונה.\par  וי"מ דהכא הכי פרכינן ליתנייה גבי מעלות קמייתא דאינון בין לקדש בין לחולין שנעשו על טהרת הקדש דאלו בהדי בתריית' כיון דליתנהו אלא לקדש לא מצי למיתנייה דהא מעת לעת שבנדה לחולין שנעשו על טהרת הקדש נמי איתא כדאמר בשילהי שמעתין ולהכי מפרקינן כי קתני בהנהו היכא דאית ליה דררא דטומאה אבל היכא דלי ליה דררא דטומאה ואפ"ה החמירו בהו לא קתני.\par והלשון משובש הוא לדעתי דההוא דאמרינן בשלהי שמעתין לחולין שנעשו על טהרת הקדש לתרוצא לברייתא דקתני לקדש אבל לא לתרומה איתמר אבל השתא להאי לישנא לקדש ולא לתרומה ולא לחולין שנעשו על על טהרת הקדש קאמרינן דאי לת"ה לא הוו צריכין בדר' לתרוצא כדעולא דהא איכא אוכלין חוליהן בטהרת הקדש בימיו ממש אלא ודאי צ"ל להאי לישנא דאף לחולין שנעשו על טהרת קודש לא גזרו במעל"ע שבנדה לכך צריך לשנוי כדעולא כיון שהיו עושין על טהרת הקרש על מנת שהיו מתנסכין ממש על גבי מזבח היינו קדשי מזבח גמורין. }
\twocol{הא ד\textbf{אמר רבי חנינא בן גמליאל מתפלל י"ח מפני שצריך לומר הבדלה בחונן הדעת.}  איכא דמקשו עלה וניכללה מכלל כדמקשינן בגמרא במסכת ברכות (דף כ"ט) דאמרינן כל השנה כולה מתפלל אדם הביננו חוץ ממוצאי שבתות ומוצאי י"ט שצריך לומר הבדלה בחונן הדעת והוינן בה וניכללה מכלל ואסיקנא בקשיא והכא נמי תיקשי וניכללה מכלל.\par  ומתרצין התם כיון דאמר כל השנה כולה מתפלל הביננו מפקע פקיע ליה למכלל כמה דבעי אבל הכא כיון דלא רגיל בהביננו אלא במוצאי יום כפורים בלבד הוא דמתפלל ליה מפני הטורח אי אתי למיכלל ביה מידי אתי למיטעא. עוד י"ל התם והכא קשיא לגמרא ומיהו לא מידחי מימרא בקשיא ואפשר הוה התם למימר ולטעמי' הא דתניא תיקשי לך אלא איכא כמה דוכתי דיכול למימר ולטעמיך ולא אמר. }
\twocol{ הא דאמר רב ששת בריה דרב אידי \textbf{כי קתני מידי דתלי במעשה.}  פירש אליבא דר"ש קסבר האי תנא בוגרת מותרת לכהן גדול א"נ נפקא מינה לכתובה ולא לכהן תניא ומוכת עץ מילתא דתלי במעשה הוא והאי דקתני כל זמן שלא נבעלה לאו דוקא אלא שלא נטלו בתוליה בין בעץ בין באדם. א"נ קסבר מוכת עץ מותרת לכהן גדול וכתובתה מאתים ומחלוקת היא ביבמות ובכתובות. וכן הא דאמר נ"מ לנחל איתן סבר לה כר' יאשיה אשר לא יעבד בו לשעבר ואיתא בפלוגתא בפ' עגלה ערופה. }
\newchap{דף \hebrewnumeral{9}}
\twocol{\textbf{ראתה ואח"כ הוכר עוברה מהו.}  יש להקשות והלא כל מדות חכמים כך הם במ' סאה הוא טובל בחסר קרטוב אינו יכול לטבול אף כאן מכיון שנתנו שיעור לדבר בהכרת העובר ראתה ולא הוכר פשיטא שמטמאה מעת לעת אע"פ שאח"כ הוכר בסמוך.\par ויש לפרש ראתה ואח"כ הוכר עוברה בו ביום ששלמו לה שלשה חדשים מי אמרינן כי מצפרא נמי הויא היכירא ואנן הוא דלא בקאינן או דילמא גבול יש לה וא"ל מידי הוא טעמא אלא משום דראשה כבד עליה בעידנא דחזאי אין ראשה כבד עליה בתמיה הילכך דיה שעתא.\par  וזה הלשון אינו נכון מפני שהיה להם לפרש ובו ביום הוכר עוברה ולימא נמי כיון דבו ביום חזיא לא מטמיא.\par  ויש לפרש אותה כפשוטה ור' ירמיה הכי בעי מיני' גבול שנתנו לה חכמים משיהא ראשה ואיבריה כבדין והיינו משעת הכרת העובר וקרוב לו מלפניו כשהיא מרגשת בעצמה או דילמא גבול שנתנו לה הכרת העובר ממש הוא וקודם לכן אפילו בסמוך אינו מטהרתה וא"ל אף איבריה אינן כבידין עליה אלא משעת הכרת העובר ממש שאין הולד חי ומכביד עליה אלא מזמן זה ואילך וזה הלשון עיקר. }
\twocol{ גרסת ר"ח ז"ל \textbf{רבא אמר רב חסדא שלשה ועשרים יום ולא פליגי מר וכו'.}  ופשוטה היא. והספרים גורסין כגרסת רש"י ז"ל כ' יום ופירש דלא פליגי מר חשיב ימי טומאה ז' ימי נידה וג' ימי זיבה ומכאן ואילך כיון דבעינן נקיים ימי טהרה הם ולפיכך לא מנה אלא עשרים שדבר בסתם נשים שהן טמאות לנדה ולזיבה. }
\twocol{\textbf{ועוד עברו עליה ג' עונות וראתה דיה שעתה.}  י"מ דהא מני רבי היא דאמר בתרי זימנא הויא חזקה והא דבעינן הכא עד תלתא משום שכל שעברו עליה ג' עונות סמוך לזקנתה הוחזקה בזקנה שדמיה מסולקין ואין ראיה זו מוציאה מכלל זקנו' שכן דרך סלוק דמיהן של זקנות מפסקת ורואה ופוסקת ושוב אינה רואה וכשהיא רואה פעם לסוף ג' עונות עכשיו הוא שמתחלת להחזיק עצמה ברוא' ואפילו בזקנתה לפיכך דיה שעתה בראיה זו שהיא תחלת למניין ובשנייה שהיא שלישית הוחזקה ומטמאה מעת לעת ומתני' דקתני במה שאמרו דיה שעתה בראיה ראשונה אבל בראיה שניה מטמאה מעל"ע התם כשקרבה ראיתה בשנייה דאיגלא מילתא דג' עונות קמייתא לאו משום סלוק דמים הוו ולא הגיעה זו לכלל זקנה. אבל רחקה אף ראיה שניה זקנה היא אלא שרואה וצריכה חזקה, וזה לשון נכון.\par  ול"נ דזקנה צריכה להחזיק עצמה בראיות והא דקתני מתני' אבל בראיה שנייה מטמאה מעת לעת אבתולה קאי והוא דאיתמר בגמרא עלה רב אמר אכולהו לומר שכולן ישנן בדין הזה שאם החזיקו עצמן בדמים מטמאות מעת לעת, ושמואל אמר ל"ש אלא בתולה וזקנה שישנן בדין הזה שמחזקות עצמן בראיות ומטמאות אח"כ מעת לעת, אבל מעוברת ומניקה דיין שעתן כל ימי מניקותיהן ועוברן ואפילו הן שופעות פעמים הרבה. ומיהו מתני' דקתני דבראיה אחת הוחזקה לדמים ודאי הכל מודים דלאו אזקנה קאי דזקנה הא קתני לה בבתרייתא בתרי זימני אלא בדין הזה להחזיק עצמה כשאר הנשים מעת לעת וזה פירש מדוקדק.\par  וי"מ אותה לרשב"ג דבתלתא זימני הוי חזקה ומטמאה מעת לעת כשור המועד מה שור המועד בשלשה זימני אתחזק, ואידך כי נגח משלם אף זו בג' פעמים הוחזקה הילכך מטמאה בשלישי עצמה מעת לעת ולהאי פי' אמרינן דמתני' דקתני אבל בראיה שני' מטמאה מעת לעת סתמא כרבי דהא זקנה בכללה לדברי הכל ואינו מחוור. ועוד דשטתא דוסתו' לא מוקמינן להו כרבי דלקמן תנן סתמא כרשב"ג וביבמות אמרינן וסתות ושור המועד כרשב"ג.\par  ולכל הלשונות נמי קשיא כיון שהוחזקה זו ולבסוף דלאו סלוק דמים הוה בה כלל נטמא למפרע מעת לעת שבכל ראיותיה והשיב רש"י מעת לעת דרבנן הוא וכל שבשעת ראיתה בחזקת טהרה אין מחמירין לטמא אותה למפרע. ואם תשאל הרי הקשו למעלה גבי היתה בחזקת מעוברת וראתה אמאי [אין] מטמאין אותה למפרע לכשהפילה רוח וזו אינה קושיא דהאיכא רב פפא דתריץ הכי הנח מעת לעת לרבנן ולרב פפא שאני התם דאיגלאי מילתא דלאו עובר הוא אבל הכא אכתי איכא למיתלי מעיקרא לא שכיחי בה דמים והשתא הוא דאכחיש' ואיתרעי א"נ התם בסמוך לראיה הפילה דאפשר לטמויה אבל לאחר עונות ליכא למימר הכי. }
\twocol{ הא דתניא \textbf{תנוקות שלא הגיע זמנה לראות וכו'.}  יפה פי' רש"י ז"ל דהיינו טעמא דלא מטמיא מעת לעת אלא בשלישית לרבי וברביעית לרשב"ג מפני שכל אשה שלא הוחזקה כבר ברואה אינה מטמאה מעת לעת דטעמא דמעת לעת כעין קנסא דרבנן הוא כדאמרו חכמים תקנו להן לבנות ישראל שיהיו בודקות עצמן שחרית וערבית וזו הואיל ולא בדקה הפסידה עונה יתירה הילכך כל שאינה צריכה לבדוק עצמה כלל אינה בכלל מעת לעת שבנדה ותנוקות שלא הגיעה זמנה לראות הרי הן בחזקת טהרה כדתניא לקמן ואין הנשים בודקות אותן הילכך פעם ראשונה ושניה שעדיין לא הוחזקה לראות ולא היתה בכלל תקנה לבדוק שחרית וערבית דיה שעתה. וכיון שראתה בשניה הוחזק ברואה לדברי רבי הילכך בג' מטמאה מעת לעת שהרי היתה צריכה לבדוק שחרית וערבית, ואע"פ שעדיין לא הגיע לכלל שנותיה, וכיון שלא בדקה הפסידה עונה יתירה ושהגיע זמנה לראות כיון שצריכה בדין היה לגזור עליה אפילו בראשונה אלא שהיא קולא לדבריהם. עברו עליה ג' עונות חזרה לכלל תנוקות שלא הגיע זמנה עד שתראה שתים ותהא מוחזקת לראות לדברי רבי דשוב צריכה בדיקה ומטמאה מעת לעת.\par  והא דאמרינן לקמן (דף י' ע"א) בין שניה לשלישית כיון דלא אתחזק בדם כתמה נמי לא מטמינן לאו אליבא דהך ברייתא דרבי אלא אליבא דהילכתא כרשב"ג. וכן פסק הר"ם ז"ל דקטנה כתמה טהור עד שתראה דם ג' וסתות.\par  וחזקיה סבר כיון (דחזיא) [דאלו חזיא] הרי היא כשאר כל הנשים אח"כ כתמה מחזיקה וטמא דבכל (מראיה שניה) [מראות משניה] ואילך הוחזקה.\par  אבל רש"י ז"ל פי' אליבא דברייתא [דרבי] ומאי כיון דלא איתחזק בדם שעדיין לא הוחזק' בה לטמא מעת לעת ולפי דבריו ז"ל ולדידן דקי"ל כרשב"ג אין כתמה טמא עד שיעברו עליה ד' וסתות. לראיה פעם ראשונה [דרב גידל] פירש רש"י ז"ל דהיינו [ראשונה שאחר ההפסקה ושניה היינו] ראיה שניה של דלוג שהיא ראשונה לראיה דעונות. ולפי שהיא עומדת בה כשרואה עכשיו בעונו' קרי לה הכי.\par  ויש לפרש "הדר קחזיא בעונות" [דקאי גם על] פעם אחרת בין דלוג ראשון ושני ראתה פעם אחת בעונה ובין שני לשלישי חזרה וראתה עוד בעונה ותרתי בעיי אהדדי איתמר ורב אשי בעא [לאפסןקי ולמפשט חדא חדא] מיניה דסד"א כיון דראתה שתים בדילוגו ושתים בעונות סלוק דמים הוא דקא מנע מינה עונות ולא תהא מוחזקת לא לדלוג ולא לעונות דכיון דשנתה כ"כ אונס בעלמא הוא. ואמר רב גידל פעם ראשונה של עונות ט) דיה שעתא כדאמרן שניה של עונות י) כיון דראיה שלישית הוא יא) מכי חזיא בעונות ואילך לעולם מטמאה מעת לעת. ומיהו בראיה (ג') [ראשונה] שלה לא מטמיא מעל"ת משום לסוף ג' עונות חזיא ואכתי לא הוחזקה ג' פעמים להפסקה דהא אנן לר' אלעזר קאמרינן ולישנא דהדרא קא חזיא דייקא כדאמרן. }
\newchap{דף \hebrewnumeral{11}}
\twocol{\textbf{אלא לקפיצות והתניא.}  פי' רש"י ז"ל דכל יום שתקפוץ מחזיקין לה רואה ואפילו בשאר ימות השנה ואין פירו' זה נכון דאי מעיקרא קס"ד דלקפיצות לחודייהו תקבע וסת הכי הוה לן לתרוצי בברייתא לא קבעה לה וסת לימים אבל קבעה לה וסת לקפיצות לחודייהו דמאי דקא אמרינן מעיקרא משנינן ועוד כי מקשינן לימים לחודייהו פשיטא לימא ליה לקפיצות לחודייהו קמ"ל.\par  אלא ה"פ אלא לקפיצות בימים קבעה והתניא אינה קובעת ומתרץ אינה קובעת לימים ולא לקפיצות לחודייהו פשיטא היא כדפרישית. }
\twocol{ הכי אשכחן בנוסחי: \textbf{לימים לחודייהו פשיטא אמר רב אשי כגון דקפץ בחד בשבת וחזאי וקפץ בחד בשבא וחזא ולשבתא נמי קפצה ולא חזאי מהו דתימא איגלי מילתא דיומא הוא דגרים קמ"ל דקפיצה נמי גרמא ומשום דאכתי לא מטאי זמן קפיצה.}  וק"ל כיון דלימים לחודייהו פשיטא ליה וה"ה לקפיצות לחודייהו נמי דפשיטא ליה דלא קבעה כדפרישית קפצה בשבא ולא חזאי מאי מהני לן פשיטא ודאי דלא תיחזי אלא בימים וקפיצה ועוד מאי קא מקשי' ומאי קא אתי רב אשי לחדותי הא מימר קאמרינן בברייתא דלא קבעה וסתות לקפיצות לחודייהו. ואי תקפוץ בשבת לא תיחזי והלכך איצטריכא ליה לאשמועינן ימים לפום מאי דקא משנינן ואדרבא פשיטא דבעיא ימים וקפיצה.\par  ונראה שרש"י ז"ל גורס ולשבתא קפצא ולא חזיא ולמחר חזאי בלא קפיצה ומהו דתימא איגלאי מילתא דיומא הוא דגרים ולא קפיצה דהא בקפיצה בלא יום לא חזאי וביום בלא קפיצה חזאי קמ"ל דקפיצה דאתמול גרמא לראיה דהאידנא ומשום דאכתי לא מטאי זמן קפיצה לא חזא מאתמול וזהו הנכון.\par  ומיהו לישנא אחרינא אמרי לה להא דרב הונא ולא ידעינן אי פליגן לישני ולמדחי' לקמא איתמר בתרא או דילמא אע"ג דלאו הכי איתמר אלא האי תרווייהו איתנהו לענין מעשה ומסתברא כיון דוסתו' דרבנן לקולא נקטינן בהו והלכתא כתרי לישני ולקולא, ואחר שכתבתי זה מצאתי להרמב"ם פאסי ז"ל שהחמיר ובטלה דעתינו מפני דעתו. }
\twocol{ מתניתין \textbf{צריכה להיות בודקת וכו' ומשמשת בעדים וכו'.}  פירש מתני' פרושי קא מפרש לה ואזיל וכיצד קתני כיצד צריכה להיות בודקת פעמים ביום שחרית וערבית ואע"פ שלא שמשה כלל וכיצד משמשת בעדים בודקת נמי בשעה שהיא עוברת משאר עסקיה לשמש את ביתה ומשמשת בעדים וע"כ מדקתני בשעה שהיא עוברת לשמש היינו עד שלפני תשמיש וש"מ דצריכה בדיקה לפני תשמיש והעד (הג') [הב'] לפני תשמיש אי אפשר אלא לאחר תשמיש הוא וכדתנן אחד לו ואחד לה אלמא צריכה בדיקה בין לפני תשמיש בין לאחר תשמיש.\par  והיינו דאמרינן לעיל [דף ה' ע"א] שתי בדיקות אצרכוה רבנן חדא לפני תשמיש וחדא לאחר תשמיש ורמינ' למתני' דקתני והמשמשת בעדים הרי זו כפקידה דהיינו עדים דקתני דאינון לפני תשמיש ולאחר תשמיש דומיא דמשמשת בעדים דהך סיפא [וכי תריץ] נמי לעיל גבי רישא דמתני' מעיקרא [אידי ואידי לאחר תשמיש] משום דקשיא להו קס"ד לפרושי ההיא דשני עדים דבסוף קא חשיב אבל בהך סיפא דכ"ע עד שלפני תשמיש קתני וכדמפרש עלה לקמן בגמרא. }
\twocol{\textbf{מדאמרינן הכא מימי טהרה לימי טומאה לא קבעא.}  נ"ל דאשה קובעת וסת בימי מניקתה דלא ממעטינן הכא אלא ימי טוהר דידה, וה"ר אברהם ז"ל היה אומר שאין קובעת לא בימי עוברה ולא בימי מניקתה ואי מוקמת שמעתין במפלת שאין לה חלב אכתי קשיא דמידי הוא טעמא אלא משום דראשה כבד עליה וכו' כל היולדות בכלל הן. }
\twocol{\textbf{דמגו דבעיא בדיקה לטהרות בעיא נמי בדיקה לבעלה.}  פירש אע"פ שתקנו חכמים לבנות ישראל לבדוק שחרית וערבית להכשיר הטהרות עוד החמירו עליהן שאם שמשו מטתן יהו צריכות בדיקה לטהרות חוששין שמא ראתה מחמת תשמיש ובדיקה שניה שהצרכוה סמוך לתשמיש מאחריו בתוך שיעור כדי שתרד מן המטה א"נ באחר אחר כפירקן דכל היד קודם שתלך או שתקנח וכן הצריכו לאיש עצמו לקנח בעד ולבדוק והכל משום חומר הטהרות שרגילה לעסוק בהן ומתוך חומר הטהרות החמירו לבעלה שתהא צריכה לבדוק לפני התשמיש וכ"ש לאחר תשמיש ואע"פ שאין דעתה לעסוק בטהרות עכשיו מאחר שהורגלה לעסוק בהן והוחזקה להיות בודקת לטהרות לאחר תשמיש.\par  נמצא שאין כאן בדיקה מן הדין אלא שלאחר תשמיש ולטהרות והשאר מדין מגו וכיון שאינו אלא משום טהרות אינה צריכה אלא עדותו של עד כלומר שמקנחת בעד לפני תשמיש ולמחר בודקת בו אם מצאה עליו דם טמאה לטהרות ומחייבת בעלה בחטאת. אבל מותר הוא לבעול משבדקה בעד ואע"פ שאינו מועיל לו עכשיו שהרי אינן יודעין אם ראתה אם לאו זהו דרכו של פי' רש"י ז"ל.\par  וי"ל דכל גבי בעלה צריכה להיות בודקת ורואה ואח"כ תשמש דאין בדיקה סתם בכל מקום אלא במקנחת ורואה מה העד מעיד בדבר דאי לא תימא הכי כל הנשים בחזקת טהרה לבעליהן הן ואפילו בעסוקה בטהרות אלא משמע דכל לפני תשמיש בודקת ורואה לגבי בעלה והכל ודאי משום מגו דטהרות והיינו דאמרי (לעיל) [לקמן ע"ב] אימר שמש עכרו, ז) וכיון שאינה צריכה בדיקה לאחר תשמיש לטהרות אף לפני תשמיש לבעלה אינה צריכה דהא ליכא מגו כך פי' הרב ז"ל. }
\twocol{\textbf{חדא מכלל דחברתא איתמר.}  נראה דהא דרב יהודה איתמר מכללא דההוא דכיון דשמעיה רבה בר ירמיה לשמואל דאמר אשה אין לה וסת אסורה לשמש עד שתבדוק בעסוקה בטהרות וש"מ נמי דכל לבעלה לא בעיא בדיקה כלל מדלא מתרץ לה לר' זירא אין לה וסת אפילו לבעלה בעיא בדיקה מ"ה דייק רב יהודה דשמעא מיניה ואמרה משמיה דשמואל דמתני' דוקא בעסוקה בטהרות היא דמכלל היא דמתני' ליכא למימר [דוקא ביש לה וסת ולא] אין לה וסת ודוקא עירה כדאיתמר לקמן.\par  וי"מ הא דאמרינן חדא מכלל דחברתא איתמר לאו אעיקר מימרא דשמואל אלא ה"ק שמואל שמעתא דהכא לא שאנו אלא בעסוקה בטהרות, ואמ' תו במימרא דלקמן דרבא בר' ירמיה ומכללא דהך אוקימנא לההוא שעסוקה בטהרות ולאשמעינן עירה וישנה אתמר ואי לא איתמר הך לא הוה ידעינן ההיא דבעסוקה בטהרות היא כדאמר רבא לקמן וכי אמרינן חדא מכלל חברתה איתמר אאוקמתין דעסוקה בטהרות קאמר וכי אמרינן [ואוקימנא אאוקימתא דרבה בר ירמיה] קאמרינן דאוקמתין קשי' ואוקמתין מתרצין. }
\twocol{הא דאמרינן \textbf{תנ"ה בד"א לטהרות אבל לבעלה מותרת וכו'.}  היינו טעמא דמשמע לן אפילו כשאין לה וסת משום דמתניתין בין שיש לה וסת וכו' ועלה קתני בד"א לטהרות אבל לבעלה מותר בין בזו בין בזו משמע ועוד דכל עיקר לא הוצרכה ברייתא זו לשנותה אלא בשאין לה וסת שאלו בשיש לה מתני' היא כל הנשים בחזקת טהורות לבעליהן. }
\newchap{דף \hebrewnumeral{12}}
\twocol{ והא ד\textbf{בעיא מיניה ר' זירא מר' יהודה מהו שתבדוק ותבדוק ומה בכך.}  לפני תשמיש קאמר ולהחמיר על עצמו היה רוצה שאע"פ שאינה עסוקה בטהרות יהא נוהג חומר כעסוקה בהן ואמר לאו שא"כ לבו נוקפו ופורש, ור' אבא בעיא מרב הונא לאחר תשמיש ולהחמיר שלא מן הדין בשאינה עסוקה ואיהו נמי [אלא דלא] מבעיא שלא יהא לבו נוקפו ופורש.\par  ופיר' הענין שהיו השואלין סבורין שבדיקות הללו של טהרות יש לחוש לספיקן וזה שלא הצריכוה חכמים אלא לטהרות קולא היא לגבי הבעל שלא להאריך עליהן את הדרך ומי שמחמיר על עצמו נקרא צנוע וכשר ופשטו להן שבדיקות של טהרות חומר הוא שתקנו בהן חכמים כדי שתהא היד מרבה לבדוק בנשים ומשובחת ולא שיהא מקום ספק לחוש להן ואף הרוצה להחמיר אין רוח חכמים נוחה הימנו מפני שלבו נוקפו ופורש ומבטל פריה ורביה בישראל.\par  ולדברי רש"י ז"ל לבו נוקפו בבדיקה שלפני תשמיש לפי שהיא בודקת ואינן יודעין מה מצאו עד למחר ונמצא בועל על הספק ולבדוק ולראות לא עלה על דעתו שאפילו בטהרות לא אמרו כן ושלאחר תשמיש נמי לבו נוקפו בו ופורש.\par ולפי דברינו [במתניתין] אפי' (בבדיקות) [בבודקות] ורואה לפני תשמיש כיון שאין אתה מחזיקה בטהורה לבו נוקפו בחששות ובחומר בדיקת חורין וסדקין ואינו סומך בבדיקת אור הנר וכל זה וכיוצא בו גורמין פרישה הן.\par ובשם ר"ח ז"ל מצאתי שאומרים דמ"ה לבו נוקפו דסבור אלמלא שלא הרגישה לא בדקה כלומר דהא קים לה דכולה לבעלה לא בעי בדיקה, וזה אינו סבור שמחמרת אלא שבודקת משום הרגשה וחוששין כיון שהרגישה ודאי בא אורח וטפה כחרדל היתה ואבדה בעד. }
\twocol{\textbf{א"ר אמי א"ר ינאי וזהו עדן של צנועות.}  פי' ודאי משנתינו עד שלפני תשמיש ולאחר תשמיש קתני ובעסוקה בטהרות וא"ר ינאי עד זה שלפני תשמיש זהו עדן של צנועות דקתני מתני' בפרק כל היד והקשו לרבי אמי הא מתניתין צריכות קתני כדתנן צריכה להיות בודקת ומשמשות בעדים ופריק שאני אומר כל המקיים דברי חכמים נקרא צנוע.\par  ולדבריו של ר' אמי פיר' משנתינו שבפרק כל היד כך הוא דרך בנות ישראל שתקנו להן חכמים להיות משמשות בשני עדים אחד לו ואחד לה לאחר תשמיש ואם לא בדקו או שאבדו עידיהן אסורות לשמש עד שיבדקוהו שמא מחמת תשמיש ראתה והצנועות שמקיימות דברי חכמים מתקינות שלישי אחר לתקן את הבית לבעליהן שכך הצריכו אותן חכמים אלא שלא אסרו להם לשמש אם אבד עד זה או (שאפשר) [שאי אפשר] להן לבדוק לאור הנר דכיון שאין עדות בכל מקום אלא עד של מגו לא החמירו בו כשנאחר תשמיש לשהוחזקה בו לטהרות מן הדין.\par  ואקשיה ליה רבא לרב אמי ופרכיה ופריש רבא הא דקאמר ר' ינאי וזהו עדן של צנועות לומר שבעד זה הוא צניעות הצנועות ששנינו במשנתינו שעד שבודקין בו לפני תשמיש זה אין בודקות בו לפני תשמיש אחר, ולדברי רבא פי' משנתינו כך הוא דרך בנות ישראל שתקנו חכמים משמשות בשני עדים אחד לו ואחד לה שלו מקנח בו לאחר תשמיש שהרי אין לו בדיקה אחרת ושלה בודקת בו עצמה כל זמן שהיא צריכה לבדוק דהיינו לפני תשמיש ולאחר תשמיש כדתנן הכא והצנועות מתקנות להן שלישי אחר לתקן את הבית כלומר חדש ולבן, והיינו דקתני תיקון והיינו נמי לשון שלישי לומר שאם רצו לשמש פעם אחרת למחר מתקינות להן שלישי שלא נשתמש בו כלל אפילו לפני תשמיש.\par  ורש"י ז"ל מפרש שני עדים אחד לו ואחד לה לאחר תשמיש, ולפי דבריו הכי מתרצא מתני' והצנועות מתקנות להן השלישי שהן צריכות א' לתקן הבית ומהו תקונן שלא נשתמשו בו ואפילו לפני תשמיש.\par  ולדברי הכל משנתינו בעסוקה בטהרו' וכדאוקי שמואל להא מתניתין דפירקן דתרווייהו בני חד ביקתא אינון וכדקתני רישא דההיא ואוכל' בתרומה ועלה קתני דרך בנות ישראל וכו', ולפום הכי קתני סיפא כל הנשים בחזקת טהרה לבעליהן כלומר אע"פ שהצריכוה בדיקה לא אמרן אלא לטהרות אבל לבעלה בחזקת טהרה הן. }
\twocol{\textbf{והא שמואל במאי מוקים לה.}  וא"ת כשאין עסוקה בטהרות ולבעלה ואפילו אין לה וסת א"ב מאי איריא חמרין ופועלין אפילו עומדין בעיר נמי ועוד מאי קמ"ל הא קתני לה אידך לעיל בד"א לטהרות וכל זה אינו מחוור דאיכא למימר ישנות קמ"ל ואכתי נמי לא קים לן דאיכא חלוק בין בא מן הדרך לשוהה בעיר אלא א"ל מדקתני סתמא ש"מ אפילו בעסוקה בטהרות הוא דכולהי סתמי לטהרות ולבעלה פרושי מפרש לה בבמה ד"א א"נ דאינהו בטויי מיבעיא ליה וגמר' מתרץ דעדיף מיניה דקאמר דדוקא נקט חמרין ופועלין ואוקימנא בשיש לה וסת. }
\twocol{\textbf{וכיון שתבעוה אין לך בדיקה גדולה מזו.}  פירש רש"י ז"ל דסתם הבא מן הדרך דרכו לפייס ולרצות ולתבוע וכי מרצו קמה ותבע רמיא אנפשה ואי הוה חזיא מרגשה. וכי אמרינן דבעינן בדיקה בשוהה עמה שאינו צריך ריצוי כ"כ ומיהו הניח בחזקת טומאה אף ריצוי לא מהני ליה עד שישמע מפיה טהורה אני והא דשאל רב כהנא אינשי דביתהו דרבנן לומר אם מחמירן על עצמן לבדוק בשאינן עסוקות בטהרות דומיא דבעיא דר' זירא דלעיל והאי דנקט כי אתו מבי רב אורחיה דמילתיה נקט שיוצאין ובאין מערב שבת לע"ש.\par  ויש לפרש דישינות בעיא מנייהו אם מחמירין בהן בבאין מן הדרך משום דכיון דאין בעלה עמה לא קפדה אנפשה ואמרו להן לאו, נמצא כלל השמועה הלכה למעש' שכל לבעלה לא בעי בדיקה לא לפני תשמיש ולא לאחר תשמיש ואפילו כשאין לה וסת לפי פירושו של רש"י ז"ל בדברי רבי חנינא בן אנטיגנוס.\par  אבל מדברי הרמב"ם הספרדי ז"ל למדנו שיש לו דרך אחרת בשמועה זו שהוא מפרש זו ששנינו דרך בנות ישראל לבעלה בשאין לה עסק בטהרות והצנועות בודקות אף לפני תשמיש לבעליהן וכל מה שהקל ר' יהודה ור' זירא משמיה דשמואל אינו אלא בבדיקה זו שלפני תשמיש שכשהן עסוקות בטהרות אפילו שאינן צנועות צריכות. וכשאין עסוקות בה הרי הן בחזקת טהרה לבעליהן לפני תשמיש.\par  וטעם לדבריו מפני שלפני תשמיש אשה מרגשת בעצמה ואפילו בישינה נמי הקלו מפני שבחזקת טהרה הן ולאחר תשמיש חוששין שמא ראתה מחמת שמש ואינה מרגשת.\par וההיא דבעיא מיניה ר' אבא מרב הונא צריך הוא לפרש שלא מנעו אלא מלבדוק בשיעור וסת ואח"כ כדי שלא תתחייבנו באשם תלוי ויהא לבו נוקפו אבל לאחר אחר בודקת קודם שתלך ותקנח ואף על פי שהוא בודק בשלו לחטאת שאני התם דא"א בבדיקה שלא תחייבנו חטאת והוא צריך בדיקה מ"מ שמא ראתה מחמת תשמיש אבל בשלה אפשר לבדיקה זו לאחר זמן של אשם תלוי שלא יהא לבו נוקפו בביאה זו ותהא מתוקנת בביאה אחרת, וכן דברי ר' זירא לר' יהודה כך הן מתפרשין לי מהו שתבדוק עצמה לבעלה לחייב בעלה שאם לפני תשמיש והלא בצנועות (הוא) [במתני' קתני] לה וזהו שאמר רבי ינאי זו עדן של צנועות לא צנועות שנשנו בפרק כל היד אלא ר' אמי פירש לומר שכל העושה כן נקרא צנועה, ורבא פירש לומר דבעד זה נכרת אם צנועה היא אם לא אבל לדברי הכל מתני' צריכות קתני ובעסוקה בטהרות ואלו בשאינה עסוקה כבר שנינו והצנועות מתקינות וכו'.\par ושאר השמועה פשוטה היא לפי דרכו לפיכך כתב אינה צריכה עד שלפני תשמיש אלא משום צנועות אבל לאחר תשמיש הכל צריכים שני עדים אחד לו ואחד לה אפילו מעוברת ומניקה זקנה וקטנה האריך עלינו את הדרך.\par  אבל דברי רש"י ז"ל יותר נכונים ומוכרעים בכמה מקומות בשמועה והחכם יבור לעצמו, ודברי רבינו יצחק אלפסי ז"ל שנראין נמי כדברי רש"י ז"ל שהוא כתב בהלכות ברייתא זו דתניא החמרין והפועלין והתיר בין עירות בין ישינות ולא הזכיר משניו' הללו של שני עדים בשאין לו וסת אלמא אין לנו עדים אלא לטהרות. }
\twocol{הא דאמר \textbf{ר' חנינא בן אנטיגנוס משמשת בשני עדים והן עיוותיה ותיקוניה.}  לפי פי' רש"י ז"ל שני עדים א' לפני תשמיש וא' לאחר תשמיש ואין עוות ותיקון אלא לחייב בעלה בחטאת ואשם תלוי או לטמא מעת לעת ולתקן אותם כדתנן בפירקן דכל היד והיינו דמקשינן אי בעסוקה בטהרות הא אמרה שמואל חדא זימנא אי בשאינה עסוקה למה לה דאמר ר' זירא וכו' ומהתם ש"מ תרתי בעסוקה שצריכה ובשאינה עסוקה שאינה צריכה ופריק מאן דמתני הא לא מתני הא, וא"ת הא ר' יהודה אמרה לעיל א"ל ההיא מכללא איתמר כדאמרי' לעיל כלומר ששמע השומע לשמואל שאמר הלכה כר' חנינא בעסוקות בטהרות ופי' בשמו שאין משנתינו אלא בעסוקה בטהרות ור' זירא שנאה מימרא לעצמו בלשון אחר, ורש"י ז"ל תירץ את זה בע"א.\par  ומדברי הר"ם ז"ל שהוא מפרש מאן דמתני הא לא מתני הא לימא לעולם בשאינה עסוקה בטהרות ואפ"ה כשאין לה וסת צריכה בדיקה ואמוראי נינהו ואליבא דרב יהודה והא דלא מקשינן דידיה אדידיה משום דלא מתפרש להו בהדיא ההיא באשה שאין לה וסת.\par ופסק הרב ז"ל כמאן דמתני הלכה כר' חנינא בן אנטיגנוס שכל אשה שאין לה וסת משמשת לעולם בשני עדים א' לפני תשמיש ואחד לאחר תשמיש בין לו בין לה.\par  ורבינו הגדול ז"ל הצריך לה בדיקה ג' פעמים עד שתהא מתוקנת והוחזקה שלא תראה מחמת תשמיש ומשם ואילך הרי היא כשאר כל הנשים ואינה צריכה כלום ועל כן יאמר בספר מלחמות ה' ג) דמשמשת בעדים היינו אחד לו ואחד לה ושלה בודקת בו לפני תשמיש ולאחר תשמיש כענין משנתינו וכמו שפי' למעלה והי עוותיה כשהורעא לה ראיה בתוך תשמיש ג' פעמים ותקוניה בשלא אירע לה כלום.\par  ואקשינן למה ליה לשמואל הלכה כר' חנינא בן אנטיגנוס אי בעסוקה בטהרות מוקי לה ומשמשת בעדים לעולם משום הטהרות והם מעותין אותה בג' פעמים של ראיה להוציא ומתקנין אותה כשהוחזקה שלא לראות ויצתה מחששא הא אמרה שמואל חדא זימנא דרב יהודה גופיה אמרה לעיל ואי בשאינה עסוקה בטהרות ואפ"ה בעיא תיקון ג' פעמים הא אמר כל לבעלה לא בעיא בדיקה ואפילו פעם אחד דאמר ר' זירא אשה שאין לה וסת אסורה לשמש עד שתבדוק ואוקי' בעסוקה בטהרות ומדלא אמר עד שתבדוק ג' פעמים לבעלה ש"מ דכל לבעלה לא בעיא כלום כדדייקינן לעיל.\par  ומפרקינן לעולם בשאינה עסוקה ואמוראי נינהו אליבא דשמואל הא רב יהודה והא ר' זירא ואידך דאמר א) לעיל נבעלה מותרת קמ"ל דמשהוחזקה ואילך אינה צריכה כלום, ב) זהו פי' השמועה לדעת רבינו הגדול ז"ל וראוי הוא לסמוך עליו. }
\twocol{\textbf{לא פירות ולא מזונות ולא בלאות.}  פירש שאינה יכולה להוציא ממנו פירות שאכל דמחילה בטעות שמה מחילה והכי מפרש בגמרא פרק איזהו נשך, ולא מזונות שאינו משלם מה שלותה ואכלה, ולא בלאות של נכסי צאן ברזל ודוקא שאינן קיימין.\par  ועל הני הוא דמפרש בגמרא מ"ט תנאי כתובה ככתובה דמי וכיון שכתובה קבל עליו נכסים הללו כצאן ברזל וקבל עליו מזונות ואין לה כתובה פטור הוא מכולם אבל הקיימים נוטלת ויוצאה שאפילו זנתה נוטלת מה שבפניה ויוצאה ודינה של זו כדין איילנות שנוטלת הקיימים בשל ברזל ומפסדת שאינן קיימים, ובשל מלוג דינה כשאר הנשים לדעת רבינו ז"ל בכתובות, וכן היא נוטלת לדעתי ודעת הגאונים תוס' כתובתה וכבר פי' בפרק אלמנה נזונות ומה שכתב רש"י ז"ל בכאן אינו נכון }
\twocol{ והא דאמר רבי מאיר\textbf{יוציא ולא יחזיר עולמית.}  משום קלקולא נ"ל והוא שאמר לה משום שאין לך וסת אני מוציאך ואם לא מפני כן לא הייתי מוציאך ואם לא אמר כן אין כאן חשש לקלקו' כדתנן המוציא אשתו משום איילנות רבי יהודה אומר לא יחזיר וחכמים אומרים יחזיר ואוקימנא מאן חכמים ר"מ ומשום דלא כפליה לתנאיה והא נמי לההיא דמיא ולדידן נמי לא בעיא כפילא כרבנן והוא שאמר סתם משום כך אני מוציאך ואף על פי שלא כפל הא גירש סתם יחזיר כדאמרינן התם גבי איילנות ומוציא משום נדר ומשום שם רע ויש לומר שאפילו לא התנה ולא אמר כלום יש לחוש לקלקל דבשלמא התם אם לא התנה כלום י"ל עילה הוא רוצה לגרש שכמה אנשים נשואים לאיילנות וע"י שיש בהם נחת רוח מהן מקיימין אותם וכך אמרו שם בירושלמי אבל זו שאסורה היא לשמש כלל בידוע שאין בעלה מגרשה אלא מחמת פיסול זה וכן במוציא משום שם רע י"ל מכיון שלא שהה לראו' אם הדברים נראין עילה מצא וגורש לפיכך אין חוששין לקלקול אלא שאמר משום שם רע אני מוציאך. }
\newchap{דף \hebrewnumeral{13}}
\twocol{\textbf{נשים דלאו בנות הרגשה נינהו משובחת.}  ק"ל לר"ת ז"ל למה ליה האי טעמא תיפוק לי' משום דלא מיפקדי אפריה ורביה ושמעתי שהיה מפרש דלאו בנות הרגשה נינהו שאינן בדין הרגשה כלל לומר שאינן מרגישות ולא מצוות. וכן פי' זו ששנינו בברייתא שלשה נשים משמשת במוך חייבות לשמש דאלו מותרות כל הנשים מותרת כן. ואין פי' נכון בכאן דבנות הרגשה לאו בדין הרגשה משמע.\par ואפשר לפרש דמשום משובחת קאמר שאלו היו בנות הרגשה אע"פ שאינן מצוות כאנשים ולא דינן ליקצף מ"מ לא היתה יד המרבה לבדוק יותר מדאי משובח' לפי שהיא משחיתה ואין שבח בהשחת' אפילו לנשים ועוד דהא מביאה עצמה לידי הרהור ואלו היתה בה הרגשה בת נדוי היא כדלקמן ולפיכך הוצרכו בגמ' לפרש דלאו בנות הרגשה נינהו.\par אבל כל עיקר אין דינו של הרב ז"ל נראה לי שאע"פ שאינן מצוות על פריה ורביה ורשאי מן התורה לבטל איסור הוא בהשחתה. ואע"פ שהאשה מותרת לעקור את עצמה מה שאין כן באיש ואע"פ שקיים מצות פריה ורביה התם מצוה אחריתי היא שנצטוו על הסירוס ואפילו מסרס אחר מסרס חייב אבל בהשחתה כל בשר כתיב. }
\twocol{הא דאקשי' בכולה שמעתין מדר"א דאמר \textbf{כל האוחז באמה וכו'.}  י"ל דהכי אקשינן וע"כ לא פליגי רבנן דאמרו לו עליה דר"א אלא בדליכא עפר תיחוח ולא מקום גבוה ומשום חשש פסול המשפחות אבל במקום אחר מודו ליה. הילכך גבי תרומ' ה"ל למימר שיפלוט ואע"פ שמפסיד' וכן בדרב יהודה שיטה וירד חוץ לכנישתא וישתין. וי"ל דא"ל לאו פלוגתא היא אלא בשואלין לפרש להן היו. וכן נמי משמע במס' ברכות פ' כיצד מברכין (מ, א). }
\twocol{הא דאמרינן ב\textbf{מקשה עצמו יהא בנדוי.}  פי' בתוספות לא שהוא מנודה בעצמו בנדוי דרבנן של רבותינו אלא שב"ד מצווין לנדותו ועד שנדוהו אינו מנודה, וראיה לדבר דקאמרינן הקורא לחבירו עבד שיהא בנידוי ואמרי' עלה בקדושין באומר לו עבד אתה ההוא שמותי משמתינן ליה דתניא הקורא לחבירו עבד יהא בנדוי. }
\newchap{דף \hebrewnumeral{14}}
\twocol{הא דתניא \textbf{רוכבי גמלים כולן רשעים.}  ובשלהי מס' קדושין אמרינן הגמלים כולן כשרים התם במילי דעלמ' ולבן לשמים הכא רשעים בדבר הזה, א"נ הכא רוכבי גמלים ורשעתם משום חמום זה התם גמלים שמחמרים אחריהם. והא דמדכר הכא הספנין ואע"ג דליתנהו ברכיבה לומר שהם כשרים וצדיקי' גמורים בכל דבר בחמום ובשאר דברים ואגב אחריני נקט להו וכן פי' החמרים יש מהן צדיקים ויש מהן רשעים בדבר הזה קאמר הא דמכף הא דלא מכף ואלו בשאר דברים רובן רשעים וליסטים. }
\twocol{\textbf{וליחוש דילמא דם מאכולת הוא.}  פי' רש"י ז"ל דעל עד שלו פריך ומיהו ה"נ קשיא לעד שלה.\par  ולפיכך הקשו מקצת המפרשים א"כ אין לך אשה שנטמאה בנדה בבדיקה. וי"ל בעלמא ודאי לא חיישינן משום דלא שכיח אבל מתוך שבדקה מיד לפני תשמיש ומצאה טהור יש לספק ולא מחוור.\par וי"ל שלא בשעת בעילה ממש ודאי אותו מקום בדוק הוא אצל מאכולת שהוא סתום מלכנוס ואם נכנסה מתה היא ואין דמה יוצא ממנה אבל כאן יש לחוש שמא בשע' בעילה דחקה ונכנסה והשמש הרגה ושפך דמה עד עקבו א"נ שמא על השמש היתה מאכולת ועמו נכנסה ופריק דחוק הוא ואין מאכולת שעל השמש נכנסת עמו וכ"ש בפני עצמה.\par  ובתוספת אמרו דעל עד שלו דוקא פריך משום דכיון שבדקה היא בשיעור וסת ומצאה טהור והוא מצא יש לחוש למאכולת שאם היה דם נדה בשלו היה נמצא בשלה נמי. אבל בשלה בבדיקה דעדים לא חיישינן למאכולת דרוב דמים מצויין בנשים ואין קנוח העד נמי ממעך מאכולת והורגה (ובמעורה) [ובמעוכא] דשמש אין לתלותה כיון שלא נמצא על שלה.\par אבל עדיין אני תמה על עד שלו דלא יהבו ליה רבנן שיעורא אלא אע"פ שיהא אחר בעילה זמן מרובה קודם קנוח טמאין ואמאי ליחוש שמא מאכול' הוא שבאה עליה אחר שבעל שהרי אינו אלא ככתם בעלמא ויש לדחוק ולומר שכל קודם קנוח כיון שעדיין שכבת זרע לחה אין המאכול' באה עליו ואין לחוש שמא נרצפה על הסדין וממנו נתקנח בו שדם מאכול' מועט הוא ואינו מתקלח מן הסדין אלא נבלע הוא בו מהכ"ש שא"א לו להתקנח (במקום) [ממקום] ששוכבת עליו לזה. }
\twocol{\textbf{בדקה בעד הבדוק לה וטחתו ביריכה ולמחר מצאת עליו דם.}  גרסינן בכולהי נוסחי וכן מצינו בשם ר"ח. ופירושו שמצאת עליו על העד דליכא למיתני אלא דילמא דם בירך הוה ואי הוה בירך נמי כתם הוא וטמא וכיון דאיכא ספיקא דירך גופיה ועוד דלא שכיח קרוב הוא יותר לתלות בעד לטומאת נדה ולא בירך הילכך בין שנמצא נמי על הירך בין שלא נמצא אלא על העד טמאה נדה.\par  ורש"י ז"ל גרס ונמצ' עליה דם ופירש על יריכ' של אשה ואין דבריו מחוורים שאם העד נמצא ואין עליו כלום נראה ודאי שדם יבש על ירכה היה ואין לטמא אותה נדה ואם דוקא כשנמצא אף על העד למה לי מציאת הירך וכי מפני שהוטח ממנה דם נקל ויש להעמידה בשאבד עדה. וי"ל לעולם כשנמצא אף על העד וכשלא נמצא על הירך כלום פשיטא בעד הוה שאלמלא על הירך הוה לא נתקנח לגמרי ממנה אלא אפילו נמצא על הירך נמי טמאה נדה כיון שנמצא אף בעד.\par  ולדברי הכל אף בזה צריכה כגריס ועוד שאלמלא כן חוששין שמא דם מאכולת הוא שאפילו בקנוח של שיעור וסת ואח"כ שהדבר קרוב ואפילו לחטאת ואשם הקשו למעלה וליחוש דילמא מאכולת הוא. ומיהו כיון דאיכא שיעורא דנפקא ליה מחשש מאכולת טמאה נדה ואפילו לר' חייא ואם היה משוך טמא בכל שהוא דלא גרע מהניחתו תחת הכר וכן בפרק הרואה כתם אליבא דהלכתא וזה דעת הראב"ד ז"ל.\par ולפי גרסת ר"ח ז"ל י"ל כמו שנמצא על העד ולא על הירך כלום שאם מאכולת נרצפה שם בתחלה על הירך נמי היה נמצא ומשהוטח העד על הירך מקום (דחוק) א) הוא אצל מאכולת אלא מקנוח היה דם ונבלע בעד ולפיכך לא הוטח ממנו על הירך כלום. וכן נראה מלישנא דגמ' דקא מקשה ר' חייא בסמוך אף אתה עשיתו כתם אלמא אין לך שצריכה שיעור כתם וטמאה נדה אלא זו לדברי רבי. }
\twocol{\textbf{בדקה בעד שאינו בדוק לה.}  פירש בתוספות כגון שהזמינה פקולין או צמר נקי ולבנים אלא שלא חזרה וראתה בהן סמוך לבדיקתה אם יש עליהם טיפי דמים מן מאכולת או משאר דברים הא בבגד שאינו בדוק כלל לא אמר ר' טמאה נדה אטו לקחה בגד מן האשפ' וקנחה בו מי מטמא רבי נדה.\par  (ואי) [ועוד אני] אומר כיון שהצריכוה כגריס ועוד הרידינו כסדין וחלוק וכל שלא היה בדוק כלל אפילו לרבי טהורה לגמרי אפילו לקחתו מן השוק סתם טהורה לבעלה. והראב"ד ז"ל סובר דבעד אפילו אינו בדוק כלל טמאה דלא דמי לחלוק דכיון שבדקה בו ממש רגלים לדבר דרוב דמים מצויין בו. }
\twocol{הא דתנן \textbf{כדי שתרד מן המטה ותדיח את פניה.}  לישנא מעליא הוא ומאי פניה שלמטה ומאי הדחה בדיקה כלומר כדי שתרד מן המטה ותבדוק והכי מוכחא שמעתא והיינו דאקשי' מדתניא כדי שתושיט ידה מתחת הכר או לתחת הכסת ותטול עד ותבדוק בו דאלמא שיעור אשם תלוי אינו אלא שתקח העד ותבדוק דקס"ד סתמא כשאין עד בידה אלא תחת הכר או תחת הכסת. }
\twocol{ ופריק רב חסדא דה"ק \textbf{איזו אחר זמן וכו'}  פי' רש"י ז"ל וחסורי מחסרא וה"ק ול"נ אלא רב חסדא פרושי מפרש לה למתניתין הכי תנן נמצא על שלה לאחר זמן טמאים מן הספק ופטורין מן הקרבן ולא פי' שיעור אחר זמן מה הוה. והדר תני איזו אחר של שיעור זה שאינן טמאין מן הספק כדי שתרד מן המטה ותבדוק שזהו אחר כך שמטמאה מעת לעת ואינה מטמאה את בועלה והאי דפריש תנא האי שיעורא לא ללמד על דין עצמו שהרי כל מעת לעת כך הוא נדון בין לרבנן בין לר' עקיבא אלא כך אמר אם ירדה מן המטה ובדקה אין בועלה טמא שזהו אחר זמן הא כל זמן שלא שהת' כשיעור הזה אלא בדקה עצמה על המטה טמאין מספק בא זה ולמד על זה.\par  אבל לדברי רש"י ז"ל שאומר חסורי מחסרא וה"ק בהדיא איזו אחר זמן כדי שתושיט ידה ותטול עד ותבדוק לא הוה תו למיתני כדי שתרד מן המטה ואפשר לתרץ לו שבא לפרש שלא תעלה על דעת שבדיקת ראשונה שלאחר זמן ראשונה כשירדה מן המט' היא לפיכך פירש שתיהן ואמר שאלו ירדה אחר אחר הוא וכ"ש לדברינו דמחוור טפי לומר שפי' אחר אחר ללמד על אחר הזמן הראשון ושלא ליתן בו מקום לטעות.\par  וברייתא נמי דייקא כדידן, דתניא איזהו אחר זמן דבר זה שאל ר' אלעזר בר צדוק לפני חכמי' באושא שמא כר"ע אתם אומרים שמטמאה את בועלה מעת לעת פירש ולפיכך אין אתם חוששין לפרש אחר זמן שהרי כל מעת לעת נמי כך הוא דינן ואע"פ שהיו צריכין לפרש לדבריו דר' אליעזר בר צדוק משום אשם תלוי יודע היה בהם דבעי חתיכה משתי חתיכות ולא תמה עליהם בזה אלא אם כדברי ר"ע שהוא יחיד הם אומרים אמרו לו לא שמענו לפי' אין אנו מפרשין אבל לא בדברי היחיד אנו אומרים ואמר להם כך פרשו חכמים ביבנה לא שהתה כדי שתרד מן המטה ותדיח פניה תוך זמן זה כלומר כל ששהתה ובדקה ולא שהתה שיעור שתרד מן המטה ותבדוק אלא על המטה בדקה אע"פ שהושיטה ידה לעד תוך זמן זה אבל ירדה מן המטה ובדקה או שהתה כשיעור הזה לעולם טהור ואע"פ שעד בידה ש"מ שבכל מקום שפי' אחר לא פירש באחר זמן דבר אחר אלא כל שלא שהתה כשיעור אחר וכן דרך משנתינו ללשון שפירשנו.\par  ואקשינן לרב אשי אמאי קא מטהרי רבנן בברייתא ביורדת מן המטה ובודק' הא במתני' מטמו כשיעור הזה וכ"ת דאין עד בידה ה"ל לפרושי שהרי אין משמעו' הלשון זה אלא כל ששהתה כדי שתרד לעולם טהור ואפילו עד בידה וכדקתני מתניתין נמי כדי שתרד ומוקמת לה בעד [בידה] דהיכי אפשר דהכא והכא חד שיעורא קתני והכא טהור והכא טמא ה"ל לפרושי במתני' עד בידה ובברייתא אין עד בידה אלא ש"מ כרב חסדא ותרווייהו שיעור לטהר וזהו דרך פירש רש"י ז"ל בשמועה כולה ויש לשונות אחרים ואין בהם ממש. }
\newchap{דף \hebrewnumeral{15}}
\twocol{\textbf{והוא שבא ומצאה בתוך ימי עונתה.}  פי' רש"י ז"ל שלשים יום לראיה ואמרו שכך נמצא במס' נדה בירושלמי ואמר רב הונא דכי מצאה תוך ימי עונתה לא בעיא בדיקה אנא שלא הגיע ימי וסת אבל הגיע קודם ביאתו מן הדרך אסורה וסתות דאורייתא הלכך אסורה עד שתאמר לו בדקתי בשעת הוסת עצמו וטהורה אני ורבה בר בר חנא אמר אפילו הגיע עת וסתה מותרו' וסתות דרבנן שהם הצריכוה לבדוק בימי וסתה שמא תראה.\par  ומיהו היכא דבעלה לא היה בעיר ולא ידעינן אי בדקה אי לא בדקה לא מחזקינן לה בטומאה. ואע"פ דאמרינן לקמן תבדוק ומשמע דאסורה עד שתבדוק וכיון דכי לא בדקה מחזיקן לה בטומאה עד שתבדוק כי לא ידעינן ודאי בההיא דחזקה נמי קיימא עד שתאמר בדקתי וטהורה אני א"ל הכא בבא מן הדרך הקל כיון דאיכא תביעה אין לך בדיקה גדולה מזו ומהניא לחששא דוסתו' כדמהניא לבדיקה דטהרות בפירקין קמא דתרווייהו מדרבנן. א"נ כיון דלא ידעינן תולין שמא בדקה ומצאה טהור או שלא ראתה והיתה טהורה מ"ה מותרת. והא דאמר ר' יוחנן בעלה מחשב ימי וסתה ובא עליה תפתר בשוהה בעיר ולשהות עמה בין עירה בין ישינה ואף ע"פ שאינה אומרת לו כלום ולא הוא תובעה כנ"ל לפי פי' רש"י ז"ל והוא נכון.\par  אלא בזו שפי' ימי עונה לאשה שיש לה וסת אינו מחוור שכיון שלא הגיע עת וסתה היאך נחוש לעונה והלא כל שיש לה וסת קבוע דמיה מסולקין ממנה עד זמן וסת [לכן נראה] דרב הונא אמתני' קאי ולדידיה יש לה וסת חוששת לוסתה ולא לעונה אין לה וסת חוששת לעונה ולרבה בב"ח אפילו יש לה וסת אינו חושש לוסתה כדאמרן ומיהו חוששת לעונה אחר הוסת כלומר שאם עבר עליה עת וסתה כיון שאין אנו חוששין לוסת נחוש לעונה אבל ודאי הגיעו ימי עונה ולא הגיע ימי הוסת אינה אסורה דאפילו למ"ד וסתו' דרבנן מסולקת דמים היא אפילו מעונות עד הוסת תדע שהרי אמרו דיה שעתה בוסתות ולא אמרו כן בעונות.\par נמצא עכשיו לדברינו כל אשה שאין לה וסת בעלה חושש לימי עונתה ושיש לה וסת חוששין לימי עונה שאחר הוסת ואפילו בבא מן הדרך ובכולן מחשב ימיה ובא עליה דהא מ"מ תרי ספיקי נינהו ואע"פ שאינו בעיר נמי סופרת היא בעצמה רוב פעמים הילכך מותרת ומסתברא נמי שאם היה הוסת רחוק שאפשר שטבלה והעונה קרובה תולין להקל שמא בשעת הוסת ראתה וטבלה ושוב אין לה עונה עכשיו דכולהי ספיקי נינהו ולקולא ולא החמירו בעונה יותר מן הוסת מפני שהיא חמורה אלא מפני שא"א לנו לומר שהאשה לא תראה לעולם לפיכך תולין בעונות וחוששין להן אבל אם בא לתלות נמי בוסת תולין. זהו מה שנ"ל.\par והראב"ד ז"ל כתב דהא דרבה פליגא אדר' יוחנן דלר' יוחנן אע"ג דוסתו' דרבנן (בפי') [צריכה] חשוב ימים לטבילה ואפילו לבא מן הדרך שאין הוסת יוצא מחזקת טומאה (נהי) [עד] שתבדוק כדלקמן ופסק הלכה כרבי יוחנן וזו דרך טובה להחמיר לענין מעשה ונמצא חוששת לוסת וחושש' לעונו' כשאין לה וסת אבל לחוש לשניהם כאחד אין לנו.\par אבל תמהוני על רבינו הגדול ז"ל שכת' זו ששנינו חמרין ופועלין וכו' נשיהן להן בחזק' טהרה ולא חלק בין הגיעו ימי עונה ועת הוסת לשלא הגיעו. והר"ם תלמידו ז"ל כתב הלך בעלה למדינה אחרת והניחה טהורה כשיבא אינו צריך לשאול ואפילו מצאה ישינה הרי זה מותר לבא עליה שלא בעונת וסתה ואינו חושש שמא נדה היא. אף הוא לא הפריש בכלום.\par ונראה שהם ז"ל מפרשים תוך ימי עונתה היינו עונת וסתה לאפוקי יום עונה עצמו ודלא שתעלה על דעתך שבבא מן הדרך לא חששו אף לעונת הוסת או שיתלו להקל לומר שמא עקרתו וקמ"ל דבעי מיחשב דלא בעונת וסת קיימא האידנא.\par והם עוד סבורין דר' יוחנן דאמר מחשב קסבר וסתו' דאורייתא והלכתא כרבה דאמר מותרת דסוגיין וסתו' דרבנן ומסתייעי מאינשי דביתיה דרב פפא ורב הונא בריה דר' יהושע דפ"ק דכי אתו מבי רב לבתר וסת ועונה אתו ולא מיחשבי ולא בדקי בין עירות בין ישינות והא דתניא ר' יהושע אומר תבדוק תפתר לטהרות וזו קולא גדולה טוב לפני האלהים ימלט ממנו. }
\twocol{הא ד\textbf{אמר רבי אושעיא וכו'}  במסכת ע"א שמעתיה (מא, ב) ושם פירשתי. }
\newchap{דף \hebrewnumeral{16}}
\twocol{הא דתניא \textbf{הרואה דם מחמת המכה אפילו בתוך ימי נדותיה טהורה.}  טעמא דמילתא דאע"ג דאמרינן תבדוק הכא כיון דלא אפשר לה למיבדק טהורה ואין דנין אפש' ממי שאי אפשר וסתו' דרבנן ובאפשר תקנו וכן זו שאמרו אף אנן נמי תנינא לר"נו דוסתות דאורייתא מדקתני שחרדה מסלק' את הדמים טעמ' דאיכא חרדה הא ליכא חרדה טמאה אלמא וסתות דאורייתא ולא אמרי' כי איכא חרדה טהורה כי ליכא חרדה תבדוק היינו טעמא משום דלמ"ד וסתו' דרבנן אע"ג דליכא חרדה טהורה לגמרי כיון שלא היה יכולה לבדוק בשעת הוסת לא הטריחו עליה לבדוק אחר כן כלל.\par ובתוספות מפרשים דדייק לה מלישנא דקתני מסלקת את הדמים משמע דמים הבאים בזמנן ואינו נכון (לא) [ועוד] אמרו דסיפא נקט דקתני לה בדר"מ וארישא סמיך דקתני הניע שעת וסתה ולא בדקה טמאה ועלה א"ר מאיר שאם היתה במחבא טהורה הא לא היתה במחבא מודה דטמאה. }
\twocol{מתני': \textbf{ב"ש אומרים צריכה שני עדים על כל תשמיש ותשמיש.}  פי' רש"י ז"ל א' לפני תשמיש וא' לאחר תשמיש ולמחר בודק בשניהם ולא עכשיו כמו שפירשתי בפ"ק לדעת הרב ז"ל, ולשון צריכה נראה כן מדלא קתני צריכים ומיהו יכולה היא לבדוק באותו שלפני תשמיש זה לפני תשמיש אחר חוץ מן הצנועו' ששנינו למעלה אבל עד שלאחר כל תשמיש ותשמיש צריך לבן וכיון שזה צריך לכל הנשים קתני נמי שלפני תשמיש שדרך הצנועו'.\par ואינו מחוור. ועוד ק"ל אמאי לא תנא צריכי' שלשה עדים וליחשוב נמי א' שלו אבל נראה שכל מקום ששנינו שני עדים א' לו וא' לה.\par  וכך פי' משנתינו לפי דעתי צריכה שני עדים א' לו וא' לה שלה בודק בו לפני תשמיש ראשון ורואה טהרה ומשמשת ואח"כ מקנח' בו לאחר תשמי' וכשבאה לשמש פעם אחרת אינה צריכה כלום אלא משמשת ולאתר תשמיש מקנחין היא ובעלה בשני עדים אחרים ומניחין אותן עד למחר שמא מחמת תשמיש ראתה וזו הבדיקה אינו מועלת להם [אלא] לטהרות חוץ מבדיקה שלפני תשמיש ראשון שהיא אף לבעל לפיכך בודקת ורואה מיד לאור היום בין השמשות או לאור הנר אם התחילה משחשיכה לגמרי ואפילו לב"ש עד שלפני תשמיש ושלאחר תשמיש עד א' הוא שאפילו הצנועות עצמן לא נהגו צניעותן אלא בעד שלפני תשמיש מתוך שהוא נקי וטיפה כל שהיא ניכרת בו אינן רוצות שיהי' בו לכלוך אפילו שלפני תשמיש אחר כדי שתהא בדיקתן מעולה לגמרי אבל של אחר תשמיש ששכבת זרע רבה עליו [אין בין עד חדש לעד] מבדיקה ראשונה כלום.\par והיינו נמי דאמר או תשמש לאור הנר ותבדוק בו דבעד שלאתר תשמיש לא בעינן לבן אלא שצריך שידע אם דם שעליו מתשמיש ראשון או מאחרון היה כדי לחייב עצמה ובועל' לידע אם טמא משום בועל נדה ולעולם לפני תשמיש [א"צ בדיקה] שלא בעי ב"ש בדיקה אלא לפני תשמיש ראשון בלבד שהרי דיה בדיקה אחת לעונה אחת ולא הוצרכו הללו שבין תשמיש לתשמיש אפילו לטהרו' [אלא] מפני חשש רואה מחמת תשמיש הילכך דיה בבדיקה שלאחר כל תשמיש ותשמיש.\par ועוד שכיון שהיא משמשת והולכת בודק' שלאחר תשמיש זה שהיא לפני תשמיש זה דסמוכין הן וב"ה אומרים דיה שני עדים כל הלילה אף אלו א' לו וא' לה שלה עולה לפני תשמיש ראשון שבודק' ורואה ולאחר תשמיש אחרון ושלו לאחר תשמיש אחרון שכיון שאף לדברי ב"ש אין בדיקו' הללו אלא לטהרו' ולאחר תשמיש דיה בסוף.\par  ולא יקשה עליך צריכה [{\small פי' ולא תנן צריכים} ] לשון שאמרנו שהרי כך שנינו דרך בנות ישראל משמשות בב' עדים ואע"פ שא' לו ולה אמרו דרך בנות ישראל ובני ישראל משמשין.\par  והר"ם הספרדי פי' שאף ב"ה מצריכים לבדוק לאחר כל תשמיש אלא שדיין באותן שני עדים כל הלילה. }
\twocol{גמרא \textbf{אמרו להם ב"ש לדבריכם.}  כיון שאתם מודים בבדיקה שלאחר תשמיש משום שמא ראתה מחמת תשמיש הוה נמי בבדיקה בין תשמיש לתשמיש שמא תראה טפת דם כחרדל בביאה ראשונה שבא אורח מחמת תשמיש ותחפנה שכבת זרע בביאה שנייה ושוב לא תמצא בעד שלאחר כל התשמישן אמרו להם ב"ה א"כ אף מתחלת ביאה לסוף ביאה ניחוש כן ויכולין היו ב"ש לומר שאין דנין אפשר מא"א אלא גדולה מזו אמרו שאינו דומה וכו'.\par ולדברי הר"מ ז"ל צריכין לפרש תראה טיפה דם כחרדל בעד של אחר ביאה ראשונה ותחפנו שכבת זרע לאחר ביאה שנייה ואמרו להם ב"ה אף לדבריכם שמא בקנוח ראשון עצמו נמוקה הטפה ובטלה בשכבת זרע.\par  ומצאתי בתוספו' שפי' כדבריו בשמו של ר' שמואל רומרוגי ז"ל והם הקשו ללשון רש"י ממה ששנינו צריכין ב' עדים על כל תשמיש ותשמיש או תשמש לאור הנר ומשמע דמשמשת לאור הנר אינה צריכה אלא כדברי ב"ה ולפירושו עדיין המשמשת לאור הנר צריכה לבדוק ולאחר כל תשמיש ותשמיש ולראו' בעד כדברי ב"ש וב"ה אינה צריכה בדיקה כלל עד סוף כל הלילה מ"מ לשון תראה מתפרש לנו יפה. }
\twocol{הא דאמר ר' זירא \textbf{בעל נפש לא יבעול וישנה.}  לא שהוא סבור שמשנתינו לבעלה שהוא עצמו אומר בפרק קמא כל לבעלה לא בעיא בדיקה ושם הביאו ברייתא זו בד"א לטהרות וכו' אלא ה"ק כיון שאע"פ שבדקה לפני תשמיש חששו כולם שמא ראתה מחמת תשמיש ומניחים העדים עד למחר הילכך בעל נפש לא יכניס עצמו לבית הספק אפילו לבדוק ולהניח דזה לא יועיל לגבי הבעל ורבא אמר שאין בחששות הללו שום ספק אלא חומר של טהרות הוא. }
\twocol{והא דאמרי' \textbf{תניא נמי הכי}  לא שהוא לבעל נפש אלא שהוא דוקא לטהרות ולא לגבי הבעל וכיוצא בזה תניא נמי הכי שאינו לראיה ממש בפרק קמא דמגילה מפני שעיניהם של עניים נשואות למקרא מגילה וכו' ובתוספות מפרשים לא יבעול וישנה בלא בדיקה הא בבדיקה מותר אפילו לחסיד שבחסידים מדאמרינן במס' שבת אמר ר' יוסי ה' בעילות בעלתי ושניתי, ואין סוגיא מתחוורת בפי' הזה. }
\twocol{\textbf{בדקה בעד ואבד אסורה לשמש עד שתבדוק.}  לדברי רש"י ז"ל בעד שלפני תשמיש ופשוט' היא, ולדברינו כגון שהחמירה ובדקה לאחר תשמיש ראשון או שהיתה דעתה שלא לשמש כל הלילה ומכיון שאין מוכיחה קיים ולא תדע אם ראתה כלום מחמת תשמיש זה אסורה לשמש עד שתבדוק לפני תשמיש לאור הנר בכל בדיקה של בעל.\par  ואקשינן אלו קנחה בו ואיתי' מי לא משמשה אע"ג דלא ידעה הא אפילו ב"ש דמחמרי שרו לבדוק ולהניח ולשמש והאי דאקשינן הכי ולא אקשינן מדב"ה דלא מצרכי בדיקה זו דמי מדמי מקשינן דאלו התם דיומא משום קולא דבעל הוא שלא להטריח עליו בין תשמיש לתשמיש אבל מכיון שבדקה דנמלך צריכה היא בדיקה ממש לאור הנר.\par  ומפרק זו מוכיחה קיים לטהרות וזו אין מוכיחה קיים לטהרות שמא ראתה לאחר תשמיש זה ולא תדע למחר הילכך אף לבעלה אסור' משום מגו של טהרות וכן נמי לב"ה דלא מצרכי הך בדיקה מוכיחה בעד שלאחר תשמי' אחרון ואין חוששין לנימוק אבל זו כבר נתקנח הדם בזה ואבד הילכך אסורה עד שתחזור ותבדוק עכשיו לפני תשמיש דבהכי ודאי מותר' שא"א להחמי' עליה יותר מכאן לא יהא זה חמור מן הוסת שאם בדקה ומצאה טהור טהור. ואפילו לטהרות עצמן אם בטלה בדיקה של עונה אחת כגון של שחרית או של ערבית בודקת עכשיו ועוסקת בהן וכן בבדיקה של אחר תשמיש בין לטהרות בין לבעלה. }
\newchap{דף \hebrewnumeral{17}}
\twocol{הא דאמרינן \textbf{רבה בר רב הונא מקרקש זגי דכלתי.}  פ' רש"י פעמונים תלויין בכילה שסביב מטתו ומקרקשן בעת תשמיש לסור בני ביתו ואין זו דרך צניעות אפילו לקלים אלא כדי שיבריחו העכברים והתרנגולים היה עושה וכעין עובדי דדידכי ופרוחי היא.\par ובספר הישר מפורש דזגי מין דבורים הם כדאמרי בבכורות דבש הגזין וצירים או גזין וצרעין, ועוד שאף לשון קרקוש אינו על הפעמונים שפי' רש"י ז"ל. }
\twocol{ הא דאמרינן לצפרנים\textbf{ולא אמרן וכו'.}  ואסיקנא ולא היא לכולי מילתא חיישינן מדגרסינן פ' ואלו מגלחין ר' יוחנן שקל טופריה בשיניה וזרקינהו בי מדרשא ואקשינן עליה היכי עביד הכי והתניא זורקן רשע ופריק אשה בבי מדרשא לא שכיחי וכ"ת דילמא כנשי להו לבראי הואיל ואישתני אישתני אלמא שאפי' בלא גנוסטרי ודידה בלחו' אסור. }
\twocol{ הא \textbf{דאמר אביי כגון שהעבירה על אויר התנור.}  חדא מתרי טעמיה נקט דה"נ אפשר לאוקומה כגון שנטמא באהל המת א"נ בהיסט הזב וזבה וכל המטמאים במשא. }
\twocol{\textbf{תינוק הנמצא בצד העיסה.}  פירש רש"י ז"ל בתינוק שאינו ודאי טמא אלא שרוב תינוקות מטפחין באשפה ושרצים מצויין שם ובצק בידו שנגע בעיסה.\par  וזה הפי' אינו נכון שאם ודאי נגע בעיסה למה אין שורפין תרומה הרי רוב זה כרוב שליא שעשאוהו כודאי דהתם נמי א"ל דרוב שליא בולד ומיעוט בלא ולד ובית זה בחזקת טהור עומד סמוך מיעוטא לחזקה ואיתרע ליה רובא דלא שרפינן עליה תרומה אלא כיון דשליא בבית הוא אין חזקתו של בית הוא כלום אלא הרי הבית כשליא ורובא ומיעוטא ובתר רובא אזלינן הוא נמי בתר רובא אזלינן.\par  אבל הפי' הנכון שבתינוק ודאי טמא נחלקו דכיון שנמצא בצד העיסה והבצק בידו אמרינן רוב תינוקות מטפחין בעיסה ומיעוטן אין מטפחין ועיסה זו בחזקת טהורה עומדת סמוך מיעוטה לחזקה שהרי שתי המדות הללו מחזיקין טהרה לעיסה לומר שלא נגעה בה תינוק וממקום אחר היה לו הבצק או אדם אחר נתנו לו ומעולם לא טפחו לבצק הזה בעיסה מה שאין כן בשליא שאי אפשר לצרף חזקת הבית למיעוט שליא שאין חזקה זו מלמדת שאין עם השליא ולד אלא הבית כשליא עצמה ואין חזקתה כלום שהרי נגעה ונעשית כמוה אבל בעיסה ותינוק שאנן מחזיקין שלא נגע תינוק בעיסה.\par ואיפשר שאף לדברי רש"י ז"ל לא בשנגע אלא שהבצק בידו ואף על פי שחזקה הוא שנגע עדיין חזקת עיסה במקומה כיון שלא נגעו ודאי בפנינו ואין זה נכון ומפורש הוא בירושלמי בסוף קדושין רוב תינוקות מטפחין בעיסה וכבר פירשתי זה בפרק כיסוי הדם והאי דלא מטמינן ברשות היחיד לעיסה ואפילו מחצה על מחצה כשאר כל הספקות משום דבר שאין בו דעת לשאול הוא ואלמלא הרוב טהור הויא וכן שליא אין דינה לשרוף אלא משום רוב דאפילו ברשות הרבים מטמינן מדקתני שאין שליא עמה ולד הא בבית ממש טמא אפילו בספיקא דהוה ליה ספק שרץ ספק צפרדע. }
\twocol{הא דאמרינן \textbf{מאי לאו לא תיובתי' אלא סייעתיה וכו'.}  ה"פ: דקסלקא דעתך מדקתני מתניתין דטועה מטבילין אותה צ"ה טבילות שמחמירין עליה כל חומרי טבילות הללו ש"מ בודאי ילדה (מצוי) א) שאלמלא יש ספק בולד ה"ל ספק ספיקא ספק זמן טבילה היום ספק אינו זמן ואת"ל הוא ספק אינו ולד ודיה טבילה אחת באחרונה כרבי יוסי לר' יהודה דלא מחמירין כולי האי בטבילה בזמנה מצוה אפילו דתרי ספיקי אי נמי בחד ספיקא בספק לידה כלל.\par ופריק דילמא לא תיובתיה ולא סייעתיה דכיון דאיכא רוב הולכין אחריו אפילו בתרי ספיקא להחמיר ואפילו לכתחלה אבל להקל לא עשאוהו בודאי.\par ויש אומרים מפני שהצריכוה למיטבל בשבוע ג' משום יולדת דזוב וספיקי טובא נינהו שמא ילדה שמא לא ילדה ואם תמצא לומר ילדה בזוב אימר עלו לה ימי שבוע שני לספירה אם הרחיקה לידתה ואינו נכון דהשתא נמי דהויא ודאי הויין ספיקות טובא אלא סלקא דעתך דבספק לידה לא מחמרינן בטבילה בזמנה. }
\twocol{\textbf{אלא למעוטי רובא דרבי יהודה.}  פירש דאפילו לרבי יהודה לאו רוב גמור הוא לשרוף אלא לתלות, וק"ל דהא איתותב ההוא לישנא ואסיקנא דבאי אפשר לפתיחת קבר בלא דם קמפלגי ובפיר' רש"י ז"ל אפילו ללישנא בתרא דר' יוחנן דאמר טעמיה משום דאי אפשר לפתיחת קבר בלא דם אפילו הכי טעמיה דר' יהודה משום רוב דברוב פתיחת קבר איכא דם ולא עשאו ברוב זה כודאי דכיון דאין עמה דם איתרע ליה.\par  וק"ל דהא אוקימנא לר' יהודה כרבי יהושע דמשוה ליה חדא ב) דאמר מביא קרבן ונאכל דאי אפשר לפתיחת קבר בלא דם וא"ל סבר לה כרבי יהושע לתלות ולא לשרוף, ויש לומר דמאן דמתני בשלשה מקומות מתני ההוא לישנא קמא\par  דר' יוחנן דמוקי פלוגתייהו דר' יהודה ורבנן בשאינה יודעת מה הפילה. וכן עיקר שאלמלא כן לא הזכירו בגמרא כאן לשון ראשון שהוא טעות במקום עיקרו. }
\newchap{דף \hebrewnumeral{19}}
\twocol{הא דתנן \textbf{עמוק מכן טמא דיהה מכן טהור.}  פירש רש"י ז"ל עמוק יותר שחור דיהה שנדחית מראיתו ואינו שחור כל כך.\par ושמעתי שה"ר שמואל ז"ל פירש בהפוך עמוק שאינו שחור כל כך דיהה שחור יותר מכן.\par  והכניסו במחלוקת הזה מה שאמרו לענין נגעים בהרת עמוקה כמראה חמה עמוקה מן הצל אלמא לובן הוא העמוק, ועוד מזה מש"כ כזית ובזפת וכעורב טהור וזהו דיהה וכדיו וכענבה טמא וזהו עמוק והוא ז"ל שיער בדעתו שהעורב שחור יותר מהענבה ועוד שהדיו שהוא עמוק זהו הדיו עצמו והשחור השנוי במשנתינו הוא חרותא דדיותא והוא שחור יותר מן הדיו עצמו.\par  וכל אלו דברים בטלין הם שהמראה חזק באותו גוון הוא נקרא עמוק בכולה גוונים והחלוש באותו צבע הוא הדיהה ממנו כדתנן עבר או שדיהה הרי זה כתם והזית והעורב שחרירותן מבהיק ושל ענבה משחיר ובוהק ג) וכן הדיו הלחה שכותבין בו ונשתהא כוהה יותר (מקום לו) מהחרת השנוי במשנה וכי היאך יעלה על דעת במזוג דשנינו (שלשה) [שני] חלקים מים ואחד יין שהעמוק מכן והוא הלבן שיש בו שלשה חלקים מים ואחד יין טמא והדיהה או דיהה דדיהה שאין בו אלא חלק אחד מים וחלק אחד יין טהור והלא אדום גמור הוא וקרוב למראה דם יותר מן הראשון כפלי כפלים וכל שכן באדום עצמו שהוא כדם המכה שהמלבין בו טהור והמתאדם ביותר טמא ודאי בלא ספק ולא ידע מר בטיבעא כלום.\par  וראיתי בתוספות שהביאו מן הירושלמי בענין עמוק דיהה הוו בעיין מימר מאן דאמר טהור במצחצחו מאן דאמר טמא בשאינו מצחצח ושמע מינה מן הדה מעשה וכו' א"ל רביח לבו כן אמר רב הונא בשם ר' שחור מקדיר (טמא) [טהור] מצחצח טמא לא אמר אלא שחור כולהון אפילו מצחצחין טהורין אלמא המצחצח קרוב לטומאה יותר מן המקדיר וזה הפך ו) שהרי מתחלה סלקא דעתך המצחצח טהור ז) שהוא המשחיר ביותר הוא הטמא ח) ומ"ש אפילו מצחצחין הכי קאמר אפילו היה במראה הטומאה הואיל ומצחצחין טהורין. }
\twocol{ הא דאמרינן \textbf{איכא בינייהו לתלות}  דתני חמשה דמים טמאים והשאר יש מהם ספק, כגון ירוק ומימי תלתן ומימי בשר צלי נראה היה לפסוק כדברי תנא קמא דהוא סתם משנה ולא כדברי בית הלל וחכמים דהיינו רבי יוסי אלא שיש לסמוך הלכה כר' יוסי מפני ששנו אותה בלשון חכמים ואף על פי שפי' דמאן חכמים ר' יוסי לכך שנו אותה תחלה לומר כן שהיא הלכה וחזרו והזכירו דבר בשם אומרו להביא גאולה לעולם.\par  ועוד יש לי סמך וראיה לדבר מפני שהיא שנויה בבחירתא דם הירוק עקביא בן מהללאל מטמא וחכמים מטהרין וכיון ששנינו שם דברי רבי יוסי בלשון חכמים לא שנו שתולין שמע מינה שהלכה כרבי יוסי כדאתמר בעלמא (ברכות כז, א) הלכתא כר' יהודה ותנן בבחירתא כוותיה והתם משמע דעקביא גופיה הדר ביה דתנן ובשעת מיתתו אמר לבנו חזור בך בד' דברים שהייתי אומר א"ל ואתה למה לא חזרת בך וכו' במתני' ומדלא קתני התם בפלוגתא אלא סבריה דר' יוסי הדר ביה עקביא לטהר לגמרי ועוד דמתניתא דרב דימי מנהרדעא וסוגיין דעלה בריש פ' המפלת כולהו כר' יוסי אמרינן.\par  ואנו עכשיו שבטלו רואי דמים כל שיש בו מראה אדמימות ואפילו כמימי בשר צלי ודיהה ממנו יושבת עליו שבעה נקיים וכן בכל מראה שחור ואפילו כזית ודיהה ממנו ואפילו דיהה מן הדיהה אבל בדם הירוק ואצ"ל לבן ושאר מראות ודאי אינה יושבת כלום שלא גזרו אלא על דם ותולדותיו אבל אלו אינן בכלל לדברינו שפסקנו הלכה כר' יוסי בירוק, וכן כתב הרמב"ם ז"ל שהאשה שראתה לובן או אודם ירוק הרי היא טהורה אף בזמן הזה אלא שהוא סומך על דברי ב"ה לכתוב בחבורו שאין לך דמים טמאים כלל אלא ה' ולא חשש לסתם. }
\twocol{\textbf{ירד ר"מ לשיטתו של עקביא וטימא וה"ק להו וכו'.}  י"מ כל שמועה זו אחר דבריו של ר' יוחנן דהוא אמר ירד ר"מ לשיטתו של עקביא וטימא משום דקאמר אם אינו מטמא משום כתם משום דדם ירוק באשה לא שכיח א"נ לרבנן דחולין קמ"ל נהי דספיקא משויתו ליה גבי כתם דלקולא גבי רואה ממש תהא טמאה נדה לשרוף והיינו כעקביא.\par  ואקשינן עליה דר' יוחנן א"ה אם אינו מטמא משום כתם וכו' משום רואה מיבעי לי' אלא ה"ק להו וכו' קושיא היא ומהדרינן לקיימא לדר' יוחנן א"ה שפיר קאמרי ליה אלא ה"ק להו וליהוי כמשקה להכשיר את הזרעים דדם נדה ודאי מכשיר את הזרעים ומטמאי' אותם כדתניא בתוספתא מסכת שבת פ"ח מנין לדם נדה שהוא משקה שנאמר ממקור דמיה ונאמר ביום ההוא יהי' מקור נפתח וכו' ואמאי לא שויתי' ליה מיהא כדם נדה להכשיר שהרי אף בשאר דמים דעלמא יש מכשירין והיאך אתם מקילין על זה שלא לעשותו כדם נדה אפילו בדבר שמצינו שאר דברים דינן כך ורבנן בעינא דם חללים וכל שאינו דם חללים טהור הילכך זה שאינו דם חללים טהור הילכך זה שאינו דם נדה לדברינו כשאר דמים שאין בהם חלל נדון אותו.\par  ואקשינן לר' יוחנן נמי שפיר קאמרי ליה ה"ק להו אלפוה בג"ש כתיב הכא שלחיך פרדס רמונים כלומר כיון שבדם זה אסורה לבעלה וקרינן בה גן נעול יכשיר דעלה כתיב שלחיך ואמרי ליה רבנן אין אדם ג"ש מעצמו ואפילו לדברינו שתולין ואוסרין אותה לבעלה גבי הכשר לקולא ואין פי' זה אלא דברי נביאות.\par  ואחרים פירשו השמועה כפשטה אלא שאמרו שר' יוחנן הוא דפריש וה"ק נהי דלא מטמא משום כתם תטמא משום רואה וכי אפרי' לההיא לישנא איפריך ליה דר' יוחנן, ול"נ שלא אמר ר' יוחנן ירד ר"מ לשיטתו של עקביא אלא מפני שאמר במשנתינו אם אינו מטמא דמשמע דלדבריהם אמר להו כלומר נהי דמקילתו בזה אודו לי מיהת בזו הא איהו כעקביא ס"ל ומטמא בזו ובזו הילכך לכולהו לישני איתא לדר' יוחנן וכולהו נמי קאמר להו כלומר לדבריכם כדפרישית וגמרא הוא דפריש ואזיל וה"ק להו וכו'.\par ומיהו הא ק"ל היכי א"ר יוחנן ירד ר"מ לשיטתו של עקביא וטמא והא לקמן בפ' בנות כותים א"ר מאיר אם יושבת הן על כל דם תקנה גדולה היא להן אלא שרואה דם אדום ומשלימתו לדם ירוק אלמא כרבנן ס"ל ואפשר דהתם לרבנן קאמר להו ונקט מראה טהור לדבריהם ולאו דוקא. }
\twocol{ והא דאמרינן \textbf{אלפוה בג"ש דכתיב הכא שלחיך פרדס רמונים.}  ודאי קשיא דר"מ לאו בכל מה שאשה משלחת מטמא משום הכשר דא"כ מאי איריא דם ירוק דנקט אלא בדם נדה בלחוד הוא דקא מכשיר מדכתיב גן נעול וכתיב פרדס מה פרדס הזה נעול להשתמר כך בנות ישראל נועלות פתחיהן לבעליהן וא"כ ה"ק נהי דלא מטמיתו משום דם נדה מכשיר והלא אין לך מכשיר אלא דם נדה והוא דם טהור הוא לדבריהם.\par  וזו שאלה רבז"ל והשיב לר"מ כל היכא דחזיא דם אדום והדר קא חזיא כל דם מכשיר שכל זמן שהיא כפרד' שלחיך קרינן ביה וה"ק להו לדידי דם טמא הוא מכשיר מתחלתו אלא לדידכו אודו לי איהי מיהת דהיכא דחזיא דם אדום מעיקרא והויא פרדס אפילו ירוק יכשיר דקרינן ביה שלחיך ואמרו ליה ג"ש לא אמרינן הילכך אפילו דם אדום עצמו אינו מכשיר. }
\twocol{לשמואל דאמר \textbf{כדם שור שחוט ולעולא דאמר של צפור חיה ולרב נחמן דאמר של הקזה.}  ל"ק הא דתנן הרגה מאכולות תולה בה ובבנה ובבעלה דא"ל חד שיעורא הוא אלא לזעירי בלחוד דפריש הוא דמקשינן מינה הילכך אית לן לפרושי דכולהו לא פליגי אלא מר בקי בהאי חזותא ומר בקי באידך ודכולהו ודברייתא נמי חד הוא כדאמרן, והא דאקשינן הרגה מאכולת הרי זה תולה בה מאי לאו דכוליה גופא ואקשינן נמי תולה בבנה ובבעלה בשלמא בנה משכחת לה משמע דלא תלינן בכתמים אלא בדדמי ומקיפין ורואין ומכאן החמירו.\par  ונמצא במקצת גליוני ראשונים דעכשיו בזמן הזה כיון שבטלו ראיית דמים ואין בקיאות במראיהן של ד' דמים אין תולין בכתמים ולא בבן ולא בבעל ולא במאכולת ושוק של טבחים ושאר כל מה ששנו חכמים בכתמים לתלות אלא בכולן אסורות לבעלה.\par ורבינו בעל התוספות השיב דכי אמרינן עברה בשוק של טבחים תולה וכן במאכולת בכולן מעצמה קאמרינן דתולה ולא מגופא אתא אלא מעלמא אתא, אלא מיהו היכא דידעינן ודאי דלא דמו ליכא למיתלי ומ"ה קאמרינן הכא בשמעתין למ"ד כדם מאכולת של ראש הרגה ודאי מאכולת של גוף היאך תולה בשאינו דומה ודאי וכן קושיא דבנה ובעלה אבל מסתמא תולין כתמים בכל מה שאמרו חכמים.\par  וכענין זה כתב הראב"ד ז"ל דתולין מן הסתם כל מיני אדום באדום וכל מיני שחור בשחור עד שיתברר לה ודאי שאין אודם הצבע דומה לאודם הכתם דהתם אינו תולה כדתניא לקמן נתעסקה באדום אין תולה בו שחור ובשמעתין הכי אקשינן היאך אפשר לתלותו במכת בעלה ובדם מאכולת הגוף והלא דבר ברור הוא שאינו דומה, אבל מי שאינה יודעת בדמיונות או שהלך הצבע מנגד פניה ואינה יכולה לדמות תולה מן הסתם כדאמרינן באשה שבאת לפני ר' עקיבא ואמרה לו ראיתי כתם שמא מכה יש בידך וטהרה והרי אשה זו לא הביאה הכתם בידה לפניו וטהרה מיד ולא הקיף ולא ידע כל זה שכתב הרב ז"ל.\par וכן יש לפרש זו שהקשו בשמעתין ממאכולת ובעלה דה"ק ודאי מדבעין תליה בהני אלמא בכתם טמא עסקינן ואלו הנך דדמו כתמים טהורים הם ומה נפשך אי לא דמו לא תליא ואי דמו לא צריכי תלייה כלל, ולא בעי לתרוצי כשראתה ואבד ואינה יודעת מה ראתה דמסתמא ברואה ובאה לפני חכם תנן וידע דאיכא טובא מתירין הלכך בזמן שאין בקיאין תולה במאכולת הגוף ובבעלה ובכל מיני אדמות עד שיתברר לפי הדעת שאינן דומין וכן נהגו ואין לחוש.\par  ושוב ראינו בכתמים שכתב הראב"ד ז"ל עיינתי בכל מילי דרבוואתא ולא אשכחית בהון דינא דכתמים אי נהיגי האידנא או לא, ואף על גב דאשכחן דמטמא את בועלה בפרק קמא דנדה דלמא לטהרות היא.\par  וזה אינו נכון, וכבר השיב הרב חתנו ז"ל מדאמרינן בכתמים פעמים שהן מביאים לידי זיבה ואיתמר עלה מהו דתימא כל כהאי גוונא מביאה קרבן ונאכל קמשמע לן מביאה קרבן לאסרו לבעלה דאי לטהרות בלבד קרבן מאי עבידתיה דאפילו להכשיר בקדשים ליכא למיחש כיון דליתיה אלא דרבנן בעלמא לחוש בטהרות ולבעלה מתירין אותה לכתחלה.\par  ועוד השיב מדאמרינן בהדיא בפרק הרואה (דף נח ע"ב) לדברי אין קץ שאין לך אשה שטהורה לבעלה שאין לך כל מטה ומטה שאין עליה כמה טיפי דם מאכולת לדברי חבירי אין סוף שאין לך אשה שאינה טהורה לבעלה וכו' אלמא כתמים לבעל נאמרו ואין צריך להאריך שהרי הוחזקו בנות ישראל שנהגו איסור בכתמים וקי"ל דמנהגא מילתא היא כרבי זירא, כל אלו דברי הרב ז"ל. }
\twocol{\textbf{ת"ש דילתא אייתא דמא לקמיה דרבה בב"ח וכו'.}  אי קשיא אדרבה איפכא מסתברא דאם כן דנאמנת אשה לומר כזה טיהר לי פלוני חכם ולטהר אפילו לחברתה, למה הביאתו ילתא קמיה דרבה תטהר לנפשה דהרו כל יומא נמי מטהר לה, א"ל קס"ד השתא דאיהי סברא דלא מהימנה לנפשה, אי נמי מחמירה על נפשה, אי נמי משום כבודן של חכמים הללו אינה רואה במקומן כדאמרן לעיל. }
\newchap{דף \hebrewnumeral{21}}
\twocol{ה"ג \textbf{וכ"ת כי פליגי רבנן אירוקה ולבנה. הא אמר כל של שאר מיני דמים דברי הכל טהורה אלא אמר רב נחמן בר יצחק וכו'.}  וה"פ: וכי תימא הכי קאמר ברייתא, המפלת חתיכה אדומה ושחורה או ירוקה ולבנה אינה טמאה לידה אלא אם יש עמה דם טמאה נדה. הלכך אדומה ושחורה הרי היא עצמה דם וירוקה ולבנה צריכה דם טמא עמה ואם לאו טהורה הא לר' יוחנן א"א לפרשה כן שהרי הוא אמר שלשאר דמים ד"ה טהורה ובודאי מצי לתרוצי כדאמרן למימר דלא תיקשי ליה אלא בחדא מיהו אנן ה"ק לר' יוחנן דודאי לא מוקי לה אלא כפשוטה קשיא בתרתי.\par  ורש"י ז"ל גורס וכ"ת כ"פ רבנן אירוקה ולבנה אלא אדומה ושחורה למאן קתני לה וכו' ולא נהירא לי דכי אמרינן דרבנן אירוקה ולבנה בלחוד פליגי, ע"כ ברייתא ה"ק אדומה ושחורה טמאה ירוקה ולבנה אם יש עמה דם טמאה ואם לאו טהורה ורבנן גופייהו קתנו לה. }
\twocol{\textbf{באפשר לפתיחת קבר בלא דם פליגי ובפלוגתא דהני תנאי וכו'.}  ואי קשיא הא לת"ק דברייתא גופיה ספיקא משוי ליה ואלו ת"ק דמתני' קאמר ואי לאו טהורה א"ל התם משום דילדה ואינה יודעת מה ילדה וחוששין ללידה וחוששין נמי לזיבה שמא עם הנפל יצא דם אבל שילדה לידה יבישתא העמד אשה על חזקתה וטהורה היא ועוד שהרי בדקו ולא מצאו דם וכן ללשון הראשון שאמר רבנן סברי לא אמרי' רוב חתיכות מד' מיני דמים הן קשיא ותהוי נמי מחצה על מחצה תהא טמאה מספק אלא משום האי טעמא הוא דהעמד אשה על חזקתה.\par וי"מ דהתם ה"ק לא אמרינן רוב חתיכות מד' מינין הן ולפיכך טמאה גמורה אלא שאין שורפין כדאיתא בפ"ק, ואין פירוש זה נכון. }
\twocol{ הא ד\textbf{אמר ר' יוחנן אם יש בה דם אגור טהורה.}  איכא דמקשו ללישנא קמא דר' יוחנן ליחזי אם חתיכה זו מד' מינין טמאה שדרכה של אשה לראות דם נדה בחתיכה וא"ל אתיא כלישנא בתרא דר' יוחנן מאן דמתני לישנא קמא מתרץ הא דידיה הא דרביה, והאי דקאמר ר' יוחנן ד"ה טהורה לרבנן דמתני' ולר' יהודה מיירי. }
\twocol{ הא דמקשינן לר' זירא \textbf{והא אמר ר' יוחנן משום רשב"י המפלת וכו'.}  וא"ל אדמקשי ליה מיניה ליסייעיה ממתני' דקתני אם יש עמה דם אין בתוכה לא. וא"ל קס"ד השתא דמתני' לאו עמה לאפוקי תוכה אלא עמה לאפוקי חתיכה גופה דאפילו היא מארבע מינין טהורה ומדר' יוחנן מיפרשא ליה קושיא וממתני' לית ליה סייעתא בהדייהו.\par  ואע"ג דאמרן לעיל בגמרא דאלו רבנן סברי עמה אין תוכה לא ההיא מימרא דגמרא היא ולא קס"ד השתא ומ"ה פריק אפילו בדר' יוחנן דהתם משום דדרכה של אשה לראות דם בחתיכה ומין במינו הוא ואינו חוצץ כדפרישית לעיל, א"נ א"ל דמתניתין לא סייעתא היא דקס"ד עמה לאפוקי תוכה משום שאין זה דם נדה אלא דם חתיכה. }
\twocol{ הא דאקשינן \textbf{ת"ק נמי טהורי מטהר.}  ומפרקי' אלא לאו דפלאי פלויי איכא בנייהו. לאו דצריך להכי דהא מצי למימר אלא שפופרת א"ב דת"ק סבר בבשרה ולא בשפיר ולא בחתיכה וכ"ש בשפופרת ואתי רבנן למימר אין זה דם נדה אלא של חתיכה הא דם נדה טמאה ואפילו בשפופרת אלא הא דאמרינן דפלי פלויי איכא בנייהו משום דקים ליה דבהא נמי פליגי לפום טעמייהו דכיון דר' אלעזר סבר דם אגור הוא א"א לטהר אלא בדלא אפלאי וכיון דרבנן סברי דם חתיכה גופה הוא אפילו איפלאי נמי ודאי טהור' הילכך מפרש ואזיל כולה פלוגתייהו. }
\twocol{ ומהדר אביי \textbf{בשפופרת דכ"ע לא פליגי כי פליגי בחתיכה מר סבר דרכה של אשה לראות דם נדה בחתיכה.}  פירש רש"י ז"ל דבפלאי פלויי פליגי מר דהו רבי אלעזר סבר דרכה של אשה לראות דם נדה בחתיכה וכיון דאיפלאי וליכא חציצה טמאה וכי לא אפלאי רחמנא מיעטה מבשרה ולא בשפיר ולא בחתיכה ורבנן סברי אין דרכה של אשה לראות דם נדה בחתיכה אלא האי דם חתיכה עצמה הוא.\par ולא מחוור דא"ה לא דמי האי דרכה של אשה וכו' לאותו שאמרו למעלה בדר' יוחנן דהתם קאמרינן דכיון דדרכה אע"ג דלא אפלאי נמי לא חוצה דהיינו אורחא ולישנא דגמרא נמי לא משמע הכי כלל.\par אלא ה"פ מרדאינהו רבנן סברי דרכה של אשה לראות דם בחתיכה ולא טהרו כאן אלא משום שאין זה דם נדה אלא דם חתיכה והיכא דהוי ודאי דם (חתיכה) [נדה] כגון מצא בה דם אגור טמאה והיינו לר' יוחנן ומר דהוא ר"א סבר אין דרכה של אשה הילכך הוי ליה כשפופרת ורחמנא אמר בבשרה.\par והאי דמדכרי' סברא דרבנן מקמי דר' אלעזרא"ל משום דאמרן לעיל דרכה של אשה כסברייהו א"נ לאו דוקא וכן בכמה דוכתי בתלמודא דלא קפדי.\par  ובודאי דה"מ אביי לתרוצי כדרבנן הא דם נדה ודאי טמאה בדאיפלאי ובשפופרת דכו"ע לא פליגי דטהורה אלא ניחא ליה לתרוצי בדידה ולא לעיולי בה פילי דהשתא לא מוספינן בפלוגתייהו איפלאי פלויי כלל אלא בדם אגור בחתיכה פליגי כדפרישית, ועוד לאוקמה כדר' יוחנן דלעיל דלא לתקום דלא כחד כנ"ל.\par ויש מפרשים דלא ניחא ליה לאביי לאוקומה פלוגתא דרבנן אפלאי פלויי בלחוד דהא לא מדכרא בהדיא במילתיה דר' אלעזר דאנן בגמרא לאו חסורי מחסרא לברייתא כלל אלא מימר קאמרינן דלר"א בודאי פלאי טמאה ולא ניחא ליה לאוקומא פלוגתייהו אמאי דלא מתפרש בברייתא בהדיא.\par  ולאו מילתא היא ור"א ורבנן תרווייהו מטהרין מר נסיב לה טעמא מבשרה ולא בחתיכה ולפום טעמיה איפלאי פלויי טמאה ומר נסיב לה טעמא אין זה דם נדה לטהורי נמי אפלאי פלויי אלא כטעמא דפרישית עיקר. }
\newchap{דף \hebrewnumeral{22}}
\twocol{\textbf{א"ד האי ממנו עד שתצא טומאתו לחוץ.}  ואי קשיא הכא איצטרך קרא לומר שלא יטמא בפנים דאלמא דינא הוא דמטמא והכא איצטרך בבשרה לומר שמטמאה בפנים אלמא אין דינה לטמא. לא קשיא דגלי רחמנא בחד ואיצטרך חבריה שלא תאמר זב וזבה הוקשו א"נ באיש דינו לטמאות משעקר שסופו לצאת מיד ולא האשה שהרי עומד בבית החיצון הרבה. }
\twocol{\textbf{והלא עצמו הוא אינו מטמא אלא בחתימת פי האמה, למימרא דנוגע הוי.}  פירש רב הונא אליבא דנפשיה פשיט ליה דס"ל כר' נתן דאמר זב אינו מטמא אלא בחתימת פי האמה ואל תתמה [דהא שמואל] רביה (דרבה) [דרב הונא] הוא דאמר נמי כר' נתן כדאיתא בפרק יוצא דופן ולפום הכי גמר רב הונא בעל קרי מיניה דזב וסבר לה נמי כר' שמעון דאמר בפ' ואלו דברים בפסחים דס"ל בזב כר' נתן דבעי פי האמה ואיתקש בעל קרי לזב ובעי נמי חתימת פי האמה כזב דהא קרא בבעל קרי לא כתיב ובזב כתיב או החתים בשרו.\par  והא דדאיק מינייהו למימרא דנוגע הוי דלהכי בעינן חתימת פי האמה דליהוי נגיעת חוץ כדפי' רש"י ז"ל, ק"ל אי הכי זב נמי נוגע הוי ולמה לא יספור בזיבה, א"ל בזב ודאי אע"ג שנוגע הוי לענין שיעוריה מיהו הוי רואה לענין טומאה דיליה דאלו נוגע בזב טומאת ערב ואלו רואה טומאת שבעה והאי דאחמיר ד) עליה רחמנא בחתימת פי האמה דליהוי נמי נוגע גזירת הכתוב הוא שלא יהא טמא טומאת שבעה עד שיראה זוב ונגע בו מגע חוץ דה"ל רואה ונוגע ומ"ה אקשי' אא"ב בעל קרי רואה הוי ואפילו במקום שאינו טמא משום נוגע טמא הוא משום רואה הרי דומה לזב מצד אחד שאף הוא יש לו טומאה בראיה שאינו מדין מגע אא"א אינו טמא אלא בנוגע וטומאתו נמי טומאת מגע היא א"כ מה הנוגע בקרי אינו סותר בזיבה אף הרואה לא יסתור שהרי שניהן טומאה אחת להן בכל ענינן ומדין מגע טימאן הכתוב, ומפרקינן התם בשביל שא"א לה בלא צחצוחי זיבה ואם תאמר והלא אין בהם חתימת פי האמה ואין הזוב מטמא אלא כן י"ל כיון שיוצא עם שכבת זרע שהוא חותם פי האמה הרי הוא כנוגע ממש שמין במינו הוא ואינו חוצץ.\par  והיינו דלא אקשינן יטמא טומא' שבעה אלא תסתור ז' דטומאה בזוב גמור לית ליה כיון דאינו רואה בשיעורו טומאת מגע זוב אית ליה וכיון דזוב הוא מיהא ובראיה דין הוא לסתור הכל שאין כאן ז' נקיים דהא הוה ליה כאלו ראה זוב בנתיים שאין אחר אחר לכולן.\par  ומפרקי' גזרת הכתוב כך הוא מאחר שאין הזוב הזה כדי ראיה אין לו טומאת שבעה ואפילו לסתור ז' אלא סתירתו כטומאתו והא נמי רב הונא אליבא דנפשיה פשט ליה דהא בפרק כיצד הרגל בב"ק איכא ר' אליעזר דסבר אפשר בלא צחצוחי זיבה כלל, ומיהו בהא כרבנן פריק ליה ורבים נינהו.\par  ואי קשיא לך לרב הונא דאמר בעל קרי נוגע הוי תרי קראי למה לי דהא כתיב רואה וכתיב נוגע וכדדרשינן בפרק יוצא דופן מדכתיב או איש, א"ל אע"ג דרואה דוקא בנוגע הוא דמטמא ה"א ה"מ ברואה דאיכא תרתי מגע וראיה אבל בנוגע לחודיה לא קמ"ל. }
\twocol{הא דתנן \textbf{המפלת מין דגים וחגבים ושרצים אם יש עמהן דם טמאה.}  אוקי' במחלוקת שנויה ורבנן היא ומדלא פרישנן הבי ברישא דקתני כמין קליפה כמין שערה כמין עפר וכמין יבחושין ש"מ דההיא ד"ה היא דכיון דמילי זוטרי נינהו אפשר לפתיחת קבר קטנה בלא דם ודמיא הא לההיא דמפרקינן בכריתות (דף ט') [דף י' {\small ואין הגירסא שם כן} ] כי אמרינן א"א לפתיחת הקבר בלא דם היכא דגמר הולד דמיפתח טפי ב) ואפשר לפתיחת הקבר בלא דם וכ"ש בכמין קליפה ויבחושין.\par ואם תאמר מ"ש ממפלת רוח דתנא באינה יודעת מה הפילה ר' יהושע אומר א"א לפתיחת הקבר בלא דם א"ל התם דהפילה שפיר יש בו פתיחת הקבר אבל יבחושין ועפר אינה לידה אלא כמקור שהפיל טיפין הוא וכרואה דם יבש בעלמא הוא והיינו נמי טעמא דלא אמרינן בדגים וחגבים ושרצים שאין עמהם דם תטיל למים ואם נמוקו טמאה דהתם בריה נינהו ודאי ונולדת היא אלא שאינה טמאה וכן בחתיכה דבשר גמור הוא אינה נמוחה אבל כאן רואה היא ומכה יש לה שממנה מפלת כן ואע"פ שהיתה בחזקת מעוברת והפילה לאו מעוברה הוא אלא של מכה הוא. }
\twocol{הא דאמרינן מעיקרא \textbf{דנין יצירה מיצירה ואין דנין בריאה מיצירה.}  לאו למימרא דסתרי אהדדי דהא אפשר לי' למיגמרינא לתרווייהו אלא ה"ק זו אינה ג"ש כלל, והיינו דאקשינן מאי נ"מ הא תנא דבי רבי ישמעאל ולא מפרק הני מילי היכא דליכא דדמי ליה אבל היכא דאיכא דדמי ליה מדדמי ליה ילפינן כדאתמר בעלמא אלא ודאי משום דלא אמרינן אלא היכא דסתרי אהדדי והכא תרווייהו דגמר.\par  והדר אקשי' ועוד נגמר בריאה מבריאה [ומשני ויברא לגופיה] וייצר לאפנויי ודנין יצירה מיצירה דהשתא ודאי ליכא למיגמר אלא חד במופנה הילכך מדדמי ילפינן דאף על גב דוייצר מופנה גבי אדם ליכא למיגמר בריאה דתנין מיני' בדין מופנה מצד אחד דאפנויי דויצר דאדם לאו להך ג"ש הוא והוה ליה כשאינו מופנה כל עיקר אי נמי השתא לא מסיק טעמיה אלא מפרש ואזיל הוא ואמסקנא ניחא דליכא למיגמר דבריאה כלל כדבעי למימר קמן. }
\twocol{ והא דאמרינן \textbf{ויברא גבי תנינים לאו מופנה.}  אי קשיא הא כתיב נמי ישרצו המים ההוא אין כתוב בעשייה אלא בצוויי, ופי' רש"י ז"ל דכיון שאין מופנה משני צדדין ומשיבין ה"נ יש להשיב מה לאדם שכן מטמא מחיים.\par  ול"נ דהאי לישנא קמא לא צריך פירכא דלכ"ע מופנה משני צדדין עדיף ממופנה מצד א' וכיון דע"כ יצירה יצירה גמרינן ה"ל בריאה דאדם לגופיה וגבי תנינים נמי לגופיה ואין מופנה כל עיקר וכל ג"ש שאינו מופנה כל עיקר אין למידן הימנה.\par וא"ת וייצר האדם לגופי' ודבהמה מופנה ודגמרינן ויברא דתנין לגופיה ודאדם מופנה וגמרינן היינו דקאמרי ומאי נ"מ זה כלומר אמאי ניחא לך לאפנויי לחדא לגמרי ומיגמר מינה ולא לאפנויי תרווייהו ומיגמר מנייהו ופריק לרבנן הא עדיפא דהא אין משיבין ולר' ישמעאל נמי הא עדיפא דהיכא דאיכא מופנה משני צדדין איהי עדיף ולהכי אפנויי רחמנא לבהמה משני צדדין דשדינן מופנ' דכולהו בגוה כי היכי דלא נימא באידך מופנה מצד אחד הוא דכל היכא דאיכא למישדי שני צדדין דמופנ' בדידיה שדינן ומיניה גמרינן בין לרבי ישמעאל בין לרבנן אבל ללישנא דרב אחא הויא דבעי' והאי מאי פירכא משום דאפילו כשאנו גומרין יצירה יצירה יכולין אנו לגמור בריאה בריאה אע"פ שאינה מופנה כל עיקר אלא שמשיבין ולפום הכי בעי' מאי פירכא ורבנן דפליגי עליה דר"מ במתני' לא גמירי כדאשכחן בפרק כל היד שאין אדם ג"ש מעצמו, וכן פי' רש"י ז"ל.\par ואי קשיא לך לר"מ מאי פירכא ליהדר דינא ותיתי מכאן דכיון דגמר יצורה ואתו בהמה חיה ועוף כי פרכת גבי תנין מה לאדם שכן מטמאו מחיים נימא בהמה תוכיח א"נ נגמר מוייצר דבהמה למד מלמד א"ל מה לשניהם שכן מטמאין במגע ובמשא תאמר בדגים שאינן מטמאין ואע"פ שמקבלין טומאה טומאת עצמן אין להם. }
\newchap{דף \hebrewnumeral{23}}
\twocol{\textbf{למימרא דחיי.}  פי' רש"י ז"ל דהא אחותה לא מיתסרא אלא בחייה דאין איסור אחות אשה לאחר מיתה ותמהני א"כ יפה שאל ר' ירמיה ונימא נפקא מינה לענין אתסורי באמה ואם אמה דאמות אפילו לאחר מיתה מן התורה ועוד נפקא מיניה לאתסורי באחותה שנים וג' ימים.\par  אלא כך פירשו למימרא דחיי שאם א"א לו לחיות כלל אין קדושין תופסין בנפל גמור כגון בת שמונה והלא הרי הוא כאבן לכל דבר אלא ודאי סבר ר' ירמיה דחיי והאמר ר' יהודה אמר שמואל לא אמרה ר' מאיר אלא הואיל ובמינו מתקיים, פי' הואיל לאו דוקא דהא לא טעמא הוא לר"מ אלא משום יצירה יצירה או שגלגל עיניו כשל אדם או בשיש בו מצורת אדם אלא ה"ק לא אמר ר"מ שהוא ולד שיעלה על דעתו שהוא חי אלא ולד הוא לענין טומאה שבמינו מתקיים כנפל גמור שהוא אינו מתקיים ובמינו מתקיים. }
\twocol{\textbf{א"ר ירמיה בר אבא אמר רב הכל מודים וכו'.}  פי' ר' ירמיה משמיה דרב פליג אדשמואל ור' יוחנן דפרשי לעיל טעמיה דר"מ משום יצירה יצירה או משום גלגול עין שלדבריהם אפילו תייש גמור במעי אשה ולד מעליא הוא לטומאות לידה וכ"ש גופו אדם ופניו תייש דאיכא מקצת אדם.\par  וה"נ משמע דס"ל לרב יהודה משמיה דרב כותיה מדקאמר הואיל ובמינו מתקיים ולרב ירמיה משמיה דרב לית ליה הנהו טעמי אלא ר"מ ורבנן בסברא בעלמא פליגי בשפניו אדם ונברא בעין א' כבהמה שר"מ אומר מצור' אדם בעינן והא איכא וחכמים אומרים כל צורת ממש.\par  וה"ה לר' מאיר' דבמצח ועין וגבן העין ולסת וגבת הזקן סגי אלא להכי נקט פניו אדם ועין אחד כבהמה להודיעך כחן דרבנן והא דאמרי ליה רבנן והא איפכא תניא לאו איפכא תניא דוקא דמתהפכי תרתי סברי דהא לא (מתסרא) [מתהפכא] סברא דרבנן לר"מ אלא איפכא בלישנא לכולהו ואיפכא בסברא דר"מ דמפכא לה ברייתא לרבנן וכדפירש רש"י ז"ל.\par ולסבריה דרב הא דתני' לקמן המפל' דמות נחש אמו טמאה לידה ר' יהושע יחידאה היא ולית ליה דר"מ וכ"ש דרבנן וההיא דתניא לעיל נראין דברי ר"מ בבהמה וחיה בהכי נמי מתוקמ' בבהמה וחיה ומקצ' סימנין דאדם דכיון דהיא עצמה עיניה הולכות כשל אדם במקצת סימנין נעשית כאדם גמור מה שאין כן בעופות שאפילו כל פניו כאדם ועיניו לצדדין אינו כלום.\par  והאי דדחי רב אחא בריה דרבא לעיל תבדוק לרבנן דמודו רבנן בקריא וקפוף הואיל ויש להם לסתות כאדם דחייה בעלמא היא דדחי בסברת דר' אלעזר בר צדוק אבל לפום מסקנא לרבנן לסתו' חדא מצורות דפנים נינהו וצריך נמי גבין וגבת זקן דאדם ועין נמי דאדם ואף ע"פ שהולכות לפניהם צריך צורת דאדם באוכמא.\par  וי"מ דלרב ירמיה גופיה אית ליה אליבא דר"מ יצירה יצירה ואי כולה תייש בפניו וגופו אמו טמאה לידה הואיל ובמינו מתקיים והא דאמה גופו אדם ופניו תיש ולא כלום משום שאין זה לא מין בהמה ולא מין אדם והואיל ואין לו מין שמתקיים דברי הכל ולא כלום.\par  ולפי דבריהם קשיא, א"כ מנא ליה לר' ירמיה א"ר דר"מ בפניו אדם ונברא בעין א' כבהמה פליג דילמא בההוא כרבנן ס"ל דלאו אדם הוא ומין בהמה נמי אינו שאין לך בבהמה כמותו והם אומרים קסבר רב דר"מ ורבנן בתרתי נמי פליגי ממאי מדאמרי ליה רבנן כל שאין בו מצורת אדם ולא קתני וחכמים אומרים אמו טהורה א"נ וחכ"א אינו ולד שמעי' דר"מ דמטמא נמי במקצת צורה ואמרי ליה לא כל צורה בעי למעוטי צורה בהמה גמורה ולמעוטי נמי מקצת צורת אדם והא דאמרי ליה רבנן והא איפכא תניא איפכא לגמרי הוא שר"מ אמר כל צורת לגמרי מ"ט או כולו אדם או כולו בהמה וחכ"א מצורת אדם ולא פניו בהמה אבל במקצת צורת אדם ולד הוא והיינו דקתני מתני' כל שאין בו מצורת לאפוקי כולו בהמה ומדלא קתני אמו טהורה סתם משמע ליה דבתרתי פליגי וברייתא נמי דמסייעא להו כך מפורש בספר הישר.\par  ודברי רש"י ז"ל יותר נראין והוא הלשון הראשון שכתבנו ואע"ג דקשיא נחש דר' יהושע כדפרישית.\par  והא דתניא המפלת דמות לילית אמו טמאה לידה ולד הוא אלא שיש לו כנפים ולא אמרינן משום דגופו תיש ופניו אדם היינו נמי טעמא משום דודאי פניו אדם אע"פ שגופו תיש בתר צורת פנים אזלינן אבל בדמות לילית ס"ד אין כאן צורת אדם כלל אלא צורת לילית היא זו בין בגוף בין בפנים קמשמע לן דלילית גופה ולד הוא אלא שיש לו כנפים. }
\newchap{דף \hebrewnumeral{24}}
\twocol{\textbf{אמר רבא ושטו נקוב אמו טמאה.}  פירש"י ז"ל קסבר טרפה חיה. וקשה להעמיד דברי הרב רבא שלא כהלכה ועוד אני תמה וכי מפני שאין טרפה חיה י"ב חדש נטהר אמו של זה והלא הנפל שהוא כאבן ואינו יכול לחיות או שיצא מת ומחותך אמו טמאה זה שיצא נקוב הושט וכלו לו חדשיו לא כ"ש ואפילו הולד שנימוק בשליא אמו טמאה והאיך יטהרוה שאלו היה דבר שמתחלת ברייתו הוא אפשר לומר שאינו ראוי לבריית נשמה וטהור אבל זה שמא עכשיו נקב ומה בינו למחותך ויצא איברים אברים.\par לפיכך נ"ל שכל הנולד בטריפות ודאי אמו טמאה היא ואפילו היה טרפותו בתחלת ברייתו שהרי ראוי הוא לחיות י"ב חדש וכ"ש בטרפות נקב וחתך. והא דאמר רבא ושטו נקב טמאה לד"ה קאמר ולא בא אלא להשמיעינו שושטו אטום אמו טהורה שאין זה בכלל אדם הואיל ונברא שלא כדרך החיים. }
\twocol{ והא דאמרי' \textbf{קא מפלגי בטרפה חיה}  שהזיקיקו לרש"י ז"ל לטהר ולד טרפה נ"ל שלא הקפידו אלא על לשון הברייתא שאמרו וכמה כדי שינטל מן החי וימות דקסבר האי תנא דכל שנברא אטום בלא חיתוך איברים עד מקום שאלו ינטל מן החי וימות אינו בכלל ולד ולא שיהא זה נקרא טרפה אלא זה אינו נולד הואיל ונברא אטום אבל נחלקו האמוראין כמה הוא כדי שינטל מן החי וימות ופי' ר' זכאי עד לארכובה ודקדקו ממנו שהוא סובר טרפה חיה דהא קאמר שבכך החי מת (לרש"י נטל) [לכשינטל] ממנו ור' ינאי אמר עד לנקובה שבכך נעשה נבלה אבל טרפה אינה מתה ר' יהושע דאמר עד טבורו קסבר בין זו בין זו חיות הן וכל זה אינו אלא בשיעור כמה כדי שינטל מן החי אבל בולד שנטל ממנו לא נחלקו (במינו) [בו] ולא אמרו כאן אלא הולד כשהוא אטום.\par  ומה שפי'רש"י ז"ל אטום חסר אינו נראה אלא אטום כמשמעי שאין לו חיתוך איברים ואין לו חלק שבהן אלא כמין גולם אטום ודמיא להא דתניא לקמן בריית גוף שאינו חתוך וכו'. }
\twocol{ הא דאקשי' \textbf{ואם איתא ליתני שמא מגוף אטום (ופניו) [או ממי שפניו] המוסמסין באתה.}  א"ל איבעי למיתני טובא ליתני שמא באת מפניו תיש או אפילו פרצוף אחר או שיש לו שני גבין ושתי שדראות וכמין אפיקותא דדיקלא וכן כיוצא בהן א"ל הנהו לא שכיחי ולא ה"ל למיתני. אבל פניו ממוסמסין ה"ל למיתני משום דשכיח נמי טפי מגוף אטום. }
\twocol{הא דאמרינן \textbf{ושמואל סבר בריה בעלמא איתא וכי אגמריה רחמנא למשה בעלמא.}  פירש"י ז"ל אותו המין אסר לו וק"ל א"כ לשמואל אפילו יוצא לאויר העול' נמי לישתרי דה"ל כקלוט בן פרה דשרי ונראה מדבריו דבין לרב בין לשמואל במעי טהורה לא חיי הלכך יצא לאויר העולם משום נפל אסור אפילו לשמואל והא דפריך רב שימי ממתניתין ר' חנינא בן אנטיגנוס אומר וכו' לרב ה"ה לשמואל אלא גביה הוה קאי דבר בריה הוה.\par ולא נהירא ועוד דהתם בפ' ואלו מומין (דף מג ע"ב) תנן לה למתני' גבי מומי כהן איזהו גבן ר' חנינא בן אנטיגנוס אומר כל שיש לו שני גבין ושדראות והוי ביה למימר' דחיי והאמר רב באשה אינו לד בבהמה אסור באכילה ולא מדכרין התם דשמואל בכלום בעולם.\par  אלא הכי משמע פירושא לכ"ע מינא בעלמא ליכא כי פליגי בבריה רב סבר אפילו בריה בעולם ליכא דלא חי הילכך כי אגמריה רחמנא למשה במעי בהמה אגמריה דבחוץ לא צריך נפל הוא. ושמואל סבר בריה בעלמא איתא דחיי וכי אגמריה רחמנא בשיצא לאויר העולם דלא תימא כקלוט בן פרה הוא אבל במעי בהמה דאפילו נפל שריא איהי נמי שרי. }
\newchap{דף \hebrewnumeral{25}}
\twocol{\textbf{המפלת שפיר מלא בשר נימוח מהו.}  פי' קא מיבעיא להו לרבנן דפליגי עליה דר' יהושע מיפלגי נמי בבשר נימוח או לא אמר להם לא שמעתי אמר לפניו ר' ישמעל בר' יוסי משום אביו כך אמר אבא מלא דם טמאה נדה מלא בשר טמא' לידה שהיה ר' ישמעאל סבור שלא אמר אביו כדברי היחיד ולפיכך דחה רבי ואמר שמא כדברי ר' יהושע אמרה וזה שאמר מלא בשר לאו דוקא אלא ה"ה למחוי עכור ולא בשר אלא להוציא צלול אפילו לר' יהושע.\par  ויש שגורסין בה כמאן כסומכוס מדהא כיחידאה הא נמי דילמא כר' יהושע אמרה ואינו בספרים.\par וא"ת כיון שרבי לא קבלה אפילו בבשר נימוח שיהא ולד ריב"ל מנין לו דקאמרי בצלול מחלוקת אבל בעכור ד"ה ולד זה אינה שאלה דריב"ל כר' יוסי ס"ל וקסבר ר' יוסי לרבנן אמרה כדסבר נמי ר' ישמעאל בר' יוסי ועוד דכיון דרבי לא שמעתי אמר אינה תשובה לדברי ריב"ל שאם ר' לא שמע ריב"ל שמע לא ראינוה אינה ראיה.\par  וי"א אין אומרים בדברים אלו זו דומה לזו שאפשר למימר עכור ולד ובשר נימוח שמא אינו ולד. ואיכא למימר נמי איפכא ולפיכך נחלקו בכולן. }
\newchap{דף \hebrewnumeral{26}}
\twocol{הא ד\textbf{אמר רב הונא בר תחליפא משמיה דרבא ולד מדנפיק קבא דרישיה הויא ליה לידה סנדל עד דנפיק רוביה.}  פי' רש"י ז"ל משום שאין הראש פוטר בנפלים. ושמואל דאמר הכי איתותב במס' בכורות אלא איכא לפרושי דאפילו למ"ד הראש פוטר בנפלים ה"מ בולד שלם או אפילו במחותך שנגמר' צורתו אבל סנדל שלא נגמרה לו צורה לא חשיב רישיה למהוי ביה לידה. }
\twocol{ הא דאמרי' \textbf{תלת מתני' ותרתי שמעתא שיעורן טפח.}  ואקשי' תרתי חדא היא היינו טעמא דלא מקשינן תלת ד' הוויין משום דהוה איכא למימר רבי שילא לית ליה דר' חייא דשיעור אזוב טפח ומ"ה מקשינן אם כן [תרתי חדא היא, ועוד א"ל] תרווייהו כי הדדי נינהו וחדא נקט. }
\twocol{והא דאמרינן \textbf{השתא דאתית להכי הך נמי פלוגתא היא דקתני סיפא א"ר יהודה לא אמרו טפח אלא מן התנור לכותל.}  ק"ל וניחשוב מן התנור לכותל דמודה ר' יהודה וה"מ למיחשבי' בדברי הכל וא"נ לא חשיב חד דדברי הכל ה"מ למימר סתמא אבן היוצא מן התנו' טפח דהא קא חשיב בכה"ג טפח סוכה למר בדופן ג' ולמר בדופן ד' כיון דכולהו אית להו דופן טפח קחשיב ליה ה"נ כולהו אית להו אבן היוצא מן התנור טפח מכאן או מכאן.\par  ואיכא למימר התם הכל מודים ששיעור דופן א' בסוכה טפח והשאר כהלכתן. וכי אמר דופן סוכה טפח ליכא למטעי במידי דפלוגתא. אבל הכא אי אמר סתמא אבן היוצא מן התנור הוה משמע מכל צדדין ואתיא כרבנן ואי פריש נמי אבן היוצא בין התנור לכותל טפח הוה משמע הא בין תנור לבית אינו טפח כר' יהודה ואיהו במילתא דפלוגתא לא איירי לא כמר ולא כמר. ולא בעי לפרושי תרווייהו ובודאי משמע לכאורה דהשתא דאתינן להכי ומפרקינן דהך נמי פלוגתא היא הדר ביה מתירוציה דקאמר כי קאמרי היכא דבצר מטפח לא חזי וכו'.\par  והשתא קשיא לי טובא וליחשוב הני דתנן במס' כלים פי"ח גדד לשתי כרעים טפח על טפח לוכסן או שמעטה פחות מטפח טהורה רישא ל"ק דטפח על טפח לא קאמרינן סיפא ליחשוב. ועוד שם בפרק (בתרא) [כ"ט] חוט מאזני' של חנוני ושל בעלי בתים טפח יד קרדום מלפניו טפח שירי הפרגל טפח יד מקבת ושל מפתחי אבנים טפח. וי"ל כי קאמרינן היכא דבציר מטפח לא חזי והני כ"ש דבציר מטפ' (דידהו) [דידות] הוי וכי קאמרינן השתא דאתית להכי לאו למימרא דליתי' לשנויה דשנינן אלא למימר' דמההי' לא תיתי תיובתא בבי מדרשא כלל. }
\twocol{ הא דאמר רב \textbf{אין הולד מתעכב אחר חבירו כלום.}  ראיתי מקצת בעלי פירושין שכתבו דלית ליה לרב הא דאמרינן לקמן מעשה ונשתהא ולד אחרונה אחר חבירו שלשים יום ולית ליה נמי מעשה דיהודה וחזקיה ולית ליה נמי האי דאמרינן בכתובות וביבמות ג' נשים משמשות במוך קטנה מעוברת מניקה מעוברת שמה תעשה עוברה סנדל אלא קסבר אין אשה מתעברת וחוזרת ומתעברת ולפיכך אין ולד מתעכב אחר חבירו כלום.\par ואין דבריהם נראין דג' נשים מתניתא הוא ונימאתהוי תיובתיה דרב. ועוד מעשה דיהודה וחזקיה בני חביביה דהוא יושב לפניו והן יושבין עמו בבה"מ היכי אפשר דלא חזי ליה ואם איהו אומר דלא היו דברים מעולם מאן מהימן לאסהודי עלייהו.\par  אלא היינו טעמא דרב דקסבר אין אשה מתעברת וחוזרת ומתעברת בין נפל בין של קיימא אלא א"כ נעשה א' מהם סנדל וסנדל כרוך עם הולד הוא יוצא שחבור אתו ונדבק בו והיינו טעמא דמוך אבל כשהאשה מתעברת תאומים טפה אחת היא שמתחלקת וכשהן נגמרין לז' או לט' אין הולד מתעכב אחר חבירו כלום אלא א"כ הפילה א' נפל וא' שליא אבל פעמים שנתחלקו לשתים וא' מהן נגמר לט' וא' לז' ובזה מודה רב שהול' משתהא אחר חבירו כדי שתגמור צורתו בזמנו. והיינו מעשה דיהודה וחזקיה ומיהו לית ליה אפוכי שמעתא דלקמן (כ"ד) [כ"ג] לולד דבחד ירחא לא משתהא אלא ל"ג אית ליה.\par  והא דאמרינן לעיל סנדל מהו דתימא הואיל וא"ר יצחק עד קמ"ל שניהם הזריעו בבת אחת ודאי קשיא דהא איכא סנדל דמתעברת וחוזרת ומתעברת וזה זכר וזה נקבה כדפרישי' במשמשת במוך. ואיכא למימר אין ודאי דמצי למימר הכי אלא שמא תאמר היכא דבעל ופירש מדהאי זכר האי נמי זכר קמ"ל אפי' בכה"ג חיישינן שמא שניהם הזריעו כא' והאי זכר (נמי) והאי נקבה כנ"ל. }
\twocol{\textbf{אין תולין את השליח אלא בדבר של קיימא.}  פי' רש"י ז"ל שכיוצא בו מתקיים אם היו חדשיו כלין למעוטי שאם הפילה דבר שאינו ראוי לבריית נשמה כגון נברא בירך אחת או גוף אטו' ואח"כ הפילה שליא (פי') [אפי'] בתוך ג' חוששין לולד אחר, ופירוש חזייה לרב יהודה בישות משום דשמעה מרב ולא אמרה.\par ואינו מחוור דבן קיימ' לאפוקי כל נפל משמע וכדאמרן דילמ' כאן בנפלי כאן בבן קיימא ולא ידעתי מי הזקיקו לשנות פירושו אלא הא דתלמיד' דרב פליגא אדרב יהוד' דאמר לעיל משמיה דרב הפילה נפל ואח"כ הפיל' שליא כל שלשה ימים תולין אותה בולד ושאר תלמידים דרב אומרי' משמי' דאין תולין את השליא בנפל אפילו יום א' אלא א"כ יצאה עמו אבל בבן קיימא תולין אותה אפילו מכאן ועד י' ימים.\par ושמעתי שפירשו בירושלמי במס' זו (ג, ד) לפי שאין השליא פורשת עד שיגמר לפיכך אין תולין אותה בנפל.\par  ובשאלתות דרב אחא משבחא ז"ל כתב לכך תולין אותה בבן קיימא דאמרי' אגב חיותא דולד בזעא לשליא ונפיק. אבל נפל דלית ביה חיותא לא. ומ"ה חזייה שמואל לרב יהודה בישות דשמעיה דאמר משמיה דרב דכל ג' תליא שליא בנפל וכיון דשמעינהו לכולהו תלמידי דרב דאמרי אין תולין כלל אמר ודאי רב יהודה טעי. }
\newchap{דף \hebrewnumeral{27}}
\twocol{\textbf{מ"ט דר' שמעון וכו'.}  פירש"י ז"ל נהי נמי דנימוק מ"מ גופו של מת כאן הויא וה"ל כרקב וכנצל. וק"ל הא דאמר רשב"ל לקמן בשמעתין שפיר שטרפוהו במימיו טהור להוי כרקב וכנצל. ועוד לר"מ נמי בבי' החיצון אמאי טהור ליהוי כרקב וכנצל. וא"ת איהו נמי סבר כל טומאה שנתערב בה מין אחר בטלה אלא מאן תנא דפליג עליה דקאמרת קסבר ר' שמעון והא דתניא מלא תרוד רקב שנפל לתוכו עפר כל שהוא טמא ור' שמעון מטהר אמאי תרמייה הא דכ"ע טומאה כיון שנתערב בה מין אחר טהורה.\par  ובתוספ' הקשו לה מדאמרינן ואזדא ר"ש לטעמיה מדאמר א"א שלא ירבו שתי פרידות עפר על פרידה אחת של רקב ויבטלנו ואמאי נהי נמי דבשיעור מצומצם כגון מלא תרוד רקב איכא למימר הכי גבי שליא מ"ט דאפילו הוה בה תרי שיעורי דמלא תרוד מטהר ר"ש דסתמא תנן ואע"ג דליכא למימר א"א שלא ירבו וכו', והם מפרשים הסוגיא כולה בענין אחר ברם נראין דברי מקצת ראשונים שפירשו מ"ט דר"ש דמטהר לגמרי והרי אנו מוצאים בכל יום ולדות חתוכים בשליא והאיך אפשר שלא תהא בכולה כזית ג) שלא נמוק לגמרי קודם שתצא שאפי' נחתך כולו לחצאי זתים מצטרפים הן בתוך השליא לטמא באהל ואמאי מטהר ליה לגמרי, ומפרקי' קסבר ר"ש כל טומאה שהיא כשיעורה ולא יותר שנתערב בה מין אחר בטלה דאמרינן כיון דהיא צריכה שיעור א"א שלא ירבה מין אחר על מקצתה ומבטלה ואף כאן כיון שנמוק הולד אע"פ שנשתייר ממנו (כחצאי) ד) זתים א"א שלא ירבו שתי טיפי מים ודם על מקצת בשר שלא נמוק ומבטלו והיינו דאמרינן ואזדא ר"ש לטעמי' דאמר א"א שלא ירבו שתי פרידו' עפר על פרידה אחת של רקב ומבטלו ובצר לה שיעורא ור"מ סבר לא מבטל אלא א"כ הוציאו לבית החיצון שנטרפו מימיו לגמרי וטהור. והא דאמר לו לר"מ כשם שאינו בבית החיצון כך אינו בבית פנימי לומר שאף בבית החיצון היה לנו לחוש שמא יש בו כזית בשר (שנמוק) ה) אנא משום בטול ואמר להו אינו דומה שזה נמוק לגמרי וה"ל כמים בטריפת בני אדם. אבל דרך לידה אינה נמוק לגמרי וביטול אינו מועיל. ואקשי להו רבא לרבנן דבי רב אדרבה כיון דרקב יותר הוא מן העפר היאך יאמ' ר"ש שהמועט רבה על מקצת המרובה ומבטלו ומגרע שיעורו אדרבה יש לנו לומר שהמרובה עומ' לבד על הממועט ומבטלו לגמרי. והיינו דאמרי' לקמן והשתא דאמרת טעמיה דר"ש סופו כתחלתו גבי שליא מ"ט דקס"ד שהמים והדם שבשליא מועטין הן אלא שרבין על מקצת בשר ומבטלין אותו כדפרישית וא"ר יוחנן משום ביטול ברוב נגעו בה שאפילו היו שם שני חצאי זתים או כזית שלם. יש במי שליא ודם שבה לבטל את כולה ואין אנו צריכין לומר שרבין על מקצת ומבטלו ומגרע שיעורו אלא על כולו הם רבים ומבטלין אותו ובהא פליגי דר"מ סבר אין טומאה בטלה ברוב מלטמא במשא ואהל דהא (קאמר) ו) מאהיל על כולה ומיהו בבי' החיצון טהור שנמוק לגמרי וה"ל אפר שרופין ופחות ממנו. ור"ש סבר בטלה היא לגמרי. }
\twocol{\textbf{מלא תרווד ועוד עפר בית הקברות טמא.}  פי' רש"י ז"ל לאו רקב של מת ממש אלא כגון שנקבר בכסותו או בקרקע בלא ארון ויש כאן מלא תרוד ועוד מאותו עפר דהיו מעורבין עפר ורקב.\par ואין פי' זה נכון דהא אמרין כל שתחלתו דבר א' נעשה גנגילון ואע"פ שיש בו שיעור מן הרקב דומיא דסיפא ומדאמרינן סופו כתחלתו (מאי) [מה] תחלתו דבר א' נעשה גנגילון אלמא פשיטא מילתא דכ"ע כל דבר א' עושה גנגילון בתחלתו. ועוד מדתניא איזהו מת שאין לו רקב נקבר בכסותו אלמא אין לו רקב כלל אפילו מלא סאה דאלת"ה ליתני שאין לו תרוד רקב. ועוד מדסוגיא במסכ' נזיר פרק כהן גדול (דף נ"א) דאמרי' שני מתים שנקברים זה עם זה נעשה גנגילון זה לזה. ואם קברן זה בפני עצמו וזה בפני עצמו והרקיבו ועמדו על מלא תרוד רקב טמא אלמא אין הדבר תלוי בשיעור מן הרקב אלא כל דבר שנקבר עמו נעשה לו גנגילון. וכן בכולה סוגיא דהתם הכי משמע דכי גמירי למלא תרוד רקב דוקא דנרקב בעיניה.\par אלא הא דתניא מלא תרוד ועוד עפר קברות פירושו כגון שנקבר המת ערום על גבי רצפה של אבנים והרקיב ונפחתה מערה ונתערב עפרה ברקב דקסבר ת"ק רוב מתים יש בהן רקב מלא תרוד שבכאן של מת והמותר הוא עפר בית הקברות ואלמלא שנמצא שם ועוד על כרחינו מת זה לא היה בו מלא תרוד דא"כ עפר בית הקברות להיכן הלך אבל מאחר שנמצא כאן ועוד זה א"א בלא מלא תרוד של מת ור"ש מטהר וחזרו לטעמייהו דהיינו מלא תרוד רקב שנפל לתוכו עפר כל שהוא. }
\twocol{וא"ר יוחנן ד\textbf{ר"ש וראב"י אמרו דבר א'.}  ואליבא דר' חייא דפריש למתניתן דתקבר לומר שנפטרה מן הבכורה ולא משום טומאה. והך סוגיא דר' יוחנן הוא דהתם בדוכתא במס' בכורות (כג, א) מסיק טעמיה דר' חייא משום דה"ל טומאה סרוחה. וצ"ע. }
\twocol{\textbf{שפיר שטרפו במימיו.}  גרסי' וכן בפר"ח ז"ל, ופי' שטרף השפיר ונמוקה צורתו אבל עדיין הוא קיים נעשה כמת שנתבלבלה צורתו באור וטהורים דכיון שאין באבריו צורת בשר ולא צורת עצם נפק ליה מדין כזית ועצם כשעורה וטהורין לגמרי. }
\twocol{ והא דא"ר יוחנן \textbf{מת שנתבלבלה צורתו מנ"ל דטהור.}  לאו דוקא דלא כרבנן אלא מדר' אליעזר שמע ליה ר' יוחנן דקסבר מודו ליה רבנן בשלא נעשה אפר כדפרישי'.\par  ויש לפרש דקסב' רבינא דר"ילא מודה לי' לר"ל בשפיר שטרפו מימיו דמדלא א"ל בשלמא שפיר שנטרפו מימיו דקאמר טהור לחיי אלא נתבלבלה צורתו שלמה מנלן אלמא ה"ק מנלן דטהור דגמרת מינה לשפיר לא הא ולא הא איתנהו. ועלה קאמר רבינא דר"י דמטמא שפיר שנטרפו מימיו לגמרי כר' אליעזר אמרה דהאי כאפר שרופין הוא ומיהו במת שנתבלבל' צורתו דקא מתמה מנלן לד"ה אתיא.\par וזה הלשון לדברי מי שגורס שפיר שנטרפו מימיו דמשמע שנטרף לגמרי וחזר למים, אבל לפי גר"ח ז"ל שנטרפו במימיו. נראה דהיינו נתבלבלה צורתו בלחוד.\par ויש לי עוד לומר דר' יוחנן הלכה קא מיבעי ליה, וה"ק ליה מנלן דטהור כרבנן דילמא טמא כר' אליעזר דמסתברא טעמיה. אילימא מדרבי שבתאי קא גמרת הלכה דהוא אמורא וקא פסיק הלכה כרבנן. }
\newchap{דף \hebrewnumeral{28}}
\twocol{\textbf{מעשה היה וטהרו לו פתחים קטנים.}  פרש"י ז"ל טמאו לו פתחים גדולים של ד' טפחים וטהרו לו קטני' הפחותים מד' כשאר מתים גדולים שהפתחים הגדולים מצילין על הקטנים דקי"ל פתחו בד'. וה"מ להציל על הפחות מד' טפחים והכי אמרינן במס' אהלות המת פתחו בד' בד"א להציל על הפתחים אבל להוציא את הטומאה בפותח טפח. זה כתב הרב ז"ל.\par  ואין הדין הזה אלא כשהפתחים כולן סתומים או מגופין שבהן שנינו המת בבית ולו פתחים הרבה כולן נעולין כולן טמאין, פי' משום דסוף טומאה לצאת נפתח א' מהם אע"פ שלא חשב עליו טיהר את כולן פירש כיון דנעולין הן וה"ה למגופין אבל בפתוחין כל פותח טפח מוציא טומא' לצד ב' ואין לו הצלה כלל. }
\twocol{\textbf{המפלת יד חתוכה.}  פי' רש"י ז"ל חתוכה שיש לה חיתוך אצבעות. וק"ל בלאו הכי נמי ליחוש ללידה שהרי אפילו השפיר שאין לו אפילו חתוך ידים עצמן אמו טמאה לידה.\par  ואיכא למימר הכי ספיקא הוא ואם הפילה יד גמורה שאינה חתוכה אומרי' מגוף אטום באת כשם שהיא משונה כך באת מגוף משונה ושמא לאו מגוף באת אלא חתיכה של בשר שנעשית כמין פיסת היד היא הילכך אמו טהורה תולין להקל שרגלים לדבר.\par והרב ר' אברהם בר דוד ז"ל מפרש שלא אמרו חתוכה אלא לענין מביאה קרבן ונאכל דמדקתני ואין חוששין כלל במשמע ואלו בשאינה חתוכה אינו נאכל. (אלא) א) לענין האם טמאה מ"מ. ואין זה לשון הגון מדקתני ברייתא אמו טמא' ואין חוששין ולא קתני מביאה קרבן ונאכל ואין חוששין. }
\newchap{דף \hebrewnumeral{29}}
\twocol{והא ד\textbf{אמ' רב פפא כתנאי יצא מחותך או מסורס.}  אלישנא קמא דר"א ור' יוחנן קאי דפליגי במחותך.\par  והא דאקשי ליה רב זביד למאי דמוקי ר' יוסי אומר משיצא רובו כתקנו מכלל דמסורס רובו נמי לא פטר קשיא ולימ' ר' יוסי אכתקנו פליג דלא מיפטר בראש עד שיצא רובו ומחמיר הוא.\par  וי"ל מדקתני ברייתא בדר' יוסי משיצא כתקנו משמע דלאיפלוגי אמסורס אתא דה"ל למימר ר' יוסי אומר כתקנו משיצא רובו וליפלוג אכתקנו והשתא פלוגתא אמסור' משמע ואוקומא רב זביד משיצא לתקנו בחיים ופלוגתא בדלחיים היא.\par והיינו נמי תנאי דת"ק סבר מחותך ומסורס משיצא רובו הרי זה כילוד כתקנו אע"פ שמחותך הראש פוטר ר' יוסי אומר משיצא כתקנו לחיים. כלומר אין הראש פוטר במחותך אלא בשלם שכיוצא בו יוצא כתקנו לחיים ולכ"ע לית להו דשמואג אלא ס"ל בשלם הראש פוטר והא דקתני איזהו כתקנו לחיים משיצא ראשו ה"ק איזהו כתקנו שיוציא ראשו תחלה ויוציא כדרך שהחיים מוציאין שר' יוסי אומר רוב ראשו וזהו שיעור אבל משיצא ראשו לאו לשיעור קתני אלא לדרך לידה קתני.\par  ואם באנו לפרש משיצא כתקנו לחיים לאפוקי נפל אפילו שלם כדשמואל ומאי כתנאי אפי' ללישנא בתרא דר"א ור"י תיקשי לן במס' בכורות פ' יש בכור לנחל' סלקא דשמואל בתיובתא ולא איתוקמא התם כתנאי. אלא שיש כיוצא בה בתלמו' תיובתא בחד דוכתא ותנאי בדוכתא אחריתי במס' תמורה. ועוד יש במסכת פסחים תיובתא בדריש שמעתא ותניא כותיה בשלהי דידה בפ' ערבי פסחים. אלא שאנו קיימנו שתיהן תיובתא. ותניא כותיה בשתי שמועות של ר' יוחנן וכבר כתבנו זה בספר המלחמות. }
\twocol{ הא ד\textbf{אמר ריב"ל עברה בנהר והפילה וכו'.}  בדין הוא דנירמי עליה מהא דתניא בריש פירקין ולשלישי הפילה ואינה יודעת מה הפילה מביאה קרבן ואינו נאכל אלמא לכ"ע הלך אחר רוב נשים לא אמרינן אלא מתוקמא ההיא כדתרצינן למתני' בשלא הוחזקה עוברה לפנינו. ודמתני' עדיפא לן למירמי. }
\twocol{הא דתניא ב\textbf{אשה שיצאה מלאה ובאת ריקנית}  דמחזקי' לה ביולדת בזוב וברואה נמי לאחר לידה כדמקשי' לקמן יומא קמא דאתיא לקמן ליטבלה דילמא שומרת יום כנגד יום היא אלמא ברואה השתא בימי זיבה מחזקין לה דוקא כגון שבאת ריקנית ואמרה ראיתי שלא בשעת לידה ואינה יודעת כמה ראיתי ואלו לא אמרה כן אינן מחזקינן אותה לא ביולדת בזוב ולא בשומרת יום אע"פ שלא בדקה כל אותן הימים שאין חוששין לראיה כל זמן שלא ידעה אלא בימי הוסת.\par  וה"נ משמע בפ' בתרא דתניא בטועה ראיתי ואינה יודעת כמה ראיתי אלמא איני יודע אם ראיתי לא כלום הוא שאם אין אתה אומר כן השוטה והחרשת והקטנה נמי שראתה אסורות לשמש לעולם שמא ראו והן אינן מרגישות ולא יודעות. }
\twocol{ והא דפריך מינה ר' יוסי בר חנינא ורבין לא ידע מאי תיובתיה משום \textbf{אימר הרחיקה לידתה.}  דמשמע דלית ליה לר' יוסי בר חנינא הרחיקה לידתה קשיא טובא וכי היאך סלקה על דעתו כן. והא אי לאו משום הך תשש לא היו מבטלין אותה בשבוע ג' בליליותא דמשום טבולת יום ארוך מטבילין אות' שמא כבר עברו לה ימי טוהר וכ"ש בשבו' ד' דאיכא למימר כבר עברו וכן טבילות דב"ה נמי משום יולדת והרחיק' לידתה ז' או שבועים הן ועוד שבוע דטהור הוא תשמש דאי ילדה ולד מעליא אפי' בשבוע ד' נמי טהורה היא דדם טוהר הוא וכ"ש בה' דטהור' ואם לאו ולד מעליא הוה לספיקה דר' יוסי בר חנינא תחלת שבוע רביעי ה"ל תחלת ונדה ושבוע דטהור הוא מותרת אלא ע"כ משום הרחיקה לידת' וחוששין שמא כלו ימי טוהר בסוף שבוע רביעי ויום אחרון שבו היה לה התחלת נדה כדמפר' ואזיל בגמ'.\par  אלא ע"כ ר' יוסי בר חנינא אגב חורפיה לא עיין בה ובגמ' ה"ל למימר ולטעמיך מ"ט דכל הני אלא אשכחן כמה דוכתי בתלמודא דהוה מצי למיפרך וליטעמיך ולא פריך ביה כלל. }
\twocol{\textbf{שבוע קמא מטבילין לה בלילותא משום יולדת זכר ונקבה.}  עיינו בתוספות שאין השבועין הנמצאין כאן בטבילות הלילו' שוים עם השבועין הנמנים כאן בטביל' הימים דהא למאי דס"ד מעיקרא שבאת לפנינו ביום וכן למאי דמתרצינן כגון שבאת לפנינו בין השמשות טבילות דלילותא מושכות עד לילה של שבוע שלאחריו כגון שבאת לפנינו בין השמשות של מוצאי שבת וכן שבאת לפנינו באחד בשבת ביום וטבילה ראשונה של לילה בליל שני בשבת ואחרונה במוצאי שבת וכן בשבוע שני ואלו טבילו' דימים דמשום זיבה ראשונה באחד בשבת ואחרונה בשבת. וליכא למימר דברייתא הכי קתני שהביאה לפנינו ג' שבועין טהורין חוץ מיום שבאת לפנינו שהרי אותו היום עילה הוא למנין שבועים ונמצאת זאת מותרת לשמש בלילי עשרין וחד שהרי אינה רואה כל אותה הליל' ולא יום שלאחריו אלא ע"כ יום שבאת לפנינו הוא ממנין שלשה שבועים טהורין. כל זה עיינו בתוספות.\par ודבר ברור הוא אלא כיון דמנין לידה וזיבה מיום א' בשבת הוא וכל טבילו' דעלמא הן דכל נדה ויולדת טבילתן בלילה של שבוע שני וטבילות דזיבה ביום בסוף שבוע שלהן לא חיישי בגמרא לפרושי הכא מידי. }
\newchap{דף \hebrewnumeral{30}}
\twocol{\textbf{כגון שבאת לפנינו בין השמשות.}  פי' רש"י הוא הדין דה"ל לאוקמ' בבאה לפנינו בלילה והוה ניחא טפי דתו לא הוה קשיא לקמן לסוף שבוע לטבלה ביממא דהא לא הוו ז' ספורים שהרי לא הפסיקה טהרה בתחלת היום ואין אותו יום שבאה לפנינו עולה לה לספירת נקיים אלא מדקתני ברייתא ג' שבועין טהורים משמע דכולן טהורים ובבין השמשות משכחת בהו מיהא פסיקת טהרה ואפי' ליום ראשון. כך פי' רש"י ז"ל. }
\twocol{ ל"ה טבילות דקאמרי ב"ה קשיא לן כיון דאוקים ב\textbf{באה לפנינו בין השמשות דיהיבנא לה טבילה בתריהן.}  תלתין ושש הווין. וראיתי בפירושים דכיון דתדא בשבוע היא לא קחשיב ואינו יודע מהו שאם בא לומר דטבילה דסוף שבוע רביעי הויא חדא בשבוע לא משמע הכי דהא טבלה נמי בימים הסמוכים לה ששה עד סוף ז' ובאור שביעי של שבוע חמישי גמרה טבילותיה וטהורה ואפשר שאותה טבילה ראשונה חדא בשבוע חשיבי לה ב"ה מפני שהיא נמנת לסוף שבוע שעבר ואותו היום עצמו נמנה לנו תחל' שבוע ללידה וטבילות שאח"כ ואט"ג דב"ש מנו לה ולא חשבי חדא בשבוע אינהי דמפשי טבילו' מנו לה כיון דמצטרפא בטבילות דלילות דשבוע א' אבל ב"ה לא מנו לה.\par  וה"ר אב"ד ז"ל כתב דאיכא לתרוצי דכיון דלאו פסיקא להו דאי אתאי ביממי להא טבילה לא חשיבי ב"ה כי היכי דתרצינן בטועה בפ' בתרא. וזה הלשון נכון בעיני דב"ש דקא מפשי טבילות מהדרי לאפושי בהו טובא וב"ה דלא מפשי בהו טפי לא חשיב' לה.\par  ובשם הרב חתנו ז"ל תירץ דבין השמשות דר' יהודה אפליגי ב"ש וב"ה ב"ש סברי כר' יהודה דספיקא הוא וב"ה סברי כר' יוסי דבין השמשות דר' יהודה יממא הוא. ולכך ליתא לטבילה יתירתי' ועומק גדול הוא אלא תימה גדול הוא היכי שתיק מיניה תלמודא. זה לשון הרב ז"ל. }
\twocol{\textbf{איידי דפתח בשבוע מסיק לה איידי דתנא טמא תנא טהור.}  פי' וה"ה דלענין צ"ה טבילות דב"ש ה"נ הוה קמ"ל בעשרה שבועים כולם טמאים או טהורים אלא משום דבעי למיתנא משמשת לאור ל"ה לא קודם לכן ולא לאחר כן משום חששות דאמרן תנא הכי.\par והק' בתוספ' כיון דמנינו עשרה שבועי נפישי להו טבילות שהרי שבוע ט' דטמא הוא ג' ימים ראשונים שבו איכא לספוקינהו בסוף לידה ותחלת נדה ויום ד' תחלת נדה ונמצאת צריכה טבילה לג' ימים בשבוע עשירי. ותירצו עד סוף שמוני' קחשיב לאחר פ' לא תשיב דהא לא תננהו אלא אגב גררא. וכ"ש למאי דפרקינן בסמוך דלא מיירי ב"ש אלא בלידה. }
\twocol{ והא דאקשי\textbf{יומא קמא דאתיא לקמן לטבילה דילמא שומרת יום כנג' יום היא}  לאו לב"ש מקשינן דהא אינהו לא זיבה גדולה ולא זבה קטנה קחשיבי אלא יולדת בזוב בלחוד כדאמרן. אלא לב"ה בעינן דהא דמחרצינן זיבה גרידתא לא קחשיב לב"ה לא צרכינן למימר הכי אלא דמקמי תשמיש קחשיבי כולהו דלבתר תשמיש לא קחשיבי. א"נ השתא דאתית להכי ליומא דחדא בשבוע לא קחשיב הדרי' מההוא טעמא דטבילת זבה חדא בשבוע נינהו ולפום הכי אקשי' ליחשוב דשומרת יום וה"ל ג' טבילות בשבוע זו ביום כיון שבאה בין השמשות וליחשוב ומפרקינן זבה גדולה קחשיב כלומר יולדת בזוב. א"נ זבה הוא קחשיב אי מיתרמיא ליה לפני תשמיש אבל זבה קטנה לא חשיב.\par  וק"ל ולימא דילמא כשילדה ראתה יום א' בימי זיבה וצריכה לשמור יום כנגד יום ואין ספירת ימי לידתה עולין לה ודאי כשם שאינן עולין לםפירת זבה גדולה וליטבי' כל שבוע קמא ביממי משום שומרת יום כנג' יום זיבה שלפני לידתה ואמאי פריך יומא קמא בלחוד.\par ואיכא למימר דקסבר האומר שהימי לידה עולין לשמור דזיבה קטנה ויולדת בזוב קטן דמקש' דלטבילה יומא קמא דילמא יולד' בזוב קטן היא והרי ספרה יום זה לפנינו בין לב"ש בין לב"ה מקשי' וכן נראה לי עיקר דימי לידה אין עולין גמירי לה לקמן בפ' בנות כותיים מדכתיב כימי נדת דותה תטמא מה ימי נידתה אין ראויין לזיבה ואין ספירת ז' עולה בהן אף ימי לידתה כן, והא ליתא אלא לספירתן דזבה גדולה אבל שימור דזבה קטנה אף בימי נדה עולה דאפשר הוא כדאיתא בשלהי בא סימן (נג, א). }
\twocol{והא דאמרינן \textbf{ש"מ תלתא.}  איכא למידק ולימא נמי ש"מ ד' דהא ש"מ ימי לידה שאינה רואה בהן אין עולין לה לימי זיבת' ואיכא למימר דההיא פלוגתא דאביי ורבא היא ורבה דאמר עולין קסבר הא מני ר' אליעזר הוא דאמר מסתר נמי סתרא. ולפום הכי נמי לא אמרי' ש"מ ר' אלעזר היא כדאמרינן ש"מ ר' עקיבא היא ור' שמעון היא. משום דלאביי דברי הכל אינן עולין הלכך לא פסיקא ליה. }
\twocol{מתניתין \textbf{בנות כותיים נדות מעריסתן.}  אוקמינן בגמרא לר"מ דחייש למיעוטא וקסבר ר"מ כותיים גירי אמת הן דהכי אסיקנא בב"ק (דף לח) לדידיה וכיון שהן גירי אמת והן מטמאות בנדה מן התורה יש לחוש לספיקן והיינו נמי דקתני אין חייבין עליהן על ביאת מקדש מפני שטומאתן בספק.\par וא"ת ולמה העמידו משנתינו לר"מ לחוד דחייש למיעוטא. והא אפי' לר' יוסי נמי אית ליה בנות הכותיים נדו' מעריסתן כדאמרינן בפ"ק דשבת (דף טז ע"ב) גבי י"ח דבר לר' יוסי בצרי להו ואמר ר' נחמן בר יצחק בנות כותים נדו' מעריסתן בו ביום גזרו כלומר גזירה בעלמא כדי שלא יטמעו בהן או גזירה משום מיעוט שהן טמאות.\par  י"ל כיון דמתני' קסבר כותיים גירי אמת הן דהיינו סבריה דר"מ ור' יוסי שמעינן ליה דפליג עליה וסבר גירי אריות הן כדאיתא במנחות בפרק ר' ישמעאל (דף סו) ובמקומות אחרים הילכך ניחא לן לאוקמא לדידיה ומדינא ועוד דאיהו סתם מתני' ולא למשקל תנאי מעלמא ומשו' גזרת י"ח דבר. ועוד דקתני לה דומיא דסיפא דכותיים עצמן והתם לאו נזיר' אלא דינא הוא לחוש לספיקן.\par  וזה שכתבנו לפי גרסת מקצת הספרים אבל מהרבה מהן מספרי הגאונים שלא נמצא שם במס' שבת אותה הגרס' כלל ואעפ"כ חשבון י"ח דבר עונה להן יפה. }
\newchap{דף \hebrewnumeral{32}}
\twocol{\textbf{שמא תמצא איילנות ונמצמו פוגעין בערוה.}  פירשתיה בתחל' מסכ' יבמות.}
\twocol{\textbf{הא נמי מיעוטא דשכיח הוא דתניא מעשה והטבילוה קודם לאמה.}  וא"ת שמא משום נגיעות אמה בה הטבילוה לסוכה בתרומה. י"ל שיודעין היה בגמרא שלא בא ר' יוסי אלא להעיד על טומאת עצמה והכי קתני מעשה היה שפרשה נדה והטבילוה קודם לאמה. ולא אטבילה בלחוד אסהוד אלא אפרשה אסהיד דאי לאו הכי פשיטא ותא חזי מאן גברא רבה מסהיד עליה. א"נ אין דרכן של בני אדם להפרישה מאמה אלא לכך הטבילוה שלא תטמא את הנשים שגוגעות בה ומגפפות אותה ויחזרו ויטמאו הן תרומה שבא"י אבל בנגיעת אמה בה אין להקפיד לטומאות הנוגעים בה שהרי היא ראשון ואין מטמאה אדם. ואותה שבפומדיתא נמי משטבלה לטומא' גופה אינה צריכה להפרישה מאמה כדמפר' ואזיל. }
\twocol{\textbf{ולא יחללו את קדשי בני ישראל לרבות את הסך ואת השותה.}  י"מ שהיא אסמכתא דרבנן דהא קי"ל גבי יום הכפורים דאכילה ושתיה דאוריית' ובכרת ואין סיכה בכלל שתיה.\par  ויש לפרש אע"פ שנתרבה סיכה כשתייה לענין תרומה מריבוי הכתוב לשאר כל התורה כולה אינה כשתיה ואי משום וכשמן בעצמותיו דשייך נמי בכל התורה ההיא ודאי אסמכת' בעלמא היא מדברי קבלה ומיהו ודאי מכיון דאמרינן גבי תרומה גופה מולא יחללו ואיבעית אימא מוכשמן בעצמותיו משמע דכולה דרבנן היא.\par  ומאחר שכתבתי סברות הללו מצאתי בפ"ב ממסכת מעשר שני שאמרו בירושלמי לענין מעש' יצהרך זו סיכה והתור' קראתו אכילה ואינו מחוור וא"ת מחוו' ולקו עליו חוץ לחומ' וכו'. ומייתי נמי התם והתני שוה סיכה לשתי' לאסור ולתשלומין לא לעונש יום הכפורים ומקשי והתני לא יחללו מ"מ להביא הסך והשותה. }
\twocol{\textbf{א"ר יוחנן לית כאן לאסר וכו'.}  משמע דסיכה כשתיה דרבנן ואינה מחוורת מן התורה. }
\twocol{תמיהה לן לר' ישמעאל בנו של רבי יוחנן בן ברוקא דדריש \textbf{לזכר כל שהוא זכר לנקבה כל שהוא נקבה}  וא"ו דגבי זבה למה לי. וכי תימא לא דריש וא"ו א"כ אין איש מטמא בדם ובאודם מנא ליה. }
\twocol{\textbf{למעוטי אשה מלובן.}  מצאתי בתוספות שמקשים למה לי מיעוטא והרי מצינו ה' דמים טמאי' באשה ותו לא. וי"ל דנהי דאינו דם הוה אתי בק"ו ומה איש שאינו מטמא באודם מטמא בלובן אשה שמטמאה באודם אינו דין שתטמא בלובן ודם טהור באשה מיתוקם בירוק ודיהה כך השיב ר"ש לר' יהודה חתנו ז"ל.\par והם הקשו בתוספות והא ק"ו פירכא הוא מה לנקבה שכן אינה מטמא בראיות כבימי' כדאמרינן בסמוך ואמרו גלוי מילתא בענמא היא דאחד איש ואשה מטמאין בלובן כיון דמתוקם דם טהור שפיר. ולא מחוור לי דאם לובן טמא משום נדה כ"ש ירוק ודיהה שכולן לא נטהרו אלא משום שאינן אלא לובן.\par  ונ"ל דמש"ה איצטריך יתורא דאי משום הא דה' דמי' לא הוה ממעטי' אלא מטומאת נדה ועדיין היינו מטמאי' באשה מדין שכבת זרע או זוב של איש היינו מטמאין לטהרות ולא לבעלה ומיניה ממעט ליה. ומ"מ יפה הרב ז"ל מלמדנו דאי לא ילפי מהדדי מנא תיתי לובן באשה ואודם ודם באיש דאצטריכו קראי למעוטינהו.\par  ואמרו בתוספות שעוד שאלו בכל מקום ואוי"ן לרבות וכאן למעט השיב לו כ"ש כיון דלא אצטריכו לרבויא מפרשין להו לקרא דייתר ואיש ואשה לומר אשה דוקא אמרתי ולא איש איש דוק' אמרתי ולא אשה. }
\twocol{\textbf{פשיטה דהא קא דרס להו.}  פי' לאו פשיטא מגופא דמילתא אלא פשיטא דכל דקא דרים להו רחמנא רבינהו למדרס דתניא בת"כ אשר ישב עליו הזב אין לי אלא יושב ומגע מניין לעשרה מושבות זה על גב זה ואפילו על גבי אבן מוסמה ת"ל והיושב על הכלי אשר ישב וכו'. ומשום דמילתה רגילה היא בתלמודא הוא קאמרינן פשיטא דלא ה"ל הכא למיתני אלא שמטמ' מדרס.\par  ובפי' עליונו של זב שמעתי דברים רבים והנכון מהם מה שאמרו משם ר"ש ז"ל שהוא דבר הנשא עליו כגון הוא בכף מאזנים ומשכבות ומושבות בכף שניה וכרעו הן טמאין מדרס כרע הזב זהו עליונו של זב ומטמאין אוכלין ומשקין. ואתינן למיבעי מנלן דתניא ובל הנוגע בכל אשר' יהיה תחתיו מאי תחתיו אלימא תחתיו דזב דהיינו משכב ומושב מאיש אשר יגע במשכבו נפקא. ואי קשיא לך אדרבא הוא מטמא בגדים דכתיב ביה יכבס בגדיו והכא ליכא כבוס אה"נ אלא גמרא לא איצטרך למיחת לה כולי האי. וקאמר סתם כל טומאה דמדרס מהתם היא כדכתיבנא ביה ולא מהכא ועוד דאי הוה נקיט טעמא מהך קושיא דכבוס בגדים דילמא הוה אמרינן דכי כתיב והנושא אותם יכבס בגדיו ארישא דקרא נמי קאי ולא בעי עיוליה נפשיה בספיקא דקושיי. }
\newchap{דף \hebrewnumeral{33}}
\twocol{\textbf{אלא וכל הנוגע בכל אשר יהיה זב תחתיו ומאי נינהו נישא יטמא נתקו הכתוב וכו'.}  זו היא גרסתו של רש"י ז"ל. ולשון יתר שבספרים מ"ט הנושא והנישא כתב נתקו וכו' כתב שהוא פי' משובש. אלא ה"ק מדכתיב וכל הנוגע בכל אשר יהיה תחתיו יטמא והנושא אותם יכב' בגדיו ולא ערבינהו ונכתוב וכל הנוגע בכל אשר יהיה תחתיו והנושא אותם יכבס בגדיו ואפסקינהו ביטמא מכלל דהאי יטמא לאו באדם ובגדים קא מיירי אלא באוכלין ומשקין מאי והנושא אותם לא לעליונו של זב דסמיך ליה אלא נדרש הוא בת"כ לנושא משכבו ומושבו של זב. אלו דברי הרב ז"ל.\par ואין פי' זה נכון דהא טמא עד הערב כחיב ואם אינו מטמא אלא אוכלין ומשקין מאי עד הערב.\par ואיכא למימר דהכי דריש דכתיב לעיל מינה וכל המרכב אשר ירכב עליו הזב יטמא וסמיך ליה וכל הנוגע בכל אשר יהיה תחתיו יטמא עד הערב ומפני שנתקו הכתוב מקרא נדרש לפניו דכתיב יטמא וכל הנוגע בכל אשר יהיה תחתיו נמי והוא עצמו כלומ' מה שהזב תחתיו יטמא עד הערב לומר שהוא מטמא כלים שיש בהן טומאת ערב לפי שיש להן טהרה במקוה.\par והרב אב"ד ז"ל מפרש דהאי יטמא עד הערב אמרכב דלעיל ולא מסתברא דהא לא כתיב במשכב ומושב עד הערב.\par ואיכא למידק, עליונו של זב מאי נינהו הסיטו הא מהכא נפקא מהתם נפקא וכלי חרס אשר יגע בו הזב ישבר אי זהו מגעו שהוא ככולו הוי אומר זה היסט ולקמן במכילתן בפ' יוצא דופץ אמרינן וכל אשר יגע בו הזב וידיו לא שטף במים זה היסטו של זב שלא מצינו לו חבר בכל התורה כולה. ואיכא למימר אי מהתם הוה אמינא עליונו של זב כתחתונו מטמא אדם ובגדים ואי מהכח ה"א כלי חרס שנטמא מאוירו לא קמ"ל וכלי חרס אשר יגע בו וכו' וכל אשר יגע מיבעי ליה למימרא דהיסט ונגיעה כידיו כדמפורש בפ' יוצא דופן.\par וא"ת אי לא כתב עליונו של זב מנא לך לעשותו כתחתונו להחמירו הא ל"ק דה"א נושא ונישא כי הדדי נינהו דהא בכולה שמעתין להחמיר עליו ולעשות כיוצא בתחתונו אנו טורחין ויש שמתרצין אי מהתם ה"א ה"מ היסטו כולן א) בא הכתוב הזה וכל הנוגע בכל אשר יהיה זב תחתיו לטמא אף לטהור וזב שהסיטו.\par וגרסת הספרים יש להעמידה אלא וכל הנוגע בכל אשר יהיה תחתיו ומאי נינהו נישא יטמא מ"ט כלומר מ"ט משמע לך יטמא טומאה קלה הנושא והנישא כתיבי בהדדי ולא ערבינהו רחמנא ואפסקינהו לומר לך נתקו כלשון רש"י ז"ל עצמו.\par ובתוספות ראיתי שפי' רבינו שלמה ז"ל בתשובה ה"ג אלא וכל הנוגע בכל אתר יהיה תחתיו יטמא והנושא נמי יטמא ומאי נינהו נשא מ"ט והנישא כתיב נתקו הכתוב וה"פ וכל הנוגע בכל אשר יהיה תחתיו בא הכתוב ולימד על המרכב שיטמא במגע דהאי קרא אחר מרכב כתיב וחלק מגעו ממשאו שמגע מטמא אדם ולא בגדי' ומשאו מטמ' אדם לטמא בגדים והכי מוקי לה בת"כ דמרכב חלק הכתוב מגעו ממשאו ובמשכב לא חלק בין מגעו למשאו ומהאי קרא נפקא לן מגע מרכב. והאי דדרשינן דהא כתיב והנישא חסר וא"ו למדרש אנישא של זב דקריי ארישא דקרא דכתיב יטמא עד הערב טומאה קלה ונתקו הכתוב מטומאה חמורה של אחריו, וה"ק והנוגע במרכב הזב יטמא טומאה קלה לטמא אוכלין ומשקין וכן הנישא על הזב דע"כ דרשא דנישא אטומאה דרישא קאי ולא אטומאה דסיפא דטומאה דסיפא אותם כתיב ביה וגבי נישא לא שייך אותם אלא אותם אמקרא קאי דקרינן נושא ודרשא דמסורת ארישא קאי ע"כ תשובתו של ר"ש ז"ל.\par והוקשה לו על גרסת הספרים שכתוב בהן בכל אשר יהי' תחתיו. ב) דהאי קרא במרכב מוקי לה בת"כ ועוד היכי מוקי לה באוכלין ומשקין הנוגעין בעליונו הא אין להם טהרה במקו' והכא כתיב וטמא עד הערב. כל זה הענין כתוב בתוספות. והעלו השמועה בשבוש ועמעום ועוד שאין זה נושא כתוב חסר בכל הספרים ובמסורת מלא ואין כאן יתור לדרוש בו מסורת ולשון ראשון של פי' רש"י ז"ל יותר נכוו הוא וכמו שכתבתיו למעלה.\par ולי נראה ענין אחר שפירוש עליונו של זב זהו שהיו עשר מצעות עליו ונשאת התחתונה על ראשו כולן טמאי' כמו שעושה בתחתונה את הנוגעת בזב מדרס מרבוי הכתוב כך עושה בעליונו את כולן עליונו של זב ואלו מדין היסט תחתונה טמאה ועליונוח שהן נשאות על עליונו של זה היסט דהיסט הוא וטהורין ומ"מ כי"ו דין היסט הן ולא מצינו להן חבר בשאר טומאות ופי' נתקו הכתוב מטומאה חמורה שאמרו כאן משום דה"ל למיכתב וכל המרכב אשר ירכב עניו הזב יטמא וכל אשר יהיה תחתיו דזב יטמא למיכתב בטומאת עצמן והדר ליכתוב בתרווייהו וכל הנוגע וגו' והנושא אותם יכבס בגדיו כדכתיב במשכב ובמושב עצמן כל המשכב יטמא וכל הכלי אשר ישב עליו הזב יטמא והדר בנגיעה דידהו ואיש אשר יגע במשכבו והיושב על הכלי. מדפלגינהו רחמנא ש"מ שאין עליונו של זב שוה למרכב הסמוך לו ולא למשכב ומושב דלעיל נתקו הכתוב מכולן שהן מטמאין אדם וזה אינו מטמא אלא אוכלין ומשקין ולהכי רהיט קרא ונסיב נוגע לפי שאינו נעשה אב הטומאה אלא שיש בו שם טומאה לנוגע בו וקרי ביה וכל אשר יהיה זב תחתיו יטמא עד הערב כלומר עליונו של זב עצמו טמא טומאת ערב ודרש בת"כ דחמור עליונו של זב מתחתונו דאוכלין ומשקין אינן נעשין תחתיו מדף ונעשין על גביו מדף כלומר טומאה קלה דעליונו של זב. וכן מפורש במשניות דתנן האוכלין והמשקין והמדף מלמטה טהורין והאוכלין והמשקין והמשכב והמושב והמדף מלמעלה מטמאין א' ופוסלין אחד. כלומר שעושין ראשון ושני באוכלין.\par  ואפשר לפי זה הפירוש שתתקיים גרסת הספרים וה"ק מ"ט הנושא והנישא כתיבי כלומר מ"ט משתמע קרא הכי נימא דהיינו משכב ומרכב וכל שתחתיו ונישא דהיינו עליונו תרווייהו כתיבי הכא ונתקו הכתוב לנישא מנושא ופי' מעלמא הוא ולא מגופה דברייתא כדפרישו נמי מאי תחתיו וכו' דהאי לישנא דגמרא ואקשי' אימא נתקו הכתוב מטומאה חמורה שבמרכב דלא לטמא אדם ובגדים אפילו במשא ומפרקי' יטמא טומאה קלה משמע פי' משום דכל דלא מתפרש ביה מגע אחר לא משמע אלא שהוא טמא בעלמא דכל דמטמא אדם כתיב בהו ורחץ בשרו או יכבס בגדיו או והנוגע בהם יטמא כדכתיב במשכב ומושב הילכך הכא דלא כתיב אלא יטמא לחודיה טומאה קלה דלית ליה טהרה במקוה משמע מדלא כתיב ורחץ כדכתיב בכולהו נוגעים והיכא דלא כתיב ביה כגון בנדה דמפרש במשכב דידיה כתביה רחמנא סתם וכן לענין משכב דבועל נדה יטמא הוא עצמו משמע לומר שאינו מטמא אחרי' כשאר משכב ומושב דזב דמפרש בהו ואיש אשר יגע במשכבו ולדברי הרב אב"ד ז"ל חזר ופי' בו טומאת ערב במרכב. }
\twocol{\textbf{מתקיף לה רמי בר חמא ותספרנו ואנן נמי ניספריה וכו'.}  פי' רמי בר חמא טעמא הוה בעי אבל ודאי ליכא דסליק אדעתיה דדינא הכי הני ספרה אנן כדמקשינן בפ' בתרא דמכילתן א"ל רב ששת לרב ירמיה רב ככותאי אמרה לשמעתיה דאמרי' יום שפוסקת בו סופרת למנין ז'.\par  ואי קשיא ההיא דגרסינן בפסחים פ' כיצד צולין (דף כא) ר' יוסי אומר שומרת יום כנגד יום ששחטו וזרקו עליה בשני שלה ואח"כ ראתה אינה אוכלת ופטורה מלעשות פסח שני ומפרשינן טעמיה דקסבר מכאן ולהבא מיטמיא דמקצת היום ככולו ובעינן עלה אלא לר' יוסי זבה גמורה היכי משכחת לה בשופעת ואיבעית אימא בגון שראתה שני בין השמשו' אלמא אמרינן מקצת היום ככולו.\par לאו מילתא היא דבסוף מנין אית ליה לר' יוסי מקצת תחלת היום ככולו בין זבה גדולה ובין בקטנה דשני שלה סוף מנין הוא דהא אנן נמי בזבה גדולה קי"ל כר"ש דאמר אחר מעשה תטהר אלא לדידן סותרת בכל היום ולר' יוסי לית ליה סתירה לאחר מקצת יום דהא שלימה היא טהרתה אבל בסוף יום ותחלת מנין דכ"ע לית להו מקצ' היום ככולו אלא לכותאי.\par וראיתי מי שמקשה כאן מאותה שאמרו בפ"ק דר"ה (דף י) אמר רבא ק"ו ומה נדה שאין תחלת היום עולה לה בסופה סוף היום עולה לה בתחלת שנה שיום א, עולה לה בתחלתה. וא"כ לר' יוסי נימא ק"ו ויהיה סוף היום עולה לה בתחלתה. וזה המקשה יכול להקשות כן בזבה גדולה לרבנן (ובין א) בקטנה דליכא בינייהו אלא סתירה ולפי דעתי שאין זו הקושיא דמקצת יום טמא ככולו טמא ומקצת יום טהור סוף היום כתחלתו בין בתחלתה בין בסופה הילכך לענין זיבה ביום נקי ליכא למיספריה אבל לענין נדה אפילו כולו נמי כימא סופרתו כנ"ל.\par  ומיהו מקצת היום שעולה בספירה דזבה דוקא ביום אבל לילה אינה עולה לספירה כלל כדאמרינן בפ' בתרא דמכילתן ושוין בטבילות לילה לזבה שאינה טבילה ותנן נמי במס' מגילה דאינה טובלת עד הנץ החמה.\par וההיא דאמרינן מפ' כיצד צולין דמוקי לדר' יוסי לרואה בין השמשות וכן נמי איתא במס' נזיר (דף כז) וגרסי' בה הכי בנוסחי לדר' יוסי מכדי קסבר מקצת היום ככולו זבה גמורה דמייתי קרבן היכי משכח' לה כגון דחזאי פלגיה דיומא אידך פלגא דלמפרע סליק ליה שימור פי' דלמפרע היינו פלגא דיומא בתחלתו שעבר עליה בטהרה ומתרצי איבעית אימא דקא שפעא ג' יומי בהדי הדדי ואיבעית אימא דחזאי תלתא יום סמוך לשקיעת התמה דלא הוה שהות סליק ליה למנינא. ההיא לרוחא דמילתא איתמר דלא בעי לאתויי עלה התם קרא דמגלה דאמרינן כיון דבעי ספירה ספירה ביממא היא ואוקמוה בסוף היום ותחלתו דליכא שהות דספירה בין ראיה לראיה כלל.\par וי"מ דלא בעיא לאוקמי זבה גדולה בלילואתא דוקא משום דקראי ביממא כתיבי דכתיב ימים רבים כל ימי זובה ולקמן בשלהי מכילתין ואימא ביממי תהוי זיבה בלילואתא תהי נדה ובפ"ק דהוריות נמי אמרינן גבי צדוקין דאמר דזבה לא הויא אלא ביממא דכתיב כל ימי זובה הילכך אע"פ דמפקינן מקראי אפילו לילותא לא מפקינן קרא מימים. }
\twocol{ והא דאסיקנא \textbf{שלא תהא טומאת זיבה מפסקת ביניהם.}  לאו דוקא אלא שלא תהא טומאת ז' מפסקת ביניהם דהא טומאת לידה נמי אמר רבא לקמן בפירקן דדינה למיפסק וכדבעי' למימר קמן, א"נ אוקמתין דלא כרבא אלא כאביי דאמר טומאת זיבה דוקא, ומיהו בטומאת ערב ליכא למימר דפליג רבא דהא לא פריק הא דאמרי ולטעמיך זב גופיה היכי סתר וכו', אלא ע"כ קבולי מקבל דטומאת ערב מיהא לא סתרא כדפרישית, ולענין בעיין דפולטת בעינן למיכתב קמן טפי בפרק יוצא דופן (מב, א). }
\twocol{\textbf{אלמא אספיקא לא שרפינן תרומה.}  פי' לאו אכל ספיקא קאמר דהא למסקנא נמי אספיקא ודאי שרפינן אלא ה"ק אלמא אהך ספיקא דעם הארץ ואפילו בכותי לא שרפינן. }
\twocol{\textbf{ורמינהו על ספק בגדי עם הארץ.}  פי' שחכמים גזרו עליהם שיהיו זבים לכל דבריהם ובגדיהם יהיו מדרס לפרושין, והא דאמרינן בפרק השוחט מדרסות קאמרת שאני מדרסות גזירה שמא תשב עליהם אשתו נדה אבגדי אוכלי תרומה מדרס לקודש קאמר והיינו נמי דמיטמי' צינורא דידהו מדבריהם משום משקה הזב והזבה.\par ואי קשיא לך האי דאמרינן בפרק הניזקין (דף סא ע"ב) וליחוש שמא תסטנו אשתו נדה ולא חיישי' להיסט שלו, ועוד אמרו שם גבי חלה מניחה בכפישה או באנחותא וכשיבא עם הארץ ליטול נוטל את שתיהן ואינו חושש משום דלא נגע בהו ולא מטמאין בפשוטי כלי עץ ולא חשש להסיטו וכן נמי בפרק אין דורשין משמע גבי חמרין ופועלין שהן טוענין טהרות דלא מטמאין משום הסיטן.\par ותירץ ר"ת ז"ל שלא גזרו על עמי הארץ היםט שא"כ אין לך אדם מעביר לחבירו חבית ממקום למקום.\par ועוד הביאו ראיה ממשנת מסכת טהרות שאין עמי הארץ מטמאין בהסיטו ולא עושין נמי משכב ומושב דתנן בפרק קמא דטהרות הגנבים שנכנסו לבית אין טמא אלא מקום רגלי הגנבים ומה הן מטמאין אוכלין ומשקין וכלי חרס הפתוחין אבל משכבות ומושבות וכלי חרס המוקפין צמיד פתיל טהורי' ואם יש עמהם נכרי או אשה הכל טמא.\par  ויש לי לדחות דהתם כיון דלא נגעו ודאי הקלו באלו שטומאתן רחוקה וספק.\par  ועוד הביא ראיה מתוך שמעתן גופה שאין ע"ה מטמאין משכב ומושב דתנן משכב התחתון כעליון ואם היו ע"ה עושין משכב ומושב מטמאין הן התחתון כתחתונו של זב שהרי עשאום כזבים לכל דבריהם.\par  ואע"פ שיש לי לפקפק אף בראיה זו נקבל אותם מפני שלא הזהירו חכמים על ע"ה שיהיו כזבים ממש אלמא דין חדש יש להם אבל מ"מ תמה הוא אם גזרו עליהם שיהיו כזבים למקצת היאך הטילו עליהם למחצה וטיהרו משכבות והיסט א"כ הקלת בשל תורה.\par וי"מ שאין ע"ה טמא טומאת עצמן כלל אלא חשש שמא נגעו בנשותיהן ובמדרסן שהן אבות הטומאה והוא נעשה ראשון.\par  והא דמטמאינן צינורא דע"ה בשמעתין וכן במסכת חגיגה לאו משום משקה הזב והזבה אלא כיון שהחזיקו אותם חכמים בטמאי' משום משא מדרס טמאו המשקין בשפתי' דכלים מטמאין משקים מדבריהם בפ"ק דשבת אבל לא שיהיו כזבים מדבריהם ובגדיהם שהן מדרס לפרושין נמי משום תשש מדרס אשתו נדה הוא והא דא"ל מ"ט לא תשני ליה בכותי שטבל ועלה ה"ג לה שטבל ועלה (ואכל) [ונגע] בתרומה וכן כתבו בתוספות בנוסחאות ולא כגרסת רש"י שהוא גורס ודרס על בגדי חבר ונגעו בתרומה שאפילו לא טבל אין לו מדרס ואפילו נגע בהן ממש ואצ"ל בעשר מצעות שהרי לא עשיתו אלא ראשון משום טומאת ע"ה דהיא משום נושא מדרס ואינו מטמא כלים לאחר שפירש מן המדרס כלל.\par  ואי קשיא ולימא ליה מאי ואין שורפין עליה את התרומה על התחתונות דעליונו קתני דאי משום טומאת ע"ה לא מטמא משכב ואי משום נדה תרי ספיקי נינהו, איכא למימר מפני שטומאתן ספק אפילו אגופייהו משמע ליה.\par תו קשיא לי ולימא ליה ברגל דטומאת ע"ה ברגל כטהרה שוויה רבנן מאי איכא בכותי משום בועל נדה תרי ספיקי נינהו, ול"ק דבשלמא ע"ה שוויה בטהרה שלא להרחיקן אבל בכותי לאו חברים קרינן ביה ואין זה דומה לצדוקי שהם בכלל ישראל הם. }
\twocol{והא דמפרקינן ב\textbf{כותי ערום}  ולא אמרינן בשנגע ביד וברגל בלא בגד מפני ששנינו הנוגע במשכב ובמושב מטמא שנים ופוסל אחד פירש מטמא אחד ופוסל אחד בפרק בתרא דזבים והילכך מטמא את התרומה אלא בכותי ערום קודם שיגע בבגדיו עסקינן,\par ובמסכת חגיגה מצאתי בירושלמי סוגיא גדולה בענין זה ובתוס' נמי הזכירוה ומשמע מינה שלא גזרו על עם הארץ שיהיו כזבין וכך הסוגיא שם על מתניתין בגדי עם הארץ מדרס לפרושין וכו'. }
\twocol{\textbf{מתני רבי יוסי בשם רבי יוחנן במגעות שנינו,}  פירוש אלו השנויין כאן אינן עושין מדרס בלא נגיעה אלא שאם נגעו בבגד עשאוהו כמדרס מדבריהם, רבי זעירא בעי קומי רבי יוסי מהיכן נטמא הבגד הזה מדרס א"ל תפתר שהיתה אשתו של עם הארץ יושבת עליה ערומה, פירוש ר' זעירא מקשי על רבי יוסי לדבריך מהיכן נטמא מדרס לא היה לנו לטמאן אלא מגע הזב פריק שאם נגעה בו אשתו בישיבה עשאוהו כמדרס אבל ישבה עליו בבגדיה לא גזרו עליו, שמואל בר בא בעי קומי ר' זעירא כמה דתימר תמן אין היסט בחולין ויש היסט בחולין על ידי מגע ודכותה אין משא בחולין ויש משא בחולין על ידי מגע וכו' גופו של פרוש מהו שיעשה כזב אצל תרומה מתיב ר' תנן והתנן המניח עם הארץ בתוך ביתו בזמן שהוא רואה את הנכנסים ואת היוצאים האוכלין והמשקין וכלי חרס הפתוחין טמאין אבל המשכבות ומושבות וכלי חרס מוקפין צמיד פתיל טהורים אין תימר עשו גופו כזב אצל תרומה אפילו מוקפין צמיד פתיל יהיו טמאין אמר רב ר' יהודה בר פזי תפתר בעם הארץ אצל הפרוש לא עשאוהו כזב אלא שגזרו על בגדיו מדרס במגע אשתו כדאמרן ואקשי' אמר ר' מונא כן אמר ר' יוסי רבי כל מה דאנן קיימין הכא בתרומה אנן קיימן תדע לך שהוא כן דתנינן אפילו מובל ואפילו כפות הכל טמא כלום אמרו יהו הן טמאין אלא משום היסט לא כן אמר ר' יוחנן לאו חציצות ולא הסיטו ולא רשות היחיד ולא רשות עם הארץ אצל תרומה, ע"כ גמר'.\par  וה"פ דקא מקשי ליה רבי מונא לר"י בן פזי דאוקמא למתני' בחולין דודאי בתרומה קיימי' מדקתני סיפא הכל טמא ואי בחולין מדרסות והיסטות טהורין הן דאמר רבי יוחנן שלא אמרו שיהא דבר חוצץ במדרסות ולא טהרו היסטות ולא חלקו בספק רשות היחיד ולא רשות עם הארץ אצל תרומה הא אצל חולין הכל טהורין.\par וברייתא היא אצל זו ששנויה בתוספתא דחגיגה דקתני ספק רשות עם הארץ מדרסו וחצירו והיסטו טהורין לחולין וטמאין לתרומה, אלמא מתניתין דקתני הכל טמא בתרומה היא ושמע מינה שלא עשו גופו של פרוש ולא של עם הארץ כזב לטמא משכבות ומושבות והיסט אלא שחששו בזמן שאינו רואה את הנכנסים לאשה או לכותי לתרומה ולחולין הכל טהור ואפילו ספק רשותו עד שיתברר לך שנכנסה אשתו לשם אי נמי בגדים שלו שאי אפשר שלא נגעה בהם אשתו במדרס, ע"כ הארכתי לכתוב מן התוספת והן מגיהין ב) ולא רשות עם הארץ לחולין אלא אצל תרומה ודבריהם הללו כולן כתבתים מפני שדברים ברורים הם וצריכין הן לכמה סוגיות שבגמרא. }
\twocol{ה"ג רש"י ז"ל: \textbf{אמר לך בית שמאי האי לזכר מיבעי ליה כל שהוא זכר בין גדול בין קטן}  ולא גריס נקבה אלא לנקבה מיבעי ליה למעיינות ואתי זכר בקל וחומר ולא פרכי' במצורע מילי דזב.\par  ובתוספות מעמידין הספרים ומפרשים ואי בעית אימא בית שמאי לית הך דרשא דר' יצחק כלל אלא לזכר כל שהוא זכר לנקבה כל שהוא נקבה ודקא קשיא לך מעיינות מנא להו לבית שמאי ולטעמיך לר' ישמעאל בנו של ר' יוחנן בן ברוקא דריש לעיל תרווייהו לקטן וקטנה מעיינות מנא ליה אלא נפקא להו משום דרשא בעלמא דלא מתפרש במכילתין ואדרבה משמע דבית שמאי כר' ישמעאל בנו של ר' יוחנן בן ברוקה אמרי לגמרי.\par  ולי נראה דבין לבית שמאי בין לר' ישמעאל לזכר כל שהוא זכר ולנקבה למעיינות אתא ונקבה קטנה לרבי ישמעאל נפקא ליה מוא"ו יתירה דגבי זבה והא דנקט לעיל דנקבה כל שהיא נקבה סירכא נקט ולא דנפיק מינה אנא משום דהכי הוא אמר קרא ולא מהאי קרא מתרבן, וכבר כתבתי מכיוצא בזו הרבה בפרק קמא דקדושין ועכשיו תירצנו הקושיא שהקשינו למעלה דלכולי עלמא הנך ואו"י מדרשי בזב ובזבה דתרי נינהו. }
\twocol{גרסת הספרים כך היא וכן בפירושי ר"ח ז"ל: \textbf{תא שמע זובו טמא לימד על הזוב שהוא טמא במאי אלימא בזב גרידא לאחרים גורם טומאה לעצמו לא כל שכן אלא פשיטא בזב ומצורע ומדאיצטריך לרבויי בראיה ראשונה שמע מינה מקום זיבה לאו מעין הוא.}  ובודאי יפה פירש רש"י ז"ל שראיה ראשונה של אדם אחר אינו מטמא במשא אלא במגע (בקרי) [כקרי] וראיה שנייה מטמא אפילו במשא לקמן בפרק דם הנדה, והא דקאמר לאחרים גורם טומאה הכא קאמר לאחרים גורם שיהיו מטמאין במשא לעצמו לא כל שכן שיטמא במשא ומדין משא למשא פריך דאי גרס טומאה בעלמא קאמר אף בזב מצורע גורס טומאה דמשכב ומושב לטמא אדם ולטמא בגדים ולטמא נמי בהיסט שאין מצורע עושה כן אלא מדין משא גופיה פריך כדפרישית.\par  ומיהו צריכין אנו לישב גרסת הספרים, ור"ש אומר פירוש שהיא בספרים והכי קאמר ומדאיצטריך לרבויי בשנייה שמע מינה דבראשונה לא מטמא במשא דלאו מעין הוא ואין וה הלשון גמרא.\par  אבל יש לפרש שכך היא הצעה זו דאמר רבא תא שמע זובו טמא לימד על הזוב שהוא טמא במאי אלימא בזב גרידא ובראיה שניה דאלו בראשונה ולמגע ודאי לא צריכא קרא דלא גרע משכבת זרע ולא עדיף מיניה אלא פשיטא בשנייה ואכתי למה לי קרא לאחרים גורס טומאה ואפילו למשא עצמו לא כ"ש אלא פשיטא בזב ומצורע ואי בשניה מי גרע מזב גרידא אלא בראשונה ולטמויי במשא וש"מ תרתי ש"מ ראיה ראשונה של מצורע מטמא במשא וש"מ לאו משום דמעיין הוא כדרב יוסף דאי הכי לא איצטרך רחמנא לרבויי הכא דממעינות נפקא אלא דרחמנא רבייה לראיה ראשונה של מצורע כראיה שניה של זב גרידא.\par ואי קשיא נימא קרא לראיה שניה והא קמ"ל דוקא בשניה אבל בראשונה אינה מטמא דלאו מעיין הוא, זו אינה תורה דמי איכא ספיקא קמי שמיא במקום זיבה אי מעין הוא או לא ואיצטרך ליתורי קרא למיגמר מיניה דלאו מעין הוא, ועוד דכל היכא דקרא מרבה כגון זה דכתיב זובו טמא דרשינן ליה לרבויי כגון לרבו' ראי' ראשונה למשא ולא מוקמינן ליה ליתורא למימר בשניה כתיב ולמעוטי ראשונה איצטרך כנ"ל.\par ומה שהקשה רש"י ז"ל מי איכא לאוקומי להאי קרא בראיה ראשונה והא מהכא נפקא לן בכל דוכתא מנה הכתוב שתים וקרא טמא, אינה קושיא דהאמרינן בפרק יוצא דופן דלמאן דאית ליה מנה הכתוב שתים וקרא טמא לית ליה זובו טמא לימד על הזוב שיהא טמא ותנאי היא.\par וכן זה שאמר הרב ז"ל דגבי מצורע איצטרך לרבוייה לטיפ' עצמה דלא אתי בק"ו משום דלא גרמה ליה טומא' שהרי מחמת נגעו הוא מטמא אין זה מחוור דכיון דאי לאו מצורע הוא נמי הות מטמי' איתא לק"ו מ"מ וכ"ש דאיכא לפרושי גרס טומאה בהיסט ומדרסות כדאמרן לעיל ומהסט למשא גמרינן ודאי דחד אורח הוא למשאות, ובמסקנא פשט אביי דמטמא במשא דהא אקשייה רחמנא למצורע אזב גמור, ולא פשט במעיין כלום משום דלא מרבוייא דקרא יתירא אתי דנימא למאי איצטרך אלא דמ"מ מטמא במשא הוא. }
\newchap{דף \hebrewnumeral{35}}
\twocol{הא ד\textbf{אמר רבא לאחרים גורם טומאה.}  ק"ל א"כ שכבת זרע דכתב רחמנא מלטמא לימא לאחרים גורם טומאה לעצמו לא כ"ש ולמה לי הא דתניא מנין לנוגע בשכבת זרע ת"ל או איש וכו' כדאיתא בפ' יוצא דופן, וכן (נמי קושיא) [דם עושה] משכב ומושב לאחרים והוא עצמו אינו עושה משכב ומושב כדאמרינן בפרק דם הנדה וכל המשכב אשר תשכב עליו נדה ולא דמה ובהא איכא למימר התם מיעטיה רחמנא דלאו בר משכב ומושב הוא ולית ביה אלא נגיעה בעלמא.\par תו קשיא והאיכא נמי היסט שהזוב גורם טומאת היסט והוא עצמו אינו מטמא בהיסט.\par ואיכא למימר נמי התם מיעטיה רחמנא מדכתיב והנושא אותם מיעוטא הוא בפרק דם הנדה (נה, א) א"נ בכל היינו פרכיה דעדיף מינייהו אמר ליה שעיר המשתלח יוכיח שאין לו טומאה כלל וגורם טומאה חמורה והך פירכא ופירוקא דרב יהודה מדסקרתא בברייתא תניא בהו בפרק דם הנדה ומדלא מייתו לה אינהו ש"מ לא שמיעא לה ובגמרא לא שמיע' להו ובגמרא לא אמרינן תניא נמי הכי דבשקלא וטריא בעלמא לא מיתמר הכי. }
\twocol{\textbf{ולוי אמר שני מעיינות הן.}  מצאתי בשם חכמי הצרפתים שהם מקשים והתניא בפרק יש בכור (דף מו ע"ב) גיורת שיצאת פדחת ולדה בהיותה נכרית ואח"כ נתגיירה אין נותנין לה ימי טומאה וימי טהרה ואמאי ללוי דאמר נסתם הטמא נפתח הטהור אמאי אין לה ימי טהרה הא ממעיין טהור אתו, ומתרצים אין לה ימי טהרה לאו דוקא שאין לה כלום אלא לומר שאם ראתה ביום ז' לזכר וביום שבועיים לנקבה טמאה נדה ומונה בתוך ימי טוהר ימי נדות.\par ויש שמעמידין אותה אליבא דלוי כב"ש דאמרי מעין אחד הוא, עוד פירשו ללוי אף על פי ששני מעיינות הן לא טהרה התורה מעיין זה אלא בנולדות וזו כיון שאין לה דין לידה אף אותו מעיין טמא הוא ורואין את דמה אם מחמשה דמים הוא וזה פשוט ונכון. }
\twocol{\textbf{בשלמא לרב דאמר מעין אח' הוא מ"ה מטמא לח ויבש.}  פי' לבית הלל פשיטא ולבית שמאי נמי כיון שראיה זו מטמאתה מלספור נקיים נמצא שהיא גורמת טומאה וכדם הנדה הוא שמטמא לח ויבש ולא דמי לרואה בתוך ימי טוהר בלא זוב שאין ראייתה כלום אלא ללוי אמאי מטמא לח ויבש בין לבית שמאי בין לבית הלל, ופריק בשופעת.\par  אי בשופעת למאי איצטרך וק"ל ולרב גופיה אמאי איצטרך ודאי לבית שמאי ללוי נמי לבית שמאי ואיכא למימר בשלמא לרב טעמיה לבית שמאי קמשמע לן דלא תימא טעמייהו משום דשני מעיינות הן ובהא פליגי קמשמע לן ומודים ואי שני מעיינות הן ביולדת בזוב נמי מטהרו בית שמאי אלא ללוי אמאי איצטרך האי טעמא בתרווייהו מכל מקום שמע מינה דהא לית ליה לאוקמינהו אלא בהך פלוגתא ופריק אפילו הכי איכא למיטעי בה לבית שמאי דסד"א אף על פי שופעות לא תטמא קמשמע לן.\par  כיון דמפורש בשמעתין דלרב ימי טוהר שרואה בהן אין עולין לה לספירת זיבה, וקיימא לן בנות ישראל החמירו על עצמן שאפילו רואות טיפת דם כחרדל יושבות עליה שבעה נקיים וקיימא לן אי אפשר לפתיחת הקבר בלא דם אם כן היולדות צריכות שבעה נקיים בתוך ימי טוהר שלהן אלא שימי לידה נמי אם אינה רואה בהן עולין לספירת זיבתה כדלקמן וכן כחב הרמב"ם ז"ל שהיולדת בזמן הזה הרי היא כיולדת בזוב וצריכה שבעה נקיים. }
\newchap{דף \hebrewnumeral{36}}
\twocol{הא דתניא \textbf{ושוין ברואה אחר דם טוהר שדיה שעתה}  הקשו בתוספות אם כן מניקה שאמרו צריכה שתפסוק שלשה עונות והלא אחר דם טוהר היא רואה לעולם וכל שכן בזו שאמרו בברייתא שתים בימי עוברה ואחת בימי מניקתה וכו' למה לי הפסקה דעונות והא אחר ימי טוהר היא רואה ומוקמי לה כרבי מאיר וכשאינה מניקה וכגון שמת בנה ומשום רואה אחר ימי טוהר דיה שעתא.\par  ולי נראה דה"ק: קיימא לן דלא אמרו דיין שעתן אלא בראיה ראשונה וקאמר השתא שאם ראתה אחר דם טוהר אינה כראיה שנייה אלא כראיה ראשונה ודיה שעתה והוא שהפסיקה כדינה בעונות, וכן נראה פירש"י ז"ל ונכון הוא לומר דשויין אפלוגתא דר' מאיר ור' יוסי בדין כל ימי עבורן וימי מניקתן קאי ומתרץ לה רב בדליכא שהות ובין שראתה בימי טוהר בין שלא ראתה ליכא לטמוייה כלל, ואף על גב דלא הפסיקה נמי כדמפרש לה ואזיל. }
\twocol{\textbf{אמר רב נדה ליומא ושמואל אמר חיישינן שמא תשפה.}  איכא דקשיא ליה לרב ור' יצחק אמאי לא חיישינן שמא תשפה והא חיישינן שמא יבקע הנאד, ותנן הרי זה גיטך שעה אחת קודם מיתתי אסורה לאכול בתרומה מיד אלמא חיישינן שמא ימות.\par ולאו מילתא היא, דהתם סופו למות שהכל למיתה הן עומדין אבל כאן אדרבה סופה להקשות כשמתקרבת ללידה והא דהדר ביה רב לקמן נ"ל דלגבי שמואל הדר בי' וחייש שמא תשפה ואף על גב דמתניתין כר' יצחק מתוקמא דתניא הכי מיהו דשמואל לא מיעקרא בהכי דאכתי איכא למימר מסבר' דחיישינן שמא תשפה דאי ס"ד רב לקולא הדר ביה אמאי גדייה רב אסי ומאי האי דאמרינן עבד עובדא כותיה הא חומרא בעלמא הוא דעבד. }
\twocol{\textbf{כל שחל קישויה להיות בג' שלה וכו'.}  פירש רש"י ז"ל כל שקשתה אפילו שעה א' בליל כניסת ג' אפילו כל היום כולו בשופי ושעה א' מליל ד' להשלמה מעת לעת וילדה אין זו יולדת בזוב דבעינן שופי כל יום ג' המביא לידי זיבה.\par  פי' לפי' לאו משום (דכפי) [דבעי] חנניא לילה ויום כלילי שבת ויומו דאם כן היינו דר' יהושע אלא משום דבעי כל יום ג' בשופי ואם קשתה בג' אפילו שפת בד' כלילי שבת ויומו אינה זבה שאין קושי שבג' קובע אותה זבה ובכה"ג לא הוה זבה עד דחזיא ג' אח"כ בשופי דקושי שבשלישי אינו קובע בזיבהולא מצטרף (עד) [עם] יום ד' לקבעה בזיבה והיינו דאמרי לקמן לעולם כדקתני והא קמשמע לן אף ע"ג דאתחיל קושי בג' אם שפתה מעת לעת טמאה לאפוקי מדחנניא בן אחי ר' יהושע ואלו לאפוקי מדר' יהושע בהדיא קתני לה אם שפתה מעת לעת ר' יהושע אומר כלילי שבת ויומו. }
\newchap{דף \hebrewnumeral{37}}
\twocol{\textbf{או דילמא דבר הגורם לטומאה סותר והא לאו גורם הוא.}  כתב רש"י ז"ל הרי קרי דבר שאינו גורם וסותר ההיא לאו סתירה היא דחד יומא הוא דסתר וכי בעי רבא לסתור את הכל ופי' דבר הגורם לידי זיבה.\par ולא מחוור לי, חדא דקאמרינן דבר הגורם לטומאת שבעה סותר ולא אמרינן דבר הגורם לידי זיבה, ועוד דתניא הכא קריו יום א' לפיכך סותר יום אחד.\par  אלא זהו טעמו של דבר קרי שמטמא יומו סותר יומו ולא מפני שאין אותו היום עולה לשבעה נקיים דהא מעין א' הוא ודינו שיעלה אלא האי דאינו עולה לספירה דיומיה הוא דכיון דרואה הוא אמר רחמנא יסתור ולפיכך הוצרכו לומר שסתירו כטומאתו אבל קושי אם דינו לסתור משום טומאת נדה סותר הוא שבעה אבל משום טומאתו לא היה דינו לסתור ואפילו יומו שהרי עכשיו אינו מטמא בימי זיבה כלל ואף על פי שמטמא בימי נדה אין דינו לסתור ואפילו יומו דדבר הגורם עכשיו סותר ולא דבר שטהור עכשיו וגורס בזמן אחר ומיהו ודאי כיון דמעיין אחד אתי אינו עולה דלא גרע מימי טוהר דבתר לידה דאמרן לעיל לרב דאמר מעין אחד הוא אינן עולין והא לא סתירה מקריא אלא דבעי' נקיים מכל דם של אותו מעין, והא דאמרת הגורם לטומאת שבעה ודאי משום דבעייה לסתירת שבעה אמר שבעה. }
\twocol{והא דתניא \textbf{ר' מרינוס אומר אין לידה סותרת בזיבה,}  אפילו ברואה קאמר לפי' שאין דם ראיה זו גורם כלום ואי קסבר נמי דאי אפשר לפתיחת קבר בלא דם ההוא בקושי חזיתיה ואין קושי סותר כר' מרינוס וכיון שאין דם הקושי שלפני הלידה ולא שלאחר הלידה ראוי להיות סותר אף הלידה אינה סותרת שאין סתירה אלא בראיה דהויא לה לטומאה זו כנגיעה וכנוגעת בטומאתה שאין לה סתירה כלל, ומיהו אם ראתה בלידה אין יומא עולה לדברי הכל אלא שאין זה נקרא סתירה כדפיר' רש"י זכרונו לברכה, והכי נמי מפרשינן בפלוגתא דאמוראי והכי נמי מתוקמא למר אינה סותרת לעולם ואינה עולה לעולם ולמר אינה סותרת לעונם ועולה כשהימים הם ראויים לעלות כגון שאינה רואה דלכולי עלמא לעלות נקיים בעינן ואפילו בתוך ימי טוהר כדאמרן לעיל, והיינו דאמרן מ"ל נקיים מלידה כלומר נקיים אף מלידה ורבא לא נקיים מדם לומר אעפ"י שאינו גורם נקיים בעינן ואף בימי טוהר וכדרב. }
\twocol{והא דאמר רבא \textbf{א"א בשלמא עולה היינו דלא מפסקת טומאה.}  ה"ק א"א בשלמא דינה לעלות בשאינה רואה אפילו ברואה נמי אינה סותרת שאין כאן טומאת ז' מפסקת אלא יומו הוא דלא חזי לעלות דומיא דרואה קרי שסותר יומו ואינו מפסיק כדאמרן בריש פירקן וכ"ש הכא דהאי דם לאו כגורם הוא טומאה כלל ולא מוסיף ביה טומאה דכלום אלא א"א אינו עולה האיכא טומאת ז' ושבועיים דמפסקא, כן נ"ל לפי' שמוע' זו ובתוספת מאריכין בה בענינים הרבה שאינן עולין.\par  ואיכא למידק אשמעתין דהא בשילהי בא סימן (דף כ"ד) איבעי להו ימי לידתה שאינה רואה בהן מהו שיעלה לה לספירת זיבתה, ואמר רב כהנא ת"ש ומסקנא ש"מ עולין ש"מ, וי"מ דהכא אליבא דר' מרינוס איירינן דאביי דאיק מדקאמר אינה סותרת מכלל דאינה עולה דה"ל למיתנא רבותא דעולה ורבא אמר אפילו לר' מרינוס עולה והא דאמר רבא מנא אמינא לה מואח' תטהר לומר דכיון דקרא קא דרשינן אפילו לר' מרינוס אית ליה, וכן הא דתניא מזובה ולא מנגעה מזובה ולא מלידתה קרא קא דייק והיינו דקאמר ליה אביי תני חדא כלומר לר' מרינוס דוקא חדא אבל ברייתא תרתי קתני והא דאמר אביי מנא אמינא לה לומר דכיון דמשכחת תנא דאמר אינה עולה ר' מרינוס היא דהא לרבנן עולה, וזה הפי' שמעתי ולא נתקבל לי.\par ועכשיו מצאתי בתוספות בשמו של ר"ש ז"ל שכתבו בתשובותיו ואמר הרב ז"ל תדע דהא בעי לה לקמן בפרק בא סימן איבעי להו וכו', ואין דרך התלמוד לשאול בעיא אחת שתי פעמים ולא מצינו כן בשום מקום תלאוהו באילן גדול, ועדיין אינו מחוור לפי שאם היה אביי מודה לרבנן דעולין לא הוה משוי ליה לר' מרינוס טועה וחולק דליכא למידק מלישנא דידיה הכי כלל כדפרישית, ועוד הא דפרישו בדרבא דאמר מנא אמינא לה דמקרא דייק ולא מצי רבי מרינוס למפלג עלה הא לאו מילתא היא דאי איכא למידק מקרא תיקשי לר' אלעזר דאמר דאינה עולה אלא היינו טעמא דאביי דאיהו סבר לתרוצא לההיא ברייתא דבשלהי בא סימן כדמתרץ לה רב פפא א' שאני התם וכו'.\par  ומסקנא דעולין ודאי כרבא אתיא דקי"ל כוותיה ולא כדברי הרב ר' יעקב ז"ל שפי' שהלמ"ד לידה כהכא אלא כפי קבלת הגאונים שהוא לחי במסכת עירובין (דף ט"ו) דאיתותב מיניה התם בגמרא ת"ש מעובדא דרב וההיא דתניא בפרק המפלת אינן עולין לרבא אתיא כר' אליעזר דאמר מסתר נמי סתרא והא דלא מייתינן לכולהו בשמעתין כמה איכא בתלמודא דכוותייהו שדברי תורה עניים במקומן ועשירים במקום אחר ומה שאמרו שלא מצאו בתלמוד בעיא א' בשני מקומות כאן מצינו.\par  ועי"ל לו דהתם אמוראי בחראי אתו למיפשט אי כאביי או כרבא והלכה או אין הלכה מיבעיא להו וכן מצינו בפרק המגרש (דף פה ע"ב) דאיבעי להו מי בעינן ודן או לא עביד בעיא סתם בפלוגתא ברבי יהודה ורבנן [ועוד בעיא בפלוגתא דמתני'] דמתני' בפרק המקבל (דף קי"ד) מהו שיסדרו בבעל חוב וכן בפרק חזקת הבתים (דף מ ע"ב(איבעיא להו סתמא מאי קא מיבעי ליה מתרי לישני דרב יוסף דלעיל הי מינייהו הלכה וזו כן ופשטו מברייתא דעולין ואידתי ליה דאביי דסבר לכ"ע בין לרבנן בין לר' אליעזר אין עולין. }
\twocol{ והא דתניא \textbf{הכא מה ימי נדתה אין ראויין לזיבה ואין ספירת ז' עולה בהן.}  לקמן בשילהי בא סימן אמרינן זאת אומרת ימי נדתה שאינה רואה בהן עולין לה לימי זיבתה אלא שהן חלוקין בפירושיהן דהכא קאמרינן אין ספירת ז' של זבה גדולה עולה בימי נדה לפי שאינה נעשית תחלת נדה משראת נ' בזיבה עד שתספור שבעה נקיים והתם בזבה קטנה קאמרינן שאע"פ שראתה שנים בימי זיבה ונעשת נדה מונה יום אחד טהור מאותן ז' של נדה לזיבתה ודיה, ולשון אחר פירש שם רש"י ז"ל והכל שוין בדבר זה שמשעה שנעשת זבה גדולה כל ראיות שתראה אינה עולה בהן אלא סותרת ואינן ראויין למנות מהן ימי נדה וזיבה. }
\newchap{דף \hebrewnumeral{38}}
\twocol{והא דאמרינן \textbf{הא קמשמע לן דאפשר לפתיחת קבר בלא דם.}  אלמא אי הוה דם בפתיחת הקבר הויא זבה ליכא לאוקמא אלא בנפלים דלית להו קושי דאי בולד מעליא כי הוה דם בפתיחת הקבר נמי לא הויא זיבה דהא בקושי חזיתיה שאין לך קושי גדול מפתיחת הקבר אלא בנפלים הוא דמתוקמא ליה דמאה יום בלא קושי קתני, ואין זה נכון לומר דאין פתיחת הקבר בלידה קושי דלא מסתבר הכי ועוד דהא מכל מקום דמה מחמת עצמה ולא מחמת ולד קרינן ביה, וה"ה דמצו בגמרא למימר דא"א לפתיחת הקבר בלא דם ויש קושי לנפלים קמ"ל אלא רואה בלא קושי קתני דאלו בקושי יש רואה כל ימיה. }
\twocol{הא דתנן \textbf{אמר להן דיו לבא מן הדין וכו'.}  לא דסבר רבי אליעזר בימי נדה נדה בימי זיבה טהורה דהא רבי אליעזר מטמא תנן בכל עת משמע, אלא לומר כיון שיש לנו לדרוש דיו על כרחנו ונטמא ימי נדתה והכתוב אומר ישיבה אח' לכולן נמצא שמטמאה בין בימי נדה בין בימי זיבה, ואם תאמר אם כן מיפרך קל וחומר והיכא דמיפרך קל וחומר לא אמרינן דיו כדאיתא בבבא קמא, איכא למימר הכא משום דיו לחודיה לא מפרך דהא ב) טהורה בימי זיבה אלא ודאי בא דיו וטמא (יהיה) ימי נדתה ובא הכתוב וטמא אף ימי זיבה וכה"ג אמרינן דיו והיינו דמקשינן לקמן רבי אליעזר ואימא בימי נדה נדה בימי זיבה טהורה ופריק אמר קרא תשב וכו'.\par ויש מפרשים דלר' אליעזר אית ליה הכי ודאי דבימי נדה נדה וביומי זיבה טהורין כדין דיו, וה"ג לקמן דמיה מחמת עצמה ולא מחמת ולד ואימא בימי נדה נדה בימי זיבה טהורה בניחותא כלומר זכותא דקל וחומר לא ליפרך ומקיים קרא וקל וחומר וגרסי בתמיה ורבנן אמר קרא תשב ישיבה אחת לכולן, וזו היא גרסתו של ר"ח ז"ל וקבלת הנוסחא. }
\twocol{ והא ד\textbf{אמר רבא בהא זכנהו ר' אליעזר.}  טעמייהו קא מפרש דודאי ר' אליעזר אפילו פרכיה לק"ו אינו בדין עד שיטמא ימי טוהר שהתורה טהרתם סתם, ועוד שאין לך טעם להחמיר עליו יותר מן השופי ורבנן נמי לא צריכי לק"ו אלא למיפרך טעמיה דר"א. }
\newchap{דף \hebrewnumeral{39}}
\twocol{\textbf{לומר שאין אשה קובע לה וסת בתוך ימי זיבתה.}  פירש זיבתה ממש וכל שכן בימי נדתה שאם ראתה נדה בריש ירחא אינה קובעת וסת עד שמונה עשר יום שבעה ימי נדה ואחד עשר של ימי זיבה, והא דקתני אחד עשר משום דבאחד עשר לא קבעה לעולם אלא ביומי הראוין לנדה קבעה אבל בימי נדתה עצמן לא קבעה כלל והכי נמי איתמר בפרק קמא דמכילתין דתנן חוץ מן הנדה וכו', כדאיתא התם.\par  ואי קשיא דשמואל אדריש לקיש ור' יוחנן דאמרי אשה קובעת לה וסת בימי זיבתה התם אי לרב פפא כגון דחזיא בריש ירחא ובתמניא בירחא וחזיא בריש ירחא ובתמניא בירחא והשתא חזאי בתמניא בירחא ובריש ירחא לא חזיא דכיון דעקרא דריש ירחא וקבעה דתמניא אמרינן הך דתמניא עיקר ואינך תוספו' דמים הוו ואי לרב הונא בריה דר' יהושע בהכי לא קבעה אלא כגון דחזאי תרתי קמייתא ממעין סתום ושלשה בימי זיבתה דאמרינן הך הקדמה תוספו' דמים הוות וכדפרשינן בגמרא גופא למימרא דריש לקיש ור' יוחנן דימי נדה הכי נמי מתפרש רישא דימי זיבה אבל חזיינוהו בימי זיבה ממש דכ"ע אין אשה קובעת בהן וסת כדשמואל נמצא שאין האשה קובעת לה וסת אלא במפלגת אחד עשר יום ורואה בי"ט וביותר מכאן, אבל רבה ראיותיה בפחות מכאן אינה קובעת.\par אלא שהרב ר' אברהם בר דוד ז"ל כתב אם רגילה לראות מט"ו לט"ו לראיתה דכל חדא וחדא קיימא לחברתה בתוך ימי זיבה רואין את האמצעיות כאלו אינן והשאר קובעת לה וסת מכ"ט לכ"ט דקיימי אהדדי בימי נדה ואי חזיא בהון ארבע ראיות מכ"ט לכ"ט קבעה לה וסת להפלגת כ"ט ובעיא עקירה תלתא זימני ובדיקה אבל אמצעית דקיימי להו בימי זיבה לא בעיא בדיקה ועקרא להו בחדא זימנא, אלו דברי הרב ז"ל, והם צ"ע, ואין אני מוצא בהם טעם שהרי אשה זו לא הפלינה כ"ט כדי שתקבע לה וסת להפלגת כ"ט.\par  ועוד מצא הרב ז"ל קביעת וסת במקרבת ראותיה ואמר שלא אמר שמואל אין אשה קובעת וסת בימי זיבתה אלא בתוך אחד עשר כדקתני מתניתין אבל בימים הראויים לספירת אחד עשר קובעת הילכך אם ראתה שלשה ימים בתוך אחד עשר ויום א' בתחלת [ימי נדה] אף על גב דלא מתחלת נדה עד דספרה שבעה נקיים קובעת דלעניין וסת ימי נדה חשיבי שלא טהרו אלא אחד עשר וה"ט משום דקים להו לרבנן דבשבעה ימי נדה היא מתמרקת מדמיה ולא הדרי עד אחד עשר יום וכי מטיא לשמונה עשר יום בין שהיא ראוייה לנדה בין שהיא מספירת זיבה ראויין הן לקביעת וסת, וכבר הורה זקן ז"ל.\par  אלא שעכשיו אין הנשים משמרות פתחי נדה וזיבות וכטועות משוינן להו ומכל מקום אם ראתה שלשה פעמים כל א' וא' בתוך י"ט יום זה דבר [ברור] הוא שעדיין לא קבעה וסת שהרי אי אפשר שלא תהא אח' מהן בתוך אחד עשר ממש אבל אם ראתה על הסדר הזה שמונה פעמים קבעה לה וסת מפני האמצעיות כמו שכתבנו לדעת הרב דהאיכא ארבע ראיות וקמייתא לאו בהפלגה חזיתא ולא הוו תלתא הפלגות עד דאיכא ארבע ראיות וחוששת היא פעם אחת לכולן.\par ולפי דעתי מכאן שהחמירו בנות ישראל על עצמן להיות כטועות, ורבי נמי דאתקן להן הכי אין מונין לעולם ואין משגיחין על שמונה עשר ולא על ימי נדה והאשה קובעת וסת בכל זמן שתראה בין במקרבת ראיותיה בין במפלגת אותן ולא נצטרך ללמד לבנות ישראל ימי נדה וי"א שכבר נשתכח מהן לגמרי וגזרו עליו ולא נחלק נמי בין רואה בהן שלש פעמים לרואה שמונה שלא יבואו לטעות בקביעת הוסת והרבה קלקולין שיהיו קרובין לבא בדבר.\par  וחכמי צרפתים זכרונם לברכה כתבו בתוספות דרבי יוחנן וריש לקיש פליגי אדשמואל והלכה כמוהם ואשה קובעת לה וסת בין בימי נדה בין בימי זיבה (ודין) [לדין] התלמוד ומכל מקום בתוך שבעה ימים שראתה בהן נדה אינן יכולין להחמיר לדין הגמרא כדחנן בפרק קמא חוץ מן הנדה וכו'. ועכשיו לעולם קובעת. }
\twocol{\textbf{מקבע לא קבעה.}  פרש"י ז"ל דתיבעי ג"פ לעקרן, מיחש מהו דניחוש לה אם היתה רנילה מט"ו לט"ו דהיינו ימי זוב מיבעי' למיחש ולא תשמ' ליום ט"ו קודם ראיה שמא תראה ואינו יודע היכי אתינן למיפשט הא מילתא מדשמואל החס לר' פפא נמי ראית עשרין ותרין קמייאתא בימי נדה הוו וראית עשרין וז' דהאידנ' בימי נדה הויא, ולהאי פירושא אמאי לא תיחוש להו בכל זמן שיבא לה וסתה כיון שהוקבע הוסת כראוי בזמן נדה.\par  אלא אפשר לפרש מקבע לא קבעה מיחש מהו דתיחוש לה אם היה לה וסת בין קבוע בין שאינו קבוע בימים הראוים לוסת ואירע לה אותו היום בימי זיבה מהו שתחוש שמא בימים הללו תראה ולא תשמש או דילמא כשם שאינה קובעת וסת בתוך י"א כך אינה חוששת לוסת הראשון שלה בימי י"א והא מילתא מיפשטא בהדיא מדשמואל לפירושיה דרב פפא.\par  ולענין גמר' ודאי מסתברא דלית הלכתא כרב פפא אלא כרב הונא ברי' דר' יהושע ואיהו כיון דאידחיא לראיה דרב פפא מדשמואל ודאי מיפלג פליג עלי' וכיון דוסתות דרבנן אע"ג דלא איפשטא ליה לרב הונא לקולא אנן לקול' נקיטו בה ואע"ג דאמר רב פפא בשלהי האשה דלק' דחיישא, רב פפא לא מהימן בה דאיהו מדשמואל אמרה והא אידחי, אבל הרב רבי אברהם בר דוד ז"ל פסק בספרו כר' פפא ואף אנו עליו נסמוך וכ"ש מאחר שכתבנו שאין הנשים יודעת פתחי נדה שכל ראיה שהן רואות חוששין לה בכל זמן שהוא בתוך נדה הן. }
\twocol{ גרסת רש"י ז"ל: \textbf{א"ל רב פפא אלא הא דאמר ר"ל וכו' אלמא מריש ירחא מנינן.}  כלומר מדקרי ליה לפעם ג' בתוך ימי נדה לפי חשבון הראוי מנינן.\par  וק"ל והא ר' פפא גופיה נמי אית ליה נדה ופתחה מכ"ז מנינן ולא לפי חשבון הראוי ועוד דילמא משום תרי זימני קמא דקביעינהו בימי נדה קרי ליה הכי.\par  ואפשר דהכי קאמר ליה והא הכא (דמעשרין וחמשה) [דמחמישה בירחא] לריש ירחא קמייאתא [{\small פי' לריש ירחא ב' וכונתו הפסקה קמייתא לריש ירחא} ] לא הפליגה אלא כ"ה יום ואע"פ כן כיון שאנו רואין לה עכשיו שהפליגה למ"ד מנינן מריש ירחא ולא מנינן מחמשא בירחא לרישי ירחא ונאמר עיקר הוסת בראשון ו) ירחא לריש ירחא הוא דאיכא למ"ד ובשני הויא ליה עיקרו של וסת מחמשה לחמשה דהא ודאי השתא בהפלגת למ"ד חזאי וקבעה לה וסת להפלגות למ"ד וראיות דחמשה בירחא קמא לא מפסקא, ולא מנינן מינייהו אלמא דמנין כ"ב מעיקרא נקטא והתם נמי כיון שכבר קבעה לה וסת ואנו צריכין לחוש לו מכ"ב לכ"ב מעיקרא וסת ראשון מנינן ואין ראיה שבאמצע מפסקת, ופריק ר"ה בריה דר"י דלא אמרי' הכי אלא ברואה מעיקרא (ממנין) [ממעין] סתום ועכשיו קרבה בתוך כך דאפילו בימי נדה אמרי תוספת דמים הוא, וכ"ש בימי זיבה דכיון דקמייאתא (ממנין) [ממעין] סתום והפליגה שתיים ועכשיו נמי ראתה יש לנו לומר תוספות הואי אבל לחוש אינה חוששת אלא למנין ראיות.\par  וה"ה ודאי דה"ל לתרוצי שאני התם דכיון דחזאי [בפעם ג' בחמישה בירחא] אמרי' דחמשה בירחא עיקר ודרישי ירחי (עיקר) [מיקרי] תוספות ואע"פ כן לחוש אין חוששין אלא בהפלגות ולא למנין הראוי. אלא מסתברא ליה דקביעות ראיות קמייאתא לא אמרי תוספת דמים הוו. וגם זה הפי' אינו מחוור.\par  ומ"מ היינו ריש ירחי דנקט לאו דוקא אלא הפלגו' שוות בהן דאלו בוסת החדש ודאי למנין הראוי חוששת דהא לאו בהפלגות שוות חזיא מעיקרא אלא בימי דחדש הוא דהשוות ראיותיה וכדבעינן לברורי קמן בפרק האשה (סד, א).\par ושמע מינה לרב הונא דאינה מונה כ"ב אלא מכ"ז ואי חזאי בכ"ב חזר הוסת הראשון למקומו ונעקר החדש לגמרי אבל אם לא ראתה בעשרין ותרין שחששת לו חוששת לסוף חמשה ימים דה"ל כ"ז מכ"ז שראתה בו תחלה והיינו דתרנגולת'. [{\small ע' ביאור דברי הרמב"ן בריטב"א} ] וכן דעת ה"ר משה ולא כן פסק הרב ר' אברהם בספרו ז"ל.  א"ל רב פפא אלא הא דאמר ר"ל וכו' אלמא מריש ירחא מנינן. כלומר מדקרי ליה לפעם ג' בתוך ימי נדה לפי חשבון הראוי מנינן.\par  וק"ל והא ר' פפא גופיה נמי אית ליה נדה ופתחה מכ"ז מנינן ולא לפי חשבון הראוי ועוד דילמא משום תרי זימני קמא דקביעינהו בימי נדה קרי ליה הכי.\par  ואפשר דהכי קאמר ליה והא הכא (דמעשרין וחמשה) [דמחמישה בירחא] לריש ירחא קמייאתא [{\small פי' לריש ירחא ב' וכונתו הפסקה קמייתא לריש ירחא} ] לא הפליגה אלא כ"ה יום ואע"פ כן כיון שאנו רואין לה עכשיו שהפליגה למ"ד מנינן מריש ירחא ולא מנינן מחמשא בירחא לרישי ירחא ונאמר עיקר הוסת בראשון ו) ירחא לריש ירחא הוא דאיכא למ"ד ובשני הויא ליה עיקרו של וסת מחמשה לחמשה דהא ודאי השתא בהפלגת למ"ד חזאי וקבעה לה וסת להפלגות למ"ד וראיות דחמשה בירחא קמא לא מפסקא, ולא מנינן מינייהו אלמא דמנין כ"ב מעיקרא נקטא והתם נמי כיון שכבר קבעה לה וסת ואנו צריכין לחוש לו מכ"ב לכ"ב מעיקרא וסת ראשון מנינן ואין ראיה שבאמצע מפסקת, ופריק ר"ה בריה דר"י דלא אמרי' הכי אלא ברואה מעיקרא (ממנין) [ממעין] סתום ועכשיו קרבה בתוך כך דאפילו בימי נדה אמרי תוספת דמים הוא, וכ"ש בימי זיבה דכיון דקמייאתא (ממנין) [ממעין] סתום והפליגה שתיים ועכשיו נמי ראתה יש לנו לומר תוספות הואי אבל לחוש אינה חוששת אלא למנין ראיות.\par  וה"ה ודאי דה"ל לתרוצי שאני התם דכיון דחזאי [בפעם ג' בחמישה בירחא] אמרי' דחמשה בירחא עיקר ודרישי ירחי (עיקר) [מיקרי] תוספות ואע"פ כן לחוש אין חוששין אלא בהפלגות ולא למנין הראוי. אלא מסתברא ליה דקביעות ראיות קמייאתא לא אמרי תוספת דמים הוו. וגם זה הפי' אינו מחוור.\par  ומ"מ היינו ריש ירחי דנקט לאו דוקא אלא הפלגו' שוות בהן דאלו בוסת החדש ודאי למנין הראוי חוששת דהא לאו בהפלגות שוות חזיא מעיקרא אלא בימי דחדש הוא דהשוות ראיותיה וכדבעינן לברורי קמן בפרק האשה (סד, א).\par ושמע מינה לרב הונא דאינה מונה כ"ב אלא מכ"ז ואי חזאי בכ"ב חזר הוסת הראשון למקומו ונעקר החדש לגמרי אבל אם לא ראתה בעשרין ותרין שחששת לו חוששת לסוף חמשה ימים דה"ל כ"ז מכ"ז שראתה בו תחלה והיינו דתרנגולת'. [{\small ע' ביאור דברי הרמב"ן בריטב"א} ] וכן דעת ה"ר משה ולא כן פסק הרב ר' אברהם בספרו ז"ל. }
\newchap{דף \hebrewnumeral{40}}
\twocol{\textbf{ורבי שמעון ההוא דאפילו לא ילדה אלא כעין שהזריע' אמו טמאה לידה.}  וא"ת לרבנן נמי מבעי להו להכי כיון דאינהו אמרי הבית טמא ולית להו נימוק הולד עד שלא יצא כדאיתא בפרק המפלת פשיטא דאמו טמאה לידה זה כתב רש"י ז"ל.\par  וק"ל נהי נמי דלית להו במפלת שליא נימוק הולד היכא דחזינן ודאי דנימוק מנא להו דאמו טמאה לידה ואפשר דלר"ש דאית ליה ביטול ברוב לענין טומאה איצטרך קרא לענין לידה אבל לרבנן לית להו ביטול בכה"ג, ואיכא למימר נמי לרבנן ממילא ש"מ דא"כ לימא קרא כי תבעל וילדה מאי כי תזריע דאפילו לא ילדה אלא כעין שהזריעה טמאה לידה. }
\twocol{\textbf{מ"ט גמר לידה מבכור.}  פי' דאי לאו ג"ש כיון שנתרבה יוצא דופן בכלל לידה גבי אדם א"א למעטו מכי יולד ולהכי צריך ג"ש דלידה לידה והא דאמרינן שכן אמו מאמו לא דהיא ג"ש אלא מסתברא לידה לידה דאדם גמר כן משום דדמי מדמי הוא דהכא כתיב אמו והכא כתיב אמו והא דאמרינן לקמן מאמו אמו נפקא לאו דוקא אלא חדא מטעמיה דג"ש נקט : }
\twocol{\textbf{זאת תורת העולה היא העולה הרי אלו ג' מעוטין פרט לנשחטה בלילה וכו'.}  פירש לר' יהודה כיון דממעט הני אף על פי שפסולן בקדש ומכשר הלן והיוצ' ושארא כדבעינן למימר צריכי מעוטא לכל חד וחד אבל לר' שמעון כיון דקרא א' מרבה וקרא א' ממעט הריבוי ריבה הכל והמיעוט מיעט הכל, ול"ק כאן שפיסולו בקדש כאן שאין פיסולו בקדש.\par  והא דאמרינן הרי אלו [ג'] מיעוטין ולא אמרינן אין מיעוט אחר מיעוט אלא לרבות משום דכיון דכל אחד ממעט את שלו אין כאן מיעוט אחר מיעוט שאין מיעוט אתר מיעוט לרבו' אלא כשהן ממעט' דבר אחד כגון שאמרו (סנהדרין טו, א) עשרה כהנים כתובים בפרשה כהן ולא ישראל ואין מיעוט אחר מיעוט אלא לרבות אבל כאן כל מיעוט הוא צריך למעט את שלו.\par וכיוצא בזו בב"ק (מד, ב) שור שור שור ז' פעמים להוציא שור האשה ושור היתומין ושור האפוטרופסין וכו' ולא היו מיעוט אתר מיעוט והני תלתא מיעוטי נמי ממעטי הני תלתא פסולי כדפרישית.\par אבל במס' הוריות מצאתי בפרק ראשון בירושלמי (ה"א) גבי נפש כי תחטא אחת תחטא בעשותה [תחטא] הרי אלו מיעוטין דמקשי בכל אתר את אמרת מיעוט אחר מיעוט לרבו' וכאן את אמרת מיעוט אתר מיעוט למעט א"ר מתניא שניה היא דכתיב מיעוט אחר מיעוט לאחר מיעוטי ולדעת זו ההיא דאמרי' בסנהדרין כהן ולא ישראל מפני שכולן צריכין לכתב לומר דעשרה בעינן ד) אבל בשאר דוכתי ג' מיעוטין או יותר נדרשין הן כולן למיעוט כמשמען ולא נאמרה מדה זו בתורה אלא בשני מיעוטן מיעוט אחר מיעוט.\par  מ"מ כל הנך פסולי דמכשיר ר' שמעון מודה בהו ר' יהודה וטעמא מפרש במסכת זבחים (דף פ"ד) מפני מה אמרו לן בדם כשר שהרי לן כשר באימורין לן באימורין כשר שהרי לן כשר בבשר, יוצא הואיל וכשר בבמה, טמא הואיל ואשתרי לגבי צבור, ונשחט חוץ לזמנו הואיל ומרצה לפיגול, חוץ למקומו הואיל ואתקו' לחוץ לזמנו ושקבלו פסולין וזרקו את דמו בהנך פסולין דחזו לעבוד' צבור וכי דנין דבר שלא בהכשרו מדבר שהוא בהכשרו תנא אזאת תורת העולה ריבה קא סמיך הדין גמרא דהתם (ובתכפה) [וכתבנוה] מפני שהיא (תמה) [סתומה] ויש לדקדק בה דא"כ נשחטה בלילה נמי כשרה שהרי כשר בבמת יחיד כדאי' בזבחים (דף ק"כ), איכא למימר אתיא כמ"ד התם שחיטת לילה פסולה בבמת יחיד.\par  אלא הא קשיא יצא דמה חוץ לקלעים נימא דכשר שהרי כשר בבמה שאין יוצא בבמה לא בבשר ולא בדם וכן חוץ למקומו נמי דקאמר הואיל ואתקוש לחוץ לזמנו לימא הואיל וכשר בבמה, ועוד דקאמר ושקבלו פסולין וזרקו את דמו בהנך פסולין דחזו לעבודת צבור אפילו זר גמור נמי יהא כשר שהרי כשר בבמת יחיד.\par  ואיכא למימר נשחטה בלילה ויוצא דמה ליכא לאכשורי (בשרן) דאי הכי מיעוטין מאי אהנו לי ומסתבר' ליה לאוקמי בהני דבעיקר הכשרן נפסלו מאינך והא דאוקי בהנך פסולי דחזו לצבור ולא אמר משום דכשרין בבמה דבהא פשיטא ליה דליכא למילף מינה שזה ודאי דבר שעקר הכשרו כן הוא ולית להו הכשר בכשרין טפי מפסולין לעולם ודקאמרת חוץ למקומו הואיל ואתקוש משום דכל היכא דמשכח הואיל בפנים לא מייתי ליה מבמה, כנ"ל.\par ובתוספות מאריכין בע"א, וניתנין למעלה שנתנן למטה וכן בחוץ ובפנים ופסח וחטאת כולהו כיון דבפנים נמי אשכחן בהו הכשרא לא צריכא ליה למימר' דודאי לא ירדו. }
\newchap{דף \hebrewnumeral{41}}
\twocol{והא דאמרינן \textbf{חד לבהמת קדשים.}  פי' דאע"ג דהאי וזאת לא מיתר ביוצא דופן שאם עלה ירד מיהו אין במשמע תורת העולה לרבות כל העולי' לגמרי כיון שמקרא א' מרבה ומקרא אחר ממעט וא"א להעמי' המיעוט בבהמת תולין (דכולהו מעלמא) [דחולין מאמו] נפקי אפילו לענין שאם עלו ירדו הילכך ע"כ מוקמינן מיעוטא בבהמת קדשים וכיון דשקולים הם יבואו כולם וכל שהוא פסול בבהמת חולין מיעט בבהמת קדשים לפסול ולומר שאפילו אם עלה ירד דמיעוטא הכי משמע שאם עלו נמי ירדו. }
\twocol{הא דאמרינן \textbf{מקור מקומו טמא.}  לא דמקור בנגיעה דידיה מטמא דם דהא בית הסתרים הוא ודם גופה אינו לא אוכל ולא משקה אלא גזרת הכתוב שטומאה נבראת שם מתחלה וכל שנברא בו טמא הוא ומטמא (ביציאותיו) [בנגיעתו] טומאת ערב במשהו כדם הנדה דבמקומו הוא נעשה דם הנדה, וה"נ מוכח בפ' בתרא דמכילתן דקאמרינן באשה שמתה ויצאה ממנה דם במשהו מטמא משום כתם משום דמקור מקומו טמא ולא מטמא ליה נגיעה דמת. }
\twocol{והא דאמרינן \textbf{ואזדא ר' יוחנן לטעמיה שאמר משום רשב"י וכו'.}  אף ע"ג דר"י להא משום רשב"י נמי אמרה ור"ל נמי לר"ש הוא מודה דאשה טהורה היא משום דנסיב ליה קרא מפורש קאמרינן הכי לימא דאין אשה טמאה קרא מפורש הוא ובאי דם עצמו טמא פליגי מר גמר לדם מאשה ומר לא גמר. }
\twocol{הא דאקשי' לר' שמעון \textbf{פולטת תיפוק לי' דהא שמשה.}  ה"ק למה לי טומאה פולטת בחוץ הא שמשה ונטמאת אפילו בפנים, ואלו לרבנן י"ל שלא הצריכה התורה טבילת משמשת אלא מפני פליטתה שאלו מפני שמוש' טהורה היא דהא מגע הוא ואותו מקום בית הסתרים הוא למגע אבל כשהיא חוזרת ופולטת עשאה הכתוב כרואה מדכתיב יהיה וההוא גלי אורחצו במים דמשום פליטה הוא דטמאים אלא לר"ש קשיא. ומהדרינן בטבלה לשמושה. }
\twocol{ואקשינן \textbf{למימרא דמשמשת בטומאת ערב סגי לה ולהכי טבלה והאמרת רבא וכו'.}  ואע"ג דרבא משום פליטה קאמר ואפשר לך לומר שבזה בא ר"ש ללמד שאינה טמאה עד שתצא טומאתה לחוץ והך קושיא לרבנן נמי היא אלא כיון דאיירי בדר"ש מפרש ואזיל בהדי' ומהדרי' בשהטבילוה במטה כלומר וטהורה לשמושה ואח"כ פלטה טומאת' לבית החיצון ולא יצא לחוץ שלא הלכה ולא נתהפכ'. }
\twocol{ואקשינן \textbf{מכלל וכו'.}  וכי תימא דילמא אשתייר ומספיק' אסרינן לה בטומא' אי הכי חיישינן שמא נשתיי' מיבעי ליה אלא ודאי מדלא קאמר הכי ש"מ דכל היכא דאזלא בכרעא מותרת בתרומה דודאי שדתיה לכוליה ודרבא בשלא הלכה הוא דקאמר והתם לא הוה צריך למימר חיישינן אלא א"א הוא. }
\newchap{דף \hebrewnumeral{42}}
\twocol{\textbf{אלא לרבא נמי כשהטבילוה במטה.}  ורבא אקרא קאי ופריש דכי כתיב עד הערב לאו במהלכת ולאו במכבדת את הבית דא"כ תיפוק לי משום נגיעת חוץ אלא בשוכבת על מטתה ואינה מתהפכת אבל מתהפכת כל ג' ימים נמי אסור' ולאו מגזרת הכתוב דמשמשת אלא משום פליטה וי"ל דאפילו במהלכת חיישינן שמא נשתייר ולהכי לא אוקמוה לקרא (בהדי) [בהכי] וכי אקשינן אי הכי חיישינן שמא נשתייר מיבעי ליה לאו למימרא דלא חיישינן אלא לומר דרבא לא בכי האי גוונא מיירי.\par ולאו מילתא היא דא"כ לישמועינן רבא דעדיפא ולימא כי כתיב בשאינה מתהפכת אבל במתהפכת אפילו מהלכת כל ג' ימים אסורה לאכול בתרומה שמא תפליט ועוד אי משום חששא דשיור לא מוקי קרא במהלכת לוקמיה משמשת במכבדת שהיא טהורה משיור לגמרי אלא הקרא לא מתוקס בהנך גווני משום דאיכא נגיעת חוץ בפליטה כדפי' רש"י ז"ל.\par  והא דתנן במסכת מקואות (ח, ד) האשה ששמשה את ביתה ירדה וטבלה ולא כבדה את הבית כאלו לא טבלה תפתר בשלא הלכה ואעפ"כ אם כבדה את הבית מותר' דודאי נפק כוליה. א"נ כשהלכה דפולטתו בבית החיצון וכותלי בית הרחם העמידוהו ומ"ה צריכה כבוד לטהרות משום נגיעה דשכבת זרע אלא דלית ליה דין פולטת לראיה ולטמא בפנים אלא ראשון דמגע שכבת זרע הוי. }
\twocol{הא ד\textbf{בעא מיניה רב שמואל בר ביזנא מאביי.}  פולטת ש"ז אי רואה הויא או נוגעת. ואסיקנא דלרבנן רואה הויא לכל מילי ואפילו לר"ש נמי לסתור ולטמא במה שהוא רואה הויא משמע לי דפשטין ודאי דלא כרב הונא דאמר בפ' המפלת דאפילו בעל קרי לא סתר אלא משום דא"א לו משום צחצוחי זיבה דמדבעי חתימת פי האמה אלמא נוגע הוי וכ"ש בפולטת דלא מגופ' הויא ואינה מטמאה אלא בחוץ דאית לן למימר למימר נוגעת הויא ולא סתרה דהא אין בפליטה דאשה צחצוחי דם כלל.\par  וא"ת אי הכי קשיא לאביי היכי פשיטא לר"ש לסתור ולטמא במשהו הויא דהא שמעינן ליה לר"ש דס"ל כר' נתן דאמר זב צריך חתימת פי האמה ואיתקוש ב"ק לזב וצריך נמי חתימת פי האמה כדאיתא בפ' אלו דברים. וי"ל קסבר אביי דבעל קרי אע"פ דצריך חתימת פי האמה רואה הוי ושעורא בעלמא הוא דאצרכיה רחמנא דומיא דזב וכדפרישית בפ' המפלת. ומיהו גבי אשה דליכא למימר תחימת פי הרחם דשיעורא אחרינא הויא במשהו כרוא' דם בזיבה ולא אמרינן בהא דייה כבעלה משום דלא אפשר.\par  וי"מ דבמשהו דקאמרינן כעדשה דשרץ קאמרינן מ"מ מסקי השתא דרואה הויא.\par ויש מחכמי הצרפתים ז"ל שחדשו בה ואמרו כיון שהיא רואה וסותרת יומה בזמן הזה שכל הנשים ספק זבות משוינן להו אשה ששמשה בלילי שבת וראתה בשב' ופסקה טהרה בו ביום או למחרתו אינ' מתחלת ספירתה עד יום ד' בשבת שהוא ג' ימים לאחר שמושה מ"ט דכל ג' ימים פולטת היא וסותר' וקי"ל נמי דשש עונות שלימות מטמאה בפולטת והלכה למעשה הורו והנהיגו הדור בחומרא זו אלא שמתירין בכבוד הבית יפה.\par ויש לדקדק אחר דבריהם שאפילו הדבר כן שהפולטת סותרת לבעלה אין לחוש כן במהלכ' שכבר נתפרשה לנו שאם הלכה מותרת בתרומה לערב ואין חוששין שמא נשתייר אבל אם האשה הרגישה בעצמה שפלטה בשני שלה או בג' או בעומדת על מטתה ומתהפכת בכאן יש מקום להורא' זו.\par  והרב ר' אברהם בר דוד ז"ל נשאל בהוראה זו ואמר שלא אמרו פולטת סותרת אלא בענין תרומה וקדשים ולענין טהרות אבל לבעלה אינה סותר' שהרי אמרו דבר הגורם סותר דבר שאינו גורם אינו סותר. וכדתניא מה גורם לו זובו ז' לפיכך סותר ז' מה גרם לו קריו יום א' לפיכך סותר יום א' ואפילו קושי אינו סותר בזיבה מפני שאינו גורם עכשיו כל שכן פולטת שאינה גורמת טומאה לבעלה שאינה סותרת לעולם ועולה. ומצאתי בתוס' שהוזכרה ביניהם סברא זו ודחוה בשתי ידים ולא קבלו אותה כלל.\par  והרב ז"ל הביא ראיה לדבריו מדתניא ד"ש אומר ואחר תטהר אחר מעשה תטהר אבל אמרו חכמים אסור לעשות כן שמא תבא לידי הספק דאלמא אי לאו חששה דראיה משמשת והולכת ואמאי והא יש לחוש לפולטת שסותרת.\par  וזו אינה ראיה, די"ל התורה לא חששה לפולטת שאפשר לה שלא להתהפך, וכן זו שהקשו בפ' המפלת בעשרין וחד תשמש לרוחא דמילתא מתרצי משום דמשמע דאסר לה לשמש בכל ענין ואע"פ שלא תתהפך כל היום וכ"ש למאן דסבר נוגעת הויא דצרכינין לאוקומא כר' שמעון וחכמים אף לזה חששו ואסרוה לשמש, א"נ תטהר לתרומה וקדשים ואסרו חכמים לעשות כן שמא תבא לםפק ראיה. והכי תניא בהדיא בספרי אחר מעש' תטהר כיון שטבלה טהורה להתעסק בטהרו' אבל אמרו חכמים וכו'.\par אלא מהא דאמרינן בשלהי בא סימן יום א' טמא ויום א' טהור משמשות שמיני ולילו וד' לילות מתוך י"ח ימים והא הכא דמשמשת ליל ט' והתשיעי טמא וסופרת עשירי ואין חוששין לה ג' ימים משום פליטה. גם זו אינה ראיה למה שכתבנו דמהלכת אינה חוששת לשייר. א"נ במקנחת ומכבדת את הבית.\par  ובודאי שדברי הרב ר' אברהם ז"ל מכריען בטעמן שכל שאינו גורם אינו סותר ולא מצינו לובן באשה לבעלה. אלא שיש לבעל דין לחלוק ולומר שלא נתנה דבריה לשעורן וכיון דסתרה לטהרת סותרת לבעלה דנקיים מכל ראיה דטומאה בעינן וקראי נמי דייקי דכתיב ואתר תטהר וביום השמיני תקח לה וכו'. ולפי מדה זו יש שטהורה בשביעי לביתה ובח' (עראי) [היא] סופרת ודברי הרמב"ם ז"ל מטין כן שהפולטת סותרת יומא לכל דבר אבל רבינו הגדול והגאונים ז"ל לא תששו לכתוב הדבר ובעל נפש יחוש לעצמו. }
\twocol{\textbf{יולדת שירדה לטבול מטומאה לטהרה ונעקר ממנה דם.}  פי' רש"י ז"ל לבית החיצון. וק"ל דא"כ היכי אקשי אמאי טומאה בלועה היא דהא אמר רבא לקמן דטומאת בית הסתרים הוה ומטמא במשא שהרי דם נדה מטמא במגע ובמשא.\par ואיכא לפרושי דנעקר לאו לבית החיצון משמע אלא רגישה בעלמא שנעקר מן המקור אע"פ שהוא בין השניים ולפנים טמאה ולכך אקשינן טומאה בלועה הוא דע"כ לא אמר רבא בית הסתרים הוי אלא בבית החיצון דהיינו בין השינים לר' יוחנן אבל לפנים בלוע הוי.\par וא"ת כי מוקמינן לדר' זירא בשיצא לחוץ נמי אמאי קשיא לן אלא יולדת אי בימי נדה נדה אי בימי זיבה זיבה נתריץ נמי ברייתא כגון שנעקר דם לבית החיצון ומטמ' התם במשא.\par  לאו מילתא היא דהאמרינן דאפילו נעקר קצת מטמא לכשיוצא ולא מהני ליה טבילה ואי לאשמועינן דמעכשיו נמי היא טמאה בבית החיצון כמו שיצא הוה ליה למיתני עקירה ממקומו טומאה בבית החיצון ולא קתני ברייתא הכי אלא אי ברייתא כדר' זירא משמע מינה דלא הויא עקירה דלא להני לה טבילה עד דהוי במקום טומאה דהיינו בבית החיצון. אי נמי ניחא לן לתרוצה אליבא דכ"ע דלא תיקשי לן הניחא לרבא אלא לאביי מאי איכא למימר. ומיהו כי מתרצינן מעיקרא ברייתא דקתני כולן מטמאו' בפני' כבחוץ כי הא דר' זירא מצינן למידק עלה והא דומיא דנדה וזבה קתני והתם בבית החיצון והכא אפילו בפנים אלא אעיקר מילתא דייקינן למיקם אפירושה א"נ דהוה ליה למימר מידי איריא לטמויי בפנים כבחוץ הן שוין. אבל בשיעור מקומן הא כדאיתא והא כדאיתא.\par  וי"מ פירכא אליבא דאביי ופירוקה נמי לאביי דאמרינן עשאוה כנבלת עוף טהור שמטמאה בגדים בבית הבליעה ומדמי לה אלמא התם נמי טומא' בלועה הוי דאי התם בית הסתרים הוי מאי דומיא הכא בבית הסתרים ודאי מטמיא במשא ובטומא' בלועה לא מטמיא כלל דלא דמיא נמי לנבלת עוף טהור. וזה הדרך ראיתי בתוספות.\par ולשון שלי הראשון מחוור ממנו לפי שנבלת עוף טהור מטמא בכל מקום מבית הבליעה ואפילו פנים דהוי טומאה בלועה וכי פליגי לקמי בתחלת בית הבליעה דלאביי ליכא מקום דתטמא בית הבליעה בנבלת בהמה במשא ועוף במגע ולרבא [איכא] היכא נמי דמטמא עוף בבית הבליעה מטמאה נמי בהמה משום משא הא לעוף אין לך מקום בושט ואפילו בלועה דלא ליטמיה ביה דלא יאכל לטמאה אמר רחמנ' כל דאכיל מטמא וברייתא דמסתייע אביי מינ' משום דמשמע דלעולם אין לבהמה טומאה בבית הבליעה ולא משמע להו לדחוקה בסוף הבליעה בלחוד דכתיב בה ולא באחרת. }
\twocol{\textbf{עשאוה כנבלת עוף טהור.}  ויש לפרש דה"ק חכמים גזרו על טומאה זו מפני שדרכה לצאת ועשאו' כנבל' עוף טהור ולמיסמך גזירה דרבנן אכעין דאורייתא קאמר שאלמלא שמצינו טומאה בלועה מטמאה בדאורייתא לא הוו גוזרין בהו שמטמא. }
\newchap{דף \hebrewnumeral{43}}
\twocol{הא דאמרינן \textbf{כל שכבת זרע שאינו יורה כחץ אינה מטמאה.}  ק"ל סריס חמה אינה מטמא בביאה ואין שכבת זרעו מטמא דהא אינו יורה כחץ ומשמע לי שהסריס יש לו הרגשה בין בעקירה בין ביציאה כדאמרינן ביבמות (סה, א) איהי קים לה כיורה כחץ איהו לא קים לי' כיורה כחץ. ואם אינו מרגי' הא קים ליה דלא מרגיש.\par  אלא הכי קתני כל שאין בו הרגשה בעקירה וביציאה כיורה כחץ אינה מטמאה הא הרגיש בה ואין בה כח בירידתה כיורה חץ טמאה אע"פ שאינה מתעברת בה דההוא חולי בעלמא הוא דשכבת זרעו דיהה.\par  והיינו דקאמרינן מאי איכא בין האי לישנא וכו'. דלאו למעוטי סריס ומי ששהה עם אשתו ואינו יורה בה כחץ אתינן ובי דרשינן נמי שכבת זרע בראויה להזריעה לומר שנעקרה ויוצאה בכח וראויה להזריע דההיא שעתא הוא דמטמיא אע"פ שחוזרת להיות דיהה ואינו יורה כחץ באשה.  כל שכבת זרע שאינו יורה כחץ אינה מטמאה. ק"ל סריס חמה אינה מטמא בביאה ואין שכבת זרעו מטמא דהא אינו יורה כחץ ומשמע לי שהסריס יש לו הרגשה בין בעקירה בין ביציאה כדאמרינן ביבמות (סה, א) איהי קים לה כיורה כחץ איהו לא קים לי' כיורה כחץ. ואם אינו מרגי' הא קים ליה דלא מרגיש.\par  אלא הכי קתני כל שאין בו הרגשה בעקירה וביציאה כיורה כחץ אינה מטמאה הא הרגיש בה ואין בה כח בירידתה כיורה חץ טמאה אע"פ שאינה מתעברת בה דההוא חולי בעלמא הוא דשכבת זרעו דיהה.\par  והיינו דקאמרינן מאי איכא בין האי לישנא וכו'. דלאו למעוטי סריס ומי ששהה עם אשתו ואינו יורה בה כחץ אתינן ובי דרשינן נמי שכבת זרע בראויה להזריעה לומר שנעקרה ויוצאה בכח וראויה להזריע דההיא שעתא הוא דמטמיא אע"פ שחוזרת להיות דיהה ואינו יורה כחץ באשה. }
\twocol{הא דאמרינן \textbf{את"ל בתר עקירה אזלינן לחומרא.}  ק"ל אמאי לא פשטה להא מילתא מהא דאמרן ונעקר ממנה דם בירידה טמאה בשלמא בעייא דרבא ל"ק דאיהו לא משום ספיקא דבתר עקירה אזלינן בלחוד מספקא ליה אלא משום דשמואל דכיון דאינו יורה כחץ בשעת טומאתו אינו מטמא אבל הא דאמרינן גבי ירד וטבל את"ל בתר עקירה אזלינן קשיא.\par  ואיכא למימר משום דכיון שטבל בנתיים א"א שלא בטלה הרגשתו ומיהו זבה שנעקרו מימי רגליה דמיא לההיא אלא דהתם מצי נקיט להו והא דקאמרינן ביה את"ל סרכא נקט. }
\newchap{דף \hebrewnumeral{44}}
\twocol{הא ד\textbf{אמר רב ששת הב"ע בכהן שיש לו שתי נשים וכו'.}  פי' רש"י ז"ל ודוקא בן יום א' אבל עובר לא משום דאין זכייה לעובר. וכן דעת רבינו אלפסי ז"ל וכתוב בהלכתא בפ' אלמנה לכ"ג (יבמות סז, א).\par וק"ל א"כ למה ליה לאוקומא בכהן שיש לו שתי נשים א' גרושה וכו' לימא שפוסל בעבדי אביו מלאכול בשביל המשפחה ועוד דר' ששת דקא מתרץ לה איהו הוא דאמר בפרק מי שמת (דף קמ"ב) המזכה לעובר קנה וכ"ש בירושה הבאה מאליה ומקשינן עלה התם מסוף דמתני' דקתני נוחל ומנחיל ומתרץ כי הכא משו' דאיהו מאית ברישא.\par אלא ה"פ: דרב ששת לטעמיה דאמר יש זכייה לעובר ומיהו בכהן שיש לו שתי נשים ויש לו בנים משאינ' גרושה כיון דאי עובר שבמעי הגרוש' נקבה היא לא פסלה ואי נפל הוא לא פסול הוו להו זכרים מיעוט' ולמיעוטא לא חיישינן ולאפוקי מדר' יוסי דאמר חוששין למיעוט ואפילו כולן זכרים לא יאכלו בשביל חלקו של עובר קמ"ל בן יום אחד אין עובר לא ואפילו כלו חדשיו נמי כיון שלא נולד י"ל שמא נפל הוא שאינו ראוי לברית נשמה.\par  והך פלוגתא מפרש בגמרא ביבמות פרק אלמנה לכהן גדול וכבר פירשתי שם בפ' מי שמת הארכתי בכל הסוגיא הזו והרוצה לסמוך על העיקר ע"ש. }
\twocol{\textbf{מ"ט דאיהו מיית ברישא.}  פיר' ודוקא מתה דקאמרינן בערכין פ"ק (דף ז) אשה היוצאת ליהרג מכין אותה כנגד בית הריון שלה כדי שימות הולד תחלה והוינן בה למימרא דאיהי מתה ברישא והא קי"ל דאיהו מאית ברישא דתנן בן יום א' וכו'. ומפרקינן ה"מ מתה דאגב דולד זוטר חיותיה עולה ביה טיפה דמלאך המות ותתיך לה לסימנים אבל נהרגה איהי מתה ברישא ואע"ג דהאי תירוצא לאוקומי יש זכייה לעובר אתמר בפ' מי שמת מיהו קושטא הוא ולהכי הוינן מינה ומפרקי לה התם. א"נ למ"ד הכי מקשי מברייתא דקתני מכין אותה כנגד בית הריון ולאו למימרא דהכי הוא בודאי. }
\twocol{והא דתנן \textbf{וההורגו חייב.}  ודוקא בן יום א' אבל עובר לא דלא קרינא ביה נפש אדם וה"נ אמרינן בסנהדרין (עב, ב) האשה שמקשה לילד מביאין סכין ומחתכין אותו אבר אבר יצא ראשו אין נוגעין בו שאין דוחין נפש מפני נפש אלמא מעיקרא ליכא משום הצלת נפש וקרא נמי כתיב דמשלם דמי ולדות.\par  ואיכא דקשיא ליה מההיא דגרסינן בערכין (ז, א) האשה שהיא יושבת על המשבר ומתה בשבת מביאין סכין וקורעין אותה ומוציאין ממנה הולד ואמאי מחללין שבת כיון שאינו קרוי' נפש.\par וליכא למימר התם ביושבת על המשבר דוקא משום דכיון דעקר גופא אחרינא הוא כדאיתמר התם בערכין במקשה לילד לא בעינן יושבת על המשבר ג) ועוד דהכא בן יום א' תנן וקרא דגבי דמי ולדות אפילו ביושבת על המשבר היא ולא אמרינן התם דכילוד הוא אלא גופא אחרינא הוא קאמרינן לומר שממתינן לה עד שתלד ואח"כ ממיתין אותה ולא מיתרבי מגם שניהם דאפילו קודם שתש' על המשבר כלל אי לאו קרא דגם לא הוה קטלינן לולד כדמפור' התם אבל לענין לידה דבר ברור הוא שאינו בכלל נפש אדם עד שנולד כדאמרינן.\par ולאו קושיא היא התם אמרה תורה חלל עליו שבת אחת כדי שיקיים שבתות הרבה והאי דאמרי' במס' שבת (קנא, ב) תינוק בן יום א' מחללין עליו את השבת לאו לאפוקי עובר אלא גוזמא היא כדי לומר דוד מלך ישראל מת אין מחללין עליו. }
\newchap{דף \hebrewnumeral{45}}
\twocol{\textbf{למאי נפקא מינה כגון שבעל בתוך ג' ומצא דם לאחר ג' ולא מצא דם.}  פי' שאלו בעל בתוך ג' ולא מצא דם ולאחר ג' נמי לא מצא ודאי ליכא ספיקא דאחר בא עליה דהא מכיון שלא בא דם בתוך ג' ודאי לא איתצדו הילכך לאחר ג' בשלא מצא היינו טענת בתולים דאיכא למיחש אחר בא עליה ובאשת כהן כדאיתא בכתובת פ"ק (ט, א).\par  והא דאמרינן ללישנא בתרא כגון שבעל בתוך ג' ומצא דם ולאחר ג' ומצא דם לאו דוקא דאפילו לא בעל לאחר שלשה נמי היינו בעיין אלא סירכא דלישנא קמא נקט כשבעל וחזר ובעל כשמצא דם שאלו לא מצא הא אמרן דאחר בא עליה ואסורה. }
\twocol{מדתנן \textbf{בן ט' שנים ויום א' שבא על יבמתו קנאה.}  משמע קנאה לגמרי ליורשה וליטמא לה אלא שאינו נותן גט עד שיגדיל וכן נמי מדהוינן בה ולכשיגדיל בגט סגי לה והתניא עשו ביאת בן ט' כמאמר משמע דמדאורייתא קנאה לגמרי ובגט סגי לה אלא שהם גרעו כת ביאתו ועשאוה כמאמר וכן בדין שהרי ביאתו ביאה לכל דבר ואע"פ שאין לו דעת הא רבי רחמנא ביבמה ביאת שוגג כדמזיד וכבר פירשתי בפ' האשה רבה (יבמות צו, ב). }
\newchap{דף \hebrewnumeral{46}}
\twocol{\textbf{[הא תוך זמן כלפני זמן תיובתא].}  הא דאסיקנא תיובתא למ"ד תוך זמן כלאחר זמן ואתי רב נחמן לאוקמא בתר הכי כתנאי ולא קמא ק"ל דהא לקמן בפר' בא סימן אשכחן בה פלוגתא דר' יהודה ור"ש דר"י סבר תוך זמן כלאחר זמן ור"ש סבר כלפני זמן והיכי אסיקנא בתיובתא ולא אוקמינהו כתנאי.\par  אלא שמצינו כיוצא בה בתלמוד והרי בזו המסכתא עצמה בפ' ג' (דף כ"ה ע"א) אלמא בעור תליא מילתא ל"ש עכור ול"ש צלול ואסיקנא תיובתא והדר אמר ר' נחמן מחלוק' בעכו' והוינן עלה ממתני' ולא אשגח אתיובתא קמייתא ובפ' אין צדין (דף לה) מדברי רבינו נלמד חיה שקנה בפרדס אינה צריבה זימון ואסיקנא בתיובתא והדר מתרצין ברייתא אחריתי כאן בסמוכה לעיר כאן בשאינו סמוכה ומצי לפרוקי תיובתין וכולהו שקלא דבתראי בתר מסקנא הוא והא נמי דכוותייהו. }
\twocol{\textbf{וכי קאמר רבא חזקה למיאון.}  אי קשי' למיאון למה לי חזקה בחששא בעלמא סגי והוה ליה למימר קטנה שהגיע לכלל שנותי' אינה ממאנת שמא הביאה שתי שערות. איכא למימר כי קאמר רבא חזקה לומר שאין ב"ד מטריחין עצמן לבדוק שלא תמאן ואם חששו היינו בודקין. א"נ אע"ג דאמרינן כי קאמר רבא חזקה למיאון לאו דחזקה אצטריכא ליה להכי אלא לומר דלא ממאנה וחזקה נמי היא ומהניא חזקה לנשים בודקת אותן כדלקמן בפ' בא סימן.\par והראשונים שאלו א"כ הא דאמרינן התם בודקין לה לחליצה ולמיאונין היכי משכחת לה ורבי' הגדול השיב נפקא מינה להגיעה לכלל שנותיה וקדש בתוך זמן ולא בעל אתר זמן דהוה דרבנן.\par  ואלמלא שזה דבר ברור יכולין אנו לפטור עצמינו משאלה זו במה שאמרו מקצת המחברים בודקין למיאונין היינו כדאיתמר עלה לאפוקי מדר' יהודה דאמר עד שירבה שחור על הלבן קמ"ל בודקין ומכי אתיא שערות לא ממאנה ולאו למימרא דבדיקה צריך אלא בדיקה זו היינו שערות לומר דמשהביאה אותן בין בבדיקה בין בחזקה אינה ממאנת וקטנה שלא נודע אם הגיעה לכלל שנותיה והביאה סימנין לא מצינו בגמ' דינה מפורש ויש שכוללין אף בזו בכלל בודקין למיאונין ואם הביאה סימנין אינ' ממאנת ואע"פ שלא בעל אלא קוד' זמן ולא תלינן בשומא לקולא וה"נ לשאר הדברים מטילין אותה כחומרא כדין הספקות. }
\twocol{\textbf{והא אין הבעל מפר בקודמין כדר' פנחס וכו'.}  פי' והא נמי על דעת כן נדרה שאם הקפי' הבעל לא יחול נדרה ומיהו משלא בעל משהגדילה אינו מיפר שכיון שלא קנה קנין גמור אינו מפר שמתחלה מתלא תלי נדרה אם נתקיימו קידושיה יפר נדרה שעל דעת כן נדרה ואם לא נתקיימו הקדושין אף הוא אינו מפר שלא נדרה על דעתו שמא סבורה היא לצאת ומיהו האי תירוצא לא אתיא כרב הונא דאמר הקדיש ואכל לוקה שא"כ היאך היא אוכלת תחלה בהפרתו הרי עדיין קידושי' תלויין והגדר גדר גמור.\par ויש שגורסין אלא כדר' פנחסולא וכו'. כלומר לעולם בשלא בעל ולא מפר אלא שעה ראשונה ואעפ"כ אתיא הפרה דידיה ומבטל נדרה דאורייתא שעל דעתו נדרה מכיון שהוא חיי' במזונתיה ועומדת תחתיו ומשמשתו והאי פירושא עיקר ואפילו למאן דלא גריס אלא ה"ג מתפרש דבכמה דוכתא בתלמודא דהדר ביה מתירוציה קמאי ולא אתמר בהו "אלא". }
\newchap{דף \hebrewnumeral{47}}
\twocol{הא דתנן \textbf{איזהו סימניה ר' יוסי הגלילי אומ' משיעלה קמט תחת הדד וכו'.}  פי' רש"י ז"ל דאצמל קאי. וא"כ צריכין אנו לומר דשיעורא דמשתקיף העטרה ושיעלה הקמט תחת הדד חד שיעורא הוא או לומר דתרי תנאי אליבא דר' יוסי ועדיין אין הדברים נראין שנשנו שיעורין במשנה ואחרים בברייתא ולא הוזכרו של זה בזה כלל ועוד בוחל זה שאמרו בידוע שהביא שתי שערות והלא לא פי' אותו כלל לא במשנה ולא בברייתא וכן בגמ' לא הזכירו בפרק מהן. אלא איזה סימניה אבוחל קאי. ופי' מתני' סימן נערות דעליה קאי ברישא דמתני' ובוגרת לא קתני משום דממילא ידוע שאין בין נערות לבגרות אלא ו' חדשים בלבד ומאי שלא פי' במתניתין פי' בברייתא אלו הן סימניה בגרות וכו'. ועלה דהא מתני' קתני באידך פירקין בא סימן התחתון עד שלא בא העליון כלומר העליון השנוי שאלו לדברי רש"י סימן עליון אצמל דסליק מיניה משמע וליתא אלא אבוחל. }
\twocol{הא דתניא \textbf{שנה האמורה בקדשים בבתי ערי חומה וכו'.}  כולן מעת לעת מיום ליום קאמר וקראי דמיום אל יום נסיב בגמרא והני כולהו דמיום אל יום שוין הן אבל מעת לעת ממש בעינן בקדשים כדאמרינן בפ' שני דזבחים (דף כה ע"ב) זאת אומרת שעות פוסלות בקדשים. וכן בבתי ערי חומה בעינן מעת לעת ממש כדאמרי' בפ' בתרא דערכין. אבל שבבן ושבבת דפירקין דיוצא דופן לא בעינן מעת לעת כדאיתמר לעיל (דף מד ע"ב) ערב ראש השנה דג' איכא בנייהו וכן לענין עבד עברי לא שמענו. כך מפורש בתוספות. }
\newchap{דף \hebrewnumeral{48}}
\twocol{הא דתנן \textbf{ר' מאיר אומר לא חולצת ולא מתיבמת.}  ר"מ לטעמיה דאמר (לעיל לב, א) קטן וקטנה לא חולצין ולא מיבמין. והא דקתני סיפא ותכ"א או חולצת או מתיבמת מפני שאמרו לאו דוקא מתיבמת דהא לרבנן קטנה נמי מתייבמת אלא איידי דקתני בדר' מאיר חליצה ויבום וקתני להו בכל דוכתא נקט נמי הכא חולצת או מתייבמת. }
\twocol{גמרא: הא דתניא \textbf{כל הנבדקות נבדקות על פי נשים.}  היה נראה דלת"ק בין להקל בין להחמיר נשים נאמנות בין לפני הפרק בין תוכו בין לאחריו וטעמא דמילתא משום דבמילתא דאיתא קמן ומצינן לגלויה מהימנן ודמיא הא מילתא לההיא דאמרינן בפ' החולץ (דף לט ע"ב) ואשתמודענוהו לקמן דאחוה דמיתנא דמן אבוה הוא וקי"ל אפילו בקרוב ואפילו אשה דגלויי מילתא הוא.\par והוסיף רבינו הגדול ז"ל בפי' דמילתא ואמר טעמא משום דלאו אמילתא דאסורא קא מסהדי ולא אממונא קא מסהדי אלא מילתא הוא דמגלו דהדין הוא גבר פלן והא ניהי איתתיה וכו' כדכתיבא בהלכות והא נמי לההיא דמיא ועדיפא מינה משום דמילתא קמן היא לגלוייה הילכך נשים מהימני בין להקל בין להחמיר. כך נראה פי' דבר זה.\par  והלכה כת"ק משום דהוא סתם ברייתא ור"י ור"ש יחידאי נינהו ועוד מעשה רב וכן היה ר' אליעזר מוסר לאשתו ור' ישמעאל מוסר לאמו. אבל רבינו הגדול ז"'ל פי' טעמא {\small [ברי"ף לפנינו ליתא שם טעמא דת"ק]}  דת"ק משום דקסבר לפני פרק בדקן נשים דאי משתכחן לאחר הפרק שומא נינהו וכדר' יהודה תוך הפרק משום דכלפני הפרק דמי וכדר' שמעון ולאחר הפרק משום דאיכא חזקה דרבא וסמכינן אנשי' וכדר' יהודה ופםק הלכה כן כדאיתא בהלכות בפ' ב"ש ביבמות.\par  ופי' לאחר הפרק בכל השמועה היינו זמן הנעורי'. ותוכו היינו שנת י"ב שהוא עונות הנדרים וכך פי' ר"ש ז"ל וכן היא קבל' רבינו וכל הגאוני' ז"ל ולפי שהזמן הזה שנוי במשנ' שנו בה סתם הפרק כלומ' הפר' שהזכירו א"נ ב"מ ומי שחולק בזה אין שומעין לו. }
\twocol{\textbf{בשלמא לפני הפרק בעיא בדיק' דאי משתכחי לאחר הפרק שומא נינהו.}  שאלמלא שבדקו הנשים בתוך הזמן היינו אומרים לאחר זמן הביאו ולא קודם לכן שאורח בזמנו בא. וה"ה אפי' לתוך זמן שאין חוששין שמא לפני זמן הביא' אותן למ"ד כלאחר זמן דמי וחולצת היא ועכשיו הנשים נאמנות וקטנה היא שלא תחלוץ ואפילו לומר קטנה היא שתמאן אינו נאמנות מ"ט כיון דקידש בתוך זמן ובעל לאחר זמן ה"ל ספיקא דאורייתא אפילו היו שם ק' עדים ששערות הללו שומא הן חוששין שמא הביאה שערות לאתר זמן ונשרו ולעולם אינה ממאנת.\par ואי קשיא היכי ניחא לן השתא בדיקה דלפני הפרק משום דאי משתכחי לאחר הפרק שומא נינהו והא למאי דקס"ד השתא דתזקה דרבא בין למיאון בין לחליצה הוא לאחר הפרק אפילו תאמר שאלו שומא הן מ"מ גדולה היא דחזקה הביאה שתי שערות.\par  לאו קושיא היא ואנן מחזא חזינא בהדיא בבריית דאית ליה בדיק' בלאחר הפרק ולפום הכי קאמרינן בשלמא בדיקה דלפני הפרק לפום מאי דקאמרת בברייתא מהניא ודאי דלא עבדין עובדא בהנך שערות אלא בדיקה דלאחר הפרק גופיה קשיא למה לי ומשום דלא בעינן לאקשויי אדיוקא מאי דקתני בהדיא קאמרינן הכי והשתא לא נחתינן למידק בנשרו כלום דקתני וסתמא פרכינן.\par  ומתרצינן לכ"ע כדמתרץ במסקנא לעיל באידך פירקין ולפרושא לברייתא בעלמא אתינן השתא דאי דייקי בנשירה הוה יכול למימר בדיקה דלאחר הפרק להחזיקה בקטנה כשלא נמצאו בה שערות אלא פירש ברייתא כמסקנא דלעיל הכא ולישנא דקאמר בעיא בדיקה לאו דוקא דאנן לא צרכינן למיבדק קודם זמן שומא הן. אלא מהניא בדיקה דנשים להכי. וכן ברייתא דקתני נשים בודקות לאו דוקא אלא לומר שהן נאמנות אם בדקו וה"נ משמע בכל מקום בתלמוד שאין חוששין לשערות שנמצא לאחר זמן שמא הביאו אותן קודם זמן ושומא הן כדאמרי' בפ' מי שמת (דף קנד) דמעשה דבני ברק בתינוקות וכו' ובאו ושאלי לר"ע מהו שיבדקו וכו'. }
\twocol{\textbf{וסיפא דקתני ונאמנת אשה להחמיר.}  אוקימנא אב"א ר' יהודה ואתוך הפרק. וק"ל בשלמא נאמנת לומר גדולה היא שלא תמאן ואינה נאמנת לומר גדולה היא שתחלוץ ניחא אלא קטנה שלא תחלוץ פשיטא דנאמנות דאפילו שתקא א"נ אמרה גדולה היא אינה חולצת וקטנה היא שתמאן אמאי אינה נאמנת ואפילו נאמנת אמאי צריכה והא אמרת צריכה לומר גדולה היא שלא תמאן.\par  ואיכא למימר כולה ברייתא נאמנת ואינה נאמנ' במקום שהוצרכנו לעדותה היא והכי קתני נאמנת לומר גדולה היא שלא תמאן במקום שאנו צריכין לעדות (גדול) שלה שאלמלא עדות אשה זו ממאנת היא שבחזקת קטנה עומדת ואפילו בדקו עדים עכשיו ולא ראו בה שערו' נאמנות אשה זו לומר הביאה אותן ואינה ממאנת שמא נשרו.\par וכן נאמנ' לומר קטנה היא שלא תחלוץ במקום שאנו צריכין לעדות קטנותה כגון שבדקנו אותה ומצינו בה שערות אם אמרה אשה לפני זמן הביאתן נאמנת והיינו בדיקה דלפני הפרק ואגב אחרינא נקט להא. א"נ שלא תאמר כשהיא עדיין לפני הפרק נאמנת דעדיין בחזקת קטנה היא אבל לאחר שהביאה שערות והיא בזמנה והוחזקה גדולה בפנינו שמא תאמר אינה נאמנת לומר קטנה הוא שתוך זמן היו בה. וקמ"ל.\par אבל אין אשה נאמנת לומר תוך זמן קטנה היא שתמאן אם הוצרכנו לעדות זו כגו' שנמצאו בה שערות והוא שבעל בתוך זמן דה"ל ספק דאורייתא ואע"ג דהכא ליכא למימר שמא נשרו דהא אכתי תוך זמן זה הוא מיהו כיון שהיא גדולה בפנינו אין האשה נאמנת להקל בשל תורה לומר שומא הן וכן אינה נאמנת לומר גדולה היא שתחלוץ.\par  ולהך לישנא דאמרינן ואב"א ר' שמעון ולאחר הפרק ולית ליה חזקה דרבא וה"נ קתני נאמנת לומר גדול' היא שלא תמאן ואפילו אין בה עכשיו שערות וקטנה היא שלא תחלוץ אפילו היו בה אבל אין נאמנת לומר קטנה היא שתמאן כשהיו בה ובעל כדפרישית ולא לומר גדולה היא שתחלוץ.\par  ומצינו נוסחא אחרת. "ונאמנת אשה להחמיר אבל לא להקל כיצד גדולה היא שתמאן גדולה היא שתחלוץ". וכן גרסת רבינו הגדול ז"ל בהלכות. ונוסתא ישרה היא ופירושא נאמנת להחמיר לומר גדולה היא לענין מיאון ואינה נאמנת להקל לומר גדולה היא לענין חליצה כלומר מיאון וחליצה היינו להחמיר ולהקל מיאון היינו להחמיר חליצה היינו להקל. }
\newchap{דף \hebrewnumeral{49}}
\twocol{\textbf{מוציא כשר למי חטאת ופסול משום גסטרא.}  יש מקשים, ואפילו בלא מכניס ומוציא היאך יהיה כלי חרס כשר למי חטאת והלא שנינו במס' פרה כל מעשיה אינן נעשין אלא בכלי אבנים ובכלי גללים ובכלי אדמה. וי"ל ההיא מעלה בעלמא היא משום שמטמאין היו הכהן השורף אותה להוציא מלבן של צדוקים שהיו אומרים במעורבי שמש היא נעשית אבל מכניס פסול דין תורה הוא.\par וי"מ שעד שלא נתנו אפר במים היו מעשיה בכלים הללו שאין מקבלין טומאה אבל משנתנו אפר לתוך המים שוב אינה מקבלת טומא' כדאמרינן בעלמא א) מי תטא' שנגע בהם שרץ טהורין וכיון שכן בכל כלים היו נותנים ואפילו בשל חרם ולכך פסלו מכניס. }
\twocol{הא דאקשי' \textbf{תנינא חדא זימנא הכל כשרין לדון דיני ממונו' וכו'.}  ק"ל דהא כולהו תננהו כל חדא וחדא בדוכתא כדפריש בגמ' (לקמן נ, א) כל שחייב בפאה חייב במעשרות ממתני' דהתם וכן כל שיש לו ביעור יש לו שביעית והכא ודאי אגב גררא דכיוצא בו קתני להו וא"ת לרב יהודה פרכי' הא לאו מילתא היא דהא רב יהודה פרושי קא מפרש להו דהכא והתם חדא קתני ועוד דהא עיקר מתני' מצרכי' חדא לגר וחדא לממזר.\par וי"ל למימרא דרב יהודה ודאי ק"ל למה לי לפרושי תרתי זימנא ומתרצי' דלא איתמר ממזר אלא גר איתמר כך פי' בתוספת והביאו דומה לה משבת פ' ח"ר עקיבא דקאמר וצריכא על מימרא דרב יהודה אמר שמואל דאמר הלכה כר' עקיבא דכל מלאכה שאפשר לעשותה מע"ש אינו דוחה את השבת וכו' ואינו מחוור לי שאם אין המשניות מיותרת מנין לו לרב יהודה לפרושי חדא לגר וחדא לממזר. וי"ל דרב יהודה סברא דלפשיה קאמר דכולהו פסולין שהן ישראל כשרין לדיני ממונות ולא מיפסלי יוחסין אנא לדיני נפשות.\par וה"ג וכן בנוסח' אי' בסנהדרין (לו, ב) וצריכא דכי אשמועינן ממזר משום דאתי מטפה כשרה ואי אשמועינן גר משו' דראוין לבא בקהל וכשירין לדיני ממונו' מצרכינן דרב יהודה אהכל כשרין דרישא קאי. וקאמר לאתויי ממזר דכשר אבל אין הכל כשרין לדון דיני נפשות אלא כהנים וכו' פשיטא לנר ולממזר ואפילו חלל כולן פסולין הן וכדקתני סיפח התם ואין הכל כשרין לדון דיני נפשות אלא כהנים ולוים וישראלי' המשיאין לכהונה וחלל אינו מן המשיאין לכהונה.\par וי"א דאסיפא קיימינן לומר דממזר וגר פסולין לדון נפשות וכן כתוב כאן בנוסחאות וצריכא דאי אשמועינן גר משום דאתי מטפה פסולה וכו' ומפרשי' מדקאמרינן לאתויי גר וממזר ש"מ דפסולי כהונה כגון חלל כשרין לדיני נפשות דלאשמועינן דאפילו גר וממזר כשרין לדיני ממונות לא קאמרינן דהא בגמר' דיני נפשות מצרכינן כדאמרינן דאי אשמועינן גר משום דקאתי מטפה פסולה ולהכי פסול לדיני נפשות אבל ממזר דאתי מטפה כשרה אימא לא וכו' וכיון דלא אשמועינן אלא גר וממזר ש"מ דחלל כשר אפילו לדיני נפשות. והא דקתני בסנהדרין ואין הכל כשרין לדון דיני נפשות אלא כהנים לוים וישראלים המשיאין לכהונה הא פריש רב יהודה דלמעוטי ממזר וגר אתי אבל חלל אינו בכלל.\par ואינו נכון כלל דחלל נמי בהדיא ממעיט מהתם ועוד דתנן אין בודקין מן הסנהדרין ולמעלה ומפקינן לה מדכתיב ונשאו אתך בדומין לך. והם דוחין לזו דהתם בממונין סנהדרין קבועים אבל כשר הוא להושיבו בדיני נפשות ולמנות עמהן אע"פ שאינו משיאו לכהונה.\par וכן ב) פירש"י ז"ל הא שאמרו ביבמות (קב, א) שאם היתה אמו מישראל דן ואפילו ישראל [דהוא דיני נפשות] ורבינו יצחק בעל הלכות ז"ל פי' לזו [דמתני'] בשאמו מישראל במס' סנהדרין. וכבר פרשתיה שם ביבמות (מה, ב). }
\twocol{\textbf{נבלת בהמה טהורה בכל מקום ונבלת העוף הטהור והחלב בכרכים אינם צריכים לא מחשבה ולא הכשר.}  פי' נבלת בהמה ונבלת העוף אינם צריכים לא הכשר מים ולא הכשר שרץ מפני שסופן לטמא טומאה חמורה. והחלב אינו צריך הכשר מים שכבר הוכשר בדם שחיטה אבל הכשר שרץ צריך לצדדין קתני ואין אתה יכול לאומר' בחלב נבלה דא"כ צריך הוא הכשר מים ושרץ שהוא אין סופו לטמא טומאה חמורה דכתיב וחלב נבלה וחלב טרפה יעשה לכל מלאכה וא"א נמי בחלב של טמאה שהיא צריך מחשבה שהרי נבלת טמאה בכל מקום צריכין מחשבה ואין חלבה חלוק מבשרה אלא בחלב של שחיטה הוא ואינו צריך הכשר מים עכשו מפני שכבר הוכשר בדם שחיטה אבל הכשר שרץ צריך ולצדדין קתני הכשר כדפרישי'.\par  ובסיפא גרס' נבלת בהמה טמאה בכל מקום ונבלת עוף טהור בכפרים צריכה מחשבה ואין צריכין הכשר ולא גרם בה חלב כיון דחלב בכפרים צריך מחשבה הכשר נמי צריך שלא הוכשר בשחיטה מפני שקדם הכשר למחשב' והבשר קודם למחשבה לא הוי הבשר כדאיתא בהעור והרוטב. ובנוסח המשניות נמי אין בהם חלב בסיפא. }
\twocol{\textbf{והאי שבת מדחייבא בפאה מחייבה במעשר.}  איכא למידק למה ליה לאתויי מכללא משנה שלימה היא שהשבת חייבת במעשרות דתנן במס' מעשרות ומייתינן לה בגמ' בפ"ק דע"א (ז, ב) ר"א אומר השבת מתעשרת זרע וירק וזירין וחכ"א אינו מתעשר ירק וזרע אלא השתלים והגרגיר בלבד הא הכל מודים שהוא מתעשר.\par ואיכא למימר אי מהתם ה"א דה"ק אינו מתעשר זרע וירק אלא השתלים והגרגיר אבל השבת אע"פ שהזרע והירק ממנו שוין אינו בכלל לפי שהוא פטור לגמרי. א"ג איידי דבעי מכללא ומדאיחייבה במעשר מטמא טומאת אוכלין דייק נמי מדמחייבא בפאה מחייב' במעשר וה"ה ודאי דמצי למיפשט תרווייהו בהדיא מההיא דמייתי בשלהי שמעתין השבת משנתנה טעם בקדרה וכ' אלא יגדיל תורה ויאדיר מתרץ לה ומסייעא ליה מההיא והכי אורחא דתלמוד' בכמה דוכתי'. }
\twocol{ הא דאמרי' \textbf{לבני מערבא דמברכין בתר דסליקו תפילייהו לשמור חקיו.}  פי' ר"ת ז"ל בספר הישר שלו שלא אמרו אלא בתפילין אבל בציצית ושאר מצות אין מברכין לאחר עשייתן.\par  והביא ראיה ממה שאמרו בירושלמי בפ' היה קורא בתורה כיצד הוא מברך עליהן ר' זירקן בשם ר' יעקב בר אידי כשהוא נותן של יד מהו אומר בא"י אמ"ה על מצות תפילין וכשהוא נותן לראש מהו אומר אקב"ו על הנחת תפילין. וכשהוא חולצן מהו אומר ברוך וכו' לשמור חקיו ואתיא כמ"ד בחוקת תפילין הכתוב מדבר ברם כמ"ד בחוקת הפסח הכתוב מדבר לא כר"א [{\small לפנינו שם }  לא בדא {\small ואם הגירסא נכונה }  כר"א קאי על למטה ע"ש] והטעם לזה מפני שמניח תפילין לאחר שקיעת התמה עובר בעשה הילכך מברך בשעת סילוקן בלילה שהוא מקיים עשה ואין לך כן בכל המצוות, כך פי' חכמי הצרפתים בשמו ז"ל.\par ועדיין אינו מחוור, דא"כ הא דאמרינן בשמעתין לאתויי מצות ומקשי ולבני מערבא דמברכין בתר דמסלקי תפילייהו מאי איכא למימר מאי קושי' מתני' לאתויי כל שאר המצות. ועוד יש נסחאות שכחוב בהן ולבני מערבא דמברכי אמצות וכו'.\par  אלא נראה לבני מערבא ה"ה לכל מצות שטעונו' ברכה לאחריהן וז"ש בירושלמי אתיא כמ"ד בחוקי התפילין לא הקפידו אלא על הלשון דלשמור חקיו אבל שאר כל המצות אין מברכין אלא לשמור מצותיו ובודאי נראה לומר שאין בני מערבא מברכין אלא כשהן מסלקין אותן בזמן ערבית ולא משום עשה שבהן אלא משום שכבר נגמרה מצותן דקסברי לילה לאו זמן תפילין הוא וא"כ סלקו אותן בע"ש ובערבי י"ט לד"ה מברכין היו אבל אם היו מסלקין ביום היאך יברך הלא מצוה להניחן ולא לסלקן. ולפיכך אמרו בירושלמי דאתיא כמ"ד בחוקת תפילין הכתוב מדבר וכתיב מימים ימימה ולא לילות וכך סמכו שם בירושלמי ר' אבהו בשם ר' אלעזר הנותן תפילין בלילה עובר\par בעשה מה טעם ושמרת וכו'. אבל נאמר לפי"ז הענין ולפ"ז הפי' שאין מברכין על כל מצוה שאין סילוקה גמר עשייתה כגון פושט ציצית ביום והיוצא מן הסוכה אבל בלילה מברכין על ציצית וכן לאחר שופר ולולב וכל כיוצא בהן שעשייתן גמר מלאכתן מברכין וזה שלא העמידו משנתינו דיש טעון במצות כיוצא באלו שאינן טעונות ברכה מפני שאין לשין לאחריו אלא לאחר שנגמר המעשה.\par וזה הלשון נכון הוא שאין הדין נותן לברך לאחרי' במצוה שעדיין הוא חייב בה והוא מסלקה ממנו שא"כ מצינו חוטא ומברך ואין לך כן אלא בקורא בתורה ובצבור מפני שהוא מצוה לגמור כדי שיהיו ג' או ז' קוראים כתקנת חכמים. אבל בגמר מצוה בכל מצוה נגמרת מברכין היו ודמיא להו להלל ומגלה ותורה בצבור וראינו לרבינו האי גאון ז"ל שכתב בהא דבני מערבא לא נהגינן הכי במתיבתא ומיהו אי בעי אינש למיעבד כבני מערבא שפיר דמי.\par ולשון הירושלמי שכתבנו נראה שמכריע כדברי בעל הלכות ז"ל שהצריך לברך א' של יד וא' של ראש אף על פי שלא שח. וכן החזירו שם הענין הזה בפרק הרואה ואמרו העושה תפילין לעצמו אומר בא"י אמ"ה לעשות תפילין לשמו כשהוא לובשן אומר בא"י אמ"ה על מצות תפילין וכשהוא מניחן אומר אקב"ו על הנחת תפילין בכל מקום מזכירין כן אע"פ שלא שח ולא כדברי רבי' הגדול ז"ל שפירש לא שח מברך א' בלבד על שתיהן.\par  אלא שיש לנו פתחון פה לומר דגמרא ירושלמי ס"ל כדקס"ד מעיקרא בגמרא דילן אבל במסקנא אסיקו אביי ורבא לא שח מברך א' ואנן כמסקנא דגמ' דילן עבדינן או שענין הירושלמי במניח א' מהן ולא במניח שתיהן. }
\newchap{דף \hebrewnumeral{52}}
\twocol{הא דאמרינן \textbf{ומודה ר' יהודה שאם נבעלה לאחר שהביאה שתי שערות שוב אינה יכולה למאן.}  משום דודאי מודה ר' יהודה שהיא גדולה משהביאה שתי שערות והילכך אמרינן אדם יודע שאין קידושי קטנה כלום וגמר ובעל לשם קדושין ומאן דבעי למעבד עובדא כר' יהודה ואע"ג דנבעלה ליכא לפרושי משום דקסבר לר' יהודה קטנה היא ואין קידושין תופסין בה עד שירבה השחור דהא קתני בהדיא תינוקת שהביאה שתי שערות חייבת בכל מצות האמורות בתורה ולא פליג ר' יהודה. וכן בתינוק לא מצינו לר' יהודה מחלוקת בגדלות שלו אלא הכל מודים דשתי שערות גומרות בו ועוד דא"כ לר' יוסי דאמר במיאון עד שתקיף העטרה והוא זמן גדלות לדבריו כדאית' בפירקין דיוצא דופן א"כ עקרת זמן נערות מכל התורה כולה.\par  אלא שהכל מודים דזמן גדלות היינו שתי שערות וטעמיה דרבי יהודה משום דיהיב ליה זמן מעתה משהיא בת דעת דהא קידושיה דרבנן הן דלא ס"ל כמ"ד כי גדלי גדלה קדושי בהדה ומשרבה השחור אע"פ שמדאורייתא אינה מקודשת תקינו לה רבנן נשואין ומאן דקסבר אפילו נבעלה סבר לא אמרינן אדם יודע שאין קדושי קטנה כלום אלא אמרינן לשום קדושין ראשונים בעל. והילכך אפילו גדולה ממש מן הדין אינה מקודשת אלא שתקנו חכמים זמן למיאוניה עד שירבה השחור תוך הזמן הזה אפי' בעל ממאנת לאחר הזמן הזה אפילו לא בעל אינה ממאנת. וזה הלשון נכון ועיקר הוא. }
\twocol{והא ד\textbf{אמר ר' ישמעאל ויש לך אחרת שאפילו לא נתפסה מותרת ואיזו זו שקידושי' קידושי טעות.}  פירש רש"י ז"ל כגון ע"מ שאני כהן והרי הוא ישראל וכגון קטנה שאין מעשיה כלום והכי ודאי פשטה דשמעתא דאמרי' דבר שאמר אותו צדיק יכשל בו זרעו. וא"כ לר' ישמעאל מצינו חמות ממאנת ושמואל אי סבר ליה כרביה ההוא דאיתמר בכתובות (דף ע"ג) קטנה שלא מיאנה והגדילה ועמדה ונשאת רב אמר אינה צריכה גט משני ושמואל אמר צריכה גט משני ולא מראשון קאמר.\par  וההיא דאמרינן בפרק מי שמת (דף קנ"ו ע"א) אמר ר' נחמן אמר שמואל בודקין לקידושין ולגיטין ולחליצ' ולמיאונין ואיתמר עלה למיאוניו לאפוקי מדר' יהודה וכו' דאלמא משהביאה שתי שערות אינה ממאנת ההיא דלא כר' ישמעאל. אי משום דהא דידיה והא דרביה. אי משום דאמוראי נינהו אליבא דשמואל.\par וי"מ דוקא קידושי טעות אבל בממאנת בקדושי קטנות לא א"ר ישמעאל ופי' ממאנת והולכת לה לומר שאם לא רצתה בבעל הולכת לה ולא שתהא צריכה גט מיאון אלא מעשה כשהיה בבתו שנכנסה למאן היו סבורין לומר כשם שלר' ישמעאל ממאנת בקדושי תנאי ולא אמרינן אין אדם עושה בעילתו זנות ולשום קדושין בעל כך בקטנה שהגדילה ונמנו וגמרו שאפילו לר' ישמעאל בקדושי קטנות אינה ממאנת אלא עד שתביא שתי שערות ומיהו דרבנן הוא מפני שנראית כגדולה שנתקדשה או מפני שלא מיחת בקידושי דרבנן שעה ראשונה שוב אינה יכולה למחות מדבריהם. וכן משמע בפרק נושאין על האנוסה (יבמות ק, ב) כלשון הזה וכבר פירשתיה ביבמות בפרק ב"ש. }
\twocol{אע"ג דקיימא לן כרבנן \textbf{עד שיהו שתי שערות במקום אחד.}  מיהו שתים על גבי קשרי אצבעותי' של יד ושתים ע"ג קשרי אצבעותיה של רגל גדולה היא דלא פליגי רבנן עליה דר"ש בהאי.\par ותמהני על הרב רמב"ם פאסי ז"ל שכתב ב' שערות אלו צריכים שיהיו במקום הערוה ובשמעתין משמע אפילו על יד ורגל או בגבה. וי"מ גבה וכריסה במקום ערוה וכריסה למעלה עד מקום ערוה וכן שמעתי בשם ר"ת מיהו ביד ורגל סגי. }
\twocol{ הא דאמרינן \textbf{מ"ד כל כה"ג מביאה קרבן ונאכל קמ"ל.}  נראה לי שאין לפרש "קמ"ל" דאינו נאכל אבל מביאה קרבן כמו שכתבו רבים. דהא כתמים דרבנן הם ואפילו בידוע שמגופ' חזאי דבר תורה טהורה ואינן מביאין לא לידי זיבה ולא לידי נדה כדאית' לקמן בפרק הרואה. אלא ע"כ נפרש קמשמע לן דאינה מביאה קרבן. ולא נאכל שני ימים וחלוק דומיא דג' חלוקין.\par  ויש לדחוק שכיון שראתה שנים וצריכה שימור בשלישיאף על פי שאין דין הכתם לטמא בתחלה כיון שדבר ברור הוא דמגופה חזאי בדין הוא שתעשה זבה גמור' שהרי לא עלתה לה שימור. ואין זה נכון. }
\twocol{\textbf{מסמאה עצמה וקדשים למפרע.}  פרש"י ז"ל עצמה לטהרות ובודאי שיש בכלל עצמה קדשים דאיהו נמי בכלל טהרות הן אלא משום פלוגתא דרשב"א נקט הכי.\par  ואי קשי' לרשב"א הא מצינו כתמה תמור מראיתה דאלו כתמה מקולקלת למנינה ומטמא בועלה ואלו ראיתה אינה מקלקל' מנינה ואינה מטמא את בועלה כדאיתא בפרק קמא.\par ויש לומר דלרשב"א ל"ק ליה אלא שלא נאריך זמן טומאה בכתם למפרע יותר מזמן טומא' דראיה לפי שאינו בדין אבל אם נחמיר במנין דכתם יותר במנין דמעת לעת דראיה לענין קלקול ל"ק ליה לפי שזמן ראיה כיון דשעת ראיה דאוריית' מתחילין ממנו שאין זמן ספק מוציא מידי זמן ודאי אבל בכתם אף שעת מציאה ספק לפיכך לא התחילו למנות ממנו וכיון שאף בזמן דמע' לעת יש קלקול במנינה גבי כתם האריכו לענין מניין מקולקל' משעת הכבוס למניינה. ואע"פ שאינה מטמא טהרות ולא קדשים אלא עד מעת לעת של מציאה. כנ"ל. והא דנקט עצמה משמע לי מפני שמטמאה נמי את בועלה נקט הכי ולא קתני בהדיא טהרו' וקדשים.\par ובתוספ' מפרשי' היינו למניין שלה לומר שהיא מקלקל' למניינה. ורשב"א עצמה אינה מטמאה למפרע ומודה במעל"ע. ואין פירושם נכון לפי כוונתם שהרי בזמן מעל"ע מיהת חמור כתמה מראי' שהרי מודה רשב"א שמקולקל' למעלה בכתמים כדתניא בסמוך ואלו בראיה אינה מקולקל' כלל ' אלא טעמא כדפרישית. }
\twocol{\textbf{שהוא מתקן הלכותיה לידי זיבה.}  פ' רש"י ז"ל לענין זיבה הוא מיקל לדידיה היכא דלא חזאי ביום לא תלינן כתמה בראיתה ומונה ימי נדה מיום ראיתה ואין ימי זיבה מתחילין עד יום ח' לראיתה ולרבי מונה מיום מציאת כתמה ואף להקל ולטבול לילי ז' לכתמה אם פסקה ומיום ח' לכתמה אמרינן יום זוב הוא ונמצא רבי מחמיר לענין זיבה דכי חזיא בח' לכתמה אמרינן יום זיבה הוא וצריכה לשמור יום כנגד יום ולרשב"א סוף נדה הוא ואין צריך שימור ולא נראה דהא רשב"א כיון דלא תלינן כתמה בראיתה מקולקל' היא לכתמה אמרינן.\par  ול"א פי' בה שהוא מתקן הלכותיה לידי זיבה שהוא מחמיר וחושש לכתם משום זוב בג' גריסין ועוד אי נמי שאם ראתה שנים והוא צריכה שימור. וכן בכל שלשה רצופים שתראה חוששת לזיבה וצריכה נקיים נמצא שהוא מתקנה ומוציאה מכל ספק זיבה ואני מעותה שאיני מוציאה מידי ספק כלומר נראין ומטין כדברי המחמיר.\par ואף לשון זה אינו עולה דלמה לידי זיבה לכל דבר הוא מחמי' שהרי רבי מטהר' ליום ששי לראי' ולרשב"א ליום שביעי. ועוד ק"ל כיון דקי"ל (נט, א) כתמים דרבנן ובראית כתמה אינה מטמאה היאך רבי מונה לה משעת כתמה והלא ביום ראיתה היא תחלת נדה וממנו ראוי למנו' דבר תורה. וכדאמרינן בפרק קמא (ו, א) ברואה כתם ומקולקל' למנינה ואינה מונה אלא משעת שראתה.\par  לפיכך נ"ל שלא תלה רבי אלא כתמה בראיתה אבל ראיתה בכתמה לא, כתמה בראיתה לומר שאינה מטמאה עצמה וקדשים למפרע ואינה מקלקלת למנינה מיום לבישת החלוק אבל מכל מקום עיקר מנין נדה וזיבה מיום ראיה בדין תורה וחוששת נמי ליום. מציאת כתמה כדין דבריהם.\par  לפיכך אמרו שהוא מתקן הלכותיה לידי זיבה כלומר שאינה תולה כתם בראיה אלא במקום שאין חילוק ספיר' זיבה ביניהם כך דהיינו אותו יום שמנין ימי נדה וזבה אחד הוא בין לכתם בין לראיה ונמצאו כל הספירו' ראויו' כדין תורה משעת ראיה וכשהוא מעת לעת הוא רואה אינו תולה ונמצא' מקולקל' לכתם ומונה משעת ראיה נמצא כשהוא אומר תולה מתוקנת לגמרי. וכשהוא אומר אינו תולה היא מקולקל' לגמרי.\par  אבל רבי אפי' בשעה שהוא תולה כתמה בראיתה הוא מעותה לידי זיבה שהרי אסורה לשמש עד יום ז' לראיה שהוא ח' לכתמה. ואם ראתה בו ביום חוששת לזיבה בודאי נמצא לרבי שאפילו בשעת תקונה כלומר שהוא תולה הוא מעותה שתולה כתמה בראיתה ואינו תולה ראיתה בכתמה ולא השוה מדותיו כנ"ל וסליק שפיר. }
\newchap{דף \hebrewnumeral{54}}
\twocol{הא דאמרינן \textbf{ותשמש נמי בתשעה עשר.}  ק"ל והיכי ס"ד דעשירי לא בעי שימור והאמר רבי יוחנן בפרק תינוקו' עשירי כתשיעי ובעי שימור. ועוד רב ששת דמתרץ לה משום גרגרן שביק ר' יוחנן ואמר כריש לקיש והא קיימא לן הלכה כרבי יוחנן בר מהלי תלת ועוד אמאי נטר לה לסיפא והא מרישא ש"מ דיום א' טמא ויום א' טהור ביום העשירי ראתה ויום אחד עשר היא שומר' כנגדו כדקתני שאינה משמשת אלא שמיני ולילו וד' לילות מתוך שמונה עשר יום אלמא רואה בשבעה עשר ושומר' בשמונה עשר.\par וי"ל קסבר מקשה דעשירי יש לו שימור בתוך ימי זיבה דהיינו ביום אחד עשר והיינו רישא אבל זו כיון דיש בה עשירי וי"א וראתה בהן אין ספירה לעשיר' מעתה שאין ספירה אלא בימי זיבה וסוף [י"ח] ימי נדה היא [ולא בעי ספירה] דומיא ספיר' ר'א והיינו נמי דרב ששת. ורב אשי מתרץ שלא בטלה שמירת י"א את של עשירי אלא נהי די"א לא בעי שימור עשירי מיהת בעי לעולם.\par  ואי קשיא ותיקשי ברייתא לר"ל דאמר אף לעשירי לא בעי שימור. ויש לומר סיפא מתרץ לה משום גרגרן אםור כר"ש וקסבר דרואה בעשירי ומשמש' לי"א כל שכן דהוי גרגרן ואע"ג דלא תנן.\par ויש מפרשים דלא אמר ר"ל עשירי לא בעי שימור אלא למ"ד הלכות י"א דהיינו ר' אלעזר בן עזריה אבל לר"ע דאמר קראי נינהו עשירי נמי בעי שימור חוץ מי"א דכיון שאין שימור שלו בימי זיבה אין לו שימור וכדאוקמא לפלוותא דר"ל ור"י בפרק בתרא, ואין זה מחוור. }
\twocol{מדאמרינן \textbf{משמש' רביע ימיה מתוך שמונה ועשרים יום.}  דלפי זה פתחה של זו מכ"ח לכ"ח. שמע מינה שאין האשה נעשי' תחלת נדה משנעשי' זבה גדולה עד שתספור נקיים שלה שהרי בשבוע שלישי שהוא טמא כל שבעה אין בו מימי זיבה אלא ארבע ימים הראשונים ואם תאמר בג' האחרונים נעשי' תחלת נדה נמצאת בשבוע חמישי' שהוא טמא זבה גדולה וצריכה שבעה נקיים ואם כן היאך פתחה של זו בתחלת כ"ח והרי כל אותו שבוע ה' בימי זיבה הוא ונעשת בו זבה גדולה ובשבוע ו' משמרת נקיים ובשבעי שהוא טמא ששה ימים שבו ימי זיבה הן. ואינה משמש' בשמיני נמצאת שלא שמשה בכ"ח שניים כלום.\par וכן נמי מדקתני סיפא משמשת חמשה עשר יום מתוך מ"ח ש"מ כה"ג דאי לא תימא הכי הרי שמנה חמישי' ארבעה האחרונים מתתל' ימי נדה נמצאו ימי זיבתן כלין בשבעה של שמוגה ימים השביעיים ואנו אומרים תחל' נדתה של זו שחוזרת חלילה.\par  וכן סיפא דקתני וכן למאה וכן לאלף כלומר דמאה טהורין שבעה הראשונים תחלת נדה והשאר כולן ימי זיבה הן ומאה טהורין ז' לספירה וכולן לתשמיש והיינו ימי שמושה כימי זיבתה אלמא כולן ימי זיבה הן שמשנעש' זבה גדולה עד שתספור שבעה נקיים איו ראייתה אלא סתירה לספירתה ואינה מונה מהם ימי נדה.\par וז"ש בפרק בנות כותיים מה ימי נדתה אין ראויין לזיבה ואין ספירת שבעה עולה בהן כלומר לפי שא"א ושם אמרו וכי דנין אפשר משא"א לומר שא"א לספירת זיבה בימי נדה למ"ש וכן פי' שם רש"י ז"ל. }
\twocol{והא דאקשינן \textbf{הני ארביס' הוו.}  מפורש בדברי הר"ר אב"ד ז"ל דהכי מקשה בשמנה ימים הרביעיי' למה תשמש שבעה והלא צריכה היא לשמור יום א, לספיר' עשירי ואחד עשר של ימי זיבה שראתה בהן בשמונה השלישיים וכדאמרן ברישא דהיינו שימור בעו ופריק רב אדא זאת אומרת ימי נדה שאינה רואה בהן עולה לה לימי זיבתה כלומר של זוב קטן. ולפיכך יום א' של ח' רביעיים שהשלימה בהן ימי נדתה עולה לה לספיר' שמיר' של יום עשירי שאמרנו. ואין דברי רש"י ז"ל נוחין בזה.\par  אבל דבר שהכל מודים בו שאין ימי נדה מתחילין עד שתספור נקיים.\par  ובואו ונצווח על הרמב"ם פאסי ז"ל שכתב בחבורו שהאשה שראתה תחלה מונה שבעה לנידתה וסמוך להן אחד עשר ואח"כ מונה ז' לנדות אעפ"י שאינה רואה בהן ואחריהן אחד עשר ואם ראתה בהן הרי היא זבה וכן כל ימיה ואם קבעה לה וסת תחל' הוס' הוא יום נדו' וממנו מונה שמונה עשר ומונה שבעה לנדותה אף על פי שלא ראתה ואם ראתה אחריהן באחד עשר זבה היא.\par עוד שבש וכתב שאפילו ראתה ט' וי' ואחד עשר ושנים עשר הרי זו זבה ותחלת נדה וכל אלו דברי הבאי שלדבריו לא תמצא לרואה שבעה טמאים ושבעה טהורים שתשמש אלא שבוע שני ולסוף תשעה שבועות משמשת ששה ימים בשבוע העשירי וחמשה ימים בשבוע שנים עשר ופתחה של זו לסוף אחד עשר שבועות ובגמרא אמרו רביע ימיה ולא קיים אלא בתוך כ"ח הא'.\par  וכן לדבריו בשמונה ימים טמאים ושמונה טהורים אינה משמשת תמשה עשר יום אלא מתוך שמונה וארבעים ראשונים אבל בשמונה וארבעים שניים אינה משמשת אלא שלושה ימים וכיון שלא אמרו משמשת ארבע עשר יום מתוך שנים ושלשים או משמשת שמונה עשר מתוך ל"ו וכן כיוצא במנינן הללו ש"מ שפתחה של זו מ"ח ומכאן ואילך חוזרת חלילה.\par וכן האשה שראתה עשרה ימים טמאים ועשרה ימים טהורים אין זיבתה ושימושה שוים אלא פעם אחת בלבד לפי דברי הרב ז"ל שהרי כשהיא חוזרת ורואה כן בשניה בשמונה ימים טהורים נשלמו ימי זיבה ראשונה והתחילו ימי נדה נמצא שבעשרה ימים טמאים השניים חמשה ימים מימי זיבה ואין ימי חמישה בטהורים אלא שלשה וכן למאה וכן לאלף למה מנה חכמים שבעה לנדה והשאר לזיבות והלא נעשה , היא נדה אף על פי שלא ספרה לזיבה. ועוד לדבריו מצינו אשה רואה יום אחד מסוף ימי נדה יושבת עליו ששה ימים מימי הזיבה ואין לנדה ספירה אלא בימיה.\par וכן שנויה בכמה מקומות במסכתא זו שהרואה יום מ"א לזכר ופ"א לנקבה הרי היא תחלת נדה ואין מונין לימים שמקודם לכן והטעם לפי שכבר נשלם המניין.\par והרב ז"ל הורה ביולדת שמפסק' ומתחלת למנות מתחל' ראיה שלאחר מלאת ולדבריו צריך הוא להביא ראיה מן התורה לשנוי זה שהוא משנה היולדת משאר נשים שאפילו כשאינן רואות הן מונות ימי נדה וזיבה כאלו הן רואות.\par ועוד דהא בפ' בנות כותיים אמרי' דלכולי עלמא נדה ופתחה מכ"ז מנינן ואם היינו מונין משעת ראיה ראשונה כ"ז בימי זיבה קאי לה.\par  ועוד מהא דתנן היתה למודה לראות יום ט"ו ואוקמה שמואל ט"ו לטבילתה שהן כ"ב לראייתה וכו' ואם אתה מונה כל ימי נדת זובם לתחלת ראיה ראשונה שראתה זו כי הדרי אותו כ"ב תליתאי בימי זיבה קיימי והיאך קבעה וסת בכך שאין האשה קובעת וסת בי"א כדאיתה התם בשלהי בנות כותיים ואין הוסת נקבע אלא בשלשה הפלגו' כדבעינן לפרושי קמן וכל שכן לרב הונא בריה דר' יהושע דקשיא דאמר אינה חוששת בתוך אחד עשר וכל זה במס' זו.\par ותמהיני עליו אם העביר עיניו בפתחי נדה במס' ערכין דתנא רבנן טועה שאמרה יום אחד טמא ראיתי פתחה שבעה עשר פירש שאפילו היו תחלת ימי נדות הרי השלימה עליו ששה ועוד י"א אחריהן נמצאת חוזרת לתחלת נדה וכל שכן אם היה בימי זיבה שכבר עברו ימי זיבתה וימים שהיתה ראוייה להיות נדה ואלו לדברי הרב ז"ל א"א דהא איכא למימר שאותו יום בתוך אחד עשר היום וכשעמדה אחריו שבעה עשר נמצא עומדת בימי הזיבה למנין הראוי וכן כל השמועה ומדאמרינן התם נמי חמשה וארבעים ימים טמאים ראיתי וכן כולם אשתמע בהדיא דמשעה שנעשית זבה גדולה אינה נעשית נדה לעולם עד שתספור שבעה נקיים שלה. ואין לי להאריך.\par  וכן יש שבושין בחבורי הראשונים בקצתם כגון רב סעדיה שכתב שכל אחד עשר יום שבין נדה לנדה בשלשה ראיית בשלשה ימים נעשית זבה גדולה בין ברצופין בין במפוזרין. וזה טעות מתפרש כאן ובכמה מקומות דרצופין בעינן ולא מפוזרין ועל כיוצא בדברים הללו ידוו כל הימים שהתורה משתכחת מלומדיה ואין אדם מוציא הלכה ברורה במקום אחד. }
\newchap{דף \hebrewnumeral{55}}
\twocol{\textbf{אבן מושמא.}  כבר פירש במסכת שבת פרק ר' עקיבא (דף פ"ב ע"ב) בתוספות בשמו של רבינו תם ז"ל שהיא אבן גדולה שמושמת על משכב הזב ומושבו ואדם טהור יושב על האבן ונטמא מדין נישא שהרי משכב הזב נושא אותו וזהו היסטו של זב שלא מצינו לו חבר בכל התורה כולה שכל טומאות המסיטות טהורות חוץ מן הזב וריבה הכתוב אף משכב שלו ומושבו הנושאים תחת האבן כדאיתא בתורת כהנים וכן אם היו בכף מאזנים וכרעו הן מטמאים בהיסט.\par  וכן דם נדה שהוא רוצה כאן לטמא תחת אבן מסמא לומר שיהא לו היסט כזב בין תחת האבן בין בכף מאזנים וממעטינן ליה מדכתיב והנושא אותם דכמשכב ומושב הוא דמטמא באבן מסמא להכלים שעליה כדמפרש בתורת כהנים וכתב אותם למעוטי דמה. ולשון רש"י הוא והלשון שפי' בתוס' כתבתי שם במסכת שבת. }
\twocol{\textbf{שמא יעשה עור אביו ואמו שטיחין.}  מפורש במסכת חולין בפרק העור והרוטב (קכב, א). }
\twocol{\textbf{אמר ר' יהודה מדסקרתא סלקא דעתך אמינא שעיר המשתלח יוכיח וכו'.}  תימא הוא למה חזר והזכיר הטעם שדוחה סברייתא קל וחומר שלו ולמה הוצרך לומר כן לפי שאלתנו זובו טמא למה לי.\par  ויש לומר שזו הברייתא השגויה למעלה שעיר המשתלח יוכיח לא היתה שנויה בבה"מ ורבא לא היה יודע אותה כמ"ש בפרק בנות כותיים וכן ר' יהודה מדסקרתא לא שמע אותה והשיב לתרץ דאיצטרך זובו טמא והיה קשה עליו בק"ו והוצרך לומר שמדין ק"ו נמי לא אתי בך מפרש בתוספת. }
\newchap{דף \hebrewnumeral{57}}
\twocol{הא דאקשינן\textbf{ודילמא כהן טמא הוא.}  לאו למימרא דכהן טמא מותר ליטמא שלא אמרו אלא בחבור אדם במת כדאמרינן במסכת נזיר היה עומד בבית הקברות והושיטו לו מתו ומת אחר ונגע בו יכול יהא חייב תלמוד לומר לא יחלל יצא זה שמחולל ועומד ומוקים לה התם בחבורי אדם במת הא לאו הכי חייב אלא הכי קאמרינן ודילמא כהן טמא וקסברי כותיים שאין טמא מוזהר על הטומאה אי נמי כהן טמא דיומי הוא וקסברי כרבי עקיבא דאמר במסכת שמחות נטמא בו ביום ר' טרפון מחייב ור' עקיבא פוטר. וכבר פירשנו בארוכה בפרק ידיעת הטומאה (שבועות יז, א).\par ובענין הכותיים בטומאת כתמים אף על פי שכתמים מדבריהם הוחזקו בהן שהן סבורין שטומאתן תורה דאינהו לא דרשי בבשרה עד שתרגיש. אי נמי בכתמים שמרגשת בהן הן נזהרו' ושל ספקות לא חששו להם. וכן בנפלים זהירין הן בטומאה כטומאתן אף על פי שאין קוברין אותן לדבריו דר' יהודה שאם לא היו זהירין בהם נמצא כולן בחזקת טמא מתים ואנן לא תנינן אלא בועלי נדות וכשטבל לאותה טומאה טהור כדאיתא בפרק בנות כותיים. }
\twocol{\textbf{הסככות.}  פירש רש"י ז"ל אילן המיסך על הארץ והוא סמוך לדרך בית הקברות וזימנין דמיתרמי בין השמשות וקברי התם והיינו ספיקייהו. }
\twocol{\textbf{הפרעות.}  אבנים גדולות ובולטות מן הגדר וקבר תחת אחת מהן ואינן יודעין תחת אזו מהן.\par ותימה הוא, אם כן ספק טומאה הוא, ואם הוא ברשות היחיד ספקו טמא ואם היה ברשות הרבים וגזרו עליהן מפני שהוחזקה שם טומאה למה אמרו בכותי מהלך על פני כולה דנאמן שמא תולה הוא בספק טומאה ברשות הרבים דספיקו טהור וגזרו דרבנן לית להו עוד השיב הרב ר' אברהם בר דוד ז"ל דאם כן כל זיזין וגזוטראות נמי ומאי שנא אבנים דנקט.\par ופירש הרב ז"ל שהסככות אילן שענפיו אחת למעלה ואחת למטה ואין שם אוהל אלא שרואין את העליונה כאלו הן למטה והתחתונות כאלו הן למעלה ואף על פי שאין העליונות כדין התחתונות אלא שעדין נשאר שם אויר מועט נעשה כולו (אויר) [אהל] שלם וכן הפרעות אבנים שיוצאות מן הגדר ואינן נוגעות זו בזו אלא שראויו' לקבל מעזיבה נעשה אהל שלם ומביא טומאה מדבריהם. ועשאום כספק טומאה וזהו ספקן שאין אהל שלהם שלם ובשקברו שם בודאי מיירי.\par  ואף על גב דאמרינן בסוכה שאין אומרים גוד אחית וגוד אסיק אלא בתוך שלשה משום לבוד וגבי קורות הבית תנן שאפילו אין ביניהם טפח טומאה תחתיהן ביניהם טהור אלמא לא כסתום דמי. שאני הבא דכיון שהכל מאילן אחד ומכותל אחד הוי חבור ומשלים אהל שלהן. כך כתב הרב הנז' ז"ל וכענין הזה שנוי בתוספות טהרות ומיהו דוקא שאין בין הענפים של סככות פותח טפח שאם היה ביניהם בודאי פותח טפח מפסיקין. }
\twocol{\textbf{אמר ר' יוחנן במהלך ובא על פני כולה.}  ה"נ איכא למיחש דלמא טמא הוא אלא באוכל (טומאה) [תרומה] הוא דמתוקמא דומיא דרישא דמתניתין ומשום דרישא מקצר ועולה.\par  ותמה הרב רבי שמואל ז"ל אלא מתניתין דקתני נאמנין לומר קברנו שם את הנפלים ואינן נאמנין על הסככות הא ודאי כשם שנאמנין על הנפלים בכהן שלהם אוכל תרומה שם כך נמי נאמנים על הסככות במהלך שם ואוכל.\par  וזו אינה קושיא שהמהלך ובא על פני כולה היכי שעובר בכל השדה שאפילו לא היה הכותי חושש לאהל הסככות נטמא בקבר עצמו אם היה שם ונאמן על גופו של קבר הא על טומאות הסככות כגון שמיסך תחת אחד מן האילנות אינו נאמן עליו דלית להו דין אהל בסככות ופרעות אבל בנפלים נאמנין הן ואף על גב דאיכא למיחש לבקיאות דיצירה כן נראה לי.\par  וליכא לפרושי מהלך ובא על פני כולה שהולך תחת הסככות אורך ורוחב שהרי פירשנו שאהל הסככות עצמו מדבריהם והם אינן גוזרין כן והלכך ודאי אינן נאמנים עליהם אפילו עושין בהם מעשה. }
\twocol{\textbf{ורבי יוסי בחד ספיקא מטהר בספק ספיקא מיבעיא.}  פירוש, לר' יוחנן פרכי' דקאמר דר' מאיר נמי מטהר דאלו לר"ל איכא למימר לא תנן ר' יוסי אלא להודיעך דבריו של ר"מ לומר לך שאין כאן מטהר אף בתרי ספיקי אלא ר' יוסי שהוא מטהר אף בחדא ומדר"ש היה לבר זוגיה ש"מ דר"מ מטמא כרישא אלא לר' יוחנן לישתוק גמי מדר' יוסי ופריק לאשמועינן דחד ספיקא דומיא דתרי ספיקי ואפילו לכתחלה וממשנה יתירה גמרינן. }
\twocol{ הא דאיבעי לן \textbf{יושבת מה לי א"ר שמעון.}  קשיא ותיפשוט ליה ממתני' כדאמרי' בסמוך כיון דאמר ר"ש חזקת דמים מן האשה ל"ש עומדת ול"ש יושבת. ואיכא למימר מעיקרא קס"ד שאין חזקת דמים שוים מי רגלים מן האשה אלא בעומדת. והשתא דאשמועינן ברייתא דר"ש אפילו ביושבת אפשר דאתי דם ממקור א"כ הלך אחר חזקתך שחזקת דמים מן האשה ולא מן האיש דל"ש עומדת ול"ש יושבת. א"נ איכא למימר דפשטה דברייתא משמע ליה טפי ועדיף מדיוקא דמתניתין. }
\newchap{דף \hebrewnumeral{60}}
\twocol{הא דתניא \textbf{ושוין שתולה בשומר' יום כנג' יום בראשון שלה וכו'.}  פי' משום דלא מקלקלא לה מידי דהא טומאתה תחלה הוא וכתמים דרבנן הם וכל כה"ג תולין קלקלה במקולקל דומיא דההיא דאמרינן בפסחים דבדרבנן אמרינן שאני אומר חולין לתוך חולין נפלו ותרומה לתוך תרומה נפלה.\par  והא דאמר ר' חסדא טמא וטהור שהלכו בשני שבילין א' טמא וא' טהור באנו למחלוק' רבי ורשב"ג. בספק טומאה ברה"ר דספקו טהורה היא ובבאין לישאל בבת אחת א"נ שבא לישאל עליו ועל חבירו שא"נ היו טהורים היינו מטמאים שניהם.\par  וראיתי למקצת המפרשים שהעמידו מימרא דרב חסדא בטמא שמנה (ימים) [ב'] וג' ימים מימי ספרו שאלו היה בראשון ד"ה תולין בטמא כדק' ושוין שתולה בשומרת יום כנגד יום בראשון שלה וכו'. ואם היה הטמא בז' שלו שהשלים ימי טומאתו היכי קאמרינן הכי מאי נפקא ליה מינה הא ודאי היינו פלוגתייהו דהכא והתם טבילה בעי דמים הן מחוסרים.\par וזה הפיר' אינו מחוור לי שא"כ היה ר' חסדא מפרש בשני שלו או בג' שלו. ועוד דרב אדא בר אהבה דפרכיה דילמא בטמא בז' שלו קאמר ר' חסדא דתרווייהו נמי כי הדדי נינהו וכן זו ששאל ר' יהודה בר' ליואי מר' יוחנן מהו לתלות כתם בכתם אינה מתחוורת לי לדברי המפרשים שאם באנו לומר כן דכתם בכתם בראשון שלה לרבי ובז' לרשב"ג היה לו לישאל. והיכי קאמר אליבא דרבי לא תיבעי לך ואם נאמר דבראשון שלה אפילו לר' תולין בלא טעם. [דמגופא קא חזיא] (אי נמי) [א"כ] ברייתא מיתרנא (לר' יוחנן) [לר' יהודה בר ליואי] שפיר לרבי כאן בראשון שלה והיכי קאמרינן מכל מקום קשיא.\par  לפיכך נ"ל דהא דקתני ושוין שתולה כשומרת יום כנגד יום בראשון שלה לרב חסדא לא מפני שתולין קלקול במקולקל לר' דלית ליה האי סברא כלל אלא מפני שהוחזקה להיות רואה. וכן בתולה מפני ששירפה מצוי וכן היושבת על דם טוהר והנכרית מתוך שאינן מקפידות אינן מרגישות ויודעות לפיכך תולין בהן שאלו ראתה זו הרגישה.\par  הילכך להאי טעמא טמא וטהור שהלכו בשני שבילין אפילו בראשון שלו באנו למחלוקת ורב אדא פליג אהך סברא ומוקי פלוגתייהו משום דכי הדדי נינהו ור"י בר ליואי נמי כרב חסדא סבר לה הילכך בבעלת הכתם אפילו בראשון אינה תולה לרבי. וכל השמועה על דרך זו תישב אותה. ושוין שתולה בשומר' יום כנג' יום בראשון שלה וכו'. פי' משום דלא מקלקלא לה מידי דהא טומאתה תחלה הוא וכתמים דרבנן הם וכל כה"ג תולין קלקלה במקולקל דומיא דההיא דאמרינן בפסחים דבדרבנן אמרינן שאני אומר חולין לתוך חולין נפלו ותרומה לתוך תרומה נפלה.\par  והא דאמר ר' חסדא טמא וטהור שהלכו בשני שבילין א' טמא וא' טהור באנו למחלוק' רבי ורשב"ג. בספק טומאה ברה"ר דספקו טהורה היא ובבאין לישאל בבת אחת א"נ שבא לישאל עליו ועל חבירו שא"נ היו טהורים היינו מטמאים שניהם.\par  וראיתי למקצת המפרשים שהעמידו מימרא דרב חסדא בטמא שמנה (ימים) [ב'] וג' ימים מימי ספרו שאלו היה בראשון ד"ה תולין בטמא כדק' ושוין שתולה בשומרת יום כנגד יום בראשון שלה וכו'. ואם היה הטמא בז' שלו שהשלים ימי טומאתו היכי קאמרינן הכי מאי נפקא ליה מינה הא ודאי היינו פלוגתייהו דהכא והתם טבילה בעי דמים הן מחוסרים.\par וזה הפיר' אינו מחוור לי שא"כ היה ר' חסדא מפרש בשני שלו או בג' שלו. ועוד דרב אדא בר אהבה דפרכיה דילמא בטמא בז' שלו קאמר ר' חסדא דתרווייהו נמי כי הדדי נינהו וכן זו ששאל ר' יהודה בר' ליואי מר' יוחנן מהו לתלות כתם בכתם אינה מתחוורת לי לדברי המפרשים שאם באנו לומר כן דכתם בכתם בראשון שלה לרבי ובז' לרשב"ג היה לו לישאל. והיכי קאמר אליבא דרבי לא תיבעי לך ואם נאמר דבראשון שלה אפילו לר' תולין בלא טעם. [דמגופא קא חזיא] (אי נמי) [א"כ] ברייתא מיתרנא (לר' יוחנן) [לר' יהודה בר ליואי] שפיר לרבי כאן בראשון שלה והיכי קאמרינן מכל מקום קשיא.\par  לפיכך נ"ל דהא דקתני ושוין שתולה כשומרת יום כנגד יום בראשון שלה לרב חסדא לא מפני שתולין קלקול במקולקל לר' דלית ליה האי סברא כלל אלא מפני שהוחזקה להיות רואה. וכן בתולה מפני ששירפה מצוי וכן היושבת על דם טוהר והנכרית מתוך שאינן מקפידות אינן מרגישות ויודעות לפיכך תולין בהן שאלו ראתה זו הרגישה.\par  הילכך להאי טעמא טמא וטהור שהלכו בשני שבילין אפילו בראשון שלו באנו למחלוקת ורב אדא פליג אהך סברא ומוקי פלוגתייהו משום דכי הדדי נינהו ור"י בר ליואי נמי כרב חסדא סבר לה הילכך בבעלת הכתם אפילו בראשון אינה תולה לרבי. וכל השמועה על דרך זו תישב אותה. }
\twocol{ הא דאמרינן \textbf{רב אשי אמר הא והא רשב"ג. ול"ק כאן למפרע כאן להבא.}  כך פירש שאם לבשו הן שתיהן החלוק הזה ואחר שפשטו אותו מצאה אחת מהן כתם א' בחלוק שלה אין תולין כתם בכתם אבל היתה אחת מהן כבר בעל' כתם ולבשו חלוק זה ונמצא בו כתם תולין בבעלת הכתם שהיתה כבר וזה הפי' נכון ולשון הגמרא מוכיח אבל הפי' שפירש ר"ש אינו נכון כלל. }
\twocol{ כיון ד\textbf{דרש רב חייא בר רב מתנה משמיה דרב}  כר' נחמיה ותנא ר' יעקב מטמא ור' נחמיה מטהר והורו חכמים כר' נחמיה שמע מינה דהלכתא כותיה. ועוד דהא רב הונא ורב חנינא ואביי דהוא בתרא מתרצי אליביה דאמרינן התם מדקמתרץ ר' יוחנן אליבא דר' יהודה ש"מ הילכתא כותיה. וההיא איתתא דבפרק הרואה דאשתכח לה דם במשתיתא משתיתא דבר המקבל טומאה הוא דהיינו טווי. וכן פסק הרמב"ם ז"ל כר' נחמיה בכתמים. }
\twocol{\textbf{שוע טווי ונוז כתיב.}  פירש רש"י ז"ל שוע חלק כדמתרגמינן שעיט כלומ' שיהיו חלוקין יחד במסרק וכן פי' טווי שיהיו טווין יחד ונוז לשון אריגה לומר שיהו ארוגין יחד.\par  ורבינו תם ז"ל השיב אם כן בכלאים בציצית דשרא רחמנא היכי משכחת לה לא שוע ולא טווי ולא נוז איכא אלא שתי תכיפות בעלמא איכא כדאמרינן במנחו' ש"מ קשר עליון דאורייתא דאי סלקא דעתך לאו דאורייתא כלאים דשרא רחמנא גבי ציצית למה לי והא קיימא לן התוכף תכיפה אחת אינו חבור אלא ש"מ דאורייתא והוו להו שתי תכיפות אלמא מדאורייתא ואפילו (בבגדי כהונה) [בב' תכיפות] הוי חבור. וכן בפרק קמא דיבמות (דף ה') מפקינן שתי תכיפות מדאורייתא ואפילו בבגדי כהונה נמי שחוטן כפול ששה דנוז איכא שוע מיהת ליכא ולא הוי חלוקין במסרק יחד ואפ"ה אסורין משום כלאים כדאיתא במסכ' יומא פרק בא לו (סט, ט).\par  אלא כך פירש רבינו תם ז"ל: שוע שיהא כל אחד חלוק במסרק לעצמו ושיהא כל אחד טווי לעצמו ושיהא כל א' שזור לעצמו ומכיון שהן כך אם תכף בהן שתי תכיפות חבור הוא דכתיב יחדיו והכי דייק לישנא דקרא לא תלבש שוע טווי ונוז צמר ופשתים יחדיו כלו' מחוברין ונוז לשון שזירה הוא ולא לשון אריגה ולא לשון חבור כמו שפי' אחרים מדקאמרינן הכא ואימא או שוע או טווי או נוז ומפרקי' כמר זוטרא מדאפקינהו רחמנא בחד לישנא אלמא כי היכי דשוע טווי קאי אכל חד באפי נפשה ה"נ נוז אכל חד באפי נפשה קאי שיהו שזורין ולאו לשון חבור הוא.\par  והאי חוטא דכיתנא דקאמרינן דמדרבנן הוא כשאינו שזור דקאמר שסתם חוטין אינן שזורין וכדרומרינן בפרק בתרא דערובין (דף צו ע"ב) המוציא תכלת בשוק לשונות פסולה חוטין כשרה מאי שנא לשונות דפסולה דאמרינן אדעתא דגלימא צבעינהו חוטין נמי נימא אדעתא דגלימא טוינהו בשזורין שזורין נמי נימא אדעתא דשיפתא דגלימא עייפינהו וכו'. וש"מ דסתם חוטין היינו פשוטין ולא כפולין ושזורין והאי חוטא דכיתנא דאבד בנלימא דעמרא בשאינו שזור וגלימא גופא אינה שזורה שאין דרכן בשזורין אלא בשיפתא דגלימא מ"ה הוו כלאים דרבנן. ואפילו תפרש בחוטין בין שזורין בין שאינן שזורין במשמע מכל מקום כאן בחוט פשוט דאינו נוז דהיינו שזור קאמרינן.\par וזו ששנינו במס' כלאים (ט, ח) אין אסור משום כלאים אלא טווי ואריג שנאמר לא תלבש שעטנז דבר שהוא שוע טווי ונוז רשב"א אומר נלוז ומליז הוא אביו שבשמים עליו אלא לאו למימרא דמשוע טווי ונוז נפיק אריג אלא טווי נפיק משעטנז ואריג משום חבור הוא ומלשון יחדיו נפיק. והתם קתני הלבדין אסורין מפני שהן שוע אע"פ שאינו טוי ואריג ומדרבנן קתני או שוע או טווי או נוז ואסורין והיינו דרבנן וכן עיקר המשנה מוכית והיינו דקתני נלוז ומליז אביו שבשמים עליו אלמא משמע לשון נוז היינו ענין פתלתולות ועקש כדרך השזורין שפותלין אותן והיינו דשרא רחמנא כלאים בציצית משום דלגבי ציצית שזורין בעינן כדאמרינן בסיפרי פתיל תכלת טווי ושזור אין לי אלא תכלת לבן מנין וכו'.\par מכאן תשובה לאומרים שאין לשזור חוטי ציצית זימנין דמפרקי והוו להו י"ו חוטין ואין לחוש אי מפרקי דכיון שתחלתן שזורין תו לא מיפסלי דהוו להו כגרדומי ציצית דכשרין אלו דברי רבינו תם ז"ל. וצ"ע בגרדומא דקאמר רבי דהתם בעי שיעור כדי לעניבן ואם כן ה"נ בעינן שנשתייר בשיזור שבהן כדי עניבה. מכאן אתה למד שהתופר בגדי צמר בחוטין של פשתן ושניהן נצמדין שהוא כלאים של תורה. }
\twocol{הא דתנן \textbf{שבעה סממנין מעבירין על הכתם.}  לטהרות קאמר ותני והדר מפרש הטבילו ועשה על גבי טהרו' העביר עליו שבעה סממנין ולא עבר הרי זה צבע. כלומר תולין אותו להקל ונאמר שהוא צבע שכן דרך הצבע שלא לעבור בסימנין ואף על פי שאפשר שהוא דם כיון דבלוע כ"כ שאינו יכול לצאת על ידי סמנין הללו טומאה בלועה היא ואינה מטמאה.\par ומיהו אם לא הטבילו תחלה טהרותיו תלויות שהרי יש לו לחוש לדם ואף על פי שאין סופו לצאת מכל מקום הבגד טמא שכבר נטמא בשעת נפילה ומטמא אותן והיינו דקתני הטבילו ומיהו אם דם נדה ודאי הוא אף על פי שלא עבר טמא לפי שדרך בני אדם להקפיד בו ולהעביר עליו סימנין אלו הילכך לא עלתה לו טבילה ראשונה עד שיעבירם ויבטלנו. }
\newchap{דף \hebrewnumeral{62}}
\twocol{\textbf{עבר או שדיהה הרי זה ודאי כתם וטהרותיו טמאות.}  לפי שסוף טומאה זו לצאת וכל שיוציא על ידי סממנין אדם עשוי להוציאו והיינו דפריך מיניה ר"י לריש לקיש דקאמר טומאה שסופה לצאת חמירה אף על פי שלא יצתה ואף על פי שאין אדם עשוי להוציאה א) רק על ידי צפון (אין) תולין בה להקל וטהורה שהכל היו מודים דמתניתין הכי קתני לא עבר שמא צבע ותולין להקל עבר ודאי דם ובטומאה ודאי בעינן בטול סממנין מפני שדרך בני אדם להקפיד עליהן ולבטלן כדפרישית. והאי דלא פרכיה ממתניתין דקתני או שדיהה וצריך להטביל דאלמא סלקא ליה טבילה משום דאיכא למימר צריך להטביל מאחר שיעביר הכתם לגמרי על ידי צפון אי נמי בההיא נמי קולא היא משום דכתמים דרבנן זהו תורף של רש"י ז"ל. ואין במשנתינו בדיקה לגבי בעלה אם הכתם טמא או טהור אלא אם יש מקום לתלות כגון שצבעה תולה ואם לאו אינו תולה.\par והרמב"ם ז"ל פירש רישא דמתניתא לגבי בעלה שאין הכתם מטמא עד שיודע שהוא דם לפיכך בודקין אותו בסממנין הללו אם עבר או שדיהה טמאה ואם לאו טהורה ולדבריו מעשה שתלו בקלור ובשרף שקמה אינו אלא שיהו צריכין סממנין. וק' אם כן מאי פרכיה דר"י לריש לקיש ממתני' כשם שבדיקת סמנין מטהרת לבעלה למה לא תטהר לטהרות. ועוד למה לי למיתני והטבילו ואיכא למימר טהרות חמירי והא דקתני הטבילו משום סיפא ולא דייק. }
\twocol{הא דאמרינן \textbf{ל"ש אלא טהרות שנעשו בין תכבוסת ראשונה וכו'.}  פי' רש"י ז"ל תכבוסת העברת סממנין שהרי הקפיד עליו כשהחזירן והעבירן עליו וגלה דעתו שמקפיד עליו בספק דם ועבר ע"י העברה זו ונעשה בו מעשה דם שכן דרך דם לעבור ע"י סמנין ואין פי' מחוור לי שאין קפידה זו דומה להא דתני ר' חייא.\par  אלא כך נראה פי' דכי מטהרינן כתם בטבילה ראשונה כשלא עבר בסמנין מפני שאין סופו לצאת בדרך כבוסו וכדפרישית וזה כיון שגלה דעתו שהוא רוצה להוציאו מ"מ אין זו טומאה בלועה אלא סופו לצאת היא וצריך טבילה לאחר שתצא לגמרי. }
\twocol{הא ד\textbf{אמר שמואל הרי אמרו לימים שנים.}  "אמרו" קאמר וליה לא סבירא דהא לקמן (סד, ב) בוסת דילוג אמרי' דשמואל כרשב"ג בוסתו' דיומי ס"ל והכי קי"ל.\par  ומיהו בוסתות דגופה ק"ל היכי אשכח בהו פלוגתא דרשב"ג ורבנן הא לא אשכחן חזקה אלא לר' בתרי זימני ולרשב"ג בתלתא אלמא היכי דבעי חזקה לרבי נמי תרי בעינן ואיכא למימר שמואל גמרא גמיר דלרבנן בחד ואשכח מתני' דקתני בוסתו' דגופה וכל שתקבע לה וסת ג' פעמים הרי זה וסת. ואמר אמאן תרמיה ודאי לרשב"ג דאשכחן דמיקל (בחזקת חששו) [בוסתות] וכיון דסיפא ודאי ביומי רשב"ג היא רישא נמי לדידיה מוקמינן ופלוגתא אחריתי היה מ"ס בעי חזקה דהיינו בתלתא זימני ואפילו בדגופה. ומ"ס אפילו תרי לא בעיא דהיינו אורחאי. ואע"ג דלא אשכחן פסקא בוסתות דגופה כרשב"ג כיון דסתם מתני' הוא ומחלוקת בדשמואל לא עדיף ממחלוקת דבריית' והלכה כסתם מתני' ודאמר ר' יהודה אמר שמואל זו דברי ר"ג הוא וס"ל היא דהא ודאי ס"ל כוותיה ביומי כדפרישית. }
\newchap{דף \hebrewnumeral{64}}
\twocol{\textbf{ראתה יום ט"ו בחדש זה וכו'.}  אם באנו לחשב חדשים מלאים בכולן אין כאן דילוג אלא הוסת שוה להפלגת ל"ב ואם באנו לפרש אותן בכסדרן חסר ומלא אין סדר לדילוג הזה שהראשונה מט"ו בניסן לט"ז באייר שוה להפלגת ל"ב והשנייה שהיא מט"ז באייר לי"ז בסיון להפלג' ל"א הוא נמצא שלא דילגה אלא קרבה ראיתה וכשהיא משלשת בדילוג מי"ז בסיון לי"ח בתמוז חזרה להפלגה שוה לל"ב ואין כאן דלוג אלא א"כ דילגה עד י"ט בחדש. ולקמן מתרץ שמואל לברייתא כגון דרגילה למחזי ליום כ' וקתני כ"ג בחדש זה ולפי חשבינך ה"ל למיתני כ"ה.\par לפיכך פ' הראב"ד ז"ל שכשם שהאשה קובעת וסת להפלגת שוות כך קובעת וסת בימי החדש שאם ראתה ריש ירחא וריש ירחא וריש ירחא קבעה לה וסת ול"ח אע"פ שאין ההפלגות שוות שאחד מלא וא' חסר. הילכך בזו שדילגה כיון שאין בראיותיה צד השוה לה לא בהפלגות ולא בדילוגין אומרים לימי החדש היא קובעת ולא בהשואה אלא בדילוג.\par  ופסק ר"ח ז"ל כרב באיסורי ומיהו דוקא בזו שהיא קובעת בימי החדש אבל בימי' שקבעה להפלגות אין הראשונה מן המנין דהא לאו בהפלגה חזיתא. וכן כתב ה"ר אברהם.\par  וכזה מורין חכמי הצרפתים בתוספות וראיה נתנו לדבריהם דהא מתניתין דהיתה למודה לראות יום ט"ו ושנתה פעמים ליום כ' הרי לה ג' ראיות ואינה קובעת עד ששינתה ג' פעמים ליום כ' כדי שיהיו לה שלש הפלגות של כ'. וא"ת למודה שאני א"כ דקארי ליה לפירכיה דרב מברייתא אמאי קא מייתי לה הא מתנ' היא דלמודה שאני ועוד מדקתני עלה שאין האשה קובעת וסת וכו' ומשמע להו שאין חלוק בין קביעותיה של זו לקביע' אחרת שאינ' למודה והיינו פלוגתייהו דרב סבר כיון דזו לימי החדש היא קובעת אף הראשונה מונין לה שהרי היתה ביום ידוע מן החדש ומתחלת' לדילוג כונה. ושמואל סבר אם השות' ליום החדש מונין לה הראשונה אבל כיון שדילגה צריכה ג' דילוגין.\par  וק"ל לרב דאמר למודה שאני מתני' אמאי קבעה לה וסת בשינוי של ג' פעמים קמייתא דט"ו שדי לה בתר ראיות ראשונות של ט"ו וליכא אלא שתי הפלגות. ואיכא למימר לרב לא בעינן אלא שוה מחד צד וכיון שנודעו הפלגותיה של זו בג' פעמים כבר הוקבע. וכי קאמר רב למודה שאני משום דהתם ליכא למימר כטעמיה דמשעה ראשונה כוונה לדילוג דהא לשמואל לא אמרינן למודה שאני כדמתריץ כגון דרגילה למיחזי ליום כ' וכו' אלמא קמייתא ממנינא כיון דאיכא ג' דילוגין מ"מ קבעה לה דהא איכא הכירה דהפלנה בראשונה דהיינו שלש.\par  והראשונים הקשו על מה שאמרו שהכל מודים דבהפלגות שלש הפלגות שהן ד' ראיות בעינן מההיא דתניא בב"ק (דף לז) ראה שור נגח שור לא נגח שור נגח שור לא נגח שור נגח שור לא נגח נעשה מועד לסירוגין לשוורים ואע"ג דקמייתא לאו בסירוגין הוות מצטרפי. ומתרצין שאני הכא דכיון דאין הפלגה ידועה אלא בשתי ראיות ג' הפלגות בעינן דהיינו ד' ראיות אבל התם האיכא ג' נגיחות.\par ומיהו לדידי קשיא לי הא דאמ' בפ"ק (יא, א)קפצה וראתה קפצה וראתה וקבעה לה וסת לימים ואוקמה רב אשי כגון דקפצה בחד בשבאי וחזי וקפץ בחד בשבאי וחזאי ולחד בשבאי אחרינא חזאי בלא קפיצה ובודאי ימי שבוע לא קבעי וסת אלא בהפלגות שוות כגון דקפץ בחד בשבא ולאחר כ"ב קפץ נמי הכי ולאשנוי ראיות בהפלגה נקט חד בשבא וקא מני ג' ראיות וקאמר דקבעה.\par ואיכא למימ' התם לאו לאשמועינן בכמה הוסת נקבע אתא ומ"ה לא דק ונקט תלת ראיות בלחוד דאי לדברי רבי קבעה לה וסת אפילו להפלגות ואי לרשב"ג אפיך סדרא מחד בשבא לחד בירחי.\par וחכמי הצרפתים מוסיפין שאף היום גורם וסת כדאשכחן בשור המועד שבת ושבת ושבת נעשה מועד לשבתות הלכך חד בשבא וחד בשבא וחד בשבא קבעה וסת לימי השבוע נמצאו לדבריהם ג' דרכים בוסתות של ימים קביעו' היום והחדש וההפלג' ולדברי הרב ר' אברהם שנים הן וקורא אני עצמי מקרא זה פליאה דעת ממני נשגבה לא אוכל לה. וכי הוסת מזל יום גורם או מזל שעה גורם שיהא תלוי ביום השבוע או ביום החדש שהרי להפלגות שוות הדין נותן כן שכבר נתמלא' סאתה של זו וכן דרכן של נשים כולן וכן של אנשים בחלאים של הפסקות שהן באין בהפסקות שוות או בשעת המולד של לבנה ובמלואה אלא שיהא שיפורא גורם תמה הוא.\par ולפי דעתי בעניותי לא יפה כוונו הראשונים בחלוקי הוסתות שאני אומר אין וסת אלא להפלגה שוה לפי שהאורח בזמנו הוא בא מתמלא ונופצ' לזמן הקבוע שכיון שטבען של בני אדם וזו שאמרו בט"ו בחדש זה וי"ו בחדש זה וי"ז בחד' זה בחדשי' השוים הוא או בחסרי' וההפלגה שלה שוה היא ליום ל"ב. והדילוג שהורו בו אינו אלא דילוג הימים לומר שהיא קובעת וסת לדלג לי"ח לי"ט ולכ', וכן לעולם.\par  ולעיקר המחלוקת של רב ושמואל לפי שהוסת הקבוע להפלגה שוה ולימים השוים כגון מט"ו לט"ו בג' ראיות הוא נקבע כשור המועד שבשלש נגיחות נעשה מועד ואלו נגח בט"ו לחדש ג' פעמים וכן בהפלגה שאין בה חדש מכ' לכ' בג' נגיחו' הוא מתיעד דאפי' בסירוגין נעשה מועד בג' נגיחות כמו שפרשתי וכשנגח בט"ו בחדש זה וי"ו בזה וי"ז בזה אינו נעשה מועד לשמואל עד שהוא בדילוג. וכן בוסתות (בהפלגה שוה) בג' ראיות קבעה אותו ואע"ג דקמייתא לאו בהפלגה חזיתה שהרי אף וסת הדילוג וסת של הפלגה הוא לדברינו ואין יום החדש גורמת קבעותו כלל ב) וטעמא דאיכא דילוג הא לאו הכי קבע' בג' ראיות וכדאמ' רב אשי בפ"ק ואע"ג דראיה קמייתא נמי לאו בהפלגה הות אפ"ה מודה שמואל שקובעת אותו בג' ראיות שהרי מתחלה כיון שראתה [ב' פעמים] בט"ו בחדש זה נכרת הפלגת ווסתה מעתה וכן בכל הפלגה [שוה] שתפליג בשתי ראיות ראשונות נודע וסתה לפיכך ראשונה מן המנין אבל וסת הדילוג כשראתה בט"ו בחדש זה וי"ו בחדש זה לא יודע וסתה של זו שהרי הפליגה ולא דילגה כלל ומנין לנו שלדילוג היא מכוונ' שמא תראה בחדש הבאה בי"ו והוםת שוה ה) וכשרא' למחרתו והוסת של דילוג הוא היינו שלישית {\small [לפי שראייה שלישית היא שביררה הדילוג]}  ואינה ראויה לקביע בתחלתה והיינו דקאמר קמייתא לא בדילוג חזיתה כלומר בהפלגה ראשונה עדיין לא היה לה כלל וסת של דילוג אבל בהפלגה שוה מתחלתה כשחזרה וראתה [ניכרת ההפלגה] ויודע וסתה.\par [והא דאמרינן] שינתה למיום עשרים דבעי ג' פעמים למודה שאני לכ"ע דלט"ו היתה למודה לראות בקביעות הך ראיה בתרייתא בתר ראשונו' שדינן לה ועכשיו ששנתה ליום אחר לגמרי כמי שמתחלת לראות דמיא. וכי קס"ד לרב דלא אמרינן למודה שאני ולשמואל נמי לית ליה במדלגת בלחוד היא משום דעכשיו גמי אינו שינוי גמור אלא על הוסת ראשון עצמו מדלגת והולכת מט"ו לי"ו ומראיה דוסת ראשון ניכר וסת ב' של דילוג לרב כמו שפי'. וראיה לדבר ההיא דאמרינן בב"ק (דף לז) נגח שור שור שור וחמר וגמל מהו האי שור קמא בתר שורים שדינן ליה ואכתי לשוורים הוא דאיעד לשאר מיני לא איעד או דילמא בתר חמור וגמל שדינן להו זהו ואיעד לכולהו מיני. וה"נ אי חזיא ט"ו וט"ו וט"ו תלת למני והדר דילגה י"ו וי"ז היינו בעיין אבל למודה לראות ט"ו וט"ו ד' זימני כיון דאי צמי שדית בתרייתא בתר הני דדילוג אכתי קבעה לט"ו שדינן כולהו בתר מעיקר' דסרכא נקט וכן נמי בשור שור שור ושור (ושור) וחמו' וגמל שור בתריית' בתר שוורים שדינן ליה דהא איעד להו והשתא הוא דקא מיעד נפשיה לשאר מיני זהו הדרך שנראה לי בדברים הללו.\par  ועדיין לבי מהסס, מה טעם אמרו בשור המועד נגח בט"ו בחדש זה וי"ו בחדש זה וכו' דאלמא מתיעד הוא בדילוג ובהפלגה. וכי מזל יום גורם נגיחות שאפילו בהפלגה שוה אין הטעם מתחוור בכך כמו שהוא מתחוויר בוסתו' אלא י"ל שראו חכמים בכל נגיחות שהן לזמן שוה כגון בין בסירוגין בין בדילוג שאין מתיעד אלא לאותו ענין שעשה לנגחותן שכך וסתו של זה ליגח ושמא מלמ"ד ללמ"ד מוסיף כח ונוגח וכן כיוצא בדבר זה לעולם למה שהשוה הוא מתיעד ולא לדבר אחר.\par  ומדברי הרב ר' משה הספרדי ז"ל משמע שאין לו קביעות וסת אלא בהפלגה שוה שאם היה סובר כדברי הרב ר' אברהם או כדעת חכמי התוס' ז"ל היה לו לפרש (בחסדו) [בספרו] והוא ה"ר משה ז"ל פסק כשמואל בדילוג משום דאמרינן דיקא נמי וש"מ ומסתברא כותיה וה"ר אברהם ז"ל דן בה להחמיר. ובעל נפש יחוש להחמיר בענין הוסתות בין בדברינו בין לדברי ראשונים עד יערה עלינו או על אחרים רוח ממרום להכריע איזו היא הדרך הישר' שיבור לו האדם ואם ימצא בחבורי הגאונים או בתשובותיהם ענין מורה על א' מאלו הדרכים בה ראוי ללכת ולצאת בעקבותיהם. }
\twocol{מהא דקתני בברייתא \textbf{שינתה לי"ז הותר י"ו ונאסר ט"ו וי"ז.}  ולא מיתסר נמי י"ח דנימא זו כבר דילגה ונחוש בפעם א' לוסת של דילוג ש"מ שאין חוששין לוסת של דילוג כלל עד שתקבענו לגמרי וזה כתוב בתוספות. וכן הורה ה"ר אברהם ז"ל. }
\twocol{\textbf{היתה למודת להיות רואה יום כ' ושנתה ליום ל'.}  מדקתני האי לישנא ש"מ דוסת הפלגה הוא דקבעה מכ' לכ'. והשתא ק"ל כיון דקי"ל מראיה לראיה מנינן ולא לפי מנין הראוי כדאיתא בשלהי בנות כותיים כשהגיע יום כ' ולא ראתה ומנו עשרה לתשלום ולא ראתה הגיע יום כ' וראתה דקתני כי אורח בזמנו בא מאי נינהו הא ליכא הפלגה דעשרים השתא.\par ואיכא למימר הכא מנינן למנין הראוי ויום מ' לראיה אחרונה זו היא יום כ' דקתני שאם ראתה בעונות הראשונו' ביום זה תראה ואפילו הרחיקה יותר מונין לראיה אחרונה שפסקה בו עכשיו ולא שאלו ראתה מאותה ראיה ואילך בעונות של כ' יארע לה ראיתה ביום עשרי' אורח בזמנו בא דהכא רגלים לדבר שלמנין הראוי חוזרת אלא ש"ל זמן שרואה בוסת השינוי מונין להן מאות' ראיה אבל מכיון שהפסיקתו וחזרה לראות ביום (א') [אחר] אם למנין הראוי חזרה מונין לוסת הראשון לפי אותו מנין ואין אומרין הפלגה של מ' היא זו שרגלים לדבר.\par  אבל הרב ר' אברהם בר דוד ז"ל פי' לזו בוסת החדש לפי דעתו ולמודה ליום כ' בחדש ושנתה ליום ל' בחדש קתני ולפי פי' בוסת של הפלגה אין אומרים חזר הוסת למקומו עד שתראה עכשיו ותחזור ותראה לסוף כ' שחזר האורח בזמנו. }
\newchap{דף \hebrewnumeral{65}}
\twocol{ מנימון סקסנאה דעבר \textbf{מיעבד כרב ואפילו ראתה}  לית ליה אידך דרב דאמר בועל בעילת מצוה ופורש משום דההיא אתיא כרבותינו דחזדו ונמנו ואיהו דאמר כסתם מתני' ושמואל לא קבילי ליה דמנימון (מידי) א) מ"ה דמאן דעבד כסתם מתני וכמעש' דר' לאו בר עונשין הוא אלא משום דבעי למיעב' דלא כחד ולמיתל' ברב מ"ה איענש דלא יאונה לצדיק כל און. }
\twocol{\textbf{כולן צריכות לבדוק את עצמן.}  פי' רש"י ז"ל שמא נשתנו מראה דמים שלה ולא סמכינן למימר הואיל ושופעת הכל דם א' הוא וטהורות.\par ואי קשיא מ"ש לאחר ד' לילות והא בתוך ד' לילות נמי אמרינן בפ"ק נשתנו מראה דמים שלה טמאה והתם נמי אמרינן ותבדוק בעדים דילמא נשתנו מראה דמים שלה.\par  י"ל הכא כיון ששופעת אין צריך לבדוק כל זמנן אבל מתוך זמנן לאחר זמנן צריכות לבדוק. א"נ התם לטהרות הכא אפילו לבעלה צריכות לבדוק הא אם נשתנו ודאי בין לאחר זמנן בין בתוך זמנן טמאה והך רישא ד"ה היא דודאי נשתנה מראה דמים שלה טמאה. וסיפא פלוגתא דר"מ ורבנן פלוגתא אחריתי היא שהיה ר"מ אומר דם בתולים אינו אדום וזהום לפיכך בודקת בזה אם מצאתו אדום וזהום בידוע שהוא דם הנדה ואע"פ שאינו יודע תחלתו מה היה שאם מצאתו מתחלתו אדום שאעפ"כ היה טהור אע"פ שאין רגילתו בכך וחכמים אומרים כל מראה דמים א' הוא הילכך לעולם תולין בדם בתולים עד שיודע לך שנשתנה ממה שהיה בתחלה.\par ויש לפרש ברייתא כולה דר"מ היא וה"ק כולן שהיו שופעות ובאות מתוך ד' לילות ולילה א' לאחר זמנן אינן טהורות אלא בבדיקה שבכולן ר"מ מחמי' כדברי ב"ש מן הסתם ומיקל כדברי ב"ה עם הבדיק' ומה היא הבדיקה הזאת שיהא מראהו שלא כדם הנדה אינו אדום ואינו זהום הא בתוך זמנן כב"ש אינן בודקות בכך אבל אם ראתה שנשתנה מראה הדמים טמאה. }
\newchap{דף \hebrewnumeral{66}}
\twocol{ה"ג וכן גרסת רש"י ז"ל \textbf{משמשת פעם ראשונה ושנייה ושלישית}  וכן בכולהו גרסינן ושלישית. ותתגרש ותנשא לאחר וכולה רשב"ג היא דאמר בתלת זימני הוי חזקה וכי קתני דברי רבי אסיפא דמתני' דנאמנת אשה בלחוד קתני אבל ברישא כולה רשב"ג היא.\par  ורבינו ז"ל כתב בהלכות עד כמה מותרת לינשא עד ג' יותר מכאן לא תנשא עד שתבדוק את עצמה ואיני יודע אם הוא לשון גרסת הגמרא מ"מ כרשב"ג אתיא. }
\twocol{ והא דמקשינן \textbf{ותבדוק בביאה ג' של בעל ראשון.}  נראה ודאי דה"ק היאך מותרת לינשא לב' ולשמש בלא בדיקה והלא כבר הוחזקה זו לראות מחמת תשמיש בג' ביאות של בעל ראשון הילכך תבדוק ולא תתגרש ופריק מותר לשני בלא בדיקה מפני שאין כל האצבעות שוות ואין מחזיקין אותה בבעל מום ורואה מחמת תשמיש אלא לאצבע זה שהוחזקה לו. לפיכך תתגרש ממנו אבל לשאר אצבעות [לא הוחזקה] עד שתהא מוחזקת לכל בג' אצבעות.\par  והדר אקשי' כיון שהוחזקה בג' אצבעות למה לה ג' פעמים באצבע אחרון. ולמאי דפריך השתא דחזקה באצבעו' בלבד תהא חזקה אפילו שמשה פעם א' וראת ונתגרשה ושמשה עם השני וראת ומת ושמשה פעם אחרת עם ג' זה וראתה הוחזקה זו לכל.\par  ופריק לפי שאין כל הכוחות שוות. ואיפשר שראוי' לאצבע בנחת אבל בג' אצבעות וג' כוחות בכל אצבע ואצבע הוחזקה לכל אבל אם רצתה להכניס עצמה בספק ולבדוק בבעל ראשון ושלא תתגרש ודאי מותרת שאין אחר בדיקה של חכמים בית מיחוש שנאסרות אותה לא' ונתיר לג'. וכי קתני תתגרש ותנשא רבותא קמ"ל דלא מיחזקה אלא בג'. ומי שסובר להחמיר בבעלי' הראשונים אין ממש בדבריו דאם יש לחוש בראשונים כ"ש לאחרון שהוחזקה ולא התירו ספק דבר שזדונו כרת בשביל שתנשא זו.\par  אבל לענין מעשה עכשיו נראין דברי הרב ר' אברהם בר דוד ז"ל שאמר אין אנו בקיאין בבדיקה זו. ועוד שאפילו שידוע שהוא מן הצדדין הרי גזרו בנות ישראל בכל רואה טיפת דם כחרדל שיושבת עליה ז' נקיים בין מן המקור בין מן העליה בין מן הצדדין ודעת רבינו ז"ל שכתבה להנהיג בה הלכה למעשה. ואף ה"ר אברהם אמר שאין מוציאין אותה לאחר בדיקה. }
\twocol{הא דא"ר יוחנן \textbf{לכי והבעלי לו על גב הנהר.}  משום שאין וסת זה אוסר יומו לפי שלא היה וסתה אלא בהכנסתה לעיר שהיו חברותי' מרגישו' בה ושמא אף היא היתה מתביישת מהן מפני שמדברות בה ומתחלחלת וה"ל כוסת של קפיצות ושל אכילות שום שכל זמן שאינה אוכלת אינה חוששת. א"נ דימה לא קבעה וסת דאקראי בעלמא היא וה"ק לה לאו טבילה גרמא ליך דתיתסרי אלא דימת עירך גרמא ליך ומותרת את על גב הנהר. }
\twocol{הא דאמר שמואל \textbf{ממלא ונופצת היא}  ואין לה תקנה. ק"ל א"כ עקרת בדיקה הראשונ' ששנויה בבריית' דקא אמרת דאי אפילא נתרפאת ואי לא אפילא לית לה תקנה אי הכי מכחול למה ושמה אותה שהפילה חררה בידוע שנתרפאת וזו שלא הפילה כלום ממלא ונופצת היא. אבל אם ראתה מחמת בעיתותא ולא הפיל' חררה זו היא שצריכה לבדיקה של ברייתא וכן נמי שלא ביעתוה נבדקת בכך ואינו מחוור ומרבינו הגדול ז"ל לא כתב מעשים הללו. }
\twocol{ה"ג וכן בנוסחאות \textbf{יום א' תשב ו' והוא ב' תשב ו' והן.}  שהרי בני מקום זה שאין בני תורה בידוע שאין רואים דם ויש לחוש שמא יום א' דם טהור ויום ב' טמא וצריכה ו' ועוד שהרי אין נשיהן בקיאין בימי נדה וזבה שלכך תקן להם לג' ז' נקיים בכל זמן הלכך בשנים נמי יש לחוש שמא ראשון י"א לזיבה הוא ושני תחלת נדה הילכך צריכות ו' והן. ואין זה צריך לפנים אלא שבהלכות רבינו הגדול ז"ל דמחה והן נראה כטעות סופר.\par  והלשון שכתב רש"י ז"ל תשב ו' והוא כדין תורה לומר שדין תורה כך הוא למנות ו' לאחד אבל אינו כדין תורה לכך שזו יושבת ו' נקיים שאם תראה צריכה יותר הילכך צריכה הפרשה בטהרה ובדיקה והאי דקאמר ז' נקיים ולא קאמר נמי ו' נקיים לישנא בעלמא נקט דשגירי למימר ז' ימים נקיים.\par  וא"ת לדברי האומר ימי נדה שאינו רואה בהן אין עולין לה לספירת זיבתה עדיין היה לו לר' בית מיחש לומר שמא יום א' י"א הוא ובעי שימור והז' הן התחלת נדה אין עולה לו וצריך ז' והן לשני ימים. לאו מילתא שכבר פי' בפ' בנות כותיים שימי נדה ולידה שאינה רואה בהן עולין לספירת זיבה קטנה לדברי כל אדם ואין לחוש כלום. וכתב רבי' בעל הלכות ז"ל שאפילו בימי טוהר נמי אם ראתה סופרת דהאידנא יולדות בזוב הן לפי שא"א לפתיח' הקבר בלא דם ובנות ישראל סופרת ז' לכל טיפה דם. ושמענו כדבריו בזה שהגאונים החרימו בדם טוהר.\par  והדברים נראין אף לדין הגמר' שכשם שחששו לטועות בפתחיהן ולמשלימות דם טמא לדם טהור ועשו נמי הרחקה יתירה בדבר כך יש לחוש שמא יבואו לטעות באותן שיושב' עליהן לזכר ולנקבה ולנדה שמא ינהגו בהן קולא שסוברות כל שיש לו טומאת לידה יש לו טוהר שלה וכ"ש שברוב נפלים אין בני אדם בקיאין. ואין עליהם לדון בהם אלא כך תשב לזכר ולנקבה ולנדה וקרוב הדבר לטעות בו הילכך אין ימי טוהר יוצאין מכלל ר' זירא שאף בהן החמירו בנות ישראל לישב ז' נקיים. וההיא דאמרינן דרש מרימר הילכתא כותיה דרב וכו'. דינא קאמרי כדאמרי בשמעתי' אמינא לך האי איסור ואת אמר' לי חומרא היכא דאחמור אחמור היכא דלא אחמור לא אחמור. והמקומות שבועלין עכשיו על דם טוהר הם יחושו לעצמן. }
\twocol{\textbf{חפיפה.}  פי' רש"י ז"ל חפיפת שערה ופי' לפי' חפיפת שער בכל מקום שבגוף בית השחי ובית הערוה ואצ"ל ראשה. ובודאי דלשון חפיפה לא שייך אלא בשער כדתנן נזיר חופף ומפספס אבל לא מסרק ואיתמר בעלמא הוה חייף רישיה.\par וה"נ משמע בפ' מרובה (דף סב) גבי עשר תקנות שתקן עזרא ושתהא אשה חופפת וטובלת ואקשינן דאורייתא הוא דכתיב את כל בשרו את הטפל לבשר ומאי ניהו שערו. ופריק מדאורייתא עיוני בעלמ' דילמא מיקטר א"נ מיאוס מידי משום חציצה אתא איהו ותיקן חפיפה. מדמקשינן את הטפל לבשרו ומאי ניהו שערו ש"מ שאפילו בכל הגוף צריכה לעיין דברי תורה משום דילמא מיאוס במידי משמע דלגבי הכי מקום השאר ושאר מקומות שבגוף שווין אלא דאתא עזרא וחייש דילמא אתיא למיטעי בעיוני דשער משום דשכיח ביה קטרי ואחמיר ביה חפיפה הילכך בעי עיוני בכוליה גופיה דאורייתא ובעי נמי חפיפה למקום שער מתקנתא.\par  והיינו דאמרינן בשמעתין (לקמן סז, א) נתנה תבשיל לבנה וטבלה לא עלתה לה טבילה כלומר אם לא חזרה ועיינה בנפשה בשעת טבילה ממש אבל ודאי עיינה בעצמה אע"פ שלא חזרה לחוף נראה שעלתה לה טביל' דהא אע"פ שנתנה בנתיי' מעט תבשיל לבנה סמוך לחפיפה טבילה היא ומזיא לא מיקטרי בתבשיל של בנה.\par והא דאמרינן ונמצא עליה דבר חוצץ אם סמוך לחפיפה טבלה וכו'. דמשמע עליה על גופה לאו למימרא דחפיפה בכולי גופא היא אלא משום דודאי עיינה בנפשה בשעת חפיפה ואם לא טבלה סמוך לחפיפה ולא חזרה ועיינה בשעת טבילה אע"פ שהיתה משמר' נפשה מליתן תבשיל לבנה וכיוצא בו כיון שמצא עליה דבר חוצץ חוששין דילמא נגעה ולאו אדעתה.\par  וההיא דאמרינן לקמן (סז, ב) בחופפת בע"ש וטובלת למ"ש וכולהו הרחקות דחפיפה כשחזרה ועיינה בעצמה בשעת טבילה שלא הקלו בשל תורה אלא בתקנת עזרא דהא לא אפשר מע"ש למ"ש דלא נחנה תבשיל לבנה וכיוצא בזה ואמרן דלא עלתה לה טבילה אלא התם בשלא עיינה בעצמה וכאן כשחזרה ועיינה בשעת טבילה.\par  ומיהו מנהגא דנהגן נשי למשטף כולה גופה בחמימי בשעת חפיפה משום מקומות השער שבגוף הוא דקרירי ממשרו להו. וה"נ משמע בהא דאמרינן לקמן עבדי חסרת דודי חסרת טכטקי חסרת משמע דשטיפת כל גופא עבדא מדצריכה עבדי ודודי וטכטקי וכן החמירו בנות ישראל על עצמן והמעביר מנהג זה ימתח על העמוד. }
\newchap{דף \hebrewnumeral{67}}
\twocol{ הא דאמרינן \textbf{ולית הלכתא ככל הני שמעתא. אלא כי הא דאמר ר"ל אשה לא תטבול אלא דרך גדילתה וראוי.}  לאו למימרא דהנך פלגינן אדר"ל דהא אפשר דתרווייהו איתנהו אלא גמ' קאמר דלית הלכתא בכל הני שמעתת' דלא מחמרינן כולי האי בביאת מים בדברים שדרכן להיות באותו מקום אלא כך מחמרי' בביאת מים בדר"ל מחמירים ובעי טבילה דרך גדילתה כדי שיבואו מים קצת בבית השחי ובבית הסתרים.\par וכה"ג איכא טובא בתלמודא דקאמר לית הלכתא בכל הני שמעתתא אלא כי הא ולא פליגן אהדדי. כי ההיא דאמרינן ככתובות דלית הלכתא ככל הני שמעתא דזינתה וכחלה ופירכסה ותבעוה לינשא ונתפייסה אבדה מזונותיה אלא כי הא דא"ר יהודה תובעת כתובתה בב"ד אין לה מזונות והא (דא"א) [נמי אף דאפשר] דאיתנהו לכולהו. וכן במסכ' ברכות (דף מב) סלק אסור לאכול ומר אמר גמר ומר אמר משחא מעכב לן ואמר ולית הלכתא ככל הני שמעתא אלא כי הא דא"ר יהודה תכף לנטילת ידים ברכה. וכן בפרק שלשה שאכלו (דף מז ע"ב) ולית הלכתא ככל הני שמעתתא אלא כי הא דא"ר נחמן קטן היודע למי מברכין מזמנין עליו וההיא ודאי לא פליגא אשמעתתא דלעיל דט' ונראין כעשרה וט' וארון ושנים ושבת ושני ת"ח המחדדין זה את זה כ"ש קטן ועבד נעשה אותן סניפין לעשרה. והרבה מהן כך.\par  אבל כי איתמר בתלמודו לית הלכתא ככל הני שמעתתא מהא דאמרינן ההיא ודאי פליגן ומחדא מידחיא חברתה ממש.\par  ורבינו הגדול ז"ל גורס ולית הלכתא ככל הני שמעתתא אלא כי איתמר הני לענין טהרות איתמר אבל לבעלה שפיר דמי כי הא דאמר ריש לקיש וכו'. וק"ל מנלן מדריש לקיש דלבעלה מותרת דהא אפשר דכולהו איתנהו. ואיכא למימר כי מייתי דר"ל משום פתחה עיניה ביותר ועצמה עיניה ביותר דלא מעכב בטבילה דרך גדילתה נמי הוא שאין אדם נמנע מלפתוח ולעצום עיניו פעמים הרבה כדרכו ואין המים מתעכבים מליכנס שם כדרך שנכנסין בבית השחי ובבית הסתרים בטובלת דרך גדלתה ולא מחוור.\par  תו קשה לי ההיא דגרסינן ביבמות (מז, ב) וכל דבר שחוצץ בטבילה של טהרות חוצץ בגר ובעבד משוחרר ובנדה לבעלה. זו היא גרסתו של רבינו עצמו ז"ל ומשמע דכל דבר שחוצץ בטבילה של טהרות חוצץ בגר ובעבד משוחרר ובנדה לבעלה. והא איכא הני דחצצי לטהרות ולא חצצי לנדה.\par ואיכא למימר דה"ק: כל דבר שחוצץ בטבילה אחרת חוצץ בגר ובעבד ובנדה ואף על פי שאין טבילתן מפורשת בתורה שהרי טבילת נדה מן הכתוב מפורש לא למדנו אלא מבנין אב אתיא דכתיב ורחצו במים בנין אב לכל הטמאין שיהיו בטומאתן עד שיבואו במים. ומה שאמרו שם במקום שנדה טובלת שם גר ועבד משוחרר טובל לא מפני שטבילת נדה מפורש' יותר אלא לומר דלא בעינן מעין כזב וא"נ דבעינן טבילה בבת אחת משום דסמכי לה אבמי נדה יתחטא מים שהנדה טובלת.\par  והרב ר' אברהם בר דוד ז"ל פירש ולית הלכתא ככל הני שמעתתא לפלוף וכוחלי אלא כי הא דאמרן כדר' יוחנן פתחה עיניה ביותר דאמר ר"ל אשה לא תטבול אלא דרך גדלתה ומי שפתחה עיניה ביותר אינו דרך גדלתה. וזה הפירש מוקצה מן הדעת מפני שהוא מקלקל עלינו שיטת התלמוד שאמרו בכמה מקומות ולית הלכתא ככל הני שמעתתא אלא כי הא ובכולן פירושן ידוע שאין הלכה בכל הנזכרות אלא כי הא דבעינן למימר קמן.\par וקשה לן מרייהו דהני שמעתח היכי אמרינהו והאנן תנן (מקוואות ט, ב)אלו הן שחוצצין לפלוף שחוץ לעין וגלד שהוא חוץ למכה ואלו שאין חוצצין לפלוף שבעין וגלד שעל המכה.\par  ואיכא למימר לפלוף שבעין מוקים לה רב עוקבא בלח. והיינו טעמא דלא חייץ משום דלא קפיד עליה וה"ל מיעוטו שאינו מקפיד אבל יבש ודאי מקפיד. ושחוץ לעין דקפיד עליה אפילו לח חוצץ וגלד שעל המכה נמי משום האי טעמא הוא דלא עביד אינש לקלף גלד מכתו משום דקשה למכה עד דיביש ומקליף מנפשיה וקסבר דהוא דוקא של מכה אבל ריבדא דכוסילתא עד תלתא יומין דלא קפיד איניש עלה לא חייצה מכאן ואילך חייצה. אי נמי עד תלתא יומין לחה ולא מעכבה מיהו מכאן ואילך חייצה דיבישה וקפיד עלה.\par  ואי קשיא לך מאי שנא פתחה עיניה ביותר או שעצמה דלא מעכבי בטבילה ומ"ט קרצה שפתותיה כדתנן כאלו לא טבלה. לא תיקשי דודאי קרצה שפתותיה מעכבת ביאת מים במקום הגלויי אבל פתחה עיניה אינה מעכבת כלום אלא קמטין בעלמא הוא דעבדה במקום שדרכן בכך ואפשר נמי שאינן מעכבין כלל מלבא בהם מים ממש. }
\twocol{הא דאמר ליה רב פפי לרבא \textbf{מכדי האידנא כולהו ספק זבות שותינהו ליטבלן ביממא דשבעה.}  רש"י ז"ל מפרש כמשמעה לומר שאין לחוש להטבל בלילה אלא יכולין לטבול אותן ביום כדין הזבה דאי לנדה יותר משמיני הוי ואי לזבה טבילתה ביום הז' הוא.\par  ומתרצים משום דר' שמעון דאמר אחר מעשה של ספירה תטהר מיד ומקצת היום ככולו אבל אמרו חכמים אסור לה לטבול שמא תבא לידי הספק שמא תבא לשמש כיון שטבלה ותראה.\par  והגאונים כך סוברים שהאשה בזמן הזה אינה טובלת אלא בלילה משום סרך בתה שלא תטבול בז' ותראה ותסתור.\par ואחרים פרשו דרב פפי לאו אטבילה בלחוד פריך אלא ליטבלו ביממא דז' ויהיו מותרת לבעליהן קאמר שהרי מקצת היום בספירת הזבה כולו הוא. ומתרץ לה משום דר"ש דאמר אסור לעשותה כן להחזיק עצמה בטהורה לאחר ספירה מיד כלומר אסור שתשמש ותעסוק בטהרות שמא תראה ותסתור. וראיה לפירש זה מה שאמרו בסוף המפלת (דף כט ע"ב) בכ"א תשמש ר' שמעון היא דאמר אבל אמרו חכמים אסור לעשות כן שמא תבא לידי הספק הא טובלת אפילו לר' שמעון ואסורה לשמש הא לרבנן מותרת אפילו לשמש אלמא איסורא דר"ש בתשמיש בלחוד היא.\par ור"ש עצמו ז"ל כך כתב שם ר"ש היא דאמר בת"כ אסור לעשות כן לשמש זבה ביום טבילתה. ובודאי דהתם בת"כ מוכח כן דקתני כיון שטבלה טהורה להתעסק בטהרות אבל אמרו חכמים לא תעשה כן שלא תבא לידי הספק. וש"מ דאסור לעשות כן אעסק טהרות קאי וה"ה לתשמיש ולא אטבילה קאי ולפי הפירש הזה מקילין ואומרים דהאידנא טובלות הן ביום וליכא סרך בתה כלל.\par  ובודאי שזה הפירש הוא הנכון דר"ש לא אסר אלא להחזיק עצמה בטהורה לטהרות אי נמי לתשמיש אבל שנקל לומר שיהו טובלו' ביום אינ' נראה דהא רבא דשרא במחוזא משום אבולאי הא לאו הכי אסור בתר חומרא דר' זירא הוה דקא"ל ר' פפא מכדי האידנא כולהו ספק זיבות שויתינהו ומשמע נמי דאהדא דתקון תקנתא ושרא משום אבולאי פריך ליה למה ליה אבולאי כולהו נמי ליטבלן ביומא דז'. אלמא לרבא אית ליה משום סרך ואפילו לבתר חומרא והכי פירכיה אפילו לטבול ולהתירן לבעלן יהיו מותרת, ומתרצין להתירן א"א משו' דר"ש וכיון דאסורות לטהרות ולבעלה אין טובלות אלא בלילה ואפילו בח' משום סרך בתה שמא תטבול ותטהר כאמה דודאי כשם שהיא נסרכ' אחר אמה בטבילה דנדה נסרכת אחריה בתשמיש גופה, ואם תאמר תשמיש גופיה גזירה דרבנן ואנן ניקום ונגזור גזירה לגזירה, כיון דבא לידי איסור דכרת גזרינן.\par  ואי קשיא דרבנן היכי שרו לבעלה בז' והתנן טבלה ביום שלאחריו ושמשה הרי זה תרבות רעה התם בשומרת יום רגילה היא לבא לידי זיבה גדולה הכא כיון שספרה שבעה הוחזקה במעין סתום ואין חוששין לה אלא מדבריהם לר' שמעון, ולפי דעתי שלא נחלקו חביריו עליו כלל מדלא פרכינן תינח לר"ש לרבנן ליטבלן, והא דאמרינן בהמפלת, הא מני ר"ש היא משום דלדידיה שמעינן לה וכיוצא בה בתלמוד הרבה.\par ואחרים השיבו אי ר"ש לאו אטבילה קאי היכי קאמר שמא תבא לידי ספק ודאי לידי ספק באה ואין אנו גורסים אלא שלא תבוא לידי ספק, וכן בהלכות גדולות וכן בת"כ. ויש נוסחא שכתוב בה יבואו לידי ספק. }
\newchap{דף \hebrewnumeral{68}}
\twocol{הא דתנן \textbf{וחכמים אומרים אפילו בשנים לנדתה בדקה וכו'.}  דוקא בשנים אבל בראשון הואיל והוחזק מעין פתוח לא דתחלת נדה אין דרכה לפסוק ביומה ואפילו בדקה ומצאה טהור חוששין לה שמא חזרה וראתה ואין צריך לומר כשלא בדקה כלל אלא שראתה תחלה דלעולם היא בחזקת טומאה עד שתפרש בטהרה כיון שראתה נדה וליכא מאן דפליגי, ומיהו בשני אפילו ראתה בשחרי' ופסקה טהרה באמצע יום ובין השמשו' לא הפרישה ואחר הימים מצאה טמא הרי זו בחזקת טהרה, ובברייתא תניא דרבי מטהר אפילו במצאה טהור בראשון.\par  ואין לפרש דלא פליגי אלא שני וה"ה לראשון דמכדי רבנן בתראי לטפויי מילתא אתו דתנא קמא רישא שביעי קאמר ואוסיפו אינהו אפילו בשנים אם בן לימרו ראשון וכן פי' רש"י ז"ל ומסתברא אפילו מצאה טהור כשבדקה אח"כ הרי זו חוששת דכיון שאין הפרישה של ראשון לנדה הפרשה גמורה אין בדיקה של עכשיו מחזיקה בטהרה למפרע ויש לדון בדבר אלא שהוא חומרא. }
\newchap{דף \hebrewnumeral{69}}
\twocol{הא דמקשי' \textbf{טועה דר' יוסי בר חנינא לרב דאמר סופן אף על פי שאין תתלתן.}  אף על גב דרב מכל מקום בדיקה דהפסקה בעי, שאני התם דכיון דילדה ודאי פסיקא אלא טועה דיום א' לא ידענא מאי סייעתא דאדרבא קשיא דהא התם לא הפסיקה טהרה שהרי אנן חוששין שמא עכשיו כשבאת לפנינו ראתה ולמה כל הני טבילות תפריש ותספור ותטבול לזבה, ואיכא למימר דהות משמע ליה מעיקרא דהכי קתני יום אחד טמא ראיתי והפסקתי ואיני יודעת כמה ראיתי והפסקתי דבלא הפסקה לא אפשר. }
\twocol{והא דקתני ברייתא \textbf{בין השמשות טמא, ראיתי ואמר רב ירמיה מדיפתי שבאת לפני' בין השמשות.}  נראה לי דהכי אמר ברייתא דקתני בין השמשות לאו לראיה אלא לביאה והכי קתני באה בין השמשות ואמרה טמא ראיתי, ואמר רב ירמיה מטבילין אותה אחד עשר טבילו' שהרי כל שבאה בין השמשות אפילו אמרה סתם יום אחד טמא ראיתי אחד עשר טבילות הן, אי נמי בין השמשו' טמא ראיתי דקתני שאם אמרה מבעוד יום ראיתי אין כאן י"א ולאו מילתא היא. }
\twocol{ הא דתנן \textbf{בית שמאי אומרים כל הנשים מתות נדות.}  אוקימנא בגמרא טעמייהו דבי' שמאי כדתני' בראשונ' היו מטבילין על גבי נדות מתות והיו נדות חיות מתביישות התקינו שיהו מטבילין על הכל, ולא למימרא דבית הלל פליגו אהך תקנתא אלא בית שמאי סברי גזרינן בכולהו לעשותן כנדות בין בחיים בין במיתה לטמא באבן מסמא ובית הלל סברי לענין הטבלת כלים עשאום כנדות, אבל לא לשאר טומאת דחיים ולא לאבן מסמא במיתה.\par  ויש מפרשים שחזרו בית הלל ותקנו כב"ש ואינו נכון כלל ואחרים העמידו ברייתא זו כב"ש וגם זה שבוש. }
\twocol{הא דתנן \textbf{ומקפת וקורא לה שם.}  פי' רש"י ז"ל קוצה לה חלה עד שלא תקרא שם ומנחתה בכלי ומקפת מקרבת הכלי אצל העיסה, ולא צריך לישך (אלא מנחתה בכלי ומקפת מקרבת הכל אצל העיסה).\par  ולא נראה לי שאין נקרא מוקף אלא בנשוך, אי נמי לרבי אליעזר בכלי אחד שהכלי מצרפן כדתנן שתי עיסות אחת טהורה ואחת טמאה נוטל כדי חלה מעיסה שלא הורמה חלתה ונותנין פחות מכביצה באמצע כדי ליטול מן המוקף שמע מינה נוגעין ממש בעינן למוקף. }
\twocol{והא דקתני \textbf{בית שמאי אומרים צריכה טבילה.}  לתרומה מפני שטבולת יום ארוך הוא והסיחה דעתה מן התרומה ואם ישראלית היא טובלת לביאת מקדש. פירש לפיכך טובלת כדי שתכנס לאתר כפרתה למקדש שהרי מחוסר כפורים שנכנס למקדש ענוש כרת כדאמרונן במס' מכות (דף ח ע"ב) טמא יהיה לרבות טבול יום עוד טמאתו בו לרבות מחוסר כפורים, וכן היא צריכה לטבול לנגיע' דתרומה, וב"ה אומרי' אינה צריכה אבל לקדשים מודי ב"ה דקיי"ל (חגיגה כא, א) האונן והמחוסר כפורים צריכין טבילה לקדש דמעלות דרבנן נינהו למדנו לדברי רש"י ז"ל שהחמירו באכילות קדשים יותר מביאת המקדש שיבנה במהרה בימינו אמן וכן יהי רצון. }

\end{document}
