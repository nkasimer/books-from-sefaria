\documentclass[12pt, openany]{book}
\usepackage[
paperheight=11in,
paperwidth=8.5in,
top=0.5in,
bottom=0.5in,
inner=0.7in,
outer=0.5in,
marginparsep=0.1in,
headsep=16pt
]{geometry}

\newcommand{\texttitle}{משנה ברכות}\usepackage{titlesec}
\usepackage{resources/unnumberedtotoc}

\usepackage{fancyhdr}
\pagestyle{fancy}
\fancyhf{}
\fancyhead[LO,RE]{\thepage}
\fancyhead[CO]{\chapname}
\fancyhead[CE]{\texttitle}

\usepackage{paracol}
\usepackage{anyfontsize}
\usepackage{ragged2e}
\usepackage{polyglossia}
\usepackage{multicol}
\usepackage{hyperref}

\setdefaultlanguage{hebrew}
\setotherlanguage{english}
\usepackage{fontspec}
\setmainfont{Frank Ruehl CLM}
\newfontfamily\englishfont{EB Garamond}

\newcommand{\sethebfont}{
\fontsize{10.5pt}{21.0pt} \selectfont
}

\newcommand{\hebeng}[2]{
	{\sethebfont #1\\}
	
	\begin{english}
		#2
	\end{english}
	\clearpage
}

\newcommand{\twocol}[1]{
	{\sethebfont \begin{multicols}{2}
			#1
	\end{multicols}}	
}

\newcommand{\textblock}[1]{
{\sethebfont #1\\}	
}

\setlength{\parskip}{8pt}

\newcommand{\chapname}{}
\newcommand{\sectname}{}

\newcommand{\newchap}[1]{
	\addcontentsline{toc}{chapter}{#1}
	\renewcommand{\chapname}{#1}
		\begin{center}
			\textbf{%
\fontsize{16pt}{16pt}\selectfont
				#1}
		\end{center}
}

\newcommand{\newsection}[1]{
	\addcontentsline{toc}{section}{#1}
	\renewcommand{\sectname}{#1}	
	\vspace{-\baselineskip}
	\begin{center}
		\textbf{%
\fontsize{16pt}{16pt}\selectfont
			#1}
	\end{center}
	\vspace{-\baselineskip}
	\nopagebreak
}

\begin{document}
\frontmatter
\pagenumbering{roman}

\title{\texttitle}

\author{}

\date{}

\maketitle

\begin{minipage}[b][\textheight][b]{\textwidth}\englishfont	
	\begin{english}
		\vfill
		The following book includes:
\begin{itemize}
\item[$\bullet$] Mishnah, ed. Romm, Vilna 1913
\item[$\bullet$] License: Public Domain
\item[$\bullet$] Source: \url{http://primo.nli.org.il/primo_library/libweb/action/dlDisplay.do?vid=NLI&docId=NNL_ALEPH001741739}
\end{itemize}
		It was retrieved from Sefaria on \today\space \texthebrew{(\Hebrewtoday)}.  It was typeset and formatted by Ktavi, using \LaTeX .
		\clearpage
		
	\end{english}
\end{minipage}


\tableofcontents

\clearpage
\mainmatter
\pagenumbering{arabic}

\newsection{פרק א}
\twocol{מאימתי קורין את שמע בערבית. משעה שהכהנים נכנסים לאכול בתרומתן. עד סוף האשמורה הראשונה דברי ר' אליעזר. וחכמים אומרים עד חצות. רבן גמליאל אומר עד שיעלה עמוד השחר. מעשה שבאו בניו מבית המשתה אמרו לו לא קרינו את שמע. אמר להם אם לא עלה עמוד השחר חייבין אתם לקרות ולא זו בלבד אלא כל מה שאמרו חכמים עד חצות מצותן עד שיעלה עמוד השחר. הקטר חלבים ואברים מצותן עד שיעלה עמוד השחר וכל הנאכלין ליום אחד מצותן עד שיעלה עמוד השחר. אם כן למה אמרו חכמים עד חצות כדי להרחיק אדם מן העבירה: 
\par מאימתי קורין את שמע בשחרית משיכיר בין תכלת ללבן רבי אליעזר אומר בין תכלת לכרתי וגומרה עד הנץ החמה. רבי יהושע אומר עד שלש שעות. שכן דרך בני מלכים לעמוד בשלש שעות הקורא מכאן ואילך לא הפסיד כאדם הקורא בתורה: }
\twocol{בית שמאי אומרים בערב כל אדם יטו ויקראו ובבוקר יעמדו שנאמר (דברים ו, ז) ובשכבך ובקומך ובית הלל אומרים כל אדם קורא כדרכו שנאמר (שם) ובלכתך בדרך אם כן למה נאמר ובשכבך ובקומך בשעה שבני אדם שוכבים ובשעה שבני אדם עומדים אמר ר' טרפון אני הייתי בא בדרך והטיתי לקרות כדברי בית שמאי וסכנתי בעצמי מפני הלסטים אמרו לו כדי היית לחוב בעצמך שעברת על דברי בית הלל: 
\par בשחר מברך שתים לפניה ואחת לאחריה ובערב שתי' לפניה ושתי' לאחריה אחת ארוכה ואחת קצרה מקום שאמרו להאריך אינו רשאי לקצר לקצר אינו רשאי להאריך. לחתום אינו רשאי שלא לחתום. ושלא לחתום אינו רשאי לחתום: }
\twocol{מזכירין יציאת מצרים בלילות. אמר ר' אלעזר בן עזריה הרי אני כבן שבעים שנה ולא זכיתי שתאמר יציאת מצרים בלילות עד שדרשה בן זומא שנאמר (דברים טז, ג) למען תזכור את יום צאתך מארץ מצרים כל ימי חייך ימי חייך הימים כל ימי חייך הלילות. וחכמים אומרים ימי חייך העולם הזה כל ימי חייך להביא לימות המשיח: }
\newsection{פרק ב}
\twocol{היה קורא בתורה והגיע זמן המקרא אם כיון לבו יצא ואם לאו לא יצא. בפרקים שואל מפני הכבוד ומשיב. ובאמצע שואל מפני היראה ומשיב דברי ר' מאיר. ר' יהודה אומר באמצע שואל מפני היראה ומשיב מפני הכבוד בפרקים שואל מפני הכבוד ומשיב שלום לכל אדם: 
\par אלו הן בין הפרקים בין ברכה ראשונה לשניה בין שניה לשמע ובין שמע לוהיה אם שמוע בין והיה אם שמוע לויאמר בין ויאמר לאמת ויציב. רבי יהודה אומר בין ויאמר לאמת ויציב לא יפסיק. אמר ר' יהושע בן קרחה למה קדמה שמע לוהיה אם שמוע אלא כדי שיקבל עליו עול מלכות שמים תחלה ואחר כך יקבל עליו עול מצות. והיה אם שמוע לויאמר שוהיה אם שמוע נוהג ביום ובלילה ויאמר אינו נוהג אלא ביום: }
\twocol{הקורא את שמע ולא השמיע לאזנו יצא. רבי יוסי אומר לא יצא. קרא ולא דקדק באותיותיה רבי יוסי אומר יצא. ר' יהודה אומר לא יצא. הקורא למפרע לא יצא. קרא וטעה יחזור למקום שטעה: 
\par האומנין קורין בראש האילן או בראש הנדבך מה שאינן רשאין לעשות כן בתפלה: }
\twocol{חתן פטור מקריאת שמע בלילה הראשון עד מוצאי שבת אם לא עשה מעשה. מעשה ברבן גמליאל שקרא בלילה הראשון שנשא אמרו לו תלמידיו לא למדתנו רבינו שחתן פטור מקריאת שמע בלילה הראשון אמר להם איני שומע לכם לבטל ממני מלכות שמים אפילו שעה אחת: 
\par רחץ לילה הראשון שמתה אשתו אמרו לו תלמידיו לא למדתנו רבינו שאבל אסור לרחוץ אמר להם איני כשאר כל אדם אסטניס אני: }
\twocol{וכשמת טבי עבדו קבל עליו תנחומין אמרו לו תלמידיו לא למדתנו רבינו שאין מקבלין תנחומין על העבדים אמר להם אין טבי עבדי כשאר כל העבדים כשר היה: 
\par חתן אם רצה לקרות קריאת שמע לילה הראשון קורא רבן שמעון בן גמליאל אומר לא כל הרוצה ליטול את השם יטול: }
\newsection{פרק ג}
\twocol{מי שמתו מוטל לפניו פטור מקריאת שמע מן התפלה ומן התפילין נושאי המטה וחלופיהן וחלופי חלופיהן את שלפני המטה ואת שלאחר המטה את שלמטה צורך בהן פטורין ואת שאין למטה צורך בהן חייבין. אלו ואלו פטורים מן התפלה: 
\par קברו את המת וחזרו אם יכולין להתחיל ולגמור עד שלא יגיעו לשורה יתחילו ואם לאו לא יתחילו העומדים בשורה הפנימים פטורים והחיצונים חייבין: }
\twocol{נשים ועבדים וקטנים פטורין מקריאת שמע ומן התפילין וחייבין בתפלה ובמזוזה ובברכת המזון: 
\par בעל קרי מהרהר בלבו ואינו מברך לא לפניה ולא לאחריה. ועל המזון מברך לאחריו ואינו מברך לפניו. ר' יהודה אומר מברך לפניהם ולאחריהם: }
\twocol{היה עומד בתפלה ונזכר שהוא בעל קרי לא יפסיק אלא יקצר. ירד לטבול אם יכול לעלות ולהתכסות ולקרות עד שלא תנץ החמה יעלה ויתכסה ויקרא ואם לאו יתכסה במים ויקרא. אבל לא יתכסה לא במים הרעים ולא במי המשרה עד שיטיל לתוכן מים. וכמה ירחיק מהם ומן הצואה ארבע אמות: 
\par זב שראה קרי. ונדה שפלטה שכבת זרע. והמשמשת שראתה נדה. צריכין טבילה. ורבי יהודה פוטר: }
\newsection{פרק ד}
\twocol{תפלת השחר עד חצות. רבי יהודה אומר עד ארבע שעות תפלת המנחה עד הערב. רבי יהודה אומר עד פלג המנחה. תפלת הערב אין לה קבע ושל מוספין כל היום רבי יהודה אומר עד שבע שעות: 
\par רבי נחוניה בן הקנה היה מתפלל בכניסתו לבית המדרש וביציאתו תפלה קצרה אמרו לו מה מקום לתפלה זו. אמר להם בכניסתי אני מתפלל שלא תארע תקלה על ידי וביציאתי אני נותן הודיה על חלקי: }
\twocol{רבן גמליאל אומר בכל יום מתפלל אדם שמונה עשרה. רבי יהושע אומר מעין שמונה עשרה. ר' עקיבא אומר אם שגורה תפלתו בפיו יתפלל שמונה עשרה ואם לאו מעין שמונה עשרה: 
\par רבי אליעזר אומר העושה תפלתו קבע אין תפלתו תחנונים רבי יהושע אומר המהלך במקום סכנה מתפלל תפלה קצרה אומר הושע השם את עמך את שארית ישראל בכל פרשת העבור יהיו צרכיהם לפניך ברוך אתה ה' שומע תפלה: }
\twocol{היה רוכב על החמור ירד. ואם אינו יכול לירד יחזיר את פניו. ואם אינו יכול להחזיר את פניו יכוין את לבו כנגד בית קדש הקדשים: 
\par היה יושב בספינה או בקרון או באסדא יכוין את לבו כנגד בית קדש הקדשים: }
\twocol{רבי אלעזר בן עזריה אומר אין תפלת המוספין אלא בחבר עיר וחכמים אומרים בחבר עיר ושלא בחבר עיר רבי יהודה אומר משמו כל מקום שיש חבר עיר היחיד פטור מתפלת המוספין: }
\newsection{פרק ה}
\twocol{אין עומדין להתפלל אלא מתוך כובד ראש. חסידים הראשונים היו שוהים שעה אחת ומתפללים כדי שיכונו את לבם למקום. אפילו המלך שואל בשלומו לא ישיבנו. ואפילו נחש כרוך על עקבו לא יפסיק: 
\par מזכירין גבורות גשמים בתחיית המתים. ושואלין הגשמים בברכת השנים. והבדלה בחונן הדעת. ר' עקיבא אומר אומרה ברכה רביעית בפני עצמה רבי אליעזר אומר בהודאה: }
\twocol{האומר על קן צפור יגיעו רחמיך ועל טוב יזכר שמך. מודים מודים משתקין אותו. העובר לפני התיבה וטעה יעבר אחר תחתיו ולא יהא סרבן באותה שעה מנין הוא מתחיל מתחלת הברכה שטעה בה: 
\par העובר לפני התיבה לא יענה אחר הכהנים אמן מפני הטירוף ואם אין שם כהן אלא הוא לא ישא את כפיו. ואם הבטחתו שהוא נושא את כפיו וחוזר לתפלתו רשאי: }
\twocol{המתפלל וטעה סימן רע לו. ואם שליח צבור הוא סימן רע לשולחיו. מפני ששלוחו של אדם כמותו. אמרו עליו על רבי חנינא בן דוסא כשהיה מתפלל על החולים ואומר זה חי וזה מת. אמרו לו מנין אתה יודע אמר להם אם שגורה תפלתי בפי יודע אני שהוא מקובל ואם לאו יודע אני שהוא מטורף: }
\newsection{פרק ו}
\twocol{כיצד מברכין על הפירות. על פירות האילן אומר בורא פרי העץ. חוץ מן היין. שעל היין אומר בורא פרי הגפן. ועל פירות הארץ אומר בורא פרי האדמה חוץ מן הפת שעל הפת הוא אומר המוציא לחם מן הארץ. ועל הירקות אומר בורא פרי האדמה. רבי יהודה אומר בורא מיני דשאים: 
\par ברך על פירות האילן בורא פרי האדמה יצא. ועל פירות הארץ בורא פרי העץ לא יצא. על כולם אם אמר שהכל נהיה יצא: }
\twocol{על דבר שאין גדולו מן הארץ אומר שהכל. על החומץ ועל הנובלות ועל הגובאי אומר שהכל. על החלב ועל הגבינה ועל הביצים אומר שהכל. ר' יהודה אומר כל שהוא מין קללה אין מברכין עליו: 
\par היו לפניו מינים הרבה. ר' יהודה אומר אם יש ביניהם ממין שבעה מברך עליו. וחכמים אומרים מברך על איזה מהם שירצה: }
\twocol{ברך על היין שלפני המזון פטר את היין שלאחר המזון ברך על הפרפרת שלפני המזון פטר את הפרפרת שלאחר המזון. ברך על הפת פטר את הפרפרת. על הפרפרת לא פטר את הפת. בית שמאי אומרים אף לא מעשה קדרה: 
\par היו יושבין לאכול כל אחד ואחד מברך לעצמו. הסיבו אחד מברך לכולן. בא להם יין בתוך המזון כל אחד ואחד מברך לעצמו. לאחר המזון אחד מברך לכולם. והוא אומר על המוגמר אף על פי שאין מביאין את המוגמר אלא לאחר הסעודה: }
\twocol{הביאו לפניו מליח בתחלה ופת עמו מברך על המליח ופוטר את הפת שהפת טפלה לו זה הכלל כל שהוא עיקר ועמו טפלה. מברך על העיקר ופוטר את הטפלה: 
\par אכל תאנים וענבים ורמונים מברך אחריהן שלש ברכות דברי רבן גמליאל. וחכמים אומרים ברכה אחת מעין שלש רבי עקיבא אומר אפילו אכל שלק והוא מזונו מברך אחריו שלש ברכות. השותה מים לצמאו אומר שהכל נהיה בדברו. ר' טרפון אומר בורא נפשות רבות: }
\newsection{פרק ז}
\twocol{שלשה שאכלו כאחת חייבין לזמן אכל דמאי ומעשר ראשון שנטלה תרומתו ומעשר שני והקדש שנפדו והשמש שאכל כזית והכותי מזמנין עליהם. אבל אכל טבל ומעשר ראשון שלא נטלה תרומתו ומעשר שני והקדש שלא נפדו והשמש שאכל פחות מכזית והנכרי אין מזמנין עליהם: 
\par נשים ועבדים וקטנים אין מזמנין עליהם. עד כמה מזמנין עד כזית ר' יהודה אומר עד כביצה: }
\twocol{כיצד מזמנין בשלשה אומר נברך. בשלשה והוא אומר ברכו. בעשרה אומר נברך לאלהינו. בעשרה והוא אומר ברכו. אחד עשרה ואחד עשרה רבוא. במאה אומר נברך לה' אלהינו. במאה והוא אומר ברכו. באלף אומר נברך לה' אלהינו אלהי ישראל. באלף והוא אומר ברכו. ברבוא אומר נברך לה' אלהינו אלהי ישראל אלהי הצבאות יושב הכרובים על המזון שאכלנו. ברבוא והוא אומר ברכו. כענין שהוא מברך כך עונין אחריו ברוך ה' אלהינו אלהי ישראל אלהי הצבאות יושב הכרובים על המזון שאכלנו. ר' יוסי הגלילי אומר לפי רוב הקהל הן מברכין שנאמר (תהלים סח, כז) במקהלות ברכו אלהים ה' ממקור ישראל. אמר רבי עקיבא מה מצינו בבית הכנסת אחד מרובין ואחד מועטין אומר ברכו את ה'. רבי ישמעאל אומר ברכו את ה' המבורך: 
\par שלשה שאכלו כאחד אינן רשאין ליחלק וכן ארבעה וכן חמשה. ששה נחלקין עד עשרה ועשרה אינן נחלקין עד שיהיו עשרים: }
\twocol{שתי חבורות שהיו אוכלות בבית אחד בזמן שמקצתן רואין אלו את אלו הרי אלו מצטרפים לזמון ואם לאו אלו מזמנין לעצמן ואלו מזמנין לעצמן אין מברכין על היין עד שיתן לתוכו מים דברי רבי אליעזר. וחכמים אומרים מברכין: }
\newsection{פרק ח}
\twocol{אלו דברים שבין בית שמאי ובית הלל בסעודה. בית שמאי אומרים מברך על היום ואחר כך מברך על היין ובית הלל אומרים מברך על היין ואחר כך מברך על היום: 
\par בית שמאי אומרים נוטלין לידים ואחר כך מוזגין את הכוס ובית הלל אומרים מוזגין את הכוס ואחר כך נוטלין לידים: }
\twocol{בית שמאי אומרים מקנח ידיו במפה ומניחה על השלחן ובית הלל אומרים על הכסת: 
\par בית שמאי אומרים מכבדין את הבית ואחר כך נוטלין לידים ובית הלל אומרים נוטלין לידים ואחר כך מכבדין את הבית: }
\twocol{בית שמאי אומרים נר ומזון ובשמים והבדלה. ובית הלל אומרים נר ובשמים ומזון והבדלה. בית שמאי אומרים שברא מאור האש ובית הלל אומרים בורא מאורי האש: 
\par אין מברכין לא על הנר ולא על הבשמים של עובדי כוכבים ולא על הנר ולא על הבשמים של מתים ולא על הנר ולא על הבשמים שלפני עבודת כוכבים. אין מברכין על הנר עד שיאותו לאורו: }
\twocol{מי שאכל ושכח ולא ברך בית שמאי אומרים יחזור למקומו ויברך ובית הלל אומרים יברך במקום שנזכר. עד אימתי הוא מברך עד כדי שיתעכל המזון שבמעיו: 
\par בא להם יין לאחר המזון ואין שם אלא אותו הכוס. בית שמאי אומרים מברך על היין ואחר כך מברך על המזון. ובית הלל אומרים מברך על המזון ואחר כך מברך על היין. עונין אמן אחר ישראל המברך ואין עונין אמן אחר הכותי המברך עד שישמע כל הברכה: }
\newsection{פרק ט}
\twocol{הרואה מקום שנעשו בו ניסים לישראל אומר ברוך שעשה ניסים לאבותינו במקום הזה. מקום שנעקרה ממנו עבודת כוכבים אומר ברוך שעקר עבודת כוכבים מארצנו: 
\par על הזיקין ועל הזועות ועל הברקים ועל הרעמים ועל הרוחות אומר ברוך שכחו וגבורתו מלא עולם על ההרים ועל הגבעות ועל הימים ועל הנהרות ועל המדברות אומר ברוך עושה מעשה בראשית. ר' יהודה אומר הרואה את הים הגדול אומר ברוך שעשה את הים הגדול בזמן שרואה אותו לפרקים. על הגשמים ועל הבשורות הטובות אומר ברוך הטוב והמטיב ועל שמועות רעות אומר ברוך דיין האמת: }
\twocol{בנה בית חדש וקנה כלים חדשים אומר ברוך שהחיינו. מברך על הרעה מעין הטובה ועל הטובה מעין הרעה. הצועק לשעבר הרי זו תפלת שוא כיצד היתה אשתו מעוברת ואמר יהי רצון שתלד אשתי זכר הרי זו תפלת שוא. היה בא בדרך ושמע קול צוחה בעיר ואמר יהי רצון שלא יהיו אלו בני ביתי הרי זו תפלת שוא: 
\par הנכנס לכרך מתפלל שתים. אחת בכניסתו. ואחת ביציאתו. בן עזאי אומר ארבע. שתים בכניסתו. ושתים ביציאתו. ונותן הודאה לשעבר. וצועק לעתיד לבא: }
\twocol{חייב אדם לברך על הרעה כשם שהוא מברך על הטובה שנאמר (דברים ו, ה) ואהבת את ה' אלהיך בכל לבבך ובכל נפשך ובכל מאדך. בכל לבבך בשני יצריך ביצר טוב וביצר רע. ובכל נפשך אפילו הוא נוטל את נפשך. ובכל מאדך בכל ממונך. דבר אחר בכל מאדך בכל מדה ומדה שהוא מודד לך הוי מודה לו במאד מאד. לא יקל אדם את ראשו כנגד שער המזרח שהוא מכוון כנגד בית קדשי הקדשים. לא יכנס להר הבית במקלו ובמנעלו ובפונדתו ובאבק שעל רגליו ולא יעשנו קפנדריא ורקיקה מקל וחומר כל חותמי ברכות שהיו במקדש היו אומרים מן העולם משקלקלו המינין ואמרו אין עולם אלא אחד. התקינו שיהו אומרים מן העולם ועד העולם. והתקינו שיהא אדם שואל את שלום חברו בשם שנאמר (רות ב, ד) והנה בעז בא מבית לחם ויאמר לקוצרים ה' עמכם ויאמרו לו יברכך ה' ואומר (שופטים ו, יב) ה' עמך גבור החיל. ואומר (משלי כג, כב) אל תבוז כי זקנה אמך ואומר (תהלים קיט, קכו) עת לעשות לה' הפרו תורתך. ר' נתן אומר הפרו תורתך עת לעשות לה': }

\end{document}
