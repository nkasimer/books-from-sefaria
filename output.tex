\documentclass[12pt, openany]{book}
\usepackage[
paperheight=11in,
paperwidth=8.5in,
top=0.5in,
bottom=0.5in,
inner=0.7in,
outer=0.5in,
marginparsep=0.1in,
headsep=16pt
]{geometry}

\newcommand{\texttitle}{חידושי רמב"ן על נדה}\usepackage{titlesec}
\usepackage{resources/unnumberedtotoc}

\usepackage{fancyhdr}
\pagestyle{fancy}
\fancyhf{}
\fancyhead[LO,RE]{\thepage}
\fancyhead[CO]{\chapname}
\fancyhead[CE]{\texttitle}

\usepackage{paracol}
\usepackage{anyfontsize}
\usepackage{ragged2e}
\usepackage{polyglossia}
\usepackage{multicol}
\usepackage{hyperref}

\setdefaultlanguage{hebrew}
\setotherlanguage{english}
\usepackage{fontspec}
\setmainfont{Frank Ruehl CLM}
\newfontfamily\englishfont{EB Garamond}

\newcommand{\sethebfont}{
\fontsize{10.5pt}{21.0pt} \selectfont
}

\newcommand{\hebeng}[2]{
	{\sethebfont #1\\}
	
	\begin{english}
		#2
	\end{english}
	\clearpage
}

\newcommand{\twocol}[1]{
	{\sethebfont \begin{multicols}{2}
			#1
	\end{multicols}}	
}

\newcommand{\textblock}[1]{
{\sethebfont #1\\}	
}

\setlength{\parskip}{8pt}

\newcommand{\chapname}{}

\newcommand{\newchap}[1]{
	\addcontentsline{toc}{chapter}{#1}
	\renewcommand{\chapname}{#1}
		\begin{center}
			\textbf{%
\fontsize{16pt}{16pt}\selectfont
				#1}
		\end{center}
}

\newcommand{\newsection}[1]{\addsec{
	\centering
\fontsize{16pt}{16pt}\selectfont
	#1
}}

\begin{document}
\frontmatter
\pagenumbering{roman}

\title{\texttitle}

\author{}

\date{}

\maketitle

\begin{minipage}[b][\textheight][b]{\textwidth}\englishfont	
	\begin{english}
		\vfill
		The following book includes:
\begin{itemize}
\item Chiddushei HaRamban, Jerusalem 1928\textendash 29
\item License: Public Domain
\item Source: \url{http://primo.nli.org.il/primo_library/libweb/action/dlDisplay.do?vid=NLI&docId=NNL_ALEPH001294828}
\end{itemize}
		It was retrieved from Sefaria on \today\space \texthebrew{(\Hebrewtoday)}.  It was typeset and formatted by Ktavi, using \LaTeX .
		\clearpage
		
	\end{english}
\end{minipage}


\tableofcontents

\clearpage
\mainmatter
\pagenumbering{arabic}

\newchap{פרק \hebrewnumeral{1}\quad שמאי אומר}
\newsection{דף \hebrewnumeral{2}}
\twocol{גמרא \textbf{מאי טעמא דשמאי.}  פרושי קא מפרש מתניתין ואזיל דאלו טעמיה דשמאי דכולי עלמא אית להו אלא שהחמירו לתלות בתרומה וקדשים כדפרישית, ואוקים משום דהעמד אשה על חזקתה ובחזקת טהורה עומדת שהרי טבלה לנדתה ובדוקה היא משעה שפסקה לנדתה הראשון.\par  והלל כי אמרי העמד דבר על חזקתו ואפילו לתרומה וקדשים היכא דלית ליה ריעותא מגופיה הך אשה כיון דמגופה קא חזיא לא אמרי' אוקמה אחזקה אלא חיישינן הילכך בחולין אף על גב דאיכא למיחש במילתא כיון דלא מכרעא מילתא דטומאה מוקמינן טהרות אחזקתייהו ג) ד"א מאפישי טומאה לא מפשינן וגבי תרומה וקדשים עבד בהו רבנן מעלה וכיון דליכא חזקה גמורה תולין.\par וי"מ דאף על גב דליכא חזקה מיהו ה"ל ספק טומאה ומסוטה גמרינן מה התם מכאן ולהבא ולא למפרע אף כל ספק טומאה לא מטמינן למפרע אלא משום מעלה דקדשים דתולין וברשות הרבים טהור לגמרי דגמרינן מסוטה בק"ו. 
\par ואקשינן \textbf{מ"ש ממקוה לשמאי קשיא למפרע ולהלל קשיא טומאת ודאי דאלו מעת לעת שבנדה תולין וכו'.}  פי' וכיון דתולין אלמא ספיקא בעלמא הוא וה"ה דקשיא ברשות הרבים ודבר שאין בו דעת לישאל נמי אלא חדא מספיקא נקט משום דתניא לקמן בהדיא וזה וזה תולין.\par  ואי קשיא לך מ"ש אשה מהא דתנן לקמן נגע בא' בלילה וכו' שחכמים מטמאים טומאה ודאי שכל הטומאות כשעת מציאתן וכאן נמי הרי דם לפניך. לא קשיא דשאני אשה דבחזקת טהרה עומדת שהרי בדוקה היא ואף על גב דשכיחי בה דמים מכל מקום כל שהפסיקה וטהרה בחזקתה זו היא אבל אדם זה אינו עומד בחזקת חי תדע דתניא בתוספתא ומודים חכמים לר"מ כשראוהו חי אלמא דבכה"ג בחזקת חי הוא לא מפקינן ליה מחזקתיה אף על פי שנמצא מת בשעת מציאתן ובמקום מציאתן אבל א"ל גבי אשה כיון דספק ביאה הוא ספק הוה ספק לא הוה תולין להקל. ולאו מילתא היא דהא קופה באותה זוית עצמה טהרות הראשונו' (טהורות) [טמאות] לדברי הכל. ואע"פ שהוא דומה לאשה בזה דספק הוה ספק לא הוה הוא אלא משום חזקה ראשונה היא דליתא בקופה וכדבעינא למימר קמן. מ"ש ממקוה לשמאי קשיא למפרע ולהלל קשיא טומאת ודאי דאלו מעת לעת שבנדה תולין וכו'. פי' וכיון דתולין אלמא ספיקא בעלמא הוא וה"ה דקשיא ברשות הרבים ודבר שאין בו דעת לישאל נמי אלא חדא מספיקא נקט משום דתניא לקמן בהדיא וזה וזה תולין.\par  ואי קשיא לך מ"ש אשה מהא דתנן לקמן נגע בא' בלילה וכו' שחכמים מטמאים טומאה ודאי שכל הטומאות כשעת מציאתן וכאן נמי הרי דם לפניך. לא קשיא דשאני אשה דבחזקת טהרה עומדת שהרי בדוקה היא ואף על גב דשכיחי בה דמים מכל מקום כל שהפסיקה וטהרה בחזקתה זו היא אבל אדם זה אינו עומד בחזקת חי תדע דתניא בתוספתא ומודים חכמים לר"מ כשראוהו חי אלמא דבכה"ג בחזקת חי הוא לא מפקינן ליה מחזקתיה אף על פי שנמצא מת בשעת מציאתן ובמקום מציאתן אבל א"ל גבי אשה כיון דספק ביאה הוא ספק הוה ספק לא הוה תולין להקל. ולאו מילתא היא דהא קופה באותה זוית עצמה טהרות הראשונו' (טהורות) [טמאות] לדברי הכל. ואע"פ שהוא דומה לאשה בזה דספק הוה ספק לא הוה הוא אלא משום חזקה ראשונה היא דליתא בקופה וכדבעינא למימר קמן. }
\newsection{דף \hebrewnumeral{3}}
\twocol{\textbf{משמשת במוך מא"ל.}  פירש"י ז"ל ג' נשים ולא דוקא דהא ארבע נשים דיין שעתן במתניתין אלא כל אשה שמשמשת במוך. 
\par \textbf{אי אתה מודה בקופה שנשתמשו בה טהרות וכו'.}  איכא למידק להלל גופיה קשיא טומאה ודאי דאלו קופה טומאה ודאי דקתני טמאות וכדמוכחא נמי שמעתין לקמן ואלו מעל"ע תולין. א"ל להלל גופיה זו יש לה שולים וזו אין לה שולים אלא ה"ק ליה כיון דכשיש לה שולי' טמאות ודאי דין הוא לתלות באשה מפני שהיא כמי שיש לה אוגנים. ואפילו לחזקיה דאמר התם טהורות שאני פירי דלא שרקי וקפיד עלייהו.\par וכן אתה מפרש ללשון שאמרו בקופה שאינה בדוקה א"נ שהיא מכוסה שלא בא הלל להשוות אשה לשאינה בדוקה ולשאילה מכוסה אלא שמאחר שבאלו טמאות ודאי באשה היה לנו לתלות מפני שהיא דומה במקצת לשאינה בדוקה ואינה מכוסה משום דשכיחי בה דמים. }
\newsection{דף \hebrewnumeral{4}}
\twocol{והא דאמרינן \textbf{ואב"א כי מודו שמאי והלל בזויות דקופה.}  ה"פ: לעולם בשאינה בדוקה מודו ובזויות קופה דאיכא תרתי לריעותא אבל בבדוקה אע"פ שבזויות קופה נמצא אינן טמאות דה"ל כאותה שאמרו בתוספתא ומודים חכמים לר"מ בשראוהו חי ועד כאן נמי לא מטמינן בשרץ שנמצא במבוי אלא משום דאיכא שרצים דיליה ושרצים דאתו ליה מעלמא דה"ל כתרתי לריעותא הא לאו הכי לא מטמינן ביה למפרע כלל ל"ש אותה זויות ול"ש זויות אחרת ובקופה לא שכיחי ביה שרצים כלל אלא ודאי בקופה שאינה בדוקה מודו שמאי והלל לכולהו לישני דאי לא א) אמרינן אוקמה אחזקיה ואפי' לתלות ב) וכ"ש דלא מודו בטומאה ודאי וה"נ מוכחא רישא דשמעתין. 
\par \textbf{שאני אומר אדם טהור נכנס לשם ונטלה.}  איכא למידק וליחוש נמי לאדם טמא כדאמרינן בפסחים פ"ק קרדום שאבד בבית או שהניחו בזויות זו ונמצא בזויות אחרת הבית טמא שאני אומר אדם טמא נכנס לשם ונטלו וכ"ש הכא דאיכא למיתלי בתרתי לריעותא באדם טמא ונפלה על גבי מדף. ובפ"ק דשחיטת חולין נמי אמרינן צלוחי' שהניחה מגולה ובא ומצאה מכוסה טמאה שאני אומר אדם טמא נכנס לשם וכסה.\par  ויש מתרצים דכיון דאיכא טומאת מדף תחתיה נראין הדברים שאין אדם נוטלה מלמעלה והניחה למטה אלא כדי שלא תפול ותטמא עשה כן ואלמלא שהוא טהור איך הוא מטמא בידים, ואין זה לשון מחוור.\par אבל ר"ת ז"ל פירש בספר הישר דהתם כלים נינהו וספק כלים הנמצאים טמאים כדאמרינן בפ"ק דשבת על ששה ספקות שורפין את התרומה על ספק כלים הנמצאים דשוינהו רבנן לרובא דעלמא טמאים לגבי כלים הנמצאים אבל ספק אוכלים לא קא חשיב אלמא לא גזרו בהו והוו להו רובא דעלמא טהורים לגבייהו הילכך ליכא למיחש גבי מדף אלא לאדם טהור ולנפילה והוה לו ספק טומאה בדבר שאין בו דעת לישאל וספיקו טהור.\par  ואיני יודע למה גזרו על כלים ולא גזרו על אוכלים הנמצאים שאם נאמר מפני שחששו לפסידתן מפני שאין להם טהרה במקוה אף כלי חרס כגון צלוחית דחולין אין להם טהרה במקוה, ושמא שכלים נמצאים בדרך נפילה ובכל מקום ואין אוכלים נמצאים בדרכים לפי שהן נמאסין מ"ה גזרו על הכלים הנופלים אפילו בבית כל שלא ידענו דרך הנחתו וטימאוה אבל אוכלין שאין מצויין אלא בבית ורוב מציאה דבית דרך הנחה היא ובטהורין הוי לא גזרו עליהם שאלו פירות הנמצאין בשוק דרך נפילה לא הוצרכו לגזור עליהן דרובן נפסלין הן מתורת אוכלים. }
\newsection{דף \hebrewnumeral{5}}
\twocol{\textbf{השתא מעת לעת ממעטה מפקידה לפקידה מיבעיא.}  פי' לדברי חכמים לעולם מפקידה לפקידה זמנה מועט מעת לעת שכבר מיעט מעל"ע ע"י הפקידה וכיון שעד זה ממעט על יד מעת לעת אם לא בדקה מאתמול כ"ש שדינו למעט על יד הפקידה אם בדקה עצמה היום בשעה ראשונה ולרביעית שמשה דכיון שהזמן ביניהם מועט אין לחוש כ"כ בשעת פקידה לומר עם סלוק ידיה ראתה. ופריק סד"א עד זה לא תמעט אלא על יד מעל"ע אע"פ שהחששא שלו יותר קרובה מפני הפסד טהרות הקל. אבל על יד פקידה לא ימעט דליחוש שמא תחפנו שכבת זרע קמ"ל. 
\par \textbf{ומה כלי חרס המוקף צמיד פתיל וכו'.}  הקשו בתוספות ונימא דיו לבא מן הדין להיות כנדון מהיכא מייתית ליה מכלי חרס מה כלי חרס אינו מטמא אדם לטמא בגדים אף משכב ומושב לא יטמא אדם לטמא בגדים. ולאו קושיא היא דאנן הכי קאמרינן ומה כלי חרס שטומאתו מועט' שהוא ניצל באה' המת גזרו על מעת לעת שלו כנד' עצמה משכבות ומושבות שטומאתן מרובה לכ"ש שנעשו מעת לעת שבנדה.\par  ועוד הקשו דנימא פכים קטנים יוכיח שטמאים במת ואין מטמאים במעל"ע שבנדה כדאמרינן בבבא קמא ופירש רש"י ז"ל שהוא של חרס ואי אפשר ליגע בתוכן ואעפ"י שאפשר בהיסט להכי אפקיה רחמנא להיסט בלשון נגיעה לומר שכל שאי אפשר להטמאות בנגיעה אינו מטמא בהיסט, גם זו אינה קושיא דמה לפכין קטנים שהן טהורין בנדה עצמה תאמר במשכבו' ומושבות דכיון שמטמאין בנדה עצמה עשו מעת לעת כמוה דאשכח' בכלי חרס מוקף צמיד פתיל כ"ש לדעת הגאונים שהן מפרשים פכין קטנים שאינן ראויין לישיבה וטהורין במדרס הזב אבל מן ההיסט אין לך ניצל מהן ולא ממגע תוך כגון בשערו רוקו ומשקה הזב והזבה. }
\newsection{דף \hebrewnumeral{6}}
\twocol{הא דאקשינן \textbf{א"ה ליתנייה גבי מעלות.}  ומפרקינן כי קתני היכא דאית ליה דררא דטומאה היכא דלית ליה דררא דטומאה לא קתני. קשיא עלה והא קתני התם דלית בה דררא דטומאה כדאמר התם בגמרא פרק חומר בקודש (דף כ"א ע"ב) חמש קמייתא דאית להו דררא דטומאה דאורייתא גזרו בהו רבנן בין לקדש בין לחולין שנעשו על טהרות קודש חמש בתרייתא דלית בהו דררא דטומאה מדאורייתא לקדש גזרו בהו רבנן לחולין שנעשו על טהרות הקדש לא גזרו בהו רבנן.\par  וי"ל דהתם דאורייתא לית להו אבל דררא דטומאה דרבנן אית להו הכא אפילו דררא דעלמא מדרבנן ליכא דכל שהוא חששא בעלמא למפרע לאו דררא היא כלל אלא כענין קנסא משום דלא בדקה הפסידוה עונה.\par  וי"מ דהכא הכי פרכינן ליתנייה גבי מעלות קמייתא דאינון בין לקדש בין לחולין שנעשו על טהרת הקדש דאלו בהדי בתריית' כיון דליתנהו אלא לקדש לא מצי למיתנייה דהא מעת לעת שבנדה לחולין שנעשו על טהרת הקדש נמי איתא כדאמר בשילהי שמעתין ולהכי מפרקינן כי קתני בהנהו היכא דאית ליה דררא דטומאה אבל היכא דלי ליה דררא דטומאה ואפ"ה החמירו בהו לא קתני.\par והלשון משובש הוא לדעתי דההוא דאמרינן בשלהי שמעתין לחולין שנעשו על טהרת הקדש לתרוצא לברייתא דקתני לקדש אבל לא לתרומה איתמר אבל השתא להאי לישנא לקדש ולא לתרומה ולא לחולין שנעשו על על טהרת הקדש קאמרינן דאי לת"ה לא הוו צריכין בדר' לתרוצא כדעולא דהא איכא אוכלין חוליהן בטהרת הקדש בימיו ממש אלא ודאי צ"ל להאי לישנא דאף לחולין שנעשו על טהרת קודש לא גזרו במעל"ע שבנדה לכך צריך לשנוי כדעולא כיון שהיו עושין על טהרת הקרש על מנת שהיו מתנסכין ממש על גבי מזבח היינו קדשי מזבח גמורין. 
\par  הא דאמר רב ששת בריה דרב אידי \textbf{כי קתני מידי דתלי במעשה.}  פירש אליבא דר"ש קסבר האי תנא בוגרת מותרת לכהן גדול א"נ נפקא מינה לכתובה ולא לכהן תניא ומוכת עץ מילתא דתלי במעשה הוא והאי דקתני כל זמן שלא נבעלה לאו דוקא אלא שלא נטלו בתוליה בין בעץ בין באדם. א"נ קסבר מוכת עץ מותרת לכהן גדול וכתובתה מאתים ומחלוקת היא ביבמות ובכתובות. וכן הא דאמר נ"מ לנחל איתן סבר לה כר' יאשיה אשר לא יעבד בו לשעבר ואיתא בפלוגתא בפ' עגלה ערופה. }
\newsection{דף \hebrewnumeral{9}}
\twocol{\textbf{ראתה ואח"כ הוכר עוברה מהו.}  יש להקשות והלא כל מדות חכמים כך הם במ' סאה הוא טובל בחסר קרטוב אינו יכול לטבול אף כאן מכיון שנתנו שיעור לדבר בהכרת העובר ראתה ולא הוכר פשיטא שמטמאה מעת לעת אע"פ שאח"כ הוכר בסמוך.\par ויש לפרש ראתה ואח"כ הוכר עוברה בו ביום ששלמו לה שלשה חדשים מי אמרינן כי מצפרא נמי הויא היכירא ואנן הוא דלא בקאינן או דילמא גבול יש לה וא"ל מידי הוא טעמא אלא משום דראשה כבד עליה בעידנא דחזאי אין ראשה כבד עליה בתמיה הילכך דיה שעתא.\par  וזה הלשון אינו נכון מפני שהיה להם לפרש ובו ביום הוכר עוברה ולימא נמי כיון דבו ביום חזיא לא מטמיא.\par  ויש לפרש אותה כפשוטה ור' ירמיה הכי בעי מיני' גבול שנתנו לה חכמים משיהא ראשה ואיבריה כבדין והיינו משעת הכרת העובר וקרוב לו מלפניו כשהיא מרגשת בעצמה או דילמא גבול שנתנו לה הכרת העובר ממש הוא וקודם לכן אפילו בסמוך אינו מטהרתה וא"ל אף איבריה אינן כבידין עליה אלא משעת הכרת העובר ממש שאין הולד חי ומכביד עליה אלא מזמן זה ואילך וזה הלשון עיקר. 
\par  הא דתניא \textbf{תנוקות שלא הגיע זמנה לראות וכו'.}  יפה פי' רש"י ז"ל דהיינו טעמא דלא מטמיא מעת לעת אלא בשלישית לרבי וברביעית לרשב"ג מפני שכל אשה שלא הוחזקה כבר ברואה אינה מטמאה מעת לעת דטעמא דמעת לעת כעין קנסא דרבנן הוא כדאמרו חכמים תקנו להן לבנות ישראל שיהיו בודקות עצמן שחרית וערבית וזו הואיל ולא בדקה הפסידה עונה יתירה הילכך כל שאינה צריכה לבדוק עצמה כלל אינה בכלל מעת לעת שבנדה ותנוקות שלא הגיעה זמנה לראות הרי הן בחזקת טהרה כדתניא לקמן ואין הנשים בודקות אותן הילכך פעם ראשונה ושניה שעדיין לא הוחזקה לראות ולא היתה בכלל תקנה לבדוק שחרית וערבית דיה שעתה. וכיון שראתה בשניה הוחזק ברואה לדברי רבי הילכך בג' מטמאה מעת לעת שהרי היתה צריכה לבדוק שחרית וערבית, ואע"פ שעדיין לא הגיע לכלל שנותיה, וכיון שלא בדקה הפסידה עונה יתירה ושהגיע זמנה לראות כיון שצריכה בדין היה לגזור עליה אפילו בראשונה אלא שהיא קולא לדבריהם. עברו עליה ג' עונות חזרה לכלל תנוקות שלא הגיע זמנה עד שתראה שתים ותהא מוחזקת לראות לדברי רבי דשוב צריכה בדיקה ומטמאה מעת לעת.\par  והא דאמרינן לקמן (דף י' ע"א) בין שניה לשלישית כיון דלא אתחזק בדם כתמה נמי לא מטמינן לאו אליבא דהך ברייתא דרבי אלא אליבא דהילכתא כרשב"ג. וכן פסק הר"ם ז"ל דקטנה כתמה טהור עד שתראה דם ג' וסתות.\par  וחזקיה סבר כיון (דחזיא) [דאלו חזיא] הרי היא כשאר כל הנשים אח"כ כתמה מחזיקה וטמא דבכל (מראיה שניה) [מראות משניה] ואילך הוחזקה.\par  אבל רש"י ז"ל פי' אליבא דברייתא [דרבי] ומאי כיון דלא איתחזק בדם שעדיין לא הוחזק' בה לטמא מעת לעת ולפי דבריו ז"ל ולדידן דקי"ל כרשב"ג אין כתמה טמא עד שיעברו עליה ד' וסתות. לראיה פעם ראשונה [דרב גידל] פירש רש"י ז"ל דהיינו [ראשונה שאחר ההפסקה ושניה היינו] ראיה שניה של דלוג שהיא ראשונה לראיה דעונות. ולפי שהיא עומדת בה כשרואה עכשיו בעונו' קרי לה הכי.\par  ויש לפרש "הדר קחזיא בעונות" [דקאי גם על] פעם אחרת בין דלוג ראשון ושני ראתה פעם אחת בעונה ובין שני לשלישי חזרה וראתה עוד בעונה ותרתי בעיי אהדדי איתמר ורב אשי בעא [לאפסןקי ולמפשט חדא חדא] מיניה דסד"א כיון דראתה שתים בדילוגו ושתים בעונות סלוק דמים הוא דקא מנע מינה עונות ולא תהא מוחזקת לא לדלוג ולא לעונות דכיון דשנתה כ"כ אונס בעלמא הוא. ואמר רב גידל פעם ראשונה של עונות ט) דיה שעתא כדאמרן שניה של עונות י) כיון דראיה שלישית הוא יא) מכי חזיא בעונות ואילך לעולם מטמאה מעת לעת. ומיהו בראיה (ג') [ראשונה] שלה לא מטמיא מעל"ת משום לסוף ג' עונות חזיא ואכתי לא הוחזקה ג' פעמים להפסקה דהא אנן לר' אלעזר קאמרינן ולישנא דהדרא קא חזיא דייקא כדאמרן. }
\newsection{דף \hebrewnumeral{11}}
\twocol{ מתניתין \textbf{צריכה להיות בודקת וכו' ומשמשת בעדים וכו'.}  פירש מתני' פרושי קא מפרש לה ואזיל וכיצד קתני כיצד צריכה להיות בודקת פעמים ביום שחרית וערבית ואע"פ שלא שמשה כלל וכיצד משמשת בעדים בודקת נמי בשעה שהיא עוברת משאר עסקיה לשמש את ביתה ומשמשת בעדים וע"כ מדקתני בשעה שהיא עוברת לשמש היינו עד שלפני תשמיש וש"מ דצריכה בדיקה לפני תשמיש והעד (הג') [הב'] לפני תשמיש אי אפשר אלא לאחר תשמיש הוא וכדתנן אחד לו ואחד לה אלמא צריכה בדיקה בין לפני תשמיש בין לאחר תשמיש.\par  והיינו דאמרינן לעיל [דף ה' ע"א] שתי בדיקות אצרכוה רבנן חדא לפני תשמיש וחדא לאחר תשמיש ורמינ' למתני' דקתני והמשמשת בעדים הרי זו כפקידה דהיינו עדים דקתני דאינון לפני תשמיש ולאחר תשמיש דומיא דמשמשת בעדים דהך סיפא [וכי תריץ] נמי לעיל גבי רישא דמתני' מעיקרא [אידי ואידי לאחר תשמיש] משום דקשיא להו קס"ד לפרושי ההיא דשני עדים דבסוף קא חשיב אבל בהך סיפא דכ"ע עד שלפני תשמיש קתני וכדמפרש עלה לקמן בגמרא. 
\par הא דאמרינן \textbf{תנ"ה בד"א לטהרות אבל לבעלה מותרת וכו'.}  היינו טעמא דמשמע לן אפילו כשאין לה וסת משום דמתניתין בין שיש לה וסת וכו' ועלה קתני בד"א לטהרות אבל לבעלה מותר בין בזו בין בזו משמע ועוד דכל עיקר לא הוצרכה ברייתא זו לשנותה אלא בשאין לה וסת שאלו בשיש לה מתני' היא כל הנשים בחזקת טהורות לבעליהן. }
\newsection{דף \hebrewnumeral{12}}
\twocol{\textbf{וכיון שתבעוה אין לך בדיקה גדולה מזו.}  פירש רש"י ז"ל דסתם הבא מן הדרך דרכו לפייס ולרצות ולתבוע וכי מרצו קמה ותבע רמיא אנפשה ואי הוה חזיא מרגשה. וכי אמרינן דבעינן בדיקה בשוהה עמה שאינו צריך ריצוי כ"כ ומיהו הניח בחזקת טומאה אף ריצוי לא מהני ליה עד שישמע מפיה טהורה אני והא דשאל רב כהנא אינשי דביתהו דרבנן לומר אם מחמירן על עצמן לבדוק בשאינן עסוקות בטהרות דומיא דבעיא דר' זירא דלעיל והאי דנקט כי אתו מבי רב אורחיה דמילתיה נקט שיוצאין ובאין מערב שבת לע"ש.\par  ויש לפרש דישינות בעיא מנייהו אם מחמירין בהן בבאין מן הדרך משום דכיון דאין בעלה עמה לא קפדה אנפשה ואמרו להן לאו, נמצא כלל השמועה הלכה למעש' שכל לבעלה לא בעי בדיקה לא לפני תשמיש ולא לאחר תשמיש ואפילו כשאין לה וסת לפי פירושו של רש"י ז"ל בדברי רבי חנינא בן אנטיגנוס.\par  אבל מדברי הרמב"ם הספרדי ז"ל למדנו שיש לו דרך אחרת בשמועה זו שהוא מפרש זו ששנינו דרך בנות ישראל לבעלה בשאין לה עסק בטהרות והצנועות בודקות אף לפני תשמיש לבעליהן וכל מה שהקל ר' יהודה ור' זירא משמיה דשמואל אינו אלא בבדיקה זו שלפני תשמיש שכשהן עסוקות בטהרות אפילו שאינן צנועות צריכות. וכשאין עסוקות בה הרי הן בחזקת טהרה לבעליהן לפני תשמיש.\par  וטעם לדבריו מפני שלפני תשמיש אשה מרגשת בעצמה ואפילו בישינה נמי הקלו מפני שבחזקת טהרה הן ולאחר תשמיש חוששין שמא ראתה מחמת שמש ואינה מרגשת.\par וההיא דבעיא מיניה ר' אבא מרב הונא צריך הוא לפרש שלא מנעו אלא מלבדוק בשיעור וסת ואח"כ כדי שלא תתחייבנו באשם תלוי ויהא לבו נוקפו אבל לאחר אחר בודקת קודם שתלך ותקנח ואף על פי שהוא בודק בשלו לחטאת שאני התם דא"א בבדיקה שלא תחייבנו חטאת והוא צריך בדיקה מ"מ שמא ראתה מחמת תשמיש אבל בשלה אפשר לבדיקה זו לאחר זמן של אשם תלוי שלא יהא לבו נוקפו בביאה זו ותהא מתוקנת בביאה אחרת, וכן דברי ר' זירא לר' יהודה כך הן מתפרשין לי מהו שתבדוק עצמה לבעלה לחייב בעלה שאם לפני תשמיש והלא בצנועות (הוא) [במתני' קתני] לה וזהו שאמר רבי ינאי זו עדן של צנועות לא צנועות שנשנו בפרק כל היד אלא ר' אמי פירש לומר שכל העושה כן נקרא צנועה, ורבא פירש לומר דבעד זה נכרת אם צנועה היא אם לא אבל לדברי הכל מתני' צריכות קתני ובעסוקה בטהרות ואלו בשאינה עסוקה כבר שנינו והצנועות מתקינות וכו'.\par ושאר השמועה פשוטה היא לפי דרכו לפיכך כתב אינה צריכה עד שלפני תשמיש אלא משום צנועות אבל לאחר תשמיש הכל צריכים שני עדים אחד לו ואחד לה אפילו מעוברת ומניקה זקנה וקטנה האריך עלינו את הדרך.\par  אבל דברי רש"י ז"ל יותר נכונים ומוכרעים בכמה מקומות בשמועה והחכם יבור לעצמו, ודברי רבינו יצחק אלפסי ז"ל שנראין נמי כדברי רש"י ז"ל שהוא כתב בהלכות ברייתא זו דתניא החמרין והפועלין והתיר בין עירות בין ישינות ולא הזכיר משניו' הללו של שני עדים בשאין לו וסת אלמא אין לנו עדים אלא לטהרות. 
\par  והא דאמר רבי מאיר\textbf{יוציא ולא יחזיר עולמית.}  משום קלקולא נ"ל והוא שאמר לה משום שאין לך וסת אני מוציאך ואם לא מפני כן לא הייתי מוציאך ואם לא אמר כן אין כאן חשש לקלקו' כדתנן המוציא אשתו משום איילנות רבי יהודה אומר לא יחזיר וחכמים אומרים יחזיר ואוקימנא מאן חכמים ר"מ ומשום דלא כפליה לתנאיה והא נמי לההיא דמיא ולדידן נמי לא בעיא כפילא כרבנן והוא שאמר סתם משום כך אני מוציאך ואף על פי שלא כפל הא גירש סתם יחזיר כדאמרינן התם גבי איילנות ומוציא משום נדר ומשום שם רע ויש לומר שאפילו לא התנה ולא אמר כלום יש לחוש לקלקל דבשלמא התם אם לא התנה כלום י"ל עילה הוא רוצה לגרש שכמה אנשים נשואים לאיילנות וע"י שיש בהם נחת רוח מהן מקיימין אותם וכך אמרו שם בירושלמי אבל זו שאסורה היא לשמש כלל בידוע שאין בעלה מגרשה אלא מחמת פיסול זה וכן במוציא משום שם רע י"ל מכיון שלא שהה לראו' אם הדברים נראין עילה מצא וגורש לפיכך אין חוששין לקלקול אלא שאמר משום שם רע אני מוציאך. }
\newchap{פרק \hebrewnumeral{2}\quad כל היד}
\newsection{דף \hebrewnumeral{13}}
\twocol{הא דאקשי' בכולה שמעתין מדר"א דאמר \textbf{כל האוחז באמה וכו'.}  י"ל דהכי אקשינן וע"כ לא פליגי רבנן דאמרו לו עליה דר"א אלא בדליכא עפר תיחוח ולא מקום גבוה ומשום חשש פסול המשפחות אבל במקום אחר מודו ליה. הילכך גבי תרומ' ה"ל למימר שיפלוט ואע"פ שמפסיד' וכן בדרב יהודה שיטה וירד חוץ לכנישתא וישתין. וי"ל דא"ל לאו פלוגתא היא אלא בשואלין לפרש להן היו. וכן נמי משמע במס' ברכות פ' כיצד מברכין (מ, א). 
\par הא דאמרינן ב\textbf{מקשה עצמו יהא בנדוי.}  פי' בתוספות לא שהוא מנודה בעצמו בנדוי דרבנן של רבותינו אלא שב"ד מצווין לנדותו ועד שנדוהו אינו מנודה, וראיה לדבר דקאמרינן הקורא לחבירו עבד שיהא בנידוי ואמרי' עלה בקדושין באומר לו עבד אתה ההוא שמותי משמתינן ליה דתניא הקורא לחבירו עבד יהא בנדוי. }
\newsection{דף \hebrewnumeral{14}}
\twocol{\textbf{בדקה בעד שאינו בדוק לה.}  פירש בתוספות כגון שהזמינה פקולין או צמר נקי ולבנים אלא שלא חזרה וראתה בהן סמוך לבדיקתה אם יש עליהם טיפי דמים מן מאכולת או משאר דברים הא בבגד שאינו בדוק כלל לא אמר ר' טמאה נדה אטו לקחה בגד מן האשפ' וקנחה בו מי מטמא רבי נדה.\par  (ואי) [ועוד אני] אומר כיון שהצריכוה כגריס ועוד הרידינו כסדין וחלוק וכל שלא היה בדוק כלל אפילו לרבי טהורה לגמרי אפילו לקחתו מן השוק סתם טהורה לבעלה. והראב"ד ז"ל סובר דבעד אפילו אינו בדוק כלל טמאה דלא דמי לחלוק דכיון שבדקה בו ממש רגלים לדבר דרוב דמים מצויין בו. 
\par  ופריק רב חסדא דה"ק \textbf{איזו אחר זמן וכו'}  פי' רש"י ז"ל וחסורי מחסרא וה"ק ול"נ אלא רב חסדא פרושי מפרש לה למתניתין הכי תנן נמצא על שלה לאחר זמן טמאים מן הספק ופטורין מן הקרבן ולא פי' שיעור אחר זמן מה הוה. והדר תני איזו אחר של שיעור זה שאינן טמאין מן הספק כדי שתרד מן המטה ותבדוק שזהו אחר כך שמטמאה מעת לעת ואינה מטמאה את בועלה והאי דפריש תנא האי שיעורא לא ללמד על דין עצמו שהרי כל מעת לעת כך הוא נדון בין לרבנן בין לר' עקיבא אלא כך אמר אם ירדה מן המטה ובדקה אין בועלה טמא שזהו אחר זמן הא כל זמן שלא שהת' כשיעור הזה אלא בדקה עצמה על המטה טמאין מספק בא זה ולמד על זה.\par  אבל לדברי רש"י ז"ל שאומר חסורי מחסרא וה"ק בהדיא איזו אחר זמן כדי שתושיט ידה ותטול עד ותבדוק לא הוה תו למיתני כדי שתרד מן המטה ואפשר לתרץ לו שבא לפרש שלא תעלה על דעת שבדיקת ראשונה שלאחר זמן ראשונה כשירדה מן המט' היא לפיכך פירש שתיהן ואמר שאלו ירדה אחר אחר הוא וכ"ש לדברינו דמחוור טפי לומר שפי' אחר אחר ללמד על אחר הזמן הראשון ושלא ליתן בו מקום לטעות.\par  וברייתא נמי דייקא כדידן, דתניא איזהו אחר זמן דבר זה שאל ר' אלעזר בר צדוק לפני חכמי' באושא שמא כר"ע אתם אומרים שמטמאה את בועלה מעת לעת פירש ולפיכך אין אתם חוששין לפרש אחר זמן שהרי כל מעת לעת נמי כך הוא דינן ואע"פ שהיו צריכין לפרש לדבריו דר' אליעזר בר צדוק משום אשם תלוי יודע היה בהם דבעי חתיכה משתי חתיכות ולא תמה עליהם בזה אלא אם כדברי ר"ע שהוא יחיד הם אומרים אמרו לו לא שמענו לפי' אין אנו מפרשין אבל לא בדברי היחיד אנו אומרים ואמר להם כך פרשו חכמים ביבנה לא שהתה כדי שתרד מן המטה ותדיח פניה תוך זמן זה כלומר כל ששהתה ובדקה ולא שהתה שיעור שתרד מן המטה ותבדוק אלא על המטה בדקה אע"פ שהושיטה ידה לעד תוך זמן זה אבל ירדה מן המטה ובדקה או שהתה כשיעור הזה לעולם טהור ואע"פ שעד בידה ש"מ שבכל מקום שפי' אחר לא פירש באחר זמן דבר אחר אלא כל שלא שהתה כשיעור אחר וכן דרך משנתינו ללשון שפירשנו.\par  ואקשינן לרב אשי אמאי קא מטהרי רבנן בברייתא ביורדת מן המטה ובודק' הא במתני' מטמו כשיעור הזה וכ"ת דאין עד בידה ה"ל לפרושי שהרי אין משמעו' הלשון זה אלא כל ששהתה כדי שתרד לעולם טהור ואפילו עד בידה וכדקתני מתניתין נמי כדי שתרד ומוקמת לה בעד [בידה] דהיכי אפשר דהכא והכא חד שיעורא קתני והכא טהור והכא טמא ה"ל לפרושי במתני' עד בידה ובברייתא אין עד בידה אלא ש"מ כרב חסדא ותרווייהו שיעור לטהר וזהו דרך פירש רש"י ז"ל בשמועה כולה ויש לשונות אחרים ואין בהם ממש. }
\newsection{דף \hebrewnumeral{15}}
\twocol{\textbf{והוא שבא ומצאה בתוך ימי עונתה.}  פי' רש"י ז"ל שלשים יום לראיה ואמרו שכך נמצא במס' נדה בירושלמי ואמר רב הונא דכי מצאה תוך ימי עונתה לא בעיא בדיקה אנא שלא הגיע ימי וסת אבל הגיע קודם ביאתו מן הדרך אסורה וסתות דאורייתא הלכך אסורה עד שתאמר לו בדקתי בשעת הוסת עצמו וטהורה אני ורבה בר בר חנא אמר אפילו הגיע עת וסתה מותרו' וסתות דרבנן שהם הצריכוה לבדוק בימי וסתה שמא תראה.\par  ומיהו היכא דבעלה לא היה בעיר ולא ידעינן אי בדקה אי לא בדקה לא מחזקינן לה בטומאה. ואע"פ דאמרינן לקמן תבדוק ומשמע דאסורה עד שתבדוק וכיון דכי לא בדקה מחזיקן לה בטומאה עד שתבדוק כי לא ידעינן ודאי בההיא דחזקה נמי קיימא עד שתאמר בדקתי וטהורה אני א"ל הכא בבא מן הדרך הקל כיון דאיכא תביעה אין לך בדיקה גדולה מזו ומהניא לחששא דוסתו' כדמהניא לבדיקה דטהרות בפירקין קמא דתרווייהו מדרבנן. א"נ כיון דלא ידעינן תולין שמא בדקה ומצאה טהור או שלא ראתה והיתה טהורה מ"ה מותרת. והא דאמר ר' יוחנן בעלה מחשב ימי וסתה ובא עליה תפתר בשוהה בעיר ולשהות עמה בין עירה בין ישינה ואף ע"פ שאינה אומרת לו כלום ולא הוא תובעה כנ"ל לפי פי' רש"י ז"ל והוא נכון.\par  אלא בזו שפי' ימי עונה לאשה שיש לה וסת אינו מחוור שכיון שלא הגיע עת וסתה היאך נחוש לעונה והלא כל שיש לה וסת קבוע דמיה מסולקין ממנה עד זמן וסת [לכן נראה] דרב הונא אמתני' קאי ולדידיה יש לה וסת חוששת לוסתה ולא לעונה אין לה וסת חוששת לעונה ולרבה בב"ח אפילו יש לה וסת אינו חושש לוסתה כדאמרן ומיהו חוששת לעונה אחר הוסת כלומר שאם עבר עליה עת וסתה כיון שאין אנו חוששין לוסת נחוש לעונה אבל ודאי הגיעו ימי עונה ולא הגיע ימי הוסת אינה אסורה דאפילו למ"ד וסתו' דרבנן מסולקת דמים היא אפילו מעונות עד הוסת תדע שהרי אמרו דיה שעתה בוסתות ולא אמרו כן בעונות.\par נמצא עכשיו לדברינו כל אשה שאין לה וסת בעלה חושש לימי עונתה ושיש לה וסת חוששין לימי עונה שאחר הוסת ואפילו בבא מן הדרך ובכולן מחשב ימיה ובא עליה דהא מ"מ תרי ספיקי נינהו ואע"פ שאינו בעיר נמי סופרת היא בעצמה רוב פעמים הילכך מותרת ומסתברא נמי שאם היה הוסת רחוק שאפשר שטבלה והעונה קרובה תולין להקל שמא בשעת הוסת ראתה וטבלה ושוב אין לה עונה עכשיו דכולהי ספיקי נינהו ולקולא ולא החמירו בעונה יותר מן הוסת מפני שהיא חמורה אלא מפני שא"א לנו לומר שהאשה לא תראה לעולם לפיכך תולין בעונות וחוששין להן אבל אם בא לתלות נמי בוסת תולין. זהו מה שנ"ל.\par והראב"ד ז"ל כתב דהא דרבה פליגא אדר' יוחנן דלר' יוחנן אע"ג דוסתו' דרבנן (בפי') [צריכה] חשוב ימים לטבילה ואפילו לבא מן הדרך שאין הוסת יוצא מחזקת טומאה (נהי) [עד] שתבדוק כדלקמן ופסק הלכה כרבי יוחנן וזו דרך טובה להחמיר לענין מעשה ונמצא חוששת לוסת וחושש' לעונו' כשאין לה וסת אבל לחוש לשניהם כאחד אין לנו.\par אבל תמהוני על רבינו הגדול ז"ל שכת' זו ששנינו חמרין ופועלין וכו' נשיהן להן בחזק' טהרה ולא חלק בין הגיעו ימי עונה ועת הוסת לשלא הגיעו. והר"ם תלמידו ז"ל כתב הלך בעלה למדינה אחרת והניחה טהורה כשיבא אינו צריך לשאול ואפילו מצאה ישינה הרי זה מותר לבא עליה שלא בעונת וסתה ואינו חושש שמא נדה היא. אף הוא לא הפריש בכלום.\par ונראה שהם ז"ל מפרשים תוך ימי עונתה היינו עונת וסתה לאפוקי יום עונה עצמו ודלא שתעלה על דעתך שבבא מן הדרך לא חששו אף לעונת הוסת או שיתלו להקל לומר שמא עקרתו וקמ"ל דבעי מיחשב דלא בעונת וסת קיימא האידנא.\par והם עוד סבורין דר' יוחנן דאמר מחשב קסבר וסתו' דאורייתא והלכתא כרבה דאמר מותרת דסוגיין וסתו' דרבנן ומסתייעי מאינשי דביתיה דרב פפא ורב הונא בריה דר' יהושע דפ"ק דכי אתו מבי רב לבתר וסת ועונה אתו ולא מיחשבי ולא בדקי בין עירות בין ישינות והא דתניא ר' יהושע אומר תבדוק תפתר לטהרות וזו קולא גדולה טוב לפני האלהים ימלט ממנו. 
\par הא ד\textbf{אמר רבי אושעיא וכו'}  במסכת ע"א שמעתיה (מא, ב) ושם פירשתי. }
\newsection{דף \hebrewnumeral{16}}
\twocol{גמרא \textbf{אמרו להם ב"ש לדבריכם.}  כיון שאתם מודים בבדיקה שלאחר תשמיש משום שמא ראתה מחמת תשמיש הוה נמי בבדיקה בין תשמיש לתשמיש שמא תראה טפת דם כחרדל בביאה ראשונה שבא אורח מחמת תשמיש ותחפנה שכבת זרע בביאה שנייה ושוב לא תמצא בעד שלאחר כל התשמישן אמרו להם ב"ה א"כ אף מתחלת ביאה לסוף ביאה ניחוש כן ויכולין היו ב"ש לומר שאין דנין אפשר מא"א אלא גדולה מזו אמרו שאינו דומה וכו'.\par ולדברי הר"מ ז"ל צריכין לפרש תראה טיפה דם כחרדל בעד של אחר ביאה ראשונה ותחפנו שכבת זרע לאחר ביאה שנייה ואמרו להם ב"ה אף לדבריכם שמא בקנוח ראשון עצמו נמוקה הטפה ובטלה בשכבת זרע.\par  ומצאתי בתוספו' שפי' כדבריו בשמו של ר' שמואל רומרוגי ז"ל והם הקשו ללשון רש"י ממה ששנינו צריכין ב' עדים על כל תשמיש ותשמיש או תשמש לאור הנר ומשמע דמשמשת לאור הנר אינה צריכה אלא כדברי ב"ה ולפירושו עדיין המשמשת לאור הנר צריכה לבדוק ולאחר כל תשמיש ותשמיש ולראו' בעד כדברי ב"ש וב"ה אינה צריכה בדיקה כלל עד סוף כל הלילה מ"מ לשון תראה מתפרש לנו יפה. 
\par \textbf{בדקה בעד ואבד אסורה לשמש עד שתבדוק.}  לדברי רש"י ז"ל בעד שלפני תשמיש ופשוט' היא, ולדברינו כגון שהחמירה ובדקה לאחר תשמיש ראשון או שהיתה דעתה שלא לשמש כל הלילה ומכיון שאין מוכיחה קיים ולא תדע אם ראתה כלום מחמת תשמיש זה אסורה לשמש עד שתבדוק לפני תשמיש לאור הנר בכל בדיקה של בעל.\par  ואקשינן אלו קנחה בו ואיתי' מי לא משמשה אע"ג דלא ידעה הא אפילו ב"ש דמחמרי שרו לבדוק ולהניח ולשמש והאי דאקשינן הכי ולא אקשינן מדב"ה דלא מצרכי בדיקה זו דמי מדמי מקשינן דאלו התם דיומא משום קולא דבעל הוא שלא להטריח עליו בין תשמיש לתשמיש אבל מכיון שבדקה דנמלך צריכה היא בדיקה ממש לאור הנר.\par  ומפרק זו מוכיחה קיים לטהרות וזו אין מוכיחה קיים לטהרות שמא ראתה לאחר תשמיש זה ולא תדע למחר הילכך אף לבעלה אסור' משום מגו של טהרות וכן נמי לב"ה דלא מצרכי הך בדיקה מוכיחה בעד שלאחר תשמי' אחרון ואין חוששין לנימוק אבל זו כבר נתקנח הדם בזה ואבד הילכך אסורה עד שתחזור ותבדוק עכשיו לפני תשמיש דבהכי ודאי מותר' שא"א להחמי' עליה יותר מכאן לא יהא זה חמור מן הוסת שאם בדקה ומצאה טהור טהור. ואפילו לטהרות עצמן אם בטלה בדיקה של עונה אחת כגון של שחרית או של ערבית בודקת עכשיו ועוסקת בהן וכן בבדיקה של אחר תשמיש בין לטהרות בין לבעלה. }
\newsection{דף \hebrewnumeral{17}}
\twocol{ הא \textbf{דאמר אביי כגון שהעבירה על אויר התנור.}  חדא מתרי טעמיה נקט דה"נ אפשר לאוקומה כגון שנטמא באהל המת א"נ בהיסט הזב וזבה וכל המטמאים במשא. 
\par \textbf{אלא למעוטי רובא דרבי יהודה.}  פירש דאפילו לרבי יהודה לאו רוב גמור הוא לשרוף אלא לתלות, וק"ל דהא איתותב ההוא לישנא ואסיקנא דבאי אפשר לפתיחת קבר בלא דם קמפלגי ובפיר' רש"י ז"ל אפילו ללישנא בתרא דר' יוחנן דאמר טעמיה משום דאי אפשר לפתיחת קבר בלא דם אפילו הכי טעמיה דר' יהודה משום רוב דברוב פתיחת קבר איכא דם ולא עשאו ברוב זה כודאי דכיון דאין עמה דם איתרע ליה.\par  וק"ל דהא אוקימנא לר' יהודה כרבי יהושע דמשוה ליה חדא ב) דאמר מביא קרבן ונאכל דאי אפשר לפתיחת קבר בלא דם וא"ל סבר לה כרבי יהושע לתלות ולא לשרוף, ויש לומר דמאן דמתני בשלשה מקומות מתני ההוא לישנא קמא\par  דר' יוחנן דמוקי פלוגתייהו דר' יהודה ורבנן בשאינה יודעת מה הפילה. וכן עיקר שאלמלא כן לא הזכירו בגמרא כאן לשון ראשון שהוא טעות במקום עיקרו. }
\newsection{דף \hebrewnumeral{19}}
\twocol{הא דתנן \textbf{עמוק מכן טמא דיהה מכן טהור.}  פירש רש"י ז"ל עמוק יותר שחור דיהה שנדחית מראיתו ואינו שחור כל כך.\par ושמעתי שה"ר שמואל ז"ל פירש בהפוך עמוק שאינו שחור כל כך דיהה שחור יותר מכן.\par  והכניסו במחלוקת הזה מה שאמרו לענין נגעים בהרת עמוקה כמראה חמה עמוקה מן הצל אלמא לובן הוא העמוק, ועוד מזה מש"כ כזית ובזפת וכעורב טהור וזהו דיהה וכדיו וכענבה טמא וזהו עמוק והוא ז"ל שיער בדעתו שהעורב שחור יותר מהענבה ועוד שהדיו שהוא עמוק זהו הדיו עצמו והשחור השנוי במשנתינו הוא חרותא דדיותא והוא שחור יותר מן הדיו עצמו.\par  וכל אלו דברים בטלין הם שהמראה חזק באותו גוון הוא נקרא עמוק בכולה גוונים והחלוש באותו צבע הוא הדיהה ממנו כדתנן עבר או שדיהה הרי זה כתם והזית והעורב שחרירותן מבהיק ושל ענבה משחיר ובוהק ג) וכן הדיו הלחה שכותבין בו ונשתהא כוהה יותר (מקום לו) מהחרת השנוי במשנה וכי היאך יעלה על דעת במזוג דשנינו (שלשה) [שני] חלקים מים ואחד יין שהעמוק מכן והוא הלבן שיש בו שלשה חלקים מים ואחד יין טמא והדיהה או דיהה דדיהה שאין בו אלא חלק אחד מים וחלק אחד יין טהור והלא אדום גמור הוא וקרוב למראה דם יותר מן הראשון כפלי כפלים וכל שכן באדום עצמו שהוא כדם המכה שהמלבין בו טהור והמתאדם ביותר טמא ודאי בלא ספק ולא ידע מר בטיבעא כלום.\par  וראיתי בתוספות שהביאו מן הירושלמי בענין עמוק דיהה הוו בעיין מימר מאן דאמר טהור במצחצחו מאן דאמר טמא בשאינו מצחצח ושמע מינה מן הדה מעשה וכו' א"ל רביח לבו כן אמר רב הונא בשם ר' שחור מקדיר (טמא) [טהור] מצחצח טמא לא אמר אלא שחור כולהון אפילו מצחצחין טהורין אלמא המצחצח קרוב לטומאה יותר מן המקדיר וזה הפך ו) שהרי מתחלה סלקא דעתך המצחצח טהור ז) שהוא המשחיר ביותר הוא הטמא ח) ומ"ש אפילו מצחצחין הכי קאמר אפילו היה במראה הטומאה הואיל ומצחצחין טהורין. 
\par לשמואל דאמר \textbf{כדם שור שחוט ולעולא דאמר של צפור חיה ולרב נחמן דאמר של הקזה.}  ל"ק הא דתנן הרגה מאכולות תולה בה ובבנה ובבעלה דא"ל חד שיעורא הוא אלא לזעירי בלחוד דפריש הוא דמקשינן מינה הילכך אית לן לפרושי דכולהו לא פליגי אלא מר בקי בהאי חזותא ומר בקי באידך ודכולהו ודברייתא נמי חד הוא כדאמרן, והא דאקשינן הרגה מאכולת הרי זה תולה בה מאי לאו דכוליה גופא ואקשינן נמי תולה בבנה ובבעלה בשלמא בנה משכחת לה משמע דלא תלינן בכתמים אלא בדדמי ומקיפין ורואין ומכאן החמירו.\par  ונמצא במקצת גליוני ראשונים דעכשיו בזמן הזה כיון שבטלו ראיית דמים ואין בקיאות במראיהן של ד' דמים אין תולין בכתמים ולא בבן ולא בבעל ולא במאכולת ושוק של טבחים ושאר כל מה ששנו חכמים בכתמים לתלות אלא בכולן אסורות לבעלה.\par ורבינו בעל התוספות השיב דכי אמרינן עברה בשוק של טבחים תולה וכן במאכולת בכולן מעצמה קאמרינן דתולה ולא מגופא אתא אלא מעלמא אתא, אלא מיהו היכא דידעינן ודאי דלא דמו ליכא למיתלי ומ"ה קאמרינן הכא בשמעתין למ"ד כדם מאכולת של ראש הרגה ודאי מאכולת של גוף היאך תולה בשאינו דומה ודאי וכן קושיא דבנה ובעלה אבל מסתמא תולין כתמים בכל מה שאמרו חכמים.\par  וכענין זה כתב הראב"ד ז"ל דתולין מן הסתם כל מיני אדום באדום וכל מיני שחור בשחור עד שיתברר לה ודאי שאין אודם הצבע דומה לאודם הכתם דהתם אינו תולה כדתניא לקמן נתעסקה באדום אין תולה בו שחור ובשמעתין הכי אקשינן היאך אפשר לתלותו במכת בעלה ובדם מאכולת הגוף והלא דבר ברור הוא שאינו דומה, אבל מי שאינה יודעת בדמיונות או שהלך הצבע מנגד פניה ואינה יכולה לדמות תולה מן הסתם כדאמרינן באשה שבאת לפני ר' עקיבא ואמרה לו ראיתי כתם שמא מכה יש בידך וטהרה והרי אשה זו לא הביאה הכתם בידה לפניו וטהרה מיד ולא הקיף ולא ידע כל זה שכתב הרב ז"ל.\par וכן יש לפרש זו שהקשו בשמעתין ממאכולת ובעלה דה"ק ודאי מדבעין תליה בהני אלמא בכתם טמא עסקינן ואלו הנך דדמו כתמים טהורים הם ומה נפשך אי לא דמו לא תליא ואי דמו לא צריכי תלייה כלל, ולא בעי לתרוצי כשראתה ואבד ואינה יודעת מה ראתה דמסתמא ברואה ובאה לפני חכם תנן וידע דאיכא טובא מתירין הלכך בזמן שאין בקיאין תולה במאכולת הגוף ובבעלה ובכל מיני אדמות עד שיתברר לפי הדעת שאינן דומין וכן נהגו ואין לחוש.\par  ושוב ראינו בכתמים שכתב הראב"ד ז"ל עיינתי בכל מילי דרבוואתא ולא אשכחית בהון דינא דכתמים אי נהיגי האידנא או לא, ואף על גב דאשכחן דמטמא את בועלה בפרק קמא דנדה דלמא לטהרות היא.\par  וזה אינו נכון, וכבר השיב הרב חתנו ז"ל מדאמרינן בכתמים פעמים שהן מביאים לידי זיבה ואיתמר עלה מהו דתימא כל כהאי גוונא מביאה קרבן ונאכל קמשמע לן מביאה קרבן לאסרו לבעלה דאי לטהרות בלבד קרבן מאי עבידתיה דאפילו להכשיר בקדשים ליכא למיחש כיון דליתיה אלא דרבנן בעלמא לחוש בטהרות ולבעלה מתירין אותה לכתחלה.\par  ועוד השיב מדאמרינן בהדיא בפרק הרואה (דף נח ע"ב) לדברי אין קץ שאין לך אשה שטהורה לבעלה שאין לך כל מטה ומטה שאין עליה כמה טיפי דם מאכולת לדברי חבירי אין סוף שאין לך אשה שאינה טהורה לבעלה וכו' אלמא כתמים לבעל נאמרו ואין צריך להאריך שהרי הוחזקו בנות ישראל שנהגו איסור בכתמים וקי"ל דמנהגא מילתא היא כרבי זירא, כל אלו דברי הרב ז"ל. }
\twocol{\textbf{ת"ש דילתא אייתא דמא לקמיה דרבה בב"ח וכו'.}  אי קשיא אדרבה איפכא מסתברא דאם כן דנאמנת אשה לומר כזה טיהר לי פלוני חכם ולטהר אפילו לחברתה, למה הביאתו ילתא קמיה דרבה תטהר לנפשה דהרו כל יומא נמי מטהר לה, א"ל קס"ד השתא דאיהי סברא דלא מהימנה לנפשה, אי נמי מחמירה על נפשה, אי נמי משום כבודן של חכמים הללו אינה רואה במקומן כדאמרן לעיל. }
\newsection{דף \hebrewnumeral{21}}
\twocol{\textbf{באפשר לפתיחת קבר בלא דם פליגי ובפלוגתא דהני תנאי וכו'.}  ואי קשיא הא לת"ק דברייתא גופיה ספיקא משוי ליה ואלו ת"ק דמתני' קאמר ואי לאו טהורה א"ל התם משום דילדה ואינה יודעת מה ילדה וחוששין ללידה וחוששין נמי לזיבה שמא עם הנפל יצא דם אבל שילדה לידה יבישתא העמד אשה על חזקתה וטהורה היא ועוד שהרי בדקו ולא מצאו דם וכן ללשון הראשון שאמר רבנן סברי לא אמרי' רוב חתיכות מד' מיני דמים הן קשיא ותהוי נמי מחצה על מחצה תהא טמאה מספק אלא משום האי טעמא הוא דהעמד אשה על חזקתה.\par וי"מ דהתם ה"ק לא אמרינן רוב חתיכות מד' מינין הן ולפיכך טמאה גמורה אלא שאין שורפין כדאיתא בפ"ק, ואין פירוש זה נכון. 
\par  ומהדר אביי \textbf{בשפופרת דכ"ע לא פליגי כי פליגי בחתיכה מר סבר דרכה של אשה לראות דם נדה בחתיכה.}  פירש רש"י ז"ל דבפלאי פלויי פליגי מר דהו רבי אלעזר סבר דרכה של אשה לראות דם נדה בחתיכה וכיון דאיפלאי וליכא חציצה טמאה וכי לא אפלאי רחמנא מיעטה מבשרה ולא בשפיר ולא בחתיכה ורבנן סברי אין דרכה של אשה לראות דם נדה בחתיכה אלא האי דם חתיכה עצמה הוא.\par ולא מחוור דא"ה לא דמי האי דרכה של אשה וכו' לאותו שאמרו למעלה בדר' יוחנן דהתם קאמרינן דכיון דדרכה אע"ג דלא אפלאי נמי לא חוצה דהיינו אורחא ולישנא דגמרא נמי לא משמע הכי כלל.\par אלא ה"פ מרדאינהו רבנן סברי דרכה של אשה לראות דם בחתיכה ולא טהרו כאן אלא משום שאין זה דם נדה אלא דם חתיכה והיכא דהוי ודאי דם (חתיכה) [נדה] כגון מצא בה דם אגור טמאה והיינו לר' יוחנן ומר דהוא ר"א סבר אין דרכה של אשה הילכך הוי ליה כשפופרת ורחמנא אמר בבשרה.\par והאי דמדכרי' סברא דרבנן מקמי דר' אלעזרא"ל משום דאמרן לעיל דרכה של אשה כסברייהו א"נ לאו דוקא וכן בכמה דוכתי בתלמודא דלא קפדי.\par  ובודאי דה"מ אביי לתרוצי כדרבנן הא דם נדה ודאי טמאה בדאיפלאי ובשפופרת דכו"ע לא פליגי דטהורה אלא ניחא ליה לתרוצי בדידה ולא לעיולי בה פילי דהשתא לא מוספינן בפלוגתייהו איפלאי פלויי כלל אלא בדם אגור בחתיכה פליגי כדפרישית, ועוד לאוקמה כדר' יוחנן דלעיל דלא לתקום דלא כחד כנ"ל.\par ויש מפרשים דלא ניחא ליה לאביי לאוקומה פלוגתא דרבנן אפלאי פלויי בלחוד דהא לא מדכרא בהדיא במילתיה דר' אלעזר דאנן בגמרא לאו חסורי מחסרא לברייתא כלל אלא מימר קאמרינן דלר"א בודאי פלאי טמאה ולא ניחא ליה לאוקומא פלוגתייהו אמאי דלא מתפרש בברייתא בהדיא.\par  ולאו מילתא היא ור"א ורבנן תרווייהו מטהרין מר נסיב לה טעמא מבשרה ולא בחתיכה ולפום טעמיה איפלאי פלויי טמאה ומר נסיב לה טעמא אין זה דם נדה לטהורי נמי אפלאי פלויי אלא כטעמא דפרישית עיקר. }
\newsection{דף \hebrewnumeral{22}}
\twocol{\textbf{והלא עצמו הוא אינו מטמא אלא בחתימת פי האמה, למימרא דנוגע הוי.}  פירש רב הונא אליבא דנפשיה פשיט ליה דס"ל כר' נתן דאמר זב אינו מטמא אלא בחתימת פי האמה ואל תתמה [דהא שמואל] רביה (דרבה) [דרב הונא] הוא דאמר נמי כר' נתן כדאיתא בפרק יוצא דופן ולפום הכי גמר רב הונא בעל קרי מיניה דזב וסבר לה נמי כר' שמעון דאמר בפ' ואלו דברים בפסחים דס"ל בזב כר' נתן דבעי פי האמה ואיתקש בעל קרי לזב ובעי נמי חתימת פי האמה כזב דהא קרא בבעל קרי לא כתיב ובזב כתיב או החתים בשרו.\par  והא דדאיק מינייהו למימרא דנוגע הוי דלהכי בעינן חתימת פי האמה דליהוי נגיעת חוץ כדפי' רש"י ז"ל, ק"ל אי הכי זב נמי נוגע הוי ולמה לא יספור בזיבה, א"ל בזב ודאי אע"ג שנוגע הוי לענין שיעוריה מיהו הוי רואה לענין טומאה דיליה דאלו נוגע בזב טומאת ערב ואלו רואה טומאת שבעה והאי דאחמיר ד) עליה רחמנא בחתימת פי האמה דליהוי נמי נוגע גזירת הכתוב הוא שלא יהא טמא טומאת שבעה עד שיראה זוב ונגע בו מגע חוץ דה"ל רואה ונוגע ומ"ה אקשי' אא"ב בעל קרי רואה הוי ואפילו במקום שאינו טמא משום נוגע טמא הוא משום רואה הרי דומה לזב מצד אחד שאף הוא יש לו טומאה בראיה שאינו מדין מגע אא"א אינו טמא אלא בנוגע וטומאתו נמי טומאת מגע היא א"כ מה הנוגע בקרי אינו סותר בזיבה אף הרואה לא יסתור שהרי שניהן טומאה אחת להן בכל ענינן ומדין מגע טימאן הכתוב, ומפרקינן התם בשביל שא"א לה בלא צחצוחי זיבה ואם תאמר והלא אין בהם חתימת פי האמה ואין הזוב מטמא אלא כן י"ל כיון שיוצא עם שכבת זרע שהוא חותם פי האמה הרי הוא כנוגע ממש שמין במינו הוא ואינו חוצץ.\par  והיינו דלא אקשינן יטמא טומא' שבעה אלא תסתור ז' דטומאה בזוב גמור לית ליה כיון דאינו רואה בשיעורו טומאת מגע זוב אית ליה וכיון דזוב הוא מיהא ובראיה דין הוא לסתור הכל שאין כאן ז' נקיים דהא הוה ליה כאלו ראה זוב בנתיים שאין אחר אחר לכולן.\par  ומפרקי' גזרת הכתוב כך הוא מאחר שאין הזוב הזה כדי ראיה אין לו טומאת שבעה ואפילו לסתור ז' אלא סתירתו כטומאתו והא נמי רב הונא אליבא דנפשיה פשט ליה דהא בפרק כיצד הרגל בב"ק איכא ר' אליעזר דסבר אפשר בלא צחצוחי זיבה כלל, ומיהו בהא כרבנן פריק ליה ורבים נינהו.\par  ואי קשיא לך לרב הונא דאמר בעל קרי נוגע הוי תרי קראי למה לי דהא כתיב רואה וכתיב נוגע וכדדרשינן בפרק יוצא דופן מדכתיב או איש, א"ל אע"ג דרואה דוקא בנוגע הוא דמטמא ה"א ה"מ ברואה דאיכא תרתי מגע וראיה אבל בנוגע לחודיה לא קמ"ל. 
\par  והא דאמרינן \textbf{ויברא גבי תנינים לאו מופנה.}  אי קשיא הא כתיב נמי ישרצו המים ההוא אין כתוב בעשייה אלא בצוויי, ופי' רש"י ז"ל דכיון שאין מופנה משני צדדין ומשיבין ה"נ יש להשיב מה לאדם שכן מטמא מחיים.\par  ול"נ דהאי לישנא קמא לא צריך פירכא דלכ"ע מופנה משני צדדין עדיף ממופנה מצד א' וכיון דע"כ יצירה יצירה גמרינן ה"ל בריאה דאדם לגופיה וגבי תנינים נמי לגופיה ואין מופנה כל עיקר וכל ג"ש שאינו מופנה כל עיקר אין למידן הימנה.\par וא"ת וייצר האדם לגופי' ודבהמה מופנה ודגמרינן ויברא דתנין לגופיה ודאדם מופנה וגמרינן היינו דקאמרי ומאי נ"מ זה כלומר אמאי ניחא לך לאפנויי לחדא לגמרי ומיגמר מינה ולא לאפנויי תרווייהו ומיגמר מנייהו ופריק לרבנן הא עדיפא דהא אין משיבין ולר' ישמעאל נמי הא עדיפא דהיכא דאיכא מופנה משני צדדין איהי עדיף ולהכי אפנויי רחמנא לבהמה משני צדדין דשדינן מופנ' דכולהו בגוה כי היכי דלא נימא באידך מופנה מצד אחד הוא דכל היכא דאיכא למישדי שני צדדין דמופנ' בדידיה שדינן ומיניה גמרינן בין לרבי ישמעאל בין לרבנן אבל ללישנא דרב אחא הויא דבעי' והאי מאי פירכא משום דאפילו כשאנו גומרין יצירה יצירה יכולין אנו לגמור בריאה בריאה אע"פ שאינה מופנה כל עיקר אלא שמשיבין ולפום הכי בעי' מאי פירכא ורבנן דפליגי עליה דר"מ במתני' לא גמירי כדאשכחן בפרק כל היד שאין אדם ג"ש מעצמו, וכן פי' רש"י ז"ל.\par ואי קשיא לך לר"מ מאי פירכא ליהדר דינא ותיתי מכאן דכיון דגמר יצורה ואתו בהמה חיה ועוף כי פרכת גבי תנין מה לאדם שכן מטמאו מחיים נימא בהמה תוכיח א"נ נגמר מוייצר דבהמה למד מלמד א"ל מה לשניהם שכן מטמאין במגע ובמשא תאמר בדגים שאינן מטמאין ואע"פ שמקבלין טומאה טומאת עצמן אין להם. }
\newsection{דף \hebrewnumeral{23}}
\twocol{\textbf{למימרא דחיי.}  פי' רש"י ז"ל דהא אחותה לא מיתסרא אלא בחייה דאין איסור אחות אשה לאחר מיתה ותמהני א"כ יפה שאל ר' ירמיה ונימא נפקא מינה לענין אתסורי באמה ואם אמה דאמות אפילו לאחר מיתה מן התורה ועוד נפקא מיניה לאתסורי באחותה שנים וג' ימים.\par  אלא כך פירשו למימרא דחיי שאם א"א לו לחיות כלל אין קדושין תופסין בנפל גמור כגון בת שמונה והלא הרי הוא כאבן לכל דבר אלא ודאי סבר ר' ירמיה דחיי והאמר ר' יהודה אמר שמואל לא אמרה ר' מאיר אלא הואיל ובמינו מתקיים, פי' הואיל לאו דוקא דהא לא טעמא הוא לר"מ אלא משום יצירה יצירה או שגלגל עיניו כשל אדם או בשיש בו מצורת אדם אלא ה"ק לא אמר ר"מ שהוא ולד שיעלה על דעתו שהוא חי אלא ולד הוא לענין טומאה שבמינו מתקיים כנפל גמור שהוא אינו מתקיים ובמינו מתקיים. 
\par \textbf{א"ר ירמיה בר אבא אמר רב הכל מודים וכו'.}  פי' ר' ירמיה משמיה דרב פליג אדשמואל ור' יוחנן דפרשי לעיל טעמיה דר"מ משום יצירה יצירה או משום גלגול עין שלדבריהם אפילו תייש גמור במעי אשה ולד מעליא הוא לטומאות לידה וכ"ש גופו אדם ופניו תייש דאיכא מקצת אדם.\par  וה"נ משמע דס"ל לרב יהודה משמיה דרב כותיה מדקאמר הואיל ובמינו מתקיים ולרב ירמיה משמיה דרב לית ליה הנהו טעמי אלא ר"מ ורבנן בסברא בעלמא פליגי בשפניו אדם ונברא בעין א' כבהמה שר"מ אומר מצור' אדם בעינן והא איכא וחכמים אומרים כל צורת ממש.\par  וה"ה לר' מאיר' דבמצח ועין וגבן העין ולסת וגבת הזקן סגי אלא להכי נקט פניו אדם ועין אחד כבהמה להודיעך כחן דרבנן והא דאמרי ליה רבנן והא איפכא תניא לאו איפכא תניא דוקא דמתהפכי תרתי סברי דהא לא (מתסרא) [מתהפכא] סברא דרבנן לר"מ אלא איפכא בלישנא לכולהו ואיפכא בסברא דר"מ דמפכא לה ברייתא לרבנן וכדפירש רש"י ז"ל.\par ולסבריה דרב הא דתני' לקמן המפל' דמות נחש אמו טמאה לידה ר' יהושע יחידאה היא ולית ליה דר"מ וכ"ש דרבנן וההיא דתניא לעיל נראין דברי ר"מ בבהמה וחיה בהכי נמי מתוקמ' בבהמה וחיה ומקצ' סימנין דאדם דכיון דהיא עצמה עיניה הולכות כשל אדם במקצת סימנין נעשית כאדם גמור מה שאין כן בעופות שאפילו כל פניו כאדם ועיניו לצדדין אינו כלום.\par  והאי דדחי רב אחא בריה דרבא לעיל תבדוק לרבנן דמודו רבנן בקריא וקפוף הואיל ויש להם לסתות כאדם דחייה בעלמא היא דדחי בסברת דר' אלעזר בר צדוק אבל לפום מסקנא לרבנן לסתו' חדא מצורות דפנים נינהו וצריך נמי גבין וגבת זקן דאדם ועין נמי דאדם ואף ע"פ שהולכות לפניהם צריך צורת דאדם באוכמא.\par  וי"מ דלרב ירמיה גופיה אית ליה אליבא דר"מ יצירה יצירה ואי כולה תייש בפניו וגופו אמו טמאה לידה הואיל ובמינו מתקיים והא דאמה גופו אדם ופניו תיש ולא כלום משום שאין זה לא מין בהמה ולא מין אדם והואיל ואין לו מין שמתקיים דברי הכל ולא כלום.\par  ולפי דבריהם קשיא, א"כ מנא ליה לר' ירמיה א"ר דר"מ בפניו אדם ונברא בעין א' כבהמה פליג דילמא בההוא כרבנן ס"ל דלאו אדם הוא ומין בהמה נמי אינו שאין לך בבהמה כמותו והם אומרים קסבר רב דר"מ ורבנן בתרתי נמי פליגי ממאי מדאמרי ליה רבנן כל שאין בו מצורת אדם ולא קתני וחכמים אומרים אמו טהורה א"נ וחכ"א אינו ולד שמעי' דר"מ דמטמא נמי במקצת צורה ואמרי ליה לא כל צורה בעי למעוטי צורה בהמה גמורה ולמעוטי נמי מקצת צורת אדם והא דאמרי ליה רבנן והא איפכא תניא איפכא לגמרי הוא שר"מ אמר כל צורת לגמרי מ"ט או כולו אדם או כולו בהמה וחכ"א מצורת אדם ולא פניו בהמה אבל במקצת צורת אדם ולד הוא והיינו דקתני מתני' כל שאין בו מצורת לאפוקי כולו בהמה ומדלא קתני אמו טהורה סתם משמע ליה דבתרתי פליגי וברייתא נמי דמסייעא להו כך מפורש בספר הישר.\par  ודברי רש"י ז"ל יותר נראין והוא הלשון הראשון שכתבנו ואע"ג דקשיא נחש דר' יהושע כדפרישית.\par  והא דתניא המפלת דמות לילית אמו טמאה לידה ולד הוא אלא שיש לו כנפים ולא אמרינן משום דגופו תיש ופניו אדם היינו נמי טעמא משום דודאי פניו אדם אע"פ שגופו תיש בתר צורת פנים אזלינן אבל בדמות לילית ס"ד אין כאן צורת אדם כלל אלא צורת לילית היא זו בין בגוף בין בפנים קמשמע לן דלילית גופה ולד הוא אלא שיש לו כנפים. }
\newsection{דף \hebrewnumeral{24}}
\twocol{הא דאמרינן \textbf{ושמואל סבר בריה בעלמא איתא וכי אגמריה רחמנא למשה בעלמא.}  פירש"י ז"ל אותו המין אסר לו וק"ל א"כ לשמואל אפילו יוצא לאויר העול' נמי לישתרי דה"ל כקלוט בן פרה דשרי ונראה מדבריו דבין לרב בין לשמואל במעי טהורה לא חיי הלכך יצא לאויר העולם משום נפל אסור אפילו לשמואל והא דפריך רב שימי ממתניתין ר' חנינא בן אנטיגנוס אומר וכו' לרב ה"ה לשמואל אלא גביה הוה קאי דבר בריה הוה.\par ולא נהירא ועוד דהתם בפ' ואלו מומין (דף מג ע"ב) תנן לה למתני' גבי מומי כהן איזהו גבן ר' חנינא בן אנטיגנוס אומר כל שיש לו שני גבין ושדראות והוי ביה למימר' דחיי והאמר רב באשה אינו לד בבהמה אסור באכילה ולא מדכרין התם דשמואל בכלום בעולם.\par  אלא הכי משמע פירושא לכ"ע מינא בעלמא ליכא כי פליגי בבריה רב סבר אפילו בריה בעולם ליכא דלא חי הילכך כי אגמריה רחמנא למשה במעי בהמה אגמריה דבחוץ לא צריך נפל הוא. ושמואל סבר בריה בעלמא איתא דחיי וכי אגמריה רחמנא בשיצא לאויר העולם דלא תימא כקלוט בן פרה הוא אבל במעי בהמה דאפילו נפל שריא איהי נמי שרי. }
\newsection{דף \hebrewnumeral{25}}
\twocol{\textbf{המפלת שפיר מלא בשר נימוח מהו.}  פי' קא מיבעיא להו לרבנן דפליגי עליה דר' יהושע מיפלגי נמי בבשר נימוח או לא אמר להם לא שמעתי אמר לפניו ר' ישמעל בר' יוסי משום אביו כך אמר אבא מלא דם טמאה נדה מלא בשר טמא' לידה שהיה ר' ישמעאל סבור שלא אמר אביו כדברי היחיד ולפיכך דחה רבי ואמר שמא כדברי ר' יהושע אמרה וזה שאמר מלא בשר לאו דוקא אלא ה"ה למחוי עכור ולא בשר אלא להוציא צלול אפילו לר' יהושע.\par  ויש שגורסין בה כמאן כסומכוס מדהא כיחידאה הא נמי דילמא כר' יהושע אמרה ואינו בספרים.\par וא"ת כיון שרבי לא קבלה אפילו בבשר נימוח שיהא ולד ריב"ל מנין לו דקאמרי בצלול מחלוקת אבל בעכור ד"ה ולד זה אינה שאלה דריב"ל כר' יוסי ס"ל וקסבר ר' יוסי לרבנן אמרה כדסבר נמי ר' ישמעאל בר' יוסי ועוד דכיון דרבי לא שמעתי אמר אינה תשובה לדברי ריב"ל שאם ר' לא שמע ריב"ל שמע לא ראינוה אינה ראיה.\par  וי"א אין אומרים בדברים אלו זו דומה לזו שאפשר למימר עכור ולד ובשר נימוח שמא אינו ולד. ואיכא למימר נמי איפכא ולפיכך נחלקו בכולן. }
\newsection{דף \hebrewnumeral{26}}
\twocol{ הא דאמרי' \textbf{תלת מתני' ותרתי שמעתא שיעורן טפח.}  ואקשי' תרתי חדא היא היינו טעמא דלא מקשינן תלת ד' הוויין משום דהוה איכא למימר רבי שילא לית ליה דר' חייא דשיעור אזוב טפח ומ"ה מקשינן אם כן [תרתי חדא היא, ועוד א"ל] תרווייהו כי הדדי נינהו וחדא נקט. 
\par \textbf{אין תולין את השליח אלא בדבר של קיימא.}  פי' רש"י ז"ל שכיוצא בו מתקיים אם היו חדשיו כלין למעוטי שאם הפילה דבר שאינו ראוי לבריית נשמה כגון נברא בירך אחת או גוף אטו' ואח"כ הפילה שליא (פי') [אפי'] בתוך ג' חוששין לולד אחר, ופירוש חזייה לרב יהודה בישות משום דשמעה מרב ולא אמרה.\par ואינו מחוור דבן קיימ' לאפוקי כל נפל משמע וכדאמרן דילמ' כאן בנפלי כאן בבן קיימא ולא ידעתי מי הזקיקו לשנות פירושו אלא הא דתלמיד' דרב פליגא אדרב יהוד' דאמר לעיל משמיה דרב הפילה נפל ואח"כ הפיל' שליא כל שלשה ימים תולין אותה בולד ושאר תלמידים דרב אומרי' משמי' דאין תולין את השליא בנפל אפילו יום א' אלא א"כ יצאה עמו אבל בבן קיימא תולין אותה אפילו מכאן ועד י' ימים.\par ושמעתי שפירשו בירושלמי במס' זו (ג, ד) לפי שאין השליא פורשת עד שיגמר לפיכך אין תולין אותה בנפל.\par  ובשאלתות דרב אחא משבחא ז"ל כתב לכך תולין אותה בבן קיימא דאמרי' אגב חיותא דולד בזעא לשליא ונפיק. אבל נפל דלית ביה חיותא לא. ומ"ה חזייה שמואל לרב יהודה בישות דשמעיה דאמר משמיה דרב דכל ג' תליא שליא בנפל וכיון דשמעינהו לכולהו תלמידי דרב דאמרי אין תולין כלל אמר ודאי רב יהודה טעי. }
\newsection{דף \hebrewnumeral{27}}
\twocol{\textbf{מ"ט דר' שמעון וכו'.}  פירש"י ז"ל נהי נמי דנימוק מ"מ גופו של מת כאן הויא וה"ל כרקב וכנצל. וק"ל הא דאמר רשב"ל לקמן בשמעתין שפיר שטרפוהו במימיו טהור להוי כרקב וכנצל. ועוד לר"מ נמי בבי' החיצון אמאי טהור ליהוי כרקב וכנצל. וא"ת איהו נמי סבר כל טומאה שנתערב בה מין אחר בטלה אלא מאן תנא דפליג עליה דקאמרת קסבר ר' שמעון והא דתניא מלא תרוד רקב שנפל לתוכו עפר כל שהוא טמא ור' שמעון מטהר אמאי תרמייה הא דכ"ע טומאה כיון שנתערב בה מין אחר טהורה.\par  ובתוספ' הקשו לה מדאמרינן ואזדא ר"ש לטעמיה מדאמר א"א שלא ירבו שתי פרידות עפר על פרידה אחת של רקב ויבטלנו ואמאי נהי נמי דבשיעור מצומצם כגון מלא תרוד רקב איכא למימר הכי גבי שליא מ"ט דאפילו הוה בה תרי שיעורי דמלא תרוד מטהר ר"ש דסתמא תנן ואע"ג דליכא למימר א"א שלא ירבו וכו', והם מפרשים הסוגיא כולה בענין אחר ברם נראין דברי מקצת ראשונים שפירשו מ"ט דר"ש דמטהר לגמרי והרי אנו מוצאים בכל יום ולדות חתוכים בשליא והאיך אפשר שלא תהא בכולה כזית ג) שלא נמוק לגמרי קודם שתצא שאפי' נחתך כולו לחצאי זתים מצטרפים הן בתוך השליא לטמא באהל ואמאי מטהר ליה לגמרי, ומפרקי' קסבר ר"ש כל טומאה שהיא כשיעורה ולא יותר שנתערב בה מין אחר בטלה דאמרינן כיון דהיא צריכה שיעור א"א שלא ירבה מין אחר על מקצתה ומבטלה ואף כאן כיון שנמוק הולד אע"פ שנשתייר ממנו (כחצאי) ד) זתים א"א שלא ירבו שתי טיפי מים ודם על מקצת בשר שלא נמוק ומבטלו והיינו דאמרינן ואזדא ר"ש לטעמי' דאמר א"א שלא ירבו שתי פרידו' עפר על פרידה אחת של רקב ומבטלו ובצר לה שיעורא ור"מ סבר לא מבטל אלא א"כ הוציאו לבית החיצון שנטרפו מימיו לגמרי וטהור. והא דאמר לו לר"מ כשם שאינו בבית החיצון כך אינו בבית פנימי לומר שאף בבית החיצון היה לנו לחוש שמא יש בו כזית בשר (שנמוק) ה) אנא משום בטול ואמר להו אינו דומה שזה נמוק לגמרי וה"ל כמים בטריפת בני אדם. אבל דרך לידה אינה נמוק לגמרי וביטול אינו מועיל. ואקשי להו רבא לרבנן דבי רב אדרבה כיון דרקב יותר הוא מן העפר היאך יאמ' ר"ש שהמועט רבה על מקצת המרובה ומבטלו ומגרע שיעורו אדרבה יש לנו לומר שהמרובה עומ' לבד על הממועט ומבטלו לגמרי. והיינו דאמרי' לקמן והשתא דאמרת טעמיה דר"ש סופו כתחלתו גבי שליא מ"ט דקס"ד שהמים והדם שבשליא מועטין הן אלא שרבין על מקצת בשר ומבטלין אותו כדפרישית וא"ר יוחנן משום ביטול ברוב נגעו בה שאפילו היו שם שני חצאי זתים או כזית שלם. יש במי שליא ודם שבה לבטל את כולה ואין אנו צריכין לומר שרבין על מקצת ומבטלו ומגרע שיעורו אלא על כולו הם רבים ומבטלין אותו ובהא פליגי דר"מ סבר אין טומאה בטלה ברוב מלטמא במשא ואהל דהא (קאמר) ו) מאהיל על כולה ומיהו בבי' החיצון טהור שנמוק לגמרי וה"ל אפר שרופין ופחות ממנו. ור"ש סבר בטלה היא לגמרי. 
\par  והא דא"ר יוחנן \textbf{מת שנתבלבלה צורתו מנ"ל דטהור.}  לאו דוקא דלא כרבנן אלא מדר' אליעזר שמע ליה ר' יוחנן דקסבר מודו ליה רבנן בשלא נעשה אפר כדפרישי'.\par  ויש לפרש דקסב' רבינא דר"ילא מודה לי' לר"ל בשפיר שטרפו מימיו דמדלא א"ל בשלמא שפיר שנטרפו מימיו דקאמר טהור לחיי אלא נתבלבלה צורתו שלמה מנלן אלמא ה"ק מנלן דטהור דגמרת מינה לשפיר לא הא ולא הא איתנהו. ועלה קאמר רבינא דר"י דמטמא שפיר שנטרפו מימיו לגמרי כר' אליעזר אמרה דהאי כאפר שרופין הוא ומיהו במת שנתבלבל' צורתו דקא מתמה מנלן לד"ה אתיא.\par וזה הלשון לדברי מי שגורס שפיר שנטרפו מימיו דמשמע שנטרף לגמרי וחזר למים, אבל לפי גר"ח ז"ל שנטרפו במימיו. נראה דהיינו נתבלבלה צורתו בלחוד.\par ויש לי עוד לומר דר' יוחנן הלכה קא מיבעי ליה, וה"ק ליה מנלן דטהור כרבנן דילמא טמא כר' אליעזר דמסתברא טעמיה. אילימא מדרבי שבתאי קא גמרת הלכה דהוא אמורא וקא פסיק הלכה כרבנן. }
\newsection{דף \hebrewnumeral{28}}
\twocol{\textbf{המפלת יד חתוכה.}  פי' רש"י ז"ל חתוכה שיש לה חיתוך אצבעות. וק"ל בלאו הכי נמי ליחוש ללידה שהרי אפילו השפיר שאין לו אפילו חתוך ידים עצמן אמו טמאה לידה.\par  ואיכא למימר הכי ספיקא הוא ואם הפילה יד גמורה שאינה חתוכה אומרי' מגוף אטום באת כשם שהיא משונה כך באת מגוף משונה ושמא לאו מגוף באת אלא חתיכה של בשר שנעשית כמין פיסת היד היא הילכך אמו טהורה תולין להקל שרגלים לדבר.\par והרב ר' אברהם בר דוד ז"ל מפרש שלא אמרו חתוכה אלא לענין מביאה קרבן ונאכל דמדקתני ואין חוששין כלל במשמע ואלו בשאינה חתוכה אינו נאכל. (אלא) א) לענין האם טמאה מ"מ. ואין זה לשון הגון מדקתני ברייתא אמו טמא' ואין חוששין ולא קתני מביאה קרבן ונאכל ואין חוששין. }
\newsection{דף \hebrewnumeral{29}}
\twocol{ הא ד\textbf{אמר ריב"ל עברה בנהר והפילה וכו'.}  בדין הוא דנירמי עליה מהא דתניא בריש פירקין ולשלישי הפילה ואינה יודעת מה הפילה מביאה קרבן ואינו נאכל אלמא לכ"ע הלך אחר רוב נשים לא אמרינן אלא מתוקמא ההיא כדתרצינן למתני' בשלא הוחזקה עוברה לפנינו. ודמתני' עדיפא לן למירמי. 
\par \textbf{שבוע קמא מטבילין לה בלילותא משום יולדת זכר ונקבה.}  עיינו בתוספות שאין השבועין הנמצאין כאן בטבילות הלילו' שוים עם השבועין הנמנים כאן בטביל' הימים דהא למאי דס"ד מעיקרא שבאת לפנינו ביום וכן למאי דמתרצינן כגון שבאת לפנינו בין השמשות טבילות דלילותא מושכות עד לילה של שבוע שלאחריו כגון שבאת לפנינו בין השמשות של מוצאי שבת וכן שבאת לפנינו באחד בשבת ביום וטבילה ראשונה של לילה בליל שני בשבת ואחרונה במוצאי שבת וכן בשבוע שני ואלו טבילו' דימים דמשום זיבה ראשונה באחד בשבת ואחרונה בשבת. וליכא למימר דברייתא הכי קתני שהביאה לפנינו ג' שבועין טהורין חוץ מיום שבאת לפנינו שהרי אותו היום עילה הוא למנין שבועים ונמצאת זאת מותרת לשמש בלילי עשרין וחד שהרי אינה רואה כל אותה הליל' ולא יום שלאחריו אלא ע"כ יום שבאת לפנינו הוא ממנין שלשה שבועים טהורין. כל זה עיינו בתוספות.\par ודבר ברור הוא אלא כיון דמנין לידה וזיבה מיום א' בשבת הוא וכל טבילו' דעלמא הן דכל נדה ויולדת טבילתן בלילה של שבוע שני וטבילות דזיבה ביום בסוף שבוע שלהן לא חיישי בגמרא לפרושי הכא מידי. }
\newsection{דף \hebrewnumeral{30}}
\twocol{והא דאמרינן \textbf{ש"מ תלתא.}  איכא למידק ולימא נמי ש"מ ד' דהא ש"מ ימי לידה שאינה רואה בהן אין עולין לה לימי זיבת' ואיכא למימר דההיא פלוגתא דאביי ורבא היא ורבה דאמר עולין קסבר הא מני ר' אליעזר הוא דאמר מסתר נמי סתרא. ולפום הכי נמי לא אמרי' ש"מ ר' אלעזר היא כדאמרינן ש"מ ר' עקיבא היא ור' שמעון היא. משום דלאביי דברי הכל אינן עולין הלכך לא פסיקא ליה. }
\newchap{פרק \hebrewnumeral{4}\quad בנות כותים}
\twocol{מתניתין \textbf{בנות כותיים נדות מעריסתן.}  אוקמינן בגמרא לר"מ דחייש למיעוטא וקסבר ר"מ כותיים גירי אמת הן דהכי אסיקנא בב"ק (דף לח) לדידיה וכיון שהן גירי אמת והן מטמאות בנדה מן התורה יש לחוש לספיקן והיינו נמי דקתני אין חייבין עליהן על ביאת מקדש מפני שטומאתן בספק.\par וא"ת ולמה העמידו משנתינו לר"מ לחוד דחייש למיעוטא. והא אפי' לר' יוסי נמי אית ליה בנות הכותיים נדו' מעריסתן כדאמרינן בפ"ק דשבת (דף טז ע"ב) גבי י"ח דבר לר' יוסי בצרי להו ואמר ר' נחמן בר יצחק בנות כותים נדו' מעריסתן בו ביום גזרו כלומר גזירה בעלמא כדי שלא יטמעו בהן או גזירה משום מיעוט שהן טמאות.\par  י"ל כיון דמתני' קסבר כותיים גירי אמת הן דהיינו סבריה דר"מ ור' יוסי שמעינן ליה דפליג עליה וסבר גירי אריות הן כדאיתא במנחות בפרק ר' ישמעאל (דף סו) ובמקומות אחרים הילכך ניחא לן לאוקמא לדידיה ומדינא ועוד דאיהו סתם מתני' ולא למשקל תנאי מעלמא ומשו' גזרת י"ח דבר. ועוד דקתני לה דומיא דסיפא דכותיים עצמן והתם לאו נזיר' אלא דינא הוא לחוש לספיקן.\par  וזה שכתבנו לפי גרסת מקצת הספרים אבל מהרבה מהן מספרי הגאונים שלא נמצא שם במס' שבת אותה הגרס' כלל ואעפ"כ חשבון י"ח דבר עונה להן יפה. }
\newsection{דף \hebrewnumeral{32}}
\twocol{\textbf{א"ר יוחנן לית כאן לאסר וכו'.}  משמע דסיכה כשתיה דרבנן ואינה מחוורת מן התורה. 
\par \textbf{פשיטה דהא קא דרס להו.}  פי' לאו פשיטא מגופא דמילתא אלא פשיטא דכל דקא דרים להו רחמנא רבינהו למדרס דתניא בת"כ אשר ישב עליו הזב אין לי אלא יושב ומגע מניין לעשרה מושבות זה על גב זה ואפילו על גבי אבן מוסמה ת"ל והיושב על הכלי אשר ישב וכו'. ומשום דמילתה רגילה היא בתלמודא הוא קאמרינן פשיטא דלא ה"ל הכא למיתני אלא שמטמ' מדרס.\par  ובפי' עליונו של זב שמעתי דברים רבים והנכון מהם מה שאמרו משם ר"ש ז"ל שהוא דבר הנשא עליו כגון הוא בכף מאזנים ומשכבות ומושבות בכף שניה וכרעו הן טמאין מדרס כרע הזב זהו עליונו של זב ומטמאין אוכלין ומשקין. ואתינן למיבעי מנלן דתניא ובל הנוגע בכל אשר' יהיה תחתיו מאי תחתיו אלימא תחתיו דזב דהיינו משכב ומושב מאיש אשר יגע במשכבו נפקא. ואי קשיא לך אדרבא הוא מטמא בגדים דכתיב ביה יכבס בגדיו והכא ליכא כבוס אה"נ אלא גמרא לא איצטרך למיחת לה כולי האי. וקאמר סתם כל טומאה דמדרס מהתם היא כדכתיבנא ביה ולא מהכא ועוד דאי הוה נקיט טעמא מהך קושיא דכבוס בגדים דילמא הוה אמרינן דכי כתיב והנושא אותם יכבס בגדיו ארישא דקרא נמי קאי ולא בעי עיוליה נפשיה בספיקא דקושיי. }
\newsection{דף \hebrewnumeral{33}}
\twocol{\textbf{מתקיף לה רמי בר חמא ותספרנו ואנן נמי ניספריה וכו'.}  פי' רמי בר חמא טעמא הוה בעי אבל ודאי ליכא דסליק אדעתיה דדינא הכי הני ספרה אנן כדמקשינן בפ' בתרא דמכילתן א"ל רב ששת לרב ירמיה רב ככותאי אמרה לשמעתיה דאמרי' יום שפוסקת בו סופרת למנין ז'.\par  ואי קשיא ההיא דגרסינן בפסחים פ' כיצד צולין (דף כא) ר' יוסי אומר שומרת יום כנגד יום ששחטו וזרקו עליה בשני שלה ואח"כ ראתה אינה אוכלת ופטורה מלעשות פסח שני ומפרשינן טעמיה דקסבר מכאן ולהבא מיטמיא דמקצת היום ככולו ובעינן עלה אלא לר' יוסי זבה גמורה היכי משכחת לה בשופעת ואיבעית אימא בגון שראתה שני בין השמשו' אלמא אמרינן מקצת היום ככולו.\par לאו מילתא היא דבסוף מנין אית ליה לר' יוסי מקצת תחלת היום ככולו בין זבה גדולה ובין בקטנה דשני שלה סוף מנין הוא דהא אנן נמי בזבה גדולה קי"ל כר"ש דאמר אחר מעשה תטהר אלא לדידן סותרת בכל היום ולר' יוסי לית ליה סתירה לאחר מקצת יום דהא שלימה היא טהרתה אבל בסוף יום ותחלת מנין דכ"ע לית להו מקצ' היום ככולו אלא לכותאי.\par וראיתי מי שמקשה כאן מאותה שאמרו בפ"ק דר"ה (דף י) אמר רבא ק"ו ומה נדה שאין תחלת היום עולה לה בסופה סוף היום עולה לה בתחלת שנה שיום א, עולה לה בתחלתה. וא"כ לר' יוסי נימא ק"ו ויהיה סוף היום עולה לה בתחלתה. וזה המקשה יכול להקשות כן בזבה גדולה לרבנן (ובין א) בקטנה דליכא בינייהו אלא סתירה ולפי דעתי שאין זו הקושיא דמקצת יום טמא ככולו טמא ומקצת יום טהור סוף היום כתחלתו בין בתחלתה בין בסופה הילכך לענין זיבה ביום נקי ליכא למיספריה אבל לענין נדה אפילו כולו נמי כימא סופרתו כנ"ל.\par  ומיהו מקצת היום שעולה בספירה דזבה דוקא ביום אבל לילה אינה עולה לספירה כלל כדאמרינן בפ' בתרא דמכילתן ושוין בטבילות לילה לזבה שאינה טבילה ותנן נמי במס' מגילה דאינה טובלת עד הנץ החמה.\par וההיא דאמרינן מפ' כיצד צולין דמוקי לדר' יוסי לרואה בין השמשות וכן נמי איתא במס' נזיר (דף כז) וגרסי' בה הכי בנוסחי לדר' יוסי מכדי קסבר מקצת היום ככולו זבה גמורה דמייתי קרבן היכי משכח' לה כגון דחזאי פלגיה דיומא אידך פלגא דלמפרע סליק ליה שימור פי' דלמפרע היינו פלגא דיומא בתחלתו שעבר עליה בטהרה ומתרצי איבעית אימא דקא שפעא ג' יומי בהדי הדדי ואיבעית אימא דחזאי תלתא יום סמוך לשקיעת התמה דלא הוה שהות סליק ליה למנינא. ההיא לרוחא דמילתא איתמר דלא בעי לאתויי עלה התם קרא דמגלה דאמרינן כיון דבעי ספירה ספירה ביממא היא ואוקמוה בסוף היום ותחלתו דליכא שהות דספירה בין ראיה לראיה כלל.\par וי"מ דלא בעיא לאוקמי זבה גדולה בלילואתא דוקא משום דקראי ביממא כתיבי דכתיב ימים רבים כל ימי זובה ולקמן בשלהי מכילתין ואימא ביממי תהוי זיבה בלילואתא תהי נדה ובפ"ק דהוריות נמי אמרינן גבי צדוקין דאמר דזבה לא הויא אלא ביממא דכתיב כל ימי זובה הילכך אע"פ דמפקינן מקראי אפילו לילותא לא מפקינן קרא מימים. 
\par \textbf{מתני רבי יוסי בשם רבי יוחנן במגעות שנינו,}  פירוש אלו השנויין כאן אינן עושין מדרס בלא נגיעה אלא שאם נגעו בבגד עשאוהו כמדרס מדבריהם, רבי זעירא בעי קומי רבי יוסי מהיכן נטמא הבגד הזה מדרס א"ל תפתר שהיתה אשתו של עם הארץ יושבת עליה ערומה, פירוש ר' זעירא מקשי על רבי יוסי לדבריך מהיכן נטמא מדרס לא היה לנו לטמאן אלא מגע הזב פריק שאם נגעה בו אשתו בישיבה עשאוהו כמדרס אבל ישבה עליו בבגדיה לא גזרו עליו, שמואל בר בא בעי קומי ר' זעירא כמה דתימר תמן אין היסט בחולין ויש היסט בחולין על ידי מגע ודכותה אין משא בחולין ויש משא בחולין על ידי מגע וכו' גופו של פרוש מהו שיעשה כזב אצל תרומה מתיב ר' תנן והתנן המניח עם הארץ בתוך ביתו בזמן שהוא רואה את הנכנסים ואת היוצאים האוכלין והמשקין וכלי חרס הפתוחין טמאין אבל המשכבות ומושבות וכלי חרס מוקפין צמיד פתיל טהורים אין תימר עשו גופו כזב אצל תרומה אפילו מוקפין צמיד פתיל יהיו טמאין אמר רב ר' יהודה בר פזי תפתר בעם הארץ אצל הפרוש לא עשאוהו כזב אלא שגזרו על בגדיו מדרס במגע אשתו כדאמרן ואקשי' אמר ר' מונא כן אמר ר' יוסי רבי כל מה דאנן קיימין הכא בתרומה אנן קיימן תדע לך שהוא כן דתנינן אפילו מובל ואפילו כפות הכל טמא כלום אמרו יהו הן טמאין אלא משום היסט לא כן אמר ר' יוחנן לאו חציצות ולא הסיטו ולא רשות היחיד ולא רשות עם הארץ אצל תרומה, ע"כ גמר'.\par  וה"פ דקא מקשי ליה רבי מונא לר"י בן פזי דאוקמא למתני' בחולין דודאי בתרומה קיימי' מדקתני סיפא הכל טמא ואי בחולין מדרסות והיסטות טהורין הן דאמר רבי יוחנן שלא אמרו שיהא דבר חוצץ במדרסות ולא טהרו היסטות ולא חלקו בספק רשות היחיד ולא רשות עם הארץ אצל תרומה הא אצל חולין הכל טהורין.\par וברייתא היא אצל זו ששנויה בתוספתא דחגיגה דקתני ספק רשות עם הארץ מדרסו וחצירו והיסטו טהורין לחולין וטמאין לתרומה, אלמא מתניתין דקתני הכל טמא בתרומה היא ושמע מינה שלא עשו גופו של פרוש ולא של עם הארץ כזב לטמא משכבות ומושבות והיסט אלא שחששו בזמן שאינו רואה את הנכנסים לאשה או לכותי לתרומה ולחולין הכל טהור ואפילו ספק רשותו עד שיתברר לך שנכנסה אשתו לשם אי נמי בגדים שלו שאי אפשר שלא נגעה בהם אשתו במדרס, ע"כ הארכתי לכתוב מן התוספת והן מגיהין ב) ולא רשות עם הארץ לחולין אלא אצל תרומה ודבריהם הללו כולן כתבתים מפני שדברים ברורים הם וצריכין הן לכמה סוגיות שבגמרא. }
\twocol{גרסת הספרים כך היא וכן בפירושי ר"ח ז"ל: \textbf{תא שמע זובו טמא לימד על הזוב שהוא טמא במאי אלימא בזב גרידא לאחרים גורם טומאה לעצמו לא כל שכן אלא פשיטא בזב ומצורע ומדאיצטריך לרבויי בראיה ראשונה שמע מינה מקום זיבה לאו מעין הוא.}  ובודאי יפה פירש רש"י ז"ל שראיה ראשונה של אדם אחר אינו מטמא במשא אלא במגע (בקרי) [כקרי] וראיה שנייה מטמא אפילו במשא לקמן בפרק דם הנדה, והא דקאמר לאחרים גורם טומאה הכא קאמר לאחרים גורם שיהיו מטמאין במשא לעצמו לא כל שכן שיטמא במשא ומדין משא למשא פריך דאי גרס טומאה בעלמא קאמר אף בזב מצורע גורס טומאה דמשכב ומושב לטמא אדם ולטמא בגדים ולטמא נמי בהיסט שאין מצורע עושה כן אלא מדין משא גופיה פריך כדפרישית.\par  ומיהו צריכין אנו לישב גרסת הספרים, ור"ש אומר פירוש שהיא בספרים והכי קאמר ומדאיצטריך לרבויי בשנייה שמע מינה דבראשונה לא מטמא במשא דלאו מעין הוא ואין וה הלשון גמרא.\par  אבל יש לפרש שכך היא הצעה זו דאמר רבא תא שמע זובו טמא לימד על הזוב שהוא טמא במאי אלימא בזב גרידא ובראיה שניה דאלו בראשונה ולמגע ודאי לא צריכא קרא דלא גרע משכבת זרע ולא עדיף מיניה אלא פשיטא בשנייה ואכתי למה לי קרא לאחרים גורס טומאה ואפילו למשא עצמו לא כ"ש אלא פשיטא בזב ומצורע ואי בשניה מי גרע מזב גרידא אלא בראשונה ולטמויי במשא וש"מ תרתי ש"מ ראיה ראשונה של מצורע מטמא במשא וש"מ לאו משום דמעיין הוא כדרב יוסף דאי הכי לא איצטרך רחמנא לרבויי הכא דממעינות נפקא אלא דרחמנא רבייה לראיה ראשונה של מצורע כראיה שניה של זב גרידא.\par ואי קשיא נימא קרא לראיה שניה והא קמ"ל דוקא בשניה אבל בראשונה אינה מטמא דלאו מעיין הוא, זו אינה תורה דמי איכא ספיקא קמי שמיא במקום זיבה אי מעין הוא או לא ואיצטרך ליתורי קרא למיגמר מיניה דלאו מעין הוא, ועוד דכל היכא דקרא מרבה כגון זה דכתיב זובו טמא דרשינן ליה לרבויי כגון לרבו' ראי' ראשונה למשא ולא מוקמינן ליה ליתורא למימר בשניה כתיב ולמעוטי ראשונה איצטרך כנ"ל.\par ומה שהקשה רש"י ז"ל מי איכא לאוקומי להאי קרא בראיה ראשונה והא מהכא נפקא לן בכל דוכתא מנה הכתוב שתים וקרא טמא, אינה קושיא דהאמרינן בפרק יוצא דופן דלמאן דאית ליה מנה הכתוב שתים וקרא טמא לית ליה זובו טמא לימד על הזוב שיהא טמא ותנאי היא.\par וכן זה שאמר הרב ז"ל דגבי מצורע איצטרך לרבוייה לטיפ' עצמה דלא אתי בק"ו משום דלא גרמה ליה טומא' שהרי מחמת נגעו הוא מטמא אין זה מחוור דכיון דאי לאו מצורע הוא נמי הות מטמי' איתא לק"ו מ"מ וכ"ש דאיכא לפרושי גרס טומאה בהיסט ומדרסות כדאמרן לעיל ומהסט למשא גמרינן ודאי דחד אורח הוא למשאות, ובמסקנא פשט אביי דמטמא במשא דהא אקשייה רחמנא למצורע אזב גמור, ולא פשט במעיין כלום משום דלא מרבוייא דקרא יתירא אתי דנימא למאי איצטרך אלא דמ"מ מטמא במשא הוא. }
\newsection{דף \hebrewnumeral{35}}
\twocol{הא ד\textbf{אמר רבא לאחרים גורם טומאה.}  ק"ל א"כ שכבת זרע דכתב רחמנא מלטמא לימא לאחרים גורם טומאה לעצמו לא כ"ש ולמה לי הא דתניא מנין לנוגע בשכבת זרע ת"ל או איש וכו' כדאיתא בפ' יוצא דופן, וכן (נמי קושיא) [דם עושה] משכב ומושב לאחרים והוא עצמו אינו עושה משכב ומושב כדאמרינן בפרק דם הנדה וכל המשכב אשר תשכב עליו נדה ולא דמה ובהא איכא למימר התם מיעטיה רחמנא דלאו בר משכב ומושב הוא ולית ביה אלא נגיעה בעלמא.\par תו קשיא והאיכא נמי היסט שהזוב גורם טומאת היסט והוא עצמו אינו מטמא בהיסט.\par ואיכא למימר נמי התם מיעטיה רחמנא מדכתיב והנושא אותם מיעוטא הוא בפרק דם הנדה (נה, א) א"נ בכל היינו פרכיה דעדיף מינייהו אמר ליה שעיר המשתלח יוכיח שאין לו טומאה כלל וגורם טומאה חמורה והך פירכא ופירוקא דרב יהודה מדסקרתא בברייתא תניא בהו בפרק דם הנדה ומדלא מייתו לה אינהו ש"מ לא שמיעא לה ובגמרא לא שמיע' להו ובגמרא לא אמרינן תניא נמי הכי דבשקלא וטריא בעלמא לא מיתמר הכי. 
\par \textbf{בשלמא לרב דאמר מעין אח' הוא מ"ה מטמא לח ויבש.}  פי' לבית הלל פשיטא ולבית שמאי נמי כיון שראיה זו מטמאתה מלספור נקיים נמצא שהיא גורמת טומאה וכדם הנדה הוא שמטמא לח ויבש ולא דמי לרואה בתוך ימי טוהר בלא זוב שאין ראייתה כלום אלא ללוי אמאי מטמא לח ויבש בין לבית שמאי בין לבית הלל, ופריק בשופעת.\par  אי בשופעת למאי איצטרך וק"ל ולרב גופיה אמאי איצטרך ודאי לבית שמאי ללוי נמי לבית שמאי ואיכא למימר בשלמא לרב טעמיה לבית שמאי קמשמע לן דלא תימא טעמייהו משום דשני מעיינות הן ובהא פליגי קמשמע לן ומודים ואי שני מעיינות הן ביולדת בזוב נמי מטהרו בית שמאי אלא ללוי אמאי איצטרך האי טעמא בתרווייהו מכל מקום שמע מינה דהא לית ליה לאוקמינהו אלא בהך פלוגתא ופריק אפילו הכי איכא למיטעי בה לבית שמאי דסד"א אף על פי שופעות לא תטמא קמשמע לן.\par  כיון דמפורש בשמעתין דלרב ימי טוהר שרואה בהן אין עולין לה לספירת זיבה, וקיימא לן בנות ישראל החמירו על עצמן שאפילו רואות טיפת דם כחרדל יושבות עליה שבעה נקיים וקיימא לן אי אפשר לפתיחת הקבר בלא דם אם כן היולדות צריכות שבעה נקיים בתוך ימי טוהר שלהן אלא שימי לידה נמי אם אינה רואה בהן עולין לספירת זיבתה כדלקמן וכן כחב הרמב"ם ז"ל שהיולדת בזמן הזה הרי היא כיולדת בזוב וצריכה שבעה נקיים. }
\newsection{דף \hebrewnumeral{36}}
\twocol{הא דתניא \textbf{ושוין ברואה אחר דם טוהר שדיה שעתה}  הקשו בתוספות אם כן מניקה שאמרו צריכה שתפסוק שלשה עונות והלא אחר דם טוהר היא רואה לעולם וכל שכן בזו שאמרו בברייתא שתים בימי עוברה ואחת בימי מניקתה וכו' למה לי הפסקה דעונות והא אחר ימי טוהר היא רואה ומוקמי לה כרבי מאיר וכשאינה מניקה וכגון שמת בנה ומשום רואה אחר ימי טוהר דיה שעתא.\par  ולי נראה דה"ק: קיימא לן דלא אמרו דיין שעתן אלא בראיה ראשונה וקאמר השתא שאם ראתה אחר דם טוהר אינה כראיה שנייה אלא כראיה ראשונה ודיה שעתה והוא שהפסיקה כדינה בעונות, וכן נראה פירש"י ז"ל ונכון הוא לומר דשויין אפלוגתא דר' מאיר ור' יוסי בדין כל ימי עבורן וימי מניקתן קאי ומתרץ לה רב בדליכא שהות ובין שראתה בימי טוהר בין שלא ראתה ליכא לטמוייה כלל, ואף על גב דלא הפסיקה נמי כדמפרש לה ואזיל. 
\par \textbf{כל שחל קישויה להיות בג' שלה וכו'.}  פירש רש"י ז"ל כל שקשתה אפילו שעה א' בליל כניסת ג' אפילו כל היום כולו בשופי ושעה א' מליל ד' להשלמה מעת לעת וילדה אין זו יולדת בזוב דבעינן שופי כל יום ג' המביא לידי זיבה.\par  פי' לפי' לאו משום (דכפי) [דבעי] חנניא לילה ויום כלילי שבת ויומו דאם כן היינו דר' יהושע אלא משום דבעי כל יום ג' בשופי ואם קשתה בג' אפילו שפת בד' כלילי שבת ויומו אינה זבה שאין קושי שבג' קובע אותה זבה ובכה"ג לא הוה זבה עד דחזיא ג' אח"כ בשופי דקושי שבשלישי אינו קובע בזיבהולא מצטרף (עד) [עם] יום ד' לקבעה בזיבה והיינו דאמרי לקמן לעולם כדקתני והא קמשמע לן אף ע"ג דאתחיל קושי בג' אם שפתה מעת לעת טמאה לאפוקי מדחנניא בן אחי ר' יהושע ואלו לאפוקי מדר' יהושע בהדיא קתני לה אם שפתה מעת לעת ר' יהושע אומר כלילי שבת ויומו. }
\newsection{דף \hebrewnumeral{37}}
\twocol{והא דאמר רבא \textbf{א"א בשלמא עולה היינו דלא מפסקת טומאה.}  ה"ק א"א בשלמא דינה לעלות בשאינה רואה אפילו ברואה נמי אינה סותרת שאין כאן טומאת ז' מפסקת אלא יומו הוא דלא חזי לעלות דומיא דרואה קרי שסותר יומו ואינו מפסיק כדאמרן בריש פירקן וכ"ש הכא דהאי דם לאו כגורם הוא טומאה כלל ולא מוסיף ביה טומאה דכלום אלא א"א אינו עולה האיכא טומאת ז' ושבועיים דמפסקא, כן נ"ל לפי' שמוע' זו ובתוספת מאריכין בה בענינים הרבה שאינן עולין.\par  ואיכא למידק אשמעתין דהא בשילהי בא סימן (דף כ"ד) איבעי להו ימי לידתה שאינה רואה בהן מהו שיעלה לה לספירת זיבתה, ואמר רב כהנא ת"ש ומסקנא ש"מ עולין ש"מ, וי"מ דהכא אליבא דר' מרינוס איירינן דאביי דאיק מדקאמר אינה סותרת מכלל דאינה עולה דה"ל למיתנא רבותא דעולה ורבא אמר אפילו לר' מרינוס עולה והא דאמר רבא מנא אמינא לה מואח' תטהר לומר דכיון דקרא קא דרשינן אפילו לר' מרינוס אית ליה, וכן הא דתניא מזובה ולא מנגעה מזובה ולא מלידתה קרא קא דייק והיינו דקאמר ליה אביי תני חדא כלומר לר' מרינוס דוקא חדא אבל ברייתא תרתי קתני והא דאמר אביי מנא אמינא לה לומר דכיון דמשכחת תנא דאמר אינה עולה ר' מרינוס היא דהא לרבנן עולה, וזה הפי' שמעתי ולא נתקבל לי.\par ועכשיו מצאתי בתוספות בשמו של ר"ש ז"ל שכתבו בתשובותיו ואמר הרב ז"ל תדע דהא בעי לה לקמן בפרק בא סימן איבעי להו וכו', ואין דרך התלמוד לשאול בעיא אחת שתי פעמים ולא מצינו כן בשום מקום תלאוהו באילן גדול, ועדיין אינו מחוור לפי שאם היה אביי מודה לרבנן דעולין לא הוה משוי ליה לר' מרינוס טועה וחולק דליכא למידק מלישנא דידיה הכי כלל כדפרישית, ועוד הא דפרישו בדרבא דאמר מנא אמינא לה דמקרא דייק ולא מצי רבי מרינוס למפלג עלה הא לאו מילתא היא דאי איכא למידק מקרא תיקשי לר' אלעזר דאמר דאינה עולה אלא היינו טעמא דאביי דאיהו סבר לתרוצא לההיא ברייתא דבשלהי בא סימן כדמתרץ לה רב פפא א' שאני התם וכו'.\par  ומסקנא דעולין ודאי כרבא אתיא דקי"ל כוותיה ולא כדברי הרב ר' יעקב ז"ל שפי' שהלמ"ד לידה כהכא אלא כפי קבלת הגאונים שהוא לחי במסכת עירובין (דף ט"ו) דאיתותב מיניה התם בגמרא ת"ש מעובדא דרב וההיא דתניא בפרק המפלת אינן עולין לרבא אתיא כר' אליעזר דאמר מסתר נמי סתרא והא דלא מייתינן לכולהו בשמעתין כמה איכא בתלמודא דכוותייהו שדברי תורה עניים במקומן ועשירים במקום אחר ומה שאמרו שלא מצאו בתלמוד בעיא א' בשני מקומות כאן מצינו.\par  ועי"ל לו דהתם אמוראי בחראי אתו למיפשט אי כאביי או כרבא והלכה או אין הלכה מיבעיא להו וכן מצינו בפרק המגרש (דף פה ע"ב) דאיבעי להו מי בעינן ודן או לא עביד בעיא סתם בפלוגתא ברבי יהודה ורבנן [ועוד בעיא בפלוגתא דמתני'] דמתני' בפרק המקבל (דף קי"ד) מהו שיסדרו בבעל חוב וכן בפרק חזקת הבתים (דף מ ע"ב(איבעיא להו סתמא מאי קא מיבעי ליה מתרי לישני דרב יוסף דלעיל הי מינייהו הלכה וזו כן ופשטו מברייתא דעולין ואידתי ליה דאביי דסבר לכ"ע בין לרבנן בין לר' אליעזר אין עולין. 
\par  והא דתניא \textbf{הכא מה ימי נדתה אין ראויין לזיבה ואין ספירת ז' עולה בהן.}  לקמן בשילהי בא סימן אמרינן זאת אומרת ימי נדתה שאינה רואה בהן עולין לה לימי זיבתה אלא שהן חלוקין בפירושיהן דהכא קאמרינן אין ספירת ז' של זבה גדולה עולה בימי נדה לפי שאינה נעשית תחלת נדה משראת נ' בזיבה עד שתספור שבעה נקיים והתם בזבה קטנה קאמרינן שאע"פ שראתה שנים בימי זיבה ונעשת נדה מונה יום אחד טהור מאותן ז' של נדה לזיבתה ודיה, ולשון אחר פירש שם רש"י ז"ל והכל שוין בדבר זה שמשעה שנעשת זבה גדולה כל ראיות שתראה אינה עולה בהן אלא סותרת ואינן ראויין למנות מהן ימי נדה וזיבה. }
\newsection{דף \hebrewnumeral{38}}
\twocol{והא דאמרינן \textbf{הא קמשמע לן דאפשר לפתיחת קבר בלא דם.}  אלמא אי הוה דם בפתיחת הקבר הויא זבה ליכא לאוקמא אלא בנפלים דלית להו קושי דאי בולד מעליא כי הוה דם בפתיחת הקבר נמי לא הויא זיבה דהא בקושי חזיתיה שאין לך קושי גדול מפתיחת הקבר אלא בנפלים הוא דמתוקמא ליה דמאה יום בלא קושי קתני, ואין זה נכון לומר דאין פתיחת הקבר בלידה קושי דלא מסתבר הכי ועוד דהא מכל מקום דמה מחמת עצמה ולא מחמת ולד קרינן ביה, וה"ה דמצו בגמרא למימר דא"א לפתיחת הקבר בלא דם ויש קושי לנפלים קמ"ל אלא רואה בלא קושי קתני דאלו בקושי יש רואה כל ימיה. 
\par  והא ד\textbf{אמר רבא בהא זכנהו ר' אליעזר.}  טעמייהו קא מפרש דודאי ר' אליעזר אפילו פרכיה לק"ו אינו בדין עד שיטמא ימי טוהר שהתורה טהרתם סתם, ועוד שאין לך טעם להחמיר עליו יותר מן השופי ורבנן נמי לא צריכי לק"ו אלא למיפרך טעמיה דר"א. }
\newsection{דף \hebrewnumeral{39}}
\twocol{\textbf{מקבע לא קבעה.}  פרש"י ז"ל דתיבעי ג"פ לעקרן, מיחש מהו דניחוש לה אם היתה רנילה מט"ו לט"ו דהיינו ימי זוב מיבעי' למיחש ולא תשמ' ליום ט"ו קודם ראיה שמא תראה ואינו יודע היכי אתינן למיפשט הא מילתא מדשמואל החס לר' פפא נמי ראית עשרין ותרין קמייאתא בימי נדה הוו וראית עשרין וז' דהאידנ' בימי נדה הויא, ולהאי פירושא אמאי לא תיחוש להו בכל זמן שיבא לה וסתה כיון שהוקבע הוסת כראוי בזמן נדה.\par  אלא אפשר לפרש מקבע לא קבעה מיחש מהו דתיחוש לה אם היה לה וסת בין קבוע בין שאינו קבוע בימים הראוים לוסת ואירע לה אותו היום בימי זיבה מהו שתחוש שמא בימים הללו תראה ולא תשמש או דילמא כשם שאינה קובעת וסת בתוך י"א כך אינה חוששת לוסת הראשון שלה בימי י"א והא מילתא מיפשטא בהדיא מדשמואל לפירושיה דרב פפא.\par  ולענין גמר' ודאי מסתברא דלית הלכתא כרב פפא אלא כרב הונא ברי' דר' יהושע ואיהו כיון דאידחיא לראיה דרב פפא מדשמואל ודאי מיפלג פליג עלי' וכיון דוסתות דרבנן אע"ג דלא איפשטא ליה לרב הונא לקולא אנן לקול' נקיטו בה ואע"ג דאמר רב פפא בשלהי האשה דלק' דחיישא, רב פפא לא מהימן בה דאיהו מדשמואל אמרה והא אידחי, אבל הרב רבי אברהם בר דוד ז"ל פסק בספרו כר' פפא ואף אנו עליו נסמוך וכ"ש מאחר שכתבנו שאין הנשים יודעת פתחי נדה שכל ראיה שהן רואות חוששין לה בכל זמן שהוא בתוך נדה הן. 
\par  גרסת רש"י ז"ל: \textbf{א"ל רב פפא אלא הא דאמר ר"ל וכו' אלמא מריש ירחא מנינן.}  כלומר מדקרי ליה לפעם ג' בתוך ימי נדה לפי חשבון הראוי מנינן.\par  וק"ל והא ר' פפא גופיה נמי אית ליה נדה ופתחה מכ"ז מנינן ולא לפי חשבון הראוי ועוד דילמא משום תרי זימני קמא דקביעינהו בימי נדה קרי ליה הכי.\par  ואפשר דהכי קאמר ליה והא הכא (דמעשרין וחמשה) [דמחמישה בירחא] לריש ירחא קמייאתא [{\small פי' לריש ירחא ב' וכונתו הפסקה קמייתא לריש ירחא} ] לא הפליגה אלא כ"ה יום ואע"פ כן כיון שאנו רואין לה עכשיו שהפליגה למ"ד מנינן מריש ירחא ולא מנינן מחמשא בירחא לרישי ירחא ונאמר עיקר הוסת בראשון ו) ירחא לריש ירחא הוא דאיכא למ"ד ובשני הויא ליה עיקרו של וסת מחמשה לחמשה דהא ודאי השתא בהפלגת למ"ד חזאי וקבעה לה וסת להפלגות למ"ד וראיות דחמשה בירחא קמא לא מפסקא, ולא מנינן מינייהו אלמא דמנין כ"ב מעיקרא נקטא והתם נמי כיון שכבר קבעה לה וסת ואנו צריכין לחוש לו מכ"ב לכ"ב מעיקרא וסת ראשון מנינן ואין ראיה שבאמצע מפסקת, ופריק ר"ה בריה דר"י דלא אמרי' הכי אלא ברואה מעיקרא (ממנין) [ממעין] סתום ועכשיו קרבה בתוך כך דאפילו בימי נדה אמרי תוספת דמים הוא, וכ"ש בימי זיבה דכיון דקמייאתא (ממנין) [ממעין] סתום והפליגה שתיים ועכשיו נמי ראתה יש לנו לומר תוספות הואי אבל לחוש אינה חוששת אלא למנין ראיות.\par  וה"ה ודאי דה"ל לתרוצי שאני התם דכיון דחזאי [בפעם ג' בחמישה בירחא] אמרי' דחמשה בירחא עיקר ודרישי ירחי (עיקר) [מיקרי] תוספות ואע"פ כן לחוש אין חוששין אלא בהפלגות ולא למנין הראוי. אלא מסתברא ליה דקביעות ראיות קמייאתא לא אמרי תוספת דמים הוו. וגם זה הפי' אינו מחוור.\par  ומ"מ היינו ריש ירחי דנקט לאו דוקא אלא הפלגו' שוות בהן דאלו בוסת החדש ודאי למנין הראוי חוששת דהא לאו בהפלגות שוות חזיא מעיקרא אלא בימי דחדש הוא דהשוות ראיותיה וכדבעינן לברורי קמן בפרק האשה (סד, א).\par ושמע מינה לרב הונא דאינה מונה כ"ב אלא מכ"ז ואי חזאי בכ"ב חזר הוסת הראשון למקומו ונעקר החדש לגמרי אבל אם לא ראתה בעשרין ותרין שחששת לו חוששת לסוף חמשה ימים דה"ל כ"ז מכ"ז שראתה בו תחלה והיינו דתרנגולת'. [{\small ע' ביאור דברי הרמב"ן בריטב"א} ] וכן דעת ה"ר משה ולא כן פסק הרב ר' אברהם בספרו ז"ל.  א"ל רב פפא אלא הא דאמר ר"ל וכו' אלמא מריש ירחא מנינן. כלומר מדקרי ליה לפעם ג' בתוך ימי נדה לפי חשבון הראוי מנינן.\par  וק"ל והא ר' פפא גופיה נמי אית ליה נדה ופתחה מכ"ז מנינן ולא לפי חשבון הראוי ועוד דילמא משום תרי זימני קמא דקביעינהו בימי נדה קרי ליה הכי.\par  ואפשר דהכי קאמר ליה והא הכא (דמעשרין וחמשה) [דמחמישה בירחא] לריש ירחא קמייאתא [{\small פי' לריש ירחא ב' וכונתו הפסקה קמייתא לריש ירחא} ] לא הפליגה אלא כ"ה יום ואע"פ כן כיון שאנו רואין לה עכשיו שהפליגה למ"ד מנינן מריש ירחא ולא מנינן מחמשא בירחא לרישי ירחא ונאמר עיקר הוסת בראשון ו) ירחא לריש ירחא הוא דאיכא למ"ד ובשני הויא ליה עיקרו של וסת מחמשה לחמשה דהא ודאי השתא בהפלגת למ"ד חזאי וקבעה לה וסת להפלגות למ"ד וראיות דחמשה בירחא קמא לא מפסקא, ולא מנינן מינייהו אלמא דמנין כ"ב מעיקרא נקטא והתם נמי כיון שכבר קבעה לה וסת ואנו צריכין לחוש לו מכ"ב לכ"ב מעיקרא וסת ראשון מנינן ואין ראיה שבאמצע מפסקת, ופריק ר"ה בריה דר"י דלא אמרי' הכי אלא ברואה מעיקרא (ממנין) [ממעין] סתום ועכשיו קרבה בתוך כך דאפילו בימי נדה אמרי תוספת דמים הוא, וכ"ש בימי זיבה דכיון דקמייאתא (ממנין) [ממעין] סתום והפליגה שתיים ועכשיו נמי ראתה יש לנו לומר תוספות הואי אבל לחוש אינה חוששת אלא למנין ראיות.\par  וה"ה ודאי דה"ל לתרוצי שאני התם דכיון דחזאי [בפעם ג' בחמישה בירחא] אמרי' דחמשה בירחא עיקר ודרישי ירחי (עיקר) [מיקרי] תוספות ואע"פ כן לחוש אין חוששין אלא בהפלגות ולא למנין הראוי. אלא מסתברא ליה דקביעות ראיות קמייאתא לא אמרי תוספת דמים הוו. וגם זה הפי' אינו מחוור.\par  ומ"מ היינו ריש ירחי דנקט לאו דוקא אלא הפלגו' שוות בהן דאלו בוסת החדש ודאי למנין הראוי חוששת דהא לאו בהפלגות שוות חזיא מעיקרא אלא בימי דחדש הוא דהשוות ראיותיה וכדבעינן לברורי קמן בפרק האשה (סד, א).\par ושמע מינה לרב הונא דאינה מונה כ"ב אלא מכ"ז ואי חזאי בכ"ב חזר הוסת הראשון למקומו ונעקר החדש לגמרי אבל אם לא ראתה בעשרין ותרין שחששת לו חוששת לסוף חמשה ימים דה"ל כ"ז מכ"ז שראתה בו תחלה והיינו דתרנגולת'. [{\small ע' ביאור דברי הרמב"ן בריטב"א} ] וכן דעת ה"ר משה ולא כן פסק הרב ר' אברהם בספרו ז"ל. }
\newchap{פרק \hebrewnumeral{5}\quad יוצא דופן}
\newsection{דף \hebrewnumeral{40}}
\twocol{\textbf{זאת תורת העולה היא העולה הרי אלו ג' מעוטין פרט לנשחטה בלילה וכו'.}  פירש לר' יהודה כיון דממעט הני אף על פי שפסולן בקדש ומכשר הלן והיוצ' ושארא כדבעינן למימר צריכי מעוטא לכל חד וחד אבל לר' שמעון כיון דקרא א' מרבה וקרא א' ממעט הריבוי ריבה הכל והמיעוט מיעט הכל, ול"ק כאן שפיסולו בקדש כאן שאין פיסולו בקדש.\par  והא דאמרינן הרי אלו [ג'] מיעוטין ולא אמרינן אין מיעוט אחר מיעוט אלא לרבות משום דכיון דכל אחד ממעט את שלו אין כאן מיעוט אחר מיעוט שאין מיעוט אתר מיעוט לרבו' אלא כשהן ממעט' דבר אחד כגון שאמרו (סנהדרין טו, א) עשרה כהנים כתובים בפרשה כהן ולא ישראל ואין מיעוט אחר מיעוט אלא לרבות אבל כאן כל מיעוט הוא צריך למעט את שלו.\par וכיוצא בזו בב"ק (מד, ב) שור שור שור ז' פעמים להוציא שור האשה ושור היתומין ושור האפוטרופסין וכו' ולא היו מיעוט אתר מיעוט והני תלתא מיעוטי נמי ממעטי הני תלתא פסולי כדפרישית.\par אבל במס' הוריות מצאתי בפרק ראשון בירושלמי (ה"א) גבי נפש כי תחטא אחת תחטא בעשותה [תחטא] הרי אלו מיעוטין דמקשי בכל אתר את אמרת מיעוט אחר מיעוט לרבו' וכאן את אמרת מיעוט אתר מיעוט למעט א"ר מתניא שניה היא דכתיב מיעוט אחר מיעוט לאחר מיעוטי ולדעת זו ההיא דאמרי' בסנהדרין כהן ולא ישראל מפני שכולן צריכין לכתב לומר דעשרה בעינן ד) אבל בשאר דוכתי ג' מיעוטין או יותר נדרשין הן כולן למיעוט כמשמען ולא נאמרה מדה זו בתורה אלא בשני מיעוטן מיעוט אחר מיעוט.\par  מ"מ כל הנך פסולי דמכשיר ר' שמעון מודה בהו ר' יהודה וטעמא מפרש במסכת זבחים (דף פ"ד) מפני מה אמרו לן בדם כשר שהרי לן כשר באימורין לן באימורין כשר שהרי לן כשר בבשר, יוצא הואיל וכשר בבמה, טמא הואיל ואשתרי לגבי צבור, ונשחט חוץ לזמנו הואיל ומרצה לפיגול, חוץ למקומו הואיל ואתקו' לחוץ לזמנו ושקבלו פסולין וזרקו את דמו בהנך פסולין דחזו לעבוד' צבור וכי דנין דבר שלא בהכשרו מדבר שהוא בהכשרו תנא אזאת תורת העולה ריבה קא סמיך הדין גמרא דהתם (ובתכפה) [וכתבנוה] מפני שהיא (תמה) [סתומה] ויש לדקדק בה דא"כ נשחטה בלילה נמי כשרה שהרי כשר בבמת יחיד כדאי' בזבחים (דף ק"כ), איכא למימר אתיא כמ"ד התם שחיטת לילה פסולה בבמת יחיד.\par  אלא הא קשיא יצא דמה חוץ לקלעים נימא דכשר שהרי כשר בבמה שאין יוצא בבמה לא בבשר ולא בדם וכן חוץ למקומו נמי דקאמר הואיל ואתקוש לחוץ לזמנו לימא הואיל וכשר בבמה, ועוד דקאמר ושקבלו פסולין וזרקו את דמו בהנך פסולין דחזו לעבודת צבור אפילו זר גמור נמי יהא כשר שהרי כשר בבמת יחיד.\par  ואיכא למימר נשחטה בלילה ויוצא דמה ליכא לאכשורי (בשרן) דאי הכי מיעוטין מאי אהנו לי ומסתבר' ליה לאוקמי בהני דבעיקר הכשרן נפסלו מאינך והא דאוקי בהנך פסולי דחזו לצבור ולא אמר משום דכשרין בבמה דבהא פשיטא ליה דליכא למילף מינה שזה ודאי דבר שעקר הכשרו כן הוא ולית להו הכשר בכשרין טפי מפסולין לעולם ודקאמרת חוץ למקומו הואיל ואתקוש משום דכל היכא דמשכח הואיל בפנים לא מייתי ליה מבמה, כנ"ל.\par ובתוספות מאריכין בע"א, וניתנין למעלה שנתנן למטה וכן בחוץ ובפנים ופסח וחטאת כולהו כיון דבפנים נמי אשכחן בהו הכשרא לא צריכא ליה למימר' דודאי לא ירדו. }
\newsection{דף \hebrewnumeral{41}}
\twocol{והא דאמרינן \textbf{חד לבהמת קדשים.}  פי' דאע"ג דהאי וזאת לא מיתר ביוצא דופן שאם עלה ירד מיהו אין במשמע תורת העולה לרבות כל העולי' לגמרי כיון שמקרא א' מרבה ומקרא אחר ממעט וא"א להעמי' המיעוט בבהמת תולין (דכולהו מעלמא) [דחולין מאמו] נפקי אפילו לענין שאם עלו ירדו הילכך ע"כ מוקמינן מיעוטא בבהמת קדשים וכיון דשקולים הם יבואו כולם וכל שהוא פסול בבהמת חולין מיעט בבהמת קדשים לפסול ולומר שאפילו אם עלה ירד דמיעוטא הכי משמע שאם עלו נמי ירדו. 
\par ואקשינן \textbf{מכלל וכו'.}  וכי תימא דילמא אשתייר ומספיק' אסרינן לה בטומא' אי הכי חיישינן שמא נשתיי' מיבעי ליה אלא ודאי מדלא קאמר הכי ש"מ דכל היכא דאזלא בכרעא מותרת בתרומה דודאי שדתיה לכוליה ודרבא בשלא הלכה הוא דקאמר והתם לא הוה צריך למימר חיישינן אלא א"א הוא. }
\newsection{דף \hebrewnumeral{42}}
\twocol{\textbf{עשאוה כנבלת עוף טהור.}  ויש לפרש דה"ק חכמים גזרו על טומאה זו מפני שדרכה לצאת ועשאו' כנבל' עוף טהור ולמיסמך גזירה דרבנן אכעין דאורייתא קאמר שאלמלא שמצינו טומאה בלועה מטמאה בדאורייתא לא הוו גוזרין בהו שמטמא. }
\newsection{דף \hebrewnumeral{43}}
\twocol{הא דאמרינן \textbf{את"ל בתר עקירה אזלינן לחומרא.}  ק"ל אמאי לא פשטה להא מילתא מהא דאמרן ונעקר ממנה דם בירידה טמאה בשלמא בעייא דרבא ל"ק דאיהו לא משום ספיקא דבתר עקירה אזלינן בלחוד מספקא ליה אלא משום דשמואל דכיון דאינו יורה כחץ בשעת טומאתו אינו מטמא אבל הא דאמרינן גבי ירד וטבל את"ל בתר עקירה אזלינן קשיא.\par  ואיכא למימר משום דכיון שטבל בנתיים א"א שלא בטלה הרגשתו ומיהו זבה שנעקרו מימי רגליה דמיא לההיא אלא דהתם מצי נקיט להו והא דקאמרינן ביה את"ל סרכא נקט. }
\newsection{דף \hebrewnumeral{44}}
\twocol{\textbf{מ"ט דאיהו מיית ברישא.}  פיר' ודוקא מתה דקאמרינן בערכין פ"ק (דף ז) אשה היוצאת ליהרג מכין אותה כנגד בית הריון שלה כדי שימות הולד תחלה והוינן בה למימרא דאיהי מתה ברישא והא קי"ל דאיהו מאית ברישא דתנן בן יום א' וכו'. ומפרקינן ה"מ מתה דאגב דולד זוטר חיותיה עולה ביה טיפה דמלאך המות ותתיך לה לסימנים אבל נהרגה איהי מתה ברישא ואע"ג דהאי תירוצא לאוקומי יש זכייה לעובר אתמר בפ' מי שמת מיהו קושטא הוא ולהכי הוינן מינה ומפרקי לה התם. א"נ למ"ד הכי מקשי מברייתא דקתני מכין אותה כנגד בית הריון ולאו למימרא דהכי הוא בודאי. 
\par והא דתנן \textbf{וההורגו חייב.}  ודוקא בן יום א' אבל עובר לא דלא קרינא ביה נפש אדם וה"נ אמרינן בסנהדרין (עב, ב) האשה שמקשה לילד מביאין סכין ומחתכין אותו אבר אבר יצא ראשו אין נוגעין בו שאין דוחין נפש מפני נפש אלמא מעיקרא ליכא משום הצלת נפש וקרא נמי כתיב דמשלם דמי ולדות.\par  ואיכא דקשיא ליה מההיא דגרסינן בערכין (ז, א) האשה שהיא יושבת על המשבר ומתה בשבת מביאין סכין וקורעין אותה ומוציאין ממנה הולד ואמאי מחללין שבת כיון שאינו קרוי' נפש.\par וליכא למימר התם ביושבת על המשבר דוקא משום דכיון דעקר גופא אחרינא הוא כדאיתמר התם בערכין במקשה לילד לא בעינן יושבת על המשבר ג) ועוד דהכא בן יום א' תנן וקרא דגבי דמי ולדות אפילו ביושבת על המשבר היא ולא אמרינן התם דכילוד הוא אלא גופא אחרינא הוא קאמרינן לומר שממתינן לה עד שתלד ואח"כ ממיתין אותה ולא מיתרבי מגם שניהם דאפילו קודם שתש' על המשבר כלל אי לאו קרא דגם לא הוה קטלינן לולד כדמפור' התם אבל לענין לידה דבר ברור הוא שאינו בכלל נפש אדם עד שנולד כדאמרינן.\par ולאו קושיא היא התם אמרה תורה חלל עליו שבת אחת כדי שיקיים שבתות הרבה והאי דאמרי' במס' שבת (קנא, ב) תינוק בן יום א' מחללין עליו את השבת לאו לאפוקי עובר אלא גוזמא היא כדי לומר דוד מלך ישראל מת אין מחללין עליו. }
\newsection{דף \hebrewnumeral{45}}
\twocol{מדתנן \textbf{בן ט' שנים ויום א' שבא על יבמתו קנאה.}  משמע קנאה לגמרי ליורשה וליטמא לה אלא שאינו נותן גט עד שיגדיל וכן נמי מדהוינן בה ולכשיגדיל בגט סגי לה והתניא עשו ביאת בן ט' כמאמר משמע דמדאורייתא קנאה לגמרי ובגט סגי לה אלא שהם גרעו כת ביאתו ועשאוה כמאמר וכן בדין שהרי ביאתו ביאה לכל דבר ואע"פ שאין לו דעת הא רבי רחמנא ביבמה ביאת שוגג כדמזיד וכבר פירשתי בפ' האשה רבה (יבמות צו, ב). }
\newsection{דף \hebrewnumeral{46}}
\twocol{\textbf{וכי קאמר רבא חזקה למיאון.}  אי קשי' למיאון למה לי חזקה בחששא בעלמא סגי והוה ליה למימר קטנה שהגיע לכלל שנותי' אינה ממאנת שמא הביאה שתי שערות. איכא למימר כי קאמר רבא חזקה לומר שאין ב"ד מטריחין עצמן לבדוק שלא תמאן ואם חששו היינו בודקין. א"נ אע"ג דאמרינן כי קאמר רבא חזקה למיאון לאו דחזקה אצטריכא ליה להכי אלא לומר דלא ממאנה וחזקה נמי היא ומהניא חזקה לנשים בודקת אותן כדלקמן בפ' בא סימן.\par והראשונים שאלו א"כ הא דאמרינן התם בודקין לה לחליצה ולמיאונין היכי משכחת לה ורבי' הגדול השיב נפקא מינה להגיעה לכלל שנותיה וקדש בתוך זמן ולא בעל אתר זמן דהוה דרבנן.\par  ואלמלא שזה דבר ברור יכולין אנו לפטור עצמינו משאלה זו במה שאמרו מקצת המחברים בודקין למיאונין היינו כדאיתמר עלה לאפוקי מדר' יהודה דאמר עד שירבה שחור על הלבן קמ"ל בודקין ומכי אתיא שערות לא ממאנה ולאו למימרא דבדיקה צריך אלא בדיקה זו היינו שערות לומר דמשהביאה אותן בין בבדיקה בין בחזקה אינה ממאנת וקטנה שלא נודע אם הגיעה לכלל שנותיה והביאה סימנין לא מצינו בגמ' דינה מפורש ויש שכוללין אף בזו בכלל בודקין למיאונין ואם הביאה סימנין אינ' ממאנת ואע"פ שלא בעל אלא קוד' זמן ולא תלינן בשומא לקולא וה"נ לשאר הדברים מטילין אותה כחומרא כדין הספקות. 
\par \textbf{והא אין הבעל מפר בקודמין כדר' פנחס וכו'.}  פי' והא נמי על דעת כן נדרה שאם הקפי' הבעל לא יחול נדרה ומיהו משלא בעל משהגדילה אינו מיפר שכיון שלא קנה קנין גמור אינו מפר שמתחלה מתלא תלי נדרה אם נתקיימו קידושיה יפר נדרה שעל דעת כן נדרה ואם לא נתקיימו הקדושין אף הוא אינו מפר שלא נדרה על דעתו שמא סבורה היא לצאת ומיהו האי תירוצא לא אתיא כרב הונא דאמר הקדיש ואכל לוקה שא"כ היאך היא אוכלת תחלה בהפרתו הרי עדיין קידושי' תלויין והגדר גדר גמור.\par ויש שגורסין אלא כדר' פנחסולא וכו'. כלומר לעולם בשלא בעל ולא מפר אלא שעה ראשונה ואעפ"כ אתיא הפרה דידיה ומבטל נדרה דאורייתא שעל דעתו נדרה מכיון שהוא חיי' במזונתיה ועומדת תחתיו ומשמשתו והאי פירושא עיקר ואפילו למאן דלא גריס אלא ה"ג מתפרש דבכמה דוכתא בתלמודא דהדר ביה מתירוציה קמאי ולא אתמר בהו "אלא". }
\newsection{דף \hebrewnumeral{47}}
\twocol{הא דתנן \textbf{איזהו סימניה ר' יוסי הגלילי אומ' משיעלה קמט תחת הדד וכו'.}  פי' רש"י ז"ל דאצמל קאי. וא"כ צריכין אנו לומר דשיעורא דמשתקיף העטרה ושיעלה הקמט תחת הדד חד שיעורא הוא או לומר דתרי תנאי אליבא דר' יוסי ועדיין אין הדברים נראין שנשנו שיעורין במשנה ואחרים בברייתא ולא הוזכרו של זה בזה כלל ועוד בוחל זה שאמרו בידוע שהביא שתי שערות והלא לא פי' אותו כלל לא במשנה ולא בברייתא וכן בגמ' לא הזכירו בפרק מהן. אלא איזה סימניה אבוחל קאי. ופי' מתני' סימן נערות דעליה קאי ברישא דמתני' ובוגרת לא קתני משום דממילא ידוע שאין בין נערות לבגרות אלא ו' חדשים בלבד ומאי שלא פי' במתניתין פי' בברייתא אלו הן סימניה בגרות וכו'. ועלה דהא מתני' קתני באידך פירקין בא סימן התחתון עד שלא בא העליון כלומר העליון השנוי שאלו לדברי רש"י סימן עליון אצמל דסליק מיניה משמע וליתא אלא אבוחל. 
\par הא דתניא \textbf{שנה האמורה בקדשים בבתי ערי חומה וכו'.}  כולן מעת לעת מיום ליום קאמר וקראי דמיום אל יום נסיב בגמרא והני כולהו דמיום אל יום שוין הן אבל מעת לעת ממש בעינן בקדשים כדאמרינן בפ' שני דזבחים (דף כה ע"ב) זאת אומרת שעות פוסלות בקדשים. וכן בבתי ערי חומה בעינן מעת לעת ממש כדאמרי' בפ' בתרא דערכין. אבל שבבן ושבבת דפירקין דיוצא דופן לא בעינן מעת לעת כדאיתמר לעיל (דף מד ע"ב) ערב ראש השנה דג' איכא בנייהו וכן לענין עבד עברי לא שמענו. כך מפורש בתוספות. }
\newchap{פרק \hebrewnumeral{6}\quad בא סימן}
\newsection{דף \hebrewnumeral{48}}
\twocol{הא דתנן \textbf{ר' מאיר אומר לא חולצת ולא מתיבמת.}  ר"מ לטעמיה דאמר (לעיל לב, א) קטן וקטנה לא חולצין ולא מיבמין. והא דקתני סיפא ותכ"א או חולצת או מתיבמת מפני שאמרו לאו דוקא מתיבמת דהא לרבנן קטנה נמי מתייבמת אלא איידי דקתני בדר' מאיר חליצה ויבום וקתני להו בכל דוכתא נקט נמי הכא חולצת או מתייבמת. 
\par \textbf{וסיפא דקתני ונאמנת אשה להחמיר.}  אוקימנא אב"א ר' יהודה ואתוך הפרק. וק"ל בשלמא נאמנת לומר גדולה היא שלא תמאן ואינה נאמנת לומר גדולה היא שתחלוץ ניחא אלא קטנה שלא תחלוץ פשיטא דנאמנות דאפילו שתקא א"נ אמרה גדולה היא אינה חולצת וקטנה היא שתמאן אמאי אינה נאמנת ואפילו נאמנת אמאי צריכה והא אמרת צריכה לומר גדולה היא שלא תמאן.\par  ואיכא למימר כולה ברייתא נאמנת ואינה נאמנ' במקום שהוצרכנו לעדותה היא והכי קתני נאמנת לומר גדולה היא שלא תמאן במקום שאנו צריכין לעדות (גדול) שלה שאלמלא עדות אשה זו ממאנת היא שבחזקת קטנה עומדת ואפילו בדקו עדים עכשיו ולא ראו בה שערו' נאמנות אשה זו לומר הביאה אותן ואינה ממאנת שמא נשרו.\par וכן נאמנ' לומר קטנה היא שלא תחלוץ במקום שאנו צריכין לעדות קטנותה כגון שבדקנו אותה ומצינו בה שערות אם אמרה אשה לפני זמן הביאתן נאמנת והיינו בדיקה דלפני הפרק ואגב אחרינא נקט להא. א"נ שלא תאמר כשהיא עדיין לפני הפרק נאמנת דעדיין בחזקת קטנה היא אבל לאחר שהביאה שערות והיא בזמנה והוחזקה גדולה בפנינו שמא תאמר אינה נאמנת לומר קטנה הוא שתוך זמן היו בה. וקמ"ל.\par אבל אין אשה נאמנת לומר תוך זמן קטנה היא שתמאן אם הוצרכנו לעדות זו כגו' שנמצאו בה שערות והוא שבעל בתוך זמן דה"ל ספק דאורייתא ואע"ג דהכא ליכא למימר שמא נשרו דהא אכתי תוך זמן זה הוא מיהו כיון שהיא גדולה בפנינו אין האשה נאמנת להקל בשל תורה לומר שומא הן וכן אינה נאמנת לומר גדולה היא שתחלוץ.\par  ולהך לישנא דאמרינן ואב"א ר' שמעון ולאחר הפרק ולית ליה חזקה דרבא וה"נ קתני נאמנת לומר גדול' היא שלא תמאן ואפילו אין בה עכשיו שערות וקטנה היא שלא תחלוץ אפילו היו בה אבל אין נאמנת לומר קטנה היא שתמאן כשהיו בה ובעל כדפרישית ולא לומר גדולה היא שתחלוץ.\par  ומצינו נוסחא אחרת. "ונאמנת אשה להחמיר אבל לא להקל כיצד גדולה היא שתמאן גדולה היא שתחלוץ". וכן גרסת רבינו הגדול ז"ל בהלכות. ונוסתא ישרה היא ופירושא נאמנת להחמיר לומר גדולה היא לענין מיאון ואינה נאמנת להקל לומר גדולה היא לענין חליצה כלומר מיאון וחליצה היינו להחמיר ולהקל מיאון היינו להחמיר חליצה היינו להקל. }
\newsection{דף \hebrewnumeral{49}}
\twocol{\textbf{מוציא כשר למי חטאת ופסול משום גסטרא.}  יש מקשים, ואפילו בלא מכניס ומוציא היאך יהיה כלי חרס כשר למי חטאת והלא שנינו במס' פרה כל מעשיה אינן נעשין אלא בכלי אבנים ובכלי גללים ובכלי אדמה. וי"ל ההיא מעלה בעלמא היא משום שמטמאין היו הכהן השורף אותה להוציא מלבן של צדוקים שהיו אומרים במעורבי שמש היא נעשית אבל מכניס פסול דין תורה הוא.\par וי"מ שעד שלא נתנו אפר במים היו מעשיה בכלים הללו שאין מקבלין טומאה אבל משנתנו אפר לתוך המים שוב אינה מקבלת טומא' כדאמרינן בעלמא א) מי תטא' שנגע בהם שרץ טהורין וכיון שכן בכל כלים היו נותנים ואפילו בשל חרם ולכך פסלו מכניס. 
\par הא דאקשי' \textbf{תנינא חדא זימנא הכל כשרין לדון דיני ממונו' וכו'.}  ק"ל דהא כולהו תננהו כל חדא וחדא בדוכתא כדפריש בגמ' (לקמן נ, א) כל שחייב בפאה חייב במעשרות ממתני' דהתם וכן כל שיש לו ביעור יש לו שביעית והכא ודאי אגב גררא דכיוצא בו קתני להו וא"ת לרב יהודה פרכי' הא לאו מילתא היא דהא רב יהודה פרושי קא מפרש להו דהכא והתם חדא קתני ועוד דהא עיקר מתני' מצרכי' חדא לגר וחדא לממזר.\par וי"ל למימרא דרב יהודה ודאי ק"ל למה לי לפרושי תרתי זימנא ומתרצי' דלא איתמר ממזר אלא גר איתמר כך פי' בתוספת והביאו דומה לה משבת פ' ח"ר עקיבא דקאמר וצריכא על מימרא דרב יהודה אמר שמואל דאמר הלכה כר' עקיבא דכל מלאכה שאפשר לעשותה מע"ש אינו דוחה את השבת וכו' ואינו מחוור לי שאם אין המשניות מיותרת מנין לו לרב יהודה לפרושי חדא לגר וחדא לממזר. וי"ל דרב יהודה סברא דלפשיה קאמר דכולהו פסולין שהן ישראל כשרין לדיני ממונות ולא מיפסלי יוחסין אנא לדיני נפשות.\par וה"ג וכן בנוסח' אי' בסנהדרין (לו, ב) וצריכא דכי אשמועינן ממזר משום דאתי מטפה כשרה ואי אשמועינן גר משו' דראוין לבא בקהל וכשירין לדיני ממונו' מצרכינן דרב יהודה אהכל כשרין דרישא קאי. וקאמר לאתויי ממזר דכשר אבל אין הכל כשרין לדון דיני נפשות אלא כהנים וכו' פשיטא לנר ולממזר ואפילו חלל כולן פסולין הן וכדקתני סיפח התם ואין הכל כשרין לדון דיני נפשות אלא כהנים ולוים וישראלי' המשיאין לכהונה וחלל אינו מן המשיאין לכהונה.\par וי"א דאסיפא קיימינן לומר דממזר וגר פסולין לדון נפשות וכן כתוב כאן בנוסחאות וצריכא דאי אשמועינן גר משום דאתי מטפה פסולה וכו' ומפרשי' מדקאמרינן לאתויי גר וממזר ש"מ דפסולי כהונה כגון חלל כשרין לדיני נפשות דלאשמועינן דאפילו גר וממזר כשרין לדיני ממונות לא קאמרינן דהא בגמר' דיני נפשות מצרכינן כדאמרינן דאי אשמועינן גר משום דקאתי מטפה פסולה ולהכי פסול לדיני נפשות אבל ממזר דאתי מטפה כשרה אימא לא וכו' וכיון דלא אשמועינן אלא גר וממזר ש"מ דחלל כשר אפילו לדיני נפשות. והא דקתני בסנהדרין ואין הכל כשרין לדון דיני נפשות אלא כהנים לוים וישראלים המשיאין לכהונה הא פריש רב יהודה דלמעוטי ממזר וגר אתי אבל חלל אינו בכלל.\par ואינו נכון כלל דחלל נמי בהדיא ממעיט מהתם ועוד דתנן אין בודקין מן הסנהדרין ולמעלה ומפקינן לה מדכתיב ונשאו אתך בדומין לך. והם דוחין לזו דהתם בממונין סנהדרין קבועים אבל כשר הוא להושיבו בדיני נפשות ולמנות עמהן אע"פ שאינו משיאו לכהונה.\par וכן ב) פירש"י ז"ל הא שאמרו ביבמות (קב, א) שאם היתה אמו מישראל דן ואפילו ישראל [דהוא דיני נפשות] ורבינו יצחק בעל הלכות ז"ל פי' לזו [דמתני'] בשאמו מישראל במס' סנהדרין. וכבר פרשתיה שם ביבמות (מה, ב). }
\twocol{\textbf{נבלת בהמה טהורה בכל מקום ונבלת העוף הטהור והחלב בכרכים אינם צריכים לא מחשבה ולא הכשר.}  פי' נבלת בהמה ונבלת העוף אינם צריכים לא הכשר מים ולא הכשר שרץ מפני שסופן לטמא טומאה חמורה. והחלב אינו צריך הכשר מים שכבר הוכשר בדם שחיטה אבל הכשר שרץ צריך לצדדין קתני ואין אתה יכול לאומר' בחלב נבלה דא"כ צריך הוא הכשר מים ושרץ שהוא אין סופו לטמא טומאה חמורה דכתיב וחלב נבלה וחלב טרפה יעשה לכל מלאכה וא"א נמי בחלב של טמאה שהיא צריך מחשבה שהרי נבלת טמאה בכל מקום צריכין מחשבה ואין חלבה חלוק מבשרה אלא בחלב של שחיטה הוא ואינו צריך הכשר מים עכשו מפני שכבר הוכשר בדם שחיטה אבל הכשר שרץ צריך ולצדדין קתני הכשר כדפרישי'.\par  ובסיפא גרס' נבלת בהמה טמאה בכל מקום ונבלת עוף טהור בכפרים צריכה מחשבה ואין צריכין הכשר ולא גרם בה חלב כיון דחלב בכפרים צריך מחשבה הכשר נמי צריך שלא הוכשר בשחיטה מפני שקדם הכשר למחשב' והבשר קודם למחשבה לא הוי הבשר כדאיתא בהעור והרוטב. ובנוסח המשניות נמי אין בהם חלב בסיפא. 
\par  הא דאמרי' \textbf{לבני מערבא דמברכין בתר דסליקו תפילייהו לשמור חקיו.}  פי' ר"ת ז"ל בספר הישר שלו שלא אמרו אלא בתפילין אבל בציצית ושאר מצות אין מברכין לאחר עשייתן.\par  והביא ראיה ממה שאמרו בירושלמי בפ' היה קורא בתורה כיצד הוא מברך עליהן ר' זירקן בשם ר' יעקב בר אידי כשהוא נותן של יד מהו אומר בא"י אמ"ה על מצות תפילין וכשהוא נותן לראש מהו אומר אקב"ו על הנחת תפילין. וכשהוא חולצן מהו אומר ברוך וכו' לשמור חקיו ואתיא כמ"ד בחוקת תפילין הכתוב מדבר ברם כמ"ד בחוקת הפסח הכתוב מדבר לא כר"א [{\small לפנינו שם }  לא בדא {\small ואם הגירסא נכונה }  כר"א קאי על למטה ע"ש] והטעם לזה מפני שמניח תפילין לאחר שקיעת התמה עובר בעשה הילכך מברך בשעת סילוקן בלילה שהוא מקיים עשה ואין לך כן בכל המצוות, כך פי' חכמי הצרפתים בשמו ז"ל.\par ועדיין אינו מחוור, דא"כ הא דאמרינן בשמעתין לאתויי מצות ומקשי ולבני מערבא דמברכין בתר דמסלקי תפילייהו מאי איכא למימר מאי קושי' מתני' לאתויי כל שאר המצות. ועוד יש נסחאות שכחוב בהן ולבני מערבא דמברכי אמצות וכו'.\par  אלא נראה לבני מערבא ה"ה לכל מצות שטעונו' ברכה לאחריהן וז"ש בירושלמי אתיא כמ"ד בחוקי התפילין לא הקפידו אלא על הלשון דלשמור חקיו אבל שאר כל המצות אין מברכין אלא לשמור מצותיו ובודאי נראה לומר שאין בני מערבא מברכין אלא כשהן מסלקין אותן בזמן ערבית ולא משום עשה שבהן אלא משום שכבר נגמרה מצותן דקסברי לילה לאו זמן תפילין הוא וא"כ סלקו אותן בע"ש ובערבי י"ט לד"ה מברכין היו אבל אם היו מסלקין ביום היאך יברך הלא מצוה להניחן ולא לסלקן. ולפיכך אמרו בירושלמי דאתיא כמ"ד בחוקת תפילין הכתוב מדבר וכתיב מימים ימימה ולא לילות וכך סמכו שם בירושלמי ר' אבהו בשם ר' אלעזר הנותן תפילין בלילה עובר\par בעשה מה טעם ושמרת וכו'. אבל נאמר לפי"ז הענין ולפ"ז הפי' שאין מברכין על כל מצוה שאין סילוקה גמר עשייתה כגון פושט ציצית ביום והיוצא מן הסוכה אבל בלילה מברכין על ציצית וכן לאחר שופר ולולב וכל כיוצא בהן שעשייתן גמר מלאכתן מברכין וזה שלא העמידו משנתינו דיש טעון במצות כיוצא באלו שאינן טעונות ברכה מפני שאין לשין לאחריו אלא לאחר שנגמר המעשה.\par וזה הלשון נכון הוא שאין הדין נותן לברך לאחרי' במצוה שעדיין הוא חייב בה והוא מסלקה ממנו שא"כ מצינו חוטא ומברך ואין לך כן אלא בקורא בתורה ובצבור מפני שהוא מצוה לגמור כדי שיהיו ג' או ז' קוראים כתקנת חכמים. אבל בגמר מצוה בכל מצוה נגמרת מברכין היו ודמיא להו להלל ומגלה ותורה בצבור וראינו לרבינו האי גאון ז"ל שכתב בהא דבני מערבא לא נהגינן הכי במתיבתא ומיהו אי בעי אינש למיעבד כבני מערבא שפיר דמי.\par ולשון הירושלמי שכתבנו נראה שמכריע כדברי בעל הלכות ז"ל שהצריך לברך א' של יד וא' של ראש אף על פי שלא שח. וכן החזירו שם הענין הזה בפרק הרואה ואמרו העושה תפילין לעצמו אומר בא"י אמ"ה לעשות תפילין לשמו כשהוא לובשן אומר בא"י אמ"ה על מצות תפילין וכשהוא מניחן אומר אקב"ו על הנחת תפילין בכל מקום מזכירין כן אע"פ שלא שח ולא כדברי רבי' הגדול ז"ל שפירש לא שח מברך א' בלבד על שתיהן.\par  אלא שיש לנו פתחון פה לומר דגמרא ירושלמי ס"ל כדקס"ד מעיקרא בגמרא דילן אבל במסקנא אסיקו אביי ורבא לא שח מברך א' ואנן כמסקנא דגמ' דילן עבדינן או שענין הירושלמי במניח א' מהן ולא במניח שתיהן. }
\newsection{דף \hebrewnumeral{52}}
\twocol{אע"ג דקיימא לן כרבנן \textbf{עד שיהו שתי שערות במקום אחד.}  מיהו שתים על גבי קשרי אצבעותי' של יד ושתים ע"ג קשרי אצבעותיה של רגל גדולה היא דלא פליגי רבנן עליה דר"ש בהאי.\par ותמהני על הרב רמב"ם פאסי ז"ל שכתב ב' שערות אלו צריכים שיהיו במקום הערוה ובשמעתין משמע אפילו על יד ורגל או בגבה. וי"מ גבה וכריסה במקום ערוה וכריסה למעלה עד מקום ערוה וכן שמעתי בשם ר"ת מיהו ביד ורגל סגי. 
\par  הא דאמרינן \textbf{מ"ד כל כה"ג מביאה קרבן ונאכל קמ"ל.}  נראה לי שאין לפרש "קמ"ל" דאינו נאכל אבל מביאה קרבן כמו שכתבו רבים. דהא כתמים דרבנן הם ואפילו בידוע שמגופ' חזאי דבר תורה טהורה ואינן מביאין לא לידי זיבה ולא לידי נדה כדאית' לקמן בפרק הרואה. אלא ע"כ נפרש קמשמע לן דאינה מביאה קרבן. ולא נאכל שני ימים וחלוק דומיא דג' חלוקין.\par  ויש לדחוק שכיון שראתה שנים וצריכה שימור בשלישיאף על פי שאין דין הכתם לטמא בתחלה כיון שדבר ברור הוא דמגופה חזאי בדין הוא שתעשה זבה גמור' שהרי לא עלתה לה שימור. ואין זה נכון. }
\twocol{\textbf{שהוא מתקן הלכותיה לידי זיבה.}  פ' רש"י ז"ל לענין זיבה הוא מיקל לדידיה היכא דלא חזאי ביום לא תלינן כתמה בראיתה ומונה ימי נדה מיום ראיתה ואין ימי זיבה מתחילין עד יום ח' לראיתה ולרבי מונה מיום מציאת כתמה ואף להקל ולטבול לילי ז' לכתמה אם פסקה ומיום ח' לכתמה אמרינן יום זוב הוא ונמצא רבי מחמיר לענין זיבה דכי חזיא בח' לכתמה אמרינן יום זיבה הוא וצריכה לשמור יום כנגד יום ולרשב"א סוף נדה הוא ואין צריך שימור ולא נראה דהא רשב"א כיון דלא תלינן כתמה בראיתה מקולקל' היא לכתמה אמרינן.\par  ול"א פי' בה שהוא מתקן הלכותיה לידי זיבה שהוא מחמיר וחושש לכתם משום זוב בג' גריסין ועוד אי נמי שאם ראתה שנים והוא צריכה שימור. וכן בכל שלשה רצופים שתראה חוששת לזיבה וצריכה נקיים נמצא שהוא מתקנה ומוציאה מכל ספק זיבה ואני מעותה שאיני מוציאה מידי ספק כלומר נראין ומטין כדברי המחמיר.\par ואף לשון זה אינו עולה דלמה לידי זיבה לכל דבר הוא מחמי' שהרי רבי מטהר' ליום ששי לראי' ולרשב"א ליום שביעי. ועוד ק"ל כיון דקי"ל (נט, א) כתמים דרבנן ובראית כתמה אינה מטמאה היאך רבי מונה לה משעת כתמה והלא ביום ראיתה היא תחלת נדה וממנו ראוי למנו' דבר תורה. וכדאמרינן בפרק קמא (ו, א) ברואה כתם ומקולקל' למנינה ואינה מונה אלא משעת שראתה.\par  לפיכך נ"ל שלא תלה רבי אלא כתמה בראיתה אבל ראיתה בכתמה לא, כתמה בראיתה לומר שאינה מטמאה עצמה וקדשים למפרע ואינה מקלקלת למנינה מיום לבישת החלוק אבל מכל מקום עיקר מנין נדה וזיבה מיום ראיה בדין תורה וחוששת נמי ליום. מציאת כתמה כדין דבריהם.\par  לפיכך אמרו שהוא מתקן הלכותיה לידי זיבה כלומר שאינה תולה כתם בראיה אלא במקום שאין חילוק ספיר' זיבה ביניהם כך דהיינו אותו יום שמנין ימי נדה וזבה אחד הוא בין לכתם בין לראיה ונמצאו כל הספירו' ראויו' כדין תורה משעת ראיה וכשהוא מעת לעת הוא רואה אינו תולה ונמצא' מקולקל' לכתם ומונה משעת ראיה נמצא כשהוא אומר תולה מתוקנת לגמרי. וכשהוא אומר אינו תולה היא מקולקל' לגמרי.\par  אבל רבי אפי' בשעה שהוא תולה כתמה בראיתה הוא מעותה לידי זיבה שהרי אסורה לשמש עד יום ז' לראיה שהוא ח' לכתמה. ואם ראתה בו ביום חוששת לזיבה בודאי נמצא לרבי שאפילו בשעת תקונה כלומר שהוא תולה הוא מעותה שתולה כתמה בראיתה ואינו תולה ראיתה בכתמה ולא השוה מדותיו כנ"ל וסליק שפיר. }
\newsection{דף \hebrewnumeral{54}}
\twocol{והא דאקשינן \textbf{הני ארביס' הוו.}  מפורש בדברי הר"ר אב"ד ז"ל דהכי מקשה בשמנה ימים הרביעיי' למה תשמש שבעה והלא צריכה היא לשמור יום א, לספיר' עשירי ואחד עשר של ימי זיבה שראתה בהן בשמונה השלישיים וכדאמרן ברישא דהיינו שימור בעו ופריק רב אדא זאת אומרת ימי נדה שאינה רואה בהן עולה לה לימי זיבתה כלומר של זוב קטן. ולפיכך יום א' של ח' רביעיים שהשלימה בהן ימי נדתה עולה לה לספיר' שמיר' של יום עשירי שאמרנו. ואין דברי רש"י ז"ל נוחין בזה.\par  אבל דבר שהכל מודים בו שאין ימי נדה מתחילין עד שתספור נקיים.\par  ובואו ונצווח על הרמב"ם פאסי ז"ל שכתב בחבורו שהאשה שראתה תחלה מונה שבעה לנידתה וסמוך להן אחד עשר ואח"כ מונה ז' לנדות אעפ"י שאינה רואה בהן ואחריהן אחד עשר ואם ראתה בהן הרי היא זבה וכן כל ימיה ואם קבעה לה וסת תחל' הוס' הוא יום נדו' וממנו מונה שמונה עשר ומונה שבעה לנדותה אף על פי שלא ראתה ואם ראתה אחריהן באחד עשר זבה היא.\par עוד שבש וכתב שאפילו ראתה ט' וי' ואחד עשר ושנים עשר הרי זו זבה ותחלת נדה וכל אלו דברי הבאי שלדבריו לא תמצא לרואה שבעה טמאים ושבעה טהורים שתשמש אלא שבוע שני ולסוף תשעה שבועות משמשת ששה ימים בשבוע העשירי וחמשה ימים בשבוע שנים עשר ופתחה של זו לסוף אחד עשר שבועות ובגמרא אמרו רביע ימיה ולא קיים אלא בתוך כ"ח הא'.\par  וכן לדבריו בשמונה ימים טמאים ושמונה טהורים אינה משמשת תמשה עשר יום אלא מתוך שמונה וארבעים ראשונים אבל בשמונה וארבעים שניים אינה משמשת אלא שלושה ימים וכיון שלא אמרו משמשת ארבע עשר יום מתוך שנים ושלשים או משמשת שמונה עשר מתוך ל"ו וכן כיוצא במנינן הללו ש"מ שפתחה של זו מ"ח ומכאן ואילך חוזרת חלילה.\par וכן האשה שראתה עשרה ימים טמאים ועשרה ימים טהורים אין זיבתה ושימושה שוים אלא פעם אחת בלבד לפי דברי הרב ז"ל שהרי כשהיא חוזרת ורואה כן בשניה בשמונה ימים טהורים נשלמו ימי זיבה ראשונה והתחילו ימי נדה נמצא שבעשרה ימים טמאים השניים חמשה ימים מימי זיבה ואין ימי חמישה בטהורים אלא שלשה וכן למאה וכן לאלף למה מנה חכמים שבעה לנדה והשאר לזיבות והלא נעשה , היא נדה אף על פי שלא ספרה לזיבה. ועוד לדבריו מצינו אשה רואה יום אחד מסוף ימי נדה יושבת עליו ששה ימים מימי הזיבה ואין לנדה ספירה אלא בימיה.\par וכן שנויה בכמה מקומות במסכתא זו שהרואה יום מ"א לזכר ופ"א לנקבה הרי היא תחלת נדה ואין מונין לימים שמקודם לכן והטעם לפי שכבר נשלם המניין.\par והרב ז"ל הורה ביולדת שמפסק' ומתחלת למנות מתחל' ראיה שלאחר מלאת ולדבריו צריך הוא להביא ראיה מן התורה לשנוי זה שהוא משנה היולדת משאר נשים שאפילו כשאינן רואות הן מונות ימי נדה וזיבה כאלו הן רואות.\par ועוד דהא בפ' בנות כותיים אמרי' דלכולי עלמא נדה ופתחה מכ"ז מנינן ואם היינו מונין משעת ראיה ראשונה כ"ז בימי זיבה קאי לה.\par  ועוד מהא דתנן היתה למודה לראות יום ט"ו ואוקמה שמואל ט"ו לטבילתה שהן כ"ב לראייתה וכו' ואם אתה מונה כל ימי נדת זובם לתחלת ראיה ראשונה שראתה זו כי הדרי אותו כ"ב תליתאי בימי זיבה קיימי והיאך קבעה וסת בכך שאין האשה קובעת וסת בי"א כדאיתה התם בשלהי בנות כותיים ואין הוסת נקבע אלא בשלשה הפלגו' כדבעינן לפרושי קמן וכל שכן לרב הונא בריה דר' יהושע דקשיא דאמר אינה חוששת בתוך אחד עשר וכל זה במס' זו.\par ותמהיני עליו אם העביר עיניו בפתחי נדה במס' ערכין דתנא רבנן טועה שאמרה יום אחד טמא ראיתי פתחה שבעה עשר פירש שאפילו היו תחלת ימי נדות הרי השלימה עליו ששה ועוד י"א אחריהן נמצאת חוזרת לתחלת נדה וכל שכן אם היה בימי זיבה שכבר עברו ימי זיבתה וימים שהיתה ראוייה להיות נדה ואלו לדברי הרב ז"ל א"א דהא איכא למימר שאותו יום בתוך אחד עשר היום וכשעמדה אחריו שבעה עשר נמצא עומדת בימי הזיבה למנין הראוי וכן כל השמועה ומדאמרינן התם נמי חמשה וארבעים ימים טמאים ראיתי וכן כולם אשתמע בהדיא דמשעה שנעשית זבה גדולה אינה נעשית נדה לעולם עד שתספור שבעה נקיים שלה. ואין לי להאריך.\par  וכן יש שבושין בחבורי הראשונים בקצתם כגון רב סעדיה שכתב שכל אחד עשר יום שבין נדה לנדה בשלשה ראיית בשלשה ימים נעשית זבה גדולה בין ברצופין בין במפוזרין. וזה טעות מתפרש כאן ובכמה מקומות דרצופין בעינן ולא מפוזרין ועל כיוצא בדברים הללו ידוו כל הימים שהתורה משתכחת מלומדיה ואין אדם מוציא הלכה ברורה במקום אחד. }
\newsection{דף \hebrewnumeral{55}}
\twocol{\textbf{אמר ר' יהודה מדסקרתא סלקא דעתך אמינא שעיר המשתלח יוכיח וכו'.}  תימא הוא למה חזר והזכיר הטעם שדוחה סברייתא קל וחומר שלו ולמה הוצרך לומר כן לפי שאלתנו זובו טמא למה לי.\par  ויש לומר שזו הברייתא השגויה למעלה שעיר המשתלח יוכיח לא היתה שנויה בבה"מ ורבא לא היה יודע אותה כמ"ש בפרק בנות כותיים וכן ר' יהודה מדסקרתא לא שמע אותה והשיב לתרץ דאיצטרך זובו טמא והיה קשה עליו בק"ו והוצרך לומר שמדין ק"ו נמי לא אתי בך מפרש בתוספת. }
\newsection{דף \hebrewnumeral{57}}
\twocol{\textbf{אמר ר' יוחנן במהלך ובא על פני כולה.}  ה"נ איכא למיחש דלמא טמא הוא אלא באוכל (טומאה) [תרומה] הוא דמתוקמא דומיא דרישא דמתניתין ומשום דרישא מקצר ועולה.\par  ותמה הרב רבי שמואל ז"ל אלא מתניתין דקתני נאמנין לומר קברנו שם את הנפלים ואינן נאמנין על הסככות הא ודאי כשם שנאמנין על הנפלים בכהן שלהם אוכל תרומה שם כך נמי נאמנים על הסככות במהלך שם ואוכל.\par  וזו אינה קושיא שהמהלך ובא על פני כולה היכי שעובר בכל השדה שאפילו לא היה הכותי חושש לאהל הסככות נטמא בקבר עצמו אם היה שם ונאמן על גופו של קבר הא על טומאות הסככות כגון שמיסך תחת אחד מן האילנות אינו נאמן עליו דלית להו דין אהל בסככות ופרעות אבל בנפלים נאמנין הן ואף על גב דאיכא למיחש לבקיאות דיצירה כן נראה לי.\par  וליכא לפרושי מהלך ובא על פני כולה שהולך תחת הסככות אורך ורוחב שהרי פירשנו שאהל הסככות עצמו מדבריהם והם אינן גוזרין כן והלכך ודאי אינן נאמנים עליהם אפילו עושין בהם מעשה. }
\newchap{פרק \hebrewnumeral{9}\quad האשה שהיא עושה}
\twocol{ הא דאיבעי לן \textbf{יושבת מה לי א"ר שמעון.}  קשיא ותיפשוט ליה ממתני' כדאמרי' בסמוך כיון דאמר ר"ש חזקת דמים מן האשה ל"ש עומדת ול"ש יושבת. ואיכא למימר מעיקרא קס"ד שאין חזקת דמים שוים מי רגלים מן האשה אלא בעומדת. והשתא דאשמועינן ברייתא דר"ש אפילו ביושבת אפשר דאתי דם ממקור א"כ הלך אחר חזקתך שחזקת דמים מן האשה ולא מן האיש דל"ש עומדת ול"ש יושבת. א"נ איכא למימר דפשטה דברייתא משמע ליה טפי ועדיף מדיוקא דמתניתין. }
\newsection{דף \hebrewnumeral{60}}
\twocol{ הא דאמרינן \textbf{רב אשי אמר הא והא רשב"ג. ול"ק כאן למפרע כאן להבא.}  כך פירש שאם לבשו הן שתיהן החלוק הזה ואחר שפשטו אותו מצאה אחת מהן כתם א' בחלוק שלה אין תולין כתם בכתם אבל היתה אחת מהן כבר בעל' כתם ולבשו חלוק זה ונמצא בו כתם תולין בבעלת הכתם שהיתה כבר וזה הפי' נכון ולשון הגמרא מוכיח אבל הפי' שפירש ר"ש אינו נכון כלל. 
\par  כיון ד\textbf{דרש רב חייא בר רב מתנה משמיה דרב}  כר' נחמיה ותנא ר' יעקב מטמא ור' נחמיה מטהר והורו חכמים כר' נחמיה שמע מינה דהלכתא כותיה. ועוד דהא רב הונא ורב חנינא ואביי דהוא בתרא מתרצי אליביה דאמרינן התם מדקמתרץ ר' יוחנן אליבא דר' יהודה ש"מ הילכתא כותיה. וההיא איתתא דבפרק הרואה דאשתכח לה דם במשתיתא משתיתא דבר המקבל טומאה הוא דהיינו טווי. וכן פסק הרמב"ם ז"ל כר' נחמיה בכתמים. }
\twocol{הא דתנן \textbf{שבעה סממנין מעבירין על הכתם.}  לטהרות קאמר ותני והדר מפרש הטבילו ועשה על גבי טהרו' העביר עליו שבעה סממנין ולא עבר הרי זה צבע. כלומר תולין אותו להקל ונאמר שהוא צבע שכן דרך הצבע שלא לעבור בסימנין ואף על פי שאפשר שהוא דם כיון דבלוע כ"כ שאינו יכול לצאת על ידי סמנין הללו טומאה בלועה היא ואינה מטמאה.\par ומיהו אם לא הטבילו תחלה טהרותיו תלויות שהרי יש לו לחוש לדם ואף על פי שאין סופו לצאת מכל מקום הבגד טמא שכבר נטמא בשעת נפילה ומטמא אותן והיינו דקתני הטבילו ומיהו אם דם נדה ודאי הוא אף על פי שלא עבר טמא לפי שדרך בני אדם להקפיד בו ולהעביר עליו סימנין אלו הילכך לא עלתה לו טבילה ראשונה עד שיעבירם ויבטלנו. }
\newsection{דף \hebrewnumeral{62}}
\twocol{הא דאמרינן \textbf{ל"ש אלא טהרות שנעשו בין תכבוסת ראשונה וכו'.}  פי' רש"י ז"ל תכבוסת העברת סממנין שהרי הקפיד עליו כשהחזירן והעבירן עליו וגלה דעתו שמקפיד עליו בספק דם ועבר ע"י העברה זו ונעשה בו מעשה דם שכן דרך דם לעבור ע"י סמנין ואין פי' מחוור לי שאין קפידה זו דומה להא דתני ר' חייא.\par  אלא כך נראה פי' דכי מטהרינן כתם בטבילה ראשונה כשלא עבר בסמנין מפני שאין סופו לצאת בדרך כבוסו וכדפרישית וזה כיון שגלה דעתו שהוא רוצה להוציאו מ"מ אין זו טומאה בלועה אלא סופו לצאת היא וצריך טבילה לאחר שתצא לגמרי. 
\par הא ד\textbf{אמר שמואל הרי אמרו לימים שנים.}  "אמרו" קאמר וליה לא סבירא דהא לקמן (סד, ב) בוסת דילוג אמרי' דשמואל כרשב"ג בוסתו' דיומי ס"ל והכי קי"ל.\par  ומיהו בוסתות דגופה ק"ל היכי אשכח בהו פלוגתא דרשב"ג ורבנן הא לא אשכחן חזקה אלא לר' בתרי זימני ולרשב"ג בתלתא אלמא היכי דבעי חזקה לרבי נמי תרי בעינן ואיכא למימר שמואל גמרא גמיר דלרבנן בחד ואשכח מתני' דקתני בוסתו' דגופה וכל שתקבע לה וסת ג' פעמים הרי זה וסת. ואמר אמאן תרמיה ודאי לרשב"ג דאשכחן דמיקל (בחזקת חששו) [בוסתות] וכיון דסיפא ודאי ביומי רשב"ג היא רישא נמי לדידיה מוקמינן ופלוגתא אחריתי היה מ"ס בעי חזקה דהיינו בתלתא זימני ואפילו בדגופה. ומ"ס אפילו תרי לא בעיא דהיינו אורחאי. ואע"ג דלא אשכחן פסקא בוסתות דגופה כרשב"ג כיון דסתם מתני' הוא ומחלוקת בדשמואל לא עדיף ממחלוקת דבריית' והלכה כסתם מתני' ודאמר ר' יהודה אמר שמואל זו דברי ר"ג הוא וס"ל היא דהא ודאי ס"ל כוותיה ביומי כדפרישית. }
\newsection{דף \hebrewnumeral{64}}
\twocol{\textbf{היתה למודת להיות רואה יום כ' ושנתה ליום ל'.}  מדקתני האי לישנא ש"מ דוסת הפלגה הוא דקבעה מכ' לכ'. והשתא ק"ל כיון דקי"ל מראיה לראיה מנינן ולא לפי מנין הראוי כדאיתא בשלהי בנות כותיים כשהגיע יום כ' ולא ראתה ומנו עשרה לתשלום ולא ראתה הגיע יום כ' וראתה דקתני כי אורח בזמנו בא מאי נינהו הא ליכא הפלגה דעשרים השתא.\par ואיכא למימר הכא מנינן למנין הראוי ויום מ' לראיה אחרונה זו היא יום כ' דקתני שאם ראתה בעונות הראשונו' ביום זה תראה ואפילו הרחיקה יותר מונין לראיה אחרונה שפסקה בו עכשיו ולא שאלו ראתה מאותה ראיה ואילך בעונות של כ' יארע לה ראיתה ביום עשרי' אורח בזמנו בא דהכא רגלים לדבר שלמנין הראוי חוזרת אלא ש"ל זמן שרואה בוסת השינוי מונין להן מאות' ראיה אבל מכיון שהפסיקתו וחזרה לראות ביום (א') [אחר] אם למנין הראוי חזרה מונין לוסת הראשון לפי אותו מנין ואין אומרין הפלגה של מ' היא זו שרגלים לדבר.\par  אבל הרב ר' אברהם בר דוד ז"ל פי' לזו בוסת החדש לפי דעתו ולמודה ליום כ' בחדש ושנתה ליום ל' בחדש קתני ולפי פי' בוסת של הפלגה אין אומרים חזר הוסת למקומו עד שתראה עכשיו ותחזור ותראה לסוף כ' שחזר האורח בזמנו. }
\newsection{דף \hebrewnumeral{65}}
\twocol{ מנימון סקסנאה דעבר \textbf{מיעבד כרב ואפילו ראתה}  לית ליה אידך דרב דאמר בועל בעילת מצוה ופורש משום דההיא אתיא כרבותינו דחזדו ונמנו ואיהו דאמר כסתם מתני' ושמואל לא קבילי ליה דמנימון (מידי) א) מ"ה דמאן דעבד כסתם מתני וכמעש' דר' לאו בר עונשין הוא אלא משום דבעי למיעב' דלא כחד ולמיתל' ברב מ"ה איענש דלא יאונה לצדיק כל און. }
\newchap{פרק \hebrewnumeral{10}\quad תינוקת}
\twocol{\textbf{כולן צריכות לבדוק את עצמן.}  פי' רש"י ז"ל שמא נשתנו מראה דמים שלה ולא סמכינן למימר הואיל ושופעת הכל דם א' הוא וטהורות.\par ואי קשיא מ"ש לאחר ד' לילות והא בתוך ד' לילות נמי אמרינן בפ"ק נשתנו מראה דמים שלה טמאה והתם נמי אמרינן ותבדוק בעדים דילמא נשתנו מראה דמים שלה.\par  י"ל הכא כיון ששופעת אין צריך לבדוק כל זמנן אבל מתוך זמנן לאחר זמנן צריכות לבדוק. א"נ התם לטהרות הכא אפילו לבעלה צריכות לבדוק הא אם נשתנו ודאי בין לאחר זמנן בין בתוך זמנן טמאה והך רישא ד"ה היא דודאי נשתנה מראה דמים שלה טמאה. וסיפא פלוגתא דר"מ ורבנן פלוגתא אחריתי היא שהיה ר"מ אומר דם בתולים אינו אדום וזהום לפיכך בודקת בזה אם מצאתו אדום וזהום בידוע שהוא דם הנדה ואע"פ שאינו יודע תחלתו מה היה שאם מצאתו מתחלתו אדום שאעפ"כ היה טהור אע"פ שאין רגילתו בכך וחכמים אומרים כל מראה דמים א' הוא הילכך לעולם תולין בדם בתולים עד שיודע לך שנשתנה ממה שהיה בתחלה.\par ויש לפרש ברייתא כולה דר"מ היא וה"ק כולן שהיו שופעות ובאות מתוך ד' לילות ולילה א' לאחר זמנן אינן טהורות אלא בבדיקה שבכולן ר"מ מחמי' כדברי ב"ש מן הסתם ומיקל כדברי ב"ה עם הבדיק' ומה היא הבדיקה הזאת שיהא מראהו שלא כדם הנדה אינו אדום ואינו זהום הא בתוך זמנן כב"ש אינן בודקות בכך אבל אם ראתה שנשתנה מראה הדמים טמאה. }
\newsection{דף \hebrewnumeral{66}}
\twocol{ה"ג וכן בנוסחאות \textbf{יום א' תשב ו' והוא ב' תשב ו' והן.}  שהרי בני מקום זה שאין בני תורה בידוע שאין רואים דם ויש לחוש שמא יום א' דם טהור ויום ב' טמא וצריכה ו' ועוד שהרי אין נשיהן בקיאין בימי נדה וזבה שלכך תקן להם לג' ז' נקיים בכל זמן הלכך בשנים נמי יש לחוש שמא ראשון י"א לזיבה הוא ושני תחלת נדה הילכך צריכות ו' והן. ואין זה צריך לפנים אלא שבהלכות רבינו הגדול ז"ל דמחה והן נראה כטעות סופר.\par  והלשון שכתב רש"י ז"ל תשב ו' והוא כדין תורה לומר שדין תורה כך הוא למנות ו' לאחד אבל אינו כדין תורה לכך שזו יושבת ו' נקיים שאם תראה צריכה יותר הילכך צריכה הפרשה בטהרה ובדיקה והאי דקאמר ז' נקיים ולא קאמר נמי ו' נקיים לישנא בעלמא נקט דשגירי למימר ז' ימים נקיים.\par  וא"ת לדברי האומר ימי נדה שאינו רואה בהן אין עולין לה לספירת זיבתה עדיין היה לו לר' בית מיחש לומר שמא יום א' י"א הוא ובעי שימור והז' הן התחלת נדה אין עולה לו וצריך ז' והן לשני ימים. לאו מילתא שכבר פי' בפ' בנות כותיים שימי נדה ולידה שאינה רואה בהן עולין לספירת זיבה קטנה לדברי כל אדם ואין לחוש כלום. וכתב רבי' בעל הלכות ז"ל שאפילו בימי טוהר נמי אם ראתה סופרת דהאידנא יולדות בזוב הן לפי שא"א לפתיח' הקבר בלא דם ובנות ישראל סופרת ז' לכל טיפה דם. ושמענו כדבריו בזה שהגאונים החרימו בדם טוהר.\par  והדברים נראין אף לדין הגמר' שכשם שחששו לטועות בפתחיהן ולמשלימות דם טמא לדם טהור ועשו נמי הרחקה יתירה בדבר כך יש לחוש שמא יבואו לטעות באותן שיושב' עליהן לזכר ולנקבה ולנדה שמא ינהגו בהן קולא שסוברות כל שיש לו טומאת לידה יש לו טוהר שלה וכ"ש שברוב נפלים אין בני אדם בקיאין. ואין עליהם לדון בהם אלא כך תשב לזכר ולנקבה ולנדה וקרוב הדבר לטעות בו הילכך אין ימי טוהר יוצאין מכלל ר' זירא שאף בהן החמירו בנות ישראל לישב ז' נקיים. וההיא דאמרינן דרש מרימר הילכתא כותיה דרב וכו'. דינא קאמרי כדאמרי בשמעתי' אמינא לך האי איסור ואת אמר' לי חומרא היכא דאחמור אחמור היכא דלא אחמור לא אחמור. והמקומות שבועלין עכשיו על דם טוהר הם יחושו לעצמן. 
\par \textbf{חפיפה.}  פי' רש"י ז"ל חפיפת שערה ופי' לפי' חפיפת שער בכל מקום שבגוף בית השחי ובית הערוה ואצ"ל ראשה. ובודאי דלשון חפיפה לא שייך אלא בשער כדתנן נזיר חופף ומפספס אבל לא מסרק ואיתמר בעלמא הוה חייף רישיה.\par וה"נ משמע בפ' מרובה (דף סב) גבי עשר תקנות שתקן עזרא ושתהא אשה חופפת וטובלת ואקשינן דאורייתא הוא דכתיב את כל בשרו את הטפל לבשר ומאי ניהו שערו. ופריק מדאורייתא עיוני בעלמ' דילמא מיקטר א"נ מיאוס מידי משום חציצה אתא איהו ותיקן חפיפה. מדמקשינן את הטפל לבשרו ומאי ניהו שערו ש"מ שאפילו בכל הגוף צריכה לעיין דברי תורה משום דילמא מיאוס במידי משמע דלגבי הכי מקום השאר ושאר מקומות שבגוף שווין אלא דאתא עזרא וחייש דילמא אתיא למיטעי בעיוני דשער משום דשכיח ביה קטרי ואחמיר ביה חפיפה הילכך בעי עיוני בכוליה גופיה דאורייתא ובעי נמי חפיפה למקום שער מתקנתא.\par  והיינו דאמרינן בשמעתין (לקמן סז, א) נתנה תבשיל לבנה וטבלה לא עלתה לה טבילה כלומר אם לא חזרה ועיינה בנפשה בשעת טבילה ממש אבל ודאי עיינה בעצמה אע"פ שלא חזרה לחוף נראה שעלתה לה טביל' דהא אע"פ שנתנה בנתיי' מעט תבשיל לבנה סמוך לחפיפה טבילה היא ומזיא לא מיקטרי בתבשיל של בנה.\par והא דאמרינן ונמצא עליה דבר חוצץ אם סמוך לחפיפה טבלה וכו'. דמשמע עליה על גופה לאו למימרא דחפיפה בכולי גופא היא אלא משום דודאי עיינה בנפשה בשעת חפיפה ואם לא טבלה סמוך לחפיפה ולא חזרה ועיינה בשעת טבילה אע"פ שהיתה משמר' נפשה מליתן תבשיל לבנה וכיוצא בו כיון שמצא עליה דבר חוצץ חוששין דילמא נגעה ולאו אדעתה.\par  וההיא דאמרינן לקמן (סז, ב) בחופפת בע"ש וטובלת למ"ש וכולהו הרחקות דחפיפה כשחזרה ועיינה בעצמה בשעת טבילה שלא הקלו בשל תורה אלא בתקנת עזרא דהא לא אפשר מע"ש למ"ש דלא נחנה תבשיל לבנה וכיוצא בזה ואמרן דלא עלתה לה טבילה אלא התם בשלא עיינה בעצמה וכאן כשחזרה ועיינה בשעת טבילה.\par  ומיהו מנהגא דנהגן נשי למשטף כולה גופה בחמימי בשעת חפיפה משום מקומות השער שבגוף הוא דקרירי ממשרו להו. וה"נ משמע בהא דאמרינן לקמן עבדי חסרת דודי חסרת טכטקי חסרת משמע דשטיפת כל גופא עבדא מדצריכה עבדי ודודי וטכטקי וכן החמירו בנות ישראל על עצמן והמעביר מנהג זה ימתח על העמוד. }
\newsection{דף \hebrewnumeral{67}}
\twocol{ הא דאמרינן \textbf{ולית הלכתא ככל הני שמעתא. אלא כי הא דאמר ר"ל אשה לא תטבול אלא דרך גדילתה וראוי.}  לאו למימרא דהנך פלגינן אדר"ל דהא אפשר דתרווייהו איתנהו אלא גמ' קאמר דלית הלכתא בכל הני שמעתת' דלא מחמרינן כולי האי בביאת מים בדברים שדרכן להיות באותו מקום אלא כך מחמרי' בביאת מים בדר"ל מחמירים ובעי טבילה דרך גדילתה כדי שיבואו מים קצת בבית השחי ובבית הסתרים.\par וכה"ג איכא טובא בתלמודא דקאמר לית הלכתא בכל הני שמעתתא אלא כי הא ולא פליגן אהדדי. כי ההיא דאמרינן ככתובות דלית הלכתא ככל הני שמעתא דזינתה וכחלה ופירכסה ותבעוה לינשא ונתפייסה אבדה מזונותיה אלא כי הא דא"ר יהודה תובעת כתובתה בב"ד אין לה מזונות והא (דא"א) [נמי אף דאפשר] דאיתנהו לכולהו. וכן במסכ' ברכות (דף מב) סלק אסור לאכול ומר אמר גמר ומר אמר משחא מעכב לן ואמר ולית הלכתא ככל הני שמעתא אלא כי הא דא"ר יהודה תכף לנטילת ידים ברכה. וכן בפרק שלשה שאכלו (דף מז ע"ב) ולית הלכתא ככל הני שמעתתא אלא כי הא דא"ר נחמן קטן היודע למי מברכין מזמנין עליו וההיא ודאי לא פליגא אשמעתתא דלעיל דט' ונראין כעשרה וט' וארון ושנים ושבת ושני ת"ח המחדדין זה את זה כ"ש קטן ועבד נעשה אותן סניפין לעשרה. והרבה מהן כך.\par  אבל כי איתמר בתלמודו לית הלכתא ככל הני שמעתתא מהא דאמרינן ההיא ודאי פליגן ומחדא מידחיא חברתה ממש.\par  ורבינו הגדול ז"ל גורס ולית הלכתא ככל הני שמעתתא אלא כי איתמר הני לענין טהרות איתמר אבל לבעלה שפיר דמי כי הא דאמר ריש לקיש וכו'. וק"ל מנלן מדריש לקיש דלבעלה מותרת דהא אפשר דכולהו איתנהו. ואיכא למימר כי מייתי דר"ל משום פתחה עיניה ביותר ועצמה עיניה ביותר דלא מעכב בטבילה דרך גדילתה נמי הוא שאין אדם נמנע מלפתוח ולעצום עיניו פעמים הרבה כדרכו ואין המים מתעכבים מליכנס שם כדרך שנכנסין בבית השחי ובבית הסתרים בטובלת דרך גדלתה ולא מחוור.\par  תו קשה לי ההיא דגרסינן ביבמות (מז, ב) וכל דבר שחוצץ בטבילה של טהרות חוצץ בגר ובעבד משוחרר ובנדה לבעלה. זו היא גרסתו של רבינו עצמו ז"ל ומשמע דכל דבר שחוצץ בטבילה של טהרות חוצץ בגר ובעבד משוחרר ובנדה לבעלה. והא איכא הני דחצצי לטהרות ולא חצצי לנדה.\par ואיכא למימר דה"ק: כל דבר שחוצץ בטבילה אחרת חוצץ בגר ובעבד ובנדה ואף על פי שאין טבילתן מפורשת בתורה שהרי טבילת נדה מן הכתוב מפורש לא למדנו אלא מבנין אב אתיא דכתיב ורחצו במים בנין אב לכל הטמאין שיהיו בטומאתן עד שיבואו במים. ומה שאמרו שם במקום שנדה טובלת שם גר ועבד משוחרר טובל לא מפני שטבילת נדה מפורש' יותר אלא לומר דלא בעינן מעין כזב וא"נ דבעינן טבילה בבת אחת משום דסמכי לה אבמי נדה יתחטא מים שהנדה טובלת.\par  והרב ר' אברהם בר דוד ז"ל פירש ולית הלכתא ככל הני שמעתתא לפלוף וכוחלי אלא כי הא דאמרן כדר' יוחנן פתחה עיניה ביותר דאמר ר"ל אשה לא תטבול אלא דרך גדלתה ומי שפתחה עיניה ביותר אינו דרך גדלתה. וזה הפירש מוקצה מן הדעת מפני שהוא מקלקל עלינו שיטת התלמוד שאמרו בכמה מקומות ולית הלכתא ככל הני שמעתתא אלא כי הא ובכולן פירושן ידוע שאין הלכה בכל הנזכרות אלא כי הא דבעינן למימר קמן.\par וקשה לן מרייהו דהני שמעתח היכי אמרינהו והאנן תנן (מקוואות ט, ב)אלו הן שחוצצין לפלוף שחוץ לעין וגלד שהוא חוץ למכה ואלו שאין חוצצין לפלוף שבעין וגלד שעל המכה.\par  ואיכא למימר לפלוף שבעין מוקים לה רב עוקבא בלח. והיינו טעמא דלא חייץ משום דלא קפיד עליה וה"ל מיעוטו שאינו מקפיד אבל יבש ודאי מקפיד. ושחוץ לעין דקפיד עליה אפילו לח חוצץ וגלד שעל המכה נמי משום האי טעמא הוא דלא עביד אינש לקלף גלד מכתו משום דקשה למכה עד דיביש ומקליף מנפשיה וקסבר דהוא דוקא של מכה אבל ריבדא דכוסילתא עד תלתא יומין דלא קפיד איניש עלה לא חייצה מכאן ואילך חייצה. אי נמי עד תלתא יומין לחה ולא מעכבה מיהו מכאן ואילך חייצה דיבישה וקפיד עלה.\par  ואי קשיא לך מאי שנא פתחה עיניה ביותר או שעצמה דלא מעכבי בטבילה ומ"ט קרצה שפתותיה כדתנן כאלו לא טבלה. לא תיקשי דודאי קרצה שפתותיה מעכבת ביאת מים במקום הגלויי אבל פתחה עיניה אינה מעכבת כלום אלא קמטין בעלמא הוא דעבדה במקום שדרכן בכך ואפשר נמי שאינן מעכבין כלל מלבא בהם מים ממש. 
\par הא דאמר ליה רב פפי לרבא \textbf{מכדי האידנא כולהו ספק זבות שותינהו ליטבלן ביממא דשבעה.}  רש"י ז"ל מפרש כמשמעה לומר שאין לחוש להטבל בלילה אלא יכולין לטבול אותן ביום כדין הזבה דאי לנדה יותר משמיני הוי ואי לזבה טבילתה ביום הז' הוא.\par  ומתרצים משום דר' שמעון דאמר אחר מעשה של ספירה תטהר מיד ומקצת היום ככולו אבל אמרו חכמים אסור לה לטבול שמא תבא לידי הספק שמא תבא לשמש כיון שטבלה ותראה.\par  והגאונים כך סוברים שהאשה בזמן הזה אינה טובלת אלא בלילה משום סרך בתה שלא תטבול בז' ותראה ותסתור.\par ואחרים פרשו דרב פפי לאו אטבילה בלחוד פריך אלא ליטבלו ביממא דז' ויהיו מותרת לבעליהן קאמר שהרי מקצת היום בספירת הזבה כולו הוא. ומתרץ לה משום דר"ש דאמר אסור לעשותה כן להחזיק עצמה בטהורה לאחר ספירה מיד כלומר אסור שתשמש ותעסוק בטהרות שמא תראה ותסתור. וראיה לפירש זה מה שאמרו בסוף המפלת (דף כט ע"ב) בכ"א תשמש ר' שמעון היא דאמר אבל אמרו חכמים אסור לעשות כן שמא תבא לידי הספק הא טובלת אפילו לר' שמעון ואסורה לשמש הא לרבנן מותרת אפילו לשמש אלמא איסורא דר"ש בתשמיש בלחוד היא.\par ור"ש עצמו ז"ל כך כתב שם ר"ש היא דאמר בת"כ אסור לעשות כן לשמש זבה ביום טבילתה. ובודאי דהתם בת"כ מוכח כן דקתני כיון שטבלה טהורה להתעסק בטהרות אבל אמרו חכמים לא תעשה כן שלא תבא לידי הספק. וש"מ דאסור לעשות כן אעסק טהרות קאי וה"ה לתשמיש ולא אטבילה קאי ולפי הפירש הזה מקילין ואומרים דהאידנא טובלות הן ביום וליכא סרך בתה כלל.\par  ובודאי שזה הפירש הוא הנכון דר"ש לא אסר אלא להחזיק עצמה בטהורה לטהרות אי נמי לתשמיש אבל שנקל לומר שיהו טובלו' ביום אינ' נראה דהא רבא דשרא במחוזא משום אבולאי הא לאו הכי אסור בתר חומרא דר' זירא הוה דקא"ל ר' פפא מכדי האידנא כולהו ספק זיבות שויתינהו ומשמע נמי דאהדא דתקון תקנתא ושרא משום אבולאי פריך ליה למה ליה אבולאי כולהו נמי ליטבלן ביומא דז'. אלמא לרבא אית ליה משום סרך ואפילו לבתר חומרא והכי פירכיה אפילו לטבול ולהתירן לבעלן יהיו מותרת, ומתרצין להתירן א"א משו' דר"ש וכיון דאסורות לטהרות ולבעלה אין טובלות אלא בלילה ואפילו בח' משום סרך בתה שמא תטבול ותטהר כאמה דודאי כשם שהיא נסרכ' אחר אמה בטבילה דנדה נסרכת אחריה בתשמיש גופה, ואם תאמר תשמיש גופיה גזירה דרבנן ואנן ניקום ונגזור גזירה לגזירה, כיון דבא לידי איסור דכרת גזרינן.\par  ואי קשיא דרבנן היכי שרו לבעלה בז' והתנן טבלה ביום שלאחריו ושמשה הרי זה תרבות רעה התם בשומרת יום רגילה היא לבא לידי זיבה גדולה הכא כיון שספרה שבעה הוחזקה במעין סתום ואין חוששין לה אלא מדבריהם לר' שמעון, ולפי דעתי שלא נחלקו חביריו עליו כלל מדלא פרכינן תינח לר"ש לרבנן ליטבלן, והא דאמרינן בהמפלת, הא מני ר"ש היא משום דלדידיה שמעינן לה וכיוצא בה בתלמוד הרבה.\par ואחרים השיבו אי ר"ש לאו אטבילה קאי היכי קאמר שמא תבא לידי ספק ודאי לידי ספק באה ואין אנו גורסים אלא שלא תבוא לידי ספק, וכן בהלכות גדולות וכן בת"כ. ויש נוסחא שכתוב בה יבואו לידי ספק. }
\newsection{דף \hebrewnumeral{68}}
\twocol{הא דתנן \textbf{וחכמים אומרים אפילו בשנים לנדתה בדקה וכו'.}  דוקא בשנים אבל בראשון הואיל והוחזק מעין פתוח לא דתחלת נדה אין דרכה לפסוק ביומה ואפילו בדקה ומצאה טהור חוששין לה שמא חזרה וראתה ואין צריך לומר כשלא בדקה כלל אלא שראתה תחלה דלעולם היא בחזקת טומאה עד שתפרש בטהרה כיון שראתה נדה וליכא מאן דפליגי, ומיהו בשני אפילו ראתה בשחרי' ופסקה טהרה באמצע יום ובין השמשו' לא הפרישה ואחר הימים מצאה טמא הרי זו בחזקת טהרה, ובברייתא תניא דרבי מטהר אפילו במצאה טהור בראשון.\par  ואין לפרש דלא פליגי אלא שני וה"ה לראשון דמכדי רבנן בתראי לטפויי מילתא אתו דתנא קמא רישא שביעי קאמר ואוסיפו אינהו אפילו בשנים אם בן לימרו ראשון וכן פי' רש"י ז"ל ומסתברא אפילו מצאה טהור כשבדקה אח"כ הרי זו חוששת דכיון שאין הפרישה של ראשון לנדה הפרשה גמורה אין בדיקה של עכשיו מחזיקה בטהרה למפרע ויש לדון בדבר אלא שהוא חומרא. }
\newsection{דף \hebrewnumeral{69}}
\twocol{והא דקתני ברייתא \textbf{בין השמשות טמא, ראיתי ואמר רב ירמיה מדיפתי שבאת לפני' בין השמשות.}  נראה לי דהכי אמר ברייתא דקתני בין השמשות לאו לראיה אלא לביאה והכי קתני באה בין השמשות ואמרה טמא ראיתי, ואמר רב ירמיה מטבילין אותה אחד עשר טבילו' שהרי כל שבאה בין השמשות אפילו אמרה סתם יום אחד טמא ראיתי אחד עשר טבילות הן, אי נמי בין השמשו' טמא ראיתי דקתני שאם אמרה מבעוד יום ראיתי אין כאן י"א ולאו מילתא היא. 
\par  הא דתנן \textbf{בית שמאי אומרים כל הנשים מתות נדות.}  אוקימנא בגמרא טעמייהו דבי' שמאי כדתני' בראשונ' היו מטבילין על גבי נדות מתות והיו נדות חיות מתביישות התקינו שיהו מטבילין על הכל, ולא למימרא דבית הלל פליגו אהך תקנתא אלא בית שמאי סברי גזרינן בכולהו לעשותן כנדות בין בחיים בין במיתה לטמא באבן מסמא ובית הלל סברי לענין הטבלת כלים עשאום כנדות, אבל לא לשאר טומאת דחיים ולא לאבן מסמא במיתה.\par  ויש מפרשים שחזרו בית הלל ותקנו כב"ש ואינו נכון כלל ואחרים העמידו ברייתא זו כב"ש וגם זה שבוש. }
\twocol{והא דקתני \textbf{בית שמאי אומרים צריכה טבילה.}  לתרומה מפני שטבולת יום ארוך הוא והסיחה דעתה מן התרומה ואם ישראלית היא טובלת לביאת מקדש. פירש לפיכך טובלת כדי שתכנס לאתר כפרתה למקדש שהרי מחוסר כפורים שנכנס למקדש ענוש כרת כדאמרונן במס' מכות (דף ח ע"ב) טמא יהיה לרבות טבול יום עוד טמאתו בו לרבות מחוסר כפורים, וכן היא צריכה לטבול לנגיע' דתרומה, וב"ה אומרי' אינה צריכה אבל לקדשים מודי ב"ה דקיי"ל (חגיגה כא, א) האונן והמחוסר כפורים צריכין טבילה לקדש דמעלות דרבנן נינהו למדנו לדברי רש"י ז"ל שהחמירו באכילות קדשים יותר מביאת המקדש שיבנה במהרה בימינו אמן וכן יהי רצון. }

\end{document}
