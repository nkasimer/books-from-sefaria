\documentclass[12pt, openany]{book}
\usepackage[
paperheight=9in,
paperwidth=6in,
top=0.5in,
bottom=0.5in,
inner=0.7in,
outer=0.5in,
marginparsep=0.1in,
headsep=16pt
]{geometry}

\newcommand{\texttitle}{מסכת נידה}\usepackage{titlesec}
\newcommand{\partname}[1]{}
\usepackage{resources/unnumberedtotoc}

\usepackage{fancyhdr}
\pagestyle{fancy}
\fancyhf{}
\fancyhead[LO,RE]{\thepage}
\fancyhead[CO]{}
\fancyhead[CE]{\partname\chapname \space\textendash\space \sectname}

\usepackage{paracol}
\usepackage{anyfontsize}
\usepackage{ragged2e}
\usepackage{polyglossia}
\usepackage{multicol}
\usepackage{hyperref}
\usepackage[marginal]{footmisc}
\usepackage[titles]{tocloft}
\usepackage{xifthen}
\usepackage{graphicx}

\setdefaultlanguage{hebrew}
\setotherlanguage{english}
\usepackage{fontspec}
\setmainfont{Times New Roman}
\newfontfamily\englishfont{Times New Roman}

\newcommand{\sethebfont}{
\fontsize{10.5pt}{13.1pt} \selectfont
}

\newcommand{\hebeng}[2]{
	{\sethebfont #1\\}
	
	\begin{english}
		#2
	\end{english}
	\clearpage
}

\newcommand{\twocol}[1]{
	{\sethebfont \begin{multicols}{2}
			#1
	\end{multicols}}	
}

\newcommand{\textblock}[1]{
{\sethebfont #1\\}	
}

\setlength{\parskip}{6pt}
\setlength\parindent{0in}

\newcommand{\chapname}{}
\newcommand{\sectname}{}

\newcommand{\newchap}[1]{
	\addcontentsline{toc}{chapter}{#1}
	\renewcommand{\chapname}{#1}
		\begin{center}
			\textbf{%
\fontsize{16pt}{16pt}\selectfont
				#1}
		\end{center}
}

\let\footnoterule\relax

\setlength{\columnsep}{0.25in}

\newcommand{\newsection}[1]{
	%\addcontentsline{toc}{section}{#1}
	\renewcommand{\sectname}{#1}	
	\vspace{-\baselineskip}
	\begin{center}
		\textbf{%
\fontsize{16pt}{16pt}\selectfont
			#1}
	\end{center}
	\vspace{-\baselineskip}
	\nopagebreak
}

\newcommand{\footnotecomment}[1]{
	\renewcommand\thefootnote{}
	\footnote{#1}}

\newcommand{\parencomment}[1]{\footnotesize (#1)}

\newcommand{\blockcomment}[2]{ 
\newsection{#1}
\sethebfont	#2}

\newcommand{\commenta}[1]{\footnotecomment{#1}}

\begin{document}
\frontmatter
\pagenumbering{roman}

\newcommand{\oneline}[1]{%
	\newdimen{\namewidth}%
	\setlength{\namewidth}{\widthof{#1}}%
	\ifthenelse{\lengthtest{\namewidth < \textwidth}}%
	{#1}% do nothing if shorter than text width
	{\resizebox{\textwidth}{!}{#1}}% scale down
}

\title{\oneline{\hspace*{0.5in}\texttitle\hspace*{0.5in}}}

\author{}

\date{}

\maketitle

\begin{minipage}[b][\textheight][b]{\textwidth}\englishfont\footnotesize
	\begin{english}
		\vfill
		The following book includes:
\begin{itemize}
\item[$\bullet$] Tanach with Ta'amei Hamikra
\begin{itemize}
\item[$\bullet$] License: Public Domain
\item[$\bullet$] Source: \url{http://www.tanach.us/Tanach.xml}
\end{itemize}
\item[$\bullet$] The Metsudah Five Megillot, Lakewood, N.J., 2001
\begin{itemize}
\item[$\bullet$] License: CC-BY
\item[$\bullet$] Source: \url{http://primo.nli.org.il/primo_library/libweb/action/dlDisplay.do?vid=NLI&docId=NNL_ALEPH002162036}
\end{itemize}
\item[$\bullet$] Mishnah, ed. Romm, Vilna 1913
\begin{itemize}
\item[$\bullet$] License: Public Domain
\item[$\bullet$] Source: \url{http://primo.nli.org.il/primo_library/libweb/action/dlDisplay.do?vid=NLI&docId=NNL_ALEPH001741739}
\end{itemize}
\item[$\bullet$] On Your Way
\begin{itemize}
\item[$\bullet$] License: CC-BY
\item[$\bullet$] Source: \url{http://mobile.tora.ws/}
\end{itemize}
\item[$\bullet$] The Mishna with Obadiah Bartenura by Rabbi Shraga Silverstein
\begin{itemize}
\item[$\bullet$] License: CC-BY
\item[$\bullet$] Source: \url{http://www.sefaria.org/shraga-silverstein}
\end{itemize}
\item[$\bullet$] Senlake edition 2019 based on Ben Yehoyada, Jerusalem, 1897
\begin{itemize}
\item[$\bullet$] License: CC0
\item[$\bullet$] Source: \url{http://beta.nli.org.il/he/books/NNL_ALEPH001933802/NLIl}
\end{itemize}
\end{itemize}
		It was retrieved from Sefaria on \today\space \texthebrew{(\Hebrewtoday)}.  It was typeset and formatted by Ktavi.
		\clearpage
		
	\end{english}
\end{minipage}

\titleformat{\chapter}[hang]{\huge\bfseries}{\thechapter.}{1em}{}
\titlespacing*{\chapter}{0pt}{-3em}{1.1\parskip}
\titlelabel{\thetitle\quad}
%\addtocontents{toc}{\protect\vspace{-\baselineskip}}
\addtocontents{toc}{\protect\begin{multicols}{2}}
%\vspace*{-5\baselineskip}
{\small \tableofcontents}


\clearpage
\mainmatter
\pagenumbering{arabic}

\addpart{אסתר}\renewcommand{\partname}[1]{אסתר}
\twocol{\clearpage}

\newchap{פרק א}
\twocol{וַיְהִ֖י בִּימֵ֣י אֲחַשְׁוֵר֑וֹשׁ ה֣וּא אֲחַשְׁוֵר֗וֹשׁ הַמֹּלֵךְ֙ מֵהֹ֣דּוּ וְעַד־כּ֔וּשׁ שֶׁ֛בַע וְעֶשְׂרִ֥ים וּמֵאָ֖ה מְדִינָֽה׃
\commenta{וַיְהִי בִּימֵי אֲחַשְׁוֵרוֹשׁ. מֶלֶךְ פָּרַס הָיָה, שֶׁמָּלַךְ תַּחַת כּוֹרֶשׁ לְסוֹף שִׁבְעִים שָׁנָה שֶׁל גָּלוּת בָּבֶל: }%endcomment
בַּיָּמִ֖ים הָהֵ֑ם כְּשֶׁ֣בֶת ׀ הַמֶּ֣לֶךְ אֲחַשְׁוֵר֗וֹשׁ עַ֚ל כִּסֵּ֣א מַלְכוּת֔וֹ אֲשֶׁ֖ר בְּשׁוּשַׁ֥ן הַבִּירָֽה׃
\commenta{כְּשֶׁבֶת הַמֶּלֶךְ אֲחַשְׁוֵרוֹשׁ, וגו'. כְּשֶׁנִּתְקַיֵּם הַמַּלְכוּת בְּיָדוֹ. וְרַבּוֹתֵינוּ פֵּרְשׁוּהוּ בְּעִנְיָן אַחֵר בְּמַסֶּכֶת מְגִילָּה:}%endcomment
בִּשְׁנַ֤ת שָׁלוֹשׁ֙ לְמָלְכ֔וֹ עָשָׂ֣ה מִשְׁתֶּ֔ה לְכָל־שָׂרָ֖יו וַעֲבָדָ֑יו חֵ֣יל ׀ פָּרַ֣ס וּמָדַ֗י הַֽפַּרְתְּמִ֛ים וְשָׂרֵ֥י הַמְּדִינ֖וֹת לְפָנָֽיו׃
\commenta{הַפַּרְתְּמִים. שִׁלְטוֹנִים בִּלְשׁוֹן פָּרַס:}%endcomment
בְּהַרְאֹת֗וֹ אֶת־עֹ֙שֶׁר֙ כְּב֣וֹד מַלְכוּת֔וֹ וְאֶ֨ת־יְקָ֔ר תִּפְאֶ֖רֶת גְּדוּלָּת֑וֹ יָמִ֣ים רַבִּ֔ים שְׁמוֹנִ֥ים וּמְאַ֖ת יֽוֹם׃
\commenta{יָמִים רַבִּים. עָשָׂה לָהֶם מִשְׁתֶּה:}%endcomment
וּבִמְל֣וֹאת ׀ הַיָּמִ֣ים הָאֵ֗לֶּה עָשָׂ֣ה הַמֶּ֡לֶךְ לְכָל־הָעָ֣ם הַנִּמְצְאִים֩ בְּשׁוּשַׁ֨ן הַבִּירָ֜ה לְמִגָּ֧דוֹל וְעַד־קָטָ֛ן מִשְׁתֶּ֖ה שִׁבְעַ֣ת יָמִ֑ים בַּחֲצַ֕ר גִּנַּ֥ת בִּיתַ֖ן הַמֶּֽלֶךְ׃
\commenta{גִּנַּת. מְקוֹם זֵרְעוֹנֵי יְרָקוֹת:}%endcomment
ח֣וּר ׀ כַּרְפַּ֣ס וּתְכֵ֗לֶת אָחוּז֙ בְּחַבְלֵי־ב֣וּץ וְאַרְגָּמָ֔ן עַל־גְּלִ֥ילֵי כֶ֖סֶף וְעַמּ֣וּדֵי שֵׁ֑שׁ מִטּ֣וֹת ׀ זָהָ֣ב וָכֶ֗סֶף עַ֛ל רִֽצְפַ֥ת בַּהַט־וָשֵׁ֖שׁ וְדַ֥ר וְסֹחָֽרֶת׃
\commenta{חוּר כַּרְפַּס וּתְכֵלֶת. מִינֵי בְגָדִים צִבְעוֹנִים פֵּרַס לָהֶם לְמַצָּעוֹת:}%endcomment
וְהַשְׁקוֹת֙ בִּכְלֵ֣י זָהָ֔ב וְכֵלִ֖ים מִכֵּלִ֣ים שׁוֹנִ֑ים וְיֵ֥ין מַלְכ֛וּת רָ֖ב כְּיַ֥ד הַמֶּֽלֶךְ׃
\commenta{וְהַשְׁקוֹת בִּכְלֵי זָהָב. כְּמוֹ וּלְהַשְׁקוֹת:}%endcomment
וְהַשְּׁתִיָּ֥ה כַדָּ֖ת אֵ֣ין אֹנֵ֑ס כִּי־כֵ֣ן ׀ יִסַּ֣ד הַמֶּ֗לֶךְ עַ֚ל כָּל־רַ֣ב בֵּית֔וֹ לַעֲשׂ֖וֹת כִּרְצ֥וֹן אִישׁ־וָאִֽישׁ׃
\commenta{כַדָּת. לְפִי שֶׁיֵּשׁ סְעוּדוֹת שֶׁכּוֹפִין אֶת הַמְּסֻבִּין לִשְׁתּוֹת כְּלִי גָדוֹל, וְיֵשׁ שֶׁאֵינוֹ יָכוֹל לִשְׁתּוֹתוֹ כִּי אִם בְּקֹשִׁי, אֲבָל כַּאן: "אֵין אוֹנֵס": }%endcomment
גַּ֚ם וַשְׁתִּ֣י הַמַּלְכָּ֔ה עָשְׂתָ֖ה מִשְׁתֵּ֣ה נָשִׁ֑ים בֵּ֚ית הַמַּלְכ֔וּת אֲשֶׁ֖ר לַמֶּ֥לֶךְ אֲחַשְׁוֵרֽוֹשׁ׃ (ס)
בַּיּוֹם֙ הַשְּׁבִיעִ֔י כְּט֥וֹב לֵב־הַמֶּ֖לֶךְ בַּיָּ֑יִן אָמַ֡ר לִ֠מְהוּמָן בִּזְּתָ֨א חַרְבוֹנָ֜א בִּגְתָ֤א וַאֲבַגְתָא֙ זֵתַ֣ר וְכַרְכַּ֔ס שִׁבְעַת֙ הַסָּ֣רִיסִ֔ים הַמְשָׁ֣רְתִ֔ים אֶת־פְּנֵ֖י הַמֶּ֥לֶךְ אֲחַשְׁוֵרֽוֹשׁ׃
\commenta{בַּיּוֹם הַשְּׁבִיעִי. רַבּוֹתֵינוּ אָמְרוּ: שַׁבָּת הָיָה:}%endcomment
לְ֠הָבִיא אֶת־וַשְׁתִּ֧י הַמַּלְכָּ֛ה לִפְנֵ֥י הַמֶּ֖לֶךְ בְּכֶ֣תֶר מַלְכ֑וּת לְהַרְא֨וֹת הָֽעַמִּ֤ים וְהַשָּׂרִים֙ אֶת־יָפְיָ֔הּ כִּֽי־טוֹבַ֥ת מַרְאֶ֖ה הִֽיא׃
וַתְּמָאֵ֞ן הַמַּלְכָּ֣ה וַשְׁתִּ֗י לָבוֹא֙ בִּדְבַ֣ר הַמֶּ֔לֶךְ אֲשֶׁ֖ר בְּיַ֣ד הַסָּרִיסִ֑ים וַיִּקְצֹ֤ף הַמֶּ֙לֶךְ֙ מְאֹ֔ד וַחֲמָת֖וֹ בָּעֲרָ֥ה בֽוֹ׃
\commenta{וַתְּמָאֵן הַמַּלְכָּה וַשְׁתִּי. רַבּוֹתֵינוּ אָמְרוּ: לְפִי שֶׁפָּרְחָה בָהּ צָרַעַת כְּדֵי שֶׁתְּמָאֵן וְתֵהָרֵג. לְפִי שֶׁהָיְתָה מַפְשֶׁטֶת בְּנוֹת יִשְׂרָאֵל עֲרֻמּוֹת וְעוֹשָׂה בָהֶן מְלָאכָה בַּשַּׁבָּת, נִגְזַר עָלֶיהָ שֶׁתִּפָּשֵׁט עֲרֻמָּה בַּשַּׁבָּת: }%endcomment
וַיֹּ֣אמֶר הַמֶּ֔לֶךְ לַחֲכָמִ֖ים יֹדְעֵ֣י הָֽעִתִּ֑ים כִּי־כֵן֙ דְּבַ֣ר הַמֶּ֔לֶךְ לִפְנֵ֕י כָּל־יֹדְעֵ֖י דָּ֥ת וָדִֽין׃
\commenta{כִּי כֵן דְּבַר הַמֶּלֶךְ. כִּי כֵן מִנְהַג הַמֶּלֶךְ בְּכָל מִשְׁפָּט לָשׂוּם אֶת הַדָּבָר "לִפְנֵי כָּל יוֹדְעֵי דָּת וָדִין": }%endcomment
וְהַקָּרֹ֣ב אֵלָ֗יו כַּרְשְׁנָ֤א שֵׁתָר֙ אַדְמָ֣תָא תַרְשִׁ֔ישׁ מֶ֥רֶס מַרְסְנָ֖א מְמוּכָ֑ן שִׁבְעַ֞ת שָׂרֵ֣י ׀ פָּרַ֣ס וּמָדַ֗י רֹאֵי֙ פְּנֵ֣י הַמֶּ֔לֶךְ הַיֹּשְׁבִ֥ים רִאשֹׁנָ֖ה בַּמַּלְכֽוּת׃
\commenta{וְהַקָּרֹב אֵלָיו. לַעֲרֹךְ דְּבָרָיו לִפְנֵיהֶם. אֵלּוּ הֵם: כַּרְשְׁנָא שֵׁתָר, וגו': }%endcomment
כְּדָת֙ מַֽה־לַּעֲשׂ֔וֹת בַּמַּלְכָּ֖ה וַשְׁתִּ֑י עַ֣ל ׀ אֲשֶׁ֣ר לֹֽא־עָשְׂתָ֗ה אֶֽת־מַאֲמַר֙ הַמֶּ֣לֶךְ אֲחַשְׁוֵר֔וֹשׁ בְּיַ֖ד הַסָּרִיסִֽים׃ (ס)
\commenta{כְּדָת מַה לַּעֲשׂוֹת. מוּסָב עַל "וַיֹּאמֶר הַמֶּלֶךְ לַחֲכָמִים": }%endcomment
וַיֹּ֣אמֶר מומכן [מְמוּכָ֗ן] לִפְנֵ֤י הַמֶּ֙לֶךְ֙ וְהַשָּׂרִ֔ים לֹ֤א עַל־הַמֶּ֙לֶךְ֙ לְבַדּ֔וֹ עָוְתָ֖ה וַשְׁתִּ֣י הַמַּלְכָּ֑ה כִּ֤י עַל־כָּל־הַשָּׂרִים֙ וְעַל־כָּל־הָ֣עַמִּ֔ים אֲשֶׁ֕ר בְּכָל־מְדִינ֖וֹת הַמֶּ֥לֶךְ אֲחַשְׁוֵרֽוֹשׁ׃
\commenta{עָוְתָה. לְשׁוֹן עָו‍ֹן:}%endcomment
כִּֽי־יֵצֵ֤א דְבַר־הַמַּלְכָּה֙ עַל־כָּל־הַנָּשִׁ֔ים לְהַבְז֥וֹת בַּעְלֵיהֶ֖ן בְּעֵינֵיהֶ֑ן בְּאָמְרָ֗ם הַמֶּ֣לֶךְ אֲחַשְׁוֵר֡וֹשׁ אָמַ֞ר לְהָבִ֨יא אֶת־וַשְׁתִּ֧י הַמַּלְכָּ֛ה לְפָנָ֖יו וְלֹא־בָֽאָה׃
\commenta{כִּי יֵצֵא דְבַר הַמַּלְכָּה עַל כָּל הַנָּשִׁים. זֶה שֶׁבִּזְּתָה אֶת הַמֶּלֶךְ עַל כָּל הַנָּשִׁים לְהַבְזוֹת אַף הֵן אֶת בַּעֲלֵיהֶן:}%endcomment
וְֽהַיּ֨וֹם הַזֶּ֜ה תֹּאמַ֣רְנָה ׀ שָׂר֣וֹת פָּֽרַס־וּמָדַ֗י אֲשֶׁ֤ר שָֽׁמְעוּ֙ אֶת־דְּבַ֣ר הַמַּלְכָּ֔ה לְכֹ֖ל שָׂרֵ֣י הַמֶּ֑לֶךְ וּכְדַ֖י בִּזָּי֥וֹן וָקָֽצֶף׃
\commenta{תּאמַרְנָה שָׂרוֹת פָּרַס וּמָדַי וגו'. לְכֹל שָׂרֵי הַמֶּלֶךְ אֶת הַדָּבָר הַזֶּה. וַהֲרֵי זֶה מִקְרָא קָצֵר:}%endcomment
אִם־עַל־הַמֶּ֣לֶךְ ט֗וֹב יֵצֵ֤א דְבַר־מַלְכוּת֙ מִלְּפָנָ֔יו וְיִכָּתֵ֛ב בְּדָתֵ֥י פָֽרַס־וּמָדַ֖י וְלֹ֣א יַעֲב֑וֹר אֲשֶׁ֨ר לֹֽא־תָב֜וֹא וַשְׁתִּ֗י לִפְנֵי֙ הַמֶּ֣לֶךְ אֲחַשְׁוֵר֔וֹשׁ וּמַלְכוּתָהּ֙ יִתֵּ֣ן הַמֶּ֔לֶךְ לִרְעוּתָ֖הּ הַטּוֹבָ֥ה מִמֶּֽנָּה׃
\commenta{דְבַר מַלְכוּת. גְּזֵרַת מַלְכוּת שֶׁל נְקָמָה שֶׁצִּוָּה לְהָרְגָהּ:}%endcomment
וְנִשְׁמַע֩ פִּתְגָ֨ם הַמֶּ֤לֶךְ אֲשֶֽׁר־יַעֲשֶׂה֙ בְּכָל־מַלְכוּת֔וֹ כִּ֥י רַבָּ֖ה הִ֑יא וְכָל־הַנָּשִׁ֗ים יִתְּנ֤וּ יְקָר֙ לְבַעְלֵיהֶ֔ן לְמִגָּד֖וֹל וְעַד־קָטָֽן׃
וַיִּיטַב֙ הַדָּבָ֔ר בְּעֵינֵ֥י הַמֶּ֖לֶךְ וְהַשָּׂרִ֑ים וַיַּ֥עַשׂ הַמֶּ֖לֶךְ כִּדְבַ֥ר מְמוּכָֽן׃
וַיִּשְׁלַ֤ח סְפָרִים֙ אֶל־כָּל־מְדִינ֣וֹת הַמֶּ֔לֶךְ אֶל־מְדִינָ֤ה וּמְדִינָה֙ כִּכְתָבָ֔הּ וְאֶל־עַ֥ם וָעָ֖ם כִּלְשׁוֹנ֑וֹ לִהְי֤וֹת כָּל־אִישׁ֙ שֹׂרֵ֣ר בְּבֵית֔וֹ וּמְדַבֵּ֖ר כִּלְשׁ֥וֹן עַמּֽוֹ׃ (פ)
\commenta{וּמְדַבֵּר כִּלְשׁוֹן עַמּוֹ. כּוֹפֶה אֶת אִשְׁתּוֹ לִלְמֹד אֶת לְשׁוֹנוֹ אִם הִיא בַּת לָשׁוֹן אַחֵר:}%endcomment
\clearpage}

\newchap{פרק ב}
\twocol{אַחַר֙ הַדְּבָרִ֣ים הָאֵ֔לֶּה כְּשֹׁ֕ךְ חֲמַ֖ת הַמֶּ֣לֶךְ אֲחַשְׁוֵר֑וֹשׁ זָכַ֤ר אֶת־וַשְׁתִּי֙ וְאֵ֣ת אֲשֶׁר־עָשָׂ֔תָה וְאֵ֥ת אֲשֶׁר־נִגְזַ֖ר עָלֶֽיהָ׃
\commenta{זָכַר אֶת וַשְׁתִּי. אֶת יָפְיָהּ וְנֶעֱצַב:}%endcomment
וַיֹּאמְר֥וּ נַעֲרֵֽי־הַמֶּ֖לֶךְ מְשָׁרְתָ֑יו יְבַקְשׁ֥וּ לַמֶּ֛לֶךְ נְעָר֥וֹת בְּתוּל֖וֹת טוֹב֥וֹת מַרְאֶֽה׃
וְיַפְקֵ֨ד הַמֶּ֣לֶךְ פְּקִידִים֮ בְּכָל־מְדִינ֣וֹת מַלְכוּתוֹ֒ וְיִקְבְּצ֣וּ אֶת־כָּל־נַעֲרָֽה־בְ֠תוּלָה טוֹבַ֨ת מַרְאֶ֜ה אֶל־שׁוּשַׁ֤ן הַבִּירָה֙ אֶל־בֵּ֣ית הַנָּשִׁ֔ים אֶל־יַ֥ד הֵגֶ֛א סְרִ֥יס הַמֶּ֖לֶךְ שֹׁמֵ֣ר הַנָּשִׁ֑ים וְנָת֖וֹן תַּמְרוּקֵיהֶֽן׃
\commenta{וְיַפְקֵד הַמֶּלֶךְ פְּקִידִים. לְפִי שֶׁכָּל פָּקִיד וּפָקִיד יְדוּעוֹת לוֹ נָשִׁים הַיָּפוֹת שֶׁבִּמְדִינָתוֹ:}%endcomment
וְהַֽנַּעֲרָ֗ה אֲשֶׁ֤ר תִּיטַב֙ בְּעֵינֵ֣י הַמֶּ֔לֶךְ תִּמְלֹ֖ךְ תַּ֣חַת וַשְׁתִּ֑י וַיִּיטַ֧ב הַדָּבָ֛ר בְּעֵינֵ֥י הַמֶּ֖לֶךְ וַיַּ֥עַשׂ כֵּֽן׃ (ס)
אִ֣ישׁ יְהוּדִ֔י הָיָ֖ה בְּשׁוּשַׁ֣ן הַבִּירָ֑ה וּשְׁמ֣וֹ מָרְדֳּכַ֗י בֶּ֣ן יָאִ֧יר בֶּן־שִׁמְעִ֛י בֶּן־קִ֖ישׁ אִ֥ישׁ יְמִינִֽי׃
\commenta{אִישׁ יְהוּדִי. עַל שֶׁגָּלָה עִם גָּלוּת יְהוּדָה. כָּל אוֹתָן שֶׁגָּלוּ עִם מַלְכֵי יְהוּדָה הָיוּ קְרוּיִים "יְהוּדִים" בֵּין הַגּוֹיִם, וַאֲפִילוּ מִשֵּׁבֶט אַחֵר הֵם: }%endcomment
אֲשֶׁ֤ר הָגְלָה֙ מִיר֣וּשָׁלַ֔יִם עִם־הַגֹּלָה֙ אֲשֶׁ֣ר הָגְלְתָ֔ה עִ֖ם יְכָנְיָ֣ה מֶֽלֶךְ־יְהוּדָ֑ה אֲשֶׁ֣ר הֶגְלָ֔ה נְבוּכַדְנֶאצַּ֖ר מֶ֥לֶךְ בָּבֶֽל׃
וַיְהִ֨י אֹמֵ֜ן אֶת־הֲדַסָּ֗ה הִ֤יא אֶסְתֵּר֙ בַּת־דֹּד֔וֹ כִּ֛י אֵ֥ין לָ֖הּ אָ֣ב וָאֵ֑ם וְהַנַּעֲרָ֤ה יְפַת־תֹּ֙אַר֙ וְטוֹבַ֣ת מַרְאֶ֔ה וּבְמ֤וֹת אָבִ֙יהָ֙ וְאִמָּ֔הּ לְקָחָ֧הּ מָרְדֳּכַ֛י ל֖וֹ לְבַֽת׃
\commenta{לוֹ לְבַת. רַבּוֹתֵינוּ פֵּרְשׁוּ "לְבַיִת", לְאִשָּׁה: }%endcomment
וַיְהִ֗י בְּהִשָּׁמַ֤ע דְּבַר־הַמֶּ֙לֶךְ֙ וְדָת֔וֹ וּֽבְהִקָּבֵ֞ץ נְעָר֥וֹת רַבּ֛וֹת אֶל־שׁוּשַׁ֥ן הַבִּירָ֖ה אֶל־יַ֣ד הֵגָ֑י וַתִּלָּקַ֤ח אֶסְתֵּר֙ אֶל־בֵּ֣ית הַמֶּ֔לֶךְ אֶל־יַ֥ד הֵגַ֖י שֹׁמֵ֥ר הַנָּשִֽׁים׃
וַתִּיטַ֨ב הַנַּעֲרָ֣ה בְעֵינָיו֮ וַתִּשָּׂ֣א חֶ֣סֶד לְפָנָיו֒ וַ֠יְבַהֵל אֶת־תַּמְרוּקֶ֤יהָ וְאֶת־מָנוֹתֶ֙הָ֙ לָתֵ֣ת לָ֔הּ וְאֵת֙ שֶׁ֣בַע הַנְּעָר֔וֹת הָרְאֻי֥וֹת לָֽתֶת־לָ֖הּ מִבֵּ֣ית הַמֶּ֑לֶךְ וַיְשַׁנֶּ֧הָ וְאֶת־נַעֲרוֹתֶ֛יהָ לְט֖וֹב בֵּ֥ית הַנָּשִֽׁים׃
\commenta{וַיְבַהֵל אֶת תַּמְרוּקֶיהָ. זָרִיז וּמְמַהֵר בְּשֶׁלָּהּ, מִשֶּׁל כֻּלָּן: }%endcomment
לֹא־הִגִּ֣ידָה אֶסְתֵּ֔ר אֶת־עַמָּ֖הּ וְאֶת־מֽוֹלַדְתָּ֑הּ כִּ֧י מָרְדֳּכַ֛י צִוָּ֥ה עָלֶ֖יהָ אֲשֶׁ֥ר לֹא־תַגִּֽיד׃
\commenta{אֲשֶׁר לֹא תַגִּיד. כְּדֵי שֶׁיֹּאמְרוּ שֶׁהִיא מִמִּשְׁפָּחָה בְזוּיָה וִישַׁלְּחוּהָ, שֶׁאִם יֵדְעוּ שֶׁהִיא מִמִּשְׁפַּחַת שָׁאוּל הַמֶּלֶךְ הָיוּ מַחֲזִיקִים בָּהּ: }%endcomment
וּבְכָל־י֣וֹם וָי֔וֹם מָרְדֳּכַי֙ מִתְהַלֵּ֔ךְ לִפְנֵ֖י חֲצַ֣ר בֵּית־הַנָּשִׁ֑ים לָדַ֙עַת֙ אֶת־שְׁל֣וֹם אֶסְתֵּ֔ר וּמַה־יֵּעָשֶׂ֖ה בָּֽהּ׃
\commenta{וּמַה יֵּעָשֶׂה בָּהּ. זֶה אֶחָד מִשְּׁנֵי צַדִּיקִים שֶׁנִּתַּן לָהֶם רֶמֶז יְשׁוּעָה: דָּוִד וּמָרְדְּכַי. דָּוִד, שֶׁנֶּאֱמַר "גַּם אֶת הָאֲרִי גַּם אֶת הַדּוֹב הִכָּה עַבְדֶּךָ". אָמַר: "לֹא בָא לְיָדִי דָבָר זֶה אֶלָּא לִסְמֹךְ עָלָיו לְהִלָּחֵם עִם זֶה". וְכֵן מָרְדְּכַי אָמַר: "לֹא אֵרַע לְצַדֶּקֶת זוּ שֶׁתִּלָּקַח לְמִשְׁכַּב נָכְרִי אֶלָּא שֶׁעֲתִידָה לָקוּם לְהוֹשִׁיעַ לְיִשְׂרָאֵל". לְפִיכָךְ, הָיָה מְחַזֵּר לָדַעַת מַה יְּהֵא בְסוֹפָהּ: }%endcomment
וּבְהַגִּ֡יעַ תֹּר֩ נַעֲרָ֨ה וְנַעֲרָ֜ה לָב֣וֹא ׀ אֶל־הַמֶּ֣לֶךְ אֲחַשְׁוֵר֗וֹשׁ מִקֵּץ֩ הֱי֨וֹת לָ֜הּ כְּדָ֤ת הַנָּשִׁים֙ שְׁנֵ֣ים עָשָׂ֣ר חֹ֔דֶשׁ כִּ֛י כֵּ֥ן יִמְלְא֖וּ יְמֵ֣י מְרוּקֵיהֶ֑ן שִׁשָּׁ֤ה חֳדָשִׁים֙ בְּשֶׁ֣מֶן הַמֹּ֔ר וְשִׁשָּׁ֤ה חֳדָשִׁים֙ בַּבְּשָׂמִ֔ים וּבְתַמְרוּקֵ֖י הַנָּשִֽׁים׃
\commenta{תּר. זְמַן:}%endcomment
וּבָזֶ֕ה הַֽנַּעֲרָ֖ה בָּאָ֣ה אֶל־הַמֶּ֑לֶךְ אֵת֩ כָּל־אֲשֶׁ֨ר תֹּאמַ֜ר יִנָּ֤תֵֽן לָהּ֙ לָב֣וֹא עִמָּ֔הּ מִבֵּ֥ית הַנָּשִׁ֖ים עַד־בֵּ֥ית הַמֶּֽלֶךְ׃
\commenta{כָּל אֲשֶׁר תּאמַר. כָּל שְׂחוֹק וּמִינֵי זֶמֶר:}%endcomment
בָּעֶ֣רֶב ׀ הִ֣יא בָאָ֗ה וּ֠בַבֹּקֶר הִ֣יא שָׁבָ֞ה אֶל־בֵּ֤ית הַנָּשִׁים֙ שֵׁנִ֔י אֶל־יַ֧ד שַֽׁעֲשְׁגַ֛ז סְרִ֥יס הַמֶּ֖לֶךְ שֹׁמֵ֣ר הַפִּֽילַגְשִׁ֑ים לֹא־תָב֥וֹא עוֹד֙ אֶל־הַמֶּ֔לֶךְ כִּ֣י אִם־חָפֵ֥ץ בָּ֛הּ הַמֶּ֖לֶךְ וְנִקְרְאָ֥ה בְשֵֽׁם׃
\commenta{אֶל בֵּית הַנָּשִׁים שֵׁנִי. הַשֵּׁנִי:}%endcomment
וּבְהַגִּ֣יעַ תֹּר־אֶסְתֵּ֣ר בַּת־אֲבִיחַ֣יִל דֹּ֣ד מָרְדֳּכַ֡י אֲשֶׁר֩ לָקַֽח־ל֨וֹ לְבַ֜ת לָב֣וֹא אֶל־הַמֶּ֗לֶךְ לֹ֤א בִקְשָׁה֙ דָּבָ֔ר כִּ֠י אִ֣ם אֶת־אֲשֶׁ֥ר יֹאמַ֛ר הֵגַ֥י סְרִיס־הַמֶּ֖לֶךְ שֹׁמֵ֣ר הַנָּשִׁ֑ים וַתְּהִ֤י אֶסְתֵּר֙ נֹשֵׂ֣את חֵ֔ן בְּעֵינֵ֖י כָּל־רֹאֶֽיהָ׃
וַתִּלָּקַ֨ח אֶסְתֵּ֜ר אֶל־הַמֶּ֤לֶךְ אֲחַשְׁוֵרוֹשׁ֙ אֶל־בֵּ֣ית מַלְכוּת֔וֹ בַּחֹ֥דֶשׁ הָעֲשִׂירִ֖י הוּא־חֹ֣דֶשׁ טֵבֵ֑ת בִּשְׁנַת־שֶׁ֖בַע לְמַלְכוּתֽוֹ׃
\commenta{בַּחֹדֶשׁ הָעֲשִׂירִי. עֵת צִנָּה שֶׁהַגּוּף נֶהֱנֶה מִן הַגּוּף. זִמֵּן הַקָּדוֹשׁ בָּרוּךְ הוּא אוֹתוֹ עֵת צִנָּה כְּדֵי לְחַבְּבָהּ עָלָיו:}%endcomment
וַיֶּאֱהַ֨ב הַמֶּ֤לֶךְ אֶת־אֶסְתֵּר֙ מִכָּל־הַנָּשִׁ֔ים וַתִּשָּׂא־חֵ֥ן וָחֶ֛סֶד לְפָנָ֖יו מִכָּל־הַבְּתוּלֹ֑ת וַיָּ֤שֶׂם כֶּֽתֶר־מַלְכוּת֙ בְּרֹאשָׁ֔הּ וַיַּמְלִיכֶ֖הָ תַּ֥חַת וַשְׁתִּֽי׃
\commenta{מִכָּל הַנָּשִׁים. הַבְּעוּלוֹת, שֶׁאַף נָשִׁים הַבְּעוּלוֹת קִבֵּץ: }%endcomment
וַיַּ֨עַשׂ הַמֶּ֜לֶךְ מִשְׁתֶּ֣ה גָד֗וֹל לְכָל־שָׂרָיו֙ וַעֲבָדָ֔יו אֵ֖ת מִשְׁתֵּ֣ה אֶסְתֵּ֑ר וַהֲנָחָ֤ה לַמְּדִינוֹת֙ עָשָׂ֔ה וַיִּתֵּ֥ן מַשְׂאֵ֖ת כְּיַ֥ד הַמֶּֽלֶךְ׃
\commenta{וַהֲנָחָה לַמְּדִינוֹת עָשָׂה. לִכְבוֹדָהּ הֵנִיחַ לָהֶם מִן הַמַּס שֶׁעֲלֵיהֶם:}%endcomment
וּבְהִקָּבֵ֥ץ בְּתוּל֖וֹת שֵׁנִ֑ית וּמָרְדֳּכַ֖י יֹשֵׁ֥ב בְּשַֽׁעַר־הַמֶּֽלֶךְ׃
אֵ֣ין אֶסְתֵּ֗ר מַגֶּ֤דֶת מֽוֹלַדְתָּהּ֙ וְאֶת־עַמָּ֔הּ כַּאֲשֶׁ֛ר צִוָּ֥ה עָלֶ֖יהָ מָרְדֳּכָ֑י וְאֶת־מַאֲמַ֤ר מָרְדֳּכַי֙ אֶסְתֵּ֣ר עֹשָׂ֔ה כַּאֲשֶׁ֛ר הָיְתָ֥ה בְאָמְנָ֖ה אִתּֽוֹ׃ (ס)
\commenta{אֵין אֶסְתֵּר מַגֶּדֶת מוֹלַדְתָּהּ. לְפִי שֶׁמָּרְדְּכַי יוֹשֵׁב בְּשַׁעַר הַמֶּלֶךְ, הַמְזָרְזָהּ וְהַמְרַמְּזָהּ עַל כָּךְ: }%endcomment
בַּיָּמִ֣ים הָהֵ֔ם וּמָרְדֳּכַ֖י יֹשֵׁ֣ב בְּשַֽׁעַר־הַמֶּ֑לֶךְ קָצַף֩ בִּגְתָ֨ן וָתֶ֜רֶשׁ שְׁנֵֽי־סָרִיסֵ֤י הַמֶּ֙לֶךְ֙ מִשֹּׁמְרֵ֣י הַסַּ֔ף וַיְבַקְשׁוּ֙ לִשְׁלֹ֣חַ יָ֔ד בַּמֶּ֖לֶךְ אֲחַשְׁוֵֽרֹשׁ׃
\commenta{לִשְׁלֹחַ יָד. לְהַשְׁקוֹתוֹ סַם הַמָּוֶת:}%endcomment
וַיִּוָּדַ֤ע הַדָּבָר֙ לְמָרְדֳּכַ֔י וַיַּגֵּ֖ד לְאֶסְתֵּ֣ר הַמַּלְכָּ֑ה וַתֹּ֧אמֶר אֶסְתֵּ֛ר לַמֶּ֖לֶךְ בְּשֵׁ֥ם מָרְדֳּכָֽי׃
\commenta{וַיִּוָּדַע הַדָּבָר לְמָרְדְּכַי. שֶׁהָיוּ מְסַפְּרִים דִּבְרֵיהֶם לְפָנָיו בְּלָשׁוֹן טוּרְסִי, וְאֵין יוֹדְעִים שֶׁהָיָה מָרְדְּכַי מַכִּיר בְּשִׁבְעִים לְשׁוֹנוֹת שֶׁהָיָה מִיּוֹשְׁבֵי לִשְׁכַּת הַגָּזִית: }%endcomment
וַיְבֻקַּ֤שׁ הַדָּבָר֙ וַיִּמָּצֵ֔א וַיִּתָּל֥וּ שְׁנֵיהֶ֖ם עַל־עֵ֑ץ וַיִּכָּתֵ֗ב בְּסֵ֛פֶר דִּבְרֵ֥י הַיָּמִ֖ים לִפְנֵ֥י הַמֶּֽלֶךְ׃ (פ)
\commenta{וַיִּכָּתֵב בְּסֵפֶר דִּבְרֵי הַיָּמִים. הַטּוֹבָה שֶׁעָשָׂה מָרְדְּכַי לַמֶּלֶךְ:}%endcomment
\clearpage}

\newchap{פרק ג}
\twocol{אַחַ֣ר ׀ הַדְּבָרִ֣ים הָאֵ֗לֶּה גִּדַּל֩ הַמֶּ֨לֶךְ אֲחַשְׁוֵר֜וֹשׁ אֶת־הָמָ֧ן בֶּֽן־הַמְּדָ֛תָא הָאֲגָגִ֖י וַֽיְנַשְּׂאֵ֑הוּ וַיָּ֙שֶׂם֙ אֶת־כִּסְא֔וֹ מֵעַ֕ל כָּל־הַשָּׂרִ֖ים אֲשֶׁ֥ר אִתּֽוֹ׃
\commenta{אַחַר הַדְּבָרִים הָאֵלֶּה. שֶׁנִבְרֵאת רְפוּאָה זוּ לִהְיוֹת לִתְשׁוּעָה לְיִשְׂרָאֵל: }%endcomment
וְכָל־עַבְדֵ֨י הַמֶּ֜לֶךְ אֲשֶׁר־בְּשַׁ֣עַר הַמֶּ֗לֶךְ כֹּרְעִ֤ים וּמִֽשְׁתַּחֲוִים֙ לְהָמָ֔ן כִּי־כֵ֖ן צִוָּה־ל֣וֹ הַמֶּ֑לֶךְ וּמָ֨רְדֳּכַ֔י לֹ֥א יִכְרַ֖ע וְלֹ֥א יִֽשְׁתַּחֲוֶֽה׃
\commenta{כֹּרְעִים וּמִשְׁתַּחֲוִים. שֶׁעָשָׂה עַצְמוֹ אֱלוֹהַּ, לְפִיכָךְ, וּמָרְדְּכַי לֹא יִכְרַע וְלֹא יִשְׁתַּחֲוֶה: }%endcomment
וַיֹּ֨אמְר֜וּ עַבְדֵ֥י הַמֶּ֛לֶךְ אֲשֶׁר־בְּשַׁ֥עַר הַמֶּ֖לֶךְ לְמָרְדֳּכָ֑י מַדּ֙וּעַ֙ אַתָּ֣ה עוֹבֵ֔ר אֵ֖ת מִצְוַ֥ת הַמֶּֽלֶךְ׃
וַיְהִ֗י באמרם [כְּאָמְרָ֤ם] אֵלָיו֙ י֣וֹם וָי֔וֹם וְלֹ֥א שָׁמַ֖ע אֲלֵיהֶ֑ם וַיַּגִּ֣ידוּ לְהָמָ֗ן לִרְאוֹת֙ הֲיַֽעַמְדוּ֙ דִּבְרֵ֣י מָרְדֳּכַ֔י כִּֽי־הִגִּ֥יד לָהֶ֖ם אֲשֶׁר־ה֥וּא יְהוּדִֽי׃
\commenta{הֲיַעַמְדוּ דִּבְרֵי מָרְדְּכַי. הָאוֹמֵר שֶׁלֹּא יִשְׁתַּחֲוֶה עוֹלָמִית, כִּי הוּא יְהוּדִי, וְהוּזְהַר עַל עֲבוֹדַת אֱלִילִים: }%endcomment
וַיַּ֣רְא הָמָ֔ן כִּי־אֵ֣ין מָרְדֳּכַ֔י כֹּרֵ֥עַ וּמִֽשְׁתַּחֲוֶ֖ה ל֑וֹ וַיִּמָּלֵ֥א הָמָ֖ן חֵמָֽה׃
וַיִּ֣בֶז בְּעֵינָ֗יו לִשְׁלֹ֤ח יָד֙ בְּמָרְדֳּכַ֣י לְבַדּ֔וֹ כִּֽי־הִגִּ֥ידוּ ל֖וֹ אֶת־עַ֣ם מָרְדֳּכָ֑י וַיְבַקֵּ֣שׁ הָמָ֗ן לְהַשְׁמִ֧יד אֶת־כָּל־הַיְּהוּדִ֛ים אֲשֶׁ֛ר בְּכָל־מַלְכ֥וּת אֲחַשְׁוֵר֖וֹשׁ עַ֥ם מָרְדֳּכָֽי׃
בַּחֹ֤דֶשׁ הָרִאשׁוֹן֙ הוּא־חֹ֣דֶשׁ נִיסָ֔ן בִּשְׁנַת֙ שְׁתֵּ֣ים עֶשְׂרֵ֔ה לַמֶּ֖לֶךְ אֲחַשְׁוֵר֑וֹשׁ הִפִּ֣יל פּוּר֩ ה֨וּא הַגּוֹרָ֜ל לִפְנֵ֣י הָמָ֗ן מִיּ֧וֹם ׀ לְי֛וֹם וּמֵחֹ֛דֶשׁ לְחֹ֥דֶשׁ שְׁנֵים־עָשָׂ֖ר הוּא־חֹ֥דֶשׁ אֲדָֽר׃ (ס)
\commenta{הִפִּיל פּוּר. הִפִּיל מִי שֶׁהִפִּיל, וְלֹא פִּירֵשׁ מִי. וּמִקְרָא קָצֵר הוּא: }%endcomment
וַיֹּ֤אמֶר הָמָן֙ לַמֶּ֣לֶךְ אֲחַשְׁוֵר֔וֹשׁ יֶשְׁנ֣וֹ עַם־אֶחָ֗ד מְפֻזָּ֤ר וּמְפֹרָד֙ בֵּ֣ין הָֽעַמִּ֔ים בְּכֹ֖ל מְדִינ֣וֹת מַלְכוּתֶ֑ךָ וְדָתֵיהֶ֞ם שֹׁנ֣וֹת מִכָּל־עָ֗ם וְאֶת־דָּתֵ֤י הַמֶּ֙לֶךְ֙ אֵינָ֣ם עֹשִׂ֔ים וְלַמֶּ֥לֶךְ אֵין־שֹׁוֶ֖ה לְהַנִּיחָֽם׃
\commenta{וְאֶת דָּתֵי הַמֶּלֶךְ. לָתֵת מַס לַעֲבוֹדַת הַמֶּלֶךְ:}%endcomment
אִם־עַל־הַמֶּ֣לֶךְ ט֔וֹב יִכָּתֵ֖ב לְאַבְּדָ֑ם וַעֲשֶׂ֨רֶת אֲלָפִ֜ים כִּכַּר־כֶּ֗סֶף אֶשְׁקוֹל֙ עַל־יְדֵי֙ עֹשֵׂ֣י הַמְּלָאכָ֔ה לְהָבִ֖יא אֶל־גִּנְזֵ֥י הַמֶּֽלֶךְ׃
\commenta{יִכָּתֵב לְאַבְּדָם. יִכָּתֵב סְפָרִים לִשְׁלֹחַ לְשָׂרֵי הַמְּדִינוֹת לְאַבְּדָם:}%endcomment
וַיָּ֧סַר הַמֶּ֛לֶךְ אֶת־טַבַּעְתּ֖וֹ מֵעַ֣ל יָד֑וֹ וַֽיִּתְּנָ֗הּ לְהָמָ֧ן בֶּֽן־הַמְּדָ֛תָא הָאֲגָגִ֖י צֹרֵ֥ר הַיְּהוּדִֽים׃
\commenta{וַיָּסַר הַמֶּלֶךְ אֶת טַבַּעְתּוֹ. הוּא מַתַּן כָּל דָּבָר גָּדוֹל שֶׁיִּשְׁאֲלוּ מֵאֵת הַמֶּלֶךְ, לִהְיוֹת מִי שֶׁהַטַּבַּעַת בְּיָדוֹ שַׁלִּיט בְּכָל דְּבַר הַמֶּלֶךְ: }%endcomment
וַיֹּ֤אמֶר הַמֶּ֙לֶךְ֙ לְהָמָ֔ן הַכֶּ֖סֶף נָת֣וּן לָ֑ךְ וְהָעָ֕ם לַעֲשׂ֥וֹת בּ֖וֹ כַּטּ֥וֹב בְּעֵינֶֽיךָ׃
וַיִּקָּרְאוּ֩ סֹפְרֵ֨י הַמֶּ֜לֶךְ בַּחֹ֣דֶשׁ הָרִאשׁ֗וֹן בִּשְׁלוֹשָׁ֨ה עָשָׂ֣ר יוֹם֮ בּוֹ֒ וַיִּכָּתֵ֣ב כְּֽכָל־אֲשֶׁר־צִוָּ֣ה הָמָ֡ן אֶ֣ל אֲחַשְׁדַּרְפְּנֵֽי־הַ֠מֶּלֶךְ וְֽאֶל־הַפַּח֞וֹת אֲשֶׁ֣ר ׀ עַל־מְדִינָ֣ה וּמְדִינָ֗ה וְאֶל־שָׂ֤רֵי עַם֙ וָעָ֔ם מְדִינָ֤ה וּמְדִינָה֙ כִּכְתָבָ֔הּ וְעַ֥ם וָעָ֖ם כִּלְשׁוֹנ֑וֹ בְּשֵׁ֨ם הַמֶּ֤לֶךְ אֲחַשְׁוֵרֹשׁ֙ נִכְתָּ֔ב וְנֶחְתָּ֖ם בְּטַבַּ֥עַת הַמֶּֽלֶךְ׃
וְנִשְׁל֨וֹחַ סְפָרִ֜ים בְּיַ֣ד הָרָצִים֮ אֶל־כָּל־מְדִינ֣וֹת הַמֶּלֶךְ֒ לְהַשְׁמִ֡יד לַהֲרֹ֣ג וּלְאַבֵּ֣ד אֶת־כָּל־הַ֠יְּהוּדִים מִנַּ֨עַר וְעַד־זָקֵ֜ן טַ֤ף וְנָשִׁים֙ בְּי֣וֹם אֶחָ֔ד בִּשְׁלוֹשָׁ֥ה עָשָׂ֛ר לְחֹ֥דֶשׁ שְׁנֵים־עָשָׂ֖ר הוּא־חֹ֣דֶשׁ אֲדָ֑ר וּשְׁלָלָ֖ם לָבֽוֹז׃
\commenta{וְנִשְׁלוֹחַ סְפָרִים. וְיִהְיוּ נִשְׁלָחִים אישטר"א טרמי"ש בלע"ז. וְהוּא מִגִּזְרַת "אִם נִלְחוֹם נִלְחַם" נִלְחוֹם, "הַנִגְלֹה נִגְלֵיתִי" נִגְלֹה, "נִדְמֹה נִדְמֵיתִי": נִדְמֹה: }%endcomment
פַּתְשֶׁ֣גֶן הַכְּתָ֗ב לְהִנָּ֤תֵֽן דָּת֙ בְּכָל־מְדִינָ֣ה וּמְדִינָ֔ה גָּל֖וּי לְכָל־הָֽעַמִּ֑ים לִהְי֥וֹת עֲתִדִ֖ים לַיּ֥וֹם הַזֶּֽה׃
\commenta{פַּתְשֶׁגֶן. לָשׁוֹן אֲרַמִּי פַּתְשֶׁגֶן סִפּוּר הַכְּתָב דרישמאנ"ט בְּלַעַ"ז: }%endcomment
הָֽרָצִ֞ים יָצְא֤וּ דְחוּפִים֙ בִּדְבַ֣ר הַמֶּ֔לֶךְ וְהַדָּ֥ת נִתְּנָ֖ה בְּשׁוּשַׁ֣ן הַבִּירָ֑ה וְהַמֶּ֤לֶךְ וְהָמָן֙ יָשְׁב֣וּ לִשְׁתּ֔וֹת וְהָעִ֥יר שׁוּשָׁ֖ן נָבֽוֹכָה׃ (פ)
\commenta{וְהַדָּת נִתְּנָה בְּשׁוּשַׁן. מָקוֹם שֶׁהָיָה הַמֶּלֶךְ שָׁם נִתַּן הַחֹק בּוֹ בַיּוֹם, לִהְיוֹת עֲתִידִים לְיוֹם י"ג לְחֹדֶשׁ אֲדָר. לְכַךְ, }%endcomment
\clearpage}

\newchap{פרק ד}
\twocol{וּמָרְדֳּכַ֗י יָדַע֙ אֶת־כָּל־אֲשֶׁ֣ר נַעֲשָׂ֔ה וַיִּקְרַ֤ע מָרְדֳּכַי֙ אֶת־בְּגָדָ֔יו וַיִּלְבַּ֥שׁ שַׂ֖ק וָאֵ֑פֶר וַיֵּצֵא֙ בְּת֣וֹךְ הָעִ֔יר וַיִּזְעַ֛ק זְעָקָ֥ה גְדֹלָ֖ה וּמָרָֽה׃
\commenta{וּמָרְדְּכַי יָדַע. בַּעַל הַחֲלוֹם אָמַר לוֹ שֶׁהִסְכִּימוּ הָעֶלְיוֹנִים עַל כָּךְ לְפִי שֶׁהִשְׁתַּחֲווּ לַצֶּלֶם בִּימֵי נְבוּכַדְנֶצַר וְשֶׁנֶּהֱנוּ מִסְּעוּדַת אֲחַשְׁוֵרוֹשׁ:}%endcomment
וַיָּב֕וֹא עַ֖ד לִפְנֵ֣י שַֽׁעַר־הַמֶּ֑לֶךְ כִּ֣י אֵ֥ין לָב֛וֹא אֶל־שַׁ֥עַר הַמֶּ֖לֶךְ בִּלְב֥וּשׁ שָֽׂק׃
\commenta{כִּי אֵין לָבוֹא. אֵין דֶּרֶךְ אֶרֶץ לָבוֹא אֶל שַׁעַר הַמֶּלֶךְ בִּלְבוּשׁ שָׂק:}%endcomment
וּבְכָל־מְדִינָ֣ה וּמְדִינָ֗ה מְקוֹם֙ אֲשֶׁ֨ר דְּבַר־הַמֶּ֤לֶךְ וְדָתוֹ֙ מַגִּ֔יעַ אֵ֤בֶל גָּדוֹל֙ לַיְּהוּדִ֔ים וְצ֥וֹם וּבְכִ֖י וּמִסְפֵּ֑ד שַׂ֣ק וָאֵ֔פֶר יֻצַּ֖ע לָֽרַבִּֽים׃
\commenta{דְּבַר הַמֶּלֶךְ וְדָתוֹ. כְּשֶׁהַשְּׁלוּחִים נוֹשְׂאֵי הַסְּפָרִים עוֹבְרִים שָׁם נִתְּנָה הַדָּת בָּעִיר:}%endcomment
וַ֠תָּבוֹאינָה נַעֲר֨וֹת אֶסְתֵּ֤ר וְסָרִיסֶ֙יהָ֙ וַיַּגִּ֣ידוּ לָ֔הּ וַתִּתְחַלְחַ֥ל הַמַּלְכָּ֖ה מְאֹ֑ד וַתִּשְׁלַ֨ח בְּגָדִ֜ים לְהַלְבִּ֣ישׁ אֶֽת־מָרְדֳּכַ֗י וּלְהָסִ֥יר שַׂקּ֛וֹ מֵעָלָ֖יו וְלֹ֥א קִבֵּֽל׃
וַתִּקְרָא֩ אֶסְתֵּ֨ר לַהֲתָ֜ךְ מִסָּרִיסֵ֤י הַמֶּ֙לֶךְ֙ אֲשֶׁ֣ר הֶעֱמִ֣יד לְפָנֶ֔יהָ וַתְּצַוֵּ֖הוּ עַֽל־מָרְדֳּכָ֑י לָדַ֥עַת מַה־זֶּ֖ה וְעַל־מַה־זֶּֽה׃
וַיֵּצֵ֥א הֲתָ֖ךְ אֶֽל־מָרְדֳּכָ֑י אֶל־רְח֣וֹב הָעִ֔יר אֲשֶׁ֖ר לִפְנֵ֥י שַֽׁעַר־הַמֶּֽלֶךְ׃
וַיַּגֶּד־ל֣וֹ מָרְדֳּכַ֔י אֵ֖ת כָּל־אֲשֶׁ֣ר קָרָ֑הוּ וְאֵ֣ת ׀ פָּרָשַׁ֣ת הַכֶּ֗סֶף אֲשֶׁ֨ר אָמַ֤ר הָמָן֙ לִ֠שְׁקוֹל עַל־גִּנְזֵ֥י הַמֶּ֛לֶךְ ביהודיים [בַּיְּהוּדִ֖ים] לְאַבְּדָֽם׃
\commenta{פָּרָשַׁת הַכֶּסֶף. פֵּרוּשׁ הַכֶּסֶף:}%endcomment
וְאֶת־פַּתְשֶׁ֣גֶן כְּתָֽב־הַ֠דָּת אֲשֶׁר־נִתַּ֨ן בְּשׁוּשָׁ֤ן לְהַשְׁמִידָם֙ נָ֣תַן ל֔וֹ לְהַרְא֥וֹת אֶת־אֶסְתֵּ֖ר וּלְהַגִּ֣יד לָ֑הּ וּלְצַוּ֣וֹת עָלֶ֗יהָ לָב֨וֹא אֶל־הַמֶּ֧לֶךְ לְהִֽתְחַנֶּן־ל֛וֹ וּלְבַקֵּ֥שׁ מִלְּפָנָ֖יו עַל־עַמָּֽהּ׃
וַיָּב֖וֹא הֲתָ֑ךְ וַיַּגֵּ֣ד לְאֶסְתֵּ֔ר אֵ֖ת דִּבְרֵ֥י מָרְדֳּכָֽי׃
וַתֹּ֤אמֶר אֶסְתֵּר֙ לַהֲתָ֔ךְ וַתְּצַוֵּ֖הוּ אֶֽל־מָרְדֳּכָֽי׃
כָּל־עַבְדֵ֣י הַמֶּ֡לֶךְ וְעַם־מְדִינ֨וֹת הַמֶּ֜לֶךְ יֽוֹדְעִ֗ים אֲשֶׁ֣ר כָּל־אִ֣ישׁ וְאִשָּׁ֡ה אֲשֶׁ֣ר יָבֽוֹא־אֶל־הַמֶּלֶךְ֩ אֶל־הֶחָצֵ֨ר הַפְּנִימִ֜ית אֲשֶׁ֣ר לֹֽא־יִקָּרֵ֗א אַחַ֤ת דָּתוֹ֙ לְהָמִ֔ית לְ֠בַד מֵאֲשֶׁ֨ר יֽוֹשִׁיט־ל֥וֹ הַמֶּ֛לֶךְ אֶת־שַׁרְבִ֥יט הַזָּהָ֖ב וְחָיָ֑ה וַאֲנִ֗י לֹ֤א נִקְרֵ֙אתי֙ לָב֣וֹא אֶל־הַמֶּ֔לֶךְ זֶ֖ה שְׁלוֹשִׁ֥ים יֽוֹם׃
וַיַּגִּ֣ידוּ לְמָרְדֳּכָ֔י אֵ֖ת דִּבְרֵ֥י אֶסְתֵּֽר׃ (פ)
וַיֹּ֥אמֶר מָרְדֳּכַ֖י לְהָשִׁ֣יב אֶל־אֶסְתֵּ֑ר אַל־תְּדַמִּ֣י בְנַפְשֵׁ֔ךְ לְהִמָּלֵ֥ט בֵּית־הַמֶּ֖לֶךְ מִכָּל־הַיְּהוּדִֽים׃
\commenta{אַל תְּדַמִּי בְנַפְשֵׁךְ. אַל תַּחְשְׁבִי, כְּמוֹ "וְהָיָה כַּאֲשֶׁר דִּמִּיתִי". "אַל תְּדַמִּי בְנַפְשֵׁךְ": "אַל תְּהִי סְבוּרָה לְהִמָּלֵט בְּיוֹם הַהֲרֵגָה בְּבֵית הַמֶּלֶךְ, שֶׁאֵין אַתְּ רוֹצָה לְסַכֵּן אֶת עַצְמֵךְ עַכְשָׁיו עַל הַסָּפֵק לָבֹא אֶל הַמֶּלֶךְ שֶׁלֹּא בִרְשׁוּת": }%endcomment
כִּ֣י אִם־הַחֲרֵ֣שׁ תַּחֲרִישִׁי֮ בָּעֵ֣ת הַזֹּאת֒ רֶ֣וַח וְהַצָּלָ֞ה יַעֲמ֤וֹד לַיְּהוּדִים֙ מִמָּק֣וֹם אַחֵ֔ר וְאַ֥תְּ וּבֵית־אָבִ֖יךְ תֹּאבֵ֑דוּ וּמִ֣י יוֹדֵ֔עַ אִם־לְעֵ֣ת כָּזֹ֔את הִגַּ֖עַתְּ לַמַּלְכֽוּת׃
\commenta{וּמִי יוֹדֵעַ אִם לְעֵת כָּזֹאת הִגַּעַתְּ לַמַּלְכוּת. וּמִי יוֹדֵעַ אִם יַחְפֹּץ בָּךְ הַמֶּלֶךְ לַשָּׁנָה הַבָּאָה שֶׁהוּא זְמַן הַהֲרֵגָה:}%endcomment
וַתֹּ֥אמֶר אֶסְתֵּ֖ר לְהָשִׁ֥יב אֶֽל־מָרְדֳּכָֽי׃
לֵךְ֩ כְּנ֨וֹס אֶת־כָּל־הַיְּהוּדִ֜ים הַֽנִּמְצְאִ֣ים בְּשׁוּשָׁ֗ן וְצ֣וּמוּ עָ֠לַי וְאַל־תֹּאכְל֨וּ וְאַל־תִּשְׁתּ֜וּ שְׁלֹ֤שֶׁת יָמִים֙ לַ֣יְלָה וָי֔וֹם גַּם־אֲנִ֥י וְנַעֲרֹתַ֖י אָצ֣וּם כֵּ֑ן וּבְכֵ֞ן אָב֤וֹא אֶל־הַמֶּ֙לֶךְ֙ אֲשֶׁ֣ר לֹֽא־כַדָּ֔ת וְכַאֲשֶׁ֥ר אָבַ֖דְתִּי אָבָֽדְתִּי׃
\commenta{אֲשֶׁר לֹא כַדָּת. שֶׁאֵין דָּת לִכָּנֵס אֲשֶׁר לֹא יִקָּרֵא. וּמִדְרַשׁ אַגָּדָה: "אֲשֶׁר לֹא כַדָּת" שֶׁעַד עַתָּה בְאֹנֶס וְעַכְשָׁיו בְּרָצוֹן: }%endcomment
וַֽיַּעֲבֹ֖ר מָרְדֳּכָ֑י וַיַּ֕עַשׂ כְּכֹ֛ל אֲשֶׁר־צִוְּתָ֥ה עָלָ֖יו אֶסְתֵּֽר׃ (ס)
\commenta{וַיַּעֲבֹר מָרְדְּכָי. עַל דָּת, לְהִתְעַנּוֹת בְּיוֹם טוֹב רִאשׁוֹן שֶׁל פֶּסַח, שֶׁהִתְעַנָּה י"ד בְּנִיסָן וְט"ו וְט"ז, שֶׁהֲרֵי בְּיוֹם י"ג נִכְתְּבוּ הַסְּפָרִים: }%endcomment
\clearpage}

\newchap{פרק ה}
\twocol{וַיְהִ֣י ׀ בַּיּ֣וֹם הַשְּׁלִישִׁ֗י וַתִּלְבַּ֤שׁ אֶסְתֵּר֙ מַלְכ֔וּת וַֽתַּעֲמֹ֞ד בַּחֲצַ֤ר בֵּית־הַמֶּ֙לֶךְ֙ הַפְּנִימִ֔ית נֹ֖כַח בֵּ֣ית הַמֶּ֑לֶךְ וְ֠הַמֶּלֶךְ יוֹשֵׁ֞ב עַל־כִּסֵּ֤א מַלְכוּתוֹ֙ בְּבֵ֣ית הַמַּלְכ֔וּת נֹ֖כַח פֶּ֥תַח הַבָּֽיִת׃
\commenta{מַלְכוּת. בִּגְדֵי מַלְכוּת. וְרַבּוֹתֵינוּ אָמְרוּ: שֶׁלְּבָשַׁתָּה רוּחַ הַקֹּדֶשׁ, כְּמָה דְּאַתְּ אָמַר "וְרוּחַ לָבְשָׁה אֶת עֲמָשַׂי": }%endcomment
וַיְהִי֩ כִרְא֨וֹת הַמֶּ֜לֶךְ אֶת־אֶסְתֵּ֣ר הַמַּלְכָּ֗ה עֹמֶ֙דֶת֙ בֶּֽחָצֵ֔ר נָשְׂאָ֥ה חֵ֖ן בְּעֵינָ֑יו וַיּ֨וֹשֶׁט הַמֶּ֜לֶךְ לְאֶסְתֵּ֗ר אֶת־שַׁרְבִ֤יט הַזָּהָב֙ אֲשֶׁ֣ר בְּיָד֔וֹ וַתִּקְרַ֣ב אֶסְתֵּ֔ר וַתִּגַּ֖ע בְּרֹ֥אשׁ הַשַּׁרְבִֽיט׃ (ס)
וַיֹּ֤אמֶר לָהּ֙ הַמֶּ֔לֶךְ מַה־לָּ֖ךְ אֶסְתֵּ֣ר הַמַּלְכָּ֑ה וּמַה־בַּקָּשָׁתֵ֛ךְ עַד־חֲצִ֥י הַמַּלְכ֖וּת וְיִנָּ֥תֵֽן לָֽךְ׃
\commenta{עַד חֲצִי הַמַּלְכוּת. דָּבָר שֶׁהוּא בְאֶמְצַע וּבַחֲצִי הַמַּלְכוּת. הוּא בֵית הַמִּקְדָּשׁ, שֶׁהִתְחִילוּ לִבְנוֹתוֹ בִּימֵי כֹרֶשׁ, וְחָזַר בּוֹ וְצִוָּה לְבַטֵּל הַמְּלָאכָה. וַאֲחַשְׁוֵרוֹשׁ שֶׁעָמַד אַחֲרָיו גַּם הוּא בִּטֵּל הַמְּלָאכָה. וּפְשׁוּטוֹ שֶׁל מִקְרָא: אַף אִם תִּשְׁאֲלִי מִמֶּנִּי חֲצִי הַמַּלְכוּת, אֶתֵּן לָךְ: }%endcomment
וַתֹּ֣אמֶר אֶסְתֵּ֔ר אִם־עַל־הַמֶּ֖לֶךְ ט֑וֹב יָב֨וֹא הַמֶּ֤לֶךְ וְהָמָן֙ הַיּ֔וֹם אֶל־הַמִּשְׁתֶּ֖ה אֲשֶׁר־עָשִׂ֥יתִי לֽוֹ׃
\commenta{יָבוֹא הַמֶּלֶךְ וְהָמָן. רַבּוֹתֵינוּ אָמְרוּ טְעָמִים הַרְבֵּה בַדָּבָר: מָה רָאֲתָה אֶסְתֵּר שֶׁזִּמְּנָה אֶת הָמָן כְּדֵי לְקַנְאוֹ בַּמֶּלֶךְ וּבַשָּׂרִים, שֶׁהַמֶּלֶךְ יַחְשֹׁב שֶׁהוּא חוֹשֵׁק אֵלֶיהָ וְיַהַרְגֶנּוּ, וְעוֹד טְעָמִים רַבִּים: }%endcomment
וַיֹּ֣אמֶר הַמֶּ֔לֶךְ מַהֲרוּ֙ אֶת־הָמָ֔ן לַעֲשׂ֖וֹת אֶת־דְּבַ֣ר אֶסְתֵּ֑ר וַיָּבֹ֤א הַמֶּ֙לֶךְ֙ וְהָמָ֔ן אֶל־הַמִּשְׁתֶּ֖ה אֲשֶׁר־עָשְׂתָ֥ה אֶסְתֵּֽר׃
וַיֹּ֨אמֶר הַמֶּ֤לֶךְ לְאֶסְתֵּר֙ בְּמִשְׁתֵּ֣ה הַיַּ֔יִן מַה־שְּׁאֵלָתֵ֖ךְ וְיִנָּ֣תֵֽן לָ֑ךְ וּמַה־בַּקָּשָׁתֵ֛ךְ עַד־חֲצִ֥י הַמַּלְכ֖וּת וְתֵעָֽשׂ׃
וַתַּ֥עַן אֶסְתֵּ֖ר וַתֹּאמַ֑ר שְׁאֵלָתִ֖י וּבַקָּשָׁתִֽי׃
אִם־מָצָ֨אתִי חֵ֜ן בְּעֵינֵ֣י הַמֶּ֗לֶךְ וְאִם־עַל־הַמֶּ֙לֶךְ֙ ט֔וֹב לָתֵת֙ אֶת־שְׁאֵ֣לָתִ֔י וְלַעֲשׂ֖וֹת אֶת־בַּקָּשָׁתִ֑י יָב֧וֹא הַמֶּ֣לֶךְ וְהָמָ֗ן אֶל־הַמִּשְׁתֶּה֙ אֲשֶׁ֣ר אֶֽעֱשֶׂ֣ה לָהֶ֔ם וּמָחָ֥ר אֶֽעֱשֶׂ֖ה כִּדְבַ֥ר הַמֶּֽלֶךְ׃
\commenta{וּמָחָר אֶעֱשֶׂה כִּדְבַר הַמֶּלֶךְ. מַה שֶּׁבִּקַּשְׁתָּ מִמֶּנִּי כָּל הַיָּמִים, לְגַלּוֹת לְךָ אֶת עַמִּי וְאֶת מוֹלַדְתִּי: }%endcomment
וַיֵּצֵ֤א הָמָן֙ בַּיּ֣וֹם הַה֔וּא שָׂמֵ֖חַ וְט֣וֹב לֵ֑ב וְכִרְאוֹת֩ הָמָ֨ן אֶֽת־מָרְדֳּכַ֜י בְּשַׁ֣עַר הַמֶּ֗לֶךְ וְלֹא־קָם֙ וְלֹא־זָ֣ע מִמֶּ֔נּוּ וַיִּמָּלֵ֥א הָמָ֛ן עַֽל־מָרְדֳּכַ֖י חֵמָֽה׃
וַיִּתְאַפַּ֣ק הָמָ֔ן וַיָּב֖וֹא אֶל־בֵּית֑וֹ וַיִּשְׁלַ֛ח וַיָּבֵ֥א אֶת־אֹהֲבָ֖יו וְאֶת־זֶ֥רֶשׁ אִשְׁתּֽוֹ׃
\commenta{וַיִּתְאַפַּק. נִתְחַזֵּק לַעֲמֹד עַל כַּעְסוֹ, כִּי הָיָה יָרֵא לְהִנָּקֵם בְּלֹא רְשׁוּת. וַיִּתְאַפַּק אישטנט"ר בְּלַעַ"ז: }%endcomment
וַיְסַפֵּ֨ר לָהֶ֥ם הָמָ֛ן אֶת־כְּב֥וֹד עָשְׁר֖וֹ וְרֹ֣ב בָּנָ֑יו וְאֵת֩ כָּל־אֲשֶׁ֨ר גִּדְּל֤וֹ הַמֶּ֙לֶךְ֙ וְאֵ֣ת אֲשֶׁ֣ר נִשְּׂא֔וֹ עַל־הַשָּׂרִ֖ים וְעַבְדֵ֥י הַמֶּֽלֶךְ׃
וַיֹּאמֶר֮ הָמָן֒ אַ֣ף לֹא־הֵבִיאָה֩ אֶסְתֵּ֨ר הַמַּלְכָּ֧ה עִם־הַמֶּ֛לֶךְ אֶל־הַמִּשְׁתֶּ֥ה אֲשֶׁר־עָשָׂ֖תָה כִּ֣י אִם־אוֹתִ֑י וְגַם־לְמָחָ֛ר אֲנִ֥י קָֽרוּא־לָ֖הּ עִם־הַמֶּֽלֶךְ׃
וְכָל־זֶ֕ה אֵינֶ֥נּוּ שֹׁוֶ֖ה לִ֑י בְּכָל־עֵ֗ת אֲשֶׁ֨ר אֲנִ֤י רֹאֶה֙ אֶת־מָרְדֳּכַ֣י הַיְּהוּדִ֔י יוֹשֵׁ֖ב בְּשַׁ֥עַר הַמֶּֽלֶךְ׃
\commenta{אֵינֶנּוּ שֹׁוֶה לִי. אֵינִי חָשׁ לְכָל הַכָּבוֹד אֲשֶׁר לִי:}%endcomment
וַתֹּ֣אמֶר לוֹ֩ זֶ֨רֶשׁ אִשְׁתּ֜וֹ וְכָל־אֹֽהֲבָ֗יו יַֽעֲשׂוּ־עֵץ֮ גָּבֹ֣הַּ חֲמִשִּׁ֣ים אַמָּה֒ וּבַבֹּ֣קֶר ׀ אֱמֹ֣ר לַמֶּ֗לֶךְ וְיִתְל֤וּ אֶֽת־מָרְדֳּכַי֙ עָלָ֔יו וּבֹֽא־עִם־הַמֶּ֥לֶךְ אֶל הַמִּשְׁתֶּ֖ה שָׂמֵ֑חַ וַיִּיטַ֧ב הַדָּבָ֛ר לִפְנֵ֥י הָמָ֖ן וַיַּ֥עַשׂ הָעֵֽץ׃ (פ)
\clearpage}

\newchap{פרק ו}
\twocol{בַּלַּ֣יְלָה הַה֔וּא נָדְדָ֖ה שְׁנַ֣ת הַמֶּ֑לֶךְ וַיֹּ֗אמֶר לְהָבִ֞יא אֶת־סֵ֤פֶר הַזִּכְרֹנוֹת֙ דִּבְרֵ֣י הַיָּמִ֔ים וַיִּהְי֥וּ נִקְרָאִ֖ים לִפְנֵ֥י הַמֶּֽלֶךְ׃
\commenta{נָדְדָה שְׁנַת הַמֶּלֶךְ. נֵס הָיָה. וְיֵשׁ אוֹמְרִים שָׂם אֶת לִבּוֹ עַל שֶׁזִּמְּנָה אֶסְתֵּר אֶת הָמָן שֶׁמָּא נָתְנָה עֵינֶיהָ בּוֹ וְיַהַרְגֵהוּ:}%endcomment
וַיִּמָּצֵ֣א כָת֗וּב אֲשֶׁר֩ הִגִּ֨יד מָרְדֳּכַ֜י עַל־בִּגְתָ֣נָא וָתֶ֗רֶשׁ שְׁנֵי֙ סָרִיסֵ֣י הַמֶּ֔לֶךְ מִשֹּׁמְרֵ֖י הַסַּ֑ף אֲשֶׁ֤ר בִּקְשׁוּ֙ לִשְׁלֹ֣חַ יָ֔ד בַּמֶּ֖לֶךְ אֲחַשְׁוֵרֽוֹשׁ׃
וַיֹּ֣אמֶר הַמֶּ֔לֶךְ מַֽה־נַּעֲשָׂ֞ה יְקָ֧ר וּגְדוּלָּ֛ה לְמָרְדֳּכַ֖י עַל־זֶ֑ה וַיֹּ֨אמְר֜וּ נַעֲרֵ֤י הַמֶּ֙לֶךְ֙ מְשָׁ֣רְתָ֔יו לֹא־נַעֲשָׂ֥ה עִמּ֖וֹ דָּבָֽר׃
וַיֹּ֥אמֶר הַמֶּ֖לֶךְ מִ֣י בֶחָצֵ֑ר וְהָמָ֣ן בָּ֗א לַחֲצַ֤ר בֵּית־הַמֶּ֙לֶךְ֙ הַחִ֣יצוֹנָ֔ה לֵאמֹ֣ר לַמֶּ֔לֶךְ לִתְלוֹת֙ אֶֽת־מָרְדֳּכַ֔י עַל־הָעֵ֖ץ אֲשֶׁר־הֵכִ֥ין לֽוֹ׃
וַיֹּ֨אמְר֜וּ נַעֲרֵ֤י הַמֶּ֙לֶךְ֙ אֵלָ֔יו הִנֵּ֥ה הָמָ֖ן עֹמֵ֣ד בֶּחָצֵ֑ר וַיֹּ֥אמֶר הַמֶּ֖לֶךְ יָבֽוֹא׃
וַיָּבוֹא֮ הָמָן֒ וַיֹּ֤אמֶר לוֹ֙ הַמֶּ֔לֶךְ מַה־לַעֲשׂ֕וֹת בָּאִ֕ישׁ אֲשֶׁ֥ר הַמֶּ֖לֶךְ חָפֵ֣ץ בִּיקָר֑וֹ וַיֹּ֤אמֶר הָמָן֙ בְּלִבּ֔וֹ לְמִ֞י יַחְפֹּ֥ץ הַמֶּ֛לֶךְ לַעֲשׂ֥וֹת יְקָ֖ר יוֹתֵ֥ר מִמֶּֽנִּי׃
וַיֹּ֥אמֶר הָמָ֖ן אֶל־הַמֶּ֑לֶךְ אִ֕ישׁ אֲשֶׁ֥ר הַמֶּ֖לֶךְ חָפֵ֥ץ בִּיקָרֽוֹ׃
יָבִ֙יאוּ֙ לְב֣וּשׁ מַלְכ֔וּת אֲשֶׁ֥ר לָֽבַשׁ־בּ֖וֹ הַמֶּ֑לֶךְ וְס֗וּס אֲשֶׁ֨ר רָכַ֤ב עָלָיו֙ הַמֶּ֔לֶךְ וַאֲשֶׁ֥ר נִתַּ֛ן כֶּ֥תֶר מַלְכ֖וּת בְּרֹאשֽׁוֹ׃
וְנָת֨וֹן הַלְּב֜וּשׁ וְהַסּ֗וּס עַל־יַד־אִ֞ישׁ מִשָּׂרֵ֤י הַמֶּ֙לֶךְ֙ הַֽפַּרְתְּמִ֔ים וְהִלְבִּ֙ישׁוּ֙ אֶת־הָאִ֔ישׁ אֲשֶׁ֥ר הַמֶּ֖לֶךְ חָפֵ֣ץ בִּֽיקָר֑וֹ וְהִרְכִּיבֻ֤הוּ עַל־הַסּוּס֙ בִּרְח֣וֹב הָעִ֔יר וְקָרְא֣וּ לְפָנָ֔יו כָּ֚כָה יֵעָשֶׂ֣ה לָאִ֔ישׁ אֲשֶׁ֥ר הַמֶּ֖לֶךְ חָפֵ֥ץ בִּיקָרֽוֹ׃
\commenta{וְנָתוֹן הַלְּבוּשׁ וְהַסּוּס עַל יַד אִישׁ. וְאֶת הַכֶּתֶר לֹא הִזְכִּיר, שֶׁרָאָה עֵינוֹ שֶׁל מֶלֶךְ רָעָה עַל שֶׁאָמַר שֶׁיִּתְּנוּ הַכֶּתֶר בְּרֹאשׁ אָדָם: }%endcomment
וַיֹּ֨אמֶר הַמֶּ֜לֶךְ לְהָמָ֗ן מַ֠הֵר קַ֣ח אֶת־הַלְּב֤וּשׁ וְאֶת־הַסּוּס֙ כַּאֲשֶׁ֣ר דִּבַּ֔רְתָּ וַֽעֲשֵׂה־כֵן֙ לְמָרְדֳּכַ֣י הַיְּהוּדִ֔י הַיּוֹשֵׁ֖ב בְּשַׁ֣עַר הַמֶּ֑לֶךְ אַל־תַּפֵּ֣ל דָּבָ֔ר מִכֹּ֖ל אֲשֶׁ֥ר דִּבַּֽרְתָּ׃
וַיִּקַּ֤ח הָמָן֙ אֶת־הַלְּב֣וּשׁ וְאֶת־הַסּ֔וּס וַיַּלְבֵּ֖שׁ אֶֽת־מָרְדֳּכָ֑י וַיַּרְכִּיבֵ֙הוּ֙ בִּרְח֣וֹב הָעִ֔יר וַיִּקְרָ֣א לְפָנָ֔יו כָּ֚כָה יֵעָשֶׂ֣ה לָאִ֔ישׁ אֲשֶׁ֥ר הַמֶּ֖לֶךְ חָפֵ֥ץ בִּיקָרֽוֹ׃
וַיָּ֥שָׁב מָרְדֳּכַ֖י אֶל־שַׁ֣עַר הַמֶּ֑לֶךְ וְהָמָן֙ נִדְחַ֣ף אֶל־בֵּית֔וֹ אָבֵ֖ל וַחֲפ֥וּי רֹֽאשׁ׃
\commenta{וַיָּשָׁב מָרְדְּכַי. לְשַׂקּוֹ וּלְתַעֲנִיתוֹ:}%endcomment
וַיְסַפֵּ֨ר הָמָ֜ן לְזֶ֤רֶשׁ אִשְׁתּוֹ֙ וּלְכָל־אֹ֣הֲבָ֔יו אֵ֖ת כָּל־אֲשֶׁ֣ר קָרָ֑הוּ וַיֹּ֩אמְרוּ֩ ל֨וֹ חֲכָמָ֜יו וְזֶ֣רֶשׁ אִשְׁתּ֗וֹ אִ֣ם מִזֶּ֣רַע הַיְּהוּדִ֡ים מָרְדֳּכַ֞י אֲשֶׁר֩ הַחִלּ֨וֹתָ לִנְפֹּ֤ל לְפָנָיו֙ לֹא־תוּכַ֣ל ל֔וֹ כִּֽי־נָפ֥וֹל תִּפּ֖וֹל לְפָנָֽיו׃
\commenta{אֲשֶׁר הַחִלּוֹתָ לִנְפֹּל וגו'. אָמְרָה: אֻמָּה זוּ נִמְשְׁלוּ לְכוֹכָבִים וּלְעָפָר. כְּשֶׁהֵם יוֹרְדִים יוֹרְדִים עַד לֶעָפָר, וּכְשֶׁהֵם עוֹלִים עוֹלִים עַד לָרָקִיעַ וְעַד הַכּוֹכָבִים: }%endcomment
עוֹדָם֙ מְדַבְּרִ֣ים עִמּ֔וֹ וְסָרִיסֵ֥י הַמֶּ֖לֶךְ הִגִּ֑יעוּ וַיַּבְהִ֙לוּ֙ לְהָבִ֣יא אֶת־הָמָ֔ן אֶל־הַמִּשְׁתֶּ֖ה אֲשֶׁר־עָשְׂתָ֥ה אֶסְתֵּֽר׃
\clearpage}

\newchap{פרק ז}
\twocol{וַיָּבֹ֤א הַמֶּ֙לֶךְ֙ וְהָמָ֔ן לִשְׁתּ֖וֹת עִם־אֶסְתֵּ֥ר הַמַּלְכָּֽה׃
וַיֹּאמֶר֩ הַמֶּ֨לֶךְ לְאֶסְתֵּ֜ר גַּ֣ם בַּיּ֤וֹם הַשֵּׁנִי֙ בְּמִשְׁתֵּ֣ה הַיַּ֔יִן מַה־שְּׁאֵלָתֵ֛ךְ אֶסְתֵּ֥ר הַמַּלְכָּ֖ה וְתִנָּ֣תֵֽן לָ֑ךְ וּמַה־בַּקָּשָׁתֵ֛ךְ עַד־חֲצִ֥י הַמַּלְכ֖וּת וְתֵעָֽשׂ׃
וַתַּ֨עַן אֶסְתֵּ֤ר הַמַּלְכָּה֙ וַתֹּאמַ֔ר אִם־מָצָ֨אתִי חֵ֤ן בְּעֵינֶ֙יךָ֙ הַמֶּ֔לֶךְ וְאִם־עַל־הַמֶּ֖לֶךְ ט֑וֹב תִּנָּֽתֶן־לִ֤י נַפְשִׁי֙ בִּשְׁאֵ֣לָתִ֔י וְעַמִּ֖י בְּבַקָּשָׁתִֽי׃
\commenta{תִּנָּתֶן לִי נַפְשִׁי. שֶׁלֹּא אֵהָרֵג בִּשְׁלשָׁה עָשָׂר בַּאֲדָר שֶׁגָּזַרְתָּ גְזֵרַת הֲרֵגָה עַל עַמִּי וּמוֹלַדְתִּי:}%endcomment
כִּ֤י נִמְכַּ֙רְנוּ֙ אֲנִ֣י וְעַמִּ֔י לְהַשְׁמִ֖יד לַהֲר֣וֹג וּלְאַבֵּ֑ד וְ֠אִלּוּ לַעֲבָדִ֨ים וְלִשְׁפָח֤וֹת נִמְכַּ֙רְנוּ֙ הֶחֱרַ֔שְׁתִּי כִּ֣י אֵ֥ין הַצָּ֛ר שֹׁוֶ֖ה בְּנֵ֥זֶק הַמֶּֽלֶךְ׃ (ס)
\commenta{כִּי אֵין הַצָּר שֹׁוֶה בְּנֵזֶק הַמֶּלֶךְ. אֵינֶנּוּ חוֹשֵׁשׁ בְּנֵזֶק הַמֶּלֶךְ שֶׁאִלּוּ רָדַף אַחַר הֲנָאָתְךָ, הָיָה לוֹ לוֹמַר: מְכֹר אוֹתָם לַעֲבָדִים וְלִשְׁפָחוֹת וְקַבֵּל הַמָּמוֹן אוֹ הַחֲיֵה אוֹתָם לִהְיוֹת לְךָ לַעֲבָדִים הֵם וְזַרְעָם: }%endcomment
וַיֹּ֙אמֶר֙ הַמֶּ֣לֶךְ אֲחַשְׁוֵר֔וֹשׁ וַיֹּ֖אמֶר לְאֶסְתֵּ֣ר הַמַּלְכָּ֑ה מִ֣י ה֥וּא זֶה֙ וְאֵֽי־זֶ֣ה ה֔וּא אֲשֶׁר־מְלָא֥וֹ לִבּ֖וֹ לַעֲשׂ֥וֹת כֵּֽן׃
\commenta{וַיֹּאמֶר הַמֶּלֶךְ אֲחַשְׁוֵרוֹשׁ וַיֹּאמֶר לְאֶסְתֵּר הַמַּלְכָּה. כָּל מָקוֹם שֶׁנֶּאֱמַר: "וַיֹּאמֶר, וַיֹּאמֶר" שְׁנֵי פְעָמִים, אֵינוֹ אֶלָּא לְמִדְרָשׁ, וּמִדְרָשׁוֹ שֶׁל זֶה: בַּתְּחִלָּה הָיָה מְדַבֵּר עִמָּהּ עַל יְדֵי שָׁלִיחַ. עַכְשָׁיו שֶׁיָּדַע שֶׁמִּמִּשְׁפַּחַת מְלָכִים הִיא, דִּבֵּר עִמָּהּ הוּא בְעַצְמוֹ: }%endcomment
וַתֹּ֣אמֶר־אֶסְתֵּ֔ר אִ֚ישׁ צַ֣ר וְאוֹיֵ֔ב הָמָ֥ן הָרָ֖ע הַזֶּ֑ה וְהָמָ֣ן נִבְעַ֔ת מִלִּפְנֵ֥י הַמֶּ֖לֶךְ וְהַמַּלְכָּֽה׃
וְהַמֶּ֜לֶךְ קָ֤ם בַּחֲמָתוֹ֙ מִמִּשְׁתֵּ֣ה הַיַּ֔יִן אֶל־גִּנַּ֖ת הַבִּיתָ֑ן וְהָמָ֣ן עָמַ֗ד לְבַקֵּ֤שׁ עַל־נַפְשׁוֹ֙ מֵֽאֶסְתֵּ֣ר הַמַּלְכָּ֔ה כִּ֣י רָאָ֔ה כִּֽי־כָלְתָ֥ה אֵלָ֛יו הָרָעָ֖ה מֵאֵ֥ת הַמֶּֽלֶךְ׃
\commenta{כִּי כָלְתָה. נִגְמְרָה הָרָעָה וְהַשִּׂנְאָה וְהַנְּקָמָה:}%endcomment
וְהַמֶּ֡לֶךְ שָׁב֩ מִגִּנַּ֨ת הַבִּיתָ֜ן אֶל־בֵּ֣ית ׀ מִשְׁתֵּ֣ה הַיַּ֗יִן וְהָמָן֙ נֹפֵ֔ל עַל־הַמִּטָּה֙ אֲשֶׁ֣ר אֶסְתֵּ֣ר עָלֶ֔יהָ וַיֹּ֣אמֶר הַמֶּ֔לֶךְ הֲ֠גַם לִכְבּ֧וֹשׁ אֶת־הַמַּלְכָּ֛ה עִמִּ֖י בַּבָּ֑יִת הַדָּבָ֗ר יָצָא֙ מִפִּ֣י הַמֶּ֔לֶךְ וּפְנֵ֥י הָמָ֖ן חָפֽוּ׃ (ס)
\commenta{וְהָמָן נֹפֵל. הַמַּלְאָךְ דְּחָפוֹ:}%endcomment
וַיֹּ֣אמֶר חַ֠רְבוֹנָה אֶחָ֨ד מִן־הַסָּרִיסִ֜ים לִפְנֵ֣י הַמֶּ֗לֶךְ גַּ֣ם הִנֵּה־הָעֵ֣ץ אֲשֶׁר־עָשָׂ֪ה הָמָ֟ן לְֽמָרְדֳּכַ֞י אֲשֶׁ֧ר דִּבֶּר־ט֣וֹב עַל־הַמֶּ֗לֶךְ עֹמֵד֙ בְּבֵ֣ית הָמָ֔ן גָּבֹ֖הַּ חֲמִשִּׁ֣ים אַמָּ֑ה וַיֹּ֥אמֶר הַמֶּ֖לֶךְ תְּלֻ֥הוּ עָלָֽיו׃
\commenta{גַּם הִנֵּה הָעֵץ. "גַּם" רָעָה אַחֶרֶת עָשָׂה שֶׁהֵכִין הָעֵץ לִתְלוֹת אוֹהֲבוֹ שֶׁל מֶלֶךְ שֶׁהִצִּיל הַמֶּלֶךְ מִסַּם הַמָּוֶת: }%endcomment
וַיִּתְלוּ֙ אֶת־הָמָ֔ן עַל־הָעֵ֖ץ אֲשֶׁר־הֵכִ֣ין לְמָרְדֳּכָ֑י וַחֲמַ֥ת הַמֶּ֖לֶךְ שָׁכָֽכָה׃ (פ)
\clearpage}

\newchap{פרק ח}
\twocol{בַּיּ֣וֹם הַה֗וּא נָתַ֞ן הַמֶּ֤לֶךְ אֲחַשְׁוֵרוֹשׁ֙ לְאֶסְתֵּ֣ר הַמַּלְכָּ֔ה אֶת־בֵּ֥ית הָמָ֖ן צֹרֵ֣ר היהודיים [הַיְּהוּדִ֑ים] וּמָרְדֳּכַ֗י בָּ֚א לִפְנֵ֣י הַמֶּ֔לֶךְ כִּֽי־הִגִּ֥ידָה אֶסְתֵּ֖ר מַ֥ה הוּא־לָֽהּ׃
\commenta{מָה הוּא לָהּ. אֵיךְ הוּא קָרוֹב לָהּ:}%endcomment
וַיָּ֨סַר הַמֶּ֜לֶךְ אֶת־טַבַּעְתּ֗וֹ אֲשֶׁ֤ר הֶֽעֱבִיר֙ מֵֽהָמָ֔ן וַֽיִּתְּנָ֖הּ לְמָרְדֳּכָ֑י וַתָּ֧שֶׂם אֶסְתֵּ֛ר אֶֽת־מָרְדֳּכַ֖י עַל־בֵּ֥ית הָמָֽן׃ (פ)
וַתּ֣וֹסֶף אֶסְתֵּ֗ר וַתְּדַבֵּר֙ לִפְנֵ֣י הַמֶּ֔לֶךְ וַתִּפֹּ֖ל לִפְנֵ֣י רַגְלָ֑יו וַתֵּ֣בְךְּ וַתִּתְחַנֶּן־ל֗וֹ לְהַֽעֲבִיר֙ אֶת־רָעַת֙ הָמָ֣ן הָֽאֲגָגִ֔י וְאֵת֙ מַֽחֲשַׁבְתּ֔וֹ אֲשֶׁ֥ר חָשַׁ֖ב עַל־הַיְּהוּדִֽים׃
\commenta{לְהַעֲבִיר אֶת רָעַת הָמָן. שֶׁלֹּא תִתְקַיֵּם עֲצָתוֹ הָרָעָה:}%endcomment
וַיּ֤וֹשֶׁט הַמֶּ֙לֶךְ֙ לְאֶסְתֵּ֔ר אֵ֖ת שַׁרְבִ֣ט הַזָּהָ֑ב וַתָּ֣קָם אֶסְתֵּ֔ר וַֽתַּעֲמֹ֖ד לִפְנֵ֥י הַמֶּֽלֶךְ׃
וַ֠תֹּאמֶר אִם־עַל־הַמֶּ֨לֶךְ ט֜וֹב וְאִם־מָצָ֧אתִי חֵ֣ן לְפָנָ֗יו וְכָשֵׁ֤ר הַדָּבָר֙ לִפְנֵ֣י הַמֶּ֔לֶךְ וְטוֹבָ֥ה אֲנִ֖י בְּעֵינָ֑יו יִכָּתֵ֞ב לְהָשִׁ֣יב אֶת־הַסְּפָרִ֗ים מַחֲשֶׁ֜בֶת הָמָ֤ן בֶּֽן־הַמְּדָ֙תָא֙ הָאֲגָגִ֔י אֲשֶׁ֣ר כָּתַ֗ב לְאַבֵּד֙ אֶת־הַיְּהוּדִ֔ים אֲשֶׁ֖ר בְּכָל־מְדִינ֥וֹת הַמֶּֽלֶךְ׃
כִּ֠י אֵיכָכָ֤ה אוּכַל֙ וְֽרָאִ֔יתִי בָּרָעָ֖ה אֲשֶׁר־יִמְצָ֣א אֶת־עַמִּ֑י וְאֵֽיכָכָ֤ה אוּכַל֙ וְֽרָאִ֔יתִי בְּאָבְדַ֖ן מוֹלַדְתִּֽי׃ (ס)
וַיֹּ֨אמֶר הַמֶּ֤לֶךְ אֲחַשְׁוֵרֹשׁ֙ לְאֶסְתֵּ֣ר הַמַּלְכָּ֔ה וּֽלְמָרְדֳּכַ֖י הַיְּהוּדִ֑י הִנֵּ֨ה בֵית־הָמָ֜ן נָתַ֣תִּי לְאֶסְתֵּ֗ר וְאֹתוֹ֙ תָּל֣וּ עַל־הָעֵ֔ץ עַ֛ל אֲשֶׁר־שָׁלַ֥ח יָד֖וֹ ביהודיים [בַּיְּהוּדִֽים׃]
\commenta{הִנֵּה בֵית הָמָן וגו'. וּמֵעַתָּה הַכֹּל רוֹאִים שֶׁאֲנִי חָפֵץ בָּכֶם, וְכָל מַה שֶּׁתּאמְרוּ יַאֲמִינוּ הַכֹּל שֶׁמֵּאִתִּי הוּא. לְפִיכָךְ, אֵין צְרִיכִין אַתֶּם לַהֲשִׁיבָם, אֶלָּא "כִּתְבוּ" סְפָרִים אֲחֵרִים "כַּטּוֹב בְּעֵינֵיכֶם": }%endcomment
וְ֠אַתֶּם כִּתְב֨וּ עַל־הַיְּהוּדִ֜ים כַּטּ֤וֹב בְּעֵֽינֵיכֶם֙ בְּשֵׁ֣ם הַמֶּ֔לֶךְ וְחִתְמ֖וּ בְּטַבַּ֣עַת הַמֶּ֑לֶךְ כִּֽי־כְתָ֞ב אֲשֶׁר־נִכְתָּ֣ב בְּשֵׁם־הַמֶּ֗לֶךְ וְנַחְתּ֛וֹם בְּטַבַּ֥עַת הַמֶּ֖לֶךְ אֵ֥ין לְהָשִֽׁיב׃
\commenta{אֵין לְהָשִׁיב. אֵין נָאֶה לַהֲשִׁיבוֹ וְלַעֲשׂוֹת כְּתַב הַמֶּלֶךְ בְּזִיּוּף:}%endcomment
וַיִּקָּרְא֣וּ סֹפְרֵֽי־הַמֶּ֣לֶךְ בָּֽעֵת־הַ֠הִיא בַּחֹ֨דֶשׁ הַשְּׁלִישִׁ֜י הוּא־חֹ֣דֶשׁ סִיוָ֗ן בִּשְׁלוֹשָׁ֣ה וְעֶשְׂרִים֮ בּוֹ֒ וַיִּכָּתֵ֣ב כְּֽכָל־אֲשֶׁר־צִוָּ֣ה מָרְדֳּכַ֣י אֶל־הַיְּהוּדִ֡ים וְאֶ֣ל הָאֲחַשְׁדַּרְפְּנִֽים־וְהַפַּחוֹת֩ וְשָׂרֵ֨י הַמְּדִינ֜וֹת אֲשֶׁ֣ר ׀ מֵהֹ֣דּוּ וְעַד־כּ֗וּשׁ שֶׁ֣בַע וְעֶשְׂרִ֤ים וּמֵאָה֙ מְדִינָ֔ה מְדִינָ֤ה וּמְדִינָה֙ כִּכְתָבָ֔הּ וְעַ֥ם וָעָ֖ם כִּלְשֹׁנ֑וֹ וְאֶ֨ל־הַיְּהוּדִ֔ים כִּכְתָבָ֖ם וְכִלְשׁוֹנָֽם׃
\commenta{כִּכְתָבָהּ. בָּאוֹתִיּוֹת שֶׁלָּהּ:}%endcomment
וַיִּכְתֹּ֗ב בְּשֵׁם֙ הַמֶּ֣לֶךְ אֲחַשְׁוֵרֹ֔שׁ וַיַּחְתֹּ֖ם בְּטַבַּ֣עַת הַמֶּ֑לֶךְ וַיִּשְׁלַ֣ח סְפָרִ֡ים בְּיַד֩ הָרָצִ֨ים בַּסּוּסִ֜ים רֹכְבֵ֤י הָרֶ֙כֶשׁ֙ הָֽאֲחַשְׁתְּרָנִ֔ים בְּנֵ֖י הָֽרַמָּכִֽים׃
\commenta{בְּיַד הָרָצִים. רוֹכְבֵי סוּסִים שֶׁצִּוָּה לָהֶם לָרוּץ:}%endcomment
אֲשֶׁר֩ נָתַ֨ן הַמֶּ֜לֶךְ לַיְּהוּדִ֣ים ׀ אֲשֶׁ֣ר בְּכָל־עִיר־וָעִ֗יר לְהִקָּהֵל֮ וְלַעֲמֹ֣ד עַל־נַפְשָׁם֒ לְהַשְׁמִיד֩ וְלַהֲרֹ֨ג וּלְאַבֵּ֜ד אֶת־כָּל־חֵ֨יל עַ֧ם וּמְדִינָ֛ה הַצָּרִ֥ים אֹתָ֖ם טַ֣ף וְנָשִׁ֑ים וּשְׁלָלָ֖ם לָבֽוֹז׃
\commenta{וּשְׁלָלָם לָבוֹז. כַּאֲשֶׁר נִכְתַּב בָּרִאשׁוֹנוֹת וְהֵם בַּבִּזָּה לֹא שָׁלְחוּ אֶת יָדָם, שֶׁהֶרְאוּ לַכֹּל שֶׁלֹּא נַעֲשָׂה לְשֵׁם מָמוֹן: }%endcomment
בְּי֣וֹם אֶחָ֔ד בְּכָל־מְדִינ֖וֹת הַמֶּ֣לֶךְ אֲחַשְׁוֵר֑וֹשׁ בִּשְׁלוֹשָׁ֥ה עָשָׂ֛ר לְחֹ֥דֶשׁ שְׁנֵים־עָשָׂ֖ר הוּא־חֹ֥דֶשׁ אֲדָֽר׃
פַּתְשֶׁ֣גֶן הַכְּתָ֗ב לְהִנָּ֤תֵֽן דָּת֙ בְּכָל־מְדִינָ֣ה וּמְדִינָ֔ה גָּל֖וּי לְכָל־הָעַמִּ֑ים וְלִהְי֨וֹת היהודיים [הַיְּהוּדִ֤ים] עתודים [עֲתִידִים֙] לַיּ֣וֹם הַזֶּ֔ה לְהִנָּקֵ֖ם מֵאֹיְבֵיהֶֽם׃
\commenta{פַּתְשֶׁגֶן. אִגֶּרֶת מְפֹרָשׁ:}%endcomment
הָרָצִ֞ים רֹכְבֵ֤י הָרֶ֙כֶשׁ֙ הָֽאֲחַשְׁתְּרָנִ֔ים יָֽצְא֛וּ מְבֹהָלִ֥ים וּדְחוּפִ֖ים בִּדְבַ֣ר הַמֶּ֑לֶךְ וְהַדָּ֥ת נִתְּנָ֖ה בְּשׁוּשַׁ֥ן הַבִּירָֽה׃ (פ)
\commenta{מְבֹהָלִים. מְמַהֲרִים אוֹתָם לַעֲשׂוֹת מְהֵרָה, לְפִי שֶׁלֹּא הָיָה לָהֶם פְּנַאי, שֶׁהָיָה לָהֶם לְהַקְדִּים רָצִים הָרִאשׁוֹנִים לְהַעֲבִירָם: }%endcomment
וּמָרְדֳּכַ֞י יָצָ֣א ׀ מִלִּפְנֵ֣י הַמֶּ֗לֶךְ בִּלְב֤וּשׁ מַלְכוּת֙ תְּכֵ֣לֶת וָח֔וּר וַעֲטֶ֤רֶת זָהָב֙ גְּדוֹלָ֔ה וְתַכְרִ֥יךְ בּ֖וּץ וְאַרְגָּמָ֑ן וְהָעִ֣יר שׁוּשָׁ֔ן צָהֲלָ֖ה וְשָׂמֵֽחָה׃
\commenta{וְתַכְרִיךְ בּוּץ. מַעֲטֵה בוּץ, טַלִּית הֶעָשׂוּי לְהִתְעַטֵּף: }%endcomment
לַיְּהוּדִ֕ים הָֽיְתָ֥ה אוֹרָ֖ה וְשִׂמְחָ֑ה וְשָׂשֹׂ֖ן וִיקָֽר׃
וּבְכָל־מְדִינָ֨ה וּמְדִינָ֜ה וּבְכָל־עִ֣יר וָעִ֗יר מְקוֹם֙ אֲשֶׁ֨ר דְּבַר־הַמֶּ֤לֶךְ וְדָתוֹ֙ מַגִּ֔יעַ שִׂמְחָ֤ה וְשָׂשׂוֹן֙ לַיְּהוּדִ֔ים מִשְׁתֶּ֖ה וְי֣וֹם ט֑וֹב וְרַבִּ֞ים מֵֽעַמֵּ֤י הָאָ֙רֶץ֙ מִֽתְיַהֲדִ֔ים כִּֽי־נָפַ֥ל פַּֽחַד־הַיְּהוּדִ֖ים עֲלֵיהֶֽם׃
\commenta{מִתְיַהֲדִים. מִתְגַּיְּרִים:}%endcomment
\clearpage}

\newchap{פרק ט}
\twocol{וּבִשְׁנֵים֩ עָשָׂ֨ר חֹ֜דֶשׁ הוּא־חֹ֣דֶשׁ אֲדָ֗ר בִּשְׁלוֹשָׁ֨ה עָשָׂ֥ר יוֹם֙ בּ֔וֹ אֲשֶׁ֨ר הִגִּ֧יעַ דְּבַר־הַמֶּ֛לֶךְ וְדָת֖וֹ לְהֵעָשׂ֑וֹת בַּיּ֗וֹם אֲשֶׁ֨ר שִׂבְּר֜וּ אֹיְבֵ֤י הַיְּהוּדִים֙ לִשְׁל֣וֹט בָּהֶ֔ם וְנַהֲפ֣וֹךְ ה֔וּא אֲשֶׁ֨ר יִשְׁלְט֧וּ הַיְּהוּדִ֛ים הֵ֖מָּה בְּשֹׂנְאֵיהֶֽם׃
נִקְהֲל֨וּ הַיְּהוּדִ֜ים בְּעָרֵיהֶ֗ם בְּכָל־מְדִינוֹת֙ הַמֶּ֣לֶךְ אֳחַשְׁוֵר֔וֹשׁ לִשְׁלֹ֣חַ יָ֔ד בִּמְבַקְשֵׁ֖י רָֽעָתָ֑ם וְאִישׁ֙ לֹא־עָמַ֣ד לִפְנֵיהֶ֔ם כִּֽי־נָפַ֥ל פַּחְדָּ֖ם עַל־כָּל־הָעַמִּֽים׃
וְכָל־שָׂרֵ֨י הַמְּדִינ֜וֹת וְהָאֲחַשְׁדַּרְפְּנִ֣ים וְהַפַּח֗וֹת וְעֹשֵׂ֤י הַמְּלָאכָה֙ אֲשֶׁ֣ר לַמֶּ֔לֶךְ מְנַשְּׂאִ֖ים אֶת־הַיְּהוּדִ֑ים כִּֽי־נָפַ֥ל פַּֽחַד־מָרְדֳּכַ֖י עֲלֵיהֶֽם׃
\commenta{וְעֹשֵׂי הַמְּלָאכָה. אוֹתָם שֶׁהָיוּ מְמֻנִּים לַעֲשׂוֹת צָרְכֵי הַמֶּלֶךְ:}%endcomment
כִּֽי־גָ֤דוֹל מָרְדֳּכַי֙ בְּבֵ֣ית הַמֶּ֔לֶךְ וְשָׁמְע֖וֹ הוֹלֵ֣ךְ בְּכָל־הַמְּדִינ֑וֹת כִּֽי־הָאִ֥ישׁ מָרְדֳּכַ֖י הוֹלֵ֥ךְ וְגָדֽוֹל׃ (פ)
וַיַּכּ֤וּ הַיְּהוּדִים֙ בְּכָל־אֹ֣יְבֵיהֶ֔ם מַכַּת־חֶ֥רֶב וְהֶ֖רֶג וְאַבְדָ֑ן וַיַּֽעֲשׂ֥וּ בְשֹׂנְאֵיהֶ֖ם כִּרְצוֹנָֽם׃
וּבְשׁוּשַׁ֣ן הַבִּירָ֗ה הָרְג֤וּ הַיְּהוּדִים֙ וְאַבֵּ֔ד חֲמֵ֥שׁ מֵא֖וֹת אִֽישׁ׃
וְאֵ֧ת ׀ פַּרְשַׁנְדָּ֛תָא וְאֵ֥ת ׀ דַּֽלְפ֖וֹן וְאֵ֥ת ׀ אַסְפָּֽתָא׃
וְאֵ֧ת ׀ פּוֹרָ֛תָא וְאֵ֥ת ׀ אֲדַלְיָ֖א וְאֵ֥ת ׀ אֲרִידָֽתָא׃
וְאֵ֤ת ׀ פַּרְמַ֙שְׁתָּא֙ וְאֵ֣ת ׀ אֲרִיסַ֔י וְאֵ֥ת ׀ אֲרִדַ֖י וְאֵ֥ת ׀ וַיְזָֽתָא׃
עֲ֠שֶׂרֶת בְּנֵ֨י הָמָ֧ן בֶּֽן־הַמְּדָ֛תָא צֹרֵ֥ר הַיְּהוּדִ֖ים הָרָ֑גוּ וּבַ֨בִּזָּ֔ה לֹ֥א שָׁלְח֖וּ אֶת־יָדָֽם׃
\commenta{עֲשֶׂרֶת בְּנֵי הָמָן. רָאִיתִי בְסֵדֶר עוֹלָם אֵלּוּ י' שֶׁכָּתְבוּ שִׂטְנָה עַל יְהוּדָה וִירוּשָׁלָיִם כְּמוֹ שֶׁכָּתוּב בְּסֵפֶר עֶזְרָא "וּבְמַלְכוּת אֲחַשְׁוֵרוֹשׁ בִּתְחִלַּת מַלְכוּתוֹ כָּתְבוּ שִׂטְנָה עַל ישְׁבֵי יְהוּדָה וִירוּשָׁלָיִם". וּמָה הִיא הַשִׂטְנָה? לְבַטֵּל הָעוֹלִים מִן הַגּוֹלָה בִּימֵי כוֹרֶשׁ, שֶׁהִתְחִילוּ לִבְנוֹת אֶת הַבַּיִת וְהִלְשִׁינוּ עֲלֵיהֶם הַכּוּתִים וְהֶחֱדִילוּם וּכְשֶׁמֵּת כּוֹרֶשׁ וּמָלַךְ אֲחַשְׁוֵרוֹשׁ וְהִתְנַשֵׂא הָמָן, דָּאַג שֶׁלֹּא יַעַסְקוּ אוֹתָן שֶׁבִּירוּשָׁלַיִם בַּבִּנְיָן וְשָׁלְחוּ בְשֵׁם אֲחַשְׁוֵרוֹשׁ לְשָׂרֵי עֵבֶר הַנָּהָר לְבַטְּלָן: }%endcomment
בַּיּ֣וֹם הַה֗וּא בָּ֣א מִסְפַּ֧ר הַֽהֲרוּגִ֛ים בְּשׁוּשַׁ֥ן הַבִּירָ֖ה לִפְנֵ֥י הַמֶּֽלֶךְ׃ (ס)
וַיֹּ֨אמֶר הַמֶּ֜לֶךְ לְאֶסְתֵּ֣ר הַמַּלְכָּ֗ה בְּשׁוּשַׁ֣ן הַבִּירָ֡ה הָרְגוּ֩ הַיְּהוּדִ֨ים וְאַבֵּ֜ד חֲמֵ֧שׁ מֵא֣וֹת אִ֗ישׁ וְאֵת֙ עֲשֶׂ֣רֶת בְּנֵֽי־הָמָ֔ן בִּשְׁאָ֛ר מְדִינ֥וֹת הַמֶּ֖לֶךְ מֶ֣ה עָשׂ֑וּ וּמַה־שְּׁאֵֽלָתֵךְ֙ וְיִנָּ֣תֵֽן לָ֔ךְ וּמַה־בַּקָּשָׁתֵ֥ךְ ע֖וֹד וְתֵעָֽשׂ׃
וַתֹּ֤אמֶר אֶסְתֵּר֙ אִם־עַל־הַמֶּ֣לֶךְ ט֔וֹב יִנָּתֵ֣ן גַּם־מָחָ֗ר לַיְּהוּדִים֙ אֲשֶׁ֣ר בְּשׁוּשָׁ֔ן לַעֲשׂ֖וֹת כְּדָ֣ת הַיּ֑וֹם וְאֵ֛ת עֲשֶׂ֥רֶת בְּנֵֽי־הָמָ֖ן יִתְל֥וּ עַל־הָעֵֽץ׃
\commenta{וְאֵת עֲשֶׂרֶת בְּנֵי הָמָן יִתְלוּ עַל הָעֵץ. אוֹתָן שֶׁנֶּהֶרְגוּ:}%endcomment
וַיֹּ֤אמֶר הַמֶּ֙לֶךְ֙ לְהֵֽעָשׂ֣וֹת כֵּ֔ן וַתִּנָּתֵ֥ן דָּ֖ת בְּשׁוּשָׁ֑ן וְאֵ֛ת עֲשֶׂ֥רֶת בְּנֵֽי־הָמָ֖ן תָּלֽוּ׃
\commenta{וַתִּנָּתֵן דָּת. נִגְזַר חֹק מֵאֵת הַמֶּלֶךְ:}%endcomment
וַיִּֽקָּהֲל֞וּ היהודיים [הַיְּהוּדִ֣ים] אֲשֶׁר־בְּשׁוּשָׁ֗ן גַּ֠ם בְּי֣וֹם אַרְבָּעָ֤ה עָשָׂר֙ לְחֹ֣דֶשׁ אֲדָ֔ר וַיַּֽהַרְג֣וּ בְשׁוּשָׁ֔ן שְׁלֹ֥שׁ מֵא֖וֹת אִ֑ישׁ וּבַ֨בִּזָּ֔ה לֹ֥א שָׁלְח֖וּ אֶת־יָדָֽם׃
וּשְׁאָ֣ר הַיְּהוּדִ֡ים אֲשֶׁר֩ בִּמְדִינ֨וֹת הַמֶּ֜לֶךְ נִקְהֲל֣וּ ׀ וְעָמֹ֣ד עַל־נַפְשָׁ֗ם וְנ֙וֹחַ֙ מֵאֹ֣יְבֵיהֶ֔ם וְהָרֹג֙ בְּשֹׂ֣נְאֵיהֶ֔ם חֲמִשָּׁ֥ה וְשִׁבְעִ֖ים אָ֑לֶף וּבַ֨בִּזָּ֔ה לֹ֥א שָֽׁלְח֖וּ אֶת־יָדָֽם׃
בְּיוֹם־שְׁלֹשָׁ֥ה עָשָׂ֖ר לְחֹ֣דֶשׁ אֲדָ֑ר וְנ֗וֹחַ בְּאַרְבָּעָ֤ה עָשָׂר֙ בּ֔וֹ וְעָשֹׂ֣ה אֹת֔וֹ י֖וֹם מִשְׁתֶּ֥ה וְשִׂמְחָֽה׃
והיהודיים [וְהַיְּהוּדִ֣ים] אֲשֶׁר־בְּשׁוּשָׁ֗ן נִקְהֲלוּ֙ בִּשְׁלֹשָׁ֤ה עָשָׂר֙ בּ֔וֹ וּבְאַרְבָּעָ֥ה עָשָׂ֖ר בּ֑וֹ וְנ֗וֹחַ בַּחֲמִשָּׁ֤ה עָשָׂר֙ בּ֔וֹ וְעָשֹׂ֣ה אֹת֔וֹ י֖וֹם מִשְׁתֶּ֥ה וְשִׂמְחָֽה׃
עַל־כֵּ֞ן הַיְּהוּדִ֣ים הפרוזים [הַפְּרָזִ֗ים] הַיֹּשְׁבִים֮ בְּעָרֵ֣י הַפְּרָזוֹת֒ עֹשִׂ֗ים אֵ֠ת י֣וֹם אַרְבָּעָ֤ה עָשָׂר֙ לְחֹ֣דֶשׁ אֲדָ֔ר שִׂמְחָ֥ה וּמִשְׁתֶּ֖ה וְי֣וֹם ט֑וֹב וּמִשְׁל֥וֹחַ מָנ֖וֹת אִ֥ישׁ לְרֵעֵֽהוּ׃ (פ)
\commenta{הַפְּרָזִים. שֶׁאֵינָם יוֹשְׁבִים בְּעָרֵי חוֹמָה בְּאַרְבָּעָה עָשָׂר, וּמֻקָּפִין חוֹמָה בְּט"ו כְּשׁוּשַׁן. וְהֶקֵּף זֶה צָרִיךְ שֶׁיִּהְיֶה מִימוֹת יְהוֹשֻׁעַ בִּן נוּן. כַּךְ דָּרְשׁוּ וְלָמְדוּ רַבּוֹתֵינוּ: }%endcomment
וַיִּכְתֹּ֣ב מָרְדֳּכַ֔י אֶת־הַדְּבָרִ֖ים הָאֵ֑לֶּה וַיִּשְׁלַ֨ח סְפָרִ֜ים אֶל־כָּל־הַיְּהוּדִ֗ים אֲשֶׁר֙ בְּכָל־מְדִינוֹת֙ הַמֶּ֣לֶךְ אֲחַשְׁוֵר֔וֹשׁ הַקְּרוֹבִ֖ים וְהָרְחוֹקִֽים׃
\commenta{וַיִּכְתּב מָרְדְּכַי. הִיא הַמְּגִלָּה הַזֹּאת כְּמוֹת שֶׁהִיא:}%endcomment
לְקַיֵּם֮ עֲלֵיהֶם֒ לִהְי֣וֹת עֹשִׂ֗ים אֵ֠ת י֣וֹם אַרְבָּעָ֤ה עָשָׂר֙ לְחֹ֣דֶשׁ אֲדָ֔ר וְאֵ֛ת יוֹם־חֲמִשָּׁ֥ה עָשָׂ֖ר בּ֑וֹ בְּכָל־שָׁנָ֖ה וְשָׁנָֽה׃
כַּיָּמִ֗ים אֲשֶׁר־נָ֨חוּ בָהֶ֤ם הַיְּהוּדִים֙ מֵא֣וֹיְבֵיהֶ֔ם וְהַחֹ֗דֶשׁ אֲשֶׁר֩ נֶהְפַּ֨ךְ לָהֶ֤ם מִיָּגוֹן֙ לְשִׂמְחָ֔ה וּמֵאֵ֖בֶל לְי֣וֹם ט֑וֹב לַעֲשׂ֣וֹת אוֹתָ֗ם יְמֵי֙ מִשְׁתֶּ֣ה וְשִׂמְחָ֔ה וּמִשְׁל֤וֹחַ מָנוֹת֙ אִ֣ישׁ לְרֵעֵ֔הוּ וּמַתָּנ֖וֹת לָֽאֶבְיוֹנִֽים׃
וְקִבֵּל֙ הַיְּהוּדִ֔ים אֵ֥ת אֲשֶׁר־הֵחֵ֖לּוּ לַעֲשׂ֑וֹת וְאֵ֛ת אֲשֶׁר־כָּתַ֥ב מָרְדֳּכַ֖י אֲלֵיהֶֽם׃
כִּי֩ הָמָ֨ן בֶּֽן־הַמְּדָ֜תָא הָֽאֲגָגִ֗י צֹרֵר֙ כָּל־הַיְּהוּדִ֔ים חָשַׁ֥ב עַל־הַיְּהוּדִ֖ים לְאַבְּדָ֑ם וְהִפִּ֥יל פּוּר֙ ה֣וּא הַגּוֹרָ֔ל לְהֻמָּ֖ם וּֽלְאַבְּדָֽם׃
\commenta{כִּי הָמָן בֶּן הַמְּדָתָא. חָשַׁב לְהֻמָּם וּלְאַבְּדָם:}%endcomment
וּבְבֹאָהּ֮ לִפְנֵ֣י הַמֶּלֶךְ֒ אָמַ֣ר עִם־הַסֵּ֔פֶר יָשׁ֞וּב מַחֲשַׁבְתּ֧וֹ הָרָעָ֛ה אֲשֶׁר־חָשַׁ֥ב עַל־הַיְּהוּדִ֖ים עַל־רֹאשׁ֑וֹ וְתָל֥וּ אֹת֛וֹ וְאֶת־בָּנָ֖יו עַל־הָעֵֽץ׃
\commenta{וּבְבֹאָהּ. אֶסְתֵּר אֶל הַמֶּלֶךְ לְהִתְחַנֶּן לוֹ:}%endcomment
עַל־כֵּ֡ן קָֽרְאוּ֩ לַיָּמִ֨ים הָאֵ֤לֶּה פוּרִים֙ עַל־שֵׁ֣ם הַפּ֔וּר עַל־כֵּ֕ן עַל־כָּל־דִּבְרֵ֖י הָאִגֶּ֣רֶת הַזֹּ֑את וּמָֽה־רָא֣וּ עַל־כָּ֔כָה וּמָ֥ה הִגִּ֖יעַ אֲלֵיהֶֽם׃
\commenta{עַל כֵּן עַל כָּל דִּבְרֵי הָאִגֶּרֶת הַזֹּאת. נִקְבְּעוּ הַיָּמִים הָאֵלֶּה וּלְכַךְ נִכְתְּבָה לָדַעַת דּוֹרוֹת הַבָּאִים:}%endcomment
קִיְּמ֣וּ וקבל [וְקִבְּל֣וּ] הַיְּהוּדִים֩ ׀ עֲלֵיהֶ֨ם ׀ וְעַל־זַרְעָ֜ם וְעַ֨ל כָּל־הַנִּלְוִ֤ים עֲלֵיהֶם֙ וְלֹ֣א יַעֲב֔וֹר לִהְי֣וֹת עֹשִׂ֗ים אֵ֣ת שְׁנֵ֤י הַיָּמִים֙ הָאֵ֔לֶּה כִּכְתָבָ֖ם וְכִזְמַנָּ֑ם בְּכָל־שָׁנָ֖ה וְשָׁנָֽה׃
\commenta{הַנִּלְוִים עֲלֵיהֶם. גֵּרִים הָעֲתִידִים לְהִתְגַּיֵּר:}%endcomment
וְהַיָּמִ֣ים הָ֠אֵלֶּה נִזְכָּרִ֨ים וְנַעֲשִׂ֜ים בְּכָל־דּ֣וֹר וָד֗וֹר מִשְׁפָּחָה֙ וּמִשְׁפָּחָ֔ה מְדִינָ֥ה וּמְדִינָ֖ה וְעִ֣יר וָעִ֑יר וִימֵ֞י הַפּוּרִ֣ים הָאֵ֗לֶּה לֹ֤א יַֽעַבְרוּ֙ מִתּ֣וֹךְ הַיְּהוּדִ֔ים וְזִכְרָ֖ם לֹא־יָס֥וּף מִזַּרְעָֽם׃ (ס)
\commenta{נִזְכָּרִים. בִּקְרִיאַת הַמְּגִלָּה:}%endcomment
וַ֠תִּכְתֹּב אֶסְתֵּ֨ר הַמַּלְכָּ֧ה בַת־אֲבִיחַ֛יִל וּמָרְדֳּכַ֥י הַיְּהוּדִ֖י אֶת־כָּל־תֹּ֑קֶף לְקַיֵּ֗ם אֵ֣ת אִגֶּ֧רֶת הַפּוּרִ֛ים הַזֹּ֖את הַשֵּׁנִֽית׃
\commenta{אֶת כָּל תּקֶף. תָּקְפּוֹ שֶׁל נֵס, שֶׁל אֲחַשְׁוֵרוֹשׁ וְשֶׁל הָמָן וְשֶׁל מָרְדְּכַי וְשֶׁל אֶסְתֵּר: }%endcomment
וַיִּשְׁלַ֨ח סְפָרִ֜ים אֶל־כָּל־הַיְּהוּדִ֗ים אֶל־שֶׁ֨בַע וְעֶשְׂרִ֤ים וּמֵאָה֙ מְדִינָ֔ה מַלְכ֖וּת אֲחַשְׁוֵר֑וֹשׁ דִּבְרֵ֥י שָׁל֖וֹם וֶאֱמֶֽת׃
לְקַיֵּ֡ם אֵת־יְמֵי֩ הַפֻּרִ֨ים הָאֵ֜לֶּה בִּזְמַנֵּיהֶ֗ם כַּאֲשֶׁר֩ קִיַּ֨ם עֲלֵיהֶ֜ם מָרְדֳּכַ֤י הַיְּהוּדִי֙ וְאֶסְתֵּ֣ר הַמַּלְכָּ֔ה וְכַאֲשֶׁ֛ר קִיְּמ֥וּ עַל־נַפְשָׁ֖ם וְעַל־זַרְעָ֑ם דִּבְרֵ֥י הַצֹּמ֖וֹת וְזַעֲקָתָֽם׃
וּמַאֲמַ֣ר אֶסְתֵּ֔ר קִיַּ֕ם דִּבְרֵ֥י הַפֻּרִ֖ים הָאֵ֑לֶּה וְנִכְתָּ֖ב בַּסֵּֽפֶר׃ (פ)
\commenta{וּמַאֲמַר אֶסְתֵּר קִיַּם וגו'. אֶסְתֵּר בִּקְשָׁה מֵאֵת חַכְמֵי הַדּוֹר לְקָבְעָהּ וְלִכְתּב סֵפֶר זֶה עִם שְׁאָר הַכְּתוּבִים. וְזֶהוּ "וְנִכְתָּב בַּסֵּפֶר": }%endcomment
\clearpage}

\newchap{פרק י}
\twocol{וַיָּשֶׂם֩ הַמֶּ֨לֶךְ אחשרש [אֲחַשְׁוֵר֧וֹשׁ ׀] מַ֛ס עַל־הָאָ֖רֶץ וְאִיֵּ֥י הַיָּֽם׃
וְכָל־מַעֲשֵׂ֤ה תָקְפּוֹ֙ וּגְב֣וּרָת֔וֹ וּפָרָשַׁת֙ גְּדֻלַּ֣ת מָרְדֳּכַ֔י אֲשֶׁ֥ר גִּדְּל֖וֹ הַמֶּ֑לֶךְ הֲלוֹא־הֵ֣ם כְּתוּבִ֗ים עַל־סֵ֙פֶר֙ דִּבְרֵ֣י הַיָּמִ֔ים לְמַלְכֵ֖י מָדַ֥י וּפָרָֽס׃
כִּ֣י ׀ מָרְדֳּכַ֣י הַיְּהוּדִ֗י מִשְׁנֶה֙ לַמֶּ֣לֶךְ אֲחַשְׁוֵר֔וֹשׁ וְגָדוֹל֙ לַיְּהוּדִ֔ים וְרָצ֖וּי לְרֹ֣ב אֶחָ֑יו דֹּרֵ֥שׁ טוֹב֙ לְעַמּ֔וֹ וְדֹבֵ֥ר שָׁל֖וֹם לְכָל־זַרְעֽוֹ׃
\commenta{לְרֹב אֶחָיו. וְלֹא לְכָל אֶחָיו, מְלַמֵּד שֶׁפֵּרְשׁוּ מִמֶּנּוּ מִקְצַת סַנְהֶדְרִין, לְפִי שֶׁנַּעֲשָׂה קָרוֹב לַמַּלְכוּת וְהָיָה בָטֵל מִתַּלְמוּדוֹ: }%endcomment
}
\addpart{משנה מגילה}\renewcommand{\partname}[1]{משנה מגילה}
\clearpage
\newchap{פרק א}
\hebeng{מגילה נקראת באחד עשר בשנים עשר. בשלשה עשר. בארבעה עשר. בחמשה עשר. לא פחות ולא יותר. כרכין המוקפין חומה מימות יהושע בן נון. קורין בחמשה עשר. כפרים ועיירות גדולות. קורין בארבעה עשר. אלא שהכפרים מקדימין ליום הכניסה: 
}{	The Megillah is read on the eleventh (of Adar), the twelfth, the thirteenth, the fourteenth, and the fifteenth, {[sometimes on one; sometimes, on the other, as explained below]} — not earlier (than the eleventh) and not later (than the fifteenth). Cities surrounded by a wall from the days of Joshua the son of Nun read on the fifteenth, {[it being written (Esther 9:19): "Therefore, the Jews of the outlying towns, who live in the unwalled cities, celebrate the fourteenth, etc." The unwalled cities, celebrating the fourteenth, the implication is that the walled cities celebrate the fifteenth. And "from the days of Joshua" is derived by identity: "perazi" ("unwalled," here) - "perazi" (Deuteronomy 3:5): "aside from the unwalled cities." Just as there, (perazi) from the days of Joshua the son of Nun; here, too, from the days of Joshua the son of Nun. And they ordained that the cities surrounded by a wall from the days of Joshua, even if they are not surrounded by a wall today, read on the fifteenth, like Shushan, in order to accord honor to Eretz Yisrael, which was in ruins in the days of Mordecai and Esther, that they, too, read as the men of Shushan and be regarded as if they were walled cities, even though now they are in ruins, so that there be a remembrance of Eretz Yisrael in this miracle. And Joshua is mentioned because he was the first who began to war against Amalek, viz. (Exodus 17:14): "Write this (the erasing of Amalek) as a remembrance in a scroll, and place it in the ears of Joshua, etc."]} The villages and the large cities read on the fourteenth; but the villages may advance it (the Megillah reading) to the "day of assembly" (yom haknissah). {[That is, since the walled cities read on the fifteenth, and the unwalled, on the fourteenth, all are included. How, then, could the eleventh, the twelfth, and the thirteenth obtain? The answer: The villages were permitted to advance their reading to the "day of assembly" — Monday or Thursday before the fourteenth — these (Mondays and Thursdays) being the days of assembly, when the villages assemble in the cities for judgment. For beth-din sit on Mondays and Thursdays by the ordinance of Ezra. Or it may be because the villages assemble in the cities on Mondays and Thursdays to hear the reading of the Torah. For the villagers are not so expert in the reading and need one of the men of the city to read for them; and the sages did not make them exert themselves to come back on the fourteenth, so that they be free on Purim to supply the needs of the Purim feast for the men of the cities. And they found an allusion for this in the Megillah, viz. (Esther (9:31): "to fulfill these days of Purim in their times" (bizmaneihem). If Mordecai and Esther instituted only the fourteenth and the fifteenth mentioned therein, we should have "zmanam" (connoting two times). Why "zmaneihem"? (connoting four times)? We are hereby apprised that two more times were added, aside from those mentioned in the Megillah. And it was not necessary for Scripture to include the thirteenth as fit for reading, because the miracle, essentially, occurred on that day. For it was on that day that the Jews gathered together to avenge themselves of their enemies, both in Shushan and in the other provinces. Perforce, then, Scripture adds only the eleventh and the twelfth. And it is not to be suggested that the sixteenth and seventeenth after the fourteenth and fifteenth written in the Megillah are intended, it being written (Ibid. 27): "and (the fifteenth) not to be passed by."]}}

\hebeng{כיצד. חל להיות יום ארבעה עשר בשני. כפרים ועיירות גדולות קורין בו ביום. ומוקפות חומה למחר. חל להיות בשלישי או ברביעי כפרים מקדימין ליום הכניסה. ועיירות גדולות קורין בו ביום. ומוקפות חומה למחר. חל להיות בחמישי. כפרים ועיירות גדולות קורין בו ביום. ומוקפות חומה למחר. חל להיות ערב שבת. כפרים מקדימין ליום הכניסה. ועיירות גדולות ומוקפות חומה קורין בו ביום. חל להיות בשבת. כפרים ועיירות גדולות מקדימין וקורין ליום הכניסה. ומוקפות חומה למחר. חל להיות אחר השבת כפרים מקדימין ליום הכניסה. ועיירות גדולות קורין בו ביום. ומוקפות חומה למחר: 
}{	How so? If the fourteenth falls on a Monday, the villages and the large cities read on that day, and the walled cities, the next day. If it falls on Tuesday or Wednesday, the villages advance it to the day of assembly, the large cities read it on the fourteenth, and the walled cities, the next day. If it falls on Thursday, the villages and the large cities read on that day, and the walled cities, the next day. If it falls on Sabbath eve, the villages advance it to the day of assembly, and the large cities and the walled cities read it on that day. {[For there is no Megillah reading on the Sabbath, a decree lest he take the Megillah in his hand and carry it four cubits in the public domain. And if it were delayed until Sunday, that would make it the sixteenth, whereas Scripture states: "and (the fifteenth) not to be passed by." And even though those in the walled cities read it on the fourteenth when the fifteenth falls out on a Sabbath, still, they read "Vayavo Amalek" only on Shabbath, which is the fifteenth, they read "Pakadeti" as the haftarah, and they review the halachoth of Purim that entire Shabbath. As to the Purim feast — some say they have it on the fourteenth, when they read the Megillah; and others, that they delay it until after Shabbath. And so it would appear from the Yerushalmi — that a Purim feast which falls on Shabbath is delayed and not advanced. But all agree that it is not made on Shabbath.]} If it falls on Sunday, the villages advance it to the day of assembly, {[which is the eleventh]}, the large cities read it on that day, and the walled cities, on the next day. {[The sages allowed the villages to advance it to the day of assembly only when Israel are on their land and messengers of beth-din go out to inform them when beth-din sanctified the New Moon and when Pesach falls. But nowadays, when the people count thirty days from the Megillah reading until Pesach — if the villagers advanced their reading, they would observe Pesach thirty days after the reading and would be eating chametz the last days of Pesach, for which reason it is read only in its time.]}}

\hebeng{איזו היא עיר גדולה כל שיש בה עשרה בטלנים. פחות מכאן הרי זה כפר. באלו אמרו מקדימין ולא מאחרין. אבל זמן עצי כהנים. ותשעה באב. חגיגה. והקהל. מאחרין ולא מקדימין. אף על פי שאמרו מקדימין ולא מאחרין. מותרין בהספד ובתעניות ומתנות לאביונים. אמר רבי יהודה אימתי מקום שנכנסין בשני ובחמישי. אבל מקום שאין נכנסין לא בשני ולא בחמישי. אין קורין אותה אלא בזמנה: 
}{	Which is a large city? Wherever there are "ten idlers" {[in the house of prayer, who are idle from work, being supported by the congregation in order always to be found there at the time of prayer.]} If there are fewer than that, it is a village. With these {[i.e., with the time of the Megillah reading]}, they said that it is to be advanced, but not delayed {[if it falls on a Sabbath]}; but with the times of the wood (offerings) of the Cohanim and the people, {[where certain families bring wood to the Temple for the (altar) wood pile on fixed days, every year (see Ta'anith 4:5), and they sacrifice a "wood-offering," gift burnt-offerings — if it (their appointed day) falls on a Sabbath, it is delayed to Sunday]}, and the ninth of Av {[(the same applies to all fasts which fall on a Sabbath)]}, and chagigah (the festival offering) {[If yom tov falls on a Sabbath, the festival peace-offerings are pushed off to the next day, for they can be made up all seven days]}, and hakhel {[viz. (Deuteronomy 31:12): "Assemble (hakhel) the people, etc.", where the king would read in the book of Deuteronomy, and all the people were obligated to bring their little children, viz. (Ibid.); "the men, the women, and the little children," which is impossible on Shabbath]}, (if they fall on a Sabbath,) they are delayed but not advanced, {[the time of their obligation not yet having arrived. (As to the ninth of Av, "Calamity is not advanced")]}. Even though they said (in respect to the time of the Megillah reading): "It is advanced, but not delayed," it is permitted (on a day that it is advanced) to eulogize, and to fast, and to fulfill the (Purim) obligation of matanoth la'evyonim (giving gifts to the poor). R. Yehudah said: When is this so (that the villages may advance the reading to the day of assembly)? Where they assemble (in the large cities) on Mondays and Thursdays. But where they assemble neither on Mondays nor on Thursdays, they may read it only in its time.}

\hebeng{קראו את המגילה באדר הראשון. ונתעברה השנה. קורין אותה באדר שני. אין בין אדר הראשון. לאדר השני. אלא קריאת המגילה ומתנות לאביונים: 
}{	If they read the Megillah on the first Adar, and they intercalated the year, they read it on the second Adar. There is no difference between the first Adar and the second Adar but the reading of the Megillah and matanoth la'evyonim alone. {[This is what is meant: There is no difference between the fourteenth and fifteenth of the first Adar and the fourteenth and fifteenth of the second Adar but the reading of the Megillah and matanoth la'evyonim, (which obtain on the second and not on the first). But in respect to eulogy and fasting, they are alike (i.e., they are forbidden on both.)]}}

\hebeng{אין בין יום טוב לשבת. אלא אוכל נפש בלבד. אין בין שבת ליום הכפורים. אלא שזה זדונו בידי אדם. וזה זדונו בכרת: 
}{	There is no difference between yom tov and Shabbath but food (preparation) alone, (being forbidden on Shabbath but permitted on yom tov.) {[This Mishnah is in accordance with Beth Shammai, who say (Beitzah 1:5): "Neither a minor, nor a lulav, nor a Torah scroll may be carried out to the public domain (on yom tov)," for they are not needed for purposes of eating. But this is not the halachah. We rule in accordance with Beth Hillel, who say that since carrying was permitted for eating purposes, it was permitted for other purposes, too. And there are also other things, which are forbidden on Shabbath but permitted on yom tov even though they are not for eating purposes, such as the dropping of fruits through the aperture (see Beitzah 5:1), which is permitted on yom tov, but not on Shabbath. There is no difference between Shabbath and Yom Kippur, but that wilful transgression of the first is punishable by man {[judicial death penalty]}, whereas wilful transgression of the second is punishable by kareth ("cutting-off").}

\hebeng{אין בין המודר הנאה מחבירו למודר ממנו מאכל. אלא דריסת הרגל. וכלים שאין עושין בהן אוכל נפש. אין בין נדרים לנדבות. אלא שהנדרים חייב באחריותן. ונדבות. אינו חייב באחריותן: 
}{	There is no difference between bevowing benefit from one's neighbor and bevowing food from him, but "the treading of the foot" {[Bevowing benefit is more stringent than bevowing food only in that one who bevows benefit may not enter the other's property, whereas one who bevows food may]}, and, (another difference), vessels which are not used for food preparation, {[it being permitted to lend them to one who bevows food (but not to one who bevows benefit). And this, only in a place where such vessels are not hired out; but in a place where they are hired out, it is forbidden (even to one who bevows food). For (he is forbidden) any benefit resulting in food. For if this one had not benefited him (by lending him the vessel), he would be lacking a perutah's worth of food benefit. For with that perutah (saved), he can buy food.]} There is no difference between vows and gifts, but that one must make good for vows, but he need not make good for gifts. {[("vows":) If one says: "I take it upon me (i.e., I vow) to bring a burnt-offering," after which he separated it (the animal) and it were lost, he must bring a different one. ("gifts":) If one says: "this animal is (given) as a burnt-offering," and it were lost, he need not bring a different one, for he did not take it upon himself. But as far as liability for delay is concerned, they are both the same, it being written (Deuteronomy 23:24): "…what you have vowed to the L-rd your G-d, the gift that you spoke with your mouth, etc."}

\hebeng{אין בין זב הרואה שתי ראיות. לרואה שלש. אלא קרבן. אין בין מצורע מוסגר למצורע מוחלט. אלא פריעה ופרימה. אין בין טהור מתוך הסגר. לטהור מתוך החלט. אלא תגלחת וצפרים: 
}{	There is no difference between a zav (one with a genital discharge) who has two sightings {[whether on one day or on two consecutive days]} and one who has three sightings {[whether on one day or on three consecutive days or two on one day and one the next]}, but the offering. {[A zav who has two sightings does not require an offering; but as far as rendering what he lies on and what he sits on av hatumah (proto-uncleanliness), even if he does not touch them, and (as far as) the counting of seven days from the cessation of his discharge, seven clean days being required before he can immerse (for purification), they are both alike.]} There is no difference between a quarantined leper (musgar), viz. (Leviticus 13:5): "Then the priest shall quarantine him for seven days," and a confirmed leper (muchlat), {[whom the priest confirms as unclean]}, but letting the hair grow long and rending of garments, (required by a confirmed leper, but not by a quarantined one), {[but as far as being sent away and being unclean, they are both alike.]} There is no difference between one cleansed after quarantine and one cleansed after confirmation (as a leper) but (the mitzvah of) shaving and (that of) the birds, {[it being written in this regard (Leviticus 14:3): "…and, behold, if the plague-spot of leprosy is healed from the leper" — to exclude a quarantined leper, whose leprosy does not hinge upon healing, but upon days (of quarantine). For even if it were healed, he must remain quarantined for seven days. But as far as cleansing in the mikveh, they are both alike. For in respect to cleansing after quarantine it is also written (Ibid. 13:6): "And he shall wash his clothes, and he shall cleanse himself." And even though there are the guilt-offerings and the log of oil, (required by the confirmed leper, but not by the quarantined one), our Mishnah speaks of the day of his cleansing and his healing, and not of offerings, which obtain on the eighth day.]}}

\hebeng{אין בין ספרים לתפילין ומזוזות. אלא שהספרים נכתבין בכל לשון. ותפילין ומזוזות אינן נכתבות אלא אשורית. רבן שמעון בן גמליאל אומר. אף בספרים לא התירו שיכתבו אלא יונית: 
}{	There is no difference between scrolls (of Scripture) and tefillin and mezuzoth, but that scrolls may be written in all languages {[i.e., in the script and tongue of all languages]}, but tefillin and mezuzoth must be written only ashurith (the holy script and tongue). R. Shimon b. Gamliel says: Even with scrolls, they permitted them to be written (aside from ashurith) only in Greek. {[The reason for permitting Greek above all other languages is (Genesis 9:27): "G-d will beautify Yefeth and it will dwell in the tents of Shem": The most beautiful of Yefeth; that is, the most beautiful tongue of all the sons of Yefeth, will dwell in the tents of Shem (Israel.) And there is no tongue among all the sons of Yefeth more beautiful than the Greek tongue. And the halachah is in accordance with R. Shimon b. Gamliel. However, that Greek tongue has already been lost and has become corrupted, for which reason, nowadays, we write scrolls only in the holy script and tongue.]}}

\hebeng{אין בין כהן משוח בשמן המשחה. למרובה בגדים. אלא פר הבא על כל המצות. אין בין כהן משמש לכהן שעבר. אלא פר יום הכפורים ועשירית האיפה: 
}{	There is no difference between the (high) priest anointed with the anointing oil and the "many-clothed" (high-priest), but the bullock which is brought for "all the mitzvoth" (Leviticus 4:2). {[(the "many-clothed"): These are the priests who officiated in the second Temple and also in the first Temple from Yoshiyahu on. The cruse of anointing oil was secreted in his days, so that high-priests were invested with the donning of (additional) vestments alone. If the anointed high-priest (but not the "many-clothed") rules to be permitted something for which wilful transgression is punishable by kareth, and he acts upon his ruling, he brings a bullock (as an offering), viz. (Ibid. 3): "And if the anointed priest, etc."]}. There is no difference between an officiating priest ("cohein meshamesh") and a pre-empted priest ("cohein she'avar") but the bullock of Yom Kippur and the tenth of the ephah. {[("cohein hameshamesh":) If the high-priest sustained a blemish, and another were appointed in his stead, and his blemish disappeared, and he returned to his service, and his "stand-in" stepped down — the first is called "meshamesh," and the second, "avar." ("but the bullock of Yom Kippur":) it being impossible to offer two. And, likewise, with the tenth of the ephah, the daily cakes of the high-priest, it being impossible to offer two. But in all other respects, they are alike. If he (the "avar") comes to offer the incense or to perform any service, he wears eight vestments. And both are commanded (to wed only) a virgin, and are exhorted against marrying a widow, and officiate at sacrifices even in mourning.]}}

\hebeng{אין בין במה גדולה לבמה קטנה. אלא פסחים. זה הכלל כל שהוא נידר ונידב. קרב בבמה. וכל שאינו לא נידר ולא נידב. אינו קרב בבמה: 
}{	There is no difference between a large bamah (sacrificial mound) and a small bamah but pesachim (Pesach offerings). {[This, when the bamoth were permitted. A large bamah is a congregational sacrificial mound, as that of Nov and Giveon. A small bamah is one that each individual makes for himself. Pesachim and all (offerings) like pesachim, i.e., obligatory offerings having a set time, such as temidim and mussafim, (are offered on a large bamah, but not on a small one); but obligatory offerings having no set time, such as the bullock of forgetfulness of the congregation and the goats for (unwitting) idolatry were not offered even on a large bamah.]} This is the rule: Whatever is vowed and donated may be offered on a (small) bamah; whatever is not vowed and donated may not be offered on a bamah.}

\hebeng{אין בין שילה לירושלם. אלא שבשילה אוכלים קדשים קלים. ומעשר שני. בכל הרואה. ובירושלים לפנים מן החומה. וכאן וכאן קדשי קדשים נאכלים לפנים מן הקלעים. קדושת שילה יש אחריה היתר. וקדושת ירושלים אין אחריה היתר: 
}{	There is no difference between (the sanctuary of) Shiloh and Jerusalem, but that in Shiloh lesser-order offerings and ma'aser sheni are eaten wherever {[Shiloh]} can be seen, and, in Jerusalem, (only) within the wall. And, in both places, holy of holies are eaten (only) within the enclosure (of the sanctuary). The sanctity of Shiloh permits {[of bamoth]} after it {[i.e., after its destruction.]} The sanctity of Jerusalem does not permit {[of bamoth]} after it.}

\clearpage
\blockcomment{ברטנורא על משנה מגילה}{מגילה נקראת באחד עשר בשנים עשר. פעמים בזה ופעמים בזה כדמפרש ואזיל:\\חל להיות. י״ד. בע״ש. עיירות ומוקפין קורין בו ביום, שאין קריאת מגילה בשבת, גזירה שמא יטלנה בידו ויעבירנה ד׳ אמות ברשות הרבים, ואם יאחרנה עד אחד בשבת הוה ליה י״ו, ואמר קרא ולא יעבור. ואע״פ שבני הכרכים קורין המגילה בי״ד כשחל ט״ו להיות בשבת, מ״מ אין קורין ויבא עמלק אלא בשבת שהוא יום ט״ו, ומפטירים פקדתי, ושואלין ודורשין בהלכות פורים כל אותה שבת. וסעודת פורים אית דאמרי דעבדי לה ביום י״ד שבו קורין את המגילה, ואית דאמרי דמאחרין אותה לאחר השבת. והכי משמע בירושלמי דסעודת פורים שחל להיות בשבת מאחרין ולא מקדימין. ולכ״ע אין עושים אותה בשבת:\\עשרה בטלנין. של בית הכנסת, שבטלים ממלאכתן ונזונים משל צבור כדי להיות מצויין תמיד בשעת התפילה בבהכ״נ:\\אין בין אדר ראשון לאדר שני וכו׳ הכי קאמר אין בין ארבעה עשר וחמשה עשר של אדר ראשון לארבעה עשר וחמשה עשר של אדר שני, אלא מקרא מגילה ומתנות לאביונים, הא לענין הספד ותענית זה וזה שוין: \\אין בין יו״ט לשבת אלא אוכל נפש בלבד. מתניתין ב״ש היא [ביצה יב.] דאמרי אין מוציאין את הקטן ולא את הלולב ולא את הס״ת לרשות הרבים כיון שאין בהם צורך אוכל נפש. ואין כן הלכה אלא כדברי בית הלל דאמרי מתוך שהותרה הוצאה לצורך אכילה הותרה נמי שלא לצורך אכילה. ואיכא נמי מילי אחרינא שאסורים בשבת ומותרים ביו״ט אע״פ שאינם צורך אוכל נפש, כגון משילין פירות דרך ארובה ביו״ט, אבל לא בשבת:\\אין בין המודר הנאה. אין מודר הנאה חמור ממודר מאכל אלא דריסת הרגל, שמודר הנאה אסור לו ליכנס בתוך שלו ומודר מאכל מותר:\\שתי ראיות. בין ביום א׳ בין בשני ימים רצופים. וכן ג׳ ראיות בין ביום א׳ בין בג׳ ימים רצופים, או ב׳ ביום א׳ וא׳ למחר:\\נכתבין בכל לשון. בכתב של כל אומה ובלשון של כל אומה:\\למרובה בבגדים. כהנים ששמשו בבית שני, ואף בבית ראשון מן יאשיהו ואילך שנגנזה צלוחית של שמן המשחה בימיו, ולא היו כהנים גדולים אלא בלבישת הבגדים בלבד:\\אין בין במה גדולה. בשעת היתר הבמות מיירי. במה גדולה היא במת צבור שהיתה בנוב וגבעון:\\בכל הרואה. בכל מקום שיכול לראות משם את שילה:\\\n}\clearpage %endcomment
\newchap{פרק ב}
\hebeng{הקורא את המגילה למפרע לא יצא. קראה על פה. קראה תרגום. בכל לשון. לא יצא. אבל קורין אותה ללועזות בלעז. והלועז ששמע אשורית יצא: }{	If one reads the Megillah in inverted order he does not fulfill his obligation, it being written (Esther 9:28): "And these days are commemorated and celebrated." Just as celebration cannot be inverted (it being impossible for the fifteenth to precede the fourteenth), so commemoration (the reading of the Megillah) cannot be inverted.]} If he reads it by heart, or in Targum, or in any language, he does not fulfill his obligation. {[("by heart":), it being written here "commemorated," and, elsewhere (in respect to the eradication of Amalek - Exodus 17:14): "Write this as a commemoration in a book." ("or in Targum, etc.":) This is what is meant: If a Hebrew reads it in Targum, and he does not understand it; or if he reads it in any other language that he does not understand, he does not fulfill his obligation.]} But it may be read to those speaking a foreign tongue in their tongue, {[so long as it be written in that tongue, so that he not read it by heart.]} And if one speaking a foreign tongue hears it in Ashurith, he fulfils his obligation. {[Greek is like Ashurith relative to this halachah. It is just that the original Greek has been lost and been forgotten as we wrote above (1:8)]}.}

\hebeng{קראה סירוגין ומתנמנם יצא. היה כותבה. דורשה ומגיהה. אם כיון לבו יצא. ואם לאו לא יצא. היתה כתובה בסם. ובסיקרא. ובקומוס ובקנקנתום. על הנייר. ועל הדפתרא. לא יצא. עד שתהא כתובה אשורית על הספר ובדיו: }{	If he reads it desultorily {[i.e., If he reads a little and pauses, and then reads a little more and pauses again — even if the pause is longer than that required to complete the whole]}, (or) if he "slumbers" (while reading), he fulfills the obligation. If he copied it, (or) expounded it, or proof-read it, {[("If he copied it":) as when it was all written and lying before him, and he copied it (for he does not fulfill his obligation unless he reads from a Megillah which is entirely written)]} — if he had intent {[to fulfill his obligation with such reading]}, he fulfills his obligation; if not, he does not fulfill it. If it were written with sam {[an herb]}, sikra {[a stone which dyes red]}, komos {[a type of resin]}, vitriol, on paper, or on diftera {[unfinished hide — processed with salt and flour but not with gall nut]}, he does not fulfill his obligation; but it must be written Ashurith, on a scroll, and in ink.}

\hebeng{בן עיר שהלך לכרך. ובן כרך שהלך לעיר. אם עתיד לחזור למקומו. קורא כמקומו. ואם לאו קורא עמהן. ומהיכן קורא אדם את המגילה ויוצא בה ידי חובתו. רבי מאיר אומר כולה. רבי יהודה אומר מאיש יהודי. רבי יוסי אומר מאחר הדברים האלה: }{	If an unwalled city man {[whose time (for reading the Megillah) is the fourteenth]}, went to a walled city, {[whose time is the fifteenth]}, or a walled city man went to an unwalled city — If he intends to return to his place, he reads as (in) his place; if not, he reads with them. {[If he is a walled city man who went to an unwalled city, and he intends to leave the city the night of the fourteenth before dawn — even though he spends the night in the city, since he does not intend to be there in the daytime, he is not even regarded as "unwalled for a day," for which reason he reads in his place on the fifteenth. But if he does not intend to leave there at night, he is "unwalled" for that day. Even though he intends to return the next day or some other day, he is called "unwalled" and reads with them. The same applies to an unwalled city man who went to a walled city. If he intends to return on the night of the fifteenth, he is not "walled for a day," and he reads on the fourteenth, even though he is in the walled city. But if he does not intend to return the night of the fifteenth, he does not read on the fourteenth, but waits and reads with them. This Mishnah is explained thus in the gemara.]} And from where must one read (the Megillah) to fulfill his obligation? R. Meir says: (He must read) the whole thing. R. Yehudah says: From (Esther 2:5): "Ish Yehudi." R. Yossi says: From (Ibid. 3:1): "After these things." {[The halachah is in accordance with R. Meir.]}}

\hebeng{הכל כשרין לקרות את המגילה חוץ מחרש שוטה וקטן. רבי יהודה מכשיר בקטן. אין קורין את המגילה. ולא מלין. ולא טובלין. ולא מזין. וכן שומרת יום כנגד יום לא תטבול עד שתנץ החמה וכולן שעשו משעלה עמוד השחר. כשר: }{	All {[including women]} are fit to read the Megillah, except a deaf-mute {[(This Mishnah is in accordance with R. Yossi, who says that if one reads and does not "make it heard" to his ears, he has not fulfilled his obligation)]}, an imbecile, and a minor. R. Yehudah rules it to be fit with a minor. {[The halachah is not in accordance with R. Yehudah.]} The Megillah is not read, circumcision is not performed, (ritual) immersion is not performed, sprinkling is not performed, and also a woman who observes "day against day" does not immerse until sunrise. And all of them, if they did so at dawn, it is kasher. {[("The Megillah is not read":) For one must read the Megillah at night and repeat it in the daytime. And the reading of the daytime is only after sunrise, viz. (Esther 9:28): "And these days are commemorated and celebrated." ("circumcision is not performed":), viz. (Leviticus 12:3): "And on the eighth day he shall circumcise." ("immersion and sprinkling are not performed":) it being written in respect to sprinkling (Numbers 19:19): "And the clean one shall sprinkle on the unclean one on the third day and on the seventh day," and immersion is likened to sprinkling. It is only when he immerses on the seventh day that he must immerse only in the daytime, and we do not say that he may immerse when it gets dark on the night of the seventh, even though the night is the beginning of the day. But after the seventh day has passed, it is permitted to immerse at night. ("a woman who observes 'day against day'":) during the eleven days between one niddah state and the next. If she sees blood on one of those days, she observes the next day (in cleanliness) and immerses that day itself at sunrise. ("if they did so at dawn, it is kasher:") For when the day dawns, it is called "day," viz. (Nechemiah 4:15): "And we did the work … from the dawn until the stars appeared," followed by (Ibid. 16): "…and the night for us was guarding, and the day, work." They said "until sunrise" only to insure that it was not night, for not all are expert in discriminating dawn.]}}

\hebeng{כל היום כשר לקריאת המגילה. ולקריאת ההלל. ולתקיעת שופר. ולנטילת לולב. ולתפלת המוספין. ולמוספין. ולוידוי הפרים. ולוידוי המעשר. ולוידוי יום הכפורים. לסמיכה. לשחיטה. לתנופה. להגשה. לקמיצה. ולהקטרה. למליקה. ולקבלה. ולהזייה. ולהשקיית סוטה. ולעריפת העגלה. ולטהרת המצורע. }{	The entire day (i.e., the daytime) is kasher for: the reading of the Megillah, the reciting of Hallel, the blowing of the shofar, the taking of the lulav, the reciting of the mussaf prayer, the mussaf offering, the confession over the bullocks {[the bullock of the anointed high-priest and the bullock of forgetfulness of the congregation, over which confession is made for the sins for which they are brought, the tithe-confession {[(Deuteronomy 26:13): "I have removed the holy things from the house, etc."]}, the confession of Yom Kippur, placing of the hands (s'michah) {[(Leviticus 1:4): "And he shall place his hand upon the head of the burnt-offering," slaughtering, lifting (of the omer and of parts of the peace-offering), presentation {[First he presents the meal-offering at the southwest corner of the altar; then he takes the fistful]}, burning {[the fistful, which corresponds in the meal-offering to the sprinkling of the blood in sacrifices, and which is kasher only in the daytime, as opposed to the burning of the fats and the limbs, which is kasher the entire night (2:6)]}, melikah ("pinching" a bird's head), receiving {[of the blood in the sprinkling bowl]}, sprinkling (hazayah) {[the sprinkling (on the ark cover) of the blood of bullocks which are burned and of all the inner sin-offerings; and the sprinkling of the blood on the altar is also called "hazayah."]}, the administering of the sotah's draught, the breaking of the neck of the red heifer, and the cleansing of the leper.}

\hebeng{כל הלילה כשר לקצירת העומר. ולהקטר חלבים ואיברים. זה הכלל. דבר שמצותו ביום. כשר כל היום. דבר שמצותו בלילה. כשר כל הלילה: }{	The entire night is kasher for the harvesting of the omer and the burning of the fats and the limbs {[left over from the afternoon tamid, viz. (Leviticus 6:2): "It is the burnt-offering upon its firewood on the altar all the night."]} This is the rule: Something whose mitzvah is in the daytime is kasher the entire day, {[to include the arranging of the two containers of frankincense placed on the show bread]}, and something whose mitzvah is at night is kasher the entire night, {[to include the eating of the Pesach offering, which is kasher the entire night, the sages having said "until midnight" only to keep one far from transgression.]}}

\clearpage
\blockcomment{ברטנורא על משנה מגילה}{הקורא את המגלה למפרע לא יצא. דכתיב (אסתר ט׳:כ״ח) והימים האלה נזכרים ונעשים, מה עשיית הימים א״א למפרע, דא״א שיהא חמשה עשר קודם ארבעה עשר, אך זכירה שהיא קריאת המגילה, למפרע לא:\\סירוגין. שקרא מעט ושהא, וחזר וקרא מעט ושהא, אפילו שהא יותר מכדי לגמור את כולה, יצא:\\בן עיר. שזמנו בי״ד:\\הכל כשרים לקרות את המגילה. הכל לאתויי נשים:\\ולוידוי הפרים. פר כהן משיח, ופר העלם דבר של צבור, שמתודים עליהם חטאם שהביאום עליו:\\ולהקטר חלבים ואברים. מותרי תמיד של בין הערבים, דכתיב בהו (ויקרא ו׳:ב׳) היא העולה על מוקדה על המזבח כל הלילה:\\\n}\clearpage %endcomment
\newchap{פרק ג}
\hebeng{בני העיר שמכרו רחובה של עיר לוקחין בדמיו בית הכנסת. בית הכנסת לוקחין תיבה. תיבה לוקחין מטפחות. מטפחות לוקחין ספרים. ספרים לוקחים תורה. אבל אם מכרו תורה לא יקחו ספרים. ספרים לא יקחו מטפחות. מטפחות לא יקחו תיבה. תיבה לא יקחו בית הכנסת. בית הכנסת לא יקחו את הרחוב. וכן במותריהן. אין מוכרין את של רבים ליחיד. מפני שמורידין אותו מקדושתו. דברי רבי יהודה. אמרו לו אם כן אף לא מעיר גדולה לעיר קטנה: }{	Men of a city who sold the open place of a city may buy a house of prayer with the proceeds. {[The open place of a city has sanctity, for they (the city dwellers) pray there on fast days. The sages differ with this anonymous Mishnah, saying that sanctity does not inhere in the open place of a city, since they pray there only adventitiously. The halachah is in accordance with the sages.]} (If they sold) a house of prayer, they may buy an ark. {[It is only a village house of prayer that may be sold; but a house of prayer in a large city, from which they come from all over to pray belongs to the populace and may not be sold.]} (If they sold) an ark, they may buy coverings (for the Torah scrolls). (If they sold) coverings, they may buy books {[Prophets and Writings]}. (If they sold) books, they may buy a Torah scroll. But if they sold a Torah scroll, they may not buy books. {[For "we raise in sanctity, and do not lower."]} (If they sold) books, they may not buy coverings. (If they sold) coverings, they may not buy an ark. (If they sold) an ark, they may not buy a house of prayer. (If they sold) a house of prayer, they may not buy the open place. And the same holds with what is left over. {[If they sold books and bought a Torah scroll with some of the money, they may not buy something of lesser sanctity with what is left over. And all of this applies only when the sale was not made by seven city dignitaries in the presence of the people of the city, but if it was, even to buy beer for drinking, it is permitted. And this, only in a village (as opposed to a large city) as stated above.]} It is not permitted to sell what is owned by many to an individual, this lowering its sanctity. These are the words of R. Yehudah. They said to him: If so, it should not be permitted to sell from a large city to a small one, {[but it is!]}}

\hebeng{אין מוכרין בית הכנסת אלא על תנאי. שאם ירצו יחזירוהו. דברי רבי מאיר. וחכמים אומרים מוכרים אותו ממכר עולם. חוץ מארבעה דברים למרחץ. ולבורסקי. ולטבילה. ולבית המים. רבי יהודה אומר מוכרין אותו לשם חצר והלוקח מה שירצה יעשה: }{	A house of prayer may be sold only on condition that if they (the sellers) desire, it will be returned. These are the words of R. Meir. {[Even from the many to the many it may not be sold unconditionally, this being demeaning, as if to say: "It is nothing special to us." The halachah is not in accordance with R. Meir.]} The sages say: It may be sold forever (i.e., unconditionally) {[even to an individual, for any purpose]}, except for four things: a bathhouse, a tannery, a mikveh, a "watering" house {[i.e., for washing (clothing); or, for passing water.]} R. Yehudah says: It may be sold as a courtyard, and the buyer can do what he wants with it. {[The halachah is not in accordance with R. Yehudah.]}}

\hebeng{ועוד אמר רבי יהודה בית הכנסת שחרב אין מספידין בתוכו. ואין מפשילין בתוכו חבלים. ואין פורשין לתוכו מצודות. ואין שוטחין על גגו פירות. ואין עושין אותו קפנדריא שנאמר (ויקרא כו, לא) והשימותי את מקדשיכם. קדושתן אף כשהן שוממין. עלו בו עשבים לא יתלוש. מפני עגמת נפש: }{	R. Yehudah said further: In a ruined synagogue, no eulogies are made, ropes are not twined {[The same holds for all labors, but the twining of ropes requires a large space, and the space in a house of prayer serves this purpose]}, nets are not spread in it, fruits are not spread on its roof, and it is not used as a short-cut (kapandarya) {["kapandarya," acronymic for "Ademakifna dari, a'ol beha," i.e., "Instead of circling rows" of houses, I will take a short-cut through here.]}, it being written (Leviticus 26:31): "And I will make desolate your sanctuaries" — Though desolate, they retain their sanctity. If grass cropped up in it, it may not be torn out, so that they grieve, {[remembering its former days and resolving to rebuild it if possible or (so that they grieve and) pray for its restoration. Therefore, only tearing out the grass and feeding it to animals or discarding it entirely is forbidden; but it is permitted to tear it out and leave it in its place, this sufficing for arousing grief.]}}

\hebeng{ראש חדש אדר שחל להיות בשבת. קורין בפרשת שקלים. חל להיות בתוך השבת. מקדימין לשעבר. ומפסיקין לשבת אחרת. בשניה זכור. בשלישית פרה אדומה. ברביעית החדש הזה לכם. בחמישית חוזרין לכסדרן. לכל מפסיקין. בראשי חדשים. בחנוכה. ובפורים. בתעניות ובמעמדות. וביום הכפורים: }{	If Rosh Chodesh Adar falls on Shabbath, we read (maftir) in the section of shekalim {[(Exodus 30:11-16) to apprise them to bring their shekalim in Adar so that offerings may be brought from the new donations on the first of Nissan.]} If it falls out during the week, it is read on the preceding (Shabbath), {[even if Rosh Chodesh Adar falls out on Sabbath eve.]} And we break off until the next Shabbath. {[i.e., we do not read the second section ("Zachor") so that it be read on the Shabbath preceding Purim, to link the erasing of Amalek with the erasing of Haman.]} On the second (Sabbath), "Zachor" (Deuteronomy 25:17-19). On the third, parah adumah (the red heifer) (Numbers 19) {[to exhort Israel to cleanse themselves in order to bring their Pesach offerings in cleanliness. Which is the third Sabbath? The Sabbath after Purim. And when Rosh Chodesh Nissan falls out on Shabbath, the third Shabbath is the one before Rosh Chodesh Nissan. (It is read) in order to link the exhortation to cleansing oneself from dead-body uncleanliness to the Pesach offerings.]} On the fourth, (Exodus 12:1-20): "This month is unto you" {[containing the section on Pesach]}. On the fifth, the usual order (of haftaroth) is reverted to. {[For until then, the haftarah is of the nature of the four sections, viz. "shekalim" — (I Kings 12:1); "Zachor" — (I Samuel 15:2); "Parah" — (Ezekiel 36:25); "This month" — (Ezekiel 45:18). And from that time on, the haftarah, again, is of the nature of the section of the day.]} For all (of the following) we break off {[i.e., We do not read a haftarah which is of the nature of the section, but one which is of the nature of the day]}: Rosh Chodesh, Channukah, Purim, fasts and ma'amadoth, and Yom Kippur.}

\hebeng{בפסח קורין בפרשת מועדות של תורת כהנים. בעצרת שבעה שבועות. בראש השנה. בחדש השביעי באחד לחדש. ביום הכפורים אחרי מות. ביום טוב הראשון של חג. קורין בפרשת מועדות שבתורת כהנים. ובשאר כל ימות החג. בקרבנות החג: }{	On Pesach we read in the section of the festivals in Leviticus {[(22:26) This, on the first day. Nowadays, the custom is to read (Exodus 12:21). And the haftarah is (Joshua 5:2). On the second day (Leviticus 22:26); the haftarah (II Kings 23:1). On the third day (Exodus 13:2). On the fourth day (Exodus 22:24). On the fifth day (Exodus 34:1). On the sixth day (Numbers 9:2). On the seventh day (Exodus 14:17); the haftarah (II Samuel 22:1). On the eighth day (the last day of the festival in the exile) (Deuteronomy 15:9); the haftarah (Isaiah 10:32).]} On Shavuoth, "Shivah shavuoth" (Deuteronomy 16:9). On Rosh Hashanah, "In the seventh month, on the first day of the month" (Leviticus 23:23). On Yom Kippur, "Acharei Moth" (Leviticus 16:1). On the first day of Succoth we read in the section of the festivals in Leviticus. And the rest of the days of the festival, (we read of) the offerings of the festival. {[On Shavuoth, on the first day of the festival (Exodus 19:1); the haftarah (Ezekiel 1). On the second day (Deuteronomy 16:9); the haftarah (Habakkuk 2:20). On Rosh Hashanah (Genesis 21:1): "And the L-rd remembered Sarah…" (for on Rosh Hashanah Sarah was remembered.") And the haftarah (I Samuel 1:1), concerning Channah; for she, too, was remembered on Rosh Hashanah. On the second day, (Genesis 22:1), on the binding of Isaac; the haftarah (Jeremiah 31:1). On Yom Kippur, shacharith, (Leviticus 16:1); the haftarah (Isaiah 57:14). Minchah: (Leviticus 18:1); the haftarah (Yonah 1:1). On Succoth, both festival days (Exodus 12:21); the haftarah: on the first day (Zechariah 14:1); on the second (I Kings 8:2). And all the rest of the days of the festival, we read of the offerings of the festival. How so? On the third day, the first day of Chol Hamoed, the Cohein reads (Numbers 29:17): "And on the second day." The Levite reads: "And on the third day." The Israelite reads: "And on the fourth day. The fourth goes back and reads: "And on the second day," "And on the third day." On the fourth day, the Cohein reads: "And on the third day." The Levite reads: "And on the fourth day." The Israelite reads: "And on the fifth day." And the fourth goes back and reads: "And on the third day and on the fourth day." And so with all. On the last day of the festival (i.e., Shmini Atzereth) (Deuteronomy 15:19); the haftarah (I Kings 8:54). And on the next day (Simchath Torah) (Deuteronomy 33:1); the haftarah (Joshua 1:1). And on a Sabbath that falls out on Chol Hamoed, both on Pesach and on Succoth, we read (Exodus 33:12); and, the haftarah; on Pesach the vision of the dry bones (Ezekiel 37:1); and, on Succoth (Ezekiel 38:18): "On the day that Gog comes, etc." For we have a tradition that the resurrection will occur on Pesach and the war of Gog and Magog, on Succoth.]}}

\hebeng{בחנוכה בנשיאים. בפורים ויבא עמלק. בראשי חדשים. ובראשי חדשיכם. במעמדות. במעשה בראשית. בתעניות. ברכות וקללות. אין מפסיקין בקללות. אלא אחד קורא את כולן. בשני ובחמישי ובשבת במנחה קורין כסדרן. ואין עולין להם מן החשבון. שנאמר (ויקרא כג, מד) וידבר משה את מועדי ה׳ אל בני ישראל. מצותן שיהו קורין כל אחד ואחד בזמנו: }{	On Channukah we read in the Nesi'im (the chiefs of the tribes) (Numbers 7). On Purim: "And Amalek came" (Exodus 17:8). On Rosh Chodesh: "And in the beginnings of your months" (Numbers 28:11). On the (convening of) the ma'amadoth (see Ta'anith 4:2) the reading is in the (account of the) creation, {[heaven and earth "standing" on the offerings. The order of the readings is given in Ta'anith 4:3.]} On fast days, we read in the blessings and the curses. {["If in My statutes, etc." (Leviticus 26:3) to impress upon them that troubles come to the world as a result of sin, so that they repent to escape them.]} No break is made in (the reading of) the curses, but one (reader) reads all of them. On Monday, Thursday, and the minchah of Shabbath we read in the sidrah {[of the week]}, and it is not "credited" to the (full) amount {[i.e., when Shabbath arrives, they read again what they read on those days]} — as it is written {[This refers to the entire Mishnah, the source of the mitzvah for reading about the festival on the day of the festival]} (Leviticus 23:44): "And Moses declared the appointed times of the L-rd to the children of Israel" — It is a mitzvah to read of each in its (appointed) time.}

\clearpage
\blockcomment{ברטנורא על משנה מגילה}{בני העיר. רחובה של עיר. יש בה קדושה, שמתפללין בה בתעניות. וחכמים פליגי על סתם מתניתין ואמרי רחובה של עיר אין בה משום קדושה, הואיל ואין מתפללין בה אלא באקראי בעלמא. והלכה כחכמים:\\אלא על תנאי. ואפילו משל רבים לרבים אסור למכור מכירה חלוטה, דדרך בזיון הוא, כלומר אינו בעינינו כלום. ואין הלכה כר״מ:\\ואין מפשילין. גודלים ופותלים. וה״ה לכל שאר מלאכות, אלא שהפשלת חבלים צריך מקום רחב ידים, ובהכ״נ בית גדול הוא וראוי לכך והוי אורחיה:\\קורין פרשת שקלים. כי תשא, להודיע שיביאו שקליהם באדר, כדי שיקריבו בא׳ בניסן מתרומה חדשה:\\בפרשת מועדות שבתורת כהנים. שור או כשב או עז. וביומא קמא מיירי. והאידנא נהוג עלמא שקורין ביום ראשון משכו וקחו לכם, ומפטירין בפסח גלגל. בשני שור או כשב, ומפטירין בפסח יאשיהו. בשלישי קדש לי כל בכור. ברביעי אם כסף תלוה. בחמישי פסל לך. בששי ויעשו בני ישראל את הפסח במועדו. בשביעי שירת הים, ומפטירין וידבר דוד. בח׳ שהוא יו״ט אחרון של גליות קורין כל הבכור, ומפטירין עוד היום בנוב לעמוד. בעצרת, ביו״ט ראשון, בחודש השלישי, ומפטירין במרכבה של יחזקאל. ביו״ט שני של גליות קורין כל הבכור, ומפטירין בחבקוק. בראש השנה וה׳ פקד את שרה, דבר״ה נפקדה שרה, ומפטירין בחנה שגם היא נפקדה בר״ה. ביו״ט שני בעקידה, ומפטירין הבן יקיר לי אפרים. ביוה״כ שחרית קורין באחרי מות, ומפטירין כה אמר רם ונשא. במנחה קורין בעריות, ומפטירין ביונה. בשני ימים טובים של חג קורין שור או כשב או עז ומפטירין ביו״ט ראשון הנה יום בא לה׳. וביו״ט שני ויקהלו אל המלך. ושאר כל ימות החג קורין בקרבנות החג, כיצד, יום ג׳ שהוא יום ראשון של חוה״מ, כהן קורא וביום השני, לוי קורא וביום השלישי, ישראל קורא וביום הרביעי, והרביעי חוזר וקורא וביום השני וביום השלישי. וביום הד׳ כהן קורא וביום השלישי, לוי קורא וביום הרביעי, ישראל קורא וביום החמישי, והרביעי חוזר וקורא וביום השלישי וביום הרביעי. וכן כולם. ביו״ט אחרון של חג, כל הבכור, ומפטירין ויהי ככלות שלמה. ולמחר קורין וזאת הברכה, ומפטירין ויהי אחרי מות משה. ושבת שחל להיות בחולו של מועד, בין בפסח בין בסוכות, קורין ראה אתה אומר אלי. ומפטירין, בפסח, העצמות היבשות. ובסוכות ביום בא גוג. שמסורת בידינו דתחיית המתים עתידה להיות בפסח, ומלחמות גוג ומגוג בסוכות: \\במעמדות במעשה בראשית. שבשביל הקרבנות נתקיימו שמים וארץ. וסדר קריאתן מפורש במסכת תענית בפרק אחרון:\\\n}\clearpage %endcomment
\newchap{פרק ד}
\hebeng{הקורא את המגילה. עומד. ויושב. קראה אחד. קראוה שנים יצאו. מקום שנהגו לברך יברך. ושלא לברך לא יברך. בשני ובחמישי ובשבת במנחה קורין שלשה. אין פוחתין ואין מוסיפין עליהן. ואין מפטירין בנביא. הפותח והחותם בתורה. מברך לפניה. ולאחריה: }{	One who reads the Megillah may {[either]} stand {[or]} sit. If one read it or two read it {[together]} they have fulfilled their obligation {[and we do not say that two voices together are not heard as one. For since it (the Megillah) is beloved of them, they concentrate (on hearing it).]} In a place where it is the custom to recite the {[concluding]} blessing, he does so; (where it is the custom) not to recite it, he does not do so. {[But in all places, he must recite three introductory blessings: "al mikra megillah," "she'asah nissim," and "shehecheyanu," both at night and in the daytime, the reading of the day being the essential one, viz. (Esther 9:28): "And these days are commemorated and celebrated." Some hold that since he recites "shehecheyanu" at night, he need not do so in the daytime. And this would stand to reason.]} On Monday, Thursday, and minchah on Shabbath three men read, no less and no more, {[and there is no haftarah reading in Prophets, so that the congregation not be imposed upon, these (Monday and Thursday) being working days. And with minchah on Shabbath, too, (there is an imposition), it being close to dark and it being their custom to learn the entire day. And for this reason, too, there is no haftarah reading)]}. The opener and the concluder in the Torah (reading) recite the opening and the concluding blessing, respectively. {[The first one to read in the Torah recites the opening blessing, and the last, the concluding blessing. And all the others who read in the Torah (between them) recite neither an opening nor a concluding blessing. But nowadays, the custom is for all to bless before and after — a decree, by reason of those who enter (in the middle of the reading), who, not having heard the blessing of the first reader, might come to say that there is no opening blessing for the Torah; and by reason of those who leave (in the middle), who, not having heard the concluding blessing, the first readers not having recited it, might come to say that there is no concluding blessing for the Torah.]}}

\hebeng{בראשי החדשים. ובחולו של מועד. קורין ארבעה. אין פוחתין מהן. ואין מוסיפין עליהן. ואין מפטירין בנביא. הפותח והחותם בתורה מברך. לפניה. ולאחריה. זה הכלל. כל שיש בו מוסף ואינו יום טוב. קורין ארבעה. ביום טוב חמשה. ביום הכפורים ששה. בשבת שבעה. אין פוחתין מהן. אבל מוסיפין עליהן. ומפטירין בנביא. הפותח והחותם בתורה. מברך לפניה ולאחריה: }{	On New Moon and Chol Hamoed four men read, no less and no more. And there is no haftarah reading in Prophets, {[For on New Moon and Chol Hamoed there is, likewise, (the factor of) keeping people from work, essential labor being permitted.]} The opener and the concluder in the Torah (reading) recite the opening and the concluding blessing, respectively. This is the rule: Wherever there is mussaf and no yom tov, there are four (readers); on yom tov, there are five; on Yom Kippur, six; on Shabbath, seven. {[For whatever occasion has more features than its neighbor has more readers. Therefore, on Rosh Chodesh and Chol Hamoed, where there is a mussaf offering, there are four readers; on yom tov, where work is interdicted, there are five; on Yom Kippur, where there is a punishment of kareth (cutting-off), six; on Shabbath, where there is a punishment of skilah (stoning), seven.]} There may be no fewer, but there may be more. And there is a haftarah reading in Prophets. The opener and the concluder in the Torah (reading) recite the opening and the concluding blessing, respectively.}

\hebeng{אין פורסין את שמע. ואין עוברין לפני התיבה. ואין נושאין את כפיהם. ואין קורין בתורה. ואין מפטירין בנביא. ואין עושין מעמד ומושב. ואין אומרים ברכת אבלים ותנחומי אבלים. וברכת חתנים. ואין מזמנין בשם. פחות מעשרה. ובקרקעות תשעה וכהן ואדם כיוצא בהן: }{	(The following are not done with fewer than ten:) The Shema is not "parceled" (porsin) with fewer than ten. {[If ten came to the house of prayer, after the congregation had recited the Shema, one (of them) rises and says "Kaddish," "Barchu," and the first blessing before the Shema. "porsin," from "p'rusah," half a thing, i.e., of the two blessings before the Shema, he says only one.]}, and they (the Cohanim) do not lift their hands {[for the priestly blessing]}, and they do not read the Torah {[(congregational reading)]}, and they do not read the haftarah (in Prophets), and they do not perform "standings and sittings" (over the dead), and they do not recite the mourners' blessing and the mourners' consolations and the grooms' blessing, and they do not say grace with His name (— with fewer than ten.) {[All of these are not done with fewer than ten because it is written (Leviticus 22:32): "And I shall be sanctified in the midst of the children of Israel" — Every matter of sanctity requires at least ten (participants). It is written here: "in the midst of the children of Israel," and, elsewhere (Numbers 16:21): "Separate yourselves from the midst of this congregation." Just as there, ten (there being no "congregation" fewer than ten), here, too, ten. ("and they do not perform 'standings and sittings'":) for the dead. When the dead were taken out to be buried, they would sit seven times in honor of the deceased and say at each interval of eulogy: "Rise, dear ones, rise; sit dear ones, sit." And this is not seemly with fewer than ten. ("the mourners' blessing":) the blessing in the open place (rechavah). They would recite a blessing for the consolers and a blessing for the mourners (Kethuvoth 8b). ("and the mourners' consolations":) They would stand in a row upon returning from the grave and console the mourners. And there is no row fewer than ten. ("and the grooms' blessing":) the seven blessings addressed to the groom. ("and they do not say grace, etc.":) Since "Let us bless our G-d" must be stated, this is not seemly with fewer than ten.]} And with land {[of hekdesh (consecrated to the Temple), if one wishes to redeem it]}, there must be nine and a Cohein, {[i.e., ten, (at least) one of whom is a Cohein; for "Cohein" is written ten times in the section on valuations (Leviticus 27): three (times) in respect to dedications: three in respect to valuations, three in respect to beasts, and three in respect to land.]}; and a man, like it {[i.e., If a man dedicates his worth (to the Temple), he is assessed as a bondsman. And a bondsman is likened to land, viz. (Leviticus 25:46): "And you shall cause them to be inherited, etc." So that just as land requires ten (assessors), one of them a Cohein; so, a man.]}}

\hebeng{הקורא בתורה לא יפחות משלשה פסוקים. לא יקרא למתורגמן יותר מפסוק אחד. ובנביא שלשה. היו שלשתן שלש פרשיות. קורין אחד אחד. מדלגין בנביא. ואין מדלגין בתורה. ועד כמה הוא מדלג. עד כדי שלא יפסוק המתורגמן: }{	The reader in the Torah may not read fewer than three verses. He may not read to the translator more than one verse (at a time), {[so that, translating by heart, he not err.]} And in Prophets, he may read three (at a time) if he wishes, and we are not apprehensive as to his erring, for we do not derive halachah therefrom.]} And if the three (verses in Prophets) were three (distinct) sections {[as in (Isaiah 52:3-5): "For thus said the L-rd: 'Gratis were you sold … For thus said the L-rd: 'To Egypt, My people went down in the beginning … And now, what have I here,' says the L-rd," These are three (distinct) sections in three consecutive verses.]}, they are read one (verse) at a time. We skip in Prophets {[from section to section, and even from one theme to another]}, but we do not skip in Torah {[from one theme to another; but we do skip in one theme, e.g., the high-priest's reading on Yom Kippur in "Acharei moth" (Leviticus 16) and skipping to "Ach be'asor" (Ibid. 23)]}. And how much may he skip? So long as the translator does not leave off (translating). {[One who skips, whether in Torah in one theme, or in Prophets, even in two themes, may not pause (reading in the process of turning to the next part) longer than is necessary for the translator to finish translating what he had just read, it not befitting the honor of the congregation to have them stand there in silence.]}}

\hebeng{המפטיר בנביא הוא פורס על שמע. והוא עובר לפני התיבה. והוא נושא את כפיו. ואם היה קטן. אביו או רבו עוברין על ידו: }{	The one who {[regularly]} reads the haftarah in Prophets "parcels" the Shema (see 4:3). {[The sages instituted that he "parcel" the Shema for the congregation]}, and he acts as prayer leader {[to effect for them the fulfillment of the obligation of the sanctification of the Name (kedushah) in the Amidah. Because he is forthcoming in reading the haftarah, which is not to his honor, they instituted this for him, for his honor.]}, and he lifts his hands (in the priestly blessing). And if he were a minor, {[who cannot act as prayer leader or "parcel" the Shema, his father or his teacher acts as prayer leader for him.]}}

\hebeng{קטן קורא בתורה ומתרגם. אבל אינו פורס על שמע. ואינו עובר לפני התיבה. ואינו נושא את כפיו. פוחח פורס את שמע ומתרגם. אבל אינו קורא בתורה. ואינו עובר לפני התיבה. ואינו נושא את כפיו. סומא פורס את שמע ומתרגם. רבי יהודה אומר כל שלא ראה מאורות מימיו. אינו פורס על שמע: }{	A minor may read in the Torah {[Some of the geonim say (that he may do so) only from shlishi on.]} and translate; but he does not "parcel" the Shema {[For he comes to effect the fulfillment of the obligation for others; and one who is himself not obligated in something cannot effect fulfillment of the obligation therein for others]}, and he may not act as prayer leader, and he may not lift his hands (in the priestly blessing) {[if he is a Cohein, it not befitting the honor of the congregation to be dependent upon his blessing.]} A pocheach {[one whose clothes are torn and whose arms show ("naked and barefoot" - Isaiah 20:2) - is translated: "pacheach veyachef")]} may "parcel" the Shema, {[for he himself is obligated therein]} and translate, but he does not read in the Torah, and he does not act as prayer leader, and he does not lift his hands (in the priestly blessing). {[He does not read in the Torah because of the honor of the Torah. And so, with acting as prayer leader and lifting his hands, for it is demeaning to the congregation.]} A blind man may "parcel" the Shema {[For even though he does not see the luminaries, he benefits from them. For (through them) others see him and rescue him from obstacles.]}, and he may act as translator. R. Yehudah says: One who never saw the luminaries, {[so that he never benefited from them]} may not "parcel" the Shema. {[The halachah is not in accordance with R. Yehudah.]}}

\hebeng{כהן שיש בידיו מומין. לא ישא את כפיו. רבי יהודה אומר אף מי שהיו ידיו צבועות אסטיס. ופואה. לא ישא את כפיו. מפני שהעם מסתכלין בו: }{	A Cohein who has blemishes on his hands {[likewise on his face or on his feet]} may not recite the priestly blessing. {[For Cohanim are not permitted to go up for the blessing in their shoes. And if he has blemishes on his feet, they will gaze at them, and thence, at his hands. And if one gazes at the Cohanim in their blessing, his eyes are dimmed (Chagigah 16a), the Shechinah abiding between their hands.]} R. Yehudah says: Also, one whose hands are dyed with istis {[a blue dye]} or with puah {[red roots, which produce a red dye]} may not lift his hands, for he is gazed at. {[The gemara concludes that if he were a "familiar" in his city, so that all knew about (and were indifferent to) his blemishes or to his dyed hands, or if most of the men of the city worked in dyes, it is permitted, for then he is not gazed at.]}}

\hebeng{האומר איני עובר לפני התיבה בצבועין אף בלבנים לא יעבור. בסנדל איני עובר. אף יחף לא יעבור. העושה תפלתו עגולה. סכנה ואין בה מצוה. נתנה על מצחו. או על פס ידו. הרי זו דרך המינות. ציפן זהב ונתנה על בית אונקלי שלו. הרי זו דרך החיצונים: }{	If one said: "I shall not act as prayer leader in dyed clothes," he may not do so even in white clothes. {[We fear that he may have succumbed to heresy, the idolators being solicitous of such matters.]} (If he said: "I shall not, etc.") in shoes, he may not do so even barefoot. If one made his (head) phylactery round {[like a nut or an egg]}, he has placed himself in danger {[of the phylactery piercing his head]} and he has not fulfilled the mitzvah, {[for square phylacteries are a "halachah to Moses upon Sinai."]} If he placed it on his forehead or on the palm of his hand, this is the way of heresy. {[For the heretics spurn the words of the sages and follow the literal meaning of the verse, saying that "between your eyes" and "on your hand" are to be taken literally, whereas the sages learned by identity (gzeirah shavah): "between your eyes" — on the hair site of the head, where an infant's brain throbs; "on your hand" — on the height of the hand, the biceps muscle at the top of the arm, so that it is opposite the heart.]} If he plated it with gold {[contrary to (Exodus 13:9): "so that the Torah of the L-rd be in your mouth" — from what is "permitted in your mouth" — that the whole (of the phylacteries) be of the hide of a clean animal and not of gold]} (If he plated it with gold) and placed it on the sleeve of his garment {[from the outside, contrary to (Ibid.): "to you as a sign" — and not to others as a sign]}, this is the way of the "outsiders" {[those who follow their dictates "outside of" the dictates of the sages.]}}

\hebeng{האומר יברכוך טובים. הרי זו דרך המינות. על קן צפור יגיעו רחמיך ועל טוב יזכר שמך. מודים. מודים. משתקין אותו. המכנה בעריות. משתקין אותו. האומר מזרעך לא תתן להעביר למולך (ויקרא יח, כא). ומזרעך לא תתן לאעברא בארמיותא. משתקין אותו בנזיפה: }{	If one says: "May the good (i.e., the righteous) bless You," this is the way of heresy. {[For Israel must include the sinners among them in the assembly of their fasts. For though galbanum (chelbenah) has a foul odor, Scripture included it among the spices of the incense.]} If one says: "To a nest of birds, let Your mercies extend," {[i.e., as Your mercies extended to birds and You decreed (Deuteronomy 22:6): "You shall not take the mother-bird together with the young," so be compassionate and merciful to us]}, he is to be silenced. {[For he makes the mitzvoth of the Holy One Blessed be He functions of mercy, whereas they are nothing else but decrees (of the King to His subjects)]}. Or (if he says:) "For (Your) good let Your name be remembered," {[the implication being: We shall acknowledge You for good (but not for evil)]}, he is to be silenced. {[For we must bless for the evil as well as for the good.]} Or (if he says:) "We thank you," We thank you," {[the impression being given that two deities are being acknowledged and accepted]}, he is to be silenced. If one expounds {[the section on]} illicit relations figuratively, {[e.g., If he interprets the interdict against living with one's father and mother as an exhortation against revealing their shame in public]}, he is to be silenced. If one interprets (Leviticus 18:21): "And from your seed you shall not give to pass (through fire) to Moloch" as: "Do not give of your seed for impregnation to Aramatism" {[i.e., Do not live with a gentile woman and beget a son for idolatry]}, he is to be silenced with a sharp rebuke. {[For he uproots the verse from its (true) meaning and makes one who lives with a gentile woman liable for kareth (cutting-off) if he does so wilfully, and for a sin-offering if he does so unwittingly.]}}

\hebeng{מעשה ראובן נקרא ולא מיתרגם. מעשה תמר נקרא ומיתרגם. מעשה עגל הראשון נקרא ומיתרגם. והשני נקרא ולא מיתרגם. ברכת כהנים. מעשה דוד ואמנון. לא נקראין ולא מיתרגמין. אין מפטירין במרכבה. ורבי יהודה מתיר. רבי אליעזר אומר. אין מפטירין בהודע את ירושלים:  }{	The episode of Reuven (and Bilhah) is read and not translated. The episode of Amnon and Tamar is read and translated. {[And we are not apprehensive for David's honor. This, when it is not written "Amnon son of David" (see below)]}. The first part of the episode of the golden calf is read and translated. {[And we are not apprehensive for the honor of Israel.]} The second part of the episode of the golden calf {[from (Exodus 32:21): "And Moses said to Aaron" until (Ibid. 25): "And Moses saw the people, etc." and (Ibid. 35): "And the L-rd sent a plague among the people, etc."]} is read and not translated, {[in deference to Aaron]}. The priestly blessing {[is read and not translated because it includes (Numbers 6:26): "The L-rd lift His countenance unto you." So that they not say that the Holy One Blessed be He (gratuitously) lifts His countenance (in forgiveness) — and they, not knowing that Israel merits the lifting of His countenance to them.]} The episode of David and Amnon is not read {[in the haftarah]} and not translated {[ — all those verses where it is written "Amnon son of David." But those where "Amnon" alone is written — it is stated above: "The episode of Amnon and Tamar is read and translated."]} There is no haftarah reading in the Divine Chariot (Ezekiel 1) {[lest they come to question and probe therein.]} R. Yehudah permits it. {[And the halachah is in accordance with him.]} R. Eliezer says: there is no haftarah reading in (Ezekiel 16): "Make known unto Jerusalem, etc.", {[for the honor of Jerusalem. The halachah is not in accordance with R. Eliezer.]}}

\blockcomment{ברטנורא על משנה מגילה}{הקורא את המגילה עומד ויושב. רצה עומד רצה יושב:\\ואין מוסיפין עליהן. דבראשי חדשים ובחולו של מועד נמי איכא בטול מלאכה, דמלאכת דבר האבד מותרת:\\אין פורסין על שמע. עשרה שבאו לבית הכנסת אחר שקראו צבור את שמע, עומד אחד ואומר קדיש וברכו וברכה ראשונה שלפני קריאת שמע. פורסין, לשון פרוסה, כלומר חצי דבר, משתי ברכות שלפני ק״ש אומר ברכה אחת:\\ולא יקרא למתורגמן יותר מפסוק אחד. שלא יטעה המתורגמן שמתרגם על פה:\\המפטיר בנביא. מי שרגיל להפטיר בנביא, תקנו חכמים שיהא הוא פורס על שמע ברבים:\\קטן קורא בתורה. ויש מן הגאונים שאמרו דוקא משלישי ואילך:\\כהן שיש בידו מומין. וכן בפניו או ברגליו:\\אף בלבנים לא יעבור. חיישינן שמא מינות נזרקה בו, דעובדי ע״ז מקפידין בכך:\\יברכוך טובים הרי זו דרך מינות. שצריכין ישראל לצרף עמהם פושעי ישראל באגודת תעניותיהם. שהרי חלבנה ריחה רע ומנאה הכתוב עם סממני הקטורת:\\מעשה אמנון ותמר נקרא ומיתרגם. ולא חיישינן ליקריה דדוד. והוא, דלא כתיב אמנון בן דוד, כדבעינן למימר לקמן:\\\n}\clearpage %endcomment
\addpart{בן יהוידע על מגילה}\renewcommand{\partname}[1]{בן יהוידע על מגילה}
\newchap{פרק \hebrewnumeral{2} הקורא למפרע}
\twocol{\textbf{מִנַּיִן שֶׁאוֹמְרִים ׳אָבוֹת׳? שֶׁנֶּאֱמַר {\small (תהלים כט, א)}׃ ״הָבוּ לַה׳ בְּנֵי אֵלִים״}. נראה לי בס״ד מה שנקראים האבות בְּנֵי אֵלִים על פי מה שכתב רבינו האר״י ז״ל בדרוש ראש השנה דרוש וא״ו, דשלשה שמות אֵ־ל אֵ־ל אֵ־ל שהם גימטריא מָגֵן {\small [31×3=93]} נרמזים בג׳ פעמים ׳אֱלֹקֵי׳ שיש בתפילת העמידה כמו שאומרים ׳אֱלקֵי אַבְרָהָם אֱלקֵי יִצְחָק וֵאלקֵי יַעֲקב׳ יעוין שם. נמצא ג׳ פעמים אֵ־ל שיש בג׳ פעמים אֱלֹקֵי האמורים על שם האבות הם רומזים לבחינת שלשה שמות אֵ־ל בפני עצמו ולכן נקראים האבות בְּנֵי אֵלִים על שלשה שמות אֵ־ל הנזכרים על שמם בתפילת העמידה.\par או יובן בס״ד כי האבות הם רגלי הכסא ו׳אֵלִים׳ גימטריא כִּסֵּא {\small [81]} ולכן נקראים בְּנֵי אֵלִים כלומר בְּנֵי הַכִּסֵּא.
\textbf{וּמָה רָאוּ לוֹמַר בִּרְכַּת ׳הַשָּׁנִים׳ בַּתְּשִׁיעִית? אָמַר רַבִּי אַלֶכְּסַנְדְּרִי: כְּנֶגֶד מַפְקִיעֵי שְׁעָרִים, דִּכְתִיב {\small (תהלים י, טו)}׃ ״שְׁבֹר זְרוֹעַ רָשָׁע״}. הקשה מהרש״א ז״ל כיון דפירש רש״י ז״ל דכל הפרשה נאמרה על מפקיעי שערים, אמאי מייתי רק האי קרא ד׳שְׁבֹר זְרוֹעַ רָשָׁע׳? יעוין שם.\par ונראה לי בס״ד משום דבהאי קרא רומז העונש שיעשה לו שלא יוכל להפקיע עוד מכאן ולהבא. והא דאותיות העומדים בראש וסוף התיבה נקראים זרועות, וכאן אמר ׳זְרוֹעַ רָשָׁע׳ משמע חד והוא אות עי״ן שבסוף שמו של רָשָׁע דאם מסיר אותו ישאר ׳רָשׁ׳ ורמז בזה שיעני אותו וישאר רָשׁ, ואז לא יוכל לקבץ התבואה תחת ידו כדי ליקר השער. וזהו ׳שְׁבֹר זְרוֹעַ רָשָׁע׳ הוא אות ע׳ כדי שיהיה רש ועני, וקראו בשם \textbf{רָשָׁע} שהוא בהפוך אתוון \textbf{שַׁעַר} , כלומר רשעתו אינה בעבירה של זנות וכיוצא אלא היא בענין יוקר השער של הדגן.
\clearpage}

\newsection{דף יח}
\twocol{\textbf{שְׁכָחוּם וְחָזַר וְסִדְּרָם}. אין הכונה ששכחו סדר הברכות, דזה לא יתכן, דאפילו המון העם מתפללים כל שמונה עשרה ואיך ישכחום? אך הכונה ששכח הטעמים של הסדר, למה זו ראשונה וזו שניה וזו שלישית וכיוצא וכמו שסידרם הש״ס כאן בטעמייהו וגם שכחו טעמי הסדר שיש בהם על פי הסוד.
\textbf{{\small (תהלים קו, ב)} לְמִי נָאֶה לְמַלֵּל גְּבוּרוֹת ה׳? לְמִי שֶׁיָּכוֹל לְהַשְׁמִיעַ כָּל תְּהִלָּתוֹ}. מקשים וכי אנשי כנסת הגדולה היו יכולים להשמיע כל תהלתו, ואיך תיקנו כל זה?\par ונראה לי בס״ד דאנשי כנסת הגדולה תיקנו דברים כנגד ספירות העליונים שיש לכל ברכה ושבח ספירה פרטית כנגדה והרי זה דומה למלך שיש לו כמה אלפים ורבבות ארמונים עד אין מספר ובא אחד וסיפר שיש בארמון אחד מן הארמונים כך וכך חדרים וכך וכך עליות אין זה גורע דאינו סופר כל החדרים ועליות שיש למלך בכל הארמונים אלא רק מדבר בארמון פרטי, אבל מי שבא לומר יש למלך כך וכך חדרים בסתם וכך וכך עליות בסתם שאינו מתכוין על ארמון פרטי הרי זה גורע, דבאמת אין מספר לחדרים ולעליות של המלך! וכן הענין כאן, והמניעה היא למי שמתכוין לקבץ שבחים הרבה מן הכלל ומן הפרט לזה אומרים סיימתנהו לשבחי דמרך.
\textbf{הַמְסַפֵּר בְּשִׁבְחוֹ שֶׁל הַקָּדוֹשׁ בָּרוּךְ הוּא יוֹתֵר מִדַּאי, נֶעֱקָר מִן הָעוֹלָם}. נראה לי בס״ד דאיתא באותיות דרבי עקיבא, בעבור השיר שמקלסין להקב״ה נברא העולם עיין שם, ועל כן שזה שהוא מרבה בשבח שחושב שמחזק העולם שנברא בעבור השיר לכך מדה כנגד מדה נעקר מן העולם.
\textbf{מִלָּה בְּסֶלַע, מַשְׁתּוּקָא בִּתְרֵין}. נראה לי בס״ד בתוך אותיות ׳סֶלַע׳ במילואם יש ׳מַיִם׳ כזה: ס\textbf{מ} ״ך ל\textbf{מ} ״ד ע\textbf{י} ״ן, וידוע מה שאמר רבי עקיבא לחכמים שנכנסו לפרדס כשתגיעו לאבני שיש טהור, אל תאמרו ׳מים מים׳ {\small (חגיגה יד:)} , נמצא יש מקום שאם יאמרו בו ב׳ פעמים מים צרוך להשתיקם, וזהו שאמר יש \textbf{מִלָּה בְּסֶלַע} שהיא תיבת מַיִם המונחת בתוך סֶלַע ראוי לעשות בה \textbf{מַשְׁתּוּקָא} , אם יאמרו אותה \textbf{בִּתְרֵין} וסוד הנזכר שמנע מהם הכפל של ׳מים מים׳ מפורש בדברי מהרח״ו {\small [מורנו הרב רבי חיים ויטאל]} ז״ל בביאור {\small (משנה אבות ו, א)} ׳כָּל הָעוֹסֵק בַּתּוֹרָה לִשְׁמָהּ׳ יעוין שם.\par ועוד נראה לי דרך הלצה, אם יש שטר צוואה או שטר חוב או מתנה שכתוב בו תנו סלע לפלוני או חייב אני סלע לפלוני, ורוצה בעל השטר לזייף להוסיף אותיות ים על סלע, כדי שתהיה נקראת ׳סלעים׳ שבדעתו לתבוע מאות ואלפים, לא תועיל לו ערמתו כי יכול בעל דינו להשתיקו בתרין סלעים, כי יאמר מעוט רבים שנים. וזהו שאמר מלה בסלע, רוצה לומר אם תרצה להוסיף \textbf{מִלָּה} אחת של ים במלת \textbf{בְּסֶלַע} , כדי שתהיה נקראת סלעים לשון רבים, עושים לך \textbf{מַשְׁתּוּקָא} לתביעתך, \textbf{בִּתְרֵין} סלעים שיפטרו אותך בהם דיאמרו לך מיעוט רבים שנים.\par מיהו צריך להבין למה נקטו שיעור ׳סֶלַע׳ לדבור? ונראה לי בס״ד כי תוכו של אדם כמו בור עמוק והפתח הוא הפה שיוצא ממנו הדיבור אך הדברים נחלקים לארבעה חלקים: שמדבר על עצמו, ועל זולתו, על העבר, ומדבר על עצמו ועל זולתו על העתיד, נמצא נחלק קרבו לארבע בורות לארבעה חלקים של דבור, לכך נקרא דִּבּוּר - ד׳ בּוֹר, ולכן עשה ערך לדבור ׳סֶלַע׳ שהוא ארבעה דנרים.\par \textbf{ובני ידידי} כה״ר יעקב נר״ו פירש דידוע המלאכים דברו תביעתם על התורה בעבור חלק הסוד דוקא דשייך להם, וזהו שאמר מִלָּה בְּ׳סֶלַע׳ ראשי תיבות \textbf{ס} וד \textbf{ל} אוין \textbf{ע} שין, על זה היתה מלה של תביעה אך נשתתקו בתרין בשביל שנתן הקב״ה לישראל שתים חלקים ביחד חלק הסוד וחלק הנגלה דלא שייך במלאכים.\par ועוד פירש מלה בסלע, דאמרו רבותינו ז״ל {\small (קידושין לא.)} כשאמר{\small (שמות כ, ב)} ׳אָנֹכִי וְלֹא יִהְיֶה לְךָ׳ הרהרו אומות העולם על זה ואמרו לכבוד עצמו דורש! וזהו \textbf{מִלָּה בְּסֶלַע} הוא דבור ראשון שהוא גבוה שבעשרת הדברות ולהכי קרי ליה סלע שהוא גבוה אך נעשה להם \textbf{מַשְׁתּוּקָא בִּתְרֵין} , כאשר שמעו דבור ׳כַּבֵּד אֶת אָבִיךָ וְאֶת אִמֶּךָ׳, שצוה על כבוד האב והאם שהם שנים, דאז חזרו והודו לדברות ראשונות עד כאן דבריו נר״ו.
\textbf{מִנַּיִן שֶׁקְּרָאוֹ הַקָּדוֹשׁ בָּרוּךְ הוּא לְיַעֲקֹב ׳אֵ־ל׳?}. יש להבין איך יצוייר זה בדעת? ונראה לי דלא קראו אֵ־ל בסתם אלא קראו יִשְׂרָאֶל בנקוד סגול תחת האלף כזה ׳יִשְׂרָאֶל׳ דאז יוצא בשמו שם אֵ־ל ולא קראו כמו שם ׳יִשְׁמָעֵאל׳ דאין נקוד תחת האלף דעל כן אינו יוצא במבטא אות האלף אלא העין נדבק עם הלמד, וכאשר הוא כן בשם יחזקאל הנביא ע״ה ופירש דכתיב {\small (בראשית לג, כ)} ׳וַיַּצֶּב שָׁם מִזְבֵּחַ וַיִּקְרָא לוֹ: אֵ־ל אֱלֹקֵי יִשְׂרָאֵל׳ רוצה לומר אֱלֹקֵי יִשְׂרָאֵל קרא שם אֶל כשקראו ישראל, ולכן אברהם שמו כפול, הרי יו״ד אותיות, ושם יצחק אינו כפול הרי י״ד אותיות, ושם יעקב כפול הרי כ״ב ושם ישרן בלא ווי״ן הרי כ״ו, ונמצא שם ישראל משלים ל״א אותיות כמנין שם אל.
\clearpage}

\newsection{דף יט}
\twocol{\textbf{מַאן דְּאָמַר {\small (אסתר ט, כט)}׃ ׳כֻּלָּה׳, תָּקְפּוֹ שֶׁל אֲחַשְׁוֵרוֹשׁ}. דכתיב {\small (אסתר א, ד)} ׳בְּהַרְאֹתוֹ אֶת עֹשֶׁר כְּבוֹד מַלְכוּתוֹ׳ דנמצא שהיה עשיר גדול ואיך יצוייר שימכור אומה שלימה בעשרת אלפים ככר כסף? אשר אלו לא היו נחשבים אצלו אפילו בערך אבנים! ואם כן מוכח דהמן זייף באומרו שקנה אותם בעשרת אלפים ככר כסף, וכיון דזייף בזה זייף בכל ובטלו האגרות שכתב!\par \textbf{וּמַאן דְּאָמַר}׃ \textbf{מֵ} ׳\textbf{אִישׁ יְהוּדִי}׳, \textbf{תָּקְפּוֹ שֶׁל מָרְדֳּכַי} שהיה לו כח מן יהודה ומן בנימין ובזה הכניע את המן שהיה בידו קטרוג מכח תמנע שלא קבלו אותה ומכח השבטים שהשתחוו לעשו וכמו שכתב מהר״י ז״ל דבכח אִישׁ יְהוּדִי דאתי מיהודה ביטל קטרוג תמנע, ומכח אִישׁ יְמִינִי דאתי מבנימין שלא השתחוה לעשו ביטל קטרוג ההשתחויה שהשתחוו השבטים לעשו.\par \textbf{וּמַאן דְּאָמַר}׃ \textbf{מֵ} ׳\textbf{אַחַר הַדְּבָרִים הָאֵלֶּה}׳, \textbf{תָּקְפּוֹ שֶׁל הָמָן} כי בגאותו ותוקפו נתן עיניו לכלות הכל ובזה ממילא נעשה נס ההצלה מה שאין כן אם היה נותן עיניו במקצת היה עולה בידו. \textbf{וּמַאן דְּאָמַר}׃ \textbf{מִ} ׳\textbf{בַּלַּיְלָה הַהוּא}׳, \textbf{תָּקְפּוֹ שֶׁל נֵס} כי נזדמן אותה הלילה שהיתה מוחזקת בנסים גדולים מדורות הראשונים ולכך הצליחו בה במעשה נסים.
\textbf{׳מָה רָאָה׳ אֲחַשְׁוֵרוֹשׁ שֶׁנִּשְׁתַּמֵּשׁ בַּכֵּלִים שֶׁל בֵּית הַמִּקְדָּשׁ? ׳עַל כָּכָה׳, מִשּׁוּם דְּחָשִׁיב שִׁבְעִים שְׁנִין וְלָא אִיפְרוּק}. נראה לי בס״ד נרמזו השבעים בזה דאות כף של כָּכָה כפול ואם תסיר הכפל יהיו ׳עַל כָּכָה׳ אותיות ׳ע׳ כלה׳ פירוש ראה שכלה זמן של ע׳ שנה ועל ידי כך ׳וּמָה הִגִּיעַ אֲלֵיהֶם׳? דְּקָטַל וַשְׁתִּי! שנעשה להם מהומה גדולה שנוסף אותיות ומה על אותיות מה ונעשה צירוף מְהוּמָה.
\textbf{׳מָה רָאָה׳ מָרְדֳּכַי דְּאִיקְנִי בְּהָמָן? ׳עַל כָּכָה׳, דְּשַׁוִּי נַפְשֵׁיהּ עֲבוֹדָה זָרָה}. נראה לי נרמז זה בתיבת כָּכָה שהוא מספר לוט {\small [45]} שהוא תרגום של ארירה, דאומרים ארור המן על אשר עשה עצמו עבודה זרה. ׳וּמָה הִגִּיעַ אֲלֵיהֶם׳ רוצה לומר אותיות מה של גְּאוּלָּה {\small [45]} , שמספרם גְּאוּלָּה שנעשה בה הנס.\par ואומרו ׳\textbf{מָה רָאָה} ׳ \textbf{הָמָן שֶׁנִּתְקַנֵּא בְּכָל הַיְּהוּדִים} ׳\textbf{עַל כָּכָה} ׳ \textbf{מִשּׁוּם דְּמָרְדֳּכַי לֹא יִכְרַע וְלֹא יִשְׁתַּחֲוֶה} והיינו כָּכָה גימטריא ׳לב אחד׳ {\small [45]} , פירוש שהיה לו לב אחד, ׳וּמָה הִגִּיעַ אֲלֵיהֶם׳? ׳וְתָלוּ אוֹתוֹ וְאֶת בָּנָיו עַל הָעֵץ׳! שיצאה נשמת בניו ברגע אחד בעבור זכות מרדכי שהיה לו לב אחד.\par ואמר עוד ׳\textbf{מָה רָאָה} ׳ \textbf{אֲחַשְׁוֵרוֹשׁ לְהָבִיא אֶת סֵפֶר הַזִּכְרוֹנוֹת?} ׳\textbf{עַל כָּכָה}׳, \textbf{דְּזַמִּינְתֵּיהּ אֶסְתֵּר לְהָמָן בַּהֲדֵיהּ} נראה לי דדריש כָּכָה על סעודה לשון דוק בבכי, ׳וּמָה הִגִּיעַ אֲלֵיהֶם׳? כי מן הסעודה נולד מעשה הנס.
\clearpage}

\newsection{דף כא}
\twocol{\textbf{אִלְמָלֵא מִקְרָא כָּתוּב, אִי־אֶפְשָׁר לְאָמְרוֹ, כִּבְיָכוֹל, שֶׁאֲפִלּוּ הַקָּדוֹשׁ בָּרוּךְ הוּא בַּעֲמִידָה}. הכונה כי הנה קרא כתיב {\small (הושע יב, יא)} ׳וּבְיַד הַנְּבִיאִים אֲדַמֶּה׳ דאי אפשר לשום נביא לראות פני שכינה, דכתיב על ידי משה רבינו ע״ה בעצמו {\small (שמות לג, כ)} ׳כִּי לֹא יִרְאַנִי הָאָדָם וָחָי׳, אך יצוייר לעיני הנביאים איזה דמיון כדי שיוכלו להשיג הנבואה כל אחד לפי ערכו, וכמו שכתב הרמב״ם ז״ל בהלכות יסודי התורה, דכל נביא יהיה איזה דמיון לנגד עיניו לפי שעה, והדמיון הזה אינו קבוע לאותו נביא, אלא יהיה לפי אותו הענין של הנבואה דאפילו משה רבינו ע״ה לא ראה הדמיון שוה כי בים סוף נדמה לו כגיבור ובסיני נדמה לו כזקן ובאמת אין לו יתברך דמות וצורה אלא הכל יצוייר לעיני הנביא כמראה הנבואה וכמחזה לפי שעה ואין דעתו של אדם מבין אמיתות הדבר, יעוין שם. וכן הענין כאן, שהיה נדמה לו כעומד ולא יושב ואף על פי שבאמת הציור הזה שהיה מצטער לעיני משה רבינו ע״ה בסיני אין זה אמיתות הדבר אלא הוא דרך דמיון, עם כל זה אמר רבי אבהו אלמלא מקרא שכתוב לא היינו יכולים לאומרו בפינו, וכונת בעל המאמר לומר כי נצטייר ונדמה לעיני משה רבינו ע״ה כעומד, ללמדינו דעת בזה שיהיו הרב והתלמיד התחתונים שוים, שניהם בעמידה או שניהם בישיבה או שניהם על גבי קרקע.\par ועוד נראה לי בס״ד דלעולם לא היה רואה משה רבינו ע״ה בהר בעת לימודו שום ציור ודמיון בעמידה אלא רק היה שומע הקול מדבר אליו ומלמדו, ומה שאמר לו {\small (דברים ה, כז)} ׳עֲמֹד עִמָּדִי׳ אמר לו בלשון זה המורה כאלו היה הקב״ה בעמידה, כדי ללמד לתחתונים שתהיה קריאתם בעומד. וגם ללמדם שלא יהיה הרב ע על גבי מטה ותלמידיו על גבי קרקע.
\textbf{מִימוֹת מֹשֶׁה עַד רַבָּן גַּמְלִיאֵל, הָיוּ לְמֵדִין תּוֹרָה מְעֻמָּד}. נראה לי בס״ד הטעם כדי להטריח עצמן בתורה, כמו שאמרו {\small (מגילה ו:)} ׳יָגַעְתִּי וּמָצָאתִי תַּאֲמֵן׳.\par או יובן הרגלים רומזים לנצח והוד, והתורה ניתנה על ידי לוחות שהוא סוד נצח והוד, ולמדוה מפי משה ואהרן שהם נצח והוד.\par או יובן הגוף שהוא עולם קטן עומד על שני רגלים וכן העולם עומד על התורה דכתיב {\small (ירמיה לג, כה)} ׳אִם לֹא בְרִיתִי יוֹמָם וָלָיְלָה חֻקּוֹת שָׁמַיִם וָאָרֶץ לֹא שָׂמְתִּי׳ והתורה כוללת שני עמודים שהוא עסק התורה וגם עמוד העבודה ד׳כל הקורא בפרשת עולה כאלו הקריב עולה׳ {\small (מנחות קי.)} .\par אי נמי לימוד התורה מעומד יורה דראוי לאדם ללמוד תורה מן הרב שהוא דומה למלאך, דכתיב {\small (מלאכי ב, ז)} כִּי שִׂפְתֵי כֹהֵן יִשְׁמְרוּ דַעַת וְהַתּוֹרָה יְבַקְשׁוּ מִפִּיהוּ כִּי מַלְאַךְ הֳ׳ צְבָקוֹת הוּא, וידוע כי המלאכים אין להם ישיבה למעלה אלא הם עומדים דכתיב {\small (זכריה ג, ז)} ׳בֵּין הָעֹמְדִים הָאֵלֶּה׳.
\textbf{רַכּוֹת מְעֻמָּד, קָשׁוֹת מְיֻשָּׁב}. קשה יתן לו כח שיוכל ללמוד גם הקשות מעומד מאחר שהוא בנס היה שם שהיה בלא אכילה ובלא שתיה ובלא שינה וכיון שהיה עומד שם בנס בכח אלקי אם כן מה הפרש יש לו בין ישיבה לעמידה כי באמת אינו עומד שם בכח טבעי?\par ונראה לי דודאי לא יש אצלו הפרש ורק עשה לו כן שילמד רכות מעומד קשות מיושב כדי ללמד לתחתונים שהנהגתם וכוחן הוא טבעי שככה יעשו רכות מעומד וקשות מיושב ומאן דאמר שהיה ׳שׁוֹחֶה׳ היינו כדי להראות הכנעה על דרך שאמרו גבי מלך בתפילת העמידה.}


\addtocontents{toc}{\protect\end{multicols}}
\end{document}
