\documentclass[12pt, openany]{book}
\usepackage[
paperheight=8.5in,
paperwidth=5.5in,
top=0.5in,
bottom=0.5in,
inner=0.7in,
outer=0.5in,
marginparsep=0.1in,
headsep=16pt
]{geometry}

\newcommand{\texttitle}{נדה}\usepackage{titlesec}
\usepackage{resources/unnumberedtotoc}

\usepackage{fancyhdr}
\pagestyle{fancy}
\fancyhf{}
\fancyhead[LO,RE]{\thepage}
\fancyhead[CO]{\chapname}
\fancyhead[CE]{\texttitle}

\usepackage{paracol}
\usepackage{anyfontsize}
\usepackage{ragged2e}
\usepackage{polyglossia}
\usepackage{multicol}
\usepackage{hyperref}
\usepackage[marginal]{footmisc}

\setdefaultlanguage{hebrew}
\setotherlanguage{english}
\usepackage{fontspec}
\setmainfont{Times New Roman}
\newfontfamily\englishfont{Times New Roman}

\newcommand{\sethebfont}{
\fontsize{10.5pt}{21.0pt} \selectfont
}

\newcommand{\hebeng}[2]{
	{\sethebfont #1\\}
	
	\begin{english}
		#2
	\end{english}
	\clearpage
}

\newcommand{\twocol}[1]{
	{\sethebfont \begin{multicols}{2}
			#1
	\end{multicols}}	
}

\newcommand{\textblock}[1]{
{\sethebfont #1\\}	
}

\setlength{\parskip}{8pt}
\setlength\parindent{0in}

\newcommand{\chapname}{}
\newcommand{\sectname}{}

\newcommand{\newchap}[1]{
	\addcontentsline{toc}{chapter}{#1}
	\renewcommand{\chapname}{#1}
		\begin{center}
			\textbf{%
\fontsize{16pt}{16pt}\selectfont
				#1}
		\end{center}
}

\let\footnoterule\relax

\setlength{\columnsep}{0.25in}

\newcommand{\newsection}[1]{
	\addcontentsline{toc}{section}{#1}
	\renewcommand{\sectname}{#1}	
	\vspace{-\baselineskip}
	\begin{center}
		\textbf{%
\fontsize{16pt}{16pt}\selectfont
			#1}
	\end{center}
	\vspace{-\baselineskip}
	\nopagebreak
}

\newcommand{\footnotecomment}[1]{
	\renewcommand\thefootnote{}
	\footnote{#1}}

\newcommand{\parencomment}[1]{\footnotesize (#1)}

\newcommand{\commenta}[1]{\footnotecomment{#1}}

\begin{document}
\frontmatter
\pagenumbering{roman}

\title{\texttitle}

\author{}

\date{}

\maketitle

\begin{minipage}[b][\textheight][b]{\textwidth}\englishfont	
	\begin{english}
		\vfill
		The following book includes:
\begin{itemize}
\item[$\bullet$] Wikisource Talmud Bavli
\item[$\bullet$] License: CC-BY
\item[$\bullet$] Source: \url{http://he.wikisource.org/wiki/%D7%AA%D7%9C%D7%9E%D7%95%D7%93_%D7%91%D7%91%D7%9C%D7%99}
\end{itemize}
		It was retrieved from Sefaria on \today\space \texthebrew{(\Hebrewtoday)}.  It was typeset and formatted by Ktavi.
		\clearpage
		
	\end{english}
\end{minipage}


\tableofcontents

\clearpage
\mainmatter
\pagenumbering{arabic}

\newchap{פרק \hebrewnumeral{1} שמאי אומר}
\newsection{דף ב}
\twocol{
\commenta{\textbf{הלל אומר מפקידה לפקידה ואפילו לימים הרבה.} פי' אפילו מפקידה לפקידה חוששין להם דמפקידה לפקידה חששא דרבנן היא וספק טומאה גזרו עליו לתלות בתרומה וקדשים כדאיתא בגמרא ואמרינן נמי לקמן בדבר שיש בו דעת לישאל. וממילא נמי שמעינן דדוקא ברשות היחיד אבל ברשות הרבים טהור דלא גזרו אלא דלהוי כספק טומאה דברשות היחיד ובדבר שאין בו דעת לישאל ואפילו ברה"י ספקו טהור וברה"י ודבר שיש בו דעת לישאל נמי לא גזרו אלא לתלות אבל לא טומאה ודאי דחששא בעלמא הוא ולחומרא כטעמא דמפרש בגמרא.\par וי"מ דמעת לעת שבנדה נוהג בין ברה"י בין ברה"ר גזרו לתלות בתרומה וקדשים וכן אמרו בירוש' דמעת לעת שבנדה נוהג בין ברשות היחיד בין ברשות הרבים והיינו נמי דאקשינן הכא בשמעתין להלל קשיא טומאה ודאי דאלו מעת לעת שבנדה תולין ולא אקשינן נמי ברשות הרבים טהור. ואלו הכא קתני בין ברה"י בין ברה"ר טמא והטעם לזה שלא הלכו בגזרה דלמפרע על דרך טומאה דסוטה. דטומאת סוטה מכאן ולהבא היא הילכך השוו רשויות לתלות ולא לשרוף. ואין זה מחוור כלל. }
מתני׳ {\large\emph{שמאי}} אומר כל הנשים דיין שעתן הלל אומר מפקידה לפקידה ואפילו לימים הרבה 
\commenta{גמרא \textbf{מאי טעמא דשמאי.} פרושי קא מפרש מתניתין ואזיל דאלו טעמיה דשמאי דכולי עלמא אית להו אלא שהחמירו לתלות בתרומה וקדשים כדפרישית, ואוקים משום דהעמד אשה על חזקתה ובחזקת טהורה עומדת שהרי טבלה לנדתה ובדוקה היא משעה שפסקה לנדתה הראשון.\par והלל כי אמרי העמד דבר על חזקתו ואפילו לתרומה וקדשים היכא דלית ליה ריעותא מגופיה הך אשה כיון דמגופה קא חזיא לא אמרי' אוקמה אחזקה אלא חיישינן הילכך בחולין אף על גב דאיכא למיחש במילתא כיון דלא מכרעא מילתא דטומאה מוקמינן טהרות אחזקתייהו ג) ד"א מאפישי טומאה לא מפשינן וגבי תרומה וקדשים עבד בהו רבנן מעלה וכיון דליכא חזקה גמורה תולין.\par וי"מ דאף על גב דליכא חזקה מיהו ה"ל ספק טומאה ומסוטה גמרינן מה התם מכאן ולהבא ולא למפרע אף כל ספק טומאה לא מטמינן למפרע אלא משום מעלה דקדשים דתולין וברשות הרבים טהור לגמרי דגמרינן מסוטה בק"ו. }
וחכ"א לא כדברי זה ולא כדברי זה אלא מעת לעת ממעטת על יד מפקידה לפקידה ומפקידה לפקידה ממעטת על יד מעת לעת 
כל אשה שיש לה וסת דיה שעתה והמשמשת בעדים הרי זו כפקידה וממעטת על יד מעת לעת ועל יד מפקידה לפקידה 
כיצד דיה שעתה היתה יושבת במטה ועסוקה בטהרות ופרשה וראתה היא טמאה והן טהורות 
אע"פ שאמרו מטמאה מעת לעת אינה מונה אלא משעה שראתה
{\large\emph{גמ׳}} מאי טעמיה דשמאי קסבר העמד אשה על חזקתה ואשה בחזקת טהורה עומדת והלל כי אמר העמד דבר על חזקתו היכא דלית ליה ריעותא מגופיה אבל איתתא}

\newchap{פרק \hebrewnumeral{1} שמאי אומר}
\twocol{כיון דמגופה קחזיא לא אמרינן אוקמה אחזקתה 
ומאי שנא ממקוה דתנן מקוה שנמדד ונמצא חסר כל טהרות שנעשו על גביו למפרע בין בר"ה בין ברה"י טמאות 
לשמאי קשיא למפרע 
להלל קשיא ודאי דאילו מעת לעת שבנדה תולין לא אוכלין ולא שורפין ואילו הכא טומאה ודאי 
התם משום דאיכא למימר העמד טמא על חזקתו ואימא לא טבל אדרבה העמד מקוה על חזקתו ואימא לא חסר הרי חסר לפניך 
הכא נמי הרי דם לפניך השתא הוא דחזאי הכא נמי השתא הוא דחסר 
הכי השתא התם איכא למימר חסר ואתא חסר ואתא הכא מי איכא למימר חזאי ואתא חזאי ואתא ומאי קושיא דלמא הגס הגס חזיתיה 
התם איכא תרתי לריעותא הכא איכא חדא לריעותא 
ומאי שנא מחבית דתנן היה בודק את החבית להיות מפריש עליה תרומה והולך ואח"כ נמצא חומץ כל ג' ימים (הראשונים) ודאי
מכאן ואילך ספק קשיא לשמאי 
התם משום דאיכא למימר העמד טבל על חזקתו ואימר לא נתקן אדרבה העמד יין על חזקתו ואימר לא החמיץ 
הרי החמיץ לפניך הכא נמי הרי דם לפניך השתא הוא דחזאי התם נמי השתא הוא דהחמיץ 
הכי השתא התם איכא למימר החמיץ ואתא החמיץ ואתא הכא מי איכא למימר חזאי ואתא חזאי ואתא ומאי קושיא דלמא הגס הגס חזיתיה 
התם איכא תרתי לריעותא הכא איכא חדא לריעותא 
ורמי חבית אמקוה מאי שנא הכא ודאי ומ"ש הכא ספק 
א"ר חנינא מסורא מאן תנא חבית ר"ש היא דלגבי מקוה נמי ספקא משוי ליה 
דתנן מקוה שנמדד ונמצא חסר כל הטהרות שנעשו על גביו למפרע בין בר"ה בין ברה"י טמאות 
ר"ש אומר בר"ה טהורות ברה"י תולין}

\newsection{דף ג}
\twocol{ושניהם לא למדוה אלא מסוטה 
\commenta{והא דאמרינן בטעמיה דר"ש \textbf{דגמר סוף טומאה מתחלת טומאה} ולא גמר רשות היחיד דסוף טומאה מתחלת טומאה משום דסבירא ליה דרשות היחיד גמרינן מסוטה בתחלת טומאה וגזרת הכתוב הוא רשות הרבים דינא הוא וגמרינן מיניה דלא אורועי ולטמויי מידי מספיקא. }
רבנן סברי כי סוטה מה סוטה ספק היא ועשאוה כודאי הכא נמי ספק ועשאוה כודאי 
\commenta{\textbf{ואב"א היינו טעמיה דשמאי הואיל ומרגשת בעצמה.} הקשו בתוספות רבותינו הצרפתים ז"ל הואיל אם בדקה עצמה עכשיו ומצאה טמא היאך תאמר מרגשת היתה והלא לא הרגישה עכשיו. ואם תאמר נתלה להקל ונאמר בבדיקה אירע לה אורח וכסבורה הרגשת עד הוא וכפירש"י דא"כ אם שמשה מאתמול נמי נימא הא דלא ארגישה מאתמול כסבורה הרגשי שמש הוא כדאיתמר נמי התם.\par והם פי' דטעמא דשמאי משום דכיון דרוב פעמים מרגש' לא חיישינן למיעוטא ולא גזרו חכמים מחמתן כלל דהא ללישנ' בתרא דאמרי' משום דאם איתא דהוה דם מעיקרא אתא שוכבת במטה ולא נתהפכה מא"ל אלא משום דבר שאין דרכו בכך לא החמירו לאסור מעת לעת ומיהו שוטה ומשמשת במוך כיון דלעולם כך דינן של אלו לחוש להן לפי שאין בהם הרגשה לעולם לפיכך שמאי מודה בהן ופי' טעמא דשמאי באשה מרגשת לומר דהואיל ואשה מרגשת בעצמה לכך לא חשש שמאי כלל ולא עשה סיג לדבריו. והלל אומר הרגשת מי רגלים היא אלא העמד האשה על ספיק' ודינה לתלות כדפרישי' ללישנא קמא. }
אי מסוטה אימא כי סוטה מה סוטה ברה"ר טהור הכא נמי ברה"ר טהור 
\commenta{\textbf{והאיכא שוטה מודה שמאי בשוטה.} פירש למאן מודה להלל דהוא בר פלוגתיה והיינו לתלות בתרומה וקדשים ואטעמא דלישנא קמא סמכינן בדהלל דהא ליכא למימר דטעמא דהלל משום דהרגשת מי רגלים חששא היא בעלמא ולפיכך אמר תולין דאם כן אף הלל מודה בשוטה דשורפין ואפילו ברשות הרבים וכן ללישנא בתרא במוך ואנן לא אשכחן במעת לעת אלא תולין, אלא הני לישני בטעמא דשמאי נינהו אבל הלל טעמיה משום דגריעי חזקה דאשה וכיון דדם לפניך עבוד רבנן מעלה בקדשים כדפרישית לעיל, וכן הא דאמרינן נימא תנן כתמים דלא כשמאי, אי הוו להו כתמים כמעת לעת דלא כהלל נמי הוו דהא כתמים לבעלה ולחולין ומעת לעת דוקא לקדשים אלא חזקה באשה הוא דמטהר מעת לעת בחולין להלל כלישנא קמא דגמרא והא דאמרינן נמי להך לישנא ליכא למירמי חבית ומקוה ומבוי הכי נמי פירושי' לשמאי ליכא לאקשויי למפרע אבל להלל בטומאת ודאי לא קושיא היא שהדין נותן כיון דאיכא תרתי לריעותא טומאה ודאי במקום דאיכא חדא ריעותא תולין. וכן שוטה ומשמשת במוך לשמאי אינן אלא תולות כשאר נשים להלל ולא קשיא להו טומאה ודאי דחבית ומקוה ומבוי משום האי טעמא דפרישית ולמה שפירש לעיל דכל ספק טומאה למפרע טהור מסוטה איכא לפרושי דבין שמאי ובין להלל בהאי טעמא בלחוד פליגי מר סבר אשה מרגשת לעצמה ואפילו לתלות אין תולין, ומר סבר אימר הרגשת מי רגלים הוא. והוה ליה ספיקא וכל ספק למפרע טהור מן התורה ומדבריהם החמירו בקדשים לתלות והחמירו בכתמים אפילו לחולין. }
הכי השתא התם משום סתירה הוא וסתירה ברה"ר ליכא הכא משום חסר הוא מה לי חסר ברה"ר מה לי חסר ברה"י 
\commenta{ הא דאמרינן \textbf{מודה שמאי בשוטה.} הוקשה בתוספות דבר שאין בו דעת לישאל היא ואין במעת לעת טומאה כשאין בו דעת לישאל ואומרים שיש לומר שאם נגע בה אדם תולין בו דהא יש בו דעת לישאל בנוגע אעפ"י שאין דעת לישאל בטומאת'.\par עי"ל דאיכא שוטה שיודעת לישאל אם נגעה אם לאו ואין לה הרגש' בראית דמים ואנן שוטה סתם פרכינן לעשותה כשאר הטמאות כשיש בענין דעת לישאל וכשאין בו. }
וכי תימא הא כל ספק טומאה ברה"ר טהור כיון דאיכא תרתי לריעותא כודאי טומאה דמי 
\commenta{\textbf{והאיכא כתמים לימא תנן כתמים דלא כשמאי.} פירש"י ז"ל האיכא כתמים דקי"ל דמטמא' למפרע וכו' ולאו דוקא פירכא משום למפרע דהאיכא ר"ש בן אלעזר דאמר בסוף בא סימן דכתם אינו מטמא למפרע כלל שלא יהא כתמה חמור מראיתה אלא פירכא משום מכאן ולהבא היא דלא אשכחן תנא דמטהר בכתמים להבא ולשמאי דאמר אשה מרגש' ולא מחמירין עליה אפילו בדרבנן טהורה היא בכתמי' לגמרי דלא מגופה הוא הואיל ולא ארגישה ומפרקינן מודה שמאי בכתמים אפילו בלמפרע כרבנן מאי טעמא וכו'. }
ורבי שמעון סבר כי סוטה מה סוטה ברה"ר טהור הכא נמי ברה"ר טהור 
\commenta{\textbf{משמשת במוך מא"ל.} פירש"י ז"ל ג' נשים ולא דוקא דהא ארבע נשים דיין שעתן במתניתין אלא כל אשה שמשמשת במוך. }
אי מסוטה אימא כי סוטה מה סוטה ברה"י טמאה ודאי הכא נמי ברה"י טמאה ודאי
הכי השתא התם יש רגלים לדבר שהרי קינא לה ונסתרה הכא מאי רגלים לדבר איכא 
ואי בעית אימא היינו טעמא דרבי שמעון גמר סוף טומאה מתחלת טומאה 
מה תחלת טומאה ספק נגע ספק לא נגע ברה"ר טהור אף סוף טומאה ספק טבל ספק לא טבל ברה"ר טהור 
ורבנן הכי השתא התם גברא בחזקת טהרה קאי מספקא לא מחתינן ליה לטומאה הכא גברא בחזקת טומאה קאי מספקא לא מפקינן ליה מטומאתו 
ומאי שנא ממבוי דתנן השרץ שנמצא במבוי מטמא למפרע עד שיאמר בדקתי את המבוי הזה ולא היה בו שרץ או עד שעת הכיבוד 
התם נמי כיון דאיכא שרצים דגופיה ושרצים דאתו מעלמא כתרתי לריעותא דמי 
ואב"א היינו טעמא דשמאי הואיל ואשה מרגשת בעצמה והלל כסבורה הרגשת מי רגלים היא 
ולשמאי האיכא ישנה ישנה נמי אגב צערה מיתערא מידי דהוה אהרגשת מי רגלים 
והאיכא שוטה מודה שמאי בשוטה הא כל הנשים קתני כל הנשים פקחות 
וליתני נשים לאפוקי מדרבי אליעזר דא"ר אליעזר ארבע נשים ותו לא קמ"ל כל הנשים 
והאיכא כתמים לימא תנן כתמים דלא כשמאי אמר אביי מודה שמאי בכתמים מ"ט בצפור לא נתעסקה בשוק של טבחים לא עברה האי דם מהיכא אתי 
אי בעית אימא היינו טעמיה דשמאי דאם איתא דהוה דם מעיקרא הוה אתי והלל כותלי בית הרחם העמידוהו ושמאי כותלי בית הרחם לא מוקמי דם 
משמשת במוך מאי איכא למימר אמר אביי מודה שמאי במשמשת במוך 
רבא אמר מוך נמי אגב זיעה מכויץ כויץ ומודה רבא במוך דחוק 
ומאי איכא בין הני לישני להאיך לישנא
איכא בינייהו למרמי חבית ומקוה ומבוי להאיך לישנא איכא למרמינהו להני לישני ליכא למרמי 
\commenta{\textbf{אי אתה מודה בקופה שנשתמשו בה טהרות וכו'.} איכא למידק להלל גופיה קשיא טומאה ודאי דאלו קופה טומאה ודאי דקתני טמאות וכדמוכחא נמי שמעתין לקמן ואלו מעל"ע תולין. א"ל להלל גופיה זו יש לה שולים וזו אין לה שולים אלא ה"ק ליה כיון דכשיש לה שולי' טמאות ודאי דין הוא לתלות באשה מפני שהיא כמי שיש לה אוגנים. ואפילו לחזקיה דאמר התם טהורות שאני פירי דלא שרקי וקפיד עלייהו.\par וכן אתה מפרש ללשון שאמרו בקופה שאינה בדוקה א"נ שהיא מכוסה שלא בא הלל להשוות אשה לשאינה בדוקה ולשאילה מכוסה אלא שמאחר שבאלו טמאות ודאי באשה היה לנו לתלות מפני שהיא דומה במקצת לשאינה בדוקה ואינה מכוסה משום דשכיחי בה דמים. }
ומאי איכא בין האי לישנא להאיך לישנא לאביי איכא מוך
לרבא איכא מוך דחוק 
תניא כי האי לישנא דאם איתא דהוה דם מעיקרא הוה אתי אמר לו הלל לשמאי אי אתה מודה בקופה שנשתמשו בה טהרות בזוית זו ונמצא שרץ בזוית אחרת שטהרות הראשונות טמאות אמר לו אבל 
ומה הפרש בין זו לזו לזו יש לה שולים לזו אין לה שולים 
רבא אמר טעמא דשמאי משום בטול פריה ורביה תניא נמי הכי אמר לו שמאי להלל א"כ בטלת בנות ישראל מפריה ורביה 
ומאן דתני האי לישנא הא תניא כי האיך לישנא דאם איתא דהוה דם מעיקרא הוה אתי התם הלל הוא דקטעי הוא סבר טעמא דשמאי דאם איתא דהוה דם מעיקרא הוה אתי וקא מקשי ליה קופה 
ואמר ליה שמאי טעמא דידי משום בטול פריה ורביה ולמאי דקטעית נמי דקמקשית קופה לזו יש לה שולים ולזו אין לה שולים 
ולמאן דתני האי לישנא הא תניא כי האיך לישנא משום בטול פריה ורביה 
הכי קאמר ליה הלל לשמאי אין טעמא קאמרת דאם איתא דהוה דם מעיקרא הוה אתי ומיהו עשה סייג לדבריך דמאי שנא מכל התורה כולה דעבדינן סייג 
אמר ליה א"כ בטלת בנות ישראל מפריה ורביה והלל מפריה ורביה מי קאמינא לטהרות הוא דקאמינא 
ושמאי לטהרות נמי לא דאם כן לבו נוקפו ופורש 
(שולי"ם בדוקי"ן מכוסי"ן בזוי"ת סימ"ן) איתמר קופה שנשתמשו בה טהרות בזוית זו ונמצא שרץ בזוית אחרת חזקיה אמר טהרות הראשונות טהורות רבי יוחנן אמר טהרות הראשונות טמאות והא (בית) שמאי והלל מודו בקופה דטהרות הראשונות טמאות 
כי מודו שמאי והלל בקופה שיש לה שולים כי פליגי חזקיה ורבי יוחנן בקופה שאין לה שולים 
אין לה שולים מ"ט דר' יוחנן אין לה שולים ויש לה אוגנים 
והתניא המדלה עשרה דליים מים בזה אחר זה ונמצא שרץ באחד מהן הוא טמא וכולן טהורין ואמר ריש לקיש משום רבי ינאי לא שנו אלא שאין לה אוגנים אבל יש לה אוגנים כולן טמאין 
לימא חזקיה לית ליה דר' ינאי מיא שרקי פירי לא שרקי 
אי נמי מיא לא קפיד עלייהו פירי קפיד עלייהו 
ואי בעית אימא כי מודו שמאי והלל בקופה שאינה בדוקה}

\newsection{דף ד}
\twocol{כי פליגי חזקיה ור' יוחנן בקופה בדוקה מר סבר הא בדקה ומר סבר אימור עם סילוק ידו נפל 
והא דומיא דאשה קתני ואשה בדוקה היא כיון דשכיחי בה דמים כשאינה בדוקה דמיא 
ואיבעית אימא כי מודו שמאי והלל בקופה שאינה מכוסה כי פליגי חזקיה ורבי יוחנן בקופה מכוסה מכוסה היכי נפל כגון שתשמישה ע"י כסוי 
והא דומיא דאשה קתני ואשה מכוסה היא כיון דשכיחי בה דמים כשאין מכוסה דמיא 
ואיבעית אימא כי מודו שמאי והלל בזוית קופה כי פליגי חזקיה ורבי יוחנן בזוית בית והא קופה קאמר 
ה"ק קופה שנשתמשו בה טהרות בזוית בית זו וטלטלוה בזוית אחרת ונמצא שרץ בזוית אחרת חזקיה סבר לא מחזקינן טומאה ממקום למקום ורבי יוחנן סבר מחזקינן 
ומי מחזקינן והתנן נגע באחד בלילה ואינו יודע אם חי אם מת ולמחר השכים ומצאו מת ר"מ מטהר
וחכמים מטמאין שכל הטמאות כשעת מציאתן 
ותני עלה כשעת מציאתן ובמקום מציאתן 
וכי תימא הני מילי לשרוף אבל לתלות תלינן ומי תלינן 
והתנן מחט שנמצאת מלאה חלודה או שבורה טהורה שכל הטמאות כשעת מציאתן ואמאי לימא האי מעיקרא מחט מעלייתא היא והשתא הוא דהעלה חלודה 
ועוד תנן מצא שרץ שרוף על גבי הזיתים וכן מטלית המהומהם טהור שכל הטמאות כשעת מציאתן 
וכי תימא כשעת מציאתן בין לקולא בין לחומרא ובמקום מציאתן אבל שלא במקום מציאתן משרף לא שרפינן מתלא תלינן
והתנן ככר ע"ג הדף ומדף טמא מונח תחתיו אע"פ שאם נפלה א"א אלא א"כ נגעה טהורה שאני אומר אדם טהור נכנס לשם ונטלה
עד שיאמר ברי לי שלא נכנס אדם שם וא"ר אלעזר לא נצרכה אלא למקום מדרון 
התם כדקתני טעמא
שאני אומר אדם טהור נכנס לשם ונטלה 
הכא נמי נימא עורב אתא ושדא אדם דבכונה אמרינן עורב דשלא בכונה לא אמרינן 
מכדי האי ככר ספק טומאה ברה"י הוא וכל ספק טומאה ברה"י ספקה טמא משום דהוי דבר שאין בו דעת לישאל וכל דבר שאין בו דעת לישאל בין ברה"ר בין ברשות היחיד ספקו טהור 
ואב"א הכא בטומאה דרבנן דיקא נמי דקתני מדף כדכתיב (ויקרא כו:לו) עלה נדף
וחכ"א לא כדברי זה כו' ת"ר וחכ"א לא כדברי זה ולא כדברי זה לא כדברי שמאי שלא עשה סייג לדבריו ולא כדברי הלל שהפריז על מדותיו 
אלא מעת לעת ממעטת על יד מפקידה לפקידה ומפקידה לפקידה ממעטת על יד מעת לעת 
מעת לעת ממעטת על יד מפקידה לפקידה כיצד בדקה עצמה באחד בשבת ומצאת טהורה וישבה שני ושלישי ולא בדקה ולרביעי בדקה ומצאה טמאה אין אומרים תטמא מפקידה לפקידה אלא מעת לעת 
ומפקידה לפקידה ממעטת על יד מעת לעת כיצד בדקה עצמה בשעה ראשונה ומצאת טהורה וישבה לה שניה ושלישית ולא בדקה ולרביעית בדקה ומצאה טמאה אין אומרים תטמא מעת לעת אלא מפקידה לפקידה 
פשיטא כיון דבדקה עצמה בשעה ראשונה ומצאת טהורה לא מטמינן לה מעת לעת איידי דתנא מעל"ע ממעטת על יד מפקידה לפקידה תנא נמי מפקידה לפקידה ממעטת על יד מעת לעת 
אמר רבה מאי טעמייהו דרבנן אשה מרגשת בעצמה א"ל אביי אם כן תהא דיה שעתה ורבה לחדודי לאביי הוא דבעי אלא מאי טעמייהו דרבנן 
כי הא דאמר רב יהודה אמר שמואל חכמים תקנו להן לבנות ישראל שיהו בודקות עצמן שחרית וערבית שחרית להכשיר טהרות של לילה וערבית להכשיר טהרות של יום 
וזו הואיל ולא בדקה הפסידה עונה מאי עונה עונה יתירה 
א"ל רב פפא לרבא והא זימנין משכחת לה ג' עונות במעת לעת השוו חכמים מדותיהן שלא תחלוק במעת לעת 
איבעית אימא שלא יהא חוטא נשכר 
מאי בינייהו איכא בינייהו דאתניסה ולא בדקה
כל אשה שיש לה וסת [וכו'] לימא מתני' ר' דוסא היא ולא רבנן דתניא ר"א אומר ארבע נשים דיין שעתן בתולה מעוברת מניקה וזקנה ר' דוסא אומר כל אשה שיש לה וסת דיה שעתה 
אפילו תימא רבנן עד כאן לא פליגי רבנן עליה דר' דוסא אלא שלא בשעת וסתה אבל בשעת וסתה מודו ליה ומתניתין בשעת וסתה ודברי הכל 
מכלל דר' דוסא אפילו שלא בשעת וסתה אמר מאן תנא להא דת"ר אשה שיש לה וסת כתמה טמא למפרע שאם תראה שלא בשעת וסתה מטמאה מעת לעת
נימא רבנן היא ולא רבי דוסא ... אפילו תימא רבי דוסא ע"כ לא פליג רבי דוסא עלייהו דרבנן אלא בשעת וסתה אבל שלא בשעת וסתה מודי להו ומתניתין בשעת וסתה ור' דוסא היא}

\newsection{דף ה}
\twocol{וברייתא דברי הכל 
\commenta{\textbf{השתא מעת לעת ממעטה מפקידה לפקידה מיבעיא.} פי' לדברי חכמים לעולם מפקידה לפקידה זמנה מועט מעת לעת שכבר מיעט מעל"ע ע"י הפקידה וכיון שעד זה ממעט על יד מעת לעת אם לא בדקה מאתמול כ"ש שדינו למעט על יד הפקידה אם בדקה עצמה היום בשעה ראשונה ולרביעית שמשה דכיון שהזמן ביניהם מועט אין לחוש כ"כ בשעת פקידה לומר עם סלוק ידיה ראתה. ופריק סד"א עד זה לא תמעט אלא על יד מעל"ע אע"פ שהחששא שלו יותר קרובה מפני הפסד טהרות הקל. אבל על יד פקידה לא ימעט דליחוש שמא תחפנו שכבת זרע קמ"ל. }
ולוקמא איפכא 
כיון דאיכא לאוקומי לקולא ולחומרא לחומרא מוקמינן 
קתני שאם תראה שלא בשעת וסתה מטמאה מעת לעת טעמא דאשה שיש לה וסת הוא דפליגי רבנן בין כתמה לראייתה
הא שאר נשים שאמרו חכמים דיין שעתן כתמן כראייתן 
מני רבי חנינא בן אנטיגנוס היא דאמר רב יהודה אמר שמואל משום רבי חנינא בן אנטיגנוס כל הנשים כתמן טמא למפרע ונשים שאמרו חכמים דיין שעתן כתמן כראייתן חוץ מתינוקת שלא הגיע זמנה לראות שאפילו סדינין שלה מלוכלכין בדם אין חוששין לה 
ומי אית ליה לרבי חנינא כתם כלל והתניא כל הנשים כתמן טמא ונשים שאמרו חכמים דיין שעתן כתמן טמא ר' חנינא בן אנטיגנוס אומר נשים שאמרו חכמים דיין שעתן אין להן כתם מאי לאו אין להן כתם כלל לא אין להן כתם למפרע אבל יש להן כתם מכאן ולהבא 
מכלל דתנא קמא סבר אפי' למפרע אין ר"מ היא דמחמיר גבי כתמים דתניא כל הנשים כתמן טמא למפרע ונשים שאמרו חכמים דיין שעתן כתמן טמא למפרע דר"מ 
רבי חנינא בן אנטיגנוס אומר נשים שאמרו חכמים דיין שעתן כתמן כראייתן ותינוקת שהגיע זמנה לראות יש לה כתם ושלא הגיע זמנה לראות אין לה כתם ואימתי הגיע זמנה לראות משהגיעו ימי הנעורים
והמשמשת בעדים כו' אמר רב יהודה אמר שמואל עד שלפני תשמיש אינו ממעט כפקידה 
מ"ט אמר רב קטינא מתוך שמהומה לביתה וכי מהומה לביתה מאי הוי מתוך שמהומה לביתה אינה מכנסת לחורין ולסדקין 
תנן המשמשת בעדים הרי זו כפקידה מאי לאו חד לפני תשמיש וחד לאחר תשמיש לא אידי ואידי לאחר תשמיש ואחד לו ואחד לה כדתנן דרך בנות ישראל משמשות בשני עדים אחד לו ואחד לה 
האי מאי אי אמרת בשלמא חד לפני תשמיש וחד לאחר תשמיש איצטריך סד"א מתוך שמהומה לביתה לא בדקה שפיר קמ"ל הרי זו כפקידה אלא אי אמרת אידי ואידי לאחר תשמיש פשיטא 
מהו דתימא שמא תראה טפת דם כחרדל ותחפנה שכבת זרע קמ"ל 
ואיבעית אימא שתי בדיקות אצרכוה רבנן חד לפני תשמיש וחד לאחר תשמיש וכי קתני הרי זו כפקיד' אלאחר תשמיש והא המשמשת קתני תני ומשמשת
ממעטת על יד מעת לעת השתא מעת לעת ממעטת
מפקידה לפקידה מיבעיא 
\commenta{\textbf{מכדי האי מטה דבר שאין בו דעת לישאל הוא.} פירש מדקא מקשינן הכי אלמא ברשות היחיד בלחוד הוא דגזור רבנן במעת לעת בספק טומאה שאלו ברשות הרבים אינו חלוק בין בדבר שיש בו דעת לשאין בו דעת דלעולם ספיקו טהור וכמו שפירשתי במשנה.\par ולפיכך הקשו חכמי הצרפתים היכי תרגימנא בשחברותיה נושאו' אותה אם כן הויין לה אינהו תרתי ואיהי חדא הא תלתא ה"ל רשות הרבים וספיקו טהור דהכי אמרינן בגמרא בריש פרק שני נזירים.\par ויש מי שתירץ אין רשות הרבים אלא בשלשה אנשים אבל נשים אפילו מאה נמי כאחד דמיין ורשות היחיד הוא מאי טעמא דגמרינן מסוטה מה סוטה אין סתירתה אלא באיש אחד אבל ב' אנשים והיא לא סתירה היא שהרי אשה אחת מתייחדת עם ב' אנשים א) אף רשות היחיד בלא שלשה אנשים אבל נשים אפילו עשר נשים אין אדם מתיחד הילכך הויא לה סתירה ורשות היחיד היא, וזה אינו כלום.\par ואחרים העמידוה לזו כשהיא ישנה בכילה במטה וחלקה רשות לעצמה ולי נראה שזו היא טמאה ודאי ואין הספק בה אלא שהיא מטמאה אינה נחשבת בכלל המנין אלא הרי היא כגוף השרץ. }
מהו דתימא מעת לעת חשו בה רבנן לפסידא דטהרות אבל מפקידה לפקידה לא קמ"ל
\commenta{\textbf{היה מתעטף בטליתו וטהרות וטמאות בצדו טהרות וטמאות למעלה מראשו.} יש מפרשים כגון שהוא וטליתו טהורים וטמאות בצדן שראויין לטמא בגדים כגון משכב ומושב ושאר אבות הטומאות ספק נגע טליתו בטמאות ונטמא ונגע בטהרות ונטמאו או ספק לא נגע הטלית לא בזה ולא בזה ספיקו טהור בין בטהרות שהן שתי ספיקות בין בטלית שאינן אלא ספיקא חד. ומסקנא ברה"י ספיקו טמא בשתיהן שהרי שנינו כל שאתה יכול לרבות ספיקו' וספק ספקו' ברה"י ספקו טמא אבל ברה"ר ספקו טהור אפילו הטלית שאין אלא ספק אחד.\par ואחרים פירשו דאו או קתני היה הוא טהור וטמאות בצדו ולמעלה מראשו או שהיה הוא טמא וטהרות בצדו ולמעלה מראשו וכך פי' רש"י ז"ל. }
כיצד דיה שעתה וכו' למה לי למיתני היתה יושבת במטה ועסקה בטהרות ליתני היתה עסוקה בטהרות ופרשה וראתה 
\commenta{והא דקתני \textbf{ואם א"א לו אלא א"כ נגע טמא.} לאו דוקא א"א שא"כ האיך אמר רשב"ג אומרים לו שנה והלא א"א וא"ת רשב"ג ארישא פליג, א"כ ה"ל רבנן לקולא ואיהו לחומרא ואנן איפכא אמרינן לקמן במכילתין דכי אמרי רבנן אין שונין בטהרות לחומרא אבל לקולא שונין אלא ודאי רשב"ג אסיפא פליג דה"ל רבנן לחומרא ולפיכך אמרו אין שונין וא"א לאו דוקא אלא שהדבר קרוב הרבה ליגע ורחוק שלא ליגע ובכיוצא בזה א"א דלאו דוקא לגמרי בפרק כיצד העדים.\par ומה שכתב רש"י ז"ל אין שונין חוששין שמא עכשו נגע ובתחלה לא או חלוף לאו דוקא דא"כ אפילו לחומר' אלא חוששין שמא עכשיו לא נגע ובתחלה נגע ולא חלוף. }
הא קמ"ל טעמא דדיה שעתה הא מעת לעת מטה נמי מטמיא מסייע ליה לזעירי דאמר זעירי מעת לעת שבנדה עושה משכב ומושב לטמא אדם לטמא בגדים 
\commenta{\textbf{ומה כלי חרס המוקף צמיד פתיל וכו'.} הקשו בתוספות ונימא דיו לבא מן הדין להיות כנדון מהיכא מייתית ליה מכלי חרס מה כלי חרס אינו מטמא אדם לטמא בגדים אף משכב ומושב לא יטמא אדם לטמא בגדים. ולאו קושיא היא דאנן הכי קאמרינן ומה כלי חרס שטומאתו מועט' שהוא ניצל באה' המת גזרו על מעת לעת שלו כנד' עצמה משכבות ומושבות שטומאתן מרובה לכ"ש שנעשו מעת לעת שבנדה.\par ועוד הקשו דנימא פכים קטנים יוכיח שטמאים במת ואין מטמאים במעל"ע שבנדה כדאמרינן בבבא קמא ופירש רש"י ז"ל שהוא של חרס ואי אפשר ליגע בתוכן ואעפ"י שאפשר בהיסט להכי אפקיה רחמנא להיסט בלשון נגיעה לומר שכל שאי אפשר להטמאות בנגיעה אינו מטמא בהיסט, גם זו אינה קושיא דמה לפכין קטנים שהן טהורין בנדה עצמה תאמר במשכבו' ומושבות דכיון שמטמאין בנדה עצמה עשו מעת לעת כמוה דאשכח' בכלי חרס מוקף צמיד פתיל כ"ש לדעת הגאונים שהן מפרשים פכין קטנים שאינן ראויין לישיבה וטהורין במדרס הזב אבל מן ההיסט אין לך ניצל מהן ולא ממגע תוך כגון בשערו רוקו ומשקה הזב והזבה. }
מכדי האי מטה דבר שאין בו דעת לישאל הוא וכל דבר שאין בו דעת לישאל ספקו טהור תרגמה זעירי כשחברותיה נושאות אותה במטה דהויא ליה יד חברותיה 
והשתא דא"ר יוחנן ספק טומאה הבאה בידי אדם נשאלין עליה אפי' בכלי מונח ע"ג קרקע כמי שיש בו דעת לישאל אע"פ שאין חברותיה נושאות אותה במטה 
גופא א"ר יוחנן ספק טומאה הבאה בידי אדם נשאלים עליה אפי' בכלי המונח על גבי קרקע כמי שיש בו דעת לישאל 
מיתיבי היה מתעטף בטליתו וטהרות וטומאות בצדו וטהרות וטומאות למעלה מראשו ספק נגע ספק לא נגע טהור ואם אי אפשר אא"כ נגע טמא 
רשב"ג אומר אומרים לו שנה ושונה אמרו לו אין שונים בטהרות
אמאי הא ספק טומאה הבאה בידי אדם הוא 
בר מיניה דההיא דתני רב הושעיא  ברשות היחיד ספקו טמא ברשות הרבים טהור 
גופא אמר זעירי מעת לעת שבנדה עושה משכב ומושב לטמא אדם לטמא בגדים 
איני והא כי אתא אבימי מבי חוזאי אתא ואייתי מתניתא בידיה מעת לעת שבנדה משכבה ומושבה כמגעה מאי לאו מה מגעה לא מטמא אדם אף משכבה לא מטמא אדם 
אמר רבא ותסברא קל וחומר הוא ומה כלי חרס המוקף צמיד פתיל הניצול באוהל המת אינו ניצול במעת לעת שבנדה משכבות ומושבות שאינן ניצולין באהל המת אינו דין שאין ניצולין במעת לעת שבנדה 
והא אבימי מבי חוזאי מתניתא קאמר אימא משכבה ומושבה}

\newsection{דף ו}
\twocol{כמגע עצמה מה מגע עצמה מטמא אדם לטמא בגדים אף משכבה ומושבה מטמא אדם לטמא בגדים 
תניא כוותיה דרבא הרואה דם מטמאה מעת לעת ומה היא מטמאה משכבות ומושבות אוכלין ומשקין וכלי חרס המוקף צמיד פתיל ואינה מקולקלת למנינה ואינה מטמאה את בועלה למפרע ר' עקיבא אומר מטמאה את בועלה ואינה מונה אלא משעה שראתה 
הרואה כתם מטמאה למפרע ומה היא מטמאה אוכלין ומשקין משכבות ומושבות וכלי חרס המוקף צמיד פתיל ומקולקלת למנינה ומטמאה את בועלה ואינה מונה אלא משעה שראתה 
וזה וזה תולין לא אוכלין ולא שורפין 
ורבא אי שמיע ליה מתניתא לימא מתניתא ואי לא שמיע ליה מתניתא קל וחומר מנא ליה 
לעולם שמיע ליה מתניתא ואי ממתניתא הוה אמינא או אדם או בגדים אבל אדם ובגדים לא משום הכי קאמר ק"ו 
אמר רב הונא מעת לעת שבנדה לקדש אבל לא לתרומה אי הכי ליתני גבי מעלות כי קתני היכא דאית ליה דררא דטומאה אבל היכא דלית ליה דררא דטומאה לא קתני 
מיתיבי מה היא מטמאה אוכלין ומשקין מאי לאו בין דקדש בין דתרומה לא דקדש 
תא שמע רבי יהודה אומר אף בשעת עברתן מלאכול בתרומה והוינן בה מאי דהוה הוה 
אמר רב חסדא לא נצרכה אלא לתקן שירים שבפניה 
רב הונא מתני לישרוף שירים שבידיה שבדקה עצמה כשיעור וסת 
ת"ש מעשה ועשה רבי כר"א
לאחר שנזכר אמר כדי הוא ר"א לסמוך עליו
בשעת הדחק והוינן בה מאי לאחר שנזכר אילימא לאחר שנזכר דאין הלכה כרבי אליעזר אלא כרבנן בשעת הדחק היכי עביד כותיה 
אלא (לאו) דלא איתמר הלכתא לא כמר ולא כמר וכיון שנזכר דלאו יחיד פליג עליה אלא רבים פליגי עליה אמר כדי הוא רבי אליעזר לסמוך עליו בשעת הדחק 
אי אמרת בשלמא לתרומה היינו דהואי תרומה בימי רבי אלא אי אמרת לקדש קדש בימי רבי מי הואי 
כדעולא דאמר עולא חבריא מדכן בגלילא הכא נמי בימי רבי 
ת"ש מעשה בשפחתו של רבן גמליאל שהיתה אופה ככרות של תרומה ובין כל אחת ואחת מדיחה ידה במים ובודקת באחרונה בדקה ומצאה טמאה ובאת ושאלה את רבן גמליאל ואמר לה כולן טמאות אמרה לו רבי והלא בדיקה היתה לי בין כל אחת ואחת אמר לה א"כ היא טמאה וכולן טהורות 
קתני מיהת ככרות של תרומה מאי תרומה תרומת לחמי תודה תרומת לחמי תודה באפיה מאי בעיא 
דאפרשינהו בלישייהו וכי הא דאמר רב טובי בר רב קטינא לחמי תודה שאפאן ד' חלות יצא והוינן בה והא בעינן ארבעים למצוה 
והא בעינן אפרושי תרומה מינייהו וכי תימא דמפריש פרוסה מכל חד וחד אחד אמר רחמנא שלא יטול פרוסה ואמרינן דאפרשינהו בלישייהו הכא נמי דאפרשינהו בלישייהו 
ת"ש שוב מעשה בשפחה של רבן גמליאל שהיתה גפה חביות של יין ובין כל אחת ואחת מדיחה ידיה במים ובודקת ובאחרונה בדקה ומצאה טמאה ובאת ושאלה לרבן גמליאל ואמר לה כולן טמאות אמרה לו והלא בדיקה היתה לי בין כל אחת ואחת אמר לה אם כן היא טמאה וכולן טהורות 
אי אמרת בשלמא חדא דקדש וחדא דתרומה היא היינו דהדרה ושיילה אלא אי אמרת אידי ואידי דקדש למה לה למהדר ולשייליה מעשה שהיה בשתי שפחות היה 
לישנא אחרינא אמרי לה אמר רב הונא מעת לעת שבנדה מטמאה בין לקדש ובין לתרומה ממאי מדלא קתני לה גבי מעלות א"ל רב נחמן והא תני תנא לקדש אבל לא לתרומה 
קבלה מיניה רב שמואל בר רב יצחק בחולין שנעשו על טהרת קדש ולא בחולין שנעשו על טהרת תרומה 
תנן התם נולד לה ספק טומאה עד שלא גלגלה תעשה בטומאה משגלגלה תעשה בטהרה 
עד שלא גלגלה תעשה בטומאה חולין נינהו ומותר לגרום טומאה לחולין שבארץ ישראל משגלגלה תעשה בטהרה חולין הטבולין לחלה כחלה דמו ואסור לגרום טומאה לחלה 
תנא}

\newsection{דף ז}
\twocol{וחלתה תלויה לא אוכלין ולא שורפין באיזה ספק אמרו בספק חלה מאי ספק חלה 
אביי ורבא דאמרי תרוייהו שלא תאמר בהוכחות שנינו כמו שני שבילין דהתם חולין גרידא נמי מטמו
אלא בנשען דתנן זב וטהור שהיו פורקין מן החמור או טוענין בזמן שמשאן כבד טמא משאן קל טהור וכולן טהורין לבני הכנסת וטמאין לתרומה 
וחולין הטבולין לחלה כחלה דמו והתניא אשה שהיא טבולת יום לשה את העיסה וקוצה הימנה חלתה ומניחתה בכפישה או באנחותא ומקפת וקורא לה שם
מפני שהוא שלישי ושלישי טהור בחולין ואי אמרת חולין הטבולין לחלה כחלה דמו הא טמיתנהו 
אמר אביי כל שודאי מטמא חולין גזרו על ספקו משום חולין הטבולין לחלה והאי טבול יום כיון דלא מטמא ודאי חולין לא גזרו עליו משום חולין הטבולין לחלה 
והא מעת לעת שבנדה דודאי מטמא חולין ולא גזרו על ספקה משום חולין הטבולין לחלה 
דאמר מר קבלה מיניה רב שמואל בר רב יצחק בחולין שנעשו על טהרת קדש ולא בחולין שנעשו על טהרת תרומה 
התם לא פתיכא בהו תרומה הכא פתיכא בהו תרומה 
ואיבעית אימא הנח מעת לעת דרבנן
{\large\emph{מתני׳}} רבי אליעזר אומר ארבע נשים דיין שעתן בתולה מעוברת מניקה וזקינה אמר רבי יהושע אני לא שמעתי אלא בתולה
אבל הלכה כרבי אליעזר 
איזו היא בתולה כל שלא ראתה דם מימיה אע"פ שנשואה מעוברת משיודע עוברה מניקה עד שתגמול את בנה נתנה בנה למניקה גמלתו או מת ר"מ אומר מטמאה מעת לעת וחכ"א דיה שעתה 
איזוהי זקנה כל שעברו עליה שלש עונות סמוך לזקנתה רבי אליעזר אומר כל אשה שעברו עליה שלש עונות דיה שעתה רבי יוסי אומר מעוברת ומניקה שעברו עליהן שלש עונות דיין שעתן 
ובמה אמר דיה שעתה בראייה ראשונה אבל בשניה מטמאה מעת לעת ואם ראתה הראשונה מאונס אף השניה דיה שעתה
{\large\emph{גמ׳}} תניא אמר לו רבי אליעזר לרבי יהושע אתה לא שמעת אני שמעתי אתה לא שמעת אלא אחת ואני שמעתי הרבה 
אין אומרים למי שלא ראה את החדש יבא ויעיד אלא למי שראהו כל ימיו של רבי אליעזר היו עושין כרבי יהושע לאחר פטירתו של רבי אליעזר החזיר רבי יהושע את הדבר ליושנו 
כרבי אליעזר בחייו מ"ט לא משום דרבי אליעזר שמותי הוא וסבר אי עבדינן כוותיה בחדא עבדינן כוותיה באחרנייתא
ומשום כבודו דר"א לא מצינן מחינן בהו לאחר פטירתו של ר"א דמצינו מחינן בהו החזיר את הדבר ליושנו 
אמר רב יהודה אמר שמואל הלכה כרבי אליעזר בארבע חדא דאמרן 
ואידך המקשה כמה תשפה ותהא זבה מעת לעת דברי ר"א והלכה כדבריו 
ואידך הזב והזבה שבדקו עצמן יום ראשון ומצאו טהור יום שביעי ומצאו טהור ושאר הימים לא בדקו רבי אליעזר אומר הרי אלו בחזקת טהרה רבי יהושע אומר אין להן אלא יום הראשון ויום השביעי בלבד 
רבי עקיבא אומר אין להם אלא יום שביעי בלבד ותניא ר"ש ורבי יוסי אומרים נראין דברי רבי אליעזר מדברי רבי יהושע ודברי ר"ע מדברי כולן אבל הלכה כר' אליעזר 
ואידך דתנן אחורי כלים שנטמאו במשקין ר' אליעזר אומר מטמאין את המשקין ואין פוסלין את האוכלין מטמאין את המשקין ואפילו דחולין ואין פוסלין את האוכלין ואפילו דתרומה רבי יהושע אומר מטמאין את המשקין ופוסלין את האוכלין 
א"ר יהושע ק"ו ומה טבול יום שאין מטמא משקה חולין פוסל אוכלי תרומה אחורי כלים שמטמא משקה חולין אינו דין שפוסל אוכלי תרומה 
ורבי אליעזר אחורי כלים דרבנן וטבול יום דאורייתא ורבנן מדאורייתא לא עבדינן קל וחומר דמדאורייתא אין אוכל מטמא כלי ואין משקה מטמא כלי
ורבנן הוא דגזור גזרה משום משקין דזב וזבה משקין דעלולין לקבל טומאה גזרו בהו רבנן אוכלין דאין עלולין לקבל טומאה לא גזרו בהו רבנן 
ומאי שנא אחורי כלים דנקט משום דקילי דתנן כלי שנטמא מאחוריו במשקין אחוריו טמא תוכו אזנו אוגנו ידיו טהורין נטמא תוכו כולו טמא 
מאי קמ"ל שמואל בכולהו תנן הלכתא 
וכי תימא אחורי כלים קמ"ל דלא תנן ולימא הלכה כר"א באחורי כלים אלא הא קמ"ל שאין למדין הלכה מפי תלמוד 
ותו ליכא והאיכא דתנן ר"א אומר}

\newsection{דף ח}
\twocol{מלמדין את הקטנה שתמאן בו ואמר רב יהודה אמר שמואל הלכה כר' אליעזר כי אמר שמואל הלכה כר' אליעזר בד' בסדר טהרות אבל בשאר סדרים איכא טובא 
וה"נ מסתברא דתנן ר' אליעזר אומר אף הרודה ונותן לסל הסל מצרפן לחלה ואמר רב יהודה אמר שמואל הלכה כר"א ש"מ 
ומאי אולמיה דהאי מהאי משום דקאי רבי אלעזר כותיה 
דתנן רבי אלעזר אומר מלמדין את הקטנה שתמאן בו 
ומי קאי והא אצרכו מצרכינן להו ולא דמיין להדדי אלא משום דקאי רבי יהודה בן בבא כותיה 
דתניא רבי יהודה בן בבא העיד ה' דברים שממאנים את הקטנות
ושמשיאין את האשה ע"פ עד אחד ושנסקל תרנגול בירושלים על שהרג את הנפש ועל יין בן מ' יום שנתנסך ע"ג המזבח ועל תמיד של שחר שקרב בד' שעות 
מאי קטנות לאו חדא דר' אלעזר וחד דר' אליעזר לא מאי קטנות קטנות דעלמא 
אי הכי גבי אשה נמי נתני נשים ונימא נשים דעלמא אלא מדהכא קתני אשה והכא קתני קטנות ש"מ דוקא קתני ש"מ 
וכן א"ר אלעזר הלכה כר"א בד' ותו ליכא והתנן רבי אליעזר אומר מלמדין את הקטנה שתמאן בו וא"ר אלעזר הלכה כר"א וכי תימא כי א"ר אלעזר הלכה כר"א בד' בסדר טהרות אבל בשאר סדרי איכא ומי איכא 
והתנן הורד והכופר והלטום והקטף יש להן שביעית ולדמיהן שביעית יש להן ביעור ולדמיהן ביעור וא"ר פדת מאן תנא קטפא פירא ר"א 
וא"ר זירא חזי דמינך ומאבוך קשריתו קטפא לעלמא את אמרת מאן תנא קטפא פירא ר"א ואבוך אמר הלכה כר"א בד' 
ואם איתא לימא ליה כי אמר אבא הלכה כר"א בד' בסדר טהרות אבל בשאר סדרי איכא 
אלא קשיא ההיא משום דקאי רבי אלעזר כותיה דתנן רבי אלעזר אומר מלמדים את הקטנה שתמאן בו 
ומי קאי והא אצרוכי מצרכינן להו ולא דמיין להדדי אלא משום דקאי רבי יהודה בן בבא כוותיה 
ותו ליכא והתנן ר"ע אומר אומרה ברכה רביעית בפני עצמה ר' אליעזר אומר אומרה בהודאה וא"ר אלעזר הלכה כר"א 
א"ר אבא ההוא דאמר משום רבי חנינא בן גמליאל דתניא רבי עקיבא אומר אומרה ברכה רביעית בפני עצמה רבי חנינא בן גמליאל אומר אומרה בהודאה
והא קשיש מיניה טובא אלא משום דקאי רבי חנינא בן גמליאל בשיטתיה 
\commenta{ הא דאמר רב ששת בריה דרב אידי \textbf{כי קתני מידי דתלי במעשה.} פירש אליבא דר"ש קסבר האי תנא בוגרת מותרת לכהן גדול א"נ נפקא מינה לכתובה ולא לכהן תניא ומוכת עץ מילתא דתלי במעשה הוא והאי דקתני כל זמן שלא נבעלה לאו דוקא אלא שלא נטלו בתוליה בין בעץ בין באדם. א"נ קסבר מוכת עץ מותרת לכהן גדול וכתובתה מאתים ומחלוקת היא ביבמות ובכתובות. וכן הא דאמר נ"מ לנחל איתן סבר לה כר' יאשיה אשר לא יעבד בו לשעבר ואיתא בפלוגתא בפ' עגלה ערופה. }
ומי קאי והתניא אור יוה"כ מתפלל שבע ומתודה שחרית מתפלל שבע ומתודה מוסף מתפלל שבע ומתודה מנחה מתפלל שבע ומתודה בנעילה מתפלל שבע ומתודה בערבית מתפלל שבע מעין שמנה עשרה 
רבי חנינא בן גמליאל משום אבותיו אומר מתפלל שמנה עשרה מפני שצ"ל הבדלה בחונן הדעת אמר ר"נ בר יצחק איהו אמר משום אבותיו וליה לא ס"ל 
א"ל ר' ירמיה לר' זירא ואת לא תסברא דמאן תנא קטפא פירא ר"א הוא והתנן ר"א אומר המעמיד בשרף ערלה אסור 
אפילו תימא רבנן ע"כ לא פליגי רבנן עליה דר"א אלא בקטפא דגווזא אבל בקטפא דפירא מודו ליה דתנן א"ר יהושע שמעתי בפירוש שהמעמיד בשרף העלין בשרף העיקרין מותר בשרף הפגין אסור מפני שהוא פרי 
ואיבעית אימא כי פליגי רבנן עליה דר"א באילן העושה פירות אבל באילן שאינו עושה פירות מודו דקטפו זהו פריו דתנן ר"ש אומר אין לקטף שביעית וחכ"א יש לקטף שביעית מפני שקטפו זהו פריו 
מאן חכמים לאו רבנן דפליגי עליה דר"א א"ל ההוא סבא הכי א"ר יוחנן מאן חכמים ר"א דאמר קטפו זהו פריו 
אי ר"א מאי איריא אילן שאינו עושה פרי אפילו אילן העושה פרי קטפו זהו פריו לדבריהם דרבנן קאמר להו לדידי אפי' אילן העושה פירות נמי קטפו זהו פריו לדידכו אודו לי מיהת באילן שאינו עושה פירות דקטפו זהו פריו ורבנן אמרי ליה לא שנא
איזו היא בתולה כל שלא ראתה כו' ת"ר נשאת וראתה דם מחמת נישואין ילדה וראתה דם מחמת לידה עדיין אני קורא לה בתולה שהרי בתולה שאמרו בתולת דמים ולא בתולת בתולים 
איני והאמר רב כהנא תנא ג' בתולות הן בתולת אדם בתולת קרקע בתולת שקמה בתולת אדם כל זמן שלא נבעלה נפקא מינה לכ"ג א"נ לכתובתה מאתים 
בתולת קרקע כ"ז שלא נעבדה נפקא מינה לנחל איתן א"נ למקח וממכר 
בתולת שקמה כ"ז שלא נקצצה נפקא מינה למקח וממכר אי נמי למקצצה בשביעית כדתנן אין קוצצין בתולת שקמה בשביעית מפני שהיא עבודה ואם איתא ליתני נמי הא 
אמר ר"נ בר יצחק כי קתני מידי דלית ליה שם לווי אבל מידי דאית ליה שם לווי לא קתני רב ששת בריה דרב אידי אמר כי קתני מידי דתלי במעשה מידי דלא תלי במעשה לא קתני 
רבי חנינא בריה דרב איקא אמר כי קתני מידי דלא הדר לברייתו מידי דהדר לברייתו לא קתני רבינא אמר כי קתני מידי דקפיד עליה זבינא מידי דלא קפיד עליה זבינא לא קתני 
ולא קפדי והתניא רבי חייא אומר כשם שהשאור יפה לעיסה כך דמים יפין לאשה ותניא משום ר"מ כל אשה שדמיה מרובין בניה מרובין אלא כי קתני מידי דקפיץ עליה זבינא מידי דלא קפיץ עליה זבינא לא קתני 
ת"ר איזוהי בתולת קרקע כל שמעלה רשושין ואין עפרה תיחוח נמצא בה חרס בידוע שנעבדה צונמא הרי זו בתולת קרקע
מעוברת משיודע עוברה וכמה הכרת העובר סומכוס אומר משום רבי מאיר שלשה חדשים ואע"פ שאין ראיה לדבר זכר לדבר שנאמר (בראשית לח, כד) ויהי כמשלש חדשים וגומר 
זכר לדבר קרא כתיב וראיה גדולה היא משום דאיכא דילדה לט' ואיכא דילדה לשבעה 
ת"ר הרי שהיתה בחזקת מעוברת וראתה דם ואח"כ הפילה רוח או כל דבר שאינו של קיימא הרי היא בחזקתה ודיה שעתה 
ואע"ג שאין ראיה לדבר זכר לדבר שנאמר (ישעיהו כו, יח) הרינו חלנו כמו ילדנו רוח מאי זכר לדבר הרי ראיה גדולה היא כי כתיב האי קרא בזכרים כתיב
ורמינהי קשתה שנים ולשלישי הפילה רוח או כל דבר שאינו של קיימא הרי זו יולדת בזוב ואי אמרת לידה מעלייתא היא}

\newsection{דף ט}
\twocol{קושי סמוך ללידה רחמנא טהריה אמר רב פפי הנח מעת לעת דרבנן 
רב פפא אמר מידי הוא טעמא אלא משום דראשה כבד עליה ואבריה כבדין עליה הכא נמי ראשה ואבריה כבדין עליה 
בעא מיניה רבי ירמיה מרבי זירא ראתה ואח"כ הוכר עוברה מהו כיון דבעידנא דחזאי לא הוכר עוברה מטמיא או דלמא כיון דסמוך לה חזאי לא מטמיא 
א"ל מידי הוא טעמא אלא משום דראשה כבד עליה ואבריה כבדין עליה בעידנא דחזאי אין ראשה כבד עליה ואין אבריה כבדין עליה 
בעא מיניה ההוא סבא מר' יוחנן הגיע עת וסתה בימי עבורה ולא בדקה מהו קא מיבעיא לי אליבא דמ"ד וסתות דאורייתא מאי כיון דוסתות דאורייתא בעיא בדיקה או דלמא כיון דדמיה מסולקין לא בעיא בדיקה 
א"ל תניתוה רבי מאיר אומר אם היתה במחבא והגיע שעת וסתה ולא בדקה טהורה שחרדה מסלקת את הדמים טעמא דאיכא חרדה הא ליכא חרדה והגיע וסתה ולא בדקה טמאה
אלמא וסתות דאורייתא וכיון דאיכא חרדה דמיה מסולקין ולא בעיא בדיקה הכא נמי דמיה מסולקין ולא בעיא בדיקה
מניקה עד שתגמול וכו' ת"ר מניקה שמת בנה בתוך עשרים וארבע חדש הרי היא ככל הנשים ומטמאה מעת לעת ומפקידה לפקידה לפיכך אם היתה מניקתו והולכת ארבע או חמש שנים דיה שעתה דברי ר"מ 
רבי יהודה ורבי יוסי ורבי שמעון אומרים דיין שעתן כל עשרים וארבע חדש לפיכך אם היתה מניקתו ארבע וחמש שנים מטמאה מעת לעת ומפקידה לפקידה 
כשתמצא לומר לדברי ר"מ דם נעכר ונעשה חלב לדברי רבי יוסי ורבי יהודה ורבי שמעון אבריה מתפרקין ואין נפשה חוזרת עד עשרים וארבע חדש 
לפיכך דר"מ למה לי משום לפיכך דרבי יוסי 
ולפיכך דרבי יוסי למה לי מהו דתימא רבי יוסי תרתי אית ליה קמ"ל 
תניא נמי הכי דם נעכר ונעשה חלב דברי ר"מ רבי יוסי אומר אבריה מתפרקין ואין נפשה חוזרת עליה עד עשרים וארבע חדש א"ר אלעאי מאי טעמא דר"מ דכתיב (איוב יד, ד) מי יתן טהור מטמא לא אחד 
ורבנן א"ר יוחנן זו שכבת זרע שהוא טמא ואדם הנוצר ממנו טהור 
ור"א אומר אלו מי הנדה שהמזה ומזין עליו טהור ונוגע טמא ומזה טהור והכתיב (במדבר יט, כא) ומזה מי הנדה יכבס בגדיו מאי מזה נוגע 
והכתיב מזה והכתיב נוגע ועוד מזה בעי כבוס נוגע לא בעי כבוס אלא מאי מזה נושא 
וליכתוב נושא קמ"ל דעד דדרי כשיעור הזאה הניחא למ"ד הזאה צריכה שיעור אלא למ"ד אין צריכה שיעור מאי איכא למימר 
אפילו למ"ד אינה צריכה שיעור ה"מ אגבא דגברא אבל במנא בעינא שיעור כדתנן כמה יהיו במים ויהא בהן כדי הזאה כדי שיטבול ראשי גבעולין ויזה 
והיינו דאמר שלמה (קהלת ו, ג) אמרתי אחכמה והיא רחוקה ממני
איזו היא זקנה כל שעברו עליה שלש עונות [סמוך לזקנתה] היכי דמי סמוך לזקנתה אמר רב יהודה כל שחברותיה אומרות עליה זקנה היא ורבי שמעון אומר
כל שקורין לה אמא אמא ואינה בושה ר' זירא ור' שמואל בר רב יצחק חד אמר כל שאינה מקפדת וחד אמר כל שאינה בושה  מאי בינייהו איכא בינייהו בושה ואינה מקפדת 
וכמה עונה אמר ריש לקיש משום רבי יהודה נשיאה עונה בינונית שלשים יום ורבא אמר רב חסדא עשרים יום ולא פליגי מר קחשיב ימי טומאה וימי טהרה ומר לא חשיב ימי טומאה 
ת"ר זקנה שעברו עליה שלש עונות וראתה דיה שעתה ועוד עברו עליה שלש עונות וראתה דיה שעתה ועוד עברו עליה שלש עונות וראתה הרי היא ככל הנשים ומטמאה מעת לעת ומפקידה לפקידה 
ולא (מיבעיא) שכוונה אלא אפי' פיחתה (ואפילו) והותירה 
אפי' פיחתה ולא מבעיא כוונה אדרבה כי כוונה קבעה לה וסתה ודיה שעתה 
וכי תימא רבנן היא דפליגי עליה דרבי דוסא דאמרי אשה שיש לה וסת מטמאה מעת לעת איפכא מבעי ליה ולימא ולא שפיחתה והותירה אלא אפי' כוונה 
תני לא שפיחתה והותירה אלא אפי' כוונה ואיבעית אימא ה"ק ולא שכוונה אלא שפיחתה והותירה אבל כוונה קבעה לה וסת ודיה שעתה ומני רבי דוסא היא
ר"א אומר כל אשה שעברו עליה וכו' תניא אמר להם רבי אליעזר לחכמים מעשה בריבה אחת בהיתלו שהפסיקה שלש עונות ובא מעשה לפני חכמים ואמרו דיה שעתה 
אמרו לו אין שעת הדחק ראיה מאי שעת הדחק איכא דאמרי שני בצורת הוו איכא דאמרי טהרות אפיש לעבידא וחשו רבנן להפסד דטהרות 
ת"ר מעשה ועשה רבי כר' אליעזר לאחר שנזכר אמר כדי הוא ר' אליעזר לסמוך עליו בשעת הדחק מאי לאחר שנזכר אילימא לאחר שנזכר דאין הלכה כר' אליעזר אלא כרבנן בשעת הדחק היכי עביד כוותיה 
אלא דלא איתמר הילכתא לא כמר ולא כמר ומאי לאחר שנזכר לאחר שנזכר דלאו יחיד פליג עליה אלא רבים פליגי עליה אמר כדי הוא ר' אליעזר לסמוך עליו בשעת הדחק 
ת"ר תנוקת שלא הגיע זמנה לראות וראתה פעם ראשונה דיה שעתה שניה דיה שעתה שלישית הרי היא ככל הנשים ומטמאה מעת לעת ומפקידה לפקידה 
עברו עליה שלש עונות וראתה דיה שעתה ועוד עברו עליה שלש עונות וראתה דיה שעתה ועוד עברו עליה שלש עונות וראתה הרי היא ככל הנשים ומטמאה מעת לעת ומפקידה לפקידה
וכשהגיע זמנה לראות פעם ראשונה דיה שעתה שניה מטמאה מעת לעת ומפקידה לפקידה עברו עליה שלש עונות וראתה דיה שעתה 
אמר מר עברו עליה שלש עונות דיה שעתה}

\newsection{דף י}
\twocol{הדר קחזיא בעונות מאי אמר רב גידל אמר רב פעם ראשונה ושניה דיה שעתה שלישית מטמאה מעת לעת ומפקידה לפקידה 
ועוד עברו עליה ג' עונות וראתה דיה שעתה הדר קחזיא בעונות מאי 
אמר רב כהנא אמר רב גידל אמר רב פעם ראשונה דיה שעתה שניה מטמאה מעת לעת ומפקידה לפקידה 
מני רבי היא דאמר בתרי זימני הוי חזקה 
אימא סיפא עברו עליה ג' עונות וראתה דיה שעתה אתאן לר"א 
וכי תימא רבי היא ובעונות סבר לה כר"א ומי סבר לה והא לאחר שנזכר קאמר אלא ר"א היא ובוסתות סבר לה כרבי 
כתם שבין ראשונה ושניה טהור שבין שניה ושלישית חזקיה אמר טמא רבי יוחנן אמר טהור חזקיה אמר טמא כיון דאילו חזיא מטמאה כתמה נמי טמא ורבי יוחנן אמר טהור כיון דלא אתחזקה בדם כתמה נמי לא מטמינן לה
מתקיף לה ר' אלעאי וכי מה בין זו לבתולה שדמיה טהורין א"ל ר' זירא זו שירפה מצוי וזו אין שירפה מצוי 
\commenta{\textbf{ועוד עברו עליה ג' עונות וראתה דיה שעתה.} י"מ דהא מני רבי היא דאמר בתרי זימנא הויא חזקה והא דבעינן הכא עד תלתא משום שכל שעברו עליה ג' עונות סמוך לזקנתה הוחזקה בזקנה שדמיה מסולקין ואין ראיה זו מוציאה מכלל זקנו' שכן דרך סלוק דמיהן של זקנות מפסקת ורואה ופוסקת ושוב אינה רואה וכשהיא רואה פעם לסוף ג' עונות עכשיו הוא שמתחלת להחזיק עצמה ברוא' ואפילו בזקנתה לפיכך דיה שעתה בראיה זו שהיא תחלת למניין ובשנייה שהיא שלישית הוחזקה ומטמאה מעת לעת ומתני' דקתני במה שאמרו דיה שעתה בראיה ראשונה אבל בראיה שניה מטמאה מעל"ע התם כשקרבה ראיתה בשנייה דאיגלא מילתא דג' עונות קמייתא לאו משום סלוק דמים הוו ולא הגיעה זו לכלל זקנה. אבל רחקה אף ראיה שניה זקנה היא אלא שרואה וצריכה חזקה, וזה לשון נכון.\par ול"נ דזקנה צריכה להחזיק עצמה בראיות והא דקתני מתני' אבל בראיה שנייה מטמאה מעת לעת אבתולה קאי והוא דאיתמר בגמרא עלה רב אמר אכולהו לומר שכולן ישנן בדין הזה שאם החזיקו עצמן בדמים מטמאות מעת לעת, ושמואל אמר ל"ש אלא בתולה וזקנה שישנן בדין הזה שמחזקות עצמן בראיות ומטמאות אח"כ מעת לעת, אבל מעוברת ומניקה דיין שעתן כל ימי מניקותיהן ועוברן ואפילו הן שופעות פעמים הרבה. ומיהו מתני' דקתני דבראיה אחת הוחזקה לדמים ודאי הכל מודים דלאו אזקנה קאי דזקנה הא קתני לה בבתרייתא בתרי זימני אלא בדין הזה להחזיק עצמה כשאר הנשים מעת לעת וזה פירש מדוקדק.\par וי"מ אותה לרשב"ג דבתלתא זימני הוי חזקה ומטמאה מעת לעת כשור המועד מה שור המועד בשלשה זימני אתחזק, ואידך כי נגח משלם אף זו בג' פעמים הוחזקה הילכך מטמאה בשלישי עצמה מעת לעת ולהאי פי' אמרינן דמתני' דקתני אבל בראיה שני' מטמאה מעת לעת סתמא כרבי דהא זקנה בכללה לדברי הכל ואינו מחוור. ועוד דשטתא דוסתו' לא מוקמינן להו כרבי דלקמן תנן סתמא כרשב"ג וביבמות אמרינן וסתות ושור המועד כרשב"ג.\par ולכל הלשונות נמי קשיא כיון שהוחזקה זו ולבסוף דלאו סלוק דמים הוה בה כלל נטמא למפרע מעת לעת שבכל ראיותיה והשיב רש"י מעת לעת דרבנן הוא וכל שבשעת ראיתה בחזקת טהרה אין מחמירין לטמא אותה למפרע. ואם תשאל הרי הקשו למעלה גבי היתה בחזקת מעוברת וראתה אמאי [אין] מטמאין אותה למפרע לכשהפילה רוח וזו אינה קושיא דהאיכא רב פפא דתריץ הכי הנח מעת לעת לרבנן ולרב פפא שאני התם דאיגלאי מילתא דלאו עובר הוא אבל הכא אכתי איכא למיתלי מעיקרא לא שכיחי בה דמים והשתא הוא דאכחיש' ואיתרעי א"נ התם בסמוך לראיה הפילה דאפשר לטמויה אבל לאחר עונות ליכא למימר הכי. }
אמר עולא א"ר יוחנן משום ר"ש בן יהוצדק תינוקת שלא הגיע זמנה לראות וראתה פעם ראשונה ושניה רוקה ומדרסה בשוק טהור כתמה נמי טהור ולא ידענא אם דידיה אם דרביה 
\commenta{ הא דתניא \textbf{תנוקות שלא הגיע זמנה לראות וכו'.} יפה פי' רש"י ז"ל דהיינו טעמא דלא מטמיא מעת לעת אלא בשלישית לרבי וברביעית לרשב"ג מפני שכל אשה שלא הוחזקה כבר ברואה אינה מטמאה מעת לעת דטעמא דמעת לעת כעין קנסא דרבנן הוא כדאמרו חכמים תקנו להן לבנות ישראל שיהיו בודקות עצמן שחרית וערבית וזו הואיל ולא בדקה הפסידה עונה יתירה הילכך כל שאינה צריכה לבדוק עצמה כלל אינה בכלל מעת לעת שבנדה ותנוקות שלא הגיעה זמנה לראות הרי הן בחזקת טהרה כדתניא לקמן ואין הנשים בודקות אותן הילכך פעם ראשונה ושניה שעדיין לא הוחזקה לראות ולא היתה בכלל תקנה לבדוק שחרית וערבית דיה שעתה. וכיון שראתה בשניה הוחזק ברואה לדברי רבי הילכך בג' מטמאה מעת לעת שהרי היתה צריכה לבדוק שחרית וערבית, ואע"פ שעדיין לא הגיע לכלל שנותיה, וכיון שלא בדקה הפסידה עונה יתירה ושהגיע זמנה לראות כיון שצריכה בדין היה לגזור עליה אפילו בראשונה אלא שהיא קולא לדבריהם. עברו עליה ג' עונות חזרה לכלל תנוקות שלא הגיע זמנה עד שתראה שתים ותהא מוחזקת לראות לדברי רבי דשוב צריכה בדיקה ומטמאה מעת לעת.\par והא דאמרינן לקמן (דף י' ע"א) בין שניה לשלישית כיון דלא אתחזק בדם כתמה נמי לא מטמינן לאו אליבא דהך ברייתא דרבי אלא אליבא דהילכתא כרשב"ג. וכן פסק הר"ם ז"ל דקטנה כתמה טהור עד שתראה דם ג' וסתות.\par וחזקיה סבר כיון (דחזיא) [דאלו חזיא] הרי היא כשאר כל הנשים אח"כ כתמה מחזיקה וטמא דבכל (מראיה שניה) [מראות משניה] ואילך הוחזקה.\par אבל רש"י ז"ל פי' אליבא דברייתא [דרבי] ומאי כיון דלא איתחזק בדם שעדיין לא הוחזק' בה לטמא מעת לעת ולפי דבריו ז"ל ולדידן דקי"ל כרשב"ג אין כתמה טמא עד שיעברו עליה ד' וסתות. לראיה פעם ראשונה [דרב גידל] פירש רש"י ז"ל דהיינו [ראשונה שאחר ההפסקה ושניה היינו] ראיה שניה של דלוג שהיא ראשונה לראיה דעונות. ולפי שהיא עומדת בה כשרואה עכשיו בעונו' קרי לה הכי.\par ויש לפרש "הדר קחזיא בעונות" [דקאי גם על] פעם אחרת בין דלוג ראשון ושני ראתה פעם אחת בעונה ובין שני לשלישי חזרה וראתה עוד בעונה ותרתי בעיי אהדדי איתמר ורב אשי בעא [לאפסןקי ולמפשט חדא חדא] מיניה דסד"א כיון דראתה שתים בדילוגו ושתים בעונות סלוק דמים הוא דקא מנע מינה עונות ולא תהא מוחזקת לא לדלוג ולא לעונות דכיון דשנתה כ"כ אונס בעלמא הוא. ואמר רב גידל פעם ראשונה של עונות ט) דיה שעתא כדאמרן שניה של עונות י) כיון דראיה שלישית הוא יא) מכי חזיא בעונות ואילך לעולם מטמאה מעת לעת. ומיהו בראיה (ג') [ראשונה] שלה לא מטמיא מעל"ת משום לסוף ג' עונות חזיא ואכתי לא הוחזקה ג' פעמים להפסקה דהא אנן לר' אלעזר קאמרינן ולישנא דהדרא קא חזיא דייקא כדאמרן. }
למאי נפקא מינה למיהוי דבריו של אחד במקום שנים 
כי אתא רבין וכל נחותי ימא אמרוה כר"ש בן יהוצדק 
אמר רב חלקיה בר טובי תינוקת שלא הגיע זמנה לראות אפילו שופעת כל ז' אינה אלא ראיה אחת אפילו שופעת ולא מבעיא פוסקת אדרבה פוסקת הויא לה כשתי ראיות 
אלא תינוקת שלא הגיע זמנה לראות ושופעת כל ז' אינה אלא ראיה אחת 
אמר רב שימי בר חייא מדלפת אינה כרואה והא קחזיא אימא אינה כשופעת אלא כפוסקת 
מכלל דשופעת (נמי) כי נהרא אלא אימא אינה אלא כשופעת 
תנו רבנן חזקה בנות ישראל עד שלא הגיעו לפרקן הרי הן בחזקת טהרה ואין הנשים בודקות אותן משהגיעו לפרקן הרי הן בחזקת טומאה ונשים בודקות אותן 
רבי יהודה אומר אין בודקין אותן ביד מפני שמעוותות אותן אלא סכות אותן בשמן מבפנים ומקנחות אותן מבחוץ והן נבדקות מאיליהן
רבי יוסי אומר מעוברת וכו' תני תנא קמיה דר' אלעזר רבי יוסי אומר מעוברת ומניקה שעברו עליה ג' עונות דיה שעתה א"ל פתחת בתרי וסיימת בחדא 
דלמא מעוברת והיא מניקה קאמרת ומילתא אגב אורחיה קמ"ל דימי עיבורה עולין לה לימי מניקותה וימי מניקותה עולין לה לימי עיבורה כדתניא ימי עיבורה עולין לה לימי מניקותה וימי מניקותה עולין לה לימי עיבורה 
כיצד הפסיקה שתים בימי עיבורה ואחת בימי מניקותה שתים בימי מניקותה ואחת בימי עיבורה אחת ומחצה בימי עיבורה ואחת ומחצה בימי מניקותה מצטרפות לג' עונות 
בשלמא ימי עיבורה עולין לה לימי מניקותה משכחת לה דקמניקה ואזלא ומיעברה אלא ימי מניקותה עולין לה לימי עיבורה היכי משכחת לה 
איבעית אימא בלידה יבשתא ואיבעית אימא דם נדה לחוד ודם לידה לחוד ואיבעית אימא תני חדא
במה אמרו דיה שעתה וכו' אמר רב אכולהו
ושמואל אמר ל"ש אלא בתולה וזקנה אבל מעוברת ומניקה דיין כל ימי עיבורן דיין כל ימי מניקותן 
וכן אמר ר' שמעון בן לקיש אכולהו ורבי יוחנן אמר לא שנו אלא בתולה וזקנה אבל מעוברת ומניקה דיין כל ימי עיבורן דיין כל ימי מניקותן כתנאי מעוברת ומניקה שהיו}

\newsection{דף יא}
\twocol{שופעות דם ובאות דיין כל ימי עיבורן ודיין כל ימי מניקותן דברי ר"מ רבי יוסי ור' יהודה ורבי שמעון אומרים לא אמרו דיין שעתן אלא בראייה ראשונה אבל בשניה מטמאה מעת לעת ומפקידה לפקידה
\commenta{ הכי אשכחן בנוסחי: \textbf{לימים לחודייהו פשיטא אמר רב אשי כגון דקפץ בחד בשבת וחזאי וקפץ בחד בשבא וחזא ולשבתא נמי קפצה ולא חזאי מהו דתימא איגלי מילתא דיומא הוא דגרים קמ"ל דקפיצה נמי גרמא ומשום דאכתי לא מטאי זמן קפיצה.} וק"ל כיון דלימים לחודייהו פשיטא ליה וה"ה לקפיצות לחודייהו נמי דפשיטא ליה דלא קבעה כדפרישית קפצה בשבא ולא חזאי מאי מהני לן פשיטא ודאי דלא תיחזי אלא בימים וקפיצה ועוד מאי קא מקשי' ומאי קא אתי רב אשי לחדותי הא מימר קאמרינן בברייתא דלא קבעה וסתות לקפיצות לחודייהו. ואי תקפוץ בשבת לא תיחזי והלכך איצטריכא ליה לאשמועינן ימים לפום מאי דקא משנינן ואדרבא פשיטא דבעיא ימים וקפיצה.\par ונראה שרש"י ז"ל גורס ולשבתא קפצא ולא חזיא ולמחר חזאי בלא קפיצה ומהו דתימא איגלאי מילתא דיומא הוא דגרים ולא קפיצה דהא בקפיצה בלא יום לא חזאי וביום בלא קפיצה חזאי קמ"ל דקפיצה דאתמול גרמא לראיה דהאידנא ומשום דאכתי לא מטאי זמן קפיצה לא חזא מאתמול וזהו הנכון.\par ומיהו לישנא אחרינא אמרי לה להא דרב הונא ולא ידעינן אי פליגן לישני ולמדחי' לקמא איתמר בתרא או דילמא אע"ג דלאו הכי איתמר אלא האי תרווייהו איתנהו לענין מעשה ומסתברא כיון דוסתו' דרבנן לקולא נקטינן בהו והלכתא כתרי לישני ולקולא, ואחר שכתבתי זה מצאתי להרמב"ם פאסי ז"ל שהחמיר ובטלה דעתינו מפני דעתו. }
ואם ראתה ראשונה וכו' א"ר הונא קפצה וראתה קפצה וראתה קפצה וראתה קבעה לה וסת למאי אילימא לימים הא כל יומא דלא קפיץ לא חזאי 
\commenta{ מתניתין \textbf{צריכה להיות בודקת וכו' ומשמשת בעדים וכו'.} פירש מתני' פרושי קא מפרש לה ואזיל וכיצד קתני כיצד צריכה להיות בודקת פעמים ביום שחרית וערבית ואע"פ שלא שמשה כלל וכיצד משמשת בעדים בודקת נמי בשעה שהיא עוברת משאר עסקיה לשמש את ביתה ומשמשת בעדים וע"כ מדקתני בשעה שהיא עוברת לשמש היינו עד שלפני תשמיש וש"מ דצריכה בדיקה לפני תשמיש והעד (הג') [הב'] לפני תשמיש אי אפשר אלא לאחר תשמיש הוא וכדתנן אחד לו ואחד לה אלמא צריכה בדיקה בין לפני תשמיש בין לאחר תשמיש.\par והיינו דאמרינן לעיל [דף ה' ע"א] שתי בדיקות אצרכוה רבנן חדא לפני תשמיש וחדא לאחר תשמיש ורמינ' למתני' דקתני והמשמשת בעדים הרי זו כפקידה דהיינו עדים דקתני דאינון לפני תשמיש ולאחר תשמיש דומיא דמשמשת בעדים דהך סיפא [וכי תריץ] נמי לעיל גבי רישא דמתני' מעיקרא [אידי ואידי לאחר תשמיש] משום דקשיא להו קס"ד לפרושי ההיא דשני עדים דבסוף קא חשיב אבל בהך סיפא דכ"ע עד שלפני תשמיש קתני וכדמפרש עלה לקמן בגמרא. }
אלא לקפיצות והתניא כל שתקבענה מחמת אונס אפילו כמה פעמים לא קבעה וסת מאי לאו לא קבעה וסת כלל 
לא לא קבעה וסת לימים לחודייהו ולקפיצות לחודייהו אבל קבעה לה וסת לימים ולקפיצות לימים לחודייהו פשיטא אמר רב אשי כגון דקפיץ בחד בשבת וחזאי וקפיץ בחד בשבת וחזאי [ובשבת קפצה ולא חזאי] ולחד בשבת חזאי בלא קפיצה 
מהו דתימא איגלאי מילתא למפרע דיומא הוא דקגרים ולא קפיצה קמ"ל דקפיצה נמי דאתמול גרמא והאי דלא חזאי משום דאכתי לא מטא זמן קפיצה 
לישנא אחרינא א"ר הונא קפצה וראתה קפצה וראתה קפצה וראתה קבעה לה וסת לימים ולא לקפיצות היכי דמי א"ר אשי דקפיץ בחד בשבת וחזאי וקפיץ בחד בשבת וחזאי (ובשבת קפצה ולא חזאי) ולחד בשבת (אחרינא) חזאי בלא קפיצה דהתם איגלאי מילתא דיומא הוא דקא גרים
{\large\emph{מתני׳}} אע"פ שאמרו דיה שעתה צריכה להיות בודקת חוץ מן הנדה והיושבת על דם טוהר
ומשמשת בעדים חוץ מיושבת על דם טוהר ובתולה שדמיה טהורים 
ופעמים צריכה להיות בודקת שחרית ובין השמשות ובשעה שהיא עוברת לשמש את ביתה יתירות עליהן כהנות בשעה שהן אוכלות בתרומה רבי יהודה אומר אף בשעת עברתן מלאכול בתרומה
{\large\emph{גמ׳}} חוץ מן הנדה דבתוך ימי נדתה לא בעי בדיקה 
הניחא לרבי שמעון בן לקיש דאמר אשה קובעת לה וסת בתוך ימי זיבתה ואין אשה קובעת לה וסת בתוך ימי נדתה שפיר אלא לרבי יוחנן דאמר אשה קובעת לה וסת בתוך ימי נדתה תבדוק דילמא קבעה לה וסת
 אמר לך רבי יוחנן כי אמינא אנא היכא דחזיתיה ממעין סתום אבל חזיתיה ממעין פתוח לא אמרי
והיושבת על דם טוהר קס"ד מבקשת לישב על דם טוהר 
הניחא לרב דאמר מעין אחד הוא התורה טמאתו והתורה טהרתו שפיר
אלא ללוי דאמר שני מעינות הם תבדוק דילמא אכתי לא פסק ההוא מעין טמא אמר לך לוי הא מני
בית שמאי היא דאמרי מעין אחד הוא וסתם לן תנא כב"ש סתם ואחר כך מחלוקת הוא וכל סתם ואח"כ מחלוקת אין הלכה כסתם 
\commenta{\textbf{דמגו דבעיא בדיקה לטהרות בעיא נמי בדיקה לבעלה.} פירש אע"פ שתקנו חכמים לבנות ישראל לבדוק שחרית וערבית להכשיר הטהרות עוד החמירו עליהן שאם שמשו מטתן יהו צריכות בדיקה לטהרות חוששין שמא ראתה מחמת תשמיש ובדיקה שניה שהצרכוה סמוך לתשמיש מאחריו בתוך שיעור כדי שתרד מן המטה א"נ באחר אחר כפירקן דכל היד קודם שתלך או שתקנח וכן הצריכו לאיש עצמו לקנח בעד ולבדוק והכל משום חומר הטהרות שרגילה לעסוק בהן ומתוך חומר הטהרות החמירו לבעלה שתהא צריכה לבדוק לפני התשמיש וכ"ש לאחר תשמיש ואע"פ שאין דעתה לעסוק בטהרות עכשיו מאחר שהורגלה לעסוק בהן והוחזקה להיות בודקת לטהרות לאחר תשמיש.\par נמצא שאין כאן בדיקה מן הדין אלא שלאחר תשמיש ולטהרות והשאר מדין מגו וכיון שאינו אלא משום טהרות אינה צריכה אלא עדותו של עד כלומר שמקנחת בעד לפני תשמיש ולמחר בודקת בו אם מצאה עליו דם טמאה לטהרות ומחייבת בעלה בחטאת. אבל מותר הוא לבעול משבדקה בעד ואע"פ שאינו מועיל לו עכשיו שהרי אינן יודעין אם ראתה אם לאו זהו דרכו של פי' רש"י ז"ל.\par וי"ל דכל גבי בעלה צריכה להיות בודקת ורואה ואח"כ תשמש דאין בדיקה סתם בכל מקום אלא במקנחת ורואה מה העד מעיד בדבר דאי לא תימא הכי כל הנשים בחזקת טהרה לבעליהן הן ואפילו בעסוקה בטהרות אלא משמע דכל לפני תשמיש בודקת ורואה לגבי בעלה והכל ודאי משום מגו דטהרות והיינו דאמרי (לעיל) [לקמן ע"ב] אימר שמש עכרו, ז) וכיון שאינה צריכה בדיקה לאחר תשמיש לטהרות אף לפני תשמיש לבעלה אינה צריכה דהא ליכא מגו כך פי' הרב ז"ל. }
ואבע"א מי קתני מבקשת לישב יושבת קתני אי יושבת מאי למימרא מהו דתימא תיבדוק דדילמא קבעה לה וסת קמ"ל דמעין טהור למעין טמא לא קבעה 
\commenta{\textbf{חדא מכלל דחברתא איתמר.} נראה דהא דרב יהודה איתמר מכללא דההוא דכיון דשמעיה רבה בר ירמיה לשמואל דאמר אשה אין לה וסת אסורה לשמש עד שתבדוק בעסוקה בטהרות וש"מ נמי דכל לבעלה לא בעיא בדיקה כלל מדלא מתרץ לה לר' זירא אין לה וסת אפילו לבעלה בעיא בדיקה מ"ה דייק רב יהודה דשמעא מיניה ואמרה משמיה דשמואל דמתני' דוקא בעסוקה בטהרות היא דמכלל היא דמתני' ליכא למימר [דוקא ביש לה וסת ולא] אין לה וסת ודוקא עירה כדאיתמר לקמן.\par וי"מ הא דאמרינן חדא מכלל דחברתא איתמר לאו אעיקר מימרא דשמואל אלא ה"ק שמואל שמעתא דהכא לא שאנו אלא בעסוקה בטהרות, ואמ' תו במימרא דלקמן דרבא בר' ירמיה ומכללא דהך אוקימנא לההוא שעסוקה בטהרות ולאשמעינן עירה וישנה אתמר ואי לא איתמר הך לא הוה ידעינן ההיא דבעסוקה בטהרות היא כדאמר רבא לקמן וכי אמרינן חדא מכלל חברתה איתמר אאוקמתין דעסוקה בטהרות קאמר וכי אמרינן [ואוקימנא אאוקימתא דרבה בר ירמיה] קאמרינן דאוקמתין קשי' ואוקמתין מתרצין. }
הניחא ללוי דאמר שני מעינות הם אלא לרב דאמר מעין אחד הוא תבדוק דילמא קבעה לה וסת אפילו הכי מימי טהרה לימי טומאה לא קבעה
\commenta{הא דאמרינן \textbf{תנ"ה בד"א לטהרות אבל לבעלה מותרת וכו'.} היינו טעמא דמשמע לן אפילו כשאין לה וסת משום דמתניתין בין שיש לה וסת וכו' ועלה קתני בד"א לטהרות אבל לבעלה מותר בין בזו בין בזו משמע ועוד דכל עיקר לא הוצרכה ברייתא זו לשנותה אלא בשאין לה וסת שאלו בשיש לה מתני' היא כל הנשים בחזקת טהורות לבעליהן. }
ומשמשת בעדים וכו' תנן התם תינוקת שלא הגיע זמנה לראות ונשאת ב"ש אומרים נותנין לה ארבע לילות וב"ה אומרים עד שתחיה המכה 
אמר רב גידל אמר שמואל לא שנו אלא שלא פסקה מחמת תשמיש וראתה שלא מחמת תשמיש אבל פסקה מחמת תשמיש וראתה טמאה 
עבר לילה אחת בלא תשמיש וראתה טמאה נשתנו מראה דמים שלה טמאה מתיב ר' יונה ובתולה שדמיה טהורים אמאי תשמש בעדים דדילמא נשתנו מראה דמים שלה 
אמר רבא אימא רישא חוץ מן הנדה והיושבת על דם טוהר הוא דלא בעיא בדיקה אבל בתולה שדמיה טהורין בעיא בדיקה אלא קשיין אהדדי 
כאן ששמשה דאימא שמש עכרן כאן שלא שמשה 
תניא נמי הכי בד"א שלא פסקה מחמת תשמיש וראתה שלא מחמת תשמיש
אבל פסקה מחמת תשמיש וראתה טמאה עבר לילה אחת בלא תשמיש וראתה טמאה נשתנו מראה דמים שלה טמאה 
פעמים היא צריכה וכו' א"ר יהודה אמר שמואל לא שנו אלא לטהרות אבל לבעלה מותרת פשיטא שחרית תנן 
אלא אי אתמר אסיפא אתמר ובשעה שהיא עוברת לשמש את ביתה א"ר יהודה אמר שמואל לא שנו אלא באשה עסוקה בטהרות דמגו דבעיא בדיקה לטהרות בעיא נמי בדיקה לבעלה אבל אינה עסוקה בטהרות לא בעיא בדיקה 
מאי קמ"ל תנינא כל הנשים בחזקת טהרה לבעליהן אי ממתני' הוה אמינא הני מילי באשה שיש לה וסת אבל אשה שאין לה וסת בעיא בדיקה 
והא מתני' באשה שיש לה וסת עסקינן מתני' בין שיש לה וסת בין אין לה וסת והא קמ"ל דאע"ג דיש לה וסת מגו דבעיא בדיקה לטהרות בעיא נמי בדיקה לבעלה 
והא אמרה שמואל חדא זימנא דאמר רבי זירא אמר רבי אבא בר ירמיה אמר שמואל אשה שאין לה וסת אסורה לשמש עד שתבדוק ואוקימנא בעסוקה בטהרות חדא מכלל חברתה אתמר 
תניא נמי הכי בד"א לטהרות אבל לבעלה מותרת בד"א שהניחה בחזקת טהורה אבל הניחה בחזקת טמאה לעולם היא בטומאתה עד שתאמר לו טהורה אני}

\newsection{דף יב}
\twocol{בעא מיניה רבי זירא מרב יהודה אשה מהו שתבדוק עצמה לבעלה אמר ליה לא תבדוק ותבדוק ומה בכך אם כן לבו נוקפו ופורש 
\commenta{\textbf{א"ר אמי א"ר ינאי וזהו עדן של צנועות.} פי' ודאי משנתינו עד שלפני תשמיש ולאחר תשמיש קתני ובעסוקה בטהרות וא"ר ינאי עד זה שלפני תשמיש זהו עדן של צנועות דקתני מתני' בפרק כל היד והקשו לרבי אמי הא מתניתין צריכות קתני כדתנן צריכה להיות בודקת ומשמשות בעדים ופריק שאני אומר כל המקיים דברי חכמים נקרא צנוע.\par ולדבריו של ר' אמי פיר' משנתינו שבפרק כל היד כך הוא דרך בנות ישראל שתקנו להן חכמים להיות משמשות בשני עדים אחד לו ואחד לה לאחר תשמיש ואם לא בדקו או שאבדו עידיהן אסורות לשמש עד שיבדקוהו שמא מחמת תשמיש ראתה והצנועות שמקיימות דברי חכמים מתקינות שלישי אחר לתקן את הבית לבעליהן שכך הצריכו אותן חכמים אלא שלא אסרו להם לשמש אם אבד עד זה או (שאפשר) [שאי אפשר] להן לבדוק לאור הנר דכיון שאין עדות בכל מקום אלא עד של מגו לא החמירו בו כשנאחר תשמיש לשהוחזקה בו לטהרות מן הדין.\par ואקשיה ליה רבא לרב אמי ופרכיה ופריש רבא הא דקאמר ר' ינאי וזהו עדן של צנועות לומר שבעד זה הוא צניעות הצנועות ששנינו במשנתינו שעד שבודקין בו לפני תשמיש זה אין בודקות בו לפני תשמיש אחר, ולדברי רבא פי' משנתינו כך הוא דרך בנות ישראל שתקנו חכמים משמשות בשני עדים אחד לו ואחד לה שלו מקנח בו לאחר תשמיש שהרי אין לו בדיקה אחרת ושלה בודקת בו עצמה כל זמן שהיא צריכה לבדוק דהיינו לפני תשמיש ולאחר תשמיש כדתנן הכא והצנועות מתקנות להן שלישי אחר לתקן את הבית כלומר חדש ולבן, והיינו דקתני תיקון והיינו נמי לשון שלישי לומר שאם רצו לשמש פעם אחרת למחר מתקינות להן שלישי שלא נשתמש בו כלל אפילו לפני תשמיש.\par ורש"י ז"ל מפרש שני עדים אחד לו ואחד לה לאחר תשמיש, ולפי דבריו הכי מתרצא מתני' והצנועות מתקנות להן השלישי שהן צריכות א' לתקן הבית ומהו תקונן שלא נשתמשו בו ואפילו לפני תשמיש.\par ולדברי הכל משנתינו בעסוקה בטהרו' וכדאוקי שמואל להא מתניתין דפירקן דתרווייהו בני חד ביקתא אינון וכדקתני רישא דההיא ואוכל' בתרומה ועלה קתני דרך בנות ישראל וכו', ולפום הכי קתני סיפא כל הנשים בחזקת טהרה לבעליהן כלומר אע"פ שהצריכוה בדיקה לא אמרן אלא לטהרות אבל לבעלה בחזקת טהרה הן. }
בעא מיניה רבי אבא מרב הונא אשה מהו שתבדוק עצמה כשיעור וסת כדי לחייב בעלה חטאת 
\commenta{\textbf{והא שמואל במאי מוקים לה.} וא"ת כשאין עסוקה בטהרות ולבעלה ואפילו אין לה וסת א"ב מאי איריא חמרין ופועלין אפילו עומדין בעיר נמי ועוד מאי קמ"ל הא קתני לה אידך לעיל בד"א לטהרות וכל זה אינו מחוור דאיכא למימר ישנות קמ"ל ואכתי נמי לא קים לן דאיכא חלוק בין בא מן הדרך לשוהה בעיר אלא א"ל מדקתני סתמא ש"מ אפילו בעסוקה בטהרות הוא דכולהי סתמי לטהרות ולבעלה פרושי מפרש לה בבמה ד"א א"נ דאינהו בטויי מיבעיא ליה וגמר' מתרץ דעדיף מיניה דקאמר דדוקא נקט חמרין ופועלין ואוקימנא בשיש לה וסת. }
א"ל מי משכחת לה לבדיקה כשיעור וסת והתניא איזהו שיעור וסת משל לשמש ועד שעומדים בצד המשקוף ביציאות השמש נכנס עד
\commenta{\textbf{וכיון שתבעוה אין לך בדיקה גדולה מזו.} פירש רש"י ז"ל דסתם הבא מן הדרך דרכו לפייס ולרצות ולתבוע וכי מרצו קמה ותבע רמיא אנפשה ואי הוה חזיא מרגשה. וכי אמרינן דבעינן בדיקה בשוהה עמה שאינו צריך ריצוי כ"כ ומיהו הניח בחזקת טומאה אף ריצוי לא מהני ליה עד שישמע מפיה טהורה אני והא דשאל רב כהנא אינשי דביתהו דרבנן לומר אם מחמירן על עצמן לבדוק בשאינן עסוקות בטהרות דומיא דבעיא דר' זירא דלעיל והאי דנקט כי אתו מבי רב אורחיה דמילתיה נקט שיוצאין ובאין מערב שבת לע"ש.\par ויש לפרש דישינות בעיא מנייהו אם מחמירין בהן בבאין מן הדרך משום דכיון דאין בעלה עמה לא קפדה אנפשה ואמרו להן לאו, נמצא כלל השמועה הלכה למעש' שכל לבעלה לא בעי בדיקה לא לפני תשמיש ולא לאחר תשמיש ואפילו כשאין לה וסת לפי פירושו של רש"י ז"ל בדברי רבי חנינא בן אנטיגנוס.\par אבל מדברי הרמב"ם הספרדי ז"ל למדנו שיש לו דרך אחרת בשמועה זו שהוא מפרש זו ששנינו דרך בנות ישראל לבעלה בשאין לה עסק בטהרות והצנועות בודקות אף לפני תשמיש לבעליהן וכל מה שהקל ר' יהודה ור' זירא משמיה דשמואל אינו אלא בבדיקה זו שלפני תשמיש שכשהן עסוקות בטהרות אפילו שאינן צנועות צריכות. וכשאין עסוקות בה הרי הן בחזקת טהרה לבעליהן לפני תשמיש.\par וטעם לדבריו מפני שלפני תשמיש אשה מרגשת בעצמה ואפילו בישינה נמי הקלו מפני שבחזקת טהרה הן ולאחר תשמיש חוששין שמא ראתה מחמת שמש ואינה מרגשת.\par וההיא דבעיא מיניה ר' אבא מרב הונא צריך הוא לפרש שלא מנעו אלא מלבדוק בשיעור וסת ואח"כ כדי שלא תתחייבנו באשם תלוי ויהא לבו נוקפו אבל לאחר אחר בודקת קודם שתלך ותקנח ואף על פי שהוא בודק בשלו לחטאת שאני התם דא"א בבדיקה שלא תחייבנו חטאת והוא צריך בדיקה מ"מ שמא ראתה מחמת תשמיש אבל בשלה אפשר לבדיקה זו לאחר זמן של אשם תלוי שלא יהא לבו נוקפו בביאה זו ותהא מתוקנת בביאה אחרת, וכן דברי ר' זירא לר' יהודה כך הן מתפרשין לי מהו שתבדוק עצמה לבעלה לחייב בעלה שאם לפני תשמיש והלא בצנועות (הוא) [במתני' קתני] לה וזהו שאמר רבי ינאי זו עדן של צנועות לא צנועות שנשנו בפרק כל היד אלא ר' אמי פירש לומר שכל העושה כן נקרא צנועה, ורבא פירש לומר דבעד זה נכרת אם צנועה היא אם לא אבל לדברי הכל מתני' צריכות קתני ובעסוקה בטהרות ואלו בשאינה עסוקה כבר שנינו והצנועות מתקינות וכו'.\par ושאר השמועה פשוטה היא לפי דרכו לפיכך כתב אינה צריכה עד שלפני תשמיש אלא משום צנועות אבל לאחר תשמיש הכל צריכים שני עדים אחד לו ואחד לה אפילו מעוברת ומניקה זקנה וקטנה האריך עלינו את הדרך.\par אבל דברי רש"י ז"ל יותר נכונים ומוכרעים בכמה מקומות בשמועה והחכם יבור לעצמו, ודברי רבינו יצחק אלפסי ז"ל שנראין נמי כדברי רש"י ז"ל שהוא כתב בהלכות ברייתא זו דתניא החמרין והפועלין והתיר בין עירות בין ישינות ולא הזכיר משניו' הללו של שני עדים בשאין לו וסת אלמא אין לנו עדים אלא לטהרות. }
הוי וסת שאמרו לקנוח ולא לבדיקה אלא מהו שתקנח 
איכא דאמרי הכי בעא מיניה אשה מהו שתבדוק עצמה כדי לחייב בעלה אשם תלוי אמר לו לא תבדוק ותבדוק ומה בכך א"כ לבו נוקפו ופורש
ובשעה שהיא עוברת וכו' אמר ר' אמי אמר רבי ינאי וזהו עדן של צנועות א"ל רבי אבא בר ממל לר' אמי תנא תני צריכות ואת תני צנועות אמר ליה שאני אומר כל המקיים דברי חכמים נקרא צנוע 
אמר רבא ושאינו מקיים דברי חכמים צנוע הוא דלא מקרי הא רשע לא מקרי אלא אמר רבא צנועות עד שבדקו בו עצמן לפני תשמיש זה אין בודקות בו לפני תשמיש אחר ושאינן צנועות בודקות ולא איכפת להן 
גופא אמר רבי זירא אמר רבי אבא בר ירמיה אמר שמואל אשה שאין לה וסת אסורה לשמש עד שתבדוק אמר ליה ר' זירא לרבי אבא בר ירמיה אין לה וסת בעיא בדיקה יש לה וסת לא בעיא בדיקה 
א"ל יש לה וסת ערה בעיא בדיקה ישנה לא בעיא בדיקה אין לה וסת בין ערה בין ישנה בעיא בדיקה 
אמר רבא ולימא ליה יש לה וסת לטהרות בעיא בדיקה לבעלה לא בעיא בדיקה אין לה וסת אפילו לבעלה נמי בעיא בדיקה ומדלא א"ל הכי ש"מ קסבר שמואל כל לבעלה לא בעיא בדיקה 
ת"ר חמרין ופועלין והבאין מבית האבל ומבית המשתה נשיהם להם בחזקת טהרה ובאין ושוהין עמהם בין ישנות בין ערות בד"א שהניחן בחזקת טהרה אבל הניחן בחזקת טומאה לעולם היא טמאה עד שתאמר לו טהורה אני 
והא שמואל במאי מוקי לה אי בשיש לה וסת קשיא ערה ואי בשאין לה וסת קשיא בין ערה בין ישנה 
לעולם בשיש לה וסת וכיון שתבעה אין לך בדיקה גדולה מזו אמר ליה רב פפא לרבא מהו למעבד כי הא מתניתא
א"ל סודני לא דמגניא באפיה אמר רב כהנא שאלתינהו לאינשי ביתיה דרב פפא ודרב הונא בריה דרב יהושע כי אתו רבנן מבי רב מצרכי לכו בדיקה ואמרו לי לא ולישיילינהו לדידהו דילמא אינהו קא מחמירי אנפשייהו 
\commenta{\textbf{לא פירות ולא מזונות ולא בלאות.} פירש שאינה יכולה להוציא ממנו פירות שאכל דמחילה בטעות שמה מחילה והכי מפרש בגמרא פרק איזהו נשך, ולא מזונות שאינו משלם מה שלותה ואכלה, ולא בלאות של נכסי צאן ברזל ודוקא שאינן קיימין.\par ועל הני הוא דמפרש בגמרא מ"ט תנאי כתובה ככתובה דמי וכיון שכתובה קבל עליו נכסים הללו כצאן ברזל וקבל עליו מזונות ואין לה כתובה פטור הוא מכולם אבל הקיימים נוטלת ויוצאה שאפילו זנתה נוטלת מה שבפניה ויוצאה ודינה של זו כדין איילנות שנוטלת הקיימים בשל ברזל ומפסדת שאינן קיימים, ובשל מלוג דינה כשאר הנשים לדעת רבינו ז"ל בכתובות, וכן היא נוטלת לדעתי ודעת הגאונים תוס' כתובתה וכבר פי' בפרק אלמנה נזונות ומה שכתב רש"י ז"ל בכאן אינו נכון }
ת"ר אשה שאין לה וסת אסורה לשמש ואין לה לא כתובה ולא פירות ולא מזונות ולא בלאות ויוציא ולא מחזיר עולמית דברי ר"מ 
\commenta{ והא דאמר רבי מאיר\textbf{יוציא ולא יחזיר עולמית.} משום קלקולא נ"ל והוא שאמר לה משום שאין לך וסת אני מוציאך ואם לא מפני כן לא הייתי מוציאך ואם לא אמר כן אין כאן חשש לקלקו' כדתנן המוציא אשתו משום איילנות רבי יהודה אומר לא יחזיר וחכמים אומרים יחזיר ואוקימנא מאן חכמים ר"מ ומשום דלא כפליה לתנאיה והא נמי לההיא דמיא ולדידן נמי לא בעיא כפילא כרבנן והוא שאמר סתם משום כך אני מוציאך ואף על פי שלא כפל הא גירש סתם יחזיר כדאמרינן התם גבי איילנות ומוציא משום נדר ומשום שם רע ויש לומר שאפילו לא התנה ולא אמר כלום יש לחוש לקלקל דבשלמא התם אם לא התנה כלום י"ל עילה הוא רוצה לגרש שכמה אנשים נשואים לאיילנות וע"י שיש בהם נחת רוח מהן מקיימין אותם וכך אמרו שם בירושלמי אבל זו שאסורה היא לשמש כלל בידוע שאין בעלה מגרשה אלא מחמת פיסול זה וכן במוציא משום שם רע י"ל מכיון שלא שהה לראו' אם הדברים נראין עילה מצא וגורש לפיכך אין חוששין לקלקול אלא שאמר משום שם רע אני מוציאך. }
רבי חנינא בן אנטיגנוס אומר משמשת בשני עדים הן עותוה הן תקנוה משום אבא חנן אמרו אוי לו לבעלה 
אסורה לשמש דילמא מקלקלת ליה ואין לה כתובה כיון דלא חזיא לביאה לית לה כתובה
ולא פירות ולא מזונות ולא בלאות תנאי כתובה ככתובה דמו 
ויוציא ולא יחזיר עולמית פשיטא לא צריכא דהדרה ואתקנה מהו דתימא ליהדרה קמ"ל דזימנין דאזלא ומנסבא ומתקנא
ואמר אילו הייתי יודע שכך היה אפילו הייתם נותנין לי מאה מנה לא הייתי מגרשה ונמצא גט בטל ובניה ממזרין 
משום אבא חנן אמרו אוי לו לבעלה איכא דאמרי לר"מ אמר ליה דבעי לאגבויה כתובתה איכא דאמרי לרבי חנינא בן אנטיגנוס קא"ל דמקלקלת ליה 
אמר רב יהודה אמר שמואל הלכה כר' חנינא בן אנטיגנוס ובמאי אי בעסוקה בטהרות הא אמרה שמואל חדא זימנא 
ואי בשאינה עסוקה בטהרות הא אמר כל לבעלה לא בעיא בדיקה דא"ר זירא א"ר אבא בר ירמיה אמר שמואל אשה שאין לה וסת אסורה לשמש עד שתבדוק ואוקימנא לה בעסוקה בטהרות מאן דמתני הא לא מתני הא
\par \par {\large\emph{הדרן עלך שמאי אומר}}\par \par }

\newchap{פרק \hebrewnumeral{2} כל היד}}

\newsection{דף יג}
\twocol{
\commenta{הא דאקשי' בכולה שמעתין מדר"א דאמר \textbf{כל האוחז באמה וכו'.} י"ל דהכי אקשינן וע"כ לא פליגי רבנן דאמרו לו עליה דר"א אלא בדליכא עפר תיחוח ולא מקום גבוה ומשום חשש פסול המשפחות אבל במקום אחר מודו ליה. הילכך גבי תרומ' ה"ל למימר שיפלוט ואע"פ שמפסיד' וכן בדרב יהודה שיטה וירד חוץ לכנישתא וישתין. וי"ל דא"ל לאו פלוגתא היא אלא בשואלין לפרש להן היו. וכן נמי משמע במס' ברכות פ' כיצד מברכין (מ, א). }
מתני׳ {\large\emph{כל}} היד המרבה לבדוק בנשים משובחת ובאנשים תקצץ
{\large\emph{גמ׳}} מ"ש נשים ומאי שנא אנשים נשים לאו בנות הרגשה נינהו משובחות אנשים דבני הרגשה נינהו תקצץ 
אי הכי מאי איריא מרבה כי לא מרבה נמי כי קתני מרבה אנשים 
תנא בד"א לענין שכבת זרע אבל לענין זוב אף הוא משובח כנשים 
ואפי' לענין שכבת זרע אם בא לבדוק בצרור או בחרס בודק 
ובמטלית לא והתניא בודק עצמו במטלית ובכל דבר שרוצה כדאמר אביי במטלית עבה הכא נמי במטלית עבה 
והיכא איתמר דאביי אהא דתנן היה אוכל בתרומה והרגיש שנזדעזעו איבריו אוחז באמתו ובולע את התרומה 
אוחז והתניא רבי אליעזר אומר כל האוחז באמתו ומשתין כאילו מביא מבול לעולם אמר אביי במטלית עבה 
רבא אמר אפילו תימא במטלית רכה כיון דעקר עקר ואביי חייש דלמא אתי לאוסופי ורבא לא חייש דלמא אתי לאוסופי
 ולא והתניא הא למה זה דומה לנותן אצבע בעין שכל זמן שאצבע בעין עין מדמעת וחוזרת ומדמעת 
ורבא כל אחמומי והדר אחמומי בשעתיה לא שכיח 
גופא ר"א אומר כל האוחז באמה ומשתין כאילו מביא מבול לעולם אמרו לו לרבי אליעזר והלא נצוצות נתזין על רגליו ונראה ככרות שפכה ונמצא מוציא לעז על בניו שהן ממזרים 
אמר להן מוטב שיוציא לעז על בניו שהן ממזרים ואל יעשה עצמו רשע שעה אחת לפני המקום 
תניא אידך אמר להן רבי אליעזר לחכמים אפשר יעמוד אדם במקום גבוה וישתין או ישתין בעפר תיחוח ואל יעשה עצמו רשע שעה אחת לפני המקום 
הי אמר להו ברישא אילימא קמייתא אמר להו ברישא בתר דאמר להו איסורא הדר אמר להו תקנתא 
אלא הא אמר להו ברישא ואמרו ליה אין לו מקום גבוה ועפר תיחוח מאי אמר להן מוטב שיוציא לעז על בניו ואל יעשה עצמו רשע שעה אחת לפני המקום
וכל כך למה מפני שמוציא שכבת זרע לבטלה דא"ר יוחנן כל המוציא שכבת זרע לבטלה חייב מיתה שנאמר (בראשית לח, י) וירע בעיני ה' (את) אשר עשה וימת גם אותו 
רבי יצחק ורבי אמי אמרי כאילו שופך דמים שנאמר (ישעיהו נז, ה) הנחמים באלים תחת כל עץ רענן שוחטי הילדים בנחלים תחת סעיפי הסלעים אל תקרי שוחטי אלא סוחטי 
רב אסי אמר כאילו עובד עבודת כוכבים כתיב הכא תחת כל עץ רענן וכתיב התם (דברים יב, ב) על ההרים הרמים ותחת כל עץ רענן 
רב יהודה ושמואל הוו קיימי אאיגרא דבי כנישתא דשף ויתיב בנהרדעא אמר ליה רב יהודה לשמואל צריך אני להשתין א"ל שיננא אחוז באמתך והשתן לחוץ 
היכי עביד הכי והתניא ר"א אומר כל האוחז באמתו ומשתין כאילו מביא מבול לעולם 
אמר אביי עשאו כבולשת דתנן בולשת שנכנס לעיר בשעת שלום חביות פתוחות אסורות סתומות מותרות בשעת מלחמה אלו ואלו מותרות לפי שאין להן פנאי לנסך אלמא דכיון דבעיתי לא אתי לנסוכי הכא נמי כיון דבעיתי לא אתי להרהורי 
והכא מאי ביעתותא איכא איבעית אימא ביעתותא דליליא ודאיגרא ואיבעית אימא ביעתותא דרביה ואב"א ביעתותא דשכינה ואיבעית אימא אימתא דמריה עליה דקרי שמואל עליה אין זה ילוד אשה 
ואיבעית אימא נשוי הוה דאמר רב נחמן אם היה נשוי מותר 
ואיבעית אימא כי הא אורי ליה דתני אבא בריה דרבי בנימין בר חייא אבל מסייע בביצים מלמטה ואיבעית אימא כי הא אורי ליה דאמר רבי אבהו אמר רבי יוחנן גבול יש לו מעטרה ולמטה מותר}

\newchap{פרק \hebrewnumeral{2} כל היד}
\twocol{מעטרה ולמעלה אסור 
אמר רב המקשה עצמו לדעת יהא בנדוי ולימא אסור דקמגרי יצה"ר אנפשיה ורבי אמי אמר נקרא עבריין שכך אומנתו של יצר הרע היום אומר לו עשה כך ולמחר אומר לו עשה כך ולמחר אומר לו לך עבוד עבודת כוכבים והולך ועובד 
איכא דאמרי אמר רבי אמי כל המביא עצמו לידי הרהור אין מכניסין אותו במחיצתו של הקב"ה כתיב הכא (בראשית לח, י) וירע בעיני ה' וכתיב התם (תהלים ה, ה) כי לא אל חפץ רשע אתה לא יגורך רע 
ואמר ר' אלעזר מאי דכתיב (ישעיהו א, טו) ידיכם דמים מלאו אלו המנאפים ביד תנא דבי רבי ישמעאל (שמות כ, יג) לא תנאף לא תהא בך ניאוף בין ביד בין ברגל 
ת"ר הגרים והמשחקין בתינוקות מעכבין את המשיח בשלמא גרים כדר' חלבו דא"ר חלבו קשין גרים לישראל כספחת אלא משחקין בתנוקות מאי היא 
אילימא משכב זכור בני סקילה נינהו אלא דרך אברים בני מבול נינהו 
אלא דנסיבי קטנות דלאו בנות אולודי נינהו דא"ר יוסי אין בן דוד בא עד שיכלו כל הנשמות שבגוף שנאמר (ישעיהו נז, טז) כי רוח מלפני יעטוף ונשמות אני עשיתי
באנשים תקצץ איבעיא להו דינא תנן או לטותא תנן דינא תנן כי הא דרב הונא קץ ידא או לטותא תנן 
ת"ש דתניא רבי טרפון אומר יד לאמה תקצץ ידו על טבורו אמרו לו ישב לו קוץ בכריסו לא יטלנו א"ל לא אמר להן מוטב תבקע כריסו ואל ירד לבאר שחת 
אי אמרת בשלמא דינא תנן היינו דאמרי והלא כריסו נבקעת אלא אי אמרת לטותא תנן מאי כריסו נבקעת אלא מאי דינא תנן לא סגי דלאו על טבורו 
אלא ה"ק רבי טרפון כל המכניס ידו למטה מטבורו תקצץ אמרו לו לרבי טרפון ישב לו קוץ בכריסו לא יטלנו אמר להן לא והלא כריסו נבקעת אמר להן מוטב תבקע כריסו ואל ירד לבאר שחת
{\large\emph{מתני׳}} החרשת והשוטה והסומא ושנטרפה דעתה אם יש להן פקחות מתקנות אותן והן אוכלות בתרומה
{\large\emph{גמ׳}} חרשת איהי תבדוק לנפשה דתניא אמר רבי חרשת היתה בשכונתינו לא דיה שבודקת לעצמה אלא שחברותיה רואות ומראות לה 
התם במדברת ואינה שומעת הכא בשאינה מדברת ואינה שומעת כדתנן חרש שדברו חכמים בכל מקום אינו שומע ואינו מדבר
הסומא איהי תבדוק לנפשה ותיחזי לחבירתה א"ר יוסי ברבי חנינא סומא אינה משנה
ושנטרפה דעתה היינו שוטה שנטרפה דעתה מחמת חולי 
תנו רבנן כהן שוטה מטבילין אותו ומאכילין אותו תרומה לערב ומשמרין אותו שלא יישן ישן טמא לא ישן טהור 
רבי אליעזר ברבי צדוק אומר עושין לו כיס של עור אמרו לו כל שכן שמביא לידי חימום אמר להן לדבריכם שוטה אין לו תקנה 
אמרו לו לדברינו ישן טמא לא ישן טהור לדבריך שמא יראה טפה כחרדל ותבלע בכיס 
תנא משום רבי אלעזר אמרו עושין לו כיס של מתכת 
אמר אביי ושל נחשת כדתניא רבי יהודה אומר רואין אותן גבעולין של אזוב כאילו הן של נחשת 
אמר רב פפא שמע מינה מכנסים אסורים והכתיב (שמות כח, מב) ועשה להם מכנסי בד לכסות בשר ערוה 
ההוא כדתניא מכנסי כהנים למה הן דומין כמין פמלניא של פרשים למעלה עד מתנים למטה עד ירכים ויש להם שנצים ואין להם לא בית הנקב ולא בית הערוה 
אמר אביי}

\newsection{דף יד}
\twocol{רוכבי גמלים אסורין לאכול בתרומה תניא נמי הכי רוכבי גמלים כולם רשעים הספנים כולם צדיקים
\commenta{\textbf{וליחוש דילמא דם מאכולת הוא.} פי' רש"י ז"ל דעל עד שלו פריך ומיהו ה"נ קשיא לעד שלה.\par ולפיכך הקשו מקצת המפרשים א"כ אין לך אשה שנטמאה בנדה בבדיקה. וי"ל בעלמא ודאי לא חיישינן משום דלא שכיח אבל מתוך שבדקה מיד לפני תשמיש ומצאה טהור יש לספק ולא מחוור.\par וי"ל שלא בשעת בעילה ממש ודאי אותו מקום בדוק הוא אצל מאכולת שהוא סתום מלכנוס ואם נכנסה מתה היא ואין דמה יוצא ממנה אבל כאן יש לחוש שמא בשע' בעילה דחקה ונכנסה והשמש הרגה ושפך דמה עד עקבו א"נ שמא על השמש היתה מאכולת ועמו נכנסה ופריק דחוק הוא ואין מאכולת שעל השמש נכנסת עמו וכ"ש בפני עצמה.\par ובתוספת אמרו דעל עד שלו דוקא פריך משום דכיון שבדקה היא בשיעור וסת ומצאה טהור והוא מצא יש לחוש למאכולת שאם היה דם נדה בשלו היה נמצא בשלה נמי. אבל בשלה בבדיקה דעדים לא חיישינן למאכולת דרוב דמים מצויין בנשים ואין קנוח העד נמי ממעך מאכולת והורגה (ובמעורה) [ובמעוכא] דשמש אין לתלותה כיון שלא נמצא על שלה.\par אבל עדיין אני תמה על עד שלו דלא יהבו ליה רבנן שיעורא אלא אע"פ שיהא אחר בעילה זמן מרובה קודם קנוח טמאין ואמאי ליחוש שמא מאכול' הוא שבאה עליה אחר שבעל שהרי אינו אלא ככתם בעלמא ויש לדחוק ולומר שכל קודם קנוח כיון שעדיין שכבת זרע לחה אין המאכול' באה עליו ואין לחוש שמא נרצפה על הסדין וממנו נתקנח בו שדם מאכול' מועט הוא ואינו מתקלח מן הסדין אלא נבלע הוא בו מהכ"ש שא"א לו להתקנח (במקום) [ממקום] ששוכבת עליו לזה. }
החמרים מהן רשעים מהן צדיקים איכא דאמרי הא דמכף הא דלא מכף ואיכא דאמרי הא דמטרטין הא דלא מטרטין 
\commenta{\textbf{בדקה בעד הבדוק לה וטחתו ביריכה ולמחר מצאת עליו דם.} גרסינן בכולהי נוסחי וכן מצינו בשם ר"ח. ופירושו שמצאת עליו על העד דליכא למיתני אלא דילמא דם בירך הוה ואי הוה בירך נמי כתם הוא וטמא וכיון דאיכא ספיקא דירך גופיה ועוד דלא שכיח קרוב הוא יותר לתלות בעד לטומאת נדה ולא בירך הילכך בין שנמצא נמי על הירך בין שלא נמצא אלא על העד טמאה נדה.\par ורש"י ז"ל גרס ונמצ' עליה דם ופירש על יריכ' של אשה ואין דבריו מחוורים שאם העד נמצא ואין עליו כלום נראה ודאי שדם יבש על ירכה היה ואין לטמא אותה נדה ואם דוקא כשנמצא אף על העד למה לי מציאת הירך וכי מפני שהוטח ממנה דם נקל ויש להעמידה בשאבד עדה. וי"ל לעולם כשנמצא אף על העד וכשלא נמצא על הירך כלום פשיטא בעד הוה שאלמלא על הירך הוה לא נתקנח לגמרי ממנה אלא אפילו נמצא על הירך נמי טמאה נדה כיון שנמצא אף בעד.\par ולדברי הכל אף בזה צריכה כגריס ועוד שאלמלא כן חוששין שמא דם מאכולת הוא שאפילו בקנוח של שיעור וסת ואח"כ שהדבר קרוב ואפילו לחטאת ואשם הקשו למעלה וליחוש דילמא מאכולת הוא. ומיהו כיון דאיכא שיעורא דנפקא ליה מחשש מאכולת טמאה נדה ואפילו לר' חייא ואם היה משוך טמא בכל שהוא דלא גרע מהניחתו תחת הכר וכן בפרק הרואה כתם אליבא דהלכתא וזה דעת הראב"ד ז"ל.\par ולפי גרסת ר"ח ז"ל י"ל כמו שנמצא על העד ולא על הירך כלום שאם מאכולת נרצפה שם בתחלה על הירך נמי היה נמצא ומשהוטח העד על הירך מקום (דחוק) א) הוא אצל מאכולת אלא מקנוח היה דם ונבלע בעד ולפיכך לא הוטח ממנו על הירך כלום. וכן נראה מלישנא דגמ' דקא מקשה ר' חייא בסמוך אף אתה עשיתו כתם אלמא אין לך שצריכה שיעור כתם וטמאה נדה אלא זו לדברי רבי. }
ריב"ל לייט אמאן דגני אפרקיד איני והאמר רב יוסף פרקדן לא יקרא קרית שמע קרית שמע הוא דלא יקרא הא מגנא שפיר דמי 
\commenta{\textbf{בדקה בעד שאינו בדוק לה.} פירש בתוספות כגון שהזמינה פקולין או צמר נקי ולבנים אלא שלא חזרה וראתה בהן סמוך לבדיקתה אם יש עליהם טיפי דמים מן מאכולת או משאר דברים הא בבגד שאינו בדוק כלל לא אמר ר' טמאה נדה אטו לקחה בגד מן האשפ' וקנחה בו מי מטמא רבי נדה.\par (ואי) [ועוד אני] אומר כיון שהצריכוה כגריס ועוד הרידינו כסדין וחלוק וכל שלא היה בדוק כלל אפילו לרבי טהורה לגמרי אפילו לקחתו מן השוק סתם טהורה לבעלה. והראב"ד ז"ל סובר דבעד אפילו אינו בדוק כלל טמאה דלא דמי לחלוק דכיון שבדקה בו ממש רגלים לדבר דרוב דמים מצויין בו. }
לענין מגנא כי מצלי שפיר דמי לענין ק"ש כי מצלי אסור והא ר' יוחנן מצלי וקרי ק"ש שאני רבי יוחנן דבעל בשר הוה
{\large\emph{מתני׳}} דרך בנות ישראל משמשות בשני עדים אחד לו ואחד לה והצנועות מתקנות שלישי לתקן את הבית 
נמצא על שלו טמאין וחייבין קרבן נמצא על שלה אותיום טמאין וחייבין בקרבן נמצא על שלה לאחר זמן טמאין מספק ופטורים מן הקרבן 
איזהו אחר זמן כדי שתרד מן המטה ותדיח פניה ואח"כ מטמאה מעת לעת ואינה מטמאה את בועלה ר"ע אומר אף מטמאה את בועלה 
מודים חכמים לרבי עקיבא ברואה כתם שמטמאה את בועלה
{\large\emph{גמ׳}} וניחוש דלמא דם מאכולת הוא אמר רבי זירא אותו מקום בדוק הוא אצל מאכולת ואיכא דאמרי דחוק הוא אצל מאכולת 
מאי בינייהו איכא בינייהו דאשתכח מאכולת רצופה להך לישנא דאמר בדוק הוא הא מעלמא אתאי להך לישנא דאמר דחוק הוא אימא שמש רצפה 
אתמר בדקה בעד הבדוק לה וטחתו בירכה ולמחר מצאה עליה דם אמר רב טמאה נדה א"ל רב שימי בר חייא והא חוששת אמרת לן 
איתמר נמי אמר שמואל טמאה נדה וכן מורין בי מדרשא טמאה נדה 
אתמר בדקה בעד שאינו בדוק לה והניחתו בקופסא ולמחר מצאה עליו דם א"ר יוסף כל ימיו של ר' חייא טימא ולעת זקנתו טיהר 
איבעיא להו היכי קאמר כל ימיו טימא משום נדה ולעת זקנתו טיהר משום נדה וטימא משום כתם 
או דלמא כל ימיו טימא משום כתם ולעת זקנתו טיהר מולא כלום 
תא שמע דתניא בדקה בעד שאינו בדוק לה והניחתו בקופסא ולמחר מצאה עליו דם רבי אומר טמאה משום נדה ורבי חייא אמר טמאה משום כתם
אמר לו ר' חייא אי אתה מודה שצריכה כגריס ועוד א"ל אבל אמר לו א"כ (אתה) אף אתה עשיתו כתם 
\commenta{ ופריק רב חסדא דה"ק \textbf{איזו אחר זמן וכו'} פי' רש"י ז"ל וחסורי מחסרא וה"ק ול"נ אלא רב חסדא פרושי מפרש לה למתניתין הכי תנן נמצא על שלה לאחר זמן טמאים מן הספק ופטורין מן הקרבן ולא פי' שיעור אחר זמן מה הוה. והדר תני איזו אחר של שיעור זה שאינן טמאין מן הספק כדי שתרד מן המטה ותבדוק שזהו אחר כך שמטמאה מעת לעת ואינה מטמאה את בועלה והאי דפריש תנא האי שיעורא לא ללמד על דין עצמו שהרי כל מעת לעת כך הוא נדון בין לרבנן בין לר' עקיבא אלא כך אמר אם ירדה מן המטה ובדקה אין בועלה טמא שזהו אחר זמן הא כל זמן שלא שהת' כשיעור הזה אלא בדקה עצמה על המטה טמאין מספק בא זה ולמד על זה.\par אבל לדברי רש"י ז"ל שאומר חסורי מחסרא וה"ק בהדיא איזו אחר זמן כדי שתושיט ידה ותטול עד ותבדוק לא הוה תו למיתני כדי שתרד מן המטה ואפשר לתרץ לו שבא לפרש שלא תעלה על דעת שבדיקת ראשונה שלאחר זמן ראשונה כשירדה מן המט' היא לפיכך פירש שתיהן ואמר שאלו ירדה אחר אחר הוא וכ"ש לדברינו דמחוור טפי לומר שפי' אחר אחר ללמד על אחר הזמן הראשון ושלא ליתן בו מקום לטעות.\par וברייתא נמי דייקא כדידן, דתניא איזהו אחר זמן דבר זה שאל ר' אלעזר בר צדוק לפני חכמי' באושא שמא כר"ע אתם אומרים שמטמאה את בועלה מעת לעת פירש ולפיכך אין אתם חוששין לפרש אחר זמן שהרי כל מעת לעת נמי כך הוא דינן ואע"פ שהיו צריכין לפרש לדבריו דר' אליעזר בר צדוק משום אשם תלוי יודע היה בהם דבעי חתיכה משתי חתיכות ולא תמה עליהם בזה אלא אם כדברי ר"ע שהוא יחיד הם אומרים אמרו לו לא שמענו לפי' אין אנו מפרשין אבל לא בדברי היחיד אנו אומרים ואמר להם כך פרשו חכמים ביבנה לא שהתה כדי שתרד מן המטה ותדיח פניה תוך זמן זה כלומר כל ששהתה ובדקה ולא שהתה שיעור שתרד מן המטה ותבדוק אלא על המטה בדקה אע"פ שהושיטה ידה לעד תוך זמן זה אבל ירדה מן המטה ובדקה או שהתה כשיעור הזה לעולם טהור ואע"פ שעד בידה ש"מ שבכל מקום שפי' אחר לא פירש באחר זמן דבר אחר אלא כל שלא שהתה כשיעור אחר וכן דרך משנתינו ללשון שפירשנו.\par ואקשינן לרב אשי אמאי קא מטהרי רבנן בברייתא ביורדת מן המטה ובודק' הא במתני' מטמו כשיעור הזה וכ"ת דאין עד בידה ה"ל לפרושי שהרי אין משמעו' הלשון זה אלא כל ששהתה כדי שתרד לעולם טהור ואפילו עד בידה וכדקתני מתניתין נמי כדי שתרד ומוקמת לה בעד [בידה] דהיכי אפשר דהכא והכא חד שיעורא קתני והכא טהור והכא טמא ה"ל לפרושי במתני' עד בידה ובברייתא אין עד בידה אלא ש"מ כרב חסדא ותרווייהו שיעור לטהר וזהו דרך פירש רש"י ז"ל בשמועה כולה ויש לשונות אחרים ואין בהם ממש. }
ורבי סבר בעינן כגריס ועוד לאפוקי מדם מאכולת וכיון דנפק לה מדם מאכולת ודאי מגופה אתא 
מאי לאו בזקנותו קאי הא בילדותו טימא משום נדה שמע מינה 
משתבח ליה רבי לרבי ישמעאל ברבי יוסי ברבי חמא בר ביסא דאדם גדול הוא אמר לו לכשיבא לידך הביאהו לידי 
כי אתא א"ל בעי מינאי מילתא בעא מיניה בדקה בעד שאינו בדוק לה והניחתו בקופסא ולמחר מצאה עליו דם מהו 
אמר לו כדברי אבא אימא לך או כדברי רבי אימא לך א"ל כדברי רבי אימא לי 
אמר רבי ישמעאל זהו שאומרין עליו דאדם גדול הוא היאך מניחין דברי הרב ושומעין דברי התלמיד 
ור' חמא בר ביסא סבר רבי ריש מתיבתא הוא ושכיחי רבנן קמיה ומחדדי שמעתתיה 
מאי רבי ומאי רבי יוסי אמר רב אדא בר מתנא תנא רבי מטמא ורבי יוסי מטהר 
ואמר רבי זירא כשטימא רבי כר"מ וכשטיהר רבי יוסי לעצמו טיהר 
דתניא האשה שהיתה עושה צרכיה וראתה דם ר"מ אומר אם עומדת טמאה אם יושבת טהורה 
רבי יוסי אומר בין כך ובין כך טהורה 
א"ל רב אחא בריה דרבא לרב אשי והא א"ר יוסי בר' חנינא כשטימא ר"מ לא טימא אלא משום כתם ואילו רבי משום נדה קאמר א"ל אנן הכי קאמרינן כי איתמר ההיא משום נדה איתמר
נמצא על שלה אותיום טמאין וכו' ת"ר איזהו שיעור וסת משל לשמש ועד שעומדין בצד המשקוף ביציאת שמש נכנס עד
הוי וסת שאמרו לקינוח אבל לא לבדיקה
נמצא על שלה לאחר זמן וכו' תנא וחייבין אשם תלוי ותנא דידן מ"ט 
בעינן חתיכה משתי חתיכות
איזהו אחר זמן וכו' ורמינהי איזהו אחר זמן פירש ר' אליעזר ברבי צדוק כדי שתושיט ידה תחת הכר או תחת הכסת ותטול עד ותבדוק בו 
אמר רב חסדא מאי אחר אחר אחר 
והא קתני עלה נמצא על שלה לאחר זמן טמאין מספק ופטורין מן הקרבן איזהו אחר זמן כדי שתרד מן המטה ותדיח פניה 
ה"ק איזהו אחר זמן כדי שתושיט ידה לתחת הכר או לתחת הכסת ותטול עד ותבדוק בו וכדי שתרד מן המטה ותדיח את פניה מחלוקת ר"ע וחכמים 
והא אח"כ קתני ה"ק וזהו אח"כ שנחלקו ר"ע וחכמים 
רב אשי אמר אידי ואידי חד שיעורא הוא עד בידה כדי שתרד מן המטה ותדיח את פניה אין עד בידה כדי שתושיט ידה לתחת הכר או לתחת הכסת ותטול עד ותבדוק בו 
מיתיבי איזהו אחר זמן דבר זה שאל רבי אלעזר ברבי צדוק לפני חכמים באושא ואמר להם}

\newsection{דף טו}
\twocol{שמא כרבי עקיבא אתם אומרים שמטמאה את בועלה אמרו לו לא שמענו 
אמר להם כך פרשו חכמים ביבנה לא שהתה כדי שתרד מן המטה ותדיח את פניה תוך זמן הוא זה וטמאין מספק ופטורין מקרבן וחייבין באשם תלוי 
שהתה כדי שתרד מן המטה ותדיח את פניה אחר הזמן הוא זה 
וכן כששהתה מעת לעת ומפקידה לפקידה בועלה מטמא משום מגע ואינו מטמא משום בועל רבי עקיבא אומר אף מטמא משום בועל רבי יהודה בנו של רבן יוחנן בן זכאי אומר בעלה נכנס להיכל ומקטיר קטורת 
בשלמא לרב חסדא היינו דמטהרי רבנן 
אלא לרב אשי אמאי מטהרי רבנן 
וכי תימא דאין עד בידה האי עד בידה ואין עד בידה מיבעי ליה קשיא
רבי יהודה בנו של רבן יוחנן בן זכאי אומר בעלה נכנס להיכל ומקטיר קטורת ותיפוק ליה דהוה נוגע במעת לעת שבנדה 
הוא דאמר כשמאי דאמר כל הנשים דיין שעתן 
ותיפוק ליה דהוה בעל קרי בשלא גמר ביאתו
ומודים חכמים לרבי עקיבא ברואה כתם אמר רב למפרע ורבי מאיר היא 
ושמואל אמר מכאן ולהבא ורבנן היא מכאן ולהבא פשיטא 
מהו דתימא הואיל ומעת לעת דרבנן וכתמים דרבנן מה מעת לעת לא מטמאה את בועלה אף כתמים לא מטמאה את בועלה קא משמע לן 
ואימא הכי נמי התם אין שור שחוט לפניך הכא יש שור שחוט לפניך 
וכן אמר ריש לקיש למפרע ורבי מאיר היא רבי יוחנן אמר מכאן ולהבא ורבנן היא
{\large\emph{מתני׳}} כל הנשים בחזקת טהרה לבעליהן הבאין מן הדרך נשיהן להן בחזקת טהרה
{\large\emph{גמ׳}} למה ליה למתני הבאין מן הדרך סד"א הני מילי היכא דאיתיה במתא דרמיא אנפשה ובדקה אבל היכא דליתא במתא דלא רמיא אנפשה לא קא משמע לן 
אמר ריש לקיש משום רבי יהודה נשיאה והוא שבא ומצאה בתוך ימי עונתה 
אמר רב הונא ל"ש אלא שאין לה וסת אבל יש לה וסת אסור לשמש 
כלפי לייא אדרבה איפכא מסתברא אין לה וסת אימא חזאי יש לה וסת וסת קביע לה 
אלא אי איתמר הכי איתמר אמר רב הונא ל"ש אלא שלא הגיע שעת וסתה אבל הגיע שעת וסתה אסורה קסבר וסתות דאורייתא 
רבה בר בר חנה אמר אפילו הגיע שעת וסתה נמי מותרת קסבר וסתות דרבנן 
רב אשי מתני הכי אמר רב הונא
לא שנו אלא שאין לה וסת לימים אלא יש לה וסת לימים ולקפיצות כיון דבמעשה תליא מילתא אימא לא קפיץ ולא חזאי אבל יש לה וסת לימים אסורה לשמש
קסבר וסתות דאורייתא 
רבה בר בר חנה אמר אפילו יש לה וסת לימים מותרת קסבר וסתות דרבנן 
אמר רב שמואל משמיה דרבי יוחנן אשה שיש לה וסת בעלה מחשב ימי וסתה ובא עליה 
אמר ליה רב שמואל בר ייבא לרבי אבא אמר רבי יוחנן אפילו ילדה דבזיזא למטבל 
אמר ליה אטו ודאי ראתה מי אמר רבי יוחנן אימר דאמר רבי יוחנן ספק ראתה ספק לא ראתה ואם תמצא לומר ראתה אימא טבלה
אבל ודאי ראתה מי יימר דטבלה הוה ליה ספק וודאי ואין ספק מוציא מידי ודאי 
ולא והתניא חבר שמת והניח מגורה מלאה פירות אפילו הן בני יומן הרי הן בחזקת מתוקנין והא הכא ודאי טבל ספק מעושר ספק אינו מעושר וקאתי ספק ומוציא מידי ודאי 
התם ודאי וודאי הוא כדרב חנינא חוזאה דאמר רב חנינא חוזאה חזקה על חבר שאינו מוציא מתחת ידו דבר שאינו מתוקן 
ואיבעית אימא ספק וספק הוא וכדרבי אושעיא דא"ר אושעיא מערים אדם על תבואתו ומכניסה במוץ שלה כדי שתהא בהמתו אוכלת ופטורה מן המעשר 
ואכתי אין ספק מוציא מידי ודאי והתניא מעשה בשפחתו של מסיק אחד ברימון שהטילה נפל לבור ובא כהן והציץ בו לידע אם זכר אם נקבה
ובא מעשה לפני חכמים וטהרוהו מפני שחולדה וברדלס מצויים שם 
והא הכא דודאי הטילה נפל ספק גררוהו ספק לא גררוהו וקאתי ספק ומוציא מידי ודאי 
לא תימא הטילה נפל לבור אלא אימא}

\newsection{דף טז}
\twocol{כמין נפל 
\commenta{מתני': \textbf{ב"ש אומרים צריכה שני עדים על כל תשמיש ותשמיש.} פי' רש"י ז"ל א' לפני תשמיש וא' לאחר תשמיש ולמחר בודק בשניהם ולא עכשיו כמו שפירשתי בפ"ק לדעת הרב ז"ל, ולשון צריכה נראה כן מדלא קתני צריכים ומיהו יכולה היא לבדוק באותו שלפני תשמיש זה לפני תשמיש אחר חוץ מן הצנועו' ששנינו למעלה אבל עד שלאחר כל תשמיש ותשמיש צריך לבן וכיון שזה צריך לכל הנשים קתני נמי שלפני תשמיש שדרך הצנועו'.\par ואינו מחוור. ועוד ק"ל אמאי לא תנא צריכי' שלשה עדים וליחשוב נמי א' שלו אבל נראה שכל מקום ששנינו שני עדים א' לו וא' לה.\par וכך פי' משנתינו לפי דעתי צריכה שני עדים א' לו וא' לה שלה בודק בו לפני תשמיש ראשון ורואה טהרה ומשמשת ואח"כ מקנח' בו לאחר תשמי' וכשבאה לשמש פעם אחרת אינה צריכה כלום אלא משמשת ולאתר תשמיש מקנחין היא ובעלה בשני עדים אחרים ומניחין אותן עד למחר שמא מחמת תשמיש ראתה וזו הבדיקה אינו מועלת להם [אלא] לטהרות חוץ מבדיקה שלפני תשמיש ראשון שהיא אף לבעל לפיכך בודקת ורואה מיד לאור היום בין השמשות או לאור הנר אם התחילה משחשיכה לגמרי ואפילו לב"ש עד שלפני תשמיש ושלאחר תשמיש עד א' הוא שאפילו הצנועות עצמן לא נהגו צניעותן אלא בעד שלפני תשמיש מתוך שהוא נקי וטיפה כל שהיא ניכרת בו אינן רוצות שיהי' בו לכלוך אפילו שלפני תשמיש אחר כדי שתהא בדיקתן מעולה לגמרי אבל של אחר תשמיש ששכבת זרע רבה עליו [אין בין עד חדש לעד] מבדיקה ראשונה כלום.\par והיינו נמי דאמר או תשמש לאור הנר ותבדוק בו דבעד שלאתר תשמיש לא בעינן לבן אלא שצריך שידע אם דם שעליו מתשמיש ראשון או מאחרון היה כדי לחייב עצמה ובועל' לידע אם טמא משום בועל נדה ולעולם לפני תשמיש [א"צ בדיקה] שלא בעי ב"ש בדיקה אלא לפני תשמיש ראשון בלבד שהרי דיה בדיקה אחת לעונה אחת ולא הוצרכו הללו שבין תשמיש לתשמיש אפילו לטהרו' [אלא] מפני חשש רואה מחמת תשמיש הילכך דיה בבדיקה שלאחר כל תשמיש ותשמיש.\par ועוד שכיון שהיא משמשת והולכת בודק' שלאחר תשמיש זה שהיא לפני תשמיש זה דסמוכין הן וב"ה אומרים דיה שני עדים כל הלילה אף אלו א' לו וא' לה שלה עולה לפני תשמיש ראשון שבודק' ורואה ולאחר תשמיש אחרון ושלו לאחר תשמיש אחרון שכיון שאף לדברי ב"ש אין בדיקו' הללו אלא לטהרו' ולאחר תשמיש דיה בסוף.\par ולא יקשה עליך צריכה [{\small פי' ולא תנן צריכים} ] לשון שאמרנו שהרי כך שנינו דרך בנות ישראל משמשות בב' עדים ואע"פ שא' לו ולה אמרו דרך בנות ישראל ובני ישראל משמשין.\par והר"ם הספרדי פי' שאף ב"ה מצריכים לבדוק לאחר כל תשמיש אלא שדיין באותן שני עדים כל הלילה. }
והא לידע אם זכר אם נקבה קתני 
\commenta{גמרא \textbf{אמרו להם ב"ש לדבריכם.} כיון שאתם מודים בבדיקה שלאחר תשמיש משום שמא ראתה מחמת תשמיש הוה נמי בבדיקה בין תשמיש לתשמיש שמא תראה טפת דם כחרדל בביאה ראשונה שבא אורח מחמת תשמיש ותחפנה שכבת זרע בביאה שנייה ושוב לא תמצא בעד שלאחר כל התשמישן אמרו להם ב"ה א"כ אף מתחלת ביאה לסוף ביאה ניחוש כן ויכולין היו ב"ש לומר שאין דנין אפשר מא"א אלא גדולה מזו אמרו שאינו דומה וכו'.\par ולדברי הר"מ ז"ל צריכין לפרש תראה טיפה דם כחרדל בעד של אחר ביאה ראשונה ותחפנו שכבת זרע לאחר ביאה שנייה ואמרו להם ב"ה אף לדבריכם שמא בקנוח ראשון עצמו נמוקה הטפה ובטלה בשכבת זרע.\par ומצאתי בתוספו' שפי' כדבריו בשמו של ר' שמואל רומרוגי ז"ל והם הקשו ללשון רש"י ממה ששנינו צריכין ב' עדים על כל תשמיש ותשמיש או תשמש לאור הנר ומשמע דמשמשת לאור הנר אינה צריכה אלא כדברי ב"ה ולפירושו עדיין המשמשת לאור הנר צריכה לבדוק ולאחר כל תשמיש ותשמיש ולראו' בעד כדברי ב"ש וב"ה אינה צריכה בדיקה כלל עד סוף כל הלילה מ"מ לשון תראה מתפרש לנו יפה. }
ה"ק ובא כהן והציץ בו לידע אם נפל הפילה אם רוח הפילה ואת"ל נפל הפילה לידע אם זכר אם נקבה 
ואיבעית אימא כיון דחולדה וברדלס מצויים שם ודאי גררוהו 
בעו מיניה מרב נחמן וסתות דאורייתא או דרבנן 
אמר להו מדאמר הונא חברין משמיה דרב אשה שיש לה וסת והגיע שעת וסתה ולא בדקה ולבסוף ראתה חוששת לוסתה וחוששת לראייתה אלמא וסתות דאורייתא 
איכא דאמרי הכי קא"ל טעמא דראתה הא לא ראתה אין חוששין אלמא וסתות דרבנן 
איתמר אשה שיש לה וסת והגיע שעת וסתה ולא בדקה ולבסוף בדקה אמר רב בדקה ומצאת טמאה טמאה טהורה טהורה ושמואל אמר אפילו בדקה ומצאת טהורה נמי טמאה מפני שאורח בזמנו בא 
לימא בוסתות קמיפלגי דמ"ס דאורייתא ומ"ס דרבנן 
אמר ר' זירא דכ"ע וסתות דאורייתא כאן שבדקה עצמה כשיעור וסת כאן שלא בדקה עצמה כשיעור וסת 
ר"נ בר יצחק אמר בוסתות גופייהו קמיפלגי דמ"ס וסתות דאורייתא ומר סבר וסתות דרבנן 
אמר רב ששת כתנאי ר' אליעזר אומר טמאה נדה
ורבי יהושע אומר תבדק והני תנאי כי הני תנאי דתניא רבי מאיר אומר טמאה נדה וחכ"א תבדק 
אמר אביי אף אנן נמי תנינא דתנן ר"מ אומר אם היתה במחבא והגיע שעת וסתה ולא בדקה טהורה שחרדה מסלקת את הדמים טעמא דאיכא חרדה הא ליכא חרדה טמאה אלמא וסתות דאורייתא 
לימא הני תנאי בהא נמי פליגי דתניא הרואה דם מחמת מכה אפילו בתוך ימי נדתה טהורה דברי רשב"ג 
רבי אומר אם יש לה וסת חוששת לוסתה 
מאי לאו בהא קמיפלגי דמר סבר וסתות דאורייתא ומר סבר וסתות דרבנן 
אמר רבינא לא דכ"ע וסתות דרבנן והכא במקור מקומו טמא קמיפלגי 
רשב"ג סבר אשה טהורה ודם טמא דקאתי דרך מקור 
ואמר ליה רבי אי חיישת לוסת אשה נמי טמאה ואי לא חיישת לוסת מקור מקומו טהור הוא
{\large\emph{מתני׳}} בית שמאי אומרים צריכה ב' עדים על כל תשמיש ותשמיש או תשמש לאור הנר בית הלל אומרים דיה בשני עדים כל הלילה:
{\large\emph{גמ׳}} ת"ר אע"פ שאמרו המשמש מטתו לאור הנר הרי זה מגונה בש"א צריכה שני עדים על כל תשמיש או תשמש לאור הנר ובה"א דיה בשני עדים כל הלילה 
\commenta{והא דאמרי' \textbf{תניא נמי הכי} לא שהוא לבעל נפש אלא שהוא דוקא לטהרות ולא לגבי הבעל וכיוצא בזה תניא נמי הכי שאינו לראיה ממש בפרק קמא דמגילה מפני שעיניהם של עניים נשואות למקרא מגילה וכו' ובתוספות מפרשים לא יבעול וישנה בלא בדיקה הא בבדיקה מותר אפילו לחסיד שבחסידים מדאמרינן במס' שבת אמר ר' יוסי ה' בעילות בעלתי ושניתי, ואין סוגיא מתחוורת בפי' הזה. }
תניא אמרו להם ב"ש לב"ה לדבריכם ליחוש שמא תראה טיפת דם כחרדל בביאה ראשונה ותחפנה שכבת זרע בביאה שניה 
\commenta{\textbf{בדקה בעד ואבד אסורה לשמש עד שתבדוק.} לדברי רש"י ז"ל בעד שלפני תשמיש ופשוט' היא, ולדברינו כגון שהחמירה ובדקה לאחר תשמיש ראשון או שהיתה דעתה שלא לשמש כל הלילה ומכיון שאין מוכיחה קיים ולא תדע אם ראתה כלום מחמת תשמיש זה אסורה לשמש עד שתבדוק לפני תשמיש לאור הנר בכל בדיקה של בעל.\par ואקשינן אלו קנחה בו ואיתי' מי לא משמשה אע"ג דלא ידעה הא אפילו ב"ש דמחמרי שרו לבדוק ולהניח ולשמש והאי דאקשינן הכי ולא אקשינן מדב"ה דלא מצרכי בדיקה זו דמי מדמי מקשינן דאלו התם דיומא משום קולא דבעל הוא שלא להטריח עליו בין תשמיש לתשמיש אבל מכיון שבדקה דנמלך צריכה היא בדיקה ממש לאור הנר.\par ומפרק זו מוכיחה קיים לטהרות וזו אין מוכיחה קיים לטהרות שמא ראתה לאחר תשמיש זה ולא תדע למחר הילכך אף לבעלה אסור' משום מגו של טהרות וכן נמי לב"ה דלא מצרכי הך בדיקה מוכיחה בעד שלאחר תשמי' אחרון ואין חוששין לנימוק אבל זו כבר נתקנח הדם בזה ואבד הילכך אסורה עד שתחזור ותבדוק עכשיו לפני תשמיש דבהכי ודאי מותר' שא"א להחמי' עליה יותר מכאן לא יהא זה חמור מן הוסת שאם בדקה ומצאה טהור טהור. ואפילו לטהרות עצמן אם בטלה בדיקה של עונה אחת כגון של שחרית או של ערבית בודקת עכשיו ועוסקת בהן וכן בבדיקה של אחר תשמיש בין לטהרות בין לבעלה. }
א"ל ב"ה אף לדבריכם ליחוש עד שהרוק בתוך הפה שמא נימוק והולך לו 
אמרו להם לפי שאינו דומה נימוק פעם אחת לנימוק שתי פעמים 
תניא א"ר יהושע רואה אני את דברי ב"ש אמרו לו תלמידיו רבי כמה הארכת עלינו אמר להם מוטב שאאריך עליכם בעוה"ז כדי שיאריכו ימיכם לעוה"ב 
אמר ר' זירא מדברי כולם נלמד בעל נפש לא יבעול וישנה 
רבא אמר בועל ושונה כי תניא ההיא לטהרות 
תניא נמי הכי בד"א לטהרות אבל לבעלה מותרת ובד"א שהניחה בחזקת טהרה אבל הניחה בחזקת טמאה לעולם היא בחזקתה עד שתאמר לו טהורה אני 
א"ר אבא א"ר חייא בר אשי אמר רב בדקה בעד ואבד אסורה לשמש עד שתבדוק מתקיף לה ר' אילא אילו איתא מי לא משמשה ואע"ג דלא ידעה השתא נמי תשמש 
א"ל רבא זו מוכיחה קיים וזו אין מוכיחה קיים
א"ר יוחנן  אסור לאדם שישמש מטתו ביום אמר רב המנונא מאי קרא שנאמר (איוב ג, ג) יאבד יום אולד בו והלילה אמר הורה גבר לילה ניתן להריון ויום לא ניתן להריון ריש לקיש אמר מהכא (משלי יט, טז) בוזה דרכיו ימות 
ור"ל האי קרא דר' יוחנן מאי דריש ביה מבעי ליה לכדדריש רבי חנינא בר פפא דדריש ר' חנינא בר פפא אותו מלאך הממונה על ההריון לילה שמו ונוטל טפה ומעמידה לפני הקב"ה ואומר לפניו רבש"ע טפה זו מה תהא עליה גבור או חלש חכם או טיפש עשיר או עני 
ואילו רשע או צדיק לא קאמר כדר' חנינא דא"ר חנינא הכל בידי שמים חוץ מיראת שמים שנאמר (דברים י, יב) ועתה ישראל מה ה' אלהיך שואל מעמך כי אם ליראה וגו' 
ור' יוחנן א"כ נכתוב קרא גבר הורה מאי הורה גבר לילה ניתן להריון ויום לא ניתן להריון 
ור' יוחנן האי קרא דר"ל מאי דריש ביה מבעי לי' לכדכתיב בספר בן סירא שלשה שנאתי וארבעה לא אהבתי שר הנרגל בבית המשתאות ואמרי לה שר הנרגן ואמרי לה שר הנרגז
והמושיב שבת במרומי קרת והאוחז באמה ומשתין מים והנכנס לבית חבירו פתאום אמר רבי יוחנן ואפילו לביתו 
אמר רבי שמעון בן יוחאי ארבעה דברים הקב"ה שונאן ואני איני אוהבן הנכנס לביתו פתאום ואצ"ל לבית חבירו והאוחז באמה ומשתין מים}

\newsection{דף יז}
\twocol{ומשתין מים ערום לפני מטתו והמשמש מטתו בפני כל חי אמר ליה רב יהודה לשמואל ואפי' לפני עכברים א"ל שיננא לא אלא כגון של בית פלוני שמשמשין מטותיהן בפני עבדיהם ושפחותיהם 
\commenta{ הא דאמרינן לצפרנים\textbf{ולא אמרן וכו'.} ואסיקנא ולא היא לכולי מילתא חיישינן מדגרסינן פ' ואלו מגלחין ר' יוחנן שקל טופריה בשיניה וזרקינהו בי מדרשא ואקשינן עליה היכי עביד הכי והתניא זורקן רשע ופריק אשה בבי מדרשא לא שכיחי וכ"ת דילמא כנשי להו לבראי הואיל ואישתני אישתני אלמא שאפי' בלא גנוסטרי ודידה בלחו' אסור. }
ואינהו מאי דרוש (בראשית כב, ה) שבו לכם פה עם החמור עם הדומה לחמור 
\commenta{ הא \textbf{דאמר אביי כגון שהעבירה על אויר התנור.} חדא מתרי טעמיה נקט דה"נ אפשר לאוקומה כגון שנטמא באהל המת א"נ בהיסט הזב וזבה וכל המטמאים במשא. }
רבה בר רב הונא מקרקש זגי דכילתא אביי באלי דידבי רבא באלי פרוחי 
אמר ר"ש בן יוחי ה' דברים הן שהעושה אותן מתחייב בנפשו ודמו בראשו האוכל שום קלוף ובצל קלוף וביצה קלופה והשותה משקין מזוגין שעבר עליהן הלילה והלן בבית הקברות והנוטל צפרניו וזורקן לרה"ר והמקיז דם ומשמש מטתו
האוכל שום קלוף כו' ואע"ג דמנחי בסילתא ומציירי וחתימי רוח רעה שורה עליהן ולא אמרן אלא דלא שייר בהן עיקרן או קליפתן אבל שייר בהן עיקרן או קליפתן לית לן בה
והשותה משקין מזוגין שעבר עליהן הלילה אמר רב יהודה אמר שמואל והוא שלנו בכלי מתכות אמר רב פפא וכלי נתר ככלי מתכות דמו וכן אמר רבי יוחנן והוא שלנו בכלי מתכות וכלי נתר ככלי מתכות דמו
והלן בבית הקברות כדי שתשרה עליו רוח טומאה זימנין דמסכנין ליה
והנוטל צפרניו וזורקן לרשות הרבים מפני שאשה מעוברת עוברת עליהן ומפלת ולא אמרן אלא דשקיל בגנוסטרי ולא אמרן אלא דשקיל דידיה ודכרעיה ולא אמרן אלא דלא גז מידי בתרייהו אבל גז מידי בתרייהו לית לן בה ולא היא לכולה מילתא חיישינן 
ת"ר ג' דברים נאמרו בצפרנים שורפן חסיד קוברן צדיק זורקן רשע
והמקיז דם ומשמש מטתו דאמר מר מקיז דם ומשמש מטתו הויין לו בנים ויתקין הקיזו שניהם ושמשו הויין לו בנים בעלי ראתן אמר רב ולא אמרן אלא דלא טעים מידי אבל טעים מידי לית לן בה 
אמר רב חסדא אסור לו לאדם שישמש מטתו ביום שנאמר (ויקרא יט, יח) ואהבת לרעך כמוך מאי משמע אמר אביי שמא יראה בה דבר מגונה ותתגנה עליו אמר רב הונא ישראל קדושים הם ואין משמשין מטותיהן ביום 
אמר רבא ואם היה בית אפל מותר ות"ח מאפיל בכסותו ומשמש 
תנן או תשמש לאור הנר אימא תבדוק לאור הנר 
ת"ש אע"פ שאמרו המשמש מטתו לאור הנר הרי זה מגונה אימא הבודק מטתו לאור הנר הרי זה מגונה 
תא שמע ושל בית מונבז המלך היו עושין ג' דברים ומזכירין אותן לשבח היו משמשין מטותיהם ביום ובודקין מטותיהם במילא פרהבא ונוהגין טומאה וטהרה בשלגים קתני מיהא משמשין מטותיהן ביום 
אימא בודקין מטותיהם ביום הכי נמי מסתברא דאי ס"ד משמשין מזכירין אותן לשבח אין ה"נ דאגב דאיכא אונס שינה מגניא באפיה 
ובודקין מטותיהן במילא פרהבא מסייע ליה לשמואל דאמר שמואל אין בודקין את המטה אלא בפקולין או בצמר נקי ורך אמר רב היינו דכי הואי התם בערבי שבתות הוו אמרי מאן בעי פקולי בנהמא ולא ידענא מאי קאמרי 
אמר רבא הני שחקי דכיתנא מעלי לבדיקה איני והא תנא דבי מנשה אין בודקין את המטה לא בעד אדום ולא בעד שחור ולא בפשתן אלא בפקולין או בצמר נקי ורך 
לא קשיא הא בכיתנא הא במאני דכיתנא ואיבעית אימא הא והא במאני דכיתנא הא בחדתי הא בשחקי 
נוהגין טומאה וטהרה בשלגין תנן התם שלג אינו לא אוכל ולא משקה חישב עליו לאכילה אינו מטמא טומאת אוכלין למשקה מטמא טומאת משקין 
נטמא מקצתו לא נטמא כולו נטהר מקצתו נטהר כולו 
הא גופא קשיא אמרת נטמא מקצתו לא נטמא כולו והדר תני נטהר מקצתו נטהר כולו למימרא דנטמא כולו 
אמר אביי כגון שהעבירו על אויר תנור דהתורה העידה על כלי חרס
אפילו מלא חרדל 
{\large\emph{מתני׳}} משל משלו חכמים באשה החדר והפרוזדור והעלייה 
דם החדר טמא דם העלייה טהור נמצא בפרוזדור ספקו טמא לפי שחזקתו מן המקור
{\large\emph{גמ׳}} רמי בר שמואל ורב יצחק בריה דרב יהודה תנו נדה בי רב הונא אשכחינהו רבה בר רב הונא דיתבי וקאמרי החדר מבפנים והפרוזדור מבחוץ ועלייה בנויה על שתיהן ולול פתוח בין עלייה לפרוזדור
 נמצא מן הלול ולפנים ספקו טמא מן הלול ולחוץ ספקו טהור 
אתא ואמר ליה לאבוה ספקו טמא אמרת לן מר והא אנן שחזקתו מן המקור תנן 
א"ל אנא הכי קאמינא מן הלול ולפנים ודאי טמא מן הלול ולחוץ ספקו טמא 
אמר אביי מאי שנא מן הלול ולחוץ דספקו טמא דדלמא שחתה ומחדר אתא מן הלול ולפנים נמי אימא אזדקרה ומעלייה אתא 
אלא אמר אביי אי בתר חששא אזלת אידי ואידי ספק הוא ואי בתר חזקה אזלת מן הלול ולפנים ודאי טמא מן הלול ולחוץ ודאי טהור 
תני רבי חייא דם הנמצא בפרוזדור חייבין עליו על ביאת מקדש ושורפין עליו את התרומה ורב קטינא אמר אין חייבין עליו על ביאת מקדש ואין שורפין עליו את התרומה 
להך לישנא דאמר אביי אי בתר חששא אזלת מסייע ליה לרב קטינא ופליגא דרבי חייא
להך לישנא דאמרת אי בתר חזקה אזלת מסייע ליה לרבי חייא}

\newsection{דף יח}
\twocol{ופליגא דרב קטינא 
לרב הונא לא פליגי כאן מן הלול ולפנים כאן מן הלול ולחוץ 
אלא לרמי בר שמואל ולרב יצחק בריה דרב יהודה דאמרי מן הלול ולחוץ ספקו טהור מן הלול ולפנים ספקו טמא הני במאי מתוקמא מן הלול ולפנים
לימא פליגא דרבי חייא 
לא קשיא כאן כשנמצא בקרקע פרוזדור וכאן שנמצא בגג פרוזדור 
אמר רבי יוחנן בשלשה מקומות הלכו בו חכמים אחר הרוב ועשאום כודאי מקור שליא חתיכה מקור הא דאמרן
שליא דתנן שליא בבית הבית טמא ולא שהשליא ולד אלא שאין שליא בלא ולד ר"ש אומר נמוק הולד עד שלא יצא 
חתיכה דתנן המפלת יד חתוכה ורגל חתוכה אמו טמאה לידה ואין חוששין שמא מגוף אטום באת 
ותו ליכא והאיכא תשע חנויות 
דתניא תשע חנויות כולן מוכרות בשר שחוטה ואחת מוכרת בשר נבלה ולקח מאחת מהן ואינו יודע מאיזה מהן לקח ספקו אסור
ובנמצא הלך אחר הרוב 
טומאה קאמרינן איסור לא קאמרינן 
והאיכא תשע צפרדעין ושרץ אחד ביניהם ונגע באחד מהן ואינו יודע באיזה מהן נגע ברה"י ספקו טמא ברה"ר ספקו טהור
ובנמצא הלך אחר הרוב 
טומאה דאשה קאמרינן טומאה בעלמא לא קאמרינן 
והאיכא הא דאמר רבי יהושע בן לוי עברה בנהר
והפילה מביאה קרבן ונאכל
\commenta{הא דאמרינן \textbf{מאי לאו לא תיובתי' אלא סייעתיה וכו'.} ה"פ: דקסלקא דעתך מדקתני מתניתין דטועה מטבילין אותה צ"ה טבילות שמחמירין עליה כל חומרי טבילות הללו ש"מ בודאי ילדה (מצוי) א) שאלמלא יש ספק בולד ה"ל ספק ספיקא ספק זמן טבילה היום ספק אינו זמן ואת"ל הוא ספק אינו ולד ודיה טבילה אחת באחרונה כרבי יוסי לר' יהודה דלא מחמירין כולי האי בטבילה בזמנה מצוה אפילו דתרי ספיקי אי נמי בחד ספיקא בספק לידה כלל.\par ופריק דילמא לא תיובתיה ולא סייעתיה דכיון דאיכא רוב הולכין אחריו אפילו בתרי ספיקא להחמיר ואפילו לכתחלה אבל להקל לא עשאוהו בודאי.\par ויש אומרים מפני שהצריכוה למיטבל בשבוע ג' משום יולדת דזוב וספיקי טובא נינהו שמא ילדה שמא לא ילדה ואם תמצא לומר ילדה בזוב אימר עלו לה ימי שבוע שני לספירה אם הרחיקה לידתה ואינו נכון דהשתא נמי דהויא ודאי הויין ספיקות טובא אלא סלקא דעתך דבספק לידה לא מחמרינן בטבילה בזמנה. }
הלך אחר רוב נשים ורוב נשים ולד מעליא ילדן 
\commenta{\textbf{אלא למעוטי רובא דרבי יהודה.} פירש דאפילו לרבי יהודה לאו רוב גמור הוא לשרוף אלא לתלות, וק"ל דהא איתותב ההוא לישנא ואסיקנא דבאי אפשר לפתיחת קבר בלא דם קמפלגי ובפיר' רש"י ז"ל אפילו ללישנא בתרא דר' יוחנן דאמר טעמיה משום דאי אפשר לפתיחת קבר בלא דם אפילו הכי טעמיה דר' יהודה משום רוב דברוב פתיחת קבר איכא דם ולא עשאו ברוב זה כודאי דכיון דאין עמה דם איתרע ליה.\par וק"ל דהא אוקימנא לר' יהודה כרבי יהושע דמשוה ליה חדא ב) דאמר מביא קרבן ונאכל דאי אפשר לפתיחת קבר בלא דם וא"ל סבר לה כרבי יהושע לתלות ולא לשרוף, ויש לומר דמאן דמתני בשלשה מקומות מתני ההוא לישנא קמא\par דר' יוחנן דמוקי פלוגתייהו דר' יהודה ורבנן בשאינה יודעת מה הפילה. וכן עיקר שאלמלא כן לא הזכירו בגמרא כאן לשון ראשון שהוא טעות במקום עיקרו. }
מתניתין קאמרינן שמעתתא לא קאמרינן 
והא כי אתא רבין אמר מתיב רבי יוסי בר רבי חנינא טועה ולא ידענא מאי תיובתיה
מאי לאו לא תיובתא אלא סייעתא 
לא דלמא לא תיובתא ולא סייעתא 
למעוטי מאי 
אילימא למעוטי רובא דאיכא חזקה בהדיה דלא שרפינן עליה את התרומה והא אמרה ר' יוחנן חדא זימנא 
דתנן תינוק הנמצא בצד העיסה ובצק בידו רבי מאיר מטהר וחכמים מטמאין שדרכו של תינוק לטפח 
ואמרינן מאי טעמא דר"מ קסבר רוב תינוקות מטפחין ומיעוט אין מטפחין ועיסה זו בחזקת טהורה עומדת סמוך מיעוטא לחזקה ואיתרע ליה רובא 
ורבנן מיעוטא כמאן דליתיה דמי ורובא וחזקה רובא עדיף 
ואמר ריש לקיש משום רבי אושעיא זו היא חזקה ששורפין עליה את התרומה ורבי יוחנן אמר אין זו חזקה ששורפין עליה את התרומה 
אלא למעוטי רובא דרבי יהודה דתנן המפלת חתיכה אם יש עמה דם טמאה ואם לאו טהורה רבי יהודה אומר בין כך ובין כך טמאה 
ואמר רב יהודה אמר שמואל לא טימא רבי יהודה אלא בחתיכה של ארבע מיני דמים אבל שאר מיני דמים טהורה ורבי יוחנן אמר של ארבע מיני דמים דברי הכל טמאה ושל שאר דמים דברי הכל טהורה לא נחלקו אלא כשהפילה}

\newsection{דף יט}
\twocol{ואינה יודעת מה הפילה רבי יהודה סבר זיל בתר רוב חתיכות ורוב חתיכות של ארבע מיני דמים הויין ורבנן סברי זיל בתר רוב חתיכות לא אמרינן
{\large\emph{מתני׳}} חמשה דמים טמאים באשה האדום והשחור וכקרן כרכום וכמימי אדמה וכמזוג בש"א אף כמימי תלתן וכמימי בשר צלי וב"ה מטהרים הירוק עקביא בן מהללאל מטמא וחכמים מטהרין 
אמר רבי מאיר אם אינו מטמא משום כתם מטמא משום משקה רבי יוסי אומר לא כך ולא כך 
איזהו אדום כדם המכה שחור כחרת עמוק מכן טמא דיהה מכן טהור וכקרן כרכום כברור שבו 
וכמימי אדמה מבקעת בית כרם ומיצף מים וכמזוג שני חלקים מים ואחד יין מן היין השרוני
{\large\emph{גמ׳}} מנלן דאיכא דם טהור באשה דלמא כל דם דאתי מינה טמא 
אמר רבי חמא בר יוסף אמר רבי אושעיא אמר קרא (דברים יז, ח) כי יפלא ממך דבר למשפט בין דם לדם בין דם טהור לדם טמא 
אלא מעתה בין נגע לנגע הכי נמי בין נגע טמא לנגע טהור וכי תימא ה"נ נגע טהור מי איכא וכי תימא (ויקרא יג, יג) כולו הפך לבן טהור הוא ההוא בוהק מקרי 
אלא בין נגעי אדם לנגעי בתים ולנגעי בגדים וכולן טמאין הכא נמי בין דם נדה לדם זיבה וכולן טמאין 
האי מאי בשלמא התם איכא לאפלוגי בנגעי אדם ובפלוגתא דרבי יהושע ורבנן
דתנן אם בהרת קודם לשער לבן טמא ואם שער לבן קודם לבהרת טהור ספק טמא ורבי יהושע אומר כהה ואמר רבה כהה וטהור 
בנגעי בתים כי הא פלוגתא דרבי אלעזר ברבי שמעון ורבנן דתנן ר"א בר"ש אומר לעולם אין הבית טמא עד שיראה כשני גריסין על שני אבנים בשני כותלים בקרן זוית ארכו כשני גריסין ורחבו כגריס 
מ"ט דר"א בר"ש כתיב (ויקרא יד) קיר וכתיב קירות איזהו קיר שהוא כשני קירות הוי אומר זה קרן זוית 
בנגעי בגדים בפלוגתא דר' יונתן בן אבטולמוס ורבנן דתניא ר' יונתן בן אבטולמוס אומר מנין לפריחת בגדים שהיא טהורה
נאמר {ויקרא י״ג:ל״ט } קרחת וגבחת בבגדים ונאמר קרחת וגבחת באדם
מה להלן פרח בכולו טהור אף כאן נמי פרח בכולו טהור 
אלא הכא אי דם טהור ליכא במאי פליגי 
וממאי דהני טהורין והני טמאין אמר רבי אבהו דאמר קרא (מלכים ב ג:כב) ויראו מואב את המים אדומים כדם למימרא דדם אדום הוא אימא אדום ותו לא 
א"ר אבהו אמר קרא {ויקרא יב, ז} דמיה {ויקרא כ, יח} דמיה הרי כאן ארבעה 
והא אנן חמשה תנן אמר רבי חנינא שחור אדום הוא אלא שלקה 
תניא נמי הכי שחור כחרת עמוק מכן טמא דיהה אפי' ככחול טהור ושחור זה לא מתחלתו הוא משחיר אלא כשנעקר הוא משחיר משל לדם מכה לכשנעקר הוא משחיר
בש"א אף כמימי תלתן ולית להו לב"ש דמיה דמיה הרי כאן ארבעה 
אב"א לית להו ואב"א אית להו מי לא א"ר חנינא שחור אדום הוא אלא שלקה ה"נ מלקא הוא דלקי
וב"ה מטהרין היינו תנא קמא 
איכא בינייהו
לתלות 
\commenta{\textbf{ירד ר"מ לשיטתו של עקביא וטימא וה"ק להו וכו'.} י"מ כל שמועה זו אחר דבריו של ר' יוחנן דהוא אמר ירד ר"מ לשיטתו של עקביא וטימא משום דקאמר אם אינו מטמא משום כתם משום דדם ירוק באשה לא שכיח א"נ לרבנן דחולין קמ"ל נהי דספיקא משויתו ליה גבי כתם דלקולא גבי רואה ממש תהא טמאה נדה לשרוף והיינו כעקביא.\par ואקשינן עליה דר' יוחנן א"ה אם אינו מטמא משום כתם וכו' משום רואה מיבעי לי' אלא ה"ק להו וכו' קושיא היא ומהדרינן לקיימא לדר' יוחנן א"ה שפיר קאמרי ליה אלא ה"ק להו וליהוי כמשקה להכשיר את הזרעים דדם נדה ודאי מכשיר את הזרעים ומטמאי' אותם כדתניא בתוספתא מסכת שבת פ"ח מנין לדם נדה שהוא משקה שנאמר ממקור דמיה ונאמר ביום ההוא יהי' מקור נפתח וכו' ואמאי לא שויתי' ליה מיהא כדם נדה להכשיר שהרי אף בשאר דמים דעלמא יש מכשירין והיאך אתם מקילין על זה שלא לעשותו כדם נדה אפילו בדבר שמצינו שאר דברים דינן כך ורבנן בעינא דם חללים וכל שאינו דם חללים טהור הילכך זה שאינו דם חללים טהור הילכך זה שאינו דם נדה לדברינו כשאר דמים שאין בהם חלל נדון אותו.\par ואקשינן לר' יוחנן נמי שפיר קאמרי ליה ה"ק להו אלפוה בג"ש כתיב הכא שלחיך פרדס רמונים כלומר כיון שבדם זה אסורה לבעלה וקרינן בה גן נעול יכשיר דעלה כתיב שלחיך ואמרי ליה רבנן אין אדם ג"ש מעצמו ואפילו לדברינו שתולין ואוסרין אותה לבעלה גבי הכשר לקולא ואין פי' זה אלא דברי נביאות.\par ואחרים פירשו השמועה כפשטה אלא שאמרו שר' יוחנן הוא דפריש וה"ק נהי דלא מטמא משום כתם תטמא משום רואה וכי אפרי' לההיא לישנא איפריך ליה דר' יוחנן, ול"נ שלא אמר ר' יוחנן ירד ר"מ לשיטתו של עקביא אלא מפני שאמר במשנתינו אם אינו מטמא דמשמע דלדבריהם אמר להו כלומר נהי דמקילתו בזה אודו לי מיהת בזו הא איהו כעקביא ס"ל ומטמא בזו ובזו הילכך לכולהו לישני איתא לדר' יוחנן וכולהו נמי קאמר להו כלומר לדבריכם כדפרישית וגמרא הוא דפריש ואזיל וה"ק להו וכו'.\par ומיהו הא ק"ל היכי א"ר יוחנן ירד ר"מ לשיטתו של עקביא וטמא והא לקמן בפ' בנות כותים א"ר מאיר אם יושבת הן על כל דם תקנה גדולה היא להן אלא שרואה דם אדום ומשלימתו לדם ירוק אלמא כרבנן ס"ל ואפשר דהתם לרבנן קאמר להו ונקט מראה טהור לדבריהם ולאו דוקא. }
הירוק עקביא בן מהללאל מטמא ולית ליה לעקביא דמיה דמיה הרי כאן ארבעה 
\commenta{ והא דאמרינן \textbf{אלפוה בג"ש דכתיב הכא שלחיך פרדס רמונים.} ודאי קשיא דר"מ לאו בכל מה שאשה משלחת מטמא משום הכשר דא"כ מאי איריא דם ירוק דנקט אלא בדם נדה בלחוד הוא דקא מכשיר מדכתיב גן נעול וכתיב פרדס מה פרדס הזה נעול להשתמר כך בנות ישראל נועלות פתחיהן לבעליהן וא"כ ה"ק נהי דלא מטמיתו משום דם נדה מכשיר והלא אין לך מכשיר אלא דם נדה והוא דם טהור הוא לדבריהם.\par וזו שאלה רבז"ל והשיב לר"מ כל היכא דחזיא דם אדום והדר קא חזיא כל דם מכשיר שכל זמן שהיא כפרד' שלחיך קרינן ביה וה"ק להו לדידי דם טמא הוא מכשיר מתחלתו אלא לדידכו אודו לי איהי מיהת דהיכא דחזיא דם אדום מעיקרא והויא פרדס אפילו ירוק יכשיר דקרינן ביה שלחיך ואמרו ליה ג"ש לא אמרינן הילכך אפילו דם אדום עצמו אינו מכשיר. }
אב"א לית ליה ואב"א אית ליה מי לא א"ר חנינא שחור אדום הוא אלא שלקה הכא נמי מלקא הוא דלקי
\commenta{לשמואל דאמר \textbf{כדם שור שחוט ולעולא דאמר של צפור חיה ולרב נחמן דאמר של הקזה.} ל"ק הא דתנן הרגה מאכולות תולה בה ובבנה ובבעלה דא"ל חד שיעורא הוא אלא לזעירי בלחוד דפריש הוא דמקשינן מינה הילכך אית לן לפרושי דכולהו לא פליגי אלא מר בקי בהאי חזותא ומר בקי באידך ודכולהו ודברייתא נמי חד הוא כדאמרן, והא דאקשינן הרגה מאכולת הרי זה תולה בה מאי לאו דכוליה גופא ואקשינן נמי תולה בבנה ובבעלה בשלמא בנה משכחת לה משמע דלא תלינן בכתמים אלא בדדמי ומקיפין ורואין ומכאן החמירו.\par ונמצא במקצת גליוני ראשונים דעכשיו בזמן הזה כיון שבטלו ראיית דמים ואין בקיאות במראיהן של ד' דמים אין תולין בכתמים ולא בבן ולא בבעל ולא במאכולת ושוק של טבחים ושאר כל מה ששנו חכמים בכתמים לתלות אלא בכולן אסורות לבעלה.\par ורבינו בעל התוספות השיב דכי אמרינן עברה בשוק של טבחים תולה וכן במאכולת בכולן מעצמה קאמרינן דתולה ולא מגופא אתא אלא מעלמא אתא, אלא מיהו היכא דידעינן ודאי דלא דמו ליכא למיתלי ומ"ה קאמרינן הכא בשמעתין למ"ד כדם מאכולת של ראש הרגה ודאי מאכולת של גוף היאך תולה בשאינו דומה ודאי וכן קושיא דבנה ובעלה אבל מסתמא תולין כתמים בכל מה שאמרו חכמים.\par וכענין זה כתב הראב"ד ז"ל דתולין מן הסתם כל מיני אדום באדום וכל מיני שחור בשחור עד שיתברר לה ודאי שאין אודם הצבע דומה לאודם הכתם דהתם אינו תולה כדתניא לקמן נתעסקה באדום אין תולה בו שחור ובשמעתין הכי אקשינן היאך אפשר לתלותו במכת בעלה ובדם מאכולת הגוף והלא דבר ברור הוא שאינו דומה, אבל מי שאינה יודעת בדמיונות או שהלך הצבע מנגד פניה ואינה יכולה לדמות תולה מן הסתם כדאמרינן באשה שבאת לפני ר' עקיבא ואמרה לו ראיתי כתם שמא מכה יש בידך וטהרה והרי אשה זו לא הביאה הכתם בידה לפניו וטהרה מיד ולא הקיף ולא ידע כל זה שכתב הרב ז"ל.\par וכן יש לפרש זו שהקשו בשמעתין ממאכולת ובעלה דה"ק ודאי מדבעין תליה בהני אלמא בכתם טמא עסקינן ואלו הנך דדמו כתמים טהורים הם ומה נפשך אי לא דמו לא תליא ואי דמו לא צריכי תלייה כלל, ולא בעי לתרוצי כשראתה ואבד ואינה יודעת מה ראתה דמסתמא ברואה ובאה לפני חכם תנן וידע דאיכא טובא מתירין הלכך בזמן שאין בקיאין תולה במאכולת הגוף ובבעלה ובכל מיני אדמות עד שיתברר לפי הדעת שאינן דומין וכן נהגו ואין לחוש.\par ושוב ראינו בכתמים שכתב הראב"ד ז"ל עיינתי בכל מילי דרבוואתא ולא אשכחית בהון דינא דכתמים אי נהיגי האידנא או לא, ואף על גב דאשכחן דמטמא את בועלה בפרק קמא דנדה דלמא לטהרות היא.\par וזה אינו נכון, וכבר השיב הרב חתנו ז"ל מדאמרינן בכתמים פעמים שהן מביאים לידי זיבה ואיתמר עלה מהו דתימא כל כהאי גוונא מביאה קרבן ונאכל קמשמע לן מביאה קרבן לאסרו לבעלה דאי לטהרות בלבד קרבן מאי עבידתיה דאפילו להכשיר בקדשים ליכא למיחש כיון דליתיה אלא דרבנן בעלמא לחוש בטהרות ולבעלה מתירין אותה לכתחלה.\par ועוד השיב מדאמרינן בהדיא בפרק הרואה (דף נח ע"ב) לדברי אין קץ שאין לך אשה שטהורה לבעלה שאין לך כל מטה ומטה שאין עליה כמה טיפי דם מאכולת לדברי חבירי אין סוף שאין לך אשה שאינה טהורה לבעלה וכו' אלמא כתמים לבעל נאמרו ואין צריך להאריך שהרי הוחזקו בנות ישראל שנהגו איסור בכתמים וקי"ל דמנהגא מילתא היא כרבי זירא, כל אלו דברי הרב ז"ל. }
וחכמים מטהרין היינו ת"ק איכא בינייהו לתלות
א"ר מאיר אם אינו מטמא משום כתם כו'
א"ר יוחנן ירד ר"מ לשיטת עקביא בן מהללאל וטימא וה"ק להו לרבנן נהי דהיכא דקא משכחת כתם ירוק אמנא לא מטמאיתו היכא דקחזיא דם ירוק מגופה תטמא 
אי הכי אם אינו מטמא משום כתם מטמא משום משקה משום רואה מבעיא ליה 
אלא ה"ק להו נהי היכא דקא חזיא דם ירוק מעיקרא לא מטמאיתו היכא דחזיא דם אדום והדר חזיא דם ירוק תטמא מידי דהוה אמשקה זב וזבה 
ורבנן דומיא דרוק מה רוק שמתעגל ויוצא אף כל שמתעגל ויוצא לאפוקי האי דאין מתעגל ויוצא אי הכי שפיר קאמרי ליה רבנן לר' מאיר 
אלא ה"ק להו להוי כמשקה להכשיר את הזרעים ורבנן בעי (במדבר כג:כד) דם חללים וליכא אי הכי שפיר קאמרי ליה רבנן לר' מאיר 
אלא הכי קאמר להו אלפוה בג"ש כתיב הכא (שיר השירים ד׳:י״ג) שלחיך פרדס רמונים וכתיב התם (איוב ה, י) ושולח מים על פני חוצות 
ורבנן אדם דן ק"ו מעצמו ואין אדם דן ג"ש מעצמו
רבי יוסי אומר לא כך וכו' היינו ת"ק הא קמ"ל מאן ת"ק רבי יוסי וכל האומר דבר בשם אומרו מביא גאולה לעולם
איזהו אדום כדם המכה מאי כדם המכה אמר רב יהודה אמר שמואל כדם שור שחוט 
ולימא כדם שחיטה אי אמר כדם שחיטה הוה אמינא ככולה שחיטה קמ"ל כדם המכה כתחילת הכאה של סכין 
עולא אמר כדם צפור חיה איבעיא להו חיה לאפוקי שחוט או דלמא לאפוקי כחוש תיקו 
זעירי אמר רבי חנינא כדם מאכולת של ראש מיתיבי הרגה מאכולת הרי זה תולה בה מאי לאו דכוליה גופה לא דראשה 
אמי ורדינאה א"ר אבהו כדם אצבע קטנה של יד שנגפה וחייתה וחזרה ונגפה ולא של כל אדם אלא של בחור שלא נשא אשה ועד כמה עד בן עשרים 
מיתיבי תולה בבנה ובבעלה בשלמא בבנה משכחת לה אלא בעלה היכי משכחת לה 
אמר ר"נ בר יצחק כגון שנכנסה לחופה ולא נבעלה 
ר"נ אמר כדם הקזה מיתיבי מעשה ותלה ר"מ}

\newsection{דף כ}
\twocol{בקילור ורבי תלה בשרף שקמה מאי לאו אאדום 
לא אשאר דמים 
אמימר ומר זוטרא ורב אשי הוו יתבי קמיה אומנא שקלי ליה קרנא קמייתא לאמימר חזייה אמר להו אדום דתנן כי האי שקלי ליה אחריתי אמר להו אשתני אמר רב אשי כגון אנא דלא ידענא בין האי להאי לא מבעי לי למחזי דמא
שחור כחרת אמר רבה בר רב הונא חרת שאמרו דיו תניא נמי הכי שחור כחרת ושחור שאמרו דיו ולימא דיו אי אמר דיו הוה אמינא כי פכחותא דדיותא קמ"ל כי חרותא דדיותא 
איבעיא להו בלחה או ביבשתא תא שמע דרבי אמי פלי קורטא דדיותא ובדיק בה 
אמר רב יהודה אמר שמואל כקיר כדיו וכענב טמאה וזוהי ששנינו עמוק מכן טמאה אמר רבי אלעזר כזית כזפת וכעורב טהור וזוהי ששנינו דיהה מכן טהור 
עולא אמר כלבושא סיואה עולא אקלע לפומבדיתא חזייה לההוא טייעא דלבוש לבושא אוכמא אמר להו שחור דתנן כי האי מרטו מיניה פורתא פורתא יהבו ביה ארבע מאה זוזי 
רבי יוחנן אמר אלו כלים האוליירין הבאים ממדינת הים למימרא דאוכמי נינהו והאמר להו רבי ינאי לבניו בני אל תקברוני לא בכלים שחורים ולא בכלים לבנים שחורים שמא אזכה ואהיה כאבל בין החתנים לבנים שמא לא אזכה ואהיה כחתן בין האבלים אלא בכלים האוליירין הבאים ממדינת הים 
אלמא לאו אוכמי נינהו לא קשיא הא בגלימא הא בפתורא 
אמר רב יהודה אמר שמואל וכולם אין בודקין אלא על גבי מטלית לבנה אמר רב יצחק בר אבודימי ושחור על גבי אדום 
אמר רב ירמיה מדפתי ולא פליגי הא בשחור הא בשאר דמים מתקיף לה רב אשי אי הכי לימא שמואל חוץ משחור אלא אמר רב אשי בשחור גופיה קמיפלגי 
אמר עולא כולן עמוק מכן טמא דיהה מכן טהור כשחור 
ואלא מאי שנא שחור דנקט סד"א הואיל ואמר רבי חנינא שחור אדום הוא אלא שלקה הילכך אפילו דיהה מכן נמי ליטמא קמשמע לן 
רבי אמי בר אבא אמר וכולן עמוק מכן טמא דיהה מכן נמי טמא חוץ משחור אלא מאי אהני שיעוריה דרבנן לאפוקי דיהה דדיהה 
ואיכא דאמרי רמי בר אבא אמר וכולן עמוק מכן טהור דיהה מכן טהור חוץ משחור ולהכי מהני שיעוריה דרבנן 
בר קפרא אמר וכולן עמוק מכן טמא דיהה מכן טהור חוץ ממזג שעמוק מכן טהור דיהה מכן טהור בר קפרא אדיהו ליה ודכי אעמיקו ליה ודכי אמר רבי חנינא כמה נפיש גברא דלביה כמשמעתיה
וכקרן כרכום תנא לח ולא יבש 
תני חדא כתחתון ולא כעליון ותניא אידך כעליון ולא כתחתון ותניא אידך כעליון וכל שכן כתחתון ותניא אידך כתחתון וכל שכן כעליון 
אמר אביי תלתא דרי ותלתא טרפן הויין
נקוט דרא מציעאה וטרפא מציעתא בידך 
כי אתו לקמיה דרבי אבהו אמר להו בגושייהו שנינו
וכמימי אדמה תנו רבנן כמימי אדמה מביא אדמה שמנה מבקעת בית כרם ומציף עליה מים דברי רבי מאיר רבי עקיבא אומר מבקעת יודפת רבי יוסי אומר מבקעת סכני רבי שמעון אומר אף מבקעת גנוסר וכיוצא בהן 
תניא אידך וכמימי אדמה מביא אדמה שמנה מבקעת בית כרם ומציף עליה מים כקליפת השום ואין שיעור למים משום דאין שיעור לעפר ואין בודקין אותן צלולין אלא עכורין צללו חוזר ועוכרן וכשהוא עוכרן אין עוכרן ביד אלא בכלי 
איבעיא להו אין עוכרין אותן ביד אלא בכלי דלא לרמיה בידיה ולעכרינהו אבל במנא כי עכר ליה בידיה שפיר דמי או דלמא דלא לעכרינהו בידיה אלא במנא 
ת"ש כשהוא בודקן אין בודקן אלא בכוס ועדיין תבעי לך בדיקה בכוס עכירה במאי תיקו 
כי אתו לקמיה דרבה בר אבוה אמר להו במקומה שנינו רבי חנינא פלי קורטא דגרגשתא ובדיק ביה לייט עליה רבי ישמעאל ברבי יוסי באסכרה
רבי חנינא הוא דחכים כולי עלמא לאו חכימי הכי 
אמר רבי יוחנן חכמתא דרבי חנינא גרמא לי דלא אחזי דמא מטמינא מטהר מטהרנא מטמא אמר רבי אלעזר ענוותנותא דרבי חנינא גרמא לי דחזאי דמא ומה רבי חנינא דענותן הוא מחית נפשיה לספק וחזי אנא לא אחזי 
אמר רבי זירא טבעא דבבל גרמא לי דלא חזאי דמא דאמינא בטבעא לא ידענא בדמא ידענא 
למימרא דבטבעא תליא מלתא והא רבה הוא דידע בטבעא ולא ידע בדמא כל שכן קאמר ומה רבה דידע בטבעא לא חזא דמא ואנא אחזי 
עולא אקלע לפומבדיתא אייתו לקמיה דמא ולא חזא אמר ומה רבי אלעזר דמרא דארעא דישראל הוה כי מקלע לאתרא דר' יהודה לא חזי דמא אנא אחזי 
ואמאי קרו ליה מרא דארעא דישראל דההיא אתתא דאייתא דמא לקמיה דרבי אלעזר הוה יתיב רבי אמי קמיה ארחיה אמר לה האי דם חימוד הוא בתר דנפקה אטפל לה רבי אמי אמרה ליה בעלי היה בדרך וחמדתיו קרי עליה (תהלים כה, יד) סוד ה' ליראיו 
אפרא הורמיז אמיה דשבור מלכא שדרה דמא לקמיה דרבא הוה יתיב רב עובדיה קמיה ארחיה אמר לה האי דם חימוד הוא אמרה ליה לבריה תא חזי כמה חכימי יהודאי א"ל דלמא כסומא בארובה 
הדר שדרה ליה שתין מיני דמא וכולהו אמרינהו ההוא בתרא דם כנים הוה ולא ידע אסתייע מילתא ושדר לה סריקותא דמקטלא כלמי אמרה יהודאי בתווני דלבא יתביתו 
אמר רב יהודה מרישא הוה חזינא דמא כיון דאמרה לי אמיה דיצחק ברי האי טיפתא קמייתא לא מייתינן לה קמייהו דרבנן משום דזהימא לא חזינא
בין טמאה לטהורה ודאי חזינא 
ילתא אייתא דמא לקמיה דרבה בר בר חנה וטמי לה הדר אייתא לקמיה דרב יצחק בריה דרב יהודה ודכי לה 
והיכי עביד הכי והתניא חכם שטימא אין חברו רשאי לטהר אסר אין חבירו רשאי להתיר 
מעיקרא טמויי הוה מטמי לה כיון דא"ל דכל יומא הוה מדכי לי כי האי גונא והאידנא הוא דחש בעיניה דכי לה 
ומי מהימני אין והתניא נאמנת אשה לומר כזה ראיתי ואבדתיו 
איבעיא להו כזה טיהר איש פלוני חכם מהו 
תא שמע נאמנת אשה לומר כזה ראיתי ואבדתיו שאני התם דליתיה לקמה 
תא שמע דילתא אייתא דמא לקמיה דרבה בר בר חנה וטמי לה לקמיה דרב יצחק בריה דרב יהודה ודכי לה והיכי עביד הכי והתניא חכם שטימא אין חבירו רשאי לטהר וכו' 
ואמרינן טמויי הוה מטמי לה כיון דאמרה ליה דכל יומא מדכי לה כי האי גונא והאידנא הוא דחש בעיניה הדר דכי לה אלמא מהימנא לה 
רב יצחק בר יהודה אגמריה סמך 
רבי ראה דם בלילה וטימא ראה ביום וטיהר המתין שעה אחת חזר וטימא אמר אוי לי שמא טעיתי 
שמא טעיתי ודאי טעה דתניא לא יאמר חכם אילו היה לח היה ודאי טמא
אלא אמר אין לו לדיין אלא מה שעיניו רואות מעיקרא אחזקיה בטמא כיון דחזא לצפרא דאשתני אמר (ליה) ודאי טהור הוה ובלילה הוא דלא אתחזי שפיר כיון דחזא דהדר אשתני אמר האי טמא הוא ומפכח הוא דקא מפכח ואזיל 
רבי בדיק לאור הנר רבי ישמעאל ברבי יוסף בדיק ביום המעונן ביני עמודי אמר רב אמי בר שמואל וכולן אין בודקין אותן אלא בין חמה לצל רב נחמן אמר רבה בר אבוה בחמה ובצל ידו
וכמזוג שני חלקים כו' תנא}

\newsection{דף כא}
\twocol{השרוני נידון ככרמלי חי ולא מזוג חדש ולא ישן 
\commenta{\textbf{באפשר לפתיחת קבר בלא דם פליגי ובפלוגתא דהני תנאי וכו'.} ואי קשיא הא לת"ק דברייתא גופיה ספיקא משוי ליה ואלו ת"ק דמתני' קאמר ואי לאו טהורה א"ל התם משום דילדה ואינה יודעת מה ילדה וחוששין ללידה וחוששין נמי לזיבה שמא עם הנפל יצא דם אבל שילדה לידה יבישתא העמד אשה על חזקתה וטהורה היא ועוד שהרי בדקו ולא מצאו דם וכן ללשון הראשון שאמר רבנן סברי לא אמרי' רוב חתיכות מד' מיני דמים הן קשיא ותהוי נמי מחצה על מחצה תהא טמאה מספק אלא משום האי טעמא הוא דהעמד אשה על חזקתה.\par וי"מ דהתם ה"ק לא אמרינן רוב חתיכות מד' מינין הן ולפיכך טמאה גמורה אלא שאין שורפין כדאיתא בפ"ק, ואין פירוש זה נכון. }
אמר רב יצחק בר אבודימי וכולן אין בודקין אותן אלא בכוס טבריא פשוט מאי טעמא אמר אביי של כל העולם כולו מחזיק לוג עושין אותו ממנה שני לוגין עושין אותו ממאתים כוס טבריא פשוט אפי' מחזיק שני לוגין עושין אותו ממנה ואיידי דקליש ידיע ביה טפי
\par \par {\large\emph{הדרן עלך כל היד}}\par \par 
מתני׳ {\large\emph{המפלת}} חתיכה אם יש עמה דם טמאה ואם לאו טהורה ר' יהודה אומר בין כך ובין כך טמאה 
המפלת כמין קליפה כמין שערה כמין עפר כמין יבחושין אדומים תטיל למים אם נמוחו טמאה ואם לאו טהורה 
המפלת כמין דגים חגבים שקצים ורמשים אם יש עמהם דם טמאה ואם לאו טהורה 
המפלת מין בהמה חיה ועוף בין טמאין בין טהורין אם זכר תשב לזכר ואם נקבה תשב לנקבה
ואם אין ידוע תשב לזכר ולנקבה דברי רבי מאיר וחכמים אומרים כל שאין בו מצורת אדם אינו ולד
{\large\emph{גמ׳}} אמר רב יהודה אמר שמואל לא טימא רבי יהודה אלא בחתיכה של ארבעת מיני דמים אבל של שאר מיני דמים טהורה 
ור' יוחנן אמר של ארבעת מיני דמים דברי הכל טמאה של שאר מיני דמים דברי הכל טהורה 
לא נחלקו אלא שהפילה ואינה יודעת מה הפילה רבי יהודה סבר זיל בתר רוב חתיכות ורוב חתיכות של (מיני) ארבעת מיני דמים הויין ורבנן סברי לא אמרינן רוב חתיכות של ארבעת מיני דמים 
איני והא כי אתא רב הושעיא מנהרדעא אתא ואייתי מתניתא בידיה המפלת חתיכה אדומה שחורה ירוקה ולבנה אם יש עמה דם טמאה ואם לאו טהורה רבי יהודה אומר בין כך ובין כך טמאה קשיא לשמואל בחדא ולרבי יוחנן בתרתי 
לשמואל בחדא דאמר שמואל לא טימא רבי יהודה אלא בחתיכה של ארבעת מיני דמים והא קתני ירוקה ולבנה ופליג רבי יהודה 
וכי תימא כי פליג רבי יהודה אאדומה ושחורה ואירוקה ולבנה לא אלא ירוקה ולבנה למאן קתני לה 
אילימא רבנן השתא אדומה ושחורה מטהרי רבנן ירוקה ולבנה מיבעיא אלא לאו לרבי יהודה ופליג 
ותו לרבי יוחנן דאמר של ארבעת מיני דמים דברי הכל טמאה הא קתני אדומה ושחורה ופליגי רבנן 
וכי תימא כי פליגי רבנן אירוקה ולבנה אבל אאדומה ושחורה לא אלא אדומה ושחורה למאן קתני לה 
אילימא רבי יהודה השתא ירוקה ולבנה טמאה אדומה ושחורה מיבעיא אלא לאו רבנן ופליגי 
אלא אמר רב נחמן בר יצחק באפשר לפתיחת הקבר בלא דם קמיפלגי ובפלוגתא דהני תנאי דתניא קשתה שנים ולשלישי הפילה ואינה יודעת מה הפילה}

\newchap{פרק \hebrewnumeral{3} המפלת חתיכה}
\twocol{
\commenta{ הא דמקשינן לר' זירא \textbf{והא אמר ר' יוחנן משום רשב"י המפלת וכו'.} וא"ל אדמקשי ליה מיניה ליסייעיה ממתני' דקתני אם יש עמה דם אין בתוכה לא. וא"ל קס"ד השתא דמתני' לאו עמה לאפוקי תוכה אלא עמה לאפוקי חתיכה גופה דאפילו היא מארבע מינין טהורה ומדר' יוחנן מיפרשא ליה קושיא וממתני' לית ליה סייעתא בהדייהו.\par ואע"ג דאמרן לעיל בגמרא דאלו רבנן סברי עמה אין תוכה לא ההיא מימרא דגמרא היא ולא קס"ד השתא ומ"ה פריק אפילו בדר' יוחנן דהתם משום דדרכה של אשה לראות דם בחתיכה ומין במינו הוא ואינו חוצץ כדפרישית לעיל, א"נ א"ל דמתניתין לא סייעתא היא דקס"ד עמה לאפוקי תוכה משום שאין זה דם נדה אלא דם חתיכה. }
הרי זו ספק לידה ספק זיבה מביאה קרבן ואינו נאכל 
\commenta{ הא דאקשינן \textbf{ת"ק נמי טהורי מטהר.} ומפרקי' אלא לאו דפלאי פלויי איכא בנייהו. לאו דצריך להכי דהא מצי למימר אלא שפופרת א"ב דת"ק סבר בבשרה ולא בשפיר ולא בחתיכה וכ"ש בשפופרת ואתי רבנן למימר אין זה דם נדה אלא של חתיכה הא דם נדה טמאה ואפילו בשפופרת אלא הא דאמרינן דפלי פלויי איכא בנייהו משום דקים ליה דבהא נמי פליגי לפום טעמייהו דכיון דר' אלעזר סבר דם אגור הוא א"א לטהר אלא בדלא אפלאי וכיון דרבנן סברי דם חתיכה גופה הוא אפילו איפלאי נמי ודאי טהור' הילכך מפרש ואזיל כולה פלוגתייהו. }
רבי יהושע אומר מביאה קרבן ונאכל שאי אפשר לפתיחת הקבר בלא דם 
\commenta{ ומהדר אביי \textbf{בשפופרת דכ"ע לא פליגי כי פליגי בחתיכה מר סבר דרכה של אשה לראות דם נדה בחתיכה.} פירש רש"י ז"ל דבפלאי פלויי פליגי מר דהו רבי אלעזר סבר דרכה של אשה לראות דם נדה בחתיכה וכיון דאיפלאי וליכא חציצה טמאה וכי לא אפלאי רחמנא מיעטה מבשרה ולא בשפיר ולא בחתיכה ורבנן סברי אין דרכה של אשה לראות דם נדה בחתיכה אלא האי דם חתיכה עצמה הוא.\par ולא מחוור דא"ה לא דמי האי דרכה של אשה וכו' לאותו שאמרו למעלה בדר' יוחנן דהתם קאמרינן דכיון דדרכה אע"ג דלא אפלאי נמי לא חוצה דהיינו אורחא ולישנא דגמרא נמי לא משמע הכי כלל.\par אלא ה"פ מרדאינהו רבנן סברי דרכה של אשה לראות דם בחתיכה ולא טהרו כאן אלא משום שאין זה דם נדה אלא דם חתיכה והיכא דהוי ודאי דם (חתיכה) [נדה] כגון מצא בה דם אגור טמאה והיינו לר' יוחנן ומר דהוא ר"א סבר אין דרכה של אשה הילכך הוי ליה כשפופרת ורחמנא אמר בבשרה.\par והאי דמדכרי' סברא דרבנן מקמי דר' אלעזרא"ל משום דאמרן לעיל דרכה של אשה כסברייהו א"נ לאו דוקא וכן בכמה דוכתי בתלמודא דלא קפדי.\par ובודאי דה"מ אביי לתרוצי כדרבנן הא דם נדה ודאי טמאה בדאיפלאי ובשפופרת דכו"ע לא פליגי דטהורה אלא ניחא ליה לתרוצי בדידה ולא לעיולי בה פילי דהשתא לא מוספינן בפלוגתייהו איפלאי פלויי כלל אלא בדם אגור בחתיכה פליגי כדפרישית, ועוד לאוקמה כדר' יוחנן דלעיל דלא לתקום דלא כחד כנ"ל.\par ויש מפרשים דלא ניחא ליה לאביי לאוקומה פלוגתא דרבנן אפלאי פלויי בלחוד דהא לא מדכרא בהדיא במילתיה דר' אלעזר דאנן בגמרא לאו חסורי מחסרא לברייתא כלל אלא מימר קאמרינן דלר"א בודאי פלאי טמאה ולא ניחא ליה לאוקומא פלוגתייהו אמאי דלא מתפרש בברייתא בהדיא.\par ולאו מילתא היא ור"א ורבנן תרווייהו מטהרין מר נסיב לה טעמא מבשרה ולא בחתיכה ולפום טעמיה איפלאי פלויי טמאה ומר נסיב לה טעמא אין זה דם נדה לטהורי נמי אפלאי פלויי אלא כטעמא דפרישית עיקר. }
לישנא אחרינא אמרי לה אמר רב יהודה אמר שמואל לא טימא רבי יהודה אלא בחתיכה של ארבעה מיני דמים אבל של שאר מיני דמים טהורה 
איני והא כי אתא רב הושעיא מנהרדעא אתא ואייתי מתניתא בידיה המפלת חתיכה אדומה ושחורה ירוקה ולבנה אם יש עמה דם טמאה ואם לאו טהורה ורבי יהודה אומר בין כך ובין כך טמאה 
קתני אדומה ושחורה ירוקה ולבנה ופליג ר' יהודה 
וכי תימא כי פליג ר' יהודה אאדומה ושחורה אבל ירוקה ולבנה לא אלא ירוקה ולבנה מאן קתני לה 
אילימא לרבנן השתא אדומה ושחורה קא מטהרי רבנן ירוקה ולבנה מיבעיא אלא לאו לר' יהודה ופליג
אלא אמר רב יהודה באפשר לפתיחת הקבר בלא דם קמיפלגי ובפלוגתא דהני תנאי דתניא קשתה שנים ולשלישי הפילה ואינה יודעת מה הפילה הרי זו ספק לידה ספק זיבה מביאה קרבן ואינו נאכל 
רבי יהושע אומר מביאה קרבן ונאכל לפי שאי אפשר לפתיחת הקבר בלא דם 
ת"ר המפלת חתיכה סומכוס אומר משום רבי מאיר וכן היה רבי שמעון בן מנסיא אומר כדבריו קורעה אם יש דם בתוכה טמאה ואם לאו טהורה 
כרבנן ועדיפא מדרבנן כרבנן דאמרי אפשר לפתיחת הקבר בלא דם ועדיפא מדרבנן דאינהו סברי עמה אין בתוכה לא וסומכוס סבר אפילו בתוכה 
ותניא אידך המפלת חתיכה ר' אחא אומר קורעה אם תוכה מאדים טמאה ואם לאו טהורה 
כסומכוס ועדיפא מסומכוס 
ותניא אידך המפלת חתיכה רבי בנימין אומר קורעה אם יש בה עצם אמו טמאה לידה אמר רב חסדא ובחתיכה לבנה וכן כי אתא זוגא דמן חדייב אתא ואייתי מתניתא בידיה המפלת חתיכה לבנה קורעה אם יש בה עצם אמו טמאה לידה 
אמר רבי יוחנן משום רבי שמעון בן יוחי המפלת חתיכה קורעה אם יש בה דם אגור טמאה ואם לאו טהורה כסומכוס וקילא מכולהו 
בעא מיניה רבי ירמיה מרבי זירא הרואה דם בשפופרת מהו (ויקרא טו:ז) בבשרה אמר רחמנא ולא בשפופרת או דלמא האי בבשרה מיבעי ליה שמטמאה מבפנים כבחוץ 
אמר ליה בבשרה אמר רחמנא ולא בשפופרת דאי בבשרה מבעי ליה שמטמאה מבפנים כבחוץ א"כ נימא קרא (בבשר) מאי בבשרה שמע מינה תרתי 
והא"ר יוחנן משום רבי שמעון בן יוחי המפלת חתיכה קורעה אם יש בה דם אגור טמאה ואם לאו טהורה 
הכי השתא התם דרכה של אשה לראות דם בחתיכה הכא אין דרכה של אשה לראות דם בשפופרת 
לימא שפופרת תנאי היא דתניא המפלת חתיכה אף על פי שמלאה דם אם יש עמה דם טמאה ואם לאו טהורה רבי אליעזר אומר בבשרה ולא בשפיר ולא בחתיכה 
ר' אליעזר היינו תנא קמא אימא שרבי אליעזר אומר בבשרה ולא בשפיר ולא בחתיכה
וחכמים אומרים אין זה דם נדה אלא דם חתיכה תנא קמא נמי טהורי מטהר אלא דפלי פלויי איכא בינייהו
תנא קמא סבר בבשרה ולא בשפיר ולא בחתיכה והוא הדין לשפופרת והני מילי היכא דשיעא אבל פלי פלויי טמאה מאי טעמיה בבשרה קרינא ביה 
ואתו רבנן למימר אף על גב דפלי פלויי אין זה דם נדה אלא דם חתיכה הא דם נדה ודאי טמא ואפילו בשפופרת נמי 
אמר אביי בשפופרת כולי עלמא לא פליגי דטהורה}

\newsection{דף כב}
\twocol{כי פליגי בחתיכה מר סבר דרכה של אשה לראות דם בחתיכה ומר סבר אין דרכה של אשה לראות דם בחתיכה 
\commenta{\textbf{והלא עצמו הוא אינו מטמא אלא בחתימת פי האמה, למימרא דנוגע הוי.} פירש רב הונא אליבא דנפשיה פשיט ליה דס"ל כר' נתן דאמר זב אינו מטמא אלא בחתימת פי האמה ואל תתמה [דהא שמואל] רביה (דרבה) [דרב הונא] הוא דאמר נמי כר' נתן כדאיתא בפרק יוצא דופן ולפום הכי גמר רב הונא בעל קרי מיניה דזב וסבר לה נמי כר' שמעון דאמר בפ' ואלו דברים בפסחים דס"ל בזב כר' נתן דבעי פי האמה ואיתקש בעל קרי לזב ובעי נמי חתימת פי האמה כזב דהא קרא בבעל קרי לא כתיב ובזב כתיב או החתים בשרו.\par והא דדאיק מינייהו למימרא דנוגע הוי דלהכי בעינן חתימת פי האמה דליהוי נגיעת חוץ כדפי' רש"י ז"ל, ק"ל אי הכי זב נמי נוגע הוי ולמה לא יספור בזיבה, א"ל בזב ודאי אע"ג שנוגע הוי לענין שיעוריה מיהו הוי רואה לענין טומאה דיליה דאלו נוגע בזב טומאת ערב ואלו רואה טומאת שבעה והאי דאחמיר ד) עליה רחמנא בחתימת פי האמה דליהוי נמי נוגע גזירת הכתוב הוא שלא יהא טמא טומאת שבעה עד שיראה זוב ונגע בו מגע חוץ דה"ל רואה ונוגע ומ"ה אקשי' אא"ב בעל קרי רואה הוי ואפילו במקום שאינו טמא משום נוגע טמא הוא משום רואה הרי דומה לזב מצד אחד שאף הוא יש לו טומאה בראיה שאינו מדין מגע אא"א אינו טמא אלא בנוגע וטומאתו נמי טומאת מגע היא א"כ מה הנוגע בקרי אינו סותר בזיבה אף הרואה לא יסתור שהרי שניהן טומאה אחת להן בכל ענינן ומדין מגע טימאן הכתוב, ומפרקינן התם בשביל שא"א לה בלא צחצוחי זיבה ואם תאמר והלא אין בהם חתימת פי האמה ואין הזוב מטמא אלא כן י"ל כיון שיוצא עם שכבת זרע שהוא חותם פי האמה הרי הוא כנוגע ממש שמין במינו הוא ואינו חוצץ.\par והיינו דלא אקשינן יטמא טומא' שבעה אלא תסתור ז' דטומאה בזוב גמור לית ליה כיון דאינו רואה בשיעורו טומאת מגע זוב אית ליה וכיון דזוב הוא מיהא ובראיה דין הוא לסתור הכל שאין כאן ז' נקיים דהא הוה ליה כאלו ראה זוב בנתיים שאין אחר אחר לכולן.\par ומפרקי' גזרת הכתוב כך הוא מאחר שאין הזוב הזה כדי ראיה אין לו טומאת שבעה ואפילו לסתור ז' אלא סתירתו כטומאתו והא נמי רב הונא אליבא דנפשיה פשט ליה דהא בפרק כיצד הרגל בב"ק איכא ר' אליעזר דסבר אפשר בלא צחצוחי זיבה כלל, ומיהו בהא כרבנן פריק ליה ורבים נינהו.\par ואי קשיא לך לרב הונא דאמר בעל קרי נוגע הוי תרי קראי למה לי דהא כתיב רואה וכתיב נוגע וכדדרשינן בפרק יוצא דופן מדכתיב או איש, א"ל אע"ג דרואה דוקא בנוגע הוא דמטמא ה"א ה"מ ברואה דאיכא תרתי מגע וראיה אבל בנוגע לחודיה לא קמ"ל. }
רבא אמר דכולי עלמא אין דרכה של אשה לראות דם בחתיכה
והכא באשה טהורה ומקור מקומו טמא קמיפלגי דר' אליעזר סבר אשה טהורה ודם טמא דהא אתי דרך מקור ורבנן סברי אשה טהורה ומקור מקומו טהור 
בעא מיניה רבה מרב הונא הרואה קרי בקיסם מהו {ויקרא טו } ממנו אמר רחמנא עד דנפיק מבשרו ולא בקיסם או דלמא האי ממנו עד שתצא טומאתו לחוץ ואפי' בקיסם נמי 
אמר ליה תיפוק ליה דהוא עצמו אינו מטמא אלא בחתימת פי האמה
למימרא דנוגע הוי אלא מעתה אל יסתור בזיבה 
אלמה תניא (ויקרא טו, לב) זאת תורת הזב ואשר תצא ממנו שכבת זרע מה זיבה סותרת אף שכבת זרע נמי סותר 
אמר ליה סתירה היינו טעמא דסותר לפי שאי אפשר לה בלא צחצוחי זיבה 
אלא מעתה תסתור כל שבעה אלמה תניא זאת תורת הזב וגו' מה זיבה סותרת אף שכבת זרע סותר
אי מה זיבה סותרת כל ז' אף שכבת זרע נמי סותר כל ז' ת"ל (ויקרא טו, לב) לטמאה בה אין לך בה אלא מה שאמור בה סותרת יום אחד 
אמר ליה גזירת הכתוב היא זיבה גמורה דלא ערבה בה שכבת זרע סותרת כל שבעה צחצוחי זיבה דערבה בה שכבת זרע לא סותרת אלא יום אחד 
בעא מיניה ר' יוסי ברבי חנינא מרבי אלעזר דם יבש מהו (ויקרא טו, כה) כי יזוב זוב דמה אמר רחמנא עד דמידב דייב ליה לח אין יבש לא או דלמא האי כי יזוב זוב דמה אורחא דמילתא היא ולעולם אפילו יבש נמי 
א"ל תניתוה דם הנדה ובשר המת מטמאין לחים ויבשים אמר ליה לח ונעשה יבש לא קא מיבעיא לי כי מיבעיא לי יבש מעיקרא 
הא נמי תניתוה המפלת כמין קליפה כמין שערה כמין עפר כמין יבחושין אדומין תטיל למים
אם נמוחו טמאה אי הכי בלא נמוחו נמי אמר רבה כי לא נמוחו בריה בפני עצמה היא 
\commenta{הא דאמרינן מעיקרא \textbf{דנין יצירה מיצירה ואין דנין בריאה מיצירה.} לאו למימרא דסתרי אהדדי דהא אפשר לי' למיגמרינא לתרווייהו אלא ה"ק זו אינה ג"ש כלל, והיינו דאקשינן מאי נ"מ הא תנא דבי רבי ישמעאל ולא מפרק הני מילי היכא דליכא דדמי ליה אבל היכא דאיכא דדמי ליה מדדמי ליה ילפינן כדאתמר בעלמא אלא ודאי משום דלא אמרינן אלא היכא דסתרי אהדדי והכא תרווייהו דגמר.\par והדר אקשי' ועוד נגמר בריאה מבריאה [ומשני ויברא לגופיה] וייצר לאפנויי ודנין יצירה מיצירה דהשתא ודאי ליכא למיגמר אלא חד במופנה הילכך מדדמי ילפינן דאף על גב דוייצר מופנה גבי אדם ליכא למיגמר בריאה דתנין מיני' בדין מופנה מצד אחד דאפנויי דויצר דאדם לאו להך ג"ש הוא והוה ליה כשאינו מופנה כל עיקר אי נמי השתא לא מסיק טעמיה אלא מפרש ואזיל הוא ואמסקנא ניחא דליכא למיגמר דבריאה כלל כדבעי למימר קמן. }
ומי איכא כי האי גוונא אין והתניא א"ר אלעזר בר' צדוק שני מעשים העלה אבא מטבעין ליבנה 
\commenta{ והא דאמרינן \textbf{ויברא גבי תנינים לאו מופנה.} אי קשיא הא כתיב נמי ישרצו המים ההוא אין כתוב בעשייה אלא בצוויי, ופי' רש"י ז"ל דכיון שאין מופנה משני צדדין ומשיבין ה"נ יש להשיב מה לאדם שכן מטמא מחיים.\par ול"נ דהאי לישנא קמא לא צריך פירכא דלכ"ע מופנה משני צדדין עדיף ממופנה מצד א' וכיון דע"כ יצירה יצירה גמרינן ה"ל בריאה דאדם לגופיה וגבי תנינים נמי לגופיה ואין מופנה כל עיקר וכל ג"ש שאינו מופנה כל עיקר אין למידן הימנה.\par וא"ת וייצר האדם לגופי' ודבהמה מופנה ודגמרינן ויברא דתנין לגופיה ודאדם מופנה וגמרינן היינו דקאמרי ומאי נ"מ זה כלומר אמאי ניחא לך לאפנויי לחדא לגמרי ומיגמר מינה ולא לאפנויי תרווייהו ומיגמר מנייהו ופריק לרבנן הא עדיפא דהא אין משיבין ולר' ישמעאל נמי הא עדיפא דהיכא דאיכא מופנה משני צדדין איהי עדיף ולהכי אפנויי רחמנא לבהמה משני צדדין דשדינן מופנ' דכולהו בגוה כי היכי דלא נימא באידך מופנה מצד אחד הוא דכל היכא דאיכא למישדי שני צדדין דמופנ' בדידיה שדינן ומיניה גמרינן בין לרבי ישמעאל בין לרבנן אבל ללישנא דרב אחא הויא דבעי' והאי מאי פירכא משום דאפילו כשאנו גומרין יצירה יצירה יכולין אנו לגמור בריאה בריאה אע"פ שאינה מופנה כל עיקר אלא שמשיבין ולפום הכי בעי' מאי פירכא ורבנן דפליגי עליה דר"מ במתני' לא גמירי כדאשכחן בפרק כל היד שאין אדם ג"ש מעצמו, וכן פי' רש"י ז"ל.\par ואי קשיא לך לר"מ מאי פירכא ליהדר דינא ותיתי מכאן דכיון דגמר יצורה ואתו בהמה חיה ועוף כי פרכת גבי תנין מה לאדם שכן מטמאו מחיים נימא בהמה תוכיח א"נ נגמר מוייצר דבהמה למד מלמד א"ל מה לשניהם שכן מטמאין במגע ובמשא תאמר בדגים שאינן מטמאין ואע"פ שמקבלין טומאה טומאת עצמן אין להם. }
מעשה באשה שהיתה מפלת כמין קליפות אדומות ובאו ושאלו את אבא ואבא שאל לחכמים וחכמים שאלו לרופאים ואמרו להם אשה זו מכה יש לה בתוך מעיה שממנה מפלת כמין קליפות תטיל למים אם נמוחו טמאה 
ושוב מעשה באשה שהיתה מפלת כמין שערות אדומות ובאה ושאלה את אבא ואבא שאל לחכמים וחכמים לרופאים ואמרו להם שומא יש לה בתוך מעיה שממנה מפלת כמין שערות אדומות תטיל למים אם נמוחו טמאה 
אמר ריש לקיש ובפושרין תניא נמי הכי תטיל למים ובפושרין רשב"ג אומר ממעכתו ברוק על גבי הצפורן מאי בינייהו אמר רבינא מעוך על ידי הדחק איכא בינייהו 
התם תנן כמה היא שרייתן בפושרין מעת לעת הכא מאי מי בעינא מעת לעת או לא 
שרץ ונבלה דאקושי בעינן מעת לעת אבל דם דרכיך לא או דלמא לא שנא תיקו
המפלת כמין דגים וליפלוג נמי רבי יהודה בהא 
אמר ריש לקיש במחלוקת שנויה ורבנן היא ורבי יוחנן אמר אפילו תימא רבי יהודה עד כאן לא קאמר רבי יהודה התם אלא גבי חתיכה דעביד דם דקריש והוי חתיכה אבל בריה לא הוי 
ולהך לישנא דא"ר יוחנן באי אפשר לפתיחת הקבר בלא דם קמיפלגי לפלוג נמי ר' יהודה בהא 
מאן דמתני הך לישנא מתני הכי רבי יוחנן וריש לקיש דאמרי תרוייהו במחלוקת שנויה ורבנן היא
המפלת כמין בהמה [וכו']
אמר רב יהודה אמר שמואל מ"ט דר' מאיר הואיל ונאמרה בו יצירה כאדם 
אלא מעתה המפלת דמות תנין תהא אמו טמאה לידה הואיל ונאמר בו יצירה כאדם שנאמר (בראשית א, כא) ויברא אלהים את התנינים הגדולים 
אמרי דנין יצירה מיצירה ואין דנין בריאה מיצירה 
מאי נפקא מינה הא תנא דבי רבי ישמעאל (ויקרא יד:לט) ושב הכהן (ויקרא יד, מד) ובא הכהן זו היא שיבה זו היא ביאה 
ועוד נגמר בריאה מבריאה דכתיב (בראשית א:כז) ויברא אלהים את האדם בצלמו 
אמרי ויברא לגופיה וייצר לאפנויי ודנין יצירה מיצירה 
אדרבה וייצר לגופיה ויברא לאפנויי ודנין בריאה מבריאה 
אלא וייצר מופנה משני צדדין מופנה גבי אדם ומופנה גבי בהמה ויברא גבי אדם מופנה גבי תנינים אינו מופנה 
מאי מופנה גבי בהמה אילימא מדכתיב (בראשית א:כה) ויעש אלהים את חית הארץ וכתיב {בראשית ב } ויצר [ה'] אלהים מן האדמה כל חית השדה גבי תנין נמי אפנויי מופנה דכתיב (בראשית א:כה) ואת כל רמש האדמה וכתיב (בראשית א:כא) ויברא אלהים את התנינים הגדולים 
רמש דכתיב התם דיבשה הוא ומאי נפקא מינה בין מופנה מצד אחד למופנה משני צדדין 
נפקא מינה דאמר רב יהודה אמר שמואל משום רבי ישמעאל כל גזרה שוה שאינה מופנה כל עיקר אין למדין הימנה מופנה מצד אחד לרבי ישמעאל למדין ואין מושיבין לרבנן למדין ומשיבין מופנה משני צדדין דברי הכל למדין ואין משיבין 
ורבי ישמעאל מאי איכא בין מופנה מצד אחד למופנה משני צדדין נפקא מינה דהיכא דאיכא מופנה מצד אחד ומופנה משני צדדין שבקינן מופנה מצד אחד}

\newsection{דף כג}
\twocol{וילפינן מופנה משני צדדין ולהכי אפניה רחמנא לבהמה משני צדדין כי היכי דלא נגמר מן מופנה מצד אחד 
רב אחא בריה דרבא מתני לה משמיה דרבי אלעזר לקולא כל גזרה שוה שאינה מופנה כל עיקר למדין ומשיבין מופנה מצד אחד לרבי ישמעאל למדין ואין משיבין לרבנן למדין ומשיבין מופנה משני צדדין דברי הכל למדין ואין משיבין 
ולרבנן מאי איכא בין מופנה מצד אחד לשאינה מופנה כל עיקר 
נ"מ היכא דמשכחת לה מופנה מצד אחד ושאינה מופנה כל עיקר ולאו להאי אית ליה פירכא ולאו להאי אית ליה פירכא שבקינן שאינה מופנה כל עיקר וגמרינן ממופנה מצד אחד 
והכא מאי פירכא איכא משום דאיכא למיפרך מה לאדם שכן מטמא מחיים 
וכן א"ר חייא בר אבא א"ר יוחנן היינו טעמא דר"מ הואיל ונאמרה בו יצירה כאדם 
א"ל רבי אמי אלא מעתה המפלת דמות הר אמו טמאה לידה שנאמר (עמוס ד:יג) כי הנה יוצר הרים ובורא רוח אמר ליה הר מי קא מפלת אבן היא דקא מפלת ההוא גוש איקרי
אלא מעתה המפלת רוח תהא אמו טמאה לידה הואיל ונאמרה בו בריאה כאדם דכתיב {עמוס ד } ובורא רוח וכי תימא לא מופנה מדהוה ליה למכתב יוצר הרים ורוח וכתיב ובורא רוח ש"מ לאפנויי 
א"ל דנין דברי תורה מדברי תורה ואין דנין דברי תורה מדברי קבלה 
(אמר) רבה בר בר חנה אמר רבי יוחנן היינו טעמא דר"מ הואיל ועיניהם דומות כשל אדם 
אלא מעתה המפלת דמות נחש תהא אמו טמאה לידה הואיל וגלגל עינו עגולה כשל אדם וכי תימא הכי נמי ליתני נחש 
אי תנא נחש הוה אמינא בנחש הוא דפליגי רבנן עליה דר"מ דלא כתיב ביה יצירה אבל בהמה וחיה לא פליגי דכתיבא ביה יצירה 
והא גבי מומין קתני לה את שגלגל עינו עגול כשל אדם לא קשיא הא באוכמא הא בציריא 
רבי ינאי אמר היינו טעמא דר"מ הואיל ועיניהם הולכות לפניהם כשל אדם והרי עוף דאין עיניו הולכות לפניו וקאמר ר"מ דטמא אמר אביי בקריא וקיפופא ובשאר עופות לא 
מיתיבי ר' חנינא בן (אנטיגנוס) אומר נראין דברי ר"מ בבהמה וחיה ודברי חכמים בעופות
מאי עופות אילימא בקריא וקיפופא מ"ש בהמה וחיה דעיניהן הולכות לפניהן כשל אדם קריא וקיפופא נמי 
אלא פשיטא בשאר עופות מכלל דר"מ פליג בשאר עופות 
חסורי מיחסרא והכי קתני ר' חנינא בן אנטיגנוס אומר נראין דברי ר"מ בבהמה וחיה והוא הדין לקריא וקיפופא ודברי חכמים בשאר עופות שאף ר"מ לא נחלק עמהם אלא בקריא וקיפופא אבל בשאר עופות מודי להו 
והתניא א"ר אלעזר בר' צדוק המפלת מין בהמה וחיה לדברי ר"מ ולד ולדברי חכמים אינו ולד ובעופות תיבדק 
למאן תיבדק לאו לדברי ר"מ דאמר קריא וקיפופא אין שאר עופות לא 
אמר רב אחא בריה דרב איקא לא תיבדק לרבנן דאמרי קריא וקיפופא אין שאר עופות לא 
ומ"ש קריא וקיפופא מבהמה וחיה הואיל ויש להן לסתות כאדם 
בעא מיניה רבי ירמיה מר' זירא לר"מ דאמר בהמה במעי אשה ולד מעליא הוא קבל בה אביה קידושין מהו למאי נפקא מינה לאיתסורי באחותה 
למימרא דחיי והאמר רב יהודה אמר רב לא אמרה ר"מ אלא הואיל ובמינו מתקיים אמר רב אחא בר יעקב עד כאן הביאו רבי ירמיה לר' זירא לידי גיחוך ולא גחיך 
גופא אמר רב יהודה אמר רב לא אמרה רבי מאיר אלא הואיל ובמינו מתקיים אמר רב ירמיה מדפתי
אף אנן נמי תנינא המפלת כמין בהמה חיה ועוף (ולד מעליא הוא) דברי ר"מ וחכ"א עד שיהא בו מצורת אדם 
והמפלת סנדל או שליא או שפיר מרוקם והיוצא מחותך הבא אחריו בכור לנחלה ואינו בכור לכהן ואי ס"ד דחיי הבא אחריו בכור לנחלה מי הוי 
אמר רבא לעולם דחיי ושאני התם דאמר קרא {דברים כח } ראשית אונו מי שלבו דוה עליו יצא זה שאין לבו דוה עליו 
בעא מיניה רב אדא בר אהבה מאביי לרבי מאיר דאמר בהמה במעי אשה ולד מעליא הוא אדם במעי בהמה מאי למאי נפקא מיניה לאשתרויי באכילה 
ותפשוט ליה מהא דר' יוחנן דא"ר יוחנן השוחט את הבהמה ומצא בה דמות יונה אסורה באכילה 
הכי השתא התם לא פרסות איכא ולא פרסה איכא הכא נהי דפרסות ליכא פרסה מיהא איכא
וחכ"א כל שאין בו כו' אמר רב ירמיה בר אבא אמר רב הכל מודים גופו תייש ופניו אדם אדם גופו אדם ופניו תייש ולא כלום
לא נחלקו אלא שפניו אדם ונברא בעין אחת כבהמה שרבי מאיר אומר מצורת אדם וחכ"א כל צורת אדם 
אמר לו לרב ירמיה בר אבא והא איפכא תניא ר"מ אומר כל צורת וחכ"א מצורת אמר להו אי תניא תניא 
אמר ר' ירמיה בר אבא אמר רבי יוחנן מצח והגבינים והעינים והלסתות וגבות הזקן עד שיהו כולם כאחד רבא אמר חסא מצח והגבן והעין והלסת וגבת הזקן עד שיהו כולם כאחת 
ולא פליגי הא כמ"ד כל צורת הא כמ"ד מצורת 
מיתיבי צורת פנים שאמרו אפילו פרצוף אחד מן הפרצופין חוץ מן האוזן למימרא דמחד נמי סגי 
אמר אביי כי תניא ההיא לעכב תניא וכמ"ד כל צורת ואיבעית אימא לעולם כמ"ד מצורת ומאי אחד אחד אחד 
אמר רבא נברא בעין אחת ובירך אחד מן הצד אמו טמאה באמצע אמו טהורה 
אמר רבא ושטו נקוב אמו טמאה ושטו אטום אמו טהורה 
ת"ר המפלת גוף אטום אין אמו טמאה לידה ואיזהו גוף אטום רבי אומר כדי שינטל מן החי וימות 
וכמה ינטל מן החי וימות רבי זכאי אומר}

\newsection{דף כד}
\twocol{עד הארכובה רבי ינאי אומר עד לנקביו ר' יוחנן אומר משום רבי יוסי בן יהושע עד מקום טבורו 
\commenta{ והא דאמרי' \textbf{קא מפלגי בטרפה חיה} שהזיקיקו לרש"י ז"ל לטהר ולד טרפה נ"ל שלא הקפידו אלא על לשון הברייתא שאמרו וכמה כדי שינטל מן החי וימות דקסבר האי תנא דכל שנברא אטום בלא חיתוך איברים עד מקום שאלו ינטל מן החי וימות אינו בכלל ולד ולא שיהא זה נקרא טרפה אלא זה אינו נולד הואיל ונברא אטום אבל נחלקו האמוראין כמה הוא כדי שינטל מן החי וימות ופי' ר' זכאי עד לארכובה ודקדקו ממנו שהוא סובר טרפה חיה דהא קאמר שבכך החי מת (לרש"י נטל) [לכשינטל] ממנו ור' ינאי אמר עד לנקובה שבכך נעשה נבלה אבל טרפה אינה מתה ר' יהושע דאמר עד טבורו קסבר בין זו בין זו חיות הן וכל זה אינו אלא בשיעור כמה כדי שינטל מן החי אבל בולד שנטל ממנו לא נחלקו (במינו) [בו] ולא אמרו כאן אלא הולד כשהוא אטום.\par ומה שפי'רש"י ז"ל אטום חסר אינו נראה אלא אטום כמשמעי שאין לו חיתוך איברים ואין לו חלק שבהן אלא כמין גולם אטום ודמיא להא דתניא לקמן בריית גוף שאינו חתוך וכו'. }
בין רבי זכאי לרבי ינאי איכא בינייהו טרפה חיה מר סבר טרפה חיה ומר סבר טרפה אינה חיה 
\commenta{ הא דאקשי' \textbf{ואם איתא ליתני שמא מגוף אטום (ופניו) [או ממי שפניו] המוסמסין באתה.} א"ל איבעי למיתני טובא ליתני שמא באת מפניו תיש או אפילו פרצוף אחר או שיש לו שני גבין ושתי שדראות וכמין אפיקותא דדיקלא וכן כיוצא בהן א"ל הנהו לא שכיחי ולא ה"ל למיתני. אבל פניו ממוסמסין ה"ל למיתני משום דשכיח נמי טפי מגוף אטום. }
בין ר' ינאי לר' יוחנן איכא בינייהו דר"א דאמר רבי אלעזר ניטל ירך וחלל שלה נבלה 
\commenta{הא דאמרינן \textbf{ושמואל סבר בריה בעלמא איתא וכי אגמריה רחמנא למשה בעלמא.} פירש"י ז"ל אותו המין אסר לו וק"ל א"כ לשמואל אפילו יוצא לאויר העול' נמי לישתרי דה"ל כקלוט בן פרה דשרי ונראה מדבריו דבין לרב בין לשמואל במעי טהורה לא חיי הלכך יצא לאויר העולם משום נפל אסור אפילו לשמואל והא דפריך רב שימי ממתניתין ר' חנינא בן אנטיגנוס אומר וכו' לרב ה"ה לשמואל אלא גביה הוה קאי דבר בריה הוה.\par ולא נהירא ועוד דהתם בפ' ואלו מומין (דף מג ע"ב) תנן לה למתני' גבי מומי כהן איזהו גבן ר' חנינא בן אנטיגנוס אומר כל שיש לו שני גבין ושדראות והוי ביה למימר' דחיי והאמר רב באשה אינו לד בבהמה אסור באכילה ולא מדכרין התם דשמואל בכלום בעולם.\par אלא הכי משמע פירושא לכ"ע מינא בעלמא ליכא כי פליגי בבריה רב סבר אפילו בריה בעולם ליכא דלא חי הילכך כי אגמריה רחמנא למשה במעי בהמה אגמריה דבחוץ לא צריך נפל הוא. ושמואל סבר בריה בעלמא איתא דחיי וכי אגמריה רחמנא בשיצא לאויר העולם דלא תימא כקלוט בן פרה הוא אבל במעי בהמה דאפילו נפל שריא איהי נמי שרי. }
אמר רב פפא מחלוקת מלמטה למעלה אבל מלמעלה למטה אפי' כל דהו טהורה וכן אמר רב גידל אמר רבי יוחנן המפלת את שגולגלתו אטומה אמו טהורה 
ואמר רב גידל אמר רבי יוחנן המפלת כמין אפקתא דדיקלא אמו טהורה 
איתמר המפלת מי שפניו מוסמסים רבי יוחנן אמר אמו טמאה ר"ל אמר אמו טהורה 
איתיביה ר' יוחנן לריש לקיש המפלת יד חתוכה ורגל חתוכה אמו טמאה לידה ואין חוששין שמא מגוף אטום באתה ואם איתא ליתני שמא מגוף אטום או ממי שפניו מוסמסין 
אמר רב פפי בפניו מוסמסין כולי עלמא לא פליגי דטמאה כי פליגי בפניו טוחות ואיפכא איתמר רבי יוחנן אמר אמו טהורה וריש לקיש אמר אמו טמאה 
ולותביה ר"ל לרבי יוחנן מהא משום דשני ליה היינו גוף אטום היינו מי שפניו טוחות 
בני רבי חייא נפיק לקרייתא אתו לקמיה דאבוהון אמר להם  כלום בא מעשה לידכם אמרו לו פנים טוחות בא לידינו וטימאנוה 
אמר להם צאו וטהרו מה שטמאתם מאי דעתייכו לחומרא חומרא דאתיא לידי קולא היא דקיהביתו לה ימי טוהר 
איתמר המפלת בריה שיש לה ב' גבים וב' שדראות אמר רב באשה אינו ולד בבהמה אסור באכילה ושמואל אמר באשה ולד בבהמה מותר באכילה 
במאי קמיפלגי בדרב חנין בר אבא דאמר רב חנין בר אבא השסועה בריה שיש לה ב' גבין וב' שדראות 
רב אמר בריה בעלמא ליתא וכי אגמריה רחמנא למשה במעי אמה אגמריה ושמואל אמר בריה בעלמא איתא וכי אגמריה רחמנא למשה בעלמא אגמריה אבל במעי אמה שריא 
איתיביה רב שימי בר חייא לרב רבי חנינא בן אנטיגנוס אומר כל שיש לו ב' גבין ושני שדראות פסול לעבודה אלמא דחיי (וקשיא לרב) א"ל שימי את ששדרתו עקומה 
מיתיבי יש בעוברין שהן אסורין בן ארבעה לדקה בן שמנה לגסה הימנו ולמטה אסור יצא מי שיש לו שני גבין ושני שדראות 
מאי יצא לאו יצא מכלל עוברין שאפילו במעי אמן אסורין 
רב מתרץ לטעמיה ושמואל מתרץ לטעמיה רב מתרץ לטעמיה בן ארבעה לדקה בן ח' לגסה הימנו ולמטה אסור 
במה דברים אמורים כשיצא לאויר העולם אבל במעי אמו שרי יצא מי שיש לו שני גבין ושני שדראות דאפילו במעי אמו נמי אסור
ושמואל מתרץ לטעמיה בן ארבעה לדקה בן שמנה לגסה הימנו ולמטה אסור במה דברים אמורים בשלא כלו לו חדשיו אבל כלו לו חדשיו מותר יצא מי שיש לו ב' גבין וב' שדראות דאע"ג דכלו לו חדשיו אם יצא לאויר העולם אסור במעי אמו שרי 
תני תנא קמיה דרב המפלת בריית גוף שאינו חתוך ובריית ראש שאינו חתוך יכול תהא אמו טמאה לידה ת"ל (ויקרא יב, ב) אשה כי תזריע וילדה זכר וגו' וביום השמיני ימול וגו'
מי שראוי לברית שמנה יצאו אלו שאינן ראויין לברית שמנה א"ל רב וסיים בה הכי ושיש לו שני גבין ושני שדראות 
רבי ירמיה בר אבא סבר למעבד עובדא כוותיה דשמואל אמר ליה רב הונא מאי דעתיך לחומרא חומרא דאתי לידי קולא הוא דקיהבת לה דמי טוהר עביד מיהא כותיה דרב דקיימא לן הלכתא כרב באיסורי בין לקולא בין לחומרא 
אמר רבא הרי אמרו אשה יולדת לתשעה ויולדת לשבעה בהמה גסה יולדת לתשעה יולדת לשבעה או לא ילדה 
אמר רב נחמן בר יצחק ת"ש הימנו ולמטה אסור מאי לאו אגסה לא אדקה 
האי מאי אי אמרת בשלמא אגסה אצטריך סלקא דעתך אמינא הואיל ובאשה חיי בבהמה נמי חיי קמ"ל דלא חיי 
אלא אי אמרת אדקה איתמר פשיטא בת תלתא ירחי מי קא חיי 
אצטריך סד"א כל בציר תרי ירחי חיי קמ"ל 
אמר רב יהודה אמר שמואל  המפלת דמות לילית אמו טמאה לידה ולד הוא אלא שיש לו כנפים תנ"ה א"ר יוסי מעשה בסימוני באחת שהפילה דמות לילית ובא מעשה לפני חכמים ואמרו ולד הוא אלא שיש לו כנפים 
המפלת דמות נחש הורה חנינא בן אחיו של רבי יהושע אמו טמאה לידה הלך ר' יוסף וספר דברים לפני ר"ג שלח לו רבי יהושע הנהג בן אחיך ובא 
בהליכתן יצתה כלת (ר') חנינא לקראתו אמרה לו רבי המפלת כמין נחש מהו אמר לה אמו טהורה אמרה לו והלא משמך אמרה לי חמותי אמו טמאה ואמר לה מאיזה טעם הואיל וגלגל עינו עגול כשל אדם מתוך דבריה נזכר רבי יהושע שלח לו לרבן גמליאל מפי הורה חנינא 
אמר אביי ש"מ צורבא מרבנן דאמר מילתא לימא בה טעמא דכי מדכרו ליה מדכר
{\large\emph{מתני׳}} המפלת שפיר מלא מים מלא דם מלא גנונים אינה חוששת לולד ואם היה מרוקם תשב לזכר ולנקבה המפלת סנדל או שליא תשב לזכר ולנקבה
{\large\emph{גמ׳}} בשלמא דם ומים לא כלום היא אלא גנונים ניחוש שמא ולד הוה ונימוח אמר אביי כמה יין חי שתת אמו של זה שנמוח עוברה בתוך מעיה 
רבא אמר מלא תנן ואם איתא דאתמוחי אתמח מחסר חסר רב אדא בר אהבה אמר גוונים תנן ואם איתא דאתמוחי אתמח כולה בחד גוונא הוי קאי 
תניא אבא שאול אומר קובר מתים הייתי והייתי מסתכל בעצמות של מתים השותה יין חי עצמותיו שרופין מזוג עצמותיו סכויין כראוי עצמותיו משוחין וכל מי ששתייתו מרובה מאכילתו עצמותיו שרופין אכילתו מרובה משתייתו עצמותיו סכויין כראוי עצמותיו משוחין 
תניא אבא שאול אומר ואיתימא רבי יוחנן קובר מתים הייתי פעם אחת רצתי אחר צבי ונכנסתי בקולית של מת ורצתי אחריו שלש פרסאות וצבי לא הגעתי וקולית לא כלתה כשחזרתי לאחורי אמרו לי של עוג מלך הבשן היתה 
תניא אבא שאול אומר קובר מתים הייתי פעם אחת נפתחה מערה תחתי ועמדתי בגלגל עינו של מת עד חוטמי כשחזרתי לאחורי אמרו עין של אבשלום היתה 
ושמא תאמר אבא שאול ננס הוה אבא שאול ארוך בדורו הוה ורבי טרפון מגיע לכתפו ור' טרפון ארוך בדורו הוה ור"מ מגיע לכתפו רבי מאיר ארוך בדורו הוה ורבי מגיע לכתפו רבי ארוך בדורו הוה
ורבי חייא מגיע לכתפו ורבי חייא ארוך בדורו הוה ורב מגיע לכתפו רב ארוך בדורו הוה ורב יהודה מגיע לכתפו ורב יהודה ארוך בדורו הוה ואדא דיילא מגיע לכתפו}

\newsection{דף כה}
\twocol{פרשתבינא דפומבדיתא קאי ליה לאדא דיילא עד פלגיה וכולי עלמא קאי לפרשתבינא דפומבדיתא עד חרציה
שאלו לפני רבי המפלת שפיר מלא בשר מהו אמר להם לא שמעתי 
אמר לפניו ר' ישמעאל בר' יוסי כך אמר אבא מלא דם טמאה נדה מלא בשר טמאה לידה 
א"ל אילמלי דבר חדש אמרת לנו משום אביך שמענוך עכשיו
מדהא קמייתא כיחידאה קאמר כסומכוס שאמר משום ר"מ הא נמי שמא כרבי יהושע אמרה ואין הלכה כר' יהושע 
דתניא המפלת שפיר שאינו מרוקם ר' יהושע אומר ולד וחכ"א אינו ולד 
אמר ר"ש בן לקיש משום ר' אושעיא מחלוקת בעכור אבל בצלול דברי הכל אינו ולד ור' יהושע בן לוי אמר בצלול מחלוקת 
איבעיא להו בצלול מחלוקת אבל בעכור דברי הכל ולד או דלמא בין בזה ובין בזה מחלוקת תיקו 
מיתיבי את זו דרש ר' יהושע בן חנניא (בראשית ג, כא) ויעש ה' אלהים לאדם ולאשתו כתנות עור וילבישם מלמד שאין הקב"ה עושה עור לאדם אלא א"כ נוצר 
אלמא בעור תליא מילתא לא שנא עכור ול"ש צלול 
אי אמרת בשלמא בצלול מחלוקת היינו דאיצטריך קרא אלא אי אמרת בעכור מחלוקת למה לי קרא סברא בעלמא הוא אלא שמע מינה בצלול מחלוקת שמע מינה 
וכן אמר ר"נ אמר רבה בר אבוה מחלוקת בעכור אבל בצלול דברי הכל אינו ולד 
איתיביה רבא לר"נ אלא אמרו סימן ולד בבהמה דקה טינוף בגסה שליא באשה שפיר ושליא
ואילו שפיר בבהמה לא פטר אי אמרת בשלמא בצלול מחלוקת משום הכי 
אשה דרבי בה קרא פטר בה שפיר בבהמה דלא רבי קרא לא פטר בה שפיר 
אלא אי אמרת בעכור מחלוקת מכדי סברא הוא מאי שנא אשה ומאי שנא בהמה 
מי סברת רבי יהושע מפשט פשיט ליה רבי יהושע ספוקי מספקא ליה ואזיל הכא לחומרא והכא לחומרא 
גבי אשה דממונא הוא ספק ממונא לקולא 
גבי בהמה דאיסורא הוא דאיכא לגבי גיזה ועבודה ספק איסורא לחומרא ה"נ גבי אשה ספק טומאה לחומרא 
ומי מספקא ליה והא קרא קאמר מדרבנן וקרא אסמכתא בעלמא הוא 
א"ל רב חנינא בר שלמיא לרב הא רבי הא ר' ישמעאל בר' יוסי והא רבי אושעיא והא רבי יהושע בן לוי מר כמאן ס"ל 
א"ל אני אומר אחד זה ואחד זה אינה חוששת 
ושמואל אמר אחד זה ואחד זה חוששת ואזדא שמואל לטעמיה דכי אתא רב דימי אמר מעולם לא דכו שפיר בנהרדעא לבר מההוא שפירא דאתא לקמיה דשמואל דמנח עליה חוט השערה מהאי גיסא וחזיא מהאי גיסא אמר אם איתא דולד הואי לא הוה זיג כולי האי
ואם היה מרוקם וכו' תנו רבנן איזהו שפיר מרוקם אבא שאול אומר תחלת ברייתו מראשו ושתי עיניו כשתי טיפין של זבוב תני רבי חייא מרוחקין זה מזה שני חוטמין כשתי טיפים של זבוב 
תני רבי חייא ומקורבין זה לזה ופיו מתוח כחוט השערה וגויתו כעדשה ואם היתה נקבה נדונה כשעורה לארכה 
וחתוך ידים ורגלים אין לו ועליו מפורש בקבלה (איוב י, י) הלא כחלב תתיכני וכגבינה תקפיאני עור ובשר תלבישני ועצמות וגידים תסוככני חיים וחסד עשית עמדי ופקודתך שמרה רוחי 
ואין בודקין אותו במים שהמים עזין
וטורדין אותו אלא בודקין אותו בשמן שהשמן רך ומצחצחו ואין רואין אלא בחמה 
כיצד בודקין אותו כיצד בודקין אותו כדאמרינן אלא במה בודקין אותו לידע אם זכר הוא אם נקבה היא 
אבא שאול בר נש ואמרי לה אבא שאול בר רמש אומר מביא קיסם שראשו חלק ומנענע באותו מקום אם מסכסך בידוע שזכר הוא ואם לאו בידוע שנקבה היא 
א"ר נחמן אמר רבה בר אבוה ל"ש אלא מלמטה למעלה אבל מן הצדדין אימא כותלי בית הרחם נינהו 
א"ר אדא בר אהבה תנא אם היתה נקבה נדונה כשעורה סדוקה מתקיף לה ר"נ ודילמא חוט של ביצים נינהו אמר אביי השתא ביצים גופייהו לא ידיעי חוט של ביצים ידיע 
א"ר עמרם תנא ב' ירכותיו כב' חוטין של זהורית וא"ר עמרם עלה כשל ערב ושני זרועותיו כב' חוטין של זהורית וא"ר עמרם עלה כשל שתי 
א"ל שמואל לרב יהודה שיננא לא תעביד עובדא עד שישעיר ומי אמר שמואל הכי והאמר שמואל אחת זו ואחת זו חוששת 
אמר רב אמי בר שמואל לדידי מפרשא לי מיניה דמר שמואל לחוש חוששת ימי טהרה לא יהבינן לה עד שישעיר 
למימרא דמספקא ליה לשמואל והא ההוא שפירא דאתאי לקמיה דמר שמואל אמר הא בר ארבעין וחד יומא וחשיב מיומא דאזלא לטבילה עד ההוא יומא ולא הוה אלא ארבעין יומין
ואמר להו האי בנדה בעל כפתיה ואודי שאני שמואל דרב גובריה
המפלת סנדל וכו' ת"ר סנדל דומה לדג של ים מתחלתו ולד הוא אלא שנרצף רשב"ג אומר סנדל דומה ללשון של שור הגדול משום רבותינו העידו סנדל צריך צורת פנים 
א"ר יהודה אמר שמואל הלכה סנדל צריך צורת פנים ... א"ר אדא א"ר יוסף א"ר יצחק סנדל צריך צורת פנים ואפי' מאחוריו משל לאדם שסטר את חבירו והחזיר פניו לאחוריו 
בימי רבי ינאי בקשו לטהר את הסנדל שאין לו צורת פנים אמר להם ר' ינאי טיהרתם את הוולדות 
והתניא משום רבותינו העידו סנדל צריך צורת פנים אמר רב ביבי בר אביי אמר רבי יוחנן מעדותו של רבי נחוניא נשנית משנה זו אמר רבי זעירא זכה בה רב ביבי בשמעתיה דאנא והוא הוינא יתבינן קמיה דרבי יוחנן כי אמרה להא שמעתא וקדם איהו ואמר וזכה בה 
למה הזכירו סנדל והלא אין סנדל שאין עמו ולד 
אי דאתילידא נקבה בהדיה ה"נ הכא במאי עסקינן דאתיליד זכר בהדיה 
מהו דתימא הואיל ואמר רב יצחק בר אמי אשה מזרעת תחילה יולדת זכר איש מזריע תחלה יולדת נקבה מדהא זכר הא נמי זכר
קמ"ל אימא שניהם הזריעו בבת אחת האי זכר והאי נקבה 
דבר אחר שאם תלד נקבה לפני שקיעת החמה וסנדל לאחר שקיעת החמה
מונה תחלת נדה לראשון ותחלת נדה לאחרון 
סנדל דתנן}

\newsection{דף כו}
\twocol{גבי בכורות למאי הלכתא 
\commenta{ הא דאמרי' \textbf{תלת מתני' ותרתי שמעתא שיעורן טפח.} ואקשי' תרתי חדא היא היינו טעמא דלא מקשינן תלת ד' הוויין משום דהוה איכא למימר רבי שילא לית ליה דר' חייא דשיעור אזוב טפח ומ"ה מקשינן אם כן [תרתי חדא היא, ועוד א"ל] תרווייהו כי הדדי נינהו וחדא נקט. }
לבא אחריו בכור לנחלה ואין בכור לכהן 
סנדל דתנן גבי כריתות למאי הלכתא 
שאם תלד ולד דרך דופן וסנדל דרך רחם דמייתא קרבן אסנדל 
ולרבי שמעון דאמר יוצא דרך דופן ולד מעליא הוא מאי איכא למימר 
אמר רבי ירמיה שאם תלד ולד בהיותה עובדת כוכבים וסנדל לאחר שנתגיירה דמייתא קרבן אסנדל 
אמרוה רבנן קמיה דרב פפא ומי איתנהו להני שינויי והא תניא כשהן יוצאין אין יוצאין אלא כרוכין 
אמר רב פפא שמע מינה מכרך כריך ליה ולד לסנדל אפלגיה ומשלחיף ליה כלפי רישיה גבי בכורות כגון שיצאו דרך ראשיהם דסנדל קדים ונפיק גבי כריתות שיצאו דרך מרגלותיהם דולד קדים ונפיק 
רב הונא בר תחליפא משמיה דרבא אמר אפילו תימא מצומצמין ואיפוך שמעתתא גבי בכורות שיצאו דרך מרגלותיהם ולד דאית ביה חיותא סריך ולא נפיק סנדל דלית ביה חיותא שריק ונפיק גבי כריתות שיצאו דרך ראשיהן ולד דאית ביה חיותא מדנפיק רישיה הויא לידה סנדל עד דנפיק רוביה
{\large\emph{מתני׳}} שליא בבית הבית טמא לא שהשליא ולד אלא שאין שליא בלא ולד 
רבי שמעון אומר נימוק הולד עד שלא יצא
{\large\emph{גמ׳}} תנו רבנן שליא תחלתה דומה לחוט של ערב וסופה דומה כתורמוס וחלולה כחצוצרת ואין שליא פחותה מטפח רבי שמעון בן גמליאל אומר שליא דומה לקורקבן של תרנגולין שהדקין יוצאין ממנה 
תניא רבי אושעיא זעירא דמן חבריא חמשה שיעורן טפח ואלו הן שליא שופר שדרה דופן סוכה והאזוב 
שליא הא דאמרן שופר דתניא כמה יהא שיעור שופר פירש רבי שמעון בן גמליאל כדי שיאחזנו בידו ויראה לכאן ולכאן טפח 
שדרה מה היא דא"ר פרנך אמר רבי יוחנן שדרו של לולב צריך שיהא יוצא מן ההדס טפח דופן סוכה דתניא שתים כהלכתן שלישית אפילו טפח אזוב דתני רבי חייא אזוב טפח 
אמר רבי חנינא בר פפא דריש שילא איש כפר תמרתא תלת מתניתא ותרתי שמעתתא שיעורא טפח תרתי חדא היא אמר אביי אימא אמר רבי חייא אזוב טפח 
ותו ליכא והאיכא טפח על טפח על רום טפח מרובע מביא את הטומאה וחוצץ בפני הטומאה 
טפח קאמרינן טפח על טפח לא קאמרינן 
והא איכא אבן היוצא מן התנור טפח ומן הכירה שלש אצבעות חבור 
כי קאמרינן היכא דבציר מטפח לא חזי אבל הכא כ"ש דבציר מטפח יד תנור הוא 
והאיכא
תנורי בנות טפח דתניא תנור תחלתו ארבעה ושיריו ד' דברי רבי מאיר 
\commenta{ הא דאמר רב \textbf{אין הולד מתעכב אחר חבירו כלום.} ראיתי מקצת בעלי פירושין שכתבו דלית ליה לרב הא דאמרינן לקמן מעשה ונשתהא ולד אחרונה אחר חבירו שלשים יום ולית ליה נמי מעשה דיהודה וחזקיה ולית ליה נמי האי דאמרינן בכתובות וביבמות ג' נשים משמשות במוך קטנה מעוברת מניקה מעוברת שמה תעשה עוברה סנדל אלא קסבר אין אשה מתעברת וחוזרת ומתעברת ולפיכך אין ולד מתעכב אחר חבירו כלום.\par ואין דבריהם נראין דג' נשים מתניתא הוא ונימאתהוי תיובתיה דרב. ועוד מעשה דיהודה וחזקיה בני חביביה דהוא יושב לפניו והן יושבין עמו בבה"מ היכי אפשר דלא חזי ליה ואם איהו אומר דלא היו דברים מעולם מאן מהימן לאסהודי עלייהו.\par אלא היינו טעמא דרב דקסבר אין אשה מתעברת וחוזרת ומתעברת בין נפל בין של קיימא אלא א"כ נעשה א' מהם סנדל וסנדל כרוך עם הולד הוא יוצא שחבור אתו ונדבק בו והיינו טעמא דמוך אבל כשהאשה מתעברת תאומים טפה אחת היא שמתחלקת וכשהן נגמרין לז' או לט' אין הולד מתעכב אחר חבירו כלום אלא א"כ הפילה א' נפל וא' שליא אבל פעמים שנתחלקו לשתים וא' מהן נגמר לט' וא' לז' ובזה מודה רב שהול' משתהא אחר חבירו כדי שתגמור צורתו בזמנו. והיינו מעשה דיהודה וחזקיה ומיהו לית ליה אפוכי שמעתא דלקמן (כ"ד) [כ"ג] לולד דבחד ירחא לא משתהא אלא ל"ג אית ליה.\par והא דאמרינן לעיל סנדל מהו דתימא הואיל וא"ר יצחק עד קמ"ל שניהם הזריעו בבת אחת ודאי קשיא דהא איכא סנדל דמתעברת וחוזרת ומתעברת וזה זכר וזה נקבה כדפרישי' במשמשת במוך. ואיכא למימר אין ודאי דמצי למימר הכי אלא שמא תאמר היכא דבעל ופירש מדהאי זכר האי נמי זכר קמ"ל אפי' בכה"ג חיישינן שמא שניהם הזריעו כא' והאי זכר (נמי) והאי נקבה כנ"ל. }
וחכמים אומרים במה דברים אמורים בגדול אבל בקטן תחלתו כל שהוא משתגמר מלאכתו ושיריו ברובו 
\commenta{\textbf{אין תולין את השליח אלא בדבר של קיימא.} פי' רש"י ז"ל שכיוצא בו מתקיים אם היו חדשיו כלין למעוטי שאם הפילה דבר שאינו ראוי לבריית נשמה כגון נברא בירך אחת או גוף אטו' ואח"כ הפילה שליא (פי') [אפי'] בתוך ג' חוששין לולד אחר, ופירוש חזייה לרב יהודה בישות משום דשמעה מרב ולא אמרה.\par ואינו מחוור דבן קיימ' לאפוקי כל נפל משמע וכדאמרן דילמ' כאן בנפלי כאן בבן קיימא ולא ידעתי מי הזקיקו לשנות פירושו אלא הא דתלמיד' דרב פליגא אדרב יהוד' דאמר לעיל משמיה דרב הפילה נפל ואח"כ הפיל' שליא כל שלשה ימים תולין אותה בולד ושאר תלמידים דרב אומרי' משמי' דאין תולין את השליא בנפל אפילו יום א' אלא א"כ יצאה עמו אבל בבן קיימא תולין אותה אפילו מכאן ועד י' ימים.\par ושמעתי שפירשו בירושלמי במס' זו (ג, ד) לפי שאין השליא פורשת עד שיגמר לפיכך אין תולין אותה בנפל.\par ובשאלתות דרב אחא משבחא ז"ל כתב לכך תולין אותה בבן קיימא דאמרי' אגב חיותא דולד בזעא לשליא ונפיק. אבל נפל דלית ביה חיותא לא. ומ"ה חזייה שמואל לרב יהודה בישות דשמעיה דאמר משמיה דרב דכל ג' תליא שליא בנפל וכיון דשמעינהו לכולהו תלמידי דרב דאמרי אין תולין כלל אמר ודאי רב יהודה טעי. }
וכמה כל שהוא אמר רבי ינאי טפח שכן עושין תנורי בנות טפח בפלוגתא לא קמיירי 
השתא דאתית להכי הא נמי פלוגתא היא דקתני סיפא אמר ר' יהודה לא אמרו טפח אלא מן התנור ולכותל 
והאיכא מסגרת טפח בדכתיבן לא קא מיירי והאיכא כפורת טפח בקדשים לא קמיירי 
והאיכא דיה לקורה שהיא רחבה טפח בדרבנן לא קמיירי אלא בדכתיבן ולא מפרשי שיעורייהו 
יתיב רב יצחק בר שמואל בר מרתא קמיה דרב כהנא ויתיב וקאמר אמר רב יהודה אמר רב כל שלשה ימים הראשונים תולין את השליא בולד מכאן ואילך חוששין לולד אחר 
אמר ליה ומי אמר רב הכי והאמר רב אין הולד מתעכב אחר חבירו כלום אישתיק אמר ליה דלמא כאן בנפל כאן בבן קיימא 
אמר ליה את אמרת לשמעתתיה דרב בפירוש אמר רב הכי הפילה נפל ואחר כך הפילה שליא כל שלשה ימים תולין את השליא בולד מכאן ואילך חוששין לולד אחר ילדה ואח"כ הפילה שליא אפילו מכאן ועד עשרה ימים אין חוששין לולד אחר 
שמואל ותלמידי דרב ורב יהודה הוו יתבי חליף ואזיל רב יוסף בריה דרב מנשיא מדויל לאפייהו באלי ואתי אמר אתי לן גברא דרמינן ליה בגילא דחטתא ומרמי ומדחי 
אדהכי אתא אמר ליה שמואל מאי אמר רב בשליא א"ל הכי אמר רב אין תולין את השליא אלא בדבר של קיימא שיילינהו שמואל לכל תלמידי דרב ואמרי ליה הכי הדר חזייה לרב יהודה בישות 
בעא מיניה רבי יוסי בן שאול מרבי המפלת דמות עורב ושליא מהו אמר ליה אין תולין אלא בדבר שיש במינו שליא 
קשורה בו מהו אמר ליה דבר שאינו שאלת איתיביה המפלת מין בהמה חיה ועוף ושליא עמהן בזמן שהשליא קשורה עמהן אין חוששין לולד אחר אין שליא קשורה עמהן חוששין לולד אחר הריני מטיל עליהן}

\newsection{דף כז}
\twocol{חומר שני ולדות שאני אומר שמא נמוח שפיר של שליא ונמוח שליא של שפיר תיובתא 
אמר רבה בר שילא אמר רב מתנה אמר שמואל מעשה ותלו את השליא בולד עד עשרה ימים ולא אמרו תולין אלא בשליא הבאה אחר הולד 
אמר רבה בר בר חנה אמר רבי יוחנן מעשה ותלו את השליא בולד עד כ"ג ימים אמר ליה רב יוסף עד כ"ד אמרת לן 
אמר רב אחא בריה דרב עוירא א"ר יצחק מעשה ונשתהה הולד אחר חבירו ל"ג יום א"ל רב יוסף ל"ד אמרת לן 
הניחא למאן דאמר יולדת לתשעה יולדת למקוטעין משכחת לה אחד נגמרה צורתו לסוף שבעה ואחד נגמרה צורתו לתחלת תשעה אלא למ"ד יולדת לתשעה אינה יולדת למקוטעין מאי איכא למימר 
איפוך שמעתתא ל"ג לשליא כ"ג לולד 
א"ר אבין בר רב אדא אמר רב מנחם איש כפר שערים ואמרי לה בית שערים מעשה ונשתהה ולד אחד אחר חבירו ג' חדשים והרי הם יושבים לפנינו בבית המדרש ומאן נינהו יהודה וחזקיה בני רבי חייא 
והא אמר מר אין אשה מתעברת וחוזרת ומתעברת אמר אביי טיפה אחת היתה ונתחלקה לשתים אחד נגמרה צורתו בתחלת ז' ואחד בסוף ט'
שליא בבית הבית טמא תנו רבנן שליא בבית הבית טמא לא שהשליא ולד אלא שאין שליא שאין ולד עמה דברי רבי מאיר רבי יוסי ורבי יהודה ורבי שמעון מטהרין 
אמרו לו לרבי מאיר אי אתה מודה שאם הוציאוהו בספל לבית החיצון שהוא טהור אמר להן אבל ולמה לפי שאינו 
אמרו לו כשם שאינו בבית החיצון כך אינו בבית הפנימי אמר להן אינו דומה נמוק פעם אחת לנמוק ב' פעמים 
יתיב רב פפא אחורי דרב ביבי קמיה דרב המנונא ויתיב וקאמר מאי טעמא דרבי שמעון קסבר כל טומאה שנתערב בה ממין אחר בטלה 
אמר להו רב פפא היינו נמי טעמייהו דרבי יהודה ורבי יוסי אחיכו עליה מאי שנא פשיטא 
אמר רב פפא אפילו כי הא מילתא לימא איניש ולא נשתוק קמיה רביה משום שנאמר (משלי ל, לב) אם נבלת בהתנשא ואם זמות יד לפה 
ואזדא רבי שמעון לטעמיה דתניא מלא תרוד רקב שנפל לתוכו עפר כל שהו טמא ורבי שמעון מטהר 
מאי טעמא דרבי שמעון אמר רבה אשכחתינהו לרבנן דבי רב דיתבי וקאמרי אי אפשר שלא ירבו שתי פרידות עפר על פרידה אחת של רקב וחסיר ליה 
ואמינא להו אדרבה א"א שלא ירבו שתי פרידות רקב על
פרידה אחת עפר (ונפיל) ליה שיעורא 
\commenta{וא"ר יוחנן ד\textbf{ר"ש וראב"י אמרו דבר א'.} ואליבא דר' חייא דפריש למתניתן דתקבר לומר שנפטרה מן הבכורה ולא משום טומאה. והך סוגיא דר' יוחנן הוא דהתם בדוכתא במס' בכורות (כג, א) מסיק טעמיה דר' חייא משום דה"ל טומאה סרוחה. וצ"ע. }
אלא אמר רבה היינו טעמא דרבי שמעון סופו כתחלתו מה תחלתו נעשה לו דבר אחר גנגילון אף סופו נעשה לו דבר אחר גנגילון 
\commenta{\textbf{שפיר שטרפו במימיו.} גרסי' וכן בפר"ח ז"ל, ופי' שטרף השפיר ונמוקה צורתו אבל עדיין הוא קיים נעשה כמת שנתבלבלה צורתו באור וטהורים דכיון שאין באבריו צורת בשר ולא צורת עצם נפק ליה מדין כזית ועצם כשעורה וטהורין לגמרי. }
מאי היא דתניא איזהו מת שיש לו רקב ואיזהו מת שאין לו רקב נקבר ערום בארון של שיש או ע"ג רצפה של אבנים זהו מת שיש לו רקב
\commenta{ והא דא"ר יוחנן \textbf{מת שנתבלבלה צורתו מנ"ל דטהור.} לאו דוקא דלא כרבנן אלא מדר' אליעזר שמע ליה ר' יוחנן דקסבר מודו ליה רבנן בשלא נעשה אפר כדפרישי'.\par ויש לפרש דקסב' רבינא דר"ילא מודה לי' לר"ל בשפיר שטרפו מימיו דמדלא א"ל בשלמא שפיר שנטרפו מימיו דקאמר טהור לחיי אלא נתבלבלה צורתו שלמה מנלן אלמא ה"ק מנלן דטהור דגמרת מינה לשפיר לא הא ולא הא איתנהו. ועלה קאמר רבינא דר"י דמטמא שפיר שנטרפו מימיו לגמרי כר' אליעזר אמרה דהאי כאפר שרופין הוא ומיהו במת שנתבלבל' צורתו דקא מתמה מנלן לד"ה אתיא.\par וזה הלשון לדברי מי שגורס שפיר שנטרפו מימיו דמשמע שנטרף לגמרי וחזר למים, אבל לפי גר"ח ז"ל שנטרפו במימיו. נראה דהיינו נתבלבלה צורתו בלחוד.\par ויש לי עוד לומר דר' יוחנן הלכה קא מיבעי ליה, וה"ק ליה מנלן דטהור כרבנן דילמא טמא כר' אליעזר דמסתברא טעמיה. אילימא מדרבי שבתאי קא גמרת הלכה דהוא אמורא וקא פסיק הלכה כרבנן. }
ואיזהו מת שאין לו רקב נקבר בכסותו או בארון של עץ או ע"ג רצפה של לבנים זהו מת שאין לו רקב ולא אמרו רקב אלא למת בלבד למעוטי הרוג דלא 
גופא מלא תרוד רקב שנפל לתוכו עפר כל שהוא טמא ור' שמעון מטהר מלא תרוד רקב שנתפזר בבית הבית טמא ורבי שמעון מטהר 
וצריכא דאי אשמעינן קמייתא בההיא קאמרי רבנן משום דמכניף אבל נתפזר אימא מודו לו לרבי שמעון דאין מאהיל וחוזר ומאהיל 
ואי אשמעינן בהא בהא אמר רבי שמעון דאין מאהיל וחוזר ומאהיל אבל בהא אימא מודה להו לרבנן צריכא 
תניא אידך מלא תרוד ועוד עפר בית הקברות טמא ורבי שמעון מטהר מאי טעמייהו דרבנן לפי שא"א למלא תרוד ועוד עפר בית הקברות שאין בו מלא תרוד רקב 
השתא דאמרת טעמא דרבי שמעון משום סופו כתחלתו גבי שליא מאי טעמא אמר רבי יוחנן משום בטול ברוב נגעו בה 
ואזדא רבי יוחנן לטעמיה דאמר רבי יוחנן רבי שמעון ור"א בן יעקב אמרו דבר אחד רבי שמעון הא דאמרן רבי אליעזר דתניא רבי אליעזר בן יעקב אומר בהמה גסה ששפעה חררת דם הרי זו תקבר ופטורה מן הבכורה 
ותני רבי חייא עלה אינה מטמאה לא במגע ולא במשא ומאחר שאינה מטמאה לא במגע ולא במשא אמאי תקבר כדי לפרסמה שהיא פטורה מן הבכורה 
אלמא ולד מעליא הוא ואמאי תני רבי חייא אינה מטמאה לא במגע ולא במשא אמר רבי יוחנן משום בטול ברוב נגעו בה 
א"ר אמי אמר רבי יוחנן ומודה רבי שמעון שאמו טמאה לידה 
אמר ההוא סבא לרבי אמי אסברא לך טעמא דרבי יוחנן דאמר קרא (ויקרא יב, ב) אשה כי תזריע וילדה זכר וגו' אפילו לא ילדה אלא כעין שהזריעה טמאה לידה 
ריש לקיש אמר שפיר שטרפוהו במימיו נעשה כמת שנתבלבלה צורתו 
אמר ליה רבי יוחנן לריש לקיש מת שנתבלבלה צורתו מנלן דטהור אילימא מהא דאמר רבי שבתאי אמר ר' יצחק מגדלאה ואמרי לה א"ר יצחק מגדלאה א"ר שבתאי מת שנשרף ושלדו קיימת טמא מעשה היה וטמאו לו פתחים גדולים}

\newsection{דף כח}
\twocol{וטהרו לו פתחים קטנים וקא דייקת מינה טעמא דשלדו קיימת הא לאו הכי טהור 
\commenta{\textbf{המפלת יד חתוכה.} פי' רש"י ז"ל חתוכה שיש לה חיתוך אצבעות. וק"ל בלאו הכי נמי ליחוש ללידה שהרי אפילו השפיר שאין לו אפילו חתוך ידים עצמן אמו טמאה לידה.\par ואיכא למימר הכי ספיקא הוא ואם הפילה יד גמורה שאינה חתוכה אומרי' מגוף אטום באת כשם שהיא משונה כך באת מגוף משונה ושמא לאו מגוף באת אלא חתיכה של בשר שנעשית כמין פיסת היד היא הילכך אמו טהורה תולין להקל שרגלים לדבר.\par והרב ר' אברהם בר דוד ז"ל מפרש שלא אמרו חתוכה אלא לענין מביאה קרבן ונאכל דמדקתני ואין חוששין כלל במשמע ואלו בשאינה חתוכה אינו נאכל. (אלא) א) לענין האם טמאה מ"מ. ואין זה לשון הגון מדקתני ברייתא אמו טמא' ואין חוששין ולא קתני מביאה קרבן ונאכל ואין חוששין. }
אדרבה דוק מינה להאי גיסא שלדו קיימת הוא דטהרו לו פתחים קטנים הא לאו הכי פתחים קטנים נמי טמאין דכל חד וחד חזי לאפוקי חד חד אבר 
א"ל רבינא לרב אשי ר' יוחנן דאמר כמאן כר' אליעזר דתנן אפר שרופין ר"א אומר שיעור' ברובע 
היכי דמי מת שנשרף ושלדו קיימת אמר אביי כגון ששרפו על גבי קטבלא רבא אמר כגון ששרפו על גבי אפודרים רבינא אמר כגון דאיחרכי אחרוכי 
ת"ר המפלת יד חתוכה ורגל חתוכה אמו טמאה לידה ואין חוששין שמא מגוף אטום באו 
רב חסדא ורבה בר רב הונא דאמרי תרוייהו אין נותנין לה ימי טוהר מ"ט אימא הרחיקה לידתה 
מתיב רב יוסף המפלת ואין ידוע מה הפילה תשב לזכר ולנקבה ואי ס"ד כל כהאי גוונא אימא הרחיקה לידתה לתני ולנדה 
אמר אביי אי תנא לנדה הוה אמינא מביאה קרבן ואינו נאכל קמ"ל דנאכל 
אמר רב הונא הוציא עובר את ידו והחזירה אמו טמאה לידה שנאמר (בראשית לח, כח) ויהי בלדתה ויתן יד 
מתיב רב יהודה הוציא עובר את ידו אין אמו חוששת לכל דבר אמר רב נחמן לדידי מיפרשא לי מיניה דרב הונא לחוש חוששת ימי טוהר לא יהבינן לה עד דנפיק רוביה 
והא אין אמו חוששת לכל דבר קאמר אמר אביי אינה חוששת לכל דבר מדאורייתא אבל מדרבנן חוששת והא קרא קאמר מדרבנן וקרא אסמכתא בעלמא
{\large\emph{מתני׳}} המפלת טומטום ואנדרוגינוס תשב לזכר ולנקבה
טומטום וזכר אנדרוגינוס וזכר תשב לזכר ולנקבה טומטום ונקבה אנדרוגינוס ונקבה תשב לנקבה בלבד 
יצא מחותך או מסורס משיצא רובו הרי הוא כילוד יצא כדרכו עד שיצא רוב ראשו ואיזהו רוב ראשו משיצא פדחתו
{\large\emph{גמ׳}} השתא טומטום לחודיה ואנדרוגינוס לחודיה אמר תשב לזכר ולנקבה טומטום וזכר אנדרוגינוס וזכר מיבעיא
איצטריך מהו דתימ' הואיל וא"ר יצחק אשה מזרעת תחל' יולדת זכר איש מזריע תחלה יולדת נקבה אימא מדהאי זכר האי נמי זכר קמ"ל אימא שניהם הזריעו בבת אחת זו זכר וזה נקבה 
אמר ר"נ אמר רב טומטום ואנדרוגינוס שראו לובן או אודם אין חייבין על ביאת מקדש ואין שורפין עליהם את התרומה 
ראו לובן ואודם כאחד אין חייבין על ביאת מקדש אבל שורפין עליהם את התרומה שנאמר {במדבר ה׳:ג׳ } מזכר ועד נקבה
תשלחו זכר ודאי נקבה ודאית ולא טומטום ואנדרוגינוס .
לימא מסייע ליה טומטום ואנדרוגינוס שראו לובן או אודם אין חייבין על ביאת מקדש ואין שורפין עליהם את התרומה ראו לובן ואודם כאחת אין חייבין על ביאת מקדש אבל שורפין עליהם את התרומה 
מ"ט לאו משום שנאמר (במדבר ה, ג) מזכר ועד נקבה תשלחו זכר ודאי נקבה ודאית ולא טומטום ואנדרוגינוס אמר עולא לא הא מני ר' אליעזר היא 
דתנן רבי אליעזר אומר השרץ (ויקרא ה, ב) ונעלם ממנו על העלם שרץ הוא חייב ואינו חייב על העלם מקדש 
רבי עקיבא אומר ונעלם ממנו והוא טמא על העלם טומאה הוא חייב ואינו חייב על העלם מקדש 
ואמרינן מאי בינייהו ואמר חזקיה שרץ ונבלה איכא בינייהו דרבי אליעזר סבר בעינן עד דידע אי בשרץ איטמי אי בנבילה איטמי ור' עקיבא סבר לא בעינן 
לאו אמר רבי אליעזר התם בעינן דידע אי בשרץ איטמי אי בנבלה איטמי הכא נמי בעינן דידע אי בלובן איטמי אי באודם איטמי 
אבל לרבי עקיבא דאמר משום טומאה מיחייב הכא נמי משום טומאה מיחייב 
ורב מאי שנא ביאת מקדש דלא דכתיב מזכר ועד נקבה תשלחו זכר ודאי נקבה ודאית ולא טומטום ואנדרוגינוס 
אי הכי תרומה נמי לא נשרוף דכתיב (ויקרא טו, לג) והזב את זובו לזכר ולנקבה זכר ודאי נקבה ודאית ולא טומטום ואנדרוגינוס 
ההוא מבעי ליה לכדרבי יצחק דאמר רבי יצחק לזכר לרבות את המצורע למעינותיו ולנקבה לרבות את המצורעת למעינותיה 
האי נמי מבעי ליה במי שיש לו טהרה במקוה פרט לכלי חרס דברי רבי יוסי 
אם כן נכתוב רחמנא אדם
וכי תימא אי כתב רחמנא אדם הוה אמינא כלי מתכות לא מכל טמא לנפש נפקא זכר ונקבה למה לי לכדרב 
ואימא כוליה לכדרב הוא דאתא אם כן נכתוב זכר ונקבה מאי מזכר ועד נקבה עד כל דבר שיש לו טהרה במקוה 
אי הכי כי איטמי בשאר טומאות לא לישלחו אמר קרא מזכר מטומאה הפורשת מן הזכר 
וכל היכא דכתיב מזכר עד נקבה למעוטי טומטום ואנדרוגינוס הוא דאתא והא גבי ערכין דכתיב {ויקרא כז } הזכר
ותניא הזכר ולא טומטום ואנדרוגינוס יכול לא יהא בערך איש אבל יהא בערך אשה תלמוד לומר הזכר ואם נקבה זכר ודאי נקבה ודאית ולא טומטום ואנדרוגינוס 
טעמא דכתיב הזכר ואם נקבה הא מזכר ונקבה לא ממעט ההוא מבעי ליה}

\newsection{דף כט}
\twocol{לחלק בין ערך איש לערך אשה
\commenta{ הא ד\textbf{אמר ריב"ל עברה בנהר והפילה וכו'.} בדין הוא דנירמי עליה מהא דתניא בריש פירקין ולשלישי הפילה ואינה יודעת מה הפילה מביאה קרבן ואינו נאכל אלמא לכ"ע הלך אחר רוב נשים לא אמרינן אלא מתוקמא ההיא כדתרצינן למתני' בשלא הוחזקה עוברה לפנינו. ודמתני' עדיפא לן למירמי. }
יצא מחותך או מסורס וכו' א"ר אלעזר אפילו הראש עמהן 
ור' יוחנן אמר לא שנו אלא שאין הראש עמהן אבל הראש עמהן הראש פוטר 
לימא בדשמואל קמיפלגי דאמר שמואל אין הראש פוטר בנפלים 
בשלם דכולי עלמא לא פליגי כי פליגי במחותך דמר סבר בשלם הוא דקחשיב במחותך לא קחשיב ומר סבר במחותך נמי חשיב 
לישנא אחרינא טעמא דיצא מחותך או מסורס הא כתקנו הראש פוטר תרוייהו לית להו דשמואל דאמר שמואל אין הראש פוטר בנפלים 
איכא דמתני לה להא שמעתתא באפי נפשה א"ר אלעזר אין הראש כרוב אברים ורבי יוחנן אמר הראש כרוב אברים וקמיפלגי בדשמואל 
תנן יצא מחותך או מסורס משיצא רובו הרי הוא כילוד מדקאמר מסורס מכלל דמחותך כתקנו וקאמר משיצא רובו הרי זה כילוד קשיא לרבי יוחנן 
אמר לך רבי יוחנן אימא יצא מחותך ומסורס 
והא או קתני הכי קאמר יצא מחותך או שלם וזה וזה מסורס משיצא רובו הרי זה כילוד 
אמר רב פפא כתנאי יצא מחותך או מסורס משיצא רובו הרי הוא כילוד רבי יוסי אומר משיצא כתקנו מאי קאמר 
אמר רב פפא הכי קאמר יצא מחותך ומסורס משיצא רובו הרי הוא כילוד הא כתקנו הראש פוטר רבי יוסי אומר משיצא רובו כתקנו 
מתקיף לה רב זביד מכלל דבמסורס רובו נמי לא פוטר הא קי"ל דרובו ככולו 
אלא אמר רב זביד הכי קאמר יצא מחותך ומסורס משיצא רובו הרי זה כילוד הא כתקנו הראש פוטר רבי יוסי אומר משיצא כתקנו לחיים 
תניא נמי הכי יצא מחותך (או) מסורס משיצא רובו הרי זה כילוד הא כתקנו הראש פוטר ר' יוסי אומר משיצא כתקנו לחיים 
ואיזהו כתקנו לחיים משיצא רוב ראשו ואיזהו רוב ראשו ר' יוסי אומר משיצאו צדעיו אבא חנן משום ר' יהושע אומר משיצא פדחתו וי"א משיראו קרני ראשו
{\large\emph{מתני׳}} המפלת ואין ידוע מהו תשב לזכר ולנקבה אין ידוע אם ולד היה אם לאו תשב לזכר ולנקבה ולנדה
{\large\emph{גמ׳}} א"ר יהושע בן לוי עברה נהר והפילה מביאה קרבן ונאכל הלך אחר רוב נשים ורוב נשים ולד מעליא ילדן 
תנן אין ידוע אם ולד היה תשב לזכר ולנקבה ולנדה אמאי תשב לנדה לימא הלך אחר רוב נשים ורוב נשים ולד מעליא ילדן 
מתני' בשלא הוחזקה עוברה וכי קאמר ריב"ל כשהוחזקה עוברה 
ת"ש בהמה שיצאה מלאה ובאה ריקנית הבא אחריו בכור מספק 
ואמאי הלך אחר רוב בהמות ורוב בהמות ולד מעליא ילדן והאי פשוט הוא 
אמר רבינא משום דאיכא למימר רוב בהמות יולדות דבר הפוטר מבכורה ומעוטן יולדות דבר שאינו פוטר מבכורה וכל היולדות מטנפות וזו הואיל ולא טנפה אתרע לה רובא 
אי כל היולדות מטנפות הא מדלא מטנפה בכור מעליא הוא אלא אימא רוב יולדות מטנפות וזו הואיל ולא טנפה אתרע לה רובא 
כי אתא רבין אמר מתיב רבי יוסי ברבי חנינא טועה ולא ידענא מאי תיובתא מאי היא דתניא
אשה שיצתה מלאה ובאה ריקנית והביאה לפנינו שלשה שבועין טהורין ועשרה שבועות אחד טמא ואחד טהור
\commenta{ והא דפריך מינה ר' יוסי בר חנינא ורבין לא ידע מאי תיובתיה משום \textbf{אימר הרחיקה לידתה.} דמשמע דלית ליה לר' יוסי בר חנינא הרחיקה לידתה קשיא טובא וכי היאך סלקה על דעתו כן. והא אי לאו משום הך תשש לא היו מבטלין אותה בשבוע ג' בליליותא דמשום טבולת יום ארוך מטבילין אות' שמא כבר עברו לה ימי טוהר וכ"ש בשבו' ד' דאיכא למימר כבר עברו וכן טבילות דב"ה נמי משום יולדת והרחיק' לידתה ז' או שבועים הן ועוד שבוע דטהור הוא תשמש דאי ילדה ולד מעליא אפי' בשבוע ד' נמי טהורה היא דדם טוהר הוא וכ"ש בה' דטהור' ואם לאו ולד מעליא הוה לספיקה דר' יוסי בר חנינא תחלת שבוע רביעי ה"ל תחלת ונדה ושבוע דטהור הוא מותרת אלא ע"כ משום הרחיקה לידת' וחוששין שמא כלו ימי טוהר בסוף שבוע רביעי ויום אחרון שבו היה לה התחלת נדה כדמפר' ואזיל בגמ'.\par אלא ע"כ ר' יוסי בר חנינא אגב חורפיה לא עיין בה ובגמ' ה"ל למימר ולטעמיך מ"ט דכל הני אלא אשכחן כמה דוכתי בתלמודא דהוה מצי למיפרך וליטעמיך ולא פריך ביה כלל. }
משמשת לאור שלשים וחמש ומטבילין אותה תשעים וחמש טבילות דברי ב"ש וב"ה אומרים שלשים וחמש רבי יוסי בר' יהודה אומר דיה לטבילה שתהא באחרונה 
\commenta{\textbf{שבוע קמא מטבילין לה בלילותא משום יולדת זכר ונקבה.} עיינו בתוספות שאין השבועין הנמצאין כאן בטבילות הלילו' שוים עם השבועין הנמנים כאן בטביל' הימים דהא למאי דס"ד מעיקרא שבאת לפנינו ביום וכן למאי דמתרצינן כגון שבאת לפנינו בין השמשות טבילות דלילותא מושכות עד לילה של שבוע שלאחריו כגון שבאת לפנינו בין השמשות של מוצאי שבת וכן שבאת לפנינו באחד בשבת ביום וטבילה ראשונה של לילה בליל שני בשבת ואחרונה במוצאי שבת וכן בשבוע שני ואלו טבילו' דימים דמשום זיבה ראשונה באחד בשבת ואחרונה בשבת. וליכא למימר דברייתא הכי קתני שהביאה לפנינו ג' שבועין טהורין חוץ מיום שבאת לפנינו שהרי אותו היום עילה הוא למנין שבועים ונמצאת זאת מותרת לשמש בלילי עשרין וחד שהרי אינה רואה כל אותה הליל' ולא יום שלאחריו אלא ע"כ יום שבאת לפנינו הוא ממנין שלשה שבועים טהורין. כל זה עיינו בתוספות.\par ודבר ברור הוא אלא כיון דמנין לידה וזיבה מיום א' בשבת הוא וכל טבילו' דעלמא הן דכל נדה ויולדת טבילתן בלילה של שבוע שני וטבילות דזיבה ביום בסוף שבוע שלהן לא חיישי בגמרא לפרושי הכא מידי. }
בשלמא שבוע ראשון לא משמשת אימר יולדת זכר היא שבוע שני אימר יולדת נקבה היא
שבוע שלישי אימר יולדת נקבה בזוב היא
אלא שבוע רביעי אע"ג דקא חזיא דם תשמש דהא דם טהור הוא לאו משום דלא אזלינן בתר רובא 
אלא מאי לא ידענא מאי תיובתא אימר הרחיקה לידתה 
הך שבוע חמישי דטהור הוא תשמש 
הך שבוע רביעי כל יומא ויומא מספקין בסוף לידה ובתחלת נדה ועשרין ותמניא גופיה אימר תחלת נדה היא ובעיא למיתב שבעה לנדתה 
בעשרים וחד תשמש 
רבי שמעון היא דאמר אסור לעשות כן שמא תבא לידי ספק לאורתא תשמש כשראתה בערב 
ומטבילין אותה תשעים וחמש טבילות שבוע קמא מטבילין אותה בלילותא אימר יולדת זכר היא 
שבוע שני מטבילין אותה בלילותא אימר יולדת נקבה היא ביממא אימר יולדת זכר בזוב היא 
שבוע שלישי מטבילין לה ביממא אימר יולדת נקבה בזוב היא
בלילותא ב"ש לטעמייהו דאמרי טבולת יום ארוך בעי טבילה}

\newsection{דף ל}
\twocol{מכדי ימי טהרה כמה הוו שתין ושיתא דל שבוע ג' דאטבלינן לה פשו להו שתין נכי חדא שתין נכי חדא ותלתין וה' תשעין וד' הויין תשעין וחמש מאי עבידתייהו 
\commenta{ ל"ה טבילות דקאמרי ב"ה קשיא לן כיון דאוקים ב\textbf{באה לפנינו בין השמשות דיהיבנא לה טבילה בתריהן.} תלתין ושש הווין. וראיתי בפירושים דכיון דתדא בשבוע היא לא קחשיב ואינו יודע מהו שאם בא לומר דטבילה דסוף שבוע רביעי הויא חדא בשבוע לא משמע הכי דהא טבלה נמי בימים הסמוכים לה ששה עד סוף ז' ובאור שביעי של שבוע חמישי גמרה טבילותיה וטהורה ואפשר שאותה טבילה ראשונה חדא בשבוע חשיבי לה ב"ה מפני שהיא נמנת לסוף שבוע שעבר ואותו היום עצמו נמנה לנו תחל' שבוע ללידה וטבילות שאח"כ ואט"ג דב"ש מנו לה ולא חשבי חדא בשבוע אינהי דמפשי טבילו' מנו לה כיון דמצטרפא בטבילות דלילות דשבוע א' אבל ב"ה לא מנו לה.\par וה"ר אב"ד ז"ל כתב דאיכא לתרוצי דכיון דלאו פסיקא להו דאי אתאי ביממי להא טבילה לא חשיבי ב"ה כי היכי דתרצינן בטועה בפ' בתרא. וזה הלשון נכון בעיני דב"ש דקא מפשי טבילות מהדרי לאפושי בהו טובא וב"ה דלא מפשי בהו טפי לא חשיב' לה.\par ובשם הרב חתנו ז"ל תירץ דבין השמשות דר' יהודה אפליגי ב"ש וב"ה ב"ש סברי כר' יהודה דספיקא הוא וב"ה סברי כר' יוסי דבין השמשות דר' יהודה יממא הוא. ולכך ליתא לטבילה יתירתי' ועומק גדול הוא אלא תימה גדול הוא היכי שתיק מיניה תלמודא. זה לשון הרב ז"ל. }
אמר רב ירמיה מדפתי כגון שבאת לפנינו בין השמשות דיהבינן לה טבילה יתירתא 
\commenta{\textbf{איידי דפתח בשבוע מסיק לה איידי דתנא טמא תנא טהור.} פי' וה"ה דלענין צ"ה טבילות דב"ש ה"נ הוה קמ"ל בעשרה שבועים כולם טמאים או טהורים אלא משום דבעי למיתנא משמשת לאור ל"ה לא קודם לכן ולא לאחר כן משום חששות דאמרן תנא הכי.\par והק' בתוספ' כיון דמנינו עשרה שבועי נפישי להו טבילות שהרי שבוע ט' דטמא הוא ג' ימים ראשונים שבו איכא לספוקינהו בסוף לידה ותחלת נדה ויום ד' תחלת נדה ונמצאת צריכה טבילה לג' ימים בשבוע עשירי. ותירצו עד סוף שמוני' קחשיב לאחר פ' לא תשיב דהא לא תננהו אלא אגב גררא. וכ"ש למאי דפרקינן בסמוך דלא מיירי ב"ש אלא בלידה. }
ולב"ה דאמרי טבולת יום ארוך לא בעי טבילה ל"ה מאי עבידתייהו 
\commenta{ והא דאקשי\textbf{יומא קמא דאתיא לקמן לטבילה דילמא שומרת יום כנג' יום היא} לאו לב"ש מקשינן דהא אינהו לא זיבה גדולה ולא זבה קטנה קחשיבי אלא יולדת בזוב בלחוד כדאמרן. אלא לב"ה בעינן דהא דמחרצינן זיבה גרידתא לא קחשיב לב"ה לא צרכינן למימר הכי אלא דמקמי תשמיש קחשיבי כולהו דלבתר תשמיש לא קחשיבי. א"נ השתא דאתית להכי ליומא דחדא בשבוע לא קחשיב הדרי' מההוא טעמא דטבילת זבה חדא בשבוע נינהו ולפום הכי אקשי' ליחשוב דשומרת יום וה"ל ג' טבילות בשבוע זו ביום כיון שבאה בין השמשות וליחשוב ומפרקינן זבה גדולה קחשיב כלומר יולדת בזוב. א"נ זבה הוא קחשיב אי מיתרמיא ליה לפני תשמיש אבל זבה קטנה לא חשיב.\par וק"ל ולימא דילמא כשילדה ראתה יום א' בימי זיבה וצריכה לשמור יום כנגד יום ואין ספירת ימי לידתה עולין לה ודאי כשם שאינן עולין לםפירת זבה גדולה וליטבי' כל שבוע קמא ביממי משום שומרת יום כנג' יום זיבה שלפני לידתה ואמאי פריך יומא קמא בלחוד.\par ואיכא למימר דקסבר האומר שהימי לידה עולין לשמור דזיבה קטנה ויולדת בזוב קטן דמקש' דלטבילה יומא קמא דילמא יולד' בזוב קטן היא והרי ספרה יום זה לפנינו בין לב"ש בין לב"ה מקשי' וכן נראה לי עיקר דימי לידה אין עולין גמירי לה לקמן בפ' בנות כותיים מדכתיב כימי נדת דותה תטמא מה ימי נידתה אין ראויין לזיבה ואין ספירת ז' עולה בהן אף ימי לידתה כן, והא ליתא אלא לספירתן דזבה גדולה אבל שימור דזבה קטנה אף בימי נדה עולה דאפשר הוא כדאיתא בשלהי בא סימן (נג, א). }
עשרים ותמניא כדאמרן הך שבוע ה' מטבלינן כל ליליא וליליא אימר סוף נדה היא 
\commenta{והא דאמרינן \textbf{ש"מ תלתא.} איכא למידק ולימא נמי ש"מ ד' דהא ש"מ ימי לידה שאינה רואה בהן אין עולין לה לימי זיבת' ואיכא למימר דההיא פלוגתא דאביי ורבא היא ורבה דאמר עולין קסבר הא מני ר' אליעזר הוא דאמר מסתר נמי סתרא. ולפום הכי נמי לא אמרי' ש"מ ר' אלעזר היא כדאמרינן ש"מ ר' עקיבא היא ור' שמעון היא. משום דלאביי דברי הכל אינן עולין הלכך לא פסיקא ליה. }
י' שבועין למה לי בתמניא ופלגא סגי 
איידי דתנא פלגא דשבוע מסיק ליה ואיידי דתנא שבוע טמא תנא נמי שבוע טהור 
והאיכא טבילת זבה 
דלפני תשמיש קחשיב דלאחר תשמיש לא קחשיב 
ולב"ש דחשיב דלאחר תשמיש ניחשוב נמי טבילת זבה בלידה קמיירי בזיבה לא קמיירי 
והאיכא יולדת בזוב יולדת בזוב קחשיב זיבה גרידתא לא קחשיב 
שבועתא קמא דאתיא לקמן ליטבלה ביומא דילמא כל יומא ויומא שלימו לה ספורים דידה 
הא מני ר"ע היא דאמר בעינן ספורים בפנינו 
סוף שבוע קמא ליטבלה חד בשבוע לא קמיירי 
יומא קמא דאתיא לקמן ליטבלה דילמא שומרת יום כנגד יום היא בזבה גדולה קמיירי בזבה קטנה לא קמיירי 
ש"מ תלת ש"מ ר"ע היא דאמר בעינן ספורים בפנינו 
וש"מ ר"ש היא דאמר אבל אמרו חכמים אסור לעשות כן שמא תבא לידי ספק 
וש"מ טבילה בזמנה מצוה ורבי יוסי בר' יהודה אומר דיה לטבילה באחרונה ולא אמרינן טבילה בזמנה מצוה
{\large\emph{מתני׳}} המפלת ליום מ' אינה חוששת לולד ליום מ"א תשב לזכר ולנקבה ולנדה
רבי ישמעאל אומר יום מ"א תשב לזכר ולנדה יום פ"א תשב לזכר ולנקבה ולנדה שהזכר נגמר למ"א והנקבה לפ"א וחכ"א אחד בריית הזכר ואחד בריית הנקבה זה וזה מ"א
{\large\emph{גמ׳}} למה הוזכר זכר
אי לימי טומאה הא קתני נקבה ואי לימי טהרה
הא קתני נדה 
שאם תראה יום ל"ד ותחזור ותראה יום מ' ואחד תהא מקולקלת עד מ"ח 
וכן לענין נקבה שאם תראה יום ע"ד ותחזור ותראה יום פ"א תהא מקולקלת עד פ"ח
רבי ישמעאל אומר יום מ"א תשב לזכר ולנדה כו' תניא רבי ישמעאל אומר טימא וטיהר בזכר וטימא וטיהר בנקבה
מה כשטימא וטיהר בזכר יצירתו כיוצא בו אף כשטימא וטיהר בנקבה יצירתה כיוצא בה אמרו לו אין למדין יצירה מטומאה 
אמרו לו לר' ישמעאל מעשה בקליאופטרא מלכת אלכסנדרוס שנתחייבו שפחותיה הריגה למלכות ובדקן ומצאן זה וזה למ"א אמר להן אני מביא לכם ראייה מן התורה ואתם מביאין לי ראייה מן השוטים 
מאי ראיה מן התורה אילימא טימא וטיהר בזכר וטימא וטיהר בנקבה כו' הא קאמרי ליה אין דנין יצירה מטומאה 
אמר קרא תלד הוסיף לה הכתוב לידה אחרת בנקבה 
ומאי ראיה מן השוטים אימר נקבה קדים ואיעבור ארבעין יומין קמי זכר 
ורבנן סמא דנפצא אשקינהו ור' ישמעאל איכא גופא דלא מקבל סמא
אמר להם ר' ישמעאל מעשה בקלפטרא מלכת יוונית שנתחייבו שפחותיה הריגה למלכות ובדקן ומצאן זכר לארבעים ואחד ונקבה לפ"א אמרו לו אין מביאין ראיה מן השוטים 
מאי טעמא הך דנקבה אייתרה ארבעין יומין והדר איעבר 
ורבי ישמעאל לשומר מסרינהו ורבנן אין אפוטרופוס לעריות אימא שומר גופיה בא עליה 
ודילמא אי קרעוהו להך דנקבה בארבעין וחד הוה משתכחא כזכר אמר אביי בסימניהון שוין
וחכ"א אחד בריית זכר ואחד בריית נקבה וכו' חכמים היינו ת"ק 
וכי תימא למסתמא רישא כרבנן ויחיד ורבים הלכה כרבים פשיטא 
מהו דתימא מסתברא טעמא דרבי ישמעאל דקמסייע ליה קראי קמ"ל
דרש רבי שמלאי למה הולד דומה במעי אמו לפנקס שמקופל ומונח ידיו על שתי צדעיו שתי אציליו על ב' ארכובותיו וב' עקביו על ב' עגבותיו וראשו מונח לו בין ברכיו ופיו סתום וטבורו פתוח ואוכל ממה שאמו אוכלת ושותה ממה שאמו שותה ואינו מוציא רעי שמא יהרוג את אמו וכיון שיצא לאויר העולם נפתח הסתום ונסתם הפתוח שאלמלא כן אינו יכול לחיות אפילו שעה אחת 
ונר דלוק לו על ראשו וצופה ומביט מסוף העולם ועד סופו שנאמר (איוב כט, ג) בהלו נרו עלי ראשי לאורו אלך חשך ואל תתמה שהרי אדם ישן כאן ורואה חלום באספמיא 
ואין לך ימים שאדם שרוי בטובה יותר מאותן הימים שנאמר (איוב כט, ב) מי יתנני כירחי קדם כימי אלוה ישמרני ואיזהו ימים שיש בהם ירחים ואין בהם שנים הוי אומר אלו ירחי לידה 
ומלמדין אותו כל התורה כולה שנאמר (משלי ד ד) ויורני ויאמר לי יתמך דברי לבך שמור מצותי וחיה ואומר (איוב כט, ד) בסוד אלוה עלי אהלי 
מאי ואומר וכי תימא נביא הוא דקאמר ת"ש בסוד אלוה עלי אהלי 
וכיון שבא לאויר העולם בא מלאך וסטרו על פיו ומשכחו כל התורה כולה שנאמר (בראשית ד, ז) לפתח חטאת רובץ 
ואינו יוצא משם עד שמשביעין אותו שנאמר (ישעיהו מה, כג) כי לי תכרע כל ברך תשבע כל לשון כי לי תכרע כל ברך זה יום המיתה שנאמר (תהלים כב, ל) לפניו יכרעו כל יורדי עפר תשבע כל לשון זה יום הלידה שנאמר (תהלים כד, ד) נקי כפים ובר לבב אשר לא נשא לשוא נפשו ולא נשבע למרמה
ומה היא השבועה שמשביעין אותו תהי צדיק ואל תהי רשע ואפילו כל העולם כולו אומרים לך צדיק אתה היה בעיניך כרשע והוי יודע שהקב"ה טהור ומשרתיו טהורים ונשמה שנתן בך טהורה היא אם אתה משמרה בטהרה מוטב ואם לאו הריני נוטלה ממך 
תנא דבי ר' ישמעאל משל לכהן שמסר תרומה לעם הארץ ואמר לו אם אתה משמרה בטהרה מוטב ואם לאו הריני שורפה לפניך 
א"ר אלעזר}

\newsection{דף לא}
\twocol{מאי קרא (תהלים עא, ו) ממעי אמי אתה גוזי מאי משמע דהאי גוזי לישנא דאשתבועי הוא דכתיב (ירמיהו ז, כט) גזי נזרך והשליכי 
ואמר רבי אלעזר למה ולד דומה במעי אמו לאגוז מונח בספל של מים אדם נותן אצבעו עליו שוקע לכאן ולכאן 
תנו רבנן שלשה חדשים הראשונים ולד דר במדור התחתון אמצעיים ולד דר במדור האמצעי אחרונים ולד דר במדור העליון וכיון שהגיע זמנו לצאת מתהפך ויוצא וזהו חבלי אשה 
והיינו דתנן חבלי של נקבה מרובין משל זכר 
ואמר רבי אלעזר מאי קרא (תהלים קלט, טו) אשר עשיתי בסתר רקמתי בתחתיות ארץ דרתי לא נאמר אלא רקמתי
מאי שנא חבלי נקבה מרובין משל זכר זה בא כדרך תשמישו וזה בא כדרך תשמישו זו הופכת פניה וזה אין הופך פניו 
תנו רבנן שלשה חדשים הראשונים תשמיש קשה לאשה וגם קשה לולד אמצעיים קשה לאשה ויפה לולד אחרונים יפה לאשה ויפה לולד שמתוך כך נמצא הולד מלובן ומזורז 
תנא המשמש מטתו ליום תשעים כאילו שופך דמים מנא ידע אלא אמר אביי משמש והולך (תהלים קטז, ו) ושומר פתאים ה' 
תנו רבנן שלשה שותפין יש באדם הקב"ה ואביו ואמו אביו מזריע הלובן שממנו עצמות וגידים וצפרנים ומוח שבראשו ולובן שבעין אמו מזרעת אודם שממנו עור ובשר ושערות ושחור שבעין והקב"ה נותן בו רוח ונשמה וקלסתר פנים וראיית העין ושמיעת האוזן ודבור פה והלוך רגלים ובינה והשכל 
וכיון שהגיע זמנו להפטר מן העולם הקב"ה נוטל חלקו וחלק אביו ואמו מניח לפניהם אמר רב פפא היינו דאמרי אינשי פוץ מלחא ושדי בשרא לכלבא 
דרש רב חיננא בר פפא מאי דכתיב (איוב ט, י) עושה גדולות עד אין חקר ונפלאות עד אין מספר בא וראה שלא כמדת הקב"ה מדת בשר ודם מדת בשר ודם נותן חפץ בחמת צרורה ופיה למעלה ספק משתמר ספק אין משתמר ואילו הקב"ה צר העובר במעי אשה פתוחה ופיה למטה ומשתמר 
דבר אחר אדם נותן חפציו לכף מאזנים כל זמן שמכביד יורד למטה ואילו הקב"ה כל זמן שמכביד הולד עולה למעלה 
דרש רבי יוסי הגלילי מאי דכתיב {תהילים קל״ט:י״ד } אודך (ה') על כי נוראות נפליתי נפלאים מעשיך ונפשי יודעת מאד בא וראה שלא כמדת הקב"ה מדת בשר ודם מדת בשר ודם אדם נותן זרעונים בערוגה כל אחת ואחת עולה במינו ואילו הקב"ה צר העובר במעי אשה וכולם עולין למין אחד 
דבר אחר צבע נותן סמנין ליורה כולן עולין לצבע אחד ואילו הקב"ה צר העובר במעי אשה כל אחת ואחת עולה למינו 
דרש רב יוסף מאי דכתיב (ישעיהו יב, א) אודך ה' כי אנפת בי ישוב אפך ותנחמני במה הכתוב מדבר 
בשני בני אדם שיצאו לסחורה ישב לו קוץ לאחד מהן התחיל מחרף ומגדף לימים שמע שטבעה ספינתו של חבירו בים התחיל מודה ומשבח לכך נאמר ישוב אפך ותנחמני 
והיינו דאמר רבי אלעזר מאי דכתיב (תהלים עב, יח) עושה נפלאות (גדולות) לבדו וברוך שם כבודו לעולם אפילו בעל הנס אינו מכיר בנסו 
דריש רבי חנינא בר פפא מאי דכתיב (תהלים קלט, ג) ארחי ורבעי זרית וכל דרכי הסכנת מלמד שלא נוצר אדם מן כל הטפה אלא מן הברור שבה תנא דבי רבי ישמעאל משל לאדם שזורה בבית הגרנות נוטל את האוכל ומניח את הפסולת 
כדרבי אבהו דרבי אבהו רמי כתיב (שמואל ב כב, מ) ותזרני חיל וכתיב (תהלים יח, לג) האל המאזרני חיל אמר דוד לפני הקב"ה רבש"ע זיריתני וזרזתני 
דרש רבי אבהו מאי דכתיב (במדבר כג, י) מי מנה עפר יעקב ומספר את רובע ישראל מלמד שהקב"ה יושב וסופר את רביעיותיהם של ישראל מתי תבא טיפה שהצדיק נוצר הימנה 
ועל דבר זה נסמית עינו של בלעם הרשע אמר מי שהוא טהור וקדוש ומשרתיו טהורים וקדושים יציץ בדבר זה מיד נסמית עינו דכתיב (במדבר כד, ג) נאם הגבר שתום העין 
והיינו דאמר רבי יוחנן מאי דכתיב (בראשית ל, טז) וישכב עמה בלילה הוא מלמד שהקב"ה סייע באותו מעשה שנאמר (בראשית מט, יד) יששכר חמור גרם חמור גרם לו ליששכר 
אמר רבי יצחק אמר רבי אמי אשה מזרעת תחילה יולדת זכר איש מזריע תחילה יולדת נקבה שנאמר (ויקרא יג, כט) אשה כי תזריע וילדה זכר 
תנו רבנן בראשונה היו אומרים אשה מזרעת תחילה יולדת זכר איש מזריע תחלה יולדת נקבה ולא פירשו חכמים את הדבר עד שבא רבי צדוק ופירשו (בראשית מו, טו) אלה בני לאה אשר ילדה ליעקב בפדן ארם ואת דינה בתו תלה הזכרים בנקבות ונקבות בזכרים 
(דברי הימים א ח, מ) ויהיו בני אולם אנשים גבורי חיל דורכי קשת ומרבים בנים ובני בנים וכי בידו של אדם להרבות בנים ובני בנים אלא מתוך
שמשהין עצמן בבטן כדי שיזריעו נשותיהן תחלה שיהו בניהם זכרים מעלה עליהן הכתוב כאילו הם מרבים בנים ובני בנים והיינו דאמר רב קטינא יכולני לעשות כל בני זכרים אמר רבא הרוצה לעשות כל בניו זכרים יבעול וישנה 
ואמר רבי יצחק אמר רבי אמי אין אשה מתעברת אלא סמוך לוסתה שנאמר (תהלים נא, ז) הן בעון חוללתי 
ורבי יוחנן אמר סמוך לטבילה שנאמר (תהלים נא, ז) ובחטא יחמתני אמי 
מאי משמע דהאי חטא לישנא דדכויי הוא דכתיב (ויקרא יד, מט) וחטא את הבית ומתרגמינן וידכי ית ביתא ואי בעית אימא מהכא (תהלים נא, ט) תחטאני באזוב ואטהר 
ואמר רבי יצחק אמר רבי אמי כיון שבא זכר בעולם בא שלום בעולם שנאמר (ישעיהו טז, א) שלחו כר מושל ארץ זכר זה כר 
ואמר ר' יצחק דבי רבי אמי בא זכר בעולם בא ככרו בידו זכר זה כר דכתיב (מלכים ב ו, כג) ויכרה להם כירה גדולה 
נקבה אין עמה כלום נקבה נקייה באה עד דאמרה מזוני לא יהבי לה דכתיב (בראשית ל, כח) נקבה שכרך עלי ואתנה 
שאלו תלמידיו את רבי שמעון בן יוחי מפני מה אמרה תורה יולדת מביאה קרבן אמר להן בשעה שכורעת לילד קופצת ונשבעת שלא תזקק לבעלה לפיכך אמרה תורה תביא קרבן 
מתקיף לה רב יוסף והא מזידה היא ובחרטה תליא מילתא ועוד קרבן שבועה בעי איתויי 
ומפני מה אמרה תורה זכר לשבעה ונקבה לארבעה עשר זכר שהכל שמחים בו מתחרטת לשבעה נקבה שהכל עצבים בה מתחרטת לארבעה עשר 
ומפני מה אמרה תורה מילה לשמונה שלא יהו כולם שמחים ואביו ואמו עצבים 
תניא היה ר"מ אומר מפני מה אמרה תורה נדה לשבעה מפני שרגיל בה וקץ בה אמרה תורה תהא טמאה שבעה ימים כדי שתהא חביבה על בעלה כשעת כניסתה לחופה 
שאלו תלמידיו את רבי דוסתאי ברבי ינאי מפני מה איש מחזר על אשה ואין אשה מחזרת על איש משל לאדם שאבד לו אבידה מי מחזר על מי בעל אבידה מחזיר על אבידתו 
ומפני מה איש פניו למטה ואשה פניה למעלה כלפי האיש זה ממקום שנברא וזו ממקום שנבראת 
ומפני מה האיש מקבל פיוס ואין אשה מקבלת פיוס זה ממקום שנברא וזו ממקום שנבראת 
מפני מה אשה קולה ערב ואין איש קולו ערב זה ממקום שנברא וזו ממקום שנבראת שנאמר {שיר השירים ב } כי קולך ערב ומראך נאוה
\par \par {\large\emph{הדרן עלך המפלת חתיכה}}\par \par 
מתני׳ {\large\emph{בנות}} כותים נדות מעריסתן והכותים מטמאים משכב תחתון כעליון מפני שהן בועלי נדות
והן יושבות על כל דם ודם 
ואין חייבין עליהן על ביאת מקדש ואין שורפין עליהם את התרומה מפני שטומאתן ספק 
{\large\emph{גמ׳}} ה"ד אי דקא חזיין אפילו דידן נמי ואי דלא קחזיין דידהו נמי לא
אמר רבא בריה דרב אחא בר רב הונא אמר רב ששת הכא במאי עסקינן בסתמא דכיון דאיכא מיעוטא דחזיין חיישינן ומאן תנא דחייש למיעוטא}

\newchap{פרק \hebrewnumeral{4} בנות כותים}}

\newsection{דף לב}
\twocol{
\commenta{\textbf{הא נמי מיעוטא דשכיח הוא דתניא מעשה והטבילוה קודם לאמה.} וא"ת שמא משום נגיעות אמה בה הטבילוה לסוכה בתרומה. י"ל שיודעין היה בגמרא שלא בא ר' יוסי אלא להעיד על טומאת עצמה והכי קתני מעשה היה שפרשה נדה והטבילוה קודם לאמה. ולא אטבילה בלחוד אסהוד אלא אפרשה אסהיד דאי לאו הכי פשיטא ותא חזי מאן גברא רבה מסהיד עליה. א"נ אין דרכן של בני אדם להפרישה מאמה אלא לכך הטבילוה שלא תטמא את הנשים שגוגעות בה ומגפפות אותה ויחזרו ויטמאו הן תרומה שבא"י אבל בנגיעת אמה בה אין להקפיד לטומאות הנוגעים בה שהרי היא ראשון ואין מטמאה אדם. ואותה שבפומדיתא נמי משטבלה לטומא' גופה אינה צריכה להפרישה מאמה כדמפר' ואזיל. }
ר"מ היא דתניא קטן וקטנה לא חולצין ולא מיבמין דברי ר"מ 
\commenta{\textbf{ולא יחללו את קדשי בני ישראל לרבות את הסך ואת השותה.} י"מ שהיא אסמכתא דרבנן דהא קי"ל גבי יום הכפורים דאכילה ושתיה דאוריית' ובכרת ואין סיכה בכלל שתיה.\par ויש לפרש אע"פ שנתרבה סיכה כשתייה לענין תרומה מריבוי הכתוב לשאר כל התורה כולה אינה כשתיה ואי משום וכשמן בעצמותיו דשייך נמי בכל התורה ההיא ודאי אסמכת' בעלמא היא מדברי קבלה ומיהו ודאי מכיון דאמרינן גבי תרומה גופה מולא יחללו ואיבעית אימא מוכשמן בעצמותיו משמע דכולה דרבנן היא.\par ומאחר שכתבתי סברות הללו מצאתי בפ"ב ממסכת מעשר שני שאמרו בירושלמי לענין מעש' יצהרך זו סיכה והתור' קראתו אכילה ואינו מחוור וא"ת מחוו' ולקו עליו חוץ לחומ' וכו'. ומייתי נמי התם והתני שוה סיכה לשתי' לאסור ולתשלומין לא לעונש יום הכפורים ומקשי והתני לא יחללו מ"מ להביא הסך והשותה. }
אמרו לו לר"מ יפה אמרת שאין חולצין איש כתוב בפרשה ומקשינן אשה לאיש ומה טעם אין מיבמין 
\commenta{\textbf{א"ר יוחנן לית כאן לאסר וכו'.} משמע דסיכה כשתיה דרבנן ואינה מחוורת מן התורה. }
אמר להן קטן שמא ימצא סריס קטנה שמא תמצא אילונית ונמצאו פוגעין בערוה שלא במקום מצוה 
ורבנן זיל בתר רובא דקטנים ורוב קטנים לאו סריסים נינהו זיל בתר רובא דקטנות ורוב קטנות לאו אילונית נינהו 
אימר דשמעת ליה לר"מ מיעוטא דשכיח אבל מיעוטא דלא שכיח מי שמעת ליה 
הא נמי מיעוטא דשכיח הוא דתניא א"ר יוסי מעשה בעין בול והטבילוה קודם לאמה ואמר רבי מעשה בבית שערים והטבילוה קודם לאמה וא"ר יוסף מעשה בפומבדיתא והטבילוה קודם לאמה 
בשלמא דר' יוסי ודרבי משום תרומת א"י אלא דרב יוסף למה לי והא אמר שמואל אין תרומת חו"ל אסורה אלא במי שטומאה יוצאה מגופו והני מילי באכילה אבל בנגיעה לא 
אמר מר זוטרא לא נצרכה אלא לסוכה שמן של תרומה דתניא (ויקרא כב, טו) ולא יחללו את קדשי בני ישראל אשר ירימו לה' לרבות את הסך ואת השותה
שותה למה לי קרא שתיה בכלל אכילה אלא לרבות את הסך כשותה ואיבעית אימא מהכא (תהלים קט, יח) ותבא כמים בקרבו וכשמן בעצמותיו 
אי הכי דידן נמי 
אנן דדרשינן אשה ואשה וכי חזיין מפרשי להו לא גזרו בהו רבנן אינהו דלא דרשי אשה ואשה וכי חזיין לא מפרשי להו גזרו בהו רבנן 
מאי אשה ואשה דתניא אשה אין לי אלא אשה תינוקת בת יום אחד לנדה מנין ת"ל ואשה 
אלמא כי מרבי קרא בת יום אחד מרבי ורמינהו אשה אין לי אלא אשה תינוקת בת ג' שנים ויום אחד לביאה מנין ת"ל ואשה 
אמר רבא הלכתא נינהו ואסמכינהו רבנן אקראי הי קרא והי הלכתא אילימא בת יום אחד הלכתא בת שלש שנים ויום אחד קרא קרא סתמא כתיב 
אלא בת ג' שנים ויום אחד הלכתא בת יום אחד קרא ומאחר דהלכתא קרא ל"ל
למעוטי איש מאודם 
\commenta{\textbf{למעוטי אשה מלובן.} מצאתי בתוספות שמקשים למה לי מיעוטא והרי מצינו ה' דמים טמאי' באשה ותו לא. וי"ל דנהי דאינו דם הוה אתי בק"ו ומה איש שאינו מטמא באודם מטמא בלובן אשה שמטמאה באודם אינו דין שתטמא בלובן ודם טהור באשה מיתוקם בירוק ודיהה כך השיב ר"ש לר' יהודה חתנו ז"ל.\par והם הקשו בתוספות והא ק"ו פירכא הוא מה לנקבה שכן אינה מטמא בראיות כבימי' כדאמרינן בסמוך ואמרו גלוי מילתא בענמא היא דאחד איש ואשה מטמאין בלובן כיון דמתוקם דם טהור שפיר. ולא מחוור לי דאם לובן טמא משום נדה כ"ש ירוק ודיהה שכולן לא נטהרו אלא משום שאינן אלא לובן.\par ונ"ל דמש"ה איצטריך יתורא דאי משום הא דה' דמי' לא הוה ממעטי' אלא מטומאת נדה ועדיין היינו מטמאי' באשה מדין שכבת זרע או זוב של איש היינו מטמאין לטהרות ולא לבעלה ומיניה ממעט ליה. ומ"מ יפה הרב ז"ל מלמדנו דאי לא ילפי מהדדי מנא תיתי לובן באשה ואודם ודם באיש דאצטריכו קראי למעוטינהו.\par ואמרו בתוספות שעוד שאלו בכל מקום ואוי"ן לרבות וכאן למעט השיב לו כ"ש כיון דלא אצטריכו לרבויא מפרשין להו לקרא דייתר ואיש ואשה לומר אשה דוקא אמרתי ולא איש איש דוק' אמרתי ולא אשה. }
והא דתניא אשה אין לי אלא אשה בת י' ימים לזיבה מנין ת"ל ואשה למה לי ליגמר מנדה 
\commenta{\textbf{פשיטה דהא קא דרס להו.} פי' לאו פשיטא מגופא דמילתא אלא פשיטא דכל דקא דרים להו רחמנא רבינהו למדרס דתניא בת"כ אשר ישב עליו הזב אין לי אלא יושב ומגע מניין לעשרה מושבות זה על גב זה ואפילו על גבי אבן מוסמה ת"ל והיושב על הכלי אשר ישב וכו'. ומשום דמילתה רגילה היא בתלמודא הוא קאמרינן פשיטא דלא ה"ל הכא למיתני אלא שמטמ' מדרס.\par ובפי' עליונו של זב שמעתי דברים רבים והנכון מהם מה שאמרו משם ר"ש ז"ל שהוא דבר הנשא עליו כגון הוא בכף מאזנים ומשכבות ומושבות בכף שניה וכרעו הן טמאין מדרס כרע הזב זהו עליונו של זב ומטמאין אוכלין ומשקין. ואתינן למיבעי מנלן דתניא ובל הנוגע בכל אשר' יהיה תחתיו מאי תחתיו אלימא תחתיו דזב דהיינו משכב ומושב מאיש אשר יגע במשכבו נפקא. ואי קשיא לך אדרבא הוא מטמא בגדים דכתיב ביה יכבס בגדיו והכא ליכא כבוס אה"נ אלא גמרא לא איצטרך למיחת לה כולי האי. וקאמר סתם כל טומאה דמדרס מהתם היא כדכתיבנא ביה ולא מהכא ועוד דאי הוה נקיט טעמא מהך קושיא דכבוס בגדים דילמא הוה אמרינן דכי כתיב והנושא אותם יכבס בגדיו ארישא דקרא נמי קאי ולא בעי עיוליה נפשיה בספיקא דקושיי. }
צריכא דאי כתב רחמנא בנדה הוה אמינא נדה משום דכי חזאי חד יומא בעיא למיתב ז' אבל זבה דאי חזאי חד יומא בשומרת יום כנגד יום סגי לה אימא לא צריכא 
וליכתוב רחמנא בזבה ולא בעי בנדה ואנא ידענא דאין זבה בלא נדה אין ה"נ ואלא קרא למה לי למעוטי איש מאודם 
הא מיעטתיה חדא זימנא חד למעוטי משכבת זרע וחד למעוטי מדם 
וכן לענין זכרים דתניא (ויקרא טו, ב) איש איש מה ת"ל איש איש לרבות תינוק בן יום אחד שהוא מטמא בזיבה דברי רבי יהודה 
רבי ישמעאל בנו של ר' יוחנן בן ברוקה אומר אין צריך הרי הוא אומר (ויקרא טו, לג) לזכר ולנקבה לזכר כל שהוא זכר בין שהוא גדול בין שהוא קטן ולנקבה כל שהיא נקבה בין גדולה בין קטנה א"כ מה ת"ל איש איש דברה תורה כלשון בני אדם 
אלמא כי מרבי קרא בן יום אחד מרבי ורמינהו איש אין לי אלא איש בן תשע שנים ויום אחד מנין ת"ל {ויקרא טו } ואיש 
אמר רבא הלכתא נינהו ואסמכינהו רבנן אקראי הי הלכתא והי קרא אילימא בן יום אחד הלכתא ובן ט' שנים ויום אחד קרא קרא סתמא כתיב 
אלא בן ט' שנים ויום אחד הלכתא ובן יום א' קרא וכי מאחר דהלכתא היא קרא למה לי למעוטי אשה מלובן
למה לי למכתב בזכרים ולמה לי למכתב בנקבות
צריכי דאי כתב רחמנא בזכרים משום דמטמאו בראיות כבימים אבל נקבות דלא מטמאו בראיות כבימים אימא לא 
ואי כתב רחמנא בנקבות משום דקמטמו באונס אבל זכרים דלא מטמאו באונס אימא לא צריכא
הכותים מטמאין משכב תחתון כעליון מאי משכב תחתון כעליון אילימא דאי איכא י' מצעות ויתיב עלייהו מטמו להו פשיטא דהא דרס להו 
אלא שיהא תחתונו של בועל נדה כעליונו של זב מה עליונו של זב אינו מטמא אלא אוכלין ומשקין אף תחתונו של בועל נדה אינו מטמא אלא אוכלין ומשקין 
עליונו של זב מנלן דכתיב (ויקרא טו, י) וכל הנוגע בכל אשר יהיה תחתיו יטמא מאי תחתיו}

\newsection{דף לג}
\twocol{אילימא תחתיו דזב {ויקרא ט״ו:י׳ } מואיש אשר יגע במשכבו נפקא אלא הנוגע בכל אשר יהיה הזב תחתיו ומאי ניהו עליון של זב
\commenta{\textbf{מתקיף לה רמי בר חמא ותספרנו ואנן נמי ניספריה וכו'.} פי' רמי בר חמא טעמא הוה בעי אבל ודאי ליכא דסליק אדעתיה דדינא הכי הני ספרה אנן כדמקשינן בפ' בתרא דמכילתן א"ל רב ששת לרב ירמיה רב ככותאי אמרה לשמעתיה דאמרי' יום שפוסקת בו סופרת למנין ז'.\par ואי קשיא ההיא דגרסינן בפסחים פ' כיצד צולין (דף כא) ר' יוסי אומר שומרת יום כנגד יום ששחטו וזרקו עליה בשני שלה ואח"כ ראתה אינה אוכלת ופטורה מלעשות פסח שני ומפרשינן טעמיה דקסבר מכאן ולהבא מיטמיא דמקצת היום ככולו ובעינן עלה אלא לר' יוסי זבה גמורה היכי משכחת לה בשופעת ואיבעית אימא בגון שראתה שני בין השמשו' אלמא אמרינן מקצת היום ככולו.\par לאו מילתא היא דבסוף מנין אית ליה לר' יוסי מקצת תחלת היום ככולו בין זבה גדולה ובין בקטנה דשני שלה סוף מנין הוא דהא אנן נמי בזבה גדולה קי"ל כר"ש דאמר אחר מעשה תטהר אלא לדידן סותרת בכל היום ולר' יוסי לית ליה סתירה לאחר מקצת יום דהא שלימה היא טהרתה אבל בסוף יום ותחלת מנין דכ"ע לית להו מקצ' היום ככולו אלא לכותאי.\par וראיתי מי שמקשה כאן מאותה שאמרו בפ"ק דר"ה (דף י) אמר רבא ק"ו ומה נדה שאין תחלת היום עולה לה בסופה סוף היום עולה לה בתחלת שנה שיום א, עולה לה בתחלתה. וא"כ לר' יוסי נימא ק"ו ויהיה סוף היום עולה לה בתחלתה. וזה המקשה יכול להקשות כן בזבה גדולה לרבנן (ובין א) בקטנה דליכא בינייהו אלא סתירה ולפי דעתי שאין זו הקושיא דמקצת יום טמא ככולו טמא ומקצת יום טהור סוף היום כתחלתו בין בתחלתה בין בסופה הילכך לענין זיבה ביום נקי ליכא למיספריה אבל לענין נדה אפילו כולו נמי כימא סופרתו כנ"ל.\par ומיהו מקצת היום שעולה בספירה דזבה דוקא ביום אבל לילה אינה עולה לספירה כלל כדאמרינן בפ' בתרא דמכילתן ושוין בטבילות לילה לזבה שאינה טבילה ותנן נמי במס' מגילה דאינה טובלת עד הנץ החמה.\par וההיא דאמרינן מפ' כיצד צולין דמוקי לדר' יוסי לרואה בין השמשות וכן נמי איתא במס' נזיר (דף כז) וגרסי' בה הכי בנוסחי לדר' יוסי מכדי קסבר מקצת היום ככולו זבה גמורה דמייתי קרבן היכי משכח' לה כגון דחזאי פלגיה דיומא אידך פלגא דלמפרע סליק ליה שימור פי' דלמפרע היינו פלגא דיומא בתחלתו שעבר עליה בטהרה ומתרצי איבעית אימא דקא שפעא ג' יומי בהדי הדדי ואיבעית אימא דחזאי תלתא יום סמוך לשקיעת התמה דלא הוה שהות סליק ליה למנינא. ההיא לרוחא דמילתא איתמר דלא בעי לאתויי עלה התם קרא דמגלה דאמרינן כיון דבעי ספירה ספירה ביממא היא ואוקמוה בסוף היום ותחלתו דליכא שהות דספירה בין ראיה לראיה כלל.\par וי"מ דלא בעיא לאוקמי זבה גדולה בלילואתא דוקא משום דקראי ביממא כתיבי דכתיב ימים רבים כל ימי זובה ולקמן בשלהי מכילתין ואימא ביממי תהוי זיבה בלילואתא תהי נדה ובפ"ק דהוריות נמי אמרינן גבי צדוקין דאמר דזבה לא הויא אלא ביממא דכתיב כל ימי זובה הילכך אע"פ דמפקינן מקראי אפילו לילותא לא מפקינן קרא מימים. }
והנושא נמי יטמא ומאי ניהו נישא מ"ט והנשא כתיב 
נתקו הכתוב מטומאה חמורה והביאו לידי טומאה קלה לומר לך שאינו מטמא אלא אוכלין ומשקין 
אימר נתקו הכתוב מטומאה חמורה דלא מטמא אדם לטמא בגדים אבל אדם או בגדים ליטמא אמר קרא יטמא טומאה קלה משמע 
ותחתונו של בועל נדה מנלן דתניא (ויקרא טו, כד) ותהי נדתה עליו
יכול יעלה לרגלה ת"ל יטמא ז' ימים 
ומה ת"ל ותהי נדתה עליו שיכול לא יטמא אדם וכלי חרס ת"ל ותהי נדתה עליו מה היא מטמאה אדם וכלי חרס אף הוא מטמא אדם וכלי חרס 
אי מה היא עושה משכב ומושב לטמא אדם לטמא בגדים אף הוא עושה משכב ומושב לטמא אדם לטמא בגדים ת"ל וכל המשכב אשר ישכב עליו יטמא 
שאין ת"ל וכל המשכב אשר ישכב עליו יטמא ומה ת"ל וכל המשכב אשר וגו' נתקו הכתוב מטומאה חמורה והביאו לידי טומאה קלה לומר לך שאינו מטמא אלא אוכלין ומשקין 
פריך רב אחאי אימא נתקו הכתוב מטומאה חמורה והביאו לטומאה קלה דלא ליטמא אדם לטמויי בגדים אבל אדם ובגדים ליטמא אמר רב אסי יטמא טומאה קלה משמע 
אימא ותהי נדתה עליו כלל וכל המשכב פרט כלל ופרט אין בכלל אלא מה שבפרט משכב ומושב אין מידי אחרינא לא 
אמר אביי יטמא ז' ימים מפסיק הענין הוי כלל ופרט המרוחקין זה מזה וכל כלל ופרט המרוחקין זה מזה אין דנין אותו בכלל ופרט 
רבא אמר לעולם דנין וכל ריבויא הוא 
מתקיף לה רבי יעקב אימא כהיא מה היא לא חלקת בה בין מגעה למשכבה לטמא אדם ולטמא בגדים לחומרא אף הוא לא תחלוק בו בין מגעו למשכבו לטמא אדם ולטמא בגדים לקולא 
אמר רבא עליו להטעינו משמע
מפני שהן בועלי נדות וכו' אטו כולהו בועלי נדות נינהו א"ר יצחק מגדלאה בנשואות שנו
והן יושבות על דם וכו' תניא אר"מ אם הן יושבות על כל דם ודם תקנה גדולה היא להן
אלא שרואות דם אדום ומשלימות אותו לדם ירוק 
דבר אחר יום שפוסקת בו סופרתו למנין שבעה 
מתקיף לה רמי בר חמא ותספרנו ואנן נמי ניספריה דקיימא לן מקצת היום ככולו 
אמר רבא אם כן שכבת זרע דסתר בזיבה היכי משכחת לה והא מקצת היום ככולו 
אי דחזאי בפלגא דיומא ה"נ הכא במאי עסקינן דחזאי סמוך לשקיעת החמה 
וליקום ולימא ליה לקרא כי כתיבא סמוך לשקיעת החמה כתיבא אין על כרחך שבקיה לקרא דאיהו דחיק ומוקי אנפשיה 
בעי רמי בר חמא פולטת שכבת זרע מהו שתסתור בזיבה רואה היתה וסותרת
או דילמא נוגעת היתה ולא סתרה 
\commenta{\textbf{אלמא אספיקא לא שרפינן תרומה.} פי' לאו אכל ספיקא קאמר דהא למסקנא נמי אספיקא ודאי שרפינן אלא ה"ק אלמא אהך ספיקא דעם הארץ ואפילו בכותי לא שרפינן. }
אמר רבא לפום חורפא שבשתא נהי נמי דסתרה כמה תסתור תסתור שבעה דיה כבועלה 
\commenta{\textbf{ורמינהו על ספק בגדי עם הארץ.} פי' שחכמים גזרו עליהם שיהיו זבים לכל דבריהם ובגדיהם יהיו מדרס לפרושין, והא דאמרינן בפרק השוחט מדרסות קאמרת שאני מדרסות גזירה שמא תשב עליהם אשתו נדה אבגדי אוכלי תרומה מדרס לקודש קאמר והיינו נמי דמיטמי' צינורא דידהו מדבריהם משום משקה הזב והזבה.\par ואי קשיא לך האי דאמרינן בפרק הניזקין (דף סא ע"ב) וליחוש שמא תסטנו אשתו נדה ולא חיישי' להיסט שלו, ועוד אמרו שם גבי חלה מניחה בכפישה או באנחותא וכשיבא עם הארץ ליטול נוטל את שתיהן ואינו חושש משום דלא נגע בהו ולא מטמאין בפשוטי כלי עץ ולא חשש להסיטו וכן נמי בפרק אין דורשין משמע גבי חמרין ופועלין שהן טוענין טהרות דלא מטמאין משום הסיטן.\par ותירץ ר"ת ז"ל שלא גזרו על עמי הארץ היםט שא"כ אין לך אדם מעביר לחבירו חבית ממקום למקום.\par ועוד הביאו ראיה ממשנת מסכת טהרות שאין עמי הארץ מטמאין בהסיטו ולא עושין נמי משכב ומושב דתנן בפרק קמא דטהרות הגנבים שנכנסו לבית אין טמא אלא מקום רגלי הגנבים ומה הן מטמאין אוכלין ומשקין וכלי חרס הפתוחין אבל משכבות ומושבות וכלי חרס המוקפין צמיד פתיל טהורי' ואם יש עמהם נכרי או אשה הכל טמא.\par ויש לי לדחות דהתם כיון דלא נגעו ודאי הקלו באלו שטומאתן רחוקה וספק.\par ועוד הביא ראיה מתוך שמעתן גופה שאין ע"ה מטמאין משכב ומושב דתנן משכב התחתון כעליון ואם היו ע"ה עושין משכב ומושב מטמאין הן התחתון כתחתונו של זב שהרי עשאום כזבים לכל דבריהם.\par ואע"פ שיש לי לפקפק אף בראיה זו נקבל אותם מפני שלא הזהירו חכמים על ע"ה שיהיו כזבים ממש אלמא דין חדש יש להם אבל מ"מ תמה הוא אם גזרו עליהם שיהיו כזבים למקצת היאך הטילו עליהם למחצה וטיהרו משכבות והיסט א"כ הקלת בשל תורה.\par וי"מ שאין ע"ה טמא טומאת עצמן כלל אלא חשש שמא נגעו בנשותיהן ובמדרסן שהן אבות הטומאה והוא נעשה ראשון.\par והא דמטמאינן צינורא דע"ה בשמעתין וכן במסכת חגיגה לאו משום משקה הזב והזבה אלא כיון שהחזיקו אותם חכמים בטמאי' משום משא מדרס טמאו המשקין בשפתי' דכלים מטמאין משקים מדבריהם בפ"ק דשבת אבל לא שיהיו כזבים מדבריהם ובגדיהם שהן מדרס לפרושין נמי משום תשש מדרס אשתו נדה הוא והא דא"ל מ"ט לא תשני ליה בכותי שטבל ועלה ה"ג לה שטבל ועלה (ואכל) [ונגע] בתרומה וכן כתבו בתוספות בנוסחאות ולא כגרסת רש"י שהוא גורס ודרס על בגדי חבר ונגעו בתרומה שאפילו לא טבל אין לו מדרס ואפילו נגע בהן ממש ואצ"ל בעשר מצעות שהרי לא עשיתו אלא ראשון משום טומאת ע"ה דהיא משום נושא מדרס ואינו מטמא כלים לאחר שפירש מן המדרס כלל.\par ואי קשיא ולימא ליה מאי ואין שורפין עליה את התרומה על התחתונות דעליונו קתני דאי משום טומאת ע"ה לא מטמא משכב ואי משום נדה תרי ספיקי נינהו, איכא למימר מפני שטומאתן ספק אפילו אגופייהו משמע ליה.\par תו קשיא לי ולימא ליה ברגל דטומאת ע"ה ברגל כטהרה שוויה רבנן מאי איכא בכותי משום בועל נדה תרי ספיקי נינהו, ול"ק דבשלמא ע"ה שוויה בטהרה שלא להרחיקן אבל בכותי לאו חברים קרינן ביה ואין זה דומה לצדוקי שהם בכלל ישראל הם. }
תסתור יום אחד (ויקרא טו, כח) ואחר תטהר אמר רחמנא אחר אחר לכולן שלא תהא טומאה מפסקת ביניהם 
\commenta{והא דמפרקינן ב\textbf{כותי ערום} ולא אמרינן בשנגע ביד וברגל בלא בגד מפני ששנינו הנוגע במשכב ובמושב מטמא שנים ופוסל אחד פירש מטמא אחד ופוסל אחד בפרק בתרא דזבים והילכך מטמא את התרומה אלא בכותי ערום קודם שיגע בבגדיו עסקינן,\par ובמסכת חגיגה מצאתי בירושלמי סוגיא גדולה בענין זה ובתוס' נמי הזכירוה ומשמע מינה שלא גזרו על עם הארץ שיהיו כזבין וכך הסוגיא שם על מתניתין בגדי עם הארץ מדרס לפרושין וכו'. }
וליטעמיך זב גופיה היכי סתר לטהרתו אמר רחמנא שלא תהא טומאה מפסקת ביניהן 
\commenta{\textbf{מתני רבי יוסי בשם רבי יוחנן במגעות שנינו,} פירוש אלו השנויין כאן אינן עושין מדרס בלא נגיעה אלא שאם נגעו בבגד עשאוהו כמדרס מדבריהם, רבי זעירא בעי קומי רבי יוסי מהיכן נטמא הבגד הזה מדרס א"ל תפתר שהיתה אשתו של עם הארץ יושבת עליה ערומה, פירוש ר' זעירא מקשי על רבי יוסי לדבריך מהיכן נטמא מדרס לא היה לנו לטמאן אלא מגע הזב פריק שאם נגעה בו אשתו בישיבה עשאוהו כמדרס אבל ישבה עליו בבגדיה לא גזרו עליו, שמואל בר בא בעי קומי ר' זעירא כמה דתימר תמן אין היסט בחולין ויש היסט בחולין על ידי מגע ודכותה אין משא בחולין ויש משא בחולין על ידי מגע וכו' גופו של פרוש מהו שיעשה כזב אצל תרומה מתיב ר' תנן והתנן המניח עם הארץ בתוך ביתו בזמן שהוא רואה את הנכנסים ואת היוצאים האוכלין והמשקין וכלי חרס הפתוחין טמאין אבל המשכבות ומושבות וכלי חרס מוקפין צמיד פתיל טהורים אין תימר עשו גופו כזב אצל תרומה אפילו מוקפין צמיד פתיל יהיו טמאין אמר רב ר' יהודה בר פזי תפתר בעם הארץ אצל הפרוש לא עשאוהו כזב אלא שגזרו על בגדיו מדרס במגע אשתו כדאמרן ואקשי' אמר ר' מונא כן אמר ר' יוסי רבי כל מה דאנן קיימין הכא בתרומה אנן קיימן תדע לך שהוא כן דתנינן אפילו מובל ואפילו כפות הכל טמא כלום אמרו יהו הן טמאין אלא משום היסט לא כן אמר ר' יוחנן לאו חציצות ולא הסיטו ולא רשות היחיד ולא רשות עם הארץ אצל תרומה, ע"כ גמר'.\par וה"פ דקא מקשי ליה רבי מונא לר"י בן פזי דאוקמא למתני' בחולין דודאי בתרומה קיימי' מדקתני סיפא הכל טמא ואי בחולין מדרסות והיסטות טהורין הן דאמר רבי יוחנן שלא אמרו שיהא דבר חוצץ במדרסות ולא טהרו היסטות ולא חלקו בספק רשות היחיד ולא רשות עם הארץ אצל תרומה הא אצל חולין הכל טהורין.\par וברייתא היא אצל זו ששנויה בתוספתא דחגיגה דקתני ספק רשות עם הארץ מדרסו וחצירו והיסטו טהורין לחולין וטמאין לתרומה, אלמא מתניתין דקתני הכל טמא בתרומה היא ושמע מינה שלא עשו גופו של פרוש ולא של עם הארץ כזב לטמא משכבות ומושבות והיסט אלא שחששו בזמן שאינו רואה את הנכנסים לאשה או לכותי לתרומה ולחולין הכל טהור ואפילו ספק רשותו עד שיתברר לך שנכנסה אשתו לשם אי נמי בגדים שלו שאי אפשר שלא נגעה בהם אשתו במדרס, ע"כ הארכתי לכתוב מן התוספת והן מגיהין ב) ולא רשות עם הארץ לחולין אלא אצל תרומה ודבריהם הללו כולן כתבתים מפני שדברים ברורים הם וצריכין הן לכמה סוגיות שבגמרא. }
אלא מאי אית לך למימר שלא תהא טומאת זיבה מפסקת ביניהן הכא נמי שלא תהא טומאת זיבה מפסקת ביניהן
ואין חייבין עליהן על ביאת מקדש וכו' רב פפא איקלע לתואך אמר אי איכא צורבא מרבנן הכא איזיל אקבל אפיה אמרה ליה ההיא סבתא איכא הכא צורבא מרבנן ורב שמואל שמיה ותני מתניתא יהא רעוא דתהוי כוותיה 
אמר מדקמברכי לי בגוויה ש"מ ירא שמים הוא אזל לגביה רמא ליה תורא רמא ליה מתני' אהדדי תנן אין חייבין עליהן על ביאת מקדש ואין שורפין עליהן את התרומה מפני שטומאתה ספק אלמא מספיקא לא שרפינן תרומה
ורמינהי על ששה ספקות שורפין את התרומה על ספק בגדי עם הארץ 
אמר רב פפא יהא רעוא דלתאכיל האי תורא לשלמא הכא במאי עסקינן בכותי חבר 
כותי חבר בועל נדה משוית ליה 
שבקיה ואתא לקמיה דרב שימי בר אשי אמר ליה מאי טעמא לא משנית ליה בכותי שטבל ועלה ודרס על בגדי חבר ואזלו בגדי חבר ונגעו בתרומה 
דאי משום טומאת עם הארץ הא טביל ליה ואי משום בועל נדה ספק בעל בקרוב ספק לא בעל בקרוב 
ואם תמצי לומר בעל בקרוב ספק השלימתו ירוק ספק לא השלימתו והוי ספק ספיקא ואספק ספיקא לא שרפינן תרומה 
ותיפוק ליה משום בגדי עם הארץ דאמר מר בגדי עם הארץ מדרס לפרושין אמר ליה בכותי ערום
{\large\emph{מתני׳}} בנות צדוקין בזמן שנהגו ללכת בדרכי אבותיהן הרי הן ככותיות פרשו ללכת בדרכי ישראל הרי הן כישראלית רבי יוסי אומר לעולם הן כישראלית עד שיפרשו ללכת בדרכי אבותיהן
{\large\emph{גמ׳}} איבעיא להו סתמא מאי ת"ש בנות צדוקין בזמן שנוהגות ללכת בדרכי אבותיהן הרי הן ככותיות הא סתמא כישראלית אימא סיפא פרשו ללכת בדרכי ישראל הרי הן כישראלית הא סתמא ככותיות אלא מהא ליכא למשמע מיניה 
ת"ש דתנן ר' יוסי אומר לעולם הן כישראלית עד שיפרשו ללכת בדרכי אבותיהן מכלל דת"ק סבר סתמא ככותיות ש"מ
תנו רבנן מעשה בצדוקי אחד שספר עם כהן גדול בשוק ונתזה צנורא מפיו ונפלה לכהן גדול על בגדיו והוריקו פניו של כהן גדול וקדם אצל אשתו 
אמרה לו אף על פי שנשי צדוקים הן מתיראות מן הפרושים ומראות דם לחכמים 
אמר רבי יוסי בקיאין אנו בהן יותר מן הכל והן מראות דם לחכמים חוץ מאשה אחת שהיתה בשכונתינו שלא הראת דם לחכמים ומתה 
ותיפוק ליה משום צנורא דעם הארץ אמר אביי בצדוקי חבר אמר רבא צדוקי חבר בועל נדה משוית ליה אלא אמר רבא}

\newsection{דף לד}
\twocol{רגל הוה וטומאת עם הארץ ברגל כטהרה שוינהו רבנן דכתיב (שופטים כ, יא) ויאסף כל איש ישראל אל העיר כאיש אחד חברים הכתוב עשאן כולן חברים
{\large\emph{מתני׳}} דם עובדת כוכבים ודם טהרה של מצורעת ב"ש מטהרים ובית הלל אומרים כרוקה וכמימי רגליה 
דם היולדת שלא טבלה ב"ש אומרים כרוקה וכמימי רגליה וב"ה אומרים מטמא לח ויבש 
ומודים ביולדת בזוב שהיא מטמאה לח ויבש
{\large\emph{גמ׳}} ולית להו לב"ש (ויקרא טו, ב) דבר אל בני ישראל ואמרת אליהם איש איש כי יהיה זב בני ישראל מטמאין בזיבה ואין העובדי כוכבים מטמאין בזיבה אבל גזרו עליהן שיהו כזבין לכל דבריהם
אמרי לך ב"ש (ההוא בזכרים איתמר דאי בנקבות) היכי לעביד ליטמא לח ויבש עשיתו כשל תורה ליטמי לח ולא ליטמי יבש חלקת בשל תורה 
אי הכי רוקה ומימי רגליה נמי כיון דעבדינן היכרא בדמה מידע ידיע דרוקה ומימי רגליה דרבנן 
ולעביד היכרא ברוקה ומימי רגליה ולטמויי לדמה רוקה ומימי רגליה דשכיחי גזרו בהו רבנן דמה דלא שכיחא לא גזרו ביה רבנן 
אמר רבא זובו טמא אפילו לב"ש קריו טהור אפילו לב"ה 
זובו טמא אפילו לב"ש דהא איכא למעבד היכרא בקריו
קריו טהור אפי' לב"ה עבוד ביה רבנן היכרא כי היכי דלא לשרוף עליה תרומה וקדשים 
ולעביד היכרא בזובו ולטמויי לקריו זובו דלא תלי במעשה גזרו ביה רבנן קריו דתלי במעשה לא גזרו ביה רבנן 
לימא מסייע ליה עובדת כוכבים שפלטה שכבת זרע מישראל טמאה ובת ישראל שפלטה שכבת זרע מן העובד כוכבי' טהורה מאי לאו טהורה גמורה לא טהורה מדאורייתא טמאה מדרבנן 
ת"ש נמצאת אומר שכבת זרע של ישראל טמאה בכל מקום
ואפי' במעי עובדת כוכבים ושל עובד כוכבים טהור' בכל מקום ואפי' במעי ישראלית חוץ ממי רגלים שבה 
\commenta{גרסת הספרים כך היא וכן בפירושי ר"ח ז"ל: \textbf{תא שמע זובו טמא לימד על הזוב שהוא טמא במאי אלימא בזב גרידא לאחרים גורם טומאה לעצמו לא כל שכן אלא פשיטא בזב ומצורע ומדאיצטריך לרבויי בראיה ראשונה שמע מינה מקום זיבה לאו מעין הוא.} ובודאי יפה פירש רש"י ז"ל שראיה ראשונה של אדם אחר אינו מטמא במשא אלא במגע (בקרי) [כקרי] וראיה שנייה מטמא אפילו במשא לקמן בפרק דם הנדה, והא דקאמר לאחרים גורם טומאה הכא קאמר לאחרים גורם שיהיו מטמאין במשא לעצמו לא כל שכן שיטמא במשא ומדין משא למשא פריך דאי גרס טומאה בעלמא קאמר אף בזב מצורע גורס טומאה דמשכב ומושב לטמא אדם ולטמא בגדים ולטמא נמי בהיסט שאין מצורע עושה כן אלא מדין משא גופיה פריך כדפרישית.\par ומיהו צריכין אנו לישב גרסת הספרים, ור"ש אומר פירוש שהיא בספרים והכי קאמר ומדאיצטריך לרבויי בשנייה שמע מינה דבראשונה לא מטמא במשא דלאו מעין הוא ואין וה הלשון גמרא.\par אבל יש לפרש שכך היא הצעה זו דאמר רבא תא שמע זובו טמא לימד על הזוב שהוא טמא במאי אלימא בזב גרידא ובראיה שניה דאלו בראשונה ולמגע ודאי לא צריכא קרא דלא גרע משכבת זרע ולא עדיף מיניה אלא פשיטא בשנייה ואכתי למה לי קרא לאחרים גורס טומאה ואפילו למשא עצמו לא כ"ש אלא פשיטא בזב ומצורע ואי בשניה מי גרע מזב גרידא אלא בראשונה ולטמויי במשא וש"מ תרתי ש"מ ראיה ראשונה של מצורע מטמא במשא וש"מ לאו משום דמעיין הוא כדרב יוסף דאי הכי לא איצטרך רחמנא לרבויי הכא דממעינות נפקא אלא דרחמנא רבייה לראיה ראשונה של מצורע כראיה שניה של זב גרידא.\par ואי קשיא נימא קרא לראיה שניה והא קמ"ל דוקא בשניה אבל בראשונה אינה מטמא דלאו מעיין הוא, זו אינה תורה דמי איכא ספיקא קמי שמיא במקום זיבה אי מעין הוא או לא ואיצטרך ליתורי קרא למיגמר מיניה דלאו מעין הוא, ועוד דכל היכא דקרא מרבה כגון זה דכתיב זובו טמא דרשינן ליה לרבויי כגון לרבו' ראי' ראשונה למשא ולא מוקמינן ליה ליתורא למימר בשניה כתיב ולמעוטי ראשונה איצטרך כנ"ל.\par ומה שהקשה רש"י ז"ל מי איכא לאוקומי להאי קרא בראיה ראשונה והא מהכא נפקא לן בכל דוכתא מנה הכתוב שתים וקרא טמא, אינה קושיא דהאמרינן בפרק יוצא דופן דלמאן דאית ליה מנה הכתוב שתים וקרא טמא לית ליה זובו טמא לימד על הזוב שיהא טמא ותנאי היא.\par וכן זה שאמר הרב ז"ל דגבי מצורע איצטרך לרבוייה לטיפ' עצמה דלא אתי בק"ו משום דלא גרמה ליה טומא' שהרי מחמת נגעו הוא מטמא אין זה מחוור דכיון דאי לאו מצורע הוא נמי הות מטמי' איתא לק"ו מ"מ וכ"ש דאיכא לפרושי גרס טומאה בהיסט ומדרסות כדאמרן לעיל ומהסט למשא גמרינן ודאי דחד אורח הוא למשאות, ובמסקנא פשט אביי דמטמא במשא דהא אקשייה רחמנא למצורע אזב גמור, ולא פשט במעיין כלום משום דלא מרבוייא דקרא יתירא אתי דנימא למאי איצטרך אלא דמ"מ מטמא במשא הוא. }
וכי תימא ה"נ טהור' מדאוריית' אבל טמאה מדרבנן אטו מי רגליה מדאורייתא מי מטמאו אלא ש"מ טהורה אפילו מדרבנן ש"מ
אמר מר שכבת זרע של ישראל טמאה בכ"מ אפי' במעי עובדת כוכבים תפשוט דבעי רב פפא דבעי רב פפא שכבת זרע של ישראל במעי עובדת כוכבים מהו 
בתוך ג' לא קמיבעיא ליה לרב פפא כי קמיבעיא ליה לאחר ג' מאי 
ישראל דדייגי במצות חביל גופייהו ומסריח עובדי כוכבים דלא דייגי במצות לא חביל גופייהו ולא מסריח או דילמא כיון דאכלי שקצים ורמשים חביל גופייהו ומסריח תיקו
דם טהרה של מצורעת ב"ש כו' מאי טעמא דב"ה אמר ר' יצחק לזכר לרבות מצורע למעינותיו ולנקבה לרבות מצורעת למעינותיה 
מאי מעינותיה אילימא שאר מעינותיה מזכר נפקא אלא לדמה לטמא דם טהרה שלה 
וב"ש נקבה מזכר לא אתיא דאיכא למיפרך מה לזכר שכן טעון פריעה ופרימה ואסור בתשמיש המטה תאמר בנקבה דלא 
וב"ה לכתוב רחמנא בנקבה ולא בעי זכר ואנא אמינא ומה נקבה שאינה טעונה פריעה ופרימה ואינה אסורה בתשמיש המטה רבי רחמנא מעינותיה זכר לא כ"ש 
אם אינו ענין לזכר תנהו ענין לנקבה ואם אינו ענין למעינותיה תנהו ענין לדמה לטמא דם טהרה שלה 
וב"ש זכר מנקבה לא אתיא דאיכא למיפרך מה לנקבה שכן מטמאה מאונס תאמר בזכר דלא 
וב"ה קיימי במצורע ופרכי מילי דזב וב"ש שום טומאה פרכי 
ואיבעית אימא אמרי לך ב"ש האי לזכר מיבעי ליה לזכר כל שהוא זכר (האי) בין גדול בין קטן ובית הלל נפקא להו מזאת תורת הזב בין גדול בין קטן 
אמר רב יוסף כי פשיט רבי שמעון בן לקיש בזב בעי הכי ראייה ראשונה של זב קטן מהו שתטמא במגע (ויקרא טו, לב) זאת תורת הזב ואשר תצא ממנו שכבת זרע אמר רחמנא
כל ששכבת זרע שלו מטמא ראייה ראשונה שלו מטמאה והאי כיון דשכבת זרע שלו לא מטמאה ראייה ראשונה נמי לא תטמא או דילמא כיון דאילו איהו חזי תרתי מצטרפא מטמיא 
אמר רבא ת"ש זאת תורת הזב בין גדול בין קטן מה גדול ראייה ראשונה שלו מטמא אף קטן ראייה ראשונה נמי מטמא 
בעי רב יוסף ראייה ראשונה של מצורע מהו שתטמא במשא מקום זיבה מעין הוא ומטמא או דילמא לאו מעין הוא 
אמר רבא ת"ש (ויקרא טו, ב) זובו טמא הוא לימד על הזוב שהוא טמא במאי אילימא בזב גרידא}

\newsection{דף לה}
\twocol{לאחרים גורם טומאה לעצמו לא כל שכן אלא פשיטא בזב מצורע
ומדאיצטריך קרא לרבויי בראייה שניה שמע מינה מקום זיבה לאו מעין הוא 
אמר ליה רב יהודה מדסקרתא לרבא ממאי דילמא לעולם אימא לך בזב גרידא ודקאמרת לאחרים גורם טומאה לעצמו לא כל שכן שעיר המשתלח יוכיח שגורם טומאה לאחרים והוא עצמו טהור 
אמר אביי מאי תבעי ליה והא הוא דאמר זאת תורת הזב בין גדול בין קטן וכיון דנפקא ליה מהתם אייתר ליה לזכר לרבות מצורע למעינותיו נקבה לרבות מצורעת למעינותיה
ואקשיה רחמנא מצורע לזב גמור מה זב גמור מטמא במשא אף ראייה ראשונה של מצורע מטמא במשא 
א"ר הונא ראייה ראשונה של זב מטמאה באונס שנאמר (ויקרא טו, לב) זאת תורת הזב ואשר תצא ממנו שכבת זרע מה שכבת זרע מטמא באונס אף ראייה ראשונה של זב מטמאה באונס 
תא שמע ראה ראייה ראשונה בודקין אותו מאי לאו לטומאה לא לקרבן 
ת"ש בשניה בודקין אותו למאי אילימא לקרבן אבל לטומאה לא אקרי כאן מבשרו ולא מחמת אונסו אלא לאו לטומאה ומדסיפא לטומאה רישא נמי לטומאה 
מידי איריא הא כדאיתא והא כדאיתא 
תא שמע רבי אליעזר אומר אף בשלישי בודקין אותו מפני הקרבן מכלל דתנא קמא מפני הטומאה קאמר 
לא דכולי עלמא לקרבן והכא באתים קא מיפלגי רבנן לא דרשי אתים ורבי אליעזר דריש אתים 
רבנן לא דרשי אתים הזב חדא זובו תרתי לזכר בשלישי אקשיה רחמנא לנקבה 
ורבי אליעזר דריש אתים הזב חדא את תרתי זובו תלת ברביעי אקשיה רחמנא לנקבה 
תא שמע רבי יצחק אומר והלא זב בכלל בעל קרי היה ולמה יצא להקל עליו ולהחמיר עליו להקל עליו שאין מטמא באונס ולהחמיר עליו
שהוא עושה משכב ומושב 
\commenta{\textbf{בשלמא לרב דאמר מעין אח' הוא מ"ה מטמא לח ויבש.} פי' לבית הלל פשיטא ולבית שמאי נמי כיון שראיה זו מטמאתה מלספור נקיים נמצא שהיא גורמת טומאה וכדם הנדה הוא שמטמא לח ויבש ולא דמי לרואה בתוך ימי טוהר בלא זוב שאין ראייתה כלום אלא ללוי אמאי מטמא לח ויבש בין לבית שמאי בין לבית הלל, ופריק בשופעת.\par אי בשופעת למאי איצטרך וק"ל ולרב גופיה אמאי איצטרך ודאי לבית שמאי ללוי נמי לבית שמאי ואיכא למימר בשלמא לרב טעמיה לבית שמאי קמשמע לן דלא תימא טעמייהו משום דשני מעיינות הן ובהא פליגי קמשמע לן ומודים ואי שני מעיינות הן ביולדת בזוב נמי מטהרו בית שמאי אלא ללוי אמאי איצטרך האי טעמא בתרווייהו מכל מקום שמע מינה דהא לית ליה לאוקמינהו אלא בהך פלוגתא ופריק אפילו הכי איכא למיטעי בה לבית שמאי דסד"א אף על פי שופעות לא תטמא קמשמע לן.\par כיון דמפורש בשמעתין דלרב ימי טוהר שרואה בהן אין עולין לה לספירת זיבה, וקיימא לן בנות ישראל החמירו על עצמן שאפילו רואות טיפת דם כחרדל יושבות עליה שבעה נקיים וקיימא לן אי אפשר לפתיחת הקבר בלא דם אם כן היולדות צריכות שבעה נקיים בתוך ימי טוהר שלהן אלא שימי לידה נמי אם אינה רואה בהן עולין לספירת זיבתה כדלקמן וכן כחב הרמב"ם ז"ל שהיולדת בזמן הזה הרי היא כיולדת בזוב וצריכה שבעה נקיים. }
אימת אילימא בראייה שניה היכא הוה בכלל בעל קרי אלא פשיטא בראייה ראשונה וקתני להקל עליו שאינו מטמא באונס 
ותסברא להחמיר עליו שהוא עושה משכב ומושב בראייה ראשונה בר משכב ומושב הוא 
אלא הכי קאמר רבי יצחק אומר והלא זב בכלל בעל קרי היה בראייה ראשונה ולמה יצא בראייה שנייה להקל עליו ולהחמיר עליו להקל עליו שאינו מטמא באונס ולהחמיר עליו שהוא עושה משכב ומושב 
אמר רב הונא זוב דומה למי בצק של שעורים זוב בא מבשר המת שכבת זרע בא מבשר החי זוב דיהה ודומה ללובן ביצה המוזרת שכבת זרע קשורה ודומה ללובן ביצה שאינה מוזרת
דם היולדת שלא טבלה וכו'
תניא אמרו להן בית הלל לבית שמאי אי אתם מודים בנדה שלא טבלה וראתה דם שהיא טמאה אמרו להם בית שמאי לא אם אמרתם בנדה שאפילו טבלה וראתה טמאה תאמרו ביולדת שאם טבלה וראתה שהיא טהורה 
אמרו להם יולדת בזוב תוכיח שאם טבלה וראתה לאחר ימי ספירה טהורה לא טבלה וראתה טמאה 
אמרו להם הוא הדין והיא התשובה 
למימרא דפליגי והתנן ומודים ביולדת בזוב שהיא מטמאה לח ויבש 
לא קשיא כאן שספרה כאן שלא ספרה 
והתניא יולדת בזוב שספרה ולא טבלה וראתה הלכו בית שמאי לשיטתן וב"ה לשיטתן 
איתמר רב אמר מעין אחד הוא התורה טמאתו והתורה טהרתו 
ולוי אמר שני מעינות הם נסתם הטמא נפתח הטהור נסתם הטהור נפתח הטמא 
מאי בינייהו איכא בינייהו שופעת מתוך שבעה לאחר שבעה ומתוך ארבעה עשר לאחר ארבעה עשר ומתוך ארבעים לאחר ארבעים ומתוך שמנים לאחר שמנים
לרב רישא לקולא וסיפא לחומרא
ללוי רישא לחומרא וסיפא לקולא 
מיתיבי דם היולדת שלא טבלה בית שמאי אומרים כרוקה וכמימי רגליה וב"ה אומרים מטמא לח ויבש 
קא ס"ד דפסקה בשלמא לרב דאמר מעין אחד הוא משום הכי מטמא לח ויבש אלא ללוי דאמר שני מעינות הן אמאי מטמא לח ויבש 
אמר לך לוי הכא במאי עסקינן בשופעת אי בשופעת מ"ט דב"ש קסברי ב"ש מעין אחד הוא 
בשלמא ללוי היינו דאיכא בין ב"ש וב"ה אלא לרב מאי בינייהו 
איכא בינייהו יומי וטבילה דבית שמאי סברי ביומי תלה רחמנא וב"ה סברי ביומי וטבילה 
ת"ש ומודים ביולדת בזוב שהיא מטמאה לח ויבש ס"ד הכא נמי דפסקה
בשלמא לרב דאמר מעין אחד הוא משום הכי מטמא לח ויבש אלא ללוי דאמר שני מעינות הן אמאי מטמא לח ויבש 
אמר לך הכא נמי בשופעת אי בשופעת למאי איצטריך 
לב"ש איצטריך אף על גב דקאמרי בית שמאי מעין אחד הוא וביומי תלה רחמנא הני מילי ביולדת גרידתא דשלימו להו יומי אבל יולדת בזוב דבעי ספירה לא 
תא שמע (ויקרא יב, ב) דותה תטמא לרבות את בועלה
דותה תטמא לרבות הלילות דותה תטמא לרבות היולדת בזוב שצריכה שתשב שבעה ימים נקיים בשלמא לרב דאמר מעין אחד הוא משום הכי בעיא שבעה ימים נקיים}

\newsection{דף לו}
\twocol{אלא ללוי דאמר שני מעינות הן למה לי שבעה במשהו סגיא 
הכי קאמר צריכה שתפסוק משהו שיעלו לה לשבעה נקיים 
ת"ש ימי עיבורה עולים לה לימי מניקותה וימי מניקותה עולים לה לימי עיבורה 
כיצד הפסיקה שתים בימי עיבורה ואחת בימי מניקותה שתים בימי מניקותה ואחת בימי עיבורה אחת ומחצה בימי עיבורה ואחת ומחצה בימי מניקותה עולין לה לג' עונות 
בשלמא לרב דאמר מעין אחד הוא משום הכי בעי הפסק שלש עונות אלא ללוי דאמר שני מעינות הן למה לי הפסק שלש עונות במשהו סגי 
הכי קאמר צריכה שתפסוק משהו כדי שיעלו לה לשלש עונות 
ת"ש ושוין ברואה אחר דם טוהר שדיה שעתה 
בשלמא ללוי דאמר שני מעינות הן משום הכי דיה שעתה אלא לרב דאמר מעין אחד הוא אמאי דיה שעתה תטמא מעת לעת 
דליכא שהות 
ותטמא מפקידה לפקידה כיון דמעת לעת ליכא מפקידה לפקידה נמי לא גזרו בה רבנן 
תא שמע יולדת בזוב שספרה ולא טבלה וראתה הלכו ב"ש לשיטתן ובית הלל לשיטתן 
בשלמא לרב דאמר מעין אחד הוא משום הכי מטמא לח ויבש אלא ללוי דאמר שני מעינות הן אמאי מטמא לח ויבש 
אמר לך לוי אנא דאמרי כתנא דשוין 
ואיבעית אימא בשופעת והא ספרה קתני 
הכא ביולדת נקבה בזוב עסקינן דשבוע קמא פסקה שבוע בתרא לא פסקה וקסבר ימי לידתה שאין רואה בהן עולין לה לספירת זיבתה 
אמר ליה רבינא לרב אשי אמר לן רב שמן מסכרא אקלע מר זוטרא לאתרין ודרש הילכתא כוותיה דרב לחומרא והלכתא כוותיה דלוי לחומרא 
רב אשי אמר הלכתא כוותיה דרב בין לקולא בין לחומרא דריש מרימר הלכתא כוותיה דרב בין לקולא בין לחומרא והלכתא כוותיה דרב בין לקולא בין לחומרא
{\large\emph{מתני׳}} המקשה נדה קשתה שלשה ימים בתוך י"א יום ושפתה מעת לעת וילדה הרי זו יולדת בזוב דברי רבי אליעזר 
\commenta{\textbf{כל שחל קישויה להיות בג' שלה וכו'.} פירש רש"י ז"ל כל שקשתה אפילו שעה א' בליל כניסת ג' אפילו כל היום כולו בשופי ושעה א' מליל ד' להשלמה מעת לעת וילדה אין זו יולדת בזוב דבעינן שופי כל יום ג' המביא לידי זיבה.\par פי' לפי' לאו משום (דכפי) [דבעי] חנניא לילה ויום כלילי שבת ויומו דאם כן היינו דר' יהושע אלא משום דבעי כל יום ג' בשופי ואם קשתה בג' אפילו שפת בד' כלילי שבת ויומו אינה זבה שאין קושי שבג' קובע אותה זבה ובכה"ג לא הוה זבה עד דחזיא ג' אח"כ בשופי דקושי שבשלישי אינו קובע בזיבהולא מצטרף (עד) [עם] יום ד' לקבעה בזיבה והיינו דאמרי לקמן לעולם כדקתני והא קמשמע לן אף ע"ג דאתחיל קושי בג' אם שפתה מעת לעת טמאה לאפוקי מדחנניא בן אחי ר' יהושע ואלו לאפוקי מדר' יהושע בהדיא קתני לה אם שפתה מעת לעת ר' יהושע אומר כלילי שבת ויומו. }
רבי יהושע אומר לילה ויום כלילי שבת ויומו ששפתה מן הצער ולא מן הדם 
כמה היא קישויה ר' מאיר אומר אפילו ארבעים וחמשים יום רבי יהודה אומר דיה חדשה ר' יוסי ור' שמעון אומרים אין קישוי יותר משתי שבתות
{\large\emph{גמ׳}} אטו כל המקשה נדה היא
אמר רב נדה ליומא ושמואל אמר חיישינן שמא תשפה 
ור' יצחק אמר המקשה אינה כלום והקתני המקשה נדה 
אמר רבא בימי נדה נדה בימי זיבה טהורה והתניא המקשה בימי נדה נדה בימי זיבה טהורה 
כיצד קשתה יום אחד ושפתה שנים או שקשתה שנים ושפתה יום אחד או ששפתה וקשתה וחזרה ושפתה הרי זו יולדת בזוב 
אבל שפתה יום אחד וקשתה שנים או ששפתה שנים וקשתה יום אחד או שקשתה ושפתה וחזרה וקשתה אין זו יולדת בזוב כללו של דבר קושי סמוך ללידה אין זו יולדת בזוב שופי סמוך ללידה הרי זו יולדת בזוב 
חנניא בן אחי ר' יהושע אומר כל שחל קישויה בשלישי שלה אפילו כל היום כולו בשופי אין זו יולדת בזוב 
כללו של דבר לאתויי מאי לאתויי דחנניא 
מה"מ דת"ר דמה דמה מחמת עצמה ולא מחמת ולד 
אתה אומר מחמת ולד או אינו אלא מחמת אונס כשהוא אומר (ויקרא טו, כה) ואשה כי יזוב זוב דמה הרי אונס אמור הא מה אני מקיים דמה דמה מחמת עצמה ולא מחמת ולד 
ומה ראית לטהר את הולד ולטמא באונס מטהר אני בולד שיש טהרה אחריו ומטמא אני באונס שאין טהרה אחריו 
אדרבה מטהר אני באונס שכן אונס בזב טהור השתא מיהא באשה קיימינן ואונס באשה לא אשכחן 
ואיבעית אימא מאי דעתיך לטהורי באונס ולטמויי בולד אין לך אונס גדול מזה 
אי הכי נדה נמי נימא זובה זובה מחמת עצמה ולא מחמת ולד 
אתה אומר ולד או אינו אלא אונס כשהוא אומר (ויקרא טו, יט) ואשה כי תהיה זבה הרי אונס אמור הא מה אני מקיים זובה זובה מחמת עצמה ולא מחמת ולד 
אמר ר"ל אמר קרא תשב יש לך ישיבה אחרת שהיא כזו ואיזו זו זו קושי בימי זיבה ואימא זו קושי בימי נדה 
אלא אמר אבוה דשמואל אמר קרא (ויקרא יב, ה) וטמאה שבועים כנדתה ולא כזיבתה מכלל דזיבתה טהור ואיזו זו זו קושי בימי זיבה 
והשתא דכתיב וטמאה שבועים כנדתה דמה למה לי אי לאו דמה הוה אמינא כנדתה ולא כזיבתה ואפילו בשופי קמ"ל 
שילא בר אבינא עבד עובדא כוותיה דרב כי קא נח נפשיה דרב א"ל לרב אסי זיל צנעיה ואי לא ציית גרייה הוא סבר גדייה א"ל 
בתר דנח נפשיה דרב א"ל הדר בך דהדר ביה רב א"ל אם איתא דהדר ביה לדידי הוה אמר לי לא ציית גדייה א"ל ולא מסתפי מר מדליקתא 
א"ל אנא איסי בן יהודה דהוא איסי בן גור אריה דהוא איסי בן גמליאל דהוא איסי בן מהללאל אסיתא דנחשא דלא שליט ביה רקבא א"ל ואנא שילא בר אבינא בוכנא דפרזלא דמתבר אסיתא דנחשא 
חלש רב אסי עיילוה בחמימי אפקוה מקרירי עיילוה בקרירי אפקוה מחמימי נח נפשיה דרב אסי}

\newsection{דף לז}
\twocol{אזל שילא אמר לדביתהו צבית לי זוודתא דלא ליזיל ולימא ליה לרב מילי עילואי צביתה ליה זוודתא נח נפשיה דשילא חזו דפרחא אסא מהאי פוריא להאי פוריא אמרי ש"מ עבדו רבנן פייסא 
\commenta{והא דתניא \textbf{ר' מרינוס אומר אין לידה סותרת בזיבה,} אפילו ברואה קאמר לפי' שאין דם ראיה זו גורם כלום ואי קסבר נמי דאי אפשר לפתיחת קבר בלא דם ההוא בקושי חזיתיה ואין קושי סותר כר' מרינוס וכיון שאין דם הקושי שלפני הלידה ולא שלאחר הלידה ראוי להיות סותר אף הלידה אינה סותרת שאין סתירה אלא בראיה דהויא לה לטומאה זו כנגיעה וכנוגעת בטומאתה שאין לה סתירה כלל, ומיהו אם ראתה בלידה אין יומא עולה לדברי הכל אלא שאין זה נקרא סתירה כדפיר' רש"י זכרונו לברכה, והכי נמי מפרשינן בפלוגתא דאמוראי והכי נמי מתוקמא למר אינה סותרת לעולם ואינה עולה לעולם ולמר אינה סותרת לעונם ועולה כשהימים הם ראויים לעלות כגון שאינה רואה דלכולי עלמא לעלות נקיים בעינן ואפילו בתוך ימי טוהר כדאמרן לעיל, והיינו דאמרן מ"ל נקיים מלידה כלומר נקיים אף מלידה ורבא לא נקיים מדם לומר אעפ"י שאינו גורם נקיים בעינן ואף בימי טוהר וכדרב. }
בעי רבא קושי מהו שתסתור בזיבה 
\commenta{והא דאמר רבא \textbf{א"א בשלמא עולה היינו דלא מפסקת טומאה.} ה"ק א"א בשלמא דינה לעלות בשאינה רואה אפילו ברואה נמי אינה סותרת שאין כאן טומאת ז' מפסקת אלא יומו הוא דלא חזי לעלות דומיא דרואה קרי שסותר יומו ואינו מפסיק כדאמרן בריש פירקן וכ"ש הכא דהאי דם לאו כגורם הוא טומאה כלל ולא מוסיף ביה טומאה דכלום אלא א"א אינו עולה האיכא טומאת ז' ושבועיים דמפסקא, כן נ"ל לפי' שמוע' זו ובתוספת מאריכין בה בענינים הרבה שאינן עולין.\par ואיכא למידק אשמעתין דהא בשילהי בא סימן (דף כ"ד) איבעי להו ימי לידתה שאינה רואה בהן מהו שיעלה לה לספירת זיבתה, ואמר רב כהנא ת"ש ומסקנא ש"מ עולין ש"מ, וי"מ דהכא אליבא דר' מרינוס איירינן דאביי דאיק מדקאמר אינה סותרת מכלל דאינה עולה דה"ל למיתנא רבותא דעולה ורבא אמר אפילו לר' מרינוס עולה והא דאמר רבא מנא אמינא לה מואח' תטהר לומר דכיון דקרא קא דרשינן אפילו לר' מרינוס אית ליה, וכן הא דתניא מזובה ולא מנגעה מזובה ולא מלידתה קרא קא דייק והיינו דקאמר ליה אביי תני חדא כלומר לר' מרינוס דוקא חדא אבל ברייתא תרתי קתני והא דאמר אביי מנא אמינא לה לומר דכיון דמשכחת תנא דאמר אינה עולה ר' מרינוס היא דהא לרבנן עולה, וזה הפי' שמעתי ולא נתקבל לי.\par ועכשיו מצאתי בתוספות בשמו של ר"ש ז"ל שכתבו בתשובותיו ואמר הרב ז"ל תדע דהא בעי לה לקמן בפרק בא סימן איבעי להו וכו', ואין דרך התלמוד לשאול בעיא אחת שתי פעמים ולא מצינו כן בשום מקום תלאוהו באילן גדול, ועדיין אינו מחוור לפי שאם היה אביי מודה לרבנן דעולין לא הוה משוי ליה לר' מרינוס טועה וחולק דליכא למידק מלישנא דידיה הכי כלל כדפרישית, ועוד הא דפרישו בדרבא דאמר מנא אמינא לה דמקרא דייק ולא מצי רבי מרינוס למפלג עלה הא לאו מילתא היא דאי איכא למידק מקרא תיקשי לר' אלעזר דאמר דאינה עולה אלא היינו טעמא דאביי דאיהו סבר לתרוצא לההיא ברייתא דבשלהי בא סימן כדמתרץ לה רב פפא א' שאני התם וכו'.\par ומסקנא דעולין ודאי כרבא אתיא דקי"ל כוותיה ולא כדברי הרב ר' יעקב ז"ל שפי' שהלמ"ד לידה כהכא אלא כפי קבלת הגאונים שהוא לחי במסכת עירובין (דף ט"ו) דאיתותב מיניה התם בגמרא ת"ש מעובדא דרב וההיא דתניא בפרק המפלת אינן עולין לרבא אתיא כר' אליעזר דאמר מסתר נמי סתרא והא דלא מייתינן לכולהו בשמעתין כמה איכא בתלמודא דכוותייהו שדברי תורה עניים במקומן ועשירים במקום אחר ומה שאמרו שלא מצאו בתלמוד בעיא א' בשני מקומות כאן מצינו.\par ועי"ל לו דהתם אמוראי בחראי אתו למיפשט אי כאביי או כרבא והלכה או אין הלכה מיבעיא להו וכן מצינו בפרק המגרש (דף פה ע"ב) דאיבעי להו מי בעינן ודן או לא עביד בעיא סתם בפלוגתא ברבי יהודה ורבנן [ועוד בעיא בפלוגתא דמתני'] דמתני' בפרק המקבל (דף קי"ד) מהו שיסדרו בבעל חוב וכן בפרק חזקת הבתים (דף מ ע"ב(איבעיא להו סתמא מאי קא מיבעי ליה מתרי לישני דרב יוסף דלעיל הי מינייהו הלכה וזו כן ופשטו מברייתא דעולין ואידתי ליה דאביי דסבר לכ"ע בין לרבנן בין לר' אליעזר אין עולין. }
דבר המטמא סותר והאי נמי מטמא כימי נדה הוא או דילמא דבר הגורם סותר והאי לאו גורם הוא 
א"ל אביי אונס בזיבה יוכיח שאינו גורם וסותר 
אמר ליה לאיי האי נמי גורם הוא דתנן ראה ראייה ראשונה בודקין אותו שניה בודקין אותו שלישית אין בודקין אותו 
ולרבי אליעזר דאמר אף בשלישי' בודקין אותו ה"נ כיון דלא גרים לא סתר אמר ליה לרבי אליעזר ה"נ 
ת"ש רבי אליעזר אומר אף בשלישית בודקין אותו ברביעית אין בודקין אותו מאי לאו לסתירה 
לא לטמויה לההיא טיפה במשא 
ת"ש בשלישית רבי אליעזר אומר בודקין אותו ברביעית אין בודקין אותו לקרבן אמרתי ולא לסתירה 
אלא לר"א תפשוט דדבר שאינו גורם סותר לרבנן מאי 
ת"ש דתני אבוה דרבי אבין מה גרם לו זובו שבעה לפיכך סותר שבעה מה גרם לו קריו יום אחד לפיכך סותר יום אחד 
מאי שבעה אילימא דמטמא שבעה האי מה זובו טמא שבעה מבעי ליה אלא לאו דבר הגורם סותר דבר שאינו גורם אינו סותר ש"מ 
אמר אביי נקטינן אין קושי סותר בזיבה ואי משכחת תנא דאמר סותר ההוא ר"א היא 
תניא רבי מרינוס אומר אין לידה סותרת בזיבה איבעיא להו מהו שתעלה אביי אמר אינה סותרת ואינה עולה רבא אמר אינה סותרת ועולה 
אמר רבא מנא אמינא לה דתניא (ויקרא טו, כח) ואחר תטהר אחר אחר לכולן שלא תהא טומאה מפסקת ביניהם 
אי אמרת בשלמא עולה היינו דלא מפסקת טומאה אלא אי אמרת אינה עולה אפסיק ליה לידה ואביי אמר לך שלא תהא טומאת זיבה מפסקת ביניהם 
אמר רבא מנא אמינא לה דתניא מזובה מזובה ולא מנגעה מזובה ולא מלידתה ואביי אמר לך תני חדא מזובה ולא מנגעה ולא תתני ולא מלידתה 
ורבא האי מאי אי אמרת בשלמא מזובה ולא מלידתה איידי דאצטריך ליה לידה תנא נגעה אטו לידה אלא אי אמרת מזובה ולא מנגעה האי {ויקרא ט״ו:י״ג } מוכי יטהר הזב מזובו נפקא מזובו ולא מנגעו 
ואביי חד בזב וחד בזבה וצריכי דאי כתב רחמנא
בזב משום דלא מטמא באונס אבל זבה דמטמיא באונס אימא לא צריכא 
ואי כתב רחמנא בזבה משום דלא מטמיא בראיות כבימים אבל זב דמטמא בראיות כבימים אימא לא צריכא 
אמר אביי מנא אמינא לה דתניא (ויקרא יב, ב) דותה תטמא לרבות את בועלה
דותה תטמא לרבות את הלילות דותה תטמא לרבות את היולדת בזוב שצריכה שתשב שבעה נקיים 
מאי לאו נקיים מלידה לא מדם 
ואמר אביי מנא אמינא לה דתניא כימי נדתה כך ימי לידתה מה ימי נדתה אין ראוין לזיבה ואין ספירת שבעה עולה מהן אף ימי לידתה שאין ראוין לזיבה אין ספירת שבעה עולה מהן 
ורבא הא מני רבי אליעזר היא דאמר מסתר נמי סתרה 
וכי דנין אפשר משאי אפשר 
אמר רב אחדבוי בר אמי ר' אליעזר היא דאמר דנין אפשר משאי אפשר ורב ששת אמר על כרחך הקישן הכתוב איכא דאמרי אמר רב אחדבוי בר אמי אמר רב ששת רבי אליעזר היא דאמר דנין אפשר משאי אפשר ורב פפא אמר על כרחך הקישן הכתוב
קשתה שלשה ימים וכו'
איבעיא להו שפתה מזה ומזה מהו רב חסדא אמר טמאה רבי חנינא אמר טהורה 
א"ר חנינא משל למלך שיצא וחיילותיו לפניו בידוע שחיילותיו של מלך הן 
ורב חסדא אמר כל שכן דבעי נפיש חיילות טפי 
תנן רבי יהושע אומר לילה ויום כלילי שבת ויומו ששפתה מן הצער ולא מן הדם טעמא דמן הצער ולא מן הדם הא מזה ומזה טהורה תיובתא דרב חסדא 
אמר לך רב חסדא לא מבעיא מזה ומזה דטמאה דפסקי להו חיילות לגמרי אבל מן הצער ולא מן הדם אימר כי היכי דמדם לא פסקה מקושי נמי לא פסקה והא תונבא בעלמא הוא דנקט לה קמ"ל 
תנן קשתה שלשה ימים בתוך אחד עשר יום ושפתה מעת לעת וילדה הרי זו יולדת בזוב 
היכי דמי אילימא כדקתני למה לי שלש בתרי בקושי וחד בשופי סגי 
אלא לאו הכי קאמר קשתה שלשה ושפתה מזה ומזה או שקשתה שנים ושפתה מעת לעת הרי זו יולדת בזוב ותיובתא דר' חנינא 
אמר לך רבי חנינא לא לעולם כדקתני והא קא משמע לן דאע"ג דמתחיל קישוי בשלישי ושפתה מעת לעת טמאה לאפוקי מרבי חנינא
כמה היא קשויה ר"מ אומר וכו' השתא חמשים מקשיא ארבעים מיבעיא אמר רב חסדא ל"ק כאן לחולה כאן לבריאה 
א"ר לוי אין הולד מטהר אלא ימים הראויין להיות בהן זבה ורב אמר אפי' בימים הראויין לספירת זבה אמר רב אדא בר אהבה ולטעמיה דרב}

\newsection{דף לח}
\twocol{אפי' ימים הראויין לספירת סתירת זבה 
תנן כמה הוא קשויה ר"מ אומר ארבעים וחמשים יום 
בשלמא לרב משכחת לה כרב אדא בר אהבה אלא ללוי קשיא
אמר לך לוי מי קתני טהורה בכולן בימי נדה נדה בימי זיבה טהורה 
לישנא אחרינא אמרי א"ר לוי אין הולד מטהר אלא ימים הראויין להיות בהן זבה גדולה מ"ט (ויקרא טו, כה) דמה ימים רבים כתיב 
אבא שאול משמיה דרב אמר אפילו ימים הראויין להיות בהן זבה קטנה מ"ט ימי וכל ימי התם כתיבי 
תנן כמה הוא קשויה ר"מ אומר אפי' ארבעים וחמשים יום קשיא לתרוייהו מי קתני טהורה בכולן קשתה בימי נדתה נדה בימי זיבתה טהורה 
תניא היה ר"מ אומר יש מקשה ק"נ יום ואין זיבה עולה בהן כיצד שנים בלא עת
ושבעה נדה ושנים של אחר הנדה וחמשים שהולד מטהר
ושמונים של נקבה ושבעה נדה ושנים של אחר הנדה 
אמרו לו א"כ יש מקשה כל ימיה ואין זיבה עולה בהן 
אמר להן מאי דעתייכו משום נפלים אין קושי לנפלים 
ת"ר יש רואה מאה יום ואין זיבה עולה בהן כיצד שנים בלא עת ושבעה נדה ושנים של אחר הנדה ושמונים של נקבה ושבעה נדה ושנים של אחר הנדה 
מאי קמ"ל לאפוקי ממ"ד אי אפשר לפתיחת הקבר בלא דם קמ"ל דאפשר לפתיחת הקבר בלא דם
ר' יהודה אומר דיה וכו' תניא רבי יהודה אומר משום רבי טרפון דיה חדשה ויש בדבר להקל ולהחמיר 
כיצד קשתה שנים בסוף שמיני ואחד בתחלת תשיעי ואפילו בתחלת תשיעי ילדה הרי זו יולדת בזוב
אבל קשתה יום אחד בסוף שמיני ושתים בתחלת תשיעי ואפילו בסוף תשיעי ילדה אין זו יולדת בזוב 
אמר רב אדא בר אהבה ש"מ קסבר רבי יהודה שיפורא גרים איני והא אמר שמואל אין אשה מתעברת ויולדת אלא למאתים ושבעים ואחד יום או למאתים ושבעים ושנים יום או למאתים ושבעים ושלשה 
הוא דאמר כחסידים הראשונים דתניא חסידים הראשונים לא היו משמשין מטותיהן אלא ברביעי בשבת שלא יבואו נשותיהן
לידי חלול שבת ברביעי ותו לא אימא מרביעי ואילך 
\commenta{ והא ד\textbf{אמר רבא בהא זכנהו ר' אליעזר.} טעמייהו קא מפרש דודאי ר' אליעזר אפילו פרכיה לק"ו אינו בדין עד שיטמא ימי טוהר שהתורה טהרתם סתם, ועוד שאין לך טעם להחמיר עליו יותר מן השופי ורבנן נמי לא צריכי לק"ו אלא למיפרך טעמיה דר"א. }
אמר מר זוטרא מאי טעמייהו דחסידים הראשונים דכתיב {רות ד׳:י״ג } ויתן [ה'] לה הריון הריון בגימטריא מאתן ושבעים וחד הוו 
אמר מר זוטרא אפי' למ"ד יולדת לתשעה אינה יולדת למקוטעים יולדת לשבעה יולדת למקוטעים שנאמר (שמואל א א, כ) ויהי לתקופות הימים ותהר חנה ותלד בן מיעוט תקופות שנים מיעוט ימים שנים
רבי יוסי ור"ש אומרים אין קושי יותר מב' שבתות אמר שמואל מאי טעמייהו דרבנן דכתיב (ויקרא יב, ה) וטמאה שבועים כנדתה כנדתה ולא כזיבתה מכלל דזיבתה טהורה וכמה שבועים 
ת"ר יש מקשה עשרים וחמשה יום ואין זיבה עולה בהן כיצד שנים בלא עת ושבעה נדה וב' שלאחר נדה וארבעה עשר שהולד מטהר 
ואי אפשר שתתקשה עשרים וששה יום בלא ולד ולא תהא יולדת בזוב 
בלא ולד בתלתא נמי סגי אמר רב ששת אימא במקום שיש ולד אמר ליה רבא והא בלא ולד קתני 
אלא אמר רבא הכי קאמר אי אפשר שתתקשה עשרים וששה יום במקום שיש ולד ולא תהא יולדת בזוב ובמקום שאין ולד אלא נפל בתלתא נמי הויא זבה מ"ט אין קושי לנפלים
{\large\emph{מתני׳}} המקשה בתוך שמונים של נקבה כל דמים שהיא רואה טהורין עד שיצא הולד ורבי אליעזר מטמא
אמרו לו לרבי אליעזר ומה במקום שהחמיר בדם השופי היקל בדם הקושי מקום שהיקל בדם השופי אינו דין שנקל בדם הקושי 
אמר להן דיו לבא מן הדין להיות כנדון ממה היקל עליה מטומאת זיבה אבל טמאה טומאת נדה
{\large\emph{גמ׳}} תנו רבנן תשב לרבות המקשה בתוך שמונים של נקבה שכל דמים שהיא רואה טהורין עד שיצא הולד ור"א מטמא 
אמרו לו לר"א ומה במקום שהחמיר בשופי שלפני הולד היקל בשופי שלאחר הולד מקום שהיקל בקושי שלפני הולד אינו דין שנקל בקושי שלאחר הולד 
אמר להם דיו לבא מן הדין להיות כנדון ממה היקל עליה מטומאת זיבה אבל מטמאה טומאת נדה 
אמרו לו הרי אנו משיבין לך לשון אחר ומה במקום שהחמיר בשופי שלפני הולד היקל בקושי שעמו מקום שהיקל בשופי שלאחר הולד אינו דין שנקל בקושי שעמו 
אמר להם אפילו אתם משיבין כל היום כולו דיו לבא מן הדין להיות כנדון ממה היקל עליה מטומאת זיבה אבל מטמאה טומאת נדה 
אמר רבא בהא זכינהו ר"א לרבנן לאו אמריתו דמה דמה מחמת עצמה ולא מחמת ולד ה"נ (ויקרא יב, ז) וטהרה ממקור דמיה דמיה מחמת עצמה ולא מחמת ולד 
אימא בימי נדה נדה בימי זיבה טהורה אמר קרא תשב ישיבה אחת לכולן
{\large\emph{מתני׳}} כל אחד עשר יום בחזקת טהרה}

\newsection{דף לט}
\twocol{ישבה לה ולא בדקה שגגה נאנסה הזידה ולא בדקה טהורה 
\commenta{\textbf{מקבע לא קבעה.} פרש"י ז"ל דתיבעי ג"פ לעקרן, מיחש מהו דניחוש לה אם היתה רנילה מט"ו לט"ו דהיינו ימי זוב מיבעי' למיחש ולא תשמ' ליום ט"ו קודם ראיה שמא תראה ואינו יודע היכי אתינן למיפשט הא מילתא מדשמואל החס לר' פפא נמי ראית עשרין ותרין קמייאתא בימי נדה הוו וראית עשרין וז' דהאידנ' בימי נדה הויא, ולהאי פירושא אמאי לא תיחוש להו בכל זמן שיבא לה וסתה כיון שהוקבע הוסת כראוי בזמן נדה.\par אלא אפשר לפרש מקבע לא קבעה מיחש מהו דתיחוש לה אם היה לה וסת בין קבוע בין שאינו קבוע בימים הראוים לוסת ואירע לה אותו היום בימי זיבה מהו שתחוש שמא בימים הללו תראה ולא תשמש או דילמא כשם שאינה קובעת וסת בתוך י"א כך אינה חוששת לוסת הראשון שלה בימי י"א והא מילתא מיפשטא בהדיא מדשמואל לפירושיה דרב פפא.\par ולענין גמר' ודאי מסתברא דלית הלכתא כרב פפא אלא כרב הונא ברי' דר' יהושע ואיהו כיון דאידחיא לראיה דרב פפא מדשמואל ודאי מיפלג פליג עלי' וכיון דוסתות דרבנן אע"ג דלא איפשטא ליה לרב הונא לקולא אנן לקול' נקיטו בה ואע"ג דאמר רב פפא בשלהי האשה דלק' דחיישא, רב פפא לא מהימן בה דאיהו מדשמואל אמרה והא אידחי, אבל הרב רבי אברהם בר דוד ז"ל פסק בספרו כר' פפא ואף אנו עליו נסמוך וכ"ש מאחר שכתבנו שאין הנשים יודעת פתחי נדה שכל ראיה שהן רואות חוששין לה בכל זמן שהוא בתוך נדה הן. }
הגיע שעת וסתה ולא בדקה הרי זו טמאה ר"מ אומר אם היתה במחבא והגיע שעת וסתה ולא בדקה הרי זו טהורה מפני שחרדה מסלקת את הדמים 
אבל ימי הזב והזבה ושומרת יום כנגד יום הרי אלו בחזקת טומאה
{\large\emph{גמ׳}} למאי הלכתא אמר רב יהודה לומר שאינה צריכה בדיקה והא מדקתני סיפא ישבה ולא בדקה מכלל דלכתחלה בעיא בדיקה 
סיפא אתאן לימי נדה וה"ק כל י"א בחזקת טהרה ולא בעיא בדיקה אבל בימי נדתה בעיא בדיקה ישבה ולא בדקה שגגה נאנסה הזידה ולא בדקה טהורה 
רב חסדא אמר לא צריכא אלא לר"מ דאמר אשה שאין לה וסת אסורה לשמש ה"מ בימי נדתה אבל בימי זיבתה בחזקת טהרה קיימא 
א"ה אמאי א"ר מאיר יוציא ולא יחזיר עולמית דלמא אתיא לקלקולא בימי נדה 
הא מדקתני סיפא הגיע שעת וסתה ולא בדקה מכלל דבאשה שיש לה וסת עסקינן חסורי מחסרא והכי קתני כל י"א בחזקת טהרה ושריא לבעלה ובימי נדה אסורה 
בד"א באשה שאין לה וסת אבל יש לה וסת מותרת וצריכה בדיקה ישבה ולא בדקה שגגה נאנסה הזידה ולא בדקה טהורה הגיע שעת וסתה ולא בדקה טמאה 
הא מדסיפא ר"מ רישא לאו ר"מ כולה ר"מ היא וה"ק אם לא היתה במחבא והגיע שעת וסתה ולא בדקה טמאה שר"מ אומר אם היתה במחבא והגיע שעת וסתה ולא בדקה טהורה שחרדה מסלקת את הדמים 
רבא אמר לומר שאינה מטמאה מעת לעת 
מיתיבי הנדה והזבה והשומרת יום כנגד יום והיולדת כולן מטמאות מעת לעת תיובתא 
רב הונא בר חייא אמר שמואל לומר שאינה קובעת לה וסת בתוך ימי זיבתה אמר רב יוסף לא שמיע לי הא שמעתתא 
א"ל אביי את אמרת ניהלן ואהא אמרת לן היתה למודה להיות רואה יום ט"ו (יום) ושינתה ליום כ' זה וזה אסורין לשמש שינתה פעמים ליום כ' זה וזה אסורין 
ואמרת לן עלה אמר רב יהודה אמר שמואל ל"ש אלא ט"ו לטבילתה שהן כ"ב לראיתה דהתם בימי נדתה קאי לה אבל ט"ו לראיתה דבימי זיבתה קאי לא קבעה 
אמר רב פפא אמריתא לשמעתא קמיה רב יהודה מדסקרתא מקבע לא קבעה מיחש מהו דניחוש לה 
אישתיק ולא א"ל ולא מידי אמר רב פפא נחזי אנן היתה למודה להיות רואה ליום ט"ו ושינתה ליום כ' זה וזה אסורין
ואמר רב יהודה אמר שמואל ל"ש אלא ט"ו לטבילתה שהן כ"ב לראייתה
ושינתה ליום כ"ז דכי הדרי ואתו עשרין ותרתי קיימא לה בתוך ימי זיבתה וקתני זה וזה אסורין אלמא דחיישינן לה 
וקסבר רב פפא עשרין ותרתין מעשרין ותרתין מנינן נדה ופתחה מעשרין וז' מנינן 
א"ל רב הונא בריה דרב יהושע לרב פפא ממאי דלמא עשרין ותרתין נמי מעשרין וז' מנינן דכי הדרי ואתו עשרין ותרתין קיימא לה בתוך ימי נדותה 
וה"נ מסתברא דאי לא תימא הכי האי תרנגולתא דרמיא יומא וכבשה יומא ורמיא יומא וכבשה יומא וכבשה תרי יומי ורמיא חד יומא
כי הדרה נקטה כדלקמיה נקטה או כדמעיקרא נקטה על כרחך כדלקמיה נקטה 
א"ל רב פפא אלא הא דאמר ר"ל אשה קובעת לה וסת בתוך ימי זיבתה ואין אשה קובעת לה וסת בתוך ימי נדותה ורבי יוחנן אמר אשה קובעת לה וסת בתוך ימי נדותה ה"ד 
לאו כגון דחזאי ריש ירחא וחמשא בירחא וריש ירחא וחמשא בירחא והשתא חזאי בחמשא בירחא ובריש ירחא לא חזאי
וקאמר אשה קובעת לה וסת בתוך ימי נדותה אלמא מריש ירחא מנינא 
א"ל לא הכי א"ר יוחנן כגון דחזאי ריש ירחא וריש ירחא ועשרין וחמשה בירחא וריש ירחא דאמרינן דמי יתירי הוא דאתוספו בה 
וכן כי אתא רבין וכל נחותי ימא אמרוה כרב הונא בריה דרב יהושע
\par \par {\large\emph{הדרן עלך בנות כותים}}\par \par }

\newchap{פרק \hebrewnumeral{5} יוצא דופן}}

\newsection{דף מ}
\twocol{
\commenta{\textbf{מ"ט גמר לידה מבכור.} פי' דאי לאו ג"ש כיון שנתרבה יוצא דופן בכלל לידה גבי אדם א"א למעטו מכי יולד ולהכי צריך ג"ש דלידה לידה והא דאמרינן שכן אמו מאמו לא דהיא ג"ש אלא מסתברא לידה לידה דאדם גמר כן משום דדמי מדמי הוא דהכא כתיב אמו והכא כתיב אמו והא דאמרינן לקמן מאמו אמו נפקא לאו דוקא אלא חדא מטעמיה דג"ש נקט : }
מתני׳ {\large\emph{יוצא}} דופן אין יושבין עליו ימי טומאה וימי טהרה ואין חייבין עליו קרבן ר"ש אומר הרי זה כילוד 
\commenta{\textbf{זאת תורת העולה היא העולה הרי אלו ג' מעוטין פרט לנשחטה בלילה וכו'.} פירש לר' יהודה כיון דממעט הני אף על פי שפסולן בקדש ומכשר הלן והיוצ' ושארא כדבעינן למימר צריכי מעוטא לכל חד וחד אבל לר' שמעון כיון דקרא א' מרבה וקרא א' ממעט הריבוי ריבה הכל והמיעוט מיעט הכל, ול"ק כאן שפיסולו בקדש כאן שאין פיסולו בקדש.\par והא דאמרינן הרי אלו [ג'] מיעוטין ולא אמרינן אין מיעוט אחר מיעוט אלא לרבות משום דכיון דכל אחד ממעט את שלו אין כאן מיעוט אחר מיעוט שאין מיעוט אתר מיעוט לרבו' אלא כשהן ממעט' דבר אחד כגון שאמרו (סנהדרין טו, א) עשרה כהנים כתובים בפרשה כהן ולא ישראל ואין מיעוט אחר מיעוט אלא לרבות אבל כאן כל מיעוט הוא צריך למעט את שלו.\par וכיוצא בזו בב"ק (מד, ב) שור שור שור ז' פעמים להוציא שור האשה ושור היתומין ושור האפוטרופסין וכו' ולא היו מיעוט אתר מיעוט והני תלתא מיעוטי נמי ממעטי הני תלתא פסולי כדפרישית.\par אבל במס' הוריות מצאתי בפרק ראשון בירושלמי (ה"א) גבי נפש כי תחטא אחת תחטא בעשותה [תחטא] הרי אלו מיעוטין דמקשי בכל אתר את אמרת מיעוט אחר מיעוט לרבו' וכאן את אמרת מיעוט אתר מיעוט למעט א"ר מתניא שניה היא דכתיב מיעוט אחר מיעוט לאחר מיעוטי ולדעת זו ההיא דאמרי' בסנהדרין כהן ולא ישראל מפני שכולן צריכין לכתב לומר דעשרה בעינן ד) אבל בשאר דוכתי ג' מיעוטין או יותר נדרשין הן כולן למיעוט כמשמען ולא נאמרה מדה זו בתורה אלא בשני מיעוטן מיעוט אחר מיעוט.\par מ"מ כל הנך פסולי דמכשיר ר' שמעון מודה בהו ר' יהודה וטעמא מפרש במסכת זבחים (דף פ"ד) מפני מה אמרו לן בדם כשר שהרי לן כשר באימורין לן באימורין כשר שהרי לן כשר בבשר, יוצא הואיל וכשר בבמה, טמא הואיל ואשתרי לגבי צבור, ונשחט חוץ לזמנו הואיל ומרצה לפיגול, חוץ למקומו הואיל ואתקו' לחוץ לזמנו ושקבלו פסולין וזרקו את דמו בהנך פסולין דחזו לעבוד' צבור וכי דנין דבר שלא בהכשרו מדבר שהוא בהכשרו תנא אזאת תורת העולה ריבה קא סמיך הדין גמרא דהתם (ובתכפה) [וכתבנוה] מפני שהיא (תמה) [סתומה] ויש לדקדק בה דא"כ נשחטה בלילה נמי כשרה שהרי כשר בבמת יחיד כדאי' בזבחים (דף ק"כ), איכא למימר אתיא כמ"ד התם שחיטת לילה פסולה בבמת יחיד.\par אלא הא קשיא יצא דמה חוץ לקלעים נימא דכשר שהרי כשר בבמה שאין יוצא בבמה לא בבשר ולא בדם וכן חוץ למקומו נמי דקאמר הואיל ואתקוש לחוץ לזמנו לימא הואיל וכשר בבמה, ועוד דקאמר ושקבלו פסולין וזרקו את דמו בהנך פסולין דחזו לעבודת צבור אפילו זר גמור נמי יהא כשר שהרי כשר בבמת יחיד.\par ואיכא למימר נשחטה בלילה ויוצא דמה ליכא לאכשורי (בשרן) דאי הכי מיעוטין מאי אהנו לי ומסתבר' ליה לאוקמי בהני דבעיקר הכשרן נפסלו מאינך והא דאוקי בהנך פסולי דחזו לצבור ולא אמר משום דכשרין בבמה דבהא פשיטא ליה דליכא למילף מינה שזה ודאי דבר שעקר הכשרו כן הוא ולית להו הכשר בכשרין טפי מפסולין לעולם ודקאמרת חוץ למקומו הואיל ואתקוש משום דכל היכא דמשכח הואיל בפנים לא מייתי ליה מבמה, כנ"ל.\par ובתוספות מאריכין בע"א, וניתנין למעלה שנתנן למטה וכן בחוץ ובפנים ופסח וחטאת כולהו כיון דבפנים נמי אשכחן בהו הכשרא לא צריכא ליה למימר' דודאי לא ירדו. }
כל הנשים מטמאות בבית החיצון שנאמר (ויקרא טו, יט) דם יהיה זובה בבשרה אבל הזב ובעל קרי אינן מטמאין עד שתצא טומאתן לחוץ 
היה אוכל בתרומה והרגיש שנזדעזעו אבריו אוחז באמה ובולע את התרומה ומטמאין בכל שהוא אפילו כעין החרדל ובפחות מכן
{\large\emph{גמ׳}} א"ר מני בר פטיש מאי טעמייהו דרבנן אמר קרא (ויקרא יב, ב) אשה כי תזריע וילדה זכר עד שתלד במקום שהיא מזרעת 
ור"ש ההיא דאפילו לא ילדה אלא כעין שהזריעה אמו טמאה לידה 
ור"ש מאי טעמיה אמר ר"ל אמר קרא תלד לרבות יוצא דופן 
ורבנן האי מבעי ליה לרבות טומטום ואנדרוגינוס דסלקא דעתך אמינא זכר ונקבה כתיב זכר ודאי נקבה ודאית ולא טומטום ואנדרוגינוס קמ"ל 
ורבי שמעון נפקא ליה מדתני בר ליואי דתני בר ליואי לבן לבן מכל מקום לבת לבת מ"מ 
ורבנן האי מבעי ליה לחייב על כל בן ובן ולחייב על כל בת ובת 
ורבי שמעון נפקא ליה מדתני תנא קמיה דרב ששת (ויקרא יב, ז) זאת תורת היולדת מלמד שמביאה קרבן אחד על ולדות הרבה יכול תביא על לידה ועל זיבה כאחת 
אלא יולדת דאכלה דם ויולדת דאכלה חלב בחד קרבן תסגי לה 
אלא יכול תביא על לידה שלפני מלאת ועל לידה שלאחר מלאת כאחת ת"ל זאת ורבנן אע"ג דכתיב זאת איצטריך לבן או לבת
סד"א בתרי עיבורי (דחד הוי נפל) אבל בחד עבורא כגון יהודה וחזקיה בני ר' חייא אימר בחד קרבן סגי לה קמ"ל 
א"ר יוחנן ומודה רבי שמעון בקדשים שאינו קדוש מאי טעמא גמר לידה לידה מבכור מה התם פטר רחם אף כאן פטר רחם 
ולגמר לידה לידה מאדם מה התם יוצא דופן אף כאן יוצא דופן 
מסתברא מבכור הוה ליה למילף שכן אמו מאמו אדרבה מאדם הוה ליה למילף שכן פשוט מפשוט 
אלא מבכור הוה ליה למילף שכן אם בהמת קדשים פגול נותר וטמא 
אדרבה מאדם הוה ליה למילף שכן פשוט זכר קדוש במתנה הנך נפישן 
אמר רב חייא בריה דרב הונא משמיה דרבא תניא דמסייע ליה לר' יוחנן ר' יהודה אומר (ויקרא ו, ב) זאת תורת העולה היא העולה הרי אלו ג' מיעוטין}

\newchap{פרק \hebrewnumeral{5} יוצא דופן}
\twocol{פרט לנשחטה בלילה ושנשפך דמה ושיצא דמה חוץ לקלעים שאם עלתה תרד 
רבי שמעון אומר עולה אין לי אלא עולה כשרה מנין לרבות שנשחטה בלילה ושנשפך דמה ושיצא דמה חוץ לקלעים והלן והיוצא והטמא והנותר ושנשחט חוץ לזמנו וחוץ למקומו
ושקבלו פסולין וזרקו את דמן והנתנין למעלה שנתנן למטה והנתנין למטה שנתנן למעלה והנתנין בחוץ שנתנן בפנים והנתנין בפנים שנתנן בחוץ והפסח והחטאת ששחטן שלא לשמן מנין 
ת"ל (ויקרא ו, ב) זאת תורת העולה ריבה תורה אחת לכל העולין שאם עלו לא ירדו 
יכול שאני מרבה את הרובע והנרבע והמוקצה והנעבד ואתנן והמחיר והכלאים והטרפה ויוצא דופן ת"ל זאת 
ומה ראית לרבות את אלו ולהוציא את אלו}

\newsection{דף מא}
\twocol{אחר שריבה הכתוב ומיעט אמרת מרבה אני את אלו שהיה פסולן בקדש ומוציא אני את אלו שלא היה פסולן בקדש 
קתני מיהת יוצא דופן דלא מאי לאו יוצא דופן דקדשים אמר רב הונא בריה דרב נתן לא יוצא דופן דבכור 
בכור מפטר רחם נפקא 
אלא מאי דקדשים מאמו אמו נפקא 
האי מאי אי אמרת בשלמא דקדשים היינו דאצריכי תרי קראי חד לבהמת חולין דאוליד דרך דופן ואקדשה
וחד לבהמת קדשים דאוליד דרך דופן וקסבר ולדות קדשים בהוייתן הן קדושים אלא אי אמרת דבכור מפטר רחם נפקא 
הכי נמי מסתברא מדקתני הרובע והנרבע והמוקצה והנעבד והכלאים
הני מהכא נפקא מהתם נפקא (ויקרא א, ב) מן הבהמה להוציא הרובע והנרבע מן הבקר להוציא את הנעבד מן הצאן להוציא את המוקצה ומן הצאן להוציא את הנוגח 
ותו כלאים מהכא נפקא מהתם נפקא (ויקרא כב, כז) שור או כשב או עז שור פרט לכלאים או עז פרט לנדמה 
אלא אצטריכו תרי קראי חד לבהמת חולין וחד לבהמת קדשים הכא נמי איצטריך תרי קראי 
ת"ר המקשה שלשה ימים ויצא ולד דרך דופן הרי זו יולדת בזוב ורבי שמעון אומר אין זו יולדת בזוב ודם היוצא משם טמא ורבי שמעון מטהר 
בשלמא רישא רבי שמעון לטעמיה ורבנן לטעמייהו אלא סיפא במאי פליגי אמר רבינא כגון שיצא ולד דרך דופן
ודם דרך רחם ואזדא ר' שמעון לטעמיה ורבנן לטעמייהו 
\commenta{והא דאמרינן \textbf{ואזדא ר' יוחנן לטעמיה שאמר משום רשב"י וכו'.} אף ע"ג דר"י להא משום רשב"י נמי אמרה ור"ל נמי לר"ש הוא מודה דאשה טהורה היא משום דנסיב ליה קרא מפורש קאמרינן הכי לימא דאין אשה טמאה קרא מפורש הוא ובאי דם עצמו טמא פליגי מר גמר לדם מאשה ומר לא גמר. }
מתקיף לה רב יוסף חדא דהיינו רישא ועוד משם מקום ולד משמע 
\commenta{הא דאקשי' לר' שמעון \textbf{פולטת תיפוק לי' דהא שמשה.} ה"ק למה לי טומאה פולטת בחוץ הא שמשה ונטמאת אפילו בפנים, ואלו לרבנן י"ל שלא הצריכה התורה טבילת משמשת אלא מפני פליטתה שאלו מפני שמוש' טהורה היא דהא מגע הוא ואותו מקום בית הסתרים הוא למגע אבל כשהיא חוזרת ופולטת עשאה הכתוב כרואה מדכתיב יהיה וההוא גלי אורחצו במים דמשום פליטה הוא דטמאים אלא לר"ש קשיא. ומהדרינן בטבלה לשמושה. }
אלא אמר רב יוסף כגון שיצא ולד ודם דרך דופן
\commenta{ואקשינן \textbf{למימרא דמשמשת בטומאת ערב סגי לה ולהכי טבלה והאמרת רבא וכו'.} ואע"ג דרבא משום פליטה קאמר ואפשר לך לומר שבזה בא ר"ש ללמד שאינה טמאה עד שתצא טומאתה לחוץ והך קושיא לרבנן נמי היא אלא כיון דאיירי בדר"ש מפרש ואזיל בהדי' ומהדרי' בשהטבילוה במטה כלומר וטהורה לשמושה ואח"כ פלטה טומאת' לבית החיצון ולא יצא לחוץ שלא הלכה ולא נתהפכ'. }
ובמקור מקומו טמא קמיפלגי מר סבר מקור מקומו טמא ומ"ס מקור מקומו טהור 
\commenta{ואקשינן \textbf{מכלל וכו'.} וכי תימא דילמא אשתייר ומספיק' אסרינן לה בטומא' אי הכי חיישינן שמא נשתיי' מיבעי ליה אלא ודאי מדלא קאמר הכי ש"מ דכל היכא דאזלא בכרעא מותרת בתרומה דודאי שדתיה לכוליה ודרבא בשלא הלכה הוא דקאמר והתם לא הוה צריך למימר חיישינן אלא א"א הוא. }
אמר ר"ל לדברי המטמא בדם מטמא באשה לדברי המטהר בדם מטהר באשה ור' יוחנן אמר אף לדברי המטמא בדם מטהר באשה 
ואזדא ר' יוחנן לטעמיה דאמר רבי יוחנן משום ר"ש בן יוחי מנין שאין אשה טמאה עד שיצא מדוה דרך ערותה שנאמר (ויקרא כ, יח) ואיש אשר ישכב את אשה דוה וגלה את ערותה את מקורה הערה מלמד שאין אשה טמאה עד שיצא מדוה דרך ערותה 
אמר ריש לקיש משום רבי יהודה נשיאה מקור שנעקר ונפל לארץ טמאה שנאמר (יחזקאל טז, לו) יען השפך נחושתך ותגלי ערותך 
למאי אילימא לטומאת שבעה דם אמר רחמנא ולא חתיכה אלא לטומאת ערב
אמר רבי יוחנן מקור שהזיע כשתי טיפי מרגליות טמאה למאי אילימא לטומאת שבעה חמשה דמים טמאין באשה ותו לא אלא לטומאת ערב ודווקא תרתי אבל חדא אימא מעלמא אתיא
כל הנשים מטמאין בבית החיצון הי ניהו בית החיצון אמר ריש לקיש כל שתינוקת יושבת ונראת 
א"ל רבי יוחנן אותו מקום גלוי הוא אצל שרץ אלא אמר רבי יוחנן עד בין השינים 
איבעיא להו בין השינים כלפנים או כלחוץ ת"ש דתני רבי זכאי עד בין השינים בין השינים עצמן כלפנים 
במתניתא תנא מקום דישה מאי מקום דישה אמר רב יהודה מקום שהשמש דש 
תנו רבנן בבשרה מלמד שמטמאה בפנים כבחוץ ואין לי אלא נדה זבה מנין ת"ל זובה בבשרה 
פולטת ש"ז מנין ת"ל יהיה ור' שמעון אומר דיה כבועלה מה בועלה אינו מטמא עד שתצא טומאה לחוץ אף היא אינה מטמאה עד שתצא טומאתה לחוץ 
וסבר רבי שמעון דיה כבועלה והתניא (ויקרא טו, יח) ורחצו במים וטמאו עד הערב אמר ר' שמעון וכי מה בא זה ללמדנו אם לענין נוגע בשכבת זרע הרי כבר נאמר למטה או איש
אלא מפני שטומאת בית הסתרים היא וטומאת בית הסתרים אינה מטמאה אלא שגזרת הכתוב הוא 
לא קשיא כאן במשמשת כאן בפולטת 
פולטת תיפוק ליה דהא שמשה בשטבלה לשמושה 
למימרא דמשמשת בטומאת ערב סגי לה והא אמר רבא משמשת כל שלשה ימים אסורה לאכול בתרומה שאי אפשר לה שלא תפלוט 
הכא במאי עסקינן שהטבילוה במטה מכלל דכי קאמר רבא דאזלה איהי בכרעה וטבלה דילמא בהדי דקאזלה שדיתא}

\newsection{דף מב}
\twocol{וכי תימא דילמא אשתייר אי הכי חיישינן שמא נשתייר מבעי ליה 
\commenta{הא ד\textbf{בעא מיניה רב שמואל בר ביזנא מאביי.} פולטת ש"ז אי רואה הויא או נוגעת. ואסיקנא דלרבנן רואה הויא לכל מילי ואפילו לר"ש נמי לסתור ולטמא במה שהוא רואה הויא משמע לי דפשטין ודאי דלא כרב הונא דאמר בפ' המפלת דאפילו בעל קרי לא סתר אלא משום דא"א לו משום צחצוחי זיבה דמדבעי חתימת פי האמה אלמא נוגע הוי וכ"ש בפולטת דלא מגופ' הויא ואינה מטמאה אלא בחוץ דאית לן למימר למימר נוגעת הויא ולא סתרה דהא אין בפליטה דאשה צחצוחי דם כלל.\par וא"ת אי הכי קשיא לאביי היכי פשיטא לר"ש לסתור ולטמא במשהו הויא דהא שמעינן ליה לר"ש דס"ל כר' נתן דאמר זב צריך חתימת פי האמה ואיתקוש ב"ק לזב וצריך נמי חתימת פי האמה כדאיתא בפ' אלו דברים. וי"ל קסבר אביי דבעל קרי אע"פ דצריך חתימת פי האמה רואה הוי ושעורא בעלמא הוא דאצרכיה רחמנא דומיא דזב וכדפרישית בפ' המפלת. ומיהו גבי אשה דליכא למימר תחימת פי הרחם דשיעורא אחרינא הויא במשהו כרוא' דם בזיבה ולא אמרינן בהא דייה כבעלה משום דלא אפשר.\par וי"מ דבמשהו דקאמרינן כעדשה דשרץ קאמרינן מ"מ מסקי השתא דרואה הויא.\par ויש מחכמי הצרפתים ז"ל שחדשו בה ואמרו כיון שהיא רואה וסותרת יומה בזמן הזה שכל הנשים ספק זבות משוינן להו אשה ששמשה בלילי שבת וראתה בשב' ופסקה טהרה בו ביום או למחרתו אינ' מתחלת ספירתה עד יום ד' בשבת שהוא ג' ימים לאחר שמושה מ"ט דכל ג' ימים פולטת היא וסותר' וקי"ל נמי דשש עונות שלימות מטמאה בפולטת והלכה למעשה הורו והנהיגו הדור בחומרא זו אלא שמתירין בכבוד הבית יפה.\par ויש לדקדק אחר דבריהם שאפילו הדבר כן שהפולטת סותרת לבעלה אין לחוש כן במהלכ' שכבר נתפרשה לנו שאם הלכה מותרת בתרומה לערב ואין חוששין שמא נשתייר אבל אם האשה הרגישה בעצמה שפלטה בשני שלה או בג' או בעומדת על מטתה ומתהפכת בכאן יש מקום להורא' זו.\par והרב ר' אברהם בר דוד ז"ל נשאל בהוראה זו ואמר שלא אמרו פולטת סותרת אלא בענין תרומה וקדשים ולענין טהרות אבל לבעלה אינה סותר' שהרי אמרו דבר הגורם סותר דבר שאינו גורם אינו סותר. וכדתניא מה גורם לו זובו ז' לפיכך סותר ז' מה גרם לו קריו יום א' לפיכך סותר יום א' ואפילו קושי אינו סותר בזיבה מפני שאינו גורם עכשיו כל שכן פולטת שאינה גורמת טומאה לבעלה שאינה סותרת לעולם ועולה. ומצאתי בתוס' שהוזכרה ביניהם סברא זו ודחוה בשתי ידים ולא קבלו אותה כלל.\par והרב ז"ל הביא ראיה לדבריו מדתניא ד"ש אומר ואחר תטהר אחר מעשה תטהר אבל אמרו חכמים אסור לעשות כן שמא תבא לידי הספק דאלמא אי לאו חששה דראיה משמשת והולכת ואמאי והא יש לחוש לפולטת שסותרת.\par וזו אינה ראיה, די"ל התורה לא חששה לפולטת שאפשר לה שלא להתהפך, וכן זו שהקשו בפ' המפלת בעשרין וחד תשמש לרוחא דמילתא מתרצי משום דמשמע דאסר לה לשמש בכל ענין ואע"פ שלא תתהפך כל היום וכ"ש למאן דסבר נוגעת הויא דצרכינין לאוקומא כר' שמעון וחכמים אף לזה חששו ואסרוה לשמש, א"נ תטהר לתרומה וקדשים ואסרו חכמים לעשות כן שמא תבא לםפק ראיה. והכי תניא בהדיא בספרי אחר מעש' תטהר כיון שטבלה טהורה להתעסק בטהרו' אבל אמרו חכמים וכו'.\par אלא מהא דאמרינן בשלהי בא סימן יום א' טמא ויום א' טהור משמשות שמיני ולילו וד' לילות מתוך י"ח ימים והא הכא דמשמשת ליל ט' והתשיעי טמא וסופרת עשירי ואין חוששין לה ג' ימים משום פליטה. גם זו אינה ראיה למה שכתבנו דמהלכת אינה חוששת לשייר. א"נ במקנחת ומכבדת את הבית.\par ובודאי שדברי הרב ר' אברהם ז"ל מכריען בטעמן שכל שאינו גורם אינו סותר ולא מצינו לובן באשה לבעלה. אלא שיש לבעל דין לחלוק ולומר שלא נתנה דבריה לשעורן וכיון דסתרה לטהרת סותרת לבעלה דנקיים מכל ראיה דטומאה בעינן וקראי נמי דייקי דכתיב ואתר תטהר וביום השמיני תקח לה וכו'. ולפי מדה זו יש שטהורה בשביעי לביתה ובח' (עראי) [היא] סופרת ודברי הרמב"ם ז"ל מטין כן שהפולטת סותרת יומא לכל דבר אבל רבינו הגדול והגאונים ז"ל לא תששו לכתוב הדבר ובעל נפש יחוש לעצמו. }
אלא לרבא נמי שהטבילוה במטה ולא קשיא כאן במתהפכת כאן בשאינה מתהפכת 
\commenta{\textbf{יולדת שירדה לטבול מטומאה לטהרה ונעקר ממנה דם.} פי' רש"י ז"ל לבית החיצון. וק"ל דא"כ היכי אקשי אמאי טומאה בלועה היא דהא אמר רבא לקמן דטומאת בית הסתרים הוה ומטמא במשא שהרי דם נדה מטמא במגע ובמשא.\par ואיכא לפרושי דנעקר לאו לבית החיצון משמע אלא רגישה בעלמא שנעקר מן המקור אע"פ שהוא בין השניים ולפנים טמאה ולכך אקשינן טומאה בלועה הוא דע"כ לא אמר רבא בית הסתרים הוי אלא בבית החיצון דהיינו בין השינים לר' יוחנן אבל לפנים בלוע הוי.\par וא"ת כי מוקמינן לדר' זירא בשיצא לחוץ נמי אמאי קשיא לן אלא יולדת אי בימי נדה נדה אי בימי זיבה זיבה נתריץ נמי ברייתא כגון שנעקר דם לבית החיצון ומטמ' התם במשא.\par לאו מילתא היא דהאמרינן דאפילו נעקר קצת מטמא לכשיוצא ולא מהני ליה טבילה ואי לאשמועינן דמעכשיו נמי היא טמאה בבית החיצון כמו שיצא הוה ליה למיתני עקירה ממקומו טומאה בבית החיצון ולא קתני ברייתא הכי אלא אי ברייתא כדר' זירא משמע מינה דלא הויא עקירה דלא להני לה טבילה עד דהוי במקום טומאה דהיינו בבית החיצון. אי נמי ניחא לן לתרוצה אליבא דכ"ע דלא תיקשי לן הניחא לרבא אלא לאביי מאי איכא למימר. ומיהו כי מתרצינן מעיקרא ברייתא דקתני כולן מטמאו' בפני' כבחוץ כי הא דר' זירא מצינן למידק עלה והא דומיא דנדה וזבה קתני והתם בבית החיצון והכא אפילו בפנים אלא אעיקר מילתא דייקינן למיקם אפירושה א"נ דהוה ליה למימר מידי איריא לטמויי בפנים כבחוץ הן שוין. אבל בשיעור מקומן הא כדאיתא והא כדאיתא.\par וי"מ פירכא אליבא דאביי ופירוקה נמי לאביי דאמרינן עשאוה כנבלת עוף טהור שמטמאה בגדים בבית הבליעה ומדמי לה אלמא התם נמי טומא' בלועה הוי דאי התם בית הסתרים הוי מאי דומיא הכא בבית הסתרים ודאי מטמיא במשא ובטומא' בלועה לא מטמיא כלל דלא דמיא נמי לנבלת עוף טהור. וזה הדרך ראיתי בתוספות.\par ולשון שלי הראשון מחוור ממנו לפי שנבלת עוף טהור מטמא בכל מקום מבית הבליעה ואפילו פנים דהוי טומאה בלועה וכי פליגי לקמי בתחלת בית הבליעה דלאביי ליכא מקום דתטמא בית הבליעה בנבלת בהמה במשא ועוף במגע ולרבא [איכא] היכא נמי דמטמא עוף בבית הבליעה מטמאה נמי בהמה משום משא הא לעוף אין לך מקום בושט ואפילו בלועה דלא ליטמיה ביה דלא יאכל לטמאה אמר רחמנ' כל דאכיל מטמא וברייתא דמסתייע אביי מינ' משום דמשמע דלעולם אין לבהמה טומאה בבית הבליעה ולא משמע להו לדחוקה בסוף הבליעה בלחוד דכתיב בה ולא באחרת. }
ורבא אקרא קאי והכי קאמר כי כתב רחמנא ורחצו במים וטמאו עד הערב בשאינה מתהפכת אבל במתהפכת כל שלשה ימים אסורה לאכול בתרומה שאי אפשר לה שלא תפלוט 
\commenta{\textbf{עשאוה כנבלת עוף טהור.} ויש לפרש דה"ק חכמים גזרו על טומאה זו מפני שדרכה לצאת ועשאו' כנבל' עוף טהור ולמיסמך גזירה דרבנן אכעין דאורייתא קאמר שאלמלא שמצינו טומאה בלועה מטמאה בדאורייתא לא הוו גוזרין בהו שמטמא. }
בעא מיניה רב שמואל בר ביסנא מאביי פולטת שכבת זרע רואה הויא או נוגעת הויא 
נפקא מינה לסתור ולטמא במשהו ולטמא בפנים כבחוץ 
מה נפשך אי שמיע ליה מתניתין לרבנן רואה הויא ולר' שמעון נוגעת הויא 
ואי לא שמיע ליה מתניתין מסתברא נוגעת הויא 
לעולם שמיע ליה מתניתין ואליבא דרבנן לא קמיבעיא ליה כי קא מיבעיא ליה אליבא דר"ש 
ולטמא בפנים כבחוץ לא קמיבעיא ליה כי קמיבעיא ליה לסתור ולטמא בכל שהוא מאי 
כי קאמר רבי שמעון דיה כבועלה הני מילי לטמויי בפנים כבחוץ אבל לסתור ולטמא בכל שהוא רואה הויא או דילמא לא שנא 
איכא דאמרי לעולם לא שמיע ליה מתניתא והכי קמיבעיא ליה מדאחמיר רחמנא אבעלי קריין בסיני רואה הויא 
או דילמא לא גמרינן מסיני דחדוש הוא דהא זבין ומצורעים דחמירי ולא אחמיר בהו רחמנא 
א"ל רואה הויא אתא שייליה לרבא א"ל רואה הויא אתא לקמיה דרב יוסף א"ל רואה הויא הדר אתא לקמיה דאביי א"ל כולכו ברוקא חדא תפיתו 
אמר ליה שפיר אמרי לך עד כאן לא קאמר ר"ש דיה כבועלה אלא לטמא בפנים כבחוץ אבל לסתור ולטמא בכל שהוא רואה הויא
ת"ר הנדה והזבה והשומרת יום כנגד יום והיולדת כולן מטמאות בפנים כבחוץ 
בשלמא כולהו לחיי אלא יולדת אי בימי נדה נדה אי בימי זיבה זיבה 
לא צריכא שירדה לטבול מטומאה לטהרה 
וכי הא דאמר רבי זירא א"ר חייא בר אשי אמר רב יולדת שירדה לטבול מטומאה לטהרה ונעקר ממנה דם בירידה טמאה בעלייה טהורה 
א"ל רבי ירמיה לר' זירא בירידה אמאי טמאה טומאה בלועה היא א"ל זיל שייליה לרבי אבין דאסברית ניהליה וכרכיש לי ברישיה בי מדרשא 
אזל שייליה א"ל עשאוה כנבלת עוף טהור שמטמאה בגדים בבית הבליעה מי דמי
התם אין לה טומאה בחוץ הכא כי נפיק לבראי ליטמי הכא נמי כשיצא לחוץ 
אי יצא לחוץ מאי למימרא מהו דתימא מגו דמהני טבילה לדם דאיכא גואי תהני נמי להאי קמ"ל 
שמעתין איפריק אלא יולדת אי בימי נדה נדה אי בימי זיבה זיבה 
הכא במאי עסקינן בלידה יבשתא לידה יבשתא מאי מטמא בפנים כבחוץ איכא 
כגון שהוציא ולד ראשו חוץ לפרוזדור וכדרב אושעיא דאמר רב אושעיא גזרה שמא יוציא הולד ראשו חוץ לפרוזדור 
וכי ההוא דאתא לקמיה דרבא אמר ליה מהו לממהל בשבתא אמר ליה שפיר דמי בתר דנפק אמר רבא ס"ד דההוא גברא לא ידע דשרי לממהל בשבתא אזל בתריה אמר ליה אימא לי איזי גופא דעובדא היכי הוה 
אמר ליה שמעית ולד דצויץ אפניא דמעלי שבתא ולא אתיליד עד שבתא אמר ליה האי הוציא ראשו חוץ לפרוזדור הוא והוי מילה שלא בזמנה וכל מילה שלא בזמנה אין מחללין עליה את השבת 
איבעיא להו אותו מקום של אשה בלוע הוי או בית הסתרים הוי 
למאי נפקא מינה כגון שתחבה לה חבירתה כזית נבלה באותו מקום אי אמרת בלוע הוי טומאה בלועה לא מטמאה ואי אמרת בית הסתרים הוי נהי דבמגע לא מטמיא במשא מיהא מטמיא 
אביי אמר בלוע הוי רבא אמר בית הסתרים הוי אמר רבא מנא אמינא לה דתניא אלא מפני שטומאת בית הסתרים היא
וטומאת בית הסתרים לא מטמאה אלא שגזרת הכתוב היא 
ואביי חדא ועוד קאמר חדא דטומאה בלועה היא ועוד אפילו אם תמצי לומר טומאת בית הסתרים היא אינה מטמאה אלא שגזרת הכתוב היא 
איבעיא להו מקום נבלת עוף טהור בלוע הוי או בית הסתרים הוי 
למאי נפקא מינה כגון שתחב לו חבירו כזית נבלה לתוך פיו אי אמרת בלוע הוי טומאה בלועה לא מטמיא (אלא אי) אמרת בית הסתרים הוי נהי נמי דבמגע לא מטמא במשא מיהא מטמא
אביי אמר בלוע הוי ורבא אמר בית הסתרים הוי אמר אביי מנא אמינא לה דתניא יכול תהא נבלת בהמה מטמאה בגדים אבית הבליעה ת"ל (ויקרא כב, ח) נבלה וטרפה לא יאכל לטמאה בה
מי שאין לה טומאה אלא אכילתה יצתה זו שטמאה קודם שיאכלנה 
ותיתי בק"ו מנבלת עוף טהור ומה נבלת עוף טהור שאין לה טומאה בחוץ יש לה טומאה בפנים זו שיש לה טומאה בחוץ אינו דין שיש לה טומאה בפנים 
אמר קרא בה בה ולא באחרת 
אם כן מה תלמוד לומר {ויקרא יא } והאוכל 
ליתן שיעור לנוגע ולנושא כאוכל מה אוכל בכזית אף נוגע ונושא בכזית 
אמר רבא שרץ בקומטו טהור נבלה בקומטו טמא 
שרץ בקומטו טהור שרץ בנגיעה הוא דמטמא ובית הסתרים לאו בר מגע הוא נבלה בקומטו טמא נהי דבמגע לא מטמא במשא מיהא מטמא 
שרץ בקומטו והכניסו לאויר התנור טמא פשיטא מהו דתימא תוכו אמר רחמנא}

\newsection{דף מג}
\twocol{ולא תוך תוכו קמ"ל 
\commenta{הא דאמרינן \textbf{את"ל בתר עקירה אזלינן לחומרא.} ק"ל אמאי לא פשטה להא מילתא מהא דאמרן ונעקר ממנה דם בירידה טמאה בשלמא בעייא דרבא ל"ק דאיהו לא משום ספיקא דבתר עקירה אזלינן בלחוד מספקא ליה אלא משום דשמואל דכיון דאינו יורה כחץ בשעת טומאתו אינו מטמא אבל הא דאמרינן גבי ירד וטבל את"ל בתר עקירה אזלינן קשיא.\par ואיכא למימר משום דכיון שטבל בנתיים א"א שלא בטלה הרגשתו ומיהו זבה שנעקרו מימי רגליה דמיא לההיא אלא דהתם מצי נקיט להו והא דקאמרינן ביה את"ל סרכא נקט. }
אמר ר"ל קנה בקומטו של זב והסיט בו את הטהור טהור קנה בקומטו של טהור והסיט בו את הזב טמא 
מאי טעמא דאמר קרא (ויקרא טו, יא) וכל אשר יגע בו הזב וידיו לא שטף במים זהו הסיטו של זב שלא מצינו לו טומאה בכל התורה כולה 
ואפקיה רחמנא בלשון נגיעה למימרא דהיסט ונגיעה כידיו מה התם מאבראי אף הכא מאבראי
אבל הזב ובעל קרי אינן מטמאין וכו' זב דכתיב (ויקרא טו, ב) כי יהיה זב מבשרו עד שיצא זובו מבשרו בעל קרי דכתיב (ויקרא טו, טז) ואיש כי תצא ממנו שכבת זרע
היה אוכל בתרומה והרגיש וכו' אוחז והתניא ר"א אומר כל האוחז באמה ומשתין כאילו מביא מבול לעולם 
אמר אביי במטלית עבה רבא אמר אפילו תימא במטלית רכה כיון דעקר עקר ואביי חייש דילמא אתי לאוסופי ורבא לאוסופי לא חייש 
והתניא למה זה דומה לנותן אצבע בעין שכל זמן שאצבע בעין מדמעת וחוזרת ומדמעת 
ורבא כל אחמומי והדר אחמומי בשעתא לא שכיח 
אמר שמואל כל שכבת זרע שאין כל גופו מרגיש בה אינה מטמאה מ"ט שכבת זרע אמר רחמנא בראויה להזריע 
מיתיבי היה מהרהר בלילה ועמד ומצא בשרו חם טמא תרגמא רב הונא במשמש מטתו בחלומו דאי אפשר לשמש בלא הרגשה 
לישנא אחרינא אמר שמואל כל שכבת זרע שאינו יורה כחץ אינה מטמאה מאי איכא בין האי לישנא להאי לישנא איכא בינייהו נעקרה בהרגשה ויצאה שלא בהרגשה 
מילתא דפשיטא ליה לשמואל מיבעיא ליה לרבא דבעי רבא נעקרה בהרגשה ויצתה שלא בהרגשה מהו 
ת"ש בעל קרי שטבל ולא הטיל מים לכשיטיל מים טמא שאני התם דרובה בהרגשה נפק 
לישנא אחרינא אמרי לה אמר שמואל כל שכבת זרע שאינו יורה כחץ אינה מזרעת אזרועי הוא דלא מזרעא הא טמויי מטמיא שנאמר (דברים כג, יא) כי יהיה בך איש אשר לא יהיה טהור מקרה אפילו קרי בעולם 
בעי רבא עובד כוכבים שהרהר וירד וטבל מהו 
אם תמצי לומר בתר עקירה אזלינן הני מילי לחומרא אבל הכא דלקולא לא אמרינן או דילמא לא שנא תיקו 
בעי רבא זבה שנעקרו מימי רגליה וירדה וטבלה מהו 
אם תמצא לומר בתר עקירה אזלינן הני מילי שכבת זרע דלא מצי נקיט לה אבל מימי רגליה דמצי נקיט לה לא או דילמא לא שנא תיקו 
בעי רבא עובדת כוכבים זבה שנעקרו מימי רגליה
וירדה וטבלה מהו 
אם תמצי לומר בתר עקירה אזלינן אע"ג דמצי נקיט להו ה"מ ישראלית דטמאה דאורייתא אבל עובדת כוכבים זבה דטמאה דרבנן לא או דילמא לא שנא תיקו
ומטמאין בכל שהן אמר שמואל זב צריך כחתימת פי האמה שנאמר (ויקרא טו, ג) או החתים בשרו מזובו 
והאנן תנן מטמאין בכל שהן הוא דאמר כרבי נתן דתניא רבי נתן אומר משום רבי ישמעאל זב צריך כחתימת פי האמה ולא הודו לו 
מ"ט דרבי ישמעאל דאמר קרא או החתים בשרו מזובו 
ורבנן ההוא מבעי ליה לח מטמא ואינו מטמא יבש 
ורבי ישמעאל ההוא מרר נפקא 
ורבנן ההוא למנינא הוא דאתא זובו חדא רר בשרו תרי את זובו תלת לימד על זב בעל שלש ראיות שחייב בקרבן 
או החתים בשרו מזובו טמא מקצת זובו טמא לימד על זב בעל שתי ראיות שמטמא משכב ומושב ורבי ישמעאל מנינא מנא ליה נפקא ליה מדרבי סימאי
דתניא רבי סימאי אומר מנה הכתוב שתים וקראו טמא שלש וקראו טמא הא כיצד שתים לטומאה ושלש לקרבן 
ולמאן דנפקא ליה תרוייהו {ויקרא טו } מזאת תהיה טומאתו בזובו (ויקרא טו, ב) איש איש כי יהיה זב מבשרו מאי עביד ליה מבעי ליה עד שיצא מבשרו 
זובו טמא למה לי לימד על הזוב שהוא טמא 
אמר רב חנילאי משום ר"א בר"ש שכבת זרע לרואה במשהו לנוגע בכעדשה והאנן מטמאין בכל שהן תנן מאי לאו לנוגע לא לרואה 
ת"ש חומר בשכבת זרע מבשרץ וחומר בשרץ מבשכבת זרע חומר בשרץ שהשרץ אין חלוקה טומאתו מה שאין כן בשכבת זרע חומר בשכבת זרע שהשכבת זרע מטמא בכל שהוא מה שאין כן בשרץ 
מאי לאו לנוגע  לא לרואה 
והא דומיא דשרץ קתני מה שרץ בנגיעה אף שכבת זרע בנגיעה אמר רב אדא בר אהבה שום שרץ קתני ושום שכבת זרע קתני 
ושרץ לא מטמא במשהו והא אנן תנן האברים אין להם שיעור פחות מכזית בשר המת ופחות מכזית בשר נבלה ופחות מכעדשה מן השרץ 
שאני אבר דכוליה במקום עדשה קאי דהא אילו חסר פורתא אבר מי קמטמיא 
שכבת זרע דחלוקה טומאתו מאי היא אילימא בין ישראל לדנכרים ה"נ איכא עכבר דים ועכבר דיבשה 
אלא בין קטן לגדול 
אמר רב פפא כתנאי מנין לרבות נוגע בש"ז ת"ל (ויקרא כב, ד) או איש 
ופליגי תנאי בעלמא דאיכא דאמרי דון מינה ומינה ואיכא דאמרי דון מינה ואוקי באתרא 
למ"ד דון מינה ומינה מה שרץ בנגיעה אף שכבת זרע בנגיעה ומינה מה שרץ בכעדשה אף ש"ז בכעדשה 
ולמ"ד דון מינה ואוקי באתרא מה שרץ בנגיעה אף ש"ז בנגיעה ואוקי באתרא מה ש"ז לרואה במשהו אף לנוגע במשהו 
א"ל רב הונא בריה דרב נתן לרב פפא ממאי דמאו איש דשרץ קמרבי ליה דילמא מאו איש אשר תצא ממנו שכבת זרע קמרבי ליה ודכ"ע דון מינה ומינה 
שיילינהו לתנאי איכא דתני כרב פפא ואיכא דתני כרב הונא בריה דרב נתן
{\large\emph{מתני׳}} תנוקת בת יום אחד מטמאה בנדה בת י' ימים מטמאה בזיבה
תנוק בן יום אחד מטמא בזיבה ומטמא בנגעים ומטמא בטמא מת וזוקק ליבום ופוטר מן היבום ומאכיל בתרומה ופוסל (את) [מן] התרומה}

\newsection{דף מד}
\twocol{ונוחל ומנחיל וההורגו חייב והרי הוא לאביו ולאמו ולכל קרוביו כחתן שלם 
\commenta{\textbf{מ"ט דאיהו מיית ברישא.} פיר' ודוקא מתה דקאמרינן בערכין פ"ק (דף ז) אשה היוצאת ליהרג מכין אותה כנגד בית הריון שלה כדי שימות הולד תחלה והוינן בה למימרא דאיהי מתה ברישא והא קי"ל דאיהו מאית ברישא דתנן בן יום א' וכו'. ומפרקינן ה"מ מתה דאגב דולד זוטר חיותיה עולה ביה טיפה דמלאך המות ותתיך לה לסימנים אבל נהרגה איהי מתה ברישא ואע"ג דהאי תירוצא לאוקומי יש זכייה לעובר אתמר בפ' מי שמת מיהו קושטא הוא ולהכי הוינן מינה ומפרקי לה התם. א"נ למ"ד הכי מקשי מברייתא דקתני מכין אותה כנגד בית הריון ולאו למימרא דהכי הוא בודאי. }
{\large\emph{גמ׳}} מנהני מילי דת"ר אשה אין לי אלא אשה בת יום אחד לנדה מנין ת"ל ואשה
בת י' ימים לזיבה מנא ה"מ דת"ר אשה אין לי אלא אשה בת י' ימים לזיבה מנין ת"ל ואשה
תינוק בן יום אחד כו' מנא הני מילי דת"ר (ויקרא טו, ב) איש איש מה ת"ל איש איש לרבות בן יום אחד שמטמא בזיבה דברי רבי יהודה 
רבי ישמעאל בנו של רבי יוחנן בן ברוקא אומר אינו צריך הרי הוא אומר (ויקרא טו, לג) והזב את זובו לזכר ולנקבה לזכר כל שהוא בין גדול בין קטן לנקבה כל שהיא בין גדולה בין קטנה אם כן מה ת"ל איש איש דברה תורה כלשון בני אדם
ומטמא בנגעים דכתיב (ויקרא יג, ב) אדם כי יהיה בעור בשרו אדם כל שהו
ומטמא בטמא מת דכתיב (במדבר יט, יח) ועל הנפשות אשר היו שם נפש כל דהו
וזוקק ליבום דכתיב (דברים כה, ה) כי ישבו אחים יחדיו אחים שהיה להם ישיבה אחת בעולם
ופוטר מן היבום (דברים כה, ה) ובן אין לו אמר רחמנא והא אית ליה
ומאכיל בתרומה דכתיב (ויקרא כב, יא) ויליד ביתו הם יאכלו בלחמו קרי ביה יאכילו בלחמו
ופוסל מן התרומה (ויקרא כב, יג) וזרע אין לה אמר רחמנא והא אית לה 
מאי איריא זרע אפילו עובר נמי דכתיב כנעוריה פרט למעוברת 
צריכי דאי כתב רחמנא וזרע אין לה משום דמעיקרא חד גופא והשתא תרי גופי אבל הכא דמעיקרא חד גופא והשתא חד גופא אימא תיכול כתב רחמנא כנעוריה 
ואי כתב רחמנא כנעוריה משום דמעיקרא גופה סריקא והשתא גופה מליא אבל הכא דמעיקרא גופה סריקא והשתא גופה סריקא אימא תיכול צריכא 
קראי אתרוץ אלא מתניתין מאי אריא בן יום אחד אפי' עובר נמי אמר רב ששת הב"ע בכהן שיש לו שתי נשים אחת גרושה ואחת שאינה גרושה ויש לו בנים משאינה גרושה ויש לו בן יום אחד מן הגרושה
דפוסל בעבדי אביו מלאכול בתרומה ולאפוקי מדר' יוסי דאמר עובר נמי פוסל קמ"ל בן יום אחד אין עובר לא
נוחל ומנחיל נוחל ממאן מאביו ומנחיל למאן לאחיו מאביו אי בעי מאבוה לירתי ואי בעי מיניה לירתי 
אמר רב ששת נוחל בנכסי האם להנחיל לאחיו מן האב ודוקא בן יום אחד אבל עובר לא מ"ט דהוא מיית ברישא ואין הבן יורש את אמו
בקבר להנחיל לאחיו מן האב 
איני והא הוה עובדא ופרכס עד תלת פרכוסי אמר מר בריה דרב אשי מידי דהוה אזנב הלטאה דמפרכסת 
מר בריה דרב יוסף משמיה דרבא אמר לומר שממעט בחלק בכורה ואמר מר בריה דרב יוסף משמיה דרבא בן שנולד אחר מיתת האב אינו ממעט בחלק בכורה מאי טעמא (דברים כא, טו) וילדו לו בעינן 
בסורא מתנו הכי בפומבדיתא מתנו הכי אמר מר בריה דרב יוסף משמיה דרבא בכור שנולד לאחר מיתת אביו אינו נוטל פי שנים מאי טעמא {דברים כ״א:י״ז } יכיר בעינן והא ליכא
והלכתא ככל הני לישני דמר בריה דרב יוסף משמיה דרבא
וההורגו חייב דכתיב (ויקרא כד, יז) ואיש כי יכה כל נפש מ"מ
והרי הוא לאביו ולאמו ולכל קרוביו כחתן שלם למאי הלכתא אמר רב פפא לענין אבלות 
כמאן דלא כרשב"ג דאמר כל ששהה שלשים יום באדם אינו נפל הא לא שהה ספק הוי הכא במאי עסקינן דקים ליה שכלו לו חדשיו
{\large\emph{מתני׳}} בת שלש שנים ויום אחד מתקדשת בביאה ואם בא עליה יבם קנאה וחייבין עליה משום אשת איש
ומטמאה את בועלה לטמא משכב תחתון כעליון 
נשאת לכהן תאכל בתרומה בא עליה אחד מן הפסולין פסלה מן הכהונה בא עליה אחד מכל העריות האמורות בתורה מומתין עליה והיא פטורה 
פחות מכן כנותן אצבע בעין
{\large\emph{גמ׳}} ת"ר בת ג' שנים מתקדשת בביאה דברי רבי מאיר וחכ"א בת ג' שנים ויום אחד מאי בינייהו אמרי דבי רבי ינאי ערב ראש השנה איכא בינייהו 
ור' יוחנן אמר ל' יום בשנה חשובין שנה איכא בינייהו 
מיתיבי בת ג' שנים ואפי' בת שתי שנים ויום אחד מתקדשת בביאה דברי רבי מאיר וחכמים אומרים בת שלשה שנים ויום אחד}

\newsection{דף מה}
\twocol{בשלמא לר' יוחנן כי היכי דאיכא תנא דקאמר יום אחד בשנה חשוב שנה הכי נמי איכא תנא דאמר ל' יום בשנה חשובין שנה 
\commenta{מדתנן \textbf{בן ט' שנים ויום א' שבא על יבמתו קנאה.} משמע קנאה לגמרי ליורשה וליטמא לה אלא שאינו נותן גט עד שיגדיל וכן נמי מדהוינן בה ולכשיגדיל בגט סגי לה והתניא עשו ביאת בן ט' כמאמר משמע דמדאורייתא קנאה לגמרי ובגט סגי לה אלא שהם גרעו כת ביאתו ועשאוה כמאמר וכן בדין שהרי ביאתו ביאה לכל דבר ואע"פ שאין לו דעת הא רבי רחמנא ביבמה ביאת שוגג כדמזיד וכבר פירשתי בפ' האשה רבה (יבמות צו, ב). }
אלא לר' ינאי קשיא קשיא
פחות מכאן כנותן אצבע בעין איבעיא להו הני בתולין מיזל אזלי ואתו או דלמא אתצודי הוא דלא מתצדי עד לאחר ג' 
למאי נפקא מינה כגון שבעל בתוך ג' ומצא דם ובעל לאחר שלש ולא מצא דם אי אמרת מיזל אזלי ואתו שהות הוא דלא הויא להו
אלא אי אמרת אתצודי הוא דלא מתצדי עד לאחר ג' הא אחר בא עליה מאי 
מתקיף לה רב חייא בריה דרב איקא ומאן לימא לן דמכה שבתוך ג' אינה חוזרת לאלתר שמא חוזרת לאלתר והא אחר בא עליה 
אלא נפקא מינה כגון שבעל בתוך ג' ומצא דם ובעל לאחר ג' ומצא דם אי אמרת מיזל אזלי ואתו האי דם בתולין הוא אלא אי אמרת אתצודי הוא דלא מתצדי אלא עד לאחר ג' האי דם נדה הוא מאי 
אמר רב חסדא ת"ש פחות מכאן כנותן אצבע בעין למה לי למתני כנותן אצבע בעין לתני פחות מכאן ולא כלום מאי לאו הא קמ"ל מה עין מדמעת וחוזרת ומדמעת אף בתולין מיזל אזלי ואתי 
ת"ר מעשה ביוסטני בתו של אסוירוס בן אנטנינוס שבאת לפני רבי אמרה לו רבי אשה בכמה ניסת אמר לה בת ג' שנים ויום אחד 
ובכמה מתעברת אמר לה בת י"ב שנה ויום אחד אמרה לו אני נשאתי בשש וילדתי בשבע אוי לשלש שנים שאבדתי בבית אבא 
ומי מעברה והתני רב ביבי קמיה דרב נחמן ג' נשים משמשות במוך קטנה מעוברת ומניקה 
קטנה שמא תתעבר ותמות מעוברת שמא תעשה עוברה סנדל מניקה שמא תגמול את בנה וימות 
ואיזוהי קטנה מבת י"א שנה ויום אחד ועד י"ב שנה ויום אחד פחות מכאן או יתר על כן משמשת והולכת דברי ר"מ 
וחכ"א אחת זו ואחת זו משמשת כדרכה והולכת ומן השמים ירחמו שנאמר (תהלים קטז, ו) שומר פתאים ה' 
איבעית אימא (יחזקאל כג, כ) אשר בשר חמורים בשרם ואיבעית אימא (תהלים קמד, ח) אשר פיהם דבר שוא וימינם ימין שקר 
ת"ר מעשה באשה אחת שבאת לפני ר"ע אמרה לו ר' נבעלתי בתוך שלש שנים מה אני לכהונה אמר לה כשרה את לכהונה 
אמרה לו רבי אמשול לך משל למה הדבר דומה לתינוק שטמנו לו אצבעו בדבש פעם ראשונה ושניה גוער בה שלישית מצצה אמר לה אם כן פסולה את לכהונה 
ראה התלמידים מסתכלים זה בזה אמר להם למה הדבר קשה בעיניכם [אמרו ליה] כשם שכל התורה הלכה למשה מסיני כך פחותה מבת שלש שנים כשרה לכהונה הלכה למשה מסיני ואף רבי עקיבא לא אמרה אלא לחדד בה את התלמידים
{\large\emph{מתני׳}} בן תשע שנים ויום אחד שבא על יבמתו קנאה ואין נותן גט עד שיגדיל
ומטמא בנדה לטמא משכב תחתון כעליון 
ופוסל ואינו מאכיל בתרומה ופוסל את הבהמה מע"ג המזבח ונסקלת על ידו ואם בא על אחת מכל העריות האמורות בתורה מומתין על ידו והוא פטור
{\large\emph{גמ׳}} ולכשיגדיל בגט סגי לה והתניא עשו ביאת בן ט' כמאמר בגדול
מה מאמר בגדול צריך גט למאמרו וחליצה לזיקתו אף ביאת בן ט' צריך גט למאמרו וחליצה לזיקתו 
אמר רב הכי קאמר
לכשיגדיל יבעול ויתן גט
{\large\emph{מתני׳}} בת אחת עשרה שנה ויום א' נדריה נבדקין בת שתים עשרה שנה ויום א' נדריה קיימין ובודקין כל שתים עשרה
בן שתים עשרה שנה ויום אחד נדריו נבדקין בן י"ג שנה ויום אחד נדריו קיימין ובודקין כל שלש עשרה 
קודם לזמן הזה אע"פ שאמרו יודעין אנו לשם מי נדרנו לשם מי הקדשנו אין נדריהם נדר ואין הקדשן הקדש לאחר הזמן הזה אע"פ שאמרו אין אנו יודעין לשם מי נדרנו לשם מי הקדשנו נדרן נדר והקדשן הקדש
{\large\emph{גמ׳}} וכיון דתנא בת אחת עשרה שנה ויום א' נדריה נבדקין בת י"ב שנה ויום א' נדריה קיימין למה לי סד"א בודקין לעולם קמ"ל 
וכיון דתני בת י"ב שנה ויום אחד נדריה קיימין בודקין כל שתים עשרה למה לי סלקא דעתך אמינא הואיל ואמר מר ל' יום בשנה חשובים שנה היכא דבדקנא ל' ולא ידעה להפלות אימא תו לא ליבדוק קמ"ל 
ולתני הני תרתי בבי בת י"ב שנה ויום א' נדריה קיימין ובודקין כל י"ב בת אחת עשרה ויום א' נדריה נבדקין למה לי 
איצטריך סד"א סתמא בשתים עשרה בעיא בדיקה באחת עשרה לא בעיא בדיקה והיכא דחזינן לה דחריפא טפי מיבדקה באחת עשרה קמ"ל 
קודם הזמן הזה ואחר הזמן הזה למה לי סד"א הנ"מ היכא דלא קאמרי אינהו אבל היכא דקאמרי אינהו נסמוך עלייהו קמ"ל 
ת"ר אלו דברי רבי ר"ש בן אלעזר אומר דברים האמורים בתינוקת בתינוק אמורים דברים האמורים בתנוק בתנוקת אמורים 
א"ר חסדא מ"ט דרבי דכתיב {בראשית ב׳:כ״ב } ויבן ה' [אלהים] את הצלע מלמד שנתן הקב"ה בינה יתירה באשה יותר מבאיש 
ואידך ההוא מבעי ליה לכדריש לקיש דאמר ריש לקיש משום ר"ש בן מנסיא ויבן ה' [אלהים] את הצלע אשר לקח מן האדם לאשה ויביאה אל האדם מלמד שקלעה הקב"ה לחוה והביאה אצל אדם הראשון שכן בכרכי הים קורין לקלעיתא בנייתא 
ור"ש בן אלעזר מ"ט אמר רב שמואל בר רב יצחק מתוך שהתינוק מצוי בבית רבו נכנסת בו ערמומית תחלה 
איבעיא להו תוך זמן כלפני זמן או כלאחר זמן 
למאי הלכתא אי לנדרים לאו כלפני זמן דמיא ולאו כלאחר זמן דמיא
אלא לעונשין מאי רב ור' חנינא דאמרי תרווייהו תוך זמן כלפני זמן ר' יוחנן ור' יהושע בן לוי דאמרי תרווייהו תוך זמן כלאחר זמן 
אמר רב נחמן בר יצחק וסימניך (רות ד, ז) וזאת לפנים בישראל 
מתיב רב המנונא אחר זמן הזה אע"פ שאמרו אין אנו יודעים לשם מי נדרנו לשם מי הקדשנו נדריהם נדר והקדשן הקדש הא תוך זמן כלפני זמן 
אמר ליה רבא אימא רישא קודם הזמן הזה אע"פ שאמרו יודעים אנו לשם מי נדרנו לשם מי הקדשנו אין נדריהם נדר ואין הקדשן הקדש הא תוך זמן כלאחר זמן 
ולא היא רבא קטעי הוא סבר רב המנונא ממשנה יתירה קדייק ואדדייק מסיפא לידוק מרישא 
ולא היא רב המנונא מגופא דמתניתין קא דייק הא לאחר זמן היכי דמי אי דלא אייתי שתי שערות קטן הוא אלא לאו דאייתי שתי שערות}

\newsection{דף מו}
\twocol{וטעמא דלאחר זמן הוא דגמר' לה למילתיה הא תוך זמן כלפני זמן 
\commenta{\textbf{וכי קאמר רבא חזקה למיאון.} אי קשי' למיאון למה לי חזקה בחששא בעלמא סגי והוה ליה למימר קטנה שהגיע לכלל שנותי' אינה ממאנת שמא הביאה שתי שערות. איכא למימר כי קאמר רבא חזקה לומר שאין ב"ד מטריחין עצמן לבדוק שלא תמאן ואם חששו היינו בודקין. א"נ אע"ג דאמרינן כי קאמר רבא חזקה למיאון לאו דחזקה אצטריכא ליה להכי אלא לומר דלא ממאנה וחזקה נמי היא ומהניא חזקה לנשים בודקת אותן כדלקמן בפ' בא סימן.\par והראשונים שאלו א"כ הא דאמרינן התם בודקין לה לחליצה ולמיאונין היכי משכחת לה ורבי' הגדול השיב נפקא מינה להגיעה לכלל שנותיה וקדש בתוך זמן ולא בעל אתר זמן דהוה דרבנן.\par ואלמלא שזה דבר ברור יכולין אנו לפטור עצמינו משאלה זו במה שאמרו מקצת המחברים בודקין למיאונין היינו כדאיתמר עלה לאפוקי מדר' יהודה דאמר עד שירבה שחור על הלבן קמ"ל בודקין ומכי אתיא שערות לא ממאנה ולאו למימרא דבדיקה צריך אלא בדיקה זו היינו שערות לומר דמשהביאה אותן בין בבדיקה בין בחזקה אינה ממאנת וקטנה שלא נודע אם הגיעה לכלל שנותיה והביאה סימנין לא מצינו בגמ' דינה מפורש ויש שכוללין אף בזו בכלל בודקין למיאונין ואם הביאה סימנין אינ' ממאנת ואע"פ שלא בעל אלא קוד' זמן ולא תלינן בשומא לקולא וה"נ לשאר הדברים מטילין אותה כחומרא כדין הספקות. }
ועוד מתיב רבי זירא {במדבר ו } איש כי יפליא לנדור נדר מה ת"ל איש לרבות בן י"ג שנה ויום אחד שאע"פ שאינו יודע להפליא נדריו קיימין 
ה"ד אי דלא אייתי שתי שערות קטן הוא אלא לאו דאייתי שתי שערות וטעמא דבן י"ג ויום אחד הוא דהוה ליה איש הא תוך זמן כלפני זמן תיובתא 
אמר ר"נ כתנאי בן ט' שנים שהביא ב' שערות שומא מבן ט' ועד י"ב שנה ויום אחד שומא רבי יוסי ברבי יהודה אומר סימן בן י"ג שנה ויום אחד דברי הכל סימן 
הא גופא קשיא אמרת מבן ט' ועד י"ב שנה ויום אחד שומא הא י"ג שנה גופא סימן והדר תני בן י"ג שנה ויום אחד סימן הא י"ג שנה גופא שומא 
מאי לאו תנאי היא דמר סבר תוך זמן כלאחר זמן ומר סבר תוך זמן כלפני זמן 
לא דכ"ע תוך זמן כלפני זמן ואידי ואידי בתינוקת ורישא רבי וסיפא ר"ש בן אלעזר 
ואיבעית אימא הא והא בתינוק ורישא ר"ש בן אלעזר וסיפא רבי 
ואיבעית אימא הא והא רבי הא בתינוק הא בתינוקת ואב"א הא והא ר"ש בן אלעזר הא בתינוק הא בתינוקת 
רבי יוסי ברבי יהודה אומר סימן א"ר כרוספדאי בריה דרבי שבתאי והוא שעודן בו 
תניא נמי הכי בן ט' שנים ויום אחד שהביא ב' שערות שומא מבן ט' ועד י"ב שנה ויום אחד ועודן בו שומא ר' יוסי בר' יהודה אומר סימן 
אמר רבא הילכתא תוך זמן כלפני זמן רב שמואל בר זוטרא מתני לה לשמעתא דרבא בהאי לישנא אמר רבא קטנה כל י"ב שנה ממאנת והולכת מכאן ואילך אינה ממאנת ואינה חולצת 
הא גופא קשיא אמרת אינה ממאנת אלמא גדולה היא אי גדולה היא תחלוץ 
וכי תימא מספקא ליה ומי מספקא ליה והאמר רבא קטנה שהגיעה לכלל שנותיה אינה צריכה בדיקה חזקה הביאה סימנין 
ה"מ בסתמא אבל הכא דבדקו ולא אשכחו לא 
אי הכי תמאן חוששין שמא נשרו 
הניחא למ"ד חוששין אלא למ"ד אין חוששין מאי איכא למימר דאיתמר רב פפא אמר אין חוששין שמא נשרו רב פפי אמר חוששין הני מילי לענין חליצה אבל לענין מיאון חוששין 
מכלל דמ"ד חוששין חולצת והא חוששין בעלמא קאמר 
אלא לעולם דלא בדקה ולענין חליצה חיישינן וכי קאמר רבא חזקה למיאון אבל לחליצה בעיא בדיקה 
אמר רב דימי מנהרדעא הלכתא חוששין שמא נשרו 
והני מילי היכא דקדשה בתוך זמן ובעל לאחר זמן דאיכא ספיקא דאורייתא אבל מעיקרא לא 
אמר רב הונא הקדיש ואכל לוקה
שנאמר {במדבר ו׳:ב׳ } איש כי יפליא לנדור (במדבר ל, ג) ולא יחל דברו כל שישנו בהפלאה ישנו בבל יחל וכל שאינו בהפלאה אינו בבל יחל 
מתיב רב הונא בר יהודה (לרבא) לסיועי לרב הונא
לפי שמצינו שהשוה הכתוב הקטן כגדול לזדון שבועה ולאיסר ולבל יחל יכול יהא חייב על הקדשו קרבן 
ת"ל (במדבר ל, ב) זה הדבר 
קתני מיהת לאיסר ולבל יחל חייב אימא לאיסור בל יחל 
איסור בל יחל מה נפשך אי מופלא סמוך לאיש דאורייתא מילקא נמי לילקי ואי מופלא סמוך לאיש לאו דאורייתא איסור נמי ליכא לאותן המוזהרים עליו 
שמע מינה קטן אוכל נבלות ב"ד מצווין עליו להפרישו הכא במאי עסקינן כגון שהקדיש הוא ואכלו אחרים 
הניחא למ"ד הקדיש הוא ואכלו אחרים לוקין אלא למ"ד אין לוקין מאי איכא למימר דאיתמר הקדיש הוא ואכלו אחרים רב כהנא אמר אין לוקין רבי יוחנן ור"ל דאמרי תרוויהו לוקין 
מדרבנן וקרא אסמכתא בעלמא 
גופא הקדיש ואכלו אחרים רב כהנא אמר אין לוקין רבי יוחנן ור"ל דאמרי תרוייהו לוקין במאי קמיפלגי מר סבר מופלא סמוך לאיש דאורייתא ומר סבר מופלא סמוך לאיש מדרבנן 
מתיב רב ירמיה יתומה שנדרה בעלה מפר לה אי אמרת בשלמא מופלא סמוך לאיש דרבנן אתו נשואין דרבנן ומבטלי נדרא דרבנן אלא אי אמרת דאורייתא אתו נשואין דרבנן ומבטלי נדרא דאורייתא 
אמר רב יהודה אמר שמואל בעלה מפר לה ממה נפשך אי דרבנן דרבנן הוא אי דאורייתא קטן אוכל נבלות הוא ואין ב"ד מצווין עליו להפרישו 
והא כי גדלה אכלה בהפרה קמייתא 
אמר רבה בר ליואי בעלה מפר לה כל שעה ושעה והוא שבעל 
והא אין בעל מפר בקודמין כדרב פינחס משמיה דרבא דאמר רב פנחס משמיה דרבא כל הנודרת על דעת בעלה היא נודרת 
אמר אביי ת"ש קטן שלא הביא ב' שערות רבי יהודה אומר אין תרומתו תרומה רבי יוסי אומר עד שלא בא לעונת נדרים אין תרומתו תרומה משבא לעונת נדרים תרומתו תרומה 
סברוה קסבר ר' יוסי תרומה בזמן הזה דאורייתא אי אמרת בשלמא מופלא סמוך לאיש דאורייתא אתי גברא דאורייתא ומתקן טבלא דאורייתא אלא אי אמרת דרבנן אתי גברא דרבנן ומתקן טבלא דאורייתא לא קסבר רבי יוסי תרומה בזמן הזה דרבנן 
וסבר ר' יוסי תרומה בזמן הזה דרבנן והתניא בסדר עולם (דברים ל, ה) אשר ירשו אבותיך וירשתה
ירושה ראשונה ושניה יש להן שלישית אין להן 
וא"ר יוחנן מאן תנא סדר עולם ר' יוסי 
ר' יוסי תני לה ולא סבר לה ה"נ מסתברא דתניא עיסה שנדמעה או שנתחמצה בשאור של תרומה}

\newsection{דף מז}
\twocol{חייבת בחלה ואינה נפסלת בטבול יום דברי ר"מ ור' יהודה ר' יוסי ור"ש פוטרין מן החלה 
סברוה מאן דאמר תרומה דאורייתא חלה דאורייתא מאן דאמר תרומה דרבנן חלה דרבנן אי אמרת בשלמא קסבר רבי יוסי חלה בזמן הזה דרבנן אתי דמוע דרבנן ומפקע חלה דרבנן 
אלא אי אמרת חלה דאורייתא אתי דמוע דרבנן ומפקע חלה דאורייתא 
ודלמא קסבר רבי יוסי תרומה בזמן הזה דאורייתא וחלה דרבנן 
וכדאהדר רב הונא בריה דרב יהושע דאמר רב הונא בריה דרב יהושע אשכחתינהו לרבנן דבי רב דיתבי וקאמרי אפילו למ"ד תרומה בזמן הזה דרבנן חלה דאורייתא
שהרי שבע שכבשו ושבע שחלקו נתחייבו בחלה ולא נתחייבו במעשר 
ואמינא להו אנא אפילו למ"ד תרומה בזמן הזה דאורייתא חלה דרבנן דתניא אי בבואכם יכול משנכנסו לה שנים ושלשה מרגלים ת"ל בבואכם בביאת כולכם אמרתי ולא בביאת מקצתכם
וכי אסקינהו עזרא לא כולהו סלוק
{\large\emph{מתני׳}} משל משלו חכמים באשה פגה בוחל וצמל פגה עודה תנוקת בוחל אלו ימי נעוריה
בזו ובזו אמרו אביה זכאי במציאתה ובמעשה ידיה ובהפרת נדריה צמל כיון שבגרה שוב אין לאביה רשות בה 
איזהו סימנין ר' יוסי הגלילי אומר משיעלה הקמט תחת הדד ר"ע אומר משיטו הדדים בן עזאי אומר משישחיר הפיטומת רבי יוסי אומר כדי שיהא נותן ידו על העוקץ והוא שוקע ושוהא לחזור
{\large\emph{גמ׳}} פגה עודה תנוקת כדכתיב {שיר השירים ב׳:י״ג } התאנה חנטה פגיה בוחל אלו ימי הנעורים כדתנן התאנים משיבחלו ואמר רבה בר בר חנה אמר רב משילבין ראשיהן 
ואיבעית אימא מהכא (זכריה יא, ח) ותקצר נפשי בהם וגם נפשם בחלה בי צמל כמ"ד יצתה מלאה
ואיזהו סימנים ר' יוסי הגלילי אומר משיעלה הקמט אמר שמואל לא משיעלה הקמט ממש אלא כדי שתחזיר ידיה לאחוריה ונראית כמי שיעלה הקמט תחת הדד 
שמואל בדק באמתיה ויהב לה ד' זוזי דמי בושתה שמואל לטעמיה דאמר שמואל (ויקרא כה, מו) לעולם בהם תעבודו לעבודה נתתים ולא לבושה 
שמואל מייחד להן רב נחמן מחליף להן רב ששת מסר להן לערבי ואמר להן אזדהרו מישראל
רבי יוסי אומר כו' מאי עוקץ אמר שמואל עוקצו של דד 
ת"ר אלו הן סימני בגרות ר"א בר' צדוק אומר משיתקשקשו הדדין ר' יוחנן בן ברוקה אומר משיכסיף ראש החוטם משיכסיף אזקונה לה אלא א"ר אשי משיפציל ראש החוטם ר' יוסי אומר משתקיף העטרה ר"ש אומר משנתמעך
הכף 
וכן היה רבי שמעון (בן יוחי) אומר שלשה סימנין נתנו חכמים באשה מלמטה וכנגדן מלמעלה פגה מלמעלה בידוע שלא הביאה שתי שערות בוחל מלמעלה בידוע שהביאה שתי שערות צמל מלמעלה בידוע שנתמעך הכף 
מאי כף אמר רב הונא מקום תפוח יש למעלה מאותו מקום כיון שמגדלת מתמעך והולך שאלו את רבי הלכה כדברי מי שלח להו כדברי כולן להחמיר 
רב פפא ורב חיננא בריה דרב איקא חד מתני אהא וחד מתני אחצר צורית דתנן איזוהי חצר צורית שחייבת במעשר ר"ש אומר חצר הצורית שהכלים נשמרים בתוכה 
מאי חצר הצורית אמר רבה בר בר חנה א"ר יוחנן שכן בצור מושיבין שומר על פתח החצר ר"ע אומר כל שאחד פותח ואחד נועל פטורה 
ר' נחמיה אומר כל שאין אדם בוש לאכול בתוכה חייבת רבי יוסי אומר כל שנכנסים לה ואין אומרים לו מה אתה מבקש פטורה 
ר' יהודה אומר שתי חצרות זו לפנים מזו הפנימית חייבת והחיצונה פטורה 
שאלו את רבי הלכה כדברי מי אמר להו הלכה כדברי כולן להחמיר
{\large\emph{מתני׳}} בת עשרים שנה שלא הביאה שתי שערות תביא ראיה שהיא בת עשרים שנה והיא איילונית לא חולצת ולא מתיבמת
בן עשרים שנה שלא הביא שתי שערות יביאו ראיה שהוא בן עשרים שנה והוא סריס לא חולץ ולא מיבם אלו דברי בית הלל בית שמאי אומרים זה וזה בן שמונה עשרה 
ר' אליעזר אומר הזכר כדברי בית הלל והנקבה כדברי בית שמאי שהאשה ממהרת לבא לפני האיש
{\large\emph{גמ׳}} ורמינהי אחד לי בן תשע שנים ויום אחד ואחד לי בן עשרים שלא הביא שתי שערות 
אמר רב שמואל בר רב יצחק אמר רב והוא שנולדו בו סימני סריס אמר רבא דיקא נמי דקתני והוא סריס ש"מ 
וכי לא נולדו לו סימני סריס עד כמה תני ר' חייא עד רוב שנותיו 
כי אתו לקמיה דרבי חייא אי כחיש אמר להו אבריוה אי בריא אמר להו אכחשוה דהני סימנים זימנין דאתו מחמת כחישותא זימנין דאתו מחמת בריאותא 
אמר רב הלכתא בכולי פרקא מעת לעת ועולא אמר דתנן תנן ודלא תנן לא תנן 
בשלמא לעולא היינו דקתני הכא יום אחד והכא לא קתני אלא לרב ליתני 
ועוד תני רבי יוסי בן כיפר אומר משום רבי אליעזר שנת עשרים שיצאו ממנה שלשים יום הרי היא כשנת עשרים לכל דבריה וכן הורה רבי בלוד שנת שמנה עשרה שיצאו ממנה שלשים יום הרי היא כשנת שמנה עשרה לכל דבריה 
בשלמא דרבי ודרבי יוסי בן כיפר לא קשיא הא כבית שמאי הא כבית הילל אלא לרב קשיא 
תנאי היא דתניא שנה האמורה בקדשים שנה האמורה בבתי ערי חומה שתי שנים שבשדה אחוזה
שש שנים שבעבד עברי וכן שבבן ושבבת כולן מעת לעת 
שנה האמורה בקדשים מנא לן אמר רב אחא בר יעקב אמר קרא (ויקרא יב, ו) כבש בן שנתו שנתו שלו ולא שנה של מנין עולם 
שנה האמורה בבתי ערי חומה מנלן אמר קרא (ויקרא כה, כט) עד תום שנת ממכרו ממכרו שלו ולא שנת של מנין עולם שתי שנים שבשדה אחוזה מנלן אמר קרא {ויקרא כה } במספר}

\newsection{דף מח}
\twocol{שני תבואות ימכר לך פעמים שאתה מוכר שלשה תבואות בשתי שנים 
שש שנה שבעבד עברי מנלן אמר קרא (שמות כא, ב) שש שנים יעבוד ובשביעית ובשביעית נמי יעבוד 
שבבן ושבבת למאי הלכתא אמר רב גידל אמר רב לענין ערכין ורב יוסף אמר לפרקין דיוצא דופן 
א"ל אביי מי פליגת א"ל לא הוא אמר חדא ואנא אמינא חדא ולא פליגינן 
והכי נמי מסתברא דאי ס"ד פליגי מאן דאמר לערכין לא אמר ליוצא דופן והאמר רב הלכתא בכולה פרקין מעת לעת 
אלא למ"ד לערכין מ"ט לא אמר ליוצא דופן דומיא דהנך מה הנך דכתיבן אף הני נמי דכתיבן 
ואידך האי שבבן ושבבת שבזכר ושבנקבה מבעי ליה 
אמר רב יצחק בר נחמני א"ר אלעזר הלכה כר' יוסי בן כיפר שאמר משום ר' אליעזר א"ר זירא אזכה ואיסק ואגמר לשמעתא מפומיה דמרא 
כי סליק אשכחיה לר' אלעזר אמר ליה אמרת הלכה כרבי יוסי בן כיפר אמר ליה מסתברא אמרי מדכוליה פירקין תני יום אחד והכא לא קתני שמע מינה מסתברא כותיה:
\par \par {\large\emph{הדרן עלך יוצא דופן}}\par \par 
מתני׳ {\large\emph{בא}} סימן התחתון עד שלא בא העליון או חולצת או מתיבמת
בא העליון עד שלא בא התחתון אף על פי שאי אפשר ר' מאיר אומר לא חולצת ולא מתיבמת
וחכ"א או חולצת או מתיבמת מפני שאמרו אפשר לתחתון לבא עד שלא בא העליון אבל אי אפשר לעליון לבא עד שלא בא התחתון
{\large\emph{גמ׳}} אע"פ שאי אפשר והלא בא בא לר' מאיר אע"פ שאי אפשר לרבנן 
ולתני בא העליון ר"מ אומר לא חולצת ולא מתיבמת וחכ"א או חולצת או מתיבמת ואנא ידענא משום דאי אפשר הוא 
אי לא תנא אע"פ שאי אפשר הוה אמינא רוב נשים תחתון אתי ברישא ומיעוט עליון אתי ברישא ורבי מאיר לטעמיה דחייש למיעוטא ורבנן לטעמייהו דלא חיישי למיעוטא 
והני מילי בסתמא אבל היכא דבדקן ולא אשכחן אימר מודו ליה רבנן לר"מ דעליון קדים
קמ"ל דאי אפשר ודאי אתי ומנתר הוא דנתר 
בשלמא לר"מ היינו דכתיב (יחזקאל טז, ז) שדים נכונו ושערך צמח אלא לרבנן איפכא מבעי ליה ה"ק כיון ששדים נכונו בידוע ששערך צמח 
בשלמא לר"מ היינו דכתיב (יחזקאל כג, כא) בעשות ממצרים דדיך למען שדי נעוריך אלא לרבנן איפכא מבעי ליה 
ה"ק כיון שבאו דדיך בידוע שבאו נעוריך ואיבעית אימא מאי שדי כולה בדדי כתיב וה"ק הקב"ה לישראל}

\newchap{פרק \hebrewnumeral{6} בא סימן}
\twocol{
\commenta{\textbf{בשלמא לפני הפרק בעיא בדיק' דאי משתכחי לאחר הפרק שומא נינהו.} שאלמלא שבדקו הנשים בתוך הזמן היינו אומרים לאחר זמן הביאו ולא קודם לכן שאורח בזמנו בא. וה"ה אפי' לתוך זמן שאין חוששין שמא לפני זמן הביא' אותן למ"ד כלאחר זמן דמי וחולצת היא ועכשיו הנשים נאמנות וקטנה היא שלא תחלוץ ואפילו לומר קטנה היא שתמאן אינו נאמנות מ"ט כיון דקידש בתוך זמן ובעל לאחר זמן ה"ל ספיקא דאורייתא אפילו היו שם ק' עדים ששערות הללו שומא הן חוששין שמא הביאה שערות לאתר זמן ונשרו ולעולם אינה ממאנת.\par ואי קשיא היכי ניחא לן השתא בדיקה דלפני הפרק משום דאי משתכחי לאחר הפרק שומא נינהו והא למאי דקס"ד השתא דתזקה דרבא בין למיאון בין לחליצה הוא לאחר הפרק אפילו תאמר שאלו שומא הן מ"מ גדולה היא דחזקה הביאה שתי שערות.\par לאו קושיא היא ואנן מחזא חזינא בהדיא בבריית דאית ליה בדיק' בלאחר הפרק ולפום הכי קאמרינן בשלמא בדיקה דלפני הפרק לפום מאי דקאמרת בברייתא מהניא ודאי דלא עבדין עובדא בהנך שערות אלא בדיקה דלאחר הפרק גופיה קשיא למה לי ומשום דלא בעינן לאקשויי אדיוקא מאי דקתני בהדיא קאמרינן הכי והשתא לא נחתינן למידק בנשרו כלום דקתני וסתמא פרכינן.\par ומתרצינן לכ"ע כדמתרץ במסקנא לעיל באידך פירקין ולפרושא לברייתא בעלמא אתינן השתא דאי דייקי בנשירה הוה יכול למימר בדיקה דלאחר הפרק להחזיקה בקטנה כשלא נמצאו בה שערות אלא פירש ברייתא כמסקנא דלעיל הכא ולישנא דקאמר בעיא בדיקה לאו דוקא דאנן לא צרכינן למיבדק קודם זמן שומא הן. אלא מהניא בדיקה דנשים להכי. וכן ברייתא דקתני נשים בודקות לאו דוקא אלא לומר שהן נאמנות אם בדקו וה"נ משמע בכל מקום בתלמוד שאין חוששין לשערות שנמצא לאחר זמן שמא הביאו אותן קודם זמן ושומא הן כדאמרי' בפ' מי שמת (דף קנד) דמעשה דבני ברק בתינוקות וכו' ובאו ושאלי לר"ע מהו שיבדקו וכו'. }
איכרפו דדיך לא הדרת בך אישתדו דדיך נמי לא הדרת בך 
\commenta{\textbf{וסיפא דקתני ונאמנת אשה להחמיר.} אוקימנא אב"א ר' יהודה ואתוך הפרק. וק"ל בשלמא נאמנת לומר גדולה היא שלא תמאן ואינה נאמנת לומר גדולה היא שתחלוץ ניחא אלא קטנה שלא תחלוץ פשיטא דנאמנות דאפילו שתקא א"נ אמרה גדולה היא אינה חולצת וקטנה היא שתמאן אמאי אינה נאמנת ואפילו נאמנת אמאי צריכה והא אמרת צריכה לומר גדולה היא שלא תמאן.\par ואיכא למימר כולה ברייתא נאמנת ואינה נאמנ' במקום שהוצרכנו לעדותה היא והכי קתני נאמנת לומר גדולה היא שלא תמאן במקום שאנו צריכין לעדות (גדול) שלה שאלמלא עדות אשה זו ממאנת היא שבחזקת קטנה עומדת ואפילו בדקו עדים עכשיו ולא ראו בה שערו' נאמנות אשה זו לומר הביאה אותן ואינה ממאנת שמא נשרו.\par וכן נאמנ' לומר קטנה היא שלא תחלוץ במקום שאנו צריכין לעדות קטנותה כגון שבדקנו אותה ומצינו בה שערות אם אמרה אשה לפני זמן הביאתן נאמנת והיינו בדיקה דלפני הפרק ואגב אחרינא נקט להא. א"נ שלא תאמר כשהיא עדיין לפני הפרק נאמנת דעדיין בחזקת קטנה היא אבל לאחר שהביאה שערות והיא בזמנה והוחזקה גדולה בפנינו שמא תאמר אינה נאמנת לומר קטנה הוא שתוך זמן היו בה. וקמ"ל.\par אבל אין אשה נאמנת לומר תוך זמן קטנה היא שתמאן אם הוצרכנו לעדות זו כגו' שנמצאו בה שערות והוא שבעל בתוך זמן דה"ל ספק דאורייתא ואע"ג דהכא ליכא למימר שמא נשרו דהא אכתי תוך זמן זה הוא מיהו כיון שהיא גדולה בפנינו אין האשה נאמנת להקל בשל תורה לומר שומא הן וכן אינה נאמנת לומר גדולה היא שתחלוץ.\par ולהך לישנא דאמרינן ואב"א ר' שמעון ולאחר הפרק ולית ליה חזקה דרבא וה"נ קתני נאמנת לומר גדול' היא שלא תמאן ואפילו אין בה עכשיו שערות וקטנה היא שלא תחלוץ אפילו היו בה אבל אין נאמנת לומר קטנה היא שתמאן כשהיו בה ובעל כדפרישית ולא לומר גדולה היא שתחלוץ.\par ומצינו נוסחא אחרת. "ונאמנת אשה להחמיר אבל לא להקל כיצד גדולה היא שתמאן גדולה היא שתחלוץ". וכן גרסת רבינו הגדול ז"ל בהלכות. ונוסתא ישרה היא ופירושא נאמנת להחמיר לומר גדולה היא לענין מיאון ואינה נאמנת להקל לומר גדולה היא לענין חליצה כלומר מיאון וחליצה היינו להחמיר ולהקל מיאון היינו להחמיר חליצה היינו להקל. }
דכולי עלמא מיהא אתחתון סמכינן מנלן אמר רב יהודה אמר רב וכן תנא דבי ר' ישמעאל אמר קרא (במדבר ה, ו) איש או אשה כי יעשו מכל חטאות האדם השוה הכתוב אשה לאיש לכל עונשין שבתורה מה איש בסימן אחד אף אשה בסימן אחד 
ואימא או האי או האי כאיש מה איש תחתון ולא עליון אף אשה תחתון ולא עליון 
תניא נמי הכי א"ר אליעזר בר' צדוק כך היו מפרשין ביבנה ואמרו כיון שבא תחתון שוב אין משגיחין על עליון 
תניא רשב"ג אומר בנות כרכים תחתון ממהר לבא מפני שרגילות במרחצאות בנות כפרים עליון ממהר לבא מפני שטוחנות ברחים 
ר"ש בן אלעזר אומר בנות עשירים צד ימין ממהר לבא שנישוף באפקריסותן בנות עניים צד שמאל ממהר לבא מפני ששואבות כדי מים עליהן ואיבעית אימא מפני שנושאין אחיהן על גססיהן 
ת"ר צד שמאל קודם לצד ימין רבי חנינא בן אחי ר' יהושע אומר מעולם לא קדם צד שמאל לצד ימין חוץ מאחת שהיתה בשכונתי שקדם צד שמאל לצד ימין וחזר לאיתנו 
ת"ר כל הנבדקות נבדקות על פי נשים וכן היה רבי אליעזר מוסר לאשתו ורבי ישמעאל מוסר לאמו 
רבי יהודה אומר לפני הפרק ולאחר הפרק נשים בודקות אותן תוך הפרק אין נשים בודקות אותן שאין משיאין ספקות על פי נשים ר"ש אומר אף תוך הפרק נשים בודקות אותן ונאמנת אשה להחמיר אבל לא להקל 
כיצד גדולה היא שלא תמאן קטנה היא שלא תחלוץ 
אבל אין נאמנת לומר קטנה היא שתמאן וגדולה היא שתחלוץ 
אמר מר רבי יהודה אומר לפני הפרק ולאחר הפרק נשים בודקות אותן בשלמא לפני הפרק בעי בדיקה דאי משתכחי לאחר הפרק שומא נינהו 
אלא לאחר הפרק למה לי בדיקה והאמר רבא קטנה שהגיעה לכלל שנותיה אינה צריכה בדיקה חזקה הביאה סימנין כי אמר רבא חזקה למיאון אבל לחליצה בעיא בדיקה 
תוך הפרק אין נשים בודקות אותן קסבר תוך הפרק כלאחר הפרק (דמי)
ולאחר הפרק דאיכא חזקה דרבא סמכינן אנשים ובדקי תוך הפרק דליכא חזקה דרבא לא סמכינן אנשים ולא בדקי נשים 
ר"ש אומר אף תוך הפרק נשים בודקות אותן קסבר תוך הפרק כלפני הפרק ובעיא בדיקה דאי משתכחי לאחר הפרק שומא נינהו 
ונאמנת אשה להחמיר אבל לא להקל האי מאן קתני לה איבעית אימא רבי יהודה ואתוך הפרק}

\newsection{דף מט}
\twocol{ואיבעית אימא רבי שמעון ולאחר הפרק ולית ליה חזקה דרבא
מפני שאמרו אפשר כו' הא תו למה לי הא תנא ליה רישא 
וכי תימא משום דקא בעי למסתמה כרבנן פשיטא יחיד ורבים הלכה כרבים 
מהו דתימא מסתברא טעמא דר"מ דקא מסייע ליה קראי קמ"ל ואיבעית אימא משום דקא בעי למתני כיוצא בו
{\large\emph{מתני׳}} כיוצא בו כל כלי חרס שהוא מכניס מוציא ויש שמוציא ואינו מכניס 
כל אבר שיש בו צפורן יש בו עצם ויש שיש בו עצם ואין בו צפורן 
כל המטמא מדרס מטמא טמא מת ויש שמטמא טמא מת ואינו מטמא מדרס
{\large\emph{גמ׳}} מכניס פסול למי חטאת ופסול משום גסטרא מוציא כשר למי חטאת ופסול משום גסטרא
אמר רב אסי שונין כלי חרס שיעורו בכונס משקה ולא אמרו מוציא משקה אלא לענין גסטרא בלבד מאי טעמא אמר מר זוטרא בריה דרב נחמן לפי שאין אומרים הבא גסטרא לגסטרא 
תנו רבנן כיצד בודקין כלי חרס לידע אם ניקב בכונס משקה אם לאו יביא עריבה מלאה מים ונותן קדרה לתוכה אם כנסה בידוע שכונס משקה ואם לאו בידוע שמוציא משקה
רבי יהודה אומר כופף אזני קדרה לתוכה ומציף עליה מים ואם כונס בידוע שכונס משקה ואם לאו בידוע שמוציא משקה 
או שופתה על גבי האור אם האור מעמידה בידוע שמוציא משקה ואם לאו בידוע שמכניס משקה 
ר' יוסי אומר אף לא שופתה על גבי האור מפני שהאור מעמידה אלא שופתה על גבי הרמץ אם רמץ מעמידה בידוע שמוציא משקה ואם לאו בידוע שכונס משקה היה טורד טיפה אחר טיפה בידוע שכונס משקה 
מאי איכא בין ת"ק לר' יהודה אמר עולא כינוס על ידי הדחק איכא בינייהו
כל אבר שיש בו צפורן וכו' יש בו צפורן מטמא במגע ובמשא ובאהל יש בו עצם ואין בו צפורן מטמא במגע ובמשא ואינו מטמא באהל 
אמר רב חסדא דבר זה רבינו הגדול אמרו המקום יהיה בעזרו אצבע יתרה שיש בו עצם ואין בו צפורן מטמא במגע ובמשא ואינו מטמא באהל 
אמר רבה בר בר חנה א"ר יוחנן וכשאינה נספרת על גב היד
כל המטמא מדרס וכו' כל דחזי למדרס מטמא טמא מת 
ויש שמטמא טמא מת ואין מטמא מדרס לאתויי מאי לאתויי סאה ותרקב
דתניא (ויקרא טו, ו) והיושב על הכלי יכול כפה סאה וישב עליה או תרקב וישב עליו יהא טמא
ת"ל (ויקרא טו, ו) אשר ישב עליו הזב מי שמיוחד לישיבה יצא זה שאומרים לו עמוד ונעשה מלאכתנו
{\large\emph{מתני׳}} כל הראוי לדון דיני נפשות ראוי לדון דיני ממונות ויש שראוי לדון דיני ממונות ואינו ראוי לדון דיני נפשות
{\large\emph{גמ׳}} אמר רב יהודה לאתויי ממזר 
תנינא חדא זימנא הכל כשרין לדון דיני ממונות ואין הכל כשרין לדון דיני נפשות והוינן בה לאתויי מאי ואמר רב יהודה לאתויי ממזר חדא לאתויי גר וחדא לאתויי ממזר 
וצריכי דאי אשמעינן גר משום דראוי לבא בקהל אבל ממזר דאין ראוי לבא בקהל אימא לא 
ואי אשמעינן ממזר משום דקאתי מטפה כשרה אבל גר דקאתי מטפה פסולה אימא לא צריכא
{\large\emph{מתני׳}} כל הכשר לדון כשר להעיד ויש שכשר להעיד ואינו כשר לדון
{\large\emph{גמ׳}} לאתויי מאי א"ר יוחנן לאתויי סומא באחת מעיניו ומני}

\newsection{דף נ}
\twocol{רבי מאיר היא דתניא היה רבי מאיר אומר מה ת"ל (דברים כא, ה) על פיהם יהיה כל ריב וכל נגע וכי מה ענין ריבים אצל נגעים מקיש ריבים לנגעים מה נגעים ביום דכתיב (ויקרא יג, יד) וביום הראות בו אף ריבים ביום 
ומה נגעים שלא בסומא דכתיב (ויקרא יג, יב) לכל מראה עיני הכהן אף ריבים שלא בסומא ומקיש נגעים לריבים מה ריבים שלא בקרובים אף נגעים שלא בקרובים 
אי מה ריבים בשלשה אף נגעים בשלשה ודין הוא ממונו בשלשה גופו לא כ"ש ת"ל (ויקרא יג, ב) והובא אל אהרן הכהן או אל אחד מבניו הכהנים הא למדת שאפילו כהן אחד רואה את הנגעים 
ההוא סמיא דהוה בשבבותיה דרבי יוחנן דהוה קדיין דינא ולא קאמר ליה ולא מידי היכי עביד הכי והאמר רבי יוחנן הלכה כסתם משנה
ותנן כל הכשר לדון כשר להעיד ויש כשר להעיד ואין כשר לדון ואמרינן לאתויי מאי ואמר רבי יוחנן לאתויי סומא באחת מעיניו 
רבי יוחנן סתמא אחרינא אשכח דתנן דיני ממונות דנין ביום וגומרין בלילה 
ומאי אולמיה דהאי סתמא מהאי סתמא איבעית אימא סתמא דרבים עדיף ואיבעית אימא משום דקתני לה גבי הלכתא דדיני
{\large\emph{מתני׳}} כל שחייב במעשרות מטמא טומאת אוכלין ויש שמטמא טומאת אוכלין ואינו חייב במעשרות
{\large\emph{גמ׳}} לאתויי מאי לאתויי בשר ודגים וביצים
{\large\emph{מתני׳}} כל שחייב בפאה חייב במעשרות ויש שחייב במעשרות ואינו חייב בפאה
{\large\emph{גמ׳}} לאתויי מאי לאתויי תאנה וירק שאינו חייב בפאה דתנן כלל אמרו בפאה כל שהוא אוכל ונשמר וגידולו מן הארץ ולקיטתו כאחד ומכניסו לקיום חייב בפאה 
אוכל למעוטי ספיחי סטים וקוצה ונשמר למעוטי הפקר וגידולו מן הארץ למעוטי כמהים ופטריות ולקיטתו כאחד למעוטי תאנה ומכניסו לקיום למעוטי ירק 
ואילו גבי מעשר תנן כל שהוא אוכל ונשמר וגידולו מן הארץ חייב במעשרות ואילו לקיטתו כאחד ומכניסו לקיום לא קתני 
אם היו בהם שומים ובצלין חייבין דתנן מלבנות בצלים שבין הירק ר' יוסי אומר פאה מכל אחת ואחת וחכ"א מאחת על הכל 
אמר רבה בר בר חנה א"ר יוחנן עולשין שזרען מתחילה לבהמה ונמלך עליהן לאדם
צריכות מחשבה לכשיתלשו קסבר מחשבת חבור לא שמה מחשבה 
אמר רבא אף אנן נמי תנינא י"ג דברים נאמרו בנבלת עוף טהור וזה אחד מהן צריכה מחשבה ואינה צריכה הכשר אלמא מחשבת חיים לא שמה מחשבה הכא נמי מחשבת חבור לא שמה מחשבה 
רבי זירא אמר הכא בגוזל שנפל מן הרום עסקינן דלא הוה קמן דלחשוב עליה 
א"ל אביי תרנגולת שביבנה מאי איכא למימר א"ל תרנגול ברא הוה 
אחיכו עליה תרנגול ברא עוף טמא הוא ועוף טמא מי קמטמא אמר להו אביי גברא רבה אמר מילתא לא תחיכו עליה בתרנגולת שמרדה ומאי ברא דאיבראי ממרה 
רב פפא אמר תרנגולתא דאגמא הואי רב פפא לטעמיה דאמר רב פפא תרנגול דאגמא אסור תרנגולתא דאגמא שריא
וסימניך עמוני ולא עמונית דרש מרימר תרנגולתא דאגמא אסירא חזיוה רבנן דדרסה ואכלה והיינו גירותא 
ת"ר גוזל שנפל לגת וחשב עליו להעלותו לכותי טמא לכלב טהור ר' יוחנן בן נורי אומר אף לכלב טמא 
א"ר יוחנן בן נורי ק"ו אם מטמא טומאה חמורה שלא במחשבה לא יטמא טומאה קלה שלא במחשבה 
אמרו לו לא אם אמרת בטומאה חמורה שכן אינה יורדת לכך תאמר בטומאה קלה שכן יורדת לכך 
אמר להן תרנגולת שביבנה תוכיח שיורדת לכך וטמאוה שלא במחשבה אמרו לו משם ראיה כותים היו שם וחשבו עליה לאכילה 
במאי עסקינן אילימא בכרכים למה לה מחשבה והתנן נבלת בהמה טהורה בכל מקום ונבלת עוף טהור והחלב בכרכים אין צריכין לא מחשבה ולא הכשר 
אלא בכפרים ומי איכא למ"ד דלא בעיא מחשבה והתנן נבלת בהמה טמאה בכל מקום ונבלת עוף טהור בכפרים צריכה מחשבה ואינה צריכה הכשר 
א"ר זעירא בר חנינא לעולם בכרך וגתו מאסתו ועשאתו ככפר 
א"ר יוחנן בן נורי קל וחומר אם מטמאה טומאה חמורה שלא במחשבה לא תטמא טומאה קלה שלא במחשבה 
אמרו לו לא אם אמרת בטומאה חמורה שכן אינה יורדת לכך 
מאי אינה יורדת לכך אמר רבא הכי קאמרי ליה לא אם אמרת}

\newsection{דף נא}
\twocol{בטומאה חמורה שכן אינה עושה כיוצא בה תאמר בטומאה קלה שעושה כיוצא בה 
אמר ליה אביי כל דכן הוא ומה טומאה חמורה דקילא דאינה עושה כיוצא בה מטמאה שלא במחשבה טומאה קלה דחמירא דעושה כיוצא בה אינו דין שמטמאה שלא במחשבה 
אלא אמר רב ששת הכי קאמר לא אם אמרת בטומאה חמורה שכן אינה צריכה הכשר תאמר בטומאה קלה שצריכה הכשר 
ומי צריכה הכשר והתנן שלשה דברים נאמרו בנבלת עוף טהור צריכה מחשבה ואינה מטמאה אלא בבית הבליעה ואינה צריכה הכשר 
נהי דהכשר שרץ לא בעיא הכשר מים בעיא 
מאי שנא הכשר שרץ דלא בעיא כדתנא דבי רבי ישמעאל הכשר מים נמי לא תבעי כדתנא דבי רבי ישמעאל 
דתנא דבי רבי ישמעאל (ויקרא יא, לז) על כל זרע זרוע אשר יזרע
מה זרעים שאין סופן לטמא טומאה חמורה צריכין הכשר אף כל שאין סופן לטמא טומאה חמורה צריכין הכשר יצתה נבלת עוף טהור שסופה לטמא טומאה חמורה שאין צריך הכשר 
אלא אמר רבא ואיתימא רב פפא שום טומאה חמורה בעולם שום טומאה קלה בעולם 
אמר רבא ומודה רבי יוחנן לענין מעשר דמחשבת חיבור שמה מחשבה אמר רבא מנא אמינא לה דתנן הסיאה והאזוב והקורנית שבחצר אם היו נשמרין חייבין 
היכי דמי אילימא דזרעינהו מתחלה לאדם צריכא למימר אלא לאו דזרעינהו מתחלה לבהמה וקתני אם היו נשמרין חייבין 
אמר רב אשי הכא בחצר שעלו מאיליהן עסקינן וסתמא לאדם קיימי והכי קאמר אם החצר משמרת פירותיה חייבין ואם לאו פטורין 
מתיב רב אשי כל שחייבין במעשרות מטמאין טומאת אוכלין ואם איתא הא איכא הני דקחייבין במעשר ואין מטמאין טומאת אוכלין 
אמר רבא הכי קאמר כל מין שחייב במעשר מטמא טומאת אוכלין 
ה"נ מסתברא מדקתני סיפא כל שחייב בראשית הגז חייב במתנות ויש שחייב במתנות ואין חייב בראשית הגז 
ואם איתא האיכא טרפה דחייבת בראשית הגז ואינה חייבת במתנות 
אמר רבינא הא מני רבי שמעון היא דתנן ר"ש פוטר את הטרפה מראשית הגז 
אמר רב שימי בר אשי תא שמע המפקיר את כרמו והשכים בבקר ובצרו חייב בפרט ובעוללות ובשכחה ובפאה ופטור מן המעשר 
והא אנן תנן כל שחייב בפאה חייב במעשרות אלא לאו שמע מינה מין קתני שמע מינה 
תנן התם מודים חכמים לר' עקיבא בזורע שבת או חרדל בשנים ושלשה מקומות שנותן פאה מכל אחד ואחד
והא שבת דמיחייב בפאה ומיחייב במעשר דתנן כל שחייב בפאה חייב במעשר 
\commenta{ הא דאמרי' \textbf{לבני מערבא דמברכין בתר דסליקו תפילייהו לשמור חקיו.} פי' ר"ת ז"ל בספר הישר שלו שלא אמרו אלא בתפילין אבל בציצית ושאר מצות אין מברכין לאחר עשייתן.\par והביא ראיה ממה שאמרו בירושלמי בפ' היה קורא בתורה כיצד הוא מברך עליהן ר' זירקן בשם ר' יעקב בר אידי כשהוא נותן של יד מהו אומר בא"י אמ"ה על מצות תפילין וכשהוא נותן לראש מהו אומר אקב"ו על הנחת תפילין. וכשהוא חולצן מהו אומר ברוך וכו' לשמור חקיו ואתיא כמ"ד בחוקת תפילין הכתוב מדבר ברם כמ"ד בחוקת הפסח הכתוב מדבר לא כר"א [{\small לפנינו שם } לא בדא {\small ואם הגירסא נכונה } כר"א קאי על למטה ע"ש] והטעם לזה מפני שמניח תפילין לאחר שקיעת התמה עובר בעשה הילכך מברך בשעת סילוקן בלילה שהוא מקיים עשה ואין לך כן בכל המצוות, כך פי' חכמי הצרפתים בשמו ז"ל.\par ועדיין אינו מחוור, דא"כ הא דאמרינן בשמעתין לאתויי מצות ומקשי ולבני מערבא דמברכין בתר דמסלקי תפילייהו מאי איכא למימר מאי קושי' מתני' לאתויי כל שאר המצות. ועוד יש נסחאות שכחוב בהן ולבני מערבא דמברכי אמצות וכו'.\par אלא נראה לבני מערבא ה"ה לכל מצות שטעונו' ברכה לאחריהן וז"ש בירושלמי אתיא כמ"ד בחוקי התפילין לא הקפידו אלא על הלשון דלשמור חקיו אבל שאר כל המצות אין מברכין אלא לשמור מצותיו ובודאי נראה לומר שאין בני מערבא מברכין אלא כשהן מסלקין אותן בזמן ערבית ולא משום עשה שבהן אלא משום שכבר נגמרה מצותן דקסברי לילה לאו זמן תפילין הוא וא"כ סלקו אותן בע"ש ובערבי י"ט לד"ה מברכין היו אבל אם היו מסלקין ביום היאך יברך הלא מצוה להניחן ולא לסלקן. ולפיכך אמרו בירושלמי דאתיא כמ"ד בחוקת תפילין הכתוב מדבר וכתיב מימים ימימה ולא לילות וכך סמכו שם בירושלמי ר' אבהו בשם ר' אלעזר הנותן תפילין בלילה עובר\par בעשה מה טעם ושמרת וכו'. אבל נאמר לפי"ז הענין ולפ"ז הפי' שאין מברכין על כל מצוה שאין סילוקה גמר עשייתה כגון פושט ציצית ביום והיוצא מן הסוכה אבל בלילה מברכין על ציצית וכן לאחר שופר ולולב וכל כיוצא בהן שעשייתן גמר מלאכתן מברכין וזה שלא העמידו משנתינו דיש טעון במצות כיוצא באלו שאינן טעונות ברכה מפני שאין לשין לאחריו אלא לאחר שנגמר המעשה.\par וזה הלשון נכון הוא שאין הדין נותן לברך לאחרי' במצוה שעדיין הוא חייב בה והוא מסלקה ממנו שא"כ מצינו חוטא ומברך ואין לך כן אלא בקורא בתורה ובצבור מפני שהוא מצוה לגמור כדי שיהיו ג' או ז' קוראים כתקנת חכמים. אבל בגמר מצוה בכל מצוה נגמרת מברכין היו ודמיא להו להלל ומגלה ותורה בצבור וראינו לרבינו האי גאון ז"ל שכתב בהא דבני מערבא לא נהגינן הכי במתיבתא ומיהו אי בעי אינש למיעבד כבני מערבא שפיר דמי.\par ולשון הירושלמי שכתבנו נראה שמכריע כדברי בעל הלכות ז"ל שהצריך לברך א' של יד וא' של ראש אף על פי שלא שח. וכן החזירו שם הענין הזה בפרק הרואה ואמרו העושה תפילין לעצמו אומר בא"י אמ"ה לעשות תפילין לשמו כשהוא לובשן אומר בא"י אמ"ה על מצות תפילין וכשהוא מניחן אומר אקב"ו על הנחת תפילין בכל מקום מזכירין כן אע"פ שלא שח ולא כדברי רבי' הגדול ז"ל שפירש לא שח מברך א' בלבד על שתיהן.\par אלא שיש לנו פתחון פה לומר דגמרא ירושלמי ס"ל כדקס"ד מעיקרא בגמרא דילן אבל במסקנא אסיקו אביי ורבא לא שח מברך א' ואנן כמסקנא דגמ' דילן עבדינן או שענין הירושלמי במניח א' מהן ולא במניח שתיהן. }
ומדחייב במעשר מטמא טומאת אוכלין אלמא כל מילי דעביד לטעמא מטמא טומאת אוכלין דהאי שבת לטעמא עבידא 
ורמינהי הקושט והחימום וראשי בשמים והתיאה והחלתית והפלפלים וחלת חריע נקחין בכסף מעשר ואין מטמאין טומאת אוכלין דברי רבי עקיבא 
אמר לו רבי יוחנן בן נורי אם נקחין בכסף מעשר מפני מה אין מטמאין טומאת אוכלין ואם אינן מטמאין אף הם לא ילקחו בכסף מעשר 
וא"ר יוחנן בן נורי נמנו וגמרו שאין נקחין בכסף מעשר ואין מטמאין טומאת אוכלין 
אמר רב חסדא כי תניא ההיא בשבת העשויה לכמך 
אמר רב אשי אמריתה לשמעתי' קמיה דרב כהנא (אמר) לא תימא בשבת העשויה לכמך הא סתמא לקדרה אלא סתם שבת לכמך עשויה דתנן השבת משנתנה טעם בקדרה אין בה משום תרומה ואינה מטמאה טומאת אוכלין 
הא עד שלא נתנה טעם בקדרה יש בה משום תרומה ומטמאה טומאת אוכלין ואי ס"ד סתמא לקדרה כי לא נתנה נמי סתמא לקדרה אלא לאו ש"מ סתמא לכמך עשויה ש"מ
{\large\emph{מתני׳}} כל שחייב בראשית הגז חייב במתנות ויש שחייב במתנות ואינו חייב בראשית הגז
כל שיש לו ביעור יש לו שביעית ויש שיש לו שביעית ואין לו ביעור
{\large\emph{גמ׳}} כגון עלה הלוף שוטה והדנדנה יש שיש לו שביעית ואין לו ביעור עיקר הלוף שוטה ועיקר הדנדנה
דכתיב (ויקרא כה, ז) ולבהמתך ולחיה אשר בארצך תהיה כל תבואתה לאכול כל זמן שחיה אוכלת מן השדה אתה מאכיל לבהמתך בבית כלה לחיה מן השדה כלה לבהמתך שבבית והני לא כלו להו
{\large\emph{מתני׳}} כל שיש לו קשקשת יש לו סנפיר ויש שיש לו סנפיר ואין לו קשקשת כל שיש לו קרנים יש לו טלפים ויש שיש לו טלפים ואין לו קרנים
{\large\emph{גמ׳}} כל שיש לו קשקשת דג טהור יש שיש לו סנפיר ואין לו קשקשת דג טמא מכדי אנן אקשקשת סמכינן סנפיר דכתב רחמנא למה לי 
אי לא כתב רחמנא סנפיר הוה אמינא מאי קשקשת דכתיב סנפיר ואפילו דג טמא כתב רחמנא סנפיר וקשקשת 
והשתא דכתב רחמנא סנפיר וקשקשת מנלן דקשקשת לבושא הוא דכתיב (שמואל א יז, ה) ושריון קשקשים הוא לבוש 
ולכתוב רחמנא קשקשת ולא בעי סנפיר א"ר אבהו וכן תנא דבי רבי ישמעאל (ישעיהו מב, כא) יגדיל תורה ויאדיר
{\large\emph{מתני׳}} כל הטעון ברכה לאחריו טעון ברכה לפניו ויש שטעון ברכה לפניו ואין טעון ברכה לאחריו
{\large\emph{גמ׳}} לאתויי מאי לאתויי ירק ולרבי יצחק דמברך אירק לאתויי מאי לאתויי מיא 
ולרב פפא דמברך אמיא לאתויי מאי לאתויי מצות ולבני מערבא דמברכי בתר דסליקו תפילייהו אשר קדשנו במצותיו וצונו לשמור חוקיו לאתויי מאי לאתויי}

\newsection{דף נב}
\twocol{ריחני
\commenta{והא ד\textbf{אמר ר' ישמעאל ויש לך אחרת שאפילו לא נתפסה מותרת ואיזו זו שקידושי' קידושי טעות.} פירש רש"י ז"ל כגון ע"מ שאני כהן והרי הוא ישראל וכגון קטנה שאין מעשיה כלום והכי ודאי פשטה דשמעתא דאמרי' דבר שאמר אותו צדיק יכשל בו זרעו. וא"כ לר' ישמעאל מצינו חמות ממאנת ושמואל אי סבר ליה כרביה ההוא דאיתמר בכתובות (דף ע"ג) קטנה שלא מיאנה והגדילה ועמדה ונשאת רב אמר אינה צריכה גט משני ושמואל אמר צריכה גט משני ולא מראשון קאמר.\par וההיא דאמרינן בפרק מי שמת (דף קנ"ו ע"א) אמר ר' נחמן אמר שמואל בודקין לקידושין ולגיטין ולחליצ' ולמיאונין ואיתמר עלה למיאוניו לאפוקי מדר' יהודה וכו' דאלמא משהביאה שתי שערות אינה ממאנת ההיא דלא כר' ישמעאל. אי משום דהא דידיה והא דרביה. אי משום דאמוראי נינהו אליבא דשמואל.\par וי"מ דוקא קידושי טעות אבל בממאנת בקדושי קטנות לא א"ר ישמעאל ופי' ממאנת והולכת לה לומר שאם לא רצתה בבעל הולכת לה ולא שתהא צריכה גט מיאון אלא מעשה כשהיה בבתו שנכנסה למאן היו סבורין לומר כשם שלר' ישמעאל ממאנת בקדושי תנאי ולא אמרינן אין אדם עושה בעילתו זנות ולשום קדושין בעל כך בקטנה שהגדילה ונמנו וגמרו שאפילו לר' ישמעאל בקדושי קטנות אינה ממאנת אלא עד שתביא שתי שערות ומיהו דרבנן הוא מפני שנראית כגדולה שנתקדשה או מפני שלא מיחת בקידושי דרבנן שעה ראשונה שוב אינה יכולה למחות מדבריהם. וכן משמע בפרק נושאין על האנוסה (יבמות ק, ב) כלשון הזה וכבר פירשתיה ביבמות בפרק ב"ש. }
{\large\emph{מתני׳}} תינוקת שהביאה שתי שערות או חולצת או מתיבמת וחייבת בכל מצות האמורות בתורה
\commenta{אע"ג דקיימא לן כרבנן \textbf{עד שיהו שתי שערות במקום אחד.} מיהו שתים על גבי קשרי אצבעותי' של יד ושתים ע"ג קשרי אצבעותיה של רגל גדולה היא דלא פליגי רבנן עליה דר"ש בהאי.\par ותמהני על הרב רמב"ם פאסי ז"ל שכתב ב' שערות אלו צריכים שיהיו במקום הערוה ובשמעתין משמע אפילו על יד ורגל או בגבה. וי"מ גבה וכריסה במקום ערוה וכריסה למעלה עד מקום ערוה וכן שמעתי בשם ר"ת מיהו ביד ורגל סגי. }
וכן תינוק שהביא שתי שערות חייב בכל מצות האמורות בתורה וראוי להיות בן סורר ומורה משיביא שתי שערות עד שיקיף זקן 
התחתון ולא העליון אלא שדברו חכמים בלשון נקיה 
תינוקת שהביאה שתי שערות אינה יכולה למאן רבי יהודה אומר עד שירבה השחור
{\large\emph{גמ׳}} וכיון דתנן חייבת בכל מצות האמורות בתורה או חולצת או מתיבמת למה לי
לאפוקי מדרבי יוסי דאמר איש כתוב בפרשה אבל אשה בין גדולה ובין קטנה קמ"ל דאי אייתי שתי שערות אין אי לא לא מאי טעמא אשה כאיש
וכיון דתנא וכן התינוק שהביא ב' שערות חייב בכל המצות האמורות בתורה ל"ל 
וכי תימא משום דקבעי למתני וראוי להיות בן סורר ומורה תנינא חדא זימנא אימתי הוא בן סורר ומורה משיביא שתי שערות ועד שיקיף זקן התחתון ולא העליון אלא שדברו חכמים בלשון נקיה 
אין ה"נ אלא איידי דפריש מילי דתינוקת קמפרש נמי מילי דתינוק
תינוקת שהביאה כו' א"ר אבהו א"ר אלעזר הלכה כרבי יהודה 
ומודה רבי יהודה שאם נבעלה לאחר שהביאה שתי שערות שוב אינה יכולה למאן 
חברוהי דרב כהנא סבור למעבד עובדא כרבי יהודה ואע"ג דנבעלה 
אמר להו רב כהנא לא כך היה מעשה בבתו של רבי ישמעאל שבאת לבית המדרש למאן ובנה מורכב לה על כתפה ואותו היום הוזכרו דבריו של רבי ישמעאל בבית המדרש ובכתה בכייה גדולה בבית המדרש 
אמרו דבר שאמר אותו צדיק יכשל בו זרעו 
דאמר רב יהודה אמר שמואל משום רבי ישמעאל (במדבר ה, יג) והיא לא נתפשה אסורה הא נתפשה מותרת ויש לך אחרת שאע"פ שלא נתפשה מותרת ואיזו זו שקדושיה קדושי טעות שאע"פ שבנה מורכב על כתפה ממאנת והולכת לה 
ונמנו וגמרו עד מתי הבת ממאנת עד שתביא שתי שערות פרוש ולא עבוד עובדא 
רבי יצחק ותלמידי דרבי חנינא עבוד עובדא כרבי יהודה ואע"ג דנבעלה אזל רב שמן בר אבא אמרה קמיה דר' יוחנן אזל רבי יוחנן אמרה קמיה דרבי יהודה נשיאה שדר בלשא ואפקוה 
אמר רב חסדא אמר מר עוקבא לא שירבה השחור ממש אלא כדי שיהיו שתי שערות שוכבות ונראות כמי שירבה השחור על הלבן רבא אמר שתי שערות המקיפות משפה לשפה 
א"ר חלבו אמר רב הונא שתי שערות שאמרו צריך שיהא בעיקרן גומות רב מלכיו אמר רב אדא בר אהבה גומות אע"פ שאין שערות 
אמר רב חנינא בריה דרב איקא שפוד שפחות וגומות רב מלכיו בלורית אפר מקלה וגבינה רב מלכיא 
רב פפא אמר מתני' ומתניתא רב מלכיא שמעתתא רב מלכיו וסימנא מתניתא מלכתא 
מאי בינייהו איכא בינייהו שפחות 
אמר רב אשי אמר לי מר זוטרא קשה בה רבי חנינא מסורא לא לישתמיט תנא ואשמועי' גומות אי אשמועינן גומות ה"א עד שיהו שתי שערות בשתי גומות קמ"ל דאפילו שתי שערות בגומא אחת 
ומי איכא כה"ג והכתיב (איוב ט, יז) אשר בשערה ישופני והרבה פצעי חנם ואמר רבא איוב בסערה חירף בסערה השיבוהו בסערה חירף אמר לפניו רבש"ע שמא רוח סערה עברה לפניך ונתחלפה לך בין איוב לאויב בסערה השיבוהו {איוב לח } ויען ה' את
איוב מן הסערה ויאמר אליו שוטה שבעולם הרבה נימין בראתי בראשו של אדם ולכל נימא ונימא בראתי לו גומא בפני עצמה שלא יהיו שתים יונקות מגומא אחת שאלמלא שתים יונקות מגומא אחת מכחיש מאור עיניו של אדם גומא בגומא לא נתחלף לי איוב באויב נתחלף לי 
לא קשיא הא בגופא הא ברישא 
אמר רב יהודה אמר שמואל שתי שערות שאמרו אפילו אחת על הכף ואחת על הביצים 
תניא נמי הכי שתי שערות שאמרו אפילו אחת בגבה ואחת בכריסה אחת ע"ג קשרי אצבעותיה של יד ואחת ע"ג קשרי אצבעותיה של רגל דברי ר' שמעון בן יהודה איש כפר עכו שאמר משום רבי שמעון ורבנן אמר רב חסדא עד שיהו ב' שערות במקום אחד 
ת"ר עד מתי הבת ממאנת עד שתביא שתי שערות דברי רבי מאיר ר' יהודה אומר עד שירבה השחור רבי יוסי אומר עד שתקיף העטרה בן שלקות אומר עד שתכלכל 
ואמר רבי שמעון מצאני חנינא בן חכינאי בצידן ואמר כשאתה מגיע אצל ר"ע אמור לו עד מתי הבת ממאנת אם יאמר לך עד שתביא שתי שערות אמור לו והלא בן שלקות העיד במעמד כולכם ביבנה עד שתכלכל ולא אמרתם לו דבר 
כשבאתי אצל רבי עקיבא אמר לי כלכול זה איני יודע מהו בן שלקות איני מכיר עד מתי הבת ממאנת עד שתביא ב' שערות
{\large\emph{מתני׳}} שתי שערות האמורות בפרה ובנגעים והאמורות בכל מקום כדי לכוף ראשן לעיקרן דברי רבי ישמעאל ר"א אומר כדי לקרוץ בציפורן ר' עקיבא אומר כדי שיהו ניטלות בזוג
{\large\emph{גמ׳}} אמר רב חסדא אמר מר עוקבא הלכה כדברי כולן להחמיר
{\large\emph{מתני׳}} הרואה כתם הרי זו מקולקלת 
וחוששת משום זוב דברי רבי מאיר וחכ"א אין בכתמים משום זוב
{\large\emph{גמ׳}} מאן חכמים ר' חנינא בן אנטיגנוס היא דתניא ר"ח בן אנטיגנוס אומר כתמים אין בהן משום זוב ופעמים שהכתמים מביאין לידי זיבה 
כיצד לבשה ג' חלוקות הבדוקות לה ומצאה עליהם כתם או שראתה ב' ימים וחלוק אחד הן הן הכתמים המביאין לידי זיבה 
השתא שלשה חלוקות דלאו מגופה קחזיא חיישינן ב' ימים וחלוק אחד מיבעיא 
מהו דתימא כל כי האי גוונא מביאה קרבן ונאכל קא משמע לן 
אמר רבא בהא זכנהו ר' חנינא בן אנטיגנוס לרבנן מאי שנא פחות מג' גריסין במקום אחד דלא חיישינן דאמרי' בתרי יומי חזיתיה שלשה גריסין במקום אחד נמי נימא תרתי ופלגא מגופה חזיתיה ואידך אגב זוהמא דם מאכולת הוא 
ורבנן כיון דאיכא לפלוגי בגריס ועוד לכל יומא לא תלינן 
ור"ח בן אנטיגנוס ג' גריסין במקום א' הוא דלא חיישינן הא בג' מקומות חיישינן הא אמרת בג' חלוקות אין בג' מקומות לא 
לדבריהם דרבנן קאמר להו לדידי בג' חלוקות אין בג' מקומות לא אלא לדידכו אודו לי מיהת דהיכא דחזאי ג' גריסין במקום אחד דאמרינן תרי ופלגא מגופה חזיתיה ואידך אגב זוהמא דם מאכולת הוא 
ורבנן כיון דאיכא לפלוגי בגריס ועוד לכל יומא לא תלינן 
ת"ר הרואה כתם אם יש בו כדי לחלק ג' גריסין שהן כגריס ועוד חוששת ואם לאו אינה חוששת 
ר' יהודה בן אגרא אומר משום רבי יוסי אחת זו ואחת זו חוששת}

\newsection{דף נג}
\twocol{אמר רבי נראין דברי רבי יהודה בן אגרא בשלא בדקה ודברי חכמים בשבדקה 
מאי בדקה ומאי לא בדקה אמר רבא אשכחתינהו לרבנן דבי רב דיתבי וקאמרי הכא במאי עסקינן כגון שבדקה עצמה ולא בדקה חלוקה ואף עצמה לא בדקה אלא בין השמשות דרבי יהודה ובבין השמשות דר' יוסי לא בדקה 
דרבנן סברי בבין השמשות דרבי יוסי ליליא הוא והא בדקה בבין השמשות דרבי יהודה ור' יוסי לטעמיה דאמר בין השמשות ספיקא הוי 
ואמינא להו אנא אילמלי ידיה בעיניה כל בין השמשות יפה אתם אומרים עכשיו שמא עם סלוק ידיה ראתה ואמרו לי כי קאמרינן כשנתנה ידיה בעיניה כל בין השמשות 
אמר רבי נראין דברי רבי יהודה בן אגרא כשלא בדקה מאי לא בדקה 
אילימא דבדקה בדרבי יהודה ולא בדקה בדרבי יוסי מכלל דרבי יהודה סבר אע"ג דבדקה בתרוייהו חיישא הא בדקה 
אלא פשיטא דלא בדקה לא בדרבי יהודה ולא בדרבי יוסי אבל בדקה בדר' יהודה ולא בדקה בדר' יוסי לא חיישא 
אלמא בין השמשות דר' יוסי לרבי ליליא הוא אימא סיפא ודברי חכמים כשבדקה מאי בדקה 
אילימא דבדקה בדרבי יהודה ולא בדקה בדרבי יוסי מכלל דרבנן סברי אע"ג דלא בדקה בתרוייהו לא חיישינן הא לא בדקה 
אלא פשיטא דבדקה בין בדר' יהודה ובין בדרבי יוסי אבל בדקה בדר' יהודה ולא בדקה בדר' יוסי חיישינן 
אלמא בין השמשות דרבי יוסי לרבי ספקא הוי קשיא דרבי אדרבי 
ה"ק נראין דברי רבי יהודה בן אגרא לרבנן דלא בדקה כלל לא בדרבי יהודה ולא בדרבי יוסי שאף חכמים לא נחלקו עליו אלא דבדקה בדר' יהודה ולא בדקה בדר' יוסי אבל היכא דלא בדקה כלל מודו ליה 
ורמינהו הרואה כתם לראיה מרובה חוששת לראיה מועטת אינה חוששת זו דברי רבי יהודה בן אגרא שאמר משום רבי יוסי 
אמר רבי אני שמעתי ממנו שאחת זו ואחת זו חוששת ומן הטעם הזה אמר לי ומה אילו נדה שלא הפרישה בטהרה מן המנחה ולמעלה לא תהא בחזקת טמאה ונראין דבריו כשבדקה 
מאי בדקה אילימא דבדקה בדר' יהודה ולא בדקה בדרבי יוסי מכלל דרבי יהודה בן אגרא סבר אע"ג דלא בדקה לא בדר' יהודה ולא בדר' יוסי לא חיישא והא לא בדקה 
אלא פשיטא דבדקה בין בדר' יהודה ובין בדרבי יוסי מכלל דרבי יהודה בן אגרא סבר בדקה בדר' יהודה ולא בדקה בדר' יוסי לא חיישא
אלמא בין השמשות דרבי יוסי לר' יהודה בן אגרא ליליא הוא קשיא דרבי יהודה בן אגרא אדר' יהודה בן אגרא 
בשלמא בלא רבי לא קשיא התם דבדקה בדר' יהודה ולא בדקה בדר' יוסי הכא דבדקה נמי בדר' יהודה ובדר' יוסי אלא בדרבי קשיא 
תרי תנאי ואליבא דרבי יהודה בן אגרא האי תנא סבר שלים בין השמשות דר' יהודה
והדר חייל בין השמשות דר' יוסי והאי תנא סבר בין השמשות דר' יוסי מישך שייך בדר' יהודה 
\commenta{\textbf{שהוא מתקן הלכותיה לידי זיבה.} פ' רש"י ז"ל לענין זיבה הוא מיקל לדידיה היכא דלא חזאי ביום לא תלינן כתמה בראיתה ומונה ימי נדה מיום ראיתה ואין ימי זיבה מתחילין עד יום ח' לראיתה ולרבי מונה מיום מציאת כתמה ואף להקל ולטבול לילי ז' לכתמה אם פסקה ומיום ח' לכתמה אמרינן יום זוב הוא ונמצא רבי מחמיר לענין זיבה דכי חזיא בח' לכתמה אמרינן יום זיבה הוא וצריכה לשמור יום כנגד יום ולרשב"א סוף נדה הוא ואין צריך שימור ולא נראה דהא רשב"א כיון דלא תלינן כתמה בראיתה מקולקל' היא לכתמה אמרינן.\par ול"א פי' בה שהוא מתקן הלכותיה לידי זיבה שהוא מחמיר וחושש לכתם משום זוב בג' גריסין ועוד אי נמי שאם ראתה שנים והוא צריכה שימור. וכן בכל שלשה רצופים שתראה חוששת לזיבה וצריכה נקיים נמצא שהוא מתקנה ומוציאה מכל ספק זיבה ואני מעותה שאיני מוציאה מידי ספק כלומר נראין ומטין כדברי המחמיר.\par ואף לשון זה אינו עולה דלמה לידי זיבה לכל דבר הוא מחמי' שהרי רבי מטהר' ליום ששי לראי' ולרשב"א ליום שביעי. ועוד ק"ל כיון דקי"ל (נט, א) כתמים דרבנן ובראית כתמה אינה מטמאה היאך רבי מונה לה משעת כתמה והלא ביום ראיתה היא תחלת נדה וממנו ראוי למנו' דבר תורה. וכדאמרינן בפרק קמא (ו, א) ברואה כתם ומקולקל' למנינה ואינה מונה אלא משעת שראתה.\par לפיכך נ"ל שלא תלה רבי אלא כתמה בראיתה אבל ראיתה בכתמה לא, כתמה בראיתה לומר שאינה מטמאה עצמה וקדשים למפרע ואינה מקלקלת למנינה מיום לבישת החלוק אבל מכל מקום עיקר מנין נדה וזיבה מיום ראיה בדין תורה וחוששת נמי ליום. מציאת כתמה כדין דבריהם.\par לפיכך אמרו שהוא מתקן הלכותיה לידי זיבה כלומר שאינה תולה כתם בראיה אלא במקום שאין חילוק ספיר' זיבה ביניהם כך דהיינו אותו יום שמנין ימי נדה וזבה אחד הוא בין לכתם בין לראיה ונמצאו כל הספירו' ראויו' כדין תורה משעת ראיה וכשהוא מעת לעת הוא רואה אינו תולה ונמצא' מקולקל' לכתם ומונה משעת ראיה נמצא כשהוא אומר תולה מתוקנת לגמרי. וכשהוא אומר אינו תולה היא מקולקל' לגמרי.\par אבל רבי אפי' בשעה שהוא תולה כתמה בראיתה הוא מעותה לידי זיבה שהרי אסורה לשמש עד יום ז' לראיה שהוא ח' לכתמה. ואם ראתה בו ביום חוששת לזיבה בודאי נמצא לרבי שאפילו בשעת תקונה כלומר שהוא תולה הוא מעותה שתולה כתמה בראיתה ואינו תולה ראיתה בכתמה ולא השוה מדותיו כנ"ל וסליק שפיר. }
ת"ר הרואה כתם מטמאה עצמה וקדשים למפרע דברי רבי 
ר"ש בן אלעזר אומר קדשים מטמאה עצמה אינה מטמאה שלא יהא כתמה חמור מראייתה 
והא מצינו כתמה חמור מראייתה לענין קדשים 
אלא תני הכי ר"ש בן אלעזר אומר אף קדשים אינה מטמאה שלא יהא כתמה חמור מראייתה לכל דבר 
ת"ר ראתה כתם ואחר כך ראתה דם תולה כתמה בראייתה מעת לעת דברי רבי 
ר"ש בן אלעזר אומר יומו א"ר נראין דבריו מדברי שהוא מתקנה ואני מעוותה 
מתקנה עוותי מעוית לה אמר רבינא איפוך 
רב נחמן אמר לעולם לא תיפוך שהוא מתקן הלכותיה לידי זיבה
ואני מעוות הלכותיה לידי זיבה 
בעי מיניה ר' זירא מר' אסי כתמים צריכין הפסק טהרה או לא אשתיק ולא א"ל ולא מידי 
זימנין אשכחיה דיתיב וקאמר תולה כתמה בראייתה מעת לעת דברי רבי 
אמר ר"ל והוא שבדקה ורבי יוחנן אמר אע"פ שלא בדקה 
א"ל מכלל דכתמים צריכין הפסקת טהרה א"ל אין והא זימנין סגיאין בעא מינך ולא אמרת ולא מידי דלמא אגב שיטפך אתיא לך א"ל אין אגב שיטפאי אתיא לי
{\large\emph{מתני׳}} הרואה יום י"א בין השמשות תחלת נדה וסוף נדה תחלת זיבה וסוף זיבה 
יום ארבעים לזכר ויום שמונים לנקבה בין השמשות לכולן הרי אלו טועות
א"ר יהושע עד שאתם מתקנים את השוטות באו ותקנו את הפקחות
{\large\emph{גמ׳}} תחלת נדה וסוף נדה תחלת נדה וסוף זיבה היא 
אמר רב חסדא הכי קאמר הרואה יום י"א בין השמשות תחילת נדה וסוף זיבה
ובשביעי לנדתה סוף נדה ותחלת זיבה
א"ר יהושע עד שאתם מתקנין את השוטות כו' הני}

\newsection{דף נד}
\twocol{שוטות נינהו טועות נינהו אלא תני טועות 
\commenta{מדאמרינן \textbf{משמש' רביע ימיה מתוך שמונה ועשרים יום.} דלפי זה פתחה של זו מכ"ח לכ"ח. שמע מינה שאין האשה נעשי' תחלת נדה משנעשי' זבה גדולה עד שתספור נקיים שלה שהרי בשבוע שלישי שהוא טמא כל שבעה אין בו מימי זיבה אלא ארבע ימים הראשונים ואם תאמר בג' האחרונים נעשי' תחלת נדה נמצאת בשבוע חמישי' שהוא טמא זבה גדולה וצריכה שבעה נקיים ואם כן היאך פתחה של זו בתחלת כ"ח והרי כל אותו שבוע ה' בימי זיבה הוא ונעשת בו זבה גדולה ובשבוע ו' משמרת נקיים ובשבעי שהוא טמא ששה ימים שבו ימי זיבה הן. ואינה משמש' בשמיני נמצאת שלא שמשה בכ"ח שניים כלום.\par וכן נמי מדקתני סיפא משמשת חמשה עשר יום מתוך מ"ח ש"מ כה"ג דאי לא תימא הכי הרי שמנה חמישי' ארבעה האחרונים מתתל' ימי נדה נמצאו ימי זיבתן כלין בשבעה של שמוגה ימים השביעיים ואנו אומרים תחל' נדתה של זו שחוזרת חלילה.\par וכן סיפא דקתני וכן למאה וכן לאלף כלומר דמאה טהורין שבעה הראשונים תחלת נדה והשאר כולן ימי זיבה הן ומאה טהורין ז' לספירה וכולן לתשמיש והיינו ימי שמושה כימי זיבתה אלמא כולן ימי זיבה הן שמשנעש' זבה גדולה עד שתספור שבעה נקיים איו ראייתה אלא סתירה לספירתה ואינה מונה מהם ימי נדה.\par וז"ש בפרק בנות כותיים מה ימי נדתה אין ראויין לזיבה ואין ספירת שבעה עולה בהן כלומר לפי שא"א ושם אמרו וכי דנין אפשר משא"א לומר שא"א לספירת זיבה בימי נדה למ"ש וכן פי' שם רש"י ז"ל. }
דתניא יום אחד טמא ויום אחד טהור משמשת שמיני ולילו עמו
\commenta{והא דאקשינן \textbf{הני ארביס' הוו.} מפורש בדברי הר"ר אב"ד ז"ל דהכי מקשה בשמנה ימים הרביעיי' למה תשמש שבעה והלא צריכה היא לשמור יום א, לספיר' עשירי ואחד עשר של ימי זיבה שראתה בהן בשמונה השלישיים וכדאמרן ברישא דהיינו שימור בעו ופריק רב אדא זאת אומרת ימי נדה שאינה רואה בהן עולה לה לימי זיבתה כלומר של זוב קטן. ולפיכך יום א' של ח' רביעיים שהשלימה בהן ימי נדתה עולה לה לספיר' שמיר' של יום עשירי שאמרנו. ואין דברי רש"י ז"ל נוחין בזה.\par אבל דבר שהכל מודים בו שאין ימי נדה מתחילין עד שתספור נקיים.\par ובואו ונצווח על הרמב"ם פאסי ז"ל שכתב בחבורו שהאשה שראתה תחלה מונה שבעה לנידתה וסמוך להן אחד עשר ואח"כ מונה ז' לנדות אעפ"י שאינה רואה בהן ואחריהן אחד עשר ואם ראתה בהן הרי היא זבה וכן כל ימיה ואם קבעה לה וסת תחל' הוס' הוא יום נדו' וממנו מונה שמונה עשר ומונה שבעה לנדותה אף על פי שלא ראתה ואם ראתה אחריהן באחד עשר זבה היא.\par עוד שבש וכתב שאפילו ראתה ט' וי' ואחד עשר ושנים עשר הרי זו זבה ותחלת נדה וכל אלו דברי הבאי שלדבריו לא תמצא לרואה שבעה טמאים ושבעה טהורים שתשמש אלא שבוע שני ולסוף תשעה שבועות משמשת ששה ימים בשבוע העשירי וחמשה ימים בשבוע שנים עשר ופתחה של זו לסוף אחד עשר שבועות ובגמרא אמרו רביע ימיה ולא קיים אלא בתוך כ"ח הא'.\par וכן לדבריו בשמונה ימים טמאים ושמונה טהורים אינה משמשת תמשה עשר יום אלא מתוך שמונה וארבעים ראשונים אבל בשמונה וארבעים שניים אינה משמשת אלא שלושה ימים וכיון שלא אמרו משמשת ארבע עשר יום מתוך שנים ושלשים או משמשת שמונה עשר מתוך ל"ו וכן כיוצא במנינן הללו ש"מ שפתחה של זו מ"ח ומכאן ואילך חוזרת חלילה.\par וכן האשה שראתה עשרה ימים טמאים ועשרה ימים טהורים אין זיבתה ושימושה שוים אלא פעם אחת בלבד לפי דברי הרב ז"ל שהרי כשהיא חוזרת ורואה כן בשניה בשמונה ימים טהורים נשלמו ימי זיבה ראשונה והתחילו ימי נדה נמצא שבעשרה ימים טמאים השניים חמשה ימים מימי זיבה ואין ימי חמישה בטהורים אלא שלשה וכן למאה וכן לאלף למה מנה חכמים שבעה לנדה והשאר לזיבות והלא נעשה , היא נדה אף על פי שלא ספרה לזיבה. ועוד לדבריו מצינו אשה רואה יום אחד מסוף ימי נדה יושבת עליו ששה ימים מימי הזיבה ואין לנדה ספירה אלא בימיה.\par וכן שנויה בכמה מקומות במסכתא זו שהרואה יום מ"א לזכר ופ"א לנקבה הרי היא תחלת נדה ואין מונין לימים שמקודם לכן והטעם לפי שכבר נשלם המניין.\par והרב ז"ל הורה ביולדת שמפסק' ומתחלת למנות מתחל' ראיה שלאחר מלאת ולדבריו צריך הוא להביא ראיה מן התורה לשנוי זה שהוא משנה היולדת משאר נשים שאפילו כשאינן רואות הן מונות ימי נדה וזיבה כאלו הן רואות.\par ועוד דהא בפ' בנות כותיים אמרי' דלכולי עלמא נדה ופתחה מכ"ז מנינן ואם היינו מונין משעת ראיה ראשונה כ"ז בימי זיבה קאי לה.\par ועוד מהא דתנן היתה למודה לראות יום ט"ו ואוקמה שמואל ט"ו לטבילתה שהן כ"ב לראייתה וכו' ואם אתה מונה כל ימי נדת זובם לתחלת ראיה ראשונה שראתה זו כי הדרי אותו כ"ב תליתאי בימי זיבה קיימי והיאך קבעה וסת בכך שאין האשה קובעת וסת בי"א כדאיתה התם בשלהי בנות כותיים ואין הוסת נקבע אלא בשלשה הפלגו' כדבעינן לפרושי קמן וכל שכן לרב הונא בריה דר' יהושע דקשיא דאמר אינה חוששת בתוך אחד עשר וכל זה במס' זו.\par ותמהיני עליו אם העביר עיניו בפתחי נדה במס' ערכין דתנא רבנן טועה שאמרה יום אחד טמא ראיתי פתחה שבעה עשר פירש שאפילו היו תחלת ימי נדות הרי השלימה עליו ששה ועוד י"א אחריהן נמצאת חוזרת לתחלת נדה וכל שכן אם היה בימי זיבה שכבר עברו ימי זיבתה וימים שהיתה ראוייה להיות נדה ואלו לדברי הרב ז"ל א"א דהא איכא למימר שאותו יום בתוך אחד עשר היום וכשעמדה אחריו שבעה עשר נמצא עומדת בימי הזיבה למנין הראוי וכן כל השמועה ומדאמרינן התם נמי חמשה וארבעים ימים טמאים ראיתי וכן כולם אשתמע בהדיא דמשעה שנעשית זבה גדולה אינה נעשית נדה לעולם עד שתספור שבעה נקיים שלה. ואין לי להאריך.\par וכן יש שבושין בחבורי הראשונים בקצתם כגון רב סעדיה שכתב שכל אחד עשר יום שבין נדה לנדה בשלשה ראיית בשלשה ימים נעשית זבה גדולה בין ברצופין בין במפוזרין. וזה טעות מתפרש כאן ובכמה מקומות דרצופין בעינן ולא מפוזרין ועל כיוצא בדברים הללו ידוו כל הימים שהתורה משתכחת מלומדיה ואין אדם מוציא הלכה ברורה במקום אחד. }
וארבעה לילות מתוך שמונה עשר יום ואם היתה רואה מבערב אינה משמשת אלא שמיני בלבד 
שני ימים טמאין ושני ימים טהורין משמשת שמיני ושנים עשר וששה עשר ועשרים 
ותשמש נמי בתשסר אמר רב ששת זאת אומרת גרגרן דתנן אסור 
רב אשי אמר נהי דחד עשר לא בעי שימור עשירי מיהא בעי שימור 
שלשה ימים טמאין ושלשה ימים טהורין משמשת שני ימים ושוב אינה משמשת לעולם 
ארבעה ימים טמאים וארבעה ימים טהורין משמשת יום אחד ושוב אינה משמשת לעולם 
חמשה ימים טמאים וחמשה ימים טהורין משמשת שלשה ימים ושוב אינה משמשת לעולם ששה ימים טמאין וששה ימים טהורין משמשת חמשה ימים ושוב אינה משמשת לעולם 
שבעה ימים טמאין ושבעה ימים טהורין משמשת רביע ימיה מתוך כ"ח ימים 
שמונה ימים טמאין ושמונה ימים טהורין משמשת חמשה עשר יום מתוך ארבעים ושמונה 
הרי ארביסר הוו 
אמר רב אדא בר יצחק זאת אומרת ימי נדתה שאין רואה בהן עולין לספירת זיבתה דאיבעיא להו
ימי לידה שאינה רואה בהן מהו שיעלו לספירת זיבתה 
אמר רב כהנא ת"ש ראתה שנים ולשלישי הפילה ואינה יודעת מה הפילה
הרי זו ספק זיבה ספק לידה 
מביאה קרבן ואינו נאכל וימי לידתה שאין רואה בהן עולין לה לספירת זיבתה 
אמר רב פפא שאני התם כיון דאיכא למימר יולדת זכר היא וכל הני שבעה יתירי דקיהבינן לה סלקי לה לספירת זיבתה 
אמר ליה רב הונא בריה דרב יהושע לרב פפא ביולדת זכר איכא לספוקי ביולדת נקבה ליכא לספוקי אלא לאו שמע מינה עולין שמע מינה 
תשעה ימים טמאין ותשעה ימים טהורין משמשת שמונה ימים מתוך שמונה עשר 
עשרה ימים טמאין ועשרה ימים טהורים ימי שמושה כימי זיבתה וכן למאה וכן לאלף
\par \par {\large\emph{הדרן עלך בא סימן}}\par \par 
מתני׳ {\large\emph{דם}} הנדה ובשר המת מטמאין לחין ומטמאין יבשין אבל הזוב והניע והרוק והשרץ והנבלה והשכבת זרע מטמאין לחין ואין מטמאין יבשין ואם יכולין להשרות ולחזור לכמות שהן מטמאין לחין ומטמאין יבשין 
וכמה היא שרייתן בפושרין מעת לעת רבי יוסי אומר בשר המת יבש ואינו יכול להשרות ולחזור לכמות שהיה טהור
{\large\emph{גמ׳}} מנא הני מילי אמר חזקיה דאמר קרא (ויקרא טו, לג) והדוה בנדתה מדוה כמותה מה היא מטמאה אף מדוה מטמאה 
אשכחן לח יבש מנלן אמר רבי יצחק אמר קרא {ויקרא טו } יהיה בהויתו יהא 
ואימא הני מילי בלח ונעשה יבש יבש מעיקרו מנלן ותו הא דתנן המפלת כמין קליפה כמין עפר כמין שערה כמין יבחושין אדומים תטיל למים אם נמוחו טמא מנלן יהיה רבויא הוא 
אי מה היא עושה משכב ומושב לטמא אדם ולטמא בגדים אף דמה נמי עושה משכב ומושב לטמא אדם ולטמא בגדים אטו דמה בר משכב ומושב הוא 
ולטעמיך אבן מנוגעת בת משכב ומושב היא דאיצטריך קרא למעוטי דתניא יכול תהא אבן מנוגעת עושה משכב ומושב לטמא אדם לטמא בגדים 
ודין הוא ומה זב שאינו מטמא בביאה עושה משכב ומושב לטמא אדם לטמא בגדים אבן מנוגעת שמטמאה בביאה אינו דין שמטמאה משכב ומושב לטמא אדם לטמא בגדים
ת"ל הזב הזב ולא אבן מנוגעת טעמא דמעטיה קרא הא לאו הכי מטמאה 
ומינה לאו מי אמרת הזב ולא אבן מנוגעת ה"נ אמר קרא אשר היא יושבת עליו היא ולא דמה}

\newchap{פרק \hebrewnumeral{7} דם הנדה}}

\newsection{דף נה}
\twocol{
\commenta{\textbf{שמא יעשה עור אביו ואמו שטיחין.} מפורש במסכת חולין בפרק העור והרוטב (קכב, א). }
אי מה היא מטמאה באבן מסמא אף מדוה נמי מטמאה באבן מסמא 
\commenta{\textbf{אמר ר' יהודה מדסקרתא סלקא דעתך אמינא שעיר המשתלח יוכיח וכו'.} תימא הוא למה חזר והזכיר הטעם שדוחה סברייתא קל וחומר שלו ולמה הוצרך לומר כן לפי שאלתנו זובו טמא למה לי.\par ויש לומר שזו הברייתא השגויה למעלה שעיר המשתלח יוכיח לא היתה שנויה בבה"מ ורבא לא היה יודע אותה כמ"ש בפרק בנות כותיים וכן ר' יהודה מדסקרתא לא שמע אותה והשיב לתרץ דאיצטרך זובו טמא והיה קשה עליו בק"ו והוצרך לומר שמדין ק"ו נמי לא אתי בך מפרש בתוספת. }
אמר רב אשי אמר קרא (ויקרא טו, י) והנושא אותם אותם מיעוטא הוא
ובשר המת מנלן אמר ר"ל אמר קרא (ויקרא כב, ה) לכל טומאתו לכל טומאות הפורשות ממנו 
רבי יוחנן אמר (במדבר יט, טז) או בעצם אדם או בקבר אדם דומיא דעצם מה עצם יבש אף כאן יבש 
מאי בינייהו איכא בינייהו דאפריך אפרוכי 
מיתיבי בשר המת שהופרך טהור התם דאקמח והוי עפרא 
מיתיבי כל שבמת מטמא חוץ מן השינים והשער והצפורן ובשעת חבורן הכל טמא 
אמר רב אדא בר אהבה דומיא דעצם מה עצם שנברא עמו אף כל שנברא עמו והאיכא שער וצפורן שנבראו עמו וטהורין 
אלא אמר רב אדא בר אהבה דומיא דעצם מה עצם שנברא עמו ואין גזעו מחליף אף כל שנברא עמו ואין גזעו מחליף יצאו השינים שלא נבראו עמו יצאו שער וצפורן שאף על פי שנבראו עמו גזעו מחליף 
והרי עור דגזעו מחליף ותנן הגלודה רבי מאיר מכשיר וחכמים פוסלין ואפילו רבנן לא קפסלי אלא דאדהכי והכי שליט בה אוירא ומתה ולעולם גזעו מחליף ותנינן אלו שעורותיהם כבשרן עור האדם 
הא איתמר עלה אמר עולא דבר תורה עור אדם טהור ומאי טעמא אמרו טמא גזרה שמא יעשה אדם עורות אביו ואמו שטיחין לחמור 
ואיכא דאמרי הרי עור דאין גזעו מחליף ותנן וחכמים פוסלין ואפי' רבי מאיר לא קא מכשר אלא דקריר בשרא וחייא ולעולם אין גזעו מחליף ואמר עולא דבר תורה עור אדם טהור 
כי איתמר דעולא אסיפא איתמר וכולן שעבדן או שהילך בהן כדי עבודה טהורין חוץ מעור אדם ואמר עולא דבר תורה עור אדם כי עבדו טהור ומה טעם אמרו טמא גזרה שמא יעשה אדם עור אביו ואמו שטיחין 
והרי בשר דגזעו מחליף וטמא אמר מר בר רב אשי בשר נעשה מקומו צלקת
אבל הזוב זוב מנלן דתניא (ויקרא טו, ב) זובו טמא לימד על הזוב שהוא טמא 
והלא דין הוא לאחרים גורם טומאה לעצמו לא כ"ש שעיר המשתלח יוכיח שגורם טומאה לאחרים והוא עצמו טהור אף אתה אל תתמה על זה שאע"פ שגורם טומאה לאחרים הוא עצמו טהור ת"ל זובו טמא לימד על הזוב שהוא טמא 
ואימא ה"מ במגע אבל במשא לא מידי דהוה אשרץ אמר רב ביבי בר אביי במגע לא איצטריך קרא דלא גרע משכבת זרע
כי איצטריך קרא למשא ואימא במשא מטמא אדם ובגדים במגע אדם מטמא בגדים לא לטמא מידי דהוה אמגע נבלה 
לא ס"ד דתניא אחרים אומרים (ויקרא טו, לג) הזב את זובו לזכר ולנקבה מקיש זובו לו מה הוא לא חלקת בין מגעו למשאו לטמא אדם ולטמא בגדים אף זובו כן 
והשתא דנפקא לן מהזב את זובו זובו טמא למה לי 
אמר רב יהודה מדסקרתא איצטריך סד"א שעיר המשתלח יוכיח שגורם טומאה לאחרים והוא עצמו טהור ואי משום הזב את זובו למניינא הוא דאתא 
זוב חד זובו תרתי ובשלישי אקשיה רחמנא לנקבה 
כתב רחמנא זובו טמא והשתא דאמר רחמנא זובו טמא הוא דרוש ביה נמי האי
והרוק רוק מנלן דתניא {ויקרא טו } וכי ירוק יכול אע"פ שלא נגע ת"ל בטהור עד שיגע בטהור 
אין לי אלא רוקו כיחו וניעו ומי האף שלו מנין ת"ל וכי ירוק 
אמר מר יכול אע"פ שלא נגע מהיכא תיתי 
סד"א נילף רוק רוק מיבמה מה התם אע"פ דלא נגע אף ה"נ דלא נגע קמ"ל 
ואימר הני מילי במגע אבל במשא לא מידי דהוה אשרץ אמר ריש לקיש תנא דבי רבי ישמעאל אמר קרא בטהור מה שביד טהור טמאתי לך 
ואימא במשא מטמא אדם ובגדים במגע אדם לטמא בגדים לא לטמא מידי דהוה אמגע נבלה 
אמר ריש לקיש וכן תנא דבי רבי ישמעאל אמר קרא בטהור טהרה שטהרתי לך במקום אחר טמאתי לך כאן ואיזה זה זה מגע נבלה 
ואימא כמשא דשרץ א"כ נכתוב קרא באדם מאי בטהור ש"מ תרתי
ומי האף מאי מי האף אמר רב בנגררין דרך הפה לפי שאי אפשר למי האף בלא צחצוחי הרוק ור' יוחנן אמר אפילו בנגררין דרך החוטם אלמא קסבר מעיין הוא ורחמנא רבייה 
ורב נחשוב נמי דמעת עינו דאמר רב האי מאן דבעי דלסתמיה לעיניה ליכחול מעובד כוכבים ולוי אמר האי מאן דבעי דלימות ליכחול מעובד כוכבים 
ואמר רב חייא בר גוריא מ"ט דרב דלא אמר האי מאן דבעי דלימות הואיל ויכול לגוררן ולהוציאן דרך הפה ורב נהי דזיהרא נפיק דמעתא גופא לא נפיק 
ת"ש תשעה משקין הזב הן הזיעה והליחה סרוחה והריעי טהורין מכלום דמעת עינו ודם מגפתו וחלב האשה מטמאין טומאת משקין ברביעית אבל זובו רוקו ומימי רגליו מטמאין טומאה חמורה ואילו מי האף לא קתני 
בשלמא לרב לא קתני דלא פסיקא ליה למתני זימנין דאתי דרך הפה וזימנין דאתי דרך החוטם אלא לר' יוחנן ליתני 
ולטעמיך כיחו וניעו מי קתני אלא תנא רוק וכל דאתא מרבויא הכא נמי תנא רוקו וכל דאתא מרבויא 
דמעת עינו דכתיב (תהלים פ, ו) ותשקמו בדמעות שליש ודם מגפתו דכתיב (במדבר כג, כד) ודם חללים ישתה מה לי קטליה כוליה מה לי קטליה פלגיה חלב האשה דכתיב (שופטים ד, יט) ותפתח את נאד החלב ותשקהו 
מימי רגליו מנלן דתניא זובו טמא וזאת לרבות מימי רגליו לטומאה והלא דין הוא ומה רוק הבא ממקום טהרה טמא מימי רגליו הבאין}

\newsection{דף נו}
\twocol{ממקום טמא אינו דין שיהו טמאין דם היוצא מפי האמה יוכיח שבא ממקום טמא וטהור אף אתה אל תתמה על זה שאע"פ שבא ממקום טומאה יהיה טהור ת"ל זובו טמא וזאת לרבות מימי רגליו לטומאה 
דם היוצא מפי האמה מנלן דטהור דתניא יכול יהא דם היוצא מפיו ומפי האמה טמאין ת"ל (ויקרא טו, ב) זובו טמא הוא הוא טמא ואין דם היוצא מפיו ומפי האמה טמא אלא טהור 
ואיפוך אנא אמר רבי יוחנן משום רבי שמעון בן יוחי דומיא דרוק מה רוק שמתעגל ויוצא אף כל שמתעגל ויוצא יצא דם שאין מתעגל ויוצא 
והרי חלב שבאשה שמתעגל ויוצא ואמר מר חלב שבאשה מטמא טומאת משקין טומאת משקין אין אבל לא טומאה חמורה 
אלא אמר ר' יוחנן משום רבי שמעון בן יוחי דומיא דרוק מה רוק מתעגל ויוצא וחוזר ונבלע אף כל מתעגל ויוצא וחוזר ונבלע יצא דם שאינו מתעגל ויוצא יצא חלב שבאשה שאע"פ שמתעגל ויוצא אינו חוזר ונבלע 
ונילף מזובו מה זובו שאין מתעגל ויוצא מטמא אף כל אמר רבא מזובו ליכא למילף שכן גורם טומאה לאחרים
והשרץ אמר ריש לקיש שרץ שיבש ושלדו קיימת טמא והאנן תנן מטמאין לחין ואין מטמאין יבשין אמר רבי זירא לא קשיא הא בכולן הא במקצתן 
דתניא א"ר יצחק ברבי ביסנא אמר רבי שמעון בן יוחי {ויקרא י״א:כ״ו } בהם יכול בכולן ת"ל מהם 
אי מהם יכול במקצתן תלמוד לומר בהם הא כיצד כאן בלח כאן ביבש 
אמר רבא הני זבוגי דמחוזא כי שלדן קיימת טמאין ואמר ריש לקיש שרץ שנשרף ושלדו קיימת טמא 
מיתיבי נמצא שרץ שרוף על גבי הזיתים וכן מטלית המהוהא טהורין שכל הטמאות כשעת מציאתן א"ר זירא לא קשיא הא בכולן הא במקצתן 
דתניא אמר רבי יצחק בר' ביסנא משום ר"ש בן יוחי בהם יכול בכולן ת"ל מהם 
אי מהם יכול במקצתן ת"ל בהם הא כיצד כאן בשרוף כאן בשאינו שרוף
מטמאין לחין זב דכתיב (ויקרא טו, ג) רר בשרו כיחו וניעו ורוקו דכתיב (ויקרא טו, ח) כי ירוק הזב כעין רוק 
שרץ {ויקרא יא } במותם אמר רחמנא כעין מיתה שכבת זרע הראויה להזריע נבלה דכתיב (ויקרא יא, לט) כי ימות כעין מיתה
אם יכולין להשרות בעי רבי ירמיה תחילתו וסופו בפושרין או דלמא תחילתו אף על פי שאין סופו 
ת"ש דתניא כמה היא שרייתן בפושרין יהודה בן נקוסא אומר מעת לעת תחילתו אף על פי שאין סופו רשב"ג אומר צריכין שיהו פושרין מעת לעת
רבי יוסי אומר בשר המת כו' אמר שמואל טהור מלטמא בכזית אבל מטמא טומאת רקב תניא נמי הכי רבי יוסי אומר בשר המת שיבש ואין יכול לשרות ולחזור כמות שהיה טהור מלטמא בכזית אבל טמא טומאת רקב
{\large\emph{מתני׳}} השרץ שנמצא במבוי מטמא למפרע עד שיאמר בדקתי את המבוי הזה ולא היה בו שרץ או עד שעת כבוד
וכן כתם שנמצא בחלוק מטמא למפרע עד שיאמר בדקתי את החלוק הזה ולא היה בו כתם או עד שעת הכבוס 
ומטמא בין לח בין יבש ר"ש אומר היבש מטמא למפרע והלח אינו מטמא אלא עד שעת שיהא יכול לחזור ולהיות לח
{\large\emph{גמ׳}} איבעיא להו עד שעת כבוד חזקתו בדוק או דלמא חזקתו מתכבד 
ומאי נפקא מינה דאמר כביד ולא בדיק אי אמרת חזקתו בדוק הא לא בדק אי אמרת חזקתו מתכבד הא מתכבד
אי נמי דאשתכח בגומא אי אמרת חזקתו בדוק מאן דבדק בגומא נמי בדיק אי אמרת חזקתו מתכבד גומא לא מתכבדא
וכן הכתם וכו' איבעיא להו עד שעת כבוס חזקתו בדוק או דלמא חזקתו מתכבס 
למאי נפקא מינה דאמר כיבס ולא בדק אי אמרת חזקתו בדוק הא לא בדק אי אמרת חזקתו מתכבס הא מתכבס 
אי נמי דאשתכחה בסטרא אי אמרת חזקתו בדוק מאן דבדק בסטרא נמי בדיק אי אמרת חזקתו מתכבס בסטרא לא מתכבס 
מאי תא שמע דתניא א"ר מאיר מפני מה אמרו השרץ שנמצא במבוי מטמא למפרע עד שיאמר בדקתי את המבוי הזה ולא היה בו שרץ או עד שעת כיבוד מפני שחזקת בני ישראל בודקין מבואותיהן בשעת כבודיהם ואם לא בדקו הפסידוהו למפרע 
ומפני מה אמרו כתם שנמצא בחלוק מטמא למפרע עד שיאמר בדקתי את החלוק ולא היה בו כתם או עד שעת הכבוס מפני שחזקת בנות ישראל בודקות חלוקיהן בשעת כבוסיהן ואם לא בדקו הפסידו למפרע 
ר' אחא אמר תחזור ותכבסנו אם נדחה מראיתו בידוע שלאחר כבוס ואם לאו בידוע שלפני הכבוס 
רבי אומר אינו דומה כתם שלאחר הכבוס לכתם שלפני הכבוס שזה מקדיר וזה מגליד ש"מ חזקתו בדוק ש"מ
ומטמא בין לח וכו' א"ר אלעזר לא שנו אלא שרץ אבל כתם לח נמי מטמא למפרע אימר יבש היה ומיא נפיל עליה 
שרץ נמי אימר יבש היה ומיא נפיל עליה אם איתא דהכי הוא אמרטוטי אימרטט
{\large\emph{מתני׳}} כל הכתמין הבאין מרקם טהורין רבי יהודה מטמא מפני שהם גרים וטועין הבאין מבין העובדי כוכבים טהורין מבין ישראל ומבין הכותים רבי מאיר מטמא וחכמים מטהרים מפני שלא נחשדו על כתמיהן
{\large\emph{גמ׳}} קפסיק ותני אפילו מתרמוד א"ר יוחנן זאת אומרת מקבלין גרים מתרמוד 
איני והא רבי יוחנן וסביא דאמרי תרוייהו אין מקבלין גרים מתרמוד 
וכי תימא זאת ולא סבירא ליה והאמר רבי יוחנן הלכה כסתם משנה 
אמוראי נינהו ואליבא דרבי יוחנן
מבין ישראל וכו' ורבנן אי דישראל מטהרי דמאן מטמו 
חסורי מחסרא והכי קתני מבין ישראל טמא מבין הכותים רבי מאיר מטמא דכותים גרי אמת הן וחכמים מטהרין דכותים גרי אריות הן 
אי הכי שלא נחשדו על כתמיהן גרי אריות מבעי ליה 
אלא הכי קאמר מבין ישראל ומבין הכותים טמאין דכותים גרי אמת הן הנמצאין בערי ישראל טהורין שלא נחשדו על כתמיהם ואצנועי מצנעי להו 
הנמצאין בערי כותים רבי מאיר מטמא דנחשדו על כתמיהם וחכמים מטהרין שלא נחשדו על כתמיהן 
{\large\emph{מתני׳}} כל הכתמים הנמצאים בכל מקום טהורין חוץ מן הנמצאים בחדרים ובסביבות בית הטמאות
בית הטמאות של כותים מטמאין באהל מפני שהם קוברין שם את הנפלים ר' יהודה אומר לא היו קוברין אלא משליכין וחיה גוררתו 
נאמנים לומר קברנו שם את הנפלים או לא קברנו נאמנים לומר על הבהמה אם בכרה אם לא בכרה נאמנים על ציון קברות
ואין נאמנין לא על הסככות ולא על הפרעות ולא על בית הפרס 
זה הכלל דבר שחשודים בו אין נאמנין עליו {\large\emph{גמ׳}} }

\newsection{דף נז}
\twocol{מאי דרוש (דברים יט, יד) לא תסיג גבול רעך אשר גבלו ראשונים בנחלתך
\commenta{\textbf{הסככות.} פירש רש"י ז"ל אילן המיסך על הארץ והוא סמוך לדרך בית הקברות וזימנין דמיתרמי בין השמשות וקברי התם והיינו ספיקייהו. }
כל שיש לו נחלה יש לו גבול כל שאין לו נחלה אין לו גבול 
\commenta{\textbf{הפרעות.} אבנים גדולות ובולטות מן הגדר וקבר תחת אחת מהן ואינן יודעין תחת אזו מהן.\par ותימה הוא, אם כן ספק טומאה הוא, ואם הוא ברשות היחיד ספקו טמא ואם היה ברשות הרבים וגזרו עליהן מפני שהוחזקה שם טומאה למה אמרו בכותי מהלך על פני כולה דנאמן שמא תולה הוא בספק טומאה ברשות הרבים דספיקו טהור וגזרו דרבנן לית להו עוד השיב הרב ר' אברהם בר דוד ז"ל דאם כן כל זיזין וגזוטראות נמי ומאי שנא אבנים דנקט.\par ופירש הרב ז"ל שהסככות אילן שענפיו אחת למעלה ואחת למטה ואין שם אוהל אלא שרואין את העליונה כאלו הן למטה והתחתונות כאלו הן למעלה ואף על פי שאין העליונות כדין התחתונות אלא שעדין נשאר שם אויר מועט נעשה כולו (אויר) [אהל] שלם וכן הפרעות אבנים שיוצאות מן הגדר ואינן נוגעות זו בזו אלא שראויו' לקבל מעזיבה נעשה אהל שלם ומביא טומאה מדבריהם. ועשאום כספק טומאה וזהו ספקן שאין אהל שלהם שלם ובשקברו שם בודאי מיירי.\par ואף על גב דאמרינן בסוכה שאין אומרים גוד אחית וגוד אסיק אלא בתוך שלשה משום לבוד וגבי קורות הבית תנן שאפילו אין ביניהם טפח טומאה תחתיהן ביניהם טהור אלמא לא כסתום דמי. שאני הבא דכיון שהכל מאילן אחד ומכותל אחד הוי חבור ומשלים אהל שלהן. כך כתב הרב הנז' ז"ל וכענין הזה שנוי בתוספות טהרות ומיהו דוקא שאין בין הענפים של סככות פותח טפח שאם היה ביניהם בודאי פותח טפח מפסיקין. }
נאמנים לומר קברנו והא לית להו (ויקרא יט, יד) ולפני עור לא תתן מכשול א"ר אבהו בכהן עומד שם 
\commenta{\textbf{אמר ר' יוחנן במהלך ובא על פני כולה.} ה"נ איכא למיחש דלמא טמא הוא אלא באוכל (טומאה) [תרומה] הוא דמתוקמא דומיא דרישא דמתניתין ומשום דרישא מקצר ועולה.\par ותמה הרב רבי שמואל ז"ל אלא מתניתין דקתני נאמנין לומר קברנו שם את הנפלים ואינן נאמנין על הסככות הא ודאי כשם שנאמנין על הנפלים בכהן שלהם אוכל תרומה שם כך נמי נאמנים על הסככות במהלך שם ואוכל.\par וזו אינה קושיא שהמהלך ובא על פני כולה היכי שעובר בכל השדה שאפילו לא היה הכותי חושש לאהל הסככות נטמא בקבר עצמו אם היה שם ונאמן על גופו של קבר הא על טומאות הסככות כגון שמיסך תחת אחד מן האילנות אינו נאמן עליו דלית להו דין אהל בסככות ופרעות אבל בנפלים נאמנין הן ואף על גב דאיכא למיחש לבקיאות דיצירה כן נראה לי.\par וליכא לפרושי מהלך ובא על פני כולה שהולך תחת הסככות אורך ורוחב שהרי פירשנו שאהל הסככות עצמו מדבריהם והם אינן גוזרין כן והלכך ודאי אינן נאמנים עליהם אפילו עושין בהם מעשה. }
ודילמא כהן טמא הוא דנקיט תרומה בידיה ודילמא תרומה טמאה היא דקאכיל מינה 
אי הכי מאי למימרא מהו דתימא לא בקיאי ביצירה קמ"ל
נאמנין על הבהמה וכו' והא לית להו ולפני עור לא תתן מכשול א"ר חייא בר אבא א"ר יוחנן בגוזז ועובד 
אי הכי מאי למימרא מהו דתימא לא בקיאי בטינוף קמ"ל
נאמנין על ציון וכו' ואע"ג דמדרבנן הוא כיון דכתיבא מזהר זהירי ביה דכתיב (יחזקאל לט, טו) וראה עצם אדם ובנה אצלו ציון
אבל אין נאמנין לא על הסככות וכו' סככות דתנן אלו הן סככות אילן המיסך על הארץ פרעות דתנן אבנים פרעות היוצאות מן הגדר 
בית הפרס א"ר יהודה א"ר שמואל מנפח אדם בית הפרס והולך 
רב יהודה בר אמי משמיה דרב יהודה אמר בית הפרס שנידש טהור ותנא החורש בית הקברות הרי זה עושה בית הפרס ועד כמה הוא עושה מלא מענה מאה אמה בית ארבעת סאין רבי יוסי אומר חמש 
ולא מהימני והתניא שדה שאבד בה קבר נאמן כותי לומר אין שם קבר
לפי שאינו מעיד אלא על גופו של קבר אילן שהוא מיסך על הארץ נאמן לומר אין תחתיו קבר לפי שאינו מעיד אלא על גופו של קבר 
א"ר יוחנן במהלך ובא על פני כולה 
אי הכי מאי למימרא מהו דתימא רצועה נפקא קמ"ל
זה הכלל כו' זה הכלל לאתויי מאי לאתויי תחומין ויין נסך
\par \par {\large\emph{הדרן עלך דם הנדה}}\par \par }

\newchap{פרק \hebrewnumeral{8} הרואה כתם}
\twocol{מתני׳ {\large\emph{הרואה}} כתם על בשרה כנגד בית התורפה טמאה ושלא כנגד בית התורפה טהורה על עקבה ועל ראש גודלה טמאה
על שוקה ועל פרסותיה מבפנים טמאה מבחוץ טהורה ועל הצדדין מכאן ומכאן טהורה 
ראתה על חלוקה מן החגור ולמטה טמאה מן החגור ולמעלה טהורה ראתה על בית יד של חלוק אם מגיע כנגד בית התורפה טמאה ואם לאו טהורה 
היתה פושטתו ומתכסה בו בלילה כל מקום שנמצא בו כתם טמאה מפני שהוא חוזר וכן בפוליוס
{\large\emph{גמ׳}} אמר שמואל בדקה קרקע עולם וישבה עליה ומצאה דם עליה טהורה שנאמר (ויקרא טו, יט) בבשרה עד שתרגיש בבשרה 
האי בבשרה מיבעי ליה שמטמאה בפנים כבחוץ א"כ לימא קרא בבשר מאי בבשרה שמע מינה עד שתרגיש בבשרה 
ואכתי מיבעי ליה בבשרה ולא בשפיר ולא בחתיכה תרתי שמע מינה 
תא שמע האשה שהיא עושה צרכיה וראתה דם רבי מאיר אומר אם עומדת טמאה ואם יושבת טהורה 
היכי דמי אי דארגשה יושבת אמאי טהורה אלא לאו דלא ארגשה וקתני עומדת טמאה 
לעולם דארגשה ואימור הרגשת מי רגלים הואי עומדת הדור מי רגלים למקור ואייתי דם ויושבת טהורה 
ת"ש עד שהיה נתון תחת הכר ונמצא עליו דם אם עגול טהור ואם משוך טמא 
היכי דמי אי דארגישה עגול אמאי טהור אלא לאו דלא ארגישה וקתני משוך טמא 
לא לעולם דארגישה ואימור הרגשת עד הואי משוך ודאי מגופה אתא עגול טהור 
תא שמע נמצא על שלו טמאין וחייבין בקרבן נמצא על שלה אתיום טמאין וחייבין בקרבן נמצא על שלה לאחר זמן טמאים מספק ופטורין מן הקרבן 
היכי דמי אי דארגישה לאחר זמן אמאי פטורין מן הקרבן אלא לאו דלא ארגישה וקתני נמצא על שלה אתיום טמאין וחייבין בקרבן לא לעולם דארגישה ואימא הרגשת שמש הוה 
תא שמע נמצאת אתה אומר ג' ספקות באשה על בשרה ספק טמא ספק טהור טמא על חלוקה ספק טמא ספק טהור טהור ובמגעות ובהיסטות הלך אחר הרוב 
מאי הלך אחר הרוב לאו אם רוב ימיה טמאין טמאה ואע"ג דלא ארגשה 
לא אם רוב ימיה בהרגשה חזיא טמאה דאימור ארגשה ולאו אדעתה 
אמר מר על בשרה ספק טמא ספק טהור טמא על חלוקה ספק טמא ספק טהור טהור 
ה"ד אי מחגור ולמטה על חלוקה אמאי טהור והא תנן מן החגור ולמטה טמא ואי מחגור ולמעלה על בשרה אמאי טמא והתנן ראתה דם על בשרה שלא כנגד בית התורפה טהורה 
אב"א מחגור ולמטה ואב"א מחגור ולמעלה אי בעית אימא מחגור ולמטה כגון שעברה בשוק של טבחים על בשרה מגופה אתאי דאי מעלמא אתאי על חלוקה מיבעי ליה אשתכוחי על חלוקה מעלמא אתא דאי מגופה אתא על בשרה מיבעי ליה אשתכוחי 
ואיבעית אימא מחגור ולמעלה כגון דאזדקרה על בשרה ודאי מגופה אתאי דאי מעלמא אתאי על חלוקה איבעי ליה אשתכוחי על חלוקה מעלמא אתאי דאי מגופה אתאי על בשרה איבעי ליה אשתכוחי 
קתני מיהת על בשרה ספק טמא ספק טהור טמא ואע"ג דלא הרגישה ועוד תנן הרואה כתם על בשרה כנגד בית התורפה טמאה ואע"ג דלא הרגישה אמר רב ירמיה מדפתי מודה שמואל שהיא טמאה}

\newchap{פרק \hebrewnumeral{8} הרואה כתם}}

\newsection{דף נח}
\twocol{מדרבנן 
רב אשי אמר שמואל הוא דאמר כר' נחמיה דתנן ר' נחמיה אומר כל דבר שאינו מקבל טומאה אינו מקבל כתמים 
בשלמא לרב אשי היינו דקאמר קרקע אלא לרב ירמיה מאי איריא קרקע אפילו גלימא נמי לא מיבעיא קאמר
לא מיבעיא גלימא דלא מבדק שפיר ואיכא למימר מעלמא אתא אלא אפילו קרקע דמבדק שפיר דאיכא למימר מגופה אתיא טהור
על עקבה ועל ראש גודלה טמאה וכו' בשלמא עקבה עביד דנגע באותו מקום אלא ראש גודלה מאי טעמא וכי תימא זימנין דנגע בעקבה ומי מחזקינן טומאה ממקום למקום 
והתניא היתה לה מכה בצוארה שתוכל לתלות תולה על כתפה שאינה יכולה לתלות אינה תולה
ואין אומרים שמא בידה נטלתו והביאתו לשם אלא שאני ראש גודלה דבהדי דפסעה עביד דמתרמי 
ולא מחזקינן טומאה ממקום למקום והתניא נמצאת על קשרי אצבעותיה טמאה מפני שידים עסקניות הן 
מאי טעמא לאו משום דאמרינן בדקה בחד ידא ונגעה באידך ידא לא שאני ידה דכולה עבידא דנגעה
על שוקה ועל פרסותיה מבפנים וכו' מבפנים עד היכא אמרי דבי רבי ינאי עד מקום חבק 
איבעיא להו מקום חבק כלפנים או כלחוץ ת"ש דתני רב קטינא עד מקום חבק וחבק עצמו כלפנים רב חייא בריה דרב אויא מתני לה בהדיא אמרי דבי רבי ינאי עד מקום חבק וחבק עצמו כלפנים 
בעי רבי ירמיה כשיר מהו כשורה מהו טיפין טיפין מהו לרוחב ירכה מהו 
ת"ש על בשרה ספק טמא ספק טהור טמא על בשרה מאי לאו כי האי גוונא לא דלמא דעביד כרצועה 
ההיא איתתא דאשתכח לה דמא במשתיתא אתאי לקמיה דרבי ינאי אמר לה תיזיל ותיתי 
והתניא אין שונין בטהרות כי אמרינן אין שונין לקולא אבל לחומרא שונין
היתה פושטתו וכו' תניא אר"א בר' יוסי דבר זה הוריתי בעיר רומי לאיסור וכשבאתי אצל חכמים שבדרום אמרו לי יפה הוריתה 
ת"ר ארוכה שלבשה חלוקה של קצרה וקצרה שלבשה חלוקה של ארוכה אם מגיע כנגד בית התורפה של ארוכה שתיהן טמאות ואם לאו ארוכה טהורה וקצרה טמאה 
תניא אידך בדקה חלוקה והשאילתו לחבירתה היא טהורה וחבירתה תולה בה אמר רב ששת ולענין דינא תנן אבל לענין טומאה היא טהורה וחבירתה טמאה
מאי שנא מהא דתניא שתי נשים שנתעסקו בצפור אחד ואין בו אלא כסלע דם ונמצא כסלע על זו וכסלע על זו שתיהן טמאות שאני התם דאיכא סלע יתירה 
ת"ר לבשה שלשה חלוקות הבדוקין לה אם יכולה לתלות תולה ואפילו בתחתון אין יכולה לתלות אינה תולה ואפי' בעליון 
כיצד עברה בשוק של טבחים תולה אפילו בתחתון לא עברה בשוק של טבחים אף בעליון אינה תולה
{\large\emph{מתני׳}} ותולה בכל דבר שהיא יכולה לתלות שחטה בהמה חיה ועוף נתעסקה בכתמים או שישבה בצד העסוקין בהן הרגה מאכולת הרי זו תולה בה 
עד כמה תולה רבי חנינא בן אנטיגנוס אומר עד כגריס של פול ואף ע"פ שלא הרגה ותולה בבנה או בבעלה אם יש בה מכה והיא יכולה להגלע ולהוציא דם הרי זו תולה 
מעשה באשה אחת שבאת לפני ר"ע אמרה לו ראיתי כתם אמר לה שמא מכה היתה ביך אמרה לו הן וחיתה אמר לה שמא יכולה להגלע ולהוציא דם אמרה לו הן וטהרה ר"ע 
ראה תלמידיו מסתכלין זה בזה אמר להם מה הדבר קשה בעיניכם שלא אמרו חכמים הדבר להחמיר אלא להקל שנאמר (ויקרא טו, יט) ואשה כי תהיה זבה דם יהיה זובה בבשרה דם ולא כתם 
עד שהוא נתון תחת הכר ונמצא עליו דם עגול טהור משוך טמא דברי ר"א ברבי צדוק
{\large\emph{גמ׳}} תנינא להא דת"ר מעשה ותלה ר"מ בקילור ורבי תלה בשרף שקמה
או שישבה ישבה אין לא ישבה לא 
תנינא להא דת"ר עברה בשוק של טבחים ספק ניתז עליה ספק לא ניתז עליה תולה ספק עברה ספק לא עברה טמאה:
הרגה מאכולת: הרגה אין לא הרגה לא מתני' מני רשב"ג היא דתניא הרגה תולה לא הרגה אינה תולה דברי רשב"ג וחכ"א בין כך ובין כך תולה 
אמר רשב"ג לדברי אין קץ ולדברי חברי אין סוף 
לדברי אין קץ שאין לך אשה שטהורה לבעלה שאין לך כל מטה ומטה שאין בה כמה טיפי דם מאכולת 
לדברי חברי אין סוף שאין לך אשה שאינה טהורה לבעלה שאין לך כל סדין וסדין שאין בו כמה טיפי דם 
אבל נראין דברי ר' חנינא בן אנטיגנוס מדברי ומדבריהם שהיה אומר עד כמה היא תולה עד כגריס של פול ולדבריו אנו מודים ולרבנן דאמרי תולה עד כמה אמר ר"נ בר יצחק תולה בפשפש ועד כתורמוס 
ת"ר פשפש זה ארכו כרחבו וטעמו כריחו ברית כרותה לו שכל המוללו מריח בו ארכו כרחבו לענין כתמים
טעמו כריחו לענין תרומה דתנן או שטעם טעם פשפש בפיו ה"ז יפלוט מנא ידע טעמו כריחו ואכתי מנא ידע ברית כרותה לו שכל המוללו מריח בו 
אמר רב אשי עיר שיש בה חזירים אין חוששין לכתמים אמר ר"נ בר יצחק והא דדוקרת כעיר שיש בה חזירים דמיא
עד כמה היא תולה וכו' אמר רב הונא כגריס אינה תולה פחות מכגריס תולה ורב חסדא אמר כגריס תולה יתר מכגריס אינה תולה 
לימא בעד ועד בכלל קא מיפלגי דרב הונא סבר עד ולא עד בכלל ורב חסדא סבר עד ועד בכלל 
אמר לך רב הונא איכא עד ועד בכלל ואיכא עד ולא עד בכלל והכא לחומרא והכא לחומרא 
ורב חסדא אמר לך בעלמא אימא לך לחומרא אמרינן לקולא לא אמרינן והכא כדרבי אבהו דא"ר אבהו כל שיעורי חכמים להחמיר חוץ מכגריס של כתמים להקל 
איכא דאמרי לה להא שמעתא באפי נפשה רב הונא אמר כגריס כיתר מכגריס ורב חסדא אמר כגריס כפחות מכגריס וקמיפלגי בעד ועד דהכא כדאמרינן מיתיבי}

\newsection{דף נט}
\twocol{היו עליה טיפי דמים למטה וטיפי דמים למעלה תולה בעליון עד כגריס מאי לאו כגריס מלמטה לא כגריס מלמעלה 
איתמר נמצא עליה כגריס ועוד ואותו עוד רצופה בו מאכולת ר' חנינא אומר טמאה ר' ינאי אומר טהורה רבי חנינא אומר טמאה כי תליא בכגריס בכגריס ועוד לא תליא 
רבי ינאי אומר טהורה הני מילי היכא דלא רצופה בו מאכולת אבל היכא דרצופה בו מאכולת מוכחא מילתא דהאי ועוד דם מאכולת הוא פש ליה כגריס כיון דבעלמא תליא הכא נמי תליא 
בעי רבי ירמיה נתעסקה בכגריס ונמצא עליה בכגריס ועוד מהו תבעי לר' חנינא תבעי לר' ינאי 
תבעי לר' חנינא עד כאן לא קאמר ר' חנינא התם טמאה אלא דלא נתעסקה אבל הכא דנתעסקה תליא או דלמא אפילו לרבי ינאי דאמר טהורה הני מילי היכא דרצופה בו מאכולת אבל היכא דאין רצופה בו מאכולת לא תליא 
תא שמע נתעסקה באדום אין תולה בה שחור במועט אין תולה בו מרובה היכי דמי לאו כי האי גוונא 
לא כגון דנתעסקה בכגריס ונמצא עליה שני גריסין ועוד אי הכי מאי למימרא 
מהו דתימא שקול כגריס צפור שדי בי מצעי זיל הכא ליכא שיעורא זיל הכא ליכא שיעורא קמ"ל 
אמר רבא נמצא עליה מין אחד תולה בו כמה מינין מיתיבי נתעסקה באדום אין תולה בו שחור נתעסקה שאני 
איכא דאמרי אמר רבא נתעסקה במין אחד תולה בו כמה מינין מיתיבי נתעסקה באדום אין תולה בו שחור כי קאמר רבא דאתעסקה בתרנגולת דאית בה כמה מיני דמא
מעשה באשה [וכו'] והתניא לא אמרו חכמים את הדבר להקל אלא להחמיר 
אמר רבינא לא להקל על דברי תורה אלא להחמיר על דברי תורה וכתמים עצמן דרבנן
עד שהוא נתון איבעיא להו מי פליגי רבנן עליה דר"א ברבי צדוק או לא 
תא שמע כתם ארוך מצטרף טפין טפין אין מצטרפין מני אי רבי אליעזר בר' צדוק למה לי צירוף האמר משוך כל שהוא טמא 
אלא לאו רבנן שמע מינה פליגי לא לעולם ר' אליעזר ברבי צדוק וכי אמר רבי אליעזר ברבי צדוק בעד אבל בכתם לא 
ת"ש דאמר רב יהודה אמר שמואל הלכה כרבי אליעזר ברבי צדוק הלכה מכלל דפליגי שמע מינה
\par \par {\large\emph{הדרן עלך הרואה כתם}}\par \par }

\newchap{פרק \hebrewnumeral{9} האשה שהיא עושה}
\twocol{
\commenta{ הא דאיבעי לן \textbf{יושבת מה לי א"ר שמעון.} קשיא ותיפשוט ליה ממתני' כדאמרי' בסמוך כיון דאמר ר"ש חזקת דמים מן האשה ל"ש עומדת ול"ש יושבת. ואיכא למימר מעיקרא קס"ד שאין חזקת דמים שוים מי רגלים מן האשה אלא בעומדת. והשתא דאשמועינן ברייתא דר"ש אפילו ביושבת אפשר דאתי דם ממקור א"כ הלך אחר חזקתך שחזקת דמים מן האשה ולא מן האיש דל"ש עומדת ול"ש יושבת. א"נ איכא למימר דפשטה דברייתא משמע ליה טפי ועדיף מדיוקא דמתניתין. }
מתני׳ {\large\emph{האשה}} שהיא עושה צרכיה וראתה דם רבי מאיר אומר אם עומדת טמאה ואם יושבת טהורה ר' יוסי אומר בין כך ובין כך טהורה 
איש ואשה שעשו צרכיהן לתוך הספל ונמצא דם על המים רבי יוסי מטהר ורבי שמעון מטמא שאין דרך האיש להוציא דם אלא שחזקת דמים מן האשה
{\large\emph{גמ׳}} מאי שנא עומדת דאמרינן מי רגלים הדור למקור ואייתי דם יושבת נמי נימא מי רגלים הדור למקור ואייתי דם
אמר שמואל במזנקת מזנקת נמי דלמא בתר דתמו מיא אתא דם 
אמר ר' אבא ביושבת על שפת הספל ומזנקת בתוך הספל ונמצא דם בתוך הספל דאם איתא דבתר דתמו מיא אתא על שפת הספל איבעי ליה לאשתכוחי 
אמר שמואל ואמרי לה אמר רב יהודה אמר שמואל הלכה כר' יוסי וכן אורי ליה רבי אבא לקלא הלכה כרבי יוסי
איש ואשה [כו'] איבעיא להו איש ואשה עומדין מה לי א"ר מאיר 
כי אמר רבי מאיר בחד ספקא אבל בספק ספקא לא מטמא או דלמא לא שנא 
אמר ריש לקיש היא היא ממאי מדלא קתני ר' מאיר ורבי יוסי מטהרין 
א"ה השתא רבי מאיר בספק ספקא מטמא בחד ספקא מיבעיא להודיעך כחו דרבי יוסי דאפילו בחד ספקא מטהר 
ואדמיפלגי בחד ספק להודיעך כחו דר' יוסי ליפלגו בספק ספקא להודיעך כחו דר' מאיר כח דהיתרא עדיף ליה 
ור' יוחנן אמר כי קאמר רבי מאיר בחד ספקא אבל בספק ספקא לא אמר אם כן ליתני ר"מ ור' יוסי מטהרין אין הכי נמי ואיידי דסליק מרבי יוסי פתח בדרבי יוסי 
ורבי יוסי בחד ספקא מטהר בספק ספקא מיבעיא מהו דתימא הני מילי דיעבד אבל לכתחלה לא קא משמע לן 
תניא כוותיה דרבי יוחנן איש ואשה שעשו צרכיהן לתוך הספל ונמצא דם על המים רבי מאיר ורבי יוסי מטהרין ור' שמעון מטמא 
איבעיא להו אשה יושבת מה לי אמר רבי שמעון כי אמר רבי שמעון בעומדת דדחיק לה עלמא אבל יושבת לא או דלמא לא שנא 
ת"ש דתניא יושבת תולה עומדת אינה תולה דברי ר"מ רבי יוסי אומר בין כך ובין כך תולה ר"ש אומר בין כך ובין כך אינה תולה 
איבעיא להו איש ואשה יושבין מה לי א"ר שמעון כי אמר רבי שמעון עומדת דדחיק לה עלמא ויושבת דחד ספק אבל בספק ספקא לא אמר או דלמא לא שנה 
ת"ש כיון דא"ר שמעון חזקת דמים מן האשה ל"ש עומדין ולא שנא יושבין
{\large\emph{מתני׳}} השאילה חלוקה לנכרית או לנדה הרי זו תולה בה
ג' נשים שלבשו חלוק אחד או שישבו על ספסל אחד ונמצא עליו דם כולן טמאות 
ישבו על ספסל של אבן או על האיצטבא של מרחץ רבי נחמיה מטהר שהיה רבי נחמיה אומר כל דבר שאינו מקבל טומאה אינו מקבל כתמים
{\large\emph{גמ׳}} אמר רב בנכרית}

\newchap{פרק \hebrewnumeral{9} האשה שהיא עושה}}

\newsection{דף ס}
\twocol{
\commenta{ הא דאמרינן \textbf{רב אשי אמר הא והא רשב"ג. ול"ק כאן למפרע כאן להבא.} כך פירש שאם לבשו הן שתיהן החלוק הזה ואחר שפשטו אותו מצאה אחת מהן כתם א' בחלוק שלה אין תולין כתם בכתם אבל היתה אחת מהן כבר בעל' כתם ולבשו חלוק זה ונמצא בו כתם תולין בבעלת הכתם שהיתה כבר וזה הפי' נכון ולשון הגמרא מוכיח אבל הפי' שפירש ר"ש אינו נכון כלל. }
הרואה 
ממאי דומיא דנדה מה נדה דקחזיא אף נכרית דקא חזיא 
אמר רב ששת כי ניים ושכיב רב אמרה להא שמעתא דתניא תולה בנכרית רבי מאיר אומר בנכרית הראויה לראות ואפילו ר"מ לא קאמר אלא בראויה לראות אבל רואה לא איצטריך 
אמר רבא ותסברא ר"מ לחומרא רבי מאיר לקולא 
דתניא אינה תולה בנכרית רבי מאיר אומר תולה ואלא קשיא הך תריץ הכי והיא שרואה ר' מאיר אומר בראויה לראות ואף ע"פ שאינה רואה 
ת"ר תולה בשומרת יום כנגד יום בשני שלה
ובסופרת שבעה שלא טבלה לפיכך היא מתוקנת וחברתה מקולקלת דברי רשב"ג רבי אומר אינה תולה לפיכך שתיהן מקולקלות 
ושוין שתולה בשומרת יום כנגד יום בראשון שלה
וביושבת על דם טוהר ובבתולה שדמיה טהורין 
לפיכך דרשב"ג למה לי משום דרבי 
לפיכך דרבי למה לי מהו דתימא ההיא דאשתכח כתם גבה תתקלקל אידך לא תתקלקל קמשמע לן 
אמר רב חסדא טמא וטהור שהלכו בשני שבילין אחד טהור ואחד טמא באנו למחלוקת רבי ורשב"ג 
מתקיף לה רב אדא עד כאן לא קאמר רבי התם אלא דתרוייהו כי הדדי נינהו הכא מאי נפקא לן מינה 
ורב חסדא סוף סוף איהי טבילה בעיא 
איתמר א"ר יוסי בר' חנינא טמא וטהור ואפילו טהור ותלוי שהלכו בשני שבילין אחד טמא ואחד טהור תולה טמא בתלוי וטהור בטהור לדברי הכל 
בעא מיניה ר' יוחנן מרבי יהודה בר ליואי מהו לתלות כתם בכתם אליבא דרבי לא תבעי לך
השתא ומה התם דקא חזיא מגופה אמרת אינה תולה הכא דמעלמא קא אתי לא כל שכן 
כי תבעי לך אליבא דרשב"ג התם הוא דקא חזיא מגופה תליא הכא דמעלמא קאתי לא תליא או דלמא לא שנא 
א"ל אין תולין מה טעם לפי שאין תולין 
איתיביה אין תולין כתם בכתם השאילה חלוקה לנכרית או ליושבת על הכתם הרי זו תולה בה 
הא גופה קשיא רישא אמרת אין תולין סיפא אמרת תולין הא לא קשיא הא רבי והא רשב"ג 
איכא דאמרי הא והא רבי הא בראשון שלה הא בשני שלה 
רב אשי אמר הא והא רשב"ג ולא קשיא
כאן למפרע כאן להבא 
מכל מקום קשיא אמר רבינא לא קשיא הכי קאמר השאילה חלוקה לנכרית בעלת כתם הרי זו תולה בה
והא או ליושבת על הכתם קתני הכי קאמר או ליושבת על דם טוהר בעלת כתם תולה בה
שלש שלבשו כו' שהיה ר' נחמיה כו' אמר רב מתנה מ"ט דר' נחמיה דכתיב (ישעיהו ג, כו) ונקתה לארץ תשב כיון שישבה לארץ נקתה 
אמר רב הונא אמר רבי חנינא מטהר היה רבי נחמיה אפילו באחורי כלי חרס פשיטא 
מהו דתימא ליגזור גבו אטו תוכו קמ"ל 
אמר אביי מטהר היה ר' נחמיה במטלניות שאין בהן שלש על שלש דלא חזיין לא לעניים ולא לעשירים 
דרש רב חייא בר רב מתנה משמיה דרב הלכה כר' נחמיה אמר ליה רב נחמן אבא תני מעשה בא לפני חכמים וטמאום ואת אמרת הלכה כרבי נחמיה 
מאי היא דתניא שתי נשים שהיו טוחנות ברחיים של יד ונמצא דם תחת הפנימית שתיהן טמאות תחת החיצונה החיצונה טמאה והפנימית טהורה בינתים שתיהן טמאות 
היה מעשה ונמצא דם על שפתה של אמבטי ועל עלה של זית בשעה שמסיקות את התנור ובא מעשה לפני חכמים וטמאום 
תנאי היא דתניא ר' יעקב מטמא ורבי נחמיה מטהר והורו חכמים כרבי נחמיה
{\large\emph{מתני׳}} שלש נשים שהיו ישנות במטה אחת ונמצא דם תחת אחת מהן כולן טמאות בדקה אחת מהן ונמצאת טמאה היא טמאה ושתיהן טהורות ותולות זו בזו ואם לא היו ראוין לראות רואין אותן כאילו הן ראויות
{\large\emph{גמ׳}} אמר רב יהודה אמר רב והוא שבדקה עצמה בשיעור וסת 
סבר לה כבר פדא דאמר כל שבעלה בחטאת טהרותיה טמאות
בעלה באשם תלוי טהרותיה תלויות בעלה פטור טהרותיה טהורות 
ורבי אושעיא אמר אפילו בעלה בחטאת טהרותיה תלויות 
בשלמא התם אימר שמש עכביה לדם [אבל] הכא אם איתא דהוי דם מאן עכביה 
א"ר ירמיה משל דר' אושעיא למה הדבר דומה לילד וזקן שהיו מהלכין בדרך כל זמן שהיו בדרך ילד שוהא לבא נכנסו לעיר ילד ממהר לבא ואמר אביי משל דר' אושעיא למה הדבר דומה לאדם שנותן אצבע בעין כל זמן שאצבע בעין דמעה שוהא לבא נטל האצבע דמעה ממהרת לבא
ותולות זו בזו ת"ר כיצד תולות זו בזו עוברה ושאינה עוברה תולה עוברה בשאינה עוברה
מניקה ושאינה מניקה תולה מניקה בשאינה מניקה זקנה ושאינה זקנה תולה זקנה בשאינה זקנה בתולה ושאינה בתולה תולה בתולה בשאינה בתולה 
היו שתיהן עוברות שתיהן מניקות שתיהן זקנות שתיהן בתולות זו היא ששנינו לא היו ראויות לראות רואין}

\newsection{דף סא}
\twocol{כאילו הן ראויות
{\large\emph{מתני׳}} שלש נשים שהיו ישנות במטה אחת ונמצא דם תחת האמצעית כולן טמאות תחת הפנימית שתים הפנימיות טמאות והחיצונה טהורה תחת החיצונה שתים החיצונות טמאות והפנימית טהורה 
אימתי בזמן שעברו דרך מרגלות המטה אבל אם עברו דרך עליה כולן טמאות בדקה אחת מהן ונמצאת טהורה היא טהורה ושתים טמאות בדקו שתים ומצאו טהורות הן טהורות ושלישית טמאה שלשתן ומצאו טהורות כולן טמאות 
למה הדבר דומה לגל טמא שנתערב בין שני גלים טהורים ובדקו אחת מהן ומצאו טהור הוא טהור ושנים טמאים שנים ומצאו טהורין הם טהורין ושלישי טמא 
שלשתן ומצאו טהורין כולן טמאים דברי ר"מ שר"מ אומר כל דבר שהוא בחזקת טומאה לעולם הוא בטומאתו עד שיודע לך טומאה היכן היא 
וחכמים אומרים בודק עד שמגיע לסלע או לבתולה
{\large\emph{גמ׳}} מאי שנא רישא דלא מפליג ומאי שנא סיפא דקמפליג אמר רבי אמי במשולבות
בדקה אחת [וכו'] למה ליה למתני למה זה דומה 
הכי קאמר להו ר' מאיר לרבנן מ"ש בדם דלא פליגיתו ומ"ש בגל דפליגיתו 
ורבנן בשלמא התם אימא עורב נטלה אלא הכא האי דם מהיכא אתא 
תניא אמר ר"מ מעשה בשקמה של כפר סבא שהיו מחזיקין בה טומאה ובדקו ולא מצאו לימים נשבה בו הרוח ועקרתו ונמצא גולגולת של מת תחובה לו בעיקרו אמרו לו משם ראיה אימר לא בדקו כל צרכו 
תניא א"ר יוסי מעשה במערה של שיחין שהיו מחזיקין בה טומאה ובדקו עד שהגיעו לקרקע שהיתה חלקה כצפורן ולא מצאו לימים נכנסו בה פועלים מפני הגשמים ונתזו בקרדומותיהן ומצאו מכתשת מלאה עצמות אמרו לו משם ראיה אימר לא בדקו כל צרכו 
תניא אבא שאול אומר מעשה בסלע בית חורון שהיו מחזיקין בה טומאה ולא יכלו חכמים לבדוק מפני שהיתה מרובה והיה שם זקן אחד ורבי יהושע בן חנניא שמו אמר להן הביאו לי סדינים הביאו לו סדינים ושראן במים ופרסן עליהם מקום טהרה יבש מקום טומאה לח ובדקו ומצאו בור גדול מלא עצמות 
תנא הוא הבור שמילא ישמעאל בן נתניה חללים דכתיב (ירמיהו מא, ט) והבור אשר השליך שם ישמעאל את כל פגרי אנשים אשר הכה ביד גדליה 
וכי גדליה הרגן והלא ישמעאל הרגן אלא מתוך שהיה לו לחוש לעצת יוחנן בן קרח ולא חש מעלה עליו הכתוב כאילו הרגן 
אמר רבא האי לישנא בישא אע"פ דלקבולי לא מבעי מיחש ליה מבעי 
הנהו בני גלילא דנפק עלייהו קלא דקטול נפשא אתו לקמיה דרבי טרפון אמרו ליה לטמרינן מר אמר להו היכי נעביד אי לא אטמרינכו חזו יתייכו אטמרינכו הא אמור רבנן האי לישנא בישא אע"ג דלקבולי לא מבעי מיחש ליה מבעי זילו אתון טמרו נפשייכו 
(במדבר כא, לד) ויאמר ה' אל משה אל תירא מכדי סיחון ועוג אחי הוו דאמר מר סיחון ועוג בני אחיה בר שמחזאי הוו מאי שנא מעוג דקמסתפי ומאי שנא מסיחון דלא קמסתפי 
א"ר יוחנן אר"ש בן יוחי מתשובתו של אותו צדיק אתה יודע מה היה בלבו אמר שמא תעמוד לו זכות של אברהם אבינו
שנאמר (בראשית יד, יג) ויבא הפליט ויגד לאברם העברי ואמר רבי יוחנן זה עוג שפלט מדור המבול 
תנו רבנן בגד שאבד בו כתם מעביר עליו שבעה סממנין ומבטלו רבי שמעון בן אלעזר אומר
בודקו שכונות שכונות 
\commenta{הא דתנן \textbf{שבעה סממנין מעבירין על הכתם.} לטהרות קאמר ותני והדר מפרש הטבילו ועשה על גבי טהרו' העביר עליו שבעה סממנין ולא עבר הרי זה צבע. כלומר תולין אותו להקל ונאמר שהוא צבע שכן דרך הצבע שלא לעבור בסימנין ואף על פי שאפשר שהוא דם כיון דבלוע כ"כ שאינו יכול לצאת על ידי סמנין הללו טומאה בלועה היא ואינה מטמאה.\par ומיהו אם לא הטבילו תחלה טהרותיו תלויות שהרי יש לו לחוש לדם ואף על פי שאין סופו לצאת מכל מקום הבגד טמא שכבר נטמא בשעת נפילה ומטמא אותן והיינו דקתני הטבילו ומיהו אם דם נדה ודאי הוא אף על פי שלא עבר טמא לפי שדרך בני אדם להקפיד בו ולהעביר עליו סימנין אלו הילכך לא עלתה לו טבילה ראשונה עד שיעבירם ויבטלנו. }
אבדה בו שכבת זרע חדש בודקו במחט שחוק בודקו בחמה תנא אין שכונה פחותה משלש אצבעות 
ת"ר בגד שאבד בו כלאים הרי זה לא ימכרנו לעובד כוכבים ולא יעשנו מרדעת לחמור אבל עושה ממנו תכריכין למת אמר רב יוסף זאת אומרת מצות בטלות לעתיד לבא 
א"ל אביי ואי תימא רב דימי והא א"ר מני א"ר ינאי לא שנו אלא לספדו אבל לקוברו אסור א"ל לאו איתמר עלה א"ר יוחנן אפילו לקוברו 
ור' יוחנן לטעמיה דא"ר יוחנן מאי דכתיב (תהלים פח, ו) במתים חפשי כיון שמת אדם נעשה חפשי מן המצות 
אמר רפרם בר פפא אמר רב חסדא בגד שאבד בו כלאים צובעו ומותר א"ל רבא לרפרם בר פפא מנא ליה לסבא הא 
א"ל מתני' היא דתנן בודק עד שמגיע לסלע ואי ליכא אימר עורב נטלה הכי נמי עמרא וכיתנא בהדדי לא סליק להו צבעא וכיון דלא ידיע אימר מנתר נתר 
אמר רב אחא בריה דרב ייבא משמיה דמר זוטרא האי מאן דרמי חוטא דכיתנא בגלימיה דעמרא ונתקיה ולא ידע אי נתיק אי לא נתיק שפיר דמי 
מ"ט מדאורייתא שעטנז כתיב עד שיהיה שוע טווי ונוז ורבנן הוא דגזרו ביה וכיון דלא ידע אי נתקיה שרי 
מתקיף לה רב אשי אימר או שוע או טווי או נוז והלכתא כמר זוטרא מדאפקינהו רחמנא בחדא לישנא 
ת"ר בגד צבוע מטמא משום כתם רבי נתן בר יוסף אומר אינו מטמא משום כתם שלא תקנו בגדי צבעונין לאשה אלא להקל על כתמיהן 
תקנו מאי תקנינהו אלא שלא הותרו בגדי צבעונין לאשה אלא להקל על כתמיהן הותרו מכלל דאסירי 
אין דתנן בפולמוס של אספסינוס גזרו על עטרות חתנים ועל האירוס בקשו לגזור על בגדי צבעונין אמרי הא עדיפא כדי להקל על כתמיהן
{\large\emph{מתני׳}} שבעה סמנין מעבירין על הכתם רוק תפל ומי גריסין ומי רגלים ונתר ובורית}

\newsection{דף סב}
\twocol{קמוניא ואשלג
\commenta{הא דאמרינן \textbf{ל"ש אלא טהרות שנעשו בין תכבוסת ראשונה וכו'.} פי' רש"י ז"ל תכבוסת העברת סממנין שהרי הקפיד עליו כשהחזירן והעבירן עליו וגלה דעתו שמקפיד עליו בספק דם ועבר ע"י העברה זו ונעשה בו מעשה דם שכן דרך דם לעבור ע"י סמנין ואין פי' מחוור לי שאין קפידה זו דומה להא דתני ר' חייא.\par אלא כך נראה פי' דכי מטהרינן כתם בטבילה ראשונה כשלא עבר בסמנין מפני שאין סופו לצאת בדרך כבוסו וכדפרישית וזה כיון שגלה דעתו שהוא רוצה להוציאו מ"מ אין זו טומאה בלועה אלא סופו לצאת היא וצריך טבילה לאחר שתצא לגמרי. }
הטבילו ועשה על גביו טהרות העביר עליו שבעה סמנין ולא עבר הרי זה צבע הטהרות טהורות ואינו צריך להטביל עבר או שדיהה הרי זה כתם והטהרות טמאות וצריך להטביל 
איזהו רוק תפל כל שלא טעם כלום מי גריסין לעיסת גריסין של פול חלוקת נפש מי רגלים שהחמיצו 
וצריך לכסכס שלש פעמים לכל אחד ואחד העבירן שלא כסדרן או שהעביר שבעה סמנין כאחת לא עשה ולא כלום
{\large\emph{גמ׳}} תנא נתר אלכסנדרית ולא נתר אנטפטרית
בורית אמר רב יהודה זה אהלא והתניא הבורית והאהל אלא מאי בורית כבריתא 
ורמינהי הוסיפו עליהן הלביצין והלעונין הבורית והאהל ואי בורית כבריתא מי אית ליה שביעית והתנן זה הכלל כל שיש לו עיקר יש לו שביעית וכל שאין לו עיקר אין לו שביעית אלא מאי בורית אהלא והתניא הבורית והאהל תרי גווני אהלא
קמוניא אמר רב יהודה שלוף דוץ ואשלג אמר שמואל שאלתינהו לנחותי ימא ואמרו אשלגא שמיה ומשתכח ביני נקבי מרגניתא ומפקי לה ברמצא דפרזלא
הטבילו ועשה [כו'] תנו רבנן העביר עליו שבעה סמנין ולא עבר צפון ועבר טהרותיו טמאות 
צפון צבע נמי מעבר אלא העביר עליו ששה סמנין ולא עבר העביר עליו צפון ועבר טהרותיו טמאות שאם העביר שביעי מתחילה שמא עבר 
תניא אידך העביר עליו שבעה סמנין ולא עבר שנאן ועבר טהרותיו טהורות 
א"ר זירא לא שנו אלא הטהרות שנעשו בין תכבוסת ראשונה לשניה אבל טהרות שנעשו אחר תכבוסת שניה טהרותיו טמאות שהרי הקפיד עליו ועבר
אמר ליה רבי אבא לרב אשי מידי בקפידא תליא מילתא 
א"ל אין דתניא רבי חייא אומר דם הנדה ודאי מעביר עליו ז' סמנין ומבטלו
ואמאי הא דם נדה הוא אלמא בקפידא תליא מילתא ה"נ בקפידא תליא מילתא 
תנן התם חרסין שנשתמש בהן זב שבלעו משקין ונפלו לאויר התנור והוסק התנור התנור טמא שסוף משקה לצאת 
אמר ר"ל לא שנו אלא משקין קלים אבל משקין חמורין טמא אע"פ שלא הוסק התנור רבי יוחנן אמר אחד משקין קלין ואחד משקין חמורין אם הוסק התנור אין אי לא לא 
איתיביה רבי יוחנן לריש לקיש הטבילו ועשה על גביו טהרות והעביר עליו ז' סמנין ולא עבר הרי זה צבע וטהרותיו טהורות ואין צריך להטביל 
אמר ליה הנח לכתמים דרבנן 
והתני רבי חייא דם הנדה ודאי מעביר עליו ז' סמנין ומבטלו 
אמר ליה רבי לא שנה רבי חייא מנא ליה 
איתיביה רבי יוחנן לריש לקיש רביעית דם שנבלע בבית הבית טמא ואמרי לה הבית טהור ולא פליגי הא בכלים דמעיקרא הא בכלים דבסוף 
נבלעה בכסות אם מתכבסת ויוצא ממנה רביעית דם טמאה ואם לאו טהורה 
אמר רב כהנא מקולי רביעיות שנו כאן שאני דם תבוסה דרבנן 
איתיביה ר"ל לרבי יוחנן כל הבלוע שאינו יכול לצאת טהור הא יכול לצאת טמא ואף ע"ג דלא נפיק 
א"ר פפא כל היכא דאין יכול לצאת ולא הקפיד עליו דברי הכל טהור יכול לצאת והקפיד עליו דברי הכל טמא
כי פליגי דיכול לצאת ולא הקפיד עליו מר סבר כיון דיכול לצאת אף על גב דלא הקפיד עליו ומר סבר אע"ג דיכול לצאת}

\newsection{דף סג}
\twocol{אם הקפיד עליו אין אי לא לא
איזהו רוק תפל תנא כל שלא טעם כלום מבערב סבר רב פפא קמיה דרבא למימר כמאן דאמר לא טעם מידי באורתא אמר ליה רבא מי קתני בערב מבערב קתני לאפוקי היכא דקדים ואכיל 
אמר רבה בר בר חנה אמר רבי יוחנן איזהו רוק תפל כל שעבר עליו חצות לילה ובשינה למימרא דבשינה תליא מילתא והתנן ישן כל היום אין זה רוק תפל ניעור כל הלילה הרי זה רוק תפל התם במתנמנם 
היכי דמי מתנמנם אמר רב אשי נים ולא נים תיר ולא תיר דקרו ליה ועני ולא ידע לאהדורי סברא וכי מדכרו ליה מדכר 
תנא השכים ושנה פרקו אין זה רוק תפל ועד כמה אמר רב יהודה בר שילא אמר רב אשי אמר רבי אלעזר כל שיצא רוב דבורו של שלש שעות
מי גריסין לעיסת גריסין של פול וכו' לימא מסייע ליה לריש לקיש דאמר ר"ל רוק תפל צריך שיהא עם כל אחד ואחד דלמא הבלא דפומא מעלי 
מתניתין דלא כרבי יהודה דתניא ר' יהודה אומר מי גריסין רותח ועובר שיתן לתוכו מלח 
מאי משמע דהאי עובר לישנא דאקדומי הוא אמר ר"נ בר יצחק דאמר קרא (שמואל ב יח, כג) וירץ אחימעץ דרך הככר ויעבור את הכושי אביי אמר מהכא (בראשית לג, ג) והוא עבר לפניהם ואיבעית אימא מהכא {מיכה ג } ויעבור מלכם לפניהם וה' בראשם
מי רגלים שהחמיצו תנא כמה חימוצן שלשה ימים 
א"ר יוחנן כל שיעורי חכמים בכתמים צריך שיעור לשיעורן דילד או דזקן דאיש או דאשה מכוסים או מגולים בימות החמה או בימות הגשמים
וצריך לכסכס שלש פעמים בעי רבי ירמיה אמטויי ואתויי חד או דלמא אמטויי ואתויי תרתי מאי תיקו
העבירן שלא כסדרן ת"ר הקדים שניים לראשונים תני חדא שניים עלו לו ראשונים לא עלו לו ותניא אידך ראשונים עלו לו שניים לא עלו לו 
אמר אביי אידי ואידי שניים עלו לו ולא ראשונים ומאי ראשונים ראשונים לכסדרן ושניים להעברתן
{\large\emph{מתני׳}} כל אשה שיש לה וסת דיה שעתה ואלו הן הוסתות מפהקת ומעטשת וחוששת בפי כריסה ובשפולי מעיה ושופעת וכמין צמרמורות אוחזין אותה וכן כיוצא בהן וכל שקבעה לה שלשה פעמים הרי זה וסת
{\large\emph{גמ׳}} תנינא חדא זימנא כל אשה שיש לה וסת דיה שעתה התם בוסתות דיומי הכא בוסתות דגופא 
כדקתני אלו הן וסתות היתה מפהקת מעטשת וחוששת בפי כריסה ובשפולי מעיה ושופעת 
שופעת הא שפעה ואזלא אמר עולא בריה דרב עלאי
בשופעת דם טמא מתוך דם טהור 
וכמין צמרמורות וכו' וכן כיוצא בהן לאתויי מאי אמר רבה בר עולא לאתויי אשה שראשה כבד עליה ואבריה כבדים עליה ורותתת וגוסה 
אמר רב הונא בר חייא אמר שמואל הרי אמרו לימים שנים לוסתות אחת למה שלא מנו חכמים שלשה 
למה שלא מנו חכמים לאתויי מאי אמר רב יוסף לאתויי ראשה כבד עליה ואבריה כבדין עליה ורותתת וגוסה א"ל אביי מאי קא משמע לן מתני' היא דהא פרשה רבה בר עולא אלא אמר אביי לאתויי אכלה שום וראתה ואכלה בצלים וראתה כססה פלפלים וראתה 
אמר רב יוסף לא שמיע לי הא שמעתא 
אמר ליה אביי את אמריתה ניהלן ואהא אמריתה ניהלן היתה למודה להיות רואה יום חמשה עשר ושינתה ליום עשרים זה וזה אסורין שלש פעמים ליום עשרים הותר יום חמשה עשר וקבעה לה יום עשרים שאין אשה קובעת לה וסת עד שתקבענה שלש פעמים 
ואמרת לן עלה אמר רב יהודה אמר שמואל זו דברי ר"ג בר רבי שאמר משום רשב"ג אבל חכמים אומרים ראתה אינה צריכה לא לשנות ולא לשלש 
ואמרינן לך לשנות אמרת לן לשלש מיבעיא ואמרת לן לשנות בוסתות לשלש בימים 
ונימא זו דברי רשב"ג הא קמ"ל שמואל דר"ג ברבי כרשב"ג סבירא ליה
{\large\emph{מתני׳}} היתה למודה להיות רואה בתחלת הוסתות כל הטהרות שעשתה בתוך הוסתות טמאות בסוף הוסתות כל הטהרות שעשתה בתוך הוסתות טהורות 
רבי יוסי אומר אף ימים ושעות וסתות היתה למודה להיות רואה עם הנץ החמה אינה אסורה אלא עם הנץ החמה רבי יהודה אומר כל היום שלה
{\large\emph{גמ׳}} תנא כיצד א"ר יוסי ימים ושעות וסתות היתה למודה להיות רואה מיום עשרים ליום עשרים ומשש שעות לשש שעות הגיע יום עשרים ולא ראתה אסורה לשמש כל שש שעות ראשונות דברי רבי יהודה ורבי יוסי מתיר עד שש שעות וחוששת בשש שעות 
עברו שש שעות ולא ראתה אסורה לשמש כל היום כולו דברי ר' יהודה ורבי יוסי מתיר מן המנחה ולמעלה
היתה למודה והתניא רבי יהודה אומר כל הלילה שלה 
לא קשיא הא דרגילה לראות בתחלת יממא והא דרגילה לראות בסוף ליליא 
תני חדא רבי יהודה אוסרה לפני וסתה ומתירה לאחר וסתה ותניא אידך אוסרה לאחר וסתה ומתירה לפני וסתה 
ולא קשיא הא דרגילה למחזי בסוף ליליא הא דרגילה למחזי בתחלת יממא 
אמר רבא הלכה כרבי יהודה ומי אמר רבא הכי והתניא (ויקרא טו, לא) והזרתם את בני ישראל מטומאתם מכאן א"ר ירמיה אזהרה לבני ישראל שיפרשו מנשותיהן סמוך לוסתן 
וכמה אמר רבא עונה מאי לאו עונה אחריתי לא אותה עונה 
ותרתי למה לי צריכא דאי אשמועינן הא הוה אמינא ה"מ לטהרות אבל לבעלה לא קמ"ל 
ואי מההיא הוה אמינא סמוך לוסתה עונה אחריתי קמ"ל אותה עונה
{\large\emph{מתני׳}} היתה למודה להיות רואה יום ט"ו ושינתה להיות רואה ליום כ' זה וזה אסורין שינתה פעמים ליום כ' זה וזה אסורין 
שינתה ג' פעמים ליום כ' הותר ט"ו וקבעה לה יום כ' שאין אשה קובעת לה וסת עד שתקבענה ג' פעמים ואינה מטהרת מן הוסת עד שתעקר ממנה ג' פעמים}

\newsection{דף סד}
\twocol{{\large\emph{גמ׳}} איתמר ראתה יום חמשה עשר לחדש זה ויום ט"ז לחדש זה ויום שבעה עשר לחדש זה רב אמר קבעה לה וסת לדילוג ושמואל אמר עד שתשלש בדילוג 
\commenta{מהא דקתני בברייתא \textbf{שינתה לי"ז הותר י"ו ונאסר ט"ו וי"ז.} ולא מיתסר נמי י"ח דנימא זו כבר דילגה ונחוש בפעם א' לוסת של דילוג ש"מ שאין חוששין לוסת של דילוג כלל עד שתקבענו לגמרי וזה כתוב בתוספות. וכן הורה ה"ר אברהם ז"ל. }
נימא רב ושמואל בפלוגתא דרבי ורשב"ג קמיפלגי דתניא ניסת לראשון ומת לשני ומת לשלישי לא תנשא דברי רבי רשב"ג אומר לג' תנשא לד' לא תנשא 
\commenta{\textbf{היתה למודת להיות רואה יום כ' ושנתה ליום ל'.} מדקתני האי לישנא ש"מ דוסת הפלגה הוא דקבעה מכ' לכ'. והשתא ק"ל כיון דקי"ל מראיה לראיה מנינן ולא לפי מנין הראוי כדאיתא בשלהי בנות כותיים כשהגיע יום כ' ולא ראתה ומנו עשרה לתשלום ולא ראתה הגיע יום כ' וראתה דקתני כי אורח בזמנו בא מאי נינהו הא ליכא הפלגה דעשרים השתא.\par ואיכא למימר הכא מנינן למנין הראוי ויום מ' לראיה אחרונה זו היא יום כ' דקתני שאם ראתה בעונות הראשונו' ביום זה תראה ואפילו הרחיקה יותר מונין לראיה אחרונה שפסקה בו עכשיו ולא שאלו ראתה מאותה ראיה ואילך בעונות של כ' יארע לה ראיתה ביום עשרי' אורח בזמנו בא דהכא רגלים לדבר שלמנין הראוי חוזרת אלא ש"ל זמן שרואה בוסת השינוי מונין להן מאות' ראיה אבל מכיון שהפסיקתו וחזרה לראות ביום (א') [אחר] אם למנין הראוי חזרה מונין לוסת הראשון לפי אותו מנין ואין אומרין הפלגה של מ' היא זו שרגלים לדבר.\par אבל הרב ר' אברהם בר דוד ז"ל פי' לזו בוסת החדש לפי דעתו ולמודה ליום כ' בחדש ושנתה ליום ל' בחדש קתני ולפי פי' בוסת של הפלגה אין אומרים חזר הוסת למקומו עד שתראה עכשיו ותחזור ותראה לסוף כ' שחזר האורח בזמנו. }
לא דכ"ע כרשב"ג והכא בהא קמיפלגי רב סבר חמשה עשר ממנינא ושמואל סבר כיון דלאו בדילוג חזיתיה לאו ממנינא הוא 
איתיביה היתה למודה להיות רואה יום ט"ו ושינתה ליום ששה עשר זה וזה אסורין שינתה ליום שבעה עשר הותר ששה עשר ונאסר חמשה עשר ושבעה עשר 
שינתה ליום שמונה עשר הותרו כולן ואין אסור אלא משמונה עשר ואילך קשיא לרב אמר לך רב למודה שאני 
ודקארי לה מאי קארי לה למודה אצטריכא ליה מהו דתימא כיון דלמודה ועקרתיה בתרי זימני עקרה ליה קמ"ל 
מיתיבי ראתה יום עשרים ואחד בחדש זה יום עשרים ושנים בחדש זה יום עשרים ושלשה בחדש זה קבעה לה וסת סירגה ליום עשרים וארבעה לא קבעה לה וסת תיובתא דשמואל 
אמר לך שמואל הכא במאי עסקינן כגון דרגילה למחזי ביום עשרים ושינתה ליום עשרים ואחד דיקא נמי דשבקינן ליום עשרים ונקט ליום עשרים ואחד ש"מ
שאין האשה קובעת לה וסת עד שתקבענה וכו' א"ר פפא לא אמרן אלא למקבעה אבל למיחש לה בחדא זימנא חיישא 
מאי קמ"ל תנינא היתה למודה להיות רואה יום חמשה עשר ושינתה ליום עשרים זה וזה אסורין 
אי מהתם ה"א ה"מ היכא דקיימא בתוך ימי נדתה אבל היכא דלא קיימא בתוך ימי נדתה אימא לא קמ"ל
ואינה מטהרת מן הוסת וכו' א"ר פפא לא אמרן אלא דקבעתיה תלתא זימני דצריכי תלתא זימני למעקריה אבל תרי זימני בחדא זימנא מיעקר 
מאי קמ"ל תנינא אין האשה קובעת לה וסת עד שתקבענה שלש פעמים מהו דתימא חדא לחד תרי לתרתי ותלתא לתלתא קא משמע לן 
תניא כותיה דרב פפא היתה למודה להיות רואה יום עשרים ושינתה ליום שלשים זה וזה אסורין הגיע יום עשרים ולא ראתה מותרת לשמש עד יום שלשים וחוששת ליום שלשים 
הגיע יום שלשים וראתה הגיע יום עשרים ולא ראתה והגיע יום שלשים ולא ראתה והגיע יום עשרים וראתה הותר יום שלשים
ונאסר יום עשרים מפני שאורח בזמנו בא 
{\large\emph{מתני׳}} נשים בבתוליהם כגפנים יש גפן שיינה אדום ויש גפן שיינה שחור ויש גפן שיינה מרובה ויש גפן שיינה מועט ר' יהודה אומר כל גפן יש בה יין ושאין בה יין ה"ז דורקטי
{\large\emph{גמ׳}} תנא דור קטוע תני רבי חייא כשם שהשאור יפה לעיסה כך דמים יפין לאשה תנא משום ר"מ כל אשה שדמיה מרובין בניה מרובין
\par \par {\large\emph{הדרן עלך האשה}}\par \par 
מתני׳ {\large\emph{תינוקת}} שלא הגיע זמנה לראות וניסת ב"ש אומרים נותנין לה ארבעה לילות בית הלל אומרים עד שתחיה המכה
הגיע זמנה לראות וניסת ב"ש אומרים נותנין לה לילה הראשון וב"ה אומרים עד מוצאי שבת ארבע לילות
ראתה ועודה בבית אביה ב"ש אומרים נותנין לה בעילת מצוה וב"ה אומרים כל הלילה כולה:
{\large\emph{גמ׳}} אמר רב נחמן בר יצחק ואפילו ראתה ממאי מדקא מפליג בסיפא בין ראתה בין בשלא ראתה מכלל דרישא לא שנא הכי ולא שנא הכי 
תניא נמי הכי ב"ה אומרים עד שתחיה המכה בין ראתה בין לא ראתה
עד שתחיה המכה עד כמה אמר רב יהודה אמר רב כל זמן שנוחרת כי אמריתה קמיה דשמואל אמר לי נחירה זו איני יודע מה היא אלא כל זמן שהרוק מצוי בתוך הפה מחמת תשמיש 
נחירה דקאמר רב היכי דמי אמר רב שמואל בר רב יצחק לדידי מפרשא לי מיניה דרב עומדת ורואה יושבת ואינה רואה בידוע שלא חיתה המכה על גבי קרקע ורואה על גבי כרים וכסתות ואינה רואה בידוע שלא חיתה המכה על גבי כולם ורואה ע"ג כולם ואינה רואה בידוע שחיתה המכה
הגיע זמנה וכו' איתמר שימשה בימים רב אמר לא הפסידה לילות ולוי אמר הפסידה לילות 
רב אמר לא הפסידה לילות עד מוצאי שבת תנן ולוי אמר הפסידה לילות מאי ארבע לילות דקתני ארבעה עונות 
ולרב למה לי למיתנא ארבע לילות אורח ארעא קמ"ל דדרכה דביאה בלילות וללוי ליתני ארבע לילות עד מוצאי שבת למה לי הא קמ"ל דשרי למבעל לכתחלה בשבת 
כדשמואל דאמר שמואל פרצה דחוקה מותר ליכנס בה בשבת ואע"פ שמשיר צרורות 
איתמר בעל ולא מצא דם וחזר ובעל ומצא דם רבי חנינא אמר טמאה ורבי אסי אמר טהורה 
ר' חנינא אמר טמאה דאם איתא דהוה דם בתולים מעיקרא הוי אתי ורבי אסי אמר טהורה דילמא אתרמי ליה כדשמואל דאמר שמואל יכולני לבעול כמה בעילות בלא דם ואידך שאני שמואל דרב גובריה 
אמר רב בוגרת נותנין לה לילה הראשון וה"מ שלא ראתה אבל ראתה אין לה אלא בעילת מצוה ותו לא 
מיתיבי מעשה ונתן לה רבי ארבע לילות מתוך י"ב חדש ה"ד אילימא דיהיב לה כולהו בימי קטנות}

\newchap{פרק \hebrewnumeral{10} תינוקת}}

\newsection{דף סה}
\twocol{עד שתחיה המכה תנן 
אלא דיהיב לה כולהו בימי נערות נערות י"ב חדש מי איכא והא אמר שמואל אין בין נערות לבגרות אלא ו' חדשים בלבד וכי תימא בציר מהכי הוא דליכא הא טפי איכא הא בלבד קאמר 
אלא דיהיב לה שתים בימי קטנות ושתים בימי נערות הא בעא מיניה רב חיננא בר שלמיא מרב הגיע זמנה לראות תחת בעלה מהו 
אמר ליה כל בעילות שאתה בועל אינן אלא אחת והשאר משלימין לד' לילות 
אלא דיהיב לה אחת בימי קטנות ושתים בימי נערות ואחת בימי בגרות אי אמרת בשלמא בוגרת בעלמא יהבינן לה טפי כי היכי דאהני קטנות בימי נערות למבצר לה חדא אהני נמי נערות לבגרות למבצר לה חדא 
אלא אי אמרת בוגרת דעלמא לא יהבינן לה טפי לא ליתב לה אלא בעילת מצוה ותו לא 
לעולם דיהיב לה אחת בימי קטנות וג' בימי נערות מי סברת כל תלתא ירחי חדא עונה כל תרי ירחי חדא עונה 
מנימין סקסנאה הוה שקיל ואזיל לאתריה דשמואל סבר למעבד עובדא כוותיה דרב אפילו ראתה אמר לא פליג רב בין ראתה בין לא ראתה 
קדים שכיב באורחא קרי שמואל עליה דרב (משלי יב, כא) לא יאונה לצדיק כל און 
אמר רב חיננא בר שלמיא משמיה דרב כיון שנתקו שניו של אדם נתמעטו מזונותיו שנאמר {עמוס ד } גם (אנכי נתתי לך) נקיון שנים בכל עריכם וחוסר לחם בכל מקומותיכם
ראתה ועודה תנו רבנן ראתה ועודה בבית אביה בית הלל אומרים כל הלילה שלה ונותנין לה עונה שלמה וכמה עונה שלמה פירש רבן שמעון בן גמליאל לילה וחצי יום 
ומי בעינן כולי האי ורמינהי הרי שהיו גתיו ובית בדיו טמאות ובקש לעשותן בטהרה כיצד הוא עושה הדפין והלולבין והעדשין מדיחן
העקלים של נצרים ושל בצבוץ מנגבן של שיפא ושל גמי מיישנן וכמה מיישנן י"ב חדש רשב"ג אומר מניחן מגת לגת ומבד לבד 
היינו ת"ק איכא בינייהו חרפי ואפלי 
רבי יוסי אומר הרוצה לטהר מיד מגעילן ברותחין או חולטן במי זיתים רשב"ג אומר משום ר' יוסי מניחן תחת הצינור שמימיו מקלחין או במעיין שמימיו רודפין וכמה עונה כדרך שאמרו ביין נסך כך אמרו בטהרות 
כלפי לייא בטהרות קיימינן אלא כדרך שאמרו בטהרות כך אמרו ביין נסך 
וכמה עונה א"ר חייא בר אבא א"ר יוחנן או יום או לילה ר' חנה שאונא ואמרי לה רבי חנה בר שאונא אמר רבה בר בר חנה א"ר יוחנן חצי יום וחצי לילה 
ואמר רב שמואל בר רב יצחק ולא פליגי הא בתקופ' ניסן ותשרי הא בתקופת תמוז וטבת 
הכא נמי אימא גבי נדה חצי יום וחצי לילה והא לילה וחצי יום קאמר אלא אי לילה דניסן ותשרי אי חצי יום וחצי לילה דטבת ותמוז 
ואב"א שאני כתובה דמגבי בה טפי עד דחתמי 
רב ושמואל דאמרי תרוייהו הלכה בועל בעילת מצוה ופורש 
מתיב רב חסדא מעשה ונתן לה רבי ד' לילות מתוך י"ב חדש 
א"ל רבא הדורי אפירכא למה לי אותיב ממתני' הוא סבר מעשה רב 
מ"מ לרב ושמואל קשיא אינהו דעבדו כרבותינו דתניא רבותינו חזרו ונמנו בועל בעילת מצוה ופורש 
אמר עולא כי הוו בה ר' יוחנן ור"ל בתינוקת לא הוו מסקי מינה אלא כדמסיק תעלא מבי כרבא ומסיימי בה הכי בועל בעילת מצוה ופורש 
א"ל ר' אבא לרב אשי אלא מעתה בעל נפש לא יגמור ביאתו א"ל א"כ לבו נוקפו ופורש 
ת"ר וכולן שהיו שופעות דם ובאות מתוך ד' לילות לאחר ד' לילות מתוך הלילה לאחר הלילה כולן צריכות לבדוק את עצמן 
ובכולן ר"מ מחמיר כדברי ב"ש 
ושאר ראיות שבין ב"ש וב"ה הלך אחר מראה דמים 
שהיה ר"מ אומר מראה דמים משונים הן זה מזה כיצד דם נדה אדום דם בתולים אינו אדום דם נדה זיהום דם בתולים אינו זיהום דם נדה בא מן המקור דם בתולים בא מן הצדדין 
אמר רבי יצחק בר רבי יוסי אמר רבי יוחנן זו דברי ר' מאיר אבל חכמים אומרים כל מראה דמים אחת הן 
תנו רבנן הרואה דם מחמת תשמיש משמשת פעם ראשונה ושניה ושלישית מכאן ואילך לא תשמש עד שתתגרש}

\newsection{דף סו}
\twocol{ותנשא לאחר 
\commenta{ והא דמקשינן \textbf{ותבדוק בביאה ג' של בעל ראשון.} נראה ודאי דה"ק היאך מותרת לינשא לב' ולשמש בלא בדיקה והלא כבר הוחזקה זו לראות מחמת תשמיש בג' ביאות של בעל ראשון הילכך תבדוק ולא תתגרש ופריק מותר לשני בלא בדיקה מפני שאין כל האצבעות שוות ואין מחזיקין אותה בבעל מום ורואה מחמת תשמיש אלא לאצבע זה שהוחזקה לו. לפיכך תתגרש ממנו אבל לשאר אצבעות [לא הוחזקה] עד שתהא מוחזקת לכל בג' אצבעות.\par והדר אקשי' כיון שהוחזקה בג' אצבעות למה לה ג' פעמים באצבע אחרון. ולמאי דפריך השתא דחזקה באצבעו' בלבד תהא חזקה אפילו שמשה פעם א' וראת ונתגרשה ושמשה עם השני וראת ומת ושמשה פעם אחרת עם ג' זה וראתה הוחזקה זו לכל.\par ופריק לפי שאין כל הכוחות שוות. ואיפשר שראוי' לאצבע בנחת אבל בג' אצבעות וג' כוחות בכל אצבע ואצבע הוחזקה לכל אבל אם רצתה להכניס עצמה בספק ולבדוק בבעל ראשון ושלא תתגרש ודאי מותרת שאין אחר בדיקה של חכמים בית מיחוש שנאסרות אותה לא' ונתיר לג'. וכי קתני תתגרש ותנשא רבותא קמ"ל דלא מיחזקה אלא בג'. ומי שסובר להחמיר בבעלי' הראשונים אין ממש בדבריו דאם יש לחוש בראשונים כ"ש לאחרון שהוחזקה ולא התירו ספק דבר שזדונו כרת בשביל שתנשא זו.\par אבל לענין מעשה עכשיו נראין דברי הרב ר' אברהם בר דוד ז"ל שאמר אין אנו בקיאין בבדיקה זו. ועוד שאפילו שידוע שהוא מן הצדדין הרי גזרו בנות ישראל בכל רואה טיפת דם כחרדל שיושבת עליה ז' נקיים בין מן המקור בין מן העליה בין מן הצדדין ודעת רבינו ז"ל שכתבה להנהיג בה הלכה למעשה. ואף ה"ר אברהם אמר שאין מוציאין אותה לאחר בדיקה. }
ניסת לאחר וראתה דם מחמת תשמיש משמשת פעם ראשונה ושניה ושלישית מכאן ואילך לא תשמש עד שתתגרש ותנשא לאחר ניסת לאחר וראתה דם מחמת תשמיש משמשת פעם ראשונה ושניה ושלישית מכאן ואילך לא תשמש עד שתבדוק עצמה 
\commenta{הא דא"ר יוחנן \textbf{לכי והבעלי לו על גב הנהר.} משום שאין וסת זה אוסר יומו לפי שלא היה וסתה אלא בהכנסתה לעיר שהיו חברותי' מרגישו' בה ושמא אף היא היתה מתביישת מהן מפני שמדברות בה ומתחלחלת וה"ל כוסת של קפיצות ושל אכילות שום שכל זמן שאינה אוכלת אינה חוששת. א"נ דימה לא קבעה וסת דאקראי בעלמא היא וה"ק לה לאו טבילה גרמא ליך דתיתסרי אלא דימת עירך גרמא ליך ומותרת את על גב הנהר. }
כיצד בודקת את עצמה מביאה שפופרת ובתוכה מכחול ומוך מונח על ראשו אם נמצא דם על ראש המוך בידוע שמן המקור הוא בא לא נמצא דם על ראשו בידוע שמן הצדדין הוא בא 
\commenta{הא דאמר שמואל \textbf{ממלא ונופצת היא} ואין לה תקנה. ק"ל א"כ עקרת בדיקה הראשונ' ששנויה בבריית' דקא אמרת דאי אפילא נתרפאת ואי לא אפילא לית לה תקנה אי הכי מכחול למה ושמה אותה שהפילה חררה בידוע שנתרפאת וזו שלא הפילה כלום ממלא ונופצת היא. אבל אם ראתה מחמת בעיתותא ולא הפיל' חררה זו היא שצריכה לבדיקה של ברייתא וכן נמי שלא ביעתוה נבדקת בכך ואינו מחוור ומרבינו הגדול ז"ל לא כתב מעשים הללו. }
ואם יש לה מכה באותו מקום תולה במכתה ואם יש לה וסת תולה בוסתה 
\commenta{ה"ג וכן בנוסחאות \textbf{יום א' תשב ו' והוא ב' תשב ו' והן.} שהרי בני מקום זה שאין בני תורה בידוע שאין רואים דם ויש לחוש שמא יום א' דם טהור ויום ב' טמא וצריכה ו' ועוד שהרי אין נשיהן בקיאין בימי נדה וזבה שלכך תקן להם לג' ז' נקיים בכל זמן הלכך בשנים נמי יש לחוש שמא ראשון י"א לזיבה הוא ושני תחלת נדה הילכך צריכות ו' והן. ואין זה צריך לפנים אלא שבהלכות רבינו הגדול ז"ל דמחה והן נראה כטעות סופר.\par והלשון שכתב רש"י ז"ל תשב ו' והוא כדין תורה לומר שדין תורה כך הוא למנות ו' לאחד אבל אינו כדין תורה לכך שזו יושבת ו' נקיים שאם תראה צריכה יותר הילכך צריכה הפרשה בטהרה ובדיקה והאי דקאמר ז' נקיים ולא קאמר נמי ו' נקיים לישנא בעלמא נקט דשגירי למימר ז' ימים נקיים.\par וא"ת לדברי האומר ימי נדה שאינו רואה בהן אין עולין לה לספירת זיבתה עדיין היה לו לר' בית מיחש לומר שמא יום א' י"א הוא ובעי שימור והז' הן התחלת נדה אין עולה לו וצריך ז' והן לשני ימים. לאו מילתא שכבר פי' בפ' בנות כותיים שימי נדה ולידה שאינה רואה בהן עולין לספירת זיבה קטנה לדברי כל אדם ואין לחוש כלום. וכתב רבי' בעל הלכות ז"ל שאפילו בימי טוהר נמי אם ראתה סופרת דהאידנא יולדות בזוב הן לפי שא"א לפתיח' הקבר בלא דם ובנות ישראל סופרת ז' לכל טיפה דם. ושמענו כדבריו בזה שהגאונים החרימו בדם טוהר.\par והדברים נראין אף לדין הגמר' שכשם שחששו לטועות בפתחיהן ולמשלימות דם טמא לדם טהור ועשו נמי הרחקה יתירה בדבר כך יש לחוש שמא יבואו לטעות באותן שיושב' עליהן לזכר ולנקבה ולנדה שמא ינהגו בהן קולא שסוברות כל שיש לו טומאת לידה יש לו טוהר שלה וכ"ש שברוב נפלים אין בני אדם בקיאין. ואין עליהם לדון בהם אלא כך תשב לזכר ולנקבה ולנדה וקרוב הדבר לטעות בו הילכך אין ימי טוהר יוצאין מכלל ר' זירא שאף בהן החמירו בנות ישראל לישב ז' נקיים. וההיא דאמרינן דרש מרימר הילכתא כותיה דרב וכו'. דינא קאמרי כדאמרי בשמעתי' אמינא לך האי איסור ואת אמר' לי חומרא היכא דאחמור אחמור היכא דלא אחמור לא אחמור. והמקומות שבועלין עכשיו על דם טוהר הם יחושו לעצמן. }
ואם היה דם מכתה משונה מדם ראייתה אינה תולה ונאמנת אשה לומר מכה יש לי במקור שממנה דם יוצא דברי רבי 
רשב"ג אומר דם מכה הבא מן המקור טמא ורבותינו העידו על דם המכה הבא מן המקור שהוא טהור 
מאי בינייהו אמר עולא מקור מקומו טמא איכא בינייהו 
שפופרת אפגורי מפגרא לה אמר שמואל בשפופרת של אבר ופיה רצוף לתוכה 
אמר ליה ריש לקיש לרבי יוחנן ותבדוק עצמה בביאה שלישית של בעל הראשון אמר ליה לפי שאין כל האצבעות שוות 
אמר ליה ותבדוק עצמה בביאה ראשונה של בעל שלישי לפי שאין כל הכחות שוות 
ההיא דאתאי לקמיה דרבי אמר ליה לאבדן זיל בעתה אזל בעתה ונפל ממנה חררת דם אמר רבי נתרפאה זאת 
ההיא אתתא דאתאי לקמיה דמר שמואל אמר ליה לרב דימי בר יוסף זיל בעתה אזל בעתה ולא נפל ממנה ולא מידי אמר שמואל זו ממלאה ונופצת היא וכל הממלאה ונופצת אין לה תקנה 
ההיא דאתאי לקמיה דרבי יוחנן דכל אימת דהות סלקא מטבילת מצוה הות קחזיא דמא א"ל שמא דימת עיריך עלתה ביך לכי והבעלי לו ע"ג הנהר 
איכא דאמר אמר לה תגלי לחברותיך כי היכי דתהוו עליך להך גיסא נתהוו עלך להך גיסא ואיכא דאמר אמר לה גלי לחברותיך כי היכי דלבעו עליך רחמים דתניא (ויקרא יג, מה) וטמא טמא יקרא צריך להודיע צערו לרבים ורבים מבקשים עליו רחמים 
אמר רב יוסף הוה עובדא בפומבדיתא ואתסי 
אמר רב יוסף אמר רב יהודה אמר רב התקין רבי בשדות ראתה יום אחד תשב ששה והוא
שנים תשב ששה והן שלשה תשב שבעה נקיים 
אמר ר' זירא בנות ישראל החמירו על עצמן שאפילו רואות טפת דם כחרדל יושבות עליה שבעה נקיים 
אדבריה רבא לרב שמואל ודרש קשתה שני ימים ולשלישי הפילה תשב שבעה נקיים קסבר אין קשוי לנפלים ואי אפשר לפתיחת הקבר בלא דם 
א"ל רב פפא לרבא מאי אריא קשתה שני ימים אפילו משהו בעלמא דהא א"ר זירא בנות ישראל החמירו על עצמן שאפילו רואות טפת דם כחרדל יושבות עליה שבעה נקיים 
א"ל אמינא לך איסורא ואת אמרת מנהגא היכא דאחמור אחמור היכא דלא אחמור לא אחמור 
(תבעוה נתר בחמין לטבול קמטים ע"ג נמל סי') אמר רבא תבעוה לינשא ונתפייסה צריכה שתשב שבעה נקיים 
רבינא איעסק ליה לבריה בי רב חנינא א"ל סבר ליה מר למכתב כתובה לארבעה יום א"ל אין כי מטא לארבעה נטר עד ארבעה אחרינא איעכב שבעה יומי בתר ההוא יומא 
א"ל מאי האי א"ל לא סבר לה מר להא דרבא דאמר רבא תבעוה לינשא ונתפייסה צריכה לישב שבעה נקיים א"ל אימר דאמר רבא בגדולה דקחזיא דמא אבל בקטנה דלא חזיא דמא מי אמר 
א"ל בפירוש אמר רבא ל"ש גדולה לא שנא קטנה גדולה טעמא מאי משום דמחמדא קטנה נמי מחמדא 
אמר רבא אשה
לא תחוף לא בנתר ולא בחול בנתר משום דמקטף ובחול משום דמסריך 
ואמר אמימר משמיה דרבא אשה לא תחוף אלא בחמין אבל לא בצונן ואפילו בחמי חמה צונן מאי טעמא לא משום דקרירי ומשרו מזייא 
ואמר רבא לעולם ילמד אדם בתוך ביתו שתהא אשה מדיחה בית קמטיה במים מיתיבי בית הקמטים ובית הסתרים אינן צריכין לביאת מים 
נהי דביאת מים לא בעינן מקום הראוי לביאת מים בעינן כדר' זירא דא"ר זירא כל הראוי לבילה אין בילה מעכבת בו ושאין ראוי לבילה בילה מעכבת בו 
אמר רבין בר רב אדא אמר רבי יצחק מעשה בשפחתו של רבי שטבלה ועלתה ונמצא לה עצם חוצץ בין שיניה והצריכה רבי טבילה אחרת 
ואמר רבא טבלה ועלתה ונמצא עליה דבר חוצץ אם סמוך לחפיפה טבלה אינה צריכה לחוף ולטבול ואם לאו צריכה לחוף ולטבול 
איכא דאמרי אם באותו יום שחפפה טבלה אינה צריכה לחוף ולטבול ואם לאו צריכה לחוף ולטבול 
מאי בינייהו איכא בינייהו למסמך לחפיפה טבילה למיחף ביממא ולמטבל בליליא 
אמר רבא אשה לא תעמוד על גבי כלי חרס ותטבול סבר רב כהנא למימר טעמא מאי משום גזירת מרחצאות הא על גבי סילתא שפיר דמי 
א"ל רב חנן מנהרדעא התם טעמא מאי משום דבעית סילתא נמי בעיתא 
אמר רב שמואל בר רב יצחק אשה לא תטבול}

\newsection{דף סז}
\twocol{בנמל אע"ג דהשתא ליכא אימר ברדיוני נפל 
אבוה דשמואל עבד לבנתיה מקוואות ביומי ניסן ומפצי ביומי תשרי 
אמר רב גידל אמר רב נתנה תבשיל לבנה וטבלה ועלתה לא עלתה לה טבילה אף על גב דהשתא ליכא אימר ברדיוני נפל 
אמר רמי בר אבא הני רבדי דכוסילתא עד תלתא יומי לא חייצי מכאן ואילך חייצי 
אמר מר עוקבא לפלוף שבעין לח אינו חוצץ יבש חוצץ אימתי נקרא יבש משעה שמתחיל לירק 
אמר שמואל כחול שבתוך העין אינו חוצץ ושעל גבי העין חוצץ אם היו עיניה פורחות אפי' על גבי העין אינו חוצץ 
א"ר יוחנן פתחה עיניה ביותר או עצמה עיניה ביותר לא עלתה לה טבילה 
אמר ריש לקיש האשה לא תטבול אלא דרך גדילתה כדתנן האיש נראה כעודר ומוסק זיתים אשה נראת כאורגת וכמניקה את בנה 
אמר רבה בר רב הונא נימא אחת קשורה חוצצת
שלש אינן חוצצות שתים איני יודע ור' יוחנן אמר אנו אין לנו אלא אחת 
אמר ר' יצחק דבר תורה רובו המקפיד עליו חוצץ רובו ואינו מקפיד עליו אינו חוצץ וגזרו על רובו שאינו מקפיד משום רובו המקפיד וגזרו על מיעוטו המקפיד משום רובו המקפיד 
ולגזור נמי על מיעוטו שאינו מקפיד משום מיעוטו המקפיד היא גופה גזרה ואנן ניקום ונגזור גזרה לגזרה 
אמר רב נדה בזמנה אינה טובלת אלא בלילה ושלא בזמנה טובלת בין ביום בין בלילה רבי יוחנן אמר בין בזמנה בין שלא בזמנה אינה טובלת אלא בלילה משום סרך בתה 
ואף רב הדר ביה דאמר רבי חייא בר אשי אמר רב נדה בין בזמנה בין שלא בזמנה אינה טובלת אלא בלילה משום סרך בתה
אתקין רב אידי בנרש למטבל ביומא דתמניא משום אריותא רב אחא בר יעקב בפפוניא משום גנבי
רב יהודה בפומבדיתא משום צנה רבא במחוזא משום אבולאי 
אמר ליה רב פפא לרבא ולאביי מכדי האידנא כולהו ספק זבות שוינהו רבנן ליטבלינהו ביממא דשביעאה 
משום דרבי שמעון דתניא (ויקרא טו, כח) אחר תטהר אחר אחר לכולן שלא תהא טומאה מפסקת ביניהן ר' שמעון אומר אחר תטהר אחר מעשה תטהר 
אבל אמרו חכמים אסור לעשות כן שמא תבא לידי ספק
אמר רב הונא אשה חופפת באחד בשבת וטובלת בשלישי בשבת שכן אשה חופפת בערב שבת וטובלת במוצאי שבת 
אשה חופפת באחד בשבת וטובלת ברביעי בשבת שכן אשה חופפת בערב שבת וטובלת במוצאי יו"ט שחל להיות אחר השבת 
אשה חופפת באחד בשבת וטובלת בחמישי בשבת שכן אשה חופפת בערב שבת וטובלת במוצאי שני ימים טובים של ראש השנה שחל להיות אחר השבת 
ורב חסדא אמר כולהו אמרינן שכן לא אמרינן היכא דאפשר אפשר היכא דלא אפשר לא אפשר 
ורב יימר אמר אפילו שכן נמי אמרינן לבר מאשה חופפת באחד בשבת וטובלת בחמישי בשבת דלמוצאי שני ימים טובים של ראש השנה שלאחר השבת ליתא דאפשר דחופפת בלילה וטובלת בלילה 
דרש מרימר הלכתא כרב חסדא וכדמתרץ רב יימר 
איבעיא להו אשה מהו שתחוף בלילה ותטבול בלילה מר זוטרא אוסר ורב חיננא מסורא שרי 
א"ל רב אדא [לרב חיננא מסורא] לאו הכי הוה עובדא בדביתהו דאבא מרי ריש גלותא דאיקוט אזל ר"נ בר יצחק לפיוסה ואמרה ליה מאי איתיה השתא}

\newsection{דף סח}
\twocol{תסגי אייתי למחר וידע מאי קאמרה ליה אמר דודי חסרת טשטקי חסרת עבדי חסרת 
דרש רבא אשה חופפת בערב שבת וטובלת במוצאי שבת אמר ליה רב פפא לרבא והא שלח רבין באגרתיה אשה לא תחוף בערב שבת ותטבול במוצאי שבת 
ותמה על עצמך היאך חופפת ביום וטובלת בלילה הא בעינן תכף לחפיפה טבילה וליכא 
הדר אוקי רבא אמורא עליה ודרש דברים שאמרתי לפניכם טעות הן בידי ברם כך אמרו משמיה דרבי יוחנן אשה לא תחוף בערב שבת ותטבול במוצאי שבת ותמה על עצמך היאך חופפת ביום וטובלת בלילה הא בעינן סמוך לחפיפה טבילה וליכא 
והלכתא אשה חופפת ביום וטובלת בלילה והלכתא אשה לא תחוף אלא בלילה (אלא) קשיא הלכתא אהלכתא 
לא קשיא הא דאפשר הא דלא אפשר
{\large\emph{מתני׳}} נדה שבדקה עצמה יום שביעי שחרית ומצאה טהורה ובין השמשות לא הפרישה ולאחר ימים בדקה ומצאה טמאה הרי היא בחזקת טהורה
בדקה עצמה ביום שביעי שחרית ומצאה טמאה ובין השמשות לא הפרישה ולאחר זמן בדקה ומצאה טהורה הרי זו בחזקת טמאה
ומטמאה מעת לעת ומפקידה לפקידה ואם יש לה וסת דיה שעתה 
ור' יהודה אומר כל שלא הפרישה בטהרה מן המנחה ולמעלה הרי זו בחזקת טמאה וחכמים אומרים אפילו בשנים לנדתה בדקה ומצאה טהורה ובין השמשות לא הפרישה ולאחר זמן בדקה ומצאה טמאה הרי זו בחזקת טהורה
{\large\emph{גמ׳}} איתמר רב אמר זבה ודאי ולוי אמר זבה ספק 
אהייא אילימא ארישא הרי זו בחזקת טהורה קתני 
אלא אסיפא בשלמא ספק זבה אמרינן אלא זבה ודאי נמי הרי בדקה ומצאה טהורה 
אלא כי איתמר דרב ולוי שמעתא באפי נפשה איתמר נדה שבדקה עצמה ביום השביעי שחרית ומצאה טמאה ובין השמשות לא הפרישה ולאחר ימים בדקה ומצאה טמאה רב אמר זבה ודאי ולוי אמר זבה ספק 
רב אמר זבה ודאי כיון דמעיקרא נמצאת טמאה ועכשיו נמצאת טמאה טמאה ודאי ולוי אמר ספק זבה אימר פסקה ביני וביני
וכן תנא לוי במתניתא אחר הימים בין בדקה ומצאה טהורה בין בדקה ומצאה טמאה הרי זו ספק זבה
ומטמאה מעת לעת לימא תהוי תיובתא דרבא דאמר רבא לומר שאין האשה מטמאה מעת לעת בתוך ימי זיבתה 
ולאו אותביניה לרבא חדא זימנא הכי קאמרינן לימא תהוי תיובתא דרבא נמי מהא 
אמר לך רבא כי קתני מטמאה מעת לעת אריש פרקין קאי אראתה ועודה בבית אביה 
סד"א כיון דמפסקי להו ימים טהורין כתחלת נדתה דמיא ולא תטמא מעת לעת קמ"ל
אם יש לה וסת נימא תהוי תיובתא דרב הונא בר חייא אמר שמואל דאמר רב הונא בר חייא אמר שמואל לומר שאין האשה קובעת לה וסת בימי זיבתה 
אמר לך רב הונא בר חייא כי אמרינן אין אשה קובעת לה וסת בימי זיבתה דלא בעיא תלתא זימני למיעקר דאמרינן דמיה מסולקין וכיון דדמיה מסולקין דיה שעתה
רבי יהודה אומר תניא אמרו לו לר' יהודה אלמלי ידיה מונחות בעיניה כל בין השמשות יפה אתה אומר
עכשיו אימר עם סלוק ידיה ראתה מה לי הפרישה בטהרה בז' מן המנחה ולמעלה מה לי הפרישה בטהרה בראשון 
בראשון מי איכא למאן דאמר 
אין והתניא אמר רבי שאלתי את רבי יוסי ור' שמעון כשהיו מהלכים בדרך נדה שבדקה עצמה יום ז' שחרית ומצאה טהורה ובין השמשות לא הפרישה ולאחר הימים בדקה ומצאה טמאה מהו 
אמרו לו הרי זו בחזקת טהרה ששי חמישי רביעי שלישי שני מאי א"ל לא שנא 
בראשון לא שאלתי וטעיתי שלא שאלתי אטו כולהו לאו בחזקת טומאה קיימי וכיון דפסק פסק ראשון נמי כיון דפסק פסק 
ומעיקרא מאי סבר הואיל והוחזק מעין פתוח
{\large\emph{מתני׳}} הזב והזבה שבדקו עצמן ביום ראשון ומצאו טהור וביום השביעי ומצאו טהור ושאר ימים שבינתיים לא בדקו רבי אליעזר אומר הרי הן בחזקת טהרה ר' יהושע אומר אין להם אלא יום ראשון ויום שביעי בלבד ר' עקיבא אומר אין להם אלא יום ז' בלבד
{\large\emph{גמ׳}} תניא אמר לו רבי אליעזר לר' יהושע לדבריך אתה מונה בסירוגין והתורה אמרה (ויקרא טו, כח) אחר תטהר אחר אחר לכולן שלא תהא טומאה מפסקת ביניהן 
אמר לו רבי יהושע ואתה אי אתה מודה בזב שראה קרי ובנזיר שהילך סככות ופרעות שמונה בסירוגין והתורה אמרה (במדבר ו, יב) והימים הראשונים יפלו 
ורבי אליעזר בשלמא התם (ויקרא טו, לב) לטמאה בה אמר רחמנא שאינה סותרת אלא יומה ואי משום איחלופי זב בבעל קרי לא מיחלף 
נזיר שהילך על גבי סככות ופרעות נמי מדאורייתא אהל מעליא בעינן ורבנן הוא דגזור ורבנן בדאורייתא לא מיחלף 
אבל הכא אי חיישינן דלמא חזאי בספק אתי לאיחלופי בודאי 
תני ר' יוסי ור' שמעון אמרי נראין דברי רבי אליעזר מדברי רבי יהושע ודברי רבי עקיבא מדברי כולן אבל הלכה כרבי אליעזר
איבעיא להו הזב והזבה שבדקו עצמן יום ראשון ויום שמיני ומצאו טהור ושאר הימים לא בדקו}

\newsection{דף סט}
\twocol{לרבי אליעזר מהו תחלתן וסופן בעינן והכא תחלתן איכא סופן ליכא או דילמא תחלתן אף על גב שאין סופן 
\commenta{והא דקתני ברייתא \textbf{בין השמשות טמא, ראיתי ואמר רב ירמיה מדיפתי שבאת לפני' בין השמשות.} נראה לי דהכי אמר ברייתא דקתני בין השמשות לאו לראיה אלא לביאה והכי קתני באה בין השמשות ואמרה טמא ראיתי, ואמר רב ירמיה מטבילין אותה אחד עשר טבילו' שהרי כל שבאה בין השמשות אפילו אמרה סתם יום אחד טמא ראיתי אחד עשר טבילות הן, אי נמי בין השמשו' טמא ראיתי דקתני שאם אמרה מבעוד יום ראיתי אין כאן י"א ולאו מילתא היא. }
אמר רב היא היא תחלתן אע"פ שאין סופן ורבי חנינא אמר תחלתן וסופן בעינן הכא תחלתן איכא סופן ליכא 
מיתיבי ושוין בזב ובזבה שבדקו עצמן יום ראשון ויום שמיני ומצאו טהור שאין להם אלא שמיני בלבד מאן שוין לאו רבי אליעזר ורבי יהושע 
לא ר' יהושע ור' עקיבא 
אמר רב ששת אמר רב ירמיה בר אבא אמר רב נדה שהפרישה בטהרה בשלישי שלה סופרתו למנין שבעה נקיים 
נדה ספירה למה לה אלא אימא זבה שהפרישה בטהרה בשלישי שלה סופרתו למנין ז' נקיים 
אמר ליה רב ששת לרב ירמיה בר אבא רב ככותאי אמרה לשמעתיה דאמרי יום שפוסקת בו סופרתו למנין ז' 
כי קאמר רב לבר משלישי בר משלישי פשיטא לא צריכא כגון דלא בדקה עד שביעי
ואשמועינן התם תחלתן אע"פ שאין סופן והכא קמ"ל סופן אע"פ שאין תחלתן 
דמהו דתימא תחלתן אף על פי שאין סופן הוא דאמרינן דאוקמינהו אחזקייהו אבל סופן אע"פ שאין תחלתן לא קמ"ל 
איני והא כי אתא רבין אמר מתיב ר' יוסי ברבי חנינא טועה
ולא ידענא מאי תיובתיה דקי"ל שבוע קמא דאתיא לקמן בלילותא מטבלינן לה ביממא לא מטבלינן לה 
ואי ס"ד לא בעינן ספורין לפנינו ביממא נמי נטבלינה דילמא יולדת זכר בזוב היא ועבדה לה ספורין אלא לאו שמע מינה בעינן ספורין בפנינו 
ולאו מי אוקימנא כר"ע דאמר בעינן ספורין לפנינו 
ומנא תימרא דלרבנן לא בעינן ספורין לפנינו דתנן טועה שאמרה יום אחד טמא ראיתי מטבילין אותה ט' טבילות
ז' לנדה ותרי לזיבה בין השמשות טמא ראיתי מטבילין אותה י"א טבילות 
י"א מאי עבידתייהו אמר רב ירמיה מדפתי כגון שבאת לפנינו בין השמשות
והויין תמני לנדה ותלת לזיבה 
לא ראיתי כל עיקר מטבילין אותה ט"ו טבילות אמר רבא האי דינא דלא דינא דייני בגלחי דאית ליה תורא לירעי חד יומא דלית ליה תורא לירעי תרי יומי 
אתרמי להו יתמא בר ארמלתא יהבי ליה תורי אזל נכסינהו אמר להו דאית ליה תורא לשקול חד משכא דלית ליה תורא לשקול תרי משכי אמרי ליה מאי האי דקאמרת אמר להו סוף דינא כתחלת דינא תחלת דינא לאו מאן דלית ליה עדיף סוף דינא נמי מאן דלית ליה עדיף 
הכא נמי ומה היכא דאמרה ראיתי סגי לה אי בתשע טבילות אי בי"א טבילות היכא דקאמרה איהי לא ראיתי בעיא חמש עשרה טבילות 
אלא אימא הכי ראיתי ואיני יודע כמה ראיתי אי בימי נדה ראיתי או בימי זיבה ראיתי מטבילין אותה ט"ו טבילות אתאי קמן ביממא יהבינן לה שב לנדה
ותמני לזיבה אתאי קמן בלילותא יהבינן לה תמני לנדה ושב לזיבה
זיבה תמני בעיא אלא אידי ואידי שב לנדה ותמני לזיבה
בלילותא תמני לנדה בעי 
זיבה דפסיקא ליה דלא שנא כי אתיא קמן ביממא לא שנא כי אתיא קמן בליליא חשיב לה נדה דלא פסיקא ליה דכי אתיא קמן בלילותא בעי תמני ביממא לא קבעי תמני לא קחשיב לה 
ואי ס"ד ספורין לפנינו בעינן כל הני טבילות למה לי תספור ז' והדר תטבול אלא לאו שמע מינה רבנן היא דאמרי לא בעינן ספורין לפנינו 
אמר ליה רב אחא בריה דרב יוסף לרב אשי לאו תרוצי קמתרצינן לה תריץ ואימא הכי ספרתי ואיני יודעת כמה ספרתי אם בימי נדה ספרתי ואם בימי זיבה ספרתי מטבילין אותה ט"ו טבילות 
ספרתי ואיני יודעת כמה ספרתי חד יומא מיהא אי אפשר דלא ספרה חסרה לה טבילה 
אלא אימא איני יודעת אם ספרתי אם לא ספרתי
{\large\emph{מתני׳}} הזב והזבה והנדה והיולדת והמצורע שמתו מטמאין במשא עד שימוק הבשר עובד כוכבים שמת טהור מלטמא 
בית שמאי אומרים כל הנשים מתות נדות וב"ה אומרים אין נדה אלא שמתה נדה
{\large\emph{גמ׳}} מאי במשא אילימא במשא ממש אטו כל מת מי לא מטמא במשא 
אלא מאי במשא באבן מסמא
דכתיב (דניאל ו, יח) והיתית אבן חדא ושומת על פום גובא 
מאי טעמא אמר רב גזרה שמא יתעלפה 
תנא משום ר' אליעזר אמרו עד שיבקע כריסו
עובד כוכבים שמת [כו'] תניא אמר רבי מפני מה אמרו עובד כוכבים שמת טהור מלטמא במשא לפי שאין טומאתו מחיים מדברי תורה אלא מדברי סופרים 
ת"ר שנים עשר דברים שאלו אנשי אלכסנדריא את רבי יהושע בן חיננא ג' דברי חכמה ג' דברי הגדה ג' דברי בורות ג' דברי דרך ארץ 
ג' דברי חכמה הזב והזבה והנדה והיולדת והמצורע שמתו עד מתי מטמאין במשא אמר להן עד שימוק הבשר 
בת משולחת מה היא לכהן 
מי אמרינן קל וחומר ומה אלמנה לכ"ג שאין איסורה שוה בכל בנה פגום זו שאיסורה שוה בכל אינו דין שבנה פגום או דילמא מה לאלמנה לכהן גדול שהיא עצמה מתחללת 
אמר להן}

\newsection{דף ע}
\twocol{היא תועבה ואין בניה תועבין 
שני מצורעין שנתערבו קרבנותיהן זה בזה וקרב קרבנו של אחד מהן ומת אחד מהן השני מה תהא עליו 
אמר להן כותב נכסיו לאחרים והוי עני ומביא חטאת העוף הבא על הספק 
והאיכא אשם אמר שמואל כשקרב אשמו 
אמר רב ששת גברא רבה כשמואל לימא כי האי מילתא כמאן אי כר' יהודה דאמר אשם קבעה וכיון דקבעה לה אשם בעשירות לא מצי מייתי חטאת בדלות 
דתנן מצורע שהביא קרבן עני והעשיר או עשיר והעני הכל הולך אחר חטאת דברי ר' שמעון 
ורבי יהודה אומר הכל הולך אחר אשם רבי אליעזר בן יעקב אומר הכל הולך אחר צפורים 
ואי כרבי שמעון דאמר חטאת קבעה אע"ג דלא קרב אשם ניתי אחר דהא שמעינן ליה לרבי שמעון דאמר לייתי ולתני 
דתניא אמר ר"ש למחרת מביא אשמו ולוגו עמו ומעמידו בשער נקנור ומתנה עליו ואומר אם מצורע הוא הרי אשמו ולוגו עמו ואם לאו אשם זה יהא שלמי נדבה 
ואותו אשם טעון
שחיטתו בצפון וטעון מתנת בהונות וסמיכה ונסכים ותנופה וחזה ושוק ונאכל לזכרי כהונה ליום ולילה 
ולא הודו לו חכמים לר' שמעון מפני שמביא קדשים לבית הפסול 
שמואל סבר לה כרבי שמעון בחדא ופליג עליה בחדא
שלשה דברי אגדה כתוב אחד אומר (יחזקאל יח, לב) כי לא אחפוץ במות המת וכתוב אחד אומר (שמואל א ב, כה) כי חפץ ה' להמיתם כאן בעושין תשובה כאן בשאין עושין תשובה 
כתוב אחד אומר {דברים י } כי לא ישא פנים ולא יקח שוחד וכתוב אחד אומר (במדבר ו, כו) ישא ה' פניו אליך כאן קודם גזר דין כאן לאחר גזר דין 
כתוב אחד אומר (תהלים קלב, יג) כי בחר ה' בציון וכתוב אחד אומר (ירמיהו לב, לא) כי על אפי ועל חמתי היתה העיר הזאת למן היום אשר בנו אותה עד היום הזה כאן קודם שנשא שלמה את בת פרעה כאן לאחר שנשא שלמה את בת פרעה
שלשה דברי בורות אשתו של לוט מהו שתטמא אמר להם מת מטמא ואין נציב מלח מטמא 
בן שונמית מהו שיטמא אמר להן מת מטמא ואין חי מטמא 
מתים לעתיד לבא צריכין הזאה שלישי ושביעי או אין צריכין אמר להן לכשיחיו נחכם להן איכא דאמרי לכשיבא משה רבינו עמהם
שלשה דברי דרך ארץ מה יעשה אדם ויחכם אמר להן ירבה בישיבה וימעט בסחורה אמרו הרבה עשו כן ולא הועיל להם אלא יבקשו רחמים ממי שהחכמה שלו שנאמר (משלי ב, ו) כי ה' יתן חכמה מפיו דעת ותבונה 
תני ר' חייא משל למלך בשר ודם שעשה סעודה לעבדיו ומשגר לאוהביו ממה שלפניו 
מאי קמ"ל דהא בלא הא לא סגיא 
מה יעשה אדם ויתעשר אמר להן ירבה בסחורה וישא ויתן באמונה אמרו לו הרבה עשו כן ולא הועילו אלא יבקש רחמים ממי שהעושר שלו שנאמר (חגי ב, ח) לי הכסף ולי הזהב 
מאי קמ"ל דהא בלא הא לא סגי
מה יעשה אדם ויהיו לו בנים זכרים אמר להם ישא אשה ההוגנת לו}

\newsection{דף עא}
\twocol{ויקדש עצמו בשעת תשמיש 
אמרו הרבה עשו כן ולא הועילו אלא יבקש רחמים ממי שהבנים שלו שנאמר (תהלים קכז, ג) הנה נחלת ה' בנים שכר פרי הבטן 
מאי קא משמע לן דהא בלא הא לא סגי 
מאי שכר פרי הבטן א"ר חמא ברבי חנינא בשכר שמשהין עצמן בבטן כדי שתזריע אשתו תחילה נותן לו הקב"ה שכר פרי הבטן
בית שמאי אומרים [וכו'] מאי טעמייהו דבית שמאי אי נימא משום דכתיב (אסתר ד, ד) ותתחלחל המלכה ואמר רב מלמד שפרסה נדה הכא נמי אגב ביעתותא דמלאכא דמותא חזיא והאנן תנן שחרדה מסלקת את הדמים הא לא קשיא פחדא צמית ביעתותא מרפיא 
אלא הא דתנן ב"ש אומרים כל האנשים מתים זבין וב"ה אומרים אין זב אלא מי שמת זב 
אקרי כאן מבשרו ולא מחמת אונסו 
אלא טעמא דב"ש כדתניא בראשונה היו מטבילין כלים על גבי נדות מתות והיו נדות חיות מתביישות התקינו שיהו מטבילין על גבי כל הנשים מפני כבודן של נדות חיות 
בראשונה היו מטבילין על גבי זבין מתין והיו זבין חיין מתביישין התקינו שיהו מטבילין על גבי כל האנשים מפני כבודן של זבין חיים
{\large\emph{מתני׳}} האשה שמתה ויצאה ממנה רביעית דם מטמאה משום כתם ומטמאה באהל
רבי יהודה אומר אינה מטמאה משום כתם מפני שנעקר משמתה ומודה רבי יהודה ביושבת על משבר ומתה ויצאה ממנה רביעית דם שהיא מטמאה משום כתם אמר רבי יוסי לפיכך אינה מטמאה באהל
{\large\emph{גמ׳}} מכלל דתנא קמא סבר אף על גב דנעקר דם משמתה מטמאה משום כתם 
אמר (רבי) זעירי מקור מקומו טמא איכא בינייהו
ומודה רבי יהודה
מכלל דתנא קמא סבר באהל נמי מטמא אמר רב יהודה דם תבוסה איכא בינייהו 
דתניא איזהו דם תבוסה פירש ר"א ברבי יהודה הרוג שיצא ממנו דם בין בחייו בין במותו ספק בחייו יצא ספק במותו יצא ספק בחייו ובמותו זהו דם תבוסה 
וחכמים אומרים ברה"י ספקו טמא ברה"ר ספקו טהור 
אלא איזהו דם תבוסה הרוג שיצא הימנו רביעית דם בחייו ובמותו ועדיין לא פסק ספק רובו בחייו ומיעוטו במותו ספק מיעוטו בחייו ורובו במותו זהו דם תבוסה 
רבי יהודה אומר הרוג שיצא ממנו רביעית דם והיה מוטל במטה ודמו מטפטף לגומא טמא מפני שהטפה של מיתה מעורבת בו וחכמים מטהרין מפני
שראשון ראשון נפסק שפיר קאמרי ליה רבנן לר' יהודה ר' יהודה לטעמיה דאמר אין דם מבטל דם 
\commenta{והא דקתני \textbf{בית שמאי אומרים צריכה טבילה.} לתרומה מפני שטבולת יום ארוך הוא והסיחה דעתה מן התרומה ואם ישראלית היא טובלת לביאת מקדש. פירש לפיכך טובלת כדי שתכנס לאתר כפרתה למקדש שהרי מחוסר כפורים שנכנס למקדש ענוש כרת כדאמרונן במס' מכות (דף ח ע"ב) טמא יהיה לרבות טבול יום עוד טמאתו בו לרבות מחוסר כפורים, וכן היא צריכה לטבול לנגיע' דתרומה, וב"ה אומרי' אינה צריכה אבל לקדשים מודי ב"ה דקיי"ל (חגיגה כא, א) האונן והמחוסר כפורים צריכין טבילה לקדש דמעלות דרבנן נינהו למדנו לדברי רש"י ז"ל שהחמירו באכילות קדשים יותר מביאת המקדש שיבנה במהרה בימינו אמן וכן יהי רצון. }
ר"ש אומר צלוב על העץ שדמו שותת לארץ ונמצא תחתיו רביעית דם טמא רבי יהודה מטהר שאני אומר טפה של מיתה עומדת לו על גב העץ 
ורבי יהודה נימא איהו לנפשיה שאני אומר טפה של מיתה עומדת על גב המטה שאני במטה דמחלחלה
{\large\emph{מתני׳}} בראשונה היו אומרים היושבת על דם טהור היתה מערה מים לפסח 
חזרו לומר הרי היא כמגע טמא מת לקדשים כדברי ב"ה ב"ש אומרים אף כטמא מת
{\large\emph{גמ׳}} מערה אין נוגעת לא אלמא חולין שנעשו על טהרת הקדש כקדש דמו 
אימא סיפא חזרו לומר הרי היא כמגע טמא מת לקדשים לקדשים אין לחולין לא אלמא חולין שנעשו על טהרת הקדש לאו כקדש דמו 
מתני' מני אבא שאול היא דתניא אבא שאול אומר טבול יום תחילה לקדש לטמא שנים ולפסול אחד
{\large\emph{מתני׳}} ומודים שהיא אוכלת במעשר וקוצה לה חלה ומקפת וקורא לה שם 
ואם נפל מרוקה ומדם טהרה על ככר של תרומה שהוא טהור
ב"ש אומרים צריכה טבילה באחרונה ובית הלל אומרים אינה צריכה טבילה באחרונה
{\large\emph{גמ׳}} דאמר מר טבל ועלה אוכל במעשר
וקוצה לה חלה חולין הטבולין לחלה לאו כחלה דמו
ומקפת דאמר מר מצוה לתרום מן המוקף
וקורא לה שם סד"א נגזור דלמא אתי למנגע בה מאבראי קמ"ל
ואם נפל מרוקה דתנן משקה טבול יום (משקין היוצאין ממנו) כמשקין הנוגע בהם ואלו ואלו אינן מטמאין חוץ ממשקה הזב שהוא אב הטומאה
בית שמאי מאי בינייהו אמר רב קטינא טבול יום ארוך איכא בינייהו
{\large\emph{מתני׳}} הרואה יום אחד עשר וטבלה לערב ומשמשה 
ב"ש אומר מטמאין משכב ומושב וחייבין בקרבן}

\newsection{דף עב}
\twocol{וב"ה אומרים פטורים מן הקרבן 
טבלה ביום של אחריו ושמשה את ביתה ואח"כ ראתה ב"ש אומרים מטמאין משכב ומושב ופטורין מן הקרבן 
וב"ה אומרים ה"ז גרגרן ומודים ברואה בתוך י"א יום וטבלה לערב ושמשה שמטמאין משכב ומושב וחייבין בקרבן 
טבלה ביום של אחריו ושמשה ה"ז תרבות רעה ומגען ובעילתן תלויין
{\large\emph{גמ׳}} ת"ר ושוין בטובלת לילה לזבה שאינה טבילה ושוין ברואה בתוך י"א יום וטבלה לערב ושמשה שמטמאה משכב ומושב וחייבין בקרבן 
לא נחלקו אלא ביום י"א יום שב"ש אומרים מטמאין משכב ומושב וחייבין בקרבן ובית הלל פוטרין מקרבן 
אמרו להן ב"ש לב"ה מ"ש יום י"א מיום תוך י"א אם שיוה לו לטומאה לא ישוה לו לקרבן 
אמרו להן ב"ה לב"ש לא אם אמרת בתוך י"א יום שכן יום שלאחריו מצטרף עמו לזיבה תאמרו ביום י"א שאין יום שלאחריו שנצטרף עמו לזיבה 
אמרו להם בית שמאי השוו מדותיכם אם שיוה לו לטומאה ישוה לו לקרבן ואם לא שיוה לו לקרבן לא ישוה לו לטומאה 
אמרו להם ב"ה אם הביאנוהו לידי טומאה להחמיר לא נביאהו לידי קרבן להקל 
ועוד מדבריכם אתם נושכין שאתם אומרין טבלה יום שלאחריו ושמשה ואח"כ ראתה מטמא משכב ומושב ופטורה מן הקרבן אף אתם השוו מדותיכם אם שיוה לו לטומאה ישוה לו לקרבן
ואם לא שיוה לו לקרבן לא ישוה לו לטומאה אלא להחמיר ולא להקל הכא נמי להחמיר ולא להקל 
אמר רב הונא משכבה ומושבה שבשני ב"ש מטמאין אע"פ שטבלה אע"פ שלא ראתה מאי טעמא כיון דאילו חזיא מטמאה השתא נמי מטמיא 
אמר רב יוסף מאי קמ"ל תנינא טבלה יום שלאחריו ושמשה את ביתה ואח"כ ראתה ב"ש אומרים מטמאה משכבות ומושבות ופטורה מן הקרבן 
אמר רב כהנא ראתה שאני 
אמר רב יוסף וכי ראתה מאי הוי ראייה דנדה היא 
א"ל אביי לרב יוסף רב כהנא הכי קא קשיא ליה בשלמא היכא דראתה גזרינן ראייה דנדה אטו ראייה דזבה אלא היכא דלא ראתה מאי נגזר בה 
ועוד תנן הרואה ראייה אחת של זוב ב"ש אומרים כשומרת יום כנגד יום ובה"א כבעל קרי
ותניא המסיט את הראייה ב"ש אומרים תולין וב"ה מטהרין 
משכבות ומושבות שבין ראייה ראשונה לראייה שנייה ב"ש תולין וב"ה מטהרין 
וקתני רישא הרואה ראייה אחת של זוב ב"ש אומרים כשומרת יום כנגד יום אלמא שומרת יום כנגד יום לב"ש תולין 
לא תימא שומרת יום כנגד יום אלא אימא כבועל שומרת יום כנגד יום 
מאי שנא איהו דלא מטמא משכב ומושב ומאי שנא איהי דמטמיא ליה 
איהו דלא שכיחי ביה דמים לא גזור ביה רבנן איהי דשכיחי בה דמים גזור בה רבנן 
ומאי שנא משכב ומושב דמטמיא ליה ומאי שנא בועל דלא מטמיא ליה 
משכב ומושב דשכיח מטמיא ליה בועל דלא שכיח לא מטמיא 
תנן טבלה יום שלאחריו ושמשה הרי זו תרבות רעה
מגען ובעילתן תלויין מאי לאו דברי הכל היא 
לא ב"ה היא דתניא אמר להם רבי יהודה לב"ה וכי לזו אתם קורין תרבות רעה והלא לא נתכוון זה אלא לבעול את הנדה נדה ס"ד 
אלא אימא לבעול את הזבה זבה ס"ד אלא אימא לבעול שומרת יום כנגד יום 
איתמר עשירי רבי יוחנן אמר עשירי כתשיעי מה תשיעי בעי שימור אף עשירי בעי שימור 
ר"ל אמר עשירי כאחד עשר מה אחד עשר לא בעי שימור אף עשירי לא בעי שימור 
איכא דמתני לה אהא אמר לו רבי אלעזר בן עזריה לר"ע אפי' אתה מרבה בשמן [בשמן] כל היום כולו איני שומע לך אלא חצי לוג שמן לתודה ורביעית יין לנזיר ואחד עשר יום שבין נדה לנדה הלכה למשה מסיני 
מאי הלכה ר' יוחנן אמר הלכה י"א ר"ל אומר הלכות אחד עשר 
ר' יוחנן אמר הלכה אחד עשר אחד עשר הוא דלא בעי שימור הא לאחריני עביד שימור ור"ל אמר הלכות אחד עשר לא אחד עשר בעי שימור ולא שימור לעשירי הוי 
הני הלכות נינהו הני קראי נינהו דתניא יכול הרואה ג' ימים בתחילת נדה רצופים תהא זבה 
ומה אני מקיים (ויקרא טו, יט) אשה כי תהיה זבה דם יהיה זובה ברואה יום אחד (אבל הרואה ג' ימים בתחילה תהיה זבה) תלמוד לומר}

\newsection{דף עג}
\twocol{(ויקרא טו, כה) בלא עת נדתה (על נדתה) סמוך לנדתה 
ואין לי אלא סמוך לנדתה מופלג לנדתה יום אחד מנין ת"ל (ויקרא טו, כה) או כי תזוב 
אין לי אלא יום אחד מנין לרבות מופלג שנים שלשה ארבעה חמשה ששה ושבעה שמונה תשעה עשרה מנין 
אמרת מה מצינו ברביעי שראוי לספירה וראוי לזיבה אף אני אביא העשירי שראוי לספירה וראוי לזיבה 
ומנין לרבות אחד עשר ת"ל בלא עת נדתה יכול שאני מרבה אף שנים עשר אמרת לאו 
ומה ראית לרבות אחד עשר ולהוציא שנים עשר מרבה אני אחד עשר שראוי לספירת או כי תזוב ומוציא אני י"ב שאין ראוי לספירת או כי תזוב 
ואין לי אלא שלשה ימים שני ימים מנין ת"ל ימי יום אחד מנין ת"ל כל ימי
טמאה מלמד שמטמאה את בועלה כנדה היא היא מטמאה את בועלה ואין הזב מטמא מה שהוא בועל 
והלא דין הוא ומה היא שאינה מטמאה בראיות כבימים מטמאה את בועלה הוא שמטמא בראיות כבימים אינו דין שמטמא מה שהוא בועל ת"ל היא היא מטמאה את בועלה ואין הזב מטמא מה שהוא בועל 
ומנין שהוא עושה משכב ומושב ת"ל (ויקרא טו, כו) כמשכב נדתה 
ואין לי אלא שלשה ימים שני ימים מנין ת"ל ימי יום אחד מנין ת"ל כל ימי 
ומנין שסופרת אחד לאחד ת"ל יהיה לה יכול תספור שבעה לשנים ודין הוא ומה הוא שאין סופר אחד לאחד סופר שבעה לשנים היא שסופרת אחד לאחד אינו דין שתספור שבעה לשנים ת"ל יהיה לה אינה סופרת אלא יומה
אלמא קראי נינהו לר"ע קראי לר' אלעזר בן עזריה הלכתא 
א"ל רב שמעיה לר' אבא אימא ביממא תהוי זבה בליליא תהוי נדה א"ל עלך אמר קרא (ויקרא טו, כה) על נדתה סמוך לנדתה סמוך לנדתה אימת הוי בליליא וקא קרי לה זבה 
תנא דבי אליהו כל השונה הלכות בכל יום מובטח לו שהוא בן העולם הבא שנאמר (חבקוק ג, ו) הליכות עולם לו אל תקרי הליכות אלא הלכות
\par \par {\large\emph{הדרן עלך תינוקת וסליקא לה מסכת נדה}}\par \par 
}

\end{document}
