\documentclass[12pt, openany]{book}
\usepackage[
paperheight=9in,
paperwidth=6in,
top=0.5in,
bottom=0.5in,
inner=0.7in,
outer=0.5in,
marginparsep=0.1in,
headsep=16pt
]{geometry}

\newcommand{\texttitle}{שו״ת בנין ציון}\usepackage{titlesec}
\renewcommand{\partname}[1]{}
\usepackage{resources/unnumberedtotoc}

\usepackage{fancyhdr}
\pagestyle{fancy}
\fancyhf{}
\fancyhead[LO,RE]{\thepage}
\fancyhead[CO]{\chapname}
\fancyhead[CE]{\texttitle}

\usepackage{paracol}
\usepackage{anyfontsize}
\usepackage{ragged2e}
\usepackage{polyglossia}
\usepackage{multicol}
\usepackage{hyperref}
\usepackage[marginal]{footmisc}
\usepackage[titles]{tocloft}
\usepackage{xifthen}
\usepackage{graphicx}
\usepackage{dblfnote}\DFNalwaysdouble

\setdefaultlanguage{hebrew}
\setotherlanguage{english}
\usepackage{fontspec}
\setmainfont{Times New Roman}
\newfontfamily\englishfont{Times New Roman}
\setsansfont{Aharoni}

\newcommand{\sethebfont}{
\fontsize{11pt}{13.8pt} \selectfont
}

\newcommand{\setengfont}{
\fontsize{11pt}{13.8pt} \selectfont
}

\newcommand{\LTRmark}{‎}

\newcommand{\hebeng}[2]{
	{\sethebfont #1}
	
	%\vspace{0.5\baselineskip}
	{\beginL\englishfont\sethebfont{\raggedright #2 \hfill} \LTRmark\endL}
	
	\vspace{\baselineskip}
}

\newcommand{\twocol}[1]{
	{\sethebfont \begin{multicols}{2}
			#1
	\end{multicols}}	
}

\newcommand{\textblock}[1]{
{\sethebfont #1\\}	
}

\setlength{\parskip}{6pt}
\setlength\parindent{0in}

\newcommand{\chapname}{}
\newcommand{\sectname}{}

\newcommand{\newchap}[1]{
	\addcontentsline{toc}{chapter}{#1}
	\renewcommand{\chapname}{#1}
		\begin{center}
			\textbf{%
\fontsize{16pt}{16pt}\selectfont
				#1}
		\end{center}
}

\let\footnoterule\relax
\setlength\premulticols{10\baselineskip}
\setlength{\columnsep}{0.25in}

\newcommand{\newsection}[1]{
	%\addcontentsline{toc}{section}{#1}
	\renewcommand{\sectname}{#1}	
	\vspace{-\baselineskip}
	\begin{center}
		\textbf{%
\fontsize{16pt}{16pt}\selectfont
			#1}
	\end{center}
	\vspace{-\baselineskip}
	\nopagebreak
}

\newcommand{\footnotecomment}[1]{
	\renewcommand\thefootnote{}
	\footnote{\textsf{#1}}}

\newcommand{\parencomment}[1]{\footnotesize (#1)}

\newcommand{\blockcomment}[2]{ 
\vspace{\baselineskip}
\newsection{#1}
\sethebfont	\textsf{#2}
\vspace{\baselineskip}}

\newcommand{\commenta}[1]{\footnotecomment{#1}\hspace{0em}}

\newcommand{\vsnum}[1]{(\hebrewnumeral{#1})\space}
\newcommand{\vsnumeng}[1]{(#1)\space}

\begin{document}
\frontmatter
\pagenumbering{roman}

\newcommand{\oneline}[1]{%
	\newdimen{\namewidth}%
	\setlength{\namewidth}{\widthof{#1}}%
	\ifthenelse{\lengthtest{\namewidth < \textwidth}}%
	{#1}% do nothing if shorter than text width
	{\resizebox{\textwidth}{!}{#1}}% scale down
}

\title{\oneline{\hspace*{0.5in}\texttitle\hspace*{0.5in}}}

\author{}

\date{}

\maketitle

\begin{minipage}[b][\textheight][b]{\textwidth}\englishfont\footnotesize
	\begin{english}
		\vfill
		The following book includes:
\begin{itemize}
\item[$\bullet$] Binyan Tziyon, Altona, 1868
\item[$\bullet$] License: Public Domain
\item[$\bullet$] Source: \url{http://primo.nli.org.il/primo_library/libweb/action/dlDisplay.do?vid=NLI&docId=NNL_ALEPH001167127 }
\end{itemize}
		It was retrieved from Sefaria on \today\space \texthebrew{(\Hebrewtoday)}.  It was typeset and formatted by Ktavi.
		\clearpage
		
	\end{english}
\end{minipage}

\titleformat{\chapter}[hang]{\huge\bfseries}{\thechapter.}{1em}{}
\titlespacing*{\chapter}{0pt}{-3em}{1.1\parskip}
\titlelabel{\thetitle\quad}
%\addtocontents{toc}{\protect\vspace{-\baselineskip}}
\addtocontents{toc}{\protect\begin{multicols}{2}}
%\vspace*{-5\baselineskip}
\tableofcontents


\clearpage
\mainmatter
\pagenumbering{arabic}

\newchap{סימן א}
\begin{multicols}{2}
אלטאנא, תמוז תרכ״ב. להרה״ג וכו׳ מה׳ צבי הירש קאלישער נ״י בק״ק טהארן יע״א.\\\vspace{0pt}

על דבר אשר העלה מעכ״ת נ״י בספרו היקר שחובה מוטל על כל ישראל להשתדל גם בזמן הזה טרם ביאת משיחנו להקריב קרבנות במקום המקדש על ידי בנין המזבח כמו שהקריבו בימי עזרא טרם בנין בית שני כדתנן סוף עדיות ורצה בחסדו להביע לו דעתי על זה.\\\vspace{0pt}

הנני להשיב כפי עניות דעתי: ראיתי למעכ״ת נ״י שמביא שלש ראיות לדעתו ולענ״ד לא בלבד שאינן מחזיקות דעתו באמת אלא לפי ענ״ד יש מהן ראיות להיפך ואפתח במה שקרא ראיה גדולה וז״ל תנן בעדיות פ״ח אמר רבי יהושע מקובל אני מר׳ יוחנן בן זכאי ששמע מרבו ורבו מרבו הלכה למשה מסיני שאין אליהו בא לטמא ולטהר לרחק ולקרב אלא לקרב את המרוחקים בזרוע כו׳ ופי׳ הרע״ב שאין אליהו בא לברר ספק המשפחות מי נטמע אלא יניחם והם כשרים כו׳ ויש מן הפלא על מאמר זה וכי הלכתא למשיחא ולמה הוצרכו הלכה למשה מסיני על דבר שהוא לעתיד לבא הלא בעת יבא אליהו וכל קדושים עמו הלא ידעו אז מה יעשו ואין לנו שום נפקותא בהלמ״מ כלל וכן מצינו בכמה מקומות שמקשה הש״ס הילכתא למשיחא אך לפי דרכינו יהי׳ תועלת עצום בהלכה זו כי כאשר יהי׳ ראשית הקבוץ בלי נביא ונזכה לרשות להקריב קרבנות ויהי׳ מזבח בנוי בלא נביא ואנחנו לא נדע מה נעשה לריח נחוח לד׳ בשביל שנחכה לנביא להודיענו טהרת הכהנים ויחוסם וכאשר לא יהי׳ קרבן לכפר לא יופיע אור הנבואה וא״כ תהי׳ גאולה קצת מן הנמנעות שאין קרבן בלי אלי׳ ואין אלי׳ בלי קרבן לזה גדול העצה ורב העליליה הקדים להודיענו ע״י הללמ״מ שלא נחכה לאלי׳ בענין זה כי הוא לא יברר טהרת המשפחות רק יניח כל הכהנים בחזקת כשרותם ואז נקריב קרבן בלי מיחש שמא יופסל כהן העובד עכ״ל דמר נ״י ותמהני שכתב שמקשה הש״ס בכמה מקומות הילכתא למשיחא ולא נודע לי רק בשני מקומות שמקשה בגמרא כן (בסנהדרין דף נ״א ובזבחים דף מ״ה) ובשני המקומות האלה כבר הקשה התוספ׳ כעין קושיתו בסנהדרין כתבו בזה הלשון והא דפסקינן הלכה כר׳ יוסי בפרק עשרה יוחסין (דף ע״ב) בעתידים ממזרים ליטהר נפקא מינה בזמן הזה שלא להתרחק ממשפחות שאינן ידועות דבממזר שאינו ידוע מיירי וכעין זה כתבו גם בזבחים הרי שבתירוץ זה מתורצת גם קושיתו מאי נפקותא בהלכה למשה מסיני שאין אלי׳ בא לטמא ולטהר לרחק ולקרב בזמן הזה הלא הנפקותא הוא כמו שכתבו התוספ׳ שלא להתרחק ממשפחות הספיקות שמא יבא אלי׳ לטמא ונמצאו בניו ממזרים וא״כ לא בלבד שאין ראיה מקושיא זו לדעתו דמר אלא אדרבא מדלא תירצו התוספ׳ כן כתירוצו דנפקותא בפסק הלכה כרבי יוסי דעתידין ממזרין לטהר בממזר שאינו ידוע הוא שלא נחוש להעמיד כהנים להקריב קרבנות טרם יבא הנביא מחשש ספק ממזירות משמע דלא סבירא להתוספ׳ כן שיקריבו קרבנות טרם יעמוד הכהן לאורים ותומים על פי נביא.\\\vspace{0pt}

עוד כתב מר נ״י ועתה נביא עוד ראיה מן הספרי המובא בדברי רבינו הגדול הרמב״ן בפ׳ ראה בפסוק לשכנו תדרשו ובאת שמה וז״ל וטעם לשכנו תדרשו שתלכו לו מארץ מרחקים ותשאלו אנה דרך בית ד׳ ותאמרו איש אל רעהו לכו ונלכו הר בית אלקי יעקב ובספרי לשכנו תדרשו דרוש על פי נביא יכול תמתין עד שיאמר לך נביא ת״ל לשכנו תדרשו ובאת שמה דרוש ומצא ואח״כ יאמר לך נביא וכן אתה מוצא בדוד וכו׳ עכ״ל הרי שהזהיר לבל נמתין עד שיבוא נביא לאמר לנו עלו ודרשו ה׳ וזבחו לאלקים תודה רק אנחנו נעלה ונדרוש כאשר לאל ידינו ואח״כ יזכינו וישלח לנו נביא ונתחייבנו להשתדל בעוז ותעצומות וחיל עד מקום שידינו מגעת כמו שנאמר זכור ה׳ לדוד את כל עונותו כו׳ עד אמצא מקום לה׳ עכ״ל דמר נ״י וגם בזה אתמה שמהספרי אדרבה ראיה להיפך שזה לשון הספרי דרוש על פי נביא יכול תמתין עד שיאמר לך נביא ת״ל לשכנו תדרשו ובאת שמה דרוש ואתה מוצא ואח״כ יאמר לך נביא וכן אתה מוצא בדוד זכור ה׳ לדוד אשר נשבע לד׳ נדר לאביר יעקב אם אבא באוהל ביתי אם אעלה על ערש יצועי אם אתן שנת לעיני עד אמצא מקום לה׳ שלא תעשה אלא על פי נביא שנאמר ויבוא גד החוזה ביום ההוא ויאמר לו עלה והקם לה מזבח בגורן ארונה היבוסי ואומר ויחל שלמה לבנות בית ה׳ בירושלים בהר המוריה אשר נראה לדוד אביו עכ״ל הספרי הרי בפירוש אף על פי דיליף מקרא לשכנו תדרשו שלא נמתין על נביא לעלות לירושלים ולבקש מקום המקדש כמו שעשה דוד עם כל זה אסור להקריב אלא על פי נביא כדיליף מדוד שלא הקריב אפילו במזבח עד שבא גד החוזה ואמר לו עלה והקם לה׳ מזבח וכמו שכתב הספרי בפירוש שלא תעשה אלא על פי הנביא. והנה בתחלה עלה בדעתי שאולי יש לחלק שזה דוקא בעוד שלא נודע מקום המזבח לדוד לא הותר לו להקריב רק ע״פ נביא אבל מאחר שנודע מאז מותר להקריב גם שלא על פי נביא אמנם ראיתי שז״א שהרי דוד ידע מאז שישב עם שמואל בניות ברמה כל מקומות בנין הבית כאשר מיד ד׳ עליו השכיל וכן אמרינן סוכה (דף נ״ג) כשכרה דוד שיתין קפא תהומא וכו׳ וכתב רש״י דלא ס״ל דמששת ימי בראשית נבראו דהיינו מה דאמרינן שם (דף מ״ט) דשיתין של מזבח נבראו מששת ימי בראשית הרי דס״ל לרש״י דדוד כרה שיתין של מזבח בימי אחיתופל שקדם להך מעשה של ארונה וא״כ כבר ידע דוד מכוון מקום המזבח ואעפ״כ לא הותר לו להקריב רק על פי נביא א״כ כל שכן בזה״ז שאפילו מקום המזבח לא נודע לנו במכוון ועוד נלענ״ד דבלא״ה הספרי אינו ענין לזמן הזה דמהפסוק נראה שהי׳ הציווי לישראל כאשר כבר נחלו ארץ הקדושה לדרוש למקום שישכון ד׳ ולבא שמה עוד טרם יתגלה להם ע״פ נביא שיהי׳ שם משכן ד׳ וכן עשה דוד שבחר ירושלם למושב לו ואף שכבר ידע ע״י רוח הקדש שבהר המורי׳ תשכון כבוד ד׳ עם כל זה לא בנה מזבח ולא הקריב עליו עד שנאמר לו בשם ד׳ ע״י גד החוזה. ומכל זה ראי׳ שאסור לבנות מזבח ולהקריב עליו עד שיבא לנו ע״פ הנביא ציווי הקב״ה לזה:\\\vspace{0pt}

ועתה אבא אל ראי׳ שלישית שהביא מן הפסוק לדעתו שכתוב בישעי׳ נ״ה והביאותים אל הר קדשי ושמחתים בבית תפלתי עולותיהם וזבחיהם לרצון על מזבחי וגו׳ אמר בכל פעם שני הסוגים ראשית הגאולה ותכלית הגאולה והביאותים אל הר קדשי הוא ראשית הגאולה שיהי׳ רק קיבוץ אל הר הקודש ולא ביה״מ ואח״כ ושמחתים בבית תפלתי כי יתפללו שם רבים צדיקי עולם המתאספים ונקבצים שם ואמר עולותיהם וזבחיהם לרצון על מזבחי שיהי׳ רק מזבח בנוי בראשית הגאולה ולא בית המקדש ואח״כ ביתי בית תפלה יקרא לכל העמים עכ״ד והנה מה שפירש מר נ״י בזה הוא מפורש נגד גמרא דמגילה (דף י״ח) דאמרינן שם וכיון שנבנית ירושלים בא דוד שנאמר אחר ישובו בני ישראל ובקשו את ה׳ אלקיהם ואת דוד מלכם וכיון שבא דוד באתה תפלה שנאמר והביאותים אל הר קדשי ושמחתים בבית תפלתי וכיון שבא תפלה בא העבודה שנאמר עולותיהם וזבחיהם לרצון על מזבחי הרי בפירוש שקודם שיקויים עולותיהם וזבחיהם לרצון על מזבחי יבוא דוד שהוא משיח ויבנה ביה״מ והרי סידר ברכות י״ח של אנשי כנסת הגדולה מורה על ביטול דעתו דמר נ״י:\\\vspace{0pt}

אמנם בלא כל הנ״ל נ״ל ראי׳ שאסור להקריב בזמן הזה ואעתיק למעכ״ת מה שכתבתי בזה בשנת תר״ז להגאון אב״ד דק״ק ווארמס מ״ה קאפל הלוי נ״י בזה״ל: מה שהשיג מר נ״י על מה שהעיד בשאילת יעבץ (סי׳ פ״ט) שאולי בימי הנשיאים שהיו סמוכים לחרבן הבית וקרובים ואהובים למלכות עדיין הי׳ המזבח בנוי והקריבו קרבנות ציבור על דעת ר״י וסיעתו דמקריבין אע״פ שאין בית והביא ראיות שלא היו מקריבין הדין עמו בזה אך מה שנתן טעם מפני שלא הי׳ מזבח בנוי שבלי ספק נחרב בשעת חרבן לענ״ד טעם זה אינו מספיק דאכתי יקשה למה לא בנו מזבח דאף דמקדש לא יכלו לבנות מפני הגלות וגם אחר שניבא יחזקאל על בנין העתיד שיהי׳ ע״פ הדבור דוקא אבל כל זה לא עצר אותם לבנות מזבח במקום המקדש ולהקריב עליו וכמו שכתב בספר כפתור ופרח שר״ח מפריש ז״ל אמר לבוא לירושלים בשנת י״ז לאלף הששי ושיקריב קרבנות הקרבין בטומאה כגון קרבנות צבור ופסח והלא עוד יותר נקל הי׳ זה להתנאים שהיו בא״י והי׳ להם עוד אפר פרה (כדמוכח פ״ג דחגיגה) להקריב בטהרה אבל נלענ״ד שמן הכתוב חדלו מלעשות כן ע״פ מה שנ״ל לפרש מתניתן מגילה (דף כ״ח) ועוד אר״י ביה״כ שחרב אין מספידין בתוכו וכו׳ שנאמר והשמותי את מקדשיכם קדושתן אף כשהן שוממין וכבר יגע התוספ׳ י״ט לפרש ממה שדרש ר״י כן וכתב דה״ל למיכתב ואת מקדשיכם אשומם ע״ש ולא ידעתי דיוק בזה שהרי כל הפרשה מדבר לפעמים בלשון עתיד ולפעמים בלשון עבר עם ו׳ המהפך והרי מיד אח״כ כתוב כזה והשמותי אני את הארץ אכן העירני לפרש פי׳ אחר במשנה זו מה שראיתי ברמב״ם ה׳ בית הבחירה (פ״ו) שהביא דרשה זו דוהשמותי גם לענין מקדש ויליף מזה דקדושת מקדש לא בטלה ונ״ל שהוציא הרמב״ם כן אף דבמתניתן לא כתוב כן רק על ב״כ ממה דאיתא בת״כ מקדש מקדשיכם לרבות בתי כנסיות ובתי מדרשות פי׳ דדרש כן מדלא כתיב מקדשי או מקדשכם לשון יחיד דאז לא הי׳ במשמעו רק בית מקדש אבל מדכתיב מקדשיכם לשון רבים דרשינן גם בתי כנסיות והשתא מדקאמר בת״כ לרבות ב״כ נראה בפי׳ דעיקר קרא אמקדש קאי ולכן נ״ל דהוציא רבי יהודה דרשתו ממה דכתיב והשמותי את מקדשיכם ולא אריח בריח ניחחכם ואיך שייך ולא אריח דאם מקדש חרב ריח ניחח מניין ואין לומר באם יקריבו בבמה דז״א שהרי משבאו לירושלים לא הותרו במות עוד כדאמרינן זבחים (דף קי״ב) ועוד דאפילו בשעת היתר הבמות אין ריח ניחח בבמה כדאמרינן שם (דף קי״ט) ולכן דייק ר׳ יהודה מזה דקדושה ראשונה קדשה לע״ל וכיון דמאן דס״ל כן ס״ל בעדיות (פ׳ ח׳) מקריבין אע״פ שאין בית שייך שפיר ולא אריח דאפי׳ תקריבו במקום מקדש אחר שיחרב דאין בזה איסור חוץ כיון שקדושה ראשונה קדשה לע״ל וא״צ מחיצות מכ״מ לא אריח בר״נ וא״כ מוכח מזה דקדושת מקדש היא גם לאחר חרבן ומרבוי דמקדשיכם אתיא גם בתי כנסיות לכן שפיר דרש ר׳ יהודה קדושתן אף כשהן שוממין שהרי אתקשו ב״כ למקדש ולפ״ז עיקר דרשת רבי יהודה היא מדדרש מקרא דוהשמותי דקדושה ראשונה קדשה לע״ל וזהו מה שכתב הרמב״ם דמזה אתיא דקדשה לע״ל שלא מלבו הוציא דרשה זו אלא היא דרשת ר׳ יהודה במתניתן דפסק כוותי׳ [ומזה תשובה למה שכתב הטורי אבן במגילה דרבי יהודה ס״ל דלא קדשה לע״ל ולפ״ד אדרבה מוכח דסבירא ליה קדשה לע״ל ולמה שהקשה מברכות (דף ס״א) כתבתי ישוב במקום אחר] והשתא מה דקאמר רבי יהודה במתניתן שנאמר והשמותי את מקדשיכם לא נקט דרשה זו רק להשמיענו דבכלל זה גם ב״כ ואזיל לשיטתו לדרשתו שבת״כ דסתם ספרא רבי יהודה וממילא נשארו בקדושתן גם כשהם שוממין כמו מקדש דבדידי׳ אתיא כן מסיפא דקרא דולא אריח. והיוצא מדברינו דאפילו למ״ד דמקריבין אע״פ שאין בית דיליף כן מוהשמותי מכ״מ אמר הקב״ה ולא אריח בריח ניחחכם וכיון דאמרינן בזבחים (דף מ״ו) לשם ששה דברים הזבח נזבח ובכללם לשם ריח ולשם נחוח לא יכלו להקריב קרבנות לאחר חרבן כיון שאי אפשר לשם ריח ולשם ניחח ואם תאמר כיון שכן לענין מה אמרינן מקריבין אע״פ שאין בית הרי לא יכלו להקריב משום ולא אריח יש לומר דיש נפקותא לענין מה דאמרינן בזבחים (דף ק״ז) המעלה בזמן הזה ר״י אמר חייב ר״ל אמר פטור וכו׳ ע״ש א״כ חייב המעלה בחוץ א״נ י״ל דולא אריח הוא דוקא כשביהמ״ק שמם ע״פ רצון הקב״ה לא כן כשצוה הקב״ה לבנותו ועדיין לא נגמר אז מותר להקריב בלא בית כמו שהקריבו הרבה שנים בימי עזרא ועל זה קאי עיקר דברי ר״י שמעתי שהרי אמר אדברי ר״א שאמר שמעתי כשהיו בונין וכו׳ דהיינו בימי עזרא ועוד יש נפקותא לענין שאם הקריב בזה״ז שהזבח כשר ונאכל אם הקריב בטהרה ופסח אפילו הקריב בטומאה וכן יצא ידי נדרו וחובתו אם הקריבו דלשם ריח ולשם ניחח לא גרע מלשם זבח ולשם זובח דהיינו שינוי קדש ושינוי בעלים דג״כ כשר בדיעבד וכן בכל המקומות דמייתי הגמרא להקריב בזמן הזה כגון במכות (דף י״ט) דפריך אי ס״ל קדשה לע״ל אפילו בכור נמי הפירוש כן שמצד קדושה לע״ל יכול להקריב ואפילו לכתחלה כשיצו׳ נביא על כך אבל בלא זה לכתחלה אסור להקריב כיון דבפי׳ אמר הכתוב שלא יהי׳ לריח ניחח ולפענ״ד מזה הטעם לא הקריבו קרבנות אפילו התנאים דס״ל דקדושה ראשונה קדשה לע״ל.\\\vspace{0pt}

וגם אין לחלק בין זמן החרבן לזמן הזה דמאין לנו לומר שעתה יהי׳ לריח ניחח הלא כל דור שאין ביהמ״ק נבנה בימיו כאלו נחרב בימיו והנה זמן רב אחר כתבי כן להגאב״ד דק״ק ווארמס כתב לי הרה״ג וכו׳ מ״ה בנימין אויעראבאך נ״י הגאב״ד דק״ק האלבערשטאדט כשהעתיק ספר האשכול כ״י לרבינו אברהם ב״ד אב״ד ז״ל הקדמון מצא בהגהה פירוש המשנה דמגילה הנ״ל כאשר כתבתי ושמחתי מאוד שכוונתי לדעת הגאון ז״ל עכ״פ מזה ראי׳ שאין להקריב אפילו במזבח לבד עד שיבא נביא ויצו׳ על כך ומסכים זה עם דברי ספרי הנ״ל ועם סידור ברכות ש״ע ע״י אנשי כנסת הגדולה שבתחלה יבנה המקדש ע״י ב׳ החרשים שהוא משיח בן יוסף ומשיח בן דוד כדאמרינן סוכה (דף נ״ד) ויתקבצו נדחי ישראל ויקויים והביאותים וגו׳ ושמחתים בבית תפלתי ואח״כ יקריבו קרבנות לקיים עולותיהם וזבחיהם לרצון על מזבחי כן יה״ר בבא״ס. הקטן יעקב יוקב בלאאמ״ו מ״ה אהרן עטטלינגער ז״ל.\\\vspace{0pt}

\end{multicols}\newpage

\newchap{סימן ב}
\begin{multicols}{2}
ומדי דברי בקדושת הבית אחקור עוד אם יש תקנה לכנוס למקום הר הבית בזה״ז ועד איזה מקום מותר לכנוס:\\\vspace{0pt}

אם קדושת ירושלם ומקדש נוהגת בזמן הזה בזה יש פלוגתא בין הרמב״ם להראב״ד דהרמב״ם ה׳ בית הבחירה (פ׳ ו׳) פסק שקדושה ראשונה קדשה לעתיד לבוא שקדושת ירושלם והמקדש מפני השכינה ושכינה אינה בטלה והראב״ד שם השיג עליו וס״ל לא קדשה לע״ל ומסיים לפיכך הנכנס עתה שם אין בו כרת. והנה ראיות הראב״ד הם ממה דאמרינן מ״ש (פ׳ ב׳) דאם אין מקדש ירקב וממה דאמרינן בבא מציעא (דף נ״ג) דנפול מחיצות אמנם על ראי׳ ראשונה כבר תרצו המגן אברהם (סי׳ תקס״א) והמשנה למלך שם במקומו דאף דקדושה ראשונה קדשה לע״ל מכ״מ אם אין מזבח אין מקריבין ומקשינן מעשר לבכור כדאמרינן מכות (דף י״ט) ולכן ירקב והמ״ל הוסיף שכן כתב בפירוש גם הטור בי״ד (סי׳ של״א) וגם על ראי׳ השני׳ של הראב״ד רמז תשובה מדברי הרמב״ם (ספ״ו דה׳ מעשר שני) וכוונתו דדוקא היכי שקלטוהו מחיצות כבר ויצא ושוב נפול מחיצות בזה אמרינן דאפילו הושב המעשר לתוך המקום לא יאכל שם כיון שקלטוהו מחיצות כבר אבל אם בתחלה לא נתקדש המעשר ע״י מחיצות נאכל בירושלם אע״פ שאין חומה ועיין בפרי מגדים א״ח (סי׳ תקס״א) ועל מה שכתב הראב״ד שלא ידע מניין להרמב״ם הביא הכס״מ ב׳ ראיות להרמב״ם ממה דפליגי ר״א ור״י בקדשה לע״ל ור״י ס״ל דקדשה לע״ל וכוותי׳ קיי״ל לגבי ר״א ועוד הביא ראי׳ ממה דפליגי רבי יוחנן ור״ל בזבחים (דף ק״ז) במעלה בחוץ בזמן הזה ור״י ס״ל דחייב דקדושה ראשונה קדשה לע״ל וקיי״ל כר״י לגבי ר״ל ובאמת יש לתמו׳ איך נעלמו מהראב״ד ראיות מפורשות האלה עד שכתב על סברת הרמב״ם שלא ידע מניין לו ואולי י״ל כיון שהביא ראי׳ לשיטתו מסתם משנה דאם אין מקדש ירקב ומסתם גמרא דמוקי בנפל מחיצות לכן ס״ל דמה שיוצא מהכלל דר׳ אליעזר ור׳ יהושע הלכה כר״י נסתר מכח כמה סתם משנה דס״ל דלא קדשה לע״ל לפי שיטתו ומה שיצא מהכלל דר״י ור״ל הלכה כר״י נסתר ממה דס״ל סתם גמרא דנפל מחיצות אין אוכלין מ״ש לכן כתב שלא ידע מניין לו להרמב״ם לסמוך על ראיות הנ״ל כיון שיש ראיות מתנגדות. אמנם על מה שכתב הראב״ד לפיכך הנכנס עתה שם אין בו כרת כתב הכס״מ אני תמה דמעיקרא משמע כמספק אמרה ובסוף דבריו נראה דפשיטא לי׳ וחידוש גדול הוא והי׳ צריך להביא ראי׳ עכ״ל ולענ״ד י״ל דגם מסוף דברי הראב״ד נראה דאין לו פשיטות בדבר דיש לדקדק למה תלה הדבר באין בו כרת מה שאין לנו נפקותא דקמי׳ שמיא גליא ולמה לא כתב לפי שיטתו פסק הלכה לדידן שמותר עתה ליכנוס שם ולכן נלענ״ד ע״פ מה שכתבתי בספרי ע״ל על מה דאמרינן במכות (דף כ״ג) הלכה כרחב״ג דחייבי כריתות שלקו נפטרו מידי כריתתן שהקשתי שם דמה נפקותא בפסק הלכה זה במה שמסור לשמים אם פוטרין אותו מכריתות או לא ותרצתי דיש נפקותא רבה גם לדידן ע״פ מה שהוכחתי דחייב כרת אפילו אינו בר מלקות כגון מי שלא מל מכ״מ נקרא רשע ופסול לעדות ודלא כפרי מגדים ולכן אם לא קיי״ל כרחב״ג יהיו חייבי כריתות שלקו פסולים לעדות אם לא שבו לכן קמ״ל דקיי״ל כוותי׳ דפטורים וא״כ כשרים לעדות וכן נ״ל פה בביאור דברי הראב״ד דבאמת גם הראב״ד מסופק אם יש לסמוך על ראיותיו או על ר׳ יהושע ור׳ יוחנן כנראה מתחלת דבריו ולכן לא התיר לכנוס לשם דמידי ספק דאורייתא לא נפקא רק על מה שיוצא משיטת הרמב״ם דמי שנכנס עתה לשם חייב כרת וא״כ רשע הוא ופסול לעדות על זה כתב דאין בו כרת דהיינו מספק אין לדונו כחייב כרת שפסול לעדות כן נלענ״ד להשוות סוף דברי הראב״ד עם תחלת דבריו. אכן אפילו אם כוונת הראב״ד להתיר לכתחילה דעתו דעת יחידית היא שעם שיטת הרמב״ם הסכימו הספר החינוך והטור שהזכרתי לעיל וגם המג״א הנ״ל כתב שכשיטת הרמב״ם משמע בתוספ׳ בשבועות וכן כתוב בסה״ת וכ״כ באגודה ולכן הסכימו כל הפוסקים שמי שנכנס בזה״ז למקום המקדש חייב כרת. אמנם יש לחקור אם יש עוד שאר חיובים בזה. והנה אף שטמא מת מותר לכנוס להר הבית מכ״מ מי שטומאה יוצאה מגופו כגון בעל קרי שלא טבל וזב שלא טבל במים חיים וספר ז׳ נקיים וכן זבה נדה ויולדת שלא ספרו וטבלו כראוי אם נכנסו להר הבית חייבין מלקות מולא יבוא אל תוך המחנה כמבואר בגמרא בכמה מקומות וברמב״ם ה׳ ביאת מקדש (פ׳ ג׳) ואפילו טבלו כראוי ונכנסו כשהם טמאי מתים לתוך העזרה שהיא מחנה שכינה חייבין מלקות מולא יטמאו את מחניהם וחייבין כרת משום כי את מקדש ה׳ טמא כמבואר ברמב״ם (שם) ואפילו טהור גמור שנכנס שלא לצורך עבודה להיכל לוקה ואם נכנס לקדש הקדשים חייב מיתה ב״ש כן פסק הרמב״ם שם (פ״ב) אכן יש להסתפק אם זה דוקא בכהן או אפילו בזר שממה שכתב הרמב״ם שם והוזהרו כל הכהנים שלא יכנסו לקדש או לק״ק וכו׳ משמע שכהנים דוקא הוזהרו וכן משמע מת״כ שהביא הכס״מ שם שדרש מאל אהרן אחיך לרבות את הבנים משמע דכהנים דוקא הוזהרו אכן יש לדייק אם כהנים שקדושים הם חייבין ישראל לא כש״כ ולא שייך בזה שאני כהנים שרבה בהם הכתוב מצות יתירות כדאמרינן ביבמות (דף קיד) דזה שייך דוקא לענין מה שמוסיף איסור לכהנים מפני קדושתן כדאיירי שם אבל הכא הוי אפכא אם אפילו כהנים קדושים הוזהרו ישראלים לא כל שכן ואעפ״י דאין מזהירין ועונשין מן הדין זה דוקא שלא נילף עונש ואזהרה לחיובא אבל איסור שפיר ילפינן בק״ו. וכל האזהרות האלה הם להרמב״ם שעמו הסכימו הפוסקים גם בזה״ז אף שאין כאן מחיצות שקדושת המקום לא בטלה. וכן כתב גם הספר החנוך פ׳ אחרי. ולכן מי שנכנס עתה למקום הר הבית בלא טבילה כראוי ואם הוא זב בלא טבילה במים חיים וספירת ז׳ נקיים חייב מלקות ואפילו טבל ונכנס למקום שהי׳ שם עזרת ישראל חייב כרת כיון שכולנו טמאי מתים ואם נכנס למקום היכל וקדשי קדשים אם כהן הוא ודאי עוד חייב מלקות מולא יבא בכל עת אל הקדש ועל כניסה במקום ק״ק שהוא המקום ששם עדיין אבן השתי׳ נראה חייב מיתה ב״ש ואף דכבר חייב כרת משום כניסה בטומאה מכ״מ יש נפקותא אם שב בתשובה על עון טומאה ולא על עון כניסה לק״ק ועיין בשבועות (דף י״ז) ובתוספ׳ שם ואם ישראל הוא יש לספק אם נוסף לו עבירת איסור על איסור מלקות וכרת שחייב משנכנס למקום העזרה גם על הכניסה למקום היכל וקדשי קדשים. והנה סוף מכות דאמרינן כשראו שועל יוצא מבית ק״ק אמרו מקום שנאמר בו והזר הקרב יומת ע״ש משמע שגם בזר יש מיתה ואע״פ שנקט פסוק דוהזר דכתיב גבי זר העובד במקדש ולא פסוק דואל יבוא בכל עת כמו שהקשתי בספרי ע״ל שם י״ל דנקט הדרשה הפשוטה יותר דכן דרך הש״ס כמשכ׳ התוספ׳ בסוכה (דף כ״ד) אבל עכ״פ משמע משם דגם זר הנכנס למקום ק״ק חייב מיתה ב״ש.\\\vspace{0pt}

ועוד אזכיר מה שתמהתי על מה שכתב בשו״ת הרדב״ז סי׳ תרצ״א אחר שהזכיר שם שהדבר ברור שתחת כיפה של הישמעאלים שם אבן השתיה בלא ספק הנקרא אצלם אל סכרא כתב וז״ל הלכך מי שעולה בעליות אשר לרוח מערב או הנכנס לראות מהפתח אשר לצד מערב צריך לשער שיהי׳ בינו לבין הכיפה יותר מאחד עשר אמה שכך הי׳ בין כותל מערבי של העזרה לכותל ההיכל ועובי הכותל כי הכותלים נתקדשו עכ״ל וזה תימא איך נתן שיעור י״א אמה שהרי הי״א אמה היו מכותל העזרה עד כותל ההיכל כמבואר במדות (פ׳ ה׳) ומתחלת כותל ההיכל עד חלל בית ק״ק היו ד׳ כותלים שרחבן עם המקום פנוי י״ז אמות כמבואר שם (פ׳ ד׳) ואח״כ הי׳ חלל בית ק״ק כ׳ אמה ובאמצעו הי׳ האבן השתי׳ כמשכ׳ התוספ׳ י״ט יומא (פ׳ ה׳) א״כ ממקום אבן השתי׳ עד סוף כותל העזרה הי׳ קרוב לשלשים ושמנה אמות ואיך נתן ברדב״ז שיעור של י״א אמה שמותר לקרוב לכיפה ואם הכיפה רחבה ומגיע לצד מערב מהלאה לאבן שתי׳ עכ״פ הי׳ לו לשער שאין לקרב אל מקום אבן השתי׳ בתוך ל״ח אמה ולא לתלות בכיפה שלא ידענו מקום רחבה וצ״ע. כנלענ״ד הקטן יעקב.\\\vspace{0pt}

\end{multicols}\newpage

\newchap{סימן ג}
\begin{multicols}{2}
ב״ה אלטאנא יום ו׳ ד׳ שבט תרט״ז לפ״ק. על מה שכתבתי בענין קדושת מקדש בזה״ז כנ״ל כתב אלי בני התורני מ״ה בן ציון שי׳.\\\vspace{0pt}

ראיתי שהביא אאמ״ו נ״י הרמב״ם שכתב שקדושה ראשונה קדשה לשעתה וקל״ל שקדושת ירושלם והמקדש מפני השכינה ושכינה אינה בטלה וקשה לענ״ד שהרי אמרינן (ס״פ קמא דיומא) דמקדש שני הי׳ חסר ה׳ דברים וקחשיב בתוכם שכינה הרי דהשראת שכינה לא הי׳ בבית שני.\\\vspace{0pt}

עוד ראיתי שתימה אאמ״ו נ״י על הרדב״ז שנתן שיעור להרחיק מכיפה שתחתי׳ אבן שתי׳ אחד עשר אמה והקשה שהרי י״א אמה היו מכותל העזרה עד כותל ראשון של היכל ומשם עד חלל קדשי קדשים היו עוד י״ז אמות ואם כי זה תמו׳ מאוד לענ״ד לא הבנתי גם דברי רש״י בברכות (דף ל׳ ע״א) שכתב ד״ה אחורי בית הכפורת במערב העזרה היו י״א אמה חצר מכותל בית קדשי הקדשים לכותל מערבי של עזרה עכ״ל הרי שגם רש״י כתב כהרדב״ז שהיו מכותל בית ק״ק עד כותל העזרה רק י״א אמה והרי ממתניתן דמדות (פ״ד) מוכח שהיו כ״ח אמות: כן הקשה לי בני שי׳. ועל כי הדברים האלה ראויים לעורר עליהם לכן אשיב בזה כפי ענ״ד.\\\vspace{0pt}

הנה מה שכתב הרמב״ם שכינה לא בטלה נראה שיצא לו כן ממה דאיתא במדרש תנחומא אמר ר׳ שמואל בר נחמני עד שלא חרב ביהמ״ק היתה השכינה נתונה בהיכל שנאמר ה׳ בהיכל קדשו משחרב ביהמ״ק ה׳ בשמים כסאו סלק שכינתו בשמים א״ר אלעזר בן פדת בין חרב בין לא חרב אינו זז ממקומו שנאמר והי׳ עיני ולבי שם כל הימים וכן הוא אומר ויענני מהר קדשו שאפילו הר הרי הוא בקדושתו אמר ר׳ אחא לעולם אין השכינה זזה מכותל מערבי של ביהמ״ק שנאמר הנה זה עומד אחר כתלנו עכ״ל וכן איתא עוד במדרש תנחומא כי עתה תצאי מקרי׳ ושכנתי בשדה אע״פ שחרב בהמ״ק לא זזה השכינה משם ושכנתי כתיב ואין שדה אלא ציון שנאמר ציון שדה תחרש עכ״ל הרי מכל אילו יצא שלא זזה שכינה ממקדש אפילו לאחר חרבנו וכמשכ׳ הרמב״ם אכן לא בלבד שקשה על זה מגמרא דיומא דאמרינן שחסרה שכינה בבית שני אלא שיש לכאורה גם סתירה לזה מתפלת ש״ע שאנו מתפללים ומברכים בכל יום ג״פ המחזיר שכינתו לציון ואם לא זזה שכינה היאך שייך שתחזור.\\\vspace{0pt}

אמנם אף שהדברים האלה עומדים ברומו של עולם עם כל זה גם לבארם ע״פ פשוטם רמזו לנו רז״ל במה דאמרינן בסנהדרין (דף ל״ט) א״ל כופר לר״ג אמריתו כל בי עשרה שכינתא שריא כמה שכינתא איכא קריא לשמעי׳ מחא בי׳ באפתקא א״ל אמאי עייל שמשא בביתא א״ל שמשא אכולי עלמא ניחא א״ל ומה שמשא דחד מן אלף אלפי רבוא שמשי דקמי׳ קב״ה ניחא לכולי עלמא שכינתא דקב״ה עאכ״ו עכ״ל וביאור המשל נלענ״ד ששני ענינים נאמר על השמש בדרך זר נאמר שהשמש בא אל הבית או החדר וכן בלשון חז״ל סוכה (דף כ״ז) הגיע חמה לסוכה והלא החמה נרחקה כמה אלפי רבבות מילין מהארץ הרי שבשמוש הלשון מכנים האור הבא מן השמש בשם השמש וכן מה שנאמר שהשמש או האור בא אל הבית וכי בא כמי שנעתק ממקום אל מקום והלא כבר זרח בכל מקום וגם על הבית אכן מפני שהבית בהסגרו מנע אורו מלהגיע לשם נאמר כשנפתח פתח או החלון שבא לשם השמש אף שאין זה אלא הסרת המניעה ומאחד מאלפי רבבות שמשי הקב״ה נשא משל שכן הענין בשנאמר שכינה שורה במקום לא שנאמר בזה ששכינה הנקרא כבוד ה׳ בעצמו שורה במקום כמו שנאמר וישכון כבוד ה׳ על הר סיני או וכבוד ה׳ מלא את המשכן אלא שאור הקדושה הנאצל והנזרח מכבוד ה׳ אשר מקומו נעלם גם ממשרתי מעלה כמו שנאמר משרתיו שואלים זל״ז אי׳ מקום כבודו הוא אשר זורח על המקום ההוא כמו שנאמר שהשמש בא והכוונה על אור השמש וגם על האור הקדושה בעצמו לא יאמר שבא אל מקום כמו שנאמר על דבר או אדם שהניע ממקום אל מקום שהרי נאמר מלא כל הארץ כבודו ומה שנאמר שבא הכוונה שאנחנו מסירים מעצמנו המניעה שמנעה הקדושה לזרוח עלינו וכמו שאמרנו באור השמש ועל ככה נאמר שהשכינה שורה בעשרה שכל מקום שנתקבצה שם עדה קדושה לעבודת ה׳ ומסירים מהם ענין העולם הזה העכור המונע זריחת אור הקדושה שם שורה השכינה שיש מקום נפתח לאור הקדושה לזרוח על מצפוני הלבבות ולהאיר נשמתם וכמו שירבה ויתמעט אור הבית לפי מספר הפתחים והחלונות אשר יפתחו אם מעט או הרבה ואם גדולים או קטנים כן הענין בהשפעת קדושת אור השכינה דכל בי עשרה שכינה שורה וברוב עם הדרת מלך וכמו שירבה אור השמש על הארץ אם זורח מול נגדו משאם יזרח ממרחק בשפוע אשר על כן יגדל האור בצהרים מבבקר ובערב כמו כן באור השכינה שמי שמקיים בעצמו שויתי ה׳ לנגדי תמיד מקבל שפע קדושה יותר מזולתו שאינו מחשיב ה׳ תמיד נגדו. וע״פ אשר ביארנו נאמר שם שכינה על ב׳ בחינות על כבוד ה׳ מקור הקדושה ועל אור הנאצל השפעת הקדושה. וזה החילוק שבין בית ראשון לבית שני שבמשכן וכן בבית ראשון שכן כבוד ה׳ בעצמו כדכתיב וכבוד ה׳ מלא את המשכן ובמקדש כי מלא כבוד ה׳ את בית ה׳ ועל ענין זה אחז״ל על פסוק בין שדי ילין שצמצם שכינתו בין שני בדי ארון וכן אמרינן סנהדרין (דף ז׳) מעיקרא כתיב ונועדתי לך וכו׳ ולבסוף השמים כסאי איזה בית וכו׳ ע״ש ועל זה אמרינן ר״ה (דף ל״א) עשר מסעות נסעה שכינה מקראי שבהפסוקים שהביא לזה נכתב שם כבוד ה׳ שהוא מקור הקדושה כדכתיב ויעל כבוד ה׳ מעל תוך העיר עד שעלה וישב במקומו כדכתיב אלכה ואשובה אל מקומי שהוא מקום הנעלם גם ממשרתיו ולא שבה בבית שני כי שם לא נאמר בשום מקום שחנה כבוד ה׳ אשר על כן אמרו חסרה שכינה בבית שני אבל שם שרתה שכינה בענין השני שממקומו השפיע אור הקדושה על הבית וזה לא נפסק גם בחרבנו ולא יפסוק לעולם כדילפינן מוהי׳ עיני ולבי שם כל הימים וגדלה השפעת אור הקדושה שם מבכל מקום כי מקום המקדש הוא מול שער השמים אשר לא יסגר ושם זורח אור הקדושה תמיד נגדו ביתר שאת וביתר עז ולכן יפה כתב הרמב״ם ששכינה אינה בטלה שאור הקדושה אשר קדש הבית לא נבטל ולא יבטל שעל זה נאמר מעולם לא זזה שכינה מכותל מערבי שהוא אור הקדושה הנעלם אבל כבוד ה׳ מקור הקדושה הוא הנגלה כדכתיב ולא יכול משה וגו׳ וכן נאמר בבית אשר בנה שלמה ולא יכלו הכהנים וגו׳ וגם בעת הנסיעה כתיב והחצר מלאה את נגה כבוד ה׳ וזה גילוי השכינה אשר עליו ניבא ישעי׳ שיחזור אלינו לעת הגאולה ב״ב ונגלה כבוד ה׳ וגו׳ ועל זה אנו מתפללים ותחזינה עיננו בשובך לציון ומברכים המחזיר שכינתו לציון שהיא השכינה בבחינה הראשונה מקור הקדושה.\\\vspace{0pt}

\end{multicols}\newpage

\newchap{סימן ד}
\begin{multicols}{2}
ומה שהקשה בני שי׳ על דברי רש״י בברכות לענ״ד י״ל דהנה שם בית הכפרת שנשנה בברייתא דברכות ודאי משמע שהוא בית קדשי קדשים הבית ששם הכפרת ושם זה מצאנו ג״כ במתניתן דמדות (פ״ה) דתנן שם ההיכל מאה אמה ואחת עשרה אמה לאחורי בית הכפרת ע״ש ובכלל מאה אמה של היכל הם גם המחיצות עם הלשכות של היכל שהיו בין מחיצת ק״ק לי״א אמה של עזרה כמבואר שם (פ״ד) ואעפ״כ קרא לי״א אמה אחורי בית הכפרת ולא אחורי ההיכל ונראה שהברטנורא שם הרגיש בזה ולכן כתב שם מכותל החצון של היכל לצד מערב עד הכותל מערבי של עזרה הי׳ י״א אמה שלחלל עם עובי הכותל קרוי אחורי בית הכפרת עכ״ל הרי שג״כ מפרש דבית הכפרת הוא קדשי קדשים ומה דקראו התנא אחורי בית הכפרת מפני שעובי הכתלים עם החלל שהם הלשכות שביניהם הכל נחשב כמחיצה אחת ונקרא אחורי בית הכפרת עד החצר של י״א אמה שבין סוף כותל ההיכל לכותל מערבי של עזרה. ולכן בברייתא בברכות שם ששנה ג״כ הלשון אחורי בית הכפרת פי׳ רש״י שהיו י״א אמה חצר מכותל בית ק״ק עד כותל העזרה שמפרש בית הכפרת בית ק״ק ולשון אחורי היינו שכל כותלי ההיכל נידונים ככותל בית ק״ק וכלשון המשנה דמדות ולכן דברי רש״י א״ש אבל ודאי גם רש״י ס״ל שהי׳ מחלל ק״ק עד כותל העזרה כ״ח אמות ולכן אין מדבריו סיוע לדברי הרדב״ז שכתב דבריו לענין המרחק שהי׳ מק״ק עד העזרה. כנלענ״ד הקטן יעקב.\\\vspace{0pt}

\end{multicols}\newpage

\newchap{סימן ה}
\begin{multicols}{2}
שאלה – ממורה אחד בעיר נייאיארק: בקהלה אחת במדינת אמעריקא יש גבר אשר נגע בו יד ד׳ זה כמה שנים אשר פרחה צרעת על פניו אשר אין יוכל להרפא בשום אופן כי צרעת נושנת היא בעור בשרו והצערת מראה בוהקיות המבואר על נשיאת כפים (סי׳ קכ״ח) הפוסל בכהן ואיש הזה רוצה להיות שליח צבור וקהל רוצים לקבלו אם אין זה נגד הדין?\\\vspace{0pt}

תשובה – דין זה אם בעל מום כשר להיות ש״ץ פלוגתא דרבוותא היא כבר הביא מהרש״ל ביש״ש (פ״א דחולין) שנשאל מהר״ם על א׳ שנטלו זרועותיו אם ראוי לש״ץ והשיב דכשר דדרך הקב״ה להשתמש בכלים שבורים והסכים המהרש״ל עמו וכ״כ גם הב״ח א״ח סי׳ נ״ג אכן המגן אברהם שם חולק על מהרש״ל ממה שכתב הזוהר שבעל מום הוא פגום ועוד הקריבהו נא לפחתך ולכן כתב שיש לזהר ואם כי הפוסק הגדול בעל מג״א אשר מימיו שותין בכל הגולה הכריע נגד המהרש״ל מי יבא אחריו להקל נגדו מבלי ראי׳ מכרעת גם ראיתי בשו״ת חות יאיר (סי׳ קע״ו) שלא ראה הפוסקים הנ״ל מסברא דנפשי׳ כתב כהמג״א דבשביל הקריבהו נא לפחתך אין ליקח בעל מום לש״ץ אכן לכאורה הי׳ ק״ל על פסק המג״א ממה דפסק בש״ע שם (סי׳ נ״ג סעיף י״ד) דסומא יורד לפני התיבה וכי יש לך בעל מום גדול מזה שוב ראיתי שגם בפרישה הסכים מכח טעם זה עם הרש״ל וא״כ ודאי קשה על המג״א אכן לענ״ד אפשר לומר דס״ל כמו שכתב מהרי ברונא בשו״ת הביאה האלי׳ רבה (סימן נ״ג) דדוקא לש״ץ קבוע אין ליקח בעל מום אבל להתפלל באקראי שרי ולזה יש הוכחה מלשון ש״ע שכתב דסומא יורד לפני התיבה ולא נקט הלשון כמו בשאר מקומות בסי׳ זה שראוי להיות ש״ץ משמע דדוקא לירד באקראי לפני התיבה שרי לבעל מום וגם לסומא אבל לא להיות ש״ץ בקביעות דבזה יש לחוש יותר ליקח שליח קבוע מכח הקריבהו נא לפחתך וביותר נלענ״ד שאפשר שגם מהר״ם ורש״ל ופרישה וב״ח ואלי׳ רבה שמכשירים בעל מום להיות ש״ץ לא איירי רק במום שאין בו משום מיאוס כגון הך שנפלו זרועותיו שעלי׳ נשאל המהר״ם אבל במום כזה שפרחה צרעת על פניו שנמאס לבני אדם להסתכל בו אולי כ״ע מודו שאין בשליח כזה כבוד המקום למנותו לש״ץ ויש בזה משום הקריבהו נא לפחתך דמי עדיף מפוחח שזרועותיו או כתפיו מגולות מחמת עוני שאין לו בגדים ללבוש ואעפ״כ מדינא דמתניתן אינו יורד לפני התיבה ולא בלבד מפני שהוא פגם לצבור כמו שפי׳ רש״י אלא לפי מה שכתב הב״ח יש בזה מפני כבוד המקום שכן כתב בסי׳ נ״ג פוחח אסור לו להיות ש״ץ מפני שאין זה כבוד המקום עכ״ל וכש״כ באיש אשר נמאס בעיני ב״א לראותו ולהסתכל בצרעת שבפניו ולכן אין למנות לאיש הלזה לש״ץ קבוע. כנלענ״ד הקטן יעקב.\\\vspace{0pt}

\end{multicols}\newpage

\newchap{סימן ו}
\begin{multicols}{2}
אלטאנא, תמוז תקצ״ז. להרה״ג וכו׳ מ״ה דוד מעלדאלא נ״י חכם דק״ק ספרדים בק״ק לאנדאן יע״א.\\\vspace{0pt}

שאלה – כהן אחד כמה שנים ייחד לו בביתו נכרית והיא יושבת עמו והוא עמה ביחוד כאיש ואשתו לכל דבר גם הוליד ממנה בנים חללים ובשאט נפש רצה לשמש בכהונה לכל דבר שבקדושה ובפרט בנשיאות כפים ולקרוא ראשון בתורה אם צריך למחות בידו? תשובה – לכאורה נדון זה מבואר בא״ח סימן קכ״ח כהן הנושא נשים בעבירה פסול מנשיאת כפים עד שידור הנאה וכיון שזה עומד במרדו אינו ראוי לנשיאות כפים אכן אכתי יש מקום עיון בזה ובתחלה נעורר על כהן שייחד לו בביתו גרושה וזונה לפלגש בלא כתובה וקדושין אם גם זה פסול לעבודה עד שידור הנאה ולכאורה תלי זה בפלוגתא שבין הרמב״ם והראב״ד בהל׳ א״ב (פ׳ י״ז הל׳ ב׳) דלדעת הרמב״ם דליכא משום פסולי כהונה רק בבעל אחר קדושין לא מקרי זה נושא נשים בעבירה כיון דליכא משום זונה וחללה דפנוי׳ לא מקרי זונה כמבואר ברמב״ם וש״ע אבל להראב״ד שם שלוקה ג״כ בלא קדושין שפיר הוי איסור כהונה והשתא בנכרית שאין קדושין תופסין בה לא משכחת בה איסור זונה כלל וכזה הי׳ נראה ג״כ מדברי הרב המגיד בה׳ א״ב (ר״פ י״ח) שכתב שם וז״ל מה שכתב רבינו כל שאינה בת ישראל היא קרוי׳ זונה ור״ל אע״פ שנתגיירה יתבאר לפנינו בזה הפרק עכ״ל והשתא יקשה לכאורה ל״ל לפרש דברי הרמב״ם דאיירי אע״פ שנתגיירה ולמה לא פירש בקיצור דאיירי בהיותה נכרית אע״כ דס״ל משום דאז להרמב״ם לשיטתו כיון שאין לו בה קדושין לא הוי זונה שוב ראיתי שזה אינו דבפי׳ כתב הרמב״ם בהל׳ א״ב (פ׳ י״ב הל׳ ג׳) דכהן הבא על העכום לוקה מן התורה משום זונה ובבעילה בלבד לוקה שהרי אינה בת קדושין עכ״ל וצ״ל מש״כ ה״ה וי״ל אע״פ שנתגיירה כוונתו שנתגיירה ג״כ בכלל זונה אבל לעולם עיקר דברי הרמב״ם משכ׳ כל שאינה בת ישראל היינו בעודה בגיותה וכמדומה שזה נעלם במכ״ה מהרב בעל חלקת מחוקק באע״ז (סי׳ ו׳ ס״ק ט׳) שכתב שם כל שאינה בת ישראל היינו סיפא דכתב וכן הגיורת והמשוחררת ולא ידעתי לאיזה טעם כפל דבריו וכן הוא בהרמב״ם עכ״ל ולפ״ז א״ש דמתחלה איירי הרמב״ם והש״ע מנכרית ולבסוף מגיורת וכיון דכן כהן זה ודאי מקרי בא על הזונה לכ״ע ובלא״ה נראה מדברי הש״ע (סי׳ ז׳ ס׳ י״ב) שדעתו כדעת הראב״ד דבביאה בלא קדושין ג״כ לוקה ע״ש אכן מכ״מ נלענ״ד בנדון שנסתפקתי דהיינו בייחד לו איסורי כהונה לפילגשים דבזה לא נפסל לעבודה כיון דבמתניתן לא קתני רק הנושא נשים בעבירה פסול ולא קתני הבא על נשים בעבירה משמע דוקא בנושא שיש לו אישות בהן דאע״ג דלפי המבואר בא״ח (סי׳ קכ״ח ס׳ ל״ט) אפילו גרשה או מתה מכ״מ פסול עד שידור הנאה מנשים פסולות י״ל כיון שפקר כ״כ לישא אותה לאשה ממש בזה פסול עד שידור אבל בא עלי׳ בלא נשואין הוי כשאר עובר עבירה שאינו נפסל על ידן לעבודה כמבואר שם ובלא״ה נלענ״ד כיון שהך הנושא נשים בעבירה פסול אינו מן התורה רק קנס ותקנת חכמים וא״כ הבו דלא לוסיף עלה ולבטל מצו׳ וכיון דכן הך כהן שייחד לו נכרית כיון שאין לו בה קדושין אפשר שאינו בכלל התקנה דהנושא נשים בעבירה פסול לדוכן גם לא ראיתי להרמב״ם שהביא בהל׳ נשיאות כפים דין דהנושא נשים בעבירה והמטמא למתים פסול לדוכן רק בהל׳ ביאת מקדש הביא כן לענין עבודה ואין לומר דהשמיט כן מפני שבגמרא לא הוזכר רק לענין עבודה ואין דרכו להביא רק מה שהוזכר בגמרא שהרי הך דעבר ע״ז פסול לדוכן הביא בהל׳ נשיאות כפים אף שגם זה לא הוזכר רק לענין עבודה בגמרא (סוף מנחות) ומזה הי׳ נלענ״ד דדעתו כיון דהנך לבד תקנות חכמים הן לא תקנו כן רק לענין עבודה דליכא בטול ע׳ אם לא יעבוד אבל כהן שאינו עולה לדוכן שעובד בעשה לא רצו חכמים לתקן שיעבור על ע׳ אפילו בשוא״ת שוב ראיתי שכבר הרגיש בזה הרב ב״י בא״ח (סי׳ קכ״ח) מה שהפוסקים לא הביאו דינים הללו לענין נשיאות כפים רק כתב כיון שהפוסקים לא התירו בפי׳ נקט כדברי מר שמואל ורשב״א שאסרו בפי׳ אבל בנדון זה שלא נשא זונה אפשר שזה יהא ג״כ סניף להתיר לו נשיאות כפים במה שנוגע לנושא נשים בעבירה כנלענ״ד להלכה ולא למעשה עד יסכימו בעלי הוראה עם זה: הקטן יעקב.\\\vspace{0pt}

\end{multicols}\newpage

\newchap{סימן ז}
\begin{multicols}{2}
ב״ה אלטאנא, יום ג׳ ב׳ אדר תרכ״א לפ״ק. לחתני הרה״ג וכו׳ מ״ה ישראל מאיר פריימאן נ״י אב״ד דק״ק פילעהנע יע״א.\\\vspace{0pt}

מה שהוקשה לך דבמנחות (דף מ״ד) אמר רב ששת כל שאינו מניח תפילין עובר בשמנה עשה וכל שאין לו ציצית בבגדו עובר בחמשה עשה כל כהן שאינו עולה לדוכן עובר בג׳ עשה כל שאין לו מזוזה בפתחו עובר בשני עשה והרמב״ם סוף ה׳ תפילין כתב שמי שאינו מניח תפילין עובר בשמנה עשה ובסוף ה׳ נשיאת כפים כ׳ כל כהן שאינו עולה לדוכן אע״פ שביטל מצות עשה אחת ה״ז כעובר על ג׳ עשה שנאמר כה תברכו את בנ״י אמור להם ושמו שמי והוקשה לך חדא למה הביא הרמב״ם דברי ר״ש דוקא לענין תפילין ונשיאת כפים ולא ג״כ לענין ציצית ולענין מזוזה דשם לא כתב דעובר בה׳ ובב׳ עשה ועוד למה לענין תפילין כ׳ דעובר בה׳ עשה ולענין נשיאת כפים דה״ז כעובר.\\\vspace{0pt}

על זה אשיב שמה שהרמב״ם כ׳ בתפילין שעובר ממש ובנשיאת כפים שהוא כעובר הטעם דבתפילין הם שמנה עשה ממש דבד׳ פרשיות כתובים שמנה עשה כמו שכתב רש״י בלישנא אחרינא ולכן כתב הרמב״ם דעובר בשמנה ע׳ אבל בנשיאת כפים אין עשה רק כה תברכו דאמור להם ושמו אין זה לשון ציווי כמש״כ הכס״מ לכן כ׳ הרמב״ם בזה רק דהוא כעובר ולא עובר ממש: ומה שלא כ׳ דברי ר״ש ג״כ לענין ציצית ומזוזה לענ״ד הטעם דהנה כל דברי ר״ש כמה עשה איכא אין נפקותא לדינא ולא אמר כן רק להגדיל ענין המצות הללו ולכן הביא הרמב״ם דבריו בסוף ה׳ תפילין ובסוף ה׳ נכ״פ כדרכו שבסוף ההלכות כותב דברי מוסר על המצות שביאר אכן לענין ציצית לא הוצרך להביא דברי ר״ש לזה שכתב סוף ה׳ ציצית להגדיל המצו׳ מה שכולל יותר שכתב לעולם יהא זהיר במצות ציצית שהרי הכתוב שקלה ותלה בה כל המצות כולן עכ״ל וכן לענין מזוזה כתב בסוף ה׳ מזוזה חייב אדם להזהר במזוזה מפני שהיא חובת הכל תמיד וכו׳ ע״ש שהגדיל ענין המצו׳ ע״י בחינות התרוממות אשר על כן לא הוצרך לכתוב שיש בה ב׳ מצות ע׳: ומה שכ׳ חתני נ״י ליישב דעת הרבינו מנוח שהביא הב״י א״ח סי׳ קכ״ח דהא דאמרינן שאין הכהן עובר כשאין אומרים לו לברך היינו שאינו עובר בג׳ עשה אבל בעשה דכה תברכו עובר ע״פ מה שראית בשאלתות דרב אחא פ׳ נשא וז״ל אריב״ל כל כהן שאינו נושא את כפיו עובר על מה שכתוב בתורה כה תברכו עכ״ל ובסוטה (דף כ״ח) כתוב אמר ריב״ל כל כהן שאינו עולה לדוכן עובר בג׳ עשה כה תברכו אמור להם ושמו את שמי ומזה רצית לומר שלהשאלתות הי׳ גרסא אחרת בלשון ריב״ל וכדי שלא לעשות פלוגתא בין ריב״ל ובין ר״ש סובר רבינו מנוח שריב״ל איירי בשלא אמר לו עלה דבזה לא עובר רק בעשה דכה תברכו ולכן נקט לשון שאינו נושא את כפיו ורב ששת איירי באומרים לו עלה ולכן נקט לשון שאינו עולה לדוכן ובזה עובר בג׳ עשה עכ״ד יפה אמרת ואני אבוא למלאות דבריך דבאמת זה לבד אינו מספיק לדון שהי׳ להשאלתות גרסא אחרת בדברי ריב״ל די״ל דמה דאמר דעובר על מה שכתוב בתורה כה תברכו כוונתו על כל מה דכתוב מכה תברכו ואילך ורישא דקרא נקט ולכן ביותר י״ל דראית רבינו מנוח היא מדברי הירושלמי דנזיר שהביאו התוספ׳ בסוטה (שם) דאמר רבי יהודה בן פזי בשם רבי אלעזר כל כהן שעמד בבית הכנסת ואינו נושא את כפיו עובר בעשה ע״ש ומדלא אמר עובר בג׳ עשה כרב ששת לכן סובר רבינו מנוח דאיירי דאין אומרים לו לעלות רק שעומד בביה״כ ואינו נושא את כפיו והוא אינו עובר רק בעשה דכה תברכו אבל רב ששת איירי בשאומרים לו לעלות לדוכן ואינו עולה והוא עובר בג׳ עשה וגם להרמב״ם י״ל דהוכיח מזה דרב ששת לא קאמר רק כעובר על ג׳ עשה אבל באמת אינו עובר רק על אחת ולכן אמר ר״א דעובר בעשה ולא נקט ג׳ עשה: כנלענ״ד הקטן יעקב.\\\vspace{0pt}

\end{multicols}\newpage

\newchap{סימן ח}
\begin{multicols}{2}
אלטאנא, סיון תרט״ז. לחתני הרה״ג וכו׳ מ״ה משולם זלמן הכהן נ״י אב״ד דק״ק מאסטריכט.\\\vspace{0pt}

שאלת – חזן שטעה וקרא במנחה בשבת פ רביעי תחת שהיה לו לקרות פ׳ ראשונה מסדר היום של שבוע הבאה וכבר עלו הקרואים קודם שנודע הטעות אם שוב חייב לקרות הפ׳ ראשונה בברכת הקרואים ורצית להביא ראי׳ שאין צריך לקרות ממה שכתב רש״י במגילה דף כ״ב ד״ה דאפשר בהכי שהוא יכול לקרות מה שירצה דהכל מענינו של יום.\\\vspace{0pt}

תשובה – גרסינן במגילה במנחה בשבת בשני ובחמשי קורין כסדרן ושם דף ל״א איתא ת״ר מקום שמפסיקין בשבת שחרית שם קורין במנחה וכו׳ דברי ר״מ ור״י אמר וכו׳ ע״ש וכיון דבין לר״מ ובין לר״י תקנת חכמים היא לקרות במנחה במקום שסיים בשבת לכן כל שלא קרא כתיקון חז״ל צריך לחזור ולקרות בברכה ומרש״י שהבאת אין ראי׳ דשם איירי שהתחיל לקרות כדין מתחלת הפ׳ ושפיר כתב רש״י שיכול לקרות כמה שירצה שהכל מסדר היום אבל כשהתחיל לקרות שלא כדינא צריך לחזור ולקרות: וכן הורתי פעם אחת שטעה הקורא וקרא ביום ב׳ פ׳ ראשונה של סדר השבוע שעברה וכבר סיים הקריאה שצריך לחזור ולקרות הפ׳ של השבוע לשלשה קרואים והם יברכו כסדר כיון שהקריאה הראשונה הית׳ שלא כדין כנלענ״ד הקטן יעקב.\\\vspace{0pt}

\end{multicols}\newpage

\newchap{סימן ט}
\begin{multicols}{2}
אלטאנא, יום ו׳ כ״ו מנחם תרט״ו לפ״ק. להרה״ג וכו׳ מ״ה קאפל ב״ב הלוי נ״י הגאב״ד דק״ק ווארמס יע״א.\\\vspace{0pt}

על דבר שאלתו דמר נ״י בקהל שקנו גן שסמוך לחצר ביה״כ למען הרחיב את החצר שבו נטוע אילן אגוז שעושה פירות הרבה עד שלקצצו אין מעולה בדמים אבל לא יחפצו שיהי׳ האילן לפני ביה״כ ועל כן ברצונם לקצצו עכ״פ ע״י עכו״ם אם יש להתיר כן.\\\vspace{0pt}

הנה מר נ״י ידיו רב לו לסמוך על הוראתו ולא צריך לדכוותי, אכן אם אביע דעתי הקלושה לא מלאני לבי להקל בענין זה, דהנה אמת אף שפשיטא שקוצץ אילנות טובים יש בו איסור דאורייתא דלא לשחית את עצה מכ״מ כשיש בהקציצה צורך מצו׳ שרי כיון שאפילו בצריך למקומו לצורך הדיוט שרי כמו שכתב הרא״ש בב״ק הביאו גם הט״ז י״ד סי׳ קט״ז דלא אסרה התורה אלא דרך השחתה, ולכן אם הי׳ דבר מצו׳ ממש בהקציצה אין כאן בית מיחוש. אכן להחשיב למצו׳ כדי שאם יתאחד הגן עם החצר של ביה״כ לא יהי׳ אילן נטוע סמוך לפתח ביה״כ, לזה לא מצאתי סמך ורמיזה בפוסקים שיהי׳ איסור לנטוע אילן על פתח בית הכנסת. ומה שהביא מעכ״ת נ״י מדברי הרמב״ן ר״פ שופטים בפסוק לא תטע לך אשירה כל אילן שנטוע על פתחי בית אלקים יקרא אשירה וכו׳ עכ״ל לא כתב כן רק לבאר לשון אשירה מלשון אשור אבל לא להשמיענו איסור בזה לנטוע על פתח בית אלקים שהרי בתוך דבריו כתב שם שאסור לנטוע בכל הר הבית וכפי הידוע קדושת הר הבית שו׳ לקדושת בית הכנסת לענין שלא לעשות קפנדרי׳ וכדומה ועוד יותר מביה״כ שהרי אסור לכנוס להר הבית במנעל וסנדל ולביה״כ שרי ואם בפתח ביה״כ יהי׳ איסור לנטוע כל שכן לפני שערי הר הבית והרי לא נאמר איסור כן רק שלא לנטוע בהר הבית עצמו ובמקום הקדוש יותר אבל לא ג״כ לפני שערי הר הבית. ועוד דאף דהראשונים נתנו לפעמים טעמים להמצות להסבירם לעם הלא ידוע דלא נלמוד הלכה מזה אפילו להחמיר וכל שכן להקל דאם נאסור הנטיעה לפי טעם הרמב״ן בפתח ביה״כ יהי׳ זה בנדון זה לקולא להתיר קציצת האילן שבלא היתר זה יש בזה איסור דאורייתא. ועוד נלענ״ד דאפילו במקדש שאסור לנטוע מכ״מ מה שהי׳ נטוע כבר בעת שנתקדשו הר הבית והעזרות או מה שעלה מאליו אחר שנתקדש אין איסור לקיימו שהרי לענין כלאים דאסרה התורה הזריעה והנטיעה צריך קרא שאסור ג״כ לקיים כלאים כמבואר במכות (דף כ״א) ובירושלמי דכלאים הביאו הר״ש שם א״כ גבי מקדש דליכא קרא רק שלא לנטוע מהיכי תיתי שיהי׳ אסור ג״כ לקיים ולפ״ז בנדון דידן גם לו יהי׳ דומה ביה״כ למה שאסרה התורה גבי מקדש אין כאן איסור לקיים אילן זה שכבר נטוע ועומד קודם שנעשה מקומו חצר ביה״כ ולכן לא מלאני לבי להקל בזה לקצצו ואפילו ע״י עכו״ם דידוע דקיי״ל ע״פ הגמרא דהשוכר את הפועלים שיש איסור שבות באמירה לנכרי גם בשאר איסורים ובפרט בענין כזה שיש סכנתא בהדי איסורא לא ידעתי היתר לקצצו. כנלענ״ד הקטן יעקב.\\\vspace{0pt}

\end{multicols}\newpage

\newchap{סימן י}
\begin{multicols}{2}
אלטאנא, יום ג׳ י״א אייר תר״ז. להרה״ג וכו׳ מ״ה יצחק דוב ב״ב הלוי נ״י הגאב״ד דק״ק ווירצבורג יע״א.\\\vspace{0pt}

מה שנסתפק למעכ״ת נ״י בשותה מי מעיינות שקורין מינעראהלוואססער לרפואה אם צריך לברך שהכל מפני שראה להגאון רבינו אלי׳ מווילנא ז״ל שכתב בחבורו שנות אליהו על מתניתן השותה מים לצמאו וכו׳ פירוש דווקא לצמאו אבל שלא לצמאו אין לברך כלל ומה דאמרו דחנקתיה אומצא לאו דווקא אלא כל שלא לצמאו עכ״ל ולפ״ז גם השותה מים לרפואה לא היה לו לברך.\\\vspace{0pt}

על זה אשיב לענ״ד דברי הגאון ראו״ו נובעים ממה שכתבו התוספ׳ ברכות (דף מ״ה) ד״ה דחנקתיה ודווקא מים שאין לו הנאה כי אם לשתות לצמאו והנאה דחנקתיה אומצא אין זה נחשב הנאה אבל שאר משקים שלעולם הגוף נהנה מהם בכל ענין מברך ואפילו חנקתיה אומצא כדאמרינן לעיל (דף ל״ו) ואע״ג דלרפואה אתי בעי ברכה עכ״ל הרי שחלקו התוספת בין מים לשאר משקים והביאו ראיה מהך דלרפואה אתי בעי ברוכי מכלל דבמים אפילו שותה לרפואה כל שאינו לצמאו לא מברך והחילוק שבין מים לשאר משקים בזה מבואר בטור וביותר בש״ע א״ח (סימן ר״ד) שמים שאין להם טעם אין הנאה בשתייתם רק בשותה לצמאו אבל שאר משקים שיש להם טעם אם טעמם טוב והחיך נהנה מהם מברך עליהם אף ששותה לרפואה ע״ש והנה יש בשותים מים לרפואה שלשה חילוקים יש ששותים המים של הבארות שאין להם טעם כלל לרפואה לקרר הדם ויש ששותים המים ממעיינות ידועים שיש להם טעם טוב והחיך נהנה מהם ויש ששותים מים שטעמם מר או מלוח והם לזרא לחיך הטועמם ולכן נראה לי לחלק ששותה מים מר או מים מבארות שאין להם טעם כלל לרפואה אינו מברך עליהם אבל השותה מאותן מים שיש להם טעם טוב והחיך נהנה מהם צריך לברך עליהם אף ששותה לרפואה שדינם כשאר משקים ששותה לרפואה שמברך עליהם אם טעמם טוב והחיך נהנה מהם כמבואר בש״ע הנ״ל. כנלענ״ד הקטן יעקב.\\\vspace{0pt}

\end{multicols}\newpage

\newchap{סימן יא}
\begin{multicols}{2}
אלטאנא, כ״ט מרחשון תרכ״א לפ״ק. למחותני הרב היקר וכו׳ מ״ה שמרי׳ צוקערמאן נ״י בק״ק מאהילעוו יע״א.\\\vspace{0pt}

על דבר מה ששאל מעכ״ת נ״י בנמצא בחדר שדר בו אמבטי שלפעמים רוחצין בו בחמין אם מותר להתפלל וללמוד בחדר הזה כשהאמבטי ריקם ונקי אחר שלפעמים בעת הרחיצה עומד בחדר זה ערום.\\\vspace{0pt}

תשובה – בא״ח סי׳ מ״ה פסק בש״ע דבית המרחץ בבית הפנימי שעומדין שם ערום אסור לכנוס בתפילין וע״ש במג״א שמחלק בין בית המרחץ למקוה דבית המרחץ כיון שרוחצין שם בחמין איכא זוהמא והבלא משא״כ במקו׳ ולכן בבית המרחץ אפילו אין שם אדם שעומד ערום צריך לחלוץ תפילין כשנכנס אבל במקוה מותר אפילו לברך שהנשים מברכות שם כדאיתא בי״ד סי׳ ר׳ וחילוק זה הביא הכס״מ בשם הר׳ מנוח וכן פסק בש״ע שם (סי׳ פ״ד) דבבית המרחץ שעומדים שם ערומים אפילו שאלת שלו׳ אסור ואסור לענות שם אמן וע״ש בט״ז ולפ״ז הי׳ נראה דבבית שרוחצים שם בחמים דינו כבית המרחץ ואסור ללמוד ולהתפלל שם דאיכא זוהמא אכן נראה לחלק דזה דוקא בחדר שהזמין להיות בית המרחץ אבל בחדר המיוחד לדירה רק שלפעמים רוחץ שם אין בזה משום מיאוס דודאי נשמר מזוהמא וכן מה דאיתא בי״ד (סי׳ רפ״ו) דבית העצים ובית הבקר שנשים רוחצות בהם כיון שעומדות שם ערומות אין כבוד שמים שיהי׳ שם מזוזה שם עיקר הטעם שאינם בית דירה ממש כמו שכתב שם הש״ך ס״ק ח׳ ואפילו לדעת הב״ח שהביא שם שפוטר גם דירה ממזוזה אם הנשים רוחצות שם ועומדות שם ערומות הטעם בזה כיון שעומדת ערומה בפני המזוזה ויש בזיון כתבי קדש אבל להתפלל ולברך בשעה שאין עומד ערום שם מותר כיון שהחדר אינו מיוחד לרחיצה כי אם לדירה. ומכ״מ טוב לכסות האמבטי בשעה שלומד ומתפלל וגם המזוזה צריך לכסות אם היא בתוך החדר שרוחץ בו שלא יעמוד בפני׳ ערום כמבואר בש״ך שם במקום שרוחצים התנוקות כנלענ״ד הקטן יעקב.\\\vspace{0pt}

\end{multicols}\newpage

\newchap{סימן יב}
\begin{multicols}{2}
עוד להרב הנ״ל נ״י.\\\vspace{0pt}

ומה ששאל מעכ״ת נ״י עוד על מה שכתב הב״י א״ח סי׳ קכ״ג ובשם רבינו האי מ״כ שטעם ג׳ פסיעות משום דתפלות כנגד תמידים תקנום וכשהכהן עולה למזבח עם אברי התמיד הי׳ עולה דרך ימין ומקיף ויורד דרך שמאל ובין כבש למזבח היו שלש רובדין של אבן ויורד בהם ג׳ פסיעות על עקב ואנן עושים כמו שהם היו עושים עכ״ל שלא ידע מנ״ל לרב האי גאון שהי׳ ג׳ רובדין בין כבש למזבח.\\\vspace{0pt}

על זה אשיב שלענ״ד יצא לו כן ממה דאמרינן זבחים (דף ס״ב ע״ב) רב פפא אמר כי דם מה דם אויר קרקע מפסיקו אף בשר אויר קרקע מפסיקו וכתבו שם התוספ׳ קרקע לאו דוקא שהי׳ פורח אמה על יסוד ואמה על סובב ובלאו קרא הי׳ צריך אויר משום שנאמר סביב כדבסמוך אלא דסגי במשהו אבל הכא בעינן אויר גדול וניכר שהוא לצורך אויר עכ״ל ויש לומר דלר״ה גאון לא מסתבר דקרקע מפסיקו לאו דוקא אלא דוקא קאמר שהי׳ עומד על הכבש במקום שיש אויר קרקע מפסיק בינו לבין המזבח וזורק משם הבשר ומה שהכבש הי׳ פורח על הסובב והיסוד זה לא נחשב לכבש רק למזבח וא״כ היו ע״כ ג׳ רובדין בין הכבש למזבח מקום המערכה האחת במקום שהי׳ הכבש פורח על הסובב והאחת במקום שהי׳ פורח על היסוד והאחת במקום שהי׳ האויר רצפת קרקע שבין הכבש להתחלת הסובב והיסוד והכהן הי׳ עומד על הכבש שחוץ לאויר וזורק משם הבשר על המזבח שהי׳ צריך להיות בזריקה ואח״כ הי׳ עולה עד מקום המערכה לסדר שם האברים על האש וכשהי׳ חוזר הי׳ צריך לעשות ג׳ פסיעות עד שבא לכבש שחוץ למזבח דהיינו על הרובד שעלי׳ פורח הסובב ועל הרובד שעלי׳ פורח היסוד ועל הרובד שמפסקת בין הכבש למזבח ולכן כשירד מן המזבח עד שהגיע לכבש הי׳ עושה ג׳ פסיעות על גבי מקום שתחתיו ג׳ רובדין שבין הכבש למזבח כנלענ״ד הקטן יעקב.\\\vspace{0pt}

\end{multicols}\newpage

\newchap{סימן יג}
\begin{multicols}{2}
ב״ה אלטאנא, כ״ט אדר שני תרי״ג לפ״ק.\\\vspace{0pt}

שאלה – ס״ת שנמצא בו טעות בספר בראשית ונתערב בס״ת אחרים כשרים וחפשו חפוש אחר חפוש אחר הטעות ולא יכלו למצוא והגיע זמן המקרא לקרות בתורה באחת מסדרות דספר בראשית אם מותר לקרות באחד מהספרים ולברך כדינו או אם צריך לטרוח למצוא ס״ת כשר ודאי.\\\vspace{0pt}

תשובה – כענין שאלה זו כבר הוזכר באחרונים דבשו״ת יד אלי׳ סי׳ פ״ח פסק דס״ת שיש בו טעות שנתערב באחרים מותר לקרות באחד מהם דחד בתרי בטיל ואע״ג דהס״ת הוי דבר חשוב מכ״מ כיון דהס״ת לא נפסל רק מחמת הטעות ה״ל כחתיכה הראוי להתכבד שנאסר ע״י בליעת איסור דבטל דאף דהחתיכה חשוב מכ״מ הבלוע אינו חשוב וה״נ אין אנו דנין על הספר אלא על הטעות ע״ש ובבית לחם יהודה בי״ד סי׳ רע״ט התיר מטעם אחר כיון שהרמ״א בא״ח סי׳ קמ״ג כתב ובשעת הדחק שאין לצבור רק ס״ת פסול י״א דיש לקרות בצבור בברכה ויש פוסלין נמצא כאן דהוי שעת הדחק מותר מטעם ספק ספקא שמא אין זה הס״ת הפסול ואת״ל שהוא הפסול שמא הלכה כי״א שמותר לקרות בשעת הדחק בס״ת פסולה ועוד כיון דהרמ״א כתב עוד שם בשם הר״ן דאם חומש אחד שלם בלא טעות יש להתיר לקרות באותו חומש אע״פ שיש טעות באחרים ולכן אפילו להיש פוסלים לקרוא בספר פסול בשעת הדחק הוי ס״ס שמא אין זה הס״ת שיש בו הטעות ואת״ל זהו שמא אינו בחומש שקורא בו עכ״ד והנה בנדון השאלה דלפנינו לא שייך ההיתר של בל״י מטעם ספק ספקא דהשני ספק ספקות אינם כאן דלסמוך על היש מכשירין בשעת הדחק לקרות בספר פסול ליכא שהרי אם יטרח אולי ימצא ס״ת כשר ולא הוי שעת הדחק והספק ספקא דאולי בחומש זה אין הטעות ליכא שהרי יודע ודאי שהטעות הי׳ בספר בראשית שצריך לקרות בו. אכן עדיין נשאר ההיתר של שו״ת יד אלי׳ שהס״ת פסול בטל בכשרים. אבל ראיתי בשו״ת חתם סופר י״ד סי׳ רע״ז שחלק על שו״ת יד אלי׳ מב׳ טעמים, האחד, דלא שייך כאן בטול כיון שאפשר לברורי ולהכיר האיסור לא אמרינן דבטל ודלא כט״ז בא״ח סי׳ תל״ב. אמנם מכח טעם זה לא יהי׳ חשש בנדון שלפנינו דמה בכך שאפשר לברורי אם יטרח עוד זמן רב הרי עתה בשעה שצריך לקרות בצבור אי אפשר לברורי וכסברא זו כתב החתם סופר בעצמו לענין היתר ספק ספקא דאף דאפשר לברורי מכ״מ כיון דאי אפשר לברורי בזמן מועט אין לבטל קריאת התורה בברכה בשביל זה כמש״כ המג״א סי׳ ח׳ לענין ציצית דאם כשבודק הציצית יתבטל מתפלת הצבור מקרי אי אפשר לברורי. אכן עוד מטעם שני חולק החתם סופר על היתר דבטול ברוב וכתב וז״ל ובלאו הכי דברי יד אלי׳ תמוהים ואינם לפי כבודו דלפי שהחליט ס״ת הוא דבר חשוב ולא בטל אלא דהכא בטל כיון שאין פסולו אלא מחמת הטעות והוא אינו חשוב ומדמי לי׳ לדבר הנאסר מחמת הבלע ודבריו תמוהים למה זה דומה לחתיכה הראוי׳ להתוכבד שבלע טפת חלב דלא בטל אע״ג דאיסורא ע״י החלב מכ״מ כיון שחתיכה עצמה נאסרה ונעשה נבלה הו׳ לי׳ חהר״ל והכא ע״י הטעות נפסל הספר או לכל הפחות החומש כולו מלקרות בו בצבור והוי דבר חשוב דלא בטל והוא דבר פשוט לכל מבין על כן אין לסמוך על דברי תשובת יד אלי׳ הנ״ל עכ״ל ולפי דברים האלה נפל גם ההיתר דבטול ברוב בבירא. אבל באמת תמה תמה אקרא על פסק החתם סופר בזה דמה שכתב והוא דבר פשוט לכל מבין במכ״ה לפענ״ד הדבר פשוט לכל מבין אפכא שדברי יד אלי׳ נכונים והם לפי כבודו שמדברי חת״ס נראה שדעתו דמה דאמרינן רק בבשר בחלב דלא בטל חתיכה ראוי להתכבד ולא בשאר דברים שנאסרו מבליעת איסור הוא כיון שהחתיכה עצמה נעשה נבלה ונאסרה ולכן דומה הך דס״ת לבב״ח שג״כ נפסל כל הספר ע״י טעות וזה אינו דהחלוק בין בליעת חלב בבשר לבליעת איסור בהיתר אינו מטעם דבב״ח החתיכה נעשה נבלה משא״כ בשאר איסורים שהרי בפי׳ כתב הש״ע י״ד סי׳ ק״א ס״ב דאפילו למי שסובר חתיכה עצמה נעשה נבלה בכל איסורים מכ״מ לא נחשב חתיכה הראוי׳ להתכבד ע״י בליעת איסור כיון דאין האיסור הנבלע בה ראוי להתכבד ורק בב״ח חשיב איסור הראוי להתכבד כיון דכל חד באפי׳ נפשי׳ שרי וע״י חבורן יחד נאסרו חזר הכל להיות כגוף אחד כן מבואר בפוסקים ראשונים ואחרונים שם וא״כ הך טעות שפוסל הס״ת אינו דומה רק לאיסור שנבלע בחתיכת היתר ועשאו נבלה ואסרו כולו ואעפ״כ לא נחשב חהר״ל כיון דהאיסור אינו ראוי להתכבד וה״נ אין הטעות האיסור ראוי להתכבד רק הס״ת ואיך ידמה זה לבב״ח דשם נעשה האיסור ע״י חבור החתיכת בשר החשוב עם החלב ולכן שפיר נחשב האיסור ראוי להתכבד משא״כ בטעות של ס״ת דהוא לבד הוא האוסר ואינו דבר חשוב ולכן לא בלבד שהנדון דטעות בס״ת לא דומה לבב״ח אלא לבליעת איסור בהיתר בשאר איסורים אלא דעוד עדיף אפילו משאר איסורים שהרי הטעם דאמרינן חתיכה נעשה נבלה הוא משום דאפשר לסוחטו אסור כמבואר בטור י״ד סי׳ צ״ב ולמ״ד דאמרינן חנ״נ גם בשאר איסורים אמרינן ג״כ אפשר לסוחטו אסור בכל איסורים כמשכ׳ הטור בשם ר״ת שם ע״ש והרי בהך דס״ת ודאי אפשר לסוחטו מותר דאם הוציא האיסור דהיינו שתיקן הטעות חזר כל הס״ת להכשרו א״כ אין שייך בזה חתיכה עצמה נעשה נבלה להחשב איסור הראוי להתכבד אפילו אם היינו אומרים דחתיכה שנאסר ע״י בליעת איסור שנעשה נבלה חשיב ראוי להתכבד וכש״כ אחר שאפילו בזה בטיל ולא נחשב ראוי להתכבד ולכן יפה דן היד אלי׳ ששייך בזה ביטול ועל כן בנדון השאלה הס״ת הפסול בטל בכשרים ומותר לקרות בכל א׳ בשעה שאי אפשר לברר ולאחר שקרא בו שוב יחפש. וכן נ״ל שודאי לכתחלה צריך לחפש בכל הספרים שנתערבו שאין לשהות ס״ת שאינה מוגה ומכ״מ אם חפש בכולן ולא מצא ושוב חפש שנית בקצת מהם ולא מצא אזי מותר לקרות בהן לכתחלה ולא צריך לחפש עוד משום אפשר לברורי דיכול לתלות הטעות באותן שנשארו כדאמרינן נדה (דף ס״א) לענין ג׳ נשים ולענין גל טמא שנתערב בין גלים טהורים דתולה הבדוק באינו בדוק וה״ה הכא לענין אפשר לברורי ורק אם חפש בכולן ולא מצא אז לא יקרא לכתחלה באחד מהם היכי דאפשר לברורי עד שיתחיל שוב לבדוק. כנלענ״ד. הקטן יעקב.\\\vspace{0pt}

\end{multicols}\newpage

\newchap{סימן יד}
\begin{multicols}{2}
ב״ה אלטאנא, ו׳ אלול תרי״ג לפ״ג. להרה״ג וכו׳ מ״ה משה שיק נ״י הגאב״ד דק״ק יערגן יע״א. עוד בענין הנ״ל.\\\vspace{0pt}

הנה מעכ״ת נ״י השיג על פסקי הנ״ל בספר תורה שיש טעות שנתערבה באחרים וכתב אלי וז״ל מר נ״י הסכים עם שו״ת יד אלי׳ והשיג על אדמ״ו הגאון חתם סופר זצ״ל וכתב דמשום טעם הראשון דאיכא לברורי אין לאסור בנדון שלו שבשעה שבעי למקרי בי׳ לא הי׳ אפשר לברורי וכמ״ש בחתם סופר שם דאע״ג דס״ס דאפשר לברורי לא סמכי׳ עליו מ״מ כיון דאי אפשר לברורי בזמן מועט אין לבטל קריאת התורה בברכה בשביל זה וכמ״ש המג״א לענין בדיקת ציצית עכ״ל שם ותמהני על מעכ״ת נ״י הא אין הנדון דומה לראי׳ דהא דאמרי׳ בס״ס או בחזקה או ברוב דהיכא דאיכא לברורי מבררינן היינו רק לכתחילה ומשום דאין עושין ס״ס בידים ואין מבטלין איסור לכתחילה וכעי׳ דשיל״מ דעד שתאכלנו באיסור תאכלנו בהיתר ולא תסמוך ארוב או חזקה או ס״ס ולכך כיון דהשתא אי אפשר לברר או יש טרחה מרובה שפיר סמכינן אס״ס ורוב וחזקה אבל הא דאמרינן דהיכא דאפשר להכיר לא שייך ביטול היינו דהוי כאילו לא נתערב כלל וכקבוע במקומו דמי וקבוע כמחצה על מחצה. וכעין שכ׳ הפרמ״ג בש״ך סי׳ ק״י סקי״ד דאפי׳ ניכר רק לאחד בסוף העולם הוי קבוע וא״כ מה מועיל דהשתא אינו יכול לברר ותו דהרי כ׳ הפרמ״ג בפתיחה להלכות תערובת ספ״ב לחלק א׳ דהיכא דבעי לברורי מה״ת אין חילוק בין צריך טורח או לא וכבר הביא אדמ״ו זצ״ל שם ראי׳ מדברי הט״ז בסי׳ תרל״ב סק״ג (ומ״ש תל״ב הוא ט״ס) ושם מה״ת קאמר ואפי׳ באיכא טורח כ׳ מר״ן זצ״ל דלא בטיל משא״כ בס״ס ורוב וחזקה וכ״כ הפר״ח בסי׳ ק״ב סק״ח דדבר דאפשר להכיר ע״י קפילא לא בטל אע״ג דליכא קפילא ולא סמכי׳ אקפילא וכיוצא בזה כ׳ בבדק הבית סוף שער רביעי מבית ד׳ דדבר שאפשר להוציא האיסור ע״י הגעלה הוי כניכר ולא בטל והרשב״א אע״ג דשם פליג ומשמע אפי׳ ניכר בטל ובחי׳ רשב״א בשבת דף כ״ט הרחיק זה בעצמו מ״מ בתשו׳ סי׳ רנ״ט מוכח דס״ל דאפשר להכיר אפי׳ מה״ת לא בטל דשם הביא לתרץ דלכך חוט של כלאים לא בטל הואיל ואפשר להכיר ע״י צבע ובמ״ל בפ״א ממטמאי משכב דין י״ד הוכיח דהפלפול שם על דין תורה הוא וא״כ מוכח מהתי׳ הזה דאפי׳ מה״ת לא בטל והרשב״א לא פליג על תי׳ זה אמנם לכאורה דברי הרשב״א בתשובה סותרין וכו׳. אמנם טעם שני שכ׳ אדמ״ו הגאון זצ״ל דלא בטל הוא נכון וברור ותמהני על מעכ״ת נ״י דהשיג עליו וכ׳ דדווקא בב״ח דהאי לחודי׳ שרי׳ ובהדדי אסור משא״כ בשאר איסורין אפי׳ אי חנ״נ לא מקרי ראוי להתכבד והה״ד הכא ותמהני דדבר גדול דיבר רבינו הגדול והמקום הי׳ בעזרו דמ״ש בב״ח דהאי לחודי׳ שרי ובהדי הדדי אסור לכך חשיב כגופו של איסור וה״נ בס״ת דהוא להיפך דהאי לחודא והאי לחודא אסור אפי׳ כל פרשיותי׳ של התורה וכל תיבותי׳ כתובין ומונחין אצלה כסדרן וכהווייתן מ״מ אסור למקרי בהן כדקיי״ל בי״ד ססי׳ רע״ח וא״כ האי לחודי׳ והאי לחודי׳ אסור למקרי בי׳ רק בהדי הדדי שתפורין ומדובקין כדינו שרי וא״כ כשחסר ממנו ואינו בהדי הדדי הוא איסור מחמת עצמו ויפה כתב מר״ן זצ״ל דפשיטא דדמי׳ לבב״ח וכיון דס״ת חשיב לא בטל ודלא כתשו׳ יד אלי׳ אלא שאני חוכך דהרי קיי״ל בי״ד סי׳ ק״י סעיף ח׳ בדבר חשוב אפי׳ ס״ס לא מהני עד דאיכא ג׳ ספיקות ואיך נסמוך כאן אהיתר בית לחם יהודה אס״ס מיהו כבר האריך הש״ך בסי׳ ק״י ס״ק נו״ן להתיר בס״ס ושם ס״ק נ״א כ׳ במידי דלאו בר אכילה איכ״ל דלכ״ע מהני ב׳ ספיקות והכא מידי דרבנן הוא ולכך סמכי׳ אספק בגופו וספק בתערובת וכל מה שכ׳ מר״ן זצ״ל שם בחתם סופר אתי׳ שפיר. עכ״ד:\\\vspace{0pt}

על זה אשיב שבכל מה שהאריך מר נ״י בזה איני רואה שום השגה על דברי כי אם אפשר לברר נחשב כקבוע איך מועיל במקום ספק ספקא הרי לדעת רוב הפוסקים היתר ספק ספקא גופא הוא מטעם רוב אלא ודאי אם סמך בשו״ת חתם סופר להתיר הוא משום דעל שעת קריאת התורה דיינינן ובאותה שעה אין אפשר להכיר ולברר ומה לי היתר ע״י ספק ספקא ותערובת חד בחד (שהרי החתם סופר כתב היתרו בנדון דבית לחם יהודה ושם איירי בחד בחד) ומה לי היתר ע״י רוב בתערובת חד בתרי ואיך שייך לזה מה שכתב הפרי מגדים דהיכי שיש א׳ בסוף העולם שמכיר האיסור שאין כאן ביטול דהכא בשעה שצריך לקרות אין כאן מכיר ואדרבא בנדון דידי יש עוד סברא יותר להתיר שאם המגן אברהם היכי שיש חזקה לחוד מתיר לסמוך עליו כיון שבאותה שעה אי אפשר לברורי כש״כ הכא דאיכא רוב וחזקה שהרי לא בלבד שהספרי תורות הכשרים הם רוב נגד הפסול אלא שיש להם ג״כ חזקת היתר שכל אחד הי׳ מוחזק לנו עד עתה בחזקת היתר ואף דנוגע זה בחזקה דתלי ברוב עכ״פ עדיף מחזקה לחוד ומרוב לחוד.\\\vspace{0pt}

ומה שכתב מעכ״ת נ״י עוד אמנם שני וכו׳ ותמהני דדבר גדול דבר רבינו הגדול וכו׳ כאשר יפלא בעיני מר נ״י גם בעיני יפלא איך העתיר דברים בכוונת רבו הגאון ח״ס זצ״ל שלא בלבד שמצד הסברא אינם אלא שאי אפשר לפרש דבריו כן שהרי כתב שע״י הטעות נפסל הספר והרי נפסל לא שייך רק במה שבלתי הטעות כשר הוא א״כ אין כאן איסור רק הטעות ומדוע העלים מעכ״ת נ״י עינו מסוף דברי שכתבתי ראי׳ לדברי שעל ידה יוכר גם למסתפק שאין דמיון כלל בין הך דטעות בס״ת ובין חתיכה נעשה נבלה דבשר בחלב שהרי בב״ח אם אפשר לסחוט הטיפת חלב האוסר מ״מ הבשר נשאר באיסורו הרי שנעשה איסור מצד עצמו אבל בס״ת אם הי׳ טעות כגון יתר או חסר תיבה או אות והסיר היתר או מלא החסר הרי ששב הס״ת להכשרו וא״כ ראי׳ שלא הס״ת היא האיסור רק הטעות שלא נחשב ראוי להתכבד והוא דבר שלפענ״ד אין להסתפק בו לכל ישר הולך ותמהני איך כתב מר נ״י בכזה: דבר גדול דבר רבינו הגדול והמקום הי׳ בעזרו. אמת כי הרבה והרבה דברים גדולים דבר הגאון זצ״ל והמקום הי׳ בעזרו אבל בדבר הזה נעלם ממני כוונתו שכתב על היד אלי׳ שדבריו אינם לפי כבודו ושהוא דבר פשוט לכל מבין: כנלענ״ד הקטן יעקב.\\\vspace{0pt}

\end{multicols}\newpage

\newchap{סימן טו}
\begin{multicols}{2}
ב״ה אלטאנא, מרחשון תרכ״ג לפ״ק. להרה״ג וכו׳ מ״ה יצחק דוב ב״ב סג״ל נ״י הגאב״ד דק״ק ווירצבורג יע״א.\\\vspace{0pt}

נשאלתי ממעכ״ת נ״י מה דעתי אם יש חשש להדפיס ספר אצל אומן נכרי שיש לו פועלים ישראלים ויש לחוש שידפיסו בשבת ועובר משום לפני עור לא תתן מכשול.\\\vspace{0pt}

הנה ידוע מה דאמרינן בע״ז (דף ו׳) דמה דתניא א״ר נתן מניין שלא יושיט אדם כוס יין לנזיר ואבמ״ה לב״נ ת״ל ולפני עור לא תתן מכשול שזה דוקא בקאי בתרי עברי דנהרא שאם לא יושיט לו לא יכול לחטוא וכן כתבו התוספ׳ שם וז״ל ולפ״ז אסור להושיט למומרים דבר איסור אע״פ שהיא שלהם כי הדבר ידוע שיאכלוהו והוא נאסר להם דכישראל גמור חשבינן לי׳ ומיירי בדקאי במקום שלא יוכל ליקח אם לא יושיט לו זה כדמסיק דקאי בתרי עברי דנהרא עכ״ל הרי דכתבו בפשיטות דבדלא בתרי עברי דנהרא מותר להושיט לו וליכא משום לפני עור וא״כ אין חשש לפני עור גם בנדון השאלה דודאי יש להמדפיס מלאכות הרבה שיעברו המחללי שבת בה גם אם לא יהי׳ להם המלאכה של ישראל אמנם מדברי התוספ׳ בשבת (דף ג׳) נראה דמכ״מ איכא איסורא דרבנן גם בתרי עברי דנהרא שהקשו שם על מה דאמרינן דפשט בעה״ב ידו לחוץ ונתן לתוך ידו של עני או שנטל מתוכה והכניס בעה״ב חייב ועני פטור דפטור ומותר קאמר הרי עובר על לפני עור ואפילו מיירי דהי׳ יכול ליטול גם אם לא הי׳ בידו הרי עכ״פ איסורא דרבנן איכא דקטן אוכל נבלות ב״ד מצווין להפרישו כש״כ גדול שלא יסייע לו ולכן תרצו דאיירי בנכרי ע״ש וכ״כ הרא״ש והר״ן שם וזה סותר לדבריהם בע״ז שנראה דס״ל דמותר לגמרי והנה כבר הביא הרמ״א בי״ד סימן קנ״א ב׳ דיעות בזה ביכול לקנות ממקום אחר אם מותר למכור לו וכתב ונהגו להקל כסברא ראשונה וכל בעל נפש יחמיר לעצמו והש״ך שם חולק על הרמ״א וכתב דב׳ הדיעות לא פליגי אלא שיש חילוק בין ישראל לנכרי ומומר דישראל צריך להפרישו ולכן אסור מדרבנן ומזה איירי התוספ׳ בשבת אבל נכרי ומומר א״צ להפרישו ולכן כתבו בע״ז דמותר והביא ראי׳ לזה דהרא״ש ורבינו ירוחם כתבו כדברי התוספ׳ דשבת ובע״ז וא״כ יהיו דבריהם סותרים זא״ז אלא ודאי החילוק כדפירש בין ישראל לנכרי ומומר עכ״ד ולענ״ד קשה למה אין במומר ג״כ מצו׳ להפרישו כמו בישראל הרי ישראל גמור הוא כמו שכתבו התוספ׳ שם בע״ז ושוב מצאתי שגם בשו״ת חות יאיר סימן קפ״ה הקשה על הש״ך כן דמאין לו לומר שאם הי׳ בידנו וכחנו להפרישו שלא נהי׳ מחוייב לעשות כן ולענ״ד יש ליישב סתירת התוספ׳ והרא״ש בדרך אחר דדוקא להושיט להעובר האיסור טרם יעשה האיסור בזה יש חילוק אם יכול מעצמו להביא לו האיסור או לא שאם יכול להביא לו האיסור גם בלא שיושיט לו אז אפילו מדרבנן מותר אבל אם בשעה שעושה האיסור יכול להפרישו ע״י שלא יושיט לו זה ודאי אסור מדרבנן שאסור לסייע ידי עוברי עבירה כדאמרינן בע״ז (דף נ״ה) ישראל שהוא עושה בטומאה לא דורכין ולא בוצרין עמו אבל מוליכין עמו חביות לגת ומביאין עמו מן הגת ע״ש הרי דבשעת עבירה עצמו כשדורך בטומאה אסור לסייעו ואפילו בשנטמא היין כבר דאסור לסייע ידי עוברי עבירה אפילו כבר התחיל בעבירה אבל מכ״מ מוליכין עמו חבית לגת אף שבזה מושיט לו החבית שיתן בו היין ומסייע לו לעשות בטומאה דאין אסור רק לסייע לו בשעת עבירה עצמו ולכן שם בשבת דהעובר עבירה נוטל מידו ונותן לתוכו ואם לא יושיט לו ידו לא יעשה העבירה בזה כתבו התוספ׳ דאסור מדרבנן דמחוייב להפרישו ולא לסייעו בשעת העבירה אבל בע״ז דאיירי במושיט למומרים דבר איסור שיודע שיאכל אותו בזה כתבו דאם יכול ליקח מעצמו אין כאן איסור דלפני עור כיון דשם לא מסייע להעבריין בשעה שעושה העבירה ולכן דוקא להושיט לו כה״ג מותר אבל ליתן למומר דבר איסור לתוך פיו אע״פ שיכול ליקח גם מעצמו אפשר שבאמת אסור וכן כתב גם המג״א סי׳ קס״ג ס״ק ב׳ על מה שכתב הרמ״א ואסור להאכיל למי שלא נטל ידיו משום לפני עור לא תתן מכשול וז״ל נ״ל דוקא כשנתן לו משלו אבל אם הפרוסה של האוכל רק שהוא מושיטו לו שרי דהא אם לא הי׳ מושיט לו נוטלו ממילא אא״כ קאי בתרי עברי דנהרא כדאיתא בע״ז פ״ק ועמשכ׳ סי׳ שמ״ז ואפשר דמכ״מ איכא מסייע ידי עוברי עבירה ועיין בתוספ׳ גטין דף ס״א ובע״ז דף צ״ה עכ״ל והמחצית השקל תמה עליו למה כתב בלשון ואפשר הרי מבואר כן בסימן שמ״ז דלכך פטור אבל אסור משום דמסייע ידי עוברי עבירה ולענ״ד ספק המג״א הוא דלא דמי לגמרי להך דשבת דשם מסייע בשעת עבירה עצמו בשעה שעוקר ומניח אבל הכא כשנותן לתוך פיו עוד לא התחילה העבירה אלא כשאוכל ולכן גם בנותן האיסור לתוך פיו של מומר עוד יש לספק אם מקרי מסייע ידי עוברי עבירה בשעת עבירה וממה שכתב הרא״ש הלכך אסור להושיט למומרים דבר איסור לפיהן שלהן היכא דקאי בתרי עברי דנהרא עכ״ל נראה שדעתו דאפילו בנותן לתוך פיו לא אסור בלא קאי בתרי עברי דנהרא אבל קודם שמתחיל העבירה כגון להביא לו הדבר איסור י״ל שמותר לגמרי אפילו בישראל בשיכול להביא לו בעצמו וא״ל הרי הרא״ש למד כן ממה דקטן אוכל נבילות ב״ד מצווין להפרישו כש״כ שלא יסייע לגדול והלא בקטן אסור להושיט לו אפילו קודם שעת העבירה כדאמרינן בשבת ס״פ ט׳ דאסור ליתן לקטן חגב טמא חי לשחוק בו שמא ימות ויאכלנו שאני התם דאם לא יתן לו לא יבא הקטן לידי אכילה ודמי לקאי בתרי עברי דנהרא דאפילו להושיט קודם שעת עבירה אסור והטעם דהוי כמאכילו בידים כמש״כ המג״א סי׳ שמ״ג וא״כ כש״כ שלא יסייע לגדול בשעת עבירה אבל להושיט לו קודם שעת עבירה בלא קאי בתרי עברי נהרא י״ל דמותר לגמרי גם בישראל ודלא כש״ך. וראיתי בספר יד מלאכי סימן שס״א שהקשה על הכלל דליכא משום לפני עור אלא היכא דקאי בתרי עברי דנהרא ממה דאמרינן נדרים (דף ס״ב) רב אשי הוי לי׳ ההוא אבא זבני׳ לבי נורא א״ל רבינא והאיכא ולפני עור לא תתן מכשול א״ל רוב עצים להסקה נתנו ע״כ והרי שם מסתמא מצאו העע״ז עוד עצים אחרים חוץ מיער של רב אשי ואעפ״כ פריך והאיכא משום ולפני עור ותירץ בספר הנ״ל על פי מה שכתבו התוספ׳ בשבת דאפילו היכא דליכא משום ולפני עור מכ״מ איסורא דרבנן איכא ולכן הקשה רבינא שפיר על רב אשי והנה לפי מה שכתבתי דהתוספ׳ לא כתבו כן אלא אם בשעת העבירה עצמה מסייע לא שייך תירוץ זה שהרי שם הי׳ קודם זמן העבירה אמנם גם לפי מה שיישב הש״ך סתירת התוספ׳ והרא״ש לחלק בין ישראל לעכו״ם לא שייך תירוץ היד מלאכי שהרי שם לעכו״ם מכר היער ואעפ״כ פריך והאיכא משום לפני עור אכן לא ידעתי למה פשיטא לי׳ להרב ז״ל דבהך דרב אשי לא הי׳ מכעין תרי עברי דנהרא שהרי לא עצים אחדים מכר להם אלא כל היער וזה אינו מצוי כ״כ לקנות ודלמא ידע רבינא שאם לא מכר להם לא מצאו לקנות אחר ולכן פריך האיכא לפני עור. שוב ראיתי בריטב״א בע״ז שם וז״ל והא דנזיר ואמ״ה שאנו חוששין לתקלה כל היכא דמצי למיעבד איסור שלא על ידינו ליתא משום לפני עור ואע״פ שאפשר שהוא מרבה באיסור על ידינו לא חיישינן אבל מכל מקום אי תבע לי׳ בפירוש לאיסורא נהי דמשום לפני עור ליכא אכתי איסורא במלתא משום מסייע ידי עוברי עבירה כל שאנו גורמין לו לעשות איסור או להרבות באיסור וכדקיימא לן שאין מסייעין ידי ישראל בשביעית ואפילו לומר להם החזקו אסור ולא עוד אלא שאנו חייבין למחות בידו דכל ישראל ערבין זה לזה וכ״ש שאסור להם לגרום להם לעשות שום איסור או להרבות באיסור כלל וכדאמרינן בפ׳ שנים אוחזין דאלת״ה אנן חיותא לרועה היכי מסרינן וכדפרישנא התם עכ״ל ומדברים האלה למדנו עוד יישוב אחר לסתירת התוספ׳ דהיכי שלא יעשה ודאי איסור על ידינו ליכא משום ולפני עור אם אפשר לו לעשות איסור שלא על ידינו אבל בתבע בפירוש לאיסורא אסור גם בכה״ג משום מסייע ידי עוברי עבירה ולכן שם בשבת שנותן בידו או נוטל מידו אין לך ודאי איסור גדול מזה שם אסור אבל ליתן למומר נבלות בידו כל שאינו תבע בפירוש לאכול לעצמו מותר וזה קרוב להחילוק שכתבנו בין מסייע קודם עשיית האיסור למסייע בשעת עבירה עצמה. והנה כפי הנראה מדברי הריטב״א כיון דאוסר ליתן לו בידו כשתבע בפירוש לאיסורא כש״כ שאסור ליתן לתוך פיו ולפ״ז חולק על הרא״ש הנ״ל שנראה מדבריו שמתיר ליתן לתוך פי המומר דבר איסור היכא דלא קאי בתרי עברי דנהרא אכן איתא בירושלמי ר״פ ג׳ דדמאי בשם רבי יוחנן רופא חבר שהי׳ מאכיל לחולה ע״ה נותן לתוך ידו ואינו נותן לתוך פיו בדמאי אבל בודאי אפילו לתוך ידו אסור עכ״ל (ונ״ל שזהו הירושלמי שרמז עליו המג״א סי׳ שמ״ז ומה שכתוב שם רפ״ג דפיאה צ״ל דדמאי) הרי שהירושלמי חילק בין ליתן לתוך ידו לנותן לתוך פיו ואע״ג דאיירי בקאי בתרי עברי דנהרא מדאסר בודאי ליתן ידו ומה שהתיר בדמאי משום דבדמאי הקילו עכ״ז יש ללמוד כמו דבדמאי אע״ג דלתוך ידו מותר מכ״מ לתוך פיו אסור דה״ל מאכיל בידים כמו כן באיסור דאורייתא היכי דלא קאי בתרי עברי דנהרא אע״ג דליתן לתוך ידו מותר מכ״מ ליתן לתוך פיו אסור. והיוצא מכל זה להלכה דליתן ולהושיט איסור למי שאינו יכול להביא לו בעצמו מדאורייתא משום ולפני עור לא תתן מכשול אבל היכי דמצי להביא לעצמו אע״פ שאפשר שעל ידי שמושיט לו מרבה באיסור מותר אפילו מדרבנן אכן לסייע לו בשעת עבירה או היכא דתבע הדבר בפירוש לאיסור או ליתן לתוך פיו אסור אפילו באיסור דרבנן אפילו אם בעצמו הי׳ יכול להביא האיסור ובכל זה אין חילוק בין ישראל למומר. ולפ״ז בנדון השאלה להדפיס אצל מדפיס נכרי שיש לו פועלים ישראלים שעושים מלאכה בשבת ודאי אין חשש כיון שאפשר להם לעשות מלאכה בלא שיתן להם וגם אין שואלין בפירוש לעשות מלאכה בשבת דבכזה אפילו לעשות מלאכה ע״י נכרי יהי׳ אסור ולכן אין חשש איסור בזה. והנה לכאורה עוד יש טעם שאין בזה משום לפני עור כיון דהוא אינו מושיט המלאכה לישראל אלא לנכרי וא״כ הוי לפני דלפני דלא חיישינן כדאמרינן בע״ז (דף י״ד ע״א) ואע״ג דלישראל חשוד מוזהר גם אלפני דלפני כמש״כ התוספ׳ שם (דף ט״ו ע״ב) ד״ה לנכרי מכ״מ הכא דנותן המלאכה לנכרי לא מוזהר אלפני דלפני אכן באמת ז״א דכל היכי שיש חשש שיבא ישראל לידי מכשול איסור מוזהר גם בזה כמו שכתב הרא״ש שם מהא דפסחים (דף מ׳) גבי ארבא דטבעא בחשתא דאסור לזבוני לנכרי דלמא הדר ומזבנא לישראל ע״ש ולכן מטעם לפני דלפני אין להתיר אבל מכ״מ אין חשש מטעמא דכתבנו דאין בזה משום לפני עור ולא משום מסייע בשעת עבירה כנלענ״ד: הקטן יעקב.\\\vspace{0pt}

\end{multicols}\newpage

\newchap{סימן טז}
\begin{multicols}{2}
ב״ה אלטאנא, ל״ג בעומר תר״ך לפ״ק. להיקר התורני מ״ה יוזל הירש נ״י בק״ק האלבערשטאדט יע״א.\\\vspace{0pt}

על דבר שאלתו דמר נ״י באחד שרוצה להשתתף עם א״י בפאבריק שעושין בה יין שרף שקורין שפריט וגם ברחים והפאבריק וכן הרחים יקראו על שם הא״י מבלי שיקרא שם ישראל על השותפות ונפשו בשאלתו אם יש לעשות כן מבלי נדנוד איסור?.\\\vspace{0pt}

שאלה זו יש לה ג׳ סניפים – א׳ – לעשות שותפות עם א״י – ב׳ – אודות מלאכת שבת וי״ט – ג׳ – אודות חמץ בפסח. והנה מה שנוגע לענין שותפות א״י יש פלוגתא בין הפוסקים כמבואר א״ח סי׳ קנ״ו ובשו״ת מעיל צדקה מחמיר וכ״נ מדברי הפ״מ י״ד סי׳ ס״ה אכן כיון שהטעם שמא יתחייב לו שבועה יש תקנה לזה שיתנו ביניהם שאם יפול בין השותפים דבר המשפט יוכרע ע״פ בוררין והם ישפטו ע״פ שכלם מבלי שיחייבו להשותפים שבועה ועין בטורי אבן מגילה (דף כ״ז) שכתב ג״כ תקנה שבאם יתחייב לו שבועה ימחול לו רק שכתב שבזה יש חשש אחר משום מתנת חנם ע״ש אבל אם יתנו בתחלה שלא יבואו לידי שבועה לכ״ע אין חשש איסור בזה.\\\vspace{0pt}

גם מה שנוגע לאיסור מלאכה בשוי״ט כיון שהא״י שותף בו ליכא כאן משום מלאכה בשביל ישראל דאדעתי׳ דנפשי׳ עבוד ומה שמקבל הישראל הריוח יש תקנה לזה כמבואר סי׳ רמ״ה שיתנו בתחלה בשעה שבאו להשתתף שיהי׳ שכר שבת וי״ט לא״י לבדו אם מעט ואם הרבה וכנגדו יטול הישראל שכר יום אחד מימי חול כנגד שבת אם מעט ואם הרבה ואז אם הא״י נתרצה בשעת חלוקה לחלוק בשו׳ מותר ואפילו לכתחלה יש לנהוג כן אם לא ידעו כמה הרויחו בימי שבת וי״ט ובימי החול וכמו שכתב המג״א שם ס״ק ב׳ בשם ר״ל חביב (ואף שמדברי המג״א סי׳ רמ״ו ס״ק י״ג משמע דרק בהפסד גדול שרי לחלק סתם ט״ס נפל במג״א ששם איירי בלא התנו בתחלה כמשכ׳ במחצית השקל) אבל בהתנו בתחלה אין חשש איסור שאם באו אח״כ לחלוק שיכולים לחלוק בשו׳ אפילו ליכא הפסד מרובה.\\\vspace{0pt}

אמנם התקון היותר קשה הוא לענין חמץ בפסח שלא יעבור על שהיית חמץ ברשותו שאם לא הי׳ נוגע רק לענין משתכר באיסורי הנאה הי׳ אפשר למצוא תקנה כמבואר בסי׳ ת״נ לענין תנור בשותפות אבל הכא יש חשש האיסור של בל יראה וב״י כיון שיש לישראל שותפות בחמץ של פאבריק וריחים ואין תקנה לזה כשיתנו בתחלה שכל ימי הפסח יהי׳ הכל נמכר לא״י ואחר כלות הפסח יחזור לישראל שמלבד שזה הערמה בלא״ה אינו מועיל כי במה קנה הא״י קנינו שיהי׳ הכל ברשותו ויצא מרשות ישראל ולכן לא מצאתי תקנה לזה אלא שבכל שנה קודם הפסח יעשה שטר מכירה גמורה שנאמר בו כי השותף ישראל מכר כל החמץ שלו שהוא בפאבריק וריחים להא״י בשער השוק ואם אי אפשר להזכיר הסכום יאמר שמאמין הקונה למקנה או המקנה לקונה על זה וישכיר לו החלק שיש לו בהמקומות שמונח עליהם החמץ גם יקבל הישראל מהקונה כסף לחזק המקח שקורין דרויף געלד ואגב המעות ושכירת קרקע יקנה החמץ להאינו ישראל שיהי׳ שלו בקנין מוחלט באופן שיכול למכרו למי שירצה ואחר הפסח יחזור הישראל ויקנה מהא״י אבל לא יאמר שהמכירה תהי׳ בטלה אלא יקנה מהא״י החצי מכל מה שבפאבריק ורחים בשער השוק ויעשו שטר שותפות מחדש ופשיטא שלא יחשב לישראל ריוח כנגד ט׳ ימי פסח דהיינו עם ערב פסח כיון שהתשעה ימים האלה לא הי׳ לישראל חלק כלל בהפאבריק והריחים ורק בשביל שכירת החלק שיש לו בפאבריק והכלים יכול לקבל מה שיגיע לזה ע״פ היושר כנגדם בימי החול ובזה אין חשש איסור עוד ואפילו יודע הא״י שהישראל יחזור ויקנה ממנו אחר הפסח מכ״מ כיון שבימי הפסח נקנה לא״י כדין והכל ברשותו ואחריותו אין כאן איסור וע״פ האופנים האלה יש להתיר שותפות זה מבלי שיש כאן נדנוד איסור כנלענ״ד הקטן יעקב.\\\vspace{0pt}

\end{multicols}\newpage

\newchap{סימן יז}
\begin{multicols}{2}
אלטאנא, יום ו׳ ערב פסח תר״ט לפ״ק. להרב וכו׳ מ״ה יהושע כהן נאשקוני נ״י מק״ק נייאשטאדט במדינת פולין.\\\vspace{0pt}

מה ששאל מעכ״ת נ״י וז״ל. זה רבות בשנים, אשר הרדיפוני מנוחה ובאתי עד פלך רוסיא הלבנה, אשר בקסרית הכללית, וראיתי אשר עירו בש״ק מים רותחין מכלי ראשון אשר היו טמונים בתנור על טהעע חדש שלא נתבשל מע״ש, ונבהלתי בראותי זאת. ואמרתי להם, מחללים שבת אתם. כאשר הלכה זו רווחת בישראל, עירוי מבשל כדי קליפה, ומבואר בהדיא בש״ע או״ח (סי׳ שי״ח סעי׳ יו״ד) אסור ליתן תבלין בקערה ולערות עליהם מכלי ראשון, ועיין במ״א שם, והשיבוני כן נוהגין אצלם רבים וכן שלימים, וכאשר שמתי לבי לחקור ע״ז, איזה שורש דבר היתר מצאו בזה, לא מצאתי מענה, עד אשר הוגד לי אחרי זה, מפי איש אחד מהנוהגין כן, היות שנודע להם מפי השמועה שכבר נתבשל הטהע במדינות חינא טרם נשלח לפה, וא״כ אין בישול אחר בישול, ובאמת הי׳ אז הדבר בעיני כמחזה שוא, עד אשר הוגד לי מאחד ממכירי, אשר יש לו עסק טהע בהמבורג, ואמר לי כן הדבר שבחינא מערין חמין על הטהע, ומגלגלים אותם בכריכה ע״י כלים מיוחדים לזה, אבל לא ידע בבירור אם מערין עליהם רותחין או פושרין וכאשר ראה לבי כי דברים בגו, אמרתי נכון הדבר לעמוד על העיקר.\\\vspace{0pt}

תשובה – הנה חקרנוהו כן הוא שהטהעע בא בלתי מבושל אלינו, ולכן אין ספק שיש בזה איסור מלאכת שבת לערות חמין מכלי ראשון על טהעע, ורבים מהמון עם גם מהיראי ה׳ אינם יודעים כי יש מלאכת בישול גם ע״י חמי אור ונכשלים באיסור סקילה שכן שמעתי שלפעמים מערין מרק בשר מכלי ראשון על חלמון ביצה חי וכן כשעושים משקה ידוע הנעשה ממי לימוניש ודבש ויין שרף ומים חמין מערין החמין מכלי ראשון על המי לימוניש שלא נתבשלו עדיין ובכל זה יש איסור סקילה ואפילו הנזהרים לבשל קודם השבת ולערות חמין עליהן בשבת יש איסור שהרי לפי המבואר בסי׳ שי״ח ס״ד בדבר לח יש בישול אחר בישול ולכן בדבר לח אין תקנה רק שיבשל לגמרי קודם שבת ויטמין בדרך היתר הטמנה המבואר סי׳ רצ״ג ובטהעע יש היתר אם יערה חמין מכ״ר עליו מע״ש ויוריקו המים עד שנשארו העלים יבש ממש ואז יוכל לערות חמין עליהם בשבת שביבש אין בישול אחר בישול וכ״כ הפרי מגדים ויש לפרסם דברים האלה להרים מכשול מדרך בית ישראל. כנלענ״ד הקטן יעקב.\\\vspace{0pt}

\end{multicols}\newpage

\newchap{סימן יח}
\begin{multicols}{2}
אלטאנא, יום ו׳ י״ט אייר תר״ט. להרה״ג וכו׳ מ״ה יצחק דוב ב״ב הלוי נ״י הגאב״ד דק״ק ווירצבורג יע״א.\\\vspace{0pt}

על מה שכתב מעכ״ת נ״י – וז״ל. – מה שכתב מר נ״י לאסור לערות מים חמין מכלי ראשון על עלי ירקות (טהעע) אין צריך חיזוק והענין ברור באין פקפוק לכל מעיין במקור ההלכה דמאחר דאין שום הפרש בין מבשל באש עצמו למבשל בתולדות אש כגון מים חמין או דופני כלי שהוחמו באש וכה״ג כולם בסקילה כדאי׳ להדיא בשבת (דף ל״ח ע״ב ודף קמ״ה ע״ב) ופסקוהו כל הפוסקי׳. וכיון דלר״ת עירוי מבשל לפחות כדי קליפה, עיי׳ פרמ״ג י״ד סי׳ ס״ח, ממילא דהנדון שדברו עליו הרבנים הנ״ל הוא אסור גמור. ואין ר״ת יחיד בסברתו זאת דרבים מאבות העולם סוברים כן, הלא המה: רש״י, רמב״ם כמ״ש הה״מ בכוונתו פ׳ כ״ב, ובס׳ ארוך מש״ך סי׳ ק״ה כתב שגם הסמ״ג סובר כן ושכ״נ דעת סה״ת והג״מ ושכן דעת הסמ״ק והרוקח והרא״ם בסי׳ וראב״ן סי׳ כ״א ור״מ ריקנטי והאגודה ושכן מסיק רבנו ירוחם ושגם מסקנת הרא״ש נראה כן. ובש״ע סי׳ שי״ח סתם כן, א״כ מי יוכל להרים יד נגד כל הנך אריוותא וממילא דהאופן שבאה השאלה עליו הוא נוגע לאיסורא דאורייתא ממש. (ועי׳ מ״ש הגרא״ו בסי׳ ס״ח ס״ק כ״ז שלענ״ד דבריו תמוהין מאד דהוכחתו כבר דחו תוספ׳ עצמם בזבחים שם). אמנם גם התקנה שדבר עליה מעכ״ת נ״י והוזכרה גם בפרמ״ג דהיינו לערות חמין מכ״ר על העלי טהעע מע״ש ליבשם אח״כ לגמרי ואז אם מערין עליהן בשבת הוי א״ב אחר בשול צריכה עיון גדול לפענ״ד, ואלה הדברים שקשים למיעוט בינתי עליה.\\\vspace{0pt}

(א׳) לפמ״ש האו״ה כלל ל״ד סימן ט״ו דקערה צריך ס׳ נגד כולה לפי שמערין עליה כמה פעמים ונשאר בה הבליעה כל פעם והולך ובולע מידי יום יום בכולה דוקא בערוי של פ״א אמרינן שאין בולע אלא כדי קליפה וכו׳ אבל בכה״ג ודאי בולע בכולו וכו׳ יעייש״ה וקלסיה הב״ח בסי׳ צ״ה, גם הט״ז שם ס״ק י״ב מביאו וכן נראה דעת רמ״א בת״ח כלל ל״ג דין ד׳ שכתב וז״ל כתבתי לדין זה ללמוד ממנו לדין ערוי דאוסר כדי קליפה דיותר יש לזהר בפעם ב׳ מבפעם א׳ וכו׳ עכ״ל ואע״ג דהחוו״ד בסי׳ צ״ה ס״ק ט׳ תמה על האו״ה אין דבריו מוכרחים כ״כ ועיין פרמ״ג י״ד סי׳ ס״ח בשפ״ד ס״ק כ״ז ולבוש א״ח סי׳ תנ״א סעיף ד׳ ובחק יעקב שם ס״ק כ״ט ובהגעלה אחר שנפגם פשיטא דאין קו׳ הפרמ״ג חמורה כ״כ ודו״ק ועיי׳ או״ה כלל נ״ח דין מ״א. וממילא דכמו דלהאו״ה וסייעתו יש חלוק בין עירוי אחת להרבה לענין בליעה ופליטה הה״נ דלענין בשול יש לומר כן, ואפשר דלגבי מאכל כ״ע מודו, ויהי׳ איך שיהי׳ מידי ספקא לא נפקא ומנ״ל להקל בדבר הנוגע לדאורייתא.\\\vspace{0pt}

(ב׳) לפמ״ש תוס׳ פ׳ כירה ד׳ ל״ט ד״ה כל בחד תירוצא דכל שלא נתבשל ממש מע״ש אין שורין אותו אפילו בכ״ש דמחזי כמבשל ומשמע דאפילו בנשרה גם מע״ש בכ״ש אפ״ה אין שורין אותו בכ״ש בשבת ולא אמרינן ממנ״פ אם כ״ש מבשל הרי כבר נתבשל מע״ש ואם אינו מבשל הלא גם בשבת אינו פועל כלום זה לא אמרינן, דעכ״פ מחזי כמבשל, וזו היא הכוונה האמתית של הגהת מרדכי פ׳ כירה (ודלא כשהבינו הב״ח שבמג״א סי׳ שי״ח ס״ק י״ד ועיי״ש בב״ח עצמו דלפירושו הוה ההגמ״ר נגד המשנה לכאורה) וכן הבינו הא״ר שם ס״ק י״ב וכן משמע מפרמ״ג בא״א ס״ק י״ד (ותיבת והגמרא שכתב הוא ט״ס וצ״ל והגהת מרדכי) ואפילו לתירוץ האחר של תוס׳ ולהמתירי׳ בכ״ש עיי׳ א״ר ס״ק י״ב י״ל דבעירוי מודו דכל שלא נתבשל מע״ש ממש בכלי ראשון כי אם בערוי מכ״ר אסור לערות עליו בשבת וכדעת התו׳ וש״ע עיי׳ מג״א ס״ק י״ד וט״ו וא״ר ס״ק י״ב וי״ג.\\\vspace{0pt}

(ג׳) דעל ידי הערוי של ע״ש לא נתבשל כי אם כדי קליפה מצד העליון של אותם העלין שקלוח המים יורד עליו ממש אבל לגבי צד התחתון מקרי נפסק הקלוח והיא לא נתבשלה עדיין, וכן בשני עלים התחובין זב״ז דשכיח טובא יש לחוש לזה ע״ד הא דאמרינן פ׳ כל שעה דף ל״ט ע״ב לא ליחליט אינש תרי חטי וכו׳, ועוד היכא שהרבה מהעלין מונחים בקדרה ודאי דאין הערוי שמע״ש מבשלת רק העליוני׳ ולגבי התחתוני׳ מקרי נפסק הקלוח וממילא דאם בשבת מתהפכים (כטבע דברים הצפים במים) ונעשו עליונים למטה ותחתונים למעלה הוי בשול חדש. אע״כ לפענ״ד הדבר צע״ג מאד: עכ״ד דמר נ״י.\\\vspace{0pt}

על זה אשיב: מה שהשיג מעכ״ת נ״י על התקנה שכתבתי בשם הפרי מגדים לערות מע״ש על טהעע מכ״ר ואז יכול לערות גם בשבת – יצדק אם הטהעע יהי׳ דבר גוש אבל לפענ״ד אין הדבר כן כי אחר שכל עלה ועלה לעצמו והעלה דבר דק מאוד ואיננו יותר מכדי קליפה א״כ כשנתבשל העלה כדי קליפה ע״י עירוי כבר נתבשל כל העלה מע״ש ואף דע״י שנתייבשו ונצטמקו העלים לפעמים נראים עבים קצת מה בכך הא חלולים הם ונכנס הקילוח לתוכן ולכן אין ראי׳ מלא לחלוט אינש תרי חטי דשם החטה דבר קשה עב ויבש ואם נדבקה בצירה של אחרת אין נכנס שם הקילוח אבל העלים הם קלים וצפים ודאי נכנס קילוח המים ביניהם אפילו מונחים הרבה זע״ז (מה שאינו שכיח כלל כי לא נצרכים הרבה לעשות המשקה) וכל שכן שאין לחוש לצד התחתון של העלה לומר ששם נפסק הקילוח שהרי זה אפילו בחטה לא חיישינן דא״כ אפילו אחת אחת לא היו יכול לחלוט בדרך חליטה לשפוך הרותחים עלי׳ ולכן כשנחשוב בישול כדי קליפה מלמעלה ומלמטה לא נשאר בו עוד מה שלא נתבשל אפי׳ עב קצת. גם מה שפשוט למעכ״ת נ״י דהקלוח שבא מצד העליון לתחתון וכש״כ לעלים שתחתיו מקרי נפסק הקלוח לא נלענ״ד שהרי לפי המבואר בי״ד סי׳ צ״ב ס״ז אמרינן דקלוח מכלי ראשון שנזחל ע״ג כירה עד שהגיע לקדרה מקרי לא נפסק הקלוח כל זמן שהיד סולדת בו והוא מתה״ד סימן קפ״א שלמד כן מהא דאמבטי דפרק כירה ע״ש הרי דגם לענין בישול של שבת מקרי לא נפסק הקילוח אפילו נמשך דרך קרקע מכלי ראשון כש״כ כשבא הקלוח מעלה עליון לתחתון הן אמת שמדברי האו״ה שהביא הש״ך סי׳ צ״ה ס״ק י״ח דעירוי אינו אוסר רק אותו כלי שנח הקילוח עליו לא משמע כן וכבר הקשה הפ״מ שם סתירה זו ע״ש שהאריך והניח בצ״ע אכן לענ״ד גם האו״ה מודה בנדון זה דלא מקרי נפסק הקלוח דהוא לא איירי רק כשנפל הקלוח על הכלי שהוא דבר קשה שנח שם מעט ומשם ירד לכלי אחר זה דן ככלי שני אבל בנפל על דבר רך וחלל מעליון לתחתון ודאי לכ״ע לא מקרי נפסק הקלוח. ולכן לענ״ד אפילו הי׳ כאן איסור בישול ודאי לדינא אין חשש בתקנת הפ״מ כיון שנתבשל הכל מע״ש וכש״כ שדי בכך כיון דלהרבה פוסקים ובראשם הרשב״ם והרמב״ן והרשב״א והר״ן עירוי אין מבשל כלל וכבר פליגי אמוראי בירושלמי בזה וגם הפ״מ (סי׳ ס״ח) לא קרא לי׳ רק ספק איסור דכתב ואנן קיי״ל עירוי מבשל כ״ק מספק לחומרא שמא הלכה כר״ת אבל מכ״מ למעשה אני מסכים עם מר נ״י שטוב יותר לבשל העלים מעט מע״ש ממש בכלי ראשון ולייבשם למען לצאת כל החששים שאפשר שיתהוו. רק אזכיר עוד למען הצדיק את ישראל שמה שכתב הרב השואל נ״י שהעושים כן לערות על טהעע בלתי מבושל מע״ש נכשלים בספק איסור סקילה שזה אינו לענ״ד דאיסור סקילה ליכא שהרי שיעור בישול שבת הוא כגרוגרות ושיעור כזה ודאי ליכא בבישול כ״ק מהטעע שרגילין ליקח אפילו לבשל קדרה גדולה ועכ״פ ליכא בזה יותר מאיסור חצי שיעור וכבר נסתפק בשו״ת חכם צבי (סי׳ פ״ז) אם חצי שיעור בשאר איסורים שאינם איסורי אכילה דאורייתא או דרבנן ואף שהגהת משנה למלך ה׳ שבת (פ׳ י״ח) השיג עליו והוכיח מרש״י שבת (דף ע״ד) דאסור מן התורה (ועיין בפרי מגדים א״ח בפתיחה כוללת שהאריך בזה) לענ״ד שיטת רש״י בזה לא דכולי עלמא הוא ותלי בפלוגתא דראשונים שהביא המ״ל ה׳ שבועות (פ׳ ד׳) אם בשבועה ח״ש אסור מה״ת דלדעת הרמב״ם והר״ן שאסור משום דחזי לאצטרופי גם במלאכת שבת אמרינן כן כיון דחזי לאצטרופי אבל לשטת הרמב״ן והיש מי שאומר שהביא הר״ן דטעם דחזי לאצטרופי לחוד לא סגי לאסור ע״ש גם בשבת לא אסור מן התורה רק מדרבנן ואף דמדברי רמב״ן נראה דבשבועה אפילו מדרבנן לא אסור י״ל דמשום חומרא דשבת אסור וא״כ אפשר דליכא בזה רק ספק איסור דרבנן אבל עכ״פ איסור סקילה ליכא. כנלענ״ד הקטן יעקב.\\\vspace{0pt}

\end{multicols}\newpage

\newchap{סימן יט}
\begin{multicols}{2}
אלטאנא, יום ו׳ י״ט שבט תר״י לפ״ק. להרב וכו׳ מ״ה יהושע כהן נשקוני נ״י מק״ק נייאשטאדט במדינת פולין יע״א.\\\vspace{0pt}

על מה שכתבתי במכתבי אל הגאב״ד דק״ק ווירצבורג נ״י הנ״ל שאיסור סקילה ודאי ליכא כתב מעכ״ת נ״י וז״ל – מה שכתב מר נ״י דשיעור בישול הוא בגרוגרות אמת ויציב הוא, כדאיתא בסוגין דהמוציא יין, (דף פ׳ ע״ב) כל שיעורי שבת כגרוגרות וערש״י שם וכ״כ בהדיא הרמב״ם ז״ל בהל׳ שבת (פרק ט׳ הל׳ א׳) אולם מה יענה מר נ״י ביום שידובר בו ממתני׳ דפרק ר״ע (דף פ״ט ע״ב) דשנינן בה תבלין כדי לתבל ביצה קלה וערש״י שם ולפי המתבאר מסוגיין דהמוציא יין הנ״ל דכל שיעורי שבת באוכלין שוין כפירש״י שם א״כ כמו דחייב על הוצאת תבלין כדי לתבל בהן כגרוגרות מביצה קלה ה״נ חייב על בישולן בשיעור זה וע״כ צ״ל טעמא דמלתא דכיון דתבלין אינן עומדין למיכלן בעינייהו רק לתבל בהן אוכלין להכי שיערו בהן כדי צורכן לתבל בהו כגרוגרות ממאכל קל כביצה קלה א״כ ה״נ בנידון דידן כיון דהטהעע ג״כ אינו ראוי לאכילה רק הוא עשוי להכשיר המים הנשפכים עליו א״כ שיעורו כדי לתבל רביעית שהוא שיעור משקין ושיעור זה איכא אפי׳ במבשל קדירה קטנה ביותר דעכ״פ רביעית איכא וא״כ שפיר אתינן עלה בלתא דאיסור סקילה.\\\vspace{0pt}

ובדברים אלה יתישבו דברי הרמב״ם ז״ל (הלכה הנ״ל) שכתב ושיעור המבשל סממנים כדי שיהיו ראויין לדבר שמבשלין אותן לו והגיה הראב״ד ז״ל כדי לצבוע בגד קטן פי סבכה כמתני׳ (דף פ״ט ע״ב) והרב פרי מגדים בפתיחתו הכוללת (הלכות שבת ד״ה עוד רגע אדבר) הוסיף להפליא עוד מדוע לא השוה הרמב״ם ז״ל מדותיו למה שכתב במלאכת הוצאה (פרק י״ח הלכה ח׳) ששם כתב כדברי הראב״ד ז״ל ע״ש דבאמת לפי דרכינו הנ״ל י״ל שפיר דכאן גבי בישול שכתב רבינו סתם המבשל סממנים הנה מלת סממנים כוללת ג״כ תבלין העומדין לאכילה ולריח כלשון המקרא קח לך סמים א״כ לא שייך בהו כדי לצבוע בדברים העומדין להכשיר מאכל או משקה וע״כ כתב כדי שיהיו ראויין לדבר שמבשלין אותן לו ואם הוא באוכלין שיעורן לתבל כגרוגרות מביצה קלה ואם במשקין לתבל רביעית כנ״ל לאפוקי בפרק ח״י גבי מלאכת הוצאה ששם כתב בהדיא סמנים העומדין לצביעה שפיר כתב בהו כדי לצבוע פי סבכה ולזה נתכווין גם בעל מגדול עוז והפר״מ כתב עליו ולא הבנותיו ולדברינו דבריו מבוארין עכ״ד.\\\vspace{0pt}

ועל זה אשיב: הנה מעכ״ת נ״י רצה להמציא דבר חדש אשר לא הוזכר בשום מקום ששיעור בישול תבלין בשיש בהם לתבל כגרוגרות ואשר עשה סמוכין לדבריו אדרבא לענ״ד משם ראיות להיפך דמה דאמרינן שבת (דף פ׳) כל שיעורי שבת באוכלין שווין כגרוגרות הרי אדרבא שיווי שלהם הוזכר לענין כגרוגרות והיינו דפריך דהשיעור היותר גדול לענין אוכלין הוא כגרוגרות דהיינו בסתם אוכלין אבל יותר מזה לא מצינו ולמה ישוער בעצים יותר מזה אבל לא הוזכר בזה שהיכי ששיעור הוצאה פחות מכגרוגרות שיהי׳ ג״כ השיעור בישול וקצירה וטחינה פחות מכגרוגרות אם יהי׳ בו לתבל כגרוגרות דכה״ג לא הוי שתיק השס׳ להשמיענו בשום מקום ובפי׳ נראה מדברי הרמב״ם ה׳ שבת (פ׳ ח׳ ה׳ ט״ו) דגם בשוחק תבלין וסמנים לא מחייב משום טוחן בפחות מכגרוגרות. גם מדברי הרמב״ם ה׳ שבת (פ׳ ט׳) הוא ראי׳ להיפך דאם איתא דכל השיעורים שהם לענין הוצאה הם ג״כ לענין בישול א״כ יתחייב בחלב בהמה טהורה בכדי גמיעה ובטמאה בכדי לכחול עין א׳ ובשמן בכדי לסוך אצבע קטנה ובדבש בכדי ליתן על הכתית בבישול כמו בהוצאה ומכל זה לא הוזכר ברמב״ם ואפילו במים שכתב השיעור לענין בישול בכדי לרחוץ אבר קטן לא שוי השיעור לענין הוצאה מלבד שגם לשיעור של המים ושל הסממנים לא מצאנו מקורו כמו שכבר הקשה הכס״מ בשם רמ״ך וכנגד זה יהי׳ קולא דבשאר משקים ששיעור הוצאתן ברביעית גם בבישול לא יתחייב רק בשיעור רביעית וגם מזה לא הזכיר הרמב״ם מאומה ומה שרצה מר נ״י לומר שמשכ׳ הרמב״ם ושיעור המבשל סממנים כדי שיהיו ראויין לדבר שמבשלים אותם לו שכוונתו גם לתבלין שנקראו סממנים כדכתיב סמים – מלבד שלא מצאנו בשום מקום תבלין נקראו בשם סממנים ולשון תורה לחוד ולשון חכמים לחוד וכן סמים וסממנים הם לשונות חלוקות גם בלא זה אי אפשר לומר כן שהרי לשון סמים נאמר על הבשמים וא״כ יהיו גם בשמים בכלל סממנים והרי במיני בשמים חייב בהוצאה בכל שהן וא״כ יתחייב גם בבשול בכל שהן ולא יהי׳ בהם שיעור כלל גם איך אפשר לפרש דמה שכתב הרמב״ם שחיוב סממנים תלי אם יהיו ראויין לדבר שמבשלים אותן לו שכוונתו בתבלין לתבל כגרוגרות דאיך נכלל זה בלשון לדבר שמבשלים אותן לו ואיזה גבול יש בזה שהרי לקדרה גדולה צריך הרבה ולקדרה קטנה מעט, ולכן ודאי שלא נתכוון הרמב״ם בזה רק לענין צביעה וכמו שפי׳ הלח״מ כוונתו אבל לענין בישול תבלין שיעור החיוב כגרוגרות דוקא כמו בשאר אוכלין וכמו שכתב הרמב״ם לענין טוחן תבלין וכנ״ל. כנלענ״ד הקטן יעקב.\\\vspace{0pt}

\end{multicols}\newpage

\newchap{סימן כ}
\begin{multicols}{2}
ב״ה אלטאנא, אלול תר״ט לפ״ק. לאחי הרה״ג וכו׳ מ״ה ליב עטטלינגער נ״י אב״ד דמדינת לאדענבורג.\\\vspace{0pt}

שאלת להודיע לך דעתי אם מותר לישא קטן שיכול לילך על רגליו בשבת בזמן הזה שאין לנו ר״ה רק כרמלית?\\\vspace{0pt}

תשובה – מצאתי בזה פלוגתא קדומה. ובתחלה אזכיר מה שראיתי בשו״ת פרי תבואה (מהרב הגאון יהודה ליב מרגליות נדפסת תקנ״ה) שכתוב שם שאלה אני שואל וכו׳ מה שאמר איש א׳ בקהלתנו שבזה הזמן שר״ה שלנו הוא כמו כרמלית במקום שאין עירוב אסור לישא תינוק שלנו מאחר שברה״ר אף שחי נושא את עצמו הוא פטור אבל אסור לכן אף בכרמלית אסור לכתחלה ואמר שכן מבואר במגן אברהם ואמת שמלשון הרמב״ם (פ׳ י״ח ה׳ ט״ז) ומג״א סי׳ ש״ח וסי׳ שמ״ח משמע קצת לאסור שכתב שם דבדברים האסורים משום גזירה דאסורים אף בכרמלית מכ״מ מצינו גם בדברים האסורים משום גזירה דמותרים בכרמלית כגון תכשיטי נשים ולפענ״ד הוא נמנע שיהיו רוב העולם טועים ונכשלים בזה לכן בקשתי וכו׳ והנה בתשובה האריך המחבר שאף שלפי דברי המג״א בסי׳ שמ״ח לכאורה תלי בב׳ תירוצי התוספ׳ אי גזרינן בכרמלית גזירה לגזירה מכ״מ בדברי סופרים הולכין אחר המיקל ועוד הביא הרבה ראיות שלא גזרינן גזל״ג ובפרט דבזה״ז שאין לנו רשה״ר כלל גזירת כרמלית אטו רה״ר היא גזירה רחוקה ובודאי לא גזרינן גזירה לגזירה והביא ראי׳ משו״ת עבודת הגרשוני סי׳ ק״ד שכתב ג״כ דמותר לעשות בכרמלית עקירה או הנחה לחוד מאחר דפסקינן כרבא בעירובין (דף צ״ט) דלא גזרינן גזל״ג וסיים המחבר וז״ל והרי מבואר ממה שכתבתי דאין לאסור בנדון השאלה שלנו על פי הראי׳ שהביא השואל מאחר דבכל דוכתא לא גזרינן גזל״ג אם לא היכי דמצינו בשס מפורש לאסור ברם סוגיא אחרת עומדת לנגדי לכאורה דברייתא מפורשת שנינו שבת (דף קכ״ח) האשה מדדה את בנה בר״ה ואצ״ל בחצר דמשמע אפילו בחצר מדדה אין אבל לשאת לא אך עכ״ז מאחר שגם בברייתא זו לאו בפי׳ אתמר אלא מדיוקא וקליש אסורא דהוי גזירה לגזירה ובפרט דנדון דידן מקרי שעת הדחק שאם תניח האשה עולה כאשר תלך לבית אבי׳ איכא צערא דינוקא נ״ל דאין להחמיר לאחרים אלא לעצמו עכ״ל והנה באמת מה שדייק הרב בזה מהך דאצ״ל בחצר לענ״ד לאו דיוקא הוא דאפילו נימא דבכרמלית אסור מכ״מ בחצר מותר לשאת דהך חצר חצר המעורבת הוא כנראה מדברי הטור דדייק גבי בהמה מדלא מדדין רק בחצר משמע הא בכרמלית לא הרי שחילק בין חצר לכרמלית ולכן פשוט מה דנקט ואצ״ל בחצר שנקט כן אגב דקתני בבהמה בחצר ולא בר״ה ובאדם אפילו בר״ה מותר נקט ואצ״ל בחצר אבל לעולם י״ל דבאדם לישא ממש גם בכרמלית מותר. אבל מה שפשוט לאחרונים הנ״ל שבכרמלית מותר לישא קטן ששייך בו חי נושא א״ע נ״ל שיש פלוגתא בזה בין גדולי הראשונים דהרשב״א שבת (דף צ״ד ע״ב) אהא דאמרינן שם מי קאמינא לר״ה לכרמלית קאמינא כתב בשם הרמב״ן דנראה כיון שצריך להוציאו לכרמלית שרי אפילו בלא ככר כדי שלא ירבה בהוצאה שהוצאת התינוק עצמה אסורה היא לכרמלית ואם המת הותר מפני כבודו יתירו בהוצאת התינוק והככר עכ״ל הרי שכתב בפי׳ שהוצאת התינוק אסורה לכרמלית דא״ל דזה דוקא בתינוק קטן שלא יוכל לילך ברגליו דא״כ אכתי למה התיר בלא תינוק הלא יש תקנה לטלטול ע״י תינוק ששייך בו חי נושא א״ע ועיין במג״א (סי׳ שי״א) שכתב אהך שיטה דבר״ה מותר ע״י תינוק דהיינו ע״י תינוק שגדול קצת הרי ששייך טלטול על ידו וא״ע ראיתי בתוספ׳ שבת שם שנבוך בדברי הרמב״ן מפני קושיא זו דלמה לא יטלטל ע״י תינוק שגדול קצת ומקונן שהאחרונים קצרו ושתורת האדם לרמב״ן לא ראה מעולם ובאמת יפה דיבר שאלו ראה דבריו הי׳ רואה מה שיטת הרמב״ן בזה שאחר שהביא שם מה שכתב הרשב״א בשמו כנ״ל כתב וז״ל ואיכא למידק אשמעתין לר״ש דאמר הוצאת המת אפילו לרה״ר איסורא דרבנן הוא במוטל לחמה והא מותר להוציאו אפילו לר״ה דומיא דכרמלית לר׳ יהודה ועוד קשיא לן לר״ש כיון דלרה״ר איסורא דרבנן הוא לכרמלית יהא מותר לכתחלה אפילו בלא כבודו דהא אמרינן בפ׳ יציאת השבת גבי לא יעמוד אדם ברה״ר וישתה ברה״י איבעיא להו לכרמלית מהו אמר אביי היא היא רבא אמר היא גופא גזירה ואנן ניקום ונגזור גזל״ג אלמא כל מלתא דבר״ה שריא מדאורייתא לכרמלית מותר לכתחלה לכל אדם ויש לומר שהכרמלית עשאוהו מדבריהם כר״ה ולא התירו בו אלא משום כבוד הבריות אבל כשהתיר רבא לעומד ברה״י לשתות בכרמלית מפני שאינו עושה מלאכה כלל ואינה אסורה בשום מקום וכן התיר בר״ה יציאת החייט במחט התחובה לו בבגדו וכיוצא בו סמוך לחשיכה שאין גוזרין שמא ישכח ויוציא כיון שהוצאתו בשבת עצמה אינה אלא משום שבות כללו של דבר הגזירות שאין בהם מעשה כלל התיר הא במוציא ממש לכרמלית דברים שהוצאתן לר״ה משום שבות אסור שלא מצינו היתר להוציא פחות מכשיעור לכרמלית וכן כזית מן המת וכזית מן הנבלה אלא עשו הכרמלית כר״ה למלאכות אפילו לדבריהם אלא שלא עשאוהו כר״ה לגזירת שמא יוציא עכ״ל וסברה עמוקה היא ראוי׳ למי שאמרה ובזה מובן למה אסר הוצאת תינוק לכרמלית כיון דחי נושא א״ע בר״ה פטור אבל אסור כמו מלאכה שא״צ לגופה לר״ש א״כ הוי שבות דאסור בכרמלית. שוב ראיתי בחדושי הרמב״ן לשבת שנדפסו מחדש שמביא גם שיטת ראשון החולקת בזה שכתב (דף צ״ד) אהא דשרי ר״נ לאפוקי לכרמלית וז״ל פירשו המפרשים דע״י ככר או תנוק וכו׳ ויש מי שאומר שהטלטול ע״י ככר הי׳ ולא ע״י תינוק שא״כ אפילו לר״ה ליכא בהוצאת התינוק אלא איסורא דרבנן בעלמא כדאמרן וכיון דלר״ה ליכא בהוצאת התינוק אלא איסורא לכרמלית מותר לכתחלה ואפילו בלא כבוד הבריות וכיון שהתינוק מותר להוציאו לכרמלית המת עם התינוק נמי מותר כדחזינן בפ׳ במה אשה דכובלת וצלוחית של פלייטון מותר לכתחלה דכיון שהבושם מותר הכלי נמי שמוציאם אגב הבושם מותר וזו היא דעת ה״ר יצחק בר אבא מרי ז״ל ואינו נכון כלל וכו׳ עכ״ל ואח״כ כתב כדבריו בתורת אדם הנ״ל. הרי דלשיטת הרבינו יצחק הותר לשאת תינוק בכרמלית בשבת שיש בו משום חי נושא א״ע ולשיטת הרמב״ן אסור. והשתא כיון שהרמב״ן שראה דברי ר״י חלק עליו וגם הרשב״א והר״ן שהעתיקו דברי הרמב״ן לא חלקו עליו בזה מי ירים ראשו להקל נגדו. שוב ראיתי שגם מדברי המג״א (סי׳ ש״ח ס״ק ע״א) מוכח כן שדעתו לאסור נשיאת תינוק בכרמלית וע״י דברי הרמב״ן הנ״ל מובנים דבריו שמוקשים בלא״ה שהרי אהא דאשה מדדה את בנה בר״ה פירש כיון דאפילו אם תגביהנו פטורה דחי נושא א״ע ועל זה קאמר בש״ע ובלבד שלא תגררהו ופי׳ המג״א שמפני שכתב הר״ן דגוררת הוא נושאת ממש לכן אסור גם בכרמלית דאי לא הוי אלא משום גזירה שמא תנשא מותר בכרמלית מטעם גזירה לגזירה ולכאורה מה בכך הא אכתי הוי גזל״ג כרמלית אטו ר״ה ור״ה הוי דרבנן משום חי נושא א״ע ומה לי תרי גזירות מה לי תלתא גזירות אבל לפי סברת הרמב״ן הנ״ל א״ש דודאי אם גזירה היא שבות דנושא אין חילוק בין כרמלית לר״ה אבל אם היא מעשה היתר גמור ולא אסור בר״ה רק שמא יבוא לידי נושא אז הוי בכרמלית גזל״ג וא״כ מוכח גם דעת המג״א כן לאסור לישא תינוק בכרמלית. גם משום צערא דינוקא לענ״ד אין ללמוד היתר כמשכ׳ הבה״מ וכמו דאמרינן (סי׳ ש״ט) לענין טלטול קטן באבן בידו דכבר כתבו התוספ׳ דאין לדמות גזירות חכמים זו לזו ואם התירו איסור טלטול משום צ״ד מנ״ל להתיר גם איסור הוצאה בכרמלית ובפרט אחר ששם גופא מחלקינן בין אבן ובין דינר דבדינר שחיישינן דלמא אתי לאתויי לא מותר משום צ״ד וגם מאן לימא לן דצערא דשם דמי להך דהכא דשם שיש געגועין הוא קצת חולה כמבואר בפוסקים אבל מה צער יש כ״כ אם התינוק ישאר בבית וגם הבעל המאור לא כתב רק דמטעם צ״ד לא רצו חכמים לגזור שלא לדדות אבל מנ״ל שלא גזרו כן בגדול קצת שלא לנשאו לכן להלכה נלענ״ד אחר שכבר הורגל בפי העולם שתינוק שיכול לילך מותר לנשאו והדבר מסור לנשים שדעתן קלות ובדבר דרבנן אמרינן מוטב שיהיו שוגגין וכו׳ לכן אין למחות ביד הנושאות תינוק בכרמלית בשבת אבל לכתחלה אין להורות היתר בדבר וכש״כ שאין להקל לעצמו. כנלענ״ד הקטן יעקב.\\\vspace{0pt}

\end{multicols}\newpage

\newchap{סימן כא}
\begin{multicols}{2}
ב״ה אלטאנא, יום ג׳ ז׳ אב תר״ז לפ״ק. להרה״ג וכו׳ מ״ה אברהם זוטרא נ״י הגאב״ד דק״ק מינסטער יע״א.\\\vspace{0pt}

בדיק לן מר נ״י לחוות דעתי אם למול בן ישראלית שנתעברה מא״י בשבת.\\\vspace{0pt}

נדרשתי לזה ואבאר הדין ממקורו. הפוסק שהעיר בראשונה לזה שלא למול בשבת בן ישראלית שנתעברה מא״י הוא הר׳ משה שהביאו בשו״ת רבי מנחם הובא בדרישה י״ד סי׳ רס״ו והוא חולק עליו ועל זה כתב הגאון מ״ה וואלף האמבורג נ״י בספרו נחלת בנימין בקונטרוס המילה וז״ל ואמרתי דטעמו של ה״ר משה הוא עפ״מ דפירש״י בפ׳ אמור בן איש מצרי בתוך ב״י מלמד שנתגייר וקשה למה הוצרך להתגייר הא קיימ״ל נו״ע הבא על בת ישראל הולד כשר או כמ״ד הולד ממזר וא״כ לד״ה א״צ גירות וכן הקשה הרמב״ן והרא״מ אך דעת רש״י דטעמא של מ״ד הולד כשר כיון דבנכרים הולד הולך אחר הזכר כדאיתא בקידושין (דף ס״ז) וכו׳ ולהכי אצטריך קרא לאשמעינן שנתגייר ומתחלה רציתי לומר דסברת ה״ר משה לאסור ע״פ מש״כ התוספ׳ בקידושין (דף ע״ה) דנכרי הבא על בת ישראל הולד כשר וצריך להתגייר וכן דעת פסקי תוספ׳ וכמש״כ המהרש״א וכן רש״י ואם כן כל כמה דאינו מתגייר נכרי הוא ואסור למול וכו׳ ע״ש שהאריך ולבסוף מסיק וז״ל היוצא מכל דברינו דה״ר משה בדרישה אם סמך עצמו ע״ד פסקי תוספ׳ ורש״י אינו ראי׳ וכו׳ אלא דעיקר סמיכתו ע״ד רש״י בפ׳ אמור גבי איש מצרי דב״ע בזנות ואפ״ה צריך להתגייר כללא דמלתא דיש לה״ר משה סיוע מש״ס ופוסקים ע״כ אין להקל באיסור תורה ובנדון כזה יש להמתין למהלו עד אחר שבת עכ״ל הרב אוה״ב נ״י: ובתחלה אחקור אם באמת יש סיוע לשיט׳ ה״ר משה ואח״כ אם יש לסמו׳ עליו. הנה במה שכתב דיש סיוע משיט׳ רש״י ותוספ׳ דס״ל דאם אמרינן הולד כשר צריך גירות לבאר זה אעתיק פה מה שכתבתי בחדושי ליבמות על מה שכתבו התוספ׳ שם (דף ט״ז ע״ב) ד״ה אמוראי המהרש״א פי׳ מה שכתבו משום דסברה דהולד כשר משום דאי הולד כשר ע״כ הולך אחר הנכרי וכמו שכתבו התוספ׳ בקידושין ולענ״ד קשה דהתוספ׳ לא כתבו כן רק בלשון ושמא ובאמת שיטה זו צ״ע דמה שהביאו ראי׳ מסוגיא דקידושין (דף ס״ח) אדרבא שם מוכח להיפך דאף דהס״ד דמקשה דקאמר (שם) לימא קסבר רבינא הי׳ כן דלמ״ד הולד כשר אזיל בתר נכרי מכ״מ כי משני ממזר נמי לא הוי ע״כ מוכח אפכא דגם למ״ד הולד כשר ישראל הוא וכן נלענ״ד להוכיח ג״כ ממה דאמרינן לקמן (דף מ״ה) הולד פגום לכהונה ק״ו מאלמנה ע״ש והשתא אי ס״ד דלמ״ד כשר בתר נכרי אזיל א״כ ל״ל טעמא מק״ו לפסול אותה לכהונה תיפוק לי׳ דגיורת פסולה לכהונה מן התורה כדאמרינן לקמן (דף ס״א) והן אמת דיש ליישב זה על שיטת התוספ׳ ע״פ מה שפירש הרמב״ן שם דמה דאמרינן הולד פגום לכהונה היינו גם בבן דיש לו דין חלל לפסול לכהונה אם בא על ישראלית ולפ״ז א״ש דאף דהבת גיורת היא מכ״מ י״ל דצריך ק״ו משום בן דגר הבא על ב״י לא פסלה לכהונה אבל זה פוסל מק״ו אכן מדברי רש״י נראה בפי׳ דלא ס״ל כהרמב״ן וכ״כ הרשב״א והריטב״א שם והביאו הירושלמי דמבואר שם ג״כ דהק״ו רק לענין בת מייתי וא״כ ודאי יקשה דל״ל ק״ו כיון שגיורת היא אע״כ דאי אמרינן דהולד כשר הוי ישראל כשר וכ״נ גם מדברי הריטב״א ודלא כדברי התוספ׳. שוב ראיתי שהמהרש״א עצמו בקידושין (דף ע״ה) נראה כמסתפק אם דעת התוספ׳ שם כן דאזיל הולד בתר נכרי ע״ש ומה שהקשה שדברי רש״י סותרים זא״ז כפי מה שכתבתי בזה לק״מ דרש״י לא פירש כן רק בהס״ד דלימא קסבר רבינא אבל לפי האמת אדרבא מוכח דלא צריך להתגייר ולכן פי׳ רש״י שפיר כן (דף ע״ה) גם מדברי הפוסקים אהע״ז (סי׳ ד׳) דחולקים אי הבת כשר או פגום לכהונה מטעם ק״ו ע״ש מוכח דס״ל דלאו גיורת היא אלא ישראלית גם מדברי התוספ׳ ב״ב (דף ג׳) מוכח שדעתם דאי הולד כשר לא אזיל בתר דידי׳ דאל״כ לא מקשו שם מידי אפי׳ רש״י וכן מוכח ג״כ ממה שכתבו שם דצ״ל שלא נשאו ישראלית וכן כתב רש״י בפי׳ ג״כ בקידושין (דף ע׳) דמסתמא לא נשאו בנות ישראל ולכן ל״ק קושית התוספ׳ עליו די״ל באגריפס הי׳ ידוע להם דאמו ישראלית היתה כן כתבתי בחדושי. ועתה ראיתי שגם התוספ׳ נדה (דף נ״ו) כתבו כן דבין אם הולד ממזר או כשר ישראל הוא וגם ראיתי שמה שכתבתי ביישוב הסתירה של רש״י שכן כתב גם בשער המלך ומסיק ג״כ שלרש״י א״צ גירות וכן נראה ג״כ מסתימת כל הפוסקים שכתבו סתמא נו״ע הבא על ב״י הולד כשר ולא כתבו שצריך להתגייר והנך רואה שאין סיוע להר״ר משה משיטת רש״י ותוספ׳ דהולד צריך גירות דאדרבא שיטת רש״י נראה דישראל גמור הוא וגם מהתוספ׳ יש הכרע יותר לשיטת כל שאר הפוסקים דאין צריך גירות.\\\vspace{0pt}

ומעתה נחקור עוד אם יש סיוע לה״ר משה מדברי רש״י פ׳ אמור ממה שכתב שם בתוך ב״י מלמד שנתגייר כמו שכתב הגאון אוה״ב נ״י ולענ״ד דבריו בזה תמוהים דאם אמנם יש סיוע לזה למה לא הזכיר רק דברי רש״י שאין מהם הכרע נגד הפוסקים החולקים והלא דברי רש״י הם מהתורת כהנים אשר מזה כפי דעת הרה״ג נ״י יהי׳ סתירה לכל הפוסקים דס״ל שא״צ גירות דהכי אמרינן בת״כ והוא בן איש מצרי אע״פ שלא היו ממזרים באותה השעה הוא הי׳ כממזר בתוך בני ישראל מלמד שנתגייר עכ״ל הרי שדברי רש״י הם דברי ת״כ ויהי׳ מזה סתירה לשיטת כל הפוסקים ולשיטת רש״י עצמו דס״ל דנו״ע הבא על ב״י הולד כשר וא״צ גירות אבל באמת הך מלמד שנתגייר פירושו כמשכ׳ הרמב״ן שנתגייר במילה וטבילה והרצאת דמים בשעת מ״ת ככל ישראל והוצרך להשמיענו כן מדכתיב והוא בן איש מצרי שלא נטעה לומר שנשאר מצרי ולא נתגייר עם ישראל אבל אם הי׳ נולד לאחר מ״ת לא הי׳ צריך גירות וע״פ פי׳ הרמב״ן יש לפרש ג״כ רישא דת״כ מה שאמר הוא הי׳ כממזר שהקרבן אהרן נדחק בזה אבל י״ל כיון שסתם ספרא ר׳ יהודה ולפי מה שכתבו התוספ׳ יבמות (דף מ״ז) ד״ה נאמן ס״ל לר׳ יהודה דנו״ע הבא על ב״י הולד ממזר ע״ש אכן זה דוקא בבא על בת ישראל דהיינו אחר מ״ת וזהו פי׳ דברי הת״כ אע״פ שלא היו ממזרים באותה שעה פי׳ אפילו אותם שנולדו מנו״ע באותה שעה עדיין לא היו ממזרים כיון שבאו עליהם קודם מ״ת הוא הי׳ כממזר דלזה פרט הכתוב שהי׳ בן איש מצרי דאל״כ אין נפקותא באביו כיון שהעיד הכתוב שנתגייר וע״פ דברינו אילו מוכרח כדברי הרמב״ן דאיך אפשר לפרש שנתגייר משום דאי הולד כשר צריך גירות שהרי סתם ספרא ר׳ יהודה דס״ל הולד ממזר ואז לכ״ע ישראל הוי וא״צ גירות אלא ודאי כדברי הרמב״ן הרי שגם מזה אין סיוע לדעת ה״ר משה לומר שצריך גירות ושאסור למוהלו בשבת אכן גם לו ימצא סיוע לדבריו הרי ה״ר מנחם שהביא שיטתו חלק עליו וגם הפרישה חלק עליו וכתב בפי׳ דמותר למוהלו בשבת וכ״כ ג״כ בתפארת למשה וכן נראה מסתימת כל הפוסקים שכתבו דבן הנכרית אסור למוהלו בשבת משום דאזיל בתר דידה ולא כתבו חדוש יותר דאפילו בן ישראלית הבא מן הנכרי אע״כ דכולהו ס״ל דישראל גמור הוא וא״כ איך נסמוך על דעת יחידית הנדחת מכל צד לבטל מצות מילה בשמיני החמורה שדוחה שבת לדעת כל הפוסקים שישראל גמור הוא וצריך למוהלו בשבת ואני תמה על הגאון הרה״ג אוה״ב נ״י שחשש לדעת ה״ר משה שלא למוהלו בשבת למה לא חשש יותר להצריכו גם טבילה שהרי פסקינן דאין גר עד שימול ויטבול ואם צריך גירות ולא יטבול הרי יטמע בישראל ואפשר לבא עי״ז איסור אשת איש אם יקדש ב״י ויקדשה גם אחר ואפילו איסור זה בעצמו דמילה בשבת אפשר שיגרום כשיגדל וישא ב״י וימולו בנו בשבת אכן לענ״ד א״צ לחוש לכל זה דישראל גמור הוא וצריך למוהלו בשבת כסתימת כל הפוסקים וכמנהג ישראל ואפילו טבילה א״צ מן הדין רק לענין לטבלו אולי יש לחוש לספק התוספ׳. כן נלענ״ד הקטן יעקב.\\\vspace{0pt}

\end{multicols}\newpage

\newchap{סימן כב}
\begin{multicols}{2}
ב״ה אלטאנא, יום ו׳ כ״ז ניסן תרי״ב לפ״ק. להרה״ג וכו׳ מ״ה גבריאל אדלער נ״י הגאב״ד דק״ק אבערדארף יע״א.\\\vspace{0pt}

בדבר שאלת מעכ״ת נ״י בנתגיירה מעוברת וילדה בן בשבת אם דיינינן לי׳ כישראל ומותר למוהלו בשבת או כגר שאין מילתו דוחה שבת ומר נ״י כתב וז״ל דעתי נוטה שאם תלד בשבת שמילת הבן דוחה שבת וכמ״ש בש״ע א״ח סי׳ של״א סעי׳ ה׳ ונכרית שילדה ואח״כ נתגיירה אינו דוחה שבת. משמע כשנתגיירה ואח״כ ילדה דוחה מילתו שבת וכן משמע מהרמב״ם פ״ג מהל׳ אסורי מזבח הל׳ י״ב ולד נוגחת אסורה היא וולדה נגחו ועכמ״ש וכמ״ש בש״ע י״ד סי׳ רס״ח סעי׳ כ״א גוי׳ שנתגיירה כשהיא מעוברת בנה א״צ טבילה וכרבא ביבמות ע״ח ע״א. וה״נ אזלינן בתר לידתו הגם דאשכחן בכמה דוכתי׳ ביבום ובירושה דאזלינן בתר נתעברה ולא בתר לידתו, היינו במקום שתלוי׳ באב, אבל לענין מילה ד״ש דעיקר תלוי באם אם היא נכרית אם ישראלית, שפיר כ׳ הש״ע בא״ח ונכרית שילדה ואח״כ ונתגיירה עכ״ד.\\\vspace{0pt}

תשובה: דין זה תלי בפלוגתא דרבוותא ובתחלה אזכיר שהראי׳ שהבי׳ מעכ״ת נ״י ממה שכתב הש״ע נכרית ואח״כ נתגיירה מילת בנה אינה דוחה שבת דמשמע הא נתגיירה ואח״כ ילדה דוחה לענ״ד לאו הוכחה היא דהש״ע נקט לשון הטור והטור כתב כן מלשון רש״י שבת דף קל״ה כמשכ׳ הב״י דאמרינן שם אמר רב אסי כל שאמו טמאה לידה נימול לשמונה וכל שאין אמו טמאה לידה אין נימול לשמונה וכתב רש״י כל שאין אמו טמאה לידה כגון יוצא דופן ונכרית שילדה ולמחר נתגיירה אין בנה ממתין עד שמונה אלא נימול מיד עכ״ל ושם נקט רש״י ע״כ שילדה ואח״כ נתגיירה דאי נתגיירה ואח״כ ילדה הרי טמאה לידה אבל לעולם יש לומר דאפילו נתגיירה ואח״כ ילדה גם כן אין מילת בנה דוחה שבת מפני שהוא כגר ולקמן יבואר ג״כ עוד טעם אחר למה נקט הטור והש״ע ילדה ואח״כ נתגיירה דהנה גם מה שהוכיח מר נ״י ממה דפסק בש״ע י״ד (סי׳ רס״ח) נכרית שנתגיירה כשהיא מעוברת בנה אין צריך טבילה לכאורה הראי׳ להיפך דאין מילתו דוחה שבת דלפי מסקנת הגמרא ביבמות (דף ע״ח) טעמא דרבא דאמר כן דמעוברת שנתגיירה בנה אין צריך טבילה הוא משום דסלקא לי׳ טבילת אמו לשם גירות ולא הוי חציצה דעובר היינו רביתי׳ וא״כ דגר הוא פשיטא דצריך גם מילה לשם גירות ומילת גר אינה דוחה שבת אלא אפילו לסלקא דעתך דגמרא שם דטעמא דרבא משום דעובר ירך אמו הוא ג״כ יש לצדד דדינו כגר כמו שהיא גיורת ולא נחשב לידתו לידת ישראל שתדחה מילתו שבת וכן מוכח דעת התוספ׳ דדינו כגר דכתבו אמה דאמרינן ביבמות (דף מ״ז) דמילת גר צריכה קודם שיטבול וז״ל משמע שהמילה קודם טבילה וכדאמר אין גר עד שימול ויטבול ותימא דאמר בהערל מעוברת שנתגיירה בנה אין צריך טבילה ואומר ר״י דשאני התם דאכתי לא חזי למילה עכ״ל וכן כתב גם הרמב״ן שם ויעיין שם ברשב״א דמפני קושיא זו נדחק לומר דמה דאמרינן בנה אין צריך טבילה בנה לאו דוקא אלא בתה קאמר משום דסבירא ליה דמילה קודם טבילה מעכב בגר הרי עכ״פ כולהו סבירא להו דמילת העובר שנולד אחר נתגיירה מעוברת היא מילת הגר ופשיטא דאינה דוחה שבת אבל ראיתי בריטב״א שכתב שם וזה לשונו ואם תאמר ולמה משהין אותו למוהלו בתחלה ולהמתין עד שיתרפא יטבילוהו בתחילה ויש לומר כשהוא ערל אין טבילה מועלת דהוי לי׳ כטובל ושרץ בידו דאפילו בדעבד מעכב אבל הרמב״ן ז״ל כתב דבדעבד אם טבל קודם מילה עלתה לו טבילה והביא ראי׳ מדאמרינן לקמן בפרק הערל גיורת מעוברת שטבלה בנה א״צ טבילה ואמרינן לה התם אפילו למ״ד דעובר לאו ירך אמו הוא וכן כתבו קצת רז״ל ומרן הרא״ה ז״ל דחה ראי׳ בו דשאני התם דכשהוא במעי אמו אינו בן מילה והרי הוא כנקיבה דסגי׳ לה בטבילה וכשנולד ומלין אותו אינו אלא כמו שמלין ישראל שהוא ערל דבלאו הכי נמי ישראל הוא הא בכל שצריך מילה בשעת גירותו אם קדמה טבילה למילה חוזר וטובל עכ״ל הריטב״א וכן כתב הנ״י הרי לפי דעת הרא״ה מדס״ל דטבילה קודם מילה אפילו בדעבד לא מהני מילת בן מעוברת שנתגיירה אינה מילת גר אלא מילת ישראל וא״כ גם מילתו דוחה שבת ובש״ע י״ד (סי׳ רס״ח) הביא הרמ״א טבל קודם שמל מועיל דבדעבד הוי טבילה ויש אומרים דלא הוי טבילה ומדהביא דעה ראשונה שהיא דעת הרמב״ן בסתמא ודעת הרא״ה דלא הוי טבילה בשם יש אומרים משמע דפסק כדעת הרמב״ן שהיא ג״כ דעת התוספ׳ והרשב״א שהוכיחו כן דבדיעבד הוי טבילה ממעוברת שנתגיירה בנה אין צריך טבילה ולכן אף שלדעת הרא״ה י״ל דמילתו דוחה שבת כיון שהוא כישראל גמור מכל מקום להדעה הראשונה שהביא הרמ״א בסתמא מילתו היא מילת גר שאינה דוחה שבת ולכן יש לומר עוד טעם למה הטוש״ע כתבו דילדה ואח״כ נתגייר אין מילת הבן דוחה שבת ולא כתבו דאפילו נתגיירה ואח״כ ילדה אין המילה דוחה שבת דלא נחתו בזה במה שתלי בפלוגתת הראשונים כמו שכתבתי אבל מכל מקום כיון שהתוספ׳ והרמב״ן והרשב״א הם רבים נגד דעת הרא״ה וגם הרמ״א נקט דיעות הרבים בסתמא נראה שהלכה כן שאין מילת בן שנולד ממעוברת שנתגיירה דוחה שבת ושוב מצאתי שגם הגאון אוה״ב נ״י בספרו נחלת בנימין פסק כן והוא הנכון כנלענ״ד הקטן יעקב.\\\vspace{0pt}

\end{multicols}\newpage

\newchap{סימן כג}
\begin{multicols}{2}
אלטאנא, יום ג׳ י׳ אלול תר״ו לפ״ק.\\\vspace{0pt}

שאלה: מה ענין מציצה שתקנו חכמים במשנה כשאמרו מוצצין אם דוקא מציצה בפה או גם ביד.\\\vspace{0pt}

תשובה: הנה הרמב״ם הל׳ מילה (פ״ב) כתב ואח״כ מוצץ את המילה עד שיצא הדם ממקומות רחוקים כדי שלא יבא לידי סכנה עכ״ל ומזה נראה דמציצה בפה דוקא דלהוציא דם ממקומות רחוקים צריך כח משיכה ואי אפשר בדחיקה ביד אלא שראיתי בשו״ת דבר שמואל (סי׳ צ״ח) על דבר עשיית המציצה ביין בט״ב וביוה״כ שכתב וכבר חשבתי למנהג קצת המוהלים לעשות תחלת המציצה בזילוף היין שבפיהם הם שני הפכים בלתי הגון להעשות בבת א׳ כי המציצה היא להוציא דם המוכן אז בטבעו לצאת ע״י החבורה ממקומות הרחוקים שעיכובו בפנים יגרום נזק או סכנה ת״ו והיין בטבעו עוצר יציאת הדם אלא הדרך הנכון הוא כמנהג אותם המוהלים שעושים בתחלה המציצה בלי יין להוציא במהירות הדם המוכן לצאת ואח״כ לעכב ולעצור שלא יצא וימשך יותר מכדי הצורך בזליפת היין ובסמים העוצרים וזה אפשר להעשות ביד או במוך מבלי אמצעות הפה והמוהלים פה נהגו זהירות בעצמם בימי התעניות שלא לזלף היין בפה עכ״ל והנה בהשקפה ראשונה הי׳ נראה מדבריו שמתיר לעשות המציצה ביד או במוך ג״כ ולפ״ז יסתור דברי עצמו שכתב דהמציצה היא להוציא הדם ממקומות הרחוקים שזה ודאי אי אפשר ביד אבל באמת אחר העיון נראה ממה שכתב וזה אפשר אדרבא היפך זה דפשיטא לו ג״כ דעיקר המציצה צריך לעשות בפה דוקא רק שזליפת היין שהיא רק לעצור הדם עלי׳ כתב שאפשר להעשות ביד או במוך בלי אמצעות הפה וזה ברור אלא שעדיין יש לחקור באמת מנ״ל להרמב״ם ואחריו להפוסקים כן שמציצה היא להוציא הדם ממקומות הרחוקים ושלזה צריך לעשות בפה דוקא ואין לומר דנפקא להו כן ממה דאמרינן שבת (דף קל״ג) מהו דתימא האי דם מיפקד פקיד קמ״ל חבורי מיחבר ע״ש וס״ל להרמב״ם דחבורי מיחבר לא מקרי רק כשיוצא בכח ממקומות רחוקים דז״א דחבורי מיחבר שייך גם בכל דם שיוצא מעצמו ע״י חבורה שנעשית אפילו אינו מוציאו במשיכה ממקום רחוק כדמוכח ממה דאמרינן בכתובות (דף ה׳) דם מיפקד פקיד או חבורי מחבר ע״ש אבל נ״ל דמהלשון משנה עצמו יצא לו להרמב״ם כן דהנה השרש מץ שהוראתו הוצאה דבר מדבר לא נמצא רק במקומות מעטים בכתוב אבל נמצא על שלש גזרות שרש מוץ מנחי עין ומזה מיץ חלב (משלי ל׳:ל״ג) ושרש מצה מנחי ל׳ ה׳ ומזה ונמצה דמו (ויקרא א׳:ט״ו) שמרי׳ ימצו (תהילים ע״ה:ט׳) ושרש מצץ מהכפולים ומזה למען תמוצו (ישעיהו ס״ו:י״א) והמדקדקים האחרונים חשבו כי שלשתם נרדפים אבל אחר התבוננות ראיתי הבדל רב בהוראותיהם (וזה מפאר לשוננו הקדושה) ששורש מיץ הוראתו הוצאה ע״י דחיקה וסחיטה ומזה מיץ חלב ומיץ אף ושרש מצץ הוראתו הוצאה ע״י כח המשיכה והעד למען תמוצו שנמשך אחר למען תינקו ושבעתם הרי ששו׳ בהוראתו ליניקה וכן פי׳ רש״י שם ושרש מצה הוראותיו גם שתיהן ע״י דחיקה כמו ונמצה דמו וע״י כח המשיכה כמו שתית מצית והנה שרש מצץ לא נמצא רק פעם א׳ בכתוב למען תמוצו הנזכר (ואולי מגזרה זו גם אפס המץ [ישעיהו ט״ז:ד׳] שפירושו יניקה ויהי׳ מץ מן מצץ במשקל צל מן צלל אבל יתכן ג״כ מגזרת מצה במשקל אש מן אשה) וד׳ פעמים נמצא במשנה הכא ופורעין ומוצצין ובפרה (פ״ט) חוץ מן היונה מפני שהיא מוצצת ובטבול יום (פ״ג) למוץ את גרעינתו שנשנה שם ב׳ פעמים ולא אדע יותר מזה במשנה ובגמרא מצינו במקומות מעוטים כמו כריתות (דף כ״א) של בין השינים מוצצו (אבל מייץ בלשון ארמי אינו מזה הגזרה דזה מצינו בב׳ הוראות כמו שרש מצה דכל אומן דלא מייץ הנ״ל וכן בגטין [דף נ״ו] מייתי לי׳ גרוגרות מייץ מייהי הוא לשון מצץ משיכה בפה ובשבת [דף פ״ח] וקא מייץ בהו הוא לשון דחיקה ומיעוך כמו שפי׳ רש״י שם) ובכל המקומות שנמצא שרש מצץ הוראתו רק הוצאה ע״י כח המשיכה ולכן מדנקט המשנה גבי מילה ומוצצין ולא נקט לשון ומצין מגזרת מוץ וגם לא נקט ומוצין מגזרת מצה שהוא השרש שנמצא ביותר במקרא ונקט ומוצצין מגזרת מצץ שלא נמצא רק פעם א׳ או שתים בכתוב מזה דן הרמב״ם דע״כ רצתה המשנה להורות דצריך דוקא כח המשיכה להוציא הדם ולא די בכח הדחיקה והסחיטה וזה ודאי להוציא הדם ממקומות הרחוקים ולזה צריך לעשות המציצה בפה דוקא ולא בדחיקת הידים כי זה מתנגד ללשון המשנה כאשר הראתי.\\\vspace{0pt}

ואחרי ביארנו את זאת נבא לחקור עוד אם יש לחוש לאשר חדשים מקרוב באו מקצת הרופאים ואחריהם מקצת רבנים באמרם שראוי לבטל המציצה שחכמינו לא הבינו הדבר על נכון כי לא בלבד לא יולד סכנה בחדול המציצה אדרבא תגרום נזק לפעמים אם יש בפה המוצץ ארס וליחות מורסא שעי״ז תתהו׳ חבורת המילה מכה טרי׳ ואם יש ארס כזה בהתינוק לפעמים יגרום חלי גם להמוצץ אך אחרי אשר הבינו מתנגדי המציצה כי נגד סכנה כזה יש להשתמר ע״י שיבדקו את המוהל טרם יותן לו רשיון למול ובחדול המציצה בתינוק ארסי לבד הוסיפו עוד כי עכ״פ המציצה היא שלא לצורך אחרי שהחכמים הקדמונים חשבו שהמציצה מעכבת קילוח הדם ע״י שנצמתו שפתי החבורה ונהפוך הוא כי המציצה תרבנו ורק היין גורם עכוב וזה אפשר בלא מציצה ואחד מן המתנגדים הוסיף שהמוצץ בשבת עובר על איסור סקילה כי לא הותרה מפי חכמינו המציצה בשבת אלא מפני הסכנה כדאמר רב פפא (בשבת קל״ג) דהאי אומן דלא מייץ סכנתא היא תדע דמחללינן עלי׳ שבתא ואחרי אשר נודע דליכא סכנתא יש בזה חלול שבת. אכן על כל זה אשיב בראשונה טעו המתנגדים בחשבם שנתקנה המציצה לעכב הדם ואדרבא נתקנ׳ להוציא הדם כמו שביארנו מדברי הרמב״ם וביאורו לשון מציצה ושגם הם ידעו שהיין עוצר הדם כמו שמבואר בשו״ת דבר שמואל הנ״ל. גם מה שנאמר דבמוצץ בשבת יש איסור סקילה הוא טעות דבשלמא הגמרא קאמר שפיר תדע דקא מחללינן עלי׳ שבתא כיון דברור הי׳ להם שהמציצ׳ היא תיקון ולא קלקול שהרי תקנו אותה שחשבוה לצורך שפיר דייק דע״כ סכנה איכא בדלא מייץ דאל״כ אלא שיש תקון בלא סכנה איכא משום חובל בשבת אבל לדעת המתנגדים למציצה באמרם שהיא שלא לצורך ולפעמי׳ מזקת א״כ אפילו לפי דעתם הוי לי׳ המוצץ בשבת מקלקל בחבורה וקיי״ל דפטור כמבואר ברמב״ם הל׳ שבת (פ״א) ואף דכל פטורי שבת אסורים מדרבנן עכ״ז יש נפקותא במה שהראנו שגם לדעת המתנגדים להמציצה אין בה איסור מן התורה רק מדרבנן למי שלבו נוקפו שמסופק אולי הדין עמם שעכ״פ יצא הדבר מכלל ספק איסור דאורייתא וליכא רק ספק איסור דרבנן דלקולא אבל נגד זה מי ישקיט את לב הנוקף בקרבו אולי באמת דברו חכמינו ז״ל נכונה שיש סכנה בחדול המציצה וכי יצא אצל המתנגדים הדבר מכלל ספק לגמרי שמה שהי׳ נהוג עכ״פ מזמן המשנה ואילך זה קרוב לאלפים שנה בכל המדינות ממזרח שמש עד מבואו בכל תפוצות ישראל שהי׳ מיוסד על אדני השוא והטעות ומי יכריע אם לא באחד מני אלף או מני רבבות יתהו׳ סכנה ע״י חדול המציצה וכי יש חילוק בסכנה בין רב למעט בין נפש א׳ לבין אלפי רבבות ומי יכריע עד כמה למעלה יגיע תקון המציצה אם לא תקון נביאים הי׳ ואם לא משה רבינו תקנה שהרי אנו רואין עכ״פ שבימי המשנה היתה המציצה כבר דבר ידוע ומפורסם אצלם ומי יכריע מאיזה טעם נתקנה שיאמר עלי׳ שבביטול הטעם נתבטלה התקנה שהרי במשנה לא הוזכר טעם המציצה כלל וגם רב פפא שהזכיר טעם הסכנה לא אמר שזה לבדו טעם המציצה וגם ענין הסכנה מה היא לא ביאר לנו עד שיאמר עליו שטעה בדבר ושאין סכנה ומלבד כל זה הלא רבות כהנה יש שאין חכמי הטבע מסכימים עם דברי רז״ל וכי עלה או יעלה על דעת המאמין להכחיש ח״ו דבריהם מפאת מניעת הסכמת הטבעיים הלא תראה מה שעבד עובדא רבה תוספאה ביבמות (דף פ׳) באשה שהלך בעלה למד״ה ואשתהי עד תריסר ירחי שתא ואכשרי׳ וכוותי׳ פסקו הפוסקים ראשונים ואחרונים וכבר שאלתי על זה את פי הרופאים והייתי כמצחק בעיניהם באמרם שעכ״פ אי אפשר לאשה שתתעבר יותר מקרוב לעשרה חדשים ואף שכבר הביאו הח״מ והב״ש באהע״ז (סי׳ ד׳) דעת התוספ׳ שחולקת ג״כ על זה מכ״מ גם עם דעת התוספ׳ שפוסקי׳ כשמואל דאינ׳ מתעברת רק רע״א או רע״ב או רע״ג הטבעיים אינם מסכימים וכן מה שנפסק (שם) ברמ״א שאפשר שתלד בן קיימא לה׳ חדשים וב׳ ימים דשפורא גרים הוא לשחוק בעיני הרופאים ובכל אילו וכיוצא בהם אנו הולכים אחר קבלת רבותינו ז״ל ואפילו לדעת הי״א שהביא הרמ״א שם (סי׳ קנ״ו) זה דוקא בשהחוש מכחיש כמש״כ שם אבל לא בשיש ספק עיין שו״ת נ״ב אהע״ז (סי׳ כ״ב וסי׳ ס״ט) ושמע נא מה שכתב על ענין כזה א׳ מגדולי האחרונים אשר הי׳ נודע גם בין העמים לחוקר ואוהב ויודע חכמות הגאון ר׳ יהונתן זצ״ל בספרו בני אהובה (הל׳ אישות פ׳ ט״ו) כאשר הזכיר שם דברי רבותינו ז״ל שאפשר לאשה שתתעבר באמבטי כתב וז״ל ואל תשגיח בדברי רופאים אחרונים שרצו להשיב וכחשו ואמרו שלא יתכן כי נואלו וטפשו בשכלם וחשבו מה שלא השיגו בדמיונם הכוזב שאי אפשר להיות ולא השיגו במציאות הטבע ככלב המלקק מהים וכהנה רבות שחשבו להשיג על הראשונים וכולו כזב ופעולתם יוכיח שכאב קטן אינם יכולים לרפאות על נקלה מה שהראשונים פעלו בכל חלי גדול ומחלה כבדה עד שהוצרכו לגנוז ספרי רפואות כי לא שמו בה׳ כסלם עכ״ל סוף דבר אני אומר כי אין לנו לזוז מדברי חכמינו ז״ל מפני דעות הרופאים אשר חדשים מקרוב באו ובפרט בדבר גדול כזה בהמציצה אשר מצו׳ עלינו מפי המשנה ואולי מפי משה רבינו ע״ה ואשר נהגו בה כל ישראל בכל קצוי ארץ זה אלפים שנה ויותר ולכן אין לחדול למצוץ גם בשבת ובפה דוקא כי זה עיקר המציצה כאשר הראתי. אכן עכ״ז אין להעלים עין גם מהסכנה אשר ע״פ הרופאים החדשים אפשר שתתהו׳ ע״י מציצה ואפילו באחד מני רבבות כאשר נחלה המוצץ או יש מכה טרי׳ בפיו ויש לשמור מאוד ולהפקיד משמרת לבל יקרב למלאכת המילה ובפרט להמציצה איש אשר חולי בו ובזה תסור תלונת המתנגדים מעל המציצה לבל יניעו עוד את לבב אחינו ב״י לסור מאחרי דברי רבותינו ז״ל אשר הם חיינו ובאורם נראה אור. כנלענ״ד הקטן יעקב.\\\vspace{0pt}

\end{multicols}\newpage

\newchap{סימן כד}
\begin{multicols}{2}
הנה מעת נתפרסם פסקי הנ״ל על דבר המציצה יצאו ללחום עמדי מורים בקשת שקר וכזב בחרות אפם בי ודבריהם סובבים על שלש בקשות – א׳ – דוברים עתק בגאו׳ ובוז על דברי רז״ל – ב׳ – שופכים בוז ולצון עלי וישימוני כמטרה לחץ שחוט לשונם הן בהעלים הן בהזכיר שמי – ג׳ – מחזיקים בכל עוז בדעתם הנכזבת על אודות בטול המציצה. ואנכי בהעלותי על רעיוני אם אליהם אשים דברתי אם אחדול, ראיתי כי על בקשתם הראשונה כבר מוזהרים אנחנו מפי רבותינו הקדושים ז״ל לבלתי השיב באמרם דע ומה שתשיב לאפקורוס וכו׳ אבל אפקורוס ישראל כש״כ דפקיר טפי וגם על בקשתם השנית אמרתי טוב אחריש אתאפק כי הלא כזה קרה מאז ששוחרי כזב דרכו קשתם שקר לירות במו אופל לישרי לב וישטנום תחת רדפם טוב ורק נגד בקשתם השלישית אטיף מלתי אבל גם בזה לא עם המכחישים והמלעיגים אתוכח כי אם עם איש ישר הולך ואשר לא דרך בזה בעקבותם הלא הוא הרב מ״ה אלעזר הורוויץ נ״י אב״ד דק״ק וויען יע״א באשר במה שנוגע לענין המציצה בכלל דבריו דבריהם אף כי גם עליו אתפלא מדוע בקנאו את קנאתו בדבר בטול המציצה – אם לאהבת האמת ולא לאהבת הנצוח – עזב לשון ערומים לדבר מרורות נגד המחזיקים בה, הלא דברי חכמים בנחת נשמעים. והנה פתח דברי מלחמתו באמרו כי האינם יודעים לא יבושו לומר שהמציצה היא חלק ממעשה המילה אשר לא יפרד ממנה ולא ידעתי אל מה כוון בזה לגלות אולת האינם יודעים אם דעתו שאומרים שהמציצה היא עיקרית במעשה המילה כמו הפריעה ומי שלא מצץ כאלו לא מל וכמי שלא פרע לא ראיתי ולא שמעתי מי אשר אמר כן רק שמחזיקי המציצה אמרו שבלי ספק כבר נתקנה משנים קדמוניות ואולי נתקנה ממשה רבינו כאשר צוה לישראל על המילה וכן אמרתי גם אני ואם יחשוב זה לאולת, הלא יתמה בהראותי לו שלזקנו גאון ישראל הרב מה׳ פנחס הורוויץ זצ״ל לא נפלאת לומר עוד יותר מזה שכאשר צוה הקב״ה בהר סיני וביום השמיני ימול כבר כוון להמציצה שתהא הכרחית בעסק המילה שכן כתב בספרו הפלאה כתובות (דף ה׳ ע״ב) ואפשר לומר דבאמת המ״ל שם דלר״י אצטריך גבי מילה משום תקון דמפרק כדס״ל לר״י בחלזון וכדאמר בשבת כל אומנא דלא מייץ סכנתא ומחללינן שבת עלוהי א״כ אפילו נחשוב לגוף המילה לקלקול מכ״מ בפירוק הדם הוי תקון ואצטריך קרא למישרי מילה אף דאתי מיני׳ ע״כ חילול שבת במפרק הדם עכ״ל הרי שלא הרחיק לומר שמה שנכתב קרא להתיר מילה בשבת הוא רק משום עשות המציצה שהיא פירוק הדם. עוד השיג הרב דק״ק וויען נ״י על אשר לבם נוקפם לבטל המציצה הכתובה במשנה שהרי כבר בטלנו גם אספלנית וכמון אשר נתקנו מפני הסכנה מפי המשנה?.\\\vspace{0pt}

על זה אשיב הכי בטלנו בזה שלא להשגיח בהסכנה אשר הורונו רבותינו ז״ל הלא מה שפשיטא להם שאספלנית וכמון הם מפני הסכנה עד שלמדו מזה דגם מציצה היא מפני הסכנה כדאמרינן שבת (דף קל״ג) מה אספלנית וכמון כי לא עביד סכנה הוא אף ה״נ וכו׳ ע״כ הוא מפני שצריך להשקיט הדם שלא יצא יותר מן הראוי והכי חדלנו לנשקיט הדם זה בכה וזה בכה והרי אספלנית עצמו הכתוב במשנה אינו דבר פרטי ומסויים אלא פירושו דבר רפואה (כמו שפרשו רש״י תחבושת) כנראה ממה שאמר אביי (שם) אמרה לי אם אספלניתא דכולהון ביבי וכו׳ רבא אמר וכו׳ ע״ש הרי שאספלנית אינו רק שם כללי של מיני רפואה ולכן גם הרמב״ם לא הקפיד בזה להעתיק לשון המשנה שכתב רטיי׳ תחת כמון והוסיף וכיוצא בהן אחר שאין קפידא במה נסיר סכנה של נזילת הדם אבל הכי יש דמיון לזה ביטול המציצה באמרם המבטלים חכמנו יותר מהמשנה אשר הורה לנו שיש סכנה בחדול המציצה וידענו כי אין סכנה? עוד הרבה הרב הנ״ל לדבר מרורות נגד מקיימי המציצה בחששם לדברי רבותינו ז״ל שאמרו שיש סכנה בחדול אותה בהעתיק להם דברי הב״י שכתב בא״ח (סי׳ של״א) בזמן חכמי הגמרא אם לא היו רוחצים את הילד לפני המילה ולאחר המילה וביום שלישי למילה במים חמין הי׳ מסוכן לפיכך נזקקו לכתוב משפטו כשחל להיות בשבת והאידנא לא נהגו ברחיצה כלל עכ״ל ושאל אם בזמן המשנה חשבו הרחיצה לצורך גדול ושיש סכנה בחדול אותה ואח״כ בטלוה בידעם שאין עוד סכנה או שלא הי׳ סכנה גם אז מדוע לא נאמר גם בדבר המציצה כן? אשיב, אם בעיני הרב יפלא זה גם בעיני יפלא איך לא ראה דברי הדרכי משה שכבר הקשה זה על הב״י וכתב ואיני מבין דבריו ומה בין זמנינו לזמנם בדבר סכנה עכ״ל ולא עוד אלא שאם כדברי מתנגדי המציצה אין תימא כי יאמרו חכמנו יותר מחכמי המשנה אך לא כן שפטו רבותינו ז״ל אשר מימיהם אנחנו שותים אכן באמת בלא״ה מן בטול רחיצה אין ראי׳ טל המציצה שמסוגיא דשבת (דף קל״ד) ממה דפליג ראב״ע עם ת״ק אם יש צורך להרחיץ גם ביום שלישי או לא וממה דפליגי אמוראים אם מרחיצין או מזלפין ואם מותר הרחצת כל גופו או הרחצת מילה לבד נראה שבענין הרחיצה לא הי׳ דבר פשוט ומקובל אצלם אלא שדנו בה על פי שהי׳ נראה להם צורך לפי שינוי הטבעים ולכן יפה דן הבית יוסף שאחרי שידוע אצלנו שאין סכנה בביטול הרחיצה אסור לחלל שבת עלי׳ שהרי גם בזמן המשנה והגמרא לא היתה הרחיצה רפואה המקוימת וברורה בכל אופני׳ אבל איך נדון כזה על המציצה שאין פלוגתא בצורך שלה ואין ספק באופן עשייתה במשנה וגמרא ובפוסקים. ואחרי הגענו עד כה אעיר עוד איך נתן הרב הנ״ל מקום לטעות בתקוע יתדו במקום נאמן והודיע לכל שהגאון הרב מה׳ משה סופר זצ״ל הסכים עמו על ביטול המציצה והרכיב בזה שני דיעות הרחוקות זו מזו מאוד שהתנגדות להמציצה סובב על שתי בחינות או שמה שכתוב במשנה ומוצצין בידינו לבטל כמו שנתבטל אספלנית ורחיצה אחרי שנודע שאין סכנה בביטול המציצה או שהמשנה לא זזה ממקומה אבל אין צריך למצוץ בפה דוקא אלא תעשה המציצה גם ע״י המצאה אחרת והנה הרב הנ״ל אחרי שכתב ככל המבואר למעלה שסובב על הבחינה הראשונה אשר הראנו ביטולה ואחרי הודיע שרופאים מומחים כבר גזרו שאין צורך להמציצה הזכיר שגם הגאון מ״ה מ״ס זצ״ל הסכים עם זה. ואולם דברי הגאון זצ״ל לא ראיתי אבל ממה שהעתיק הרב דק״ק וויען נ״י עצמו בשמו מבואר התנגדות למה שכתב בזה שהגאון זצ״ל לא הסכים עם דיעה הנכזבת שע״פ דברי הרופאים יש לנו לבטל משנה שלמה ושיש לבטל מציצה כמו אספלנית ורחיצה רק שהלך אחרי בחינה השנית שלפי דעתו ומוצצין במשנה לא צריך שיהי׳ בפה דוקא אלא מותר להעשות גם ע״י הוצאה אחרת שכן העתיק בשמו – על תשובתי בענין מציצה הוספתי לומר אע״פ שאני מתיר מציצה ע״י המצאה אחרת שלא בפיו מכל מקום מותר למצוץ בפיו בשבת כי גם ע״י הספוג נמי מחלל שבת כמ״ש מג״א סי׳ שכ״ח ס״ק כ״ג ויעיין א״ע שם – הרי שרק בפירוש לשון מציצה נתחדש להגאון זצ״ל שאינו בפה דוקא אבל לא שיש בידינו לבטלה לגמרי ואם שאין משיבין הארי וגו׳ אם הי׳ בחיים עודנה הייתי דן לפניו בקרקע ואשאלה ממנו להראות לי במקרא במשנה ובגמרא אפילו פעם א׳ לשון מציצה שאין פירושו המצאה בכח היניקה והמשיכה כאשר כבר בארתי בתשובתי הראשונה ואחר שאין ספק בזה דמוצצין הוא לשון משיכה בכח ממילא אין ספק ג״כ דאין בידינו לבטל המציצה בפה וגם תמהתי על דברי הגאון רמ״ס זצ״ל הנ״ל שממה שכתב שמותר למצוץ בפיו בשבת משמע דעל ידי הספוג פשיטא דמותר ולענ״ד צ״ע כיון דמציצה שהיא משום סכנה אפשר לעשות ע״י פה איך יותר סחיטת הספוג הרי בשבת (דף קל״ד) מייתי ברייתא דאין נותנין חמין ושמן על גבי מוך שע״ג מכה בשבת ומפרש בגמרא משום סחיטה ואף שלפי מה שכתב המג״א (סי׳ שכ״ח ס״ק כ״ד) לא שייך חשש סחיטה בזה רק במים אבל לא בשמן לבד ע״ש וא״כ לא שייך ג״כ בספוג שנבלע מיין ודם מכ״מ כבר כתב הב״ח ועמו הסכימו הת״ש והפ״מ דגם בשמן לבד שייך איסור סחיטה והביא ראי׳ מתוספ׳ שבת (דף קי״א) ע״ש וכיון דאם דוחק ע״י ספוג נסחטי׳ היין והדם הנבלעים בו אף שאינו מכוון לסחיטה מכ״מ הוי פסיק רישא כיון שמהדק בכח וכמשכ׳ התוספ׳ שם ויש בזה חלול שבת שלא לצורך סכנה וזה אפילו לדעת הרב זצ״ל שיכול לצאת ידי הסכנה ע״י דחיקה בלא משיכה אבל לפי מה שבארתי דלשון מציצה הוא בפה ובזה דוקא יוצא מידי סכנה אבל בדחיקה לבד לא א״כ יש חלול שבת אפי׳ בלא ספוג אם עושה המציצה בדחיקה לבד שלא בפיו כיון שמוציא דם בכוונה ה״ל חובל ומפרק דרך תקון ומשום סכנה לא הותר כיון דבזה אינו יוצא מידי סכנה ע״פ דברי המשנה שצות׳ מציצה דוקא שהיא הוצאה בכח המשיכה. ולכן אליכם אישים אקרא אשר יראת ד׳ על פניכם התבוננו נא והשימו על לב עד אנה תגיע התחדשות אשר חדשתם בעזבכם את מנהג ישראל אשר מימי קדם לא בלבד שאתם מביאים את נערי בני ישראל לידי סכנה בעברכם על ציווי המשנה אלא תבואו ותביאו ג״כ לידי חלול שבת אם טעית׳ בפירוש מציצ׳ והאמת אתנו שהיא במשיכה דוקא ולא בדחיקה ואם תאמרו אחר עצת הרופאים והטבעיים נלך שכל ענין המציצה להוציא דם הוא שלא לצורך וא״כ אין בזה משום חלול שבת דאין זה תקון אלא קלקול, עוד הפעם אקרא אליכם התבוננו נא עד אנה תגיעו אם תכריעו דיעות הטבעיים על מה שמקובל בידינו מפי חכמי התלמוד; שאלו נא את פיהם אם יאמרו שבהמה שחסר או יתר אבר היא טרפה ואינה יכולה לחיות וראו נא מה שכתב על זה הרשב״א בשו״ת (סימן צ״ח) וז״ל ואם יתחזק בטעותו ויאמר לא כי אהבתי דברים זרים והם אשר ראו עיניהם ואחריהם אלך נאמר אליו להוציא לעז על דברי חכמים אי אפשר ויבטל המעיד ואלף כיוצא בו ואל תבטל נקודה אחת ממה שהסכימו בו חכמי ישראל הקדושים הנביאים ובני הנביאים ודברים שנאמרו למשה מסיני עכ״ל ואם תאמרו לא במה שמקובל מסיני נכריע דברי הטבעיים על דברי חכמינו ז״ל רק במה שנתיסד מפיהם על פי הטבע כפי אשר הכירוה בעת ההיא ולא נאמין שהמציצה מקובלת מסיני ולכן נבטלה ע״פ דברי הטבעיים בעת הזאת, עוד אקרא אליכם, שאלו נא את פי הרופאים והטבעיים אם יסכימו עם החכמים שכל הבהמות שאנו מטריפין משום סרכות הריאה אינם יכולים לחיות והרי גם טרפות האלה נתיסדו מפי חכמינו ז״ל ע״פ הטבע אשר הכירוה שמסיני לא נתקבל רק נקיבת הריאה בכלל וכהנה עוד רבות מלבד מה שכתבתי כבר במכתבי הראשון מנשתהה י״ב חדש ושיפורא גרים ובכזה ח״ו תהפך הקערה על פיה ויבטל רוב התורה כפי התנהגותה בכל תפוצות ישראל ואם תאמרו הלא לא לחנם בטלנו המציצה רק באשר שראינו שנתהו׳ סכנה על ידה בשיש חולי בפה המוצץ וכי אפשר לבדוק בכל פעם את פיו כשמוצץ על זה אשיב הלא החזקה היא אחת מהיסודות אשר כל התורה נשענת עליהן וסוקלין ושורפין על החזקות ולמה נדאג שמי שהוא בחזקת בריא וכשרות ע״י הבדיקה הראשונה שנתרע אח״כ ובפרט בדבר שאם נתרע אי אפשר להמוחזק שלא ידע בריעותא שלו וכי בשופטני עסקינן שיסכנו נפשות על חנם ובדבר שעבידא לאגלויי דבלא״ה לא משקרי בה אינשי. ולכן גם אנכי אסיים בדברי הרב דק״ק וויען נ״י קבלו האמת ממי שאמרו. כנלענ״ד הקטן יעקב.\\\vspace{0pt}

\end{multicols}\newpage

\newchap{סימן כה}
\begin{multicols}{2}
אלטאנא, יום ו׳ י״ד מרחשון תרכ״ב. לבני חביבי מ״ה בן ציון נ״י בק״ק מאהילעוו יע״א.\\\vspace{0pt}

מה שהקשת בסתירת דברי הרא״ש דבשבת פ׳ ר״א דמילה כתב לפי׳ הלכות גדולות דליחם אגב אמי׳ איירי ע״י נכרי אחר ג׳ ימים אחר המילה ובב״ק פ׳ מרובה כ׳ דאיירי ע״י ישראל ובתוך ז׳ פעמים מעת לעת להלידה, אם כוונתך באשר שכתב אחר ג׳ ימים דמשמע שמספר הימים גורם ולא בעינן לענין פקוח נפש מעל״ע כמו שכתב בב״ק, לא ידעתי סתירה בזה דאם מה שנאמר בגמרא שבת (דף קכ״ט) לענין יולדת עד ג׳ ימים בין אמרה צריכה וכו׳ נפרש דג׳ ימים היינו ג׳ פעמים מעל״ע, גם מה שכתב הרא״ש אחר ג׳ ימים של מילה נפרש כן, ואם כוונת להקשות מה דלא כתב הרא״ש בשבת כמו בב״ק לפרש שיטת ה״ג דאיירי ע״י ישראל ובתוך ז׳ מעל״ע של לידה ומזה יהי׳ מוכח דהרא״ש נסתפק בזה אם חשבינן לענין פקוח נפש הימים מעת לעת, גם בזה אין סתירה דהרא״ש בעצמו גילה דעתו בזה במה שכתב בשו״ת כלל כ״ו סי׳ ג׳ דבתחלה הביא שם הפי׳ שכתב בב״ק דאיירי תוך ז׳ מעת לעת והשיב על זה מגמרא דמנחות דאסור בזה להחם ע״י ישראל ואח״כ מסיק דאיירי לאחר ג׳ ימים של מילה שכבר נתרפא הילד. והנה מדבריו בשו״ת זו הי׳ אפשר לדון דהרא״ש עצמו לא החליט דחשבינן מעל״ע ונסתפק בזה שלא כתב שם אלא בלשון אם תמצא לומר ע״ש ובלשון זה יש לספק אם ספקו רק על האוקימתא שנוקי בתוך מעל״ע או על עיקר הדין אם חשבינן מעל״ע אבל צדקת בדבריך דממה שכתב בפ׳ ר״א דמילה על יום הברותו כיום הולדו דכיון דספק נפשות הוא יהבינן לי׳ מעל״ע, ובשו״ת הביא ראי׳ מזה להך דיולדת מזה נראה דפשיטא לי׳ הך דחשבינן לענין פקוח נפש הימים מעל״ע דאל״כ מנ״ל דזה הוא הטעם דחשבינן לענין קטן חולה שנתרפא מעל״ע דדלמא הלכה למשה מסיני הוא דנחשוב בזה השיעור דז׳ פעמים מעל״ע כמו שאר השיעורים דהל״מ הם ולא ידענו טעם להם דא״ל דס״ל דזה השיעור אינו מן התורה אלא שחכמים הציבו שיעור זה כיון דבלא״ה כבר עבר יום השמיני ולזה נתן הטעם למה אמרו חכמים כן דז״א דביבמות (דף ע״א) מוכח דשיעור זה הוא מן התורה דלחד תירוץ שם מוקי מה דצריך קרא דמילת זכריו מעכב גם בשעת אכילה בענין זה דנתרפא וכלו ז׳ פעמים מעל״ע בין שחיטת לאכילת הפסח הרי דמוכח דפשיטא להגמרא דשיעור זה הוא מן התורה דאף שיש שם גם תירוצים אחרים הרי לא חלקו על דין זה אלא שהם אמרו דמשכחת ג״פ באופנים אחרים ומדלא חלקו בפי׳ ילמוד סתום מן המפורש.\\\vspace{0pt}

והנה כתבת לבאר לך מה כוונת רש״י במה שכתב בשבת (דף קל״ז) ד״ה מאי לאו ומהא ליכא למשמע מידי עכ״ל כוונתו נראה פשוט דהרי אבעי׳ לי׳ אם בעינן מעל״ע ורצה לפשוט ממה דקאמר כיום הולדו דמשמע יום הולדו ממש והשיב על זה די״ל דעדיף מיום הולדו, וא״כ היוצא מזה דהאבעי׳ לא נפשטה דודאי גם לאידך גיסא לא נוכיח דבעינן מעל״ע דמי כתיב בפי׳ בברייתא דעדיף מיום הולדו וזה מה שכתב רש״י דמהא ליכא למשמע מידי לא להקל ולא להחמיר. וזה ג״כ כוונת הר״ן שכתב על מה שכתב הרי״ף ואפשטא דבעינן מעת לעת דלא מהסוגיא דשבת אפשטא אלא מסוגיא דיבמות וכוונתו כמשכ׳ לעיל דשם מוכח דפשיטא להגמרא דמדין הל״מ הוא דחשבינן הז׳ ימים מעל״ע. ובאמת זה מחליש הראי׳ מהרא״ש הנ״ל די״ל דגם הרא״ש לא כוון במה שכתב דכיון דספק נפשות הוא בעינן מעל״ע אלא לומר דאע״ג דהאבעי׳ לא נפשטה מכ״מ כיון דס״נ הוא אזלינן לקולא ובעינן מעל״ע ועיין משכ׳ בזה בספרי ע״ל ביבמות אבל מכ״מ יותר נראה כמו שכתבתי לעיל דאל״כ יקשה למה לא כ׳ הרא״ש כמו הרי״ף דמיבמות נפשטה אלא ע״כ כוונתו ליתן טעם דלכך בעינן מעל״ע כיון דס״נ הוא ולדון מזה גם על שאר ימים דנוגעים לס״נ. ולכן כיון דעכ״פ הראש לחד שיטה כ׳ בפי׳ דחשבינן הז׳ ימים דיולדת מעל״ע ואין לנו ראי׳ שחזר מזה וגם מתוספ׳ גטין (דף ח׳) שחשבו פי׳ זה דאגב אמי׳ הי׳ תוך ז׳ מעל״ע לדוחק אין ראי׳ דפליגי על הדין אלא שי״ל שהוקשה להם קושית הרא״ש בשו״ת הנ״ל וכן משמע מהריטב״א בעירובין שאזכיר לקמן שכתב בפשיטות בשם התוספ׳ שחושבין ז׳ ימים מעל״ע וא״כ ה״ה ג׳ ימים לכן עכ״פ לא יצא פסק התרומת הדשן (סי׳ קמ״ח) הביאו המג״א סי׳ ש״ל שלא חשבינן מעל״ע מכלל ספק ושפיר עבדי גדולי האחרונים שבזמננו שלא חששו לו דס״נ להקל.\\\vspace{0pt}

אמנם דע בני נ״י שמצאתי ראי׳ שדעת הר״ן ג״כ כדעת תרומת הדשן דלא חשבינן הג׳ ימים מעת לעת דעל מה דאמרינן נדה (דף ל״ח) אהא דאמר שמואל אין אשה מתעברת ויולדת אלא לרע״א או לרע״ב או לרע״ג יום דהוא דאמר כחסידים הראשונים דתניא חסידים הראשונים לא היו משמשין מטותיהן אלא מרביעי בשבת ואילך שלא יבואו נשותיהן לידי חילול שבת פי׳ רש״י אבל ליל שלישי וליל שני וליל מוצאי שבת לא דאי מתעברה למ״ש שמא תלד לרע״ג דמטי בשבת ונמצא שבת מתחלל שניתן רשות לחלל מפני סכנת הגוף ואם נתעברה ליל שני דלמא תלד לרע״ב יום דמטי בשבת ואם תתעבר ליל שלישי דלמא תלד לרע״א יום דמטי בשבת אבל הר״ן כתב כדי שלא יבואו נשותיהן לידי חילול שבת פי׳ כדי שלא יבא שבת תוך שלשה ללידתן ומשום דיולדת כל ג׳ ימים הראשונים מחללין עלי׳ את השבת ואע״ג דהיכא דלא קלט הזרע לאלתר לא אתי שבת תוך ג׳ כיון דברוב הזרע נקלט לא חיישי להכי עכ״ל הרי שדעת הר״ן כיון שלרוב הזרע נקלט לאלתר לא חיישינן רק לג׳ ימים שמיום התשמיש אחר רע״א יום שלא יבואו בשבת אכן קשה דאכתי איך שמשו בשבת דאם תתעבר תלד ביום רביעי וג׳ פעמים מעת לעת לא ישלמו עד ביום השבת ואכתי איכא מקצת חילול שבת וכל שכן אם תלד בערב של יום הרביעי שאיכא היתר חילול שבת עד סמוך למוצאי שבת ואפילו ליכא רק מקצת יום השבת תוך מעל״ע הרי ע״כ למקצת ג״כ חששו שהרי לפירוש רש״י שחששו שמא תלד בשבת ג״כ ליכא אלא מקצת שבת שיש לחוש בו לחילול ולא כל השבת וא״כ ה״ה לפי׳ הר״ן ניחוש לזה אע״כ דדעת הר״ן דלא חשבינן הג׳ ימים מעת לעת ולכן אם תלד ביום הרביעי שלמו הג׳ ימים בכניסת שבת. ואע״פ שלענ״ד ראי׳ זו נכונה שדעת הר״ן להחמיר מכ״מ נגד זה ראיתי בריטב״א עירובין (דף ס״ח) שכתב בפשיטות בשם התוספ׳ דז׳ ימים ליולדת חשבינן מעל״ע כאשר הזכרתי לעיל ולכן המיקל בספק נפשות לא הפסיד. כנלענ״ד הקטן יעקב.\\\vspace{0pt}

\end{multicols}\newpage

\newchap{סימן כו}
\begin{multicols}{2}
ב״ה אלטאנא, סיון תר״ט. הנה זה מקרוב באה לידי שו״ת מהגאון החסיד שבכהונה מ״ה נתן אדלער זצ״ל מק״ק פראנקפורט יע״א שהועתקה ע״י אחד מתלמידיו מכ״י בענין מה שקורין שלאגבוים לצורת הפתח בשבת וז״ל.\\\vspace{0pt}

שאלה – אם יש לסמוך למחיצת שבת על מה שקורין בל״א שלאגבוים שאם פתוח אין שום מחיצה גם יש שאין גבוהים יוד. –\\\vspace{0pt}

תשובה – הדין פשוט כביעותא בכותחא היכא דליתנהו עתה מחיצות אף שראוי׳ לעשות בקל אינם מועילים והגם דאי׳ סי׳ שס״ד בא״ח די״א דסגי בדלתות ראוי׳ לנעול זה מיירי שעושים כדרך דלתות שיש להם גיפופים דומי׳ דדלתות ירושלים דאי׳ בעירובי׳ (דף ו׳) דודאי הי׳ להם גיפופי׳ ומשקוף כדאי׳ בתוספ׳ (דף כ״ב) אבל בנדון דידן אין שום צד היתר, וכן פסק אדמ״ו הגאב״ד כאן נר״ו הלכה למעשה בבוקנהיים ואסר להם לטלטל עד שהשתדלו אצל השררה לעשות צה״פ כהגון ואם עושים צה״פ ע״י קנה מכאן ומכאן וחוט טרוט על גביהן צריך לתקוע יתידות בקנים על גביהם באמצע ולכרוך החוט סביב היתד דאם כורך בקנה עצמו הו׳ צה״פ שעשאה מן הצד ולא מהני כדאי׳ סי׳ שס״ב סי״א ואם אין גבו׳ יוד הגם דבש״ע סי׳ שס״ג סכ״ו משמע דבצה״פ א״צ שיהי׳ המבוי גבו׳ יוד זו דעת הרמב״ם וכבר הסכים המ״מ לדעת הרשב״א דלא מהני כלל אם אין גבו׳ יוד וכ״נ הסכמת הפוסקים ונראה דלפני הב״י היתה גי׳ מ״מ בשיבוש וכמו שהעתיק בב״י סי׳ שס״ג וכבר הגהתי שם בב״י כנוסח שלפנינו במ״מ ועיקר.\\\vspace{0pt}

אחרי ראיתי השו״ת הנ״ל שהשיב הגאון מ״ה נתן אדלער זצ״ל לאחד מתלמידיו נזכרתי בפלפול שהי׳ לי בנדון כיוצא בזה עם אדמ״ו הרב הגאון הגדול מ״ה אברהם בינג זצ״ל אב״ד דק״ק ווירצבורג בימי בחרותי בשנת תקפ״א. והנני מעתיקו בכאן, רק אזכיר שכפי הנראה מהתשובה הנ״ל שלאגבוים אשר עליו השיב הגאון ר״נ אדלער זצ״ל הי׳ עשוי באופן שכנפתח היה פתוח לגמרי ולא הי׳ כאן צה״פ כלל ועל זה השיב דלא מהני אבל אני דנתי על שלאגבוים כמו שנעשים עתה דהיינו שיש שני עמודים בזקיפה ועל גבן מונח קורה לרחבו וכשנפתח אז הקורה עולה למעלה ועומד בזקיפה ובשפוע לצד העמוד האחד וממנו יורד חבל בשפוע עד העמוד השני שכנגדו באופן שכשנפתח יש צורת הפתח כעין צריף השנוי בסוכה דף י״ט, וכזה הי׳ נלענ״ד דמהני לצורת הפתח כאשר אבאר.\\\vspace{0pt}

הנה אם העמודים אשר עליהם נשען השלאגבוים הם גבוהים עשרה טפחים א״כ השתי מזוזות של הפתח עומדין נזקפין כהוגן, אלא שהמשקוף לא שוכב בשוה לרוחב המזוזות, רק מתרחב למטה והולך ומתקצר למעלה כעין צריף לכאורה יש ללמוד שכעין זה מקרי צורת הפתח ממקרא מלא במלכים א׳ – וי״ו – ל״א – ואת פתח הדביר וגו׳ ולהפי׳ הראשון שהביא רש״י ז״ל שם בלשון כך שמעתי צורת פתח הדביר הי׳ כעין צריף ולמה לא נילף דכעין זה קרוי פתח מדקרא הכתוב כן, וכמו דילפינן בבבא בתרא (דף צ״ח ע״ב) דצריך לעשות בית רומה כחצי ארכו וכחצי רחבו מבנין היכל אלמא מה דנקרא בנין ובית לענין היכל נקרא ג״כ לענין אחר וא״כ ה״ה לענין פתח, אך דמזה אין ראי׳ כל כך כיון דלפי פירושים אחרים בפסוק זה לא קאי חמישית אפתח רק אמזוזות אבל מכ״מ על פי עיקר הדין נלענ״ד דאם העמודים אשר עליהם נשען השלאגבוים מפה ומפה הם גבוהין יוד נקרא צורת הפתח דהוא ממש הדין דכיפה שנזכר בש״ע א״ח סי׳ שס״ב סי״ב, דהטעם דכיפה הוי צורת הפתח משום דרואין העיגול כאילו הוא סתום והוי צורת הפתח מרובע כמו שפי׳ רש״י בעירובין (דף י״א ע״ב) בד״ה חייבת ע״ש ומ״ש בזה אם המשקוף הוא עגול או כעין צריף. אך בזה יש לדון אם העמודים הנזקפים אינם גבוהים רק ג׳ טפחים או אפילו פחות מג׳ טפחים ומרוחקים זה מזה יותר מד׳ טפחים באופן שהפתח יהי׳ יותר רחב מד׳ אם זה מהני לענין צורת הפתח, דהנה לפמשכ׳ הב״י בהל׳ מזוזה סוף סימן רפ״ז תלוי זה בפלוגתא שבין הרמב״ם ורש״י ואחריו נמשכו גם הש״ך וט״ז שם, עד שהט״ז מסופק כיון דלא איפסק הלכתא כחד מינייהו איך לעשות המזוזה ואני בעניי לא זכיתי להבין דעת הגדולים בזה דאיך אפשר לומר דהרמב״ם והרשב״א סוברים דגם ברחבה ד׳ צריך שיהי׳ ברגלי׳ בלא עיגול עשרה הא בעירובין שם דף י״א אמרינן כי פליגי ביש ברגלי׳ ג׳ וגבו׳ יו״ד ואין רחבה ד׳ וכו׳ ומשמע הא יש ברחבה ד׳, אף דאין חוקקין להשלים מכ״מ הוי פתח הרי דמוכח בפי׳ כשיטת רש״י וטור והיאך אפשר שהרמב״ם יפלוג על זה, ולכן נלענ״ד שגם הרמב״ם והרשב״א שנמשך אחריו סוברים כשיטת רש״י ומה שהביא הרמב״ם בכל מקום רק התנאי של גובה הרגלים אשר ממנו הוציא הב״י שהרמב״ם יפלוג על רש״י לענ״ד אין ראי׳ מזה דהנה הוקשה לי על אשר לא ראיתי שהביא הרמב״ם בפ״ו מהל׳ מזוזה ג״כ התנאי שיהי׳ רוחב הפתח ד׳ ואחרי כך ראיתי שגם הרב המגיד פ׳ ט״ז מהל׳ שבת הלכה כ׳ עמד בזה וכתב דעד כאן לא הזכירו רוחב ד׳ רק משום ר״מ אבל בשיש ברגלים יוד ודאי אינה צריכה רחב ד׳ ע״ש דכתב שם מדעתו כן אבל נראה דהרמב״ם הוציאו כן בראי׳ ברורה מהגמרא דעירובין הנ״ל כיון דהגמרא מוקי פלוגתא דר״מ ורבנן באין רחבה ד׳ א״כ איך איירי הך דקתני ושוין שאם יש ברגלי׳ י׳ בסיפא דברייתא שם, אי איירי ברחבה ד׳ ל״ל יש ברגלי׳ עשרה הא גם אם יש ברגלי׳ רק ג׳ כ״ע מודו כיון דרחבה ד׳, אע״כ דגם האי ושוין איירי כמו רישא דברייתא באין רחבה ד׳, א״כ הרי מוכח שיטת הרמב״ם דביש ברגלים י״ט לא בעינן רחב ד׳ ואף שה״ה הביא שהרשב״א פליג על הרמב״ם, הנה דברי הרשב״א לא ראיתי אבל תמה אני איך ישיב הרשב״א על הראי׳ שהבאתי שלענ״ד היא מוכרחת וכיון דלפי היוצא מזה הך דין דושוין איירי ג״כ באינה רחב ד׳ א״כ לא מוכח מדברי הרמב״ם מידי מה שהוכיח הב״י דהרמב״ם נקט לשון הברייתא כדרכו בכל מקום, ואיירי ג״כ כמו הברייתא באין רחבה ד׳ ולכן לא חילק רק בין יש ברגלי׳ עשרה לאין ברגלי׳ עשרה אבל ודאי דגם הוא סובר דברחבה ד׳ לא בעינן גובה הרגלים עשרה כמו שמוכח מהגמרא ומפני שלא הוזכר דין זה בפי׳ לכן לא הביאו הרמב״ם ולפ״ז רש״י ורמב״ם וטור ורשב״א שיטה אחת להם בדבר זה וא״כ יחשב השלאגבוים לצורת הפתח גם אם אינם גבוהים העמודים אשר נשען עליהם עשרה טפחים. כנלענ״ד הקטן יעקב.\\\vspace{0pt}

\end{multicols}\newpage

\newchap{סימן כז}
\begin{multicols}{2}
וכאשר הצעתי פסק זה לפני אדמ״ו הרב הגאון מ״ה אברהם בינג זצ״ל אב״ד דק״ק ווירצבורג יע״א השיב לי כזה.\\\vspace{0pt}

הנה שאלתך ע״ד מה שקורין בל״א שלאגבוים אי מהני במקום צה״פ לענין טלטול שבת התעוררת שני דברי׳ א׳ מצד המשקוף שהוא משופע כעין צריף ובזה הכרעת דמהני והבאת תחלה ראי׳ מפסוק מלכי׳ א׳ וי״ו וסתרת אותו דאין פי׳ ראשון דרש״י ברור כי יש שם פרושי׳ אחרי׳ ולכן אין לסמוך ע״ז הפי׳ לדינא עכת״ד אמנם אף להך פי׳ אין לך ראי׳ מכרחת כי י״ל ביש דלתות עדיף טפי ונקרא על ידיהם פתח אף במשקוף משופע וכיוצא בזה כתבו תוספ׳ ערובין ב׳ ע״ב ד״ה אלא מעתה דאף אי ילפי׳ מפתח היכל מ״מ פשוט הוא דדלתות (דלא עבדי לצניעות) מהני אפילו לגבו׳ מכ׳ כיון דננעלות ע״כ ופי׳ דבריהם לפי עומק הפשט כיון דכשננעלו ה״ל מחיצה גמורה ועשוין לפתוח ולנעול ה״ל פתח על ידיהם אף בשעה שפתוחין וא״כ ה״נ י״ל לענין דביר דשם לא עבידי לצניעות ובסמוך נדבר עוד מראיתך הנ״ל.\\\vspace{0pt}

שוב הכרעת מסוגי׳ דכיפה דברור דמהני ברחב ד׳ כשגבו׳ יו״ד ובזו יש לדבר כי לדעתי מסוגי׳ הנ״ל מוכח בהדיא אפכא דהא רב ששת הוכיח שם מהך ברייתא דא״צ ליגע וההוכח׳ הנ״ל היא מרבנן דכיון דסברו אין חוקקין א״כ המשקוף הנוגע ברגלי׳ הוא משופע ואינו משקוף רק השיווי שהוא בשמי הכיפה הוא הקורה הנחשב למשקוף וממילא מוכח שא״צ ליגע וכ״כ הריטב״א בהדי׳ וזה אינו שייך אלא בצורה ראשונה שבסוגי׳ דכיפה בעירובין שם דהכיפה היא שוה מבחוץ מצדדיו ומלמעלה א״כ ה״ל משקוף שוה למעלה אבל צורה ב׳ שבסוגי׳ הנ״ל אין כאן משקוף כלל וכן שלאגבוים אינו חייב במזוזה ואינו צה״פ ואפי׳ לר״מ דסבר חוקקין דא״א לחוק דאם נחוק נעשה אויר וממילא דאינו מתיר וכ״ז מבואר בריטב״א בסוגי׳ הנ״ל כשתעיין בו. ובאמת אגיד דדברי טור יו״ד סי׳ רפ״ז נשארו אצלי בצ״ע דהטור כשהביא דין דכיפה נקט בלשונו דהאסקופ׳ עליונה עשוי׳ כקשת והיא הצור׳ ב׳ הנ״ל ופסק דחייב׳ בגבו׳ יו״ד ולהנ״ל בעשוי׳ כקשת פטור ממזוזה לכ״ע והרא״ש בהל׳ מזוזה לא כ׳ רק כלשון רי״ף דכיפה חייבת בגבו׳ יו״ד ורי״ף בעירובין הביא הגמ׳ דהכא דמוכח מזה ג״כ דצורה ב׳ הנ״ל פטור.\\\vspace{0pt}

והנה לשון רמב״ם שהביא לשונו בש״ע שם מורה ג״כ כהטור שכ׳ מפני שאין לו משקוף באינו גבו׳ יו״ד פסול ומשמע מדיוקו דאם גבו׳ יו״ד חייב אף בעשוי׳ כקשת אבל הלא דבריו בעצמותן תמוהין וכמו שתמה הכ״מ והניח דבריו בצ״ע ואף שהבאר הגולה רוצה ליישב אין ישובו מספיק והט״ז ג״כ רוצה ליישב דבריו ואף שיש להטעים דבריו כמו שאכתוב בסמוך מ״מ לא העלה ארוכה לדברי רמב״ם דהנה מדברי רמב״ם יוצאים ב׳ דיוקים דאם הי׳ לו משקוף שוה וכמו צורה א׳ הנ״ל דהמשקוף שוה למעלה הי׳ חייב לרמב״ם אף זו שאינם גבוהי׳ יו״ד וז״א דהא הרמב״ם מצריך רגלים שוים גבוהי׳ יו״ד ותו אם היו הרגלי׳ גבוהי׳ יו״ד בשוה אף אם המשקוף עקום הי׳ חייב וזה הי׳ כדעת הטור הנ״ל אבל הלא הוא נגד פשטות הגמ׳ וכנ״ל ולב׳ קושיות הנ״ל לא הועיל תי׳ הט״ז מידי ואין דרך הטור לסמוך ארמב״ם ולהביא דבריו בסתם לפסק הלכה אם הוא נגד פשטות הגמ׳ ורי״ף ורא״ש וע׳ עוד בסמוך מזה. והנה יש עוד מבוכה ברמב״ם בענין הנ״ל דאף דפשטות הדין בכל מקום דפתח שאינה גבו׳ יו״ד כלל אינו פתח והוי כחור בעלמא ורמב״ם עצמו מלבד כאן שפסק שהרגלים צריכי׳ להיות גבוהי׳ יו״ד כ׳ עוד פי״ו ה׳ י״ט צה״פ צריכי׳ בו לחיים שיהיו גבוהין יו״ד ועם כל זה כ׳ פי״ז ה׳ י״ד דצה״פ מועיל בלמטה מיו״ד והוא תמי׳ גדולה והה״מ אף שנתעורר קצת אבל לא תמה עליו כדבעי למעבד ואני כתבתי בזה דבר החדשות בחדושי ואין רצוני להטריחך בזה והדברי׳ ארוכי׳ ותמו׳ שטור וש״ע כתבו דבריו סתם לפסק הלכה. עכ״פ זה הוא ודאי דרגלי׳ צריכין להיות גבוהין יו״ד (וק׳ כקו׳ א׳ הנ״ל) וכ״כ רמ״א בש״ע אדברי רמב״ם הנ״ל דאפי׳ בפתח שוה אם אינה גבו׳ יו״ד דפטור׳ וכ׳ דזה הוא כל שכן מדינא דרמב״ם הנ״ל ולפי פשטות לשונו הי׳ חייב בזה במזוזה וזה אינו מיושב בתי׳ ט״ז הנ״ל. וכיון דדברי רמב״ם סתומי׳ לא ה״ל לטור לפסוק כדיוק הב׳ הנ״ל היוצא מדבריו ובפרט לשטת ב״י שהטור לא ס״ל כרמב״ם לענין רגלים שוות רק כרש״י א״כ למה פסק בסתמ׳ דינו לענין קשת דחייב. והנה דברי רי״ף ברור מללו דמשקוף עגול לא מהני לענין צה״פ ורמב״ן במלחמות הסכים עמו ורא״ש הביא דברי רי״ף בסתמ׳ בהל׳ מזוזה וכן ריטב״א כ׳ כן בפירוש ודברי רמב״ם אינם מובנים ודברי טור קשים א״כ לא נוכל להקל בזה למעשה ומ״מ נר׳ דהך דינ׳ תלי באשלי רברבי דהנה ריטב״א כ׳ בסוגין וז״ל וי״מ (דלא) ממתניית׳ ולא מפי׳ אביי פשטי מידי בהא אלא מתניית׳ בעלמ׳ הוא דא״צ בצה״פ דדלמ׳ שאני הכא שאין אויר מפסיק וכו׳ ולא גרסו הא דא״ל אי תשכחנהו וכו׳ ולפ״ז הלכה כר״ש במקום ר״נ עכ״ל ושיטה הלז היא שיטת בעל המאור ולא נזכר שם הך תי׳ שכ׳ ריטב״א לדחות הראי׳ דאין צריך ליגע מבריי׳ ובאמת תי׳ ריטב״א אף דמצאנו כיוצא בו ב״ב דף קס״ב ע״ב גבי מלאהו בקרובי׳ ע״ש מ״מ ה״ל לבעה״מ להזכיר תי׳ הנ״ל והוא לא כ׳ שום טעם למה לא מוכח מבריי׳ כר״נ ופסק כר״ש דצריך ליגע אבל לא הבנתי למה ריטב״א מכניס עצמו לדוחק הלז ולהמציא ס׳ חדשה דנר׳ בפשיטות דבעה״מ ס׳ משקוף עגול מהני וחייבת במזוזה וא״כ אין שום ראי׳ דהא משקוף העגול הוא נוגע וזה פשוט מאוד. וא״כ תלי הך דינ׳ דקשת בזה לבעה״מ דפסק כר״ש כשר לצה״פ ולרי״ף וסיעתו דפסקו כר״נ פסול וכנ״ל ולפ״ז הטור ואנן דפסקי׳ כר״נ דא״צ ליגע מוכח כרי״ף דפסול. ולפ״ז י״ל קצת דעת הטור דגבי מזוזה חשש לחומר׳ לדעת רז״ה ופסק חייב אבל גבי מבוי מורה כעיקר הסכמתו דא״צ ליגע וממילא משקוף עגול פסול.\\\vspace{0pt}

ולפ״ז י״ל אף לשיטתך דהנך ב׳ פרושים דרש״י במלכי׳ תלי׳ בהנך ב׳ שיטות פי׳ א׳ דרש״י במלכי׳ הוא כדעת רז״ה הנ״ל ושארי פירושי׳ כדעת רי״ף. והנה רמב״ן ושאר פוסקי׳ הרבה ופסק דידן הכל כרי״ף כמו דפשוט בכל מקומות דצה״פ אינו נוגע וע״כ דפסקי׳ כר״נ וא״כ צ״ל דגרסי׳ אי תשכחנהו וכוי דמוכח מבריי׳ כר״נ וממילא צ״ל לפסק דידן דמשקוף עגול אינו משקוף. וא״כ אי נסמך להקל בדרבנן ולהכשיר גם משקוף עגול כדעת רז״ה ה״ל תרי קולי דסתרי אהדדי ויותר מזה אם יארע משקוף עגול וגם אינו נוגע במזוזות היינו צריכין להכשיר אם נפסוק בשניהם לקולא וזה פסול לכל הדעות הנ״ל. ומ״מ אחר כל הנ״ל אם המשקוף משופע באופן שצד אחד גבו׳ יותר מחברו מוכח מדברי מג״א סימן שס״ב ס״ק י״ח בשם הרא״ם דמהני והחילוק שביניהם נר׳ דעיקר פסול משקוף עקום לפוסקים הנ״ל הוא מטעם דצריך להיות ניכר היכן המשקוף מתחיל אכן בכיפה אין ניכר לא סיום רגלים ולא התחלת המשקוף אבל בצורה הנ״ל ניכר היטיב וכשר. ולפ״ז אם המשקוף הוא כמו ח׳ שיש לו חטוטרות ג״כ נר׳ דכשר לכ״ע וראי׳ לזה נראה דהא קורה שיש עקמימות באמצעיתו מוכח מדברי רז״ה בעירובין י״ד ע״א דכשר אם הוא כולו בתוך כ׳ ובתוך המבוי ומשמע דכשר לכ״ע אף למ״ד קורה משום פתח. וא״כ ראייתך מרש״י במלכים בלא״ה אינו די״ל דהי׳ עשוי כנ״ל בדביר. ועוד מאן ימר דבדביר לא הי׳ משקוף שוה רק הי׳ עשוי משתי חתיכות מחוברי׳ יחד. אבל אם המשקוף בעגול אף אם הרגלי׳ שוים מ״מ כיון דאין זוית בין סוף רגלי׳ לתחלת המשקוף בזה פליגי הפוסקי׳ הנ״ל. וזה הוא לשון הרמב״ם הנ״ל מפני שאין לו משקוף ר״ל דאין ניכר היכן המשקוף מתחיל וכעין פי׳ הט״ז הנ״ל דאף אם המשקוף כשר באופן זה וכדעת רז״ה הנ״ל כיון דרגלי׳ עכ״פ נכרין לעצמן מ״מ כיון דאין גבו׳ יו״ד בסוף הרגלי׳ וצריכין לצרף מקצת העגול לרגלי׳ כדי שיהיו גבוהין יו״ד א״כ אין ניכר לא הרגלי׳ ולא המשקוף ולכן פסול. רק דמ״מ ק׳ עוד דהא רמב״ם פסק כר״נ א״צ ליגע וא״כ להנ״ל צריכין לומר אף ברגלי׳ שוות גבוהי׳ יו״ד מ״מ פסול אם המשקוף עגול וצ״ל דרמב״ם כגי׳ רז״ה ס״ל ולכן מכשיר עשוי׳ כקשת ומ״מ בזה סובר דלא כרז״ה דהוא פסק כר״ש ורמב״ם פסק כר״נ אף דאין לו ראי׳ מבריי׳. ולפ״ז שלאגבוים דג״כ אין לו זוית הוא כמו משקוף עקום ואפשר אפילו דגרע טפי ולא מהני אפילו לדעת רז״ה הנ״ל כיון דעומד קרוב לזקוף ומלבד זה לא העירות דכפי הנשמע שלאגבוים הוא צה״פ שעשא׳ מן הצד דצד השלשלת אינו מושכב על העמוד רק נקרס בצד העמוד וא״כ בוודאי לא מהני לכ״ע.\\\vspace{0pt}

שוב הארכת אם לפסוק כרש״י שאין צריך רגלים שוים או לא ולא אדקדק על כל דבור ודבור שבדבריך אחר שאתה בעצמך יישבת כל קושייתך אחר שראית דברי ריטב״א בסוגין ותירוצי׳ שלך נכונים והיינו אם הגרס׳ בגמר׳ ואין בה רחב ד׳ אז נאמר כן לרבות׳ דר״מ דאף דר״מ מחלק לענין חוקקין בין רחב ד׳ ברגלי׳ או לא מ״מ למעלה מג׳ א״צ רחב ד׳. ולרש״י שם רגלי׳ שבכל הסוגי׳ אין הכוונה אלא שרחב ד׳ ועי״ז ניכר שהם רגלי׳ עומדי׳ ולכן אם רחב זו בגובה יו״ד ה״ל רגלי׳ גבוהי׳ יו״ד ואין סתיר׳ בדברי רש״י אמנם לרמב״ם וסיעתו צ״ל דאין בה רחב ד׳ רק לרבות׳ לר״מ ופליגי רבנן בין אין רחב בין רחב והך ושוין איירי ברחב ד׳ ואין מזה ראי׳ לרמב״ם ולרז״ה צ״ל דלא גרס בגמ׳ ויש בה לחוק לארבעה רק סתם ויש בה לחוק והיינו לחוק שיהא רגלים שוין.\\\vspace{0pt}

ולדינא הנה ברור דרמב״ם מצריך רגלי׳ שוים וכן רשב״א וריטב״א ומ״ש ב״י סי׳ רפ״ז דדברי טור כרש״י אינו ברור דהא תחלת דברי טור כלשון רמב״ם דצריך רגלי׳ וא״כ הרי הקיל הטור אפילו לענין מזוזה דאורייתא ואף דהזכיר הטור בסוף דבריו אינו רחב ד׳ אין זה כדאי עכ״פ הש״ע פסק דצריך רגלי׳ שוים (רק בזה אין דעתו כרמב״ם ומצריך רחב ד׳ כמ״ש הש״ך ובאמת נר׳ דדעת רמב״ם יחיד בזה ופשטות הסוגי׳ דערובין ה׳ ע״א ואי בד׳ היכי משכחת לה הוא דכל פתח צריך רחב ד׳ וכ״כ הה״מ פי״ז משבת ה׳ ח׳ ראי׳ הנ״ל) והב״י בש״ע שלו הסכי׳ להרמב״ם והרמ״א לא חלק עליו א״כ א״א להכשיר בשום ענין אם אין עמודים שוין גבוהי׳ יו״ד: כן השיב לי אדמ״ו הגאון הרב מ״ה אברהם בינג זצ״ל.\\\vspace{0pt}

ועל זה השבתי – מה שכתב אדמ״ו נ״י לסתור הראי׳ ממלכים ע״פ מה שכתבו התוספ׳ עירובין (דף ב׳) דביש דלתות מהני אפילו לגבוה מכ׳ וה״נ י״ל לענין דביר דשם הדלתות ג״כ לא עבידי לצניעות – לא הבנתי דלפ״ז לא מייתי הגמרא ראי׳ מכיפה דאין צריכין ליגע די״ל ג״כ דמה דהוי בכיפה צורת הפתח אף דאינו נוגע משום דהתם הוי דלתות דהא לענין מזוזה בעינן דלתות לשיטת הרמב״ם כמו שכתב ה׳ מזוזה (פרק ו׳) משא״כ בצורת הפתח לענין עירובין אבל לפענ״ד התוספ׳ לא כתבו כן דמהני לענין גבוה מכ׳ כמו שהזכיר אדמ״ו דמזה לא איירינן התם רק לענין רחב יותר מעשר ובזה שייך שפיר דהדלתות כיון דננעלו יש להם דין מחיצות למעט הרוחב יותר מיו״ד דדלתות הם משום מחיצה כדאמרינן לענין שערי ירושלים בערובין דף כ״ב ודף ק״א, אבל לא שנעילת הדלתות יתקן צה״פ מה שבלא״ה אינו צה״פ – ועוד גם לו יהיה כדברי אדמ״ו מכ״מ לא זכיתי להבין מה שכתב דה״נ לענין דביר דשם הדלתות נמי לא עבידי לצניעות הא דלתי הדביר לא ננעלו מעולם כי אריכות הבדים מנעו אותם מלהסגר וכמו שכתבו רד״ק ורלב״ג במלכים ח׳ וכן כתבו רש״י ורד״ק בדברי הימים יע״ש.\\\vspace{0pt}

מה שכתב אדמ״ו נ״י דכיפה לא הוי צה״פ אלא ביש עלי׳ קירוי וכמו שכתב הריטב״א – הנה אמת שדברי הריטבא נעלמו ממני בעת כתבתי דברי הראשונים ולא עלתה על דעתי כלל שיטה אחרת כי נטיתי אחר שיטת הרמב״ם והטור והב״י שכתבו בפי׳ שהכיפה העשוי׳ כמין קשת מהני לצה״פ אך לא ידעתי למה לא הזכיר אדמ״ו איך דעתו בשיטת רש״י כי לפי מה שחשבתי שיטת רש״י ז״ל ג״כ כשיטת הרמב״ם דהנה בענין פתחא שימאי שהוזכר במנחות דף ל״ג ובעירובין דף י״א ע״א פי׳ רש״י וגם הרמב״ם דהפסול הוא אי לית לי׳ משקוף כלל וכן כתב הב״י ריש סי׳ רפ״ז אך הריטב״א בחידושיו לעירובין חולק על פי׳ רש״י ומפרש כפי׳ התוספ׳ דאבן יוצא ואבן נכנס ג״כ לא מקרי משקוף – והשתא לפי שיטת רש״י והרמב״ם דאבן יוצא ונכנס כשר למשקוף ולא פסול רק כשפתוח מלמעלה לגמרי, א״כ איך תסיק אדעתין דמשקוף עגולה יגרע מאבן יוצא ונכנס ולכן נראה דרש״י וגם הרמב״ם סוברים דמשקוף עגולה באמת מהני. (ורק הריטב״א אזיל לשיטתו דסבר דצריך משקוף דוקא וגם אבן יוצא ונכנס לא מהני) אך דלא נקרא משקוף רק מה שמונח על רוחב הפתח ובין שיהי׳ עגול בין שיהי׳ שוה אבל במקום שמתחיל להתעגל שם אכתי אינו משקוף כיון שאינו מונח על הרוחב ומזוזות נמי אינן לשיטת רש״י כיון שרגלים הם רחבים ד׳ א״כ במקום שמתחיל להתעגל הוא מצר ואינו רחב ד׳ וכמו שכתב ד״ה וחכמים א״כ נחשב התחלת העגול כמו אויר ומזה הוכיח הגמרא דאין צריך ליגע, וזה מה שכתב רש״י וגם הרי״ף התקרה שבשמי הכיפה היינו אותו חלק העליון של הכיפה דאי כוונו לתקרה שעל הכיפה לא הי׳ להם לקרוא זה תקרה העליונה או קירוי שבשמי הכיפה דמלשון זה משמע שהוא חלק מעגול הכיפה – וכן הוא ג״כ לשיטת רמב״ם רק דהוא נבדל בזה מרש״י דלדידי׳ העגול לא נקרא מזוזות כלל אפילו ברחב ד׳ וכמו שכתב ב״י סוף סי׳ רפ״ז – ועוד יש סמך קצת לשיטה זו ממה דאמרינן בעירובין דף י״ד ע״א היו א׳ למעלה וא׳ למטה ריב״י אומר רואין כו׳ וכוותי׳ פסקינן, וא״כ ה״נ י״ל דרואין כל חלק מהעגול כאילו הוא למטה שוה לגובה המזוזות ואז נעשה משקוף שוה אך דמכ״מ מוכח דאין צריך ליגע דהא אויר העגול מפסיק בין מה שנעשה תקרה להמזוזות, וזה הטעם דמ״ד דאין צריך ליגע, משום דרואין המשקוף כאילו הוא למטה והמזוזות כאילו הם למעלה, והא דמייתי מברייתא דכיפה ולא מברייתא דקורה י״ל דניחא לי׳ להביא ראי דגם בצה״פ אמרינן רואין דמקורה אין ראי׳ כ״כ דהא איכא מ״ד דקורה הוא משום מחיצה, ואז פשיטא דרואין כאילו כל הקורה מגיע למטה דאמרינן גוד אחית או חבוט רמי כן הבנתי פי׳ רש״י ושיטת הראשונים אך אם גם יהי׳ פשט הזה בשיטת רש״י ורי״ף בגדר הספק וגם אם נאמר שהרמב״ם חולק על הרי״ף מה שאין דרכו בשאר מקומות אכתי לא ידעתי למה נניח הרמב״ם והטור והב״י ומה שפסק גם בש״ע וכפי מה שכתב אדמ״ו גם הרז״ה ס״ל כשיטה זו ונפסוק להחמיר כשיטה האחרת שאין לנו ברור שסובר כן רק הריטב״א לבד.\\\vspace{0pt}

מה שהקשה אדמ״ו בדיוק לשון הרמב״ם נלענ״ד לומר דכיון דאותו חלק מהפתח הנזקף לאורכו הם המזוזות והשוכב לרחבו הוא המשקוף והשתא העגול שהולך מארכר לרחבו פעם נחשוב למזוזה ופעם נחשב למשקוף והיינו אם יש כאן צורת הפתח במזוזות דהיינו שהם גבוהים יו״ד אזי העגול נחשב למשקוף דאמרינן רואין וכנ״ל אבל אם אין גבוהין הרגלים יו״ד אזי ע״כ לא שייך לומר דהעגול הוא המשקוף דמשקוף בלא מזוזות מי איכא וע״כ נחשב העגול לרגלים וא״כ אין כאן משקוף (והוא כעין תי׳ הט״ז ואין זה סותר למה שכתבנו לעיל בשם הב״י דלהרמב״ם העגול לא נחשב לרגלים דהאמת כן להרמב״ם) והשתא לא קשה הקושיא א׳ שהקשה אדמ״ו עליי דשפיר נדייק דאם יהי׳ כאן עוד משקוף שוה על העגול כשר לצה״פ גם באין ברגלים עשרה דאז נחשוב העגול לרגלים להשלים מנין יו״ד וגם מהדיוק הב׳ לא קשה ע״פ מה דכתבנו לעיל.\\\vspace{0pt}

עוד כתב אדמ״ו דאין ראי׳ מפתח הדביר דדלמא הי׳ צורתה כקורה שיש לו עקמימות באמצעיתה או שהי׳ משקוף שוה רק עשוי מב׳ חתיכות מחוברים יחד ולא זכיתי להבין דלפי׳ הראשון הי׳ לו ז׳ זויות ולפי הב׳ לא הי׳ לו רק ד׳ ובשניהם לא שייך לומר חמישית ובלא״ה לשון עשוי כשתים לא משמע כאחד מהנך פירושים ולכן לא נראה לומר גם כן, מה שלא הזכיר אדמ״ו, דהי׳ קירוי עליו מלמעלה וע״י זה נקרא פתח הדביר פתח דא״כ עיקר צורת הפתח הי׳ רביעית ולא חמישית.\\\vspace{0pt}

אך בלא כל הנ״ל ראיתי צד הכשר לשלאגבוים דלפי מה שחשבתי בתחילה השלאגבוים יש לו צורת צריף אך עתה ראיתי עכ״פ ברוב השלאגביימע דהקורה הנזקף מגיע עד על העמוד אשר השלשלת בו והשלשלת שוה לצד העמוד אשר נקשרת בו וא״כ יש לו ממש הצורה שהזכיר המג״א בשם הרא״מ ס״ס שס״ב דאפילו קנה א׳ גבו׳ מאחר כשר ולפ״ז אפילו נקשרת השלשלת בצד העמוד (אף דבאותן שראיתי היא קשורה באמצע) זה לא מעלה ולא מוריד דלא צריך השלשלת כלל לצורת הפתח כיון דסוף הקורה מגיע עד על פני העמוד האחר בגובה והרי פסקינן דאין צריכין ליגע גם בלא״ה לא הבנתי למה לא חילק אדמ״ו בין עגול ככיפה לכמין צריף דגם אם בעיגול אינו ניכר מקום שהתחיל להתעגל מ״מ בכמין צריף ניכר היטב מקום שמתחיל להשתפע ונעשה שם זוית ממש וכן הוא בשלאגבוים ומכל זה נלענ״ד דכשר לצה״פ אכן בתנאי או שהשלשלת קשורה באמצע ולא מן הצד או שהקורה כשהוא נזקף מגיע ע״פ העמוד האחר שכנגדו וגם בתנאי שיהיו העמודים העומדים זקופים גבוהים י׳ טפחים. כנלענ״ד הקטן יעקב.\\\vspace{0pt}

\end{multicols}\newpage

\newchap{סימן כח}
\begin{multicols}{2}
ב״ה מאננהיים, ניסן תקפ״ט לפ״ק. להרה״ג וכו׳ מ״ה זעקל ליב נ״י בק״ק מיכעלשטאדט יע״א.\\\vspace{0pt}

נשאלתי ממעכ״ת נ״י – יותר מבן י״ג שנה ויום א׳ הביא ביום י״ג ניסן גדי לשוחט ואמר בל״א זה יהי׳ קרבן פסח (דאס גיעבט איין קרבן פסח) והשוחט מיאן לשחוט מפני שידע חשש איסור בזה מה יהי׳ דין הגדי אם מותר בהנאה או לא?.\\\vspace{0pt}

תשובה – הנה אם הטלה שלו תלי בפלוגתת הפוסקים שהביא המג״א בסי׳ תס״ט דלדעת הש״ך והע״ש והט״ז מותר אפילו באכילה ולדעת הב״ח אסור בהנאה והמג״א הכריע דבגדי וטלה יש להחמיר בדיעבד והחוק יעקב חולק עליו וס״ל כיון דהלכה כרבי טרפון דבלא התנדב כדרך המתנדבים אינו קדוש לכן הכא אין קפידא בדיעבד ואף דזה כתב התם לענין בשר זה לפסח והכא גרע דאמר בפי׳ זה יהי׳ קרבן פסח והרי הקדישו מכ״מ נראה דאין חילוק דהטעם שלא התנדב כדרך המתנדבים כתבו התוספ׳ דה״ל כמי שאמר ע״מ שאקריבנה בבית חוניו כיון שעתה אין בית המקדש להקריבו וא״כ ה״נ שהביאו להשוחט ואמר שישחטנו לקרבן פסח ה״ל כמו ע״מ שאקריבנה בחוץ ולא התנדב כדרך המתנדבים ואינו קדוש – אכן כל זה באם הטלה שלו – אכן בנדון זה דמסתמא אינו שלו דאין לנער כזה טלה שישחוט לו דמסתמא סמוך על שולחן אביו או אחרים וכן שמעתי שלא הי׳ שלו רק של אביו בזה יש לדון דאין כאן איסור כלל דהא אין אדם מקדיש דבר שאינו שלו ואין אדם אוסר דבר שאינו שלו. שוב ראיתי בכה״ג שכתב ג״כ הכי בשם דמשק אליעזר דכל זה דוקא בשהוא שלו אבל באינו שלו לא אבל הוסיף עוד דכמדומה שראה שפסקו כן אפילו באחד שלא הי׳ הטלה שלו והיינו משום חומרא דקדשים ע״כ והפר״ח חולק עליו מטעם שאין אדם אוסר דבר שאינו שלו. ולכאורה תלי זה בפלוגתת הפוסקים בי״ד (סימן ה׳) דלדעת הש״ע שהיא שיטת הרמב״ם בשוחט בהמת אחרים לשם קדשים אינו אסור אבל הרמ״א בשם הב״י כתב דמשום מראית העין יש להחמיר גם בשל אחרים והיינו שהב״י רצה לבאר כן דעת הרא״ש והטור, והט״ז חולק על הב״י והרמ״א וכתב שאין ראי׳ שהרא״ש והטור חולקים על הרמב״ם ולכן כתב להקל להלכה ולא למעשה. אכן ביש עוד צד היתר אפילו למעשה גם הפר״ח הסכים עם הט״ז. והנה זה בשוחט לשם קדשים דעביד מעשה רבה אבל בנדון דידן שלא עשה מעשה רק הקדיש י״ל דכ״ע מודים דאינו אסור דאין אדם מקדיש דש״ש דאל״כ היאך משכחת לעולם דאין אדם מקדיש דבר שאינו שלו וכן נראה מדברי הט״ז דאם הקדיש בהמת אחרים פשיטא לו לכ״ע דאינו אסור אפילו משום מראית העין ואפשר דבנדון דידן לא שייך מראית העין כלל דכ״ע ידעו דאין לנער כזה בהמה להקדישו ואפילו שייך משום מראית העין בנדון זה מכ״מ כיון שהט״ז הקיל למעשה ביש עוד טעם להתיר והכא יש טעם שלא התנדב כדרך המתנדבים דבזה מקיל החוק יעקב אפי׳ בשלו לכן יש טעם להקל: גם לענין מראית העין יש תקנה שישאול על הקדשו דאפילו יסברו העולם שהי׳ שלו מכ״מ הרי נשאל על הקדשו. ועיין בכרתי ופלתי (סימן ה׳) וכש״כ שיש תקנה בנדון דידן אם ימכור לנכרי דבזה אין איסור כ״כ גם אם הי׳ קודש משישחוט בחוץ ויאכלנו. לכן הורתי שישאל הנער על הקדשו ואח״כ ימכרנו לנכרי והדמים מותרים. כנלענ״ד הקטן יעקב.\\\vspace{0pt}

\end{multicols}\newpage

\newchap{סימן כט}
\begin{multicols}{2}
אלטאנא, יום ה׳ ל״ה למב״י תרכ״ד לפ״ק. לבני חביבי הרבני מ״ה בן ציון נ״י בק״ק מאהילעוו יע״א.\\\vspace{0pt}

אשר נסתפקת מי שמחמת חסרון שינים אינו יכול ללעוס מצה אם מותר לו לפררם לפירורין דקים ע״י שידוך במדוכה ואם יוצא בזה ידי חובת מצה מפני שהוקשה לך שלא ראית ברמב״ם ופוסקים שהביאו הברייתא דמייתי הגמרא בברכות (דף ל״ז) ובמנחות (דף ע״ה) לקט מכולן כזית ואכלן אם חמץ הוא ענוש כרת ואם מצה היא אדם יוצא בו ידי חובתו בפסח שמשם מוכיח בגמרא דפורכן למנחות עד שמחזירן לסלתן מברך המוציא.\\\vspace{0pt}

הנה כבר הביא הרא״ש שם דברי רבינו יונה דלחם שאינו לא מבושל ולא מחובר אלא שהוא מפורר דק דק אע״פ שאין בהן כזית ואין בהן תואר לחם מברך עליו המוציא דאינו יוצא מידי לחם וכן פסק הטוש״ע א״ח (סי׳ קס״ח) ולכן אין ספק שיוצא במצה מפוררת גם ידי חובת מצה ומה שלא הביאו הפוסקים הברייתא דיוצא אדם בו י״ח בפסח י״ל כיון שהביאו לענין ברכת המוציא ממילא משתמע מדמקרי לחם לענין המוציא מקרי ג״כ לחם לענין מצה אלא שיש להסתפק אם מותר לכל אדם לכתחלה לפרר המצה או רק לזקן וחולה שהרי בפסחים (דף מ״א) אמרינן יוצא אדם במצה שרוי׳ וכתב הרי״ף דלא שרינן אלא לזקן או לחולה ובסי׳ תס״א ס״ק ז׳ כתב המג״א דמשמע להרי״ף כן ממה שכתוב יוצא אדם במצה שרוי׳ דמשמע דיעבד ע״ש וא״כ הכי נמי לענין מצה מפוררת דתני בברייתא אדם יוצא בו י״ח בפסח משמע ג״כ מלשון יוצא דיעבד דוקא אלא דק״ל היאך נדייק מלשון יוצאין דדוקא בדיעבד קאמר הרי בפסחים (דף ל״ח) תנן אילו דברים שאדם יוצא בהן ידי חובת מצה בחטין וכו׳ וכן תנן שם (דף ל״ט) ואילו ירקות שאדם יוצא בהן ידי חובתו בפסח וכו׳ ובאילו ודאי לכתחלה קאמר ובאמת לולא דברי המג״א הי׳ נלענ״ד דגם הרי״ף לא רצה לומר דלא מותר גם לכל אדם לשרות מצה במים אלא דרצה להורות עצה טובה לזקן ולחולה שאין יכולים לאכול מצה שיש להם תקנה ליצא י״ח בהיתר שכן משמע מטור וש״ע שכתבו הברייתא סתמא שיוצאין במצה שרוי׳ ואי הוי ס״ל דהרי״ף לא התיר רק בדיעבד למי שאינו יכול לאכול מצה לא הוי שתקי להשמיענו כן ועכ״פ מדהשמיטו תנאי זה משמע שלא חששו לזה. אמנם לפי דעת המג״א שהרי״ף לא התיר רק לזקן ולחולה ויליף כן מדתניא יוצאין לשון דיעבד צריך לחלק בין לשון יוצאין סתמא ובין לשון אדם יוצא ידי חובתו דמשמע טפי שמותר לכל אדם אפילו לכתחלה וא״כ לענין פירורין דנקטה ג״כ הברייתא לשון אדם יוצא בו ידי חובתו בפסח מותר ג״כ לכל אדם ולכן מי שלא יכול לאכול מצה מחמת חסרון שנים טוב יותר לפרר המצה מלשרות אותה מה שלדעת המג״א לא מותר רק בדיעבד למי שאינו יכול לאכול בענין אחר. כנלענ״ד הקטן יעקב.\\\vspace{0pt}

\end{multicols}\newpage

\newchap{סימן ל}
\begin{multicols}{2}
ב״ה אלטאנא, ט״ו אייר תרי״ב לפ״ק.\\\vspace{0pt}

בליל זמן אכילת פסח שני למען קיים כל העוסק וכו׳ ראיתי לחקור איזה חקירות בענין מצוה זו, הנה בפסחים (ד׳ צ״ה) במתניתן מה בין פסח ראשון לפסח שני קחשיב בל יראה והלל בשעת אכילה דליתא בשני ושוין לענין הלל בעשיי׳ ולענין אכילת צלי במצות ומרורים ודחיית שבת ע״ש והתוספ׳ (שם) הביאו בשם התוספתא ג״כ החילוק שהראשון נשחט בג׳ כתות ולא השני וכן שהראשון טעון בקור ולא השני וכתבו דתני ושייר והתוספ׳ ביומא (דף כ״ט) מסופקים אם פסח שני נקרב כמו פ״ר אחר תמיד של בין הערבים כיון דהטעם דיאוחר פ״ר הוא משום שנאמר בערב וביה״ע ובתמיד לא כתיב רק בין הערבים וזה לא שייך לענין פ״ש דשם ג״כ לא נאמר רק בין הערבים ומה שלא חשיב ליה במה שבין פ״ר לפ״ש אין ראי׳ שהרי שייר ג״כ כרת למ״ד דחייב כרת על הראשון ולא על השני ושייר ג״כ נשים למ״ד נשים בראשון חובה ובשני רשות ע״ש ולא ידעתי למה הוצרכו התוספ׳ להני שיורי ולא נקטי מה שהוא לכ״ע דהיינו שלש כתות דליתא בשני וכן בקור וכן יש שיור דראשון בא בטומאה ולא השני. וכל הני דשייר במתניתן תני בתוספתא דהיינו החילוק בג׳ כתות ודבא בטומאה ודחיוב כרת ודחגיגה דהשני אין טעון חגיגה, והרמב״ם (סוף ה׳ ק״פ) חשב החילוקים שבין פ״ר לפסח שני בל יראה ושאין נשחט על החמץ ושאין מוציאין חוץ לחבורה ושטעון הלל באכילתו ושמביאין עמו חגיגה ושבא בטומאה דכל הני הם בראשון לבד. ולא חשב הרמב״ם הך דג׳ כתות וגם החילוק של נשים ה״ל לחשוב אחר שפסק נשים בשני רשות שם (פ״ה) ולא ידעתי למה השמיטם. ולפ״ז יש בין ראשון לשני בל יראה ושאין נשחט על החמץ ושטעון בקור ושטעון ג׳ כתות ושמביאין עמו חגיגה ושאין מוציאין חוץ לחבורה ושבא בטומאה וחיוב כרת למ״ד דכרת בראשון ולא בשני וחיוב נשים למ״ד נשים בשני רשות ומה שמסופקים התוספ׳ אם בא פ״ש קודם התמיד.\\\vspace{0pt}

והנה בכל החילוקים מפורש הטעם דמה שהשני אינו בבל יראה ולא תשחט יליף פסחים (דף צ״ה) מקרא ומה שאין טעון בקור יליף שם (דף צ״ו) מהזה וכן לשאר הדברים שפרטנו שהם דוקא בפ״ר ולא בפ״ש מפורשים הטעמים במקומם בגמרא ע״ש אכן מה שקחשיב בתוספתא שגם ג׳ כתות אינם בשני לזה לא ראיתי טעם לא בגמרא ולא בראשונים ולא באחרונים מ״ש ג׳ כתות משאר מצות של עשיית פסח וא״ל שמפני שהמקריבים פ״ש מועטין לא הוצרכו לכך דז״א שהרי גם בפסח ראשון אפילו לא היו רק חמשים המקריבים הוצרכו להתחלק לג׳ כתות כדאמרינן שם (דף ס״ד) ולענ״ד אולי י״ל כיון דמה שמצות שבגופו נוהגים גם בפ״ש ילפינן מככל חקת הפסח יעשו אותו ואמרינן במנחות (דף י״ט) דכל מקום שנאמר חקה אינה אלא לעכב ולכן לא ילפינן לפ״ש רק מצות שבגופו שמעכבין בפ״ר אבל ג׳ כתות אינן מעכבות בפ״ר כמו שפסק הרמב״ם ה׳ ק״פ (פ׳ א׳) וע״ש הטעם בכס״מ ועל כן ל״צ בפ״ש אפילו לכתחילה. ואף שקריאת ההלל ג״כ אינו מעכב מכ״מ צריך בהקרבה דשיר צורך עבודה הוא. ולפ״ז יש ראי׳ למה שנראה מדברי התוספ׳ סוכה (דף מ״ב) דבקור מעכב דאי אינו מעכב בתמיד ודאי ג״כ שאינו מעכב בפסח א״כ ל״ל הזה למעט פ״ש מבקור כיון דאפילו בראשון לא מעכב ובמקום אחר הוכחתי משיטת רש״י דס״ל דבקור לא מעכב וגם תמהתי על הרמב״ם שלא הביא כלל דפסח ראשון טעון בקור כמו שהביא לענין תמיד ואין כאן מקומו. אמנם במה ששוין פ״ר ופ״ש קחשיב במתניתן הנ״ל הלל בעשיי׳ ונאכל צלי על מצות ומרורים ודוחין את השבת ובתוספתא הוסיף דבשניהם שה תמים זכר בן שנה ואיסור אכילת נא ומובשל וטעונין לינה והרמב״ם הוסיף על אילו שנאכלין בבית א׳ ואין מותירין מהן ואין שוברין בהן את העצם ע״ש אכן אכתי שייר טובא שלא חשב במה ששוין פ״ר ופ״ש וחלוקין משאר קדשים דהיינו שאינן נקרבין עד אחר חצות ושנשחטין שלא לשמן או שלא לשם בעלים פסולים ושאין שוחטין על היחיד לכתחלה ושאינן נאכלין אלא למנויין ושאינן נאכלין אלא בלילה ושנתנין מתנה א׳ ואין בהן סמיכה ותנופת חזה ושוק ונסכים כשלמים ואף דלא ראיתי מפורש שכל דינים הללו הם גם בפ״ש מכ״מ מוכח כן מסוגיא דפסחים (דף פ״ט) ע״ש ועוד דכל אילו הם מצות שבגוף ושעל גוף הפסח דמרבינן בפ״ש מכל חקת הפסח יעשו אותו ואפילו מצות דרבנן שבגוף הפסח נהגו גם בשני וכן מוכח ממה שכתבו הותספ׳ (ר״פ ערבי פסחים) דלכך נקט ערבי פסחים לשון רבים משום פסח ראשון ושני הרי דס״ל שמה שתקנו שלא יאכל מן המנחה ולמעלה כדי לאכול פסח ומצה לתאבון גם לערב פסח שני תקנו אכן זה פשוט שכל מה שהי׳ בפ״ר משום זכר לגאולה ולא הי׳ בגוף הפסח לא הי׳ בפ״ש ולכן לא הי׳ לא ספור יציאת מצרים ולא ד׳ כוסות ולא הסבה שכל זה לא נוגע לפסח.\\\vspace{0pt}

ואם צריך טבול בחרוסת בפסח שני לכאורה תלי בפלוגתא שבין ת״ק ורבי אלעזר ב״ר צדוק שם (דף קי״ד) דלת״ק שאין חרוסת מצו׳ רק משום קפא ודאי נוהג גם בשני דמ״ש לענין קפא ראשון משני אבל לראב״צ דס״ל מצו׳ זכר לתפוח וזכר לטיט אין זה חיוב רק בפסח ראשון אמנם נ״ל דודאי גם ראב״צ לא פליג אמציאות שמועיל חרוסת לקפא רק דס״ל דאפילו לא יקפיד אקפא או אם יכול להסיר קפא ע״י דבר אחר מכ״מ יש חיוב בחרוסת ולכן אפילו לראב״צ דפסק הרמב״ם כוותי׳ מכ״מ יש טבול בחרוסת גם בפ״ש ומכ״מ יש חילוק בין פ״ר לפ״ש לענין חרוסת לשיטת הרמב״ם ה׳ חו״מ (פ׳ ח׳) שגם המצה טובל בחרוסת ע״ש שזה ודאי לא הי׳ בפ״ש כיון שהטבול הוא אז רק משום קפא וזה לא שייך רק במרור.\\\vspace{0pt}

אמנם יש לחקור במה שאמר ר״ג שם (דף קט״ז) שצריך לפרש טעם פסח ומצה ומרור אם הי׳ זה גם בפ״ש דלפי מה שכתבו התוספ׳ שם ילפינן כן מדכתיב ואמרתם זבח פסח הוא פי׳ באמירה שצ״ל פסח זה שאנו אוכלין ואתקש מצה ומרור לפסח עכ״ל ולפ״ז הוי זה מצוה שבגופו או עכ״פ על גופו של פסח שנוהג גם בשני ועוד שהרי מקרא דזבח פסח הוא ילפינן בזבחים (דף ז׳) דפסח שלא לשמו פסול וכפי מה שכתבתי לעיל נוהג זה גם בפ״ש וא״כ ה״ה האמירה דילפינן מקרא זה.\\\vspace{0pt}

אכן מלבד שהדרשה שהביאו התוספ׳ אינה במשמעות הפסוק גם הלשון לא יצא ידי חובתו קשה דאם לא קיים מצות אמירה למה לא יצא ידי אכילת פסח מו״מ והר״ן נדחק לפרש דקאמר כאלו לא יצא וכעין מה שנאמר בסוכה (פ״ב) אם כן היית נוהג לא קיימת מצות סוכה מימיך ולענ״ד אין זה דומה לשם דהתם עשה בישיבת הסוכה נגד תקנת חכמים וא״כ שייך לא קיימת ישיבת סוכה מימיך כמצות חכמים אבל הכא האכילה מצוה לחוד והאמירה מצוה לחוד ואיך שייך שלא יצא במצות אכילה כיון שלא קיים מצות אמירה בהדה ולכן לולא דברי הראשונים הי׳ נלענ״ד שמה שאמר ר״ג לא יצא י״ח לא לענין אכילת פסח מצה ומרור קאמר אלא לענין מצות ספור יציאת מצרים דכבר הקשו הפוסקים מה ענין מצות ע׳ של ספור יציאת מצרים בליל פסח כיון דבכל לילה חיוב להזכיר יצ״מ כדרשת בן זומא דפסקינן כוותי׳ ותרצו דבכל לילה די בזכירה לחוד אבל בליל פסח צריך להרבות לספר והנה למעלה אין לזה שיעור ולכן אמרו רז״ל כל המרבה לספר בי״מ ה״ז משובח אבל הרי צריך שיעור למטה כמה יספר ויהי׳ יוצא י״ח מצות ספור יציאת מצרים ונלענ״ד דלזה בא ר״ג ומפרש ב׳ דברים האחד שהמועט להיות יוצא הוא לפרש טעמי פסח מצה ומרור כדדרשינן מוהגדת לבנך וגו׳ בעבור זה עשה ד׳ לי בעבור זה לא נאמר אלא בשמראה על פסח מצה ומרור וגם מפורשים טעמים של ג׳ דברים הללו בג׳ פסוקים שצותה התורה לספר לבנים טעם פסח בפסוק ואמרתם זבח פסח הוא וטעם מצה בפסוק ואמרת אליו בחוזק יד הוציאנו וגו׳ שהוא שלא הספיק בצקם להחמיץ וטעם מרור בפסוק ואמרת לבנך עבדים היינו לפרעה במצרים ועוד שנית נשמע מדברי ר״ג שאפילו ספר כל הלילה מיציאת מצרים ולא פירש טעמים הללו לא יצא י״ח סיפור י״מ ולכן נסמכו דברי ר״ג במתניתן אחר מתחיל בגנות ומסיים בשבח וכו׳ להודיענו שגם זה מדבר מענין מצות הגדה ופי׳ זה נכון לענ״ד ולפי״ז לא שייך זה רק בפ״ר ולא בפסח שני ויה״ר שנזכה לעשות פסח ב״ב כעתירת הקטן יעקב.\\\vspace{0pt}

\end{multicols}\newpage

\newchap{סימן לא}
\begin{multicols}{2}
ב״ה אלטאנא, יום ה׳ ז׳ אב תרכ״ג לפ״ק. לחתני הרה״ג וכו׳ מ״ה ישראל מאיר פריימאנן נ״י אב״ד דק״ק פילעהנע יע״א.\\\vspace{0pt}

על דבר שאלתך באירע מילה בשבוע שחל ט״ב להיות בשבת אם יש היתר למוהל ולסנדק ולב״ב לספר זקנם.\\\vspace{0pt}

תשובה – כבר הערת לנכון שדין זה אין לו שורש בפוסקים ראשונים אמנם באחרונים נמצאו בזה דיעות חלוקות האלי׳ רבה סי׳ תקנ״א התיר מר״ח עד שבוע שחל בה ט״ב דבה אוסר להם לספר אלא שכתב שהעיקר כדעה ראשונה שבש״ע דבחל ט״ב בשבת אין לשבוע דין שחל בו ט״ב ובשו״ת פנים מאירות ח״ג סי׳ ל״ז התיר לספר להם אפילו בשבוע שחל בו ט״ב ע״ש טעמו ובשו״ת נודע ביהודה ח״ר א״ח (סי׳ כ״ח) מפקפק קצת בהיתר אפילו מר״ח ואילך אבל בשבוע שחל ט״ב כתב פשיטא שאסור ובשו״ת חתם סופר א״ח סי׳ קנ״ח הסכים עם פנים מאירות וכתב אילו ראה הנ״ב מה שכתב הפנים מאירות הוי הדר בי׳ כי אוסר בלי שום ראי׳ כלל ותמהני שלא הזכיר שגם האלי׳ רבה אוסר בשבוע שחל בה ט״ב וגם בשערי תשובה הביא בשם שו״ת זרע אמת וגם בשם בני יהודה שהשיגו על פ״מ ואסרו תספורת ולכן לענ״ד אין לזוז מפסק הא״ר לאסור בשבוע שחל בו ט״ב באמצע השבוע אבל בחל ט״ב בשבת מותר שבכזה ודאי יש לסמוך על הך דעה שאין בזה דין שבוע שחל בו ט״ב ולכן נראה שיש להתיר לספר אפילו בע״ש כשהמילה היא בט״ב בשבת. כנלענ״ד הקטן יעקב.\\\vspace{0pt}

\end{multicols}\newpage

\newchap{סימן לב}
\begin{multicols}{2}
ב״ה אלטאנא, בחודש אב תרי״ג לפ״ק. להרה״ג וכו׳ מ״ה אלכסנדר אדלער נ״י אב״ד דק״ק ליבעק יע״א.\\\vspace{0pt}

מעכ״ת נ״י שאל לא מצאתי לאחד מן האחרונים דין קבלת תענית כשחל תענית באחד בשבת והוא בפירוש בתוספ׳ ע״ז (דף ל״ד) שכתבו ור״י הי׳ רגיל כשהי׳ מתענה באחד בשבת לקבל התענית באלקי נצור עכ״ל ומה שכתב העטרת זקנים והמג״א סימן רפ״ח ס״ק ב׳ ויקבלו עליהם בשבת שני ימים תענית בימי החול עכ״ל אין ראי׳ לזה כי שם מיירי בתענית חלום שמדינא מותר בשבת עצמו גם מסימן תצ״ב אין ראי׳ דשם איירי שנכלל הקבלה במי שבירך ואין זה קבלה גמורה, וכ״ת מהיכי תיתי שלא יותר קבלת תענית בשבת הרי ממה שחדשו לנו התוספ׳ במה שהי׳ הר״י רגיל לקבל תענית משמע ששאר העולם לא נהגו כן עכ״ד.\\\vspace{0pt}

תשובה – נראה פשוט אחר שאמרינן בתענית (דף י״ב) כל תענית שלא קבלו מאתמול אינו תענית ודאי שגם בשבת צריך לקבל ואין לומר דיקבל מערב שבת דז״א דהא בעינן קבלה סמוך לתענית כדאמרינן (שם) אימתי מקבלו רב אמר במנחה פי׳ לעת המנחה ושמואל אמר בתפלת מנחה ופי׳ רש״י שם דדוקא נקט מנחה משום דסמוך לתחלת יום תעניתו לאפוקי תפלת יוצר עכ״ל וא״כ כש״כ דקבלה מע״ש אינו מועיל וע״כ צריך לקבל בשבת במנחה ולא הוצרכו הפוסקים להזכיר מרוב פשיטתו ומה שכתבו התוספ׳ דר״י הי׳ רגיל לקבל באלקי נצור דמשמע שיש חידוש דין בזה י״ל דר״י ס״ל כפי׳ ר״ח שהביא הטור (סי׳ תקס״ג) דמה דקאמר שמואל דפסקינן כוותי׳ בתפלת מנחה היינו בשומע תפלה וא״כ בשבת דלא מתפללים שומע תפלה לא ידענו היכן יקבל לזה כתבו דר״י הי׳ רגיל לקבל בשבת באלקי נצור משא״כ בחול שהי׳ מקבל בש״ת אבל אחר דפסק הטור ואחריו הש״ע כיון דאיכא פלוגתא דרבוותא טוב שלא להפסיק בתפלה אלא יקבל גם בחול באלקי נצור לכן לא הוצרכו להזכיר הדין בשבת שפשוט הוא ששו׳ שבת לחול דלענין חיוב קבלה גם הר״י לא הוצרך להשמיענו. כנלענ״ד הקטן יעקב.\\\vspace{0pt}

\end{multicols}\newpage

\newchap{סימן לג}
\begin{multicols}{2}
ב״ה אלטאנא, יום ד׳ א״ח של סוכות תרכ״ד לפ״ק.\\\vspace{0pt}

שאלה – מי שלא מצא שופר אלא בפחות מכשיעור טפח אם יכול לתקוע בו בי״ט של ר״ה ולברך עליו.\\\vspace{0pt}

תשובה – הדין ששיעור שופר טפח אתי מברייתא דנדה (דף כ״ו) דתנא ר׳ אושעי׳ דמן חבריא חמשה שיעורין טפח ואילו הן שליא שופר שדרה דופן סוכה והאיזוב ואמרינן שם שופר דתניא כמה יהא שיעור שופר פירש רשב״ג כדי שיאחזנו בידו ויראה לכאן ולכאן טפח שדרה מאי היא דאמר ר׳ פרנך אר״י שדרו של לולב צריך שיהא יוצא מן ההדס טפח ע״ש וראיתי בברכי יוסף א״ח סי׳ תרמ״ט על מה שהביא הרמא שם דעת הרמב״ם שיכול ליטול לולב יבש היכא דליכא אחר ולברך עליו כתב בשם רבינו ישעי׳ הראשון בתשובה כ״י שיכול לברך על לולב שאין בו אלא שלשה טפחים היכא דליכא אחר וגוף התשובה לא העתיק לנו אכן ממה שהביא הב״י זה בסימן תרמ״ט נראה דעתו שדעת הר״י הראשון כשיטת הפוסקים שהובאו בטור שם שיכול לברך על כל הפסולים היכא דליכא כשר ולפ״ז לפי מה דפסקו הש״ע והרמ״א שם דלא יברך על הפסולים אין נפקותא לדידן בדין זה אלא שהוקשה לי דאם זה טעם הרבינו ישעי׳ למה כתב על לולב של ג׳ טפחים דוקא ולא סתם על כל הפסולים וגם בלולב עצמו למה התנה על של ג׳ טפחים דמשמע אבל בפחות לא ומ״ש פסול דג״ט מפחות מג׳ טפחים כיון דשיעור לולב להכשר הוא ד׳ טפחים ולכן לענ״ד יש לישב פסק זה גם לשיטת הפוסקים דפסקינן כוותייהו דלא יברך על הפסולים דיש לחקור אם ה׳ שיעורי טפח דחשיב ר״א בנדה הם כולם הל״מ או אם גם שיעורי דרבנן חשיב בהדייהו והנה מדברי התוספ׳ סוכה (דף ז׳) ד״ה סיכך נראה שדעתם שכל שיעורי טפח דאורייתא ננהו שכתבו דאין לומר דטפח סוכה דרבנן ומדאורייתא בדופן כל שהוא סגי דהא בפרק המפלת קחשיב דופן סוכה בהדי חמשה ששיעורן טפח ואמרינן התם דלא קחשיב קורה טפח משום דהוי מדרבנן ולא חשיב אלא הנך דכתיבן ולא מפרש שיעורייהו עכ״ד ומזה נראה דס״ל דכל הני שיעורי טפח דאורייתא הם אמנם כבר השיב הפני יהושע על ראי׳ זו דמשם לא מוכח רק דדופן סוכה היא מדאורייתא ולא כקורה דמדרבנן אבל ששיעור טפח ג״כ דאורייתא לזה ליכא הוכחה (גם בלא״ה תמהתי על התוספ׳ בספרי שם למה להו ראי׳ מרחוק דשיעור טפח דאורייתא הוא הא בפי׳ אמרינן בסוכה [דף ו׳] אתאי הלכתא ואוקמתא אטפח) ולענ״ד יש להוכיח דלא כל שיעורי טפח דחשיב שם הל״מ הם ממה שכתב הרא״ש בר״ה אמה דקאמר שיעור שופר פירש רשב״ג כדי שיאחזנו בידו ויראה לכאן ולכאן וז״ל והיינו טפח כדאמרינן פי המפלת דשיעור שופר טפח ונקט האי לישנא הכא כדי שיאחזנו התוקע בידו ויראה לכאן ולכאן ולא יאמרו לתוך ידו הוא תוקע עכ״ל וכ״כ גם הר״ן וז״ל וכי תימא לימא טפח בהדיא י״ל דטעמא אתי לאשמעינן דמשום הכי הוי שיעורא טפח כדי שיאחזנו בידו ויראה לכאן ולכאן עכ״ל הרי דס״ל דשיעור טפח דשופר לאו הל״מ הוא אלא רבנן תקנו כן משום מראית העין שלא יאמרו לתוך ידו הוא תוקע ואפילו נימא דלדעת הרי״ץ גיאות שהביא הטור סי׳ תקפ״ו שצריך טפח שוחק כדי שיראה לכאן ולכאן ופי׳ הב״י שרצה לתרץ ג״כ למה נקט האי לישנא כדי שיאחזנו ולא נקט טפח סתם ע״ש י״ל דס״ל ששיעור טפח דשופר הוא הל״מ ושזה ג״כ דעת התוספ׳ ולכן לא יקשה עליהם מטפח דשופר מכ״מ הרי הטור דחה דעת הרי״ץ גיאות וא״כ להלכה נקטינן כדעת הרא״ש והר״ן דשיעור טפח הוא רק מדרבנן. וכן יש להוכיח ג״כ דלא כל השיעורים הל״מ הם מטפח דלולב דמה דבעי טפח אתי מהא דתנן כל לולב שיש בו ג׳ טפחים כדי לנענע בו כשר ומפרש בגמרא סוכה (דף ל״ב) וכדי לנענע בו דהיינו שצריך טפח יותר כדי לנענע הרי שהטפח הוא משום נענוע והנענוע עצמו לכאורה אינו מן התורה שהרי שם (דף מ״ב) אמרינן והא מדאגבהי׳ נפיק בי׳ ואי ס״ד דצריך נענוע מן התורה מאי קושיא הא אכתי טריד טרדא דמצו׳ כיון דלא נענע וכ״כ גם התוספ׳ שם (דף ל״ט) דנענוע אינו רק מכשירי מצו׳ בעלמא ולא מעכב כדאמרינן סוף פרקין מדאגבהי׳ נפיק בי׳ ע״ש וכיון דהנענוע עצמו אינו מחיוב נטילת לולב דאורייתא היאך אפשר שנמסר למשה מסיני שיעור טפח בלולב כדי לנענע ובזה יש ליישב מה שדקדק הר״ן אהא דתנן לולב שיש בו ג״ט וטפח כדי לנענע בו למה לא קתני סתם דלולב בעי ד׳ טפחים והר״ן תירץ דקמ״ל אם ההדס וערבה גבוהים הרבה צריך שיהא שדרו של לולב למעלה מהן טפח כדי לנענע והר״ן בזה לשיטתו אזיל דפסק כן כמשכ׳ הב״י סי׳ תר״ן בשמו וכנראה שיצא לו זה מן הדקדוק שבמשנה אבל הרי״ף והרמב״ם והרא״ש והטור וש״ע חולקין עליו וס״ל דלמעלה אין להם שיעור וא״כ לשיטה זו קשה כנ״ל למה לא קתני שלולב צריך ד׳ טפחים ולפי דברינו א״ש דע״כ צריך לחלק דשיעור דג׳ טפחים כנגד אורך הדס וערבה הוא הל״מ ומעכב אבל הטפח כדי לנענע בו אינו רק לכתחלה ולא מעכב כיון דהנענוע עצמו אינו מעכב ולכן אתי פסק רבינו ישעי׳ שפיר אפילו לדידן דפסקינן דעל כל הפסולים לא יברך מכ״מ על לולב של ג׳ טפחים יכול לברך דהטפח של נענוע החסר אינו מעכב אבל על פחות מג׳ טפחים לא יברך כיון דזה מדאורייתא פסול. וע״פ זה גם מי שאין לו רק שופר פחות מטפח יכול לברך עליו כיון דמדאורייתא אין לו שיעור כנ״ל א״כ יכול לקיים בו מצות שופר דאורייתא ואף דמדרבנן צריך שיעור טפח ויש כח לחכמים לעקור מצו׳ דאורייתא בשב ואל תעשה זה דוקא באמרו חכמים בפי׳ כן לעשות סייג לתורה אבל בסתמא היכא דאין יכול לקיים המצו׳ רק ע״פ מצות התורה ולא גם ע״פ מצות רבנן ודאי יכול לברך עליו כנלענ״ד. הקטן יעקב.\\\vspace{0pt}

\end{multicols}\newpage

\newchap{סימן לד}
\begin{multicols}{2}
ב״ה אלטאנא, תשרי תרט״ז לפ״ק.\\\vspace{0pt}

שאלה – חולה שיש בו סכנה שע״פ הרופאים צריך לאכול ולשתות ביוה״כ יותר מכשיעור אם צריך לשער בכל אכילה ושתיי׳ אם יש סכנה כשלא יאכל וישתה או אם נאמר דמצות תענו את נפשותיכם היא למערב עד ערב וכיון דלא יכול לקיים מצו׳ זו מפני הסכנה שוב אין קפידא אם יאכל וישתה באותו יום גם בלא סכנה.\\\vspace{0pt}

תשובה – גרסינן בירושלמי (פרק א׳ דתענית) הובא ברי״ף וברא״ש שם נדר להתענות ושכח ואכל כזית אבד תעניתו רב אמר בשם רבנן דתמן והוא שאמר יום סתם אבל יום זה מתענה ומשלים וכתב הרא״ש משמע ביום סתם הוא דאיבד תעניתו ולפי שיכול להתענות יום אחר תחתיו דלוה ופורע אבל יום זה אינו לוה ופורע לפיכך משלים אעפ״י שאכל כזית וכן פסק בטוש״ע א״ח (סי׳ תקס״ח) ומשמע מזה דנחשב תענית אפילו אכל דדוקא בנדר סתם כיון דיכול ללוות ופורע צריך להתענות יום אחר אבל ביום זה וכן בתענית צבור צריך להשלים תעניתו והתרומת הדשן (סי׳ קנ״ו) כתב אמנם בתענית צבור קבוע מגזירת חכמים וחובה להתענות אם שכח ואכל צריך להשלים ע״כ דמי שאכל שום וריחו נודף אינו אוכל עוד שום ויהא ריחו נודף יותר שהרי יום זה אסור באכילה מדרבנן ואין צריך לפרוע אחר תחתיו דדוקא יום זה חובה ולא אחר ואי אפשר לתקן את אשר עוות אם לא שכוונתו להתענות לכפרה על עונתו ושגגתו עכ״ל והנה כל זה שייך במי ששגג ואכל באיסור דזה ודאי מצוה שלא יאכל יותר כמו שכתב הת״ה שלא יאכל שוב שום וריחו נודף אבל במי שהוצרך לאכול מטעם סכנה אם גם הוא מצווה שלא לאכול יותר מהסכנה עדיין צ״ע ולכאורה יש להביא ראי׳ מדברי רש״י יומא (דף פ׳ ע״ב) דעל מה דאמרינן שם האוכל אכילה גסה ביוה״כ פטור פירש רש״י כגון שאכל לילי יום הכיפורים על השובע דאל״כ חייב על האכילה שקודם אכילה גסה ע״ש ולמה הוצרך לכך הרי יש נפקותא בהודע לו בנתיים כגון שאכל בשוגג והודע לו ושוב אכל שנית אכילה גסה שפטור על אכילה שניי׳ אלא ע״כ משמע דס״ל דבכה״ג אפילו לא היתה אכילה שניי׳ אכילה גסה מ״מ לא חייב עלי׳ עוד כיון דמשום יוה״כ אינו מעונה אכן יש לדחות זה ויש לומר דפשוט יותר לומר בכה״ג כפירוש רש״י שמיד אכילה הראשונה היתה אכילה גסה. וגם אין להביא ראי׳ ממה דתנן ביומא (שם) אכל ושתה בהעלם אחת אינו חייב אלא אחת משמע הא בשתי העלמות חייב שתים ואמאי הרי אינו מעונה בשתיי׳ אחר שכבר השיב דעתו בשכבר אבל די״ל דאכילה ושתיי׳ שאני דאף דשתיי׳ בכלל אכילה וגם בחמשה ענויים לא נחשב אכילה ושתיי׳ רק לחד עינוי מ״מ כל אחד הוי עינוי לחוד שהרי אכילה ושתיי׳ אינן מצטרפין ולכן אף שאכל י״ל דהוא מעונה עדיין לענין שתיי׳ אבל מי שאכל והשיב דעתו באכילה פעם אחת י״ל דלא שייך אצלו עוד תענו את נפשותיכם באכילה באותו היום. אכן יש להביא ראי׳ לכאורה דשייך איסור כרת גם כשאכל כבר ממה דאמרינן בר״ה (דף כ״א ע״א) לוי איקלע לבבל בחדסר בתשרי אמר בסים תבשילא דבבלאי ביומא רבא דמערבא אמרי לי׳ אסהיד אמר להו לא שמעתי מפי ב״ד מקודש וכתבו התוספ׳ וא״ת והאיך יניחם לאכול ויש כאן ודאי איסור כרת דאורייתא וי״ל התנן לקמן בפרק שני אשר תקראו אותם מקראי קודש בין בזמנן בין שלא בזמנן ואמרינן בגמרא אתם אפילו שוגגים אתם אפילו מזידים ואפילו מוטעים עכ״ל ומדכתבו התוספ׳ ויש כאן ודאי איסור כרת אף שכבר אכלו משמע שבכל אכילה ואכילה יש איסור כרת אכן יש לדחות זה דהתוספ׳ שם הביאו שני פירושים לפירוש הראשון בא לוי לבבל באחד עשר לבני בבל ועשירי לבני א״י ולפירוש השני בא לבבל ביוה״כ של בני בבל סמוך לחשיכה והוא הי׳ ערב יוה״כ לבני א״י ואמרו לו אסהיד לן ונאסור לאכול גם עתה דיוה״כ הוא ועל זה אמר להם לא שמעתי מפי ב״ד מקודש וא״כ י״ל דלפירוש הראשון שכבר אכלו באחד עשר שלהם באמת לא קשה איך יניחם לאכול באיסור כרת אבל קושיתם היא לפירושם השני שבא ביוה״כ סמוך לחשיכה ורצו להתענות עוד יותר מעתה עד למחר בערב ובזה הקשו התוספ׳ האיך יניחם לאכול ויש כאן ודאי איסור כרת שהרי אז לא אכלו עדיין וכן משמע קצת מהמשך דברי התוספ׳ שם.\\\vspace{0pt}

אמנם לכאורה יש להביא ראי׳ שמי שאכל ביוה״כ שוב ליכא באכילה שניי׳ איסור כרת ממה דאמרינן כריתות (דף י״ח) דעל מה דסבירא לי׳ רבי זירא שם ידיעות ספק מחלקות לחטאת אמר לי׳ רבא בר חנא לאביי מה אילו אכל חלב כזית שחרית ביוה״כ וכזית חלב במנחה ביוה״כ ה״נ דמחייב שתי חטאות (פירוש דיוה״כ במקום אשם תלוי קאי) אמר לי׳ אביי ומאן לימא לן דיוה״כ כל שעתא מכפר דילמא כולה יומא מאורתא ע״ש והלא יפלא וכי למשכח חיוב שתי חטאות באכילת יוה״כ צריך למינקט חלב הלא גם באכילת היתר יש חיוב ביוה״כ אלא ע״כ מוכח דכיון שכבר אכל שחרית ביוה״כ ליכא עוד חטאת באכילה במנחה, אבל מה אעשה שהתוספ׳ לא סבירא להו כן שדחקו לתרץ דלכך נקט חלב ולא היתר דא״כ הוי צריך למינקט ככותבת שהוא שיעור אכילת יוה״כ לכך נקט כזית חלב שהוא פחות הרי שפשיטא להו להתוספ׳ דאע״פ שאכל שחרית מ״מ יש חיוב כרת וחטאת למי שאכל שנית ביוה״כ ובאשר לא מצאתי חולק על זה צריך לחוש לדבריהם ולכן צריך לשער באכילת חולה ביוה״כ שלא להאכילו בין כשיעור שיש בו איסור כרת ובין פחות מכשיעור שעכ״פ אסור מן התורה רק מה שמספיק להוציאו מספק סכנת נפשות וקשה מאד לשער כן בכל אכילה ושתיי׳ שיותן לחולה ותמהתי שלא ראיתי לאחד מהפוסקים ראשונים ואחרונים שהעיר על זה במה שנוגע לאיסור חמור דכרת וצריך זהירות ובקיאות יותר מעיקר היתר אכילה שלזה יספיקו דברי הרופא שצריך להאכילו דרך כלל אבל בענין זה צריך לשער דרך פרט בכל אכילה ואכילה. כנלענ״ד הקטן יעקב.\\\vspace{0pt}

\end{multicols}\newpage

\newchap{סימן לה}
\begin{multicols}{2}
אלטאנא, תשרי תרי״ט לפ״ק.\\\vspace{0pt}

ראיתי לחקור מי שאכל ביוה״כ ככותבות בשר חי אם חייב עליו ולכאורה יש ראי׳ דפטור דהרי כלל אמרו במתניתן דיומא (פ׳ ח׳) אכל אוכלין שאין ראויין לאכילה פטור ובפסחים (דף כ״ד) אמרינן אכל חלב חי פטור וכתבו התוספ׳ והא דאמר בחולין (דף ק״ב) אכל צפור טהורה בחיי׳ בכל שהו במיתתה בכזית שאני עוף שהוא רך וחזי לאומצא וחשיב כדרך הנאתן עכ״ל הרי שרק בשר עוף שהוא רך מקרי חזי לאכילה כשהוא חי משא״כ בשר בהמה חי וכן מוכח ממה דאמרינן שבת (דף קכ״ח) שאני בר אווזא דחזי לאומצא ופי׳ רש״י כשהוא חי אוכלין אותו בלי מלח הרי בפירוש דרק בר אווזא דרכיך קרי׳ לי רב חסדא שם חזי לאומצא אבל שאר בשר לא שהרי אסר לטלטל בשר תפל וזה ע״כ משום דלרוב בני אדם לא חזי לאכילה שהרי מה דחזי לאחד מותר בטלטול בשבת לאחר שישראל מותר לטלטל תרומה כיון דחזי לכהן ואעפ״כ חשיב רב חסדא בשר תפל לא חזי והיינו ע״כ משום דמי שאוכל אותו חי באומצא בטלה דעתו אצל כל אדם וא״כ ודאי גם לענין יוה״כ מקרי אוכל שאינו ראוי לאכילה ופטור ואין להביא ראי׳ ממה דאמרינן ביומא (דף פ׳) אמר רב פפא אכל אומצא ומילחא מצטרף ואע״ג דלאו אכילה היא כיון דאכלי אינשי מצטרפין הרי דחשיב אומצא בר אכילה שחייב עליו ביוה״כ דיש לומר דהתם איירי באומצא דעוף דרכיך שלא רצה להשמיענו אלא דמלח אף דלאו בר אכילה הוא מצטרף לאומצא או י״ל דרק בשר חי בלא מלח מקרי אינו ראוי לאכילה אבל ע״י מלח שנחשב כרותח נעשה ראויי׳ לאכילה קצת וכשנראה מדברי רש״י בשבת הנ״ל ואעפ״כ קרא התנא במנחות להבבליים שאכלו שעיר יוה״כ בשבת חי שדעתם יפה אף שמסתמא אכלו אותו במלח. ובהכי הי׳ נראה לי ליישב מה דאמרינן בכריתות (דף ז׳) כרת דיומא לרבי דס״ל דיוה״כ בכל שעתא ושעתא מכפר אפילו בלא שב היכי משכחת לה ומשני דילמא בהדי דקאכל נהמא חנקתיה אומצא ומית דלא הוי לי׳ שעות ביומא דלכפר לי׳ וקשה דלמאי צריך בהדי דקאכיל נהמא חנקתיה אומצא ולא אמר בקיצור דחנקתיה אומצא ומית ואף דבסוגיא דשבועות באמת נאמר כן קרוב יותר לומר דשם הוגה כן מפני קושיא זו משנאמר דהכא נוסף הך דקאכיל נהמא בטעות אבל על פי הנ״ל נראה דגירסא דכריתות עיקר דזה ודאי מה דנקט דחנקתי׳ אומצא היינו כיון דאומצא הוא בשר חי שלא נתרכך ע״י בישול ומלח הוא דבר קשה ודרך להחנק בו יותר מבשאר מאכלים אבל יקשה האיך יתחייב על אכילת אומצא כזה משום יוה״כ הרי הוי אוכלין שאינן ראויין לאכילה לזה נקט דקאכל נהמא שדרך לאוכלה עם אומצא כנראה בחולין (דף ק״ז) בתוספ׳ והחיוב הוא משום הלחם מכל זה הי׳ נלענ״ד דאוכל בשר בהמה חי בלא מלח ביוה״כ פטור.\\\vspace{0pt}

אמנם אחר עיון מצאתי בשאגת ארי׳ (סי׳ ע״ו) שכתב איפכא מזה שהביא שם קושית התוספ׳ בכריתות (דף כ״ג) שהקשו אמה דאמרינן בגמרא שם ומי אמר רבי שמעון איסור חל על איסור והתניא רבי שמעון אומר האוכל נבילה ביוה״כ פטור דמאי פריך דילמא שאני יוה״כ משום הכי לא חל על איסור נבילה משום דאינו רק איסור כולל אבל הכא גבי נותר הוי לי׳ איסור מוסיף דחמיר ועל זה כתב השאגת ארי׳ ולדידי לא קשה מידי דע״כ רבי שמעון לית לי׳ איסור מוסיף דיוה״כ הוי איסור מוסיף לגבי איסור נבלה משום דקיי״ל בפרק כל שעה כל איסורים שבתורה אין לוקה עליהם אלא דרך אכילתן וטעמא משום דאכילה כתיב בהו והא ודאי גבי יוה״כ דלא כתיב בי׳ אכילה אלא כל הנפש אשר לא תעונה אפי׳ אם אכלן שלא כדרך אכילתן כגון אוכל חלב חי וכש״כ בשר בהמה חי כמשכ׳ התוספ׳ שם אפילו הכי חייב משום לא תעונה ביוה״כ וביומא (דף פ׳) לא פטרינן אלא באוכל אכילה גסה ביוה״כ משום דכתיב אשר לא תעונה פרט למזיק אבל בשלא כדרך אכילתן דאינו מזיק הוא ודאי חייב הרי איסור יוה״כ הוי איסור מוסיף אחפצא דנבילה שלא כדר״א שריא משום דאכילה כתיבי בה ואם אכלו חי פטור ומשום יוה״כ חיב ומדפטר ר״ש אוכל נבילה ביוה״כ ש״מ דאפילו באיסור מוסיף לית לי׳ אחע״א עכ״ל השאגת ארי׳ ולענ״ד דעת התוספ׳ כמו שכתבתי דממה דאמרינן אכל אוכלין שאינן ראויין לאכילה פטור גם באוכל שלא כדרך אכילתן כגון בשר חי פטור ואין לומר דזה דוקא מה שאינו ראוי להיות אוכל לעולם כגון שאכל עשבים המרים כמו שפי׳ הרמב״ם דזה אינו דממה דקרי להו אוכלין דקאמר אכל אוכלין שאינן ראויין לאכילה משמע שאוכלין הם באמת רק כשהם עתה אינן ראויין לאכילה ובכלל זה גם אוכל בשר חי ומה דהוכיח השאגת ארי׳ דחייב גם באוכל שלא כדרך אכילתן ביוה״כ מדלא כתיב אכילה ביוה״כ לענ״ד אינו ראי׳ דהא כתיב אשר לא תעונה ומזה ילפינן דבעינן דבר דמייתב דעתי׳ אם רב אם מעט כדאמרינן ביומא (דף פ׳) מתקיף לה ר׳ זירא בשר שמן בככותבות ולולבי גפנים בככותבות אמר לי׳ אביי קים להו לרבנן דבהכי מייתבה דעתא בציר מהכי לא מייתבי דעתא מיהו בשר שמן טובא לולבי גפנים פורתא ע״ש ולכן כל שאוכל אוכל שעתה אינו ראוי לאכילה אינו בכלל אשר לא תעונה דלא מייתב דעתי׳ כמו אוכל עשבים מרים ואין להביא ראי׳ דבשר חי מקרי אכילה שהרי לענין הכשר אוכלים כתיב מכל האוכל אשר יאכל אשר יבא עליו מים דנקרא אוכל ואעפ״כ בשר אפילו חי מקבל הכשר כדמוכח ממה דאמרינן פסחים (דף כ״ב) כגון שהי׳ פרה של זבחי שלמים והעבירה בנהר ושחטה ועדיין משקה טופח עלי׳ וכן מוכח בכמה דוכתי׳ דדבר שעתה אין ראוי לאכילה מכ״מ כשיהי׳ ראוי לבסוף מקרי אוכל והכי אמרינן בעירובין (דף כ״ח) ולענין טומאת אוכלים שאני כדאמר ר״י הואיל וראוי למתקן ע״י האור די״ל דלענין טומאת אוכלים דכתיב מכל האוכל אשר יאכל דמשמע כל שיאכל לבסוף לא בעינן רק שיהי׳ שם אוכל עליו וזה ודאי דבשר חי אוכל נקרא רק שנקרא אוכל שאין ראוי לאכילה ואינו מייתב דעתי׳. כן הדעת נוטה ועכ״פ מדפשיטא להו להתוספ׳ דיוה״כ על נבילה לא הוי מוסיף נראה דדעתם כן דאוכל שלא כדרך אכילתן מקרי לא מייתבי דעתי׳ ופטור ויש נפקותא במי שאחזו בולמוס שמאכילין אותו עד שיאורו עיניו אם יש להסב הסכנה ממנו ע״י אכילת בשר חי וכדומה שאסור להאכילו דבר שחייבין עליו כרת. כנלענ״ד הקטן יעקב.\\\vspace{0pt}

\end{multicols}\newpage

\newchap{סימן לו}
\begin{multicols}{2}
ב״ה אלטאנא, יום ג׳ כ״ד אלול תרכ״ו לפ״ק. לחתני הרה״ג וכו׳ מ״ה יוסף איזאקזאהן נ״י אב״ד דק״ק ראטטערדאש יע״א.\\\vspace{0pt}

על מה שכתבת – לא ידעתי על מה נסמך המנהג לומר בראש השנה ויוה״כ וי״ט בשעת הוצאת הס״ת ג׳ פעמים י״ג מדות וג׳ פעמים ואני תפלתי הרי לפי פ׳ רש״י בהא דאמרינן האומר שמע שמע הרי זה מגונה דזה באמר מלה וכופלו אבל בשכופל הפסוק משתקין אותו ולפי׳ רב אלפס עכ״פ הרי זה מגונה ובשניהם דהיינו בי״ג מדות וגם באני תפלתי לא שייך הטעם שכתב רש״י בסוכה (דף ל״ח) שכופלין בהלל מאודך ולמטה כיון דכל ההלל כפול וגם לא טעם הרשב״ם בפסחים בזה משום דאמרו ישי ודוד ושמואל ואולי רק בשמע ומודים הוי מגונה או משתקינן מטעם דנראה כב׳ רשויות אבל הכפלת שאר פסוקים שרי עכ״ד.\\\vspace{0pt}

אשיב – הטור בא״ח (סי׳ ס״א) כתב האומר שמע שמע פי׳ שני פעמים משתקין אותו שנראה כשתי רשויות ואיתמר בירושלמי דווקא בצבור אבל ביחיד שרי ונראה שאין לחלק דבגמרא דידן אינו מחלק וכתב הב״י בשם הר״ר יונה דלפי פי׳ רש״י קשה מה שנהגו לכפול בסליחות פסוק של שמע ישראל אלא שאומרים שכיון שמנהג אבותיהם בידיהם מכמה שנים באמרו אותו כל הקהל מוכחא מילתא שאינם עושים מכוונת שתי רשויות ולא חיישינן להכי וכתב הב״י ומה שטען להתיר מטעם שאומרים אותו כל הקהל קשה בעיני דבירושלמי שכתב רבי׳ בסמוך משמע איפכא דקאמר דבצבור דוקא חששו לשתי רשויות וא״כ כל שאומרים אותו הקהל איכא למיחש טפי ושמא י״ל דהתם כשאומר אותו יחיד בצבור אבל כשהקהל כולו אומרים אותו ליכא למיחש למידי ולענין הלכה נקטינן כהנך רבוותא דאסרי עכ״ד וכן פסק גם בש״ע לאיסור וכתב עליו הב״ח דמשמעות הסוגיא פ׳ אין עומדין דוקא ביחיד האומר לפני רבים משתקין משום דמחזי כשתי רשויות אבל כשכל הצבור אומרים לתקוע בלב כל ישראל שימסור נפשו על קדושת שמו יתעלה טוב הדבר מאוד לכפול ג׳ פעמים דג׳ פעמים הוי חזקה ועל זה וכיוצא בזה לא אמרו חכמים שלא לומר שמע שמע ועל כן אין לשום חכם וגדול לבטל המנהג במקומות ומדינות שנהגו כן ודלא כמשכ׳ ב״י כאן ופסק כן בש״ע ללמד אותם שלא יאמרוהו עכ״ל וכתב עליו המג״א דאישתמיטתי׳ גמרא פ״ד דסוכה שהיו אומרים אנו לי׳ ולי׳ עינינו ופריך בגמרא והא אמרינן האומר שמע שמע משתקין אותו והתם כולם היו אומרים אותה אלא ע״כ אפילו כל הקהל אומרים אותה אסור עכ״ל גם האלי׳ רבה הקשה קושיא זו ונדחק ביישובה ולענ״ד דבריו תמוהים במה שכתב כולם היו אומרים אותה דהרי תנן סוכה (דף נ״א) עמדו שני כהנים בשער העליון שיורד מעזרת ישראל לעזרת נשים וכו׳ תקעו והריעו ותקעו וכו׳ היו תוקעין והולכין עד שמגיעין לשער היוצא ממזרח הגיעו לשער היוצא ממזרח הפכו פניהם ממזרח למערב ואמרו אבותינו שהיו במקום הזה אחוריהם אל ההיכל ופניהם קדמה ומשתחוים קדמה לשמש ואנו לי׳ עינינו ומשמע מזה דרק הכהנים שירדו מעזרת ישראל לעזרת נשים אמרו כן בפני כל ישראל שנאספו בעזרת נשים ואפילו נאמר דגם שאר כהנים ולוים ואנשי מעשה אמרו עכ״פ אין רמז שם שכל ישראל אנשים ונשים שנאספו בעזרת נשים לשמחת בית השואבה שכולם אמרו ולכן פריך שפיר הרי אמרינן האומר שמע שמע משתקין אותו כיון שלא היו רק יחידים שאמרו בפני כל הקהל כן איכא חשדא אבל מה ראי׳ מזה שגם אם יאמרו כל הקהל יחד איכא חשדא דשתי רשויות דמי יחשוד את מי ולכן לענ״ד סברת הב״ח נכונה ואם כי אין לפסוק נגד פסק הש״ע עם כל זה בדברים שנהגו כן אין לבטל המנהג ואולי מה שכופלים י״ג מידות ואני תפלתי נתייסד המנהג על פי רבינו יונה והב״ח שהחזיק בדעתו אמנם לענ״ד ליישב מנהג זה אין צורך לזה שאף שמה שרצית לחלק בין שמע שמע ומודים מודים ובין שלש עשרה מידות דדוקא בהנך משתקינן משום דנראה כב׳ רשויות אינו מבורר לי כ״כ דגם בי״ג מידות ובאני תפלתי יש לדאוג שיחשדו שנאמר כן לב׳ רשויות שהרי גם על אנו לי׳ ולי׳ עיננו פריך הגמרא משמע שמע ומודים מודים בלא״ה נראה לי לחלק דדוקא באומר שני דברים שווים דרך תחנה ובקשה או דרך שבח ותהלה שייך החשד דב׳ רשויות לא כן בקורא פסוקי תורה וכתובים וראי׳ לזה שהרי מצוה לחזור הפרשה שנים מקרא וע״פ האר״י יש לכפול כל פסוק ופסוק וקורין הפסוק שמע ישראל ב׳ פעמים זה אחר זה ואין קפידא וכיון דמה שמזכירין י״ג מידות אין זה דרך תחנה דא״כ יהי׳ אסור לאומרם ביו״ט אלא ע״כ לא אומרים רק כקורא פסוק בתורה וכן בואני תפלתי ע״כ אין בזה משום קורא שמע וכופלו. כנלענ״ד הקטן יעקב.\\\vspace{0pt}

\end{multicols}\newpage

\newchap{סימן לז}
\begin{multicols}{2}
ב״ה אלטאנא, יום ו׳ ה׳ תשרי תרי״ד לפ״ק.\\\vspace{0pt}

ראיתי בספר מטה אפרים להגאון הרב מ״ה אפרים זלמן מרגליות זצ״ל שחקר במי שאין לו לאכול בלילה ראשונה בסוכה רק פת של איסור כגון של טבל ושל כלאי כרם מה יעשה ושלכאורה זה תלי בשני תירוצי התוספ׳ דקידושין (דף ל״ח) דמה שלא אכלו מצה של חדש דתבא ע׳ ותדחה ל״ת תרצו בשם הירושלמי דמצה הוי ע׳ דלפני הדבור ולא דחי ל״ת דלאחר דבור עוד תרצו דגזירה כזית ראשון אטו כזית שני והשתא לפי תי׳ שני דהתוספ׳ גם בסוכה אסור אבל לפי תי׳ ראשון דבמצה לא דחי דהוי ע׳ דלפני הדבור זה לא שייך בסוכה דהוי ע׳ דלאחר דבור ודחי מדאורייתא אלא שיש לדחות כיון דכל עיקרה דע׳ דסוכה ממצה גמרינן א״כ י״ל דיו להלמד מגז״ש להיות כמלמד וכיון דבמצה לא דחי ה״ה בסוכה ולכן אף אם עבר ואכל הוי מצו׳ הבאה בעבירה ולא יצא ומסיק לדינא נראה שאין להקל לאכול מצה של איסור עכ״ד והנה במה שמסיק הגאון זצ״ל דלכתחלה אין לאכול ודאי הדין עמו דאפילו נימא דמדאורייתא דחי מ״מ יש איסור מכח תי׳ התוספ׳ דגזירה כזית ראשון אטו כ״ש ומי יכריע שלא לחוש לדברי התוספ׳ שאין חולק עליהם בפי׳ מהראשונים אבל ראיתי לחקור איך הדין בדיעבד כגון שאכל כזית מפת של איסור שסבר שלא ימצא עוד היתר ואח״כ מצא אם צריך לאכול שוב כזית מפת של היתר ואם שוב יברך ברכת ל״ב או לא יברך ונלענ״ד שהספק של המטה אפרים תלי בפלוגתא דתנאי דהיינו אם דבר דילפינן בגז״ש דיינינן לגמרי כאותו דבר דגמרינן מיניה או אם נימא דאחר דמייתי לי׳ בגז״ש הוי כאלו נכתב בפי׳ ודיינינן לי׳ כאותו דבר שנכתב אצלו בזה פליגי ר׳ אליעזר ור׳ יהושע בחולין דף (ק״ך) אם אמרינן דון מינה ומינה או דון מינה ואוקי באתרא ע״ש וכן אמרינן ג״כ ביבמות (דף ע״ח) לא קשיא הא כמ״ד דון מינה ומינה הא כמ״ד ד״מ ואוקי באתרה ע״ש הרי אע״ג דכל איסור ממזר לאחר עשרה דורות לא אתי רק מעמוני ומואבי ושם נקבות מותרות אעפ״כ לא אמרינן למ״ד ד״מ ואוקי באתרה דיו ללמד מגז״ש להיות כמלמד להיות נקבות מותרות אלא אמרינן אחר דילפינן בגז״ש הוי כאילו נכתב אחר עשרה דורות בפי׳ גבי ממזר ונדון כדין ממזר וא״כ ה״נ אחר דגמרינן בגז״ש דחמשה עשר מחג המצות דצריך לאכול בסוכה הוי כאילו כתוב זה בפי׳ גבי סוכה וא״כ הוי ע׳ דלאחר הדבור שדחי ל״ת אכן למ״ד דון מינה ומינה ודאי ששו׳ לגמרי למצה וכיון דפסקינן כמ״ד דון מינה ואוקי באתרה דסתם מתניתן דיבמות דפסקינן כוותי׳ סובר כן וכן מדפליגי ר״י ור״א ור״י דפסקינן כוותי׳ ס״ל ד״מ ואוקי באתרה וכן פליגי בפלוגתא זו ר״מ וחכמים בשבועות (דף ל״א) וחכמים דפסקינן כוותייהו ס״ל דאוקי באתרה א״כ ודאי דהלכה כן ולפ״ז גם בספק דסוכה אמרינן להלכה דאוקי באתרה ודיינינן לי׳ כעשה דלאחר הדבור ואמרינן דמן התורה דוחה ל״ת ורק מדרבנן אסור לאכול כזית מפת של איסור ולכן אם אכל בשוגג לא צריך לאכול שוב כיון דמדאורייתא יצא ומדרבנן לא אסור רק לכתחלה. אכן י״ל כיון דאם הי׳ יודע שימצא עוד היתר לא הי׳ דוחה לל״ת דאפשר לקיים שניהם א״כ כשמצא לבסוף למפרע אפילו מדאורייתא לא יצא כדאמרינן יבמות (דף כ׳) דבאפשר לקיים שניהם אם בעלו לא קנו ועדיין יש לחלק דהתם בשעת דחייה הי׳ אפשר לקיים שניהם משא״כ בנדון זה שבשעה שאכל לא הי׳ לו היתר וסבר שלא ימצא עוד ולכן עדיין צ״ע אבל מכ״מ נלענ״ד ברור כשלא מצא לבסוף שיצא מן התורה. הקטן יעקב.\\\vspace{0pt}

\end{multicols}\newpage

\newchap{סימן לח}
\begin{multicols}{2}
ב״ה אלטאנא, יום ו׳ י״ז מרחשון תרי״ד לפ״ק. להרב וכו׳ מ״ה מאיר ליב לעבוואהל נ״י מק״ק קראקא יע״א.\\\vspace{0pt}

על מה שכתבתי בספק אם יוצא בליל ראשון ש״ס בפת של טבל דלמ״ד דון מינה ואוקי באתרא יוצא כתב מעכ״ת נ״י וז״ל הנה בענין דון מינה ומינה ראיתי מבוכה גדולה וכבר עמדו על זה גדולים ה״ה הלח״מ פי״ו מה׳ מעה״ק הי״ד והתוי״ט פ״י דזבחים מ״ט ובפי״ג דמנחות מ״ה והתשב״צ ביבין שמועה בפ׳ איזהו מקומן דהא בכמה מקומות דפליגי תנאי בדון מינה ומינה פסק הרמב״ם אוקי באתרא ביבמות ע״ח ופ׳ הרמב״ם בפ׳ ט״ו מהל׳ א״ב אוקי באתרא וב״ק ד׳ כ״ה ופ׳ הרמב״ם ז״ל פכ״ג מה׳ כלים ה״א שבועת ד׳ ל״א ופ׳ הרמב״ם פ״ט מהל׳ שבועת הל״ב בכל אלו מקומות פסק הרמב״ם אוקי באתרא אמנם בפי״ז ופי״ז מהל׳ מעה״ק פסק כרבנן דרבי זבחים דף צ״א דדון מינה ומינה עכ״ל וגם העירו מנדה דף מ״ג ע״ב דאיתא שם אמר ר׳ חנילאי משום ראב״ש ש״ז נוגע בכעדשה וכן פ׳ הרמב״ם פ״ה מהאה״ט ואיתא שם בגמרא דזה הוא דוקא למ״ד דון מינה ומינה. גם בספר בה״ז מס׳ מנחות ק״ז העיר סתירה בדברי הרמב״ם ממנחות ס״ב דלפי פסק הרמב״ם שם משמע ג״כ דון מינה ומינה ולפענ״ד נעלם ממנו במחכ״ה דרבינא אמר שם דכ״ע אוקי באתרא וכו׳ יעי״ש והרמב״ם פסק כרבינא דהוא בתרא יעי״ש. ובספר מרכבת המשנה פט״ז מהל׳ מעה״ק הי״ד ר״ל דהרמב״ם ס״ל באמת דון מינה ומינה ומה שפסק בשבועות הפקדון כרבנן רוצה לישב לפי דרכו שם יעי״ש ונעלם ממנו במח״כ כל הני מקומות שהבאתי לעיל דהרמב״ם פסק אוקי באתרא, ולתרץ קושיות אלו שהקשו על הרמב״ם הנ״ל נלפענ״ד דהרמב״ם באמת ס״ל אוקי באתרא והא דפסק הרמב״ם שמן לא יפחות מלוג משום דאי אמרינן דבשמן בעי ג׳ לוגין א״כ למ״ל קרבן גבי מנחת נדבה לומר שמתנדבין שמן הא מצינו למילף מאזרח כי היכי דגמרינן יין וכמו שסבר רב פפא לומר אליבא דר׳ דמאזרח גמר א״ו מדכתבי רחמנא קרבן על שמן להורות דליגמר ממנחה לכל מילי מה מנחה בלוג אף שמן בלוג וא״ל הא בלא״ה איצטרך קרבן ללמד שמתנדבין עצים כדאיתא במנחות כ״א דע״כ עיקר קרא דקרבן איצטרך משום שמן דז״ל הגמרא שם קרבן מנחה מלמד שמתנדבין עצים ר׳ אומר וכו׳ ואמר רב לדברי ר׳ עצים צריכים קמיצה והקשה תוספ׳ בד״ה לדברי וא״ת מנלן דמקשינן למנחה דנקמצת האיכא מנחות דאינן נקמצות וי״ל דלגבי מנחה הנקמצת כתיב עכ״ל התוספ׳ והשתא לפ״ז לרבנן דסברי דעצים אינן טעונין קמיצה ליכא למימר דעיקר קרבן כתיב משום עצים דא״כ למה כתבה רחמנא גבי מנחת נדבה טפי ה״ל למיכתב גבי שאר מנחות דאינן נקמצות דלא למיטעי דגם עצים טעון קמיצה א״ו דעיקר קרא לשמן אתי ולהורות דהכי דון מינה ומינה ובזה י״ל דרבי ורבנן בזבחים (דף צ״א) אזלו לשיטתם במנחות (דף כ״א) אבל בעלמא גם רבנן מודים דאוקי באתרא, מכל הנ״ל יש לעשות סמוכין לדברי מעכ״ת נ״י אך עדיין נשאר לבאר מה דפסק הרמב״ם פ״ה מהל׳ אה״ט ש״ז נוגע בכעדשה דמוכח דס״ל דון מינה ומינה וגם יש לי עיון בדברי מעכ״ת נ״י מדברי תוספ׳ פסחים ל״ב ד״ה ומינה דכתבו דהכא אין לומר אוקי באתרא דכיון דאכילה דהקדש מאכילה דתרומה יליף והאיך יהי׳ בו מיתה אם לא בכשיעור אכילת תרומה דבשאר נהנה ליכא מיתה עכ״ל התוספ׳ יעי״ש וא״כ הכא ג״כ איך שייך לומר אוקי באתרא הא כיון דכל החיוב של לילה ראשונה אינו אלא ט״ו ט״ו מחג המצות. וא״כ איך יהי׳ חייב לאכול מה שאינו מצוה ואינו רשאי לאכול בחג הפסח עצמו איברא דבתוספ׳ מנחות (דף ס״ב) משמע דפליגי אמ״ש בפסחים אבל עכ״פ לפי דברי תוספ׳ פסחים (דף ל״ג) יש להעיר קצת עכ״ד מעכ״ת נ״י.\\\vspace{0pt}

ועל זה אשיב: מעכ״ת נ״י העיר בטוב טעם על כל המקומות שנזכרה פלוגתא זו דדון מינה ומינה או אוקי באתרה ומסיק ג״כ כמו שכתבתי דהלכה ד״מ ואוקי באתרה וכן מצאתי בספר כריתות שכתב דהלכה כמ״ד ד״מ ואוקי באתרה והנה לא מצאנו ברמב״ם סתירה לזה רק בב׳ מקומות במה שפסק כרבנן דרבי דשמן לא יפחות מלוג ובמה שפסק דש״ז נוגע בכעדשה והך דשמן לא יפחות מלוג רצה מר נ״י לבאר אבל עדיין נשאר לו הקושיא מהא דנוגע בכעדשה אכן לענ״ד הנכון ביישוב קושיות האלה על הרמב״ם ע״פ מה שכתב הברכת הזבח בשם חתנו ר״י כהן ז״ל דס״ל להרמב״ם דמה דאתיא בגז״ש אוקי באתרה אבל מה דאתיא בהיקש ס״ל דדון מינה ומינה ע״ש ואף שהוא לא ביאר מנ״ל להרמב״ם כן לענ״ד י״ל דהוקשו לו הסתירות דרבנן אדרבנן דקיי״ל אין הלכה כרבי מחביריו וקיי״ל דר״מ ורבנן הלכה כרבנן וכן ר״א ורבי יהושע הלכה כרבי יהושע ולכן ס״ל להלכה דבהיקש דון מינה ומינה וא״כ שפיר י״ל דרבנן דרבי דס״ל דלא יפחות מלוג ס״ל כחכמים דר״מ וכר׳ יהושע ובזה מתורץ ג״כ הקושיא מהא דנוגע בכעדשה (שלא הזכירה הברה״ז שם) אבל לפי הכלל הזה גם זה א״ש דכיון דמרבינן מאו דגבי שרץ נוגע בש״ז א״כ היקש הוא ובהיקש פסק הרמב״ם דהלכה דדון מינה ומינה אבל במה דאתיא בגז״ש פסק דאוקי באתרא ולכן מה שכתבתי שע״פ מה דפסקינן ד״מ ואוקי באתרא לענין סוכה עשה דוחה ל״ת א״ש שהרי התם גז״ש הוא דילפינן חמשה עשה מחג המצות.\\\vspace{0pt}

אבל לכאורה מצאתי סתירה לדברי מדברי הריטב״א בסוכה (דף כ״ז) דאמה דפליגי שם רבי אליעזר וחכמים דלר״א מי שלא אכל בסוכה ליל י״ט הראשון ישלים בליל י״ט האחרון ולרבנן אין לו תשלומין כתב הריטב״א וז״ל וכ״ת דמודה ר״א וכו׳ וא״כ מנ״ל שיש לו תשלומין ובמאי פליגי וי״ל דרבנן סברי דגמרינן מחג המצות לגמרי דסברי דון מינה ומינה דמה התם חייב ואין לו תשלומין אף כאן חייב ואין לו תשלומין ורבי אליעזר סבר דון מינה דחייב באכילה ואוקים באתרה דהתם דאתקש לפסח אין לו תשלומין כפסח אבל הכא דמיא לחגיגה שיש לה תשלומין כל הרגל ולילי י״ט האחרון עכ״ל והרי הרמב״ם פסק כחכמים דר״א דאין לו תשלומין וא״כ ע״כ בהך גז״ש עצמה דט״ו ט״ו מחג המצות פסק דון מינה ומינה ונסתרו דברי אבל באמת זה אינו דיש להוכיח דר״א וחכמים לאו בהכי פליגי דאיך אפשר דטעם דרבי אליעזר דס״ל דון מינה ואוקי באתרא שהרי ר״א ס״ל בחולין (דף ק״ך) דון מינה ומינה כדמוכח בסוגיא דשם ע״ש ובחידושי לסוכה בארתי במה פליגי ר״א וחכמים לענין תשלומין ועכ״פ אין ראי׳ מזה נגד מה שכתבתי דלדידן ד״מ ואוקי באתרא ודברי הריטב״א בסוכה צע״ג. ומה שהשיב מעכ״ת נ״י עוד על דברי מדברי התוספ׳ פסחים (דף ל״ג) ד״ה ומינה הנה מלבד מה שהעיר בעצמו שהתוספ׳ במנחות לא ס״ל כמש״כ בפסחים לענ״ד בלא״ה אין לו דמיון לשם שהרי ענין אוקי באתרא הוא שאנו רואין הדבר הנלמד בגז״ש כאלו נכתב אצל הלמד בפי׳ ולכן בפסחים כתבו התוספ׳ שפיר דאין לומר אוקי באתרא כיון דכל דין מיתה לא יהי׳ רק באוכל ולא בנהנה וא״כ אפילו הוי כתיב במעילה בפי׳ דאוכל במיתה ג״כ הוי דיינינן דאכזית דוקא חיוב מיתה דסתם אכילה בכזית אבל הכא לענין סוכה אם אמרינן אוקי באתרא אנו רואין כאלו כתיב בסוכה עצמו העשה דאכילה ואז ודאי דוחה ל״ת דעשה דלאחר הדבור הוא לכן שפיר אמרינן להלכה כיון דפסקינן בגז״ש כמ״ד אוקי באתרא דאם אכל טבל בסוכה יצא דעשה דוחה ל״ת כנלענ״ד הקטן יעקב.\\\vspace{0pt}

\end{multicols}\newpage

\newchap{סימן לט}
\begin{multicols}{2}
ב״ה אלטאנא, יום ו׳ כ״ד אדר תרי״ד לפ״ק. לאחי הרה״ג וכו׳ מ״ה ליב נ״י אב״ד דגליל לאדענבורג.\\\vspace{0pt}

על הספק שהערתי בשו״ת (סי׳ ל״ז) אם נקרא אפשר לקיים שניהם באכל בסוכה כזית טבל בחשבו שלא ימצא היתר מטעם עשה דוחה ל״ת שהרי אפשר שימצא עוד וענין הספק אם בשעת הדחייה אי אפשר לקיים שניהם אם נאמר שצריך להמתין שמא אפשר למצוא עוד היתר ויקיים העשה מבלי דחיית הל״ת או נאמר כיון דעתה אי אפשר לקיים שניהם דוחה העשה לל״ת ולא צריך להמתין –.\\\vspace{0pt}

על זה הערת להביא ראי׳ שאפילו יודע ודאי שאחר כך יקיים שניהם אם אי אפשר לקיים שניהם לאלתר אמרינן דעשה דוחה לל״ת ממה שכתב השיטה מקובצת בביצה (דף ח׳) בשם הרשב״א שהקשה אמה דאמרינן שם דאי עביד כתישה ביו״ט לכסות דם עשה דוחה ל״ת דאמאי דחי הא אפשר להמתין עד לערב ויכסה ותירץ כיון שאי אפשר לקיים לאלתר ודאי דחי שכן מוכח ג״כ ממה דאצטריך קרא שאין שריפת קדשים דוחה יו״ט ואיך ס״ד דדחי הא אפשר להמתין עד למחר ולשרפו אלא ע״כ כל שאי אפשר לקיים העשה לאלתר נקרא אי אפשר לקיים שניהם עכ״ד והנה לא ידעתי למה הוצאת דברי הרשב״א ממה שכתב השיטה מקובצת בשמו והלא הם דברי הרשב״א בחדושיו שבת (דף כ״ד) שכתב שם כן בשם הרמב״ן שאחר שהביא הקושיא משריפת קדשים שימתין עד למחר כדר״ל דאפשר לקיים שניהם כתב ותי׳ הרמב״ן ז״ל דלא אמר ר״ל הכי אלא היכי דאפשר לקיים שניהם עכשיו וכו׳ אבל היכי דלא אפשר לקיים העשה היום אלא בדחיית ל״ת אי אפשר לקיים את שניהם מקרי והרשב״א הביא ג״כ ראי׳ לדבריו ממה דצריך קרא דאין בנין ביהמ״ק דוחה שבת מטעם ע׳ דוחה ל״ת והרי אפשר למחר אע״כ דזה מקרי א״א לקיים שניהם אם לא אפשר לקיים עכשיו ע״ש אכן מה נכבד היום אשר נתגלה לנו מקור מים חיים דברי הרמב״ן עצמם (בחדושיו שנדפסו מחדש) לידע גבול הדבר שהביא שם הקושיא משריפת קדשים דימתין עד למחר בלשון ואיכא דקשיא לי׳ וכתב וז״ל ולדידי הא ל״ק לי משום דלא אמר רשב״ל אפשר לקיים את שניהם אלא כגון האי דאקשינן במנחות בפ׳ התכלת גבי סדין בציצית שאפשר לו לעשות לבן ממינו ולא עבדי׳ לו צמר בכיוצא בזה נאמרה אבל מצות ע׳ שאין לקיימה היום כיון דחביבה מצוה בשעתה וכמו דאמר השס במנחות אר״ש בא וראה כמה חביבה מצוה בשעתה שהרי אברים ופדרים כשרים כל הלילה וע״כ אי אפשר לקיים את שתיהן מקרי אע״פ שאפשר למחר שכל שמחוסר זמן כמחוסר הכל דמי והרי עכשיו א״א לקיים שניהם עכ״ל ומבואר כוונתו שכל שיש זמן מוגבל שאי אפשר לקיים שניהם לא נחשב אפשר לקיים שניהם אם יעביר הזמן כגון מיום ללילה שמחוסר זמן כמחוסר הכל דמי אכן זה דוקא היכי שמחוסר זמן אבל אם באותו זמן עצמו אפשר לקיים שניהם על ידי שימתין שעה או שתים זה לא מקרי מחוסר זמן כדאמרינן בזבחים (דף י״ב) אין מחוסר זמן לבו ביום ואפילו למ״ד יש מחוסר זמן לבו ביום זה דוקא כשיש בו ביום עצמו זמן מוגבל כגון מקודם חצות לאחר חצות אבל היכי שאפשר לבא לידי קיום העשה בלא עבירת הלאו באותו זמן עצמו בזה לא אמרו הראשונים שלא צריך להמתין שהרי כל הראיות שהביאו הם דוקא בענין זה בשצריך להמתין מזמן לזמן מהיום למחר או מיום ללילה למען קיים שניהם בזה אמרינן דמקרי אי אפשר לקיים שניהם וכן נראה מהראי׳ שהביא הרמב״ן מחביבה מצוה בשעתה שאין ממתינן עד שתחשך והרי אין קפידא בשבת עצמו להקדים או לאחר הקטרת אימורין שעה או שתים הרי דאין בכלל חביבה מצוה בשעתה רק שלא להעביר זמן מוגבל מזמן לזמן וכיון דאין קפידא אם יאכל מצה בליל פסח או כזית בסוכה בליל סוכות בשעה ראשונה או בשעה ששית דלילה ודאי אמרינן שיצטרך להמתין עד סוף גבול הזמן דהיינו לראב״ע עד חצות ולר״ע עד סמוך לעמוד השחר שמא ימצא היתר משום אפשר לקיים שניהם כנלענ״ד. הקטן יעקב.\\\vspace{0pt}

\end{multicols}\newpage

\newchap{סימן מ}
\begin{multicols}{2}
ב״ה אלטאנא, יום ב׳ כ׳ שבט תר״כ לפ״ק.\\\vspace{0pt}

להרה״ג וכו׳ מ״ה יצחק דוב ב״ב הלוי נ״י הגאב״ד דק״ק ווירצבורג יע״א.\\\vspace{0pt}

ראיתי למעכ״ת נ״י בספרו היקר מלאכת שמים כלל י״ח ס״ג שהביא שם התוספתא שהזכירו התוספ׳ בזבחים (דף ק״ה) אתרוג שנפרץ ותחבו בכוש או בקיסם אינו חיבור וכ׳ שנדמה לו שתוספתא זאת היתה בהעלמת עין משו״ת בית יעקב דממנה הי׳ לו להוכיח דעוקץ אתרוג שנפל לא מהני אם מחברים אותו במחט אל האתרוג והנה אף שבספרי ב״י כבר השגתי על פסק הבית יעקב ובספרי ע״ל סוכה (דף ל״ז) גם השבתי על ראיתו עכ״ז לענ״ד מתוספתא זו אין ראי׳ נגדו ובתחלה אזכיר שהרי לכאורה ממשנה שלמה אהלות (פ׳ ג׳) מוכח דלא כדבריו שהרי אמרינן שם אין חבורי אדם חבור אכן זה אינו דע״כ לא לכל הדברים אמרינן כן דאל״כ יקשה על מה דמחלקינן ב״ב (דף ס״ה) בצנור בין חקקו ואח״כ קבעו לקבעו ואח״כ חקקו וכן מחלקינן בשוקת שבסלע פרה (פ׳ ה׳) כמשכ׳ הר״ש שם ואם חבורי אדם אינו חבור א״כ גם העץ והאבן שחבר לארץ ואח״כ חקקו לא נחשב כמחובר רק כתלוש ואכתי הוי חקקו בתלוש אע״כ רק לענין טומאה אמרינן כן ומטומאה לא ילפינן כדאמרינן בכמה דוכתי׳ וכן בסוכה (דף י״ח) ובפרט לענין שיעורי טומאה דשיעורין הל״מ הם ולכן גם מהך תוספתא דאתרוג שנפרץ אין ראי׳ דדין זה לא לענין הכשר אתרוג למצו׳ נשנה כי אם לענין טומאת אוכלים כנראה בתוספתא דטבול יום (פ׳ ג׳) ובתוספתא דאהלות לא הובא רק אגב גררא דחבורי אדם אינו חבור וכן מוכח מפסק הרמב״ם שהביא התוספתא זו ה׳ טומאת אוכלים (פ׳ ו׳) ובה׳ לולב לענין הכשר אתרוג לא הביאה אלמא רק לענין צירוף שיעור טומאה נשנית ולכן אין ראי׳ מזה לפסול חבור באתרוג מטעם חבור אדם דאפילו לענין טומאה יש להוכיח ממה דתנן כלים (פ׳ י״א) חוץ מן הדלת וכו׳ שנעשו לקרקע דלא לכל ענין אמרינן דאין חבורי אדם חבור דאל״כ הרי לעולם לא יתחברו לקרקע וכן מוכח ממשכ׳ הברטנורא לענין כוורת דבורים סוף שבועות וסוף עוקצין ואף שהתוספ׳ ר״ע הקשה עליו מסוגיא דב״ב במכ״ה לא נזכר שכדברי הברטנורא מפורש בירושלמי דשביעית ויש להשוות הירושלמי עם גמרא דב״ב ואין כאן מקומו עכ״פ מוכח דלא לכל ענינים אמרינן אין חבורי אדם חבור רק לענין שיעור טומאה ולכן אין ראי׳ לענין הכשר אתרוג. שוב ראיתי אחר שנדפסו ע״י דמר נ״י פסקי ריץ גיאות ז״ל שהביא תוספתא זו בה׳ לולב אכן ממה שהפוסקים לא העתיקו דבריו בזה וגם הרמב״ם לא הביאה בה׳ לולב המוקדם כי אם בה׳ ט״א מזה נראה שדעתם כמו שכתבתי שלא נאמר זה כי אם לענין טומאה וטהרה. כנלענ״ד. הקטן יעקב.\\\vspace{0pt}

\end{multicols}\newpage

\newchap{סימן מא}
\begin{multicols}{2}
ב״ה אלטאנא, מרחשון תרט״ו לפ״ק.\\\vspace{0pt}

שאלה – מי שלקח לולב מחבירו במתנה ע״מ להחזיר ולא החזירו ונטלו ביום ראשון וגם בשאר ימי החג אם יצא בו או לא.\\\vspace{0pt}

תשובה – ביום ראשון פשיטא דלא יצא כדאמרינן סוכה (דף מ״א ע״ב) לא החזירו לא יצא וכן פסק בש״ע סי׳ תרנ״ח – אבל בשאר הימים יש לדון דהטעם דלא יצא הוא משום דאיגלי מילתא למפרע דגזול הוא בידו כמשכ׳ רש״י (שם) ובדין גזול בשאר הימים יש פלוגתא בין הפוסקים כמבואר בש״ע סי׳ תרמ״ט דלדעת הש״ע כשר ולדעת רמ״א פסול – אכן עם כל זה יש חילוק רב בין גזול ביום ראשון לגזול בשאר הימים דגזול ביום ראשון פסול מטעם לכם כדאמרינן סוכה (דף ל׳) אבל בשאר הימים לא פסול רק מטעם מצו׳ הבאה בעבירה דלכם לא כתיב רק ביום ראשון וכמבואר (שם) והנה אהך דמצו׳ הבאה בעבירה הקשו התוספ׳ דל״ל בלולב של אשירה ושל עיה״נ הטעם דכתותי תיפוק לי׳ מטעם מה״ב ותרצו דלא חשבינן מה״ב רק אם מחמת העבירה באה המצו׳ שיוצא בה כגון בגזל ע״ש והשתא דברי התוספ׳ שייכים שפיר בסתם גזל אבל בכה״ג שקבל במתנה ולא החזירו לא שייך שעי״ז הגזל נעשה המצו׳ שהרי אם לא גזלו והחזירו ג״כ הי׳ יוצא בו שהרי קבלו במתנה – וכבר אמרתי ליישב בזה מה שהעירו התוספ׳ שם ל״ל לכם משלכם להוציא הגזול תיפוק לי׳ מטעם מה״ב דיש נפקותא בכה״ג שקבל במתנה ע״מ להחזיר ולא החזירו דמטעם מה״ב הי׳ יוצא אבל מטעם לכם אינו יוצא – אלא שראיתי בזה פלוגתא במה שפשיטא להתוספ׳ שמה״ב לא אמרינן אלא אם נעשה המצוה ע״י עבירה והוכיחו כן מקושיתם למה לי באשירה ועיה״נ טעם דכתותי תיפוק לי׳ מטעם מה״ב שהרי כבר הביאו התוספ׳ בעצמם שבקצת נוסחאות כתוב הטעם דמה״ב והרב המגיד (ר״פ ח׳ מהל׳ לולב) כתב על פסק הרמב״ם דשל אשירה פסול אפילו לאחר בטול שהוא הי׳ גורס משום מה״ב ואף שהכס״מ כתוב דדוחק הוא לומר שהרמב״ם גרס גרסא שאינה בנוסחתינו במכ״ה לא ראה הפי׳ המשניות ששם הביא הרמב״ם בפי׳ לאשירה ועיה״נ ג״כ טעמא דמה״ב והוא ע״כ מפני שגרס בגמרא כן והרי לפ״ז מוכח דלשיטת הרמב״ם אמרינן מה״ב אפילו אם לא נעשה המצוה ע״י עבירה וכ״נ בפי׳ ג״כ ממה שכתב הרמב״ם בה׳ איסורי מזבח (פ׳ ה׳) דא״מ מנחות ונסכים מן הטבל וחדש וערלה וכ״כ משום מה״ב שהקב״ה שונאם ע״ש. אבל ק׳ על הרמב״ם ממה דאמרינן בירושלמי שבת (פ׳ י״ג) אמה דאמרינן דהקורע על מתו בשבת פטור בעון קומי ר׳ יוסה לא כן אמר ר׳ יוחנן בשם ר״ש בן יוצדק מצה גזולה אינו יוצא בה ידי חובתו בפסח (פי׳ מטעם מה״ב וא״כ ה״נ לא יהא יוצא י״ח קריעה מטעם מה״ב) אמר לן תמן גופי׳ עבירה ברם הכא הוא עבר עבירה כך אני אומר הוציא מצה מרשות היחיד לר״ה אינו יוצא בה י״ח בפסח עכ״ל וא״כ מה שחילק הירושלמי בין קריעה למצה גזולה הוא לכאורה חילוק התוספ׳ שהעבירה נעשית ע״י המצו׳ וכן הבין גם המפרש שם והקשה על התוספ׳ ורשב״א למה לא הביאו ראי׳ לסברתם מן הירושלמי וא״כ כש״כ שיקשה על הרמב״ם – שוב ראיתי שהקשה כן השאגת ארי׳ על הרמב״ם (סי׳ צ״ט) מהירושלמי והניח בצ״ע והנה לפי מה שכתבתי דהרמב״ם כתב כן דחשיב לולב של אשירה ושל עיה״נ מה״ב ע״פ גרסתו בהגמרא כן לכאורה אין קושיא עליו דאף דהירושלמי לא קחשיב זה מה״ב מכ״מ כיון דע״פ גמרא דילן קחשיב מה״ב פסק הרמב״ם כן – ועיין בשער המלך ה״ל לולב (פ׳ ז׳) שהביא הפרי חדש סי׳ תנ״ד שהשיג שם על מה שפסק הבית מועד דאין יוצאין במצה שהוציא מרשות היחיד לר״ה משום מה״ב מכח הירושלמי שסובר שיוצאים וכן הביא שם בשם המרדכי שאם אמר לנכרי לקצוץ לו ערבה בשבת אין חשיב זה מה״ב ויוצאין בו וכתב ג״כ ראי׳ מכח הירושלמי הלזה – אכן לפ״ז לפמשכ׳ לדעת הרמב״ם אין יוצאין בשניהם כיון שהוא פוסק כהגמרא דילן שחולק על הירושלמי – אבל באמת אי אפשר לומר כן דגם אי נימא כן שהרמב״ם סובר שהגמרא שלנו חולק על הירושלמי אכתי קשה ממה דתני בברייתא שבת (דף ק״ה ע״ב) הקורע על מתו חייב ואע״פ שמחלל את השבת יצא ידי קריעה ע״ש וכן פסק הרמב״ם והשתא כיון שמחלל השבת איך יצא ידי קריעה הא הוי מה״ב לשיטת הרמב״ם כמו לולב של אשירה – אכן גם על התוספ׳ קשה לפי מה שסברו המפרש בירושלמי וכן השער המלך והשאגת ארי׳ דהחילוק של הירושלמי בין קורע בשבת למצה גזולה משום ששם נעשה המצו׳ ע״י עבירה הא גם בקריעה הדבר כן דהא לפי המבואר בשבת כן (שם) לא חייב רק ע״י שמתקן וא״כ אי נימא שלא יצא ידי קריעה לא תקן כלום והוא פטור על שבת מטעם מקלקל ורק ע״י שאמרינן שיצא ידי קריעה הוי מתקן וחלל שבת וא״כ נעשה המצוה ע״י העבירה ובשלמא לפי׳ הרמב״ם שם בפי׳ המשניות ובחבורו שהתקון הוא ששך יצרו וחמתו בכך א״ש דלעולם הוי מתקן אפילו לא יצא ידי קריעה ועיין בתוספ׳ י״ט – אבל אכתי יקשה להתוספ׳ לשיטתם שכתבו שם דהתקון הוא מה שמקיים המצו׳ וא״כ יעשה המצו׳ ע״י עבירה שעבר וחלל שבת. ומה בכך שהי׳ יכול לצאת קריעה בבגד זה אם הי׳ קורע בחול ה״נ במצה גזולה נימא כן שהי׳ יכול לצאת בה אם ניתנה לו במתנה או קנאה אע״כ דדיינינן ע״פ מה שנעשית המצו׳ עתה וא״כ בכזה גם קריעה נעשית המצו׳ ע״י עבירה – ובלא״ה אין משמעות דברי הירושלמי כסברת התוספ׳ דא״כ לא הי׳ לו לחלק בין גופא עבירה והוא עבר עבירה אלא אם נעשה המצו׳ ע״י עבירה או לא – ולכן נלענ״ד שאין הפי׳ בירושלמי כמו שפירשו האחרונים הנ״ל אלא כחילוק שחלק הירושלמי בין מצה לקריעה הוא דלא חשיב מה״ב אלא היכי שהדבר שנעשה בה העבירה היא מתועבת והיינו כיון דילפינן מה״ב מקרא דשונא גזל בעולה א״כ לא שייך רק באותן איסורים שחלו על הדבר כגון גזל שהקב״ה שונא ומתעב הדבר עד שיחזיר לבעליו וכן של אשירה ושל עיה״נ שכבר הי׳ מתועב למקום עד שצריך שריפה וכן הדברים שנאסרים באכילה או בהנאה שבהם הרמב״ם מפרש הטעם שהקב״ה שונאם אבל היכי שהוא עבר עבירה בלבד ואין העבירה עושה רושם בהדבר שיאסר לשום דבר או שיהי׳ צריך תקון זה לא חשיב מה״ב – כן נלענ״ד שמפרש הרמב״ם הירושלמי ועל כן אין ראי׳ לסברת התוספ׳ מהירושלמי – ולפ״ז בנדון שאלה שלפנינו לכאורה תלוי בין פלוגתא שבין הרמב״ם להתוספ׳ דלהרמב״ם חשיב מה״ב ולא יצא גם בשאר הימים אבל לשיטת התוספ׳ דלא חשיב מה״ב אלא היכי שהעבירה נעשית ע״י עשיית המצו׳ לא חשיב מה״ב הכא שניתן לו במתנה רק שלא החזירו אכן להרמב״ם אפילו הוי גזול כשר בשאר הימים ולכן נלענ״ד דהכא ממנ״פ יצא להתוספ׳ משום דלא הוי מה״ב ולהרמב״ם כיון דפסק דגזול כשר בשאר הימים דבדרבנן לא חיישינן למה״ב: כנלענ״ד הקטן יעקב.\\\vspace{0pt}

\end{multicols}\newpage

\newchap{סימן מב}
\begin{multicols}{2}
ב״ה אלטאנא, יום ו׳ ט׳ מרחשון תרי״ג לפ״ק. להרה״ג וכו׳ מ״ה בער אפפענהיים נ״י הגאב״ד דק״ק אייבענשיץ יע״א.\\\vspace{0pt}

על מה שכתב מעכ״ת נ״י – הנה זה שנים רבים אשר תמהתי על האחרונים שהשיגו על רמ״א בא״ח (סי׳ תרס״ח) שכתב שאין לומר חג בשמיני עצרת והט״ז מאריך שמה כדי לסתור דבריו ובאמת דברי רמ״א מבוארים במסכת סופרים פרק י״ט ה׳ ג׳ בשביעי אומר ביום שביעי עצרת הזה ואינו מזכיר בו חג לפי שאינו חג בפני עצמו עכ״ל ולא ראיתי לאחד מהאחרונים שהביא זה ובאמת רוב עולם מזכירים חג בשהע״צ וראוי לפרסם זה כי כבר דברי הרמ״א מבוארים במסכת סופרים עכ״ד:\\\vspace{0pt}

על זה אשיב – בספרי בכורי יעקב (סי׳ תרס״ח) הבאתי ראיות מגמרא דילן דשמיני עצרת נקרא חג ובפרט הראי׳ ממה דאמרינן בפסחים (דף ע׳) וחגותם אותו חג לד׳ שבעת ימים בשנה שבעה שמנה הוי אלא מכאן לחגיגה שאינה דוחה את השבת דמוכח מזה דש״ע הוא בכלל חג לד׳ שבעת ימים ואשר מצאתי אח״כ ג״כ בשם שמלה חדשה היא ראי׳ מכרעת לענ״ד נגד הרמ״א ולכן גם אם הי׳ מוכח ממסכת סופרים דש״ע לא אקרי חג מכ״מ ע״כ אזלינן אחר ש״ס דילן שאף שכתבו התוספ׳ שם (דף מ׳) דלפעמים אנו סומכים על ספרים חצונים ומניחין גמרא שלנו הרי עכ״פ ברוב פעמים הולכין אנחנו אחר גמרא שלנו אבל תמהתי שמעכ״ת נ״י כתב שכדברי הרמ״א מבואר במסכת סופרים ולענ״ד משם ראי׳ להיפך דז״ל מסכת סופרים בפסח בין בתפלה בין בכוס צריך להזכיר בי״ט מקרא קדש הזה ביום חג המצות הזה ובחה״מ חג פלוני הזה בשביעי אומר ביום שביעי העצרת ואין מזכיר בו חג לפי שאינו חג בפני עצמו עכ״ל הרי שאיירי משביעי של פסח ונותן טעם לפי שאינו חג בפני עצמו והיינו כמו דאמרינן בסוכה (דף מ״ז) דלכך אין אומרים זמן בז׳ של פסח לפי שאינו רגל בפני עצמו ולכן לא שייך לקרותו חג עצרת שאינו חג לעצמו אלא עצרת של חג המצות ולפ״ז שמיני עצרת דאמרינן (שם) שאומרים בו זמן לפי שהוא רגל בפני עצמו לענין פז״ר קש״ב צריך להזכיר בו ג״כ חג לפי הטעם עצמו של מסכת סופרים שלא להזכיר חג בשביעי ש״פ כיון דאינו חג בפני עצמו א״כ שמיני עצרת שהיא רגל בפני עצמו צריך לקרותו ג״כ חג כנלענ״ד הקטן יעקב.\\\vspace{0pt}

\end{multicols}\newpage

\newchap{סימן מג}
\begin{multicols}{2}
ב״ה אלטאנא, יום ו׳ ל׳ תשרי תר״ט לפ״ק.\\\vspace{0pt}

למען קיים כל העוסק בתורת קרבן כאילו הקריבו עמדתי בחג העבר על החקירה אם ניסוך המים שהיו מנסכים בז ימי החג עם תמיד של שחר כדאמרינן ביומא (דף כ״ו) הי׳ מנסכי התמיד או אם הי׳ מצו׳ בפני עצמה ויש נפקותא בלא הי׳ להם תמיד להקריב אם יש מצוה לנסך המים או אם נימא שאין נסכים בלא קרבן מלבד מה שאזכיר עוד לקמן נפקותות בזה ולא ראיתי דין זה מפורש עד שמצאתי אחד קדוש מדבר והוא הריטב״א בחדושיו סוכה (דף נ׳) שאחר שהביא שם דעת הרב אב״ד ז״ל שאין ניסוך המים בלילה וחלק עליו ממה דאמרינן בתענית פ״ק ומנחתם ונסכיהם אפילו בלילה כתב וז״ל ואע״פ שאמרו במסכת תמורה שלא אמרו נסכים בלילה אלא בנסכים הבאים מעצמן שלא עם הזבח ואלו ניסוך המים קרב עם הזבח הא לא תקשי׳ דניסוך המים אינו חשוב בא עם הזבח שחובת היום הוא ולא חובת הזבח כלל שהזבח של תמיד בנסכים של יין הוא נכשר כמו בשאר ימים אלא שיש חובת היום ג״כ לעשות נסוך המים וזה כפתור ופרח לרבינו הגדול ז״ל עכ״ל והנה אף ששבח הריטב״א לסברא זו עכ״ז מצאתי בזה לא לבד פלוגתא דרבוותא אלא ג״כ סתירות בדברי הראשונים עצמם דממה שכתבו התוספ׳ בתענית (דף ב׳) שאפילו לכתחילה הי׳ מותר לנסך בלילה ראשונה ע״ש נראה שדעתם ג״כ דנסוך המים לא שייך לתמיד דאל״כ היאך ינסך בלילה ואפילו קודם הקרבת הקרבן אבל נראה דעתם היפך מזה במנחות (דף ט״ו ע״ב) שהביאו שם דעת רש״י שס״ל דנסכים הבאים עם הזבח לא נתקדשו רק בשחיטת הזבח ולא משקדשו בכלי וחלקו עליו והוכיחו דמשקדשו בכלי נפסל בלינה ממה דאמר זעירי בסוכה (דף נ׳) דלכך היו ממלאין המים מע״ש בכלי שאינה מקודשת דאי מייתי במקודשת אפסלו בלינה הרי דקדוש כלי בלא שחיטת הזבח מהני לפסול בלינה ע״ש הרי בפי׳ דס״ל להתוספ׳ דהמים מקרו נסכים הבאים עם הזבח אלמא דשייכו לתמיד דאם כדעת הריטב״א שחובת היום בפני עצמם הם פשיטא שנתקדשו בכלי והיאך שייך להוכיח דנסכים הבאים עם הזבח ג״כ נתקדשו בכלי בלא שחיטת הזבח וא״כ יש סתירה בדברי התוספ׳ בזה אכן גם בשיטת רש״י נראה סתירה דמדסבירא לי לרש״י דנסכים הבאים עם הזבח לא נתקדשו רק בשחיטת הזבח ואעפ״כ אמר זעירי דאי מייתי במקודשת אפסל בלינה לכאורה נראה שדעתו כדעת הריטב״א דנסוך המים לא שייך לתמיד אבל נראה היפך מזה ממה שכתב בסוכה (דף ל״ד) וכל קרבנות כל ימות השנה אין נסכיהם אלא יין חוץ מן החג בתמיד של שחר שצריך שני ניסוכים עכ״ל הרי דס״ל דנסוך המים הי׳ מנסכי תמיד וא״כ יקשה עליו מאד לשיטתו מהא דזעירי דאמר אפסלו בלינה אכן גם אשיטת הריטב״א ק״ל איך אפשר לומר דנסוך המים היא חובת היום ולא מנסכי תמיד שהרי בתענית (דף ב׳) פליגי תנאי מנ״ל נסוך המים מן התורה ר׳ יהודה ב״ב יליף ממה דכתיב ונסכיהם ונסכי׳ כמשפטם הרי מ׳ י׳ מ׳ ור״ע יליף מדכתיב על עולת התמיד ומנחתה ונסכי׳ בשני ניסוכין הכתוב מדבר אחד מים ואחד יין ורבי נתן יליף מדכתיב בתמיד הסך נסך בשני ניסוכים הכתוב מדבר ע״ש הרי לכל הני דרשות קרא הכתוב בפי׳ נסוך המים נסכי תמיד ומתוך כך הי׳ נלענ״ד לומר דודאי אליבי׳ דהני תנאי גם הריטב״א מודה דנסוך המים שייך לתמיד אלא כיון דמסקינן בתענית (שם) דמה דאמרינן נסוך המים כל ז ע״כ לא אתי ככל הני תנאי אלא כר׳ יהושע דס״ל דנסוך המים הל״מ היא וכיון דהך סתם מתניתן והלכתא היא לכן כתב הריטב״א שפיר דלדידן י״ל דנסוך המים חובת היום הוא ולא שייך לתמיד ובזה ימצא יישוב ג״כ לסתירת רש״י והתוספ׳ הנ״ל די״ל דהם כתבו שיטות חלוקות בזה אליבי׳ דתנאים חלוקים רק מדמסקינן להלכה דנסוך המים כל ז׳ וזה כמ״ד דנסוך המים מהל״מ אתיא ולא מהני דרשות דהתנאים כתב הריטב״א שפיר להלכה דנסוך המים היא חובת היום בפני עצמה וכן נראה ג״כ מהירושלמי הביאו המשנה למלך ה׳ תמידין (פ׳ י׳) דס״ל דמותר לנסך המים בלילה קודם התמיד וכן פסק גם הרמב״ם שם הרי דס״ל להלכה ג״כ דנסוך המים היא מצו׳ בפני עצמה לענין לנסך בלילה וא״כ ה״ה ג״כ שיפסלו בלינה בקדוש כלי לבד ושצריך לנסך גם בלא תמיד. כנלענ״ד הקטן יעקב.\\\vspace{0pt}

\end{multicols}\newpage

\newchap{סימן מד}
\begin{multicols}{2}
ב״ה אלטאנא, יום ג כ״ח מרחשון תרכ״ד לפ״ק.\\\vspace{0pt}

הרמ״א בא״ח סי׳ תרצ״ה כתב השולח מנות לחבירו והוא אינו רוצה לקבלם או מוחל לו יצא ובדרכי משה כתב כן בשם מהר״י ברין והפרי חדש כתב על זה תימא דזה מניין לו ובקרבן נתנאל תירץ שיצא לו ממה דאמרינן נדרים (דף ס״ג) האומר לחבירו קונם אם אין אתה בא ונוטל לבנך כור א׳ של חטין ושתי חביות של יין ה״ז יכול להפר נדרו שלא ע״פ חכם ויאמר לו כלום אמרת אלא מפני כבודי זהו כבודי וכתב הרשב״א הטעם שאפילו הי׳ מקבל ממנו הי׳ יכול להחזיר לו והאי טעמא שייך גם במשלוח מנות עכ״ד ולענ״ד קשה דא״כ גם במתנות לאביונים אם רצה ליתן לעני והוא אינו רוצה לקבל יצא ולמה כתב הרמ״א רק לענין משלוח מנות כן גם כבר העיר בשו״ת חתם סופר א״ח סי׳ קצ״ו דתלוי זה בב׳ טעמים של משלוח מנות אם הטעם כמו שכתב המנות הלוי להראות חבה וריעות דלפ״ז י״ל דיצא בשלוח דהראה חבתו אבל לטעם שכתב התרומת הדשן דאולי לא יספיק לו סעודתו והוא מסייעו לא שייך זה ולענ״ד טעם המהר״י ברין ורמ״א כמו שכתוב ומשלוח מנות ולא כתיב ונתון מנות כמו דכתיב ומתנות לאביונים דלשון נתינה שייך גבי מנות כדכתיב ונתן לפנינה וגו׳ מנות ולחנה יתן מנה אחת אפים מזה נראה דלא הקפיד הכתוב אלא על השילוח דהיינו שיוצא מן המשלח אבל מתנה לא אקרי רק מה שבא מיד הנותן ליד המקבל דרק אם בא לידו נקרא מתנה ולכן אם העני אינו רוצה לקבל לא יצא ידי מתנות לאביונים אפילו אם מועיל לענין נדר לומר הרי הוא כאלו התקבלתי כיון דלא קיים כאן מצות הכתוב ויכול לקיים בעני אחר אבל כאן דכתיב ומשלוח מנות דאין זה רק מצות שילוח מששלח יצא. ובהכי ניחא לי מה שנסתפקתי מי שמביא בעצמו מנות ונותן לחבירו אם יצא ידי ומשלוח מנות דאנן אמרינן שלוחו של אדם כמותו אבל אפכא לא מצאנו שיהא הוא כשלוחו וכיון דהכא כתיב ומשלוח מנות נימא דדוקא בעינן ע״י שליחות אבל על ידי נתינה לא ותמהתי שלא ראיתי לפוסקים שהעירו על זה אבל לפי מה שכתבתי י״ל כיון דב׳ טעמים דשילוח מנות שכתבתי לעיל שייכים גם בנותן הוא בעצמו ליד חבירו לכן יוצא גם בנתינה ומה דכתיב לשון שליחות הוא להורות דבשילוח לבד יצא אפילו אין חבירו רוצה לקבל ולכן לא הזכירו הפוסקים איסור נתינה ומכ״מ אולי לכתחלה טוב יותר לשלוח המנות ע״י אחר כנלענ״ד הקטן יעקב.\\\vspace{0pt}

\end{multicols}\newpage

\newchap{סימן מה}
\begin{multicols}{2}
ב״ה אלטאנא, יום ג׳ י״א אדר ראשון תר״ח לפ״ק.\\\vspace{0pt}

ראיתי לחקור חקירה אחת הנוגעת לאיסור כרת הנוהג גם בזמן הזה – שלא ראיתי לאחד מן הראשונים ומן האחרונים שעמד בה – דהיינו באיסור המעלה קדשים בחוץ מהו האיסור אם העלאה על המזבח או הקטרה על האש או שניהם יחד באופן שאם אחד העלה על המזבח בלא אש ואחד הביא את האש מי הוא החייב או אם שניהם פטורים ולכאורה הי׳ נראה כיון דקפידת הכתוב להעלות בפנים ולא בחוץ א״כ לא נקרא העלאה רק דומיא דבפנים שהיא העלאה על האש על המזבח אבל העלאה בלא אש לאו העלאה היא וכ״כ גם הרא״ה בספר החינוך המעלה בחוץ פי׳ שריפה באש עכ״ל ומכ״מ הנותן האש לבד ג״כ פטור כיון דהוא לא העלה ובקרא אשר יעלה כתיב ולא דמי למה דאמרינן בב״ק (דף נ״ט) לענין מבעיר דהכל הולך אחר העושה המעשה לבסוף דבאחד הביא את העצים ואחד את האור המביא את האור חייב ובאחד הביא את האור וא׳ את העצים המביא את העצים חייב דהתם קפידת הכתוב שיתחייב מי שעושה ההבערה וכיון דבעצים בלא אש ובאש בלא עצים ליכא הבערה לעולם לא חייב רק העושה לבסוף שהוא גרם ההבערה אבל הכא דהחיוב הוא על העלאה דומיא דבפנים לא חייב רק בשעושה שניהם ההעלאה והבערת אש אלא דלפ״ז ק״ל אמה שכתב רש״י ביבמות (דף צ׳ ע״ב) ד״ה כגון אליהו שהקריב בבמה ושעת איסור הבמות היתה ואיכא כרת דשחוטי חוץ וכרת דהעלאה עכ״ל והרי אליהו העלה בלי אש שירד אח״כ מן השמים כמאמר הכתובים והיאך הי׳ בזה כרת דהעלאה אע״כ מוכח לכאורה דעת רש״י דבהעלאה בלא אש ג״כ חייב ודע שאין להביא ראי׳ נגד שיטת רש״י בזה ממה דאמרינן בסנהדרין (דף פ״ט) היאך סמכי עלי׳ ועבדי שחוטי חוץ ע״ש ומדלא הזכיר רק שחוטי חוץ ולא ג״כ העלאה משמע דאיסור העלאה לא הי׳ שם דז״א דמהעלאה לא הי׳ יכול להביא ראי׳ דבהעלאה לא היו צריכים לסמוך על אליהו שהרי אליהו בעצמו העלה על המזבח כדכתיב וינתח את הפר וישם על העצים אבל שחיטה שלא הוזכר שאליהו עשה משמע לי׳ שאחרים עשו ע״פ דבורו ולכן מייתי ראי׳ משחוטי חוץ דוקא אבל לעולם י״ל שהי׳ שם גם איסור העלאה וכן נראה ג״כ ממה דאמרינן בירושלמי דתענית (פ׳ ב׳) ובדברך עשיתי שהקרבתי בחוץ דמדמפרש עשיתי אהקרבה בחוץ משמע שזה קאי אהעלאה שעשאה אליהו בעצמו ולא אשחיטה שלא עשה הוא בעצמו ע״פ גמר׳ דסנהדרין וא״כ מוכח שיש איסור העלאה אפילו בלא אש וכדברי רש״י ודלא כהרא״ה אבל נלענ״ד דתלי זה בפלוגתא דתנאי ע״פ מה שכתב רש״י בזבחי׳ (דף ק״י ע״א) ד״ה והאיכא חציצה וז״ל וקס״ד דמתניתן ר״י היא דאמר אינו חייב עד שיעלה לראש המזבח כהקטרת פנים עכ״ל הרי דלר׳ יוסי לא מקרי העלאה להתחייב אפי׳ בדאיכא חציצה לבד בין הנקטר לאש דבעינן דומיא דבפנים על האש דוקא אבל לר׳ יוחנן דמתרץ שם דר״ש היא ליכא קפידא במה שאינו נותן על האש ממש ולפ״ז לר׳ יוסי העלאה שלא על האש לאו העלאת חוץ היא אבל לרבי שמעון הוי העלאה ועוד ראי׳ לזה דלר״ש מקרי העלאה בלא אש שהרי ר״ש יליף דאפילו העלה על הסלע חייב מהעלאת מנוח כדאמרינן שם (דף ק״ח) ובהעלאת מנוח ג״כ לא הי׳ אש ע״פ פי׳ הרדק בפסוק ומפליא לעשות שהמלאך הוציא האש מן הצור כמו בגדעון וא״כ הי׳ שם דומיא דאליהו ואעפ״כ מקרי העלאה ולכן י״ל שמה שכתב רש״י שהי׳ אצל אליהו ג״כ איסור כרת דהעלאה אליבי׳ דר״ש כתב כן אבל לדידן דקיימא לן כר׳ יוסי העלאה על המזבח בלא אש לאו העלאה היא ופטור עלי׳. ובנין ציון מהרה יגלה, ושם על מזבח חדש ברננה נעלה, באש תוקד יורד מן השמים. נשלמו בזה תשובות מחלק אורח חיים. כנלענ״ד הקטן יעקב.\\\vspace{0pt}

\end{multicols}\newpage

\newchap{סימן מו}
\begin{multicols}{2}
ב״ה אלטאנא, יום ג׳ י״ג כסליו תרכ״ד לפ״ק. לחתני הרה״ג וכו׳ מ״ה משולם זלמן הכהן אב״ד דק״ק שווערין נ״י.\\\vspace{0pt}

על דבר שאלתך במעשה כיעור שנעשה ביישוב א׳ הסרים למשמעתך שפסולה אחת נתעברה מהשוחט שם והוא והיא מודים ורוצה לקחתה לאשה לכסות בשתה ובושת המשפחה וזה לא יוכל רק אם מותר להיות עוד שוחט בהיישוב דאם מעבירין אותו אין לו במה לפרנס אשה וגם יש חשש לתרבות רעות ואם ישא אותה יקיים המצו׳ לישא את מפותחו המבוארת באהע״ז סי׳ קע״ז.\\\vspace{0pt}

אשיב לך שאחר שבלי ספק בא עלי׳ בנדתה א״כ הוא חשוד לדבר החמור איסור כרת וחשוד לדבר החמור חשוד לדבר הקל כמבואר בי״ד סי׳ קי״ט ס״ה וא״כ לכאורה אין להאמינו עוד על איסורי נבלות וטרפות הקלים שהם איסורי לאו אמנם משום זה לא הי׳ צריך לאסרו דכבר כתב הב״י בשם שו״ת הרמב״ן והרשב״א דחשוד לדבר חמור חשוד לדבר הקל אינו רק אם שניהם מענין אחד כגון ששניהם מדבר מאכל וכיוצא בו הא לאו הכי לא והש״ך הביא ראי׳ לזה ממה שפסק המחבר סי׳ ב׳ דמומר לערלות אינו מומר לשחיטה אף דמומר לערלות הוא איסור כרת (ותמהתי דמשמע מדברי הש״ך כאילו הוא כתב ראי׳ זו מעצמו והיא מובאה כבר בשו״ת הרשב״א) וא״כ ה״נ שמי דחשוד על איסור כרת דביאת איסור לא חשיד על אכילת איסור אף דקל ממנו אכן לפי המבואר בסי׳ ב׳ ס׳ ו׳ במומר לשאר עבירות יש פלוגתא אם צריך לבדוק לו סכין דלהרמב״ם צריך ואע״ג דהמחבר הביא דיעה ראשונה שאין צריך בסתמא מכ״מ נראה מהאחרונים הפרישה והש״ך דס״ל דדעת הרמב״ם עיקר מדהקשו מהא דגניבה דעל הרמב״ם לא קשה כמשכ׳ בפלתי ובזה מתורצת קושיתו ע״ש וגם הפרי מגדים כתב על דעת הרמב״ם וכן הלכתא ושכן פסקו האחרונים והוכיח לנכון שדעת הש״ע ג״כ הכי ממה שחלק בס״ח בין מומר לערלות ובין מת אחיו מחמת מילה וזה ע״כ רק לענין בדיקה ולפ״ז אין תקנה להכשיר להעובר על איסור כרת להיות שוחט בלא שיבדק לו סכין תחלה וסוף ואין לומר כיון דרק מפיו אנו יודעים דעבר על איסור חמור הזה שהיא ודאי אינה נאמנת עליו א״כ לא נאמין לו דאין אדם משים עצמו רשע דז״א דא״כ לא נוכל להתיר לו לישא המעוברת עכ״פ עד אחר כלות ב׳ שנים משילדה משום מעוברת ומינקת חבירו אלא שיש לומר דמשום תשובה אמר כן לכסות בושתה ולקיים המצו׳ לישא מפותתו והיכא די״ל דלטוב נתכוון לא אמרינן אין אדם משים עצמו רשע כנראה מדברי התוספ׳ ב״מ (דף ד׳) אמנם אכתי הי׳ אפשר למצוא תקנה לו דלפי המבואר בש״ע שם גם להרמב״ם א״צ בדיקת סכין רק מומר לעבירה ולא הפסול לעדות והחילוק בין מומר ופסול לעדות הוא דמומר לא נקרא רק בעבר לפחות ג׳ פעמים עיין בש״ך ופ״מ וא״כ אף שנתעברה ממנו דלמא לא בא עלי׳ רק פעם או שתים ומסתמא לא הודה שכמה פעמים בא עלי׳ ואפילו הודה אפשר לומר בזה שאין אדם נאמן לעשות עצמו מומר לעבירה דקרוב אצל עצמו אלא שראיתי בשו״ת חתם סופר חי״ד סי׳ ח׳ שהביא מעשה כיוצא בזה ששוחט הודה שבא על אחות אשתו שכריסה בין שני׳ ואע״ג דאח״כ חזר וכפר מכ״מ כתב איך להאמין לכזה על שחיטות ובדיקות ע״ש אלא שלא הזכיר שם רק שפסול לעדות ומטעם זה לכאורה אין לפסלו משחיטה ולכן עכ״פ אין בנקל להכשיר עובר עבירה כזה להיות שו״ב ורק אם נראה משאר מעשיו שהולך בדרך הישר ומתחרט ועושה תשובה בסיגופים אז יש לצרף טעמים הנ״ל שלא לפסלו לגמרי אבל מכ״מ צריך לעומד על גביו ולבדוק סכינו קודם ולאחר שחיטה ולהשגיח על הבדיקה משך שנה שלימה ובתנאים הללו יש להכשירו. כנלענ״ד הקטן יעקב.\\\vspace{0pt}

\end{multicols}\newpage

\newchap{סימן מז}
\begin{multicols}{2}
ב״ה אלטאנא, יום ו׳ ט״ו סיון תרכ״א לפ״ק. לבני היקר מ״ה בן ציון נ״י בק״ק מאהילעוו יע״א.\\\vspace{0pt}

מה ששאלת ממני להביע דעתי בבהמה או עוף שיש לו ב׳ ושטים והאחד מנוקב או יש לו ב׳ כבדים והאחד אין לו ב׳ זיתים במקום מרה ובמקום חיותא אם נימא דהנקוב היתר כמאן דליתא דמי וכן בכבד שאין לה ב׳ זיתים מפני שראית בתוספ׳ ורא״ש ר״פ אילו טרפות דניקב הוא כנטול.\\\vspace{0pt}

הנה בי״ד סי׳ נ׳ פשוט דכל יתר כנטול דמי ובזה יש ג׳ פירושים לרש״י ורמב״ם הפי׳ כאלו חסר אבר השני ג״כ ולרמב״ן הפי׳ כאלו נטול ממקום שהוא מחובר שם עם המקום שהוא דבוק בו וכן דעת הרשב״א אלא שהוא מיקל דלא אמרינן דנטול עם המקום שהוא מחובר שם אלא כנחתך בשו׳ ויש נפקותא בין דעת הרמב״ן לדעת הרשב״א בשני טחולין שדבוקים במקום עובין או שני כבדים שדבוקים במקום חיותי׳ דלרמב״ן טרפה ולרשב״א כשר כמבואר בפרי חדש סי׳ מ״א ע״ש ולכן בושט לכל השיטות טרפה דבין אי נימא דהוא כניטל הושט לגמרי כשיטת רש״י ורמב״ם בין אי נימא כניטל היתר לבד הרי במקום שינטל הושט היתר יהי׳ נקב בקיבה או בושט אם נכנס היתר למטה בושט וכי ס״ד דמפני שהוא ושט נקוב לא יקרא יתר וכן לענין ב׳ כבדים תלי בפלוגתא זו לשיטת רש״י גם זה טרפה ולשיטת הרמב״ן והרשב״א יהי׳ כשר אם נראה דהכבד השלם הוא עיקר והחסר הוא היתר אכן כיון דהש״ע סי׳ מ״א פסק כרש״י להטריף ב׳ כבדים א״כ לדידן גם בכבד טרפה ואף דמדברי התוספ׳ ר״פ אילו טרפות נראה דניטל הוא בכלל ניקב ולכן בכל מקום שנאמר שניקב טרפה גם ניטל טרפה וכן פסק בש״ע סי׳ נ׳ מכ״מ מפני זה לא אמרינן אפכא ג״כ שניקב הוא כניטל ולכן אפילו ניקב הושט דינו כב׳ ושטים. כנלענ״ד הקטן יעקב.\\\vspace{0pt}

\end{multicols}\newpage

\newchap{סימן מח}
\begin{multicols}{2}
ב״ה אלטאנא, יום ג׳ כ״ו למב״י תרכ״ד לפ״ק. להרה״ג וכו׳ מ״ה שמשון הירש נ״י אב״ד דעדת ישורון בק״ק פראנקפורט דמיין יע״א.\\\vspace{0pt}

בדבר שאלת מעכ״ת נ״י אם החלב הנמצא דבוק בטחול עצמו אחר הסיר הקרום עם החלב שבקרום הוא בכלל איסור או לא כי הרבה פעמים ימצא תחת אותו קרום דבוק בטחול חלב מעט או הרבה ואפילו שמשתדלים לנקר את כל שטח הטחול הרבה פעמים אי אפשר בנקל לנקותו מכל וכל מאותו חלב הדבק בטחול עצמו.\\\vspace{0pt}

תשובה – לא ידעתי מה אדון בי׳ כי הלא בודאי כאשר דבר מעכ״ת נ״י בנכון הן בגמרא והן בפוסקים ראשונים ואחרונים לא הוזכר איסור רק מהקרום שעל הטחול ולא מחלב או שומן שתחתיו, ולכן ודאי גם מה שהעתיק מרוקח (ואשר הביאו לידי ספק זה) שכתב בסי׳ שצ״ד וסי׳ ת״ט יטול החלב והקרום שעל הטחול צריך לפרש שהחלב שעל הקרום קאמר אף שדחוק בלשונו דאם הי׳ רוצה לאסור מה שלא נאמר בשום מקום בגמרא ובשאר פוסקים ראשונים עכ״פ הי׳ משמיענו טעמו. ואפילו נימא דהוא באמת אוסר החלב שתחת הקרום ג״כ אז דעתו דעת יחידית בזה שאין צריך לחוש לה אחר שכל הפוסקים השמיטו דיעה זו. גם בשו״ת שב׳ יעקב חי״ד סי׳ כ״ו נראה ממה שכתב שם דאין איסור בטחול רק בקרום ולא בחלב שתחתיו ע״ש. אמנם כלל גדול בידינו פוק חזי מה עמא דבר וכפי הנראה מדברי מעכ״ת נ״י המנהג שם להעביר גם החלב שתחת הקרום וגם שאלתי פה מנקר מומחה ואמר לי שגם פה נוהגין כן וכן ראה במדינת פולין אף שלא ידע טעם בדבר ואולי הטעם שחוששין שמא נשאר מעט מהקרום לכן מעבירין כל אשר לבן בו ועל כן לכתחלה אין להקל דלא יהא אלא דברים המותרים ואחרים נהגו בהן איסור אי אתה רשאי להתיר בפניהם, אבל לחוש לביטול איסור לכתחלה אם ישאר מעט מהחלב בהטחול שודאי בטל בס׳ בבישול ואין כאן רק משום ביטול איסור לכתחלה כולה האי אין צריך לחוש. כנלענ״ד הקטן יעקב.\\\vspace{0pt}

\end{multicols}\newpage

\newchap{סימן מט}
\begin{multicols}{2}
ב״ה אלטאנא, יום ג׳ כ״ב תמוז תר״ז לפ״ק.\\\vspace{0pt}

נשאלתי אודות חולה שיש בו סכנה שצוו הרופאים שצריך לשתות בכל יום דם בהמה, ולמען הצילו מאיסור כרת בששותה פחות מכשיעור בפעם אחת ראיתי לחקור כמה הוא השיעור בדם להתחייב עליו כרת ולכאורה נראה פשוט דהשיעור הוא בכזית ככל שיעורי אכילה שהרי כתובה האזהרה בלשון אכילה שסתמו בכזית וכן כתב רש״י בסוכה (דף ו׳) ד״ה רוב שיעורים כזיתים, מת ונבלה ואיסור חלב ודם ופיגול ונותר וטמא וגיד הנשה עכ״ל הרי שכלל דם עם שאר איסורי׳ ששיעורם בכזית, וכן כתב הרמב״ם ה׳ מאכלות אסורות (פ׳ ו׳) האוכל כזית מן הדם במזיד חייב כרת בשוגג מביא חטאת ע״ש וכ״כ גם הסמ״ג וכן מוכח ג״כ ממה דאמרינן בכריתות (דף כ״ב) אהא דתנן הלב קורעו ומוציא את דמו לא קרעו אינו עובר עליו אר״ז א״ר ל״ש אלא בלב עוף הואיל ואין בו כזית אבל לב בהמה דיש בו כזית אסור וחייבין עליו כרת ע״ש הרי בפירוש דשיעור להתחייב כרת על אכילת דם הוא בכזית ולכן יש לתמוה ממה דאמרינן ביבמות (דף קי״ד) דבשלש אזהרות כתיב להזהיר גדולים על הקטנים בשקצים ובדם ובטומאת כהנים ומסיק בגמרא שם וצריכי דאי אשמעינן שקצים משום דאיסורן במשהו אבל דם דעד דאיכא רביעית אימא לא ואי אשמעינן דם משום דאיכא כרת אבל שרצים אימא לא וכו׳ ע״ש הרי בפי׳ דשיעור חיוב דם הוא ברביעית ולא ראיתי מי שהרגיש בזה וביותר יש לתמו׳ לפי מה שכתב המהרש״א שם דליכא לאקשויי דלא לכתוב דם ותיתי משרצים וטומאה דאיכא למימר מה להנך שכן כזית אבל דם עד דאיכא רביעית ע״ש דלפ״ז לא בלבד מדברי הגמרא מוכח דדם שיעורו ברביעית אלא גם מקרא מוכח כן מדכתיב קרא גבי דם להזהיר גדולים עה״ק ולא ילפינן משרצים וטומאה וזה תימא דמנ״ל שיעור רביעית כיון דאכילה כתיב ואכילה בכזית ועוד דבכריתות בפי׳ אמרינן דחייב בכזית וכנ״ל והי׳ נ״ל לומר ע״פ מה דאמרינן שבת (דף ע״ז) אף כשטמאו ב״ה לא טמאו אלא בדם שיש בו רביעית הואיל ויכול לקרוש ולעמוד על כזית וכן אמרינן גם במנחות (דף ק״ד) והיינו דבטבע הדם שמעורבים חלקי הדם במים וכשהוא נקרש שנפרדים חלקי המים ממנו נשאר מרביעית כזית דם והשאר הוא מים ולכן כשאוכל דם צלול לא חייב רק ברביעית דבפחות מרביעית אין בו כזית דם דהרוב מים וא״כ מה דאמרינן בכריתות וחייב בכזית דם היינו בדם קרוש שכולו דם ואין בו תערובת מים ומה דאמרינן ביבמות עד דאיכא רביעית היינו בדם צלול אבל א״ע נלענ״ד דזה אינו לא בלבד מדלא חלקו הפוסקים בין צלול לקרוש לענין שיעור אלא שגם מהגמרא מוכח דאין חילוק שהרי בחולין (דף פ״ז) אמרינן ר״א מנהרביל אומר בצללתא דדמא רבי ירמי׳ מדפתי אמר ענוש כרת והוא דאיכא כזית במתניתא תנא מטמאים באוהל והוא דאיכא רביעית ע״ש הרי משמע דרק כשכבר נקרש ונפרש המים מן הדם אז לא חייב רק כשיש כזית דם גמור ולא מצטרפים המים להדם אבל דם צלול חייב אכזית ומצטרפים חלקי המים להדם וכמו לענין רביעית דטומאת אוהל דתניא שם דג״כ הדין כן דבצלול מטמא ברביעית ובקרוש לא מטמא עד דאיכא רביעית דם גמור ע״ש בתוספ׳ ואין זה תימא שנאמר דבמעורבים חלקי המים עם הדם נחשבים דם ובנפרדו אין נחשבים דם שהרי גם לענין חלב הדבר כן דבחצי זית בשר וחצי זית חלב שבשלן זע״ז לוקה על אכילתו ועל בשולו כדאמרינן בחולין (דף ק״ח) ובהעמיד החלב עד שנפרשו מי החלב מן חלקי החלב אינו לוקה עוד דנחסר השיעור דמי חלב אינן כחלב לענין בב״ח כדאמרינן שם (דף קי״ד) אע״כ דכל זמן שלא נפרדו החלקים הכל כאחד נחשב להקרא חלב וכן לענין דם ואין זה מטעם היתר מצטרף לאיסור או משום טעם כעיקר שהרי בזה פליגי תנאי בפסחים (דף מ״ד) והכא לענין דם וחלב ליכא פלוגתא אלא ודאי דכל זמן שלא נפרש הכל נחשב כחלב ודם וא״כ אין טעם לחלק בין דם צלול לקרוש לענין שיעור ועל כן רבה התמי׳ דביבמות אמרינן דעד דאיכא רביעית לא חייב ובכריתות אמרינן דאכזית חייב ועל מה יש לסמוך ולא ראיתי כעת לאחד מהראשונים ומהאחרונים שהרגיש בזה: הקטן יעקב.\\\vspace{0pt}

\end{multicols}\newpage

\newchap{סימן נ}
\begin{multicols}{2}
ב״ה אלטאנא, יום ג׳ ד׳ תשרי תר״ח לפ״ק.\\\vspace{0pt}

אחרי הצעתי דברי הנ״ל לפני ידידי רבנים מופלגים נ״י השיבו לי ולמען ברר דבר הגדול הזה אעתיק דבריהם ומה שהשבתי עליהם. וזה יצא ראשונה מה שהשיב לי תלמידי הרבני היקר מ״ה עזריאל הילדעסהיימער נ״י (לע״ע הגאב״ד דק״ק אייזענשטאדט יע״א) וז״ל הנה אדמ״ו נסתפק כמה יאכל מן הדם ויהא חייב כי לכאורה משמע בכ״מ דשיעורא בכזית אכן ביבמות קי״ד משמע דשיעורא ברביעית זה תוכן חקירתו. ואני דעתי השפלה אציעה לפניו ואם שגיתי אתי תלין משוגתי: זה ודאי דמוכח מכל הפוסקים דשיעור אכילת דם בכזית והיינו משום דכתיב לא יאכל וסתם אכילה בכזית כמבואר בכמה דוכתא ולולי דברי מהרש״א גם סוגיא דיבמות אפשר ליישבה בדוחק דלאו דוקא רביעית דאף אם נאמר דשיעורא בכזית עכ״ז חלוק משקצים דשיעורן במשהו אכן לדברי מהרש״א ודאי מוכח דבצלול שיעורא ברביעית מטעם שבטבע מעורבים חלקי הדם במים וכשהוא נקרש נפרדים חלקי המים כפי סברת אדמ״ו הברורה אכן מה שכתב שאינו נראה לו מדלא חלקו הפוסקים בין קרוש לצלול תמי׳ לי דהר״ן פרק גיד הנשה בדין ירך שנתבשל בו גיד הנשה בד״ה אמר רבי חנינא באמת חילק כן בשם רא״ה אכן לא נראה לו לדינא לחלק בין עב לצלול ולעונש באמת אפשר שנאמר כן. וגם מה שהשיג על דברי עצמו מסוגיא חולין פ״ז לא נראה לענ״ד להסיר מזה חילוק זה הברור כשמש דמשם נראה דבקרוש בעי כזית והא קמ״ל דאעפ״י דנקרש ונפרש המים מן הדם עכ״ז בעינן כזית, וכ״ה בתוספות, ואם נקרש המים והדם סגי בכזית והיינו משום דדיינינן ליה כאכילה ובצלול אפשר לומר דבעינן רביעית משום שאם יקרש יעמוד על כזית ואף אם בטומאה אינה כן הא כדינה והא כדינה, ולפי זה הסברא במקומה עומדת דבצלול משערין ברביעית ובקרוש עם המים סגי בכזית, והיינו משום שבתורה נאמר לא יאכל ומשערין בכ״מ שיעור אכילה דהיינו כזית, וכעין זו מצאתי ברמב״ם הלכות נזירות (פרק ה׳ הלכה ד׳) וז״ל וכן אם שתה רביעית יין או אוכל כזית מיין קרוש ה״ז לוקה והוא מנזיר פרק ג׳ מינין דאקשינן לפי משנה אחרונה שתייה לאכילה. ומחול אדמו״ר נ״י לכתוב לי דעתו הרחבה בזה.\\\vspace{0pt}

וזה אשר השבתי לידידי נ״י הנ״ל.\\\vspace{0pt}

מה שתפסת בענין חקירתי על שיעור הדם בהחילוק אשר הערתי עליו לחלק בין דם צלול לקרוש ועל שתמהתי שלא חלקו הפוסקי׳ בכך שכן נמצא בר״ן בשם הרא״ה: לא כן בני. הרא״ה והר״ן לא דברו כלל מאכילת דם דז״ל הרא״ה והוי יודע דכשמשערין ברוטב משערין לכזית ביצה ומחצה של רוטב שכן שיעורו שאם יקרוש יעמוד על כזית עכ״ל וע״ז כתב הר״ן דנראה לו שיצא לו כן ממה דאמרינן כן בשבת לענין טומאת דם הואיל ואם יקרוש יעמוד על כזית אבל השיב דלא נראה לו דמשערין ברוטב סתמא קתני ולא מחלק אם הוא עב או דק ע״ש הרי לבד לענין רוטב איירו הרא״ה והר״ן אבל מדם לא הזכירו ואדרבה משם ראי׳ להיפך דאי ס״ל להראשונים מכח סתירת הגמרא לחלק לענין אכילת דם בין צלול לקרוש לא הי׳ להר״ן לחלוק מדסתמא קתני דמשערין ברוטב בששים כיון דגם בדם סתמא קתני בכזית ואעפ״כ מחלקינן בין צלול לקרוש. גם לא ידעתי מה השבת על ראיתי מסוגיא דחולין דלענ״ד ברור משם דבצלול אין השיעור ברביעית דאל״כ למה הוצרך לפרוט והוא דאיכא כזית כיון דאפילו במעורב לא מחייב עד דאיכא רביעית ובתוך רביעית פשיטא להגמרא דאיכא כזית דם גמור מדמטמא לב״ה. גם מהרמב״ם דה׳ נזירות אשר הבאת יש ראי׳ להיפך דהוא הדבר אשר דברתי דה״ל להפוסקים לחלק בדם בין צלול לקרוש וכש״כ דה״ל להרמב״ם לחלק בכך כמו דחילק בפי׳ לגבי נזיר ולא זו בלבד אלא דכתב כלל בה׳ מאכלות אסורות (פ׳ י״ד) כל איסור מאכלות שבתורה שיעורן בכזית ע״ש ועוד נלענ״ד דרק לגבי נזיר דכתיב לא ישתה וכתיב ג״כ לא יאכל בזה מחלקינן בין קרוש לצלול דסתם שתיי׳ ברביעית דילפינן שכר שכר ממקדש כמשכ׳ רש״י והתוספ׳ בנזיר (דף ל״ח ע״ב) וסתם אכילה בכזית אבל בדם דכתיב רק לשון אכילה שיעורו לעולם בכזית אף דצלול הוא ואכלו דרך שתיי׳ וכן נראה ממשכ׳ הרשב״א בחולין (דף ק״ך) אמה דאמרינן שם הקפהו לדם ואכלו חייב כיון דאקפי אחשובי׳ אחשבי׳ וז״ל ודם דאמרינן הקפה את הדם ואכלו היינו טעמא נמי משום דדרכו לאכול דרך שתיי׳ וזה שהקפה אותו ואכלו אי לאו דאחשבי׳ הא לא״ה ה״א דוקא בשאכלו כברייתו דהיינו מחוי הא בשהקפהו וגמעו פטור שלא חייבה תורה אלא כדרך ברייתן עכ״ל הרי שפשיטא לו דלאו דלא תאכל עיקרו רק על דם צלול קאי והרי בכ״מ סתם אכילה בכזית ואפילו במה שדרך אכילתו לאכלו יבש והמחהו כגון חמץ חייב בכזית כמשכ׳ הרמב״ם (ריש הל׳ חמץ ומצה) כיון דכתיב הלאו בלשון אכילה כש״כ בדם שדרך אכילתו לאכלו צלול ועוד נלענ״ד להוכיח דליכא שיעורא ברביעית גבי דם ממה דאמרינן בנזיר (דף ל״ח ע״א) א״ר אלעזר עשר רביעיות הן ונקט רב כהנא בידי׳ חמש סומקתא וחמש חווריתא ע״ש ואי ס״ד דשיעור אכילת דם צלול ברביעית ליחשוב שם סומקתא כמו דפריך באחריני שם ומתרץ בפלוגתא לא קמיירי וזה לא שייך גבי רביעית דם דבזה ליכא פלוגתא הן אמת דבש״מ שם הביא הקושיא בשם תוספו׳ הראש למה לא קחשיב רביעי׳ דם נבלות שמטמא לב״ה וליכא למימר דבפלוגתא לא קמיירי דהא בסומקתא ע״כ איירי בפלוגתא דהא חשיב רביעית דם ממת דג״כ הוא בפלוגתא אי משני מתים ג״כ או ממת א׳ דוקא ותירץ שמא אותן רביעיות לא הי׳ תופס אמת עכ״ל ולא זכיתי להבין איך שייך שהאמורא רבי אלעזר לא יתפוס אמת מה דאמרי תנאים במתניתא אכן מהא דרביעית דנבלות לענ״ד לק״מ דשפיר י״ל בפלוגתא דב״ש וב״ה לא קמיירי והא דרביעית דם מת לא הוי בפלוגתא כיון דברביעית דם ממת א׳ עכ״פ ליכא פלוגתא אכן מהא דרביעית דם לענין אכילה שפיר יקשה למה לא קחשיב אע״כ מוכח דאפילו בצלול חייב בכזית ולא ברביעית. ולכן עוד לא מצאתי ד׳ באר מה דאמרינן ביבמות אבל דם דעד דאיכא רביעית. כנלענ״ד הקטן יעקב.\\\vspace{0pt}

\end{multicols}\newpage

\newchap{סימן נא}
\begin{multicols}{2}
שוב כתב אלי על החקירה הנ״ל הרה״ג וכו׳ מ״ה גבריאל אדלער הכהן נ״י הגאב״ד דק״ק אבערדארף יע״א, וז״ל. למען ציון לא אחשה לדבר על החקירה אשר העיר עלי׳ מעכ״ת נ״י.\\\vspace{0pt}

בענין שיעור הדם להתחייב עליו כרת או מלקות? אמרתי אל דמי לו עד אכונן ועד אשים יסוד לבנין לדרז״ל ופסקי הרמב״ם לפשט הספק על דאשכחן בסוגיא דכריתות (דף כ״ב) דשיעור הדם לחייב עליו הוא בכזית וכ״כ הרמב״ם בפ׳ ו׳ מהל׳ מאכלות אסורות, וביבמות אמרי׳ סתמא דחיוב על הדם הוא ברביעית, והאריך בעיונו היטב הדק. אמנם אלו כן ק׳ דברי רבינו דידי׳ אדידי׳ דבפ״ו הנ״ל כ׳ דשיעור הדם הוא בכזית ובפי״ד מהל׳ מאכלות אסורות כ׳ סתמא דהשותה רביעית של סתם יינם מעט מעט וכו׳ או ששתה מן הדם מעט מעט אם שהה מתחלה ועד סוף כדי שתיית רביעית מצטרפין ע״ש ומוכח דשיעור דם הוא רביעית וא״כ איך כ׳ כזית ורביעית ותו הלא גם בחמץ בפסח ובחלב קשה נמי דבכל הש״ס קתני שיעור חמץ בפסח ושיעור חלב הוא בכזית והכא פ׳ רבינו השותה רביעית שהמחה חמץ או את החלב וגמאו מעט מעט אלמא דחמץ וחלב הם ברביעית אלא האמת יורה דרכו דחילוק שיעורים של כזית ורביעית הוא בין אכילה לשתי׳ דאם אכל האיסור סתם אכילה בכזית ואם שתה האיסור סתם שתית האיסורים הוא ברביעית (חוץ ביוה״כ) וכן מוכח מסוגי׳ דנזיר (ד׳ ל״ד וד׳ ל״ח) ואינו חייב עד שיאכל מן הענבים כזית משנה ראשונה אומרת עד שישתה רביעית יין וע״ש בתוספ׳ (ד׳ ל״ח) בד״ה ואינו חייב בחלופי גרסאות בסוגי׳ זו. והרמב״ם בפ״ה מהל׳ נזירות (הל׳ ב׳) כ׳ היוצא מן הגפן כיצד נזיר שאכל כזית מן הפרי שהוא ענבים וכו׳ וכן אם שתה רביעית יין או אכל כזית מיין קרוש ה״ז לוקה עכ״ל והשתא כשישאל א׳ על הלכה זו אמאי תפס רבינו כזית וגם רביעית הלא התירוץ פשוט כי אכילה הוא כזית ע״כ כשיאכל הנזיר ענבים או יין קרוש דיני׳ בכזית ושתי׳ אם שתה יין הוא ברביעית וכדין נזיר הוי בכל אסורין שבתורה אכילת איסור הוי בכזית ושתיית איסור ברביעית וה״נ לענין דם כשיאכל דם קרוש הוי כבנזיר יין קרוש ודינא הוי בכזית דשעורים הל״מ ואם שתה דם הוי ברביעית וכיון דדם ע״פ טבעו הוא מידי דשתי׳ ע״כ שפיר אמרי׳ במס׳ יבמות (דף קי״ד) בדם עד דאיכא רביעית דאם שותהו הוי ברביעית ובמס׳ חולין (דף פ״ז) דמיירי מדם קרוש ע״כ אמר רבי ירמי׳ מדפתי ענוש כרת והוא דאיכא כזית וכמ״ש רש״י דם גמור כיון דהוקרש הוא המאכל ושפיר כתבו תוס׳ בד״ה והוא דאיכא כזית והוא דאיכא רביעית דלא מחייב כרת עד דאיכא כזית היינו באכילת דם דהוי חיוב כרת משא״כ לענין טומאת מת דקיימא הלכה ברביעית אין חילוק בין קרוש או לח דצריך רביעית לטומאת מת דדוקא באכילה גזה״כ דהוי שיעורא בכזית משום דנהנה גרונו בכזית משא״כ לענין טומאה ההלכה עומדת דהן ביבש או בלח דצריך רביעית ותורת ד׳ תמימה ואין ראי׳ מטומאה לאיסור ויותר מזה כ׳ רבינו שם (פי״ד הל׳ ג׳) אפילו אכל כחצי זית וחזר ואכל אותו חצי זית עצמו שהקיא חייב שאין החיוב אלא על הנאת גרונו בכזית מדבר האסור אע״ג דגבי טומאה אין ה״א שיטמא האוהל בזה דאיסור מטומאה לא גמרינן וראיתי בהג״ה שם שתמה על רבינו על דבריו כזית שאמרנו חוץ משל בין השינים וז״ל ותימה לי איך פ׳ כר״ל דכזית חוץ משל בין השינים ר״ל אמרו ע״ש ולענ״ד אין כאן תימה דהרי רב פפא אמר שם בשל בין השינים כ״ע לא פליגי כ״פ בין החניכיים ע״כ פסק רבינו בזה ככ״ע ועל בין החניכיים פ׳ כר״י דמצטרף ועלי׳ מוסב פי׳ הכ״מ דמציין הלכה כר״י ועל בין השניים מציין לדברי הכל ודברי הרמב״ם מסכימים עם ההלכה בש״ס חולין שלהי (פ״ז ד׳ ק״ג ע״ב) אך המדפיסי׳ דברי הכ״מ שבשו קצת בזה שלא ציינו נכון ואין כ״מ אמנם הכלל דאכילה בין בדם בין בכל אסורי׳ הוא בכזית ושתי׳ בין בדם בין בנזיר יין ובהמחה חמץ או חלב כולן שוה ברביעית וע״כ אין סוגי׳ זו לזו סותרי׳ דכלם מתאימות ושכלה אין בהם ואם שגיתי יעמידני מעכ״ת נ״י על האמת ובסתום חכמה יודיעני כי מדת אמת ליעקב איש תם יושב אהלים באהל אמתי של תורה. עכ״ד הרה״ג הגאב״ד דק״ק אבערדארף נ״י.\\\vspace{0pt}

עוד כתב אלי הרה״ג וכו׳ מ״ה יעקב קאפיל הלוי ב״ב נ״י הגאב״ד דק״ק ווארמס יע״א וז״ל במה שהעיר מעכ״ת נ״י על מה שכתבו הרמב״ם והסמ״ג דהאוכל כזית מן הדם חייב ומן הש״ס יבמות (דף קי״ד ריש עמוד ב׳) משמע דבעי רביעית ועיין במהרש״א שם דיש הכרח לומר זה לפירכא לפי ברייתות אלו משום דאל״כ ליכתוב רחמנא בטומאה ובשרצים ואתי דם מיני׳ יע״ש (ועיין רש״י נזיר דף ד׳ ע״א ד״ה ור״ש כו׳ מבואר מפשטות דבריו ג״כ דדם שיעורו בכזית) והנה אמנם יש לראות דבש״ס ירושלמי ביומא (פרק ח׳ ה״ג) ובשבועות (פרק ג׳ ה״ב) בעי למילף בתחלה דשתי׳ בכלל אכילה מדאפק׳ רחמנא לאיסור דם בלשון אכילה ודחי דאיירי בדם קרוש יע״ש א״כ לפ״ז י״ל דדם בעיני׳ שיעורו ברביעית והא דאפק׳ רחמנא בלשון אכילה איירי בדם קרוש דדינו כאוכל ושיעורו בכזית וזה גופא אתי קרא לאשמעינן דדם קרוש שיעורו בכזית ולא תימא כיון דעיקרא דדם משקה היא שיעורו גם בקרוש ברביעית קמ״ל (ויש קצת דמיון לסלקא דעתין זה מנ״א על הגליון ברש״י נזיר דף ל״ז ע״ב דמדמי אכילת ענבים לשתיית יין לשיעור רביעית למשנה ראשונה אע״ג דענבים אינם משכרים ודו״ק) ואף די״ל דלפי האמת דנפ״ל שתי׳ בכלל אכילה ממ״א לא איירי קרא דדם בדם קרוש ואיסורא דדם קרוש ידעינן מכח סברא כדאי׳ בש״ס בבלי חולין (דף ק״כ) מ״מ י״ל דגם לפי האמת נשאר דאיירי בדם קרוש כיון דכתיב בלשון אכילה דהוא בכזית דדם בעיני׳ שיעורו ברביעית ואעפ״כ ידעינן דדם בעיני׳ אסור דזה א״א לומר מכח סברא דדם בעיני׳ יהי׳ מותר ודם קרוש יהי׳ אסור ועוד אפ״ל כיון דגם בדם כתיב נפש מרבינן שותה כי היכי דמרבינן בחלב וחמץ ונבילה בחולין (דף ק״כ) מקרא דנפש דכתיב בהו והנה על כרחנו אנו צריכין לומר דכך היא שיטת הברייתא (ביבמות דף קי״ד) דהביאה ג״כ לימוד בדם להזהיר וכו׳ וכן היא ג״כ שיטת התוספ׳ שהביא ג״כ ג׳ לימודים בשרצים ובדם ובטומאת כהנים להזהיר כו׳ ועיין ק״א ריש פ׳ אמור משום דאל״כ יקשה קושית מהרש״א דלא ליכתוב בדם כו׳ אבל כיון דמשמע בש״ס בכמה דוכתי דדם בעיני׳ בכזית נוסף לזה דגם בש״ס חולין (דף ק״כ ע״א) בשלמא הקפה את הדם כו׳ משמע דלא איירי קרא בדם קרוש רק ידעינן מכח סברא בעלמא דחייב עליו דאחשובי אחשבי׳ א״כ ע״כ הך ברייתא דהביאה לימוד מיוחד בדם להזהיר גדולים כו׳ לאו דהלכתא היא דידעינן זה במה הצד כו׳ וכן הרמב״ם לא הביא דין זה דהגדולים מוזהרים על הקטנים דלא להביאם להאיסור בידים ביחוד ובפרט רק גבי טומאת כהנים (פ״ג מהל׳ אבל הי״ב) אבל אינך הביא דרך כלל (פי״ז מהמא״ס הכ״ז) ממילא דנוכל לומר דס״ל דהך ברייתא הוי בגדר זה דלא כהלכתא דלא איצטריך קרא מיוחד לזה בדם רק נפיק הכל במה הצד משרצים וטומאה כנ״ל [ולפלפולא י״ל דלפי מה שפסק הרמב״ם (פ״א מהבמ״ק ה״ב) כר׳ יודא באכל דבילה קעילית כו׳ והתוספו׳ בחד תירוץ בשבועות (דף כ״ג ע״א) ד״ה גמר שכר שכר כו׳ כתבו דר׳ יודא נפ״ל דשתי׳ בכלל אכילה כדיליף בירושלמי מקרא וכל דם לא תאכלו כו׳ ולא מצי לאיירי בדם קרוש כו׳ יע״ש. ולפ״ז אתיין דבריו שפיר ביותר וכנ״ל ודוק] ומה שהקשה בספר פני משה בירושלמי שבועות ע״ד תוספ׳ הללו מברייתא דהקפה את הדם כו׳ לפענ״ד לק״מ דלפי מה דיהיב בש״ס בבלי חולין (דף ק״כ) הטעם משום דאחשובי אחשבי׳ א״כ אף אי לא הוי בכלל אוכל ומשקה חייב עליו מכח סברא הנזכרת רק קרא לא מצי איירי בהכי לר״י כיון דכתיב בי׳ אכילה והנה הק״א ריש פ׳ אמור הביא תירוץ בשם הרג״א על קושית הרא״מ ודחה אותו ואח״כ הביא תירוץ משמי׳ דנפשי׳ יע״ש והנה באמת לפי תירוצים אלו גם קושית מהרש״א הנדברת דניליף דם במה הצד מטומאה ושרצים היתה מתורצת. אך סברות הללו אין להם הכרח כלל (וע״ד הרג״א כבר השיג הק״א) והם לא חידשו אותה אלא לתרץ קושית הרא״ם והמהרש״א כבר יישב אותה קושי׳ בדבריו עכ״ד הרה״ג הגאב״ד דק״ק ווארמס נ״י. ועל הדברים הנ״ל השבתי:\\\vspace{0pt}

תשובה – תתן אמת ליעקב אזעק ואשוע ומי יתן וינחוני רבותי וחברי על דרך נכון אל מטרת האמת בביאור הסתירה של שיעור הדם בגמרא אשר עדן בקשתי ולא מצאתי גם אחרי רואי תשובות ידידי מעלת כבוד הרבנים נ״י אב״ד דק״ק אבערדארף ודק״ק ווארמס יע״א כי הדרך אשר דרכו בו לחלק בין דם קרוש דבאכילה לדם צלול דבשתיה ואשר הי׳ נראה גם לי קרוב אל האמת בהשקפה הראשונ׳ לא יתכן כלל כאשר הזכרתי (סי׳ נ״א) כי הראי׳ מנזיר שחילק הרמב״ם בין אכילה לשתיה דאכילה בכזית ושתי׳ ברביעית כבר כתבתי שם שהיא להיפך דמדלא חילק גם לענין דם כן ע״כ בדם גם שתי׳ בכזית ושרק בנזיר דכתיב לאו בלשון שתי׳ ג״כ בזה אמרינן דסתם שתי׳ ברביעית כמשכ׳ רש״י בנזיר (דף ל״ח) בגליון והתוספ׳ (שם) ורש״י הוסיף דילפינן שכר שכר ממקדש וכן כתב ג״כ השיטה מקובצת שם בשם הירושלמי הרי משמע בפי׳ דבדם דכתיב רק לשון אכילה לעולם חייב בכזית אפילו שתה ובפרט אחר דבדם שתיתו היא דרך אכילתו כמו שכתבתי בשם הרשב״א שם ולא אבא בזה עוד רק לדון בדבר החדש אשר העיר עליו הרב אב״ד דק״ק אבערדארף נ״י ממה שכתב הרמב״ם הל׳ מאכלות אסורות (פ׳ י״ד) או שתה מן הדם מעט מעט אם שהה מתחלה ועד סוף כדי שתיית רביעית מצטרפין אשר מזה הוכיח הרב נ״י דס״ל להרמב״ם דשתית דם ומיחוי חמץ וחלב שיעורם ברביעית אכן א״ע נלענ״ד שגם משם ראי׳ להיפך דאם נאמר כן דשיעור החיוב ברביעית דוקא איך שייך התנאי אם שהה מתחלה ועד סוף כדי שתית רביעית הרי אחר שהזכיר שגמאו מעט מעט ושתה מעט מעט כבר נכלל בזה ששהה בשיעור החיוב דרביעית יותר מכדי שתית רביעית וה״ל להרמב״ם לכתוב בלבד דאם שתה הרביעית מעט מעט אין מצטרף כמו שכתב באמת לענין סתם יינם שכתב בזה הלשון השותה רביעית של סתם יינם מעט מעט ועוד שלהרב נ״י נראה פשוט דשיעור חיוב מיחוי חמץ הוא ברביעית וכבר הבאתי לעיל דברי הרמב״ם (ריש הל׳ חומ״צ) שכתב בפי׳ דבהמחה כזית חמץ ושתאו חייב ואף שאי אפשר להמחות כזית חמץ שלא יהי׳ אחר המחתו יותר מכזית בעבור המשקה שמעורב בו הרי עכ״פ לא התנה הרמב״ם רק שיהי׳ כזית חמץ ולא שיהי׳ אחר שנמחה שיעור רביעית וכן נראה ג״כ מדברי הרמב״ם ה׳ מאכלות אסורות (פ׳ י׳) שכתב שם דעל יין ושמן של ערלה ושל כ״כ לוקין כדרך שלוקין על הזיתים ועל הענבים ואח״כ כתב דשיעור כל אכילה מהן כזית ע״ש משמע בפי׳ דגם על היין ועל השמן לוקה בכזית מדלא פרט בהן שיעור אחר אע״כ דכל היכי דכתיב האיסור בלשון אכילה לוקה על כזית בין שאכלו בין ששתאו רק לענין צירוף יש חילוק בין אכילה לשתי׳ דבאכילה שיעור הצירוף של כזית בכדי אכילת פרס שהוא ששה זיתים ובשתיה שיעור הצירוף של הכזית בכדי שתית רביעית שהוא שלשה זיתים ואביא ראי׳ מפורשת לכל זה ממה שכתב הרמב״ם הל׳ תרומות (פ׳ י׳) וכשם שאכילת תרומה בכזית כך שתייתה בכזית אכל וחזר ואכל שתה וחזר ושתה אם יש מתחלת אכילה ראשונה עד סוף אכילה אחרונה כדי אכילת פרס ומתחלת שתיי׳ ראשונה עד סוף שתיי׳ אחרונה כדי שתיית רביעית הרי אילו מצטרפים לכזית עכ״ל וע״ש בכ״מ שכתב הטעם בשם הר״י קורקוס דבאכילה נמשכה הנאה עד כדי אכילת פרס ובשתיי׳ רק עד כדי שתיית רביעית הרי עכ״פ מפורשים כאן ב׳ דברים – א׳ – דהיכי דכתיב האזהרה בלשון אכילה גם השתיי׳ היא בכזית – ב׳ – דאף היכי דשיעור שתיי׳ בכזית מכ״מ הצירוף הוא עד כדי שתיית רביעית ולכן מדוייק ג״כ לשון הרמב״ם שכתב גבי דם התנאי אם שהה מתחלה ועד סוף כדי שיעור רביעית דהפי׳ הוא אם שתה הכזית דם כל כך מעט מעט עד ששהה בכדי שיעור רביעית וכמו שכתב בפי׳ גבי תרומה. והיוצא מזה נוסף על מה שכתבתי כבר דבכל איסורים דכתיב האזהרה בלשון אכילה בין אכל בין שתה השיעור חיוב בכזית (חוץ מנזיר מפני ששם כתיב הלאו ג״כ בלשון שתיי׳ וגם ילפינן שכר ממקדש) רק לענין צירוף יש חילוק בין אכילה לשתיי׳ ועל כן ברור לענ״ד בלי ספק דבדם בין אכלו בין שתאו לעולם שיעור חיובו בכזית ודלא כמשכ׳ מעכ״ת הרבנים הנ״ל דבשתיית דם השיעור ברביעית ולכן עדיין לא הגענו לידע פשר הדבר מה דאמרינן ביבמות אבל דם עד דאיכא רביעית.\\\vspace{0pt}

\end{multicols}\newpage

\newchap{סימן נב}
\begin{multicols}{2}
והנה אדוני אבי מ״ו נ״י כתב אלי שנראה לו דמה דאמרינן עד דאיכא רביעית נקט כן משום דם נבלות דגם הוא בכלל שצריך להפריש קטנים ממנו ושיעורו ברביעית לא מבעי׳ לשיטת הר״ת בתוספ׳ פסחים (דף כ״ב) דפשיטא לי׳ דדם הוא בכלל נבלה אלא אפילו לשיטת הר״י שם אפשר דיש לומר כן: ולא זכיתי להבין דהן אמת דלר״ת דדם בכלל נבלה חייב על דם נבלה משום נבלה ג״כ דאיסור נבלה חל על איסור דם כיון דמוסיף על כל הבהמה וכמשכ׳ הרמב״ם הל׳ מאכלות אסורות (פ׳ ז׳) לענין חלב נבלה אלא דלא חייב משום נבלה בכזית דם רק ברביעית דאין ברביעית רק כזית בשר כמו לענין טומאה אבל משום דם דכבר קודם שנתנבלה הי׳ השיעור החיוב בכזית מהיכי תיתי שאחר שנתנבלה לא יתחייב רק על שיעור רביעית ואין לומר דאמרינן שיעורו כטומאתו וכמו דאמרינן במעילה (דף ט״ז) לענין שרצים מה טומאה בכעדשה אף אכילה בכעדשה לענ״ד זה לא אמרינן רק להחמיר דאף דבשאר שרצים לא חייב עד דאכל כזית מכ״מ בח׳ שרצים שטומאתן בכעדשה מדחמירי לענין טומאה חמירי ג״כ לענין אכילה אבל איך נאמר כן להקל היכי דבלא טומאה יש שיעור קטן משיעור טומאה דע״י שלא מטמא רק בשיעור גדול לא יתחייב ג״כ באכילה רק בשיעור גדול ועוד דהתם בשם א׳ קאמרינן דשם שרץ הוא דמטמא ודחייבין עליו באכילה אבל הכא איך נאמר מדמטמא משום נבלה ברביעית דוקא לא חייב ג״כ משום דם אלא ברביעית אלא ודאי דבדם נבלות ג״כ חייב בכזית והרי להזהיר גדולים על הקטנים גבי אזהרת דם כתיב ואכתי איך שייך מה דאמרינן אבל בדם עד דאיכא רביעית?\\\vspace{0pt}

אמנם עלה ברעיוני לפרש פי׳ אחר במה דקאמר עד דאיכא רביעית דראיתי לחקור אמה דאמרינן בחולין (דף ע״ד) דדם נוהג בשליל וכן כתב גם הרמב״ם ה׳ מ״א (פ׳ ו׳) השליל הנמצא במעי בהמה הרי דמו כדם הילוד לפיכך דם הנמצא כנוס בתוך לבו חייבין עליו כרת עכ״ל והיינו אפילו בלא כלו לו חדשיו כמבואר בלח״מ שם אכן איזו שיעור יש להתחייב על דמו וכי ס״ד דשליל בתחלת יצירתו שלא חי עדיין יתחייב על דמו משום דם הנפש ונ״ל לפשוט זה ממה דאמרינן סוטה (דף ה׳) אדם שאין בו אלא רביעית אחת וכו׳ ופי׳ רש״י שם שברביעית דם הוא מתקיים שיעור זה הל״מ שברביעית דם מת מטמא באוהל מפני שהיא נפש וקרינא בי׳ על נפשות מת לא יבא עכ״ל וכן כתב ג״כ הרמב״ם בפי׳ המשניות אהלות (פ׳ ב׳) דלכך רביעית דם מקרי דם הנפש שהמעט שיהי׳ באדם מתחלת ברייתו רביעית ע״ש והיינו באדם משעה שיש בו חיות וכן הביא הר״ש שם מתוספתא אבא שאול אומר רביעית תחלת דמו של קטן וכתב על זה אבא שאול בא ליתן טעם למה שיעורו ברביעית הואיל וזהו תחלת דמו כי האי דלעיל בספרי זוטא אמרו כזית מן המת טמא שכן הוא תחלת יצירתו וגבי שרץ שיעורו בכעדשה הואיל וחומט תחלת ברייתו כעדשה עכ״ל והנה אף דזה לא הוזכר רק גבי אדם מכ״מ מדפשיטא להגמרא בחולין (דף פ״ט) לענין יצירת בשר וגידין דדרך יצירת אדם ובהמה להיות שוין מסתמא ג״כ לענין דם הנפש כן שכל עוד שאין בבהמה רביעית דם אין בה נפש ודמה לא מקרי דם הנפש להתחייב עליו כרת ולפ״ז י״ל דלעולם אכזית דם חייב כרת כאשר הוכחנו אבל מכ״מ לא חייב אלא משעה שיש בבהמה רביעית דם הנפש ואז חייב על כזית והשתא הכי קאמר דאי אשמועינן שקצים משום דאיסורן במשהו (פי׳ משעה שהן ברי׳ במשהו כמו שפירשו התוספ׳) אבל דם דעד דאיכא רביעית (פי׳ עד שיש רביעית דם הנפש לא חייב אכזית ממנו וזה אפילו למ״ך דנוהג בשליל ודפסקינן כוותי׳ וכל שכן למ״ך שא״נ בשליל) אימא לא ובזה מדוייק הלשון דעד דאיכא רביעית ולא קאמר דאיסורו ברביעית דומיא דאיסורן במשהו ואף דגם בליכא דם הנפש מכ״מ חייב על דם מלקות משום דם האברים מכ״מ הכא לענין להזהיר גדולים על הקטנים ע״כ מדם הנפש שיש בו כרת איירי כדמוכח ממה דקאמר אחר זה ואי אשמעינן דם משום דאיכא כרת והרי יש דם האברים דאין בו כרת אע״כ דפשיטא להגמרא דמעיקר איסור דם דבכרת איירי קרא דכל נפש מכם לא תאכל דם דילפינן מיני׳ להזהיר על הקטנים מדכתיב על זה כל אוכליו יכרת ולכן קאמר שפיר ג״כ דעד דאיכא רביעית אימא לא והא דלא קאמר בקיצור דדם קיל משקצים דלא מחייב על משהו אלא על כזית י״ל דרצה לנקוט קולא בדם ששייכת גם נגד טומאה שלא יקשה קושית המהרש״א דאכתי בדם ל״ל נילף משקצים וטומאה ולכן נקט עד דאיכא רביעית דבזה קיל דם מטומאה דטומאה יש במת משעה שיש בו כזית כמבואר בדברי הר״ש הנ״ל ואיסור שקצים אפילו במשהו משא״כ איסור דם הנפש לא מתחיל עד דאיכא רביעית. כן חשבתי באולי דאפשר לפרש דברי הגמרא, והכלל היוצא מזה דעכ״פ מבואר לדינא דעל כל דם בין קרוש בין צלול בין דם שחוטה בין דם נבלה בין אכלו בין שתאו חייב בכזית רק דלענין צרוף יש חילוק בין אכלו לשתאו דבאכלו מצטרף לכדי אכילת פרס ובשתאו לכדי שתיית רביעית והיוצא לדינא בחולה מסוכן שצריך לשתות דם לרפואה על פי הכלל דמאכילין אותו הקל הקל אם סגי׳ לי׳ בכזית בשותה אותה ביותר מכדי שתיית רביעית אין להשקותו יותר מזה דבכזה לבד לא הוי רק חצי שיעור אבל כל ששותה כזית בזמן עד כדי שיעור רביעית הוי איסור כרת. כן נלענ״ד הקטן יעקב.\\\vspace{0pt}

\end{multicols}\newpage

\newchap{סימן נג}
\begin{multicols}{2}
ב״ה אלטאנא, יום ג׳ כ״ח טבת תר״ח לפ״ק. להרה״ג וכו׳ מ״ה אברהם וועכסלער נ״י הגאב״ד דק״ק שוואבאך יע״א.\\\vspace{0pt}

על מה שכתב מעכ״ת נ״י בזה הלשון – ראיתי הקושי׳ שהקשה מעכ״ת נ״י על הגמרא דיבמות (דף קי״ד) דקאמר שם ואי אשמעינן שקצים משום דשיעורן במשהו אבל דם דעד דאיכא רביעית כו׳ ובאמת אמרינן בכמה דוכתי בש״ס דדם שיעורו בכזית והיא לכאורה תמי׳ רבה ומה שפלפלו הרבנים נ״י בחכמה לחלק בין דם צלול לדם הקרוש גם אני אמרתי בהשקפה ראשונה חילוק כנ״ל אבל בטעם אחר שאמרו הם נ״י והוא דאע״ג דכתיב אכילה גבי דם מ״מ הרי הקישו הכתוב למים דכתיב על הארץ תשפכנו כמים ואין היקש למחצה אך ורק גם הלום ראיתי מה ענה אחריהם ידידי מעכ״ת נ״י ודבריו כרעי מוצקים וממילא נדחו גם דברי ובאמת בל״ה א״א לומר כן דאמרינן בפ׳ הקומץ רבה (דף כ״א ע״א) ל״ק כאן שהקפה באור כאן שהקפה בחמה באור לא הדר בחמה הדר מבואר מזה דכל עצמו של דם הקרוש אינו חייב אלא משום דדם הקרוש הדר להעשות צלול מעתה הא דתנן בפרק דם שחיטה אכל דם שחיטה כו׳ ע״כ בצלול איירי שהרי הלאווים של דם לא נאמרו אלא בצלול ומה שחייבים עליו בקרוש אינו אלא משום דהדר ואפ״ה אמרו בגמרא מפני שאין בדמו כזית ולא אמרו מפני שאין בדמו רביעית מוכח מזה דאף דם הצלול שיעורו בכזית. אכן גם בדברי ידידי מעכ״ת נ״י אעפ״י שהם דברי חכמה וחדי רחמנא בפלפולא עכ״ז לא נתקררה דעתי בהם חדא שהיא קולא גדולה שאין לה ע״מ לסמוך ואדרבה משמעות המשנה דפרק דם שחיטה משמע היפוך דבריו שהרי שנינו סתם דם בהמה חי׳ ועוף וישנן כו״כ עופות שאין בדמם כזית כש״כ רביעית בתחלת הוייתו ועוד איך יתכן לומר בפשיטות דעד דאיכא רביעית בדבר שלא נזכר בשום מקום לא במשנה ולא בברייתא ולא בגמ׳ והרי אפי׳ בדמו של קטן נחלקו ר״ע ורבנן (במשנה ב׳ פ״ב דאהלות) ומנין לנו להוסיף לדמות השיעור לענין אכילת דם לומר דיצירתן שוה אף לענין זה אטו כולהו בחדא מחתא מחתינן מה גם שהחוש מכחישו שיהי׳ יצירת אדם בהמה חי׳ ועוף שוה שיהי׳ הכל כאשר לכל כתחלת הוייתו של קטן עכ״ד מעכ״ת נ״י.\\\vspace{0pt}

ועל זה אשיב: ידידי מעכ״ת נר״ו נראה לו דוחק מה שפירשתי בדברי הגמרא דדם עד דאיכא רביעית משום דליכא סברא דבבהמה גדולה וקטנה ועוף בכלם יהי׳ שיעור דם הנפש ברביעית וגם שהרי שיעור רביעית לגבי דם הנפש לא הוזכר בשום מקום. הנה לא החלטתי את פירושי אדרבא כתבתיו בלשון אולי יש לפרש כן, אך מדברי מעכ״ת נ״י אין אני רואה סתירה לפי׳ זה בענ״ד דמה שכתב דכי ס״ד שבכלם יהי׳ השיעור שו׳ ברביעית תמו׳ לי דהכי לנו זה בלבד שהוזכר השיעור בגדול ולפי הערך צריכין אנו לשער בקטנים והרי בטרפות יש הרבה כיוצא בזה ואזכיר רק א׳ לדוגמא מה דאמרינן בי״ד (סי׳ מ״א) בכבד שנמוק שצריך שישאר כזית ע״פ הל״מ מפורש בפוסקים שם שבקטנים משערים לפי ערך זה בשיעור קטן מזה. גם מה שהקשה שהשיעור רביעית לא הוזכר בכ״מ על זה אשיב הרי עכ״פ הוכחתי שצריך שיעור א׳ ושאין לומר שנחשוב דם הנפש לשליל שעדיין לא הי׳ בו נפש וכיון שלא מוזכר שיעור הלא טוב יותר שנאמר ששיעור דם הנפש שו׳ באדם ובבהמה משנאמר שלא ניתן שיעור לדבר שע״כ צריך שיעור ובפרט אחר שמצינו בטרפות שבהרבה דברים נלמד דין בהמה מדין אדם כגון גבי חסרון בשדרה וגולגולת וכן לענין ניקב מרה וכליות כמבואר בא״ט הרי שהי׳ פשוט לחכמינו ז״ל שטבע אדם ובהמה קרובים להיות שוות כנלענ״ד הקטן יעקב.\\\vspace{0pt}

\end{multicols}\newpage

\newchap{סימן נד}
\begin{multicols}{2}
ב״ה אלטאנא, יום ג׳ ר״ח כסליו תר״ח לפ״ק. לכבוד הרה״ג וכו׳ אדוני אבי מ״ו נ״י בק״ק קארלסרוהע יע״א.\\\vspace{0pt}

כתב לי אאמ״ו נ״י וז״ל – הנה כבר הזכרתי לך בני נ״י (סימן נ״ב) דעתי שאין חייבין על דם נבלה בכזית כרת ואפרש טעמי שנ״ל אחרי שהתורה נתנה טעם לחיוב כרת על הדם כי הדם הוא בנפש יכפר אשר מזה ילפינן בכריתות (דף כ״ב) שרק על דם שהנפש יוצאת בו חייבין לכן אין חיוב רק על דם הנפש שיצא מהבהמה בשעה שהוא ראוי לכפר כגון ע״י שחיטה או נחירה ועיקור שהרי נעשה בה מעשה מה שאין כן דם נבלה שלא הי׳ לעולם ראוי לכפרה דינו כדם התמצית שאינו בכרת רק בלאו ואין להשיב שהרי בכל דם חולין הדין כן שכשנשחטה הבהמה חוץ לעזרה אינו ראוי לכפרה ומכ״מ חייבין עליו זה אינו דהתם רחמנא רביי׳ וכמו שכתב רש״י בזבחים (דף קי״א ע״א) וז״ל וא״ת כל העולין בחוץ נפסלין ביציאתן וכו׳ התם רחמנא רביי׳ אבל לענין שאר פסולין מתקבל בפנים בעינן עכ״ל א״כ ה״נ לענין חיוב דם אמרינן הכי דליכא חיוב בדם הנפש אף שאין ראוי לכפר אלא היכי דאיכא פסול חוץ לבד דלענין זה רחמנא רביי׳ שהרי כתיב קרא בפי׳ לחייב אדם חולין כדאמרינן בכריתות (דף ד׳ ע״ב) מה שאינו כן היכי דאיכא עוד פסול אחר דנבלה דזה לא עדיף מדם התמצית שאינו בכרת.\\\vspace{0pt}

על זה אשיב – הנה כבר הצעתי (סימן הנ״ל) דעתי הקלושה נוטה מדעת אאמ״ו נ״י שלפענ״ד אין חילוק בין דם נבלה לדם שחוטה או נחורה וגם על מה שהוסיף אאמ״ו נ״י פה טעם לשבח לחלק בנבילה מפני שאין דמה ראוי לכפרה ואף דבכל דם חולין אינו ראוי לכפרה מטעם יוצא שאני התם דרחמנא רביי׳ וכדאמרינן לענין העלאת חוץ – לענ״ד יש להשיב דהלא יקשה להך סברא דרחמנא רביי׳ מה הקשו רבנן לריה״ג בזבחים (דף ק״ו) אף השוחט בפנים ומעלה בחוץ כיון שהוציאו פסלו וכן (שם) אף טהור שאכל את הטמא כיון שנגעו בו טמאהו נימא ג״כ שאני התם דרחמנא רביי וכפי הנראה שם (דף ק״ח) מהא דקאמר השיב רבי תחת ריה״ג וכן ממה דקאמר שם שפיר קאמרי׳ לי׳ רבנן לריה״ג פשיטא להגמרא דאפילו לריה״ג אין סברא לומר כן אלא ודאי דסברא זו לא שייך רק לענין ב׳ פסולים חלוקים כמו בהא דכתב רש״י כן דלא נימא שיתחייב על העלאת עוף שנפסל ע״י שנשחט ולא נמלק מדחייב גם על העלאת פסול יוצא דשאני פסול יוצא דרחמנא רביי׳ אבל לענין שאר פסולים מתקבל בפנים בעינן אכן לענין פסול יוצא בעצמו אין לחלק בין היכי שנפסל ע״י שחיטה בחוץ או ע״י שהוציאו לחוץ וכן לענין טומאה כיון דרחמנא רביי׳ דחייב על דבר טמא דכיון שנגע בו טמאהו שוב אין לחלק בין טמא לטמא וכן הדבר לענין ראוי לכפרה דגבי דם דהאיך נאמר דאדם נבלה לא יתחייב כיון שעתה אין ראוי לכפרה אחרי דגלי רחמנא דאדם חולין חייב שג״כ עתה אין ראוי לכפרה אע״כ דלענין ראוי לכפרה לא הקפיד הכתוב ועוד מאי שנא דם נבלה מדם בהמה טמאה או דם חי׳ דג״כ אינם ראויים לכפר ואעפ״כ חייבים עליהם כדאמרינן כריתות (דף כ׳) ואין לומר דמקרי ראויים לכפר לנכרים בזה״ז או אפילו לישראל קודם מ״ת דהתינח לענין חי׳ אבל לגבי טמאה לא שייך זה דבפי׳ אמרינן זבחים (דף קט״ו) דבין קודם מ״ת בין לנכרים בזה״ז לא הותר להקריב רק טהורים ולא טמאים ועיין מ״ל סוף הל׳ מעה״ק אע״כ דלענין חיוב דם לא תלי בראוי לכפרה וא״כ ה״ה בדם נבלה אכן באמת לכאורה בדם נבלה בלא״ה לא משכחינן חיוב (דהיינו במתה מאלי׳ דעל דם נחירה אף שג״כ מקרי נבלה בפי׳ אמרינן שחייבין) שהרי לכ״ע בעינן בחיוב דם התנאי שהנפש יוצאת בו כמשכ׳ התוספ׳ בכריתות (דף כ׳) ד״ה דם הקזה וא״כ במתה מאלי׳ שלא יצאה הנפש ע״י הדם אין שייך חיוב כלל וא״ל שהרי גם על דם הלב אמרינן שם (דף כ״ב) דחייבין דבשעה שהנשמה יוצאת מישרף שריף ז״א דלפי מה שפי׳ רש״י שם דמושך הדם מבית השחיטה גם חיוב דדם הלב לא שייך רק בשנשחטה או נחרה ולא במתה מאלי׳ אבל מכ״מ נ״ל דמשכחת לשיטת הרי״ף והרמב״ם שפירשו שם איפכא מפי׳ רש״י דעל הדם המתכנס ללב בשעת שחיטה אין חייבים כרת רק על הדם הכנוס בתוך הלב מצד הטבע ולכן פסק הרמב״ם הל׳ מאכלות אסורות (פ׳ ו׳) דעל דם לב השליל הנמצא במעי בהמה חייבין כרת דמקרי דם הנפש וא״כ ה״ה על דם לב בהמה שמתה. ויש נפקותא לדינא גם לדידן בדין זה במסוכן שצריך להאכילו דם בהמה לרפואה כנ״ל לפי דעת אאמ״ו נ״י דדם נבלה אינו בכרת יש להסתפק איזו איסור נקרא חמור אם דם שחוטה שהוא בכרת או דם נבילה שיש בו ב׳ לאווין לאו דדם ולאו דנבילה אבל לפי מה שנלענ״ד דגם דם נבילה הוא בכרת ודאי מוטב להאכילו דם שחוטה שאין בו רק איסור אחד כנלענ״ד הקטן יעקב.\\\vspace{0pt}

\end{multicols}\newpage

\newchap{סימן נה}
\begin{multicols}{2}
ב״ה אלטאנא, יום ו׳ ה׳ אב תר״ח לפ״ק.\\\vspace{0pt}

כתב אאמ״ו נ״י וז״ל התוספ׳ בחולין (דף צ״ו ע״א) ד״ה מ״ט דרבנן וכו׳ כתבו דאבר מן החי הו׳ ברי׳ וכן משמעות רש״י ז״ל שם (דף ק״ב ע״ב) וגם שאר הראשונים מסכימים בזה ומשמע מלשונם דליכא חולק בזה ותמהני שלא הוזכר מהם שהרמב״ם חולק בה שלפענ״ד מוכח מכמה ראיות שהרמב״ם אינו סובר דהו׳ ברי׳ חדא דבפ׳ ט״ז מהל׳ מאכלות אסורות הביא כל הדברים שאינם בטלים מחמת ברי׳ ולא הביא אבר מן החי ביניהם ועוד שהראי׳ דהוכיחו מינה דהו׳ ברי׳ דהיינו מה שאמר רב בחולין (דף ק״ב ע״א) אכל צפור טהורה בחיי׳ בכל שהו מפרש הרמב״ם לענין נבלה ולא לענין אבר מן החי דבהל׳ מ״א (פ״ד הל׳ ג׳) כתב וז״ל האוכל עוף טהור חי כל שהוא לוקה משום אוכל נבלה ואע״פ שאין בו כזית הואיל ואכלו כולו עכ״ל וע״ש בה״ה – ועוד מוכח נמי מצד הסברא דהנה בחולין (דף צ״ו ע״ב) הקשו התוספ׳ ד״ה ור״י אכילה כתיב וכו׳ וז״ל וא״ת הא אמרינן בסוף פרקין אבר מן החי צריך שיאכל כזית אכילה כתיב בי׳ דילמא הא דכתיב בי׳ אכילה למימר דמחייב בכזית אע״ג דאית בי׳ ד׳ וה׳ זיתים ותרצו דבאמת אינו חייב עד שיאכל אבר שלם יע״ש וכיון שהרמב״ם כתב בהדי׳ שם (פ״ה הל׳ ג׳) וז״ל חתך מן האבר בשר כברייתו וגידין ועצמות כזית ואכלו לוקה אע״פ שאין בו בשר אלא כל שהו וכו׳ עכ״ל הרי מוכח שאינו סובר כתירוץ התוספ׳ דבעינן אבר שלם דהא כתב חתך מן האבר וכו׳ וא״כ קשה עליו קושית התוספ׳ וע״כ סובר דאבר מן החי אינו ברי׳ ועיין במה שכתב התוספ׳ י״ט טהרות (פ״א מ׳ א׳) ד״ה והאוכל אמ״ה עכ״ד אאמ״ו נ״י.\\\vspace{0pt}

הנה אאמ״ו נ״י הביא ראי׳ לדעת הרמב״ם שאבר מן החי אינו ברי׳ מקושית התוספ׳ שהיא לר׳ יהודה ולענ״ד יש להוסיף עוד שגם מרבנן דר״י דס״ל לן כוותייהו הוכיח כן שהרי שם בחולין (דף צ״ו) אמרינן כתנאי אכלו ואין בו כזית חייב ר׳ יהודה אומר עד שיהא בו כזית מאי טעמא דרבנן ברי׳ בפני עצמו היא ור״י אכילה כתיבה בי׳ וכו׳ ע״ש הרי שהוכיחו רבנן דאכלו ואין בו כזית חייב כיון דברי׳ הוא וא״כ פשיטא להו דבמה שהוא ברי׳ אפילו בפחות מכזית חייב ולכן כיון דאמר רב אבר מן החי צריך כזית ופסק הרמב״ם כן מוכח דאבר מן החי אינו ברי׳. ומכ״מ להלכה כיון דכל שאר פוסקים ס״ל דברי׳ הוא פסק הטוש״ע כוותייהו ולא חש לדעת הרמב״ם. ובזה מיושב מה שתימה הפלתי על ש״ע (סי׳ ק׳) שפסק דעוף טהור שנתנבל אינו ברי׳ למה השמיט דעת הרמב״ם שפסק דעוף טהור האוכלו בחייו אפילו בכל שהו ובמיתתו בכזית בשר גידין ועצמות ש״מ דס״ל דהוי ברי׳ כמו אבר מן החי ע״ש ולפי דברי אאמ״ו נ״י לק״מ כיון דטעם הפוסקים דאבר מ״ה הוא ברי׳ הוא מהך ברייתא דעוף טהור דס״ל דהחיוב הוא מן משום אבר מ״ה אבל להרמב״ם אין זה ראי׳ כיון שהוא מפרש דהחיוב הוא משום נבלה וכיון דהש״ע פסק דאמ״ה ברי׳ הוא דלא כרמב״ם איך יפסוק דעוף טהור שנתנבל ברי׳ הוא דהוי תרתי דסתרי כנלענ״ד הקטן יעקב.\\\vspace{0pt}

\end{multicols}\newpage

\newchap{סימן נו}
\begin{multicols}{2}
ב״ה אלטאנא, יום ג׳ כ׳ אייר תר״ח לפ״ק.\\\vspace{0pt}

ראיתי לחקור בענין הכלל שנתן הרמב״ם ה׳ שגגות (פ׳ ד׳ ה׳ ג׳) שבכל מקום שאמרינן אחע״א על ידי מוסיף צריך שיהיו האנשים מצויים בעולם כדי שתאסר עליהם אבל אם אינם מצויים אין אומרים הואיל אלו הי׳ לזה בנים או אחרים היתה נאסרת עליהם יתוסף גם עתה איסור לזה וע״ש בכס״מ שכתב שיצא להרמב״ם ממה דאמרינן בכריתות כגון דאיכא ברא לסבא אבל נסתפקתי מתי צריך שיהי׳ אותו האיש שעליו נתוסף האיסור בעולם אם בשעה שהאיסור חל או בשעה שעבר על האיסור או בשני העתים יחד למשל בהא דאמרינן בכריתות (דף י״ד) דאחותו מאמו שנשאת לאחיו מאביו ובא עלי׳ חייב שתים דאיסור אשת אח מקרי איסור מוסיף מגו דאתוסף איסורא לגבי שאר אחים אתוסף איסורא לגבי׳ דידי׳ והשתא אם בשעה שנשאת לאחיו היו לו אחים ומתו ואח״כ בא זה עלי׳ אי נימא כיון שבשעה שחל האיסור עליו הי׳ כאן איסור מוסיף אף שבשעה שעבר עליו לא הי׳ עוד מוסיף מכ״מ חייב שתים או אי אזלינן אחר שעת העבירה וכן יש נפקותא אפכא אם בשעה שנשאת לאחיו לא היו לו אחים ואח״כ נולדו לו דאי אזלינן בתר שעה שחל האיסור לא הי׳ כאן מוסיף אבל אי אזלינן אחר שעה שעבר יש כאן מוסיף או אם צריך שיהי׳ האיסור מוסיף משעה שחל עד שעבר ואז בשני הני גווני אינו חייב אלא א׳ כיון שלא הי׳ כאן מוסיף משעה שחל האיסור עד שעה שעבר עליו – והי׳ נלענ״ד לפשוט ספק זה על פי מה דאמרינן ביבמות (דף צ״ז) דהקשה למ״ד דמוקי לא יגלה כנף אביו בשומרת יבם של אביו ותיפוק לי׳ משום דודתו ומשני לעבור עליו בשני לאווין ומקשה ותיפוק לי׳ משום יבמה לשוק ומשני לעבור עליו בשלשה לאווין ואיבעית אימא לאחר מיתה ע״ש והוקשה לי איך משני לעבור עליו בשני לאווין אכתי היאך חל לאו דלא יגלה אלאו דדודתו הא אין איסור חע״א ואין לומר דשאני הכא דגלי קרא דיקשה דא״כ ליגמר מיני׳ וכמו שכתבו התוספ׳ בחולין (דף ק״א) אכן י״ל דשפיר חל מטעם איסור מוסיף דמגו דאתוסף איסורא לגבי עלמא משום יבמה לשוק אתוסף איסורא ג״כ לגבי דידי׳ משום כנף הראוי לאביו ואף שאין זה משם א׳ הרי כבר כתבו התוספ׳ ביבמות (דף ל״ב) דחשיב מוסיף גם משני שמות ואיסור יבמה לשוק עם איסור כנף אביו הוי בת אחת ולכן שפיר אמרינן לעבור עליו בג׳ לאווין אלא שלפ״ז יקשה אהא דמשני ואיב״א לאחר מיתה דהיינו שמת אביו ופקע איסור יבמה לשוק אף שמכ״מ חייב משום ולא יגלה כנף אביו כיון שהי׳ ראוי לאביו בחייו כמשכ׳ התוספ׳ שם מכ״מ השתא יקשה היאך חל שוב על איסור דודתו כיון דבמיתת אביו פקע איסור יבמה לשוק וא״כ אין כאן עוד מוסיף אע״כ יהי׳ מוכח דלענין מוסיף אזלינן אחר שעה שחל האיסור מתחלה ולא אחר שעה שעבר עליו אכן אחר עיון נראה שאין ראי׳ מזה שהרי יש לאוקמי׳ קרא באיסור בת אחת כגון שנשאת לאחי אביו ומת ונפלה לפני אביו קודם שהביא ב׳ שערות וכשהביא ב׳ שערות חלו עליו איסור דודתו ואיסור כנף אביו בבת אחת וזה חשיב בת אחת כדאמרינן יבמות (דף ל״ג) ולכך חייב שתים אבל בזה אחר זה באמת י״ל דלא חשיב עוד מוסיף לאחר מיתת אביו.\\\vspace{0pt}

שוב הי׳ נלענ״ד לפשוט ספק זה ממה דאמרינן נדרים (דף צ׳) דהאומרת נטולה אני מן היהודים יפר חלקו ותהא משמשתו ותהא נטולה מן היהודים וכתבו התוספ׳ וז״ל ותהא נטולה מן היהודים כשתתגרש ותנשא לאחרים דלא מהני הפרתו אלא לעצמו ואין לתמו׳ היאך חל נדרה על תשמיש יהודים אחרים בעודה אשת איש והא אין איסור חע״א די״ל דהוי איסור כולל מגו דאיתחל בתשמיש בעל שהוא היתר חל נמי על האחרים עכ״ל אכן אכתי יקשה הא לאחר שהפר הבעל חלקו אין כאן איסור כולל ע״י איסור הבעל ואעפ״כ אמרינן דתהא נטולה מן היהודים והרי אז אין איסור חע״א וליכא כולל אע״כ דאזלינן בתר שעת נדרה שאז האיסור הי׳ חל ע״י כולל ולכן נשאר להיות חל אף שפסק הכולל וה״ה ג״כ לענין מוסיף בכה״ג דמאי שנא אבל באמת גם על ראי׳ זו יש להשיב על פי מה שראיתי בשער המלך ה׳ א״ב (פ׳ י״ז) שהקשה אדברי התוספ׳ דמה בכך דבעודה אשת איש אין הנדר חל משום דאין אחע״א מכ״מ כשתתגרש ופקע איסור אשת איש אתי איסור נדר וחייל וכדאמרינן יבמות (דף ל״ב) איסור אחות אשה מתלי תלי וקאי ותירץ דנדר ושבועה שאני כיון דחלות האיסור הוא ע״י מבטא שפתיו ואם בשעה שנדר ונשבע לא מצא דבורו מנוח לחול מפני שהי׳ שם איסור אחר אפילו כי פקע האיסור לא חייל דבאותה שעה לא נשבע ולכן הקשו התוס׳ שפיר הא בשעה שנדרה היתה אסורה לכ״ע משום אשת איש וא״כ לא חל נדרה עכ״ד ולכן ג״כ אין ראי׳ לפשוט ספקי משום דכיון דקושיתם לא הי׳ רק משעת חלות נדר תרצו שפיר דאז הי׳ כולל וחל ואף דכשהפר הבעל פסק מלהיות כולל מכ״מ כיון דמתחלה חל הנדר לכן כשפסק איסור אשת איש כשנתגרשה אסורה לכ״ע מפני הנדר דהא אין כאן עוד איסור אחר שנצטרך שיהי׳ גם אז עדיין איסור השני כולל או מוסיף אבל כשהאיסור הראשון עדיין במקומו עומד בשעה שעובר על שניהם עדיין יש להסתפק אם בעינן שגם הכולל או המוסיף עדיין יעמוד אז אם יתחייב על שניהם או לא ואף שהדעת נוטה דאזלינן בתר זמן חלות האיסור ולא בתר זמן העבירה מכ״מ לא נזכרה לי כעת ראי׳ מכרעת לזה. כנלענ״ד. הקטן יעקב.\\\vspace{0pt}

\end{multicols}\newpage

\newchap{סימן נז}
\begin{multicols}{2}
ב״ה אלטאנא, יום ו׳ ג׳ אלול תר״ח לפ״ק. להרב וכו׳ מ״ה פנחס שיפפער נ״י בק״ק לעמבערג יע״א.\\\vspace{0pt}

כתב לי מר נ״י: וז״ל הנה נודע לי מה שחקר מעכ״ת נ״י בענין איסור מוסיף דלדעת הרמב״ם פ״ד משגגות שצריך שיהי׳ אותן האנשים מצויים בעולם ונסתפק מעכ״ת נ״י מתי צריך שיהי׳ אותו האיש שעליו נתוסף האיסור בעולם אם בשעה שהאיסור חל או בשעה שעבר על האיסור או בשני העתים יחד ורצה להביא ראיות לזה וסתר אותם וסיים לא נזכרה לי כעת ראי׳ מכרעת לזה ולי הצעיר נלפענ״ד להביא ראי׳ דאין צריך להיות בעולם רק בעת שעבר על האיסור לא בשעה שהאיסור חל מהא דגרסינן יבמות (ד׳ ל״ב) במתניתן שני אחין נשואין שתי אחיות ומת אחת וכו׳ וגרסינן עלה בגמרא פשיטא כו׳ הכא דמדחיא מהאי ביתא לגמרי וכו׳ ת״ר בא עלי׳ חייב משום אשת אח וכו׳ מיגו דאיתוסף איסור לגבי אחים איתוסף איסור לגבי׳ דידי׳ והנה כל הישר הולך בסוגית הגמרא בלשון בא עלי׳ מבואר דאמתניתן קאי דמיירי דהי׳ רק שני אחין וכדאמרינן עלה בגמרא דאדחיא מהאי ביתא לגמרי וכ״מ פשטות לשון רש״י ז״ל ד״ה בא עלי׳ וא״כ תיקשי לדעת הרמב״ם הנ״ל דצריך שיהי׳ אותן אנשים שנתוסף עליהם האיסור בעולם א״כ מאי קאמר הגמרא מיגו דאיתוסף איסור לגבי אחין כו׳ הא כאן מיירי שלא הי׳ רק שני אחין דאז בשעת נפילה לא הי׳ אח אחר מדקאמר דאדחיא מהאי ביתא לגמרי דהברייתא אמתניתין קאי כדכתיבנא אלא ודאי צ״ל דלדעת הרמב״ם העיקר תלוי בשעה שעבר על האיסור וא״כ שפיר נוכל לומר דמיירי בשעה שבא עלי׳ נולד אח ונוכל לומר מיגו דאתוסף איסור לגבי אח וזהו דוחק גדול לומר דמיירי בשעה שנשא מת הי׳ להם אח שלישי ומת קודם מיתתו דפשט המשנה מורה היפך זה ונפשט ספיקת מעכ״ת נ״י עכ״ד.\\\vspace{0pt}

ועל זה אשיב מה שכתב מעכ״ת נ״י דברייתא איירי ג״כ בשני אחין בזה ודאי הדין עמו ולא הי׳ צריך לתבוע הודאה לזה מכל הישר הולך דבפי׳ איתא בתוספתא ברישא כלשון המתניתין כמו שהביאוה הרשב״א והריטב״א וכן הוא בתוספתא דלפנינו אבל עם כל זה לא ידעתי ראי׳ מזה דאזלינן אחר זמן העבירה דכיון דע״כ לא משכחת מוסיף לר׳ יוסי אם לא שנאמר שמחוץ של ב׳ אחים שהוזכרו יש עוד אחים אחרים או בשעת נישואין או בשעת עבירה למה יהי׳ דוחק יותר שנאמר שהיו בשעת נישואין ומתו משנאמר שאיירי שנולדו אחר נישואין ואדרבא מלשון רש״י שכתב נשאת לאחד מהן נאסרה על כולן משום אשת אח עכ״ל משמע דאיירי שהיו בשעת נישואין ונאסרה אז עליהן ולכן לא זו בלבד שאין מכאן ראי׳ דאזלינן אחר שעת עבירה אלא אדרבא נלענ״ד להביא ראי׳ להיפך דאזלינן בתר שעת חלות האיסור דהנה הריטב״א הקשה אמה דקאמר מיגו דאתוסף איסורא לגבי אחים דלימא מגו דאתוסף איסורא לכל בני העולם משום אשת איש ותירץ דניחא לי׳ שם חלות דאשת אח עצמו דאתוסף על אחרים ועוד דנקט איסור שאר אחים דחמיר טפי דאיתי׳ אפילו לאחר מיתת אחיהם עכ״ל והשתא אי ס״ד דאזלינן בתר שעת עבירה מאי הוקשה לריטב״א דנימא מגו דאתוסף איסור אשת איש הרי ע״כ איירי שבא עלי׳ לאחר מיתת בעלה וכלשון התוספתא ועוד דאל״כ למה קאמר ר״י חייב משום אשת אח ואחות אשה הל״ל ג״כ משום אשת איש שהרי אשת איש ואשת אח בבת אחת חיילי ומודה רבי יוסי בבת אחת דחייב שתים כדמסקינן שם וכיון דכן היאך הוי שייך לומר מגו דאתוסף איסור אשת איש הרי בשעת עבירה ליכא עוד איסור זה כיון שכבר מת בעלה אע״כ דפשיטא לי׳ להריטב״א דאזלינן בתר שעת חלות האיסור בלבד וכמו שכתבתי כבר שהסברא נוטה לזה וזה נלענ״ד ראי׳ שאין עלי׳ תשובה. הקטן יעקב.\\\vspace{0pt}

\end{multicols}\newpage

\newchap{סימן נח}
\begin{multicols}{2}
ב״ה אלטאנא, יום ג׳ כ״ד כסליו תרכ״ג לפ״ק. לחתני הרה״ג וכו׳ מ״ה משלם זלמן הכהן נ״י אב״ד דק״ק שווערין יע״א.\\\vspace{0pt}

הוקשה לך על מה שכתב הפלתי בי״ד (סי׳ ט״ו) לתרץ קושית התוספ׳ בחולין (דף י״ב) ובכורות (דף כ׳) למה אסרינן בהמה תוך ח׳ יום ולא אזלינן בתר רובא דהוי דבר שיש לו מתירין כיון שיכול להמתין, ועל זה הוקשה לך דאין זה נקרא דשיל״מ כיון שאין ההיתר יבא בודאי דשמא לא יחי׳ הולד ח׳ ימים והיכי שאין ההיתר יבא בודאי לא נקרא דשיל״מ כמבואר י״ד (סי׳ ק״ב).\\\vspace{0pt}

תשובה – לענ״ד אין זה דומה להכלל שם שהביא הש״ע בשם יש מי שאומר ששם הענין דנסמוך על הרוב לבטל הביצה ספק טרפה ולא אמרינן כיון שנוכל לאכלו בהיתר בלא ביטול ברוב אם נמתין עד שיתברר שאין התרנגולת שהולידה הביצה טרפה הוי דבר שיש לו מתירין דבזה אין ההיתר יבא בודאי דשמא תמות קודם י״ב חודש ועדיין לא נתברר אם היתה טרפה וע״כ נסמוך על הרוב להתיר הביצה לכן מותר מיד ע״י ביטול אבל בנדון זה היאך נאמר שנתיר הספק נפל מטעם רוב ולא נמתין על ההיתר דשמא ימות אדרבא אם ימות תוך ח׳ הרי נתברר שהי׳ מן המיעוט וכל שכן שנחוש לאסור וא״כ ממנ״פ לא נסמוך על הרוב דאם יחי׳ נאמר עד שתאכלנו בספק איסור תאכלנו בהיתר ואם ימות ודאי אין כאן רוב. והנה בי״ד (סי׳ ט״ו) שהביא הש״ע שם מתשובת הרשב״א דספק אם כלה לבהמה ח׳ ימים אסור הקשה הט״ז הא הוי ספק ספיקא ספק שמא כלו לו חדשיו ואם תמצא לומר לא כלו שמא יש לו ח׳ ימים ותירץ דסבירא לי׳ כיש אומרים (בסימן ק״י) דדבר שיש לו מתירין לא מהני בי׳ ספק ספיקא וכתב עליו הש״ך בנקודת הכסף זה אינו דהא הכא אין המתיר בודאי שיבא וכי האי גוונא לא הוי דשיל״מ כלל וכדלקמן (סי׳ ק״ב סעי׳ ב׳) הרי דהש״ך כתב כדעת חתני נ״י אבל לענ״ד הדין עם הט״ז מן הטעם אשר כתבתי דממנ״פ צריך להמתין אם יחי׳ ח׳ ימים הרי בא ההיתר ואם לא יחי׳ אגלי מלתא דנפל הי׳.\\\vspace{0pt}

שוב מצאתי באור זרוע בתשובה (סי׳ תשנ״ו) שהשיב הגאון רבינו ישעיה להגאון בעל האור זרוע וז״ל: מאי דכתב מר על חתיכה שנאסרה ע״י תיקו בכל הני דא״ט אם נתערבה בחתיכות אחרות אם היא בטילה אם לאו משום שמא יבא אליהו ז״ל ויטהרנה או שמא נמצא בה טעם להתיר והוה ליה דשיל״מ, אין לחכם כמוך לשאול שאילה כזו דאפי׳ את״ל שיקרא בעבור זה שיל״מ כש״כ שיהא בטל שאם יתירהו אליהו נמצא שלא הי׳ אסור ולמה נחמיר על זה שלא יהא בטל אפי׳ באלף יותר מאותו שהוא אסור בבירור לא החמירו חכמים בדשיל״מ אלא בדבר שהיום אסור בבירור ולמחר יהא מותר כגון ביצה שנולדה ביום טוב ועצים שנשרו ביו״ט מן הדקל שהיום אסורין ולמחר מותרין וכן החדש שלפני העומר אסור ולאחר העומר מותר התם אמרו חכמים דלא ליבטל איסור דידיה כיון דאית לי׳ שריותא למחר לא אמרו ביה לבטוליה וכן נמי הטבל כיון שיש לו מתירין שיכול לתרום ולעשר עליו ממקום אחר ומתירו ואוכלו ולא נבטל האיסור לאוכלו טבל בלא תיקון אבל זו החתיכה למה לא נבטלה אפי׳ אי קים לן דאתי אליהו מחר תתבטל ממנ״פ שאם יאמר כי אסורה היתה הרי בטלה ברוב כדין כל האיסורין ואם יתירנה כל שכן דאגלאי מילתא שהכל הי׳ היתר ואין שם תערובות איסור ודשיל״מ מונע האיסור מהתבטל אבל זה אם יש לו מתירין אין כאן שום איסור שיהא צריך להתבטל, עכ״ל. ולכאורה מדברים אלה יש סתירה למה שכתבתי שהרי כתב דאפי׳ קים לן דאתי אליהו מחר ואפשר שיאמר כי אסורה היתה מ״מ נסמוך על הרוב ונתיר היום ולא נמתין עד שיתברר אבל אחר עיון נראה שזה אינו דהרבינו ישעי׳ כתב כן על הרוב דביטול דבאמת אפי׳ יפשוט אליהו דהחתיכה אסור הי׳ מ״מ הי׳ היתר ע״י ביטול ולכן כתב שפיר דממנ״פ לא נאסר מטעם דשיל״מ דאם יתיר הרי מותר גם עתה וגם אם יאסר הרי הי׳ היתר ע״י ביטול אבל זה לא שייך באם שנסמוך על רובא דעלמא שאין זה נפל דבזה הוי הממנ״פ להיפך דאם מתברר למחר ההיתר הרי הי׳ דשיל״מ ולא נסמוך היום על הרוב ואם יתברר האיסור כל שכן שלא הי׳ כאן רוב ולא היתר ולכן בכהאי גוונא לענ״ד לא נסמוך על הרוב בדשיל״מ אף שאינו בירור בוודאי לבסוף.\\\vspace{0pt}

ואין להביא ראי׳ דגם בכהאי גוונא סמכינן ארובא ולא נחשב דשיל״מ במה שאפשר להמתין ממה דסמכינן ארובא דבהמה אינה טרפה ולא מצרכינן לבדוק אחר י״ח טרפות דאע״ג דכתב רש״י חולין (דף י״ב) דהלכה למשה מסיני דסמכינן ארובא אפי׳ היכי דאפשר מ״מ אחר דאמרו רבנן דבדבר שיל״מ לא סמכינן ארובא למה לא נצריך מדרבנן לבדוק אלא ע״כ הטעם משום דאין ההיתר בודאי שיבא דדילמא תמצא טרפה א״כ יהי׳ מוכח דאפי׳ היכי שהמתנגדות לההיתר הוא האיסור מ״מ סמכינן ארוב ולא חשבינן לי׳ לדבר שיל״מ דיש לומר דבזה הטעם כיון שיש טורח בדבר ואין ההיתר בא מאליו לא נחשב דבר שיל״מ דאע״ג דממה שכתב הרשב״א הביאו הש״ע דכלי שנאסר בבליעת איסור שנתערב באחרים בטל ברוב ואין דנין אותו כדשיל״מ לפי שצריך להוציא עליו הוצאות להגעילו משמע דמשום טורח לבד לא בטל מלהיות דשיל״מ והרי גם מעשר שני מקרי דשיל״מ לדעת הר״ש והתוספ׳ והרא״ש מפני שיכול להעלותו לירושלים אע״ג שגם בזה יש טורח די״ל דאין זה דומה להטורח של בדיקת י״ח טרפות וגם מסתמא יש הפסד בזה אם נצריך שינתח הבהמה ויבדוק אחר כל הטרפות ולכך לא חשיב דשיל״מ ולא משום דאין ההיתר ברור בודאי שיבא.\\\vspace{0pt}

ובזה נלענ״ד שיש להשיב ג״כ על מה שכתב חתני נ״י להביא ראי׳ לדעת הרשב״א דהיכא שאין ההיתר בודאי שיבא לא נחשב דשיל״מ ממה דפליגי רבי מאיר ורבנן ביבמות (דף ס״א) בקטן וקטנה שלר״מ לא מיבמין קטן שמא ימצא סריס וקטנה שמא תמצא איילונית דחייש למיעוטא וחכמים מתירין דאזלינן בתר רובא שאינם סריס ואיילונית וכוותייהו פסקינן ולמה לא נאמר שיש כאן דשיל״מ שיכול להמתין עד שיגדלו ויתברר הדבר שאינם סריס ואיילונית והאיך סמכינן ארובא בדשיל״מ אלא ע״כ כיון שאין ההיתר בודאי שיבא שאפשר שימותו קודם שיתברר לא נחשב דשיל״מ ואם כדבריך יקשה על הש״ע למה כתב דעת הרשב״א בשם יש מי שאומר כיון דמוכח כן מהגמרא וכמו שהקשה הש״ך מגמרא דביצה והפלתי תירץ קושית הש״ך שאין משם ראי׳ אבל כדבריך הלא יקשה מראי׳ זו דקטן וקטנה. אמנם לפענ״ד לא נחשב דשיל״מ אם נצטרך אותם להמתין ולעגנם שנים רבות דודאי לא גרע זה מבאם צריך להוציא הוצאות להגעיל דלא נחשב דשיל״מ ולא דמי לדין הרשב״א להמתין כ״א יום לידע אם התרנגולת טרפה אף שכעת לא ידעתי ראי׳ מבוררת לזה דמשום עיגון איזה שנים לא יחשב דשיל״מ ועדיין צ״ע. כנלענ״ד הקטן יעקב.\\\vspace{0pt}

\end{multicols}\newpage

\newchap{סימן נט}
\begin{multicols}{2}
ב״ה אלטאנא, יום ג׳ ג׳ טבת תר״ז לפ״ק. להרב וכו׳ מ״ה פנחס שיפפער נ״י בק״ק לעמבערג יע״א.\\\vspace{0pt}

על מה שכתב מעכ״ת נ״י שהוקשה לו על מה שכתב הר״ן בפ׳ התקבל והרמב״ן ז״ל הקשה כו׳ לכך פי׳ דכל הנשים מותרות לינשא כו׳ אבל הוא אסור בכל הנשים שיש להם קרובות לפי שאין חזקה של קרובות שאינן באות לדון לפנינו מועלת לו כו׳ עכ״ל מפורש יוצא מפי׳ הרמב״ן ז״ל דאין חזקה של אדם שאין אנו דנין עליו מועיל לאדם אחר שאנו דנין עליו וא״כ תיקשי על הרמב״ן מסוגיא דנדה (דף מ״ו) בהקדיש הוא ואכלו אחרים דלוקין עליו משום בל יחל ואמאי לוקין הא התוספ׳ גיטין (דף ל״ג) הקשו איך מלקין לנזיר ששתה יין הא הוי התראת ספק דילמא ישאל על נזירתו ותירצו דמוקמינן אותו על חזקתו כמו שלא שאל על נזירתו עד עתה כמו כן לא ישאל על נזירתו מעתה והלאה וא״כ כאן קושית התוספ׳ במקומה למה מלקין אחרים שאכלו הקדשו דילמא ישאל הוא על הקדשו והוי התראת ספק ואין לתרץ כתירוץ התוספ׳ דמוקמינן אותו על חזקתו שלא ישאל כמו שלא שאל עד עתה דלפי דברי הרמב״ן לא שייך כאן חזקה דהא עליו אין אנו דנין עתה רק על האחרים ואין מועיל חזקת המקדיש ללקות האחרים כפי דברי הרמב״ן וזה דוחק לומר דמיירי דוקא כשבא ליד גיזבר.\\\vspace{0pt}

עוד הקשתי לדברי הרמב״ן ז״ל מהא דגרסינן נזיר (דף כ׳ ע״ב) במתניתן מי שאמר הרני נזיר ושמע חבירו ואמר ואני ואני כולם נזירים הותר הראשון הותרו כלם והקשתי ג״כ לדעת הרמב״ן קושית התוס׳ גיטין כאן במקומה עומדת דהא פשט המשנה דהוי כולם נזירים ולוקין כשעוברים על נזירותם ולמה לוקין הא הוי התראת ס׳ דלמא ישאל הראשון על נזירותו ולא שייך תי׳ התוספ׳ דמוקמינן אותו על חזקתו דלא ישאל דז״א לפי דברי הרמב״ן דהא על הראשון אין אנו דנין כעת רק על האחרונים שאמרו ואני ואני שעברו על נזירותם ולא מועיל חזקת הראשונים כפי דברי הרמב״ן.\\\vspace{0pt}

עוד הקשתי על הרמב״ן ז״ל מהא דגרסינן פסחים (דף ס״ג ע״ב) תניא כוותי׳ דר״פ השוחט את הפסח על החמץ כו׳ או לאחד מבני החבורה כו׳ ובתוספ׳ שם ד״ה או וז״ל אור״י דלא מחייב בעל החמץ אלא השוחט והזורק כו׳ ועוד הא לאו שאין בו מעשה הוא והנה ברייתא זאת מוקי לה לקמן בגמרא כר״ש ומבואר לקמן (דף פ״ט) במתניתין דר״ש ס״ל דמושכין את ידיהן עד שיזרוק הדם וא״כ מבואר דיכול למשוך את ידיו אפי׳ לאחר שחיטה לפ״ז יקשה אמאי לוקה השוחט או הזורק כשיש חמץ לאחד מבני החבורה דילמא ימשוך ידיו זה שיש לו החמץ מבני החבורה והוי התראת ס׳ דהא כשזה שיש לו החמץ מושך את ידיו מבני החבורה א״כ לא נוכל ללקות את השוחט דהא שוב אין לאחד מבני החבורה חמץ דהא בעל החמץ משך ידיו מבני החבורה ואין לתרץ כאן תירוץ התוס׳ בגיטין דמוקמינן אותו על חזקתו דלא ימשך ז״א לדברי הרמב״ן ז״ל דהא על בעל החמץ אין אנו דנין דהא אין מלקין אותו דהוי לאו שאין בו מעשה כמ״ש התוספ׳ רק על השוחט אנו דנין אם מלקין אותו א״כ אינו מועיל להעמיד בעל החמץ על חזקתו לענין השוחט כפי דברי הרמב״ן ז״ל שהביא הר״ן ז״ל שם וצ״ע עכ״ד מר נ״י.\\\vspace{0pt}

תשובה – הנה כבר הוקשה על שיטה זו של הרמב״ן שהביאה גם הבית שמואל סימן ל״ה ס״ק ל׳ להגאון בעל נו״ב מנאבד הסכין אחר שחיטה ומה שהשיב על זה (חלק י״ד סי׳ ו׳) שכוונת הרמב״ן משום דחזקת פנוי לאו חזקה אלימתא היא שכל הפנויות אין עומדות להיות כל ימיהן פנויות אדרבא עומדות להתקדש ובידיהן להתקדש לענ״ד דוחק הוא שלא מצאנו בשום מקום שחזקת פנוי׳ גרועה משאר חזקות ומ״ש דאם באה לפנינו דאנו מתירין אותה מטעם חזקת פנוי׳ ודוקא באינה לפנינו תהי׳ גרועה ובלא״ה קשה לכוון כזה בדברי הרמב״ן אכן לפ״ז מלבד קושיות מעכ״ת נ״י ומלבד מה שהוקשה להגאון בעל נו״ב מנאבד הסכין ומלבד מה שהקשה מהא דהניחו זקן או חולה הלא יקשה להרמב״ן מכל הני דתנן בגיטין (דף כ״ח) דאיך אשת כהן שהלך בעלה למד״ה אוכלת בתרומה בחזקת שהוא חי מה מועיל חזקת חי דהבעל להאשה וכן איך השולח חטאתו ממד״ה מקריבין בחזקה שהוא חי מה מועיל חזקת חי דשולח לכהן להקריב ספק חטאת שמתו בעלי׳ למזבח ועוד ממקומו הוא מוכרע הלא יקשה על הרמב״ן דאם אין אנו מחזיקין מאדם לאדם אחר להקל פשיטא דג״כ אין אנו מחזיקין להחמיר דמ״ש וא״כ היאך נאסר בכל הנשים משום חזקה שליח עושה שליחותו מה יגרום חזקה דשליח לאסרו הלא חזקה מאדם לאדם היא ולכן הנלענ״ד שמעולם לא עלה על דעת הרמב״ן ז״ל לומר שחזקה של אדם אחד לא תועיל ולא תזיק לאדם אחר אבל כוונתו נראה דהנה יש כאן שתי חזקות מתנגדות דחזקה שליח עושה שליחותו עושה כל פנויה לספק נתקדשה וחזקת פנוי׳ מוציא כל אחת מספק נתקדשה ולכן ס״ל להרמב״ן דכל שאין הפנויות לפנינו לדון עליהן ורק על הקרובות אנו דנין אז נלך אחר חזקת שליח עושה שליחותו ונאמר שהן הקרובות מאותה שקדשה השליח שאין לנו עתה לילך אחר חזקת פנוי׳ אחר שאין אנו דנין על חזקה זו רק על חזק׳ השליח אבל אם באו הפנויו׳ עצמן להעיד שלא נתקדשו אז אדרבא בכל אחת תכריע חזקת פנוי׳ שלה ונאמר דשליח עשה שליחותו באחרת שהרי ע״פ עדותה אנו מתירין אותה להנשא ממילא מותרות גם קרובותי׳ אבל במקום שאין חזקה מתנגדת לאותה חזקה שאנו דנין עלי׳ אנו הולכין אחרי׳ בין אם היא בעצמו או באחר שאינו לפנינו דהחזקה לעולם תכריע על הספק ולכן לא קשה מכל הנ״ל דבכל הני אין כאן חזקה מתנגדת לאותה חזקה שאנו דנין עלי׳ ואף דבנדון שהוקשה להגאון בעל נו״ב ג״כ חזקת הסכין מתנגדת לחזקת איסור דבהמה בחיי׳ מכ״מ כיון דעתה אנו דנין על חזקת הסכין שמוציאה מחזקת איסור שהרי הבהמה שחוטה לפניך דומה זה לבאות הפנויות להעיד דאנו הולכין אחר חזקתן גם להרמב״ן ז״ל כנלענ״ד הקטן יעקב.\\\vspace{0pt}

\end{multicols}\newpage

\newchap{סימן ס}
\begin{multicols}{2}
ב״ה אלטאנא, בחודש אלול תרכ״א לפ״ק.\\\vspace{0pt}

ספק ספקא דאית בי׳ תרי קולי דסתרן אהדדי לא מקרי ס״ס ובתרווייהו אזלינן לחומרא כן העלו הפוסקי׳ עיין בפרי חדש (סי׳ ק״י) מדברי התוספ׳ ב״ק (דף י״א) דאמרינן שם אמר רבי אלעזר שלי׳ שיצתה מקצתה ביום הראשון ומקצתה ביום שני מונין לה מן הראשון אמר לי׳ רבא מאי דעתך לחומרא חומרא דאתי לידי קולא הוא דקמטהרת לה מראשון וכתבו התוספ׳ פי׳ שסופרת י״ד של נקבה מראשון והדמים שתראה בט״ו יהיו טהורין וא״ת ואמאי לא יהיו טהורין דילמא ביום הראשון יצא ראשו או רובו ואפילו לא יצא דילמא הוא זכר וי״ל דלא מצי למשרי בס״ס דהוי תרי קולי דסתרי אהדדי שאם תראה יום מ״א תשתרי נמי מס״ס דשמא לא יצא רובא ביום הראשון ואפילו יצא רוב דילמא נקבה היא והשתא בתחלה התרנו משום דשמא זכר השתא נתירנו משום שמא נקבה הרי ממנ״פ נעשה איסור עכ״ל וכן כתבו גם בנדה (דף כ״ז) (ודף כ״ט) אמה דאמרינן דביולדת ספק זכר ספק נקבה ספק אינו וולד תשב לזכר ולנקבה ולנדה ע״ש. ולפענ״ד פסק זה צריך עיון הן מצד הסברא והן מצד הראיות על הסברא כבר תימה המהרש״ל בנדה שם וכתב מה בהך סוף סוף כל חד מכח ס״ס נשרי ומה שסותר אהדדי לחומרא נמי סתרי אהדדי ואין שייך לומר דאי אפשר להקל בתרוויהו דהא ממנ״פ נעשה איסור דמה בכך שמצטרפין לכל חד וחד עוד ספק עמו כי היכי דנשרי והמהרש״א השיב על זה אנן לא ידעינן מהיכא קפשיטא לי׳ דלא עבדינן בכי האי גוונא כתרי חומרי דסתרי אהדדי דמה דאמרינן בפ״ק דעירובין דמאן דעבד כתרי חומרי עליו הכתוב אומר הכסיל בחושך הולך היינו כגון האי דשדרה וגולגולת דאית לי׳ למעביד כחד תנא אי כב״ש כקוליהן וכחומריהן אי כב״ה כקוליהן וכחומריהן מיהו היכא דאסתפק לן עבדינן ב׳ חומרי דסתרי אהדדי כדאמרינן התם רבי עקיבא גמרא אסתפק וכו׳ אבל לא כתרי קולי כיון דסתרן אהדדי ע״ש ולענ״ד הסברא נוטה לדעת המהרש״ל דאעפ״י שמבואר בשו״ת הרא״ש (כלל כ׳ סי׳ ט״ז) לכאורה כדברי המהרש״א שבנו יחיאל השואל הי׳ רוצה לדמות כל ספקא דינא שלא נלך לחומרא כמו בשדרה וגולגולת והשיב לו הרא״ש דזה אינו דהיכא דאיכא ספקא דדינא אזלינן בכל חד לחומרא אע״ג דסתרן אהדדי ואין זה דומה לשדרה וגולגולת ע״ש מ״מ א״ע נראה שז״א דבשלמא אם הי׳ צד חומרא וקולא בספק ראשון לבד אז וודאי אזלינן לחומרא לכל צד וכמו שכתב הרא״ש אבל כיון דצד חומרא וקולא בס״ס דמדינא בס״ס אזלינן לקולא כמו ברוב מה בכך שנעשה סתירה לקולא וממנ״פ נעשה איסור כיון שאין זה בבת אחת אלא בזה אחר זה ומה בין זה למה דתנן טהרות (פ״ה) שני שבילין אחד טהור ואחד טמא הלך באחד מהן ועשה טהרות ונאכלו הזה ושנה וטבל טהור הלך בשני ועשה טהרות הרי אלו טהורות ע״ש והרי גם שם כשאוכל הטהרות שניות נמצא שעשה ממנ״פ איסור שאחת מהן ודאי טמאות היו מ״מ כיון דעל פי הדין ספק רשות הרבים טהור לא מטמאינן ודנין בכל פעם אחר כלל זה דספק רשות הרבים טהור וכן אמרינן טהרות (פ״ו) המסוכן ברשות היחיד והוציאוהו לרשות הרבים והחזירוהו לרשות היחיד כשהוא ברשות היחיד ספקו טמא כשהוא ברשות הרבים ספקו טהור ע״ש הרי אעפ״י שאנו דנין אותו כמת ברשות היחיד ושורפין על טומאתו תרומה וקדשים כדין כל ספק טומאה ברשות היחיד מ״מ כל הטהרות שנגע בהן אח״כ ברשות הרבים אנו מטהרים ונאכלות בטהרה ואין לך סתירה גדולה מזו מ״מ אנו דנין כל אחד כדינו ספק בר״ה טהור וספק ברה״י טמא ולמה לא נאמר ג״כ לענין ס״ס דאנו מטהרין אשה זו שנטהרה ביו׳ ט״ו משהפילה שלי׳ מטעם ס״ס דשמא זכר היה וכשתחזור ותראה במ״א ג״כ אנו מטהרין אותה מטעם ס״ס דשמא נקבה הי׳ אף שממ״נ באחד היתה טמאה כיון שאין זה בבת אחת אלא בזה אחר זה כן נלענ״ד להשיב מצד הסברא:\\\vspace{0pt}

אמנם גם על הראיות שהביאו התוספ׳ לסברא זו כבר השבתי לענ״ד בחדושי למסכת נדה דהנה התוספ׳ הוכיחו סברתם משלשה מקומות ממה דאמרינן ב״ק (דף י״א) גבי שלי׳ שיצתה מקצתה וחומרא דאתי לידי קולא הוא וכנ״ל וממה דאמרינן נדה (דף כ״ז) המפלת מין בהמה חי׳ ועוף ושלי׳ שאינה קשורה עמהן חוששין לוולד אחר הרני מטיל עליהם חומר שני וולדות שהקשו התוספ׳ לרבנן נימא שלא תשב רק ז׳ לטומאת לידה דאימר אין וולד בשלי׳ ואפילו יש שמא זכר הוא ותרצו דא״כ אם תראה ביום ל״ד ותחזור ותראה ביום מ״א איכא למימר נמי אימר לא ילדה כלל והוי במ״א שומרת יום כנגד יום ואפי׳ איכא וולד אימר נקבה היא ושתי ראיות דם טוהר הוא אי אפשר בתרווייהו למיזל לקולא דסתרי אהדדי אזלינן בתרווייהו לחומרא עכ״ל וכן כתבו עוד שם (דף כ״ט) אמה דתנן המפלת ואין ידוע אם וולד הי׳ אם לאו תשב לזכר ולנקבה ולנדה ע״ש ועל כל הראיות האלה יש להשיב ע״פ מה שכתב הש״ך בי״ד (סימן ק״י) בדיני ס״ס (כלל כ״ט) בשם הרא״ה ושאר פוסקים דבמקום שיש חזקת איסור לא מהני ס״ס להתיר ולפ״ז כיון דתנן שתשב לנדה ע״כ איירי בראתה דם עם הלידה או דסבירא לי׳ להאי תנא דאי אפשר לפתיחת הקבר בלא דם דאל״כ איך תשב לנדה ג״כ הרי הוי חומרא דאתי לידי קולא דאם תראה ביום ט״ו אחר שהפילה תהי׳ שומרת יום כנגד יום ותטהר לערב של יום ט״ז ודלמא ביום שהפילה לא ראתה ולא ילדה וא״כ הוי ראיתה ביום ט״ו תחלת נדה אלא ע״כ דאיירי שהיא נדה ודאי או ע״י דם שראתה או ע״י פתיחת הקבר וגם אם לאו וולד הוא מה שהפילה היא טמאה ודאי משום נדה ולפ״ז כשיגיע יום השמיני היא עומדת בחזקת טומאה ולכן לא אמרינן ס״ס לטהרה דילמא זכר הוי וכן ביצא מקצת שלי׳ כשראתה דם עמו היא בחזקת טומאה ועל כן פריך שפיר היאך קפסק ותני שתמנה משני הרי לענין ראי׳ של יום ט״ו תבא לידי קולא וכן לענין חומר ב׳ וולדות דג״כ בשעה שדנין אותה לטהרה משום ס״ס היא עומדת בחזקת טומאת נדה ולכן בכל הני לא מטהרין משום ס״ס ולא משום דסתרן אהדדי. ומצאתי ראי׳ לזה בש״מ בב״ק שם שכתב בשם גליון וז״ל הקשה בתוספ׳ למה לא תהי׳ ביום ט״ו טהורה מטעם ס״ס וקשה כיון שמנתה י״ד של טומאה הרי היא בחזקת טומאה וא״כ היאך נוכל לה לטהר יום ט״ו מכח ס״ס וי״ל דחזקת טומאה במקום דאיכא ס״ס אינה כלום עכ״ל הרי שהחזיק קושית התוספ׳ רק במה שחולק על הרא״ה וס״ל דגם במקום חזקת איסור אמרינן ס״ס ובודאי שמוכח שהתוספ׳ סבירא להו כן אבל לפ״ז להפוסקים שבמקום חזקה לא אמרינן ס״ס אין ראי׳ דלא אמרינן שני ס״ס לקולא אף כשסותרין אהדדי וכיון דפסקינן כהרא״ה דבמקום חזקת איסור לא אמרינן ס״ס י״ל דאזלינן לקולא במקום ב׳ ס״ס בזה אחר זה כיון דאזלי ראיות התוספ׳.\\\vspace{0pt}

אכן מכח ראי׳ אחרת יש להוכיח קצת סברא זו דהנה הרא״ם בתוספותיו על הסמ״ג בהל׳ מגילה הקשה אמה דאמרינן בר״ה (דף כ״ט) טומטום אינו מוציא לא את מינו ולא את שאינו מינו אמאי אינו מוציא והאיכא ס״ס ספק דילמא המוציא זכר ואת״ל נקבה שמא גם היוצא נקבה ותירץ דס״ס בתרי גופי לא אמרינן וגם המנחת יעקב הביא כלל זה בכללי ס״ס והשער המלך בהל׳ מקוואות הביא ראי׳ לזה ממה שכתבו התוספ׳ יבמות (דף ב׳) בשם רבינו אברהם מבורגווילא ע״ש אכן בחדושי למס׳ נדה הבאתי ראי׳ נגד כלל זה ממה דאמרינן שם (דף ל״ג ע״ב) ואי משום בועל נדה ספק בעל בקרוב ספק לא בעל בקרוב ואת״ל בעל בקרוב ספק השלימתו ירוק ספק לא השלימתו והוי ס״ס ואס״ס לא שרפינן תרומה ע״ש הרי דחשבינן ס״ס בתרי גופי דהספק האחד הוא בה אם השלימה לדם ירוק והספק האחר הוא בו אם בעל בקרוב ולענ״ד מזה ראי׳ נגד הכלל זה של הרא״ם אכן לפ״ז יקשה הקושיא למה טומטום אינו מוציא את מינו אבל כבר תירצה הט״א בכלל של התוס׳ הנ״ל שכתב משום דלפעמים יתהפך ספק דטומטום זה לאידך גיסא דאם טומטום אחר בא להוציאו מאיזה מצוה של תורה שאין הנשים חייבות בו כגון קריאת שמע וכיוצא בו תאמר נמי שיוצא מטעם ס״ס כזו שמא המוציא איש את״ל אשה שמא זה הטומטום נמי אשה הרי להוציא הוא טומטום אחר אזלת לקולא מחמת ספק שמא זכר הוא והו׳ לי׳ ס״ס וכשטומטום אחר בא להוציאו אתה מספקת לי׳ בנקבה והוי לי׳ ס״ס תרתי קולי דסתרן אהדדי ומשום הכי לא אמרינן כלל לס״ס זה ע״ש.\\\vspace{0pt}

היוצא מזה דמכח קושיא זו של הרא״ם יהי׳ מוכח כאחד משני הכללים או דלא אמרינן ס״ס בתרי גופי או דלא אמרינן תרי ס״ס לקולא היכא דסתרן אהדדי וכיון שהוכחנו מגמרא דנדה הנ״ל דאמרינן ס״ס גם בתרי גופי א״כ יהי׳ מוכח מכח ראי׳ זו דלא אמרינן ס״ס בתרי קולי דסתרן אהדדי אף שמראיות התוספ׳ אין להוכיח כן לשיטת הרא״ה דפסקינן כוותי׳ דבמקום חזקת איסור לא מהני ס״ס. כנלענ״ד. הקטן יעקב.\\\vspace{0pt}

\end{multicols}\newpage

\newchap{סימן סא}
\begin{multicols}{2}
ב״ה אלטאנא, י״ב אדר תרי״ח לפ״ק.\\\vspace{0pt}

נשאלתי באחד שהוצרך לבנות לו בית כדי לקבל רשיון מהשררה יר״ה לישא אשה ואי אפשר לו למצוא באותה עיר שהותר לו לישב בה אם יבנה בית רק מקום שנטועים בו אילנות העושים פירות אבל הם זקנים ולהם מעט ענפים ומקצתם יבשים אם מותר לעקור האילנות כדי לבנות לו בית במקומם.\\\vspace{0pt}

תשובה – בב״ק (דף צ״א) אמר רב דיקלא דטעון קבא אסור למקציי׳ אמר ר״ח לא שכיבא שכחת ברי אלא דקץ תאנתה בלא זימנא ואמרינן שם דאם מעולה בדמים לעצים מותר דרק דרך השחתה אסרה תורה ולכן כתב הרא״ש שם וכן אם הי׳ צריך למקומו נראה דמותר וכן נפסק בפשיטות גם בט״ז י״ד (סי׳ קט״ז) ובשו״ת צמח צדק סי׳ מ״א. אמנם בשו״ת בית יעקב סי׳ ק״מ שדא נרגא בהיתר זה שרצה להוכיח שהתוספ׳ חולקין על הראש וס״ל דבצריך למקומו אסור לקצוץ ולכן העלה להלכה כן ודלא כט״ז. ולענ״ד כל דבריו תמוהין בתחלה כתב שהרא״ש כתב ההיתר בצריך למקומו בס״פ לא יחפור וזה אינו דלא בב״ב אלא בב״ק פ׳ החובל כתב הרא״ש כן שוב הקשה מה טעמו של הרא״ש ונכנס בדרך רחוק ומתוך כך רצה להוכיח שלהתוספ׳ הראי׳ שהמציא הרא״ש אינה ראי׳ ולכן חולקין על הרא״ש ולא אכניס להשיב על ראיתו להרא״ש כי יארכו הדברים אבל תמהתי הרי הרא״ש בעצמו ביאר טעמו שבב״ק אחר שהביא הברייתא ומימרא דרבינא דאם הי׳ מעולה בדמים מותר ומעשה דשמואל ורב חסדא שצוו לעקור האילנות ע״י שהיו מכחישין הגפנים שמעולין יותר סיים וכן אם הי׳ צריך למקומו נראה דמותר וא״כ טעמו כמו דמותר לקוץ אם הי׳ האילן מעולה מפירותיו לעצים או אם מכחיש פירות אחרות מטעם דרק דרך השחתה אמרה התורה ולא כשהכריתה היא תועלת לו ה״ה ג״כ כשכורת מפני שפנוי המקום הוא תועלת לו יותר מפירות האילנות ורחוק שימצא מי שיחלוק על סברא זו. שוב כ׳ הבית יעקב ראי׳ שהתוספ׳ חולקין על הרא״ש דבברכות (דף ל״ו) הקשו על הא דאין קוצצין אילנות בשביעית דבלא שביעית נמי תיפוק לי׳ דאסור מלא תשחית את עצה ותרצו דאיירי דלא טען קבא דלא שייך בה בל תשחית א״נ איירי דמעולה בדמים לעשות ממנו קורות שאז לא שייך בל תשחית והקשה למה לא תרצו התוספ׳ דאיירי בצריך למקומו אע״כ דהתוספ׳ חולקין על הרא״ש ולזה קרא הב״י ראי׳ ברורה ולענ״ד אפילו ראי׳ קטנה ליכא שהתוספ׳ תרצו ממה שנאמר בגמרא בפי׳ דליכא בל תשחית דהיינו בלא טען קבא ובמעולה בדמים לעצים והרי הך דצריך למקומו מותר אתיא מזה דאם הי׳ מעולה בדמים ואחרי שתרצו התוספ׳ מעיקר הדין ודאי לא הוצרכו להביא גם מה שנולד מן הדין הזה דבכללו הוא ולכן כל דברי הב״י לענ״ד אין בהם ממש במכ״ה להביא ראי׳ מהם שהתוספ׳ חולקין על הרא״ש וכבר כ׳ גם בשו״ת חתם סופר י״ד סי׳ ק״ב ובתשובת בית יעקב סי׳ ק״מ כתב דברים שאינם נכונים במכ״ת עכ״ל ומסתמא כוון לדברינו האלה ולכן ודאי דברי הרא״ש הם להלכה:\\\vspace{0pt}

אכן ראיתי בשאילת יעב״ץ ח״א (סי׳ ע״ו) אף שלדינא הסכים עם הראש והט״ז מכ״מ רצה להעלות דגם במעולה בדמים ובצריך למקומו אסור משום סכנתא ומה דאמר רבינא אם מעולה בדמים מותר רק משום ל״ת קאמר אבל משום סכנה אסור ומה שאמר שמואל לאריסיה אייתי לי מקורייהו איכא לאוקמי׳ בדלא טעין קבא אי נמי דהוי מטי זימנייהו עכ״ד ולענ״ד כמה מן הדוחקים יש בזה שנאמר דרבינא אמר מותר אע״ג דאיכא סכנה ולפרש הך דשמואל בדלא טעין קבא שהרי שמואל לא שאל לאריסי׳ על זה כמה טעון האילן ומיד א״ל אייתי לי מקורייהו ומש״כ דאיירי דמטי זימנייהו לא הבנתי כלל איזה זמן יש ועוד דבהך מעשה דרב חסדא שאמר לאריסי׳ עקרינהו משום דגופני קני דיקלא נראה בפי׳ שהתיר מפני שגפנים שוים יותר והנה ראיות היעב״ץ שסכנה יש אע״ג דליכא איסורא הם ממה דאמר ר״ח לא שכיב שכחת ברי אלא דקץ תאנתא בלא זימנא והרי ודאי שכחת בנו צדיק הי׳ שלא מצא לתלות מיתתו רק בזה ואיך עבר על לא תשחית אע״כ דהי׳ בכה״ג שהי׳ מעולה בדמים או בצריך למקומו ואעפ״כ מת ולענ״ד אין מזה ראי׳ דהרי צ״ב מה דקאמר דקץ בלא זימנא איזה זמן יש לקציצה אכן זה מבואר במה שכ׳ הש״מ בשם הגאון בלא זימני׳ שהיתה עדיין טוענת תאנים עכ״ל ועפ״ז נ״ל לפרש דודאי שכחת לא עבר במזיד על לא תכרות רק שהי׳ תאנה זקנה שלא טען עוד רק מעט והוא חשב כיון שלא טען רק מעט ולזמן קצר ייבש לגמרי אין זה בכלל עץ מאכל שמוזהר שלא להשחיתו כמו לרב בשטוען פחות מקב אבל טעה בזה שאפילו לא טוען רק קב עדיין עץ מאכל מקרי לזה סמך הך דר״ח להא דרב לראי׳ על זה דאפילו לא טען רק קב עדיין נקרא עץ מאכל וגם שכחת לא שכב רק דקץ תאנה בלא זימנא שהיתה עדיין טוענת תאנים אף שלא טען רק מעט וא״כ ליכא ראי׳ מזה דגם במעולה בדמים יש בהקציצה סכנה. עוד הביא השו״ת יעב״ץ ראי׳ לזה מב״ב (דף כ״ו) ממה שאמר רבא בר רב חנן לרב יוסף אנא לא קייצנא דאמר רב וכו׳ והקשו התוספ׳ דהא מהא דשמואל מוכח דהיכא דאיכא היזק גפנים שרי לקוץ דקלים ומה שתרצו קצת דחוק אבל אי נימא דגם בכה״ג הוי סכנה א״ש ולענ״ד יקשה אפכא אי איכא סכנה היאך אמר לרב יוסף מר אי ניחא לי׳ ליקוץ וכי יתן לו עיצה להכניס עצמו לסכנה אלא ודאי דליכא סכנה היכא דליכא איסור וקושית התוספ׳ נ״ל לתרץ דעוד הקשו התוספ׳ למה אמר רבר״ח אנא לא קייצנא הרי תנן במתניתא דקוצץ ונותן דמים וכן הקשה הרמב״ן והרחיב הקושיא שהרי מצו׳ להפסיד כל מה שיש לו ולא יפסיד לחבירו פרוטה ותי׳ הרמב״ן שרבר״ח לא הודה לרב יוסף דסבר דאין חילוק בין גפנים לגפנים ובין אילנות לגפנים לכן א״ל דלדידי דסבירא לי דליכא דינא למקצץ אסור למקציי׳ כרב אבל לדידך דס״ל דדינא הוא זיל קץ ומה שלא קבל רבר״ח מרב יוסף משום דאמר בשעת מעשה שאין שומעין לו וכתי׳ זה כתב ג״כ הרשב״א ולענ״ד גם זה דוחק קצת שהי׳ רב יוסף חשוד בעיניו שיורה הוראה להנאתו בדבר שנוגע לאיסור לאו שאם דינא שלא לקוץ הרי עובר על לא תכרות ומי גרע זה ממה דאמרינן יבמות (דף ע״ז) שאני התם דשמואל וב״ד קיים ע״ש והרא״ש תירץ קושית התוספ׳ דלדחיי׳ בעלמא קא מכוון רבר״ח אכן בזה יקשה כקושית הרמב״ן וכי נחשד רבר״ח לעשות שלא כדין בשביל הנאתו שדחה לר״י בדחיי׳ בעלמא אמנם לענ״ד יש ליישב כל הקושיות בחקירה אחת דיש לחקור היכא שהדין שיקוץ אי נימא שבעל האילן המזיק צריך לקוץ כדין מזיק את חבירו שהמזיק חייב לסלק את ההיזק או אי נימא דהניזק יש לו רשות לקוץ אבל המזיק לא חייב והנה בהגהת אשרי כ׳ בשם מהרי״ח ומיהו בעל האילן יכול לומר אנא לא קציצנא אלא את אי ניחא לך קוץ עכ״ל ונראה שיצא לו כן מדברי רבר״ח לרב יוסף אכן לפ״ז ממה שבקש רב יוסף ממנו שיקוץ לכאורה משמע אפכא אבל באמת ממתניתן הוא מוכרע שהרי אמר קוצץ ונותן דמים ואם קוצץ על המזיק קאי ה״ל למימר קוצץ ומקבל דמים אע״כ דבעל האילן אפילו נטעו בלא שהי׳ לו רשות מכ״מ הוא א״צ לקצצו אבל הניזק יכול לקוצצו ואין יכול למחות בידו ואפשר ג״כ שצריך לשלם לו את הדמים כמו בנוטע אילן סמוך לבור לת״ק דמתניתן ומה שאמר רב יוסף זיל קוץ י״ל דלא רצה רק להגיד לו שהדין לקוצצו מפני שמזיק לו ואולי ניחא לי׳ טפי לקוצצו בעצמו ויש נפקותא רבתא לדינא באילן העושה פירות דאם בעל האילן צריך לקוצצו מדינא אז אין בזה דרך השחתה כיון שלא מקיים רק דין התורה וכנראה מדברי הרמב״ן הנ״ל אבל אם הוא א״צ לקוצצו אזי אסור לקוצצו משום לא תכרות דרק הניזק יכול לקוצצו דאצלו הוא מעולה בדמים כיון שניצל עי״ז מהיזק ששו׳ יותר מאילן שהרי קוצצו ונותן דמים אלמא שההיזק הוא אצלו יותר מדמי האילן אבל בעל האילן שאינו מקבל רק דמי האילן אצלו אין כאן מעולה בדמים ולכן אסור שבשביל ריוח הניזק לא מותר לעבור על ל״ת שהרי אמרינן שבת (דף ג׳) לא אומרים לאדם חטא בשביל שיזכה חברך אפילו במקום שמציל חבירו מעבירה כל שכן שאסור בשביל שיזכה חבירו בממון, וזה מה שאמר רבר״ח לרב יוסף אנא לא קייצנא דאמר רב וכו׳ דלי יש איסור וסכנתא שלא יגיע לי ריוח אבל מר אי ניחא לי׳ ליקוץ דאצלך הוא מעולה בדמים לך מותר ואולי רצה לומר לו בזה דאפילו תשלומין לא יבקש ובזה מתורצים קושיות התוספ׳ דא״ש למה רבר״ח לא רצה לקצוץ אף דתנן קוצץ ונותן דמים וגם הא דהזכיר משום הא דרב וסכנה אע״ג דבב״ק אמרינן דמשום היזק גפנים מותר לקוץ דיקלי ומה דאמרו שמואל ורב חסדא לאריסייהו למיקצץ אף שלהם לא הגיע הריוח י״ל או שהאריסים היו נכרים ואפילו היו ישראלים כיון שמושכרים לעבודתם הוי כידם ואפילו שליח בעלמא ג״כ מותר דשלוחו של אדם כמותו ולא שייך בזה אין שליח לדבר עבירה כיון דלמשלח מותר אין כאן שליחות עבירה וזה ברור ונכון לענ״ד.\\\vspace{0pt}

והיוצא מזה לדינא דהיכא דמעולה בדמים או שצריך למקומו מותר לקצוץ מי שיגיע לו הריוח או ע״י עצמו או ע״י שלוחו ואין כאן איסור או סכנה אבל לישראל אחר שרוצה לעשות בשבילו לטובתו אסור, ולכן גם בנדון השאלה כיון שברור דצריך למקומו מותר לקצוץ האילנות ואין כאן חשש איסור או סכנה כלל. ומכ״מ מה שיש לתקן להקל הענין יתקן. והנה תקנה אחת כתב השאילת יעבץ והסכים עמו החתם סופר שיעקור האילנות עם שרשיהן שיכולים לחיות ממנה כשנוטען במקום אחר ואז כשינטעו ויעשו פירות אין בזה השחתת אילן מאכל. אמנם לענ״ד בנדון השאלה שהאילנות כבר זקנים ויבשים במקצת יהי׳ זה ללא הועיל. אכן יש עוד תקנה אחרת שיעשה העקירה ע״י א״י שלדעת הראב״ד והאור זרוע אין שבות לנכרי בשאר איסורים חוץ משבת ואפילו לדעת שאר פוסקים שפסקינן כוותייהו שיש שבות עכ״פ יצאנו מחשש איסור דאורייתא וגם מחשש סכנה ועוד נ״ל כיון דבנדון השאלה עדיין לא נקנה המקום מישראל לכן טוב שיאמר לא״י המוכר שטרם יקנה המקום יעקור האילנות עם שרשיהן וינטעם במקום אחר דאז הוי אמירה לא״י במלאכת א״י שאע״פ שמהתוספ׳ בב״מ (דף צ׳) נראה קצת שמסתפקים בזה אם לא יש איסור שבות בזה מכ״מ מדברי הרמב״ן והר״ן שהביא הש״מ שם נראה שפשיטא להם שמותר. אכן גם באם לא יכול לחוש לתקנות האלה נלענ״ד שמותר לישראל בעצמו לעקור האילנות לצורך מקומו ובפרט בנדון זה שהוא לדבר מצו׳ שיבא עי״ז לכלל נישואין: הקטן יעקב.\\\vspace{0pt}

\end{multicols}\newpage

\newchap{סימן סב}
\begin{multicols}{2}
ב״ה אלטאנא, יום א׳ י״ט מרחשון תרכ״ז לפ״ק.\\\vspace{0pt}

שאלה – מי שחושב דבר להיתר ויודע שחבירו חולק עליו וחושבו לאיסור אם מותר לו להאכיל הדבר לחבירו כאשר לא ידע חבירו ולא ירגיש בו.\\\vspace{0pt}

תשובה – בדין זה מצאתי דעות חלוקות בין הראשונים ולא ראיתי לפוסקים שהביאו דין זה דהאור זרוע כתב (סי׳ תר״ג) וז״ל פשיטא לי אדם שאסר עצמו בדבר המותר לו ואחרים יודעים שהחמיר על עצמו אע״פ שידע שהוא היתר אלא שהדיר עצמו פשיטא לי שסומך על אחרים ואין לחוש שמא יאכילהו כמו שנזיר אוכל בבית ישראל ולא חיישי׳ שמא שם בו ישראל יין באותו מאכל דלא עבר משום לפני עור לא תתן מכשול וכן ישראל בבית כהן ל״ח שמא יאכילנו תרומה אלא בהא נסתפקתי ראובן שפירש עצמו ממאכלים מחמת שנראה לו שאלו המאכלים אסורים לו ושמעון נוהג בהם היתר גמור ונראה לו שראובן טועה בדבר לגמרי אם יכול ראובן לאכול בבית שמעון מי אמרינן הואיל ששמעון תופס את ראובן טועה גם ראובן עצמו אם הי׳ יודע שהוא מותר היה אוכל אלא שמחמת טעותו מניח הרי יש לחוש לראובן שמא שמעון יאכילנו וה״ל לגבי׳ דידי׳ כמו חשוד דתנן פ׳ עד כמה החשוד על הדבר לא דנו ולא מעידו או דלמא לא ספי אינש לחברי׳ מידי דלא כשירה לי׳ ויש לדקדק מההיא דאיפליגו רב ושמואל דרב אמר לא עשו ב״ש כדבריהם ושמואל אמר עשו כדבריהם ואסקינן דעשו ב״ש כדבריהם והאי דלא נמנעו ב״ה מב״ש משום דמודעו להו ופרשי אלמא שסומך עליו ולא חיישינן דילמא ספי לי׳ עכ״ל הא״ז ומזה נראה שדעתו דאסור ליתן להאוסר מה שהוא מותר לדעת הנותן אמנם על הראי׳ שהביא הא״ז לענ״ד יש להשיב דדילמא ב״ה בקשו מב״ש שיודיעו להם ואז ודאי היו יכולים לסמוך עליהם בלא חשש דילמא ספי להו איסורא דשארית ישראל לא ידבר כזב וכש״כ הגדולים אבל אכתי לא מוכח שגם אם לא ביקש ממנו כן שאסור ליתן לו:\\\vspace{0pt}

והנה לכאורה הי׳ אפשר לומר שפליגו בזה רב ושמואל דאהא דאמרינן חולין (דף קי״א) איתמר דגים שעשו בקערה רב אמר אסור לאוכלן בכותח ושמואל אמר מותר לאוכלן בכותח מייתי שם רבי אלעזר הוי קאים קמיה דמר שמואל אייתי לקמי׳ דגים שעשו בקערה וקאכיל בכותח יהיב לי׳ ולא אכל אמר לי׳ לרבך יהיבי לי׳ ואכל ואת לא אכלת אתי לקמי׳ דרב אמר לי׳ הדר בי׳ מר משמעתי׳ אמר לי׳ חס לי׳ לזרעי׳ דאבא בר אבא דליתבי לי מידי דלא סבירא לי ופירש רש״י לא היו דברים מעולם אמנם קשה לומר דשמואל שקורי קמשקר ח״ו אלא משמע דבהא פליגי דשמואל סבירא לי׳ מותר להאכיל למי שאוסר דבר שחושב להיתר ולכן נתן לרב שסבר שידע שהיו דגים שעשו בקערה אבל רב סובר שאסור ליתן ובטח על שמואל שלא יתן לו דבר שהוא אסור לדעתו אלא שראיתי בריטב״א שכתב על מה דאמרינן סוכה (דף י׳) איתמר נויי הסוכה המופלגין ממנה ארבעה רב נחמן אמר כשרה רב חסדא ורבה בר רב הונא אמרי פסולה ר״ח ורבה בר רב הונא איקלעו לבי ריש גלותא אגנינהו רב נחמן בסוכה שנוייה מופלגין ממנה ארבעה טפחים אשתיקו ולא אמרו ליה ולא מידי אמר להו הדור בהו רבנן משמעתייהו אמרו ליה אנן שלוחי מצוה אנן ופטורין מן הסוכה וז״ל רב חסדא ורבה בר ר״ה אקלעו לבי ריש גלותא אגנינהו בסוכה שנוייה מופלגין ממנה ד׳ פירש ואע״ב דאכתי לא ידע ר״נ דהדרו משמעתייהו או דהוו שלוחי מצוה אגנינהו לפום דעתיה ולא חש דהוי חתיכה דאיסורא לדידהו ויתבי בסוכה פסולה ומברכי התם שלא כראוי והוה כנותן מכשול לפני פיקח יש אומר דמהא שמעינן שהמאכיל לחבירו מה שהוא מותר לו לפי דעתו ומאכילו אין בזה משום לפני עור לא תתן מכשול ואעפ״י שיודע בחברו שהוא אסור לו לפי דעתו וחברו בעל הוראה שהמאכיל היה ג״כ ראוי להוראה וסומך על דעתו להאכיל לעצמו ולאחרים לפי דעתו ונ״ל דהכא דוקא מפני שהאיסור ניכר לחברו ואי לא ס״ל לא ליכול הא בשאינו ניכר לחברו לא ואמרינן התם חס לי׳ לזרעיה דאבא בר אבא דליספי לי מידי דלא ס״ל בפרק כל הבשר וכן הורה לי מורי הרב נר״ו עכ״ל הריטב״א הרי שהביא שני דיעות בזה והסכים עם האחרונה דרק בשניכר להאוכל אז מותר להאכילו מה שאוסר אכן לא ביאר מה דעת שמואל בזה שכפי הנראה מהגמרא האכיל לרב מבלי שידע דאל״כ האיך אמר רב חס לי׳ לזרעי׳ דאבא וכן מרב נחמן לכאורה נראה דס״ל דמותר אפי׳ אין ניכר דהלא יקשה מה דעת רב נחמן שבתחלה אגנינהו בסוכה שפסולה לדעתם ואח״כ שאלם אם חזרו משמעתתם לכן י״ל שרב נחמן אגנינהו בלילה בסוכה שלא יכלו להכיר שהנויים מופלגים מסכך ד׳ ושצלתן מרובה מחמתן מה דלדעתם פוסל הסוכה ולכן קאמר אגנינהו ולא אמר בפשטות הושיבום בסוכה להשמיענו שעשה כן מבלי שיוכלו להכיר בלילה שהיה חושך ולא היו יכולים לראות אופן עשיית הסכך משום דסבירא לי׳ מה שהוא מותר לדעתו אין כאן משום ולפני עור ורק תימה עליהם בבוקר כשראו אופן עשיית הסכך שאע״פ כן ישבו בסוכה זו ולכן עתה שאלם אם חזרו משמעתייהו.\\\vspace{0pt}

אמנם גם אי נימא דשמואל פליג ארב הלכה כרב באיסורי נגד שמואל ואפי׳ אם רב נחמן יסבור כשמואל לא יכריע זה להלכה נגד רב דמה דקיימי לן כבתראי זה דוקא מרבא ואילך כמבואר בכללי הש״ס. אבל באמת המעשה דרב בלאו הכי צ״ע דממנ״פ אם הכיר רב שהיו דגים שעלו בקערה איך אמר חס לי׳ לזרעי׳ וכו׳ ואם שמואל האכילו בלא שידע איך הביא שמואל ראי׳ לר״א דמותר ממה שהאכיל לרב ובספרי למסכת סוכה כתבתי איזה תירוצים ליישב זה אכן היותר נלענ״ד בישוב הך מעשה דרב דאמה דאמרינן בגמרא בחולין (דף קי״א) ולא היא שאני התם דנפיש מררא טפי פירש רש״י שהשומע טעה אבל התוספ׳ כתבו כלומר מכאן אין להוכיח אבל הוא בפירוש אמרה כדמוכח בסמוך עכ״ל ולכאורה ראית התוספ׳ ראי׳ גדולה היא דאמר רב חס לי׳ לזרעא דאבא וכו׳ ואין לך מפורש גדול מזה אמנם אפכא ג״כ קשה איך אמר והא דרב לאו בפירוש אתמר והרי לפי דברי התוספ׳ בפירוש אמרה וא״כ מזה יהי׳ ראי׳ לפי׳ רש״י ולכן לענ״ד י״ל דפירוש הגמרא כך הוא דהנה לעיל (דף צ״ה) בהא דאמר רב בשר שנתעלם מן העין אסור קאמר בגמרא ג״כ והא דרב לאו בפירוש אתמר אלא מכללא אתמר וכו׳ ופריך בגמרא וכי מכללא מאי ופירש רש״י מאי בין פירושא למכללא והא שפיר שמעינן מינה ומשני דאיכא בינייהו דליכא למיסמך דסבר רב כן בכל מקום כי אם התם דפרוותא דעכום הוי ע״ש והרי הכא ג״כ קשה כן וכי מכללא מאי ואין לומר דהיינו מה שהשיב ולא היא וכו׳ דזה אינו דא״כ לא ס״ל רב כלל בשום מקום ולא שייך דמכללא אתמר ולא הוי דומיא דלעיל דודאי ס״ל רב בשר שנתעלם מן העין אסור רק לא בכל מקום אלא במקום פרוותא דשכיח טרפות ונ״ל דהראש יוסף מסופק לפי מה דקיימא לן נ״ט בר נ״ט דהתירא מותר אם זה דוקא דאינו מרגיש טעם או אפילו במרגיש הטעם כיון דקלוש הוא ג״כ מותר ושוב הביא מהר״ן והפר״ח דס״ל כן דאפילו בנותן טעם ג״כ מותר ולפ״ז י״ל דגם רב לא אוסר רק במרגיש הטעם אבל בלא טעם כלל גם רב מתיר נ״ט בר נ״ט והיינו דקאמר לאו בפירוש אתמר כיון דבתחלה אמר סתמא דגים שעשו בקערה רב אמר אסור לאכלן בכותח שיש להבין מזה אפילו אין בדגים טעם בשר כלל לזה קאמר דרב לאו בפירוש אלא מכללא אתמר ממה שאמר יהיב טעמא כולי האי ומזה הי׳ אפשר להבין שעל ידי זה שפט רב דבכל נ״ט בר נ״ט נרגש כמו בטעם גמור ועל כן קאמר ולא היא שאני התם דנפיש מררא טפי ולכן הרגיש ולא בכל נותן טעם בר נ״ט נרגש כן אבל מכ״מ שמעינן מזה דגם רב לא אוסר רק בנרגש טעם מעט ממה דקאמר יהיב טעמא כולי האי ושמואל גם בכה״ג התיר כמו שכתב הר״ן ובזה מתורצת הקושיא דממנ״פ או רב או שמואל לא אמרו אמת אבל לפי דברינו אתי שפיר די״ל דשמואל האכיל לרב בכותח אחר שטעם תחלה שלא הי׳ נרגש טעם כלל דידע דרב מודה בכה״ג דמותר וככה רצה להאכיל גם לר׳ אלעזר ולא רצה לאכול משום ששמע בשם רב שאוסר ולא ידע לחלק בין נרגש טעם ללא נרגש ולזה אמר לו שמואל באמת הלא לרב האכלתי ואכל ור״א שלא ידע לחלק סבר שחזר רב משמעתי׳ ואמר לי׳ רב חס לי׳ וכו׳ ואם נתן לי בודאי הי׳ באופן שלא נרגש הטעם ולכן שתיהם אמת מה שאמר רב ומה שאמר שמואל ובזה מתורצת ג״כ הסתירה דקאמר לאו בפירוש אתמר ומה דקאמר אח״כ דאמר רב בעצמו חס וכו׳ ולא סבירא לי דגם מזה לא ידענו רק מכללא איך אוסר רב שדוקא בנרגש טעם.\\\vspace{0pt}

ולכן נלענ״ד להלכה כיון שמהנך מעשיות דרב ור״נ אין ראי׳ גמורה שיש שמתיר להאכיל לחבירו מה שלדעתו אסור ואפי׳ יפלגו שמואל ור״נ ארב מ״מ כיון שמדברי רב דאמר לי׳ חס לי׳ לאבא נראה בפירוש שאוסר והלכתא כרב באיסורי לכן אין להתיר להאכיל לחבירו מה שאסור לדעתו אפי׳ נראה בעיני הנותן להיתר גמור וכמו שנראה ג״כ מדברי האור זרוע ומדברי הריטב״א שהסכים מורו (שהוא הרא״ה או הרשב״א) עמו ורק באפשר להכיר בהדיא מותר להאכילו ואולי גם בזה צריך להודיעו. ולכן בנתארח האוסר אצל המתיר שהוא בעל הוראה אם יכול לסמוך עליו שלא יאכילנו מה דלא ס״ל אף דהא״ז מתיר מכ״מ לפי מה שכתבתי שאין ראי׳ מהא דלא נמנעו ב״ה מב״ש דדלמא בקשו מהם שיודעיו להם ולכן לענ״ד לא יסמוך עליו בסתם אלא ישאלנו בכל מקום שיש אצלו ספק ורק באי אפשר לשאלו יש לסמוך על פסק הא״ז כנלענ״ד הקטן יעקב.\\\vspace{0pt}

\end{multicols}\newpage

\newchap{סימן סג}
\begin{multicols}{2}
ב״ה אלטאנא, יום ה׳ כ״א סיון תרי״ח לפ״ק. להראש ביהמ״ד הרב מ״ה אברהם אש נ״י בק״ק נייאיארק.\\\vspace{0pt}

אשר שאל מר נ״י כי נדחקו ראשי ביהמ״ד לקנות להם בית לביהמ״ד ולא יוכלו להשיג רק בית שנבנה בתחלה לבית דירה ואח״כ קנאו אותו כהני האומות שעובדים ע״ז בשיתוף והתפללו שם לאליל שלהם איזה שנים אבל לא הביאו שם שום תמונת ע״ז ועתה מכרוהו אם מותר לעשות בית זו לבית המדרש לתורה ולתפלה.\\\vspace{0pt}

תשובה הנה מצד איסור הנאה ודאי אין חשש בזה דאפילו היו עע״ז ממש ואפילו נדון הבית כתשמישי ע״ז הרי כיון שמכרוהו בטלוהו כמבואר י״ד (סי׳ קל״ט) וא״כ מותר בהנאה אכן אם מאוס לגבו׳ או לא תלוי בפלוגתא דלפי משכ׳ המג״א (סי׳ קנ״ד ס״ק י״ז) בשם הרא״מ בבית שעבדו שם ע״ז אין בו משום מיאוס להשתמש בו דבר שבקדושה אבל העיר שם בדגול מרבבה שמהתוספ׳ מגילה (דף ו׳) משמע שס״ל שאסור ע״ש ולכן הי׳ נראה קצת לאסור אם לא מצדדי היתר דלאו לע״ז ממש נשתמש הבית דב״נ לא מצווים על השיתוף וגם לא הכניסו בתוכו ע״ז ממש אכן בזה יש לפקפק דמה בכך שלא הכניסו ע״ז מכ״מ הרי התפללו בו לאליל להם והאומר לע״ז אלי אתה והמקבלו לאלוה אפי׳ שלא בפניו ג״כ מקרי עע״ז כדילפינן בסנהדרין (דף ס״ג) מויאמרו לו אלה אלדיך וגם מה שב״נ אינם מצוים על השיתוף לא מהני שהרי ישראל נצטו׳ ואכתי לגבי ישראל מקרי עע״ז ממש כנראה (שם) ולכן לכתחלה ודאי יש להחמיר אבל לעת הצורך כדי הרא״מ לסמוך עליו בנדון זה שלא נבנה הבית לכתחלה להיות בית ע״ז רק להיות בית דירה ולא נקרא שם עע״ז עליו י״ל דגם להתוספ׳ לא נקרא מאוס כנלענ״ד הקטן יעקב.\\\vspace{0pt}

\end{multicols}\newpage

\newchap{סימן סד}
\begin{multicols}{2}
ב״ה אלטאנא, בחדש מרחשון תרכ״א לפ״ק. למחותני הרב היקר וכו׳ מ״ה שמרי׳ צוקערמאן נ״י בק״ק מאהילעו יע״א.\\\vspace{0pt}

אשר שאל מר נ״י ממני – מה נקרא בפרהסיא לענין מחלל שבת ולענין שאר דברים אם בעינן שיעשה העבירה בפני עשרה מישראל כשהם ביחד או גם בזה שלא בפני זה.\\\vspace{0pt}

תשובה: גרסינן בסנהדרין (דף ע״ד) וכמה פרהסיא אמר רבי יעקב אמר רבי יוחנן אין פרהסיא פחותה מי׳ בני אדם פשיטא ישראלים בעינן דכתיב ונקדשתי בתוך בני ישראל ע״ש ומדילפינן מונקדשתי וזה ודאי שאין דבר שבקדושה פחות מעשרה רק כשהם ביחד ולא כשבאו והלכו זה אחר זה א״כ פשיטא ג״כ שגם לענין חילול שבת לא נקרא פרהסיא אלא בפני עשרה מישראל כשהם ביחד אלא שיש להעיר דבי״ד (סי׳ קנ״ז) על מה דפסק בש״ע שם בשלש עבירות אם הוא בפרהסיא דהיינו בפני עשרה מישראל חייב ליהרג ולא יעבור כתב הש״ך (ס״ק ד׳) ואין רצה לומר בפניהם ממש אלא שיודעים מהעבירה והכי מוכח בש״ס ופוסקים גבי והא אסתר פרהסיא הוי ע״ש עכ״ל ואם ידיעה בלבד מספיק להקרא פרהסיא א״כ יקרא ג״כ כשנעשה העבירה בפני עשרה מישראל בזה אחר זה דהא עכ״פ יודעים הם אלא דלענ״ד צריך עיון כיון דילפינן מונקדשתי בתוך בני ישראל היאך יספיק ידיעה מעשרה בלבד דאם לא בעינן שיעשה העבירה בתוך עדה קדושה הלא להיות פרהסיא יספיק גם בעובר בפני שנים והם העידו עליו ברבים כיון שנאמנים עליו אפילו להרגו ודאי רבים ידעו את הדבר אלא ע״כ בעינן החילול דוקא בפני עדה ומה דהוכיח הש״ך מוהא אסתר פרהסיא הוי לענ״ד הפירוש שגם שם הי׳ בפני עשרה מישראל יחד כמו שפירש הר״ן שם דיותר מעשרה ישראל היו בשושן כשלקחה המלך לאשה עכ״ל וממה שפרט שהיו בשושן נראה שדעתו שרק לכך נקרא פרהסיא בעבור שבפניהם נלקחה ואע״ג דהבעילה לא היתה בפניהם מ״מ מפורסם היה שלקחה לאשה בפניהם והכל יודעים שתבעל ואפשר שגם הש״ך לא כתב כן רק לענין גילוי עריות ששם נקרא פרהסיא ע״י שתאנס בפני עשרה שיודעים בודאי שתבעל אפי׳ לא היתה הבעילה בפניהם שהרי גם לענין קידושין הדין כן שקידושי כסף ושטר בעינן שיראו העדים מעשה הקידושין עצמו ובקידושי ביאה נקראו עדים כשיראו היחוד לשם ביאה לבד כמבואר באבן העזר (סי׳ ל״ג) אבל בשאר עבירות אפשר שגם הש״ך מודה דלא נקרא פרהסיא ע״י ידיעה לבד כי אם שיעשה המעשה בפני עשרה מישראל כשהם יחד שנקראו עדה כנלענ״ד הקטן יעקב.\\\vspace{0pt}

\end{multicols}\newpage

\newchap{סימן סה}
\begin{multicols}{2}
ב״ה אלטאנא, בחדש סיון שנת תרי״ז לפ״ק.\\\vspace{0pt}

נשאלתי עשיר א׳ ישראל נתן במותו סך גדול שיהי׳ לסיוע לבני העיר בין ישראל בין אינם ישראל דהיינו שילוו ממנו לאשר צריכים ללות סך מסויים של אלף או אלפים דוקא לאנשים שניכרים שהם בני אומדנא כגון סוחרים ובעלי אומניות להרווחתם למען הרחיב מסחרם ואומנתם וישלמו אחר שנה ושנתים ויתנו רבית ב׳ למאה לשנה והרבית יתאחד עם הקרן להרבות אותו ותיקן כי בכל עת יהיו הממונים הנתמנים על סך הנ״ל מקצת מישראל ומקצת מאינם ישראל אם מותר לישראל ללות מסך הזה ולשלם הרבית?\\\vspace{0pt}

תשובה – הנה הרא״ש והטור ואחריהם הש״ע סי׳ ק״ס אסרו להלות מעות של עניים ברבית קצוצה דאורייתא וכן הסכים הרמ״א ואפילו ברבית דרבנן שהתירו הנך פוסקים יש מחמירין כמשכ׳ הרמ״א שם וכפי הנראה מהש״ך הסכים עם המחמירין ואפילו הרשב״א בשו״ת שהביא הב״י שם שנראה מדעתו שרצה להתיר אפילו רבית קצוצה מכ״מ הרי נתיירא להורות כן למעשה וגם מהרי מטראני אסר ברבית קצוצה כמשכ׳ המשנה למלך בשמו פ׳ ד׳ מה׳ מלו׳ ואע״פ שהמשנה למלך כ׳ עליו ולא ידעתי למה לא הזכיר הרב סברת הר׳ שמואל שמביא המרדכי פ׳ המפקיד שמתיר מעות עניים ברבית קצוצה משום דהוי כהקדש גבו׳ וכן כ׳ הגה״מ בשם רבינו ברוך מכ״מ מי יקל נגד כל עמודי הוראה הנ״ל אשר בית ישראל נשען עליהם בדבר הנוגע לדאורייתא. ומה שנוהגין עתה שקהל לוה מעות של דברים טובים ונותן אפילו רבית קצוצה נראה שסומכין על ההיתר שכתב הד״מ ואחריו ברמ״א סי׳ הנ״ל דנחשבו צרכי הקהל כפקוח נפש ואף שכבר כתב הרמ״א שאין לסמוך על זה רק לצורך גדול נראה עוד שסומכין על מה שכתב הרשב״א בשו״ת הנ״ל בענין להלוות מעות ת״ת ברבית לישראל וז״ל אבל יש מלווים כזה להתיר כדי לפרנס הקטנים והתלמידים שבמקומם לפי שעליהם להתעסק בפרנסתם ולימודם ואין זה רבית אלא שזנין ומפרנסים הלומדים שהחיוב עליהם לזונם ולפרנסם עכ״ל וכיון דכל עניני ד״ט של הקהל הם בענין זה לצורך ת״ת או לפרנס העניים שבלא״ה מוטל על הקהל לא נחשב זה רבית והנה כל היתרים שהוזכרו בזה לא שייכי בנדון השאלה שאין זה רבית דרבנן רק רבית קצוצה ואין הלוים והמלוים קהל ששייך לענין זה דמיון פקוח נפש וגם אין בזה חיוב פרנסה מצד הקהל כי הרבית שנותן הלו׳ אינו נותן לצורך פרנסת עניים אלא להרבות מעות המלו׳ להלותו שוב ברבית ולכן לענ״ד לא לבד לכל הפוסקים הנ״ל זה נאסר כי אם גם לשיטות המתירים שהביא המ״ל שהם המרדכי בשם ר״ש והגה״מ בשם ר״ב שהם כתבו שמותר משום דהוי כהקדש גבו׳ והיינו דדרשינן בר״ה (דף ה׳) בפיך זו צדקה ואתקש שם בפסוק דועשית כאשר נדרת לשאר הקדשות דדרשינן מפסוק זה שם א״כ דינו כהקדש כמו דשו׳ גם לשאר דברים להקדש כגון לענין בל תאחר וכדומה אבל בנדון השאלה לענ״ד אין לדמות זה לצורך צדקה כיון דהרבית לא ניתן לצדקה אלא להרבות הון או לשכר המתעסקים בהלואה ואף שהמעות הזה ג״כ שוב נלו׳ לאחרים ולהלות לעני בשעת דחקו ג״כ צדקה נחשב כמבואר בי״ד סי׳ רמ״ט מכ״מ זה דוקא כשמלו׳ לעני בלי רבית דאתיא ממה דאמרינן בשבת (דף ס״ג) גדול המלו׳ וכו׳ וזה ודאי בלי רבית דאתיא מוכי ימוך אחיך ומטה ידו עמך והחזקת בו והרי על זה נאמר לא תקח מאתו נשך ותרבית אבל להלות לעני ברבית אף שלוקח ממנו פחות ממה שצריך לתת לאחרים אין לדמות זה לצדקה ואפילו לגמילות חסד אין לדמות דג״ח הוא מה שנעשה בין לעניים בין לעשירים כדאמרינן סוכה (דף מ״ט) וג״ח זה אינו לעניים כלל שאין נותנים בהלואה רק סך גדול שאין בו צורך לעני רק לאומנים ולבעלי מסחר להרווחתם ולהצלחתם במסחר ובאומנות ואין זה רק כמו שאר דברים שנודבים לפעמים יושבי עיר בשביל טובת העיר ולא ידעתי בזה דמיון לצדקה שנילף מפיך זו צדקה ושדומה להקדש ולכן לענ״ד אפילו המתירים בהקדש של עניים מסכימים בענין זה לאיסור.\\\vspace{0pt}

ואם כי מצד זה לא ידעתי היתר לדבר, לענ״ד אין למצוא ג״כ היתר במה שמעות המלו׳ אינו לישראל לבד רק לישראל ולא״י בשו׳ או מפני שהרוב יושבי העיר שזכו במעות ד״ט הללו הם א״י שהנה ראיתי לחקור למה לא יהי׳ מעות שותפות אפילו של שני ישראלים מותר להלות ברבית שהרי כתיב את כספך לא תתן לו בנשך וכל מקום דכתיב כינוי דך ממעטינן שותפות כמבואר בחולין (דף קל״ה) לענין תרומה וראשית הגז ופיאה ובכורה ומזוזה ומעשר ובכורים וציצית ומעקה דבכל הני מצרכינן שם רבוי לרבות שותפין דלא נמעט מך והכא ליכא ריבוי ואפילו נימא דגם הכא כתיב ריבוי כיון דכתיב בסוף הפרשה אני ד׳ אלדיכם וגו׳ דקאי ארבית ומדכתיב לשון רבים מרבינן שותפין כמו בכל הני דשם מכ״מ יקשה ששותפות ישראל עם א״י יהא מותר להלות ברבית אפילו מהשותף ישראל דהא בכל הני דאמרינן שם דמרבינן שותפות מריבוי דלשון רבים ממעטינן שותפות נכרי ממיעוט דלשון יחיד וזה לא הוזכר בשום מקום להיתר ואפילו ללות משותף נכרי ודאי אסור כנראה מסתימת לשון הפוסקים שלא הזכירו היתר זה וגם הוקשה לי למה לא קחשיב רבית בהדי הנך דרשות בחולין בסוגיא הנ״ל. שוב ראיתי דלק״מ דז״ל ת״כ פ׳ בהר כספך ולא כסף אחרים ואכלך ולא אוכל אחרים או כספך ולא כסף מעשר ואכלך ולא אוכל בהמה כשהוא אומר נשך כסף לרבות כסף מעשר נשך אוכל לרבות אוכל בהמה עכ״ל ופי׳ הקרבן אהרן ולא כסף אחרים דהיינו של נכרי דנעשה לו ערב ליקח לו מעות ברבית מן הנכרי א״נ כספו של נכרי ביד ישראל מותר להלוותו ברבית לישראל וכן שנינו בתוספתא וכו׳ יע״ש הרי ממיעוט דכספך ממעטינן נכרי לבד שאין לו לישראל זכות בו ומנשך כסף סתמא מרבינן כל שיש לישראל בו זכות ואף דאינו נקרא ממונו שהרי לענין חלה ומצה ואתרוג למ״ד מעשר ממון גבו׳ ממעטינן מעשר כדאמרינן סוכה (דף ל״ה) מכ״מ הכא מרבינן אף שמ״ד דת״כ ע״כ ס״ל דממון גבו׳ הוא דאל״כ איך ס״ד למעט מעשר מכספך דצריך רבוי וא״כ כל שכן השותפות ישראל עם נכרי מרבינן לאיסור מרבויא דנשך כסף ולכן לא הוזכר בפוסקים מזה להיתר ועוד דאפילו נימא דשותפות ישראל ונכרי מותר ללות ברבית יהי׳ תלוי עדיין בפלוגתא דברירה כמבואר בחולין (שם) לענין ישראל ונכרי שלקחו שדה בשותפות והרי לחומרא פסקינן דיש ברירה וא״כ כאן נברר החלק של ישראל כשלו׳ הישראל ובלא״ה לפי מה שכתב רש״י שם לענין דיגון לא שייך לפטור משום שותפות נכרי אלא דבר שאי אפשר לחלק כגון הצאן לענין ראשית הגז אבל תבואה הנחלק במדה קרינן בי׳ דגנך בחלקו של ישראל יע״ש וא״כ הכא נמי המעות שאפשר לחלוק חלקו של ישראל ושל נכרי שהרי מטבעות חלוקות הם לא שייך לפטור מטעם שותפות נכרי. ולכן אפילו הי׳ נדון השאלה דומה לשותפות נכרי לא ידעתי היתר בזה דבשותפות לא שייך חילוק לענין רוב שהרי פוטרין בהמת ישראל מן הבכורה כשיש לנכרי רק חלק מעט בהאם או בהולד. אכן לענ״ד אפילו לשותפות נכרי אין לדמותו כיון שהד״ט ניתן מישראל ונקרא על שמו וכי מפני שהרשה להלות גם לנכרי נעשו בהממון שותפין. ולא נשאר נטיית היתר רק מצד שאין בעלים מיוחדים ובזה כל הפוסקים מסכימים שאין לסמוך על זה לענין רבית קצוצה כנ״ל ולכן בנדון השאלה לא אוכל למצוא היתר שילו׳ ישראל ברבית שהרי הלו׳ ג״כ עובר בלא תשיך כשיתן רבית. כנלענ״ד הקטן יעקב.\\\vspace{0pt}

\end{multicols}\newpage

\newchap{סימן סו}
\begin{multicols}{2}
ב״ה אלטאנא, יום ו׳ ב׳ דר״ח כסליו תרכ״ז לפ״ק. לחתני הרה״ג וכו׳ מ״ה משולם זלמן הכהן נ״י אב״ד דק״ק שווערין יע״א.\\\vspace{0pt}

כתב לי חתני נ״י: נשאלתי מרב אחד הלוה מחברו מעות ונתן לו אתרוג עבור הרבית שהקציץ עמו אם המלוה יכול לצאת באתרוג זה י״ח? והרב נ״י התיר והביא ראי׳ מגמ׳ דקדושין ו׳ ע״ב אמר אביי המקדש במלוה וכו׳ בהנאת מלוה וכו׳ האי הנאת מלוה וכו׳ האי רבית מעליא היא ופירש״י ואמאי קרי לה הרעמת רבית? ומדייק בספר עצמות יוסף אמאי מדייק רש״י רק בלישנא דגמ׳ ולמה לא פירש רש״י אמאי מקודשת ברבית אי הוי רבית מעלי׳ אלא מוכח שרש״י סובר המקדש ברבית מקודשת א״כ לפ״ז הא דאמרי׳ רבית קצוצה יוצאה בדיינים, היינו שצריך להחזיר דמים והגוף של רבית נעשה שלו, הלכך גבי אתרוג הואיל דקנה גוף האתרוג רק צריך להחזיר מעות אחרת, האי אתרוג מקרי לכם ויוצא בו אפי׳ בי״ט ראשון. ואי משום מצוה הבאה בעבירה זה נמי ליכא לפמ״ש התוס׳ בריש לולב הגזול כיון דצריך להחזיר רק הדמי׳ מטעם וחי אחיך עמך לא נתוסף אסור רבית מחדש כמו גבי גזל עכ״ל הרב נ״י.\\\vspace{0pt}

וזאת תשובתי – על ד״ת אשר מעכ״ת נ״י כבדני בהם אומר אני בקיצור אי הוי אתית שאילתא דא קמאי לא שרינא הוינא לצאת ידי חובתו באתרוג הניתן בעד רבית קצוצה לא בי״ט שני ומכ״ש בי״ט ראשון. דהנה הראי׳ שהביא מעכ״ת נ״י מרש״י דפ״ק דקדושין ו׳ ע״ב על מה דמקשי׳ בגמ׳ הא רבית מעלי׳ היא ופירש״י ואמאי קארי לי׳ תערומת רבית משמע דרש״י ס״ל דאשה מקודשת ברבית קצוצה לפעד״נ דמתוך דיוק זה לא נוכל עדיין להתיר לקדש אשה בר״ק. חוץ ממה שכ׳ הפ״י בסוגי׳ זו ע״ש י״ל נמי מש״ה לא פירש״י דאסור לקדש בר״ק דזה ממילא משתמע כיון דיוצאה בדיינים א״כ אינה שלו והיכי יקדש האשה אלא רש״י מפרש להקשות אביי אאביי גופא דהנה בב״מ ס״ב ע״ב מקשי׳ בגמ׳ וכי אין לו יין מה הוי דהתניא אין פוסקין וכו׳ א״ל אביי אי דלא כאיסורא וכו׳ אלא אמר אביי מתני׳ וכו׳ דתני ר׳ ספרא וכו׳ יש דברי׳ שהם מותרי׳ ואסורים מפני הערמת רבית כיצד וכו׳ ע״ש א״כ מרא דשמעתתא התם הוא אביי והוא סופר ומונה הדברי׳ שאסורי׳ משום הערמת רבית א״כ מקשו הגמ׳ הכא בסוגי׳ דקדושין היכי קאמר אביי דאזקפה דאמר לה ד׳ ב״ה הוי הערמת רבית הא ודאי רבית מעלי׳ הוא דהערמת רבית אינה אלא כגון זו שמנה אביי התם בב״מ משא״כ הכא. א״כ לעולם אימא לך דרש״י ס״ל נמי דאסור לקדש בר״ק ואי ק׳ למה לא פירש״י כן בהדי׳ ז״א דרש״י לא נחית לאקשויי אקדושין שאינם משום רבית דהא ממילא משתמע אלא ניחא לרש״י טפי לאקשויי על אביי גופי׳ דהכא קרא לי׳ תערומת רבית ושם בב״מ משמע דדוקא כגוונא דהתם מקרי הערמת רבית אבל כגון זו דהכא ר״ק מקרי ולא הערמת רבית. ומש״ה באמת לא כתב העצמות יוסף שמדייק בדברי רש״י כן דס״ל בודאי דמקודשת ע״י ר״ק אלא כתב ואפשר שדעת רש״י נוטה לדברי הריטב״א, אפשר ולא ודאי. – ועוד אפילו תימא דרש״י בודאי ס״ל דבר״ק מקודשת היינו דוקא בכסף עצמו אבל לא בחפץ אשר נתנה לו בר״ק כמש״כ בהדי׳ ב״ש אה״ע סי׳ כ״ח ס״ק כ״ו. והנה כ״ז לדעת הריטב״א אבל לפי משכ׳ הרי״ו הובא שם בב״ש אינו מקודשת אפילו בכסף הניתן בר״ק. וכן פסק הט״ז שאם מקדש ברבית קצוצה אינה מקודשת ומדמי לגזל כיון דצריך להחזיר בדיינים ע״ש באה״ע. – א״כ ממילא בנידן דידן ודאי להרי״ו דקסבר דאינה מקודשת בר״ק א״כ ה״נ אינו יוצא באתרוג, אלא אפי׳ להריטב״א הא נמי ס״ל דדוקא בכסף מקודשת אבל בחפץ לא כמ״ש הב״ש הנ״ל א״כ הכא נמי דא״י באתרוג הניתן לו בר״ק ועיי׳ ביו״ד קס״א בש״ך ס״ק י״א דכל דבר בעין מקרי דבר מסויים ואפי׳ בניו צריכים להחזיר וכיון דט״ז הנ״ל מדמי רבית לגזל לכל מילי ע״ש א״כ נמי בי״ט שני אינו יוצא משום מצוה הבאה בעברה או עכ״פ ספק הוי אי מיקרי מה״ב כמ״ש מו״ח הגאון נ״י בספרו ע״ל על סוכה פ׳ שלישי דף נ״א לספרו ד״ה שם בא״ד דמחמת עבירה וכו׳ יע״ש זהו מה שהשבתי להרב הנ״ל נ״י. ועתה אבקש ממו״ח נ״י לחוות לי דעתו הרמה בזה.\\\vspace{0pt}

תשובה – דין זה לענ״ד תלוי בפלוגתא בין פוסקים ראשונים ובין האחרונים דהנה הטבת לראות שכבר הביא הב״ש (סי׳ כ״ח ס״ק כ״ו) פלוגתא שבין הראשונים במקדש אשה ברבית קצוצה שלדעת הריטב״א מקודשת ולדעת רבינו ירוחם אינה מקודשת וכבר הובא פלוגתא זו גם בכנסת הגדולה ולכן מסכים להלכה דהוי ספק מקודשת אכן גם במה שהביא הב״ש שם דבלקח ממנה איזה חפץ ברבית ומקדש בו דיש לומר דלכ״ע אינה מקודשת די״ל דידי שקלתי מצאתי פלוגתא בין האחרונים שכדברי ב״ש כתב ג״כ המשנה למלך (פרק ח׳ מהל׳ מלוה) וז״ל מי שהיה נושה בחבירו ד׳ דינרים של רבית ונתן לו בהם חפץ כו׳ נראה שמי שהלוה לחבירו ונתן לו הלוה חפץ ברבית שצריך להחזיר לו החפץ ואין יכול המלוה ליתן לו דמיו דדוקא כשקצץ עמו בדמים ואח״כ נתן לו חפץ בשביל הדמים אז הוא דאמרינן המקח קיים ויחזיר הדמים אבל בשלא קצץ צריך להחזיר לו החפץ בעצמו שהרי הלוקח לא קנאו והמוכר לא מכרו עכ״ל הרי שהסכים ג״כ עם הב״ש שהמלוה לא קנה החפץ וא״כ אם קידש בו אשה פשיטא שאינה מקודשת לדעתו אמנם מצאתי שני נביאים מתנבאים בסיגנון אחד שחולקים על הב״ש ועל המ״ל האחד הוא הבעל קצות החושן שכתב בספרו אבני מלואים דנראה לו דאפילו קצץ לו כלי ברבית ונתן לו הכלי נמי קנה הכלי ואין צריך להחזיר משום וחי אחיך עמך אלא דמי הכלי וכן מוכח מהא דאמר רבא בבבא מציעא (דף ס״ה) גלימא מפקינן דלא לימרו אינשי מכסי גלימא דריבתא ומבואר דשאר חפצים אינו צריך להחזיר דקנינהו וקיימא לן כרבא ולפי״ז אם קידש אשה בחפץ שלקח בו רבית הרי זה מקודשת כיון שהחפץ שלו אלא דמים הוא חייב משום וחי אחיך עמך ועוד ראי׳ כתב לזה מהא דתניא הניח להן אביהן מעות של רבית אעפ״י שהן של רבית אין חייבין להחזיר הניח להן אביהן פרה וטלית וכל דבר המסויים חייבין להחזיר ואי נימא דחפץ של רבית לא קנה הלוקח והמוכר לא מכרו א״כ מעולם לא היה של אביהם וא״כ אפילו אינו דבר המסויים יהיו חייבים להחזיר ואפילו בדבר מסויים כגון פרה וטלית אין חייבין להחזיר אלא מפני כבוד אביהם אלא ודאי קנאן המלוה קנין גמור ואינו אלא חוב להחזיר דמים משום וחי אחיך ושכן מבואר להדיא בחדושי הריטב״א בקידושין שכתב וז״ל אבל ודאי אפילו ברבית גמורה אם כבר פרעתו לו וחזר וקידשה בו מקודשת דמעות דריבתא שפרע הלו׳ למלוה קנינהו לגמרי וממון גמור הם לו אלא שיש עליו חוב להחזירו וב״ד מוציאין ממנו ואם מת אין בניו חייבין להחזיר אלא מפני כבוד אביהן בדבר מסויים עכ״ל הריטב״א ומבואר להדיא דגם בחפץ הדין כן משום דרבית קנה המלוה לגמרי אלא שחייב להחזירו משום וחי אחיך עמך אין צריך להחזיר גוף החפץ אלא דמי החפץ משום וחי אחיך עמך עכ״ד האבני מלואים וכדבריו וראיותיו כתב ג״כ הבעל נתיבות המשפט בחושן משפט (סי׳ ר״ח) ובי״ד בספרו חוות דעת (סי׳ קס״א) ואם שאיני כדאי להכריע בין גדולי הפוסקים מ״מ לענ״ד יש להשיב על ראיותיהם שמה שהביא הא״מ ראי׳ ממה דאמר רבא גלימא לא מפקינן מיניה לענ״ד אינו ראי׳ כלל נגד המ״ל דשם איירי שלא התנה עמו בתחלה ליתן לו גלימא ברבית אלא מעות ולהכי אמרינן שם (ב״מ דף ס״ה) ואמר אביי האי מאן דמסיק ארבעה זוזי דרביתא בחברי׳ ויהיב ליה גלימא בגוויהו כי מפקינן מיני׳ ארבעה מפקינן מני׳ גלימא לא מפקינן מני׳ רבא אמר גלימא מפקינן מני׳ מאי טעמא כי היכי דלא לימרו גלימא דמכסי וקאי גלימא דריבתא הוא הרי דאיירי היכא שהיה לו לתבוע ממנו מעות דריבתא ארבע זוזי ויהיב לי׳ גלימא תחתיו וזה מקח בעלמא הוא ובזה גם המ״ל מודה דהמקח קיים ויחזור הדמים כמו שכתב דדוקא כשקצב עמו בדמים ואח״כ נתן לו החפץ בשביל הדמים אז הוא דאמרינן המקח קיים ויחזיר הדמים ולכן בזה מן הדין לא היה צריך להחזיר הגלימא עצמו ולזה הוצרך רבא לטעם משום דלא לימרו וכו׳ אבל כשקצץ לו החפץ בעצמו ברבית י״ל דכ״ע מודי דלא נקנה לו החפץ וצריך להחזיר החפץ עצמו וכן משמע מדנקט פלוגתא דאביי ורבא דווקא במסיק ארבעה זוזי דרביתא ויהיב ליה גלימא בגוויהו ולא נקיט הפלוגתא בקצץ עמו גלימא עצמו ברביתא משמע דבזה גם אביי מודה דגלימא מפקינן מיניה וגם על הראי׳ מהניח להן אביהן מעות של רבית וכו׳ לענ״ד יש להשיב דאעפ״י שהאב לא קנה דבר המסויים עצמו דהוי כמו גזל ביאוש דיאוש כדי לא קנה מכ״מ כשבא ליד הבנים אחר מיתת אביהן הוי אצלם יאוש ושינוי רשות דיד יורש כיד אחר דמי וקנו ולכן אי לאו משום כבוד אביהן אינן צריכים להחזיר וגם מה שכתבו מצד הסברא כיון דאין צריך להחזיר אלא משום וחי אחיך א״כ אינו צריך להחזיר רק מעות כדי להחיותו ולא החפץ עצמו לענ״ד אין זו סברא מכרעת דיש לומר כיון דהחפץ היה שלו לצרכו והתורה ציותה שלא ליקח ממנו ברבית לכן כשציותה להוציא ממנו משום וחי אחיך עמך לא מקויים וחי אלא כשמקבל מה שהיה לו דמי נתן למלו׳ רשות למכור או ליטול חפץ הלוה שהיה לצרכו ולחייו וליתן לו מעות תחתיו שהרי בודאי האי וחי אחיך לאו דוקא שיתן לו כדי חייו שהרי אינו צריך ליטול ממנו יותר ממה שקבל א״כ האי וחי אחיך לא קאי רק על מה שנטל ממנו והוא חפץ הרבית וגם מדברי הריטב״א שהביא הא״מ ראי׳ וכתב שמבואר להדיא לענ״ד אינו מבואר כל כך שאף שסיים מהך דאין בניו חייבים להחזיר אלא מפני כבוד אביהם לדבר מסויים מכ״מ בתחלת דבריו כתב דמעות דרביתא שפרע לוה למלוה קנינהו לגמרי ומעות גמור הם לו ומזה משמע דלא כתב כן רק על מעות ולא על חפץ מדלא השמיענו רבותא טפי דאפילו חפץ אינו צריך להחזיר ולכך י״ל שמה שהזכיר מהבנים לא נקט רק לומר שעליו יש חוב להחזיר מה שאין כן בבנים שאין עליהם להחזיר אפי׳ מעות רק דבר מסויים משום כבוד אביהם אבל לא מוכח מזה דדעת הריטב״א שגם המלוה עצמו אין צריך להחזיר דבר מסוים.\\\vspace{0pt}

אמנם כבר הזכרתי שאיני כדאי להכריע ולכן גם בקדשה בחפץ שנטל ממנה ברבית הוי ספק מקודשת. והנפקותא לענין נדון השאלה דלדעת הב״ש והמל״מ שצריך המלוה להחזיר חפץ עצמו א״כ לא מקרי לכם ואם נטלו ביום הראשון לא יצא אבל לדעת הא״מ והחוות דעת שקנה האתרוג וא״צ לשלם רק דמים אם נטלו יצא אפילו ביום הראשון אכן מה שכתב הרב השואל שיכול ליטלו אפילו ביום הראשון לענ״ד בודאי לא צדק בזה שאף על פי שכפי המבואר א״ח ריש (סי׳ תרמ״ט) לא מקרי מצו׳ הבא בעבירה בגזול וקנא׳ בלא סיוע מצוה דהיינו כשקנאו ביאוש ושינוי מעשה וכמו שכתב הריטב״א שהזכרתי בספרי וכן כתב הר״ן מכ״מ הרי מבואר שם שאינו יכול לברך עליו משום ובוצע בירך כיון שעדיין חייב מעות לשלם ולכן נלענ״ד להלכה שאם נטל אתרוג שניתן לו ברבית קצוצה ביום ראשון יש ספק אם יצא או לא וצריך לחזור וליטול אתרוג אחר לצאת ידי ספק ובשאר הימים לכתחלה ודאי לא יטול דאעפ״י דלא בעינן לכם מכ״מ הרי לא יכול לברך עליו משום בוצע בירך אבל אם נטלו יצא. כנלענ״ד הקטן יעקב.\\\vspace{0pt}

\end{multicols}\newpage

\newchap{סימן סז}
\begin{multicols}{2}
ב״ה אלטאנא, טבת תרי״ב לפ״ק. לק״ק אמשטרדם יע״א.\\\vspace{0pt}

שאלה – איש צדיק וחשוב נפל למשכב ויעצוהו לבקש תרופה ע״י הפעולה שקורין מאגניטיזירען שעל ידי הפעולה זו נעשה החולי כישן בלי הרגשה ועוד כפי הספור נעשה שינוי גדול בחולי שנהפך לאיש אחר ומספרים נפלאות שמתוך שינה יודע מה שנעשה ברחוק מאוד ממנו ומספר מה שנעשה בחדרי חדרים וכדומה. אכן בשביל זה לב הצדיק מהסס אם יתעסק בדבר זה שנראה לעין שפועלים בזה כחות רוחניות חוץ לטבע ויש לדאוג שח״ו יש בזה פעולות כחות הטומאות אשר כל שומר נפשו ירחק מהם ואחרי אשר הצדיק על פיו דמר נ״י ילך לכן יורנו רבנו מה דינו?\\\vspace{0pt}

תשובה – שאלתי את פי חכמי אה״ע מה דעתם בפעולת המאגניטיזירען אם יש בו ממש בשינוי הטבע כמו שמספרים או לא ומצאתים בדעות חלוקות יש שאמרו שהכל הבל וכזב ואין נעשה שינוי כלל רק שנתעלה כח המדמה של החולה עד שחושב כאלו רואה נפלאות ויש מהם שאמרו שאמת נעשו מראות נפלאות אשר בודאי יש להם מקור ודרך בטבע אבל עוד נעלם ממנו כל ענין הדברים ורק מעט יש לקרב אל השכל איך אפשר שיהי׳ לטבע מהלך בין המראות האלה. ולכן לענ״ד אפילו לו יהי כן שאין למצוא מפתח כלל איך בדרך הטבע יהי׳ שינוי גדול בכל הענינים ע״י פעולת המגניטיזירען עם כל זה אין אנו צריכים להרחיק ולחוש שמכחות הטומאה נעשה שהרי לפי המבואר בפוסקים ופסק בטוש״ע י״ד (סי׳ קנ״ה) מותר להתרפות ע״י לחש מעע״ז כשאינו בודאי שמזכיר שם ע״ז על הלחש וכשאינו כומר לע״ז דכומר ודאי מזכיר שם ע״ז אבל בספק שרי והרי בלחש ודאי אין מבוא לטבע שיפעל על החולי ואעפ״כ אין חוששין שמא מרפא ע״י כחות הטומאה אלא תלינן שיש הרבה עניני טבע שנעלמו עדיין ממנו ולמה נחוש יותר בענין מאגניטיזירען שעכ״פ העוסקים בו מאמינים שנעשה ע״י הטבע ולא ע״י פעולות רוחניים ואפילו להתרפאות ע״י כשוף יש מהפוסקים שמתירים וגם האוסרים מתירים עכ״פ ביש סכנת נפשות כמבואר בש״ך י״ד רסי׳ קע״ט וכן מצאנו בר׳ יוחנן שהניח עצמו להתרפאות ממטרוניתא נכרית אף שלא ידע איזו רפואה עושה לו כנראה בע״ז (דף כ״ח) ולא חשש שמא תרפא אותו ע״י כחות הטומאה וגם משום תמים תהי׳ אין לחוש דכל שעושין משום רפואה אין בו משום דרכי האמורי אף שהם דברים שאין להם מבוא בטבע כש״כ שאין חשש להתרפאות ע״י דבר המאגניטיזירען אשר העוסקים בו אומרים שהוא דבר נהוג ע״פ הטבע אף שעדיין לא באו עד המקור לידע הענין על בוריו והרי אין לתמו׳ על זה שגם בשאר ענינים אחר כל החקירות אשר חקרו לא ידעו מגודל פעולת הטבע כטפה מן הים ולכן אין אתנו לאסור שנויי הטבע רק מה שאסרה התורה כגון כשוף ונחוש אשר לפני הבורא המצו׳ נגלו כל תעלומות והוא יודע מה שהוא שינוי טבע באמת והרי אפילו לדרוש מן המת בקבר מותר כמו שמצינו גדולי האמוראים שעשו כן כל שאינו דורש לגוף המת רק לרוח כמבואר בי״ד (סי׳ קע״ט) וע״ש בש״ך ס״ק י״ז הרי שאין בידינו לאסור בענינים כאלה רק מה שאסרה התורה בפירוש ולכן לענ״ד מותר להתרפאות ע״י מאגניטיזירען אפילו חולה שאין בו סכנה ויקבל עזרו מקדש: הקטן יעקב.\\\vspace{0pt}

\end{multicols}\newpage

\newchap{סימן סח}
\begin{multicols}{2}
ב״ה אלטאנא, יום ו׳ א׳ דר״ח אייר תרכ״ד לפ״ק. לחחני הרה״ג וכו׳ מ״ה זלמן הכהן נ״י אב״ד דק״ק שווערין יע״א.\\\vspace{0pt}

על דבר שאלתך אודות הזכרת השם בלשון אשכנז – אמת הוא שאני מזהיר מאוד בפרט בני ביתי שלא להזכיר השם לבטלה גם בלשון אשכנז שאף שהש״ך י״ד סימן קע״ט כתב שאין זה שם קדש רק שם חול והוכיח כן ממה שאין איסור למחוק שם בלשון אשכנז לענ״ד אין מזה ראי׳ די״ל דרק לענין מחיקה אין קפידא שהלאו לא תעשון כן לד׳ אלקיך לא נאמר אלא על שם כתוב כזה והרי אפילו שם בלה״ק אם לא נכתב בקדושה דעת רוב הפוסקים שמותר למחקו ולכן כיון שנכתב בלשון אשכנז הוא שם של חול אבל מכ״מ חשיב שם כנראה גם מדברי הרמ״א ממה שכתב ומיהו ברוקק טוב לזהר בכל ענין בפרט אם מזכירין השם שאין לו חלק לעה״ב עכ״ל הרי שמחשיב אותו כשם ותדע דאם לא נחשב שם רק הנאמר בלה״ק היאך אמרינן דברכות וברכת המזון ותפלה וק״ש נאמרין בכל לשון הרי לא מזכיר השם וברכה בלא שם אינה ברכה ואפילו נימא דשאני ק״ש וברכת המזון דגלי קרא הרי עכ״פ גלי קרא בזה שהוא בכלל וברכת את ד׳ אלקיך א״כ הרי נחשב שם ועוד למה פסקינן בכל ספק ברכות להקל משום חשש הזכרת השם לבטלה הרי יש תקנה לברך בלשון אחר אע״כ שמקרי הזכרת השם בכל לשון שמזכירו וכן נראה ממה שכתב הש״ע ח״מ סי׳ כ״ז שהמקלל באחד מהשמות שקורין האומות להקב״ה לוקה שלא גרע זה מכינוי כמשכ׳ הסמ״ע שם וע״ש באורים ותומים שכתב תוכחת מגולה על המון העם שלא נזהרים להזכיר השם בלשון העמים ולכן מאוד יש לזהר ולהזהיר בפרט שהגדילו חכמינו ז״ל העונש על הזכרת שם לבטלה וחמירא סכנתא מאיסורא ולהשומע יונעם. כנלענ״ד הקטן יעקב.\\\vspace{0pt}

\end{multicols}\newpage

\newchap{סימן סט}
\begin{multicols}{2}
ב״ה אלטאנא, יום ד׳ כ״ח תמוז תר״ך לפ״ק.\\\vspace{0pt}

הראב״ד בבעל הנפש שער הפרישה הביא שו״ת רבינו האי גאון ז״ל בכלה שנבעלה בעילת מצו׳ שנוהגת כל ז׳ ימים כנדה וצריכה ז׳ נקיים אלא שאין אנו מטמאין משכב שתחתי׳ בשעת בעילה משום דלאו נדה ודאי היא אלא ספק וכ׳ הראב״ד על זה ואיני עומד על בירור דבריו במה שאמר שאין אנו מטמאין משכב שתחתי׳ בשעת בעילה משום דלאו נדה ודאי היא ואילו היתה נדה ודאי כלום יש טומאה וטהרה בזמן הזה לטמא משכב שתחתי׳ ונראה מדבריו שאסור לישן על מטתה של נדה אפילו בשעה שאינה במטה משו׳ הרגל ודוקא נדה ודאי אבל כלה מותר לישן על אותה מטה לאחר שעמדה מאצלו ואפילו באותה סדין שהדם עליו ואע״פ שאין זה מן ההלכה הרי הם שהדעת מכרעת עליהם עכ״ל וכ״כ גם הרא״ש בשמו ועפ״ז נפסק בטוש״ע סי׳ קצ״ג שאסור לישן על מטה של אשתו נדה אבל על של כלה מותר לישן ובמכ״ה מרבותינו הראשונים לענ״ד אין הפי׳ ברבינו האי כמו שחשב הראב״ד דעתה זכינו לתשובת הגאונים שלא היו לפני הראשונים ושם כתוב (סימן ה׳) שאלה: על נדה אשר רחצה אחר ימי נדותה במים שאובין והיא שומרת שבעת ימי נקיים אם יש לה משכב או מושב או לא. תשובה: אם על עיקר התורה אתם דורשים כל זמן שאין הנדה טובלת במי מקו׳ היא כשאר בעלות נדה דכתיב והדו׳ בנדתה תהי׳ בנדתה עד שתבוא במים וצריכה (נ״ל שצ״ל ומטמאה) למשכב ומושב אבל בזמן הזה שאין מתרחקין ממשכבה ומושבה של נדה אלא כדי שלא תשתכח תורת טהרה מישראל וכיון שטהרה מנדתה ורחצה במים שאובין אין מתרחקין ממשכבה וממושבה ועל זה הוא מנהג העם בדור הזה עכ״ל הרי שהיו נוהגין בימי הגאונים להתרחק ממשכב ומושב של נדה משום שלא תשתכח תורת טהרה וזה כוונת רבינו האי במה שכתב אלא שאין אנו מטמאין משכב שתחתי׳ בשעת בעילה משום דלאו נדה ודאי היא אלא ספק דכיון דאינה נדה ודאי לא החמירו לעשות זכר כמו שלא החמירו ג״כ בספרה ז״נ וטבלה במים שאובין אף שעדיין היא נדה גמורה אבל לחלק בין כלה שנבעלה לנדה גמורה לענין לשכב הבעל על מטתה זה לא שמענו לפע״ד מדברי הגאון ובפרט שעוד ראיתי שנעלמו במכ״ה מהרא״ש שו״ת רבינו האי שהרא״ש כתב בפ׳ תנוקת על מה דאמרינן דבועל בעילת מצו׳ ופורש וז״ל ונראה לי דטעם לחומרא זו לא בשביל שנחוש שמא יצא דם מן המקור עם דם בתולים דלמה נחוש בתנוקת שלא הגיעה זמנה לראות ואפילו באשה גדולה למה נחוש הא אמרינן לקמן דאפילו אשה שראתה מחמת תשמיש אם יש לה מכה תולה במכתה וכו׳ אלא טעם חומרא זו משום דבעילת מצו׳ לכל מסורה ואין הכל בקיאין וכו׳ לפיכך הסכימו רבותינו להשוות כולן וליתן להם דין חומרא שבחמורות עכ״ל הרא״ש אמנם ראיתי בשו״ת הגאונים (סי׳ קס״ה) בשם רב נטרונאי גאון וז״ל וששאלתם הא דאמר רבנן בועל בעילת מצו׳ ופורש צריכה שבעה נקיים וטבילה או לא: תנוקת חשו לה חכמים שמא עם טורח דם בתולים אי אפשר לבא דם בתולים בלא צחצוחי זיבה וכיון דמשום הכי הוא ל״ש ראתה ועודה בבית אבי׳ וניסת ול״ש וכו׳ צריכה ז׳ נקיים וטבילה כעיקר זבה עכ״ל ועוד כתוב שם (סימן קס״ח) בשם רבינו האי גאון וששאלתם הנושא אשה בתולה כו׳ הלכתא הנושא אשה בתולה כשהוא בועל בעילה הראשונה שהי׳ מצו׳ פורש ממנה ואסור לבעול בעילה שניי׳ עד שתספור ז״נ ותטבול במים חיים מפני שדם בתולים גורם לדם נדה שיצא עמו וכשהיא נבעלת בעילה ראשונה אוחזת אותה רעדה וחלחלה ובזמן שהאשה מתחלחלת היא פורסת נדה וכענין מה שכתוב באסתר ותתחלחל המלכה מאוד אמר רב מלמד שפרסה נדה עכ״ל הרי שבפי׳ כתבו הגאונים שהחומרא היא משום חשש נדה ובלי ספק אילו ראה הרא״ש דבריהם לא הי׳ כותב כדבריו וכיון דמה שפורש אחר בעילת מצו׳ משום חשש נדה הוא לענ״ד אין להקל שישכב הבעל על מטתה כמו שנהגו שלא לשכב בנדה גמורה אחר שכתבנו שדברי רבינו האי שמהם הוציא הראב״ד חילוק זה אין הפי׳ כן. כנלענ״ד הקטן יעקב.\\\vspace{0pt}

\end{multicols}\newpage

\newchap{סימן ע}
\begin{multicols}{2}
ב״ה אלטאנא, יום ד׳ ר״ח שבט תרכ״ג לפ״ק. לאחי הרה״ג מ״ה ליב אב״ד דגליל לאדענבורג נ״י.\\\vspace{0pt}

אשר שאלת במה שבא לפניך אשה בת ששים שנה ויותר שפסקה להיות לה אורח כנשים זה י״ב שנים, וזה כערך שנה מצאה כתם בחלוקה ולאחר שבדקה עצמה מצאה גם בעד הבדיקה טפות דמים וספרה ז״נ וטבלה, ומאז והלאה מצאה פעם בחדש ופעם בב׳ וג׳ חדשים טפת דם בלא הרגשה ובלא סימני וסת הרגילים בנשים בשעת וסתן ולפעמים שהרגישה כאב בשעת הבדיקה ושאלה לרופא ישראל ובדקה בכלי אומנות שלו ואמר שיש לה מכה בשלפוחית של מים מה שנקרא בל״א דריזענאויפלויף שהם אבעבועות שלפעמים מלאים דם וכשנדחק בהם יזל דם ולדעתו מזה בא הדם שהאשה רואה והמקור סתום, ועכ״ז לא יחליט שלא בא מעט דם גם מן המקור עמו אבל לדעתו רחוק לחשוב כן, ושאלת מה דין האשה אם לטהר אותה לבעלה כשתראה טפות דמים כאלה.\\\vspace{0pt}

תשובה – הנה ראיתי דעתך נוטה להקל באשר שבנדון זה ודאי אין חשש איסור דאורייתא אחר שרואה בלא הרגשה ואשה זו כבר בחזקת שאינה רואה זה י״ב שנים וגם רוב נשים אינן רואות בשהגיעו לששים שנה ומאחר שיש לה מכה בשלפוחית ודאי יש לדון שמשם לבד הדם בא. אמנם לבבי לא כן יחשוב שמה שכתבת דודאי אין חשש דאורייתא באשר שראתה בלא הרגשה אין זה רק במה שראתה בחלוקה אבל במה שמצאה בבדיקת עד אינו כן דלדעת רוב הפוסקים בבדיקת עד יש חשש הרגשה דסבורה דהרגשת עד היא ואיכא ספק דאורייתא וכ״כ ג״כ בפשיטות הגאון ר׳ יהונתן ז״ל בספרו תפארת ישראל סימן קפ״ג ע״ש וגם על מה שרוב נשים בשהגיעו לזקנה אינן רואות ועל מה שפסקה לראות אין לסמוך דא״כ בזקנה מטעם זה לא יטמא כתמה דנסמוך דודאי מעלמא אתי ואינו כן שממה שכתב הב״י בשם הרשב״א והביאו הרמ״א סימן ק״צ ס׳ מ״ה דזקנה תולה כתמה בעת שלא היתה זקנה מוכח דכתם זקנה שאין לה במה לתלות טמא. וגם בלא״ה אין לדונה בזקה שמוחזקת שאינה רואה שהרי לפי מה דאמרינן בנדה (דף ט׳) זקנה שעברו עלי׳ ג׳ עונות ודי׳ שעתה אם שוב ראתה ג׳ פעמים אפילו פחתה או הותירה מטמאה למפרע ככל הנשים והרי זקנה זו ג״כ ראתה כמה פעמים וממילא אין לדונה עוד כזקנה ולכן לא נשאר לנו סמך היתר רק ממה שיש לה מכה בשלפוחית של מי רגלים. והנה ע״פ המבואר בסימן קפ״ז להיות תולה במכתה צריך שיהי׳ לה וסת קבוע ואז תולה במכתה אפילו אינה יודעת בודאי שמכתה מוציאה דם או אפילו באין לה וסת קבוע באם שיש ס״ס ספק מן המקור ספק מן הצדדים ואת״ל מן המקור שמא מן המכה אבל בלא ס״ס לא טהורה רק ביודעת ודאי שמכתה מוציאה דם. והשתא בנדון זה האשה אין לה וסת קבוע ואפילו נדון אותה כזקנה ממש ג״כ דעת שו״ת הפני יהושע סימן ו׳ לדונה לענין מכה כאשה שאין לה וסת רק שבזה בספר צלעות הבית חולק עליו וסובר דזקנה דנין כאשה שיש לה וסת שלא בשעת וסתה ולא הוכרע הדין עם מי וא״כ כל שכן דהכא שאפשר שאין לה דין זקנה כנ״ל שנדונה כאין לה וסת ואז בעינן חדא מתרתי או שיש ס״ס להתיר או שיודעת ודאי שמכתה מוציאה דם ולדעת נו״ב לא שייך היתר ס״ס רק בספק מן המקור ג״כ אבל בודאי בא מן הצדדים אסורה דאז ליכא רק ספק אחד וא״כ בזה שיודעת ודאי שאין לה מכה במקור אין להתיר מכח ס״ס וגם אם יש לדונה כודאי יודעת שמכתה מוציאה דם יש לספק כיון דעל פי הרופא האבעבועות רק לפעמים מלאים דם וגם אז רק כשנדחק בהם מוציאים דם ובאשה זו מאן לימא לן דכשמצאה דם שאז היו האבעבועות מלאות ונדחק בהם ועוד דבכל מקום שאפשר לברר לא מתירין ספק והרי האשה שיש לה המכה בשלפוחית מי רגלים אפשר לה לבדוק אם הדם בא מן המקור או ממקום מי רגלים בבדיקה שהביא הרמ״א (סי׳ קצ״א). וע״פ זה נ״ל באשה זו שאם מצאה כתמים בבגדה שלא ע״י בדיקה טהורה דכתמים תולין בכל ענין כמשכ׳ הרמ״א סי׳ קפ״ז ס׳ ה׳ אבל מה שמצאה ע״י בדיקה אין לסמוך שבא מן מכתה ודאי דהוי ספק דאורייתא שהרי הרגישה וכדעת רוב הפוסקים הנ״ל והרי גם הרופא אומר שאפשר שבא דם המקור עם דם המכה ולכן אז צריכה לבדוק בז׳ נקיים ע״פ הבדיקה שהביא הרמ״א במוצאת דם במי רגלים ואם עברו ל׳ יום מבלי שראתה דם ודאי מגופה כיון שעכ״פ היא כאשה שאין לה וסת א״צ לבדוק ובפרט אחר שזקנה היא נוכל לסמוך ולומר שפסק וסתה ולכן טהורה בלא בדיקה לעולם. כנלענ״ד הקטן יעקב.\\\vspace{0pt}

\end{multicols}\newpage

\newchap{סימן עא}
\begin{multicols}{2}
ב״ה אלטאנא, יום ב׳ כ״ו אייר תרי״ט לפ״ק. לכבוד הרה״ג וכו׳ מ״ה יצחק דוב הלוי נ״י הגאב״ד דק״ק ווירצבורג יע״א.\\\vspace{0pt}

בדיק לן מר נ״י בדבר השאלה שבאה לפניו שאשה א׳ שיש לה וסת קבוע נתהו׳ לה מחלה ברחמה ועשו לה הרופאים טבעת עגול ברחמה לעכב נפילת המקור ולדעתם יהי׳ סכנה בהסרת הטבעת איך תתנהג בבדיקות הפסקת טהרה וז׳ נקיים ובעת טבילת מצו׳.\\\vspace{0pt}

תשובה – מעכ״ת נ״י כבר הביא לנו תשובות האחרונים שדברו בענין זה, בשו״ת זכרון יוסף סי׳ י׳ מיקל שדי לה לבדוק עד מקום שידה מגעת ולא צריכה להסיר הטבעת אמנם בסדרי טהרה סימן קצ״ו ס״ק כ״ג העלה שעכ״פ צריכה להסיר הטבעת בעת הפסקת טהרה גם בשו״ת גידולי טהרה החמיר להצריך שתסיר בעת הפסקת טהרה ובאחד מימי בדיקה לכל הפחות ובשו״ת חתם סופר חי״ד סי׳ קצ״ב נראה שהיקל שדי לה בבדיקת מוך בלי הסרת טבעת גם בשו״ת שמלת בנימין סי׳ ט״ו נראה שהיקל כן וגם בשו״ת ר״ע איגר רק שהצריכה שפופרת ומוך בתוכה ומר נ״י הראה דעתו לסמוך על המקילים ורצה שאסכים עמו. והנה זה ודאי אם הי׳ פשיטא לנו שהטבעת למעלה יותר ממקום שהשמש דש א״צ לפנים שהרי שם א״צ בדיקה אפילו להנך שיטות שס״ל דגם לבעלה צריכה בדיקה בחורין וסדקין אבל בנדון השאלה שיש ספק בזה בתחלה נדון אם הבדיקה היא מדאורייתא או מדרבנן.\\\vspace{0pt}

וראיתי בשו״ת גידולי טהרה שהוכיח מדברי הרמב״ן נדה (דף ה׳) שהבדיקה של הפסק טהרה היא מן התורה ממה שכתב דבדיקת ההפסק שהיא מעלה אותה מטומאה לטהרה ומוציא אותה מחזקה לחזקה צריכה בדיקה מעולה אמנם ממה שכתב הרמב״ן שם בתחלת דבריו שיש מי שחולק ואמר דבדיקת זבה בין ביום ההפסקה בין בבדיקת השבעה בעינן בדיקה מעולה עכ״ל משמע שרק בזבה מצריך כן מדכתיב וכי יטהר הזב ודרשינן משיפסוק אבל לא בנדה ולפ״ז בנדון השאלה שיש לה וסת ומסתמא הוסת אינה פחותה מי״ח לי״ח וגם מסתמא לא תראה בזמן הוסת יותר משבעה ימים א״כ אשה זו ודאי לא זבה היא אלא נדה ולכן אף דנשי דידן כספק זבות דיינינן להו מכ״מ זה אינו רק דרבנן ואין כאן רק ספק בדרבנן אם הוא למעלה מהמקום שהש״ד ואע״פ שמטעם זה בלבד אין להקל שבדאתחזק איסורא כגון הכא לא אמרינן ספק דרבנן להקל מכ״מ יצאנו מכלל ספק איסור דאורייתא אבל א״ע נלענ״ד דזה אינו מספיק דהתינח אבדיקות ז׳ נקיים שאילו הם רק דרבנן בנשי דידן אבל בדיקה דהפסקת טהרה משמע מלשון הפוסקים דהיא מדאורייתא גם בנדה וכן נראה ממה דאמרינן נדה (דף ס״ח) נדה שבדקה עצמה ביום השביעי שחרית ומצאה טמאה וביה״ש לא הפרישה ולאחר ימים בדקה ומצאה טמאה רב אמר זבה ודאי ע״ש וכוותי׳ פסקינן הרי דמחזקינן מדאורייתא לנדה שעדיין לא פסק מעיינה אם לא תפסוק בטהרה וכיון דאשה זו לא ידענו אם פסקה כראוי הוי ספק דאורייתא וא״ל דהוי ס״ס כיון דיש פלוגתא בין הפוסקים אם לבעלה צריכה בדיקה בחורין ובסדקין ולא סגי בקינוח ואף דפסקינן דגם לבעלה צריכה בדיקה בחוס״ד מכ״מ אפשר דנדון אותו כפלוגתא דרבוותא אחר דבשו״ת נודע ביהודה רצה להוכיח דגם הרי״ף והרמב״ם כדעת המקילין וא״כ הוי ס״ס ספק שמא א״צ כלל בדיקה בחוס״ד ואת״ל צריכה שמא בדקה כן והטבעת למעלה ממקום הבדיקה והוי ס״ס שיכול להתהפך דז״א דמלבד שגם בזה בעצמו נגענו בפלוגתא שבין הפוסקים שהביא הש״ך בכללי ס״ס סי׳ כ״ט אם באתחזק איסורא מהני ס״ס בדאורייתא בלא״ה אפשר לא נחשב זה ס״ס כיון דהספק אם הוא למעלה ממקום שהש״ד אולי הוא כספק מחמת חסרון ידיעה שלא נחשב לס״ס וא״כ אכתי לא יצאנו בבדיקה של הפסק טהרה מספק איסור כרת אבל מכ״מ נ״ל כיון שכתב הרמ״א סי׳ קצ״ו ס״ז דבדיעבד אפילו לא הגיע ודאי למקום שהשמש דש סגי לה וכפי המבואר בפוסקים הובאו ביד מלאכי סימן קל״ג כל דבר שאין לו תקנה מקרי דיעבד וכיון דאשה זו בלי סכנה אי אפשר להוציא הטבעת ואסורה לסכן את עצמה לא יהי׳ לה תקנה לטהר עצמה לבעלה אין לך דיעבד גדול מזה ואע״ג די״ל דהכא גרע כיון שיש דבר המעכב יציאת הדם מכ״מ אין זה חשש הכרח ונוכל לצרף לזה לסניף ס״ס הנ״ל ובפרט דהרי בבדיקה אשה נאמנת על עצמה כדילפינן מספרה לה ואם אומרת שלדעתה הוא למעלה ממקום שהש״ד למאי ניחוש לה אבל מכ״מ יש לחקור שכפי ששמעתי אין סכנה להוציא הטבעת לשעתו ולהחזירו מיד והסכנה היא בלבד אם תלך בלי הטבעת ואם הדבר כן אז ודאי שצריכה להסירו עכ״פ להפסקת טהרה וליום ראשון מהנקיים כמשכ׳ הס״ט ובספר גידולי טהרה כן נלענ״ד אם יסכים מר נ״י על זה הקטן יעקב.\\\vspace{0pt}

\end{multicols}\newpage

\newchap{סימן עב}
\begin{multicols}{2}
ב״ה אלטאנא, יום ו׳ צום העשירי תרי״ט לפ״ק.\\\vspace{0pt}

שאלה – אשה שילדה וטבלה לאחר לידה ושמשה וביום ל״ט מיום שטבלה ושמשה הפילה ואינה יודעת מה הפילה אם ולד אם לאו אם טהורה אחר שתמתין ה׳ ותספור ז׳ נקיים כדין רואה נדה או לא.\\\vspace{0pt}

תשובה – בנדה (דף ל״ו) תנן המפלת ליום מ׳ אינה חוששת לולד וכ׳ רש״י ליום מ׳ לטבילתה ובזה נחלקו הפוסקים האחרונים בשו״ת עבודת הגרשוני פי׳ דברי רש״י כפשוטן דלאחר שראתה נדה וטבלה לא חוששת שמא היתה מעוברת קודם שראתה נדה אכן בשו״ת שבות יעקב הביא בשם ר״י מילר ז״ל שחלק על זה שהרי אין מעוברת מסולקת דמים עד ג׳ חדשים והש״י השיב ונראה שהסכים עם עבודת הגרשוני אמנם בכרתי ופלתי וסדרי טהרה חלקו עליו וכתבו דאין לסמוך על זה לחשבה מסולקת דמים מעת שנתעברה אלא שדברי רש״י צריכים ביאור ובסדרי טהרה כ׳ דאולי כוונת רש״י בשטבלה לאחר לידתה ושוב מצאתי בשו״ת חתם סופר חי״ד סי׳ קס״ט שג״כ פי׳ כן וכ׳ שזה ברור ואמת בלי גמגום ועל זה הורה שם הלכה למעשה שהמפלת אחר לידתה שטבלה ושמשה תוך מ׳ יום אינה חוששת לולד ותמהתי מה בכך שילדה מאן לימא לן דלאו ולד הוא שאף שאין אשה מתעברת וחוזרת ומתעברת הרי לפי׳ רש״י נדה (דף מ״ה) זה רק ולד של קיימא אבל נפל חוזרת ומתעברת וכוונתו שבמקום שא׳ נפל בין הראשון בין השני אפשר שחוזרת ומתעברת כנראה מדברי התוספ׳ שם (דף כ״ה) וא״כ מה ראי׳ ממה שילדה דלמא קודם ללידתה נתעברה שנית והפילה עכשיו וכבר עברו יותר ממ׳ יום והרי מצאנו שאפשר שנשתהה ולד אחר חבירו ג׳ חדשים כדאמרינן שם (דף כ״ז) כש״כ שאפשר שנפל אשתהה אחר ולד וכן מצאנו בכריתות (דף ט׳) והמפלת תאומים שנשתהה נפל אחר חבירו אפילו בג׳ נפלים ע״ש ואף שהר״ת חולק על רש״י וס״ל דאין אשה חוזרת ומתעברת כלל וא״כ כיון שילדה ולד של קיימא ע״כ הנפל אינו ולד מאן לימא לן שהלכה כר״ת ולא כרש״י ובפרט שאין לר״ת ראי׳ מכרעת ועכ״פ דברי רש״י אי אפשר לפרש כן כמו שחשבו הס״ט והחתם סופר ומה שנלענ״ד בפי׳ דברי רש״י כתבתי בחידושי למסכת נדה שכוונתו משום דרוב נשים מתעברות סמוך לטבילתן כדאמרינן בנדה פ׳ המפלת (וכעת שנדפס ספר אור זרוע ראיתי שכוונתי להאמת ת״ל כמו שהעתקתי דבריו בהוספה שם) ובלאו הכי פי׳ הסדרי טהרה והחתם סופר שקאי אטבילה אחר לידה לענ״ד דוחק הוא דלידה מאן דכר שמי׳ וה״ל לרש״י לפרש המפלת ליום מ׳ לטבילת לידתה כיון דמלשון מתניתן לא נשמע דאחר לידתה איירי ומה שהביא החת״ס ראי׳ מהמפלת לאור שמנים וא׳ אינו ענין לכאן דשם הוא אחר לידה ממש וזה נשמע מהמנין וגם מענין פלוגתת ב״ש וב״ה אבל הכא דאין חושבין המ׳ מהלידה כי אם מהשמוש שאחר טבילת לידה זה ודאי המתניתן הי׳ צריך לפרש וכש״כ שרש״י לא ה״ל לסתום ועכ״פ להלכה נ״ל שבנדון השאלה צריכה לחוש שמא נקבה הפילה ותמנה י״ד יום מיום שהפילה כדי לצאת מספק איסור כרת. כנלענ״ד הקטן יעקב.\\\vspace{0pt}

\end{multicols}\newpage

\newchap{סימן עג}
\begin{multicols}{2}
ב״ה אלטאנא, יום ב׳ כ׳ אלול תרכ״ה לפ״ק. להרה״ג וכו׳ מ״ה יצחק דוב הלוי נ״י הגאב״ד דק״ק ווירצבורג יע״א.\\\vspace{0pt}

על דבר שאלת מר נ״י ממני להביע דעתי באשה שהי׳ לה כאב ועסקה ברפואות ולאחר זמן מצאה בשולי הספל שמטלת בו מים דם וכן רגילה לראות ודוקא בתוך הספל נמצא ולא על שפת הספל והאשה יש לה וסת בדילוג למפרע דהיינו שמקדים לבא איזה ימים אבל לא הרבה ועתה אחר שעסקה ברפואות ובמרחצאות הכאב בתוך הגוף הוא רק לפעמים ולזמן קצר ולא דוקא בשעת הטלת מ״ר מה דינה של אשה זו לטהרה לבעלה.\\\vspace{0pt}

תשובה – הגם שמעכ״ת נ״י לא צריך לדידי עכ״ז למלאות רצון צדיק כי״ב לא אמנע מלהודיע מה שנלפענ״ד ואפתח במה שהעיר מעכ״ת נ״י שהי׳ אפשר למצוא צד היתר ע״י בדיקה שהביא הרמ״א בשם מהרי״ל ומהרי״ו בסי׳ קצ״א שתכניס מוך נקי על המקור בפנים ואח״כ תשתין מים ותוציא אותו ואם תמצאנו נקי יש הוכחה שאין הדם מן המקור אכן כיון שכפי הנראה מדברי רמ״א זה דוקא ביש לה כאב אבל בנדון השאלה הכאב נפסק קצת בפרט בשעת הטלת מ״ר לא תועיל הבדיקה ואף שהנו״ב מ״ק חי״ד סי׳ מ״ח רצה לבאר דברי הרמ״א שיש לסמוך על הבדיקה גם בלא כאב הדין עם מעכ״ת נ״י שדחה זה מדברי מהרי״ל בתשובה סי׳ ר״ג ושגם משו״ת צמח צדק סי׳ צ״ב נראה שס״ל דתרתי בעינן ועל כן מסופק אם בנדון השאלה נקרא שיש כאב. אמנם לענ״ד יש לחקור למה לא נסמוך על הבדיקה לבד וכי יש הוכחה וראי׳ גדולה מזו שאין הדם בא ממקור אם תסתום המקור במוך דחוק ויצא נקי אחר הטלת מ״ר ובמ״ר יש דם וכי אפשר להיות שיצא דם מן המקור אל מי רגלים ובמוך אשר סתם המקור לא יראה אפילו טפה כחרדל והרמ״א עצמו קרא לבדיקה זו הוכחה גדולה (ומזה נראה שדעתו נגד הש״ך) ולמה לא נסמוך עלי׳ אלא שראיתי בחות דעת שכתב שמצא ראי׳ ברורה שאפשר לדם לזוב דרך סדק ולא יצטבע המוך מנדה (דף ג׳) דמקשה השס׳ מוך דחוק מא״ל ומשני מוך אגב זיעה מכוויץ כוויץ וע״כ א״א לומר דמיירי דנמצא הדם על המוך דא״כ איך משני מכוויץ כוויץ הא כשנתדבק הדם במוך ודאי דא״א שוב לירד ועוד דלא גרע מכתם הנמצא בבגד אלא ודאי דאיירי שלא נמצא הדם על המוך ואפילו הכי אמרינן מכוויץ כוויץ עכ״ד וראי׳ זו לא הבנתי דודאי אי אפשר שיצא דם ולא יעשה רושם על המוך כנראה בחוש וכן איירי ג״כ רבא שנמצא מעט על המוך ואח״כ זב דם הרבה ולכן פשיטא לי׳ כיון דלב״ש אין בית הרחם מחזיק דם ודאי אי הוי מעיקרא הי׳ זב כבר והרי ראינו שלא נדבק על המוך רק מעט וגם לכתם לא דמי דכתם ע״כ נתלה למפרע אבל בזה אפשר שעתה בשעה שזב הדם בא מעט על המוך אבל דהוי אתי מעיקרא ולא נראה רושם על המוך באמת אי אפשר ועוד אפילו נימא דהיכא דיש לומר מכוויץ כוויץ אפשר שיצא הדם מבלי שיעשה רושם על המוך הרי אמרינן שם ומודה רבא במוך דחוק הרי שבמוך דחוק גם רבא מודה שאי אפשר בשום אופן שיצא הדם לחוץ א״כ למה לא יועיל הבדיקה במוך דחוק ותמהתי על בעל חות דעת שכתב שם דאע״ג דבסי׳ קצ״ו מבואר דמוך דחוק מוציאה מכל ספק שם איירי שמניחה מבחוץ ג״כ מוך שמכסה כל בית התורפה והרי במה שאמרינן ומודה רבא במוך דחוק איירי במשמשת במוך שאין בית התורפה מכוסה ואעפ״כ פשיטא לרבא שאי אפשר לטפת דם לצאת ולכן לענ״ד לא בלבד שאין מסוגיא זו ראי׳ שלא תועיל בדיקה לבד אלא אדרבה יש ראי׳ להיפך ולכן אפשר דמה שהפוסקים לא סמכו על הבדיקה לבד איירי בבדיקה ע״י מוך דומיא דמשמשת במוך שאינו דחוק אבל בבדיקה ע״י מוך דחוק ודאי יש סברא שתועיל לבד.\\\vspace{0pt}

אמנם על זה לבד אין לסמוך להקל אבל בנדון השאלה יש עוד קולות אחרות כיון שישבה על שפת הספל ולא ראתה על שפת הספל אלא תוך הספל והרי להרמ״א כל החומרות דבדיקה ושלא תראה על העד אחר הטלת מ״ר לא צריכה רק בעומדת או ביושבת ומקלחת ומוצאת גם על שפת הספל ואף דהרמ״א לא איירי רק בפעם א׳ ולא בהרבה פעמים כמו שחילק הנו״ב מ״ת סי׳ ק״ל לענ״ד אין זה במשמעות דברי הרמא אכן מצד אחר יש לחלק ששם באותו פעם שראתה ישבה על שפת הספל ולא ראתה א״כ מוכח שמה שראתה בתוך הספל לא אתי ממקור ולכן אם אשה זו ג״כ בכל פעם שתשתין תעשה כן ודאי סברא יותר שתועיל בהרבה פעמים מבפעם אחת שעי״ז מוחזקת לרואה במי רגלים אבל בנדון השאלה לא בכל פעם תעשה כן א״כ קשה להקל מטעם זה ואולי אם תבדוק כן ג׳ פעמים שתהי׳ מוחזקת בכך שאינה רואה על שפת הספל רק בתוכו יהי׳ ג״כ סניף להקל אכן לענ״ד בלא״ה יש תקנה שהרי לפי המבואר מרמ״א וכמו שכתב וביאר דעתו בסדרי טהרה ס״ק ז׳ שמה שלא סמכינן על הבדיקה לבד באין לה כאב זה דוקא במצאה גם על העד שקנחה בו אחר הטלת מ״ר אבל במצאה במ״ר לבד אפילו בלא כאב סגי בבדיקה לבד וכפי מה שכתב הרמא אינה צריכה לבדוק אחר זה וא״כ אם אשה זו לא תבדוק בעד אחר הטלת מ״ר ולא תראה רק בשולי הספל ולא על שפת הספל ותשתין כשהיא יושבת הרי יוצאת כל החומרות אפילו אם נאמר דכאב פסק לגמרי אבל הרי לפעמים מרגשת כאב א״כ זה הוכחה קצת שעדיין לא נתרפאה מכתה ולכן לענ״ד אם אשה זו תבדוק בבדיקת מהרי״ל במוך דחוק באופן שכתב החתם סופר סי׳ קמ״ח וגם תבדוק ג׳ פעמים להשתין יושבת על שפת הספל שתהי׳ מוחזקת בכל פעם שמצאה תוך הספל שעל שפת הספל לא היתה מוצאת וגם לא תבדוק בעד אחר הטלת מ״ר יש להקל אפילו לא נחשוב לה לכאב קיים אפילו לפי מה שכתב הנ״ב בסי׳ ק״ל דבעינן מוכיחה קיים שהרי טעמו בזה כמשכ׳ סי׳ מ״ח דאל״כ עד מתי נטהר אותה והיינו מה שכתבו הפוסקים לענין שבעינן שיש לה וסת וכיון דאשה זו יש לה וסת וגם קצת כאב הרי מוכיחה קיים והלא החתם סופר היקל אפילו באשה שפסק וסתה ופסק הכאב לגמרי א״כ כש״כ בנדון זה אם לא תבדוק העד אחר הטלת מ״ר על כן לענ״ד יש לסמוך על זה עכ״פ בנדון זה אבל תבדוק הבדיקה ג׳ פעמים שלא לזוז מפסק הרמ״א כן נלענ״ד אם מעכ״ת נ״י יסכים עמי למטינא שיבא מכשורא. הקטן יעקב.\\\vspace{0pt}

\end{multicols}\newpage

\newchap{סימן עד}
\begin{multicols}{2}
ב״ה אלטאנא, בחדש כסליו שנת תרי״ח לפ״ק. להרה״ג וכו׳ מ״ה יצחק דוב הלוי נ״י הגאב״ד דק״ק ווירצבורג יע״א.\\\vspace{0pt}

ראיתי למעכ״ת נ״י בספרו היקר אמירה לבית יעקב שהורה באשה שלא הפסיקה בטהרה או לא בדקה בראשון של ז׳ נקיים או בז׳ של ז׳ נקיים אפילו בדקה בכל הימים בנתים בבקר ובערב כדין עם כל זה צריכה לספור מחדש ז׳ נקיים ולענ״ד דין זה צ״ע כי מה שפסק בש״ע י״ד סי׳ קצ״ו ס״ה כיש אומרים דאם לא בדקה יום ראשון ושביעי אין להקל זה דוקא נגד מה שמיקל דיעה ראשונה אפילו לא בדקה רק פעם א׳ בתוך ז׳ ימים על זה קאי היש אומרים דלעולם בעינן בדיקה ראשון ושביעי שיהיו ספורים לפנינו אבל ודאי כל שיש לצרף בתוך ימי הבדיקה שיהי׳ בהם הפסקת טהרה ובדיקה ראשון ושביעי מהני ולכן לא בלבד כשלא בטלה רק בדיקת ז׳ בלבד יש לה תקנה שתבדוק בשמיני וטהורה לערב שאז מחשבינן בדיקת יום ראשון להפסקת טהרה ואפילו לא היתה הבדיקה בערב בין השמשות כדבעינן לכתחלה להפסקת טהרה שהרי בדיעבד גם בדיקת שחרית מהני כשלא ראתה באותו יום כמבואר שם ס׳ א׳ בהגהה ובדיקת יום שני נחשבת לבדיקת יום ראשון ובדיקת שמיני לבדיקת שביעי אלא אפילו לא הפסיקה בטהרה ולא בדקה עצמה בראשון רק ביום שני וביום שלישי באותן הימים שיעדה להיות לה ז׳ נקיים וכל הימים שאחרי זה לא בדקה עוד עד שכלו כל ז׳ הימים נ״ל שדי לה שתבדוק בתשיעי ולכתחלה גם בשמיני שאז יעלה לה בדיקת יום שני להפסק טהרה ובדיקת יום שלישי לראשון ובדיקת יום תשיעי לשביעי ולערב טובלת וטהורה. ואחרי דנתי כן מסברא מצאתי און לי שכן כ׳ ג״כ בכיוצא בזה בשו״ת נו״ב מ״ב חלק י״ד סי׳ קכ״ח ע״ש שרצה להוסיף עוד שאפילו לא בדקה רק בראשון ובשלישי ובשמיני שג״כ מהני דהשלישי מצרף השמיני עם הראשון אמנם כפי הנראה השיב על ראיתו ולא החליט זה אמנם מה שהחליט ג״כ להיתר שאם בדקה בראשון ובשלישי ובתשיעי שזה מהני בפשיטות ששלישי נחשב כראשון ותשיעי כשביעי לענ״ד צ״ע קצת כיון שלא בדקה בשני הרי הרחיקה יום הפסקת טהרה משלישי שנחשב לה כראשון והרי נתקן יום הפסקת טהרה להיות סמוך לראשון שמונה לז׳ נקיים ולכן היתר זה עדיין לא הי׳ ברור לי אכן מצאתי בסדרי טהרה סי׳ קצ״ו ס״ק י״ח שג״כ פשוט לו שאם הרחיקה הפסקת טהרה מיום ראשון של בדיקה איזה ימים דמהני ולכן אם הפסיקה ובדקה רק בשני ובשמיני או שבדקה בשלישי ובתשיעי לענ״ד פשיטא דמהני וטובלת לערב וטהורה ופסק מעכ״ת נ״י דאפילו לא בטלה רק אחת מהבדיקות של ראשון ושביעי לבד ושאר ימים בדקה שתהי׳ צריכה לספור מחדש ז׳ נקיים לענ״ד צ״ע. הקטן יעקב.\\\vspace{0pt}

\end{multicols}\newpage

\newchap{סימן עה}
\begin{multicols}{2}
ב״ה אלטאנא, יום ב׳ ח״י אדר תרכ״ג לפ״ק. לגיסי הרבני המופלא מ״ה יחזקאל זאב קונרייטער נ״י בק״ק מינכען יע״א.\\\vspace{0pt}

על דבר השאלה אשר הובאה לפניו, אם מותר לרופא ישראל ליילד את אשת אחיו שקורין אקושירען בל״א במקום שיש רופאים ישראלים וא״י אחרים שבקיאים בדבר.\\\vspace{0pt}

תשובה – אם נחליט שיש כאן איסור ערו׳ כיון דלדעת הרמב״ם יש בנגיעה בבשר ערו׳ לאו דאל כל שאר בשרו לא תקרבו, כמבואר באהע״ז (סי׳ כ׳) והש״ע פסק כוותי׳ וא״כ אין להתיר לרופא ישראל ליילד אפילו אשת איש דעלמא שאינה ביחוד לו ערו׳ אלא במקום שיש סכנת נפשות ובאי אפשר ע״י א״י משום פקוח נפש אז לכאורה יש לחלק בין אם היא אשת אחיו או לא דזה פשיטא דב׳ האיסורים דאשת איש ואשת אח חלים כאחד וא״כ היא ערו׳ לו משום ב׳ איסורים ובכל יולדת יש ג״כ איסור נדה שג״כ חל עליהם ע״י מוסיף כדמוכח ביבמות (דף ל״ד) ובכריתות (דף י״ד) וא״כ יש באשת איש דעלמא ב׳ איסורי ערו׳ ובאשת אחיו ג׳ איסורים וכיון דאמרינן בכל פקוח נפש דלמעט באיסורים עדיף כדאמרינן ביומא (דף פ״ג) מאכילין אותו הקל הקל א״כ גם הכא עדיף שיעסוק בה רופא ישראל אחר ולא אח בעלה שלזה יש ג׳ איסורי ערו׳ ולאחר רק שנים. אמנם לא״ע נראה שזה אינו שלא שייך לחלק בין איסור א׳ לב׳ איסורים אלא אם הוא עובר בכל איסור מהם על לאו בפני עצמו אבל בנדון זה דאין כאן אלא לאו א׳ דלא תקרבו לגלות ערו׳ א״כ מה לי אם היא לו ערו׳ משום איסור א׳ או הרבה איסורים תדע שהרי גם להרמב״ם שלוקה על לאו זה מכ״מ אינו לוקה אלא א׳ אפילו אם התרו לו שהיא ערו׳ מכמה איסורים כיון שאין כאן רק לאו א׳ וא״כ גם לענין איסור חמור וקל ליכא לחלק בזה בשיש פקוח נפש בין אם היא לו באיסור א״א בלבד או גם באיסור אשת אח. והנה כל זה אם נחליט כנ״ל שיש להרמב״ם איסור ערו׳ במה שמילד הרופא לאשת איש ואז פשיטא בשיש כאן מילדת או רופא נכרי שאין להתיר לרופא ישראל לילד אפילו אשת איש דעלמא. אמנם כפי הנראה מהשאלה לא נסתפק השואל בזה אם מותר לרופאי ישראל לילד במקום שיש גם רופאים א״י מוכנים לזה עד שאין כאן היתר פקוח נפש ויש להם לסמוך על מה שכתב הש״ך בי״ד (סי׳ קצ״ה) טעם למה שנהגו הרופאים ישראל למשמש הדפק של אשת איש שגם להרמב״ם לא אסור רק נגיעת בשר לשם תאו׳ אבל בשעוסק במלאכתו לא ע״ש וא״כ הכא נמי מותר לרופא לילד כיון שבמלאכתו הוא עוסק ואע״ג ששדי נרגא בהיתר זה הבית שמואל סי׳ כ׳ וחולק על הש״ך דאפילו אינו עושה דרך חבה יש איסור להרמב״ם מן התורה מכ״מ כבר השיגו עליו העצי ארזים והכרתי ופלתי והסדרי טהרה והסכימו עם הש״ך דאם עוסק במלאכתו ליכא איסור נגיעת ערו׳ והכו״פ כתב שכן נוהגין שאשה שיש לה מכה בבטנה וכל מקום תורפה שרופאים רואים אותה ואף שלפענ״ד כל רופא ירא אלקים טוב עושה להרחיק מלילד במקום שיש אחר שעכ״פ מכוער הדבר ויבא לידי הרהור עכ״ז הנוהגין היתר בדבר יש להם על מה שיסמוכו ופשיטא שלפי טעם דהש״ך אין חילוק בין א״א דעלמא ובין אם היא אשת אחיו ולהחמיר בו יותר משום לבו גס בה לא נראה דסברא דגס בה לא שייך רק בבעל לגבי אשתו ולכן ע״פ מה שנוהגין דרופאי ישראל מילדים לא״א אף במקום שיש רופאי א״י גם לאשת אחיו מותר לילד. כנלענ״ד הקטן יעקב.\\\vspace{0pt}

\end{multicols}\newpage

\newchap{סימן עו}
\begin{multicols}{2}
ב״ה אלטאנא, יום ו׳ ב׳ דר״ח אדר תר״ט לפ״ק. להתורני המופלג מ״ה אלי׳ ווייל נ״י בק״ק פאריס יע״א.\\\vspace{0pt}

כתב לי מר נ״י וז״ל – ראיתי כאן שערורי׳ שהנשים יורדות לטהרה בלילי שבת וי״ט בחמין ואין מוחה בידם ושאלתי היש כח ביד מי להתיר כזאת בישראל נגד משנה שלימה בביצה רוחץ אדם פניו ידיו ורגליו אבל לא כל גופו וכן הורו כל הפוסקים ראשונים ואחרונים והלא אפילו בחול יש להחמיר בדבר משום גזירת מרחצאות אם לא במקום שנוהגים להקל כמבואר בי״ד (סוף סי׳ ר״א) מכש״כ בליל שוי״ט שיש עוד טעמים אחרים לאסור הגם שמצאתי בספר סדרי טהרה שהי׳ מתיר לטבול בחמין בליל שבת נגד דעת החכם צבי וכן נאמר לי שהרב הגאב״ד מ״ה קאפמן נ״י פה התיר את הדבר משום טבילת מצו׳ אפס אין דעתי נוחה בכך ומעתה אם יש את נפשו ורצונו הטוב יודיעני את דעתו הרחבה בדבר הזה:\\\vspace{0pt}

תשובה: הנה אמת שהרב חכם צבי שו״ת (סי׳ מ״א) רצה להחמיר בדבר ולא רצה להתיר רק לטבול בע״ש בין השמשות בחמין מה שבעצמו קרא דרך רחוק והסדרי טהרה (סי׳ קצ״ז) אחר שהביא דבריו והאריך להסכים עמו כתב אחר שכתבתי כל זה ראיתי בס׳ דברי יוסף (סי׳ ס״ד) שכתב שמנהג נתפשט ביניהם שטובלות בחמין בליל שבת ואין פוצה פה ומצפצף וטרח שם לקיים המנהג ואחר השקלא וטריא בסוף העלה דין זה בצ״ע והטעם שכתב הוא כיון דאיכא צערא דגופא לטבול בצונן ה״ל מצטער שהתירו לרחוץ בחמין כמבואר בא״ח (סי׳ ש״ז) דמותר לומר לנכרי להביא חמין דרך חצר שלא ערבו לרחוץ בו המצטער ע״ש והסדרי טהרה הוסיף טעם שהרי ה״ה כתב דלכך מותר לרחוץ לפני המילה בחמין דגזירה דמרחצאות במקום מצו׳ לא גזרו והלח״מ (פ״ו בשבת) ביאר דמחמין הוי שבות דשבות דהגזירה היא משום מרחצאות ומרחצאות משום גזירת הבלנין ושבות דשבות שריא במקום מצו׳ א״כ ה״ה במקום טבילת מצו׳ וסיים הס״ט מכל הלין נ״ל דבמקום שנהגו לטבול בשבת בחמין אין למחות בידם כי יש להם על מה לסמוך ובפרט בזה״ז דירדה חולשה לעולם וא״א להם בצונן עכ״ד ואם בכל זה לא נח דעת מעכ״ת נ״י עם כל זה לא ידעתי למה קרא המנהג שערורי׳ ושאל היש כח ביד מי להתיר כזאת בישראל אחר שכבר ראש ופאר משפחתו אבי זקנו הגאון בעל קרבן נתנאל ז״ל התיר הדבר שבפ׳ במה מדליקין (סי׳ כ״ב) אחר שהשיג שם על החכם צבי שהתיר לטבול ביה״ש והקשה עליו דממנ״פ אם נחשוב לילה לענין טבילה בז׳ היאך נחשוב יום לענין שבת סיים ובעיקר הדין כתבתי באריכות שטבילת מי מקו׳ בחמין לא הוי בכלל גזירת מרחצאות עכ״ל הן אמת שבשו״ת נודע ביהודה מ״ב כתב על זה ואין לנו לסמוך על דבר שלא בירר לנו טעמו ואני אמרתי אולי דוקא רחיצה אסור שעיקר הטעם שמא יחם בשבת אבל טבילה דלא משכחת אלא בכניסת ליל שבת לא משכחת טבילה בחמין שהוחמו בשבת ועוד כתב טעם שהרי רש״י בשבת (דף ל״ט) כתב טעם הגזירה דרחיצה בחמין דאתי להחם בשבת כגון נותן צונן בחמין וזה לא שייך בטבילה במקו׳ שהרי אין טובלין בשאובין וסיים ומ״מ אעפ״י שאנו מדמין אין אנו מורים להיתר רק ליתן בעוד איזה שעות עד הלילה שיהי׳ בלילה רק פושרים ואמנם הבלנין לפי הנראה אין נזהרין בזה ואני מעלים עין מזה ומוטב שיהי׳ שוגגין ואל יהיו מזידין עכ״ל אמנם חוץ מכל הנ״ל לענ״ד יש ללמוד זכות על המנהג שבודאי אין לנו להוסיף על גזירות חכמים אלא כמו שגזרו והם גזרו שלא לטבול בחמין עצמם שהוחמו בשבת או אפי׳ בע״ש ע״י אור שגזרו ע״ש אטו שבת ובשבת גזרו שמא יטיל צונן בחמין ויבשלם בשבת וחדא גזירה היא אבל מה שטובלין בחמין דמקו׳ הרי בזה אינו טובל בחמין עצמם שהוחמו ע״י אור שהרי הם נתבטלו ע״כ במ׳ סאה מי מקו׳ ונעשו ונקראו על שם מי המקו׳ דאל״כ ה״ל טבילה בשאובין ואף דמי המקו׳ נתחממו ע״י החמין שהטילו בהם לא מצינו שגזרו חכמים על החמין שנתחממו ע״י חמין ואף שאסרו לרחוץ במי הסילון שעברו תוך חמי טברי׳ זה אינו שהרי לפי הנראה בשבת (דף ל״ט) מטעם הטמנה בשבת אסרו ועוד דאם הי׳ לחמי טברי׳ דין חמי האור ה״ל במי הסילון בישול ממש וזה הוי הגזירה שכתב רש״י שיטיל צונן בחמין אבל שנגזור לרחוץ בחמין שנתחממו ע״י חמי האור זה לא מצינו שגזרו חכמים ומטעם זה נ״ל שאפילו אין נזהרין ליתן החמין במקו׳ בעוד היום גדול עד שיהיו פושרים כמו שכתב הנ״ב מכ״מ יש לזהור שיתנו עכ״פ מבעוד יום בע״ש קודם הלילה דמשנכנס שבת ה״ל על החמין שם איסור והוי לי׳ מבטל איסור לכתחלה אבל כשמערב קודם השבת בעוד שעל החמין שם היתר מבטל בטל. וכן הורתי בקהלתנו פה לזהר ליתן החמין קודם השבת אבל עכ״פ אין למחות ואין לערער בדבר שכבר נהגו היתר בדורות הקודמים הטובים כש״כ בדור הזה וד׳ יגדור פרצותנו. כנלענ״ד הקטן יעקב.\\\vspace{0pt}

\end{multicols}\newpage

\newchap{סימן עז}
\begin{multicols}{2}
ב״ה אלטאנא, יום ד׳ ז״ך טבת תרכ״ה לפ״ק. לחתני הרה״ג וכו׳ מ״ה יוסף איזאקזאהן נ״י אב״ד דק״ק ראטטערדאם יע״א.\\\vspace{0pt}

על דבר שאלתך במקו׳ גדול שמחזיק שיעור כ׳ או כ״ה פעמים מ׳ סאה וממנו יוצאות סילונות בדרך הכשר לחמשה מקואות קטנים שמחזיק כל אחד יותר מארבעים סאה אכן למען לא יחסר המים בהמקו׳ הגדול באשר שצריך להשקות תמיד המקואות הקטנים מביאין להמקו׳ הגדול לפעמים מים ע״י פומפען שהם שאובים עד שרבו השאובים הרבה על המים הכשרים שבתוכו ושמעת דיבת מערערים שהמקואות קטנים אינם כשרים כיון דבכל מקו׳ מהקטנים לא בא לפי ערך חשבון המים כשרים שבגדול רק שליש או רביע מהמים כשרים והשאר הוא מהשאובים שנשאבו לתוכו.\\\vspace{0pt}

תשובה – לא ידעתי חשש פקפוק בהכשר המקואות כיון שלעולם לא באו השאובים אל המקו׳ הגדול כי אם בשיש בתוכו הרבה יותר ממ׳ סאה וכלל גדול בידינו שמקו׳ שיש בו ארבעים סאה מים כשרים כל מים שאובים שבעולם אין פוסלין אותו, ולפי דברי המערערים איך משכחת כן הרי בכל פעם שיבואו שאובים לתוכו וינטלו ממים שלו נפחתו מים הכשרים שהיו בו מתחלה ולבסוף לא ישאר לפי ערך החשבון אפילו סאה ממים הכשרים שהיו בו בתחלה ואפילו הכי אמרינן שאין כל מים שאובים שבעולם פוסלים אותו ואם דעת המערערים שזה דוקא במקו׳ שהי׳ בו מתחלה מ׳ סאה מים כשרים אז לא נפסל עוד ע״י כל מים שאובים שבעולם אבל במקואות הקטנים כיון דמתחלה לא בא לתוכם רק ממים שרובו שאובים אף שבא בדרך הכשר ע״י סילונות ממקו׳ כשר מכ״מ באילו לא מכשירין, גם זה אינו דכיון שבא המים אליהם דרך הכשר ממקו׳ כשר מה לי אם המים הם עדיין בהמקו׳ גדול או שינו את מקומם וראי׳ לזה שגם מטעם זה אין חשש ממה דתנן מקואות (פ׳ ו׳ משנה ח׳) הי׳ בעליון מ׳ סאה ובתחתון אין כלום ממלא בכתף ונותן לעליון עד שירדו לתחתון ארבעים סאה ע״כ וכ״פ הרמב״ם הרי שבנדון זה בתחתון אין כלום ולכל היותר בא משם ממ׳ סאה של העליון החצי לפי ערך חשבון והחצי האחרת היא שאובין ואעפ״כ התחתון כשר אשר בתחלה לא הי׳ בו כלום ולכן אין חשש פקפוק היתר בהמקואות קטנים ג״כ כנלענ״ד: הקטן יעקב.\\\vspace{0pt}

\end{multicols}\newpage

\newchap{סימן עח}
\begin{multicols}{2}
ואשר שאלת עוד במקו׳ אשר מימיו באים בסלונות אל מקו׳ גדול ומשם באים אל מקו׳ קטן ע״י סילונות והמקו׳ קטן ג״כ מחזיק יותר מהשיעור מ׳ סאה אמנם לבל ישטופו המי מקו׳ הגדול לקטן צריך לסגור הסילון אם ירצה שלא יבואו המים לשם ע״י מנעל והמנעל נעשה במחובר דהיינו כשהסילון כבר הוא מחובר לקרקע נתחבר אליו המנעל אכן למען לפתוח המנעל ולסגרו צריך מפתח כמפתח שעושין לפתוח הדלת ולסגרו אם יש בזה חשש משום מעמיד המקו׳ בדבר המקבל טומאה.\\\vspace{0pt}

על זה אשיב לך: הנה אמת דברת אם הסילון מראשו עד סופו יש בו כשפופרת הנוד אין חשש בזה משום מעמיד בדבר המקבל טומאה אבל אתה מסופק בזה אם יש מראשו עד סופו שיעור שפ״ה ועל כן חששת והנה באהלות (פ׳ ו׳) תנן קוברי המת שהיו עוברים באכסדרה והגיף א׳ מהם את הדלת וסמכו במפתח אם יכול הדלת לעמוד בפני עצמו טהור ואם לאו טמא פי׳ שכלים אינן חוצצין מפני שמקבלים טומאה ומזה הקשה הט״ז בי״ד סי׳ שע״א דמזה משמע דמפתח מקבל טומאה ובכלים (פ׳ י״א) תנן דמנעל אינו מקבל טומאה ותירץ דמפתח אינו מקבל טומאה ומה שסמכו במפתח טמא הוא מפני שסמיכה בכלי אפילו אינו מקבל טומאה אינו מועיל אמנם הש״ך בנקודת הכסף חולק עליו וחילק בין מפתח שאינו מחובר לדלת שהוא מחובר ובזה איירי מתניתן דהגיף את הדלת אבל הך דמנעל אינו מקבל לטומאה איירי במחובר לדלת וכיון דמחובר לקרקע טהור וכן כתב ג״כ בתוספת חדשים ליישב הסתירה שהקשה הט״ז ולפ״ז יהי׳ חשש בנדון השאלה משום מעמיד בדבר המקבל טומאה אבל באמת זה אינו דאין זה דומה להגיף את הדלת במפתח דהתם איירי דהמפתח גופא הוא המנעול הסוגר כנראה ממה שהתנה התנא שאין הדלת יכול לעמוד בלא המפתח שבזה דוקא טמא ולכן כיון דהמפתח אינו מחוכר לדלת ומקבל טומאה הרי שחצץ ע״י דבר המקבל טומאה ולכן טמא אבל כאן מה שסוגר את הסילון הוא המנעול המחובר בו ועל ידה אל הקרקע שאינו מקבל טומאה והמפתח המקבל טומאה אינו רק המניע את המנעול וכיון דהוא אינו סותם מוצא המים לא הוי מעמיד בדבר המקבל טומאה ולכן המקו׳ כשר בכל ענין אפילו אין בסילון כולו כשפופרת הנוד: כנלענ״ד הקטן יעקב.\\\vspace{0pt}

\end{multicols}\newpage

\newchap{סימן עט}
\begin{multicols}{2}
ב״ה אלטאנא, בחדש מנחם שנת תרי״א לפ״ק. להרבנים המופלגים הצדיקים וכו׳ מ״ה מרדכי מיכאל יפו ומ״ה געטשליק שלעזינגער נ״י יושבי ביהמ״ד בק״ק האמבורג יע״א.\\\vspace{0pt}

על דבר שאלתכם וז״ל אחד שנשבע בהנחת ידו על המזוזה כי לא יעשה עוד לטובת בן אחותו ואמר עוד אם יעשה ככה או אם יניח להתיר לעצמו שבועתו (ואין בזכרונו אם כה אמר ואם כה) אזי לא יומחלו לו עונותיו וקשה הדבר לפניו לקיים שבועתו והוא מתחרט מעיקרא על כל הנזכר ואנחנו לא ידענו מה נדון בהתרתו אם מצד דעות שהובאו ברמ״א י״ד סי׳ ר״ל ואם מצד המבואר בש״ע שם סי׳ רכ״ח סעיף מ״ה אשר נדון זה דכוותי׳ בדומה עכ״ל מעכ״ת נ״י.\\\vspace{0pt}

תשובה – הנה מדברי שאלתכם נראה שאתם מדמין נדון השבועה למי שנשבע באלקי ישראל שכתב הרמ״א (סי׳ ר״ל) שלא ישאל עליו לכתחלה אלא מדוחק אכן לבבי לא כן ידמה דאף שדברי הרמ״א הם ע״פ מה שכתב רב האי הובא בטור שם שאין להתיר שבועה כזה ושם הושוה מי שנשבע בתורה המונחת בארון למי שנשבע בשם מכ״מ מאן יימא לן דנשבע במזוזה ג״כ כמי שנשבע בס״ת כיון דקדושת ס״ת חמור מקדושת מזוזה ואחרי דחומרת רבינו האי בלא״ה אין לה מקור בשם ורק חומרת הגאונים היא שלא להיות פרוץ בנדרים הבו דלא לוסיף עלי׳ ובלא״ה נ״ל דחומרת הגאונים אינה רק בנשבע בס״ת שזה שבועה חמורה מאוד כאומר כמו שהתורה אמת כן שבועתי אמת אבל מי שהניח ידו על ס״ת או על המזוזה ולא הזכיר שנשבע במזוזה אין זה רק כנשבע בנקיטת חפץ ואין בזה חומרת הגאון. וכיון דבלא״ה הש״ע כתב בפשיטות להתיר אפילו נשבע בהשם והרמ״א רק לכתחלה החמיר, בנדון זה לענ״ד אין כאן בית מיחוש שלא להתיר השבועה מצד זה.\\\vspace{0pt}

אכן עוד מצאתם בתבונתכם חשש שלא להתיר ע״פ מה שכתוב בש״ע סי׳ רכ״ח והיינו ע״פ מה שמבואר שם דמי שמנדה עצמו בעה״ב יש מי שאומר שיש לו התרה ויש מי שאומר שאין לו התרה ונראה למעלתכם דכיון שאמר הנשבע דאם יעשה ככה לא יומחלו לו עונותיו שזה דומה למי שמנדה עצמו בעה״ב ובזה יש לדון לפענ״ד ובתחלה אומר שמה שכתב הש״ע ב׳ דיעות במנדה עצמו בעה״ב אם יעשה כך והוא ע״פ מה שהביא בב״י תשובת הרשב״א שהתיר לשאול על נדרו ודעת המרדכי בשם הרבינו פרץ וכן רבינו ירוחם שאסרו להתיר לענ״ד צ״ע שאין אני רואה פלוגתא בזה דז״ל הרשב״א על מי שנדה עצמו בשני עולמים אם לא יעשה כך וכך ועבר ולא עשה לשון חכמים מרפה וחכם עוקר את הנדר ואת הנדוי מעיקרו וכאלו לא הי׳ הנדר והנדוי כלום וענין יהודה בפ׳ החובל מפני שהי׳ סבור דנדוי על תנאי אינו צריך הפרה אם נתקיים התנאי וע״כ לא שאל על נדויו הא שאל הכל הי׳ מופר ע״פ שאלה לחכם עכ״ל הובא בב״י שם ואח״כ הביא דעת ר״פ וריר״ו דהמנדה עצמו בעה״ב אין לו התרה וכתב הטעם בשם י״א משום דלא חל עדיין הנדר והוא כתב דנראה לו הטעם כיון דנכנס בנידוי המקום של עה״ב ונידוי של מקום תלוי בהיתרו של מקום והנה ב׳ הטעמים אילו לא שייכי רק במי שכבר נתחייב בנדוי כמו בנדון של הרשב״א שנשבע לעשות ועבר על הנדר ולא עשה וכמו במעשה דיהודה שהביאו ראי׳ ממנו דכשבא יעקב למצרים ויהודה לא השיב לו בנימין לארץ כנען כבר עבר על הנדר ונתחייב נדוי לשמים אבל מי שנשבע שאם יעשה כך יהא בנדוי ועדיין לא עשה הרי נתנה התורה רשות להתיר הנדר וכיון דהותר הנדר ממילא לא נתחייב בנדוי אם יעשה כך ולא הנדוי אנו מתירין אלא הנדר ובזה אפשר גם הר״פ והיר״ו מודים דיכול להתיר ואף דהי״מ כתבו אפכא כיון דלא חל עדיין הנדר גרע טפי זהו להתיר הנדוי שכבר נתחייב בו ועדיין לא בא הזמן לחול אבל באם מתיר הנדר קודם שעבר עליו הרי לא נתחייב בנידוי כלל ולא שייך לא הטעם הי״מ ולא הטעם של הב״י ולכן מה שכתב הש״ע דגם בנדר בנידוי עה״ב אם יעשה כך דמשמע שעדיין לא עשה תלוי בפלוגתת הראשונים וכן כתב הר״מ פאדו׳ בשו״ת סי׳ ע׳ ג״כ בענין כזה שלר״פ ויר״ו אין לו התרה לענ״ד צ״ע דאפשר דבזה כ״ע מודה שיכול להתיר הנדר וא״כ גם בנדון דידן לא יהי׳ קפידא להתיר כיון שעדיין לא עבר ולא חל הנידוי והרי ודאי כוונתו בשבועתו שלא יעשה כך הא אם יעבור על שבועתו הזאת ויעשה כך אז לא יהי׳ כפרה לעונותיו והרי לא בא לידי כך לעבור על שבועתו אכן אפילו אם לא נחלק חילוק זה ונאמר דאין חילוק אם כבר עבר או לא לענ״ד אינו דומה נשבע שלא יהי׳ כפרה לעונותיו למנדה עצמו בעה״ב לא מבעי׳ לטעם הי״מ לשיטת האוסרים משום דעדיין לא חל הנדר שאינו דומה שהרי לענין כפרת עונות חל הנדר כבר דבכל שעה אנו צריכים לכפרה ואדרבא עיקר כפרת עונות היא רק בעה״ז דמיתה ממרקת אפילו העון היותר חמור דחלול השם כדאמרינן ביומא (דף פ״ו) אלא גם לטעם הב״י משום דתלוי בדעת המקום ולכן אין בני אדם יכולים להתיר יש לומר דנדר כזה לא שייך רק לענין מנדה עצמו לעה״ב דבזה אסר עליו חלקו לעה״ב ואין מאכילין לאדם דבר שאוסר עליו אבל כשאמר שלא יהי׳ לו כפרת עונות וכי בדעת השוטה הדבר תלוי אם מלך מלא רחמים רוצה לכפר עונותיו והרי אפילו במבעט בכפרה שאמר שלא יכפר לו חטאתו ושלא יכפר לו יוה״כ יש מ״ד בכריתות (דף ז׳) דאעפ״כ מכפר דכפרה ממילא קאתי ואפילו למ״ד שם דלא מכפר היינו מטעם דהאומר כן לא שב מחטאתו ויוה״כ וחטאת לא מכפרים בלא תשובה אבל מטעם דבורו אין מעוכב הכפרה וא״כ מי שאמר שלא יהי׳ כפרה לעונותיו דברים בעלמא הם ואפשר דאפילו התרה אינו צריך ודמי זה למה שכתב בפסקי מהראי (סי׳ קצ״ב) והנה נראה דבטוי כה״ג שאומר אפילו בפי׳ אם יעשה דבר פלוני יכפור בהשם יתב׳ אין זו לא שבועה ולא כינוי ולא יד הואיל ואינו מוציא שום רמז שבועה או נדר או איסור מפיו ולא דמי אפילו למי שאמר אם יעשה דבר פלוני תבא עליו קללה זו דהתם נמי אשכחן בתלמוד דארור בו נדוי בו קללה בו שבועה אבל שבועה כזו לא מצינו וא״כ נוכל לומר דלא צריך התרה לגמרי עכ״ל ואפילו לדעת המרדכי שאוסר לשאול על נדרו אם אמר שאם יתיר שבועתו אינו יהודי ונפסק ברמ״א (סי׳ רכ״ט) כן הרי התם שמא יכפור ויבא תקלה ע״י התרה מה שלא שייך בנדון זה אפילו אמר הלשון השני שנסתפק בו דאם יניח להתיר עצמו שבועתו לא יומחלו לו עונותיו ובלאו כל זה אפילו דמי לשון זה לגמרי למנדה עצמו בעה״ב לכל עניניו הרי כבר פסק הרמא בסי׳ רכ״ח שיש להתירו במקום מצו׳ והוא ע״פ פסק מהר״ם פאדו׳ שכתב הטעם כיון דתלה בדעת המקום א״כ ודאי דעת המקום ב״ה שיקיים מצותיו ויסכים עם ההתרה והרי הכא עיקר שבועתו הי׳ לבטל מצות דברי הקבלה דמבשרך אל תתעלם דנדרש בגמר׳ בפרט ג״כ על בני אחותו כשאמרו דהנושא בת אחותו מקיים מצו׳ זו ולכן אף דמטעם נשבע לבטל המצו׳ השבועה חל כיון דלא מפורש בתורה כמשכ׳ בש״ע (סי׳ רל״ט ס״ו) מכ״מ להתיר שבועתו כדי שיכול לקיים מצו׳ דברי קבלה ודאי מקרי התרה לדבר מצו׳ ולכן לענ״ד ע״פ כל השיטות יש התרה לשבועה זו כמו לשאר שבועות. והנה אחרי שנסתפק איך נשבע אם דכשיעשה המעשה לא יהי׳ לו כפרה או אם יניח להתיר לו שבועתו לא תהי׳ לו כפרה צריך התרה ע״פ שני הענינים דע״פ ענין הראשון ההתרה היא על המעשה בלבד שמתחרט על שבועת המעשה ויתירו לו שבועת המעשה וע״פ ענין השני דמי זה למה דנפסק (בסי׳ רכ״ט ס״ד) ע״פ תשובת הרשב״א דמי שנשבע על דבר א׳ ונשבע שלא ישאל על שבועתו נשאל תחלה על האחרונה ואח״כ על הראשונה ולכן האיש הזה צריך ג״כ מספק התרה תחלה על מה שנשבע שלא להשאל על שבועתו ואח״כ התרה על המעשה עצמו ועל כן אם מעלתכם יסכימו בדבר נתיר לו ההתרה אכן מכ״מ אחרי שנפרץ בנדר חמור ונפלא כזה ראוי להחמיר עליו ולקנסו קצת כמשכ׳ הש״ך סי׳ ר״ל בשם מהר״ם מינץ להטיל עליו תענית או שני תעניתים לפי מה שהוא אדם חלש או חזק וגם שיתן קנס מה לצדקה וסר עונו וחטאתו תכופר. כנלענ״ד הקטן יעקב.\\\vspace{0pt}

\end{multicols}\newpage

\newchap{סימן פ}
\begin{multicols}{2}
ב״ה אלטאנא, יום ד׳ כ״ה אייר תרכ״ב לפ״ק. להרה״ג וכו׳ מ״ה בירך אברהם אויסטערליץ נ״י הגאב״ד דק״ק סקאליטץ יע״א.\\\vspace{0pt}

מה שהוקשה למעכ״ת נ״י על פסק מהרי״ק הובא בי״ד סי׳ רכ״ח ס׳ כ״א אם נשבעו ד׳ ביחד לעשות דבר אחד מקרי על דעת רבים דכל א׳ נשבע לג׳ חבריו – מגמרא דשבועות (דף כ״ט) שאמר משה לישראל לא על דעתכם אני משביע אתכם כ״א על דעתי ועל דעת המקום ומסיק השס׳ כי היכא דלא תהוי הפרה לשבועתייהו והרי לדעת מהרי״ק בלא״ה הוי עד״ר דכל ישראל נשבעו יחד על דבר אחד – לענ״ד יש לומר על פי מה שפסק בש״ע סי׳ רכ״ח ס׳ כ״א דנדר על דעת רבים דוקא בלא דעתם אין לו התרה אבל בהסכמתם אפשר להתיר וגם הרמ״א שם לא החמיר רק לכתחלה שלא להתיר וגם בזה כ׳ הש״ך דגם להמחמירין אם מתחרטים בחרטה א׳ יש להתיר ולכן קאמר שפיר כי היכא דלא תהוי כפרה לשבועתייהו דאי משום דהסכימו כלם הוי עד״ר היו יכולים להפר שבועתייהו על ידי שיסכימו כלם בחרטה כאחד ולכן נשבעו ע״ד הקב״ה ומשה דבזה לא מהני חרטה שלהם עוד:\\\vspace{0pt}

אמנם זה אינו מועיל רק ליישב פסק הרמ״א כפי המהרי״ק אבל על המהרי״ק עצמו אין זה יישוב המספיק שאחר שכתב דזה מקרי על ד״ר כתב אלא שמעט אני חוכך בדבר ממה שכתב הרמב״ן בהתרת החרמי צבור שהקשה הרי נהגו עכשיו להחרים ולהשביע על דעת המקום וה״ל עד״ר וק״ל נדר שהודר עד״ר אין לו הפרה ולפי הסברא למה לו לרמב״ן לתלות קושיתו במה שנהגו להשביע עד״ר תיפוק לי׳ דכל מה שמחרימים הקהל לצורך הקהל ותקונו וכל אחד מסכים לחבירו וצייתי לו דחשיב עד״ר מכ״מ קצת יש לתרץ דלרווחא דמלתא נקט הרמב״ן דעכשיו נהגו עכ״ד בקיצור הרי שמהריק בעצמו הקשה קושית מעכ״ת על הרמב״ן וא״כ לא ס״ל כתירוצנו וע״כ דס״ל דאפילו אם מסכימים אין התרה לנרר עד״ר וא״כ יקשה הקושיא על הגמרא דלא שייך תירוצו דלרווחא דמלתא נקט (ובאמת זה צ״ע על המהרי״ק שהרי הש״ך כתב בשם הרמב״ן דאם מסכימים כלם בחרטה אחת יש להתיר וא״כ מה קושיא עליו הרי ע״כ הוצרך להקשות ממה שנשבעים ע״ד הקב״ה) אבל עכ״פ ע״פ דעת המהריק יקשה עליו מן הגמרא אמנם לענ״ד י״ל דגם המהרי״ק לא כתב דנקרא עד״ר אלא היכא שנשבעו יחד לתועלתם כנדון השאלה שהביא שם שהרופאים עשו שותפות שכיס א׳ יהי׳ לכלם ונשבעו יחד על זה ואח״כ הלך האחד והתיר השבועה בלא דעת והסכמת אחרים אבל בנשבעו רבים יחד בדבר שאין תועלת מזה לזה בזה לא נחשב עד״ר גם למהרי״ק ולכן גם במה שכתב מחרמי צבור דייק המהריק לכתוב שמחרימין הקהל לצורך הקהל ותקונו ולכן כשנשבעו ישראל על קיום התורה יחד אפילו נחשוב לענין מצות שבין אדם לחבירו כעל דעת רבים כפסק המהריק מכ״מ לענין מצות שבין אדם למקום לא נחשוב כן דנשבע אדם לחבירו אלא דכל אחר נשבע לעצמו ולתועלתו אף שהסכימו ונשבעו יחד ולכן לשבועה כזה יהי׳ התרה גם למהריק אשר על כן קאמר שפיר דנשבעו ע״ד הקב״ה ומשה דלא נהוי הפרה לשבועתייהו גם לענין מצות שבין אדם למקום: כנלענ״ד הקטן יעקב.\\\vspace{0pt}

\end{multicols}\newpage

\newchap{סימן פא}
\begin{multicols}{2}
ב״ה אלטאנא, יום ו׳ כ״ו תשרי תר״י לפ״ק. להרה״ג וכו׳ מ״ה גבריאל אדלער הכהן נ״י הגאב״ד דק״ק אבערדארף יע״א.\\\vspace{0pt}

מעכ״ת נ״י העיר על הנוסחא שבסדר הפרת נדרים ואפי׳ נזירות שמשון שהרי נזירות שמשון אין לו התרה (וכן שמעתי ג״כ מפי אדמ״ו הגאון מ״ה אברהם בינג זצ״ל) ומה שרצה מר נ״י לומר שתיבת שמשון בלבד הוא טעות בזה לענ״ד לא הועיל שנראה שבכוונה נכתב זה שהרי כבר אמר וכל מיני נזירות שקבלתי עלי ושוב הוסיף ואפילו נזירות שמשון ולא עוד אלא במה שאמר ואפילו נראה שידע דבנזירו׳ שמשון יש חדוש יותר לבקש התרה עליו מבשארי מיני נזירות ובלי ספק גדול הי׳ המחבר סדר הזה אשר נתפשט ונתקבל ברוב תפוצות ישראל ולכן עלה ברעיוני לתור ולבקש טעם למה נזירות שמשון אין לו שאלה והנה באמת מסוגיא דמכות (דף כ״ב) שהזכיר מעכ״ת נ״י אין ראי׳ כ״כ שמפי׳ הריב״ן שם שכתב בנזיר שמשון שקבל נזירתו ע״י מלאך ע״ש הי׳ נראה דמוקי הך נזיר דשם דוקא בנזיר כזה שהזירו מלאך ולא במי שקבל עליו נזירות שמשון אבל באמת זה אינו דמסוגיא דנזיר (דף י״ד) מוכח דגם במי שקבל עליו נזירות שמשון אין לו התרה וכ״פ גם הרמב״ם ה׳ נזירות (פ׳ ג׳) וכן מוסכם מכל הפוסקים אכן הוקשה בעיני מאיזה טעם לא יהי׳ התרה לנזירות כזה דאף דלשמשון עצמו לא הי׳ אפשר להתיר נזירותו כיון דלא בא לו ע״י דבורו מכ״מ מי שהזיר להיות כשמשון הרי לא חל נזירותו רק ע״י דבורו א״כ למה לא נאמר גם בזה כבשאר נדרים לא יחל דברו אבל אחרים מוחלין לו כדאמרינן חגיגה (דף י׳) ואין לומר כיון שנזר כשמשון הוי כאומר בפי׳ שלא יהי׳ לו התרה דז״א שהרי מבואר בי״ד סי׳ (רכ״ט ס״ד) שאם נשבע או נדר שלא לעשות דבר פלוני בלא התרה ובלא הפרה נשאל תחלה על מה שאמר בלא התרה ואח״כ נשאל על עיקר הנדר ע״ש וא״כ ה״נ אפילו נחשוב מה שאמר כשמשון כאילו אמר על מנת שלא יהי׳ לו התרה ישאל תחלה על זה ואח״כ על עיקר הנזירות והנה ראיתי בשו״ת מבי״ט (סי׳ קע״ד) שמפרש מה שכתב הרמב״ם הנ״ל שמי שנדר בנ״ש אין יכול להשאל על נדרו שנזירות שמשון לעולם הי׳ דהכי קאמר כיון שהוא לא נדר רק קודם שבא לעולם נדרו מלאך א״כ לא שייך אצלו לא יחל דברו וממילא לא שייך ג״כ אבל אחרים מוחלין לו אכן גם זה לא שייך רק בנזיר שמשון עצמו שנדרו המלאך אבל מי שנדר להיות נ״ש לא שייך גם זה ובינותי בספרים ולא מצאתי טעם בזה עד שהאיר הקב״ה עיני וראיתי אחד קדוש עמד ג״כ על חקירה זו והוא בשו״ת חות יאיר (סי׳ ט״ו) ופתח דבריו אל אמנע טוב מבעליו להודיע צערי לרבים אשר מימי נסתבכתי בסבך העיון דהא מלתי׳ וכו׳ ע״ש שהאריך בזה שלא ידע טעם למה לא יהי׳ התרה למי שנדר נזירות שמשון ולבסוף כתב מפני שמצו׳ לקיים דברי חז״ל אמרתי ליתן פנים לדבריהם דמפני ב׳ טעמים אין התרה לנ״ש האחד מגזירת חז״ל דאם אתה אומר להתיר נ״ש ילעזו העולם דמעתה הי׳ נזיר לזמן וטמא למתי׳ שלא כדין והשני כיון דגוף הנדר ואיסור השאלה קבל עליו במלה א׳ במה שאמר הרני נזיר שמשון תיכף שבא לשאול עבר על קבלתו ונדרו משא״כ מי שנדר בלא התרה הם ב׳ דברים חלוקים ומסיים שם אבל כל זה אינו שו׳ לי ולא נתישבה דעתי בו ע״ש.\\\vspace{0pt}

והנה אם נאמר כטעם הראשון של ח״י כבר מצאנו התנצלות על הנוסחא של ואפילו נזירות שמשון כיון דהתכלית של התרה היא להנצל מעון נדר א״כ מהני ההתרה של נ״ש ג״כ להנצל מאיסור דאורייתא כיון דלטעם זה רק מדרבנן אין שאלה לנ״ש אכן באמת אין דעתי נוחה בזה שאף שיש קצת סמך לזה שאיסור שאלה בנ״ש הוא רק מדרבנן מכח מה שהקשו התוספ׳ בנזיר (דף ד׳) למה לא קחשיב שאלה בדברים שבין נזיר עולם לנ״ש דלפ״ז י״ל דכיון דרק מדרבנן הוא לא קחשיב מכ״מ אין משמעות הגמרא והפוסקים כן שאיסור שאלה הוא רק דרבנן וגם טעם השני של ח״י הוא מוקשה דלמה אי אפשר להפריד ג״כ כמו בנדר בלא התרה ג״כ שבתחלה יתיר מה שאמר כשמשון ואח״כ מה שאמר הרני נזיר אכן טעם זה יש להטעים קצת דבשני דברים נחלק סתם נזירות מנ״ש דבסתם נזירות אסור להטמא למתים ובנ״ש מותר וסתם נזירות ל׳ יום ונ״ש הוא לעולם ולכן אי אפשר להתיר רק כשמשון מבלי שיגע ג״כ בהך דהרני נזיר דבזה ישתנה הך הרני נזיר להיות ל׳ יום דוקא אחר שהיה כוונתו לעולם וגם יהי׳ כולל איסור טומאה ובתחלה לא הי׳ טומאה בכלל ולכן אי אפשר להתיר הך כשמשון לבד כן יש ליתן טעם קצת לאיסור התרה בנ״ש אכן כל זה דוקא באמר בפי׳ הרני נזיר שמשון או כשמשון אבל בהזיר נזירות מן התגלחת ומן היין והתנה בפי׳ שלא מן הטומאה בזה הי׳ נראה לכאורה דגם זה יהי׳ דינו כנזיר שמשון דכן משמע ממה דאמרינן בנזיר (דף י״א) הרני נזיר ע״מ שאהא שותה יין ומיטמא למתים ה״ז נזיר ואסור בכולם הרי שהתנה ב׳ דברים יין וטומאה ולכך אסור בכולם מפני שהתנה על מה שכתוב בתורה כדאמרינן בגמרא שם אבל התנה היתר טומאה לבד משמע דמהני שהרי בזה לא על מה שכתוב בתורה כיון שיש נזירות כזה דהיינו נזירות שמשון ולכן תמהתי על מה שכתב הרמב״ם ה׳ נזירות (פ׳ א׳ ה׳ י״ג) דהאומר הרני נזיר ע״מ שאהי׳ שותה יין או מטמא למתים וכו׳ ה״ז נזיר ואסור בכולם מפני שהתנה על מה שכתוב בתורה ע״ש למה נקרא מתנה על מה שכתוב בתורה שהרי נזירות שמשון קבל עליו וצ״ל דזה איירי בשהיתה כוונתו להיות נזיר סתם ובפרט אם קבל נזירות כזה על זמן קצוב דבזה אי אפשר להיות נזיר שמשון אבל לפ״ז נלענ״ד אם אמר הרני נזיר ע״מ שאהי׳ מטמא למתים והי׳ בדעתו לקבל עליו בזה נ״ש אז מועיל קבלתו להיות נ״ש ומותר לטמא למתים כיון דלא שייך בזה מתנה על מ״ש בתורה ונזירות שמשון כזה יש לו ג״כ שאלה שהרי זה הוי כשאר נדר ולא תלה נזירות בדבר שאין לו שאלה והוי כמי שנשבע על דבר לעולם שיש לו שאלה כמבואר (סי׳ רכ״ט ס׳ ז׳) וא״כ לא שייך בזה טעם הב׳ של הח״י ולטעם הראשון עכ״פ מן התורה יש לו שאלה ועל נזירות שמשון כזה י״ל שנתכוון הנוסחא בסדר ה״נ ואפילו נזירות שמשון דהיינו ענין נזירות שמשון אם אסר עליו ואחר שנאמר בסדר הזה ואין אני מבקש התרה על אותן הנדרים שאין להתיר אותם לכן יפה כוון לכלול גם נ״ש בתוך סדר זה דהיינו נ״ש שיש להתיר. כנלענ״ד הקטן יעקב.\\\vspace{0pt}

\end{multicols}\newpage

\newchap{סימן פב}
\begin{multicols}{2}
ב״ה אלטאנא, יום ד׳ כ״ד מרחשון תרכ״ה לפ״ק. לחתני הרה״ג וכו׳ מ״ה משלם זלמן הכהן נ״י אב״ד דק״ק שווערין יע״א.\\\vspace{0pt}

על דבר שאלתך בעשיר א׳ אשר נדבה רוחו סכום מעות לכתוב לו ס״ת ועתה כאשר שמע מענין בתי עניים בעיה״ק תוב״ב נכספה וגם כלתה נפשו להמיר טוב בטוב ולתת הסכום הנ״ל לבנין הנ״ל אם מותר לעשות כן.\\\vspace{0pt}

תשובה – הנה פתחת בפתחי היתר שהרי אפילו בס״ת עצמו מותר היחיד למוכרו ולעשות בדמיו כל מה שירצה כמבואר בא״ח סי׳ קנ״ג ס״י א״כ כש״כ שמותר לשנות המעות המזומן לכתיבת ס״ת שהרי הזמנה לאו מלתא היא והוי כטווי לאריג כמשכ׳ המג״א שם ס״ק ה׳ שאפילו אביי מודה דלא קדוש כש״כ לדידן דפסקינן כרבא דאפילו באריג הזמנה לאו מלתא היא ואע״ג דלפי המבואר שם ס׳ ה׳ אין לשנות מעות שגבו לס״ת רק לקדושה חמורה הרי כבר חילק שם המג״א שזה דוקא בגבו מצבור שאין לשנות נדבתם בלא דעתם אבל ליחיד מותר אלא שעומד לנגדך מה שכתב הט״ז שם ס״ק ב׳ דהזמנה לאו מלתא לא מועיל רק מטעם קדושת הדמים אבל המנדב מטעם נדר חייב לקיים דברו ולכן גם בנדון השאלה מטעם נדר אינו יכול לשנות עכ״ד. הנה צדקת בכל דבריך אילו שאע״ג שלכאורה כל האחרונים הרמ״ך שהקשה קושיא זו דאמאי אינו מותר לשנות הרי הוי כטווי לאריג והב״י והב״ח והמג״א שלא תרצו כתירוץ הט״ז לא ס״ל כן דהוי כנדר מכ״מ כיון דלא חלקו עליו בפי׳ אין לסמוך על זה לחלוק עליו ועוד דטעמו מסתבר וגם הביא ראי׳ מהמרדכי אלא דמטעם זה יהי׳ תקנה אחר שמתחרט המנדב לס״ת יכול לשאול על נדרו ואפשר שמטעם זה לא ניחא להו להפוסקים הנ״ל בתי׳ הט״ז דלא ה״ל להפוסקים הראשונים לכתוב סתמא דאסור לשנות שהי׳ להם לפרט התנאי מבלי שישאל על נדרו ולכן כשישאל המנדר על נדרו בפתח חרטה לעקור הנדר מעיקרא יהי׳ מותר לשנות לבנין בתים דאין לומר כיון דמסתמא נדר ליתן הס״ת לביה״כ לקרות בו א״כ הוי כנדר על דעת רבים שאין לו הפרה דז״א כיון דלא פרט בפי׳ על דעת רבים לית לן בה וכן העלה בשו״ת חות יאיר (סימן נ״ט) באחד שנדר חדר בביתו להתפלל לרבים ועתה רוצה ליטול החדר לדור בו שיכול לשאול על נדרו ואין בזה משום נדר ע״ד רבים כמבואר בי״ד סי׳ רכ״ח.\\\vspace{0pt}

ולכן אם קיים המנדב כבר מצות כתיבת ס״ת לענ״ד יכול לשאול על נדרו ומותר לשנות אבל בשעדיין לא קיים צע״ג אם יש להתיר לו לשנות שמלבד נדרו מוטל עליו העשה ומה תהא עלי׳ דאפילו יגמור בדעתו לקיים גם מצו׳ זו להבא הרי לזמן מרובה חיישינן שמא ימות כדאיתא בי״ד סי׳ רי״ט במי שנדר לקנות בית ואע״ג דהמג״א סי׳ תקס״ח ס״ק י״ג רצה להוכיח מפסק הש״ע שם במי שנדר תעניות שיכל לדחותם עד ימות החורף דלא חיישינן שמא ימות והניוח הסתירה בצ״ע כבר תירצתי קושיא זו בספרי ע״ל בסוכה (דף כ״ג) וחילקתי על פי פסקי הרמב״ם בין נדר לזמן מוגבל דלא חיישינן שמא ימות ובין נדר סתם דחיישינן בכל יום שמא ימות באשר הוא סוף כל האדם ולכן גם בנדון זה יש חשש שמא ימות ולא יקיים עוד מצו׳ ע׳ זו ואמת שיש פלוגתא בין הפוסקים אם בזה״ז עדיין נוהגת מצו׳ זו דלדעת הפרישה ע״פ דברי הרא״ש אינה נוהגת ולפי דברי הט״ז נוהגת כמבואר בי״ד סי׳ ע״ר אמנם אע״פ שהש״ך הסכים עם הפרישה מכ״מ כיון שהב״י והב״ח והעט״ז ובאר הגולה והט״ז הסכימו לדעת א׳ שגם להרא״ש יש מצות ע׳ בזה״ז אין לדחותה וגם מה שבשו״ת שאגת ארי׳ סי׳ ל״ו העלה שבזה״ז כיון שאין אנו בקיאים בחסרות ויתרות כדאמרינן בקידושין אין לנו ספר תורה כשרה ועל כן אזיל המצו׳ בזה״ז אולי לא ראה במכ״ה מה שכתב הר״ן בחידושיו לסנהדרין (דף ד׳) בשם הרשב״א שע״פ מסכת סופרים אזלינן בתר רוב ולכן אין תימא שרב יוסף אמר בקידושין שאין אנו בקיאים בחסרות ויתרות שהרי אפילו אבות העולם ב״ש וב״ה שפליגי בדרשה דקרנות לא היו בקיאין כמבואר בר״ן שם ומכ״מ כיון שמסורה שלנו שנקבעה ע״פ רוב כ״כ היא מבורר שלא לחוש אפילו על דרשת ב״ה ודאי הי׳ מבורר להם לאמת גמור ולמה לא נאמר ג״כ לענין כתיבת ס״ת שאמת הוא כיון שנתייסד על פי רוב שהוא כלל גדול בתורה שסוקלין ושורפין עליו וכן כ׳ בשו״ת גינת ורדים חא״ח כלל ב׳ כיון שהמסורה שלנו מיוסדת ע״פ הרוב דיינינן לה כאלו ברור לנו שכן נתנה מסיני ולכן גם מטעם השאגת אריה אין לפטור ממצות ע׳ זו בזה״ז ועל המנדר לקיים מצו׳ זו כאשר נדר ועוד דאפילו נימא כדברי הפרישה והש״ך שכוונת הרא״ש שבזה״ז תחת מצות כתיבת ס״ת היא מצות כתיבת שאר ספרים שלומדים בהם ג״כ יתחייב המנדב לקנות שאר ספרים וללמוד בהם הוא או אחרים שמצו׳ זו מוטלת עליו ואי אפשר להתקיים ע״י אחרים משא״כ בנין בתי עניים, ולכן הנלע״ד אם המנדב כבר קיים מצות כתיבת ס״ת אז יכול לשאול על נדרו ולתת המעות לבנין בתים אבל אם עדיין לא קיים יכתוב בו הס״ת לקיים מצות ע׳ של כתיבת ס״ת ולא ישנה. כנלענ״ד הקטן יעקב.\\\vspace{0pt}

\end{multicols}\newpage

\newchap{סימן פג}
\begin{multicols}{2}
ב״ה אלטאנא, בחדש תמוז תר״י לפ״ק. להרה״ג וכו׳ מ״ה יצחק דוב ב״ב הלוי נ״י הגאב״ד דק״ק ווירצבורג יע״א.\\\vspace{0pt}

מעכ״ת נ״י העירני על מה שכתבתי בספרי ערוך לנר לפרש מה דאמרינן יבמות (דף מ״ב) מסתמך ואזיל ר׳ אבו׳ אכתפי׳ דרבי נחום שמעי׳ וכו׳ שלכן למדו הלכות בעוד שנסמך עליו משום דאסור להשתמש בשונה הלכות ואפילו הוא תלמידו שכן משמע בהא דריש לקיש במגילה (דף כ״ח) ועל זה השיג מעכ״ת נ״י דזה אינו דכל מלאכות שעבד עושה לרבו תלמיד עושה לרבו כמו שפסקו הטוש״ע י״ד סי׳ רמ״ה וכ״כ הרמב״ם בפי׳ המשניות פ׳ א׳ דאבות וגם המהרש״א במגילה שם דאסור לשמש בשונה הלכות לא נאמר על תלמידו.\\\vspace{0pt}

תשובה – הדין עם מר נרו אם הוא תלמידו המובהק דהיינו שלמד רוב חכמתו ממנו כמשכ׳ הטוש״ע (סי׳ רמ״ב) דכל הדינים שנאמרו שם בתלמיד נגד הרב הם דוקא בתלמיד כזה שלמד רוב חכמתו ממנו וגם אנכי לא נתכוונתי בכתבי שאסור לשמש בשונה הלכות דמשמע אפילו הוא תלמידו אלא לתלמיד שלא למד רוב חכמתו ממנו דכל שלא נתגדל להיות קרוב לרבו מקרי ג״כ תלמידו כמו שכתב הש״ך שם ס״ק י״ב ועל תלמיד כזה כתבתי שמשמע שאסור להשתמש בו אם הוא שונה הלכות כמו אותו האיש שנשא לר״ל וכן י״ל שרבי נחום שמעי׳ דרבי אבו׳ הי׳ תלמיד כזה שלא למד רוב חכמתו מר׳ אבו׳ ולכן לא רצה ר׳ אבו׳ להשתמש בו רק לצורך הלימוד שתהי׳ דעתו צלולה יותר וכה״ג אמרינן במגילה (דף כ״ח) ברבא דעייל לבי כנשתא והתנצל שלא משום מטר נכנס אלא דשמעתתא בעי צילותא הרי מה שאסור משום כבוד ביה״כ מותר כשיש בו תועלת התורה וכן י״ל כה״ג דאף דלהשתמש בשונה הלכות אסור משום כבוד התורה מכ״מ אם יש בו צורך התורה שרי. ונלענ״ד שמה שאסרו להשתמש בשונה הלכות זה דוקא שמוש של עבדות כגון הך דריש לקיש שנשאו האיש על כתפו או כהך דר׳ אבו׳ דמסתמיך ואזיל אכתפי׳ דרבי נחום אבל שמוש בדברים שבדרך גמילות חסד רגיל אדם לעשות לחבירו אינו בכלל זה דאל״כ לא מצאנו ידינו ורגלינו לקבל שמוש מבני תורה אחר שאין אצלנו תלמיד שרוב חכמתו מרבו כיון שלומדים מהספרים, והנה לכאורה אפשר למצוא התנצלות לזה ולומר דהך דאסור להשתמש בשונה הלכות הוא דוקא באם שואל ממנו שמושו אבל אם נדבה רוחו לשמשו מותר לקבל ממנו וכן הביא בברכי יוסף סי׳ רמ״ו בשם יש מי שכתב אבל הוא השיג עליו מטעם דהשונה הלכוח אינו יכול למחול כבוד התורה ע״ש ולא ידעתי למה הוצרך לחלוק מכח סברא דממקומו הוא מוכרע שהרי ריש לקיש לא בקש מאותו האיש שישאהו אלא הוא מעצמו עשה כן כדאמרינן שם מטא עורקמא דמיא אתי׳ האי גברא ארכבי׳ אכתפי׳ ואעפ״כ כששמע ר״ל דתנא ד׳ סדרי משנה אמר לו שדי בר לקישא במיא וגם כשהשיב לו ניחא לי דאשמעי׳ למר לא קבל ממנו עד ששנה לו הלכות. והנה לכאורה גם מצד זה יהי׳ התנצלות לקבל שמוש מש״ה אם שנה לו איזה הלכות וכ״כ בברכי יוסף (שם) אם קודם שישתמש בשונה הלכות מלמדו דבר מדברי תורה נראה דיכול להשתמש בו דכיון שלמד ממנו דבר א׳ צריך לנהוג בו כבוד והכי אשכחן במגילה ד׳ כ״ח בריש לקיש עכ״ל אבל לא נלענ״ד לומר כן וכי מפני שהוא צריך לנהוג בו כבוד מותר לזה לבזותו ולהשתמש בו ועוד שהרי לפי מה שכתבתי בשם הטוש״ע כל הדינים דבסימן רמ״ב שבתוכם גם הך דכל מלאכות שעבד עושה לרבו תלמיד עושה לרבו אינם רק ברבו מובהק הרי דתלמיד שלמד מרבו הרבה רק שלא למד רוב חכמתו ממנו אף שצריך לנהוג ברבו כבוד אינו בכלל דכל מלאכות וכו׳ וכיון דאין צריך לשמשו אסור לקבל ממנו וכנ״ל. והנה עוד כ׳ הברכי יוסף שם וז״ל יש להסתפק לפי מ״ש לעיל סי׳ רמ״ד דת״ח מופלג בחכמה דינו כרבו מובהק אם יהי׳ גם לענין זה דינו כדין רבו ושונה הלכות יכול להשתמש בו ומעובדא דר״ל אין כל כך ראי׳ דדלמא בו בפרק לא הי׳ נחשב בדורו מופלג בחכמה עכ״ל ומלבד שלא הכריע ספקו וגם הצד של ספקו דר״ל לא הי׳ נחשב אז בדורו דוחק הוא כי שם באותו מקום סמוך למעשה דמייתי שם קרא לר״ל תקיפא דארעא דישראל ומשמע דמייתי גם הך מעשה להשמיענו ענותנותי׳ דר״ל בלאו הכי מזה אין התנצלות אצלנו דמאן חשיב ומאן ספון להחשיב עצמו ת״ח מופלג בחכמה בשביל להשתמש בשונה הלכות ובפרט שאין אצלנו מי שיש לו דין ת״ח גמור כמבואר סי׳ רמ״ג. ולכן עיקר ההיתר אצלנו לענ״ד הוא ע״פ ב׳ דברים דהיינו שאין אסור להשתמש רק בדבר שיש עבדות שיש בזה ביזוי תורה כמו בהך דר״ל ורבי נחום שנשאו על כתפו ושגם אותו שמוש מותר אם הוא לצורך למוד תורה כמו שפירשתי הנך מעשות דר״ל ור׳ נחום וכן נראה מהא דרב הונא במגילה (דף כ״ח) דדרי מרא אכתפי׳ וכו׳ ואי לא אתיקורי אנא בזילותא דידך לא ניחא לי הרי שאף שרב הונא הי׳ מופלג בדורו כדאמרינן שם (דף כ״ב) דאפילו רב אמי ורב אסי דכהנא חשיבי דארעא דישראל מיכף כייפו לי׳ וגם רב חנא בר חנילאי בעצמו מיכף כייפי לי כנראה (דף כ״ז) אעפ״כ לא רצה לקבל ממנו שמושו אם לא דרגיל בכך בעירו ואין לו בזה זילותא וגם אין לומר דרב הונא מחמיר על עצמו הי׳ ולא מן הדין שהרי זה חומרא דאתי לידי קולא דאסור למנוע לתלמידו לשמשו כדאמרינן בכתובות (דף צ׳) ונפסק בש״ע (סי׳ רמ״ב) וא״כ ה״ה במי שיש לו דין תלמיד נגד המופלג בחכמה ולכך מהך עובדא דר״ה ש״מ תלת ש״מ דאפילו המופלג בחכמה אין להשתמשי בשונה הלכות וש״מ דאין אסור רק תשמיש שיש בו קצת ביזו׳ להמשמש וש״מ דאפילו רצה מעצמו לשמשו אין שומעין לו שהרי רב חנא מעצמו בקש לשמש לר״ה ולא בקש לקבל ממנו. כנלענ״ד הקטן יעקב.\\\vspace{0pt}

\end{multicols}\newpage

\newchap{סימן פד}
\begin{multicols}{2}
ב״ה אלטאנא, אדר שנת תרכ״ב לפ״ק.\\\vspace{0pt}

בי״ד בי׳ רמ״ב ס׳ ט״ו פסק הרמ״א שמותר לומר רבי מורי פלוני וכתב הש״ך שזה דוקא שלא בפניו ובשערי תשובה הביא בשם ספר רגל ישרה וליקוטי פרי חדש שחולקין על הש״ך מדכתיב אדוני משה כלאם ולענ״ד אדרבה מזה ראי׳ לש״ך דבמדרש תנחומא פ׳ תצו׳ איתא כתיב במשה ומשה בן ק״ך שנה במותו ויהושע נגנז בן מאה ועשר שנים ולמה פחתו לו עשר שנים בשביל שאמר בפני משה רבו עשרה דברים שנאמר ויען יהושע בן נון משרת משה מבחוריו ויאמר אדוני משה כלאם לפיכך פחתו לו עשר שנים עכ״ל והוא תמו׳ שהרי אדוני משה כלאם אינם רק ג׳ דברים והמפרש נדחק דכלאם במ״ק עשר ולענ״ד יתיושב ע״פ מה דאיתא בספרי הובא בילקוט פ׳ בהעלותך אדוני משה כלאם אמר לו רבוני כלם מן העולם לבני אדם שבשרוני בשורה רעה זו עכ״ל והם עשרה תיבות וצ״ל שהכתוב קצר אבל באמת אילו היו דברי יהושע ומדוייק מאוד שאמר רבוני ולא רבוני משה שאסור לקרות רבו בשמו בפניו כדעת הש״ך ומה שכתוב בפסוק שאמר אדוני משה כלאם י״ל דבלא״ה יש לדקדק הכפל ויען יהושע וגו׳ ותו ויאמר לכן י״ל דב׳ אמירות היו בתחלה בפני משה אמר עשרה הדברים שבספרי ובהם לא הזכיר שם משה ואח״כ אמר אדוני משה כלאם שלא בפני משה דרך בקשה מי יתן שיהי׳ כן או אפשר אחר שאמר יהושע עשרה הדברים שבספרי שנרמזו בויען יהושע ולא הרגיש משה ולא השיב לו חשב שאולי הבין משה שאמר דברים האלה דרך תפלה אל הקב״ה לכן שוב הוצרך לרמז למשה שממנו בקש כן והוצרך להזכיר שמו הא לא״ה אסור בפניו כדעת הש״ך ובזה מיושב סתירת הספרי עם הפסוק. כנלענ״ד הקטן יעקב.\\\vspace{0pt}

\end{multicols}\newpage

\newchap{סימן פה}
\begin{multicols}{2}
ב״ה אלטאנא. יום ו׳ כ״ד אלול תרכ״ה לפ״ק.\\\vspace{0pt}

שאלה – מי שהוציא כרוז לאחינו ב״י לפרנס אחיהם הנתונים ברעב וביגון בארץ הקדושה ושלחו נדבות גם א״י מה יעשה בזה.\\\vspace{0pt}

תשובה – לפי המבואר בי״ד סי׳ רנ״ד מלך או שר ששלח ממון לישראל לחלק לצדקה אין מחזירין אותו מפני שלו׳ מלכות אלא נוטלין ומפרנסין עניי א״י בסתר והרמ״א הביא דעת רש״י ותוספ׳ שיעשה בהן מה שצו׳ המושל והטעם מפני שאין לגנוב דעת א״י ולפי המבואר בש״ך גם שיטה זו מודה דאם שלח סתם ינתן לעניי א״י. והנה בזה לא נזכר רק שאין להחזירו מפני שלו׳ מלכות ויש להסתפק אם גם כשמעורר שנאה ובפרט בפרהסיא בעיני הא״י לומר שהיהודים מבזים אותם הדין שאין להחזיר שהלא גם מפרנסין עניי א״י מפני דרכי שלו׳ במעות צדקה כמבואר סי׳ רנ״א כש״כ שיש לחוש בענין זה לדרכי שלו׳ ואפילו נימא ששו׳ נדון זה להך דשבור מלכא עוד יש לספק אם הלכה כי״א שהביא הרמ״א או כפסק הש״ע דהש״ך הביא בשם הב״ח דהלכה כי״א אכן הט״ז הקשה דרש״י ותוספ׳ שהיא שיטת הי״א לא כתבו כן רק לענין פ״ש ששם אי אפשר בלי גניבת דעת אבל בצדקה הרי יכול לזון עניי א״י ואין כאן ג״ד שהם יודעים דישראל מפרנסין גם ענייהם והשתא יש לחקור בנדון השאלה אם יש למצוא תקנה כמו בהך דשבור מלכא דכיצד יעשה אם יפרנס עניי א״י בהם יש משום גניבת דעת דבזה לא שייך שהם ג״כ יודעים שישראל מפרנסין גם ענייהם שהרי פה הי׳ הכרוז בפי׳ בשביל עניי ישראל ואם יפרנס ישראל מהם איכא משום ביבוש קצירה ולכאורה יש לומר דהיא הנותנת כיון דאי אפשר בלי גניבת דעת א״כ גם הנדון הזה דמי למעשה דרב יוסף דפ״ש ומותר ליתן לעניי ישראל אלא דלפ״ז קשה מה הקשה הט״ז על הלבוש שכתב דאם אמר בפי׳ שלא לתנם לעניי א״י יעשה כדבריו דהוא נגד סתימת הפוסקים דלא חשו כאן לג״ד דדלמא בנדון כזה מודו ואולי זה בעצמו קושית הט״ז מדלא הזכירו חילוק זה משמע דס״ל דבצדקה לעולם אסור לישראל ולכן קשה לסמוך בזה על דעת הי״א והנה הט״ז הביא קושית הדרישה דברישא התיר לקבל בצנעה והכא אוסר ותירץ לחלק בין יחיד שמותר לקבל בצנעה ובין גבאי דאפילו בצנעה לא יקבל והט״ז דחה חילוק זה וכתב חילוק אחר דדוקא כשהא״י מכוון ליתן צדקה לישראל דוקא יש משום ביבוש קצירה אבל הך דמותר לישראל לקבל בצנעה איירי שנוטל בתוך עניי א״י ואז אין זכות הא״י גדול שרק מצד טבעו הרחמן נותן לכל הפושט יד ובזה אין זכותו גדול עכ״ד ולפי ב׳ החילוקים אין בזה תקנה כיון שנותן לגבאי וגם יודע שהוא לישראל לבד אלא שלענ״ד גם חילוק של הט״ז דחוק דלפ״ז צ״ל דמותר לקבל בצנעה אינו רק כשמפרנסו בתוך שאר עניי א״י והפוסקים לא חלקו בכך דאפילו מפרנסו לבדו מותר לקבל ממנו וגם להיפך לא חלקו דאפי׳ שולח גם לגבאי א״י לפרנס ענייהם ולגבאי ישראל לפרנס עניי ישראל ג״כ אסור ולכן נלענ״ד לחלק ע״פ סברת הט״ז בענין אחר ע״פ מה שיש לפרש מה דאמרינן ב״ב פ׳ א׳ צדקה תרומם גוי אילו ישראל וחסד לאומים חטאת אילו א״ה ולא מפרש למה אבל י״ל דקרא קדריש דבישראל נקרא צדקה ובלאומים חסד, והיינו דישראל מה שנותן לעני אינו נותן מפני רחמנות לבד אלא שחושב עצמו מחוייב לזה ע״פ הדין דיודע שהקב״ה ברא העניים לזכות את העשירים ושממה שנתן לו הקב״ה יש בו חלק שניתן לעניים ושאינו עושה רק משפט צדק להחזיר לאחר מה שניתן לו בשבילו ולכן אינו נותן מפני שלבו העיר בו החשק לרחם על העניים אלא מפני שמחוייב לזה על פי היושר ולכן נקרא זה אצלו צדקה ומרוממו אבל בלאומים שאינם נותנים ע״פ בחינה זו אלא כשלבו מעיר בו חשק לרחם ולחנן וזה מה שנקרא אצלו חסד בלבד וזה חטאת דבזה כל מחשבתו אינו רק בשביל עצמו להשיב תאות לבו כמו שממלא שאר תאוותיו וגם נגד העני יש חילוק דאם העני יודע שמה שמקבל הוא חלקו שניתן לו מהקב״ה אינו מתבייש בכך אבל כשיודע שרק מחמת נדבת לב הנותן שחושבו לחסד מקבל מתנה ממנו מתבייש וזהו חטאת ובזה יש חילוק ג״כ בין רישא לסיפא דברישא איירי שהא״י מפרנס לעני יחיד מפני שמרחם עליו וגומל עמו חסד ובזה אין זכותו גדול ולכן מותר לקבל ממנו אבל בסיפא דאיירי שמבלי שמתבקש צדקה לעניים שולח הא״י לחלק לעניים בזה אין לתלות ברחמנות שאין כאן מה שמעוררו לכך אלא ודאי שנותן בבחינת צדקה שחושב שחוב לאדם להפריש מממונו לעניים וא״כ זה בגדר צדקה כמו בישראל ובזה זכותו גדול ואסור לקבל מפני ביבוש קצירה וע״פ חילוק זה כיון שבנדון השאלה לא שלחו הא״י נדבותיהם אלא באשר ששמעו גודל היגון ואנחה בזה יש לתלות שמפני רחמנות ובתורת חסד נתנו ולכן מותר לקבל מהם ואע״פ שגם בזה לא מותר רק לטול בצנעה ולא בפרהסיא וכשהא״י מפרסם שמו איכא בפרהסיא י״ל שהחילוק שבין צנעה לפרהסיא אינו רק כששואל ממנו הישראל דבזה יש בפרהסיא משום חילול השם אבל כשנותן מעצמו אין כאן חילול השם. כנלענ״ד הקטן יעקב.\\\vspace{0pt}

\end{multicols}\newpage

\newchap{סימן פו}
\begin{multicols}{2}
ב״ה אלטאנא, יום א׳ ז׳ אב תרכ״ב לפ״ק. להרה״ג וכו׳ מ״ה שמשון הירש אב״ד דק״ק ישורון בפ״פ דמיין יע״א.\\\vspace{0pt}

בדיק לן מר נ״י בשאלה שבאה לפניו באיש אחד בר לבב קרוב לשלושים ונשוי לאשה בת טובים נסתכלו באקראי במרחץ במילתו ונדמה להם שחותם ברית קודש משונה משאר ב״ב דהיינו תחת אשר העטרה שלהם גלוי מפנים ומאחור עטרה שלו גלוי׳ כולה מלפנים כלומר כל בשר ראש הגיד וגם חוט הגבוה הסובב גלוי מלמעלה ולצד פנים דהיינו ממקו׳ שהוא מתחיל לשפע כנגד הקרקע אכן מאחוריו במקום שיורד לצד הגוף הוא מכוסה ואין נראה החריץ שבין בשר שבראש הגיד והגיד, ועתה נכנסה הדאגה בלב הקרובים אם מילתו מתוקנת כהלכה או צריכה תיקון באיזמל.\\\vspace{0pt}

הנה ראיתי כי ידיו רב לו לחקור ולדרוש על כל אופני הדין על נכון, אשר על כן אבוא בקצרה להודיע למעכ״ת נ״י מה שנלענ״ד בנדון השאלה, מר נ״י חקר מה נקרא עטרה לענין מילה ומצא לצדד במשמעות לכאן ולכאן ובודאי ראה מה שכתב הב״י בשם החכם הספרדי כי נמצאו המפרשים והפוסקים מתחלפים בלשונם פעם יורו כן ופעם יורו כן, ואשר על כן החליט שעטרה נקרא כל העטרה ראש הגיד עם חוט הסובב ולכן אין לנו לדרוש רק מה נקרא חוט הסובב אם רק גבול החוט שממנו משפע לפנים לצד הגוף או כל השטח הגבו׳ עד החריץ. וגם בזה מצאתי כמו שאירע לחכם הספרדי חילוף בלשון הפוסקים כמו שראה מעכ״ח נ״י כבר על נכון ואני מוסיף על זה מה שכתב הב״ח וז״ל רש״י ומה היא עטרה שורה גבו׳ המקפת הגיד סביב שממנה הגיד משפע ויורד לצד הגוף עכ״ל הרי שעיקר העטרה הוא החוט הגבו׳ היא השורה מקפת הסובב אשר בין בשר העטרה ובין הגיד עכ״ל הב״ח וממה שביאר דהחוט הוא השורה הסובב ושלזה כוון רש״י במה שכ׳ השורה הגבו׳ מזה משמע שכל השטח הגבו׳ הוא הנקרא שורה ולא חוט הגבול לבד, דידוע דשורה הוא מלשון שור תרגום של חומה שיש לו שטח קצת. וגם מצד הסברא דזה בכלל עטרה דאיך יחשב גיד אחר שהוא גובה וסוף של בשר העטרה ודומה לו. אמנם לענ״ד בלתי אפשר לגלות החוט בקטן בן שמונה שהעטרה קטנה מאוד מבלי שיגלה כל השטח הגבו׳ עד החריץ, ואולי מטעם זה סתמו הפוסקים לכתוב חוט הסובב דבגלוי החוט יש גם גלוי השטח הגבו׳, ועכ״פ רק במיעוטא דמיעוטא יארע שיתגלה החוט לבד, ושאלתי את פי מוהל מומחה ואמר לי שכבר אירע לו לא פעם ולא שתים שאחר שמל ופרע כל העטרה כדינו חזר הבשר על העטרה עד שאפילו חוט המקיף לא נראה בעת קשוי ובפרט יארע זה באבר קטן ושעל כן מזהיר לנשים שישמרו להחזיר אבל מכ״מ אירע ג״כ שצמח הבשר על העטרה עד שהי׳ בלתי אפשר להחזיר עוד, ועכ״ז שמע שאחר איזה שנים כשנתגדל האבר חזר מעצמו לאחור ונתגלה העטרה כדינו. ולענ״ד כזה אירע ג״כ לאיש בנדון השאלה אשר קרוביו ראו באקראי במרחץ במילתו שכל בשר העטרה עם חוט הסובב מגולה אבל לא נראה החריץ שלמעלה שבין בשר העטרה והגיד ואף דמידי ספק לא יצא עכ״ז בכגון זה יש לסמוך על חזקה דרוב מצוים אצל מילה מומחים הם כמו שהעיר החתם סופר י״ד סימן רמ״ח ואם אמנם הגאון זצ״ל שדא בי׳ נרגא מדברי הט״ז סי׳ רס״ד במה שכתב שיש להעמיד מוהל אחר שישגיח שיפרע כראוי שלפעמים יארע טעות כבר העיר מר נ״י לנכון שאין דמיון לזה שבזה נקל לטעות וכמו שכתב הט״ז בעצמו וראיתי מוהלים שטעו בזה אבל שנאמר שמוהל לא פרע החוט כראוי במקום שלא שכיח טעות מהיכי תיתי שנחוש לזה ואדרבה יותר שכיח שחוזר עור הפריעה ומכסה החוט אחר שיפרע כראוי כמו שהזכרתי וכפי הנראה מדברי הגאון זצ״ל גם הוא לא החליט הדבר שכתב ולולא דמסתפינא וכו׳, ואין להקשות מהמעשה שהביא הב״י סי׳ רס״ד שבדק באחד מבניו ומצא שלא הי׳ החוט מגולה וסיים ומזה אחשוב שרבים שהם מהנמהלים צריכים הלקט בלי ספק ואמאי לא נימא שאחרי שהם נמולים ומוהלים שמצויים אצל מילה נחזוק שנמולים כראוי ושב העור וכסה העטרה דיש לומר כיון שאז הדין לא הי׳ פשוט דצריך לגלות גם החוט כנראה שהוצרך החכם הספרדי לעורר לבב הירא את דבר ד׳ על זה, לכן היו אצלו הנמצאים בריעותא כזה בחזקה שנמולו ממוהלים שלא ידעו זה. אבל אצלנו שנקבע גילוי החוט לדין ולחיוב גמור וכל מוהל יודע שאם לא יגלה היקף החוט לא מל כראוי הוי גם זה בכלל רוב מצויים אצל מילה מומחים הם.\\\vspace{0pt}

ועוד נלענ״ד דבכל כי האי גוונא שיש ספק אם מתחילה נמול כראוי וחזר ונתכסה או לא נמול כראוי הכל הולך אחר מה שנראה בעת הקשוי וכמו שכתב הש״ך סימן רס״ד ונקודת הכסף שם והביא ראי׳ מירושלמי ומהרמב״ם בפי׳ המשניות נגד החכם הספרדי שהביא הב״י דס״ל דכל הצריך תיקון מדינא לא משערים בעת הקשוי אלא באבר רך ואם אמנם בשו״ת חתם סופר סימן הנ״ל חולק על הש״ך והסכים עם החכם הספרדי לענ״ד הדין עם הש״ך כי מה שהשיב על ראית הש״ך מפי׳ המשניות וז״ל והעיקר אצלנו בכל זה שיבוקש האדם בשעה שהוא מתקשה אם נראה מהול מניחין אותו כמות שהוא ואם אינו נראה מהול חותכין הבשר סביב עד שיראה מהול כשיתקשה עכ״ל דמדכתב חותכין הבשר סביב ולא כתב חותכין העור משמע דלא קאי רק אבעל בשר שצריך תיקון מפני מראית העין ולא אציצין עכ״ד על זה אני תמה הרי גם ציצין המעכבין נזכרו במשנה בלשון בשר כדתנן אילו הן ציצין המעכבין את המילה בשר החופה את רוב העטרה ועוד איך יכתוב הרמב״ם חותכין העור כיון דקאי ודאי גם אתיקון שמפני מראית העין שאין שם עור אלא בשר, אלא ודאי דקאי בין אציצין בין אבשר המסורבל וכמו שמשמע מלשון בכל זה, גם מה שהשיב הגאון זצ״ל על הראי׳ מהירושלמי דגמרא שלנו חולק על הירושלמי דמוכח כחכם ספרדי מדנקט שמואל ורשב״ג נראה ואינו נראה במתקשה משמע שזה הדין המתחדש גבי תיקון מפני מראית העין ואי ס״ד דגם בעיקר מילה תלי כל השיעורים במתקשה א״כ מ״ש דאשמעינן זה גבי תיקון דרבנן ולא גבי דאורייתא עכ״ל על זה אשיב דלענין עיקר מילה פשיטא דהכל תלוי בנראה בעת הקשוי שאז האבר במילואו וכשצו׳ התורה לגלות העטרה הכוונה על העטרה בדמות שנבראה לא כשהבשר והעור נקמטו בעת רכות האבר אבל בבעל בשר שצריך תיקון מפני מראית העין ודי במקצת העטרה רק שיראה מהול בזה הי׳ הסברה לומר שבעינן דוקא שיראה מהול גם בלא שעת קשוי כיון דברוב העתים האבר כן ואם לא יהי׳ ניכר אז שנמול יהי׳ חשש משום מראית העין לזה קמ״ל שמואל ורשב״ג דגם בזה די בעת שמתקשה, ולכן ראי׳ זו ודאי אינה כדאי לומר שחולק גמרא שלנו על מה שמפורש בירושלמי שגם בלא נימול כראוי בודקים בעת שמתקשה. והנה בנדון השאלה שנראה במרחץ מסתמא שלא בקשוי ולא נתברר כלל אם צריך תיקון אפילו אם בודאי מתחילה לא נימול כראוי דאולי כשיבדק בקישוי יהי׳ נראה מהול כראוי לכל הדיעות ואף שאין לצרף ספק זה לס״ס כיון שאפשר להכיר מכ״מ יש סניף לומר שאפילו נצטרף כל החומרות אכתי איננו ערל ודאי.\\\vspace{0pt}

סוף דבר נלענ״ד שאע״פ שקרוב לודאי שדעת הפוסקים דגם השטח עד החריץ הוא בכלל עטרה קרוב לודאי נגד זה ג״כ שבהתגלות החוט נתגלה עד החריץ אלא שאח״כ שבה עור הפריעה וכסה השטח והחריץ וכיון שרוב מצויים אצל מילה מומחים הם נוכל לסמוך על זה שגם האיש הזה נימול כראוי מתחילה רק שאח״כ נתכסה החריץ ולא יצטרך תיקון ע״י איזמל שמלבד חשש סכנה שיש בזה יהי׳ ג״כ חומרא שמביא לידי קולא שלא ימלט האיש מקשוי אבר בעת שיתרפא ויבא לידי עון הגדול דהז״ל. אמנם באשר נוגע הפסק לאיסור כרת מת טוב ומה נעים אם יצטרף מעכ״ת נ״י עוד דעת מורה אחד עמנו שיצא הפסק מבית דין של ג׳. כנלענ״ד הקטן יעקב.\\\vspace{0pt}

\end{multicols}\newpage

\newchap{סימן פז}
\begin{multicols}{2}
ב״ה אלטאנא, יום ו׳ כ׳ טבת תר״י לפ״ק.\\\vspace{0pt}

נשאלתי – תנוק שהיה חולה ונתרפא שהדין שאין מלין אותו אלא לאחר ז׳ ימים מעת לעת משעה שנתרפא אם נחשבין אותן ז׳ ימים לשעות כגון אם לא נתרפא עד לאחר חצות היום אם אז אין מלין ביום השמיני לרפואתו עד אחר חצות ועוד נשאלתי אם התנוק נחלה ונתרפא בתוך שמנה ימים ללידתו אם גם בזה צריך להמתין ז׳ ימים מעת לעת משעה שנתרפא או אם נאמר כיון דבלא״ה תוך ח׳ לא הי׳ ראוי למילה הוי כשאר תנוק ונמול ביום שמיני ללידתו?\\\vspace{0pt}

תשובה – הנה הא דנתרפא צריך להמתין למולו ז׳ ימים מעל״ע אתי ממה דאמר שמואל שבת (דף קל״ז) חלצתו חמה נותנין לו כל ז׳ ימים להברותו ומסקינן שם דז׳ ימים היינו מעת לעת וראיתי לחקור אם זה מן התורה דהיינו הלכה למשה מסיני שנותנין לו ז׳ ימים להברותו או אם תקנת חכמים הוא ויש נפקותא בזה אם יש לדחות ע״י המתנת ז׳ ימים מילה בזמנו דאם תקנת חכמים היא אפשר שלא תקנו כן רק היכי שנתרפא אחר שכבר עברו ח׳ ימים ללידתו דבלא״ה אין קפידא עוד אם מאחרין מילה שלא בזמנו אבל לענין לדחות מילה בזמנו אפשר שלא תקנו ולכאורה הי׳ נ״ל דדעת הרא״ש שהיא מדרבנן שכתב שם דאף דהאבעי׳ אי בעינן מעת לעת או לא לא נפשטה מכ״מ פסקינן לחומרא משום דספק נפשות להקל ע״ש ואי ס״ד דהא דנותנין לו כל ז׳ הוא מן התורה ל״ל הך טעמא דספק נפשות תיפוק לי׳ משום דספק דאורייתא לחומרא כמו בכל אבעיות דלא נפשטו אם הם באיסורי תורה אלא ודאי משמע דס״ל להרא״ש דאין זה רק תקנת חכמים אכל ק״ל דמגמרא דיבמות (דף ע״א) מוכח דמן התורה הוא דאמה דמקשה שם ל״ל קרא דמילת זכריו מעכב באכילת פסח כיון דכבר כתיב קרא דמעכב בשעת עשיי׳ מתרץ הכא במאי עסקינן כגון שחלצתו חמה וכו׳ ונמהלי׳ מצפרא בעינן מעת לעת ע״ש הרי דפשיטא להגמרא דמדאורייתא הוא מדמוקי קרא להכי וא״כ ק׳ על הרא״ש ונ״ל דבלא״ה י״ל על הראש למה לא ניחא לי׳ בטעם הרי״ף שכתב דאף דהתם בשבת לא נפשטה האבעי׳ מ״מ כיון דביבמות אפשיטא הכי פסקינן אבל י״ל כיון דרק לתירוץ קמא אפשיט ביבמות כן אבל לאידך אמוראי דמתרצי בענין אחר לא מוכח לכן רצה הרא״ש ליתן טעם דשייך ג״כ אליבי׳ דשאר תירוצים וכיון דלהנך תירוצים לא מוכח ג״כ דמדאורייתא הוא לכן הוצרך הרא״ש ליתן טעם דפסקינן לחומרא משום ספק נפשות אכן אפילו יהא דעת הרא״ש דמדרבנן הוא מכ״מ כיון דהרי״ף סמך אתי׳ דסתם גמרא וגם הרמב״ם ה׳ קרבן פסח (פ״ט) הביא תי׳ זה להלכה הכי נקטינן וא״כ מוכח דמה דצריך להשהות ז׳ ימים למעל״ע מן התורה היא. ויש להסתפק בעבר ומל תוך ז׳ ימים אם צריך להטיף דם ברית אי נימא כיון דאמרינן יום הבראותו כיום הולדו א״כ כמו במל תוך ח׳ צריך להטיף ד״ב לפי מה דמסיק הש״ך (סי׳ רס״ב) ה״ה ג״כ בתוך ח׳ מיום הבראותו או אי נימא כיון דעכ״פ לא לגמרי הוי כיום הולדו דהרי בנשלמו ז׳ ימים בשבת וי״ט ודאי לא דוחה המילה שוי״ט כמו במילה בזמנו א״כ גם בזה לא הוי כמילה בזמנו להחשב כמו מל קודם זמן חיובו ובפרט אחר דראית הש״ך היא ממה דמילה דוחה שבת ועדיין צ״ע. אכן זה יוצא ברור מסוגיא דיבמות שהזכרתי דמה דקתני שצריך להמתין ז״פ מעת לעת היינו דחשבינן לשעות דאל״כ לא משכחת כלל מה שתירץ דהי׳ מוטל עליו למול בשעת אכילה ובשעת עשיי׳ לא וכמו שפי׳ רש״י שם דלשון מעת לעת היינו משעה לשעה ואף דקצת הי׳ נראה מלשון רש״י בזבחים (דף כ״ה) דבכלל מעת לעת אין נכלל שיהי׳ משעה לשעה ג״כ דכתב שם ד״ה שעות פוסלות בקדשים דשנה האמורה בקדשים קיימא לן בפ׳ יוצא דופן דאין מונין מתשרי כשאר ראשי שנים אלא מיום שנולד מעת לעת לשנה הבאה וילפינן לה מקרא וכו׳ ואשמעינן הכא דלא תימא מיום ליום הוא דבעינן ולא משעה לשעה אלא אף משעה לשעה עכ״ל הרי אחר שכבר כתב דבעינן מעת לעת ע״פ סוגיא דנדה כתב דאשמועינן הכא דבעינן בקדשים משעה לשעה ג״כ ומשמע דבכלל מעל״ע לא הוי מכ״מ צריך לפרש דברי רש״י דנקט מיד המסקנא דבעינן מעת לעת שהרי מסוגיא דנדה לא מוכח כלל רק דמחשבינן שנתו ולא שנת העולם ולא דבעינן מעל״ע ועוד דשם דלא נזכר עדיין דשנתו הוא שנת העולם בזה הי׳ אפשר לפרש מה דקתני מעת לעת דרצה להשמיענו בלבד דמנינן ע״פ שנת העולם משנה לשנה ולכן שוב צריך להשמיענו דמונין ג״כ לשעות אבל היכא דקתני על מנין הימים מעת לעת בזה ודאי אי אפשר לפרש רק דמונין לשעות וכן נראה מדברי רש״י בעצמו בערכין (דף ל״א) דאמה דאמרינן שם דממה דכתיב ימים גבי בתי ערי חומה ילפינן דבעינן מעת לעת פי׳ רש״י שאם מכרה באחד בניסן בחצי היום אין מונין לו שנה עד שיגיע חצי היום של אחד בניסן הבא עכ״ל הרי דפשיטא לי׳ ג״כ דמה דקתני התם מעל״ע הוא משעה לשעה וא״כ ה״ה הכא לענין המתנת ז׳ ימים וכדמוכח מסוגיא כנ״ל.\\\vspace{0pt}

אמנם גם ספק האחר בנתרפא תוך ח׳ ימים שצריך להמתין ז׳ ימים מעל״ע משעה שנתרפא ג״כ נפשט מסוגיא דיבמות הנ״ל דאמה דמתרץ רב שרביא שם דמשכחת זכריו בשעת אכילה ולא בשעת עשיי׳ בהוציא ראשו חוץ לפרוזדור מקשה ומי חיי פי דאין לו מחיה ומשני דזנתיה אישתא ופריך אישתא דמאן אילימא אישתא דידי׳ אי הכי כל שבעה בעי ומשני דזנתי׳ אישתא דאימי׳ ע״ש והשתא אי ס״ד דהיכי דנתרפא בתוך ח׳ שראוי למול לא צריך להמתין מעל״ע מאי פריך אי הכי כל שבעה בעי הרי התם מיירי שנתרפא ביום השמיני שעדיין ראוי למול היום אלא ע״כ דפשיטא להגמרא דגם אז צריך להמתין וא״כ ה״ה בנתרפא קודם ח׳ דמאי שנא ועוד דאם יש חילוק בין נתרפא ביום ח׳ לנתרפא קודם אכתי הוי מצי לאוקמי׳ דזנתי׳ אישתא דידי׳ עד יום ח׳ וחלצתו חמה בליל ח׳ שמאז עד שנולד בח׳ לאחר שהקריב פסחו ודאי ראוי לחיות בלא מזון אלא ודאי דכל שחלה ונתרפא מאז הוי כשעה שנולד וצריך להמתין מדאורייתא ז׳ ימים מעל״ע בין נתרפא תוך ח׳ בין נתרפא אח״כ וכמו שמשמע ג״כ מסתימת לשון הפוסקים שלא חלקו בכך וכמדומה שהעולם לא נזהרים בזה ומלין ביום השמיני אפילו בהי׳ התינוק חולה תוך ח׳ כשאומר הרופא שנתרפא ולפענ״ד לא בלבד שמגע זה לסכנת נפשות ולאיסור מילה תוך זמנו אלא לפעמים נוגע ג״כ לאיסור חלול שבת כשחל שמיני להיות בשבת דכיון דמדאורייתא צריך להמתין ז׳ ימים משעה שנתרפא ה״ל המילה בשבת יום שמיני ללידתו כמל תוך זמנו כנלענ״ד: הקטן יעקב.\\\vspace{0pt}

\end{multicols}\newpage

\newchap{סימן פח}
\begin{multicols}{2}
ב״ה אלטאנא, יום ו׳ י״א שבט תרט״ז לפ״ק. לחתני הרה״ג וכו׳ מ״ה משלם זלמן הכהן נ״י אב״ד דק״ק מאאסטריכט יע״א.\\\vspace{0pt}

על דבר אשר שאלתני – הנה חדשים מקרוב באו לעשות מדת יהודית מדושתם ובן גרנם להתעולל על מה שאמרו ודברו רז״ל עד אשר נגעו גם בברית קדש להסיר המציצה נגד מאמר רז״ל כל אומנא דלא מייץ וכו׳ ועתה דבר חדש גזרו המחדשים בצרפת לבלתי העשות עוד הפריעה בצפרנים כי אם ע״י כלי אשר המציא רופא בפאריז דהיינו אחר חתוך הערלה יחתוך עור הפריעה בכלי ויחזירה לאחורי׳ אם מותר לעשות כן ואם לא אם צריך להעביר מוהל שעושה כן כמו באומנא דלא מייץ.\\\vspace{0pt}

תשובה – יפה דבר חתני נ״י שאף שלא הוזכר בגמרא בפי׳ לעשות הפריעה בצפרנים מכ״מ כבר כתב כן הרמב״ם והסמ״ג והטור והש״ע ובלא״ה מנהגם של ישראל תורה היא. ואשר שאלת שלכאורה יש ראי׳ שמותר לעשות הפריעה בכלי ממה דאמרינן יבמות (דף ע״א) לא נתנה פריעת מילה לא״א שנאמר בעת ההיא וכו׳ א״כ מאי שוב אלא לאו לפריעה ע״ש הרי דפריעה היתה נעשית ע״י חרבות צורים הנה מלבד אשר השבת בעצמך דבין לגרסא שהביא רש״י שם ובין לגרסא ראשונה ניחא דעכ״פ או משוב או משנית ילפינן ציצין וא״כ צריך חרבות צורים לציצין בלא״ה אין ראי׳ משם דודאי במל את הגדולים דעור הפריעה קשה ואי אפשר לקרוע צריך לחתכה ע״י כלי כידוע במילת גרים וכיון דשם גדולים היו הנמולים הוצרך חרבות צורים גם לפרוע ולכן גם ממה דאמרינן במדרש פ׳ לך לך ובירושלמי פ׳ ר״א דמילה ופ׳ הערל המול ימול חד למילה וחד לפריעה אין ראי׳ שפריעה ג״כ ע״י חתיכה דגם שם איירי במילת גדולים כדכתיב המול ימול יליד ביתך ומקנת כספך הרי דאיירי במילת עבדים גדולים ובלא״ה אין ראי׳ משם דדרשה זו אסמכתא בעלמא היא שהרי מוכיח ביבמות כנ״ל דלא נתנה פריעה לא״א ודרשה דהמול ימול דרשינן שס (דף ע״ב) לציצין המעכבין ועוד דלשון מל לא משמע כריתה אלא הסרת אטימה על איזה אופן שיהי׳ כדכתיב ומלתם את ערלת לבבכם. ומה שהקשה חתני נ״י דמנ״ל להרמב״ם והפוסקים אחריו דפריעה בצפורן דע״כ לא מלשון פרע אתי דמצאנו לשון פרע גם במה שמגלה ע״י חתיכה כדאמרינן חולין (דף צ״ג) אר״י היכי דפרעי טבחי ופי׳ רש״י דמקום חתך הירך כשנחתכת ונפרשת מן האלי׳ על זה אשיב אמת דלא מלשון פריעה אתי אמנם למה נצריך ראי׳ על אופן קיום המצוה שלא פסקה מימות משה והלא כל אופני קיום המצות לא נדע רק מדרך קבלה למשל מאין נדע דאתרוג הוא הפרי עץ הדר הכתוב בתורה דאע״פ דדרשינן סוכה (דף ל״ה) מלשון הדר ע״פ ג׳ דרשות שם זה רק למעט דאינו פלפלין אף שגם כן טעם עצו ופריו שו׳ אבל שאינו לימוני ופמרנצין ואפפעלזינן לא נדע מהדר דכל הסימנים שדר משנה לשנה ושגדל ע״כ מים ושדומה לדיר גדולים וקטנים שיש באתרוג יש ג״כ בפירות הללו וע״כ לא נדע המצוה רק ע״ד קבלה איש מפי איש עד משה רבינו וכמו שכתב הרמב״ם דהל״מ הוא וכמו כן באופני קיום שאר המצות וכיון דאופן קיום הפריעה ע״י צפורן הי׳ כן מפורסם ומקובל בכל ישראל ודאי שכן קבלנו ממ״ר וכמו שכתבתי (סימן כ״ג) לענין מציצה ולא נצריך ראי׳ לזה אמנם אך ליתר אביא ראי׳ לזה שעשיית הפריעה ע״י צפורן תהי׳ שאע״פ שבגמרא לא מצאנו אבל מצאנו במדרש הובא בילקוט תהלים (סי׳ תשכ״ג) על פסוק כל עצמותי תאמרנה אמר דוד אני משבחך בכל אברי ומקיים בהם המצות וכו׳ צפרנים לעשות בהם פריעה ומליקת העוף ובהן להסתכל אור להבדלה עכ״ל (ומדרש זה הביאו הגהת מיימוני ללמוד ממנו שמצו׳ לתפוס ציצית בשעת ק״ש כמובא בב״י א״ח [סי׳ כ״ד] וגם הביאו הג״מ ללמוד ממנו מצות סנדקאות כמובא בדרכי משה י״ד [סי׳ רס״ה] ע״ש) הרי שהשו׳ עשיית פריעה בצפורן למליקת העוף דצפורן מעכב בה דמלק בסכין מליקתו פסולה כדתנן זבחים (פ׳ ז׳) ולהסתכלות צפרנים בשעת הבדלה שכבר מוזכר בפרקי ר״א (פ׳ כ׳). ולכן המשנים עשיית הפריעה ע״י צפורן לא בלבד שעכ״פ משנים מנהג ישראל אלא גם תקנה שתקנו רז״ל מימים קדמונים ושנתפשטה בכל ישראל ואפשר שמשנים קבלת משה רבינו כנ״ל ועליהם יאמר הפורעים העם לשמצה להם שומר נפשו ירחק מהם. ועל דבר אם יש לסלק המוהל המחזיק בדברי רפואי אליל נגד דברי רז״ל ודאי אם נמצא מוהל אחר יסלקו אבל באם לא נמצא אין ללמוד חיוב סילוקו ממה שאמרו באומנא דלא מייץ דשם נותן הטעם משום סכנתא ומשום זה ודאי יש לבטל מצות מילה בשמיני אבל משום תקנה זו אין בידינו לבטל רק יש לגעור בו ולהשומע יונעם כנלענ״ד הקטן יעקב.\\\vspace{0pt}

\end{multicols}\newpage

\newchap{סימן פט}
\begin{multicols}{2}
ב״ה אלטאנא, שבט תר״ה לפ״ק.\\\vspace{0pt}

בדין אם יש למול ע״י ב׳ מוהלים בשבת ידוע שיש בזה פלוגתא בין הפוסקים הרמ״א מתיר שמביא ראי׳ מקרבנות בשבת שזה שוחט וזה זורק אף שהשחיטה לחוד אינה כלום וא״כ ה״ה כשזה מל וזה פורע אכן המהרשל ביש״ש ביבמות חולק על זה וס״ל דמוהל שאינו פורע בעצמו בשבת חייב כרת. וראיתי בשו״ת מהרב הגאון מ״ה אוה״ב נ״י שהביא ראי׳ לשיטת המהרשל וסיעתו ממה דכתב רש״י ביבמות (דף ל״ג ע״ב) אמה דמקשה שחיטה בזר כשרה דאין כאן זרות דמשמע הא משום שבת מחייב אע״ג דשחיטה בזר כשרה. וכתב עוד ראי׳ לשיטה זו ממה שכתב רש״י ביומא (דף כ״ד) ד״ה עבודה תמה ולא שיש אחרי׳ עבודה כגון שחיטה והקשה הרי שחיטה לאו עבודה היא ואי בפרו של כה״ג הא מרא דשמעתתא שם הוא רב והוא ס״ל ביומא (דף מ״ב) בפרו של כה״ג שכשר בזר אלא ודאי דרש״י כוון בשוחט זר בשבת דאז אסור לשחוט אפילו תמידין ומוספין ואפילו הכי א״ח מיתה משום דל״ה עבודה תמה וכחב עוד שלזה נתכוון גם הראש בתשובה (כלל פ״ב) שכתב ואם הולד זכר מקריב ע״ג המזבח ושוחט ומבעיר בשבת והשיג עליו הח״צ בשו״ת (סי׳ מ״ד) שהרי שחיטה בזר כשרה אבל לפ״ז א״ש דדעת הראש כדעת רש״י דחייב משום שבת אע״ג דשחיטה בזר כשרה כיון דלא גמר המצו׳ והוא לא יכול לגמור דלא יכול לזרוק וא״כ ה״ה בשני מוהלים כל שהראשון לא גמר המצו׳ ע״ש שהאריך. ולענ״ד יש להשיב על זה ובתחלה אזכיר שהרב נ״י הביא ראי׳ לסתור שאם כוונת רש״י ביומא כיון דזר אינו מותר לשחוט בשבת ה״ל איסור זרות רק שאין חייב מיתה כיון דה״ל עבודה שאינה תמה א״כ מאי פריך ביבמות שחיטה בזר כשרה הא בשבת ה״ל בזר איסור זרות ואף דהוי עבודה שאינה תמה מה בכך הא מכ״מ שייך חיוב זרות דחיוב לאו מיהו הוי שהרי גם כשמתרץ בשחיטת פרו של כה״ג ג״כ אין חיוב זרות רק משום עבודה שאינה תמה וכן הוי אתי לי׳ שפיר לאוקמי׳ בקבלה והולכה אי לאו דטלטול בעלמא הוא אף דהוי עבודות שאינן תמות ואע״ג דמזה אין ראי׳ כ״כ דאכתי הוי מצי למימר אי בזריקה דהוי עבודה תמה וצריך להשיב גם בזה טלטול בעלמא הוא מכ״מ לפי מה דנקט אי בקבלה והולכה לא הוי עבודה תמה – ועוד תמו׳ לי אי כוונת רש״י ביומא בזר ששוחט בשבת וקמ״ל דאין חייב מיתה משום זרות כיון דלא הוי עבודה תמה דממנ״פ היכי איירי אי לא ידע ששבת הוא או שזר אסור לשחוט בשבת מאי קמ״ל דאין חייב מיתה פשיטא הא שוגג גמור הוא שלפי דעתו מותר לו לשחוט היום כמו בשאר הימים ואי איירי שידע ששבת הוא ושאסור לו לשחוט בשבת ואעפ״כ שחט א״כ מאי קמ״ל דאין חייב מיתה משום זרות הא חייב כרת משום שבת וכעין מה דמקשה בשבועות (דף י״ז ע״א) אי דשהה בר כרת הוא והכא לא שייך אפילו מה שהקשה הר״ת על קושית הגמרא שם שהרי על אזהרת שבת אין מלקות שניתן לאזהרת מיתת ב״ד וגם מה שהקשה דלמא איכא נפקותא בשגג באחד והזיד באחד דהכא ממנ״פ אם בשבת שגג הוא שוגג ג״כ לענין זרות ולכן נלענ״ד ברור שאין כוונת רש״י על שחיטת זר בשבת אלא אשחיטת פר ונקט כן ע״פ סוגיא דיבמות דקחשיב הני לענין זרות ואף דלרב לשיטתו לא שייך זה מכ״מ נקט כן משום לוי וסתם גמרא דס״ל שם ג״כ דעבודה תמה בעינן וקמ״ל דשחיטה אף היכי שפסולה בזר כגון בפר של כה״ג מכ״מ אינו במיתה דאינה עבודה תמה וכל שכן שיש לפרש דברי הרא״ש כן במה שכתב ששוחט ומבעיר בשבת שכתב כן ע״פ סוגיא דיבמות אשחיטת פר ולא ידעתי מה דחקו להרב ח״צ להגי׳ בדבריו ומדברי רש״י ביבמות אין ראי׳ דס״ל דאיכא משום שבת בזר ששוחט בשבת דכבר כתבתי בחדושי די״ל דמה דכתב ואין כאן זרות דמשמע הא שבת איכא דה״ק דאפילו נדחוק דהך זר ששמש בשבת בברייתא איירי ששחט קרבן יחיד בשבת דשפיר הוי משום שבת מכ״מ זרות ליכא אבל לעולם כ״ע ס״ל דק״צ מותר לשחוט גם ע״י זר בשבת ולענ״ד יש ראי׳ אולימתא לזה ממה דאמרינן בפסחים (דף ס״ד) שחט ישראל וקבל הכהן דמפרשינן בגמרא שם דקמ״ל שחיטה בזר כשרה ועלה קתני התם במתניתן כמעשהו בחול כך מעשהו בשבת ואי ס״ד דבשבת שוחט הכהן דוקא היאך קתני כך מעשהו בשבת אחר שיש שינוי רב לענין שחיטה דבשבת לא שחט ישראל אלא כהן ועוד אי ס״ד דשחיטה בשבת אינה בזר כיון דשחיטה משום זריקה והוא לא יכול לזרוק א״כ ההפשט בשבת שלא הותר רק משום הקטרת אימורין כמש״כ התוספ׳ (ר״פ אילו דברים) ג״כ לא יהי׳ מותר בזר בשבת כיון שהוא לא יכול להקטיר והרי שם (דף ס״ה) פריך אטו הוא גופא מקטיר להו ופי׳ רש״י דכהנים לא היו מפשיטים וכו׳ ע״ש הרי דפשיטא להגמרא דזרים היו מפשיטים גם בשבת ומ״ש הפשט משחיטה דתרווייהו משום עבודה אחרת שאסורה בזר אע״כ דליכא למ״ד דיש איסור זרות בשחיטה בשבת ולכן שפיר הביא הרמ״א ראי׳ ממה שזה שוחט וזה זורק שמותר למול ע״י ב׳ מוהלים בשבת. כנלענ״ד הקטן יעקב.\\\vspace{0pt}

\end{multicols}\newpage

\newchap{סימן צ}
\begin{multicols}{2}
ב״ה אלטאנא, יום ו׳ ה׳ תשרי תרי״ד לפ״ק. להרה״ג וכו׳ מ״ה גבריאל אדלער הכהן נ״י הגאב״ד דק״ק אבערדארף יע״א.\\\vspace{0pt}

הקשה מעכ״ת נ״י על מה שכתב הטורי זהב בא״ח סי׳ קכ״ח שפסק שם הרמ״א אסור להשתמש בכהן אפילו בזמן הזה דהוי כמועל בהקדש אם לא מחל על כך ומקור הדין הוא מהגהת מרדכי שכתב מעשה בכהן שיצק מים ע״י ר״ת הקשה תלמיד א׳ הא שנינו בירושלמי המשתמש בכהונה ה״ז מעל והשיב ר״ת אין קדושה בזמן הזה דקיי״ל בגדיהם עליהם קדושה עליהם ואי לא לא והקשה אם כן כל מיני קדושה לא ליעבד להו ושתק ר״ת והשיב הר׳ פטר דנהי שיש בו קדושה יכול למחול כדאמר פ״ק דקידושין אין ע״ע כהן נרצע מפני שנעשה בעל מום משמע דבלאו הכי יכול להשתעבד בו עכ״ל וע״ז הקשה הט״ז הא קיי״ל וקדשתו בעל כרחו דאם נשא גרושה כופיו אותו שיגרשה כדאיתא בהאשה רבה אם לא רצה דפנו פירוש שמיסרין אותו ביסורין ואמאי לא נימא שהוא מוחל על קדושתו ותירץ דהטעם דיכול למחול הוא משום דיש לו הנאה מזה ולא אסרה התורה עליו מידי דהוא ניחא לי׳ בכך אם לא במה שאסרה התורה בפירוש עליו ולפי״ז כל זמן שאין לכהן הנאה ממה שעושה מלאכה לאחרים נראה שאסור להשתמש בו והוי כמועל בהקדש אע״ג דמחיל עכ״ד הט״ז ועל זה הקשה מעכ״ת נ״י דזה עדיין אינו מעלה ארוכה לתרץ קושית הגהת מיימוני על הרמב״ם שפסק (בפ״ג מהל׳ עבדים) דכהן נמכר לע״ע אך לא נרצע מפני שנעשה בעל מום הא איתא בירושלמי המשתמש בכהונה מעל ואיך נמכר לע״ע ע״ש שהניח בצ״ע ומר נ״י הרחיב הקושיא שהרי דין נרצע הוא דווקא במכרוהו ב״ד דאי במוכר עצמו קיי״ל דאינו נרצע ואיך שייך במכרוהו ב״ד מחילה או נהנה הא הב״ד מכרוהו בגניבתו בע״כ ע״כ דברי הר״פ והט״ז נפלאים מאוד וצ״ע עכ״ד מר נ״י.\\\vspace{0pt}

על זה אשיב: במכירת ע״ע ודאי מטי להנמכר הנאה אפי׳ במכרוהו ב״ד שהרי כשאינו נמכר נשאר עכ״פ הקרן של גניבה חוב עליו כשיגיע לו מה לשלם כמו בכל האופנים שהדין שאינו נמכר כמבואר ברמב״ם (הל׳ גניבה פ״ג) וא״כ ע״י המכירה נפטר מהחוב וזה ודאי נחשב הנאה ואע״ג דאמרינן בנדרים (דף ל״ג) המודר הנאה מחבירו פורע את חובו וא״כ משמע דפרעון חוב לא מקרי הנאה ז״א שהרי רבא מפרש דזה דווקא כשהתנה עם המלוה שלא יכול לתובעו וגם לרב הושעי׳ דמוקי כחנן ג״כ הטעם דלא נחשב הנאה כיון דאינו יכול לתובעו וכן מבואר בש״ך (יו״ד סי׳ רכ״א) שעל מה שפסק בש״ע שם המודר הנאה מחבירו יכול לפרוע חובו כתב הש״ך היינו דווקא שלא מדעתו כן כתב הר״ן וכן פשוט בפוסקים שבסי׳ זה והב״ח פסק כרבינו חננאל ור״ת והרא״ש וטור דאסור לפרוע חובו כשהוא בשטר או משכון כיון שהוא חוב ברור ע״ש ולפי״ז חוב גניבה שחייבוהו ב״ד לשלם הוא ודאי חוב גמור של הגנב והנגנב יכול לתובעו בכל עת ממנו ולכן נחשב הנאה מה שנפטר מהחוב אעפ״י שנמכר בע״כ ולכן ליכא מעילה זה שיטת הר״פ ומתורצת קושית הג״מ ולכן ראי׳ מזה לפסק הט״ז שכל שמגיע להכהן הנאה ע״י השימוש מותר להשתמש בו: כנלענ״ד הקטן יעקב.\\\vspace{0pt}

\end{multicols}\newpage

\newchap{סימן צא}
\begin{multicols}{2}
ב״ה אלטאנא, יום ו׳ כ״ו אייר תר״ט לפ״ק. להרה״ג וכו׳ מ״ה אשר לעמיל נ״י הגאב״ד דק״ק גאלין וכעת משכן כבודו בירושלם עיה״ק תוב״ב.\\\vspace{0pt}

כתב מעכ״ת נ״י וז״ל – ילמדנו רבינו בעובדא דאתא לידן פעה״ק ירושלם ת״ו יום ג׳ כ״ג לירח אדר שני שנת תר״ח העבר לפ״ק נימול א״י אחד שבא הנה ממדינת מאראקא לשם גירות בפנינו בד״צ דקהל אשכנזים הי״ו וקיבל עליו המצות כדין וכדתה״ק ובש״ק שלאחריו עדן לא היה נתרפא ממילתו ולא טבל עודנה הגידו לי לאמר מזריזתו במצות איך הוא נזהר בשביתת שבת הגם שהוא עודנה בכלל חולה שאב״ס אינו מניח לגוי להבעיר אש בביתו והשבתי להם לדעתי לא מבעי׳ שמותר לו לעשות מלאכה בשבת אלא אפילו מחויב ומוזהר על יום ולילה לא ישבותו וחייב לעשות מלאכה בשבת כ״ז שלא טבל לשם גירות וכה עשו השומעים למשמעתי והלכו אצל הגר והגידו לו בשמי כן בש״ק לאחר תפלת המנחה וכן עשה כי כתב איזה אותיות ויהי ביום המחרת כאשר נשמע הדבר בעה״ק ת״ו פה צווחו עלי חכמי ספרד וחכמי אשכנזים הי״ו על דבר חדש הלזו אשר לא נשמע מעולם אחרי שכבר קבל עליו כל המצות בשעת מילה וכבר נימול ועומד ומצפה בכל יום לטבול לכשיתרפא שיהי׳ מותר לו לחלל וכש״כ שיהי׳ עליו חיובא ומצוה לחלל ש״ק והמה זוכרים כמה גרים שנימולו פעה״ק ת״ו ולא נשמע כזאת ומנין לי לחדש דבר אשר לא שערום הראשונים והשבתי להם אולי מקום הניחו לי להתגדר בו. ואמינא טעמא דידי האמנם מהראוי היה להתייעץ בזה עם חכמי ורבני עיה״ק פה הי״ו טרם נעשה המעשה אמנם מחמת כי כבר הי׳ אחר תפלת המנחה לעת ערב ובין כך יצא ש״ק ובעיני הי׳ הדבר פשוט שאין כאן איסור כלל ומכש״כ דלית בי׳ דררא אי׳ דאורייתא ובדרבנן עבדינן עובדא כו׳ וכש״כ שלענ״ד אין כאן איסור לא דאורייתא לא דרבנן כ״א מצוה בחילולו ש״ק אחרי שמעכת״ה אמרתם מכח סברא אמנם לדעתי אינו כן אב״ע סברא אב״ע גמרא איבע״א סברא כיון דקיי״ל כחכמים וכר׳ יוחנן יבמות (דף מ״ו) דאינו גר עד שימול ויטבול וכמ״ש הרמב״ם פי״ד מה״ל איסורי ביאה ובטוש״ע יו״ד (סי׳ רס״ח) וכיון דקיי״ל כרשב״ל סנהדרין דף נ״ח ע״ב גוי ששבת ח״מ דכתיב יום ולילה לא ישבותו ולרבינא אפי׳ בשני בשבת וכ״פ הרמב״ם ז״ל בפ׳ עשירי מהל׳ מלכים דין ט׳ אלא דפשטא דהש״ס משמע מיתה בידי אדם דאזהרתן זו היא מיתתן ודעת הרמב״ם דוקא בז״מ ב״נ ב״ד ממיתין עליהן כשידינו תקיפא אבל בגוי ששבת ב״ד מכין ועונשין אבל אין ממיתין שאינה בכלל שבע מצות ב״נ יעו״ש בכ״מ והרי כ״ז שלא טבל אינו גר ועדיין הוא ב״נ כאשר מבואר א״כ במאי נפקע ממנו מצות יום ולילה ל״י שנצטוה עליו והאיך יוכל לפטור א״ע ממצוה שנצטוה בו במה שיהי׳ גר ויהי׳ לו דין ישראל לאחר שיטבול הלא לדעת הרמב״ם ז״ל שם בהלכה י״ג מפרכסת מותר לישראל ואסור לב״נ משום אמ״ה דבב״נ במיתה תלייא רחמנא ולא בשחיטה (דלא כהרשב״א יעו״ש בכ״מ) היעלה על הדעת שיהי׳ מותר לו להקל על עצמו לאכול מפרכסת קודם הטבילה כיון דעדיין ב״נ הוא ולא גר וא״כ כן מה לי קולא דמפרכסת או קולא זו להתירו לעבור על מה שנצטוה יום ולילה ל״י וא״כ אדרבא לרומעכ״ת שרוצים להקל לעבור על מה שנצטוה להביא ראי׳ ועוד האריך מעכ״ת נ״י בהראותו רב חריפותו ובקיאתו לחזק דבריו האלה.\\\vspace{0pt}

תשובה – פסק מעכ״ת נ״י שגר שמל ולא טבל אסור לשמור שבת מפני שעדיין אינו גר ולא יצא מכלל בן נח אשר לא חשו לו חכמי ירושלם נ״י – חקרתי בשאר מקומות שמקבלים גרים ונאמר לי שגם שם מעולם לא הקפידו על זה שלא ישמור הגר שבת קודם הטבילה. ונתתי אל לבי למצוא טעם לזה אחרי שלכאורה פסק מעכ״ת נ״י מוסד על אדני הדין ואמת אבל א״ע ראיתי שהדין עם המנהג דכבר מצד הסברא יהי׳ מתנגד אל השכל אחרי שמילת הגר נקרא ברית שמברכים עלי׳ כורת הברית כדאמרינן שבת (דף קל״ז) וגם שבת נקרא ברית כדאמרינן שם (דף קל״ב) איך נאמר אחר שנכנס לברית האחת יהי׳ מוכרח להפר ברית האחרת שכרת הקב״ה עם ישראל מקיימי מצותיו ולכן נלענ״ד דאף שעדיין לא נכנס לכלל ישראל גמור עד שטבל מכ״מ משעה שנכנס לברית מילה כבר נבדל מכלל ב״נ וכעין זה כתבו התוספ׳ בכריתות (דף ט׳) אמה דאמרינן שם דאבותינו נכנסו לברית במילה וטבילה והרצאת דמים ויליף מילה ממה דכתיב כי מולים היו כל העם היוצאים וכתבו התוספ׳ ואע״פ שאותן שהיו נמולים בימי אברהם לא מלו אותם ביציאת מצרים מכ״מ מעיקרא כשמלו עצמן מלו ליכנס בברית המקום וליבדל משאר אומות וגם כי עתה טבלו עכ״ל הרי בפי׳ שכבר קודם טבילה ע״י מילה לבד נכנסו לברית ועי״ז נבדלו משאר האומות וא״כ גם זר זה שמל ולא טבל דמי לזה שנכנס לברית ועי״ז נבדל משאר האומות וע״כ אין עליו עוד מצות יום ולילה ל״י של ב״נ ולכן לא בלבד שמותר לגר כזה לקיים שבת אלא אפשר לצדד ג״כ שחובה עליו לקיים ע״פ מה שכתבתי בספרי ע״ל ביבמות (דף מ״ו) בתוספ׳ ד״ה כי פליגי לתרץ קושית התוספ׳ שם שהקשו לר״ע ל״ל תושב ושכיר למעט גר שמל ולא טבל מפסח תיפוק לי׳ דאינו גר עד שימול ויטבול ותירצתי דשפיר צריך קרא כיון דפסח אכלו במצרים לאחר שמלו וטבילה לא הי׳ עד מ״ת א״כ ה״א דגם לדורות יאכל גר שמל ולא טבל מפסח לכן צריך קרא למעט. והנה בשבת (דף פ״ז) אמרינן דעל שבת נצטוו ישראל במרה וכן מוכח מהכתובים שכבר קיימו ישראל שבת קודם שבאו להר סיני שהרי הספור של המן שעליו נאמר עד אנה מאנתם הי׳ קודם סיני כמבואר (שם) וכיון דטבילה לא הי׳ עד סיני ע״כ קיימו ישראל שבת כשמלו ולא טבלו אף שב״נ מוזהר על יום ולילה ל״י (וכבר העיר על זה בס׳ פ״ד פ׳ בשלח ע״ש) וע״כ צ״ל או שגזיה״כ הי׳ שלענין שבת יצאו מכלל ב״נ ונילף משם כמו דהוי גמרינן גם לענין פסח אי ליכא מיעוט או כאשר כתבנו שע״י שנכנסו לברית מילה נכנסו ג״כ לברית שבת ועכ״פ איכא למילף מישראל קודם מ״ת שגר שמל ולא טבל מותר לקיים שבת או אם נאמר כאופן השני ששתי הבריתות כאחת נחשבו חייב לקיים שבת. ולכן לענ״ד יפה נהגו שלא לכוף לגר שמל ול״ט לעשות מלאכה בשבת. ואם צדקתי במה שכתבתי במקום אחר שציווי שביתת ישראל ואזהרת שביתת ב״נ אינם מענין א׳ שבזה תלוי בל״ט מלאכות ובזה תלוי במלאכת טורח ויגיעה מצאנו אפילו למי שלבו נוקפו לומר שגר שמל ול״ט מותר לקיים שבת פשר דבר על ידי שיעשה מלאכת יגיעה שאינה מל״ט מלאכות כגון שישא משא ברשות היחיד כנלענ״ד הקטן יעקב.\\\vspace{0pt}

\end{multicols}\newpage

\newchap{סימן צב}
\begin{multicols}{2}
ב״ה אלטאנא, בחדש סיון תרי״א לפ״ק. להרבנים המופלגים הצדיקים וכו׳ מ״ה מרדכי מיכאל יפו ומ״ה געטשליק שלעזינגער נ״י יושבי ביהמ״ד דקלויס בק״ק האמבורג יע״א.\\\vspace{0pt}

שאלתם ממני לחוות דעתי בס״ת אשר בשם הוי׳ ב״ה ההא אחרונה נדבקה בגגה אל הויו מה יעשה בתקונה ואם יש בו משום חק תוכות.\\\vspace{0pt}

תשובה – הנה לא נעלם ממעלתכם מה שנפסק בש״ע י״ד סי׳ רע״ו ס׳ י״א נדבקה אות לחברתה באותיות השם יש לו לגררו אלא נראה מדבריכם שאתם מסופקים דשמא זה דוקא בנדבק מעט אבל בנדבק כל הגג של האותיות כשרוצה להכשיר ע״י גרירת הדיו שבנתים ה״ל חק תוכות ופסול ובזה נלענ״ד דהחשש דחק תוכות הוא בשני ענינים האחד כשנדבק עד שנדמה האות לאות אחר כגון מה שהביא הטור באהע״ז (סי׳ קכ״ה) בשם ספר התרומה אם רצה לעשות מם פתוחה ונסתמה שאין תקנה לגרור הסתימה לבד משום חק תוכות וכן הובא בפוסקים בא״ח סי׳ ל״ב ג״כ וכגון נדבק רגל הא בגגה דנראה כחית ובאמת במהר״ם פאדו׳ (סי׳ פ׳) לא רצה לחשוב לחק תוכות אם יגרר הנגיעה לבד עד ששמע קבלה מסופר מומחה דגם בכה״ג צריך לגרור כל הרגל משום חק תוכות והשני אם נפסד צורת האות לגמרי כגון בנפל טיפת דיו על האות המבואר בא״ח (סי׳ ל״ב ס׳ י״ז) או בכתב ב׳ במקום כ׳ כמבואר שם אבל בנגיעה בעלמא כל שניכר צורות האותיות לא מקרי חק תוכות כמבואר שם ס׳ י״ח ולכן לא נלענ״ד שיש חשש חק תוכות בזה כשמפריד הו׳ מההא ואין לדמות זה למה שכתב המג״א שם ס״ק כ״ז בשם הב״ח דאם כל אורך האות דבוקה לחברתה לא מהני גרירה משום ח״ת דזה דוקא אם הדבוק מלמעלה למטה דהב״ח כתב למשל כגון א׳ ב׳ שדבוקים זה לזה ובזה הטעם דנפסד צורת האותיות לגמרי אבל הכא הרי לא נדבק אורך האותיות רק גג האותיות ולא הוי רק נגיעה בעלמא ואפילו בנדון הב״ח חלק עליו בקרבן נתנאל פ״ב דגטין והכשיר לגררו דליכא בזה משום חק תוכות ולכן לענ״ד אם תנוק דלא חכים ולא טפש מכיר צורות האותיות דהם ו׳ ה׳ מותר להפריד מעט הדבוק עד שיהיו האותיות מוקפים גויל אכן פשיטא שמשנפרד מעט אין ליגע עוד באותיות להרחיב הריוח דכיון שנפרדו וחזרו להכשרם יש בזה משום מחיקת השם אבל כל שנדבקים להפריך אין בזה חשש מחיקה כמבואר במה שמתיר ברמ״א י״ד סי׳ רע״ו בנדבק רגל ההא בגגה לגרר כל הרגל על מנת לתקן השם וכן נפסק בפשיטות במהר״ם פאדו׳ סי׳ הנ״ל וגם בשו״ת נודע ביהודה מ״ק חלק י״ד סי׳ ע״ח פסק למעשה כן וכן הביא בשם שו״ת מ״ע אכן אם התנוק לא מכיר צורת האותיות צריך להתיישב בדבר אם יש לסמוך על התנוק להקל ולהתיר מחיקת אותיות השם דשמא יש תנוקות שמכירים צורות האותיות וסגי בפירוד לבד ועוד חזון למועד כנלענ״ד הקטן יעקב.\\\vspace{0pt}

\end{multicols}\newpage

\newchap{סימן צג}
\begin{multicols}{2}
ב״ה אלטאנא, אלול שנת תרי״ב לפ״ק. להרה״ג וכו׳ מ״ה גבריאל אדלער הכהן נ״י הגאב״ד דק״ק אבערדארף יע״א.\\\vspace{0pt}

ע״ד שאלת מעכ״ת נ״י בדבר הרע שנעשה מחזן א׳ שלמען הקל לו הקריאה בספר תורה מחק בכמה סדרות בתוך האותיות עד שנראה בהן כמין נגינות דהיינו כגון באות ה׳ חקק מעט שנראה הקלף בה כמו זרקא וכדומה וכן עשה בעו״ה גם בשמות הקדושים אם יש תקון לדבר הזה?\\\vspace{0pt}

תשובה: הנה כבר כתב מעכ״ת נ״י שהספרים צריכים תקון אף שממש אין העין שולט במחיקות האלה ובזה ודאי הדין עמו אכן מה שמסופק אם מהני תקון אחר שהש״ע פסק בי״ד סי׳ רע״ד בשם רבינו ירוחם דספר המנוקד פסול ואפילו הסירו הנקוד והקשה מעכ״ת מ״ט דרבינו ירוחם ורצה לומר מטעם דנראה ונדחה אינו חוזר ונראה ושכן מצא גם בספר מלאכת הקדש ואף שדחה שם טעם זה דא״כ ספר שנפסל במחיקת אות וכדומה ג״כ ישאר בפסולו לעולם זה יישב מעכ״ת נ״י שיש חילוק בין דיחוי בידים לדיחוי דממילא, ועפ״ז רצה לצדד דגם בנדון השאלה לא מהני תקון. הנה ספר מלאכת הקדש כעת אין בידי אכן לא ידעתי למה העלים מעכ״ת עין ממה שכתב הט״ז (שם ס״ק ו׳) שכבר הקשה קושיא זו למה לא מהני תקון להעביר הנקוד כמו בשאר פסולים ותירץ ונ״ל דכאן גרע טפי כיון שנראת כוונת הסופר שאינו חושש למסורת כלל ובודאי לשם זה כתבו והוי כמזוייף מתוכו ופסול לגמרי משא״כ במקום שכתב בשוגג איזה טעות כן נ״ל עכ״ל הרי שכבר נתן הט״ז טעם למה לא מהני העברת הנקודות והט״ז לא ראה שכבר כתב טעם זה הלבוש ג״כ דלא מהני העברת הנקוד כיון דכתב הסופר לשם נקוד ולא לשם המסורה יע״ש. אכן לפי טעם זה נראה בפשיטות דדוקא אם הסופר שכתב הספר עשה הנקודות בזה לא מהני תקון אבל אם לאחר שנכתב הספר בהכשר עשה אחר במזיד הנקודות ואפילו בכוונה לפסול הספר עכ״ז מהני להעביר הנקודות ולהכשירו כיון שלא שייך בזה טעם הלבוש והט״ז שהסופר כתבו בזיוף מתוכו וא״ע מצאתי שכן כתב ג״כ הנב״י מ״ת ושמחתי שכוונתי לדעתו וא״כ גם בנדון השאלה שכבר היו הספרים נכתבים בהכשר קודם שעשה החזן העבריין סימני הנגינות אין חשש מטעם זה להתיר לתקנו ובלא״ה לענ״ד אין נדון השאלה דומה להא דספר המנוקד דשם נעשו הנקודות נוסף על הכתב ונשתנה הכתב מן המסורה באופן שניכר לכל אבל כאן הרי לא נוסף רק נמחק מן הכתב וכפי אשר כתב מעכ״ת תינוק דלא חכים ולא טפש מכיר צורות האותיות א״כ הרי לא נשתנה הכתב מן המסורה ואע״פ שאין ספק שלכתחלה צריך לתקנו מכ״מ אין חשש להכשירו אחר תקון.\\\vspace{0pt}

אמנם ראיתי בשו״ת דבר שמואל סי׳ פ״א שהביא בשם בעל לחם חמודות טעם אחר שפסול אפילו מחק הנקודות משום דמחזי כמנומר. ולפי טעם זה ודאי יש לפסול אפילו לא עשה הסופר הנקודות בשעת כתיבה. אכן מלבד שטעם זה דחוק הוא לפענ״ד דמה נימור יש כשמחק הכל ולא נראה רק הקלף סביב האותיות אלא גם לפי טעם זה לא יפסל אלא אם התקון הוא למחוק הנקודות אבל לא בנדון זה דאדרבה התקון הוא להעביר הקולמוס על המחיקות והרי מעשה בכל יום שאם נתיישן הכתב באיזה מקומות שמעבירין קולמוס על האותיות שנמחקו קצת ואין בזה משום מחזי כמנומר אבל הכל כפי הענין שאם למשל נתיישן הכתב הרבה וע״י העברת הקולמוס וחדוש הדיו יהי׳ כמנומר לפענ״ד יש לפסול אפילו אין זה בכל הספר דמה דדייק מעכ״ת ממה דקאמרינן בגטין (דף נ״ד) אבל בכולה ס״ת לא משום דמחזי כמנומר דאין חשש מנומר רק אם הוא בכל הספר ולא באיזה סדרות לבד לא נלענ״ד דהרי יש לדייק מהרישא אפכא מדקאמר (שם) ע״כ לא קאמר ר״י אלא בחדא אזכרה וכו׳ משמע דוקא בחדא אזכרה אין חשש מנומר אבל בהרבה אזכרות יש חשש אפילו אינו בכל התורה ומה דנקט בכל התורה היינו משום ששם כך הי׳ שאמר על כל הספר שלא כתב האזכרות לשמן ולכן אם ע״י חדוש הכתב מחזי כמנומר אפילו אינו רק בחד סדרה לענ״ד צריך לסלק היריעה אבל אם לא מחזי כמנומר יש תקון ע״י העברת הקולמוס על המחיקות. והנה לכאורה יש עוד חשש בתיקון זה ע״פ מה שכתב בשו״ת התשבץ ח״ג סי׳ קצ״ג שאין להעביר דיו ע״ג השם שכתוב דכתב ע״ג כתב הוי מחיקה ע״ש אכן כבר כתבתי לקמן (סי׳ צ״ו) שלענ״ד אין הלכה כתשבץ בזה ולכן אין חשש להעביר הקולמוס ג״כ על שמות הקדושים כנלענ״ד הקטן יעקב.\\\vspace{0pt}

\end{multicols}\newpage

\newchap{סימן צד}
\begin{multicols}{2}
ב״ה אלטאנא, יום ג׳ י״א מרחשון תרכ״ב לפ״ק. להרב המופלג הנגיד מ״ה עקיבא לעהרען נ״י בק״ק אמשטרדם יע״א.\\\vspace{0pt}

בדיק לן מר נ״י בנדון שנזדמן בביתו שבחור א׳ קרא בתורה בפ׳ ואתחנן ונזרקה צנורא מפיו על אות ה׳ של שם הקדוש ב״ה ובראותו רצה להעביר בסודרו ונשאר במקום הצנורא טשטוש שמתחיל בחלל ה׳ ועובר גם על החלל שבין גג ה׳ ורגלה השמאלית ושם מראהו שחור יותר ממראה הטשטוש אבל לא שחרחורת כגוף הכתב באופן שנראה כחוט דק מאוד בתוך החלל ההוא שמחבר הגג עם הרגל והוא דק כ״כ שאינו נראה לעין כל רואה ומר נ״י העלה ג׳ דרכים – א׳ – להניח הדבר כמו שהוא בעבור שראה בשו״ת יעבץ ח״ב סי׳ כ״ז שכל האותיות צריכין להיות גוף א׳ וכן ראה בהגהת רוו״ה לספר מקנה אברם שהסופרים הקדמונים לא חלקו בין ה׳ ובין ח׳ כי אם ע״י תג על אות ח׳ ומה גם שההא לא נשתנה וכל איש וגם תינוק יכיר שהוא ה׳ – ב׳ – לסלק הטשטוש ויוסר החוט ולסמוך על הסמ״ק דבכה״ג לא מקרי חק תוכות אחרי שבראשונה הי׳ האות כתוב כתקונה – ג׳ – לסלק היריעה אלא שבזה חושש להוריד הס״ת מקדושתה לשעתה ע״י פירוד היריעות אף שזה אינו רק לשעה א׳ קטנה ומכ״מ זה הדרך הוא אצלו היותר מתקבל על הלב.\\\vspace{0pt}

תשובה: אתחיל במה שסיים מעכ״ת נ״י שבעיניו ישר לסלק היריעה אף שיש אצלו חשש קצת על הורדת ס״ת מקדושתה לשעה קטנה עד שתתפר היריעה החדשה וודאי צדק בזה שאין לחוש לעשות כן לצורך תקון אפילו ספק פסול אבל לא ידעתי למה לא חשש יותר ממה שעי״ז תרד יריעה ובפרט שמות הקדושים שבה מקדושה לעולם ותבא לידי גניזה מה שמצאנו לפוסקים שחששו לכזה הרבה ולא רצו בזה כי אם היכי שאי אפשר למצוא תקנה אחרת ולכן על זה נחקור והנה מה שהזכיר מעכ״ת נ״י בשם הרב יעבץ ורוו״ה שהסופרים הקדמונים חברו כרעי׳ דה׳ לגגה כבר הוזכר בשו״ת הריב״ש ומהרי״ק ורדב״ז אבל פשיטא שאין לסמוך על זה להניח הס״ת כן לא בלבד שאין אנו נוהגין כן כמו שהסכימו הראשונים הנ״ל אלא מפני ששארי ה׳ שבתורה אינם כן ובודאי אין לעשות ה׳ בשני צורות אבל התקנה לפרוד הדבוק ולסמוך על הסמ״ק כבר נפסקה בשאלה כזו בשו״ת רדב״ז ח״ב סי׳ תקצ״ו וז״ל שם ואי הוי פשיטא לן שנכתב השם בכשרות ואח״כ נדבק טוב הי׳ לגרור הדבוק ולסמוך על האומרים שלא נקרא זה חק תוכות וכו׳ וכן סיים שם וז״ל הלכך אם נתברר שהאות נכתבה כתקנה ואח״כ נפסלה כגון שנתפשט הדיו ונדבק א״נ כגון כשבא לעשות התג שעל גג ההא נמשך למטה וחבר רגל ההא עם גגה בזה אני רואה לסמוך על האומרים שמפריד הרגל מן הגג כל שהוא ואין זה חק תוכות כיון שהאות נכתבה כתקונה בהכשר כאשר כתבו בעלי סברא זו עכ״ל ומדבריו נראה שגם הוא לא פסק כהסמק לכתחלה ורק בנדון זה סומך על דבריו למען שלא להביא שמות הקדושים שביריעה לידי גניזה מה שיש בעיניו חשש גדול כנראה מדבריו שם דביש ספק אם לא הי׳ הדיבוק בתחלה בפסול מתיר לגרור כל הרגל של ה׳ אף שלהסמ״ק יש בזה חשש מחיקת השם כיון שיש תקנה בפירוד ומסיים שם ואי לאו דמסתפינא הוי אמינא דהכי עדיף מלסלק היריעה ולגונזה עם האזכרות עכ״ל ולא ידעתי מי שחולק על הרדב״ז בזה שלא לסמוך על הסמ״ק בנדון כזה שמה שכתב הב״י (סי׳ רע״ו) בשם הר״י אסכנדרני ברם איכא למיבעי היכא דחד ה׳ מאותיות השם לא תלי כרעה כדקא יאות כגון דדביק בגגה כחוט השערה וכ״ע קרו לה ה׳ היכא לעביד דאי גריר לההוא דבק ה״ל חק תוכות וכו׳ ומסיק שם דיסלק היריעה זה איירי שבתחלת כתיבה נעשה כך כנראה מתחלת דבריו שם שבזה לכ״ע הוי חק תוכות אבל בשנעשה אחר שנכתב השם בהכשר אפשר שגם הוא מודה לסמוך על הסמ״ק ועוד יש לצרף לזה דעת הרלב״ח שהביא המג״א סי׳ ל״ב ס״ק כ״ז וז״ל אם נדבק הרגל ההא כחוט השערה באופן שהתינוק יודע שהוא ה׳ מי שיקל להפריד לא אמחה בידו והרד״ך חולק בזה עכ״ל והוי כעין ס״ס שמא הלכה כרלב״ח שאין בזה משום חק תוכות כלל כיון שתינוק קורא להא ואת״ל שהלכה כרד״ך שמא הלכה כסמ״ק דאחר כתיבה ליכא משום חק תוכות ולכן יש לעשות כפסק הרדב״ז ולפרוד החוט הדק מעט עד שתהי׳ הא כתקונה ותשוב הס״ת להכשרה: כנלענ״ד הקטן יעקב.\\\vspace{0pt}

\end{multicols}\newpage

\newchap{סימן צה}
\begin{multicols}{2}
ב״ה מאננהיים, סיון תק״ץ לפ״ק.\\\vspace{0pt}

נשאלתי – ס״ת שנמצא בה שם הוי׳ ב״ה שנחלקה היוד ע״י קריעה והגיע הקריעה גם תוך אות ה׳ עד שנחלק כמעט גם כל גג הה״א אם יש תקנה לאותה יריעה?\\\vspace{0pt}

תשובה – לפי המבואר בי״ד (סי׳ ר״פ) אם נחלק שום אות ע״י קריעה פסול ולא מהני מה שמדבקו מאחוריו ע״ש אכן מטעם זה אין לפסול כאן דכבר כתב הש״ך שם דזה דוקא היכי שנחלק ונשתנה צורת האות עי״ז ולכן בנדון זה שלא נשתנה צורת האות אין כאן פסול גם אין לפסלו מהא דאמרינן בפרק הקומץ כל אות שאין גויל מקיף לה מד׳ רוחותיו פסול דכבר הקשה הב״י אהא דכתב הטור בא״ח (סי׳ ל״ב) דנפסק האות ולא נשתנה כשר מהא דפרק הקומץ ותירץ בתי׳ בתרא דזה דוקא היכי דלכתחלה לא הי׳ מוקף גויל אבל היכי דהוי מוקף בתחלה ואח״כ נפסק כשר וכן פסק בש״ע (שם) וא״כ ה״נ דהי׳ מוקף מתחלה כשר לכאורה אבל באמת נ״ל דמכ״מ אין תקנה ליריעה זו דהנה הט״ז (שם ס״ק י״ד) הקשה דברי ש״ע אהדדי דלעיל פסק כירושלמי דגם בפנים צריך להיות מוקף קלף ואם ניטל כל תוכו של הה״א פסול אפילו ניטל לאחר כתיבה וכאן פסק דלא בעינן שיהי׳ מוקף גויל לבסוף אם בתחלה הי׳ מוקף וכן הקשה באלי׳ רבה ושניהם הניחו הקושיא בצ״ע והנלענ״ד בזה דבלא״ה צריך להבין דעת הש״ע כיון דפסול דבוק אות באות הוא ג״כ מטעם מוקף גויל כמבואר שם א״כ מ״ש דלענין דבוק לא מחלקינן בין הי׳ נפרד מתחלה ואח״כ נדבק או לא ומ״ש דלענין נחלק האות דמחלקינן בכך אכן באמת החילוק פשוט דבנחלק האות עד שאין מוקף גויל דפסולו משום וכתבתם כתיבה תמה כל שהי׳ כתיבה תמה בתחלתו לא מפסל במה שנחסר אח״כ הגויל אבל בדבוק הפסול דבעינן שכל אות יהי׳ לעצמו דאי לאו הכי אין זה צורת האות זה לעולם מפסל ולכן כל שנחתך כל הקלף תוכו של הה״א הוי כאלו נגע רגל ימין ברגל שמאל או בגג שלמעלה כיון שהקלף שהי׳ חולק הכתב נסתלק א״כ נפסל מטעם דבוק ולכן אין סתירה בדברי הש״ע דהש״ע לא הכשיר רק קרע באות א׳ אבל כשהגיע הקרע לאות אחר ונסתלק הקלף שהפריד אותם הוי כאלו נגע אות באות ופסול אפילו נעשה אחר כתיבה ונ״ל להביא ראי׳ לזה מדברי תשובת באר עשק הביאו בבית לחם יהודה (סי׳ ר״פ) וז״ל שם ס״ת שאכלו עש באותיות השם ישים טלאי חוץ לנקב מאחורי הקלף או הגויל ואח״כ יגרור מעט מן האות כדי שתהא מוקפת כהלכתו ולא הוי ח״ו כמוחק אלא כמתקן כיון דההוא שעתא אינו כשר והשתא מעמידו בהכשרו עכ״ל ולכאורה דבריו תמוהים היאך חשב זה לפסול ודאי והיקל במחיקה הא מדינא כשר הוא כיון שנעשה אחר הכתיבה אע״כ דדעתו ג״כ הכי דהיכי דנקרע הקלף בין אות לאות באופן שאין הפסק בין אות לאות אפילו אחר כתיבה פסול שוב א״ע ראיתי בתשובת נודע ביהודה חלק י״ד (סי׳ ע״ה) שכתב כדברי והביא ג״כ הראי׳ מירושלמי ושמחתי שכוונתי לדעת הגאון ובר מן דין ראיתי בתשובת דבר שמואל (סי׳ של״ג) שכתב ממש בנדון זה שנכנס הקריעה באותיות אף שלא הי׳ ניכר כלל ואעפ״כ פסל משום דחשב זה לקלקול אותיות שפסל הש״ך (בסי׳ ר״פ) וכיון דבנדון דלפנינו שהפסול בשם אין תקנה בגרירה לכן אין תקנה ליריעה זו וצריכה גניזה. כנלענ״ד הקטן יעקב.\\\vspace{0pt}

\end{multicols}\newpage

\newchap{סימן צו}
\begin{multicols}{2}
ב״ה אלטאנא, בחדש שבט תר״ד לפ״ק. לק״ק אמשטרדם יע״א.\\\vspace{0pt}

שאלה ילמדנו רבינו אודות ס״ת ישנה שנמחקו אותיותי׳ ותיבותי׳ ויש בה שמות הקדושים שרשומן ניכר שתינוק דלא חכים ולא טיפש יכול לקרותם ויש שמות שאין רשומן ניכר אם מותר להעביר עליהן קולמוס להיטיב הכתב ולחדש ע״פ כולה כדי שלא יהי׳ כמנומר אם ניחוש לכתב עליון מבטל התחתון והוי כמוחק השם ולכאורה לפי מה דאמרינן בשבת דף ק״ד כתב ע״ג כתב פטור אמר ר״ח מאן תנא וכו׳ וחכ״א אין השם מן המובחר ובגטין דף כ׳ אמר ר״ח גט שכתבו שלא לשמה והעביר עליו קולמוס לשמה וכו׳ מוכח דרב חסדא משו׳ ג׳ דינים של שבת ושל כתיבת השם ושל גט ובכולם הלכה כחכמים וכר״י ור״ל דאמרי תרווייהו דיו ע״ג דיו סיקרא ע״ג סיקרא פטור וכן פסקו הרמב״ם והטור והמחבר וכיון דכתב ע״ג כתב לאו כתיבה הוא לא הוי ג״כ מחיקה לכתב התחתון וכ״כ מהר״מ מטרני בקרית ספר פ׳ י״א דיו ע״ג דיו סיקרא ע״ג סיקרא פטור דאין כאן לא כותב ולא מוחק וכן מוכח בשו״ת הרא״ש הביאו הטור בי״ד סי׳ רע״ו וז״ל יש בו דבק באותיות השם יש לגוררו וא״צ אח״כ להעביר עליו קולמוס לקדשו כי כבר נכתב כהלכתו בקדושת השם עכ״ל ומדאמר אין צריך משמע שאין איסור בדבר רק שאין צריך ואין כאן משום מוחק גם בשו״ת חתם סופר סי׳ רנ״ו מתיר בפי׳ להעביר קולמוס על ס״ת ישנה שנמחקה גם על שמות הקדושים אבל מה שהביאני להסתפק בזה הוא מפני שראיתי בשו״ת תשב״ץ ח״ג סי׳ קצ״ג שכ׳ וז״ל ועוד יש לאסור להעביר דיו ע״ג השם דקיי״ל כתב ע״ג כתב חייב משום מוחק התחתון וכן כתבתי בתשובתי הראשון סי׳ קכ״ז עכ״ל ולא ידעתי מאין נובעים דבריו שכ׳ דקיי״ל כתב ע״ג כתב חייב משום מוחק שהרי אדרבא בדיו ע״ג דיו לכ״ע פטור ובתשובתו הראשונה אינו מדבר רק בדיו ע״ג זהב שהוא כדיו ע״ג סיקרא ואפילו בדיו ע״ג סיקרא אינו רק מדרבנן לדעת רש״י בגטין דף י״ט שפי׳ אהא דאמר ר״י שם וכי מפני שאנחנו מדמין נעשה מעשה ואפילו לענין שבת אם בא מעשה לידינו בזמן המקדש לא הייתי סומך על דברי להביא חולין לעזרה ע״ש ומשמע שמה שאמר שם חייב שתים היינו מכת מרדות אף שמדברי הרמב״ם נראה שחייב ב׳ חטאת כבר הקשה עליו בס׳ מעשה רוקח ורצה לבאר שגם לרמב״ם אינו חייב חטאת אבל בדיו ע״ג דיו ודאי לכ״ע אין כאן כתיבה ומחיקה והנה בברכי יוסף א״ח סי׳ ל״ב הביא שו״ת התשב״ץ וכתב משו״ת תרומת הדשן סי׳ מ״ח דנפסקת ג״כ בא״ח סי׳ ל״ב ס׳ כ״ז מוכח דלא כתשב״ץ דכ׳ אותיות ותיבות שנמחקו קצת אם רשומן ניכר מותר להעביר עליהן הקולמוס להטיב הכתב ולחדש ולא הוי שלא כסדרן כיון שעכשיו הכתב כשר ומה שמוסיף עליו אינו רק שלא ימחוק יותר שרי עכ״ד ואם כתב עליון מבטל ומוחק התחתון הרי הוי שלא כסדרן אע״כ דלא ס״ל כתשב״ץ אבל לפום ריהטא מרמב״ן ור״ן וריטב״א נראה כהרשב״ש כן כתב הברכי יוסף אבל לא ידעתי פירושו שלא ראיתי בדבריהם שום משמעות ולכן ילמדני רבינו מה לעשות. עוד ילמדנו איך עושים בשמות הקדושים שקפץ הדיו רק שנשאר עוד מהדיו מעט ואותו מעט כשמעבירין בקולמוס אי אפשר שלא תעבור הדיו בקולמוס והוי פסיק רישא ולפי הנראה בתוס׳ ערכין דף ו׳ ד״ה יגוד שהמעט הנשאר יש בו ג״כ קדושת השם. גם ילמדנו איך עושין כשהיוד או ההא של השם נמחק והאותיות האחרונות כשרים איך מעבירין בקולמוס עליהן אם צריך לקדש כל השם ולהעביר הקולמוס על כל השם או סגי בתיקון אות הנמחק לבד כי לפי הנראה צריך לכתוב השם כסדר ואם מעבירין בקולמוס על האותיות הראשונות הוי שלא כסדרן א״ד הצעיר יעקב צבי במר״ה מנחם הכהן סופר סת״מ בק״ק אמשטרדם יע״א.\\\vspace{0pt}

תשובה – יזקיקני להכניס ראשי בין הרים גדולים ולהכריע למעשה במקום שגדולי הוראה אשר קטנם עבה ממתני לא רצו להכריע שהברכי יוסף אחר שהביא התשב״ץ וראיתו נגדו סיים וצריך להתיישב בדבר למעשה וגם בשו״ת חתם סופר לא כתב בפי׳ להתיר כמו שכ׳ מעכ״ת נ״י אלא כתב וקרוב לודאי שמותר להעביר קולמוס ע״ג השם הרי שרק קרוב לודאי כתב ולא ודאי גמור אכן מה אעשה אחר שהחמירו הפוסקים שלא להפסיד יריעות שיש בהן כמה שמות הקדושים אם אפשר לתקנם וגם אמרינן בכתובות פ״ב והובא בפוסקים שאסור להשהות ספר שאינו מוגה השתיקה והמשיכה בזה הוי חומרא דאתי לידי קולא לכן אמרתי אענה את חלקי וד׳ יצילני משגיאות. ופתח דברי אבאר כי בדיו ע״ג סיקרא הוי כתיבה ומחיקה מדאורייתא לכ״ע ולא כמו שחשב מר נ״י שלרש״י בודאי ואפשר גם להרמב״ם לא הוי רק מדרבנן שהרי ר״י ור״ל אמרו בפי׳ חייב שתים ואיך אפשר דהוי דרבנן ומה שרצה לפרש דחייב שתים מכת מרדות קאמר לא ידעתי איך שייך לכפול מכת מרדות שהרי אין למכת מרדות מנין קבוע אלא לפי ראות עיני ב״ד וביותר הי׳ אפשר לפרש דנקט חייב שתים לענין לקברו בין רשעים גמורים וכדאמרינן ביבמות (דף ל״ב) לענין חייב שתים דקתני שם אכן דוחק לומר כן במקום דהגמרא לא מפרש הכי ועוד דגם התם דאורייתא הוא אבל באמת נראה דגם לרש״י מה דקאמר חייב שתים ב׳ חטאת קאמר וכפסק הרמב״ם ומה דמפרש רש״י אדברי ר״י אם בא מעשה לידי לא הייתי סומך על דברי להביא חולין לעזרה יפורש ע״פ הירושלמי בגטין שם שהביא ג״כ דברי ר״י בשינוי לשון קצת וז״ל אמר לי׳ ר׳ יוחנן מפני שאנו עסוקים בהלכות שבת אנו מתירין את אשת איש עכ״ל והמפרש קרבן עדה רצה לפרש דה״ק וכי מפני שאנחנו סומכין על דעתנו להתיר חולין לעזרה דליכא כרת נתיר אשת איש שיש בו כרת ורצה ליישב בזה מה שפסק הרמב״ם חייב שתים ע״ש אבל לכאורה זה מתנגד למה דאמרינן ביבמות (דף קי״ט) מכדי הא דאורייתא והא דאורייתא מה לי איסור לאו מה לי איסור כרת ואף דבתרומת הדשן הקשה על זה ממה דאמרינן שם (דף פ״ב) בספרי ערוך לנר כבר יישבתי ואין להאריך ולענ״ד פי׳ דברי ר׳ יוחנן ע״פ מה דאמרינן בשבת (דף קכ״ב) מעשה דר״ג וזקנים קמ״ל וכן אמרינן בכמה דוכתי׳ מעשה רב הרי דאפילו בדינים הנוהגים בזה״ז אם אמרו תנאים ואמוראים הלכה לא הי׳ ברור כ״כ כאלו עשו מעשה ע״פ הלכה דבשעת מעשה עוד דקדקו ובררו יותר וזה ג״כ כוונת ר״י במה שאמר מפני שאנו עסוקים עתה בה׳ שבת לענין קרבנות מה שהוא רק להלכה ומדמין מלתא למלתא נתיר ג״כ אשת איש שהוא הלכה למעשה וזה ג״כ דברי רש״י אם בא עתה מעשה לידי לא הייתי סומך על דברי להתיר למעשה להביא חולין לעזרה עד שהייתי מדקדק ומברר עוד ועל כן למעשה לא נסמוך אף שלהלכה פסקנו דחייב שתים ולכן א״ש ג״כ דפסק הרמב״ם חייב שתים דלהלכה ודאי ר״י ור״ל אמרו כן. ועפ״ז נתברר דלענין דיו ע״ג סקרא לכ״ע הוי איסור דאורייתא בשבת אבל לענין דיו ע״ג דיו לכאורה אין זה כתב לכ״ע אכן אחרי עיון נראה דיפה דן הברכי יוסף שמהראשונים משמע כתשב״ץ ולא בלבד שמשמע כן אלא גם הראה לנו בזה מקום מוצא דין דהתשבץ שהרי בשבת (דף ק״ג) הקשו התוספ׳ לרב אחא בר יעקב דס״ל דרק בשם כתב ע״ג כתב לא נחשב כתיבה אבל בשאר מקומות הוי כתיבה א״כ מתניתן דשבת דכתב ע״ג כתב פטור דלא כמאן ותרצו דגם לראב״י דוקא כשכתב אחרון מתקן כלום כגון בגט ס״ל דהוי כתיבה אבל בלא״ה לא וכן כתבו התוספ׳ ג״כ בגטין דף י״ט והרמב״ן הקשה ג״כ קושית התוספ׳ ותירץ די״ל דראב״י מוקי למתניתן בסקרא ע״ג כתב דלכ״ע לאו כתיבה הוי א״נ סבר איהו שאני לשמה משבת דגבי שבת מלאכת מחשבת בעינן ולא אהנו מעשיו שכבר נכתב עכ״ל ויש נפקותא בין תירוץ התוס׳ לתירוצי הרמב״ן דלתי׳ התוספ׳ לראב״י אליבי׳ דרבנן כתב ע״ג כתב בכ״מ לא הוי כתיבה ורק לגבי גט מפני שכתב אחרון מתקן הוי כתיבה אבל לתירוצי הרמב״ן הוי אפכא דלראב״י אליבי׳ דרבנן בכ״מ כעג״כ הוי כתיבה ורק לענין שבת לא הוי כתיבה דלא הוי מלאכת מחשבת כיון דלא מהני מידי ולתירוצו הראשון אפילו גבי שבת דיו ע״ג דיו הוי כתיבה ורק סקרא ע״ג דיו לא והנה הרשב״א בגטין הביא פלוגתא לענין פסק הלכה דהר״ח פסק לחומרא כראב״י דעכ״פ לחומרא חיישינן לי׳ אבל הרמב״ם פסק כר״ח ע״ש ובזה מבואר שיטת התשבץ די״ל דס״ל כתירוץ הרמב״ן דלראב״י אליבי׳ דרבנן בכ״מ כתב ע״ג כתב הוי כתיבה חוץ מבשבת וגם ס״ל כשיטת הר״ח והרשב״א שפסק כראב״י לחומרא לכין כתב דכתב ע״ג השם הוי מחיקה כמו דכתב ע״ג זהב והו כתיבה אפילו לענין שבת כמו כן הוי לחומרא כתיבה ג״כ דיו ע״ג דיו לענין מחיקת השם ומסתמא לזה כוון ג״כ הברכי יוסף כי הר״ן בחידושיו כתב ג״כ כרמב״ן ומסתמא ג״כ הריטב״א אשר אין כעת בידי ואם אמנם זאת שיטת התשב״ץ נלענ״ד להלכה כיון דהתוספ׳ בשני המקומות לא כתבו כן אלא דרק לענין גט פליג ראב״י אבל בכ״מ ס״ל דכעג״כ לא הוי כתיבה ובאמת שיטת הרמב״ן צ״ע דאם יש לחלק בין שבת לשאר מקומות מנ״ל לר״ח בשבת שם דמתניתן דלא כר׳ יהודה דלמא התם לענין שבת גם ר״י מודה ועכ״פ התוס׳ לא ס״ל חילוק זה וגם במה שע״פ דברינו התשבץ ס״ל כשיטת ר׳ חננאל דהלכה כראב״י לחומרא לא מסכים להלכה דרוב הפוסקים פסקו סתמא כרבנן אליבי׳ דרב חסדא והטור אהע״ז סי׳ קל״א לא הביא דעת רבינו חננאל אלא כתב סתמא דאינו גט וכן פסק סתמא בש״ע שם אף שהביא שיטת ר״ח ג״כ לא הביא רק בשם י״א אחר שכתב שיטת הרמב״ם בסתם הרי דאפילו לענין איסור אשת איש החמור לא חששו הפוסקים לשיטת ר״ח כש״כ לענין ל״ת דמחיקת השם ולכן כיון דאפילו לראב״י אליבי׳ דהתוס׳ ליכא כתיבה ע״ג כתיבה בשאר מקומות וגם רוב הפוסקים לא פסקו כראב״י לחומרא הכי נקטינן דליכא כעג״כ בכתיבת השם וממילא לא הוי ג״כ מחיקה וכאשר מסכים זה ג״כ עם פסק התה״ד דפסקינן כוותי׳ דמותר להעביר קולמוס ע״ג תפילין אף דבעינן כסדרן.\\\vspace{0pt}

גם יש לצרף לזה מה שכתבו הפוסקים דמחיקת השם לצרוך תיקון אינו מחיקה והביאו ראי׳ מהא דתנינן במסכת סופרים דטפת דיו שנפל על הכתב מותר למחקו שלא הי׳ כוונתו אלא לתקן והר״י רצה להביא ראי׳ מזה לענין נגיעה באלפין ואף שהמרדכי חולק עליו היינו דוקא כיון שהשם עדיין לא נכתב כתקונו בקדושה אבל כשהשם כבר נכתב כתקונו מודה ג״כ דאם אירע דבר הפוסלו שמותר למחוק לצורך תקון וגם בשו״ת חתם סופר סי׳ רס״ו הסכים כן להלכה ולכן נ״ל שמותר להעביר קולמוס גם על שמות הקדושים ולתקנם ואפילו על אותיות שהכתב עדיין קיים. ומטעם זה נ״ל ג״כ דאין חשש מה שכתב מעכ״ת שאם יעביר הקולמוס על אות שנמחק ונשאר עוד מעט דיו שאפשר שימחקו והוי פסיק רישא דאפילו ימחוק הוי לצורך תקון וגם כבר כתב בחתם סופר סי׳ רע״א בכה״ג להתיר דלא הוי מחיקה רק גרם מחיקה דשרי יע״ש ובלא״ה נ״ל דבכה״ג דנמחק האות כבר ולא נשאר רק מעט דיו דלא שייך עוד מחיקה דבמסכת סופרים לא אמרינן רק דאסור למחוק אות משמות הקדושים והכא אין עוד אות שכבר נמחק ומה שהזכיר מעכ״ת מתוספ׳ דערכין לא ראיתי שום ראי׳ משם ומה שחשש מעכ״ת נ״י עוד דהוי ע״י העביר קולמוס כתיבת השם שלא כסדרן ולפי הנראה צריך לכתוב השם כסדר – לא ידעתי מאין נראה לו כן כי בפוסקים לא ראיתי הזכרה מזה שיש לכתוב השם בס״ת כסדר רק בשו״ת גינת וורדים הביא כן בשם גדול א׳ ובאמת ראיותיו חלשים ובשו״ת רדב״ז ב׳ אלפים קס״א כתב בפי׳ שא״צ לכתוב אותיות השם על הסדר וחכם א׳ הביא ראי׳ לזה מבר קמצר שהי׳ כותב ד׳ אותיות השם כאחד וגנוהו חכמים שלא רצה ללמד ואי בעינן כסדרן הרי כתיבתו פסולה היתה שא״א לצמצם שלא ישלים האות האחרון רגע קודם הראשון אע״כ דאין קפידא להיות כסדרן והיא ראי׳ גדולה שא״צ לדקדק בזה ולכן נלענ״ד שמותר להעביר הקולמוס גם על השמות הקדושים ולחדשם הקטן יעקב.\\\vspace{0pt}

\end{multicols}\newpage

\newchap{סימן צז}
\begin{multicols}{2}
ב״ה אלטאנא, יום ו׳ כ״ז כסליו שנת תר״ט לפ״ק.\\\vspace{0pt}

שאלה – אם מותר להניח ס״ת בלוי׳ שאי אפשר לתקן עוד בארון הקדש אצל ספרי תורות כשרות.\\\vspace{0pt}

תשובה – שאלה זו כבר נשאלה בשו״ת נודע ביהודה ח״א (א״ח סי׳ ט׳) והשואל רצה להתיר מטעם לוחות ושברי לוחות מונחים בארון אבל הגאון המחבר ז״ל השיב לו שראי׳ זו אינה כלום ששם הארון מתחלה נעשה לשם כך להניח בו גם שברי לוחות ופשיטא דתנאי מועיל בתחלה אבל לא שייך זה בס״ת שנפסל גם מסיק שיש איסור אפילו יש קדושה בס״ת שבלה שמא יקרא בו ע״ש אכן במכ״ה נעלם ממנו שדין זה כבר הוזכר בספר חסידים (סי׳ תתקל״ד) להתיר וז״ל שם לוחות ושברי לוחות מונחים בארון אם יש ספר תורה במקצתם קרועים ומחוקים ישים עם ספר תורה בארון ואם בא לגנוז יריעות של ס״ת לא יתכן שיהיו בארון בין העמודים על יריעות ויגנזו במקום אחר ופן יהו מושלכים עכ״ל הרי בפי׳ שהתיר לשים ס״ת פסולה עם הכשרות רק שחילק בין ס״ת שלמה שבלה ליריעות שבלו והתיר מטעם לוחות וש״ל מונחים בארון מה שדחה בנ״ב בשתי ידים ובאמת מה שהשיב על זה ששם מתחלה נעשה גם לשברי לוחות והוי כמו התנה לענ״ד אין זה תשובה דעכ״פ מדצוה הקב״ה להשים השברים אצל הלוחות בארון א׳ מוכח דלהשברים קדושה כמו להשלמים וא״כ ה״ה בס״ת ועוד כיון דכבר נהגו העם להשים הס״ת הפסולים אצל הכשרים בארון הקדש הוי כמו התנה בתחלה גם בארון הקדש כמו גבי ארון דלוחות ואף דאמרינן במגילה (דף כ״ו) ס״ת שבלה גונזין אותו אצל ת״ח וכ״פ בא״ח (סי׳ קנ״ד) ובי״ד (סי׳ רפ״ב) זה דוקא אם אין לו מקום להניחו וצריך לגנזו קמל״ן היאך יגנזו אבל אם רצה להניחו בארון הקדש במקום מיואד זהו גניזתו. גם מה שכתב בשו״ת נ״ב הנ״ל לאסור מטעם שמא יקראו בו והביא ראי׳ ממה דנפסק בש״ע י״ד (סי׳ רע״ט) דאסור להשהות ספר שאינו מוגה יותר מל׳ יום אלא יתקן או יגנוז לענ״ד תמו׳ שהרי דין זה נובע ממה דאמרינן כתובות (דף י״ט) אתמר ספר שאינה מוגה אר״א עד ל׳ יום מותר לשהותו מכאן ואילך אסור משום שנאמר אל תשכן באהליך עולה ע״ש הרי בפי׳ דהטעם דוקא משום אל תשכן שלא יקרא הטעיות אבל לא מטעם שמא יקרא בו ויחזיקו לכשר ועוד שהרי רש״י פי׳ שם דספר שאינו מוגה היינו תורה נביאים וכתובים וכ״כ הש״מ שם בשם שאר ראשונים והרי בהם ודאי איך שייך גזירה דשמא קרא דמה איסור יש לקרות בנביאים וכתובים שאינם מוגהים אע״כ דהטעם דוקא משום כיון דיש לתקן והוא קורא הטעיות מקרי עולה ואסור לשכון באהלו וכן משמע הלשון ספר שאינו מוגה משמע שראוי להגיהו ולתקנו אבל כיון דאי אפשר לתקן והוא אינו רוצה לקרות בו אין כאן עולה דמה לו לעשות ולכן גם מה שהוסיף הרמב״ם ואחריו הש״ע דיתקן או יגנוז הפירוש כיון שאפשר לתקן אם גנזו הרי שהרחיק העולה מביתו אבל לא שצריך לגנוז מה שאי אפשר לתקן והוא אינו קורא בו מטעם שמא יקרא בו בצבור ועוד הרי לזה יש תקנה שיניח הס״ת הפסולים במקום מיוחד באה״ק וכמנהג הרבה קהלות וזהו בעצמו גניזתם ולכן אין כאן חשש רק מטעם כבוד ס״ת הכשרים ולזה מהני הטעם של הס״ח דלוחות ושברי לוחות מונחים בארון. הן אמת דזה בעצמו ששברי לוחות היו מונחים בארון א׳ עם הלוחות השלמים אינו מוסכם מכל דמצינו בזה פלוגתא בירושלמי דשקלים (פ״ה) דלר׳ יהודה בן לקיש ב׳ ארונות היו ואותו ארון שהיו מונחות בו השברי לוחות לא הי׳ בו קדושה כ״כ והי׳ יוצא עמהם למלחמה אבל לרבנן ארון א׳ הי׳ וכבר הזכיר גם רש״י בחומש בפ׳ עקב דעת ריב״ל וע״ש ברמב״ן וגם הספרי ס״ל כן דב׳ ארונות היו דאיתא שם וארון ברית ד׳ נוסע לפניהם זה שיוצא עמהם במחנות והיו בו שברי לוחות עכ״ל וגם במדרש תנחומא פ׳ עקב ס״ל כן ע״ש ולפ״ז אין ראי׳ מהא דשברי לוחות שמותר להניח ס״ת בלוי עם הכשרים בארון א׳ אכן מגמרא שלנו נראה דפשיטא דארון א׳ הי׳ ללוחות וש״ל ושאין פלוגתא בזה שהרי בב״ב (דף י״ד) שקיל וטרי לר״מ מנ״ל דשברי לוחות מונחים בארון ע״ש ומה פריך דלמא באמת ס״ל כאידך מ״ד שלא היו מונחים בארון אע״כ מוכח דגמרא שלנו ס״ל דליכא פלוגתא בזה ועוד דגם בירושלמי ריב״ל דס״ל דב׳ ארונות היו הוא יחיד נגד רבנן דס״ל דבארון א׳ מונחים והלכה כרבים ולכן מה שנוהגים בהרבה קהלות שאינם גונזים הס״ת שבלו רק מניחים אותם באה״ק במקום מיוחד ומוציאים אותם בש״ת להרבות ההקפה יש להם לסמוך על הס״ח שהתיר כנלענ״ד הקטן יעקב.\\\vspace{0pt}

\end{multicols}\newpage

\newchap{סימן צח}
\begin{multicols}{2}
ב״ה אלטאנא, יום ד׳ ו׳ שבט תרי״ט לפ״ק.\\\vspace{0pt}

שאלה – נמצא טעות בס״ת בפסוק בסכות הושבתי את ב״י שהי׳ כתוב סכת חסר ו׳ וע״פ המסורה צ״ל מלא אם להוציא אחרת.\\\vspace{0pt}

תשובה – בא״ח (סי׳ קמ״ג) כתב הרמ״א והא דמוציאין אחר דוקא שנמצא טעות גמור אבל משום חסרות ויתרות אין להוציא אחר שאין ס״ת שלנו מדוייקים כ״כ שנאמר שהאחרת יהי׳ יותר כשר עכ״ל אמנם בנמצא טעות בחסרות ויתרות במקום שלמדו רז״ל דין מזה בזה העלה בשו״ת שער אפרים (סי׳ פב) דצריך להוציא אחרת ועמו הסכים ג״כ בשו״ת שב יעקב (סי׳ נ״ו) דהיכא דיליף מני׳ בשם הלכה אף דלא נשתנה הלשון כגון ויצאה חכם אין כסף דדרשינן מיוד אין כסף לאדון זה או על כל נפשות מת לא יבא או קרנות ודאי צ״ל שהי׳ להם קבלה בודאי דנכון להיות כן וכתב עוד ששמח שאחר כתבו זאת מצא ראי׳ לדבריו בשו״ת רדב״ז שכתב (סי׳ ק״א) וז״ל תמהתי על המגיהים האלה וכי עדיף להו מדרשו של רשב״י מהגמרא ערוכה אשר בידינו וכתב בו פלגשם כתיב ובכל הספרים הוא מלא וכן ואשמם בראשיכם וגו׳ אלא עיקרן של דברים מה שאגיד לך שכל מלא וחסר שתלה בו דין לפי מה שלמדו בגמרא כגון קרנות קרנת סכות סכת ובן אין לו עיין עליו אילו וכיוצא בהן יש להגי׳ הספרים אם נמצאו היפך ממה שאמרו בגמרא אבל כל מלא וחסר דלא נ״מ לענין דינא אלא מדרש בעלמא לא נגי׳ שום ספר ע״פ הדרשה ולא ע״פ המסורה אלא אזלינן בתר רובא וכו׳ במלתא דלא תלי לענין דינא עכ״ל ועל פי הדברים האלה גם בחסר ו׳ דבסוכות הושבתי צריך להוציא אחרת כיון דמזה נדרש הדין דמחיצות סוכה בסוכה (דף ו׳) אלא דכבר שדא נרגא בפסק זה בסדרי טהרה נדה (דף ל״ג) ממה דאמרינן שם לחד גרסא והנשא כתיב והתוספ׳ הקשו הרי במסורה הוא מלא ותרצו דלפעמים המסורה חולק על גמרא שלנו כדאמרינן בשבת גבי בני עלי מעבירם כתיב ובמקראות שלנו כתיב מעבירים מלא ע״ש והרי מהך והנשא כתיב ילפינן דין טומאת עליונו של זב ולפי דברי תשובות הנ״ל ס״ת דידן פסולים לקרות בו וצריכין תקון והניח בצ״ע ולענ״ד מזה אין אי׳ כ״כ דתלי בגרסות ורש״י דלפנינו שם מחק הך והנשא כתיב ורק רש״י בתשובה לפי משכ׳ התוספ׳ גרס כן ואפילו ע״פ הגרסא והנשא כתיב כבר פי׳ הרמב״ן שלא רצה לומר דכתיב חסר אלא דמלשון והנשא דרש ע״ש ולכן אין מזה קושיא כ״כ על ספרים שלנו שסומכים על גרסת רש״י דלפנינו אבל יש לתמו׳ על הגדולים איך לא העירו דעיקר דרשה דקרנות דדרש מני׳ בית הלל בזבחים (דף ל״ז) דבחטאת מתנה אחת מעכבת מדכתיב קרנת קרנת קרנות בחטאת יחיד אינה מסכמת עם המסורה ועם כל ס״ת שלנו שבהם כל ג׳ קרנות דחטאת יחיד חסרים הם וכבר העיר על זה הר״ן בחדושיו לסנהדרין (דף ה׳) וז״ל משמע מפורש מהכא דלפום גמרא צריך להיות בס״ת קרנות בוי״ו בפ׳ כשבה דיחיד וכתבו מקצת מפרשים ז״ל שכתבו בשם הר״ר שלמי׳ ז״ל דלא כתיב קרנות לא במכתב ולא במסורה וא״כ הוא צ״ע ויש כיוצא בזה ביום כלות משה שלם ורבותינו אמר דכלת כתיב וכן ולבני הפילגשים מלא ורבותינו אומרים פילגשם חסר וכתב הרשב״א ז״ל דהא הוי פלוגתא דמדינחאי ומערבאי בפלוגתא בבן אשר ובן נפתלי ובכל מקום שאנו מוצאים מחלוקות בספרים אמרו במסכת סופרים שהולכים אחר הרוב עכ״ל הרי מבואר מזה דמסורה שלנו אפילו עם מה דיליף ב״ה דין ממנה אינה מסכמת ובמכ״ה נעלם זה בפרט מהרדב״ז והשב יעקב שהזכירו גם קרנת באותן הטעיות שצריך להוציא עליהם ולפי דבריהם אדרבא כל ספרי תורה שלנו פסולים הם שכתוב בהם קרנת כל ג׳ פעמים חסר ואין צ״ל שבעל מסורה שלנו חולק על ב״ה אלא דס״ל דטעם דב״ה דמתנה א׳ מעכבת אינו כמו שמפרש רב הונא מקרנות מלא אלא דיליף כן מדרשה אחרת (ואולי שיליף כן מקרא דודם זבחיך ישפך שהזכירו התוספ׳ בסנהדרין שם) ובהכי ניחא ג״כ מה דאמרינן בסנהדרין (שם) רבי ור״י בן רועץ וב״ש ור״ש ור״ע כולהו ס״ל יש אם למקרא ע״ש וכי כל הני תנאי פליגי על ב״ה וס״ל כב״ש ובפרט רבי דמב״ה קאתי אלא ודאי די״ל דב״ש וב״ה לא ביש אם למקרא או למסורה פליגי ועל פי גם מסורה שלנו דקרנת חסר לא פליג על ב״ה ועכ״פ היוצא מזה דגם במה דאתי דין מני׳ אין הקבלה ודאית שנאמר שעל זה נסמוך יותר מעל מסורה אחרת ולכן אין טעם מזה להוציא על טעות כזה ס״ת אחרת יותר משאר טעיות בחסרות ויתרות ובלא״ה לא הבנתי פסק זה דמה בכך שזה ודאי טעות הרי הטעם שכתב הרמ״א שלא להוציא שאין ס״ת שלנו מדוייקים כ״כ שנאמר שהאחר יותר כשר וכ׳ המג״א דדוקא לענין חסרות ויתרות אמרינן כן דלא בקיאינן בהו כדאיתא בקידושין פ״ק ושמא האחר ג״כ אינו מדויק בחסר ויתר ע״ש הרי שאין הטעם מפני שנאמר שמא זה אינו טעות אלא שנאמר שיש טעיות אחרות כמותן גם בס״ת אחרת כיון שאין אנו בקיאין בחסר ויתר וא״כ גם בטעות ודאי בחסר ויתר דאתי דין מני׳ ג״כ אין להוציא אחרת כנלענ״ד: הקטן יעקב.\\\vspace{0pt}

\end{multicols}\newpage

\newchap{סימן צט}
\begin{multicols}{2}
ב״ה אלטאנא, יום ו׳ כ׳ תמוז תרט״ו לפ״ק.\\\vspace{0pt}

שאלה: מי שיש לו בביתו חדר קטן סמוך לגדול ויש פתח מן החדר הגדול לקטן והיכר ציר של הפתח הוא בחדר הקטן וחדר קטן זה ארכו יותר מו׳ אמות ורחבו סמוך לשלש אמות באופן שאין בו ד׳ על ד׳ בשו׳ אבל כשנצרף האורך לרחבו יש בו יותר מד׳ על ד׳ שהוא ט״ז פעמים אמה על אמה אם פתח זה חייב במזוזה ואם חייב באיזה צד יעשה אותה.\\\vspace{0pt}

תשובה: פשוט בסוכה (דף ג׳) דבית שאין בו ד׳ על ד׳ פטור מן המזוזה וכן נפסק ברמב״ם וטוש״ע י״ד (סי׳ רפ״ו) אכן אם יש פתח מחדר גדול לקטן כזה אם נאמר דג״כ פטור או אם נאמר דפתח זו אף דמשום חדר קטן פטור מכ״מ חייב משום גדול זה לא מבואר בפוסקים. והנה דעת השואל בשו״ת מהרי״ל (סי׳ צ״ח) נראה קצת שפתח כזו פטור לגמרי והמהרי״ל כשהשיב לו (סי׳ צ״ט) לא השיב על שאלה זו בפירוש ומ״מ נ״ל שפתח זה חייב במזוזה דמה בכך שהחדר קטן אינו חייב במזוזה מכ״מ הרי הפתח משמשת ג״כ לחדר הגדול לחזור בו מן החדר הקטן לתוכו וא״כ חייב משום גדול וכן נראה ממה שפסק בש״ע (סי׳ רפ״ו) דאם יש פתח מבית המדרש לצאת לבית חייב במזוזה באותו פתח וכ״נ ג״כ ממה שכתב הפרי מגדים ה׳ סוכה דמי שיש לו חדר סמוך לסוכה זה לפנים מזה אם הולך מחדרו לסוכה פשיטא שפתח הסוכה צריך מזוזה משום חדרו עכ״ל הרי דפשיטא לו ג״כ דאף דהפתח נעשה משום סוכה לכנוס מהחדר לתוכה והסוכה בימי החג פטורה ממזוזה מכ״מ צריך לעשות מזוזה משום החדר ומה לי אם הפטור הוא משום סוכה או מפני שאין בו דע״ד וכאשר הודעתי שכן דעתי להרב מ״ה מל״ל נ״י הראה לי ראי׳ לדברי מדברי שלטי הגיבורים סוף ה׳ מזוזה שכתב וכן נראה בעיני שאם הי׳ בית גדול פתוח לבית קטן שאין בו ארבע אמות על ארבע אמות בין שהי׳ היכר ציר לצד הבית הגדול בין שהי׳ היכר ציר לצד הבית הקטן חייב במזוזה שפתחו של גדול הוא נחשב והרי הוא כבית הפתוח לרשות הרבים או לגינה שהוא חייב במזוזה ואפילו הי׳ היכר ציר מבחוץ כמבואר בקונטרס הראיות עכ״ל וזה מפורש כדברי ומהתימא שלא העתיקו הפוסקים דברים האלה. ומכ״מ יש חילוק רב בין מזוזה בפתח שבין חדר גדול לקטן שפטור ובין מזוזה שבין ב׳ חדרים שחייבים שניהם דאם שניהם חייבים צריך לעשות המזוזה בימין הכניסה לחדר שיש בו היכר ציר או שעכ״פ דרך לכנוס שם ביותר כמבואר (סי׳ רפ״ט ס״ו) וע״ש בש״ך וט״ז אבל כשהקטן פטור א״כ כל חיוב המזוזה משום חדר גדול ולכן צריך לעשות המזוזה בימין הכניסה מחדר קטן לגדול אף שהיכר צור הוא בחדר קטן כמו בפתח הפתוח לרה״ר דודאי עושין המזוזה בימין הכניסה מרה״ר לבית אף שהיכר ציר מבחוץ ומטעם זה נ״ל ג״כ בנדון של הפרי מגדים בחדר פתוח לסוכה שאף שכל ימות השנה שהסוכה ג״כ חייבת במזוזה צריך לקבוע המזוזה בימין הפתח שמן החדר לסוכה כיון שדרך הכניסה היא מן החדר לסוכה וכל שכן אם היכר ציר ג״כ הוא בסוכה מכ״מ בימי הסוכות שהסוכה פטורה והחדר לבד חייב וחיוב הפתח במזוזה הוא בשביל החדר צריך להפוך מקום המזוזה לקבעה בשמאל הנכנס מן החדר לסוכה שהיא הימין של הנכנס מן הסוכה לחדר בין שהיכר ציר בסוכה בין שהוא בחדר ולכן בחדר קטן שפתוח לגדול מקום קביעות המזוזה הוא בימין הכניסה מן החדר הקטן לגדול שהוא בצד שמאל של הנכנס מן הגדול לקטן.\\\vspace{0pt}

אבל זה דוקא בחדר קטן שפטור בודאי מן המזוזה אכן בנדון השאלה שהחדר אין בו ד׳ על ד׳ אבל יש בו לרבע ד׳ על ד׳ בזה יש לדון אחר שלפי דעת הרמב״ם שפסק הש״ע כמותו (סי׳ רפ״ו סי״ג) חדר כזה חייב במזוזה ולפי דעת הרא״ש פטור ולפי מה שכתב הש״ך שם גם דעת רבינו ירוחם כהרא״ש ולכן יש לחוש לדעת הראש לקבוע מזוזה בלא ברכה ע״ש והנה הנפקותא שבין דעת הרמב״ם לדעת הרא״ש שהמציא הש״ך לענין ברכה היא בנדון דידן בחיוב מזוזה דאורייתא דלדעת הרמב״ם שהחדר קטן חייב במזוזה צריך לקבוע המזוזה בימין כניסה לחדר קטן ולהרא״ש דהחיוב הוא משום חדר גדול לבד כיון שהחדר קטן פטור צריך לקבעה בצד שכנגד שהוא הימין של כניסה מן ח״ק לח״ג ואם שינה לדעת זה או לדעת זה פסולה דמזוזה שקבעה בשמאל פסולה כמבואר בפוסקים ואף שהט״ז בא״ח (סי׳ תרל״ד) העלה שאין פלוגתא כלל ושגם דעת הרמב״ם כדעת הרא״ש וחולק על הרא״ש שהבין דעת הרמב״ם שחולק עליו מכ״מ מי יכריע נגד הרא״ש והפוסקים בכוונת הרמב״ם וכן בין דעת הרמב״ם לדעת הרא״ש בחיוב מזוזה דאורייתא ולכן לענ״ד לצאת ב׳ השיטות צריך לקבוע בפתח זה שבין ח״ג לח״ק ב׳ מזוזות אחת בימין הכניסה מן הח״ק לח״ג ואחת בצד שכנגד בימין הכניסה מן הח״ג לח״ק ואין לחוש בזה משום בל תוסיף כשעושה ב׳ מזוזות בפתח א׳ כיון דממנ״פ רק אחת מהן כשרה לכל אחת מהשיטות א״כ ליכא בל תוסיף כמו במניח ב׳ זוגי תפילין של רש״י ור״ת כאחד שאין בזה משום ב״ת כיון דרק א׳ מהן כשר כמבואר בטוש״ע א״ח (סי׳ ל״ד) וכאשר קובע ב׳ המזוזות יעשה האחת בימין הכניסה לחדר הקטן באלכסון ושמע כלפי פנים של חדר הקטן והאחת במזוזה השמאלית דהיינו בימין הכניסה לחדר הגדול באלכסון ושמע כלפי חדר הגדול ויכוון שבאיזה מהן שהיא כדין יהי׳ יוצא ידי מצות מזוזה וכן עשיתי מעשה בביתי ובאשר שדבר חדש הוא שלא ראיתי בפוסקים לעשות ב׳ מזוזות בפתח א׳ מספק העלתי פסק זה על ספר לשמוע דעת חכמים. הקטן יעקב.\\\vspace{0pt}

\end{multicols}\newpage

\newchap{סימן ק}
\begin{multicols}{2}
ב״ה אלטאנא, יום ו׳ כ״א אדר שני תרט״ז לפ״ק. להרה״ג וכו׳ מ״ה משה שיק נ״י הגאב״ד דק״ק יערגען יע״א.\\\vspace{0pt}

על מה שכתב לי מעכ״ת נ״י וז״ל: מר נ״י פסק בית גדול הפתוח לקטן שאין בו ד׳ על ד׳ דיקבע מזוזה בימין הכניסה מחדר קטן לגדול אפי׳ אם היכר ציר בקטן ומשמע שם דיקבע בברכה ואם החדר קטן ארוך ואינו רחב ד׳ ויש בו לרבע דע״ד ויש היכר ציר בקטן פסק שם דהוי ספק והעלה הלכה למעשה דיקבע ב׳ מזוזות אחת בימין ואחת בשמאל כמו ב׳ זוגי תפילין דרש״י ור״ת בא״ח סי׳ ל״ד יעו״ש ולענ״ד אין הנדון דומה לראי׳ התם חד מהנך תפילין פסול ממ״נ דאי אפיך גווייתא לברייתא פסולין וכאילו אינם תפילין והוי כמניח חפץ אחר על ראשו ולכן אין כאן איסור בל תוסיף וכמ״ש המג״א שם סק״ג אבל כאן דהמזוזות שניהם כשרין אלא דצד שמאל פטור ולכך אם מניח גם התם עובר על ב״ת מידי דהוי כמטיל חמש ציצית בחמש כנפות דעובר על ב״ת וכפירש״י בפ׳ ואתחנן בשם הסיפרי אע״ג דאינו מחוייב אלא בד׳ המרוחקות וכנף אחד מהם פטור ואם נסמוך הואיל ואינו עושה אלא משום ספק ואינו מכווין להוסיף זה דבר שאינו ברור ובאנו למחלוקת ולמ״ד לעבור בזמנו לא בעי כוונה אמרי׳ בר״ה כ״ח דאסור ליתן מתן ד׳ משום ספק ופרמ״ג בפתיחה כוללת האריך בסברות לכאן ולכאן וא״כ תקנה זו אין בה לצאת אליבא דכ״ע עכ״ד מעכ״ת נ״י.\\\vspace{0pt}

על זה אשיב בקצרה מאפס הפנאי הנה מעכ״ת נ״י דימה ב׳ מזוזות לחמש ציצית ולענ״ד אין זה דמיון כלל דשם כל אחת מהציציות כשרה דאיזה מהן שמסיר הטלית כשר ע״י ד׳ הנשארים ולכן איכא בל תוסיף כמו בקובע ב׳ מזוזות בימין אבל מזוזה בשמאל אינה מזוזה וכאילו מחבר חפץ אחר שם ודומה ממש לב׳ זוגי תפילין דרש״י ור״ת דשם הפסול לכל א׳ מפני שינוי מקום הנחת הפרשיות וכן הכא והנה זה מקרוב מצאתי בשאילת יעב״ץ סי׳ ע׳ שעשה ג״כ הלכה למעשה בספק לקבוע ב׳ מזוזות. כנלענ״ד הקטן יעקב.\\\vspace{0pt}

\end{multicols}\newpage

\newchap{סימן קא}
\begin{multicols}{2}
ב״ה אלטאנא, יום ג׳ ו׳ אייר שנת תר״ח לפ״ק.\\\vspace{0pt}

שאלה – אם מותר ליתן ערלה וכלאי כרם האסורים בהנאה לפני כלבים ועופות של הפקר שאינו נהנה בהם.\\\vspace{0pt}

תשובה – אמרינן בא״ח (סי׳ תמ״ח) אסור להאכיל חמצו בפסח אפילו לבהמת אחרים או של הפקר וכתב המג״א הטעם דמהנה הוא לבהמה ע״ש ולפ״ז גם בערלה וכ״כ אסור אבל דברי המג״א צ״ע דמה בכך דמהנה לבהמה כיון שהוא אינו נהנה אבל באמת דין זה אתי מהירושלמי כדאיתא בטור והירושלמי יליף מלא יאכל חמץ בצירי לא יאכיל לבהמת הפקר ונראה דדרש לשון נפעל לא יאכל על ידך ולא מבעי׳ לר׳ אבו׳ דפסק הרמב״ם כוותי׳ דמיותר לשון לא יאכל דגם לא יאכל בחולם משמע איסור הנאה אלא אפי׳ לחזקי׳ א״ש דכל זה בכלל דרשת חזקי׳ לא יהא בו היתר אכילה וכיון דעכ״פ נראה בפי׳ מהירושלמי דמלשון לא יאכל דרש כן א״כ בערלה וכ״כ דלא כתיב לא יאכל יהי׳ מותר ליתן לכלב של הפקר כיון דלא נהנה. אבל לכאורה מטעם אחר יש איסור בדבר שהרי בתמורה (דף ל״ג) תנן ואילו הן הנשרפין וכו׳ והערלה וכ״כ וכל הנשרפין לא יקברו ע״ש הרי דאותן דמצותן בשריפה אסור לבערן בענין אחר אכן גם בזה יש מקום עיון דראיתי בתוספ׳ ר״ע על משנה זו שתימה על הברטנורא שכתב הטעם דכל הנשרפים לא יקברו דלמא חפר אינש ומשכח להו ואכל להו וז״ל לא ידעתי למאי צריך לזה הא בפשוטו הטעם דלא יקברו דהא צריך מצות שריפה וצ״ע עכ״ל אבל לענ״ד לק״מ דהנה דברי הברטנורא הם דברי רש״י במתניתן שם ורש״י בזה לשיטתו אזיל – דבשבת (דף כ״ה) בהא דאמרינן מצו׳ לשרוף תרומה שנטמאת כתב הטעם משום תקלה וכתבו התוספ׳ דלפי הטעם הזה לאו דוקא שריפה אלא ה״ה שאר ביעור וכן פי׳ בהדיא באילו עוברין (פי׳ דכתב שם [דף מ״ו] דמותר ליתן חלה שנטמאת לכלבו) ולא נראה דבסוף תמורה קתני תרומה טמאה בהדי הנשרפין וקתני התם הנשרפין לא יקברו עכ״ל התוספ׳ ונראה דלא קשה קושיא זו דהא פי׳ רש״י באמת דהטעם דהנשרפין לא יקברו הוא משום תקלה ולא שיש מצוה בשריפה ולכן א״ש ג״כ דהוצרך רש״י לכתוב הטעם משום תקלה ולא משום דמצו׳ בשריפה דא״כ יקשה מתרומה טמאה וביותר נ״ל דלא בלבד בתרומה טמאה אמרינן כן דאין מצותו בשריפה דוקא אלא גם בערלה וכ״כ דדוקא בקדשים שנפסלו דכתיב איסור אכילה והנאה וכתיב באש תשרף בזה הוי ע׳ לשרוף והוי השריפה מצו׳ אבל בערלה וכ״כ דלא כתיב רק פן תוקד אש לאסור אכילה והנאה לא אמרינן דשריפה הוי מצו׳ דהוראת הקרא אינו רק לאסור בהנאה ולא שיהי׳ צריך שריפה דוקא וראי׳ לזה ממה דאמרינן במתניתן דתמורה הנ״ל את שדרכן לשרוף ישרוף ואת שדרכן להקבר יקבר ופי׳ רש״י אערלה וכ״כ קאי אוכלין ישרפו ומשקין יקברו עכ״ל וכ״כ גם התוס׳ בשבת (דף כ״ה) והשתא אי ס״ד דיש מצו׳ בשריפה דוקא למה לא אמרינן דגם משקים ישרפו כדאמרינן לענין נסכים שנתותרו שהם בשריפה בסוכה (דף מ״ט) ויליף מבקדש הסך נסך משמע אפילו הוא יין שאינו קרוש וכן אמרינן ביין שמזלפו ע״ג האישים אלמא דמשכחת שריפה גם במשקים אע״כ דאין בהם מצו׳ בשריפה דוקא והא דאמרינן לא יקברו הטעם דלא ליתי׳ לידי תקלה דלמא משכח אינש וזה לא שייך במשקים כששופכן ע״ג קרקע ונבלעו בה.\\\vspace{0pt}

אבל נלענ״ד דזה דוקא בערלה וכ״כ אבל בחמץ בפסח אליבי׳ דר׳ יהודה דיליף שריפה מנותר השריפה בעצמה היא מצו׳ כמו בנותר וגם המשקים צריך לשרוף דוקא ובזה מדוייק מה שכתב רש״י הנ״ל דאת שדרכן להקבר יקבר אערלה וכ״כ קאי וכן כתבו גם התוספ׳ בשבת ולמה שבקו חמץ בפסח דקתני ברישא ונקטו רק ערלה וכ״כ אלא ודאי דבחמץ בפסח לר׳ יהודה גם משקים בעי שריפה כמו בנכסים שנתותרו ושנטמאו ובזה מדוייק גם לשון המשנה בתמורה דקתני ואילו הן הנשרפין חמץ בפסח ישרף ותרומה טמאה וערלה וכ״כ את שדרכן וכו׳ ולמה חלק חמץ בפסח לעצמו לשנות בו ישרף אבל לפ״ז א״ש דנקט כן לאשמועינן דבחב״פ לעולם ישרף אבל בערלה וכ״כ דוקא באת שדרכו להשרף ומזה נראה לענ״ד להשיב אמה שכתב הגאון הרב מ״ה משה סופר זצ״ל בתשובה (ראיתי׳ מכ״י הקדושה \textit{וכעת נדפסה בשו״ת ח״ס חא״ח סי׳ ק״ד}) דלפי משכ׳ הרע״ב דכל הנשרפין לא יקברו משום דחיישינן דלמא ימצאם אדם ויהנה בהם א״כ שריפה לאו דוקא אלא שיעשו אפר או עפר וכל שנתמקמק וכלה מן העולם נתקיים מצותו וה״ה ג״כ זריעה מותר בחמץ בדבר שזרעו כלה עכ״ד ולענ״ד זה אינו דרש״י והרע״ב לא כתבו כן רק משום תרומה טמאה וערלה וכ״כ אבל חמץ דמצותו בשריפה בעינן שריפה דוקא כמו בנותר ותמהתי על הגאון זצ״ל איך ס״ד דכל שכלה מן העולם הוי כמו שריפה לר׳ יהודה הרי ליכא כילוי מן העולם יותר מהאכיל לכלבו וא״כ איך דייק רבא בפסחים (דף ה׳) דס״ל לר״ע אין ביעור חמץ אלא שריפה ופי׳ רש״י דאי השבתתו בכל דבר לוקמי׳ בי״ט ויבערנו בדבר אחר שיאכילנו לכלבים ע״ש הרי גם לפי מה דס״ל אין ביעור חמץ אלא שריפה ג״כ יקשה יאכילנו לכלבים ואיך קאמר ואי בי״ט מי שרי אלא ודאי ברור דבדבר שמצותו בשריפה בעי שריפה דוקא ומה שכתבו רש״י והרע״ב דהנשרפין לא יקברו משום תקלה שיבא לאכלו לא כתבו רק משום תרומה טמאה וערלה וכ״כ דבהם לא צותה התורה שצריך לשרפם וששריפתם מצו׳ ולכן נלענ״ד שכמו שבתרומה טמאה דחשיב בהדי נשרפין כתב רש״י בפי׳ שמותר ליתן לכלב ה״ה ג״כ בערלה וכ״כ רק שבתרומה שמותרת בהנאה מותר ליתן לכלבו ובערלה וכ״כ דוקא לכלבים של הפקר ומה דכתיב לשון שריפה בכ״כ וילפינן ערלה מני׳ אין זה רק להקל שיכול לשרפו אף שזה אפרו מותר מה שאינו כן בשאר אסורי הנאה שלא התירה התורה בדבר אחר דליכא למיחש לתקלה כגון להאכילו לכלבים לבער בדבר אחר דליכא למיחש לתקלה כגון להאכילו לכלבים שפיר דמי ודע שמדברי רש״י בפסחים הנ״ל מוכח דס״ל דאפילו חמץ בפסח מותר ליתן לכלבים של הפקר לרבנן דס״ל דהשבתתו בכל דבר ודלא כהירושלמי הנ״ל וכ״כ ג״כ בפני יהושע שם וא״ל דס״ל לרש״י כיון דהירושלמי יליף לי׳ מלשון לא יאכל בצירי א״כ לא שייך זה רק לר׳ אבו׳ דלא צריך לאיסור הנאה כיון דגם לא יאכל בחולם משמע כן אבל לחזקי׳ צריך לשון לא יאכל לאיסור הנאה וליכא דרשה לאסור לכלבים וא״כ מה שכתב רש״י להאכילו לכלבים אליבי׳ דחזקי׳ כתב כן דז״א דהא אליבי׳ דר״ע כתב כן ור׳ עקיבא ע״כ כר׳ אבו׳ ס״ל כדאמרינן בפסחים (דף כ״ג) דאל״כ הוי כתנאי ועוד דע״כ ליכא פלוגתא בין חזקי׳ לר״א בזה דאל״כ כי פריך שם מאי איכא בין חזקי׳ לר״א לימא דהא איכא בינייהו אלא ודאי משמע מדברי רש״י דלא פסק כהך ירושלמי אבל בערלה וכ״כ אפילו להירושלמי מותר כן נלענ״ד. הקטן יעקב.\\\vspace{0pt}

\end{multicols}\newpage

\newchap{סימן קב}
\begin{multicols}{2}
ב״ה אלטאנא, יום ג׳ י״ט סיון תר״ח לפ״ק. להרה״ג וכו׳ מ״ה מיכל מונק הכהן נ״י הגאב״ד דק״ק דאנציג ומ״ב יע״א.\\\vspace{0pt}

מעכ״ת נ״י השיג על פסקי שמותר ליתן ערלה וכ״כ לכלבים של הפקר וז״ל: על דברי מעכ״ת נ״י בענין נתינת ערלה וכ״כ לכלבים של הפקר מצאתי מקום להשיג ואציע פה בבקשה ממעכ״ת נ״י להראות לי דרך האמת והצדק. (א) מה שעלה על דעתו לחלק בין שריפת חמץ לערלה וכ״כ הואיל בחמץ כתיב לא יאכל ולא כן בערלה וכ״כ זאת פליאה נשגבה בעיני הלא בערלה כתיב ג״כ בפ׳ קדושים יהי׳ לכם ערלים לא יאכל? (ב) מה שרצה לומר דמה שכתבה התורה בכ״כ פן תוקד אינו למצות שריפה רק התורה שריפה בהם אבל מכ״מ גם בערלה וכ״כ אם רצה לבער בדבר אחר דליכא למיחש לתקלה כגון להאכילו לכלבים להראות איסור אכילה והנאה כמו בשאר איסורים ואין לומר להורות לנו שלוקין עליהן אפילו שלא כדרך הנאתן כדברי אביי׳ פסחים (דף כ״ד ע״ב) זה אינו דהא אמרינן כמה וכמה פעמים בשס׳ ערלה הוקשה לכ״כ ונראה לענין מאי צריכין להקיש לאיסור אכילה והנאה מפורש בערלה בפסוק לא יאכל ושלוקין על ערלה שלא כדרך הנאה אי אפשר דהא בערלה כתבה התורה לשון אכילה אלא ודאי לשריפה הוקשה כמו בכ״כ צריך שריפה דוקא כן בערלה צריך שריפה דוקא וא״כ מוכח דבכ״כ וערלה צריך דוקא שריפה ולכן הקשה בתוספ׳ ר״ע שפיר למה צריך הטעם דלמא חפר אינש ומשכח לי׳ תיפוק לי׳ דהא צריך לקיים מצות שריפה? (ג) ושלישי אני בא מה שכתב מעכ״ת נ״י על רש״י בפסחים (דף ה׳) שכתב דאי השבתתו בכל דבר יאכילנו לכלבים דלסברת הגאון רמ״ס זצ״ל דביעור של כילוי לגמרי בכל דבר מהני א״כ גם למ״ד אין ביעור אלא שריפה אכתי יקשה יאכילנו לכלבים ומזה סלל נתיב להלוך לדרכו – לענ״ד אין כאן שום קושיא לרש״י כי רש״י כתב אלה הדברים לשיטתו בביצה (דף כ״ז) שכתב שם כל מקום שצותה התורה שריפה אחשבה להבערתן הלכך מלאכה היא ואסור לתת לכלבו ולכן כתב רש״י שפיר דאי השבתתו בכל דבר וא״כ לא הקפידה התורה ומותר לבער ע״י נתינה לכלב משא״כ אם שריפה דוקא בעי׳ ע״ש: עכ״ד.\\\vspace{0pt}

על זה אשיב: מה שהשיג מעכ״ת נ״י על דברי שחלקתי בין חמץ לערלה וכ״כ שהירושלמי יליף איסור כלבים של הפקר בחמץ מלא יאכל מה שלא שייך בערלה וכ״כ שהרי בערלה כתיב ג״כ לא יאכל בצירי צדק בדבריו ואגב ריהטא לא ביארתי דעתי כראוי דזה לשון הירושלמי הביא הטור לא יאכל חמץ אפילו לכלבים מה אנן קיימין אם לכלבו היינו איסור הנאה אלא כי אנן קיימין אפילו לכלב אחרים זאת אומרת שאסור להאכילן לבהמת הפקר ופי׳ המפרש היינו איסור הנאה דשמעינן מלא תאכל חמץ אלא כי אצטריך קרא דלא יאכל חמץ לאסור להאכיל אפילו לכלבים אחרים עכ״ל הרי מבואר בהירושלמי דרק אם כבר ידעינן איסור הנאה ממקום אחר אז מוקמינן לא יאכל לכלבים של הפקר אף שאינו נהנה ולא בלבד מהירושלמי אלא גם מש״ס שלנו יש להוכיח כן דמלא יאכל לבד בלא שכבר ידענו איסור הנאה ממקום אחר ליכא למילף כלבים של הפקר שהרי בב״ק (דף מ״א) אהא דמקשה שם אימא לא יאכל את בשרו קאי אהיכא דסקלי מיסקל שאסור בהנאה מתרץ א״כ לכתוב קרא לא יהנה ע״ש ולכאורה מאי משני דלמא לכך כתיב לא יאכל ולא כתיב לא יהנה לאסור אפילו ליתן לכלבים של הפקר שאינו נהנה אע״כ דהיכי דליכא יתורא גם מלא יאכל לא ילפינן רק הנאה ולא כלבים של הפקר וזהו מה שכתבתי דבערלה וכ״כ לא כתיב לא יאכל לאסור כלבים של הפקר כמו גבי חמץ דבכ״כ לא כתיב לא יאכל כלל ובערלה לא מיותר דצריך לגופא ואין להשיב שהרי גם בערלה כבר שמענו איסור הנאה בלא לא יאכל כדאמרינן בפסחים (דף כ״ב) מניין שלא יהנה וכו׳ ת״ל וערלתם וכו׳ ע״ש דזה אינו דהא צריך לא יאכל ללאו למלקות דמוערלתם ליכא רק ע׳ ולא מבעי׳ לשיטת התוספ׳ דלוקין על הנאה כמו שהוכיח המשנה למלך בה׳ יסודי התורה (פ׳ ה׳) אלא אפילו לשיטת הרמב״ם דאין לוקין על הנאה כמו שכתב ה׳ מאכלות אסורות (פ׳ ח׳) מכ״מ צריך לא יאכל למלקות דאכילה. וגם אין לומר דכיון דפסק הרמב״ם כר׳ אבו׳ (שם) דגם לא יאכל משמע איסור הנאה א״כ זה גופא חשיב יתורא מה דכתיב לא יאכל בצירי ולא בחולם דזה אינו דאף דהגמרא בפסחים (דף כ״ג) פריך כזה דלחזקי׳ לכתוב לא יאכל גבי שרצים ולא בעי לכם היינו דוקא משום דאז לא בעי׳ לכם אבל בלא זה לא חשיב יתורא מה דכתיב בצירי ולא בחולם תדע דאל״כ כי משני לחזקי׳ טעמא דידי נמי מהכא אכתי יקשה לר׳ אבו׳ למה כתיב בצירי ולא בחולם וכן יקשה לר״א גבי שור הנסקל למה כתיב לא יאכל בצירי ולא בחולם דאין לומר דבאמת גם בשור הנסקל אתי לאסור כלבים של הפקר דהרי לפ״ז יקשה מה שהערנו לעיל איך משני לכתוב קרא לא יהנה הא בלא״ה צ״ל דאתי לאסור נתינה לכלבים ש״ה אע״כ דזה לא חשיב יתור לר׳ אבו׳ וכש״כ לחזקי׳ דלדידי׳ צריך לאיסור הנאה ולכן גבי ערלה ליכא יתורא אשר על כן נלפענ״ד נכון מה שכתבתי דדוקא גבי חמץ דמיותר לא יאכל דרשינן מני׳ כלבים של הפקר משא״כ גבי ערלה וכ״כ.\\\vspace{0pt}

ומתוך הדברים האלה נתברר לי למה גדולי הפוסקים הרי״ף והרמב״ם והרא״ש לא הביאו דין דהירושלמי דאסור ליתן חמץ לכלבים של הפקר ורש״י כתב בפי׳ נגד הירושלמי שמותר ליתן חמץ לכלבים של הפקר כמו שכתבתי למעלה וטעמם נ״ל כיון שלפי המבואר לא יליף הירושלמי כן רק מיתורא דלאו דלא יאכל חמץ וזהו לכאורה דלא כהשס׳ דילן דבפסחים (דף כ״ח) מצרכינן לר׳ יהודה תלתא קראי אלפני זמנו ותוך זמנו ולאחר זמנו ולר״ש דלא דרש להכי דרש לא יאכל חמץ לכדר׳ יוסי הגלילי ע״ש וא״כ לכ״ע לא יאכל לא מיותר לכלבים של הפקר וכן ראיתי בפני יהושע בפירושו שם (דף כ״א) שכתב ג״כ דמ״ד דירושלמי דיליף מלא יאכל דמיותר לדידי׳ לגמרי ע״כ לא ס״ל כר׳ יהודה ע״ש ולא ידעתי למה לא העיר דגם כריה״ג לא ס״ל ע״פ שס׳ שלנו וכיון דלפ״ז השס׳ שלנו חולק על הירושלמי פסקו כשס׳ שלנו והשמיטו הדין של הירושלמי וכש״כ דא״ש להרי״ף והרא״ש דפסקו כחזקי׳ וא״כ צריך לא יאכל לא״ה ולא ידעתי למה הביאוהו הגהת אשרי והטוש״ע. ודברי הג״א בשם הגהת מיימוני בלא״ה תמוהין לי שכתב ריב״ק הי׳ אוסר להשליך חמץ במקום הפקר דחזינן בתמורה כל הני איסורי הנאה מצריך להו קבורה ומייתי ראי׳ מירושלמי לא יאכל חמץ אפילו לכלבים במה אנן קיימין אי לכלבו היינו הנאה אלא אפילו לכלבים אחרים עכ״ל והנה פתח באיסור להשליך במקום הפקר מטעם דאמרינן בתמורה דכל א״ה מצריך קבורה דזהו מטעם שמא יבא לאכול ולפ״ז לא שייך זה רק במשליכו במקום הפקר והולך לו אבל באוכל הכלב לפניו לא שייך איסור שהרי זה עדיף מקבורה ואיך מייתי על זה מירושלמי שהרי לדברי הירושלמי אפילו להאכיל לכלבים אסור מטעם דרשה דלא יאכל וצ״ל דהוא מפרש להירושלמי דמלא יאכל קדריש שלא ישליך לפני כלבים ש״ה וילך לו שמא ימצאנו אחר ויבא לידי אכילה ולפ״ז אם עומד אצלו עד שיאכל באמת מותר ואפשר דגם רש״י ס״ל כן ולכן התיר להאכיל חמץ לכלבים של הפקר אכן מדברי טוש״ע לא משמע כן אלא דאפילו באוכלין לפניו אסור.\\\vspace{0pt}

והיוצא מזה דעכ״פ לכל השיטות בערלה וכ״כ דליכא יתורא דלא יאכל לאסור כלבים של הפקר מותר להאכילם רק בתנאי שיעמוד אצלם עד שיאכלו. ולפ״ז גם בבשר וחלב יהי הדין כן אלא שראיתי בשו״ת תשב״ץ חלק ג׳ סי׳ רצ״ג שאוסר ליתן כל איסורי הנאה לכלבים ש״ה וכן באו״ה כלל כ״א הי״ב הביאו הט״ז סי׳ צ״ד אוסר ליתן בשר בחלב לכלבים ש״ה ולא ידעתי למה כי לפענ״ד בכל איסורי הנאה חוץ מחמץ לכל השיטות יהי׳ מותר ליתן לפני כש״ה כשיאכלו לפניו.\\\vspace{0pt}

עוד השיג מעכ״ת נ״י על מה שכתבתי דפן תוקד אינו למצות שריפה רק לאסור בהנאה וז״ל דהא אמרינן כמה וכמה פעמים בשס׳ ערלה הוקשה לכ״כ ונראה למאי צריכין להקיש וכו׳ אלא ודאי לשריפה הוקשה כמו ב״ככ צריך שריפה כן בערלה צריך שריפה דוקא עכ״ל ותמהתי שמעכ״ת נ״י כתב דאמרינן כמה וכמה פעמים בשס׳ דהוקשה ערלה לכ״כ ואני לא ידעתי אפילו פעם אחת דאמרינן כן וגם לא ידעתי איך שייך היקש אם לא בשני דברים דהוקשו בפסוק אחד או בסמוכים והרי ערלה וכ״כ כתובים רחוקים זה מזה ומה דילפינן ערלה דבשריפה מכלאי כרם כמו שכתב רש״י בתמורה (דף ל״ג) וערלה מכלאים גמרה היינו במה מצינו משום דדמי להדדי דשניהם גדולי קרקע ואסורים באכילה ובהנאה אבל לא שיש היקש ואף שהתוספ׳ שם כתבו בשם רש״י משום דאתקש לכלאי כרם כבר תימה התוספ׳ י״ט על זה דלא הודיעו מהו ההיקש ולכן פירש ג״כ דרש״י דקדק לשון גמר דלאו היקשא הוא ע״ש וא״כ פשיטא דליכא יתורא ללמוד מזה דמצו׳ בשריפה. ומה שהשיג מעכ״ת נ״י על מה שרציתי להוכיח נגד דעת הגאון רמ״ס זצ״ל דלמ״ד חמץ בשריפה אסור לבער בדבר אחר של כילוי דאל״כ אכתי יקשה גם לפי מה דס״ל ר״ע אין ביעור חמץ אלא שריפה דיתן לכלבו דרש״י לשיטתו קאי דס״ל דרחמנא אחשבי׳ להבערתו וזה לא שייך רק אי בעור חמץ מצותו בשריפה – על זה אשיב הן אמת שהפני יהושע כתב כסברא זו ליישב סתירת רש״י מה דכתב הכא יאכילנו לכלבים ושם בביצה כתב שאסור ליתן לכלבו אבל בשו״ת נודע ביהודה דחה סברא זו והוכיח דלא שייך רחמנא אחשבי׳ להבערתו רק לענין שריפת תרומה וקדשים ולא לענין חמץ ע״ש אכן גם אי נימא כסברת הפ״י אכתי קאי ראייתי אליבי׳ שיטת התוספת דס״ל בביצה (דף כ״ז) דאין איסור ליתן לכלב בי״ט רק משום דמצותו בשריפה ולכן לשיטתם שפיר מוכח נגד דעת הגאון רמ״ס ז״ל דאל״כ היאך מוכיח רבא דאין ביעור חמץ אלא שריפה הרי גם לפ״ז אכתי יקשה לר״ע הלא בי״ט יכול ליתן לכלב דאין בזה מלאכה ומצות ביעור קיים אע״כ דלמ״ד אב״ח אלא שריפה דוקא שריפה בעי כנלענ״ד הקטן יעקב\\\vspace{0pt}

\end{multicols}\newpage

\newchap{סימן קג}
\begin{multicols}{2}
עוד השיגו על פסקי הנ״ל (סי׳ ק״א) הרה״ג וכו׳ מ״ה יעקב קאפל ב״ב הלוי נ״י הגאב״ד דק״ק ווארמס יע״א והרה״ג וכו׳ מ״ה גבריאל אדלער הכהן נ״י הגאב״ד דק״ק אבערנדארף יע״א. ואעתיק דבריהם עם מה שנלענ״ד להשיב עליהם.\\\vspace{0pt}

הרב הגאב״ד דק״ק ווארמס נ״י כתב: מעכ״ת נ״י השיג על המג״א סי׳ תמ״ח ס״ק ט׳ ובודאי שהמג״א קיצר במקום שהי׳ לו לבאר אבל בשו״ת רשב״ץ (ח״ג סי׳ רצ״ג) הביא ראי׳ מן הירושלמי ר״פ כל שעה דכל איסור הנאה אסור ליתנו לכלב של הפקר (והכנה״ג חשב שהוא מהש״ס בבלי וכבר העיר עליו זה בס׳ יד דוד) מוכח דס״ל דלאו דוקא חמץ בפסח והוא אותו ירושלמי שהביאו בטור ומג״א והנה ס׳ שו״ת הרשב״ץ לא נדפס עדיין בימי המג״א אבל כיון לדעתו והנה אף שמפשטות דברי הירושלמי לא שמענו רק היכא דכתיב לא יאכל בצירי אבל הנה הלבוש סי׳ תמ״ח כ׳ וז״ל ואסמכוה רבנן אקרא כו׳ דעכ״פ יש לו הנאה ממה שמאכיל ומשביע הבהמה ואפי׳ היא הפקר עכ״ל וס״ל להרשב״ץ והמג״א דמטעם זה יש להחמיר בכל איסורי הנאה רק דבחמץ בפסח אסמכוה אקרא כו׳ (ונראה מדברי הלבוש דסברא דאורייתא קאמר) וכ״כ באו״ה כלל כ״א הי״ב וז״ל: ובירושלמי אוסר להאכיל כו׳ אפי׳ לכלב הפקר עכ״ל וכן תבשיל שנאסר מבשר וחלב דאורייתא כו׳ יע״ש ובהי״ג וכ״כ בש״ד כלל פ״א וכ״כ בספר אפי רברבי שער ס״ח הי״א ומהרש״ל ביש״ש וכ״כ הפר״ח יו״ד סי׳ צ״ד סק״י וז״ל: ומיהו כתב האו״ה כו׳ ושאר אחרונים ז״ל כו׳ ואיסורי הנאה אסור להאכילן אפי׳ לבהמת הפקר כו׳ והנה גם רמ״א בתו״ח כו״ס פ״ה ה״ג הביא ג״כ דין זה בבב״ח בקיצור עכ״ד הגאב״ד דק״ק ווארמס נ״י הנוגעים לעניננו:\\\vspace{0pt}

וז״ל הרב הגאב״ד דק״ק אבערדארף נ״י – הגיע לידי פסק שיצא ממעכ״ת נ״י בעיון ובסברא באיסור הנאה דס״ל דמותר לתתו לפני כלבים ועופות של הפקר דלא הוי הנאה ודוקא בחמץ בפסח קימל״ן כהירושלמי דאסור מדכתיב לא יאכל והנה אחוה דעי כי בספק הזה כבר פליגי רבוותא דקמן התשב״ץ עם בעל כנה״ג ומה אדע אשר לא ידע א״ה הרב הגדול ואולם וכי כעורה זו ששנה התשב״ץ דס״ל דבכל אסורי הנאה אסור לתתו לכלב של הפקר וגם ראיתי כי בעל כנה״ג בי״ד הל׳ טרפות סי׳ כ״ט העתיק תשו׳ כ״י של התשב״ץ והשיג עליו כדבעינן למימר לקמן ואפילו לפרושו כוונת התשב״ץ בדפוס ח״ג סי׳ רצ״ג מן הירושלמי פרק כל שעה דמוכח דאסור לכלב של הפקר וכן משמע מהרא״ש בחולין פרק ג״ה על מתני׳ שולח אדם ירך לנכרי בענין ג״ה אי מותר בהנאה דכ׳ וכן ראיתי באשכנז שמוכרין ג״ה לנכרים ובשאר ארצות ראיתי שנזהרו אף להאכילו לחתול עכ״ל וע״כ איירי בחתול של הפקר דאי בחי׳ שלו מאי אף איכא דהא הוי לי הנאה מני׳ אע״כ שנזהרו להאכילו אפילו לשל הפקר משום דהוי הנאה. גם במ״ש מעכ״ת נ״י לחלק בין חמץ לערלה דדוקא חמץ אסור מטעם הירושלמי דכתיב לא יאכל בצירי משא״כ בערלה וכ״כ וכו׳ מקום שאמרו להאריך אאר״ל דהרי גם בערלה כתיב לא יאכל בצירי? ואולי כוונתו כיון דילפינן א״ה בערלה מן ערלתם ערלתו כדאמרי׳ במס׳ פסחים (ד׳ כ״ב ע״ב) ומש״כ אינו דומה לחמץ. אמנם לדינא דערלה ל״צ שריפה גם אאמ״ו הה״ג זצ״ל כ״כ בשו״ת א נדפס על ידי בספר לשון זהב ח״א ד׳ כ״ז ע״ב וכו׳ וראיתו מהא דכ׳ הראב״ד הובא בריטבא במס׳ סוכה באתרוג של ערלה פסול מדפליגי אמוראי בטעמא דאין בו היתר אכילה או דאין בו דין ממון ול״ק טעמא משום דכתותי מ״ש כיון דלשריפה קאי ע״כ מדפלגנהו מתני׳ ערלה ותרומה ותני פסול פסול ולא כללנהו עם אשירה ש״מ דמלבד טעם דאשירה יש בו טעם אחר ומטעמא דאמרי׳ בגמרא וכמ״ש רש״י במתני׳ וז״ל ובגמרא מפרש טעמא (ולא משום כמ״ש) ומש״כ נשמט בי״ד סי׳ רצ״ד דין דשריפת ערלה אבל בכלאי כרם כ׳ הש״ע בסי׳ רצ״ו ושורפין אותם וכסתמא דמתני׳ שלהי תמורה עכ״ל ואולם במאי דמסיק מעכ״ת נ״י למסקנא אבל בערלה וכ״כ אפילו לירושלמי מותר אם אמנם כח דהתירא עדיף מפשטות לישנא דגמרא פסחים בסוגי׳ דחזקי׳ ור״א לא משמע כן דהרי אמרי׳ אותו אתה משליך לכלב וא״א משליך לכלב כ״א שבתורה וכ׳ רש״י משום דאין הקב״ה מקפח שכר כל ברי׳ איצטרך אותו ע״ש ומשמע דלאו בכלבים שמזונתן עליו איירי דא״כ הרי מוטל עליו לפרנסם וא״כ בשאר איסורי הנאה אסור להשליך אף לכלבים של הפקר וכמ״ש הפ״י ע״ד רש״י שם ריש ד׳ כ״ב וכ״כ הט״ז מפורש בי״ד סי׳ צ״ד ס״ק ד׳ גבי בו״ח וז״ל אבל המאכל שנאסר כתוב באו״ה ורש״ל ות״ח שצריך להשליכו דוקא לבה״כ אבל לא לפני הכלב אפילו אין הכלב שלו עכ״ל ובלי ספק אין שום חילוק בין בו״ח לשאר אסורי הנאה. והשתא מה לן לעזוב האחרונים הנ״ל שאנו נגררים אחריהם בכל מקום ולהביא ראי׳ מדברי רש״י שכ׳ בחמץ דתשביתו מקיים ג״כ בתתו לכלבים של הפקר והיא בעצם קושי׳ הפ״י ואולם הרי רש״י ד׳ ה׳ כתב דאי השבתתו בכל דבר יאכילנו לכלבים ולא מוזכר הפקר או ישליכנו לים וע״כ כיון דתורה אפקי׳ חמץ בלשון תשביתו א״כ בכל מידי דמבער החמץ יצא וכעין מ״ש הבה״מ באכילת חמץ בע״פ הוי השבתה ואם דלא קימל״ן כוותי׳ מ״מ לקס״ד ס״ל דהוי השבתה בתתו לכלבים ובעל כנה״ג הביא ראי׳ ממכילתא דצריך לק״ו מנבילה דטריפה מותרת בהנאה אלמא דלא הוי היתר הנאה בכלב של הפקר ולענ״ד אין זה ראי׳ דאי לא הוי לן ק״ו מנבילה ה״א דטריפה אסורה בהנאה ודוקא לכלב של הפקר מותר וכמו בתרומה טמאה דמותר להנות ממנו בשעת הדלקה ה״נ הגם אי הוי טריפה אסורה בהנאה מ״מ משום דאין הקב״ה מקפח שכר כ״ב מותר לתתו לכלב ש״ה. גם מה שהשיב על התשב״ץ שהביא ראי׳ מגמרא דפסחים הנ״ל וכ׳ דאותו אתה משליך לכלב לא קאי על כלב לבד ע״ש גם אם נדחק לפרש כך מ״מ בתשובתו בדפוס כ״כ בשם הירושלמי רפ״ב דפסחים ואפשר כוונתו על חמץ הנ״ל גם ראיתי בס׳ באר יעקב בי״ד סי׳ כ״ט שהביא פלוגתת הכנה״ג עם התשב״ץ הנ״ל והביא ראי׳ בשם הגאון מהר״ר טעבלי שייאר זצ״ל לתשב״ץ דאי אמרת דנתינה לכלב של הפקר לא הוי הנאה כמ״ש בעל כנה״ג ממכילתא קשי׳ לר׳ אבו דיליף איסור הנאה מדאיצטרך קרא להתיר בנבילה הא אי לא כתיב קרא הו״א דנבלה אסור׳ בהנאה מק״ו מטריפה דאינו מטמא ואפ״ה דוקא לכלב של הפקר מותר ובהנאה אסורה ומש״כ איצטרך קרא בנבילה דמותר בהנאה וה״ה לכל אסורי אכילה הו״א דמותרים בהנאה אע״כ דגם לכלב של הפקר הוי הנאה א״כ שפיר ילפינן מנבילה והגם דבעל באר יעקב הנ״ל דחק לתרץ קושי׳ זו מ״מ לדידי קשה קושי׳ אחרת דע״כ מוכח כתשב״ץ דאי אמרת כבעל כנה״ג דנתינה לכלב של הפקר לא הוי הנאה א״כ איך אמרינן בגמרא במס׳ ב״ק ד׳ מ״א ת״ר ממשמע שנאמר סקל יסקל השור אינו יודע שנבילה היא וכו׳ ופריך אימא האי לא יאכל להיכא דסקלי׳ מסקל דאסור בהנאה כדר״א וכו׳ אמרי ה״מ היכא דנפיק איסור אכילה וא״ה מחד קרא אבל הכא נכתוב רחמנא לא יהנה ע״ש והשתא אי אמרת דלכלב של הפקר לא הוי הנאה א״כ דילמא לעולם אתי׳ קרא דאסור בהנאה להיכא דסקלי׳ ומש״כ לא כתיב לא יהנה דא״כ הו״א לכלב הפקר מותר ומש״כ כתיב לא יאכל בצרי דאז אסור אפילו לכלב של הפקר כמ״ש בירושלמי בפ״ב דפסחים וכמ״ש הפוסקים בא״ח סי׳ תמ״ח אע״כ דגם נתינה לכלב של הפקר נמי הוי הנאה א״כ שפיר משני הש״ס לכתוב לא יהנה דהוי כל הנאות בכלל וא״כ מוכח מפורש כהתשב״ץ וכמ״ש האחרונים דבכ״מ דאסור בהנאה גם נתינה לכלב של הפקר אסור דלא כבעל כנה״ג וסיעתו וצ״ע ע״כ דברי הרב הגאב״ד דק״ק אבערדארף נ״י.\\\vspace{0pt}

ועל זה אשיב – מעלת כבוד הרבנים הנ״ל נ״י העירוני על מה שכתבתי להתיר ליתן ערלה וכ״כ לפני כלבים ש״ה שהאו״ה ושאר אחרונים כתבו שאסור ליתן איסור הנאה לפני כש״ה וכן העירני הגאב״ד דק״ק יערגען נ״י והנה לא ראו אז מה שכתבתי (סי׳ ק״ב) שהבאתי כבר שם שמצאתי בשם או״ה לאיסור אבל לא ידעתי למה אכן צדקו הרבנים נ״י בזה שאין בידינו לחלוק על הפוסקים אפילו לא מסתבר לנו טעמם ובאמת בכוונה לא כתבתי כן רק על ערלה וכ״כ מה דלא שכיח אצלינו ולא על בב״ח למען יהי׳ רק להלכה ולא למעשה אבל לא חשכתי להודיע מה שנלענ״ד בעיקר הדין למען יהי׳ נפקותא עכ״פ בשעת הדחק (כמו שהתיר באמת בשו״ת פני יהושע ליתן אפילו חמץ בפסח לפני כלבים ש״ה בשעת הדחק) ולכן הראתי שבשס׳ בבלי לא נזכר האיסור שמה שהביא הבאר יעקב ראי׳ לאיסור מהא דאותו אתה משליך לכלב כבר מיושב במה שכתב הפני יהושע בפסחים (ד׳ כ״ב) ע״ש ומה שהביא הרב דק״ק אבערדארף נ״י ראי׳ ממה דאמרינן בב״ק א״כ ליכתוב קרא לא יהנה לא ראה עדיין בכתבו כן שכבר הזכרתי ראי׳ זו (סי׳ ק״ב) ואדרבא הבאתי מזה ראי׳ להיפך דרק דהיכי דמיותר לא יאכל אז דרשינן מני׳ לאסור לכש״ה והירושלמי שהזכיר האיסור נראה בפי׳ מדבריו שיליף כן מלשון לא יאכל ואם שקצת האחרונים רצו לומר שדברי הירושלמי אסמכתא הם כמה מן הדוחק לומר כן שהירושלמי קאמר או אינו וכו׳ ת״ל וכו׳ ע״ש דמכל סוגית הירושלמי נראה בפי׳ דדרשה גמורה היא ומלא יאכל יליף כן וכמו שכתב גם הפ״י כאשר כבר הזכרתי. גם כבר הזכרתי שכל גדולי הראשונים הרי״ף והרמב״ם והרא״ש ושאר ראשונים לא הזכירו איסור זה שמה שהביא הרב הנ״ל נ״י ממה שכתב הרא״ש ובשאר ארצות ראיתי שנזהרו אף להאכילו לחתול לענ״ד אינו כדאי להוכיח מזה שאסרו ליתן איה״נ לכש״ה דודאי יש הנאה מוכחת יותר כשמוכר ומקבל דמים משנותן לחתול אפילו היא בביתו דבלא״ה תמצא מזונות ולכן שפיר שייך לשון אף להאכילו לחתול גם לענין חתול שבביתו ומדברי רש״י הוכחתי כמו שכתב גם הפ״י שס״ל בפי׳ דמותר ליתן אפילו חמץ לכלבים.\\\vspace{0pt}

והנה כתב אלי הגאב״ד דק״ק פעגערסהיים נ״י וז״ל מעכ״ת נ״י הבין כמו שהבין ג״כ הגאון פ״י שרש״י חולק עם הירושלמי ולענ״ד אינני רואה מדברי רש״י שום סתירה לדברי הירושלמי די״ל דרש״י ז״ל לשיטתו אזיל דפירש הא דאמר ר״י אימתי שלא בשעת ביעורו וכו׳ דמיירי קודם זמן איסורו (ועיין במהרש״א דף ד׳ על תוספ׳ ד״ה כתיב אך) וא״כ שפיר כתב רש״י ש״מ אין ביעור חא״ש דאי השבתתו בכל דבר לוקמי׳ בי״ט ויבערנו בדבר אחר יאכילנו לכלבים דודאי אי הוי מוקמינן אך ביום הראשון אי״ט אז בע״כ מקצתו יהי׳ מותר כי ההשבתה תהי׳ קודם זמן איסורו וקודם זמן איסורו יוכל להאכילו לכלבים עכ״ל ואני תמה איך אפשר לפרש כן כוונת רש״י דא״כ למאי צריך למינקט יאכילנו לכלבים הוי מצי למינקט יאכלנו בעצמו או יסיקנו תחת תבשילו ועוד דגם השתא דס״ל אין ביעור חמץ אלא שריפה יקשה הרי גם בי״ט יכול להסיקו תחת תבשילו קודם זמן איסורו אלא ודאי דבי״ט לעולם הוי לאחר זמן איסורו ואי משום קושית הרב נ״י דרש״י ס״ל דשלא בשעת ביעורו הוא קודם זמן איסורו זה כבר הקשה הפ״י ותירץ דגם רש״י לא כתב כן אחר ששמענו דראשון מעיקרא משמע ע״ש הן אמת שהוקשה לי מטעם זה ארבא שאמר ש״מ מדר״ע תלת למה לא אמר ארבע שהרי מוכח מזה ג״כ דר״ע ס״ל חמץ בפסח אסור בהנאה דאי ס״ל כריה״ג דמותר בהנאה היכי מוכח דלא איירי בי״ט הרי יכול להסיקו תחת תבשילו גם אחר זמן איסורו אבל באמת לק״מ דלא רצה רבא להשמיענו רק מה דלא שמעינן לר״ע ממקום אחר אבל הא דס״ל חב״פ אסור בהנאה לא צריך להוכיח כיון שאמר ר״ע בפי׳ כן בפסחים (דף ל״ב) ועכ״פ מוכח דעת רש״י דס״ל איסור הנאה מותר ליתן לפני כלבים של הפקר וכסתימת שאר הראשונים ולכן מה דס״ל להתשב״ץ ושערי דורא ואו״ה ואחריהם שארי אחרונים שאסור ליתן איה״נ לפני כש״ה לענ״ד הוא מלתא בלא טעמא כי אף שהלבוש כתב ג״כ טעם המג״א (כמו שהודיעני הרב דק״ק ווארמס נ״י) לא אוכל להבין איך יחשב אפילו קצת הנאה מה שנותן לכלב שאינו מכירו ואת בעליו וביותר הי׳ נ״ל שטעם הפוסקים לאסור שס״ל כהגהת אשרי שאסר בחמץ ליתן לכלב ש״ה משום תקלה אכן כבר כתבתי (סי׳ ק״ב) דלפ״ז לא יהי׳ אסור רק להשליכו במקום הפקר על סמך שיאכלוהו הכלבים אבל אם עומד אצלם עד שיאכלו יהי׳ מותר:\\\vspace{0pt}

והנה הגאב״ד דק״ק יערגען נ״י כתב אלי אחר שהזכיר ג״כ דעת הפוסקים שאסרו ליתן לפני כש״ה וז״ל אמנם טעמא דמלתא או דס״ל שיש מצו׳ בקבורה וילפינן כל איסורי הנאה מעגלה ערופה וע״ע ממת ולפ״ז פליגי עם התוספ סוף תמורה ויותר נכון לומר דס״ל כמשכ׳ הלח״מ ה׳ שחיטה (פ״ב) ויש שם ט״ס וכוונתו דקיי״ל כהאי מרבנן פסחים (ד׳ כ״ד) דיליף מלא יאכל דכתיב גבי בשר טמא על כל איסורים שבתורה וא״כ כאילו כתיב בכולהו לא יאכל ואסור ליתן לבהמת הפקר או דס״ל כמשכ׳ התשב״ץ דזה ג״כ חשוב הנאה או טובת הנאה אם יכול להנות בהמה הפקר וזה נראה דעת הרמב״ם והר״ן בנדרים (דף מ״ז) ד״ה והוי יודע וכו׳ והב״י בי״ד סי׳ רי״ו והובא בש״ך י״ד (סי׳ רכ״ג סק״ד) הקשה עליהם דהא אין מחזיקין לו טובה ולי בלא״ה הי׳ קשה הרי אין רצונו שיחזיקו לו טובה ודבר שאינו מתכווין שריא כמש״כ התוספ׳ פסחים (דף כ״ב) וכי תימא דהוי פסיק רישא הא כ׳ הר״ן בסוגיא דריחא מלתא בהנאה לא שייך פסיק רישא וא״כ יהי׳ מותר ליתן לאחיו דבר שאסר לו אביו בהנאה אמנם אי ס״ל כסברא הנ״ל א״ש דזה עצמו חשיב טובת הנאה שיכול להנות לכך כתב הרמב״ם שאינו נותן אלא מראה מקום להם והוי כמפקירו ומיושב קושית הב״י עכ״פ מבואר הסכמת הפוסקים האחרונים דכל א״ה אסור ליתן לכלב הפקר וכ״כ בחת״ם סופר סי׳ של״ו עכ״ל ולא אאריך במה שלענ״ד יש להשיב על זה רק אזכיר בקצרה שאם כדברי הרב נ״י דהפוסקים ס״ל כההוא מרבנן וכאילו כתיב לא יאכל בכל איסורים שבתורה איך פסק הרמב״ם ושאר הפוסקים דבכלאי כרם וב״בח חייב אפילו שלא כדרך הנאתן משום דלא כתיב בהו אכילה ומאי שנא דלחומרא אמרינן דלא יאכל קאי עליהם לענין כש״ה ולא ג״כ לקולא לענין שלא כד״ה וגם מה שהביא הרב נ״י ראי׳ לסברת התשב״ץ מהא דנודר הנאה לענ״ד אינו דומה כלל דאפילו יהי׳ כסברת הרב נ״י שם דחשיב זה לו הנאה מה שמהנה היינו שמהנה בנו או אחיו ואפי׳ אינן מחזיקים לו טובה מכ״מ נהנה במה שנהנים החביבים לו מנכסי אביו אבל מה הנאה יש לו אם נהנה כלב של הפקר מנכסיו ואעיד אלי עוד עד נאמן לדברי שהרבינו יונה כתב באגרת התשובה כל דבר שאסור בהנאה אסור למוכרו וכו׳ ואסור לתנו לחתול או לכלב שמזונותיהם עליו עכ״ל הרי שהתנה בפי׳ לאסור דמזונותיהם עליו דוקא לאפוקי של הפקר דמותר לכן בהא נחתי שלא מצאתי טעם מספיק לפסק האחרונים בזה ושיטת הראשונים ומשמעות הגמר׳ נגדם ואם שאין אני כדאי לחלוק עליהם מכ״מ בשעת הדחק יש סמך רב להתיר ליתן איסורי הנאה לכלבים ועופות של הפקר ובפרט כשאי אפשר לבערם ע״י שריפה או קבורה לענ״ד טוב להאכילם לכש״ה מלהניחם שאפשר לבא לידי תקלה כנלענ״ד הקטן יעקב.\\\vspace{0pt}

\end{multicols}\newpage

\newchap{סימן קד}
\begin{multicols}{2}
ב״ה אלטאנא, יום ג׳ ג׳ דחנוכה תרכ״ד לפ״ק. לחתני הרה״ג וכו׳ מ״ה ישראל מאיר פריימאן נ״י אב״ד דק״ק פילעהנע יע״א.\\\vspace{0pt}

שאלת באשר נודע לך בשני בכורים שלא נפדו והאחד כבר הוא בן י״ח שנים והאחד בן עשרה שנים ועל שאלתך למה נהי׳ כזה בישראל הוגד לך שהם בני פנויות שילדו ונתעברו מן נכרים ואין איש שם על לב ופלפלת בחכמה אם חייבים בפדיון מדחיישינן לענין כהונה כשאומרת מישראל נתעברה מדאפקרא לגבי האי אפקרא ג״כ לאחריני כמו כן נימא לענין ממון שמא מכהן או מלוי נתעברה ואז אין צריך פדיון ואע״ג דכהנים ולוים הוי מיעוט הרי פסקינן דאין הולכין בממון אחר רוב וכהן הוי מוציא מחבירו עליו הראי׳ ואע״ג דהוי תרי רובא רוב נכרים ורוב ישראל נגד כהנים ולוים והוי מיעוטא דמיעוטא דגם ר״מ לא חייש מכ״מ כיון דאיכא חזקה דבודקת ומזנה י״ל שזינתה עם ישראל וא״כ לא הוי רק חד רובא.\\\vspace{0pt}

תשובה – גרסינן בבכורות (דף מ״ז) אמר רב פפא בדיק לן רבה כהנת שנתעברה מנכרי מהו ואמינא לי׳ לאו היינו דרב אדא ב״א דאמר לוי׳ שילדה מנכרי בנה פטור מחמש סלעים וא״ל הכי השתא בשלמא לוי׳ בקדושתה קיימא דתניא לוי׳ שנשבת או שנבעלה בעילת זנות נותנין לה מן המעשר ואוכלת אלא כהנת כיון דאי בעיל לה הוי זרה ע״ש ומזה יצא דישראלית וכהנת שילדה מנכרי בנה חייב לפדות וכן נפסק בטוש״ע (סי׳ ש״ה ס׳ י״ח) והשתא לפי דבריך אפילו הנכרי והיא מודים שממנו נתעברה מכ״מ נימא שמא מכהן ולוי נתעברה שהם ודאי אין להם נאמנות וא״כ פטור מפדיון והיאך הכהן מוציא ממון מספק אחר דקיי״ל כשמואל דאין הולכין בממון אחר הרוב ואפילו נוקי בהי׳ הוא והיא חבושים בבית האסורים הא אמרינן בחולין (דף י״א) דאעפ״כ אין אפוטרפוס לעריות ולכן נ״ל שלא שייך בזה אין הולכין בממון אחר הרוב מב׳ טעמים האחד ע״פ מה שכתבו התוספ׳ בסנהדרין (דף ג׳ ע״ב) שהקשו על מה דאמרינן שם בד״נ הלך אחר הרוב בד״מ לא כש״כ מהא דשמואל דס״ל אין הולכין בממון אחר הרוב ותרצו דרובא לרדיא לא הוי רוב גמור אבל ברוב גמור מודה שמואל דהולכין אחר הרוב וכן כתב גם הש״מ בב״ב (דף צ״ב) בשם גליון תוספ׳ והתוספ׳ בב״ק (דף כ״ז) סתמו דבריהם אבל כוונתם נראה ג״כ כמשכ׳ בסנהדרין והב״ח ח״מ סי׳ רל״ב פירש דבריהם דאף דאחד קונה מאה לרדיא מכ״מ יש ג״כ הרבה דקונין אחד לאכילה וכן לענין זרעוני גינה ע״ש ולפ״ז ברוב דנדון זה ודאי אין מבורר גדול ממנו שיש הרבה נכרים וישראלים בעולם מכהנים ולוים ולכן גם לדידן אזלינן בתר רובא ואע״ג דהתוספ׳ ספ״ק דכתובות אהא דאמרינן שם אם רוב ישראל ישראל להחזיר לו אבדה הקשו ג״כ מהא דשמואל שם לא הוי רוב ישראל רוב גמור דאף דבמקום הזה הוי רוב ישראל מכ״מ הרי רוב בעולם הם א״י אבל בנדון דלפנינו ודאי כהנים ולוים הוי מיעוטא דמיעוטא נגד שאר בני עולם ונחשב זה רוב גמור ומטעם זה צל״ע אמה שפסק בשו״ת צ״צ סי׳ קכ״ה באשה שלא ידעה אם כהנת ולוי׳ או ישראלית היא ונשאת והמליטה זכר דבנה פטור מפדיון דשמא כהנת או לוי׳ היא ואין הולכין בממון אחר הרוב ולענ״ד ע״פ דברי התוספ׳ הנ״ל זה אינו כיון דישראלים הוי רוב גמור ושוב מצאתי שגם המחבר נחלת צבי השיב עליו כן וא״כ הבו דלא לוסיף.\\\vspace{0pt}

ועוד מטעם שני נ״ל דלא שייך הכא אין הולכין בממון אחר הרוב ע״פ מה שכתבתי בספרי ע״ל ביבמות (דף כ״ב) שהקשתי אמה דאמרינן שם דממזר יורש את אחיו דדלמא לאו אחיו הוא דשמא זנתה גם עם אחר דחזקת כשרות לית לה וכן אמה דאמרינן שם (דף ל״ז) בספק ויבם שבאו לחלוק בנכסי מיתנא שהקשה הרשבא דניזל בתר חזקה ליבם ותירץ דאיכא נגד זה רוב יולדת לט׳ הוקשה לי הרי אין הולכין בממון אחר רוב ותרצתי דלא שייך כלל זה רק את נדון על ממון לבד כגון בהך דמקח וממכר דפליגי בי׳ רב ושמואל אבל היכא דעיקר הרוב הוא גם לדברים אחרים אלא שיש נפקותא גם לענין ממון בזה אמרינן כמו דאזלינן בתר רוב לשאר דברים אזלינן ג״כ לענין ממון ושוב מצאתי סברא זו גם בספר הכתובה ספ״ק דכתובות והביא כמה ראיות לזה והיא סברא נכונה מאוד ולכן כאן דודאי לא נדון שהולד כהן או לוי ליתן לו תרומה ומעשר או שימכור תרומה ומעשר שלו והדמים שלו כדין ספק כהן ולוי ואפילו להחמיר לא מספקינן לי׳ בכהן לאסרו בנשים איסורי כהונה משום דאזלינן בתר רובא ולכן בזה לענין ממון לא דיינינן להו כספקי כהן ולוי אלא אזלינן בתר רובא דישראלים ננהו.\\\vspace{0pt}

ועוד נ״ל שאפילו לא נחלק כטעמים הנ״ל מכ״מ ע״כ לענין פדיון גלי קרא דאזלינן בתר רובא מדכתיב ופדויו מבן חדש תפדה והיאך צותה התורה לפדותו מבן חדש ומעלה קודם שנשלמו לו י״ב חדש דלמא טרפה הוא ואמרינן ב״ק (דף י״א) בכור שנטרף פטור מפדיון ובשלמא לפי׳ רש״י דנטרף היינו שנהרג ל״ק אבל לר״ת שפירש נטרף שנעשה טרפה קשה אע״כ דלענין לקיים מצות פדיון גזירת הכתוב הוא דאזלינן בתר רובא ובזה מתורצת קושית השער המלך מבכור שנטרף ועיין בש״מ בב״ק שם ומכל הלין טעמין נ״ל שצריכים פדיון וצריך הבן י״ח לפדות עצמו מיד והבן עשרה כשיגדל כנלענ״ד הקטן יעקב.\\\vspace{0pt}

\end{multicols}\newpage

\newchap{סימן קה}
\begin{multicols}{2}
ב״ה אלטאנא, יום ד׳ י״ב שבט תרכ״ד לפ״ק. עוד לחתני הרב הנ״ל נ״י.\\\vspace{0pt}

במה שהשיב חתני נ״י על תשובתי בענין בכורים שלא נפדו אמנם שמחתי בראותי כי ת״ל ידך רב לך בבקיאות בפלפול ובסברא ועם כל זה בעבור העמד הפסק על אדני האמת לא אסכים עמך ואשיב על כל דבריך:\\\vspace{0pt}

על מה שכתבתי שמגמרא דבכורות ומפסק הש״ע מוכח דפנוי׳ שנתעברה מנכרי בנה צריך פדיון השבת דדלמא זה איירי שנאנסה דלא חיישינן שמא זינתה עם אחר אבל במפותה יש ספק שמא זינתה עם כהן או לוי זה ודאי יוצא מדרך האמת להעמיד דין הגמרא ופוסקים ראשונים ואחרונים שכתבו בפשיטות דישראלית שנתעברה מנכרי בנה צריך פדיון דוקא בשנתעברה באונס ובפרט דאונס לא שכיח כדאמרינן בכתובות וודאי לא הוי שתקי הפוסקים מלהזכיר זה. ומה שהשבת על דברי שכתבתי דלא שייך בזה אין הולכין בממון אחר הרוב ע״פ דברי התוספ׳ דסנהדרין דזה דוקא ברוב שאינו רוב גמור שזה דלא כהלכתא ע״פ מה שנפסק בח״מ סי׳ רצ״ב ס׳ י׳ אבל אם נתן להוליך טבעות של זהב וכו׳ וכתב הש״ך ע״פ דברי תרה״ד הטעם משום דאין הולכין בממון אחר הרוב נגד חזקת ממון המוחזק ולכן אף שהב״ח ותומים פסקו כדעת התוספ׳ שהזכרתי דברוב גמור הולכין בממון אחר הרוב מכ״מ אחר שפסק הרמ״א חולק על זה אין להוציא פדיון מהבכורים המוחזקים לענ״ד אין מפסק הרמ״א סתירה לשיטת התוספ׳ דהך דטבעות לא נחשב רוב גמור כלל דאפילו ראו המפקידים בשעה שהפקידו שטבעות הנפקד היו רבים משלהם דלמא אח״כ נטל מהם מאותם שלו עד שבשעת אבידה היו שלו המיעוט ואפילו מודה שלא נטל מכ״מ נאמן במגו דאי בעי שתיק ואז ממילא לא הי׳ רוב גמור שהרי הם לא יכולים לידע בבירור שבשעת אבידה היו שלו הרוב ובזה דוקא לא אזלינן בתר רובא נגד המוחזק אבל ברוב גמור כי הך דנדון השאלה יש לומר דגם תרומת הדשן שממנו יצא פסק הרמא לא פליג.\\\vspace{0pt}

ואשר הביא חתני נ״י ראי׳ נגד התוספ׳ ממה שכתב המרדכי סוף חולין נראה לראבי׳ הקונה הלחיים וכו׳ אע״ג דרוב בהמות כשרות הן קיי״ל כשמואל דאין הולכין בממון אחר הרוב והרי שם רוב גמור הוי מזה ודאי אין ראי׳ להלכה כיון דבעצמו חזר מתירוצו הראשון וכתב עוד תירוץ שני דרוב טרפות הריאה שכיח. וגם מה שהביא חתני נ״י ראי׳ מתוספ׳ בכורות דף כ׳ שכתבו דחזקת ממון עדיף מרובא וחזקה דאין הולכין בממון אחר הרוב ושם רוב גמור הוא וא״כ התוספ׳ בעצמם חלקו על סברתם לא ידעתי ראי׳ משם שהרי כתבו שם שאינו רוב גמור שנגד הרוב בהמות אינם מטנפים איכא רוב בהמות מתעברות כשמגיעות לזמן הראוי להתעבר. ולכן מכל זה אין ראי׳ נגד סברת התוספ׳ שהוכיחו חילוקם מכח קושיתם מסוגיא דסנהדרין דלא מצאנו לקושיתם תירוץ אחר אם לא שנאמר דברובא דאיתא קמן גם שמואל מודה דאזלינן בתר רובא וכמו שכתב התרומת הדשן לשיטת הריצב״א ואפילו נימא כן ג״כ לא שייך בנדון השאלה אין הולכין בממון אחר הרוב דאין לך רובא דאיתא קמן יותר מזה שרוב יושבי הארץ אינם כהנים ולוים.\\\vspace{0pt}

אמנם בלא״ה כל דבריך בזה לא משיגים רק הטעם הראשון שכתבתי שאין שייך בזה אין הולכין בממון אחר הרוב ולא ב׳ טעמים אחרים שכתבתי דהיכא שלא נדון על ממון לבד אלא גם על איסור הולכין אחר הרוב ועוד שהכא לענין פדיון גלי קרא שנלך אחר הרוב מדכתיב ופדויו מבן חדש תפדה. אבל עוד השיג חתני נ״י מטעם אחר על כל הטעמים שכתבתי דאין שייך בזה רוב כלל דדלמא אזלה איהי לגבי בועל ע״פ גמרא דקידושין (דף ע״ג) דאמרינן שם אי דאזלה איהי לגבייהו הוי קבוע וכמחצה על מחצה דמי וא״כ הכי נמי בנדון השאלה נימא דלמא אזלה איהי לגבי כהן או לוי הבועל וא״כ הוי ספק השקול ואין מוציאין ממון מהמוחזק עכ״ד ודבריך בזה תמוהים דלפ״ז בכל פנוי׳ שילדה ואין כאן אב שיאמר בני הוא נימא שהוא ספק כהן ואסור לטמא למתים ולישא אשה פסולה לכהן ויהי׳ בן הזונה כשר יותר מבן הכשרה יציבא בארעא וגיורא בשמי שמיא אבל באמת לא שייך בזה ספק השקול מכמה טעמים, בראשון, מה שהערת בעצמך ממה שהקשו הרשבא והריטבא שהרי בשתוקי ספק ספקא הוא דלמא לא אזלה איהי לגבי אלא הוא אצלה ותרצו דאיירי שהיו עדים שראו שהלכה אליו א״נ איירי בעיר שהיא מחצה על מחצה והרי כל זה לא שייך בנדון השאלה וא״כ לא הוי כאן ספק השקול אלא רוב ומיעוט דגם ס״ס הוא מטעם רוב ולכן הרשבא לא הזכיר בקושיתו שהוא ס״ס אלא שהוא רוב. שנית, לא שייך לומר דהוי מחצה על מחצה אלא אם מוחזק לנו שהמיעוט הוא קבוע אבל בלא מוחזק זה לא וכן נראה בפירוש ממה שכתב הש״ע אהע״ז סימן ו׳ ס׳ י״ח ראוה שנבעלה או שנתעברה בעיר אפילו לא הי׳ שוכן שם אלא עכו״ם אחד או חלל ועבד וכו׳ ה״ז לא תנשא לכתחלה לכהן שכל קבוע כמחצה על מחצה דמי עכ״ל הרי דאפילו להחמיר בעינן לידע שקבוע הפסול אם נאמר כל קבוע כמחצה על מחצה דמי וא״כ בנדון השאלה שלא ידענו אם במקום שנתעברה הי׳ כהן או לוי קבוע היאך נאמר בזה כל קבוע כמע״מ דהוי ג׳ ספקות שמא אין כאן כהן ולוי קבוע ואת״ל שקבוע דלמא אזיל הבועל לגבה ואת״ל דאזלא איהו לגבי דבועל דלמא לא הוי כהן ולוי ובג׳ ספקות ודאי הוי רוב גמור להוציא ממון דגם אותם הסוברים בשיטה דפתח פתוח שהזכרת דגם בס״ס לא מוציאין ממון הוא מטעם דאין הולכין אחר הרוב דס״ס לא עדיף מרוב אכן בג׳ ספקות אפשר דמודו ועוד דכיון דבאנו לדון מטעם רוב כבר כתבתי הטעמים דהולכין כאן אחר רוב אפילו בממון. ועוד שלישית נ״ל דלא שייך בזה לומר דלמא הבועל כהן או לוי הי׳ דאף דאמרינן בעלמא כשם שזינתה ע״ז זינתה עם אחר היינו היכא שנוכל לומר שמא זינתה עם נכרי אבל לומר שמא זינתה עם כהן ולוי הרי להם יש חזקת כשרות וא״כ נגד חזקת ממון יש ג״כ חזקת כשרות דכהן ולוי ובכגון זה ודאי אזלינן בתר רובא. ולכן נלענ״ד ברור דהבכורים שנולדו מפנויות שנתעברו מנכרים צריכים פדיון כמו שכתבתי: הקטן יעקב.\\\vspace{0pt}

\end{multicols}\newpage

\newchap{סימן קו}
\begin{multicols}{2}
ב״ה אלטאנא, יום ג׳ כ״ז שבט תר״ח לפ״ק. להרה״ג וכו׳ מ״ה אברהם זוטרא נ״י הגאב״ד דק״ק מינסטער יע״א.\\\vspace{0pt}

נדרשתי לאשר שאל מעכ״ת נ״י לחוות דעתי הקלושה על מה שכתב בענין במי שנולד אחר הנפל במה שהקשה על הטור וש״ע י״ד (סי׳ ש״ה) שפסקו בן שמנה שהוציא ראשו מת הבא אחריו הוא בכור לכהן שהרי שמואל דס״ל בבכורות (דף מ״ו) דאין הראש פוטר בנפלים אתותב אכן לענ״ד מוכח כדברי הטור ממה דתנן (שם) איזהו בכור לנחלה ואינו בכור לכהן הבא אחר הנפלים אע״פ שיצא ראשו חי ומשמע הא יצא ראשו מת הבא אחריו גם בכור לכהן הוא הן אמת דלכאורה מרש״י חולין (דף ס״ח) לא משמע כן דכתב בד״ה אע״פ וכל שכן אם יצא ראשו מת עכ״ל והיינו לפרש לשון אע״פ דקאמר וא״כ משמע דיצא ראשו חי לאו דוקא דה״ה מת אבל זה אינו דזה לא כתב רש״י רק לגבי הא דבכור לנחלה שהרי לענין שאינו בכור לכהן אדרבא הוציא ראשו מת הוי רבותא טפי ואיך שייך וכש״כ אם יצא ראשו מת אע״כ דזה דוקא לענין דהבא אחריו בכור לנחלה דבזה קאמר שכל שכן מת אבל לפי מה שכתב אח״כ דמה דנקט ראשו משום בכור לכהן לבד נקט דלענין נחלה אפילו נולד חי ומת הבא אחריו הוי בכור לנחלה דהשתא דמשום בכור לכהן נקט כן ואע״פ כן קאמר שהוציא ראשו חי ולא קתני רבותא טפי דאפילו הוציא ראשו מת פוטר הבא אחריו מכהן מזה מוכח שפיר כדעת הטור דרק חי פוטר מכהונה אבל מת לא פוטר ומה שהקשה מעכ״ת נ״י שהרי אתותב שמואל לענ״ד לק״מ ששמואל רצה להשמיענו דאין הראש פוטר בנפלים אפילו יצא חי כמו שכתב רש״י בפי׳ בבכורות (דף מ״ו) בד״ה אין הראש וז״ל בן שמנה שהוציא ראשו חי ולענין זה אתותב דהוכיח דראשו חי פוטר וכמו שכתב רש״י שם ד״ה תיובתא דודאי ראשו דנפלים דמתניתן דוקא עכ״ל וא״כ דוקא כמו דקתני במתניתן הוציא ראשו חי פוטר אבל מת לא פוטר ומה שהקשה מעכ״ת נ״י עוד ממה שכתבו התוספ׳ בד״ה תיובתא כיון דאתותב שמואל אפילו במת הראש פוטר נראה שמשמע לו כן ממה שכתבו וא״ת כיון דאתותב שמואל וכו׳ הא אפילו מתים הראש פוטר עכ״ל אכן לענ״ד אין מזה ראי׳ דכבר הוגה בצאן קדשים תחת מתים בנפלים וא״כ שפיר י״ל דבנפלים חיים איירי שהראש פוטר אבל מת לא.\\\vspace{0pt}

ומה שהקשה מעכ״ת נ״י עוד על שיטת הטור מלשון המשנה דתנן איזו בכור לנחלה ולכהן המפלת שפיר מלא מים וכו׳ ולא קתני ג״כ הבא אחר נפל שהוציא ראשו מת הן אמת שכבר הקשה כן על הטור בשו״ת פנים מאירות (ח״א סי׳ ז׳) אכן לענ״ד לק״מ די״ל דלא צריך להשמיענו זה דכבר ידענו מדיוקא דרישא מדקתני בן שמנה שיצא ראשו חי הוי בכור לנחלה ולא לכהן מכלל דביצא ראשו מת הוי בכור גם לכהן וכן מוכח גם מסיפא דקתני יוצא דופן והבא אחריו אינן בכור לנחלה ולא לכהן ולא קתני ג״כ הבא אחר בן ט׳ שיצא ראשו חי אע״כ דלא הוצרך למיתני כן דכבר שמענו מדיוקא דרישא אכן לכאורה נראה מדברי רש״י נדה (דף כ״ו) דס״ל דאין חילוק בנפל בין הוציא ראשו חי או מת לפטור דכתב שם ד״ה הויא וז״ל אבל סנדל עד דנפיק רובא דאין ראש נפל חשוב דהכי אמרינן בבכורות אין הראש פוטר בנפלים עכ״ל והשתא אי ס״ד דהוציא ראשו מת לכ״ע לא פוטר למה לו להביא מהא דשמואל הא סנדל הוי מת דלכ״ע אינו פוטר ואף דכבר הקשו התוספ׳ על רש״י שם דהביא מהא דשמואל דהא אתותב שמואל אכן גם ממה שכתבו התוספ׳ לחלק בין נפל דאית לי׳ צורת פנים לסנדל ולא חלקו דנפל נפיק ראשו חי משא״כ סנדל משמע ג״כ דס״ל דאפילו ראשו מת פוטר.\\\vspace{0pt}

והנה לכאורה מסוגיא דשם הי׳ נראה ראי׳ לשיטת הטור שהרי כבר העירו התוספ׳ מה דנקט ולד דאית ביה חיותא דלאו דוקא הוא דאפי׳ לית בי׳ חיותא כיון דאתותב שמואל ועוד הקשו דמשמע בירושלמי דולד שעם הסנדל אינו מתקיים אכן לשיטת הטור א״ש שהרי אף לפי מה שכתב הירושלמי דאינו מתקיים ע״כ אין הפי׳ דבשעת לידה כבר מת שהרי בסוגיא שם אמר ולד דאית בי׳ חיותא סריך הרי משמע דבשעת לידה חי הוא אע״כ דפי׳ הירושלמי הוא דאין מתקיים שאינו ראוי לחיות וכמו בן ח׳ ולפ״ז א״ש פי׳ הגמרא בדיוקא דולד דאית בי׳ חיותא מדנפיק רישא הוי לידה כיון דהוי כנפל שהוציא ראשו חי אבל סנדל דלית ביה חיותא הוי כנפל שהוציא ראשו מת דאפילו לפי מה דאתותב שמואל אינו פוטר עד דנפיק רובא אך מדלא פירשו רש״י ותוספ׳ כן לכאורה משמע דלא כדעת הטור אבל באמת אין זה קושיא ואדרבא בזה מתורץ מה שתימה בשו״ת פ״מ (שם) על הטור וש״ע דלענין טומאת לידה (בסי׳ קצ״ד) כתבו סתם דהוציא ראשו ה״ה כילוד ולא חלקו בין הוציא ראשו מת או חי ולענין בכור לכהן חלקו והרי פסק הרמב״ם (פ׳ י״א מהלכ׳ בכורים) כל שאמו טמאה לידה פוטר פטר רחם ע״ש שמתוך כך רצה לחלוק על הטור אבל לפ״ז א״ש כיון דמשמע בפי׳ מדברי רש״י ותוספ׳ בבכורות ומלשון המשנה שם דדוקא ראשו חי פוטר לענין בכור לכהן אבל ראשו מת לא ואעפ״כ לא פירשו רש״י ותוספ׳ בהך דסנדל כפירושינו הרי דס״ל דיש חילוק בזה בין טומאת לידה ובין בכור לכהן לכהן פסקו הטור וש״ע כשיטת רש״י ותוספ׳ לענין טומאת לידה שלא לחלק בין חי למת ולענין בכור לכהן דדוקא חי פוטר כנלענ״ד הקטן יעקב.\\\vspace{0pt}

\end{multicols}\newpage

\newchap{סימן קז}
\begin{multicols}{2}
ב״ה אלטאנא, יום א׳ י״ח תמוז תר״ך לפ״ק. לחתני הרה״ג וכו׳ מ״ה ישראל מאיר פריימאן נ״י אב״ד דק״ק פילעהנע יע״א.\\\vspace{0pt}

על דבר שאלתך באחד שבירך בשעת המילה על בנו הבכור ברכת שהחיינו והוא לא מל בעצמו אם שוב יברך גם שהחיינו בשעת פדיון הבן.\\\vspace{0pt}

הנה כבר הערת לנכון על שיטות הפוסקים בענין זה לדעת הרמב״ם צריך האב לברך שהחיינו במילה ובפדיון ורבינו שמחה וראבי׳ כתבו דדוקא כשהאב מל בעצמו מברך שהחיינו מפני שעושה המצו׳ המוטלת עליו ובא״ח בשם הרמ״ה כתב שגם הרמב״ם ס״ל כן דדוקא כשמל האב בעצמו מברך שהחיינו אבל הב״י בבדק הבית כתב שאינו נראה כן מפשט דברי הרמב״ם אלא אפילו כשמל אחר מברך האב שהחיינו אכן בהג״מ כתב שאין אנו נוהגין לברך בשעת המילה והטעם לא נתברר אלא בשם י״מ כתב הטעם מפני צערא דינוקא וזה דחה הרשב״א בתשובה דברכת שהחיינו שייך אפילו בשמחה שמתערב עמה צער והרוקח כתב מפני שעדיין לא יצא התינוק מתורת נפל וגם טעם זה דחה הרשב״א דכל שנגמרו שערו וצפרניו לא חיישינן לספק נפל שסומכין על סימנים הללו למולו בשבת החמורה כ״ש לענין ברכה עכ״ד ואע״ג דאמרינן בשבת (דף קל״ו) דממנ״פ מוהלין בשבת דאי נפל הוא הוי מחתך בשר בעלמא י״ל הרשב״א ס״ל כשיטת ר״ח וריף ורמב״ם דס״ל דלא קאי תי׳ זה לפי המסקנא וכמו שהביא שיטה זו בחדושיו שם ולכן אין מוהלין ספק נפל בשבת אמנם הרוקח י״ל דס״ל כשיטות הפוסקים דמוהלין ספק נפל בשבת ולכן ל״ק עליו מה שהקשה הרשב״א אמנם גם הרשב״א העיד שלא ראה מעולם מי שבירך שהחיינו בשעת המילה אכן לא כתב טעם לזה וגם התוספ׳ בסוכה (דף מ״ו) כתבו דלא מברכין על מילה שהחיינו והקשו מ״ש מפדיון הבן וגם הקשו למה על הלל לא מברכינן ובספרי ע״ל כתבתי יישוב לזה דברכת שהחיינו היא ברכת הודאה שזכה לחיות עד זמן ההוא ולכן לא שייך ברכה זו אלא כשהי׳ חשש מיתה והשער המלך כתב בשם הראש דל׳ יום מקרי זמן מרובה דחיישינן למיתה ומצאתי ראי׳ לזה לענין ברכת שהחיינו ממה דאמרינן בברכות הרואה חבירו מל׳ יום לל׳ יום מברך שהחיינו הרי שקבעו זמן לברכה זו ל׳ יום ולכן על כל המועדים ומצות שמזמן לזמן מברכים שמצפה על קיומם יותר מל׳ יום אבל על מילה לא מברך שקודם שנולד הבן אינו מצפה על קיום המצו׳ ומשעת לידה עד המילה אין ל׳ יום אבל על פדיון שהוא אחר ל׳ יום ללידה מברכים וכן הלל דר״ח אינו ל׳ יום משעה שבירך ועל הלל די״ט אינו מברך שהוא בכלל מקרא קדש די״ט כן כתבתי ליישב הטעם דלא מברכים על המילה שהחיינו והרמ״א בי״ד (סי׳ רס״ה) כתב בשם מהריל דאם האב מל בנו הבכור מברך בשעת מילה ולא בשעת פדיון אכן כבר דחה הש״ך שם שאדרבא במהרי״ל כ׳ דאפילו היכא דליכא פדיון כגון שהאם כהנת אינו מברך בשעת מילה וכש״כ בדאיכא פדיון שברכת שהחיינו בפדיון הוזכר בגמרא ואין לדחותה ע״ש והיוצא מזה דלכל הטעמים שהובאו בזה שלא לברך בשעת מילה אין סברא לדחות ברכת שהחיינו שבשעת פדיון דאין שייכות לברכת שהחיינו שבשעת מילה עם הברכה שבשעת פדיון ולכן אף שהאב עשה שלא כדין שבירך בשעת מילה מכ״מ לא נפטר בזה מברכת שהחיינו שבשעת פדיון שהובא בגמרא כנלענ״ד הקטן יעקב.\\\vspace{0pt}

\end{multicols}\newpage

\newchap{סימן קח}
\begin{multicols}{2}
ב״ה אלטאנא, יום ג׳ י״ט טבת תרכ״א לפ״ק. להרה״ג וכו׳ יצחק דוב הלוי נ״י הגאב״ד דק״ק ווירצבורג יע״א.\\\vspace{0pt}

על דבר השאלה באחד שמכר ראש הולד של מבכרת ודאי קודם לידה לנכרי וילדה זכר אם יש בזה קדושת בכורה.\\\vspace{0pt}

תשובה – הנה מעכ״ת נ״י כבר הביא דברי הש״ע סי׳ ש״כ דמכירת הולד אינו מועיל דהוי דבר שלא בא לעולם ושגם על דברי מהרא״י שהביא הש״ך שם ס״ק ט׳ שנעשה כאלו אמר קנה פרה לולדה פקפק הנב״י מה״ת חי״ד סי׳ קצ״א דמהראי לא סמך על סברא זו כי אם בצירוף קולות אחרות שהביא שם ומכ״מ שם סי׳ קצ״ב היקל קצת במקום שכבר הוכר עוברה דבזה יש לצרף דעת הרמב״ם דבהוכר עוברה מקרי בא לעולם ומכ״מ לא רצה לסמוך על זה להתירו לשוחטו רק לעשות בו מום קודם שחיטה דעשיית מום בזה״ז בקדשים אינו רק דרבנן אבל שחיטת קדשים בחוץ הוא דאורייתא ובעקבותיו דרך ג״כ הגאון בעל חתם סופר חי״ד סי׳ שי״ב ומעכ״ת נ״י שדא נרגא בקולות הללו שעל מה שהביאו מדברי הרמב״ם כ׳ דגם הרמב״ם לא סובר בודאי שבהוכר עוברה מקרי דבר שבא לעולם שאע״פ שבה׳ אישות באומר אם תלד אשתך נקבה ס״ל דבהוכר עוברה מקודשת הרי שם התנה הרמב״ם שכשתולד צריכה קידושין אחרים שאין בהם דופי ובפי׳ המשניות קידושין פ׳ ג׳ ביאר דעתו שאינם רק ספק קידושין וכיון דהרמב״ם בעצמו לא החליט דמקרי בא לעולם היאך נסמוך על זה עכת״ד מעכ״ת נ״י. אמנם לענ״ד הדין עם הגאונים ז״ל דלדעת הרמב״ם אין זה ספק אלא ודאי כנראה מדבריו ה׳ אישות (פ׳ ז׳) שכתב והוכר העובר ה״ז מקודשת ולא כתב מקודשת מספק כמשכ׳ בשאר מקומות שם כשהקידושין ספק וכמשכ׳ שם בהלכה הסמוכה לענין לאחר שיחלוץ יבמך וכ״נ מהרב המגיד שם דלהרמב״ם מקודשת ודאי ומה שכתב שיקדש אותה שנית זה רק שכיון שבקל יכול לחוש גם לדעת החולקים וזה מה שכתב בקידושין שאין בהם דופי פי׳ שאין חולק עליהם וכן יש לבאר ג״כ דבריו בפי׳ המשניות גם הח״מ והבית שמואל באהע״ז (סי׳ מ׳) כתבו כן דלהרמבם הוי קידושי ודאי וכן נראה ג״כ ממה שכתב הרמב״ם ה׳ מכירה (פ׳ כ״ב) תנו מה שתלד בהמה זו לפלוני לא קנה וכיון דלא כתב שם ואפילו הוכר העובר כמשכ׳ הטור משמע דאזיל לשיטתו דבהוכר העובר נקרא בא לעולם וכמשכ׳ הב״ש ואי לא הוי לדעתו רק ספק הרי מספק לא מפקינן מיד המוחזק וה״ל לכתוב דאפילו בהוכר העובר לא קנה אע״כ דלהרמב״ם חשיב ודאי בא לעולם ולכן יפה כתבו הגאונים ז״ל דלהרמב״ם בנדון השאלה אין ספק בכור ושאין דעת הרמב״ם יחידי בזה כבר הוזכר בש״ע (שם) שכתב לדעת הרמב״ם וקצת מפרשים ולכן אין נפקותא במה שחלק מעכ״ת על החתם סופר שאין דעת הא״ז כהרמב״ם.\\\vspace{0pt}

ועוד שנית: מה שכ׳ מעכ״ת נ״י שקשיא לי׳ טובא על דברי הגאונים בעל נב״י וחת״ס היאך התירו לגרום עשיית מום בספק בכור הרי יש בזה איסור צער בעל החיים דקיי״ל שהוא דאורייתא ופלפל טובא ברוחב בינתו שאין להתיר ממה שפסק הרמא אעה״ז סי׳ ה׳ שכל דבר שצריך לרפואה או לשאר דברים לית בי׳ משום צער בעל החיים שאו״ה שממנו נובע דין זה לא התיר רק היכא שיש צורך משום רפואה אפילו לחולה שאין בו סכנה אבל משום ריוח ממון לא מצאנו שהתיר ושגם ממה דאמר ר׳ יהודה בע״ז פרק קמא במוכר תרנגול לבן שקוטע אצבעו אין ראי׳ דאפשר דס״ל כריה״ג צער בעה״ח דרבנן אבל למ״ד דאורייתא לא הותר והאריך בזה, לענ״ד יש להוכיח דבעשיית מום בבהמה לצורכו אין בזה משום צער בעה״ח ממה דאמרינן בכורות (דף ל״ו) הכל נאמנין על מומי מעשר ומפרש בגמרא טעמא דאי בעי שדי מומא בכולי׳ עדרא כדי שיצא המעשר בעל מום יע״ש ומדאמרינן דאי בעי שדי מומא משמע דמותר לעשות כן דאין לומר דה״ק דנאמן דלא עביד איסור דאם הי׳ רוצה לעשות איסור הי׳ יכול לשדות מום בכולא עדרא דז״א דמה מגו הוא זה דדלמא בבהמה א׳ עביד איסורא אבל לא בבהמות הרבה ובפרט כיון שבספק מעשר איירי כמו שפי׳ רש״י שם דעשיית מום הוי ספק איסור וצער בעה״ח הוא ודאי וגם הלשון אי בעי וכן מה שכתב הרמב״ם אם ירצה משמע דברצונו לבד הדבר תלוי וגם א״ל דלמא מתניתן ריה״ג היא דס״ל צער בעה״ח דרבנן אבל למ״ד דאורייתא באמת אינו נאמן דמלבד שדוחק לאוקמי׳ סתם מתניתן דלא כהלכתא ושגם לא מצאנו פלוגתא בזה שנאמנים על מומי מעשר אלא שיקשה ג״כ על הרמב״ם שפסק כסתם מתניתן ופסק צער בעה״ח דאורייתא אלא ודאי שמה שעושה לתועלתו לית בי׳ משום צער בעה״ח ומה דאסרינן ע״ז (דף י״א) לעקור הבהמה הוא דוקא משום ששם אין לו תועלת וכן מה שאסרו להכניס בכור לכיפה משום צער בעה״ח שם ג״כ הטעם שאין לו תועלת מוחלט רק שלילות שינצל מטורח או מהיזק או אפשר ג״כ שבאילו יש צער גדול וכבר חילק הריטבא בזה שפי׳ דמה דאמרינן (שם) עיקור שיש בו טרפה אסור ושאין בו טרפה מותר דבאין בו טרפה אין צער גדול כ״כ ולכן י״ל דבעשיית מום שאין בו צער גדול וגם יש בזה תועלת מוחלט שעי״ז יותר הבהמה לאכילה לכ״ע אין בזה משום צער בעה״ח וכדמוכח מהא דבכורות. והנה מעכ״ת נ״י רצה לחדש היתר שיתן הבכור ספק לנכרי למחצית שכר על זמן מה ובכלות הזמן יהי׳ הברירה ביד הישראל המוכר לקנות ממנו החלק שמכר בעד סך מה ומסתמא בתוך הזמן יסרסנו הנכרי ובאופן זה מותר כמשכ׳ הרמ״א באהע״ז סי׳ ה׳ ושאין עומד לנגדנו אלא דעת קצת הפוסקים שהביא הש״ך סי׳ ש״ז ס״ק ב׳ ובכגון זה ודאי היתרים הנ״ל חזי לאצטרופי להיתר עכ״ד ולענ״ד יש לפקפק בזה דהפוסקים הנ״ל איירי בבכור בעל מום שאין חשש אלא שמא ימצא טרפה לאחר שחיטה ובזה רק קצת פוסקים אסרו ורובם התירו וכפסק הש״ע שם אבל למכור בכור תם לנכרי לזה לא מצאנו היתר דרק למכור לישראל הותר בזה״ז כמבואר בתמורה (דף ח׳) וכמו שפסק הש״ע סי׳ ש״ו ס״ו ואף שי״ל דשמא הטעם שלנכרי לא ימכור בכור תם הוא שמא ישחטנו ובנדון זה אין חשש לזה כיון דלא נמכר לו רק למחצית שכר מכ״מ מנ״ל להמציא טעם זה דלמא יש בזה ג״כ משום ביזוי קדשים ועוד שאפילו למכור חצי בכור לנכרי לא מצאנו היתר וא״כ בתחבולה זו לא יצאנו מחשש שמא בכור הוא ולכן באופן הזה לענ״ד אין היתר ע״י מכירה אלא שיתן הבהמה לשומר ויתנה עמו שמה שמעולה בדמים לאחר כלות השנה או החצי עלוי ישלם לו בעד שכר שמירה ואולי עי״ז יסרסה השומר להעלותה בדמים ואם לאחר כלות שנה לא עשה כן אז לענ״ד ראויים ההיתרים של הנב״י והחת״ס לסמוך עליהם שיערים עד שהנכרי יעשה בו מום ואח״כ מותר לשחטו כנלענ״ד הקטן יעקב.\\\vspace{0pt}

\end{multicols}\newpage

\newchap{סימן קט}
\begin{multicols}{2}
ב״ה אלטאנא, אייר תר״ב לפ״ק. לאחי הרה״ג וכו׳ מ״ה ליב עטטלינגער נ״י אב״ד דגליל לאדענבורג יע״א.\\\vspace{0pt}

שאלה – נוהגים העולם כשמשימין פרי האדמה כתותים שקורין קארטאפעלן בעיסה שמפרישין בתחלה מעט כשיעור חלה טרם שמערבין הקארטאפעלן ונ״ל טעם המנהג ע״פ המבואר בי״ד (סי׳ שכ״ד ס׳ י״א) בשאור של א״י שמשימין אחר הפסח בעיסה שצריך להפריש יותר ממה שיש בהשאור שמא יפריש מן הפטור על החיוב וא״כ ה״ה כשיפריש חלה אחר שערבו פרי האדמה שמא יפריש חלה מהם אבל נ״ל דלא דמי שהרי דין זה שהוא מהת״ה אינו אלא בשאור כמו שכתב שם לדעת הראש אבל בשאר דברים ואפילו בעיסה אמרינן יש בילה ואפילו לענין שאור יש לתמו׳ איך החליט דאין בילה דבפי׳ כתבו התוספ׳ זבחים (דף ע״ג ע״ב) דשאור שנפל לעיסה אין לח נתערב יותר ממנו ע״ש וא״כ מבואר שהשאור מתערב כמו לח וכל שכן בפרי האדמה כתותים שמתערבים כמו לח ואפילו אם חולק הראש עם התוספ׳ וסובר דבשאור אין בילה מכ״מ נ״ל דהכא כיון דנפרדו הקארטאפעל לפרורין קטנים הוי כמו קמח בקמח דחשיב לח בלח כמבואר י״ד (סי׳ ק״ט) וא״כ פשיטא דיש בחלה מעט קמח וחלה אין לה שיעור ושפיר יכול לערב הקארטאפעל בשעת לישת קמח קודם הפרשת עיסה לחלה ועוד נ״ל ראי׳ ברורה לזה ממה שפסק בי״ד (סי׳ שכ״ד ס׳ ט׳) העושה עיסה מחטים ואורז אם יש בה טעם דגן וכו׳ ע״ש ופירשו התוספ׳ והראש הטעם משום דטעם כעיקר דאורייתא וכיון דפרי האדמה הם אינם מינו עם העסה ויש בהם טעם דגן הם כמו קמח ולא שייך מן הפטור על החיוב ואף דלשיטת הרשבא זה דוקא באורז שנגרר אחר החטים והיא שיטת הרמב״ן שמביא הראש בה׳ קטנות סוף ה׳ חלה יע״ש מכ״מ מודה אם יש בה כזית בכדי אכילת פרס וכש״כ כשיש שיעור חלה בלי פרי האדמה ולכן הי׳ לפסוק דשפיר יכול לערב הקארטאפעל בשעת לישה ואפילו אין שיעור חלה בלי הקארטאפעל והם משלימין לשיעור אם יש כזית בכדי אכילת פרס נוטל חלה בברכה ואפילו אינו כ״כ אם עכ״פ יש להם טעם דגן צריך להפריש חלה מספק בלא ברכה ויורני נא אחי נ״י אם כוונתי בזה להלכה.\\\vspace{0pt}

תשובה – אתחיל במה שסיימת שהחלטת שגם לשיטת הרמב״ן דס״ל שדוקא באורז עם חטים אזלינן בתר הטעם מכ״מ מודה כשיש כזית בכדי אכילת פרס ואפילו אין שיעור חלה מקמח ותמהתי איך שייך לומר כן דענין כזית כא״פ הוא שיש שיעור שלם רק שמעורב במין אחר הרבה ממנו ולכן אם עכ״פ בתוך שאוכל ג׳ או ד׳ בצים ממנו אוכל בתוכו כזית שהוא השיעור הקבוע נחשב כאלו אכלו בפני עצמו אבל איך שייך לענין חלה שאין בו שיעור שיצטרף עמו כשהוא ככא״פ ומה יהי׳ הכזית ולכן ברור דלא שייך ככא״פ אלא כשיש שיעור שלם מקמח רק שהמין אחר רב עליו אם בתוכו מהדגן כדי אכילת פרס אזי נשאר עסת קמח בשמו וחייב בחלה אבל כשאין שיעור קמח לא ואפילו בכה״ג נראה מדברי הב״י בסי׳ שכ״ה במה שכתב לבאר דברי הרמב״ם דע״כ כשיעור דקתני לענין חלה ולענין מצה בתרי גווני איירי ולא ר״ל דתרווייהו איירי לענין כדי אכילת פרס ע״ש דלא ס״ד דשייך שיעור כאכ״פ לענין חלה ומה שהחלטת לדעת הרמב״ן אדרבא הרמב״ן כתב אפכא שאחר שכתב מה שהעתיק הרא״ש סיים הרמב״ן וז״ל שמעינן השתא ממתניתן ומגמרא דאתמר עלה שכל שעושה עסה אפילו מן האורז עם החטים אינו נגרר אלא בשיש בה דגן כשיעור כרשב״ג וכי יש בה כשיעור אע״ג דרובא אורז גורר ואע״פ שאין בדגן אפילו כזית בכדי אכ״פ כיון דאיכא טעמא לא בטל ולא עוד אלא שגורר ואלו שאר כל המינים עם החטים או אפילו אורז עם שאר ד׳ מינים אינו כן אלא ודאי אם עשה עיסה ויש בה כשיעור מחמשת מינים ועירב בה א׳ מכל שאר המינים אזלינן בתר רובא ואי רובא וטעמא דגן חייבת ואי לית בה רוב דגן לאו לחם הוא ופטורה וכן הלכתא עכ״ל הרי דלשיטת הרמב״ן בעינן ג׳ תנאים לחייב בחלה שנתערב שאר מינים בקמח שיהי׳ שיעור חלה מקמח לבד ושיהי׳ הקמח רובא ושיתן טעם באינו מינו וכשחסר א׳ מן התנאים האלה פטור מן החלה וכן היא שיטת הרשבא וכ״נ גם דעת הראב״ד והנה הרמב״ם חולק על שיטה זו דס״ל כשיש טעם חטים באורז אפילו אין החטים לבד כשיעור חייב בחלה וכן היא גם דעת הראש אכן אם זה באורז דוקא משום שנגרר אחר החטים או אפילו בשאר מינים לא נתבאר בדבריהם ולכן יפה העלה להלכה הרב בעל חוות דעת בדיני חלה דבערב קמח ושאר מינים אפילו אם הרוב מקמח אין מצטרפים ופטור מן החלה ואם יש בעיסה כשיעור חלה וגם הוא רוב נגד אין מינו שנתערב עמו חייב בחלה בברכה אבל אם יש שיעור בעיסה ואינו מינו רב עליו אלא שהקמח נותן בו טעם בזה יש מחלוקת הפוסקים ולכן יטול חלה בלא ברכה ע״ש ומזה תראה שהפרסת על המדה להקל ולכן לענ״ד יש טעם רב שלא לערב הפרי האדמה קודם הפרשת חלה שאם הם הרוב נגד הקמח אזי לרמב״ן וסיעתו פטור מן החלה ומברך ברכה לבטלה וכש״כ אם אין שיעור חלה מקמח לבד ואפילו יש טעם ואפילו יש כזית כא״פ דלשיטתם ודאי ואפשר גם לשאר הפוסקים פטור מן החלה אכן אפילו יש שיעור חלה מקמח וגם הקמח רוב נגד פרי האדמה ונותן טעם בהם יש סברא עכ״פ להפריש מעט מהעיסה שיהי׳ לחלה קודם שיערבם עם העיסה שמה שכתבת שהקארטאפעל אחר שנתפררו הם לח בלח כמו קמח בקמח החוש מתנגד לזה שהם מתדבקות כמו עסה בעסה ולכן אף להשיטות דבעסה יש בילה מכ״מ אין מתערב שו׳ בשו׳ כמו קמח וא״כ יש לחוש שמא באותה חתיכה שמפריש לחלה הרוב הוא מהפרי האדמה אף שבעיסה בכללו הקמח הרוב וא״כ אותו חלק שמפריש הוא ממין פטור ועוד דלכתחלה ודאי יש לחוש לדעת המרדכי בשם מהר״ם דכתב דאין להפריש מני׳ ובי׳ משום דק״ל אין בילה אלא ביין ושמן בלבד משמע אפילו בעיסה חוץ משאור ג״כ לא ומה שהבאת ראי׳ דשאור חשיב כמו לח מתוספ׳ דזבחים אין ראי׳ משם דרק לענין מה שחלקו התוספ׳ דמין במינו לא בטל לא אמרינן רק בדבר לח שמתערב לגמרי דומיא דדם הפר ודם השעיר בזה כתבו שפיר דהרי ודאי דמעט מעט מהשאור מתערב לגמרי שהרי עי״ז נתחמץ העיסה וא״כ ודאי שחמוצו של השאור נכנס ונבלע בעיסה כמו לח אבל מזה אין ראי׳ שנאמר בשאור יש בילה שמתערב לגמרי שו׳ בשו׳ כמו בילה ביין ושמן והרי על שמפריש מעט יותר לא רצו הפוסקים לסמוך לכתחלה ולכן מהר״ש עשה לאחר הפסח עסה שאין בו שאור של פטור וצרפו להעסה ונטל חלה מזה וכן כתב הש״ך ס״ק כ״א בשם התשבץ וכתב שיש להחמיר כדבריהם ועוד נ״ל שיש לחוש כיון שאם מערב הקארטאפעל מקודם אף שיש טעם דגן בהם מכ״מ גם טעמם נטעם בהעסה והרי לפי משכ׳ הט״ז (סי׳ שכ״ד ס״ק ט״ו) לא יפריש מהפשטידא כיון שיש בו טעם בשר אף שטעם הלחם לא בטל ממנה לגמרי מכ״מ כיון שנטעם בו גם טעם אחר הוי כאינו מינו וא״כ ה״ה ג״כ בכה״ג ולכן לענ״ד יפה נוהגין להפריש מעט קודם עירוב הקארטאפעל שיהי׳ לחלה ואם הקארטאפעל שמערב בו הם רב מהקמח צריך להפריש חלה לגמרי קודם עירוב דאם לא עשה כן הוי ספק ברכה כנלענ״ד. הקטן יעקב.\\\vspace{0pt}

\end{multicols}\newpage

\newchap{סימן קי}
\begin{multicols}{2}
ב״ה אלטאנא, יום ג׳ כ״ד כסליו תרכ״ג לפ״ק. לחתני הרה״ג וכו׳ מ״ה משלם זלמן הכהן נ״י אב״ד דק״ק שווערין יע״א.\\\vspace{0pt}

אשר שאלת אודות מה שנוהגים איזה ב״ב בעיר א׳ הסר למשמעתך ששולחים קמח לנחתום א״י והוא לש ועושה עיסה ממנו ואופה הפת ואח״כ מפרישין חלה מן הפת האפוי אם יאות לעשות כן.\\\vspace{0pt}

ודאי הדין עמך דלכתחלה צריך להפריש חלה מן העיסה ולא מן הפת אם אפשר בכך. אלא דיש במנהג הזה עוד חשש איסור אחר דאע״ג דבלש א״י עיסה של ישראל חייב בחלה כמבואר בטוש״ע י״ד סי׳ ש״ל נלענ״ד שזה דוקא בלש בפני ישראל אבל בשולח לא״י הקמח לביתו יש לחוש שמא החליף וכדתנן דמאי (פ׳ ג׳ מ׳ ד׳) המוליך חטין לטוחן וכו׳ לטוחן נכרי דמאי ופירשו הרמב״ם והר״ש ע״פ הירושלמי דא״י חשוד לחלוף ע״ש וא״כ הכא נמי יש לחוש שמא החליף הנחתום קמח של ישראל בקמחו וא״כ העיסה והפת הוא של נכרי ופטור מן החלה ולכן לענ״ד בכה״ג יפריש חלה בלא ברכה אם לא שיעמוד אצלו שומר ישראל עד שילוש ויכין הפת מהקמח של ישראל: כנלענ״ד הקטן יעקב.\\\vspace{0pt}

\end{multicols}\newpage

\newchap{סימן קיא}
\begin{multicols}{2}
ב״ה מאננהיים, אייר תקפ״ט לפ״ק.\\\vspace{0pt}

שאלה – מסוכן שנוטה למות וכבר נתיאשו מרפואתו ורופא מומחה רוצה לעשות לו רפואה ע״י הקזה או מרחץ וכדומה שאפשר שתצילהו ממות אכן אם לא תצילהו אז ימות במהרה יותר משהי׳ מת בלא מעשה דרפואה זו אם יש להתיר לעשות כן ע״י נכרי או על ידי ישראל לעת הצורך.\\\vspace{0pt}

תשובה – ע״י נכרי נלענ״ד פשיטא שיש להתיר דאף דאמרינן בשבת (דף קנ״א ע״ב) המעמץ עם יציאת הנפש ה״ז שופך דמים מפני שמקרב מיתתו וכן אמרינן ביומא (דף פ״ה) במי שנפלה עליו מפולת ספק חי ספק מת מפקחין עליו הגל מצאוהו חי מפקחין עליו יותר ואם מת יניחוהו ופריך מצאוהו חי פשיטא ומתרץ ל״צ אלא לחיי שעה ע״ש הרי דאפילו שבת מחללינן משום חיי שעה יותר וכש״כ דאיכא קפידא שלא להרגו בידים אף דליכא אלא חיי שעה מכ״מ הרי אמרינן בע״ז (דף כ״ז) דספק חי ספק מת אין מתרפאין מעכו״ם (דעכו״ם ודאי קטיל לי׳ ומוטב שיניח אולי יחי׳) ודאי מת מתרפאין מהן (דעכו״ם מאי עביד לי׳ הא בלא״ה מיית ושמא ירפאנו העכו״ם) ופריך האיכא חיי׳ שעה ומשני לחיי׳ שעה לא חיישינן וכבר חלקו שם התוספ׳ בין הך להאי דיומא דחיישינן לחיי׳ שעה דבכל מקום עבדינן לטובתו דבנפל עליו הגל אם לא תחוש ימות ודאי ולכן חיישינן לחיי שעה וברפואה מעכו״ם אם חיישינן לחיי׳ שעה ודאי ימות ולכן לא חיישינן ואפשר שירפא ע״ש והשתא נדון שאלה דלפנינו דמי ממש להך דע״ז דאם לא יעשו לו רפואה זו ימות ודאי ולכן לא חיישינן לחיי שעה דהא אפשר שירפא ויחי׳.\\\vspace{0pt}

וביותר נלענ״ד דאפילו ע״י ישראל עושין דהא אמרינן בב״ק (דף פ״ה) ורפא ירפא מכאן שניתן רשות לרופא לרפאות ואף שרש״י ותוספ׳ פירשו דקמ״ל שלא יאמר הקב״ה מחי ואנא מסי מכ״מ הרמב״ן בת״ה וכן הטור בי״ד (סי׳ של״ו) פירשו שלא יאמר מה לי לצער הזה שמא אטעה ונמצאתי הורג נפשות בשוגג הרי דלא צריך הרופא לחוש לזה ואף שכתב הרמב״ן שם וכן הטור מהתוספתא דרופא שטעה והרג בשוגג גולה על ידו, מכ״מ כתב הטור שם דאין למנוע משום חשש טעות דרפואה מצוה היא ובכלל פקוח נפש היא וכי חמיר איסור ספק רציחה מספק איסור חלול שבת שהרי התירו לחלל שבת לחולה שיש בו סכנה משום פקוח נפש אף דאפשר שאעפ״כ ימות או אפשר שיחי׳ גם בלא רפואה זו מכ״מ התירו לכל ספיקות לחלל שבת החמור דאסור סקילה כש״כ שיש להתיר ספק איסור רציחה דבסייף משום ספק פקוח נפש ועוד דאף דברופא נחשב שוגג אם טעה מכ״מ ישראל העושה ע״פ הרופא נחשב אונס גמור דהרופא הי׳ לו שלא לטעות אבל הישראל עושה משום פקוח נפש לרפאות להחולה ואם ממיתו אונס הוא דרחמנא פטרי׳ – ועוד דלא אפשר ולא קמכוון אמרינן בפסחים (דף כ״ז) דכ״ע ס״ל דשרי והכא לא מתכוון להורגו ובאמת צל״ע אמה שכתב הטור דרופא שהרג ע״י רפואתו גולה דהרי הטור גופא פסק בי״ד (סי׳ ש״א) בלובש כלאים דמותר מטעם דבר שאין מתכווין וה״נ הרופא אין מתכווין להמית רק להחיות ומהתוספתא אין ראי׳ דאפשר דאתיא כמ״ד דדבר שאינו מתכווין חייב וכן הא דאמרינן במכות (דף כ״ב ע״ב) הוסיף לו רצועה א׳ ומת ה״ז גולה על ידו אין ראי׳ לזה די״ל ג״כ דאתיא כמ״ד דבר שאינו מתכווין חייב ואפשר דחשיב זה דבר המתכווין כיון שעכ״פ מתכווין למעשה שעל ידו באה המיתה ועיין במכות (דף ז׳ ע״ב) אבל מכ״מ מאידך טעמי שכתבתי נלענ״ד דמותר לעשות רפואה כזה אפילו ע״י ישראל: הקטן יעקב.\\\vspace{0pt}

\end{multicols}\newpage

\newchap{סימן קיב}
\begin{multicols}{2}
שאלה – מי שמת בנו וביום שאחריו קודם קבורת בנו מתה גם אשתו וקבורת בנו תהי׳ יום אחד קודם קבורת אשתו איך יש להתנהג בקריעה ובאמירת צדוק הדין ושאר דברים הנוהגים בשעת קבורה ומיד אחרי׳.\\\vspace{0pt}

תשובה – מה שאונן פטור מכל המצות בזה יש ב׳ טעמים או שמפני שעוסק במצות קבורה פטור מכל המצות או מפני כבוד המת ולשני הטעמים לא שייך לפטור אונן על מת אחר ממצות קבורה של מת הראשון דאם משום עוסק במצו׳ הרי אדרבא נקרא עוסק במצו׳ במת הראשון שהיא מצו׳ ראשונה ולעולם במצו׳ שחלה עליו ראשון נקרא עוסק במצו׳ לפטור ממצו׳ אחרת כמשכ׳ הרן בברכות פ׳ מי שמתו ואם משום כבוד המת מאי חזית דכבוד מת שני עדיף אדרבא כבוד מת ראשון עדיף ובר מן דין אין לפוטרו ממצות קבורת מתו הראשון משום אנינות דבקבורת המת בלא עשה דקבר תקברנו יש ג״כ לא תעשה דלא תלין כדאמרינן בסנהדרין דהמלין את מתו עובר בלא תעשה והרי אונן לא פטור רק ממצות עשה אבל לא הותר לו לעבור על לא תעשה שבתורה ולכן כיון דמוטל עליו לקבור מתו הראשון מוטל עליו ג״כ לקרוע שהרי קריעה שהיא מדברי קבלה צריך להיות בפני המת עכ״פ למצו׳ מן המובחר כמבואר בפוסקים והיא משום כבוד המת וגם צדוק הדין יאמר מטעם זה ועוד דבלא״ה לא שייך לפטור מטעם אנינות מלומר צ״ה שהרי זה שייך גם למתו השני אבל לעשות שורה ולנחמו נ״ל דלענין זה נוהג אנינות שהרי אמרו אל תנחמנו בשעה שמתו מוטל לפניו ואף דהנחמה אינה על המת המוטל לפניו מכ״מ כשמנחמין אותו בלשון נחמה סתם שייך זה על כל צערו ואין לנחם מי שצער עדיין מוטל לפניו אבל סעודת הבראה צריך לאכול שהרי זה נזכר בלשון איסור כדאמרינן סוף מ״ק סעודה ראשונה אסור האבל לאכול משלו ויליף מקרא דיחזקאל דלחם אנשים לא תאכל וכן שאר דיני אבלות שאינן נוהגין ג״כ בשעת אנינות כגון נעילת הסנדל נראה שצריך לנהוג מטעם זה אפילו קודם קבורת מתו השני דכל שנזכר עליו שם איסור אין לפוטרו מטעם עוסק במצו׳ פטור מ״ה וגם משום כבוד מתו הראשון כנלענ״ד: הקטן יעקב.\\\vspace{0pt}

\end{multicols}\newpage

\newchap{סימן קיג}
\begin{multicols}{2}
ב״ה אלטאנא, יום ג׳ ג׳ תמוז שנת תר״ח לפ״ק. להרה״ג וכו׳ ש״ב מ״ה אלכסנדר אראן נ״י אב״ד דק״ק פעגערסהיים יע״א.\\\vspace{0pt}

נדרשתי לאשר שאל מעכ״ת נ״י ממני לחוות דעתי על מה שחידש שצריך לנהוג אנינות על מת תוך ל׳ ללידתו. הנה בנה יסודו על מה שחלק המג״א על הב״ח ופסק דנפל צריך קבורה מן התורה וכתב שכל שכן שצריך קבורה בספק אם כלו לו חדשיו ולכן כיון שכל שמוטל עליו לקברו חייב באנינות ה״ה ג״כ במת תוך ל׳ עכ״ד אכן לא כן אדמה ולבבי לא כן יחשוב ובתחלה אבאר שלענ״ד אין הדין עם המג״א שחלק (בסי׳ תקכ״ו) על מה שכתב הגהת מיימוני (ה׳ מילה) דאין מצו׳ לקבור נפל ומביא ראי׳ ממה דאמרינן שפחה שהטילה נפל לבור שכל הקושיות שהקשה המג״א עליו יש ליישב שמה שהקשה דהא גם זה קבורה היא כמשכ׳ הר״ש פ׳ י״ד מ״ג באהלות לא מצאתי שם כן ואדרבא במקום אחר באהלות מצאתי ראי׳ להיפך דבאהלות (פ׳ ט״ז מ״ה) תנן בור שמטילין לתוכו נפלים מלקט עצם עצם והכל טהור פי׳ דאין להם תבוסה דלא נתקנה לקבר כמשכ׳ הר״ש שם הרי בפי׳ דנטילה לבור לא מקרי קבורה. עוד השיב המג״א על ראית הג״מ מהא דהטילה נפל לבור די״ל דמיירי בנפל שהוא שפיר וזה דוקא א״צ קבורה ותמהתי איך אפשר לומר כן הא בפסחים (דף ט׳) אמרינן דבא כהן והציץ בו לידע אם זכר אם נקבה ואי בשפיר איירי הרי אין להכיר בין זכר לנקבה דהא אמרינן בנדה (דף כ״ד) דהמפלת שפיר מלא מים אינה חוששת לולד ומפלת שפיר מרוקם תשב לזכר ולנקבה ע״ש הרי דאין להכיר אע״כ דהך נפל שהשליכו לבור נפל גמור הי׳ ואעפ״כ לא נקבר. עוד הביא המג״א ראי׳ דנפל צריך קבורה מדאמרינן נדה (דף נ״ז) דהכותים לא היו קוברים את הנפלים מאי דרוש כל שיש לו נחלה יש גבול משמע דאנן לא קיימ״ל כוותייהו וגם על זה יש לתמו׳ איך לא הזכיר דגם לדידן מצינו דרשה זו שהרי הר״ש באהלות (פ׳ ט״ז משנה ה׳) הביא התוספתא דרשב״ג אומר הנפלים אינם קונים את הקבר ואין להם תבוסה ומפרש שם דכתיב לא תשיג גבול רעך אשר גבלו ראשונים בנחלתך אשר תנחל כל שיש לו נחלה יש לו גבול ושאין לו נחלה אין לו גבול עכ״ל הרי בפי׳ דדרש רשב״ג ג״כ דרשה זו וכיון דת״ק דמתניתן (שם) ס״ל כרשב״ג דבור שמטילין לתוכו נפלים אין לו תבוסה אלמא גם לדידן דרשינן כן ומה דקאמר בנדה מאי דרוש דמשמע דדוקא כותים דרשו כן אפשר דנקיט הכי משום דאשכחן תנא דפליג ארשב״ג בברייתא הובא בסנהדרין (דף מ״ח) דקבר שהטיל בו נפל אסור בהנאה וקאמר שם לאפוקי מרשב״ג דאמר אין לנפלים תפיסת הקבר קמ״ל וא״כ הך תנא ע״כ לא דרש דרשה דלא תשיג גבול אבל כיון דפסקינן כסתם מתניתן דאהלות דנפלים אין להם תבוסה ע״כ לדידן אית לן דרשה זו דכל שיש לו נחלה וכו׳. א״ע ראיתי שזה אינו שאע״ג שפסקינן כסתם משנה דאהלות כמה שכתב הרמב״ם בהל׳ ט״מ (פ׳ ט׳) מ״מ פסק כתנא דברייתא דקבר שהטיל בו נפל אסור בהנאה בה׳ אבל (פ׳ י״ד) וצ״ל דדוקא בור שהטיל בו נפלים אין להם תבוסה משא״כ בקבר אכן מ״מ הגהת מיימוני דפסק דנפל אין צריך קבורה י״ל דפסק כרשב״ג דכבר יש שיטות דס״ל הלכה כרשב״ג בכל מקום גם בברייתא (עיין בספר יד מלאכי) ובפרט שגם סתם מתניתן דאהלות מסייעו שאף שכתבתי דלדעת הרמב״ם צריך לומר דדוקא בור שמטילין בו נפל אין לו תבוסה מ״מ לא משמע כן ממה דאמר ר״ש אם התקינו לקבורה יש לו תבוסה משמע דלת״ק אפילו התקינו אין לו תבוסה וע״כ צ״ל דס״ל כרשב״ג וא״כ מוכח ג״כ להלכה דנפל א״צ קבורה מן התורה ולכן אין ראי׳ ג״כ נגד הג״מ מה שכתב המג״א ממה דאמרינן בב״ב שהיו עושין כוכין לנפלים דמלבד שי״ל שזה הי׳ רק מנהג ולא מן הדין עוד י״ל דזה אתי ע״פ תנא דפליג ארשב״ג אבל לדידן באמת לא היו צריכין כוכין. גם מה שכתב המג״א דבנדה משמע דגם על הנפלים קאי לאו דלא תלין כבר נדחק במחש״ק לפרש כוונתו דבאמת אין שם בנדה רמז מזה. גם מה שהקשה המג״א עוד על הג״מ ממה דדרשינן דאין הכהן מטמא לבנו ולבתו נפלים שמזה מוכח שמצו׳ לקוברן שהרי אין הכהן מטמא אלא לצורך קבורה לא זכיתי להבין קושיתו בזה שהרי בי״ד (סי׳ שע״ג) מייתי ב׳ דעות בזה והש״ע סתם כדעה ראשונה דמטמא הכהן אפילו שלא לצורך קבורה וגם הרמ״א לא כתב רק שנכון להחמיר כי״א משמע שע״פ עיקר הדין פוסק כסברא ראשונה וא״כ מאי קושיא על הג״מ דאפשר דס״ל ג״כ כדעה ראשונה. והיוצא מזה דלא בלבד שאין ראי׳ נגד הגהת מיימוני מכל דברי המג״א אלא אדרבא יש ראי׳ לשיטתו. ועוד נ״ל ראי׳ לדעת הג״מ ממה דתניא במסכת שמחות הביאו הרא״ש פרק ואילו מגלחין המחותך והמסורס והנפלים ובן שמנה חי ובן ט׳ מת אין מתעסקים עמו לכל דבר וכן פסק גם הרמב״ם ה׳ אבל (פ׳ א׳) ומדקאמר אין מתעסקים עמו לכל דבר הרי בפי׳ שאין צריך לקברו.\\\vspace{0pt}

והנה מעכ״ת נ״י רצה להביא ראי׳ דצריך לקבור עכ״פ את המת תוך ל׳ שהוא ספק בן קיימא ממה שכתב הרמב״ן די״א דצריך לקרוע על ספק בן קיימא ומפרש הטעם משום דקריעה דאורייתא וא״כ ה״ה לענין קבורה שהיא ודאי דאורייתא ולענ״ד אדרבא גם מזה ראי׳ להיפך שהב״י שהביא דברי הרמב״ן לא הביא סוף דבריו שם בת״ה שז״ל שם ועוד לדברי הגאונים יום ראשון של אבלות תורה הוא ואעפ״כ בנפלים ספק אינו מתאבל וכן קריעה אינו קורע עליהם עד שיהו בני קיימא ודאי עכ״ל הרי שמסיק דאפי׳ מה שהוא מן התורה אינו נוהג בנפלים ספק. אכן גם לו יהי כדעת מעכ״ת נ״י דבספק נפל צריך קבורה מן הדין מכ״מ אין לדון מזה שגם דיני אנינות נוהגין בו שהרי מה שאונן פטור מן המצות הוא מטעם עוסק במצו׳ פטור מן המצו׳ א״כ איך יפטר ממצות שמוטלין ודאי עליו מכח מצות קבורה שהיא רק ספק דאם נפל הוא הרי א״צ לקבור מן הדין כמו שהוכחתי והרי כלל בידינו שאין ספק מוציא מידי ודאי כדאמרינן פסחים (דף ט׳) ואפילו לאידך טעמא דהירושלמי שטעם אנינות מכח כבוד המת ג״כ יש בזה שאין ספק מוציא מידי ודאי שהרי אם נפל הוא לא שייך כבוד המת דאין מתעסקים עמו בשום דבר כמו שכתבתי ובודאי א״ל שדיני אנינות שהם לחומרא לבד ינהגו בנפל שלא תקנו חכמים אנינות במקצת. ומה שהביא מעכ״ת נ״י ראי׳ שבנפל נוהג אנינות מכח קושית הרשב״א שהקשה ל״ל מוטל לפניו פטור מן התפילין אפילו ביום ראשון נמי לענ״ד גם מזה ראי׳ להיפך שהרי קושית הרשב״א היא בירושלמי ומתרץ כתירוץ הרשב״א תנא אגב הא ע״ש ומדלא מתרץ בירושלמי דמשום נפל הוצרך למינקט משמע בפי׳ שכל שאינו נוהג בו אבילות גם אנינות אינו נוהג. שוב הביא מעכ״ת נ״י ראי׳ לדבריו מהרמב״ם שהביא ראי׳ דאין דין אבלות נוהג כל זמן שלא נקבר המת מדוד שרחץ וסך כשמת הילד ומדהביא ראי׳ מדוד שרחץ דאע״פ שהבן שמת לו הי׳ תוך ל׳ ללידתו מוכח דאנינות נוהג גם בנפל עכ״ד ולא ידעתי כוונתו בזה שממה ידע דמת הבן תוך ל׳ ללידתו והרי אדרבא מהרמב״ם מוכח דס״ל שהי׳ אחר ל׳ דאל״כ איך הוכיח דדיני אבלות אין נוהגין כל זמן שלא נקבר כיון דגם אחר קבורה אין נוהגין בנפל. מכל זה נלענ״ד ברור שאין אנינות נוהג לא בנפל ולא בספק נפל וכמנהג של ישראל כנלענ״ד הקטן יעקב.\\\vspace{0pt}

\end{multicols}\newpage

\newchap{סימן קיד}
\begin{multicols}{2}
ב״ה אלטאנא, יום ג׳ י״א א״ר תרי״ט לפ״ק.\\\vspace{0pt}

שאלה – אשה שמתה והי׳ לה בפי׳ שנים תותבות שנתחזקו בפה ע״י מאשינע של זהב אם השנים והמאשינע אסורים בהנאה.\\\vspace{0pt}

תשובה – גרסינן בערכין (דף ז׳) האשה שנהרגה נהנין בשערה ואמאי איסורי הנאה ננהו א״ר באומרת תנו שערי לבתי ובפאה נכרית ופריך טעמא דאמרה תנו הא לא אמרה תנו גופא הוא ומתסר והא מבעי׳ לי׳ לריבר״ח דבעי ריבר״ח שער נשים צדקניות מהו ואמר רבא בפאה נכרית קמבעי׳ לי׳ ומשני כי קמבעי׳ לי׳ לריבר״ח דתלי בסכתא הכא דמחבר בה טעמא דאמרה תנו הא לא אמרה תנו גופה הוא ומתסר קשיא לי׳ לרנב״י והא דומיא דבהמה קתני מה התם גופא אף הכא נמי גופא אלא אמר רב נחמן זו מיתתה אוסרתה וזו גמר דינה אוסרתה והיוצא מסוגיא דשם דלרב שער המת אסור בהנאה ואפילו פאה נכרית שהי׳ קשור בשערה של אשה שמתה או שנהרגה נאסר ורק באמרה תנו הפאה לבתי מותר ולרנב״י אפילו שער של מת מותר בהנאה דאין השער מת דלא עשוי להשתנות ומייתי בגמרא ברייתא כרב וברייתא כרנב״י והרמב״ם והסמ״ג פסקו כרב נחמן ב״י דשער המת מותר בהנאה אבל הרמב״ן והרשב״א והטור פסקו כרב ולכן כ׳ טהור י״ד (סי׳ שמ״ט) נויי המת המחוברים בגופו כגון פאה נכרית וכיוצא בזה אסורים כמו המת עצמו בד״א בסתם אבל אם צו׳ שיתנו נויי גופו המחוברים בו לבנו או לצורך דבר אחר מותרין אבל שערו ממש אפילו אם צו׳ עליו אסור בהנאה כמו גופו עכ״ל וכן פסק שם בש״ע וכתב בד״מ בגמרא ספ״ק דערכין פי׳ רש״י שמקשרין בשערה ומשמע דאם אינן קשורין מותרין ומזה נהגו לטול טבעות מן המתים דכה״ג לא מקרי מחובר בגופו עכ״ל וכ״כ בהגהה בש״ע וכתב הש״ך בשם הב״ח דאפילו קלועים בתוכם ואינם קשורים אסורים אלא כשאינן קשורים בה כלל דהיינו דתלי בסיכתא מותרים ע״ש ולענ״ד צ״ע דמדברי הפוסקים משמע דפיאה נכרית אסורה מפני שהיא קשורה לגופו של מת נחשבה כגופו ולכן בעינן קשורה דוקא או עכ״פ קלועה בשערה ובגמרא דסנהדרין לא משמע כן אלא דפיאה נכרית נחשב כמלבוש בעלמא דאמרינן שם (דף קי״ב) בעי ר״י שער נשים צדקניות מהו ומפרש רבא בפיאה נכרית וקאמר ה״ד אי דמחובר בגופה כגופה דמיא לא צריכא דתלי בסיבטא כנכסי צדיקים שבתוכה דמי ואבד או דלמא כיון דעייל ונפק כלבושה דמי תיקו הרי שאינו מסופק רק אי כנכסים או כלבוש דמי אבל לא שיהי׳ כגופה והי׳ אפשר לומר שיש שני מיני פיאות נכרית פיאה שקשורה לשער להיות שם לגמרי והיא באמת כגופה ופיאה שאינה קשורה ועיילא ונפקא והיא לפעמים תלי בסיבטא וזו באמת אינה רק כלבוש וכן משמע מלשון הגמרא דבתחלה קאמר אי דמחובר בגופה כגופה דמי ואח״כ קאמר כלבושא דמי והיינו משום דבתחלה איירי מקשורה שאי אפשר לתלות בסיבטא ואח״כ דאיירי מתלי בסיבטא קרי לי׳ לבושא אבל מה אעשה שמדברי רש״י לא נראה כן שגם על כגופא דמי פי׳ וכשם שאין שורפין מלבושים שעליהם שהרי צדקניות הן כך אין שורפין אותו גדיל עכ״ל משמע דמה דקאמר אי כגופא דמי הפי׳ שהוא כמלבוש ולכן נלענ״ד לחלק עוד בענין אחר בשנדקדק דבסנהדרין כתב רש״י על פיאה נכרית גדיל של שערות דעלמא שעושין לנוי עכ״ל ועל מחובר בגופא פי׳ שנקשר בה ולא הזכיר שנקלע עם השיער של ראש ובערכין פי׳ על פיאה נכריות רגילות היו נשים כששערן מועט לקשור שיער נשים נכריות לשערן והוא פיאה נכרית עכ״ל ולא הזכיר שהוא גדיל שעושין לנוי ולכן נראה דשתי מיני פיאות נכריות הן יש שעושין גדיל של שער לעצמו שלא נקלע עם השיער של ראש אלא הוא כמין מגבעת וכ״כ הרמב״ם בפי׳ המשניות שבת פ׳ ו׳ ופאה נכרית כמו מגבעת ידבקו בו שער נאה והרבה ותשים אותו האשה על ראשה דרך עראי כדי שתתקשט בשיער עכ״ל ולכן בסנהדרין שהספק אם פאה נכרית כנכסים דמי או לא ודאי לא הי׳ מסופק בפאה שקשורה בגופה שנחשבת כגופא שתהי׳ כנכסים ולכן שם פי׳ רש״י דהספק היא באותה פאה שהיא גדיל לנוי וקשור בגופה כמו שקושרין מלבוש אבל בערכין שהאיסור הוא משום שהיא כגופא ממש להיות כגוף המת שהרי מלבוש המת אינו אסור בהנאה שם פי׳ רש״י דאיירי מפאה שקושרין השיער לשערן להיות כשערן.\\\vspace{0pt}

והנה מה שנובע מזה לנדון השאלה הוא שכפי מה ששמעתי יש ג׳ מיני שינים תותבות יש שמחזקים שן מעצם בפה בתוך השינים לבל ימוט משם ויש שמחזקים השינים של עצם על כלי (מאשינע) של זהב והוא נאחז בפה בתוך השינים ע״י שנדחק להשינים הטבעיות והתותבות משמשות עם הטבעיות כל תשמישי השינים כאחד ולא נלקחו שם רק לפרקים אחר שבוע או שבועים לנקותם ומיד שוב נתחברו עם הטבעיות ויש שעושין כלי כזה רק לנוי להסתיר המום שחסרים השינים הטבעיות אבל אינם משמשים שימוש השינים דבעת אכילה וכן בלילה כשפושטת האשה תכשיטי׳ תקח גם הכלי הזה מפי׳. והנלענ״ד דמין הראשון שמחובר השן לגוף ממש לבלתי הפרד פשיטא דאסור בהנאה כמת עצמו וכן מין השני שנתחברו השינים ע״י מאשינע לשינים הטבעיות אסורים בהנאה דדומה לפאה נכרית הקלועה בשערות או נתקשרה עמהם דנחשב כגוף המת וה״נ הנך שינים כיון דמשתמשות עם השינים פעולתן הוי כגוף האשה אבל מין השלישי שאין האשה עושה רק לנוי לכסות מומה ואין משתמשת בהן לאכילה זה דומה לפאה נכרית שעושה כמו גדיל ומגבעת לראשה שאף שעושה משער להיות דומה לשער הראש וגם נקשר בגופה מכ״מ לא נדון כגוף רק כמלבוש לפי דברי רש״י בסנהדרין והכא נמי הך כלי שמחוברים בו השינים לא הוי רק לנוי ודומה לטבעות ונזמים שלא נאסרו בהנאה אף שהי׳ בגוף האשה בשעת מיתה כנלענ״ד: הקטן יעקב.\\\vspace{0pt}

\end{multicols}\newpage

\newchap{סימן קטו}
\begin{multicols}{2}
ב״ה אלטאנא, יום ג׳ כ״א שבט תרכ״ג לפ״ק. להרה״ג וכו׳ מ״ה זעליגמאן פראם נ״י אב״ד דק״ק האמבורג יע״א.\\\vspace{0pt}

שאלה – פה העירה יש בית הקברות מקדמת דנא והוא רחוק יותר מג׳ רביעי שעה מן העיר ומונח בתוך יער והדרכים לשם בפרט בעת החורף הם רעים מאוד, וגם אין לעשות שמירה מעולה להקברים והמצבות מחמת ריחוק מקום ומטעם זה אין לעשות שם בית להתאסף שמה ולצורך טהרת המת כנהוג בישראל מחמת חשש גניבה. וכעת שאין שם עוד מקום לקבורה כי נתמלא הביה״ק והוצרכו אנשי הקהלה לקנות מקום אחר (נו״נ) לקבור שם ויכלו למצוא קרקע הסמוכה להקברות הישן ולהקיף חומה סביבה ועם הקברות הישן אבל חפץ הפו״מ וכל אנשי הקהלה לקנות מקום לקברות סמוך לעיר ועי״כ יהי׳ אפשר לעשות שמירה מעולה למתים ולבנות שם בית לאסיפה וטהרה ולהקיף חומה עם שער סגור בדלתים ובריח מבלי שיגנב ואשאל אם יש שום חשש ופקפוק בעשיית בית החיים החדש ולעשות פרישה ממקום מנוחה של אבות בני הקהלה. כ״ד מעכ״ת נ״י.\\\vspace{0pt}

תשובה – בתחלה נחקור אם יש בעשיית הקברות החדש סמוך לישן תועלת לחיים או למתים ויש נפקותא דאם בשביל החיים יכולים למחול אבל אם יש תועלת המתים אינם יכולים למחול. והנה מעכ״ת נ״י הזכיר שדעת מר חמיו הצדיק הגאב״ד דק״ק ווירצבורג נ״י נוטה לאסור הרחקה מן הקברות הישן ע״פ הירושלמי פ׳ מי שהפך ערב הוא לאדם לנוח אצל אבותיו וכ״כ בספר חסידים סי׳ ת״נ ברזילי הגלעדי אמר אמות בעירי כי הנאה יש למתים שאוהבים הולכים על קבריהם וכו׳ ע״ש אמנם לענ״ד משום זה אין למחות לאנשי הקהלה לעשות בית החיים החדש סמוך לעיר כיון דטעמים האלה הם לתועלת החיים עודנה עמ״ש א״כ יכולים למחול על תועלתם וכדאמרינן בסנהדרין (דף מ״ו) אם הספידא יקרא דשיכבי הוא מי שאמר שלא יספדוהו שומעין לו מפני שיכול למחול על כבודו וה״ה על תועלת אחר ומהא דספר חסידים בלא״ה לענ״ד אין ראי׳ שהרי החיים עתה יאמרו גם כשיקברו לאחר מ״ש בסמוך לעיר יבואו אוהבים על קבריהם ואדרבא יבואו יותר ממה שיבואו כשהדרך רחוק ואם מפני התועלת של המתים שכבר מתו זה לא יגרע ע״י הקברות החדש שמי שבא על קבריהם להתפלל על קברי אבות גם אז יבוא רק שלא יבואו לשם לעת קבורת מת חדש ח״ו ואין לחשוב זה לגרעון תועלת המתים כי מצאנו בברכות פ׳ מי שמתו ובכמה דוכתי׳ שהמתים מצערים בצער החיים וכי יחשב הנאה להם כשיבואו החיים לצורך קבורה לשם.\\\vspace{0pt}

אמנם לכאורה יש בעזיבת הקברות הישן חשש פגם כבוד המתים ע״פ מה דנפסק בש״ע י״ד (סי׳ שס״ג) אין מוליכין מת מעיר שיש בה קברות לעיר אחרת אא״כ מחוצה לארץ לארץ וכ׳ הש״ך שם הטעם משום כבוד המתים הקבורים באותה העיר שמבזה אותם שלא לנוח אצלם את זה עכ״ל וא״כ זה הפגם כבוד למתים יהי׳ ג״כ בעזיבת מקום קבורתם לבחור מקום אחר לקברות. אבל א״ע נ״ל שזה אינו שמלבד מה שיש לחלק שזה דוקא באפשר לקבור אצלם אבל בנדון זה שאי אפשר עוד לקבור אצלם אע״פ שאפשר לקנות מקום לקבורה בשכונתם הרי כל זמן שלא נקנה אי אפשר לקבור שם והוי דבר שלא בא לעולם כדאמרינן ב״מ (דף ט״ז) שדה זו לכשאקנה קנוי׳ לך נקרא דבר שלא בא לעולם, בלא״ה לענ״ד יש ראי׳ מזה להיפך שדין זה הוא ממה שהביא הגהת אשרי במ״ק בשם האור זרוע שהעתיק שם בקיצור ועתה שזכינו להאור זרוע (שיצא לדפוס מקרוב) עצמו נראה שאין טעמו כמשכ׳ הש״ך שז״ל האור זרוע (סי׳ תי״ט) ועל ענין הולכת המת מעיר לעיר פרשתי אני המחבר ושאלתי מרבותי על מעשה שנעשה באחד שהתיר בי״ט שני להוליך את המת מעיר לעיר אחרת ע״י היהודים והמת לא צו׳ ובעיר שמת בה הי׳ ביה״ק ולעיר שהוליכוהו שם לא נקברו אבותיו של מת ודבר זה נראה בעיני אפילו בחול אסור כי בזיון המת הוא וגם צער הוא שמטלטלין אותו מעיר לעיר צא ולמד ממת מצו׳ וכו׳ ע״ש הרי שלא הזכיר טעם הש״ך שיש בזיון למתים שאין נקבר עמם אלא בזיון וצער לו לטלטלו למרחוק. וראיתי בשו״ת זכרון יוסף חח״מ (סי׳ ט׳) שהשיג ג״כ מסברא דנפשי׳ על הש״ך דלפי טעמו אף אם מצו׳ להוליכו לעיר אחרת לא נשמע לו והרמ״א כ׳ דאז מותר אע״כ דהטעם משום בזיון וצער המת לטלטלו למרחוק ואולי בין כך ובין כך יסריח או ינשלו אבריו ויהי׳ בזיון המת כאשר באמת קרה הדבר פעמים רבות עכ״ד ועתה נראה מדברי הא״ז שכוון אל האמת ואע״ג דבצוואת ר״י חסיד ז״ל כתוב דאם יש קברות בעיר אין להביא מת לקבור בעיר אחרת מפני בזיון המתים מכ״מ כיון דע״פ פסק הא״ז שפסק הרמ״א כוותי׳ שהיכא שצו׳ שומעין לו הכי נקטינן והנה מדברי הא״ז הנ״ל נלמוד ב׳ דברים לענין נדון השאלה האחד שאין חשש פגם כבוד המתים הקבורים בביה״ק הישן בעזוב אותו שהרי אפילו בשיש ביה״ק בעיר שאפשר לקבור המת שם לא חשש לזה כל שכן בשאי אפשר עוד לקבור שם מבלי קנות קרקע מחדש בשכונתם. והשני, שיש צער ובזיון למת להוליכו למרחוק בשאפשר לקברו במקום קרוב ולכן החיים עתה שרוצים להקבר לאחר מ״ש בקרוב לעירם במקום שישמר קברם ומקפידים על צערם ובזיונם שומעים להם אפילו אם הי׳ בזה בזיון למתים שאין אדם צריך למחול על צערו ובזיונו בשביל אחרים אם לא בשביל בזיון רבותיו ואבותיו אכן כבר הוכחתי שלא שייך בזה בזיון המתים.\\\vspace{0pt}

ומלבד זה אפשר שיגיע לאיסורי דאורייתא דבמדינות האלה שאסור לקבור בלי רשות השררה לפעמים אי אפשר לקבור אלא סמוך לשבת וי״ט כאשר כבר קרה פעמים רבות וזה אי אפשר רק בביה״ק הסמוך לעיר אבל ברחוק אי אפשר מפני שהקברנים לא יכלו לשוב לביתם קודם שבת ועי״ז ילינו המת עד לאחר שבת וכן לא יכלו לקבור בי״ט ראשון מפני התחום דמסתמא כיון שרחוק ג׳ רביעי שעה היא חוץ לתחום וזה אסור בי״ט ראשון כמבואר סי׳ תקכ״ו ועי״ז ילינו המת שלא לכבודו שהוא איסור דאורייתא וגם יבא המת לידי ניוול בפרט בימי החום וגם ניוול המת הוא איסור דאורייתא כדמוכח ממה דאמרינן בחולין (דף י״א) וכי תימא דבדקינן לי׳ הא קא מינוול. ולכן אפילו יהי׳ חשש פגם כבוד להמתים הקבורים בביה״ק הישן בעזוב המקום הרי אמרינן בסנהדרין (דף מ״ו) ניחא להו לצדיקיא דמייקרי בהו אינשי וכן אמרינן שם כל העושה לכבודו של חי אין בו בזיון למת כל שכן שלא יקפידו הצדיקים הקבורים וימחלו על כבודם למען שלא יבואו החיים לידי עבירה ולא לידי בזיון אמנם כבר העלתי שלא מצאנו בשום מקום שיש בזה פגם כבוד למתים אם לא יקברו עוד אצלם. ומה שחשש מר חמיו הגאון נ״י מפני מה שכתוב בספר חסידים (סי׳ תש״ט) קהל אחד רצו ללכת למקום אחר ובא מת לאחד מהם בחלום ואמר אל תעזבו אותנו כי יש לנו הנאה כשתלכו לביה״ק ואם תעזבונו דעו כי תהרוגו ולא חששו ונהרגו כולם עכ״ל לענ״ד אינו דומה לנדון זה כי שם עזבו לרצונם וכאן מפני הדין ועוד דכאן לא נקרא עזיבה שאשר הלכו על קברי אבות גם עתה ילכו כנ״ל ועם כל זה טוב לעשות הקהל תקנה לחוק ולא יעבור שילכו שלוחי הקהל איזה פעמים בשנה בפרט בימים שנוהגים ישראל להתפלל על הקברות בימי המצרים וקודם ר״ה וקודם יוה״כ על ביה״ק הישן ויתפללו שם וגם יבקשו כל הקהל מחילה מן המתים בעזבם הקברות הישן ואז מותר להקהל לבחור להם מקום לקברות סמוך לעיר וד׳ יתן הטוב שישלח גואל צדק ויבלע המות לנצח ב״ב: כנלענ״ד הקטן יעקב.\\\vspace{0pt}

\end{multicols}\newpage

\newchap{סימן קטז}
\begin{multicols}{2}
ב״ה אלטאנא, אלול חרי״ז לפ״ק.\\\vspace{0pt}

ראיתי בספר מי נפתוח (דף י״א) שמסופק אם עבד כנעני מטמא באוהל כיון דאמרינן עכום אינו מטמא באוהל דאתם קרויים אדם א״כ עבד דאין לו חייס דעם הדומה לחמור כתיב וגם מת עבד מותר בהנאה כמש״כ התוספ׳ ב״ק (דף י׳) א״כ אינו קרוי אדם או דילמא כיון דחייב במצות שאשה חייבת בהן הוא בכלל אדם וסיים כעת לא מצאתי גילוי לדין זה:\\\vspace{0pt}

לענ״ד יש לפשוט ספק זה ממש״כ התוספ׳ מגילה (דף כ״ג ע״ב) ד״ה ואדם דעבד כנעני קרוי אדם מדדרשינן בגטין (דף ל״ח) מכל חרם אשר יחרם מן האדם לרבות עבדים כנענים ע״ש וא״כ הרי בפי׳ דעבדים קרויים אדם. הן אמת דמה שהוכיחו התוספ׳ מהא דגטין לא שייך לשיטת ר״ת שהביאו התוספ׳ יבמות (דף ס״א) דחילק בין אדם להאדם דעכום בכלל האדם הם אבל לא בכלל אדם דלפ״ז אין ראי׳ משם כיון דשם כתיב האדם אבל במתניתן דמגילה דכתיב ואדם כיוצא בהם בכלל אדם עבדים אינם ולכן פריך שפיר ואדם מי קדוש וע״כ צ״ל דקושית התוספ׳ היא לרש״י לשיטתו דלא ס״ל חילוק של ר״ת כדמוכח ביבמות שם וכן רבינו משולם שם לא ס״ל חילוק של ר״ת ולכן אליבי׳ דהנך שיטות שפיר נפשט הספק דעבד כנעני נקרא אדם ומטמא באוהל: כנלענ״ד הקטן יעקב.\\\vspace{0pt}

\end{multicols}\newpage

\newchap{סימן קיז}
\begin{multicols}{2}
ב״ה אלטאנא, יום ו׳ כ״ב כסליו תר״י לפ״ק.\\\vspace{0pt}

ראיתי לחקור בהא דאמרינן להזהיר גדולים על הקטנים אם זה דוקא באותן מצות שהגדול עצמו מצו׳ עליהן או אפילו בשהוא אינו מוזהר ויש נפקותא אם ישראל מוזהר שלא לטמא כהן קטן או לא. והנה בפשטות הי׳ נלענ״ד כיון דהא דלהזהיר גדולים על הקטנים ילפינן ביבמות (דף קי״ד) מאמור אל הכהנים ואמרת אליהם להזהיר גדולים על הקטנים הרי דלא הזהיר רק לכהנים ולא לישראל וכן מה שהזהיר קרא עוד גבי שרצים ודם גדולים על הקטנים כדאמרינן שם היינו ג״כ בשהגדולים מוזהרים שהרי הם איסורים שוים בכל ומזה ילפינן כל שאר איסורים שבתורה א״כ לא ידענו רק בכה״ג שהגדולים מוזהרים לעצמם אז מוזהרים גם על הקטנים ובזה יש ליישב ג״כ מה דאמרינן בפסחים (דף כ״ב) ובע״ז (דף ו׳) מניין שלא יושיט אדם כוס יין לנזיר ואבר מן החי לב״נ ת״ל ולפני עור לא תתן מכשול והתוספ׳ בע״ז העירו למה נקט הני דוקא ולא מניין שלא יושיט לישראל שאר איסורים שבתורה ונדחקו ליתן טעם לזה וגם שאר הראשונים נדחקו בזה אבל לפ״ז י״ל דהנה הרא״מ פ׳ שמיני הביא בשם ר׳ אברהם ן׳ עזרא לפרש מה דאמרינן בת״כ לא יאכלו לחייב את המאכיל כאוכל דאם האכיל עוף טמא לישראל גדול לוקה ועיין בקרבן אהרן שהאריך בזה ומסיק ג״כ דמה דאמרינן ביבמות להזהיר גדולים על הקטנים היינו דלענין קטנים שאינם בני חיובא אינו רק באזהרה אבל לענין ליתן לגדול הוי לאו גמור ע״ש אכן אפילו נימא דהדרשות דמייתי הש״ס ביבמות רק אקטנים קאי מכ״מ הסברא נותנת דאם על הקטן שאינו בר חיוב מוזהר שלא לספות איסור בידים על הגדול שהוא בר חיוב לא כל שכן ולכן לענין כל איסורים שבתורה שלא לספות לגדול ידענו מטומאה ודם ושרצים אכן זה דוקא באותם איסורים שהוא בעצמו מוזהר עליהם כמו הני אבל להושיט לנזיר שהוא אינו מוזהר בעצמו וכן להושיט לב״נ דלא שייך למילף מהני תלת דלא כתיבי רק בישראל מנ״ל לזה הוצרך לומר דילפינן מלפני עור לא תתן מכשול דזה ודאי שייך גם בשהוא אינו מוזהר ולכן נקט הני תרתי כוס יין לנזיר ואבמ״ה לב״נ וא״כ גם מזה מוכח דבהוא אינו מוזהר לא שייך איסור לספות בידים רק היכי דאיכא משום לפני עור ואף דבלא קאי בתרי עברי דנהרא ליכא משום לפני עור כדאמרינן בע״ז (שם) מכ״מ מדרבנן צריך לאפרושי מאיסורא כמו שכתבו התוספ׳ שבת (דף ג׳ ע״א) ע״ש והיוצא מזה דבקטן דלא שייך גבי׳ משום לפני עור כיון דהוא אינו מצו׳ וגם משום לאפרושי מאיסורא לא שייך גבי׳ מהך טעמא כיון שאצלו אין איסור וליכא רק משום דגלי קרא להזהיר גדולים על הקטנים וכיון דלפי מה שהוכחתי לא שייך זה רק באותן איסורים שהוא עצמו מוזהר יהי׳ מותר לישראל לטמא כהן קטן אלא שמדברי הרוקח שהביא הש״ך בי״ד וגם המג״א (סי׳ שמ״ג) נראה שאינו כן שכתב שמותר לכהנת מעוברת לכנוס לאוהל המת ולא חיישינן לטומאת העובר מטעם ספק ספקא שמא נקבה הוא ושמא נפל והנה מה שהקשה המג״א עליו דתיפוק לי׳ דהוי טהרה בלועה כבר תרצו האחרונים זה בכה וזה בכה אכן לפי הנ״ל יותר יקשה דתיפוק לי׳ כיון דכהנת אינה מוזהרת על הטומאה ג״כ אינה מוזהרת לטמא כהן קטן שהרי רק לבני אהרן הזהיר הכתוב גדולים על הקטנים אלא ע״כ דפשיטא להרוקח דגם כהנת מוזהרת על כהן קטן וא״כ ה״ה ישראל אבל לא ידעתי מנ״ל כן ובפוסקים לא מצאתי דין זה מפורש: הקטן יעקב.\\\vspace{0pt}

\end{multicols}\newpage

\newchap{סימן קיח}
\begin{multicols}{2}
ב״ה אלטאנא, יום ו׳ א׳ דר״ח אייר תר״י לפ״ק.\\\vspace{0pt}

על חקיותי הנ״ל כתב לי הרה״ג וכו׳ מ״ה משה שיק נ״י הגאב״ד דק״ק יערגען וז״ל – מעכ״ת נ״י מסופק דלמא מה דפסקינן דאסור לספות בידים איסור לקטן הוא דוקא בשהגדול עצמו מוזהר על האיסור ולכאורה אמרתי להביא ראי׳ לדבריו מנזיר (נ״ז ע״ב) דלר״ה רשאי אשה להקיף קטן ואמאי הא אסור למספי ועיין בפרמ״ג (סי׳ שמ״ג) שהעיר ג״כ בזה ובשלומא לראב״א כ׳ השער המלך פי״ב מע״ז דגז״ה בהקפה דאקיש מקיף לניקף דהיכא דניקף פטור אין כאן איסור כלל אבל לר״ה קשה אלא עכח כיון דהאשה אינה באזהרת לא תקיפו לא אסור לה למספי וזה לכאורה ראי׳ גדולה אמנם לפמ״ש הכ״מ בפי״ב מע״ז היכא דניקף אינו מסייע לית בי׳ איסור כלל לניקף יש לדחות דלמא ר״ה דמתיר לאשה להקיף קטן היינו באינו מסייע אמנם יש להביא ראי׳ בהיפך בנידה מ״ו ע״ב אמרי׳ לענין נדר ושבועה תלי׳ בפלוגתא דקטן א״נ וכו׳ ומדנקט לשם א״נ הוא מוכח דאפילו בכה״ג שאין המאכיל באזהרה תלי׳ בפלוגתא וכן מוכח מתוס׳ שם ד״ה אין ב״ד וכו׳ שהקשו מר״י אלא תירצו הכא שאני ועכ״ח מדנקט אוכל נבילות הוא מוכח דשווין וכן מוכח בגיטין (נ״ה ע״א) וברש״י ובתוס׳ שם ד״ה ועל הקטנה וכו׳ דאע״ג דהבעל כהן ואינו מוזהר על עצמו מוזהר עלי׳ ועדיין אפשר לומר בוודאי למ״ד א״נ מצווין להפרישו א״כ האיסור מכח הקטן באמת אפי׳ מי שאינו מוזהר מ״מ מצווה עליו להפרישו אבל לדידן דקיי״ל דאינו מצווה עליו להפרישו אלא דאסור למספי׳ לי׳ בידים וא״כ האיסור מצד המאכיל א״כ יש לומר דווקא אם המאכיל מוזהר עליו ובסברא זו הי׳ מיושב קו׳ המ״ל שם על התוס׳ בנידה דף מ״ו ע״ב ד״ה אין ב״ד וכו׳ דלמה לא הקשו התוס׳ משום דאסור למיספי בידים וכמ״ש הרשב״א ולהנ״ל א״ש דשם דהבעל אינו מוזהר לא שייך בי׳ איסור למספי עכ״פ מדברי הרשב״א המובא במ״ל שם מוכח דאפי׳ בכה״ג אסור למספי וכן מוכח מתוספ׳ פסחים פ״ח ד״ה שם וכו׳ וכ״כ הט״ז רסי׳ שע״ג בי״ד ובא״ח סי׳ שמ״ג וכן נלענ״ד עוד מוכח מדברי הרמב״ם פ״ו מה׳ תרומות דין ד׳ דאוסר לחרשת אפי׳ השיאה אבי׳ לכהן לאכול בתרומה מגזירה דחרש בחרשת והכ״מ תמה שהוא נגד הש״ס בגיטין נ״ה ולענ״ד פשוט דהרמב״ם סובר דמה שהבעל מאכיל חשוב למספי בידים וקשי׳ לי׳ סוגי׳ דגיטין ש״מ קטן א״נ הוא ונראה דס״ל נהי דבש״ס יבמות קי״ד מסיק דאסור לכ״ע למספי מיהו לר׳ יאשי׳ דס״ל בפ״ג דערלה בירושלמי דקרא דלא יאכל חמץ בא ללמד דאסור למספי וא״כ קשיא תרתי בדם ובחמץ ל״ל וכן לר״ש דס״ל כ״ש למכות לא שייך צריכותא דש״ס יבמות קי״ד דבשרצים אסור בכ״ש הא בכל איסורין אוסר בכל שהוא ועכ״פ לדדהו הוי ב׳ כתובים ואפי׳ למספי שרי בעלמא וע״ז פריך הכא בגיטין וליכול קטן א״נ הוא ואיך סותם המשנה בלי מחלוקת אבל לדינא דקיי״ל דאסור למספי בכל איסורין שפיר י״ל דהגזירה הוא משום חרש בחרשת דהוי למספי בידים כנלענ״ד נכון ופשוט ועכ״פ מוכח דס״ל להרמב״ם כמ״ש הט״ז דאפי׳ במקום שאין המאכיל מוזהר מ״מ אסור למספי וכן מוכח עוד ביבמות ע״א דקבעי לי׳ מהו לסוך ערל קטן בתרומה ושם עכ״ח בכהן מיירי דבישראל בלא״ה הנאה של כילוי אסור ואפ״ה פשיטא לי׳ דאסור למספי׳ וזה יש לדחות מ״מ ראיות הנ״ל הם ברורות אמנם טעמא בעי מנ״ל באמת כן כמו שהקשה מעכ״ת נ״י ונראה דאם הוי כתיב האזהרה על המאכיל הי׳ אפשר לחלק אבל באמת אין מקרא יוצא מידי פשוטו האזהרה על האוכל קאי אלא מיתורא דקרא דלא תאכלום כי שקץ הם בא לרבות דהלאו גם על הקטנים וכן מקרא דוכל נפש לא תאכל דם גם על קטנים וכיון דקטן לאו בר קבולי אזהרה הוא עכ״ח הכוונה על המאכיל דוגמא דכתיב בפסח כל בן נכר לא יאכל בו ופירש הרמב״ם על המאכיל מ״מ כיון דאזהרה על הקטן נאמר בדידי׳ תליא מלתא ואפילו גדול שאינו בכלל אותו אזהרה מוזהר עליו כיון דהקטן מוזהר ומצידו כלם חייבין עכ״ד הרב הנ״ל נ״י.\\\vspace{0pt}

וכעין הדברים הנ״ל השיב לי ג״כ הרב וכו׳ מ״ה פנחס שיפפער נ״י מק״ק לעמבערג – וז״ל לפענ״ד פשוט דאפילו הוא בעצמו אינו מוזהר אסור להאכיל לקטן וראי׳ מהא דגרסינן פסחים (דף פ״ח) על מתניתן דשם יתום ששחטו עליו אפוטרופסין יאכל מאיזה שירצה ש״מ יש ברירה ואר״ז שה לבית מ״מ וכ׳ בתוס׳ ד״ה שה לבית וא״ת האיך מאכיל פסח שלא למנויו ונהי דקטן אוכל נבילות אין ב״ד מצווין להפרישו לספות לו בידים אסור כדאמרינן בפ׳ חרש ע״כ יעיי״ש מבואר להדיא דאע״ג דהאפוטרופוס מותר לו לאכול מפסח זה ואינו מוזהר עליו אעפ״כ מוזהר הוא שלא להאכילו לקטן שאינו מנוי עליו. עוד נראה להביא ראי׳ מהא דגרסינן גיטין דף נ״ה ועל קטנה בת ישראל שנישאת לכהן שאוכלת בתרומה ומקשה עלה הגמרא בפ׳ האשה רבה למ״ד אין כח ביד חכמים לעקור דבר מן התורה א״כ איך אוכלת בנשואין דרבנן בתרומה ומתרץ הגמרא בתרומה דרבנן והקשו שם בתוספת ד״ה ועל הקטנה וז״ל וא״ת ולוקמי׳ אפי׳ בתרומה דאורייתא דקטן אוכל נבילות הוא ותרצו בתי׳ אחד וז״ל ועוד דאוכלת בגינו משמע דאפי׳ מאכילה בידים ובידים אסור לספות מבואר להדיא דאפי׳ כהן דמותר בתרומה ואינו מוזהר עלי׳ אסור להאכיל קטנה בת ישראל בתרומה. עוד ראי׳ מהא דגרסינן נדה (ד׳ מ״ו ע״ב) יתומה שנדרה בעלה מפר לה ומקשינן עלה דאי מופלא סמוך לאיש דאורייתא היאך אתי נשואין דרבנן ומבטלי נדרא דאורייתא ופריק רבא בר שילא דהוי קטן אוכל נבילות והא כו׳ ומסיק כדרב פנחס משמי׳ דרבא דאמר כל הנודרת על דעת בעלה נודרת וכ׳ הרשב״א ביבמות פ׳ חרש (ד׳ קי״ד) דעיקר הגירסא דגרסינן אלא כדרב פנחס דתו לא צריכינן לתי׳ רבא בר שילא דאל״כ קשה על תי׳ רבא בר שילא דלכן מפר בעל דהוי קטן אוכל נבילות דהא כיון דמפר לה בעל הוי כמאן דספי לה בידים ואסור אפילו למ״ד קטן אוכל נבילות אין ב״ד מצוין להפרישו לכן שפיר הגירסא אלא כדר״פ כו׳ עכת״ד הרשב״א בקיצור הרי מבואר להדי׳ אע״ג דהוא אינו מוזהר אסור להאכיל לקטן דהא כאן הבעל לא נדר על הדבר שנדרה היתומה אשתו ואעפ״כ כ׳ הרשב״א דלכן גרסינן אלא כדר״פ כו׳ כדי שלא תקשי היאך יכול להפר לה הא הוי כמאן דספי בידים ואם איתא הא מותר לו לספות לה בידים דהא דבר הנדור לה אינו אסור לו והוי דומיא ממש דמשיט כוס יין לנזיר דהמושיט מקרי אינו מוזהר כיון שהוא אינו נזיר ועיין שו״ת אמונת שמואל ובאורח משור על נזיר בהגהותיו על ד״מ יו״ד סי׳ נ״ה. ואין להביא ראי׳ איפכא דהיכי דאינו מוזהר בעצמו מותר לספות לקטן מהא דגרסינן ב״מ (דף י׳ ע״ב) א״ב איש דאמר לאשה אקפי לי קטן למ״ד דתלוי באם השליח בר חיובא כאן המשלח חייב כיון דאשה אינה בת חיובא א״כ תקשי אמאי לא תתחייב האשה כשמקפת לקטן מטעמא דלא תאכילום א״ו צ״ל כיון דהיא אינה מוזהרת אינה עוברת ג״כ אלאו דלא תאכילום אמת שכן מצאתי שהקשה הפרמ״ג או״ח (סי׳ שמ״ג) במשבצת זהב וז״ל קשה דהשוה לכל עונשין כו׳ ויליף מ״ג כתובים דלא תאכילום בידים עכ״ל ובאמת אשתמטתי׳ גמרא ערוכה נזיר (נ״ח ע״ב) דמבואר שם טעם הדבר דגלי קרא בהדיא דכל שאיננו בבל תשחיתו אינו בבל תקיפו דהיינו מה שכולל לאו דלא תקיפו דהוא המקיף והניקף ואשה כיון דאינה בבל תשחיתו אינה בבל תקיפו דהיינו דמותר לה להקיף עצמה גם מותר לה להקיף אחרים אפילו לגדול וא״כ שאני התם דגלי קרא משום דאינה מוזהרת על בל תשחיתו לכן מותרת להקיף אפילו לגדול משא״כ בעלמא אפילו אינו מוזהר בעצמו אסור לספות לקטן דבר שאסור לו אם הי׳ גדול כמו שמבואר מהראיות שהבאתי עכ״ל.\\\vspace{0pt}

על דברי הרה״ג נ״י הנ״ל אשיב: תמהתי על דבריהם – הוכחתי מן הגמרא מדילפינן שלא לספות בידים מטומאת כהנים שמזהיר הכהנים גדולים על הקטנים ומשרצים ומדם דהם איסורים הנוהגים גם באותם שמוזהרים שלא לספות דרק בכה״ג יהי׳ איסור אבל כתבתי שמדברי הרוקח שנראה מדבריו שגם כהנת אסורה לטמאות את בנה מוכח אפכא ושמשמע מדברי הפוסקים שלא השיבו על דברי הרוקח שמסכימים עמו בזה אבל לא מצאתי דין זה מפורש בפוסקים – כן היו דברי – ולכן אפילו יהיו ראיות הרבנים נ״י מוכרחות אין בזה תשובה על דברי הוכיחו מדברי התוספ׳ פסחים (דף פ״ח) ומדברי התוספ׳ גטין (דף נ״ה) ומדברי הרשב״א נדה (דף מ״ז) ומדברי הט״ז שס״ל ג״כ הכי ואנכי ראיות מן הגמרא המתנגדים לראיות שהבאתי בקשתי או עכ״פ מראה מקום שמבואר דין זה מפורש בפוסקים אחרי שהוכחתי שמשמעות הפוסקים כן אכן גם בענין הראיות לענ״ד יש להשיב שמדברי התוספ׳ פסחים (דף פ״ח) פשיטא שאין ראי׳ דמה בכך שהאב אינו מוזהר על פסח זה שנמנה עליו מכ״מ הרי איסור אכילת פסח שלא למנויו גם אצלו איכא והרי יש כמה פסחים בעולם שמוזהר עליהם ואין זה דומה לישראל וכהנת שאינם מוזהרים אטומאה כלל והרי הרב אב״ד דק״ק יערגען נ״י בעצמו כתב שמדברי התוספ׳ נדה (דף מ״ו) מכח קושית המ״ל מדלא הקשו כהרשב״א דהפרה הוי כמיספי בידים משמע דס״ל דהיכי דאינו מוזהר גם למיספי מותר וא״כ יהי׳ סתירה בדבריהם אע״כ מוכח כנ״ל. גם מה שהביאו הרבנים נ״י ראי׳ מדברי התוספ׳ גטין (דף נ״ה) שכתבו ועוד דאוכלת בגינו משמע שאפילו מאכילה בידים עכ״ל לענ״ד אינו ראי׳ שאף אם מותר ליתן לקטן איסור בידים היכי שהוא אינו מוזהר מכ״מ בתרומה ודאי יש איסור בכהן להאכיל למי שאינו ראוי לאכול דלא עדיף ממפסיד תרומה או ממאכיל תרומה לבהמה דאסור כמבואר ברמב״ם ה׳ תרומות (פ׳ י״ב) ולכן גם הראי׳ שהביא הרב דק״ק יערגען נ״י ע״פ פירושו בהרמב״ם משמא יאכיל חרש בחרשת אינה ראי׳ כלל שהרמב״ם ס״ל כן ובלא״ה מלבד כל הדחוקים שבפי׳ הזה כמדומה שבמכ״ה שכח הרב נ״י שאם כל הגזירה תהי׳ שמא יאכיל חרש לחרשת בידים הרי מטעם קטן אוכל נבלות אין ב״ד מ״ל לא חיישינן שתאכל החרשת תרומה כש״כ שאין לנו לחוש שמא יאכיל החרש בידים דמה לנו אם יעשה החרש מעשה איסור לכן ראי׳ זו מהרמב״ם ודאי אינה ותמהתי איך כתב הרב נ״י על זה שהוא נכון ופשוט (ובחדושי ליבמות כתבתי יישוב אחר לשיטת הרמב״ם) וגם הראי׳ מקטן ערל מהו לסוך ע״פ דברינו לא תהי׳ ראי׳ אפילו אם בלא״ה לא תהי׳ מופרכת מכל צד כמו שכתב הרב נ״י בעצמו שיש להשיב עלי׳ וא״כ לא נשאר לנו מכל הראיות אלא מה שכתב הרשב״א לענין נדר ומה שהוציא הט״ז ממשמעות לשון הטור ולא בלבד שאין הפי׳ של הט״ז בדברי הטור מוכרח אלא אפילו יהי׳ כוונת הטור שגם ב״ד מוזהרים שלא לטמאות כהן קטן בידים לא עדיף ראי׳ זו ממה שהוכחתי מדברי הרוקח ויקשה גם עלי׳ כמו שהקשתי מנ״ל לומר כן כיון דקרא לא מזהיר רק לבני אהרן על הקטנים ולכן אפילו מה שרצה הרב אב״ד דק״ק יערגען נ״י לתרץ על קושיתי בדוחק משרצים ודם מדכתיב האזהרה על הקטן האוכל (מה שבלא״ה א״א לומר כן שלא שייך אזהרה על הקטן שאינו בן דעת) לא שייך לענין טומאה ששם האזהרה ודאי על הגדול קאי ואזהרה על הקטנים לא אתיא רק מואמרת אליהם דודאי קאי על הגדול ג״כ ומהיכי תיתי לומר שגם ישראל וכהנת יהיו מוזהרים עליו וגם לענין שאר איסורים יקשה היאך שייך למילף מדם ושרצים וטומאה מה שאינו דומה להם היכי שאינו מוזהר המאכיל כמו בהם והרי גם התוספ׳ בפסחים (דף פ״ח) כתבו כה״ג דאין ליליף מדם ושרצים אלא מה שדומה להם ואם שמדברי הרשב״א משמע דגם בנדר שייך איסור ליתן בידים הרי אדרבא מהתוספ׳ מדלא נקטו כן משמע אפכא דס״ל דכיון דהוא לא מוזהר אין איסור ג״כ לספות לה.\\\vspace{0pt}

והנה מלבד מה שכתבתי כבר קצת ראי׳ לשיטה זו ממה דנקט כוס יין לנזיר ולא שאר איסורים ומלבד מה שכתבו כבוד הרבנים נ״י עכ״פ לחד שיטה מהא דמקיף לקטן עוד נלענ״ד להוכיח כן מכח קושית המ״ל (סוף ה׳ מ״א) שהקשה אהא דקאמר לאותן המוזהרים עליו ש״מ קטן א״נ ב״ד מצווין להפרישו דמאי פריך דאימא לאותם המוזהרים עליו היינו שלא להאכילו בידים וע״ש שהניח בקושיא אבל לפי הך שיטה א״ש דלא שייך בזה איסור להאכיל כיון דבנדר ונשבע הקטן קמיירי ובזה אין המאכיל מוזהר הנה אמת לכאורה יש ראי׳ ג״כ אפכא לשיטת הרשב״א ששייך איסור ליתן בידים גם בנדר שהוא אינו מוזהר מסוגיא דנדרים (דף ל״ז) דמוקי הא דלא ילמדנו מקרא בקטן המודר וכתב הר״ן שם דאע״ג דקטן אוכל נבלות אבמ״ל אפילו הכי לא ספינן לי׳ בידים וכשמלמדו הרי הוא מהנהו עכ״ל וא״כ מוכח מהגמרא דאפילו בנדר שייך איסור ליתן בידים אבל באמת אין ראי׳ משם דא״כ יקשה בלא״ה הרי התוספ׳ פסחים (דף פ״ח) הוכיחו מהא דמאכילין פסח לקטן שאינו ממנוייו דהיכי שיש חנוך מצו׳ לא ילפינן מדם ושרצים לאסור והרי אין לך חנוך מצו׳ גדול מזה ללמד תורה לבן חבירו ואיך שייך לאסור מטעם איסור ליתן בידים ולכן יש להוכיח כפי׳ אחר שכתב הרשב״א בנדרים שם דאיירי בקטן מופלא סמוך לאיש וס״ל כרב הונא דהוא עצמו מוזהר ע״ש שבפי׳ אי נמי כתב נמי כפי׳ הר״ן ולכאורה יקשה על הרשב״א לשיטתו מה דחקו לפרש במופלא אחר שהוא לשיטתו סובר דנדר מקרי איסור ליתן בידים ובודאי אין לך נתינה בידים גדול מזה כשמלמד עמו ומעבירו על נדרו אע״כ דגם הרשב״א הוקשה לו כן דבמצו׳ לא שייך זה או שהרשב״א עצמו מסופק בשיטתו וא״כ ודאי אין מזה ראי׳ לשיטת הרשב״א שגם בהוא אינו מוזהר יהי׳ איסור לתת בידים. ולכן אף שאינני כדי לחלוק להקל אפילו על משמעות קצת מהראשונים מכ״מ נלענ״ד להלכה ולמעשה היכי שיש צורך לטמאות לכהן קטן שמוטב שיעשה ע״י ישראל או כהנת משיעשה ע״י כהן וכן היכי שצריך ליתן איסור בידים לקטן שמוטב שיעשה ע״י מי שאינו מוזהר בעצמו עליו. הקטן יעקב.\\\vspace{0pt}

\end{multicols}\newpage

\newchap{סימן קיט}
\begin{multicols}{2}
ב״ה אלטאנא, יום ו׳ ו׳ כסליו תרט״ז לפ״ק.\\\vspace{0pt}

נשאלתי – אשה ישראלית הפילה נפל שכבר נגמר באבריו ורצה רופא ישראל להשהותו ולהשימו בצלוחית וביין שרף למען יעמוד ימים רבים להתלמד בו חכמת הטבע כמנהג הרופאים אם מותר לעשות כן.\\\vspace{0pt}

תשובה: בדין זה יש ג׳ חקירות: בראשון: מצד חיוב קבורת נפל ובזה יש פלוגתא בין הפוסקים דהגהת מיימוני ה׳ מילה כתב דאין מצוה לקבור נפלים וכוותי׳ נפסק בא״ח (סי׳ תקכ״ו) ובי״ד (סי׳ רס״ג) דאין קוברים נפלים בי״ט והיינו משום דליכא מצות קבורה ולא משום לא תלין אמנם המגן אברהם בסי׳ תקכ״ו חלק על פסק הג״מ וב״י והוכיח דיש חיוב קבורה בנפלים ע״ש ולפ״ז יהי׳ איסור להשהות הנפל בביתו מטעם חיוב קבורה ומטעם לא תלין אכן באמת משום הא לא אריא דכבר השבתי (בסימן קי״ג) על כל ראיות המגן אברהם והוכחתי כהג״מ וב״י דאין חיוב קבורה בנפלים ולא אכפיל פה הדברים.\\\vspace{0pt}

בשנית: יש לומר דאפילו אין חיוב קבורה בנפלים מכ״מ אסור להשהותו משום תקלה שמא יהנה בו כיון שמת אסור בהנאה וראי׳ לזה ממה שכתב התוספ׳ י״ט בשבת (פ׳ י׳) אמה דאמרינן שם דהמוציא כזית מן המת חייב דאף דאינו חייב בהוצאה רק על המצניעין כמוהו מכ״מ חייב בכזית מן המת דכיון דמת אסור בהנאה מצניעין אותו כדי לקוברו ומזה רצה השואל בנודע ביהודה מ״ק חי״ד סי׳ צ׳ להביא ראי׳ דגם נפל צריך קבורה ואף שהנו״ב השיב לו דשאני כזית מן המת דמצו׳ לקברו מטעם בזיון מנפל שאין שייך בו בזיון מכ״מ כתב שם וז״ל ומעתה גם נפל אע״פ שאין מצו׳ בקבורתו מכ״מ אפרו אסור ומהנקברין הוא שעכ״פ צריך לקברו שלא יכשל בו להנות ממנו כמו כל איסורי הנאה הנקברין עכ״ל ולפ״ז בנדון השאלה אף דמצד חיוב קבורה לא יהי׳ איסור דהיינו שאינו עובר על לא תלין להשהותו יום או יומים מכ״מ יהי׳ איסור להשהותו לגמרי משום חשש שמא יהנה אכן גם על זה יש להשיב דמה דפשוט להנו״ב דנפל אף שאין טעון קבורה מכ״מ אסור בהנאה לענ״ד צ״ע טובא דכיון דהא דמת אסור בהנאה ילפינן בסנהדרין (דף מ״ז) מגז״ש דשם דכתיב ותמת שם מרים ותקבר שם וילפינן לאביי מעגלה ערופה ולרבא מע״ז ע״ש א״כ י״ל דדוקא מת הטעון קבורה דומיא דמרים אסור בהנאה ובפרט דלפי הנראה משם דכתיב גבי ותקבר ילפינן הגז״ש דהוא המיותר דשם ראשון לא מיותר שצריך לגופא שבקדש מתה מרים וכיון דכן מנ״ל ללמוד דגם מת שאין טעון קבורה אסור בהנאה ואין להשיב ממה דאמרינן בסנהדרין (דף מ״ח) קבר חדש מותר בהנאה הטיל בו נפל אסור בהנאה הרי דאפילו קבר נפל אסור בהנאה כל שכן נפל עצמו די״ל דהרי אמרינן שם דהך תנא פליג ארשב״ג דס״ל דאין לנפלים תפיסת הקבר וטעם דרשב״ג מביא הר״ש אהלות (פ׳ ט״ז) מלא תשיג גבול רעך וגו׳ כל שיש לו נחלה יש לו גבול ע״ש ומדרשה זו אתי ג״כ דאין קבורה לנפלים כמש״כ (בסימן הנ״ל) וא״כ תנא דברייתא דסנהדרין דקבר נפל אסור בהנאה ע״כ לא דרש דרשה זו וא״כ סובר באמת דנפל טעון קבורה אבל להג״מ דנפל אין טעון קבורה באמת י״ל דאין אסור בהנאה דלא שייך למילף בגז״ש דותקבר שם. והנה י״ל דתלי זה בפלוגתא שבין הראשונים שהביא המשנה למלך (סוף ה׳ אבל) לענין מת עכום אם מותר בהנאה דהטעם דסוברים שמותר בהנאה הוא כמש״כ שם בשם הרמב״ן כיון דילפינן א״ה הממרים א״כ בעינן דומיא דמרים בישראל דוקא ולפ״ז גם נפל שאין טעון קבורה לא הוי דומיא דמרים אבל לשיטת הסוברים דגם מת עכום אסור בהנאה וע״כ לא ס״ל דבעינן דומיא דמרים שפיר י״ל דגם נפל אסור בהנאה ולפ״ז לפי מה שהוכיח המ״ל מן הירושלמי דמת עכו״ם מותר בהנאה והיינו ע״כ משום דבעינן דומיא דמרים י״ל להלכה דגם נפל מותר בהנאה ולכן גם מטעם דשמא יהנה אפשר דאין חשש להשהותו בנדון השאלה.\\\vspace{0pt}

אמנם מטעם שלישי נלענ״ד שיש איסור גמור בדבר דבמתניתן שקלים (פ׳ א׳) ומ״ק (פ׳ א׳) תנן ומציינין את הקברות ושם (דף ה׳) יליף ר״ש בן פזי כן מקרא דיחזקאל דוראה עצם אדם ובנה אצלו ציון ועד דאתי יחזקאל גמרא גמירי לה ור׳ אבו׳ מפרש הטעם מוטמא טמא יקרא ואביי מפרש מלפני עור לא תתן מכשול ור״פ יליף מסולו סולו פנו דרך ועוד דרשות אחרות יש שם ע״ש ושם (ע״ב) אמרינן דמציינין על כל דבר המטמא באוהל ואפילו על כזית מן המת כשאינו מצומצם וכיון דנפל ג״כ מטמא באוהל אפילו לא נתקשרו אבריו בגידין כדאמרינן בחולין (דף פ״ט) וע״ש בתוספ׳ א״כ גם על נפל מציינין והנה רש״י והר״ן שם כתבו הטעם דמציינין מפני אוכלי תרומה משמע דרק משום תרומה אבל משום שמא יטמאו הכהנים לא אכן לפ״ז יקשה מנ״ל כן שהרי כל הטעמים שמפרש על חיוב ציון שייכים גם לענין טומאת כהנים וצ״ל דחדא מינייהו נקטי וה״ה דמשום טומאת כהנים נמי מציינין ועכ״פ כן מוכח דעת הרי״ף והרמב״ם והרא״ש והטור וש״ע שהרמב״ם בפי׳ המשניות מעשר שני (פ׳ ה׳) על ושל קברות בסיד כתב שופכין אותו על מקום הקבר כדי שידעו שאותו המקום טמא ויפרשו ממנו הכהנים וכן כתב בפי׳ המשניות בשקלים דמשום טומאת כהנים מציינין וכן מוכח גם דעת הרי״ף והרא״ש שהביאו הדין דמציינין קברות בחה״מ ואם רק משום טומאת תרומה מציינין הרי לא שייך אצלנו שכולנו טמאי מתים וידוע דהרי״ף והרא״ש לא מביאין רק מה שנוהג אצלנו אע״כ שס״ל דמשום טומאת כהנים ג״כ מציינין וכן כתוב בפי׳ בטוש״ע א״ח (סי׳ תקמ״ד) שמציינין קברות בחה״מ שלא יטמאו הכהנים ולפ״ז כל שכן שאסור לשהות נפל בבית שאין חשש גדול מזה וקרוב לודאי הוא שיבואו שם כהנים ברבות הימים יטמאו באוהל וכל הטעמים שמפרש במ״ק לענין ציון קברות שייכי גם בזה וא״ל דא״כ מאי נפקותא במה שפסק הגמי״י שאין צריך לקבור נפל כיון דמכ״מ צריך קבורה משום טומאת כהנים די״ל דהנפקותא הוא שאין צריך לקבור בו ביום משום לא תלין אי נמי שא״צ לקבור בקרקע אלא מותר להשליכו לבית הקברות דמצויין ועומד הוא שם וליכא חשש טומאת הכהנים שאם הי׳ צריך קבורה צריך דוקא קבורה בקרקע כמבואר בטוש״ע י״ד (סי׳ שס״ב) אבל להשהותו בביתו ודאי אסור לכ״ע. רק שיש לספק אם חיוב ציון קברות הוא מן התורה או מדרבנן דממה דקאמר במ״ק עד דאתי יחזקאל מאן קאמר ומשני גמרא גמירי לה משמע דלר״ש בן פזי עכ״פ מדאורייתא הוא וכן י״ל לכל הני טעמי דאמוראי אחריני דלכולהו גמרא גמירי לה ואסמכי׳ אקרא וכן משמע דהא על מה דקאמר רמז לציון קברות מן התורה מניין קאי כל תירוצים דאמוראים שם ואע״ג דבנדה (דף נ״ז) על מה דאמרינן דכותים נאמנים על ציון קברות דקאמר ואע״ג דמדרבנן הוא כיון דכתיב מזהיר זהירי בי׳ י״ל דקרי שם דרבנן כיון דמה שנמסר למשה מסיני להכותים הוא מדרבנן כיון דלא מודים בתורה שבע״פ הלכך אפשר דלכ״ע הוא מן התורה אבל עכ״פ מדרבנן יש חיוב ציון א״כ כל שכן שאסור להשהות בביתו נפל שמטמא באוהל וליתן מכשול לטומאת כהנים ותמהני על התוספ׳ י״ט והנודע ביהודה שלא מצאו טעם לחיוב קבורה לכזית מן המת ולנפל אלא משום חשש חלוש דאיסור הנאה ולא העירו על דין זה דציון דאפשר שיש חיוב מן התורה ועכ״פ מדרבנן כנלענ״ד הקטן יעקב.\\\vspace{0pt}

\end{multicols}\newpage

\newchap{סימן קכ}
\begin{multicols}{2}
ב״ה אלטאנא, יום ו׳ כ״א אדר שני שנת תרט״ז לפ״ק. להרב וכו׳ מ״ה משה ליב הכהן נ״י דיין דק״ק ניקאלסבורג יע״א.\\\vspace{0pt}

על תשובתי בענין להשהות נפל כתב מעכ״ת נ״י – וז״ל העיר מר נ״י דאף דאח קבורה בנפלים מכ״מ אסור להשהותו משום תקלה שמא יהנה כיון שמת אסור בהנאה רק דתמה על הא דפשיטא לי׳ גם להנוב״י מהד״ק סי׳ צ׳ דנפל אסור בהנאה והא ילפי׳ מגז״ש דשם שם ותמת שם מרים ותקבר שם א״כ י״ל דוקא מה דטעון קבורה אסור בהנאה ולא נפל שאין טעון קבורה עכת״ד הנה כעין שאלת מעכ״ת נ״י כבר הובא בס׳ בינה לעתים מהגאון מ״ה יונתן ז״ל ברמב״ם ספ״א מהל׳ י״ט והיא איש עני חסר לחם שהפילה אשתו נפל משונה בצורתו ונשאל אם מותר לסבב בכרכים להראות חזותו לרבים ויהי׳ הרואה אותו יתן פרוטה או ב׳ וחיתה נפשו בגללו והחליט שם לאסרה וז״ל אף גם דמת אסור בהנאה וזה ודאי אף נפל בכלל דאיסורו כטומאתו וזהו פשוט וא״כ הי׳ משתכר באיסורי הנאה עכ״ל הרי דגם להגאון הלז פשיטא לי׳ דנפל ג״כ אסור בהנאה ומה שהניח מעכ״ת נ״י לצ״ע טובא כבר העיר להקשות בשו״ת חתם סופר על או״ח סי׳ קמ״ד וכתב דילפינן שם שם מותמת שם וקרא דותקבר שם איצטריך לדרש אחרינא שלהי מ״ק דאין משהין מטה של נשים דסמוך למיתה קבורה אבל איה״נ לא תלי׳ בקבורה יעו״ש ובאמת גם מעכ״ת נ״י העיר בזה וכתב דגז״ש הוא משם דכתיב גבי ותקבר דהוא המיותר דשם ראשון לא מיותר ובמח״כ ז״א דהלא סוף מ״ק דף כ״ח ילפי׳ דרשה גם משם דכתיב גבי ותמת דבנשיקה מתה דילפי׳ שם שם ממשה גם בע״א (דף כ״ט ע״ב) היכי דילפי׳ דמת אסור בהנאה לא הביא הש״ס רק הך קרא ותמת שם מרים יעוש״ה הרי דילפי׳ משם דכתיב גבי ותמת.\\\vspace{0pt}

ומה שהעיר מעכ״ת נ״י בחקירה השלישית שיש איסור גמור להשהות הנפל בביתו מחמת ציון לענ״ד ז״א דהלא ברייתא ערוכה היא במ״ק דף ה׳ ע״ב אין מצייני׳ על הודאות אבל מציינין על הספיקות וכ׳ שם רש״י ד״ה על הודאות הואיל דברור לכ״ע דאית בי׳ טומאה ודאית לא מטלטלי באותו מקום טהרות וכ״כ הרמב״ם פ״ח מהל׳ טומאת מת הל׳ ט׳ יעו״ש וא״כ ה״נ בנ״ד אין לך ודאי גדול מזה דידעי כ״ע שאותו רופא יש לו נפל בביתו וכן יוודע הדבר הזה גם ברבות הימים איש מפי איש וידעו הכהנים להזהר ולא שייך ציון כלל וכלל ושפיר עבדו התי״ט והנוב״י שלא העלו טעם לחיוב קבורה לנפל רק מחמת איסור הנאה. גם מה שנסתפק מעכ״ת נ״י אם חיוב ציון קברות הוא מה״ת או רק מדרבנן והביא גמרא דנדה ד׳ נ״ז ודחה דמשם ליכא ראי׳ דמדרבנן הנה במחכ״ה אישתמיטתי׳ דכבר העיר בזה בתוס׳ ב״ב (דף קמ״ז ע״א) סוף ד״ה מנין שהוכיחו לפשיטות מהך גמ׳ דנדה דציון קברות הוא רק מדרבנן ובאמת דבריהם ברורים דליכא למימר כדעת מעכ״ת נ״י דהלל״מ הוא לכותים מדרבנן דכ״ז א״ש על המקשן דפריך סתמא אע״ג דמדרבנן י״ל דהוא קאי אליבא דהך מ״ד דגמרא גמירי לה עד דאתא יחזקאל ואסמכי׳ אקרא משא״כ על התרצן דהביא הך קרא דילפי׳ מיני׳ ציון ואפ״ה לא קאמר רק משום דכתיבי מזהר זהירי ולא קמתרץ דדאורייתא הוא מהך קרא אע״כ דמודה ג״כ לדעת המקשן שהוא רק מדרבנן רק כיון דכתיבי מזהר זהירי אף שהוא מדרבנן ע״כ דברי מעכ״ת נ״י.\\\vspace{0pt}

הנה מאפס הפנאי לא אשיב רק בקצרה על השגות הנ״ל: מעכ״ת נ״י תלה השגתו בגז״ש דשם שלא כתבתי רק לסניף אבל הלא עיקר ראיתי ממה שכתב הרמב״ן וסייעתי׳ דמת עכו״ם מותר בהנאה מדלא דומה למרים וא״כ ה״ה נפל ותימא על החתם סופר שלא העיר על זה וביותר יש לתמו׳ על הבינה לעתים שכתב דאיסורו כטומאתו והרי מת עכו״ם מטמא במגע ובמשא לכ״ע ואעפ״כ מותר בהנאה לדעת רוב הראשונים. מה שהשיג על פסקי דזה נקרא ודאי אין מן הצורך להשיב על זה שכל שופט בצדק ישפוט דאחד מני אלף יודע שיש שם נפל ואם יודע מי ידע אם מישראלית או מעכו״ם וא׳ מני אלף יודע שנפל כזה מטמא באוהל. צדק בזה שאגב ריהטא לא ראיתי דברי התוספ׳ בב״ב אבל לא ידעתי מה השיב על תשובתי דאיך יאמר התרצן כיון דדאורייתא הוא הרי הכותים אינם מודים בהל״מ וקרא דיחזקאל לא נקרא דאורייתא. כנלענ״ד הקטן יעקב.\\\vspace{0pt}

\end{multicols}\newpage

\newchap{סימן קכא}
\begin{multicols}{2}
ב״ה אלטאנא, יום ד י״ד אייר תרי״ח לפ״ק. לחתני הרה״ג וכו׳ מ״ה משלם זלמן הכהן נ״י אב״ד דק״ק שווערין יע״א. אשר שאלת במי שהלך בשגעון מביתו ולא חזר וקרוביו היו מסופקים אם נטבע במים או אם הלך בשגעונו למרחקים ואחר ו׳ חדשים נמצא מהדייגים במים והכירוהו בבגדיו ובאדרעסקארטען אשר נמצאו בשקו ונתן לקבורה אם קרוביו ינהגו ז׳ ול׳.\\\vspace{0pt}

תשובה יפה דנת שמה שנפסק בי״ד (סי׳ שע״ה) במי שנטבע במים ונתייאשו מלבקש ושוב נמצא שא״צ שוב להתאבל שזה דוקא בשכבר נהגו אבלות בשעה שנתייאשו אבל בנדון זה שעדיין לא נהגו אבלות לא שייך זה. ושגם לא שייך בזה שהוא שמועה רחוקה שכל זמן שלא נקבר לא חל אבלות עד שיסתם הגולל וא״כ דינו כשמועה קרובה שחייבין הקרובים להתאבל. אכן לא נתבאר בהשאלה אם הי׳ איש נשוי לאשה שאז פשיטא שאין להתאבל עליו שכפי מה שנראה מלשון השאלה לא הי׳ היכר בגופו רק בבגדיו ובמה שבשקו והרי לפי המבואר בגמרא ובאהע״ז סי׳ י״ז סעיף כ״ד כל שלא הכירוהו רק ע״י כליו אין משיאין את אשתו שחיישינן לשאלה ואף דהר״מ פאדואה ומהר״י מטראני פסקו דאם נכר בבגדיו שהו׳ לבוש בהם קודם הטביעה לא חיישינן לשאלה הרי הכה״ג חולק עליהם ופוסקים אחרונים הסכימו עמו שלא להתיר עגונה ע״י היכר בגדיו בלבד וגם האדרעסקארטען פשיטא שאינם סימן מובהק שהוא דבר הנתן מיד ליד ושמא נתנם לאחר וגרע זה מארנקי וטבעת וחותם וכיון שאין להתיר אשתו על סימנים האלה אסור גם להתאבל עליו כמבואר באהע״ז שם סעיף ה׳ בהגהה וכ״כ הש״ך סי׳ שע״ה ס״ק ז׳ אמנם אם הנטבע הי׳ פנוי בלא אשה זה תלוי בפלוגתא שלדעת המהר״ם בן חביב בעזרת נשים אפילו בפנוי אין להתאבל כל שלא נתיר אשה דגזרינן גזירה לגזירה אבל בשו״ת שבות יעקב ח״ד סי׳ ק״ג פסק דבפנוי צריכין להתאבל דלא גזרינן גזירה לגזירה ובשו״ת חתם סופר חלק י״ד סי׳ דש״ם הסכים עם הש״י ולכן קרוביו חייבין להתאבל ושלו׳ לכל ישראל כנלענ״ד הקטן יעקב.\\\vspace{0pt}

\end{multicols}\newpage

\newchap{סימן קכב}
\begin{multicols}{2}
ב״ה אלטאנא, יום ד׳ ט״ו כסליו תרי״ד לפ״ק. להרה״ג וכו׳ מ״ה בער אפפענהיים נ״י הגאב״ד דק״ק אייבענשיץ יע״א.\\\vspace{0pt}

כתב מעכ״ת נ״י – נשאלתי קהל אחד הי׳ להם שתי בתי כנסיות א׳ גדולה וא׳ קטנה ועוד שאר מנינים וכשנעשתה הבהכ״נ הגדולה רעוע וחשבה לפול הרסו אותה ובנו ביהכ״נ גדולה בנין נאה המחזקת כל בני הקהלה והסכימו שלא להתפלל בשום מנין רק בביהכ״נ החדשה הזאת ועוד התקינו כי קדיש יתום יאמרו כל האבלים ואפי׳ יאהרצייט ושלשים כולם יחד אמנם קמו מעוררים ואבני הקלע בידם מדברי הרמ״א ביו״ד (סי׳ שע״ו) ודברי המג״א בא״ח (סי׳ קל״ב) והקהל אומרים אם ננהיג במנהג הקודם יתרבו המחלוקת וקטטות בכל יום ויום יען שאין כאן רק ביהכ״נ אחת ורוצים לידע אי שפיר עבדי. והשבתי להם תקנה גדולה וישרה ראיתי כאן דהנה ענין קדיש יתום לא מצאנו לא בשתי תלמודיים גם לא ברמב״ם ובטור רק הב״י ביו״ד (סי׳ שע״ו) הביא בשם הכלבו שהעתיק דברי הזה״ק פ׳ אחרי פ״א פגע ר״פ באחד שהי׳ מקושש עצים וכו׳ אבל בש״ע לא הביא כלל מנהג הלז דיתומים יאמרו קדיש רק הרמ״א הביאו ועשה חיל וקים בזה ובאמת גוף המעשה שהביא הזה״ק אין לו פירוש ואפשר שבא כן לאחד בחלום והריב״ש בתשובה סי׳ קט״ו כ׳ ג״כ שלא נמצא מעשה זה בתלמוד ואפשר במדרש ובס׳ הנדפס בחדש כרם שלמה על יו״ד סי׳ שע״ו שהי׳ מלקט שו״ת ודברי אחרונים כ׳ שלא נמצא בשום מדרש אמנם הריב״ש כ׳ שם כי כל המנהג הזה הוא רק על קדיש דרבנן אחר הלימוד ולא על שארי קדישים אמנם אף שהרמ״א הביא סתם ואחריו המג״א הביא כמה וכמה דינים מי שיש לו קדימה בענין הקדיש לא אמרו רק בעיר שיש להם הרבה בתי כנסיות ובהמ״ד ואז נוכל לקיים פסקיהם וחלוקיהם בענין הקדימה שלשים ויא״צ אבל בעיר שאין להם רק ביהכ״נ אחת ובכל יום ויום שם אבלים שלשים ויא״צ הרבה כדי למעט במחלוקות טוב שיאמרו כולם בב״א כדי לתווך השלום ומה שהמערערים טוענים שלא לבטל המנהג כי כל מנהג יש לו שער בשמים ויסודתם בהררי קודש מדברי המג״א בא״ח סי׳ ס״א אמר אני להשקיט הריב נוכל לבטל וכבר אמרו חז״ל שם שנכתב בקדושה נמחק כדי לעשות שלום בין איש לאשתו מכ״ש בנדון זה ועיי׳ תשוב׳ הרדב״ז ח״ג סי׳ תרמ״ה הביא בשם הריטב״א וז״ל אבל המנהג להקל לעולם אין חוששין לו ואפי׳ הי׳ ע״פ גדולים שבעולם כל שנראה בו צד איסור לחכם בעל הוראה אשר יהי׳ בימים ההם שאין לנו אלא השופט שיהי׳ בימינו עכ״ל הכ״נ יש בו קולות שהרבה פעמים בא לידי מחלוקות במקום קדוש ואנוכי היודע ועד כי ראיתי פ״א שאחד נתן לחבירו מכת לחי ע״י הקטטה עבור קדיש ועוד גוף הדבר אין לו טעם מה חילוק יש אם אחד אומר הקדיש או הרבה יחד אדרבה כי עיקר הקדיש כ׳ בס׳ יש נוחלין הובא בס׳ בית לחם יהודה ביו״ד סי׳ שע״ו סק״ה הקדיש אינו תפלה שיתפלל הבן על האב לפני ה׳ שיעלה משאול מטה אלא זכות ומצו׳ הוא למת כשבנו מקדש ברבים והקהל יענו אחריו אמן והה״ד שאר מצות וזכיות שעושה הבן אחר מיתת אביו הוא כפרה לנפש האב עכ״ל וא״כ אם רבים אומרים ביחד קדיש בודאי יתקדש שמו ית׳ יותר וכמ״ש רש״י פ׳ בחוקותי א״ד מרובים העוסקים בתורה למעטים ואם יחיד אומר בפ״ע כמה פעמים אין עונים עשרה אחריו איש״ר אמנם מחמת דתרי קלי לא משתמעי יש לתקן שיאמר הש״ץ הקדיש והאבלים יאמרו בלחש מלה במלה אחריו ואז טוב יותר משיאמרו בב״א ולא ישמעו דבריהם והקהל אינם עונים איש״ר וראיתי אחד קדוש מדבר בספרו חתם סופר סי׳ שמ״ה בתשובה אחת כתוב אלי וזה לשונו אך נ״ל דהנה מנהג הספרדיים כל הקדשים אומרים האבלים בפ״א ועל כולם עונים איש״ר והגאון מוה׳ יעבץ בסידור שלו כ׳ אינני מטפל בדיני קדיש כי נכון מנהג הספרדים כל הקדישים אומרים כולם בפ״א ועז״כ בעל חתם סופר וכן אנו נוהגים בבית מדרשינו בקדיש דרבנן שיאמרו כל הבחורים האבלים יחד גם בביהכ״נ כשיש שום פלפול בין האבלים אומרים שניהם מה בכך על כן נלענ״ד וכו׳ עכ״ל ואם הגאון מהרמ״ס זצלל״ה אשר מרגלא בפומי׳ לומר אני אומר החדש אסור בכל מקום והי׳ שונא החדשות ואמר כי טוב לומר כולם ביחד בודאי טוב בעיר שאין להם רק ביהכ״נ אחת להנהיג מנהג ספרדיים ובפרט בביהכ״נ חדשה יכולים לתקן מנהגים חדשים עכ״ד מר נ״י.\\\vspace{0pt}

על זה אשיב – תמהתי דרך כלל איך קרא מעכ״ת נ״י תקנה גדולה וישרה לשנות מנהג ישראל אשר נהגו בכל מדינות אשכנז ופולין זה יותר משלש מאות שנה בקדיש יתום שיאמר כל אחד לבדו ולילך בעקבות המתחדשים בזמננו ששינו עניני התפלה והנהיגו גם המנהג הזה שיאמרו האבלים קדיש יחד. ועתה אתוכח עמו דרך פרט. מר נ״י כתב שענין קדיש לא נמצא בשום תלמוד ומדרש רק במעשה שהביא הב״י בסי׳ שע״ו בשם הכלבו שהעתיק דברי הזוהר הקדוש פ׳ אחרי ע״פ המעשה שהובא שם וגוף המעשה שהביא הזוהר אין לו פירוש ואפשר שבא כן לאחד בחלום עכ״ד ואשאלה ממנו למה אין לו פירוש אפילו הי׳ המעשה בהקיץ ולא בחלום וכי יפלא בדברי רז״ל שרוחות המתים נראו לחיים בפרט לאנשים גדולים ודברו עמהם מה יענה מעכ״ת נ״י לאשר אמר ריב״ז לתלמידיו פנו כסא לחזקי׳ מלך יהודה שבא כדאמרינן בברכות פ׳ תפלת השחר ולאשר אמרינן בשבת פ׳ השואל הנהו קפלאי וכו׳ מאן ניהו מר א״ל אחאי בר יאשי׳ ולאשר אמרינן בברכות פ׳ מי שמתו במעשה החסיד שלן בביה״ק ושמע המתים מספרים זע״ז ומה שאמרינן שם שהלך שמואל אצל אביו ושאלו היכן מעות היתומים והשיב לו ומה דאמרינן ביומא פ׳ יוה״כ באבו׳ דכידור שאמר לבנו שהי׳ כיס מונח בקברי׳ ומה דאמרינן בכתובות פ׳ הנושא ברבינו הקדוש שהי׳ בא מע״ש לע״ש לביתו וקדש ומה דאמרינן בב״מ פ׳ הפועלים בר״א בר׳ שמעון שהי׳ מונח מת י״ח שנה בעלייתו וספר עם אשתו והורה הוראות מלבד הרבה מעשות כאלה שנמצאים במדרשים ובזוהר הקדוש והכי ישפוט מעכ״ת נ״י שגם כל אלה דברי חלומות היו והנה חפשתי בזוהר הקדוש אחר המעשה שכתב הב״י שנמצא בפ׳ אחרי ולא מצאתיו שם וכבר חשבתי שנתחלף להב״י מעשה אחר שהובא שם בפ׳ אחרי בר׳ חזקי׳ ור׳ ייסא שיתנו סמיך לבי קברא אתרגיש חד קברא קמייהו וצוח ווי ווי דהא עלמא בצערא שכיחא וכו׳ ע״ש אכן אין זה המעשה שהובא בכלבו ועכ״פ גם ממעשה זה נראה שהיו המתים מדברים עם החכמים בהקיץ. שוב מצאתי המעשה במדרש הנעלם הובא בזוהר חדש וכפי הנראה משם הי׳ המראה בחלום ובאמת אין חילוק בזה אם הי׳ בהקיץ או בחלום אחרי שהמדרש סיפר לנו ודאי שהמעשה אמת ולא דברי חלומות אשר ישאם הרוח.\\\vspace{0pt}

עוד כתב מר נ״י שעל מה שכ׳ הריב״ש שאפשר שנמצא מעשה זה במדרש כתוב בספר כרם שלמה שלא נמצא בשום מדרש והי׳ לי תימא שהמחבר ז״ל ת״ח גדול ומופלג יכחיש דברי הרמ״א שכתב שנמצא במדרשות ועיינתי בספר הזה וראיתי שלא כתב הדברים שהועתקו בשמו שרק על מה שכתוב במראה מקום הבחי״י בשם מסכת כלה על זה כתב לא נמצא שום רמז ורמיזה במסכת כלה עכ״ל וזה אמת שהבחי״י סוף פ׳ שופטים כתב כן שנמצא במסכת כלה ובשלנו לא נמצא אבל הלא ידוע שהמסכתות החצונות לא נמצאו שלמים אתנו אבל מעולם לא עלה על דעת איש להכחיש הרמ״א ולומר שלא נמצא בשום מדרש כי באמת נמצא מעשה זה במדרש רבי תנחומא פ׳ נח (אין זה המדרש תנחומא הנדפס באמשטרדם) ונאמר שם מעשה בר׳ עקיבא שהי׳ מהלך בבית הקברות ופגע בפחם אחד שהי׳ טעון עצים על כתפיו וכו׳ עיי״ש באריכות וסיים שם וקרא באלף בית עד שהוליכו לביתו ולמדו ברכת המזון וק״ש ותפלה והעמידו והתפלל ואמר ברכו את ד׳ המבורך לעולם ועד באותה שעה התירוהו מן הפורענות ובא אותו האיש בחלום ואמר לר״ע תנוח דעתך בג״ע שהצלתני מדינה של גהינם עכ״ל הרי מה שהי׳ בחלום שם הוזכר במדרש אבל בעיקר המעשה נאמר מעשה בר״ע שהי׳ מהלך בביה״ק וכו׳ ואמר לו האיש שמעתי שהיו אומרים לי אלו הי׳ לך בן שהי׳ עומד בצבור והי׳ אומר בצבור ברכו את ד׳ המבורך היו מתירין אותו מן הפורענות וכו׳ מזה נראה שהי׳ ספור מעשה מה שנאמר בהקיץ ולא בחלום והמעשה הזה הועתק ג״כ במנורת המאור בלשון ארמי בשם מס׳ כלה פ׳ ר״י ע״ש נ״א ח״ב והנה במדרש זה לא נאמר מקדיש רק מברכו את ד׳ המבורך ואין זה תימא כי בימי ר״ע עדיין לא נתקן קדיש שנתקן בבבל אבל ברכו כבר נאמר בצבור שהוזכר במשנה ברכות (פ׳ ז׳) אכן אחר שגדולת מעלת ברכו היא שעל ידו ועל פי דבורו מקדשים רבים שם הקב״ה ועי״ז מזכה גם לאביו א״כ מעלת הקדיש עוד יותר באשר ששם מזכיר השבחים בעצמם שיאמרו אחריו ולכן יפה כתב הרמ״א שנמצא במדרש גם בזהר חדש מדרש רות נזכר מעשה כזה באריכות והי׳ הבן ההוא רבי נחום הפקולי ומסיים שם א״ר חייא בר אבא כהאי גוונא אירץ לר׳ עקיבא ע״ש הרי דנזכר ג״כ בזה מעשה דר׳ עקיבא ועכ״פ יוצא מזה דמעלת הקדיש גדול ויסודתו בקדש קדשים בר׳ עקיבא דכולהו סתומי אליבי׳.\\\vspace{0pt}

עוד כתב מר נ״י: אמנם הריב״ש כתב שם כי כל המנהג הזה הוא רק על קדיש דרבנן אחר הלמוד ולא על שארי קדישים. והא בורכא, שלא בלבד שלא נזכר מזה בריב״ש אלא אדרבא אפכא נראה שם שאף שהשואל לא הזכיר רק על מה שנהגו בספרד לומר קדיש דרבנן הוא השיב בשם רא״ה ז״ל שכתב בשם הר״מ ז״ל טעם תקנת הקדיש מפני המעש׳ וסיים ועל זה פשט המנהג לומר בנו של מת קדיש בתרא כל י״ב חודש עכ״ל וידוע דקדיש בתרא היא מה שנאמר בפוסקים על הקדיש שאומרים אחר התפלה ולשון זה עצמו הוזכר ברמ״א סי׳ שע״ו על קדיש יתומים הרי שמנהג אמירת קדיש יתום כבר הי׳ מנהג שפשט בישראל בימי הראשונים זה יותר משש מאות שנה ויסודתו בקדש.\\\vspace{0pt}

והנה מר נ״י באשר דמה שהראה שאין למנהג קדיש שרש ועיקר לא בתלמוד ולא במדרש ולא בראשונים אשר על כן יהי׳ ביד המורה כחומר ביד היוצר לאשר יחפוץ יטנו הודיע דעתו שמנהג טוב יהי׳ לתקן שכל היתומים יאמרו יחד ונגד המערערים שאין לבטל מנהג אמר שאין בזה שינוי מנהג כיון שביה״כ נבנה מחדש ועוד שאין מנהג עומד להקל ובזה יש תקנה להשבית המחלוקת שכבר אירע שהכו זא״ז על אמירת קדיש ולכן טוב ויפה ומקובל לעשות שלום ע״י שינוי התקנה עכ״ד מר נ״י תמהתי לשמוע דברים כאלה וכי ענין המנהג תלוי בבית אשר מתפללים בו הלא את אשר נהגו יושבי הקהלה עד עתה הוא הנקרא מנהגם וכי יקרא מנהג להקל שניתן ביד כל לשנות כרצונו מה שנהגו בכל תפוצות אשכנז ופולין כמה מאות שנה מפני שיש קצת פושעים שמרימים יד להכות בביה״כ הלא על כזה נאמר כי ישרים דרכי ד׳ צדיקים ילכו בם ופושעים יכשלו בם. ומה שתלה מר נ״י באילן גדול במה שכתב הגאון חתם סופר זצ״ל ואמר הלא הוא הי׳ שונא התחדשות ואעפ״כ התיר לומר קדישים יחד בהציעו דברי היעב״ץ שאמר שישר בעיניו מנהג הספרדים גם בזה לא דבר נכונה וכי הגאון חת״ס התיר לומר קדישים כל האבלים יחד הלא טרח ויגע למצוא פשר בפסקו שיאמרו האבלים כל א׳ ב׳ קדישים ורק הקדיש השלישי יאמר השכיר עם עוד אבל אחר ורק לשינוי מעט הזה מצא לו סמך מדברי היעב״ץ וממה שנהגו בישיבה לומר קדיש דרבנן יחד ואלו רצה להתיר אמירת קדישים יחד לא הצריך לפסקו וגם היעב״ץ ז״ל אשר פה הי׳ מקום כבודו והי׳ לו ביה״כ מיוחד לעצמו לא שמענו ששינה מנהג בזה להניח האבלים לומר קדיש יחד אף שאמר שישר בעיניו מנהג הספרדים בזה.\\\vspace{0pt}

ומה שנהגו הספרדים אין בידינו לידע על מה נהגו כן אבל עכ״פ יש חילוק רב בין הספרדים ובין אשכנזים כי הספרדים כל תפלתם היא במקהלות בקול שו׳ בלי איחור ומיהור ואחרי שרגילים בזה גם כשיאמרו קדישים יחד נשמע קולם יחד כדאמרינן במגילה (דף כ״א) דאף דתרי קלי לא משתמעי מכ״מ בהלל ובמגילה אפילו עשרה קורין כיון דחביבה יהבי דעתייהו ושמעי הרי דאפשר לשמוע גם הרבה קולות אי יהבי דעתייהו ובזה רגילים הספרדים מה שאין כן בנו אנשי אשכנז שאין מתפללין וקורין בקול א׳ וכשהרבה מתפללים יחד אי אפשר לשמוע ולכוון כאשר ידעתי נאמנה במה שגם פה נוהגין לומר קדיש דרבנן האבלים יחד והוא כחוכא ואטלולא דכל דאלים גבר לישא קולו קול ענות גבורה וחבירו צועק כנגדו זה פותח וזה חותם ואי אפשר ממש לענות אמן יש״ר ולולא יראתי לבטל מנהג כבר הייתי מונעם שלא לומר קדיש דרבנן יחד כל שכן שלא להנהיג כן לכתחלה ומה גם בשאר הקדישים אשר לא נשמע ולא נראה בכל גלילות אשכנז ופולין אפילו קהלה אחת שנהגו כן חוץ ממה שהנהיגו מגידי חדשים אשר מנהגי ישראל לא יחשובו. אמנם מדברי מר נ״י נראה כי בעצמו ראה שלא יאות לומר הקדיש יחד בקול דתרי קלי לא משתמעי ולכן המציא פשר דבר שהש״ץ יאמר הקדיש בקול והאבלים יאמרו בלחש עמו ואמת ויציב כי כן הנהיגו המגידי חדשות אשר להם כל עניני הקדיש דברי חלומות ואשר לא ישגיחו אם יושג תכליתו באשר מגמתם רק לסתום פה השואלים חקם לומר קדיש לכבוד אבותם אכן למעכ״ת נ״י אשר לא יחשוב כן אשאל הלא תכלית וגדולת ענין הקדיש הוא (כאשר שמע האיש שרק ע״י ברכו את ד׳ המבורך אם יאמר בנו ברבים יזכה) שכמו שעל פי דבורו יזכו רבים לקדש שם ה׳ כן יזכה אביו על ידו וכי יושג תכלית זה אם האבלים יאמרו קדיש וקולם לא ישמע הלא שתיקתם יפה מדבורם כי נראה כשקר בפיהם ותרמית בלבם לומר לרבים שבחו לד׳ אתנו ואמרו אמן יש״ר ונשמרים לבלתי הרים קול לבל ישמעו רבים דבריהם ורק על דברת הש״ץ ישבחו הצבור לכן אין טוב רק לשמור דרך ארחות חיים אשר גבלו ראשונים ולזכות כל אחד לבדו בקדיש אשר יאות לו ע״פ התורה אשר הורו הפוסקים למי הקדימה ואחד המרבה ואחד הממעיט רק שיכוון לבו באמירת קדיש וברכו כנלענ״ד, הקטן יעקב. סליק חלק יורה דעה. בעזרת חונן לאדם דעה.\\\vspace{0pt}

\end{multicols}\newpage

\newchap{סימן קכג}
\begin{multicols}{2}
ב״ה אלטאנא, יום ד׳ ט׳ תמוז תר״י לפ״ק. להרב היקר וכו׳ מ״ה עקיבא לעהרען נ״י בק״ק אמשטרדם יע״א.\\\vspace{0pt}

כתב לי מעכ״ת נ״י וז״ל – הענין מצות לשבת יצרה עלה בדעתי לומר (שלא כדברי התוס׳ וכמה מגדולי הראשונים ז״ל ואם שגיתי ד׳ ימחול לי) שמצוה זו אינה חיוב מוטל על העבד עצמו ולא על האשה הן שפחה הן בת חורין כי אם דוקא על האדון או על הב״ד מוטל להשתדל שיהי׳ העבד נשוי כדי שיקיימו הם בזה מצות שבת וכל זה דוקא בשביל העבד אבל בשביל האשה לאו חיובא רמיא מצו׳ זו כלל וכלל לא על עצמה ולא על אחר ומה שמוטל על העבד ולא על האשה זהו פשוט שכמו שמצות פו״ר לא נאמרה אלא על הזכרים כן גם דברי הקבלה באו רק על הזכרים לבד ולא על הנקבות ובאה מצו׳ זו עלינו לקיים בכל הכפויים תחתנו כמו העבדים ולענ״ד יתורצו בזה כמה קושיות – א׳ – וכי מקרא זו של לא תוהו בראה לעבדים נאמר הלא לישראל דבר הנביא בשם ד׳ – ב׳ – אם כדברי התוספ׳ והראשונים ז״ל לא לשתמיט תנא וליתני בפירוש האשה חייבת במצות שבת – ג׳ – אדרבא מצינו פלוגתא במתניתן אם האשה חייבת בפו״ר ואם לא ולמה לא אמרו חכמים דר׳ יוחנן בן ברוקא שהגם שהאשה אינה חייבת במצות פו״ר עכ״פ מצות שבת גם על האשה נאמרה – ד׳ – במעשה דרב הונא דאיתא בגטין (דף מ״ג) רצה בש״ס להוכיח כדברי ריב״ב וא״כ מוכח דלדברי חכמים בודאי אינה מחוייבת לא על מצות פו״ר ולא על שבת – ה׳ – אם דברי הקבלה דלא תוהו בראה מוטלים על העבד א״כ קשיא על מתניתן דקאמר יבטל והלא לא נברא העולם אלא לפו״ר הכי הל״ל יבטל והלא העבד חייב במצות שבת שנאמר לא תהו בראה וכו׳ – ו׳ – מדוע לא אמרה המשנה שצד עברי שבו חייב במצות פו״ר ואם נתרץ על זה כמו שתרצו הראשונים שהוא אונס א״כ לענין מצות שבת ג״כ אונס הוא ופטריה רחמנא – ז׳ – הקושיא שכבר הרגישו התוספ׳ בעצמם בגטין (מ״א ע״ב) סוף ד״ה לא תהו אחרי הוכיחו ההוכחה שגם בשפחה שייך שבת דא״כ למה כפינן בחצי׳ שפחה וחב״ח דוקא בנהגו בה מנהג הפקר ואף שתרצו קושיא זו הלא נראה לעין ששנויא דחיקא הוא – ח׳ – שבת (דף ג׳ ע״ב) הקשה הרשב״א בדברי רב ביבי ממי שחציו עבד וחציו ב״ח שכופין את רבו ומה שמתרץ בודאי נודע למעכ״ת נ״י (גם בתוספ׳ הקשו קושיא זו ותרצו באופן אחר) אכן על תרוצו ק׳ מגמרא דגטין הנ״ל שרצה הגמרא להוכיח ממעשה דרב הונא כדברי ר״י בן ברוקא ותי׳ רנב״י לא מנהג הפקר נהגו בה ולדברי הרשב״א לא הי׳ קשה ולא מידי – ט׳ – וכי מצאנו שיש רשות לעבד לקחת לו שפחה מדעת עצמו או לכבשה אדרבא מוכנע הוא לרבו ולהיות מרוצה עם מה שייטב בעיניו לתת לו וא״כ איך תהי׳ מצות שבת חלה על העבד אם אינו ברשות עצמו וידיו אסורות מלקיימה כרצונו וחפצו – י׳ – אם יאמנו דברי מיושב בזה ג״כ מש״כ באהע״ז (סי׳ ט׳ ס׳ ב׳) דהאיש מותר ליקח אחרת אם מתו שתי נשיו כי שומר מצו׳ לא ידע דבר רע לא כן האשה שאינה עושה מצו׳ בזה אבל אם נאמר שעכ״פ האשה חייבת במצות שבת יהי׳ קשה על דין זה דגם האשה בכלל שומר מצו׳ דשבת. והגם שמדברי הירושלמי דמ״ק הובאו בתוספ׳ חגיגה (ב׳ ע״ב) משמע שהמצו׳ היא על העבד אפשר שזה לאו דוקא שגם אם המצו׳ היא על האדון מכ״מ העבדים הם בכלל המחוייבים במצות פו״ר ואבקש ממעכ״ת נ״י לפרש לי דעתו הרחבה בענין מצו׳ זו ואשמחה ואעלצה בדברי מעכ״ת נ״י ותשאות חן חן אביע לו עכ״ד מעכ״ת נ״י.\\\vspace{0pt}

ועל זה אשיב: הנה מעכ״ת נ״י רצה לחדש בזה ב׳ דברים נגד התוספ׳ והראשונים (ולא ידעתי למה ערבם) רצה לומר שמצות לשבת יצרה אינה על העבד כי אם על האדון או על הב״ד ועוד רצה לומר שמצות לשבת יצרה אינה באשה כמו מצות פו״ר שאינה נוהג בה והביא ראיות לזה אכן לענ״ד במבוקשו הראשון אין הדין עמו כי לא בלבד שמדברי הירושלמי נראה שהעבד בעצמו מצו׳ כמו שהעיר בעצמו אלא מן המשנה עצמה מוכח כן דקאמרי ב״ש לב״ה תקנתם את רבו ואת עצמו לא תקנתם וכו׳ והודו ב״ה לב״ש ואי ס״ד דרק האדון מצו׳ על פו״ר ושבת דעבד ולא הוא היאך שייך ואת עצמו לא תקנתם הרי גם מצות שבת היא תקנת רבו או תקנת ב״ד והכי הל״ל תקנתם עבודת רבו ולא תקנתם מצות רבו אע״כ מוכח דהעבד עצמו מצו׳ ומה שהקשה וכי דבר הנביא לעבדים והלא לישראל דבר אני תמה והלא בעבד של ישראל שמל איירינן שחייב בכל המצות כישראל חוץ ממ״ע שהזמן גרמא וא״כ למה לא יהי׳ גם בכלל מצות לשבת יצרה.\\\vspace{0pt}

והנה אם העבד מצו׳ על פו״ר בזה יש ב׳ שיטות בתוס׳ דבחגיגה (דף ב׳) כתבו דמצו׳ ובגטין (דף מ״ב) כתבו בשם הריב״ם דלא מצו׳ ובספרי ערוך לנר ביבמות (דף ס״ב) כתבתי קצת ראי׳ שעבד אינו מצו׳ דעפ״ז מיושב קושית התוס׳ שם גם תמהתי שם דאיך אפשר דמצו׳ עבד על פו״ר כיון דאינו יכול לקיים שהרי אין זרעו מיוחס אחריו משא״כ בשבת דלא בעינן מיוחס אחריו כמש״כ התוספ׳ שם אבל מצות שבת ודאי איכא גם על העבד כמו שאר מצות שבתורה ולכן ל״ק ג״כ מה שהקשה מעכ״ת נ״י וכי העבד ברשותו הוא והלא רשות האדון עליו דז״א דודאי אין רשות האדון עליו לבטלו ממצוה. ומה שרצה להביא ראי׳ מהא דכופין את רבו וכי אומרים לאדם חטא הלא זה כבר תרצו התוספ׳. ומה שהקשה על הרשב״א מגמרא דגטין (דף מ״ג) דלתי׳ הרשב״א בשבת לחלק בין חצי עבד ובין עבד גמור לא פריך שם מידי באמת קושיא עצומה היא וראיתי בטורי אבן בחגיגה (דף ג׳) שכתב ג״כ כסברת הרשב״א (שלא ראה הרשב״א) ולא הרגיש בקושיא זו וכבר הקשה לי קושיא זו חכם א׳ והשבתי לו שי״ל שסברת הרשב״א היא כן דבחצי עבד וחצי ב״ח יש לכל צד זכות וחובה דבצד עבדות מוטל על האדון עשה דבהם תעבודו ובצד חירות מוטל על העבד ע׳ דפרי׳ ורבי׳ ולכן כמו שנאמר דאין אומרים לאדון לשחררו משום זכות העבד כמו כן נאמר ג״כ שאין לעבד לעבור בצד חירות שלו לבטל ע׳ דפו״ר כדי שיזכה האדון בע׳ דבהם תעבודו וכיון דהכא והכא יש משום וכי אומרים לאדם חטא לכן מכריע זכות העבד משום דלשבת יצרה היא מצו׳ גדולה יותר שעל ידה מתקים העולם וא״כ זה שייך דוקא לענין עבד אבל בשפחה דליכא פו״ר וליכא משום שבת פריך שפיר לאביי׳ כיון דכפו לשחררה ולבטל בהם תעבודו דצד עבדות שלה ע״כ משום איסורא הי׳.\\\vspace{0pt}

אכן במה שכתב מעכ״ת נ״י שמצות שבת אינה באשה כמו מצות פו״ר לזה הדעת נוטה הגם שהראי׳ שהביא לזה ממה דפריך גטין (דף מ״ג) כמאן כר׳ יוחנן ב״ב דמוכח דגם שבת אין באשה לרבנן לענ״ד אינה ראי׳ דכבר תרצתי על זה בספרי הנ״ל ע״פ דברי התוספ׳ שם (דף מ״א) דלכך לא כפינן בחצי׳ שפחה לשחרר כיון דלא מפקדה אפו״ר שמא גם אחר שתשתחרר לא תקיים ולפ״ז פריך שפיר ג״כ דכיון דכפו לשחרר ע״כ כריב״ב ס״ל דלאחר שחרור מצו׳ אפו״ר דאי רק משום שבת הרי אין כופין שמא אחר שחרור לא תקיים אבל מכ״מ מצד הסברא הדבר צ״ע כיון דלא מצינו מצות שבת באשה בשום מקום והנה הוכחת התוספ׳ שיש שבת באשה היא רק ע״פ מה שכתבו דפו״ר אין בעבד והיינו ע״כ משום דליכא באשה ואעפ״כ שבת איכא א״כ ע״כ מוכח ג״כ דשבת יש באשה אכן לפי דעת התוספ׳ בחגיגה דמצות פו״ר היא בעבד אף שאינה באשה שפיר י״ל דגם מצות שבת ליכא באשה אבל מכ״מ צריך טעם למה לא דאף דפו״ר מיעט קרא משום דאין דרכה של אשה לכבוש מכ״מ הרי שבת גם בנקבה שייך כדמוכח ביבמות (דף ס״ב) דקאמר והא עבד לה שבת ע״ש וכיון דשייך שבת בבת שהוליד למה לא יהי׳ ג״כ חיוב דשבת אנקבה.\\\vspace{0pt}

אמנם ראיתי בפני יהושע בגטין שם שג״כ דעתו לומר דמצות שבת אינה באשה נגד התוס׳ מדלא הוזכר כן בשום מקום והביא ג״כ הראי׳ ממה דפריך כמאן כריב״ב דמשמע דלרבנן ליכא גם משום שבת יע״ש אכן לא ידעתי למה לא הזכיר דעכ״פ משמע ביבמות דשייך שבת בבת משום דע״י ג״כ יהי׳ שבת העולם ע״ד שאמרו א״א לעולם בלא זכרים ובלא נקבות וכיון דכוונת בריאת מין האדם לא ישלם רק ע״י שניהם א״כ ודאי הסברא נוטה שגם בה יש מצות שבת העולם שוב ראיתי שגם הר״ן ר״פ האיש מקדש כתב כן דאף דאשה אינה מצו׳ על פו״ר מכ״מ מצו׳ עושה כשתנשא ומביא ראי׳ מהא דאמרינן שם מצו׳ בה יותר מבשלוחה משמע דבה ג״כ מצוה וביאר יותר בשו״ת (סי׳ ל״ב) שהביא שם כן ג״כ בשם גדול א׳ יע״ש ובברכי יוסף אהע״ז (סי׳ ב׳) ולענין זה לענ״ד גם התוס׳ כתבו דאיכא מצות שבת בה כיון דע״י נגמר השבת אבל חיוב שבת דהיינו מצות פו״ר בזה אמרינן דליכא באשה רק באיש.\\\vspace{0pt}

אלא שעדיין צ״ע שדברי הרמב״ם והפוסקים אחריו סותרים זא״ז דבה׳ איסורי ביאה (פ׳ כ״א) פסק ורשות לאשה שלא תנשא לעולם או תנשא לסריס ע״ש והוא ע״פ התוספתא כמש״כ ה״ה ובה׳ אישות (פ׳ ג׳) פסק דמצו׳ בה יותר מבשלוחה וכיון דמצו׳ עכ״פ איכא איך פסק דיש רשות לאשה אכן לענ״ד הדבר כן ע״פ מה דאמרינן קידושין (דף ב׳) דלכך כתיב כי יקח איש אשה דדרכו של איש לחזור אחר אשה ואין דרכה של אשה לחזור אחר איש ע״ש ולכן ודאי לריב״ב אם היא מצו׳ על פו״ר אזי זה חוב מוטל עלי׳ וצריכה להשתדל לצאת י״ח כשאר מצות התורה אבל לרבנן דאין כאן חיוב רק כיון שנאמר לשבת יצרה וזה אי אפשר בלעדה א״כ עלי׳ מוטל רק אם ירצה האיש לישא אותה להנשא לו ועל זה אמרינן מצו׳ בה יותר מבשלוחה אם יבא האיש לקדשה אבל היא לא צריכה להשתדל להנשא משום שבת שאין דרכה של אשה לחזור אחר איש ואין זה מענין לשבת יצרה ולזה אמרינן דרשות לה לישב בלא איש או להנשא לסריס אכן זה דוקא לענין אשה אבל לענין עבד כיון שאיש הוא ודרכו לחזור אחר אשה א״כ בו נכלל במצות לשבת יצרה להשתדל לישא אשה ולכן שפיר אמרינן דהאדון צריך לשחררו לקיים מצו׳ זו דשייכת בו כיון שהיא ג״כ באשה רק בהחילוק שחלקנו שעליו מוטל החיוב להשתדל לישא ועלי׳ רק החיוב להנשא אם ירצה הבעל לישא אותה ועל כן ל״ק ג״כ הקושיא שהקשה מעכ״ת נ״י ושהקשתי ג״כ בספרי וכעת מצאתי שגם בפ״י הקשה כן דאי מצות שבת באשה מאי פריך כמאן כריב״ב הרי גם לרבנן איכא חיוב דז״א דלשחררה ולהביאה לידי חיוב זה אין בה בכלל לשבת יצרה דזה הוי כמו לחזור אחר איש דאין דרכה של אשה בכך משא״כ בעבד דדרכו של איש לחזור אחר אשה לכן שייך אצלו החיוב משום שבת לשחררו כדי שיכול לקיים מצו׳ זו כנלענ״ד, הקטן יעקב.\\\vspace{0pt}

\end{multicols}\newpage

\newchap{סימן קכד}
\begin{multicols}{2}
ב״ה אלטאנא, בחדש סיון תר״ח לפ״ק. לק״ק אמשטרדם.\\\vspace{0pt}

מה ששאל מעכ״ת נ״י להביע דעתי בת״ח צדיק היושב על כסא הוראה ויש לו בנות אכן לא זכה להעמיד בנים שג׳ בנים שילדה לו אשתו זה אחר זה מתו ואחר שבספר חסידים סי׳ רמ״ז וסי׳ תע״ח כתב שיעזוב העיר שדר בה אם יעשה כן שיש לדאוג שאם יעזוב הרבנות שמא יבא לשם שאינו הגון כ״כ ובפרט שהוא עשיר ואין צריך לישא פנים מכל וכל.\\\vspace{0pt}

תשובה הן אמת דליכא בזה משום אין אומרים לאדם חטא בשביל שיזכה חבריך דבין לפי מה שחלקו התוספ׳ בשבת (דף ד׳) ועירובין (דף ל״ב) לחד תירוצא דלא אמרינן כן רק בפשע ובין לפי מה שחלקו לאידך תירוצא דלא אמרינן כן רק לגבי זכות יחיד ולא לגבי זכות רבים ובין לפי מה שחילק הריטב״א בעירובין (שם) דלא אמרינן כן רק לענין לפטור מחטאת אבל לא לענין לזכהו בעשיית מצו׳ ע״ש לכל הנך חילוקים לא שייך משום א״א לאדם חטוא וכו׳ אם נאמר שהרב ישאר על מקומו משום זכות הרבים התלוי בזה אף אם אצלו יחשב לחטא כיון דאין פשיעה מצד הזוכים וגם יש זכות הרבים ואף שלפי הגהת הרי״ף שהביא המג״א (ס״ס ש״ו) לא התיר לעשות איסור אפילו בשביל הרבים וכאן יש חשש איסור לפי מש״כ הספר חסידים שגורם לבניו שימותו הרי כבר כתב המג״א שכל הפוסקים חולקין עליו ועוד דהכא אין כאן מעשה איסור אלא שב ואל תעשה. אכן גם בעיקר החשש שהזכיר הספר חסידים שיש לומר דהמקום גורם ושעל כן יצא מן המקום לא הוזכר זה בשום מקום בגמרא דאף דליש אומרים בר״ה (דף ט״ז) שינוי מקום גורם קריעת גז״ד וילפי כן מאברהם הרי לרבי יצחק דלא חשיב רק ד׳ דברים ולא שינוי מקום לא ס״ל כן אלא דשאני אברהם משום זכותא דא״י והלכה כמותו דלפי מש״כ המהרש״א בח״א ביבמות (דף ס״ד) ע״כ סתם גמרא דשם ס״ל כן ולא כי״א ע״ש ואף דהרמב״ם ה׳ תשובה (פ׳ ב׳) הביא ג״כ דיגלה ממקומו הרי שכתב הטעם דאמרינן גם במקום אחר דגלות מכפר עון אבל לא ששינוי מקום ישנה מזלו להיות לו בנים ומשום כפרת עון א״צ שהרי אין מיתת בנים סימן לעבירה דהרי אמרינן בברכות (דף ה׳) שייסורין של אהבה הן למי שעסק בתורה ובג״ח וכן משמע שלא חששו חכמי הגמרא לזה שהרי אמרינן (סוף מ״ק) דבי׳ רבה שתין תכלי וכל שנותיו ארבעים שנה ולפי מה דאמרינן (סוף הוריות) מלך כ״ב שנה הרי שתכלי אילו רובם או כולם היו בעת שמלך ואעפ״כ לא גלה לעזוב נשיאותו משום זה עד שלבסוף ימיו ערק מפני גזירת מלכות כדאמרינן פרק השוכר את הפועלים אבל מה אעשה שספר חסידים רוב דבריו דברי קבלה ובכמה מקומות מתנגדים להגמרא ואעפ״כ רבים חוששים להם וגם הרבה פעמים הנסיון מעיד על דבריו ולכן ע״פ דין הגמרא לא ידעתי חיוב לצאת מן המקום אבל אין בידי להשיב את דעת מי שלבו נוקפו שלא לחוש לדברי ס״ח ובפרט דחמירא סכנתא מאיסורא אכן בלא״ה כבר אמרו רז״ל שיש שלא זכו לבנים בילדותם וזכו לבנים בזקנותם ורחמי שמים מרובים כנלענ״ד, הקטן יעקב.\\\vspace{0pt}

\end{multicols}\newpage

\newchap{סימן קכה}
\begin{multicols}{2}
ב״ה אלטאנא, בחדש כסליו תר״י לפ״ק. לק״ק מאסבאך.\\\vspace{0pt}

נשאלתי – למה השמיט הרמב״ם הלכה ברורה שחייב אדם לישא אשה לבנו וללמדו אומנות וי״א אף להשיטו בנהר והיא גמרא ערוכה קידושין (דף כ״ט) ושלשה מצות הראשונים הנזכרים שם הביא כל א׳ ואחד במקומו וביותר צ״ע מאחר שהביא בפירושו למשניות אילו השש מצות.\\\vspace{0pt}

תשובה – לישא אשה לא השמיט הרמב״ם דכן כתב בה׳ איסורי ביאה (פ׳ כ״א ה׳ כ״ה) מצות חכמים שישיא אדם בניו ובנותיו סמוך לפרקן והיא הברייתא דסנהדרין (דף ע״ו) ובה נכלל המצו׳ שחייב להשיא לבנו אשה רק שבברייתא זו מבואר יותר מתי צריך להשיא אשה לבנו גם שמצו׳ על האב ללמד בנו אומנות אף שכעת לא ראיתי להרמב״ם שכתב בפי׳ כן מכ״מ לא השמיטה והביאה לפסק הלכה בה׳ רוצח (פ׳ ה׳ ה׳ ה׳) שכתב שם אבל אם ייסר בנו כדי ללמדו תורה או חכמה או אומנות ומת פטור ע״ש שמחלק בין כבר למדו אומנות אחרת או לא והיינו ע״פ גמרא דמכות (דף ט׳) דבלא גמיר מלאכה אחריתי מצו׳ קעביד ואם השמיט הרמב״ם הך דיש אומרים דגם להשיטו בנהר בזה אין תימה דפסק כת״ק דג״כ לא חשיב הך. הקטן יעקב.\\\vspace{0pt}

\end{multicols}\newpage

\newchap{סימן קכו}
\begin{multicols}{2}
ב״ה אלטאנא, יום ה׳ כ״ב תמוז תרי״ג לפ״ק.\\\vspace{0pt}

במכשירין (פ׳ ב׳ משנה ז׳) תנן מצא בה תינוק מושלך אם רוב עכו״ם עכו״ם אם רוב ישראל ישראל מחצה למחצה ישראל ותניא על זה בתוספתא מחצה למחצה מטילין עליו שני חומרין ודין זה מוסכם מכל הפוסקים הביאו הרמב״ם ה׳ א״ב (פ׳ ט״ו) והטוש״ע אהע״ז (סי׳ ד׳) ע״ש והנה מהו הפי׳ מטילין עליו ב׳ חומרין פירש הר״ש שם וז״ל דמחמירין עליו כישראל דאסור בנבלות ומחמירין עליו ככותי כי ההיא דאמרינן בפ׳ ד׳ מיתות (דף נ״ט) דכותי העוסק בתורה א״נ אם קדש אשה ובא ישראל אחר וחזר וקדשה עכ״ל ומה דנקט בכלל החומרות גם שלא לעסוק בתורה אף דזה הוי קולא לגבי ישראל שמצוה לעסוק בתורה י״ל הרי יכול לעסוק בז׳ מצות דב״נ דלימוד זה מותר גם לב״נ כדאמרינן בסנהדרין (שם) התם בשבע מצות דדהו ולא קשה ג״כ היאך יעשה לענין קריאת שמע דאם ישראל הוא צריך לקרות ואם כותי הוא אסור לקרות די״ל דלמ״ד בברכות (דף כ״א) ק״ש דרבנן אין קפידא כ״כ אם לא יקרא ואעפ״כ שייך לומר מטילין עליו ב׳ חומרין כיון שלא עובר אמצו׳ דאורייתא ואפילו למאי דקיימא לן ק״ש דאורייתא ג״כ י״ל כיון דבענין פרשיות דק״ש הוא קבלת עול מלכות שמים ועול מצות ושלא לעבוד ע״ז וכל זה שייך גם בב״נ שפיר יכול לקרות דהרי במצות דידי׳ יכול לעסוק ואפילו פרשת ויאמר כיון דכתיב בה ולא תתורו אחרי לבבכם וגו׳ שייך ג״כ גבי ב״נ לענין מצותיו.\\\vspace{0pt}

אמנם מי יתן ונדע היאך יתנהג ב׳ חומרות לענין שבת לפי מה דאמרינן בסנהדרין (דף נ״ח) עכו״ם ששבת ח״מ שנאמר יום ולילה לא ישבותו וא״כ היאך יעשה אם ישבות שמא ב״נ הוא ואסור לשבות ואם לא ישבות שמא ישראל הוא וחייב לשבות והנה ידוע קושית המפרשים בכעין זה להנך שיטות דלאבות קודם מ״ת הי׳ דין ישראל רק להחמיר ולא להקל היאך נהגו לענין שבת והרבה כבר נדבר בזה (ולא ידעתי למה חקרו בענין זה במה שאינו לדינא ולא בהך דמצא תנוק מושלך שיש נפקותא גם לדידן) והגאון מ״ה פנחס הורוויץ זצ״ל בספרו פנים יפות בפ׳ נח תירץ על קושיא זו דאזהרת יום ולילה לא ישבותו היא שלא יהיו שובתים יום ולילה שאחריו כדכתיב יום ולילה ולפ״ז אפשר שהאבות עשו מלאכה בע״ש ובמוצאי שבת והא דכתבו התוספ׳ בסנהדרין שם דלכך הלאו דיום ולילה ל״י אינו בכלל דנאמרה ולא נשנית דלישראל נאמרה דהרי ישראל נצטוו לשבות והרי לפ״ז אפשר דנאמר גם לישראל דלא נצטו׳ ישראל לשבות יום ולילה שאחריו י״ל שהרי מכמה דרשות מקרא מוכח דאפשר שיחולו שבת ויוה״כ סמוכים זה לזה וע״כ צריך לשבות יום ולילה שאחריו עכ״ד וע״פ הדברים האלה מצאנו פשר גם לתנוק מושלך ספק ישראל ספק ב״נ כה״ג שישבות בשבת ויעשה מלאכה בע״ש ומ״ש שהרי אצלנו לא יחולו שבת ויוה״כ סמוכים זה לזה אכן כבר השיב בתשובת רבי עקיבא (בהשמטות) על דברי הגאון הנ״ל וז״ל ואולם לענ״ד אין קיום לדברי הגאון הנ״ל דא״כ עדיין יקשה לר״ל דס״ל חצי שיעור מותר מה״ת א״כ אף ביוה״כ ושבת סמוכים להדדי אפשר לקצור ביום השבת כחצי גרוגרות ובלילה שלאחריו בליל יוה״כ חצי גרוגרות דלגבי כל אחד הוי ח״ש דשבת ויוה״כ אין מצטרפין כדאיתא במתניתן בכריתות ומצד אזהרה דיום ולילה ל״י מצטרפי׳ להקצירות כגרוגרות ולא שבתו יום ולילה שלאחריו מבלי עשות מלאכה בהם עכ״ל אמנם לענ״ד משום הא ליכא פירכא אדברי הגאון ר״פ זצ״ל שלריש לקיש דס״ל חצי שיעור מותר מה״ת בלא״ה לא קשה קושית המפרשים ואין צורך להתירוץ לומר דבב״נ הלילה הולך אחר היום שהרי לפי מש״כ הרמב״ם ה׳ מלכים (פ׳ ט׳) שיעורים לא נאמרו לב״נ ולפ״ז גם בעשה הב״נ מלאכה בחצי שיעור כגון שקצר חצי גרוגרות כבר נחשב אצלו מלאכה ולא יעבור על לא ישבותו וא״כ אין קושיא על האבות שהרי אפשר שעשו מלאכה בשבת בחצי שיעור דלענין לא ישבותו נחשב מלאכה ולענין שביתת שבת לגבי ישראל לא עשו מלאכה ועל כן לא קשה קושית המפרשים רק לר״י דס״ל ח״ש אסור מה״ת ולדידי׳ תירץ שפיר דלילה הולך אחר היום ומכ״מ מוכח דלא ישבותו לא נאמר לישראל מדחלו שבת ויוה״כ סמוכים זה לזה וליכא תקנה לעשות מלאכה בחצי שיעור.\\\vspace{0pt}

אבל באמת מה שחדש הגאון ר״פ זצ״ל דבב״נ קודם מ״ת הלילה הולך אחר היום בלא״ה לא נלענ״ד שהרי ילפינן דיום הולך אחר הלילה מברייתו של עולם כדאמרינן בחולין (ד׳ פ״ג) ומהיכי תיתי שיהי׳ בב״נ שיעור אחר של חשבון היום ואי משום דכתיב יום ולילה ל״י ולא לילה ויום הרי סדר הכתוב לכתוב כן להקדים יום משום דהוא עיקר טפי כדכתיב תשבו יומם ולילה והגית בו יומם ולילה והרבה כן ולא מצאתי בכל המקרא שהקדים לילה ליום רק בד׳ מקומות ופחדת לילה ויומם (דברים כ״ח:ס״ו) להיות עיניך פתוחות אל הבית הזה לילה ויום (מלכים ח׳) תרדנה עיני דמעה לילה ויומם (ירמיהו י״ד:י״ז) שלשת ימים לילה ויום (אסתר ד׳:ט״ז) ובכל שאר מקומות הרבים הקדים יומם ללילה וגם באלה שהקדים לילה יש ליתן טעם דלענין פחד הלילה עיקר יותר שהיא זמן הפחד ולכן גם בשמירה הקדים לילה שבלילה יותר צריך שמירה מביום וכן לענין דמעה שהלילה עת הבכי׳ יותר כלשון הכתוב בכה תבכה בלילה ומה דכתיב באסתר לילה ויום כבר נדרש במדרש להורות שהפסיקו מבעוד יום ולכן כיון שבכל המקרא הקדים יום ללילה אין ראי׳ ממה דכתיב יום ולילה ל״י דיש חשבון זמן אחר בב״נ מבישראל. אמנם מה שנלענ״ד בענין קושית המפרשים וזה העיקר מה שראיתי לחקור בזה שידוע דענין מלאכת שבת אינו כענין מלאכה לפי הוראת השם מלאכה שהרי האומן שהוציא מחטו בבגדו בשבת לרה״ר נקרא עושה מלאכה בשבת ומי שנושא משא כבדה ברשות היחיד אפילו כל היום כולו נקרא שובת בשבת דכן נמסר למשה בסיני שרק מל״ט מלאכות שהיו במשכן צריך לשבות והנה כבר הזכרתי שהרמב״ם פסק דאין שיעורין לב״נ ונלענ״ד שיצא להרמב״ם כן ממה דאמרינן חולין (דף ק׳) אר״י והלא מבני יעקב נאסר גיה״נ ועדיין בהמה טמאה מותרת להם אמרו לו בסיני נאמר אלא שנכתב במקומו ע״ש והיוצא משם שמצות שנהגו קודם מ״ת נהגו על פי פשטות הדברים מבלי הקבלה ולא נשתנו ע״י נתינת המצות והקבלה ולכן למ״ד שמבני יעקב נאסר גיה״נ שעדיין לא הי׳ חילוק בין טמאה לטהור׳ נשאר איסור גיה״נ בטמאה גם לאחר מ״ת וכיון דשיעורין נמסרו למשה בסיני כדאמרינן סוכה (דף ו׳) א״כ קודם מ״ת מה שנאסר לב״נ נאסר בלא שיעור בין רב בין מעט ונשאר כן גם לאחר מ״ת ולכן שיעורין לא נאמרו לב״נ. ומזה נשפוט ג״כ כיון דל״ט מלאכות לא נמסרו רק בסיני א״כ קודם סיני הי׳ נקרא שביתה ממלאכה לשבות מכל דבר טורח ויגיעה קצת וזה הפי׳ יום ולילה לא ישבותו שלא ישבתו לגמרי ממה שנקרא מלאכה כפי הלשון דהיינו שיש בו טורח ויגיעה וזה לא נשתנה משנתנה תורה לישראל ולכן מי ששבת מכל ל״ט מלאכות ונשא משא כבדה ברשות היחיד לא עבר על יום ולילה ל״י וכן מי ששבת מכל יגיעה אף שהוציא תבלין כל שהו או מחט בבגדו לר״ה הגם שלענין שבת נקרא עושה מלאכה מכ״מ עבר על לאו דל״י ואין לומר דא״כ דלא ישבותו וביום השביעי תשבות יש חלוק שביתה לשני עניינים יקשה כקושית התוספ׳ בסנהדרין הנ״ל מנ״ל דלא ישבותו נאמר לב״נ ולא לישראל כיון שלא נשנית ולא שייך תי׳ התוספ׳ מדצו׳ הקב״ה לישראל לשבות דאכתי יכול לעשות מלאכה של יגיעה שאינה מל״ט מלאכות דז״א כיון שצו׳ התורה לשבות זכר ליציאת מצרים שנחו מעבדות וזכר לבריאת עולם ודאי אין סבר׳ שהטיל הכתוב על ישראל לעשות מלאכת יגיעה דא״כ אין זכר לדברים האלה בזה מכ״מ קושית המפרשים מתורצת בזה די״ל דהאבות קיימו שבת ועשו מלאכת יגיעה שאינה מל״ט אבות מלאכות להחמיר וכהאי גוונא משכחת ג״כ שספק ישראל ספק ב״נ יקיים שבת מבלי שעובר על יום ולילה ל״י כנלענ״ד, הקטן יעקב.\\\vspace{0pt}

\end{multicols}\newpage

\newchap{סימן קכז}
\begin{multicols}{2}
ב״ה אלטאנא, יום ד׳ י׳ שבט תרי״ד לפ״ק. להרה״ג וכו׳ מ״ה חיים יוסף פאללאק נ״י הגאב״ד דק״ק טרעביטש יע״א.\\\vspace{0pt}

מר נ״י השיב על דברי הנ״ל וז״ל – מה שרצה מעכ״ת נ״י לחלק בין השביתה שמוזהר ב״נ עלי׳ שהיא רק שלא לנוח מעמל ויגיעת הגוף ולא מל״ט מלאכות שנמסרו בסיני שיש בתוכן קלות מאוד כהוצאת תבלין כל שהו ודומיהן ובין השביתה בשבת שנצטוו ישראל עליה מכל ל״ט מלאכות ולא מיגיעת הגוף כנשיאת המשא הכבד ברה״י אף שסברתו זאת בנוי׳ על אדני השכל בכל זאת יש לעורר עלי׳ כי ממה שאמרו חז״ל בע״ז (נ״ח ע״ב) ובחולין (קל״ז ע״ב) (ויעוין בזה גם בתי״ט פ״ה דתמורה משנה א׳) לשון תורה לעצמה לשון חכמים לעצמה נראה בהיר שרק בין הוראת איזה מלה בלשון המקרא ובין הוראתה בלשון חכמים נוכל לחלק אבל בלשון המקרא הוראת כל מלה שוה בכל המקומות וא״כ אחרי שבא בקבלה שלשון השביתה הנאמרת בשבת וי״ט בלשון התורה מורה רק על המניעה מל״ט מלאכות קלות וכבדות ולא על המניעה ויגיעת הגוף שאינה מלאכה ע״כ שגם במאמר לא ישבותו הנאמר בתורה לבני נח הוזהרו בו רק שלא למנוע מעשות הל״ט מלאכות ולא מניעת היגיעה בלבד וכעין זה למדו דבי ר״י ביומא (דף ע״ד ע״ב) משיווי לשון עינוי הנאמר ביוה״כ עם הנאמר בדברים (ה׳ ג׳) על שיווי הוראתם על מניעת האכילה והשתי׳ כמש״א נאמר כאן עינוי ונאמר להלן עינוי מה להלן עינוי רעבון אף כאן עינוי רעבון ובירושלמי דאין דורשין איתא נאמר כאן ריחוף (מרחפת על פני המים) ונאמר להלן על גוזליו ירחף מה להלן מגע ואינו מגע אף כאן וכו׳ וכמו כן נמצא עוד בהרבה מקומות בש״ס. היוצא מדברינו אלה שהוראת מלת תשבות בשבת וי״ט צריכה להיות דומה להוראת יום ולילה לא ישבותו שנאמר לב״נ ואף שזה נאמר קודם מתן תורה וזה אחריו בזה אין לחלק דלשון תורה אחת היא ועוד הרי מצינו ביבמות (י״ז ע״ב) דלמדים אחים לענין יבום מאחים דבני יעקב ובחולין (פ״ה ע״א) גמר ר״ש שחיטת אותו ואת בנו שצריכה להיות ראוי׳ מטבוח טבח שנאמר ביוסף ור״מ שחלק עליו טעמו רק משום דאין דנים שחיטה מטביחה יע״ש ואי נאמר כדברי מעכ״ת נ״י דהוראת לא ישבותו הנאמר לב״נ היא להזהיר על המניעה מכל עמל ויגיעת הגוף יקשה שיגיעת גוף כזה תאסר מן התורה גם לישראל בשבת וי״ט בעבור העשה דתשבות א״ו שגם אזהרת השביתה דב״נ היא רק על המניעה מעשות א׳ מל״ט מלאכות לבד אף שהיא קלה בטבעה שלא כדברי מעכ״ת נ״י, עכ״ד דמר נ״י.\\\vspace{0pt}

ועל זה אשיב: מר נ״י חלק עלי מבלי שהשיב על ראיותי ולא אדקדק במה שכתב והחליט דרק בהוראת מלה בין לשון תורה ובין לשון חכמים חלקו חז״ל אבל בלשון המקרא הוראת כל מלה שו׳ בכל המקומות שזה מתנגד למה שכתבו התוספ׳ בקידושין (דף ל״ז) בשם ר״ת לענין ממחרת הפסח שבתורה ושביהושע דיש להם פירושים שונים דלשון תורה לחוד ולשון נביאים לחוד יע״ש וגם לא אדקדק אמה שכתב והביא ראי׳ ממה דאמרינן ביומא נאמר כאן עינוי וכו׳ דלענין לגמור פירוש המלה אדם דן גז״ש מעצמו דלענ״ד אין הדבר כן דבכל מקום שנאמר לשון זה נאמר כאן וכו׳ ונאמר כאן וכו׳ הוא גז״ש גמור דאין אדם דן מעצמו ובהא דעינוי גופא משמע מלשון ושקלי וטריא דגמרא שהוא גז״ש גמור רק אזכיר הראי׳ שהבאתי שחלק הרמב״ם בשיעורין בין ב״נ ובין ישראל מפני שכנמסרו המצות לב״נ עדיין לא נאמרו שיעורין והרי בב״נ נאמר אך בשר בנפשו דמו לא תאכלו לאזהרת אבר מן החי ובישראל נאמר לא תאכל הנפש עם הבשר לאזהרת אמ״ה ולמה לא אמרינן מה לא תאכל דגבי ישראל פירושו לא תאכל כזית אף לא תאכלו דגבי ב״נ כן אלא ע״כ דלא גמרינן פירוש מלת אכילה ממה שנאמר בסיני אלא אמרינן דגבי ב״נ דלא נמסר שיעור אזלינן אחר פירוש המלה דסתם אכילה משמע גם כלשהו כדאמרינן בשבועות (דף כ״א) ורק לישראל נמסר שיעור אחר אבל לא שינה זה מצות ב״נ וא״כ ה״נ לענין לא ישבותו דלב״נ דלא נמסר פירוש מסיני אזלינן אחר הוראת פירוש המלה דמשמע שביתה מיגיעה ורק לישראל נמסר בפירוש וביום השביעי תשבות שהוא מל״ט מלאכות אבל לא שינה זה ענין ב״נ דאם לא שינה הקבלה מסיני לענין מצות אבר מן החי עצמה הציווי לב״נ כש״כ לענין לא ישבותו שאין זה המצוה עצמה שעלי׳ באה הקבלה לישראל וכמו שהוכחתי ג״כ ממה דאמרינן לר׳ יהודה מדנאמר על כן לא יאכלו בנ״י את גיד הנשה לבני יעקב גם על של בהמה טמאה לא נשתנה זה ע״י מה שאסרה התורה אח״כ הטמאים והרי זה לא יעלה על הדעת שכשנצטו׳ נח לא ישבותו שנצטו׳ דוקא על מה שתאסור התורה אח״כ ל״ט מלאכות דילפינן ממלאכת המשכן ושענין ל״ט מלאכות שנמסרו בסיני כבר נאמר לב״נ וא״כ ע״כ גם ע״י קבלה מסיני לא נשתנה ולכן הנלענ״ד הנכון כמו שכתבתי, הקטן יעקב.\\\vspace{0pt}

\end{multicols}\newpage

\newchap{סימן קכח}
\begin{multicols}{2}
ב״ה אלטאנא, בחדש טבת שנת תר״ו לפ״ק.\\\vspace{0pt}

שאלה – כהן נשא אלמנה ואחר החתונה נודע לו שנחלצה מיבמה ובאין מוחה שהה עמה כמה שנים ואין לה בנים ממנו ועתה באו אנשי יראי השם ומבקשים ממנו שיוציא החלוצה ע״י גט אם צריך להוציא ואת״ל שצריך להוציא אם אינו רוצה להוציא אם מותר לקרותו ראשון לתורה ולהעלותו לדוכן. והנה בני משפחת הכהן אומרים שהחלוצה הזאת איילונית היא שהבעל מעיד שמעולם לא היתה לאשתו דדים כשאר נשים גם תשמיש הי׳ קשה לה בשנים קדמונים וכיון שאיילונית אינה צריכה חליצה אומר שהחליצה אינה חליצה ומותרת לו – והדין עם מי.\\\vspace{0pt}

תשובה – שאלה זו תלי לכאורה בפלוגתא קדומה אשר כבר ערכו עליו אנשי מלחמת התורה מערכה מול מערכה – בשו״ת מהר״ש ממודינא רצה להקל בכהן בזה״ז שנשא שבוי׳ שא״צ להוציא כיון שכהנים בזה״ז אין להם ייחוס א״כ אינם רק כהני ספק אכן כבר חלקו עליו בשו״ת מהרי״ט ח״א סי׳ קמ״ט ובשו״ת חוט השני סי׳ י״ז ואחריהם בא בעל שו״ת שבות יעקב ח״א סי׳ צ״ג לסתור דבריהם ולחזק דעת הרשד״ם ומכ״מ סיים דמיראי הוראה הוא ולא יעשה מעשה עד שיסכימו עמו בעלי הוראה ואחריו קם בעל כנסת יחזקאל (סי׳ נ״ו) ודחה דבריו בשתי ידים דחלילה לנו להוציא לעז על כל יחוסי כהונה בזה״ז ולהחזיק הבכורים בספק אינם פדויים וא״כ בטל דעת השבות יעקב כנגד כל הני רבוותא ובפרט אחר שלא נתקיים תנאו שיסכימו עמו גדולי חקרי לב ולא עוד אלא דבנדון דידן אפילו בעל שבות יעקב מודה שהרי כתב דלהחזיק בניו לכהונה פשיטא דלא דממנ״פ אינם כהנים דאם הוא כהן אזי הם חללים ואם אינו כהן פשיטא שבניו ג״כ אינם כהנים וא״כ גם בנדון זה פשיטא שאין לקרותו בתורה ולהעלותו לדוכן משום ממנ״פ זה גופא.\\\vspace{0pt}

אכן גם מטעם דהבעל אומר שהיא איילונית לא ראיתי לה תקנה שתשאר אצלו וכש״כ לנהוג בו קדושת כהונה שהרי כנראה מדברי הבעל והקרובים אין לה כל סימני איילונית ובפרט הסימן שקולה עבה ואין ניכר בין איש לאשה מדלא הזכירו את זה משמע שאין לה והרי לפי המבואר בש״ע אהע״ז (סי׳ קע״ב ס״ד) אינה איילונית עד שיהיו לה כל סימני איילונית ואף דהרמ״א שם הביא בשם י״א שיטת הטור שגם בסימן א׳ נקראת איילונית מכ״מ נ״ל מדכתב בשם י״א ולא כתב וכן הלכה דגם הרמ״א לא כתב כן רק להחמיר לדונה כאיילונית שאסורה להתייבם אבל לא להקל לפוטרה מחליצה וא״כ ה״ה ג״כ שאין להתירה לכהן אם חלצה ואף שהמהרש״ל ביש״ש כתב ג״כ כדעת הטור מכ״מ להלכה אין נראה להקל נגד דעת הש״ע ועוד אפילו אי סימן א׳ סגי אפילו להקל מכ״מ מאן לימא לן דיש לה סימן א׳ דהבעל אינו נאמן אפילו היכי דאינו חשוד כמו דאמרינן לענין שבויי׳ באהע״ז (סי׳ ז׳) דאף דעבד ושפחה נאמנים מכ״מ הבעל אינו נאמן על עצמו וכש״כ שחשוד הוא שנשאה ואינה מוציאה עד יתברר הדבר ועוד אפילו לפי דבריו אין כאן אפילו סימן א׳ דקישוי בשעת תשמיש פשיטא שאינו סימן לפי דבריו שהרי אומר שבשנים קדמוניות הי׳ תשמיש קשה לה משמע אבל עתה לא ואי איילונית היא הרי אין לה רפואה כמו שכתבו התוספ׳ ביבמות (דף ב׳ ע״ב) דאיילונית אין לה רפואה וכיון שנתרפאת ודאי לאו איילונית היתה וסימן דדים ג״כ נראה שאין זה אצלה כיון שאומר שאין לה דדים כשאר נשים משמע הא דדים יש לה ובגמרא וכן בש״ע קאמר סתם אין לה דדים משמע שאין לה דדים כלל מדלא הוזכר שיעור דדים ומה נקרא אין לה דדים נראה לי שיש ללמוד ממומי כהן דבבכורות (דף מ״ד) קחשיב מום דדיו שוכבין כשל אשה והרמב״ם בה׳ ביאת מקדש (פ׳ ח׳) מפרש יותר שכתב מי שדדיו שוכבין על בטנו כדדי אשה משמע שהחילוק וההיכר להיות דדי אשה הוא שגדולים מעט עד ששוכבין מעט או הרבה וכיון דסימני איילונית הן להיות שו׳ לאיש כמו דאמרינן לענין קול ביבמות (דף פ׳ ע״ב) שאין ניכר בין איש לאשה מסתמא גם באין לה דדים הפי׳ כן שאין שוכבין כלל עד ששוו לדדי האיש אבל מה שאין גדולים הדדים כ״כ כדדי שאר נשים ודאי לא מקרי סימן איילונית.\\\vspace{0pt}

אבל ביותר נ״ל דאפילו יש לה סימן א׳ עד שהיא בכלל ספק איילונית וחלצה ג״כ אסורה לכהן דאף דאמרינן ביבמות וכן פסק באהע״ז (סי׳ קע״ב) איילונית שחלצה לא נפסלה לכהונה ואמרינן ג״כ ביבמות (דף כ״ד ע״א) קדמו וכנסו אין מוציאין מידם ותני שילא ואפילו שניהם כהנים דבספק חלוצה לא גזרו רבנן מכ״מ בנדון זה יש להחמיר יותר שהרי כבר הקשו התוספ׳ (שם) ממת בתוך שלשים דלמה לא תחלוץ כיון דספק חלוצה היא לכהן ותרצו דיאמרו דקים להו בנפל כיון שחולצת אחר נישואין. והנה התוספ׳ לא בארו איך הדין אם חלצה תחתיו דכהן אם אסורה או לא. אכן מדברי הרי״ף והרמב״ם יש ללמוד זה שאסורה שהרי כתבו שאשת כהן לא תחלוץ שלא תאסר עליו משמע דאם חלצה אסורה. ואף שמדברי התוספ׳ שבת (דף קל״ו ע״ב) מבואר שס״ל דאם חלצה מותרת לבעלה הכהן רק שאינה חולצת לכתחלה שלא להתיר חלוצה לכהן כבר כתבתי בחידושי דשיטה זו צל״ע ממה דפריך ביבמות (דף ל״ז ע״א) מי עביד רבנן תקנתא לכהן דהא כאן לא עבדינן תקנתא לכהן במה שאמרו שלא תחלוץ כיון דאפילו נחלצה מותרת משא״כ במעוברת ומינקת כשתקבל גט דאז אסורה מדאורייתא ולכן מוכח לכאורה כשיטת הראשונים דאסורה וכפסק הש״ע אה״ע (סי׳ קס״ד ס״ז) שכתב ג״כ כלשון הרמב״ם. ולכן הכא דכ״ע ידעי שנחלצה ושאינה ודאי איילונית כשיראו שנשואה לכהן יאמרו חלוצה מותרת לכהן ולכן ע״כ תצא משא״כ בהא דשתי אחיות דהתם ה״ט דמותרת דיאמרו קים להו. ולא עוד אלא אפילו נודע עתה אחר נישואין שאיילונית ודאי היא ג״כ יש טעם להחמיר שהרי אמרינן ביבמות (דף קי״ט) האשה שהלך בעלה וצרתה למדינת הים וכו׳ ומפרש הטעם בגמרא דשמא אתה מצריכה כרוז לכהונה. הרי אע״פ דמשום עגונה הקילו רבנן להתיר אפילו א״א דאורייתא ולסמוך אסברא דדייקא ומנסבא כדאמרינן שם (דף פ״ח) מכ״מ בזה לא הקילו והיינו ע״כ משום דחלוצה דרבנן ודרבנן צריך חזוק יותר משל תורה כדאמרינן שם (דף ל״ו) ובתענית (דף י״ז) ולכן דוקא בנדון דתני שילא דהתם כ״ע ידעי שהוא ספק שאי אפשר להתברר כשיראו שנשואה לכהן לא יאמרו שחלוצה מותרת לכהן אלא יאמרו כיון שספק דרבנן הוא אין מוציאין או כמו שכתבו התוספ׳ (שם) שיאמרו דקים להו להאי דלאו חלוצה הוא. אבל בנדון זה דכ״ע ידעי דאיילונית אפשר להתברר ואעפ״כ חלצו לה א״כ יאמרו דקים להו לרבנן דלאו איילונית היא ויאמרו חלוצה מותרת לכהן. ואפילו נתברר ודאי דאיילונית היא אתה מצריכה כרוז לכהונה דלמא דאיכא דידע החליצה ולא שמע ההכרזה ומה דאמרינן איילונית שחלצה לא נפסלה לכהונה אפשר דהיינו בשידעו בשעת חליצה דאיילונית היא רק לא ידעו דאיילונית פטורה מחליצה. אבל בשלא נודע עד אחר החליצה אתה מצריכה כרוז לכהונה. ולכן אין אני רואה היתר בדבר להתירה לו, ואם לא יוציא לא צריכים להנהיג בו מנהג כהונה. כנלענ״ד, הקטן יעקב.\\\vspace{0pt}

\end{multicols}\newpage

\newchap{סימן קכט}
\begin{multicols}{2}
ב״ה אלטאנא, יום ב׳ כ״ג כסליו תרכ״ו לפ״ק. להרה״ג וכו׳ מ״ה אליקים געץ הכהן נ״י הגאב״ד דק״ק גאירינג במדינת אונגארן יע״א.\\\vspace{0pt}

על דבר שאלתו דמר נ״י אודות הרשעה המנאפת תחת בעלה ש״ץ ושו״ב דקהלתו והעידו על זה עדים כשרים וגם בערכאות כבר נענשה על דברי נאופי׳ וחתמה בעצמה על אשמתה באופן שאין לספק עוד וכעת היא מופקרת ואעפ״כ אינה רוצה לקבל גט כריתות מבעלה כדי לעגנו ולצערו – אני מסכים בזה עם ההיתר של ק׳ רבנים להתיר לו חרם דר״ג לישא אשה טרם קבלה גטה ע״פ הדרך המבואר באהע״ז סי׳ א׳ רק שאין צריך לסלק לה כתובה אחר דזינתה ואבדה כתובתה כמבואר שם (סי׳ קט״ו) ואעפ״י שהתחייב עצמו בכתב ידו ליתן לה ב׳ מאות זהובים אם תקבל גט ממנה עם כל זה אחרי דפעם אחר פעם מיאנה לקבל גט על תנאי זה וגרמה לו הוצאות לקבץ היתר ק׳ רבנים גם הוא פטור מלקיים לה תנאו אפילו תרצה עתה לקבל גט ממנה כשתראה שלא תוכל לעגנו עוד.\\\vspace{0pt}

ומה דהקשה מעכ״ת נ״י בסוגיא דבטלו מבוטל גטין (דף ל״ג) דלרשב״ג בטלו אינו מבוטל ושרינן אשת איש לעלמא משום דאפקעינהו רבנן לקידושין מני׳ מה יענה רשב״ג בכהן שנשא בת ישראל ואכלה תרומה תחתיו וכתב לה גט ובטלו ע״י שליח דאי אמרינן דבטלו אינו מבוטל משום דאפקעינהו רבנן לקידושין מני׳ ועשאו לבעילתו בעילת זנות הרי היתה למפרע זרה ואכלה תרומה שיש בה חיוב מיתה לענ״ד מזה אין קושיא ע״פ מה שכתבו התוספ׳ שם על קושיא דיכולים ממזרים לטהר דלתקנה עשו חכמים ולא לתקלה א״כ הכי נמי י״ל דבאשת כהן כיון דתקלה תהי׳ שאכלה זרה תרומה לא עשו חכמים תקנה לעקור קידושין למפרע אבל ביותר ה״ל להקשות על מתניתן דנדרים (דף צ׳) באומרת טמאה אני לך דתביא ראי׳ לדברי׳ ואי לא מייתי ראי׳ מותרת לבעלה והקשה הר״ן כיון דע״פ דין אסורה לבעלה כמשנה ראשונה היאך התירוה משום שמא עיני׳ נתנה באחר ותירץ דאפקעוה רבנן לקידושין מעיקרא ונמצא שבשעה שנאנסה פנוי׳ היתה ומש״ה שריא לבעלה ואוכלת נמי בתרומה יע״ש הרי דבאשת כהן ס״ל להר״ן דעקרוה רבנן לקידושין מינה ובזה ודאי יקשה היאך עקרו לקידושין ועשאוה פנוי׳ למפרע דא״כ זרה אכלה תרומה אמנם באמת לק״מ כיון דמסקינן ביבמות (דף צ׳) דבשב ואל תעשה יש כח לחכמים לעקור דבר מן התורה והיינו אפילו במקום שע״י עקירתם נעשה איסור למפרע דקחשיב שם כבשי עצרת בכלל הדברים שעקרו חכמים בשב וא״ת דאמרו לא יזרוק ואע״ג דעי״ז נעשה להשוחט חילול שבת למפרע דאם יזרוק הוי טעה בדבר מצו׳ ועשה מצו׳ דפטור וע״י שאמרו לא יזרוק לא עשה מצו׳ וחייב חטאת כדאמרינן פסחים דף ע״א מכ״מ כיון דחכמים לא אמרו שיעבור בידים יש להם כח לעקור דבר מן התורה הכא נמי כיון דראו לעשות תקנה דבטלו אינו מבוטל ושלא תהי׳ נאמנת לומר טמאה אני לך הי׳ להם כח לעקור קידושין למפרע אע״ג דעי״ז נעשה איסור למפרע באכילת תרומה כנלענ״ד, הקטן יעקב.\\\vspace{0pt}

\end{multicols}\newpage

\newchap{סימן קל}
\begin{multicols}{2}
ב״ה אלטאנא, יום ב׳ ב׳ כסליו תרכ״ו לפ״ק. להרה״ג וכו׳ מ״ה משה נפתלי יעקב נ״י הגאב״ד דק״ק נייאמעגען יע״א.\\\vspace{0pt}

אשר שאל מעכ״ת נ״י על אודות כהן שזה כמה שנים שהועבר מן הכהונה שלא לעלות לתורה ראשון ושלא לעלות לדוכן בעבור שנשא אשה שילדה בזנות... זינה הוא עצמו עמה קודם שנשאה על כן הרב... אסר לישא אותה אבל לא שמע לרב ומורה ונשא אותה ועל זה העבירו אותו מן הכהונה והיא אומרת שהולד ראשון שהי׳ לה הי׳ מפלוני ישראל ולעולם לא זנתה עם אחר ובולד השני הוא והיא אומרים שהוא מבעלה הכהן שזנתה עמו קודם נישואין וכבר עברו שנים רבים והולידו בנים ובין הוא ובין הבנים מחזיקים עצמם לכהנים שלא לטמא למתים ועתה שואל עד מתי יהי׳ מסולק מן הכהונה שלא להקרא לתורה בשם כהן ושלא לעלות לדוכן ומה יהי׳ משפט הבנים.\\\vspace{0pt}

הנה לא נעלם ממעכ״ת נ״י מה שכבר האריכו הפוסקים ראשונים ואחרונים בדין זה בפנוי׳ שזנתה והוא והיא מודים אם כשרה לכהונה ונפסק בש״ע אהע״ז סי׳ ו׳ ס׳ י״ז דברוב פסולים אצלה לא תנשא לכהן וידוע דאצלנו במדינות האלה שאין שכונת יהודים בכ״מ הוי רוב פסולים אצלה ולכן יפה פסק לו הרב שאסור לישא אותה ואע״ג דלפי המבואר שם אם נשאת לא תצא זה דוקא בנשאה בשוגג אבל בנשאה במזיד אחר שנאסרה לו ודאי שייך הקנס שצריך להוציאה דאל״כ יהי׳ חוטא נשכר ובלא״ה האריכו האחרונים אם הודאתו מועלת דניחוש שמא עיניו נתן בה ועוד דבנדון זה גרע טפי דבזנות הראשון לא הודה הבועל כלל ואולי נתברר אז להרב שנחשדה לפסול לה ובניו גרעו עוד יותר שהם ספקי חללים וכפי אשר פסק בבית שמואל סי׳ ד׳ ס״ק מ׳ בכל מילי דאורייתא אזלינן לחומרא כשיטת הרמב״ם דלא מהני הוא והיא מודים וכיון דלפי המבואר בא״ח סי׳ קכ״ח זר העולה לדוכן עובר בעשה וע״ש במג״א דנראה דעתו דאפילו בעולה עם כהנים אחרים לכן אזלינן לחומרא ולא עולין לדוכן דהוי מילי דאורייתא וא״ל דגם לאידך גיסא אם לא יעלה הוי ספק איסור דאורייתא דאם כהן הוא ואינו עולה לדוכן כשחזן קורא כהנים עובר בג׳ עשה כמבואר (שם) דז״א כיון דע״פ תקנת חכמים אינו עולה לא עובר ואפילו לא הוי רק איסור דרבנן כמש״כ המג״א שם (ס״ק ד׳) דהחזן אין כוונתו על הפסולים ולאשר אסרו חכמים לעלות לדוכן אבל זה ודאי כיון דרק ספק חללים הם לחומרא צריכים להחזיק עצמם לכהנים שלא לטמא למתים ושלא לישא נשים הפסולים לכהונה כדין כל ספק כהן כדאמרינן יבמות (דף ק׳) וכל שכן שהאב מוזהר שלא לטמא ולענין לעלות לתורה ראשון היכא דיש כהן אחר ודאי אינם עולים דהא איכא משום וקדשתו אבל היכא דליכא כהן אחר י״ל דשייך גם בהם משום ספק וקדשתו ולא גרע ממי שאינו מוחזק לכהן ואמר כהן אני אף שאינו נאמן כמבואר אהע״ז (סי׳ ג׳) מכ״מ רשאים בזמן הזה לקרותו ראשון בשם כהן כמבואר ברמ״א שם משום דליכא אצלנו תרומה וא״כ כש״כ באילו שהם ספקי כהנים ולמען יחזיקו עצמם לכהנים שלא לטמא למתים אבל להאב לענ״ד אין לקרותו כלל בשם כהן לתורה כיון דנאסר מהרב וב״ד וחכם שאסר אין חבירו רשאי להתיר אפילו ברור לו ההיתר אם לא שיכול לברר שהחכם טעה בדבר משנה או בשיקול הדעת כמבואר בש״ך סי׳ רמ״ב כש״כ בענין זה שנתברר שע״פ הדין העבירוהו ממעלות כהונה. כנלענ״ד, הקטן יעקב.\\\vspace{0pt}

\end{multicols}\newpage

\newchap{סימן קלא}
\begin{multicols}{2}
ב״ה אלטאנא, יום ה׳ כ״ו סיון תרכ״ד לפ״ק. לחתני הרה״ג וכו׳ מ״ה יוסף איזאקזאהן נ״י אב״ד דק״ק ראטטערדאם יע״א.\\\vspace{0pt}

אשר שאל חתני נ״י על מה סמכו הרבנים לסדר קידושין לנשאת לשלישי אשר נתאלמנה כבר מב׳ אנשים.\\\vspace{0pt}

מסתמא סמכו על מה שכתב התרומת הדשן (סי׳ רי״א) שכבר בזמנו ראה כמה ת״ח גדולים ואנשי מעשה הגונים דלא הוי קפדי לישא אשה שמתו לה ב׳ אנשים ואף שהוא כתב יראה דלאו שפיר עבדי מכ״מ למד זכות עליהם שסמכו על אור זרוע שכתב שאינו רק ספק חששא ולא חיישי לספק חששא כמו שאין אנו נזהרים מכמה מילי דאזהירו רבנן עלייהו משום חשש סכנה דשומר פתאים ד׳ וגם נראה מתוך שאנו מתי מעט וצריכין אנו לישא מאשר נמצאו דשו בה רבים ושומר פתאים ד׳ גם באור זרוע כתוב ועוד דמעשים בכל יום שאנו רואים שנשאת לשלישי ומאריכין ימים ומולידין בנים ובנות עכ״ל ובזמנינו יש להוסיף עוד טעם שאם נמחה בידם יזדוגו ע״פ חקי המלכות בלא קידושין או יקבלו קידושין מהפריצים שאינם יודעים בטיב קידושין ולפני עדים פסולים לכן ודאי יש להזהיר ולהודיע למי שבא לישא קטלנית שיש סכנה ואיסור בדבר ואם יעמוד במרדו אין למנוע ממנו סידור קידושין במקום שיש חששות כנ״ל. כנלענ״ד, הקטן יעקב.\\\vspace{0pt}

\end{multicols}\newpage

\newchap{סימן קלב}
\begin{multicols}{2}
ב״ה אלטאנא, יום ב׳ ז״ך שבט כתר לפ״ק. להרה״ג וכו׳ מ״ה דוד ווייסקאפף נ״י הגאב״ד דק״ק וואלערשטיין ואגפי׳ יע״א.\\\vspace{0pt}

על דבר שאלתו וז״ל אלמנה א׳ שמת בעלה ח״י מרחשון תרי״ט ובכ״ג תמוז בשנת תרי״ח ילדה בן לבעלה ומחמת חולשת גופה והטורח והיגיעה רבה שהי׳ לה עם בעלה שנפל בחולי על ערש דוי ר״ל כמשלש שבועות אחרי לידתה ומחמת צימוק דדי׳ כי היא בעצמה חלושת הגוף והיא בת ארבעים פסקה החלב ולא היתה יכולה להניק את בנה וגמלתו והחי׳ אותו במאכלים ואחר שנתאלמנה היתה פרנסתה יותר בדוחק והיא פונדקית ונתקשרה בט״ו באב העבר עם פנוי א׳ יודע ספר ועוסק במלאכת שמים בכתיבת ס״ת תו״מ וזה בקרוב קבלו קיומים ונפשם בשאלתם אם מותרים לעשות הנישואין בתוך כ״ד חדש ממיתת הבעל.\\\vspace{0pt}

תשובה – הנה מעכ״ת נ״י כבר יצא להתיר אחר שגמלתו ג׳ חדשים קודם שמת בעלה ומבואר בש״ע סי׳ י״ג ס׳ י״א ע״פ שו״ת הרא״ש בשם השר מקוצי שבכה״ג אינה צריכה להמתין ואף שהרא״ש לא רצה לסמוך ולהתיר רק בשעת הדחק מכ״מ כאן שהי׳ לה צימוק דדים ומחמת כן גמלה את בנה גם הרא״ש מודה כמבואר בתשובתו כלל נ״ג סי׳ ד׳ וז״ל אבל אשה שיש לה צימוק שדים ואינה מינקת בשום פעם אותה ודאי לא מקרי׳ מינקת ואינה צריכה להמתין כ״ד חדש וכש״כ דאיכא ג״כ שעת הדחק שהאשה אין לה פרנסתה רק ע״י הנשואין וגם יש חשש ייחוד באשר שאנשים מתאכנסים אצלה עכ״ד הרמה. והנה אף שלפענ״ד אין הנדון דומה למה שכתב הרא״ש באשה שיש לה צימוק שדים דשם איירי שלא יכולה להניק כלל מעולם כמו שכתב ואינה מינקת בשום פעם אבל בנדון השאלה האשה התחילה להניק רק שאח״כ גמלתו מחמת צימוק שדי׳ ובזה לא היקל הרא״ש כמש״כ הטעם כי אשה תערים כן שתגמול עד יהי׳ לה צימוק שדים עם כל זה לענ״ד אין פקפוק להתיר אחר שגמלתו ג׳ חדשים בחיי בעלה ואין לנו להחמיר במה שפסקו בטוש״ע וכל הפוסקים האחרונים להתיר ואם הרא״ש לא רצה להתיר באשר שלא קבל מרבותיו הרי אנו קבלנו מרבותינו ההיתר וכל שכן דאיכא שעת הצורך ג״כ שאפילו הרא״ש לא מחה בכה״ג ועוד דממה שכתב הרא״ש בכתובות פ׳ אע״פ נראה שחזר ממה שכתב בשו״ת ופסק ג״כ בפשיטות להתיר וכן העתיקו הטור אלא שעדיין יש לי ספק במה שכתב מר נ״י בפשיטות שגמלתו ג׳ חדשים קודם מיתת הבעל שהרי לפי דברי השאלה ילדה כ״ג תמוז ובעלה נפל על ערס דוי ל״ע ג׳ שבועות אחרי לידתה שהוא לערך ט״ו אב ומחמת הטורח והיגיעה שהי׳ לה עם בעלה פסקה החלב וגמלתו ומסתמא לא בימים הראשונים אחר שנפל על ערס דוי הי׳ זה וגם אם גמלתו לגמרי ה׳ ימים אחר כך שהוא כ׳ אב אין כאן ג׳ חדשים שלמים עד יום מיתת הבעל ח״י מרחשון. אמנם גם אם לא היו ג׳ חדשים מכ״מ בכה״ג נראה להתיר שהרי הרא״ש פ׳ אע״פ והטור לא הזכירו כלל מג׳ חדשים וגם הש״ע לא כתב התנאי דג׳ חדשים רק בנתנה בנה למיניקה בחיי בעלה בלי שפסקה חלבה אבל בפסקה חלבה ושכרה מינקת נראה בפי׳ מדברי הש״ע שבזה לא התנה שיהי׳ ג׳ חדשים קודם מיתת הבעל ועוד נלענ״ד שבנדון השאלה הג׳ חדשים לא בעינן כלל שהרי מה דנתנו הגאונים שיעור דג׳ חדשים כתב הרא״ש בשו״ת הטעם בשמא והריב״ש כ׳ בפשיטות כן דקים להו דעד ג׳ חדשים אפשר שתשוב חלבה ותהי׳ ראוי׳ להניק וזה שייך דוקא בנתנה בנה למינקת שיש חשש שמא תחזור המינקת והיא תהי׳ ראוי׳ להניק ולא תרצה ותמית את בנה אבל בשלא נתנה בנה למינקת אלא גמלתו לגמרי בחיי בעלה ומחי׳ אותו במאכלים בזה לכ״ע לא בעינן ג״ח דמה בכך שתשוב חלבה ותהי׳ ראוי׳ להניק הרי הולד ל״צ ליניקה ואולי יסתכן אם תרצה להניקו אחר שנגמל כבר זמן מה ונתרגל במאכלים ולכן לענין גמלתו לא הוזכר בשום מקום שיעור דג׳ חדשים ואף דגמלתו אחר מיתת בעלה לא מועיל אף שלא צריך ליניקה שם הטעם דכל אלמנה תעשה כך ותגמול את בנה וימות אבל אם בשעה שגמלתו לא הי׳ חשש איסור באשר שהי׳ בחיי בעלה גם לאחר מיתתו אין כאן חשש מינקת אפילו בפחות מג״ח. ולכן נלענ״ד שאם עברו משעה שגמלתו לגמרי ג׳ חדשים בחיי הבעל אין ספק שאין כאן חשש איסור כלל שתנשא האלמנה לאיש וגם אם לא עברו ג׳ חדשים דעתי נוטה להתירה כיון שגמלתו לגמרי וגם פסקה חלבה בחיי בעלה אמנם כל זה בתנאי שיסכים עמנו ידידנו הרה״ג הגאב״ד דק״ק ווירצבורג נ״י. הקטן יעקב.\\\vspace{0pt}

\end{multicols}\newpage

\newchap{סימן קלג}
\begin{multicols}{2}
ב״ה אלטאנא, בחדש תמוז תקצ״ט לפ״ק. להרה״ג וכו׳ מ״ה אלכסנדר אראן נ״י אב״ד דק״ק פעגערסהיים יע״א.\\\vspace{0pt}

בדיק לן מר נ״י בשאלה שבאה לפניו נערה הרה לזנונים מאיש נכרי וילדה בן ואחר איזה חדשים ללדתה נשתדכה עם בחור אחד וגם נשאה ע״פ חקי המלכות שקורין קאפולאטיאן בל״א והתחייב עצמו לזון הילד כאלו הוא בנו בדין המלכות וגם כבר מיד כשנולד שכרה לו האם מינקת ולא התחילה להניק כלל ועתה רוצה הבחור לכנסה לחופה אם יש בזה חשש מינקת חבירו.\\\vspace{0pt}

תשובה מעכ״ת בנה דעתו להתיר על ד׳ עמודים – באשר – א׳ – שכרה מינקת – ב׳ – היא זונה ודומה למופקרת – ג – לא התחילה להניק כלל – ד׳ – באשר שהמשודך כבר נעשה בעלה בדין המלכות ומחוייב לזון הולד וגם קבל עליו בפי׳ כן.\\\vspace{0pt}

הנה ג׳ התירים הראשונים כבר פלפל עליהם בשו״ת שב יעקב (סי׳ ח׳) במעשה כזה ולא רצה להתיר והביא ראי׳ ממהר״י מינץ שג״כ לא רצה להתיר במזנה רק במופקרת והשיג שם על שו״ת אמונת שמואל ומסיק שגם באמונת שמואל לא סמך על דעתו עד שיסכימו עמו עוד שני גדולי רבנים גם בשו״ת כנסת יחזקאל לא סמך על התירים אילו לבד רק בצירוף טעמים אחרי׳ ובהסכמת ב׳ גדולי הדור ומי יודע אם הסכימו גם העיד בשב יעקב שבעל שבות יעקב אסר במזנה שנתנה בנה למינקת מיד ולא הניקה כלל גם בשו״ת נודע ביהודה שאלה י״ח החמיר ולא הקל רק בדיעבד שלא להפרישם וגם משום חשש עניות המעבירים את האדם על דעת קונו הן אמת שבשו״ת זכרון יוסף סי׳ ד׳ מיקל בכעין נדון זה ע״י שתשליש מעות אכן לא רצה לסמוך ג״כ רק אם יסכים עמו הגאון החסיד ר׳ דוד ז״ל אשר מנוחתו כבוד פה קהלתנו וב״ד וגם לא ידענו בזה אם הסכימו ובאמת לא אכחיד כי קשה בעיני להקל בסברות כאלה בגזירות חכמים דא״כ אחר שהזכירו חז״ל איסור נישואין כ״ד חדש אפי׳ נתנה בנה למינקת למה לא הזכירו ג״כ קולא כזה דבהשליש מעות מותר אלא ודאי לא רצו להקל בזה אף שהסברא נותנת דבזה אין חשש שמא ימות הולד מכ״מ לא רצו ליתן מקום להקל בגזירה דרבנן ומצאנו הרבה כיוצא מזה שהחמירו אפי׳ במקו׳ שלא שייך גזירה כלל ואף שח״ו לי להרהר אחר היתר הגדולים נוחי נפש אשר קטנם עבה ממתני מכ״מ הבו דלא להוסיף ושלא להקל רק בענין שהקילו הם והרי הזכרון יוסף לא הקיל רק בסניפי׳ של שעת הדחק וחרבן בית בני משפחה ובני משפחתה אין רגילים להניק ויש לחוש לפריצי הדור ולא ידעתי אם אפילו רוב הטעמים האלה שייכי׳ בנדון דלפנינו כיון דכבר הי׳ אשת איש ע״פ ציווי המלכות ודאי משומרת היא מפני הפריצים שלא תדון כאשת איש מזנה שקשה עונשה ע״פ ציווי המלכות גם בשו״ת בגדי כהונה ראיתי שהקל והורה להתיר בכה״ג ע״י שתשליש מעות המינקת עד סוף כ״ד אכן כתב ג״כ רק מטעם סניפי׳ להתיר שתנשא לבעל אחותה שתרחם על בני אחותה יותר וג״כ מטעם פריצי הדור.\\\vspace{0pt}

ברם מעכ״ת נ״י כתב סניף להתיר שתחשב זו כמופקרת באשר שזנתה עם א״י שודאי לא תנשא לו ועל התיר זה תמהתי שהרי מה שהקיל הר״י מינץ ואחריו הרמ״א במופקרת הוא משום קלקול אחרים שיזנו עמה וכעין הא דחצי׳ שפחה שהתירו בגטין (דף מ״ג) משום קלקול אחרים עמה לשחררה וא״כ כל שאינה מופקרת לזנות לא שייך זה והרי כפי מה שהוזכר בשאלה היא בת טובים ויצרה תקפה לזנות עם זה ובודאי שמטעם זה לא תחשב מופקרת שיש לחוש לקלקול אחרים ובפרט בנדון זה שכבר נעשית אשת איש ע״פ דין המלכות ועוד דלפי סברת מעכ״ת כל שזנתה עם בעל שיש לו אשה ג״כ תדון כמופקרת כיון שלא תוכל להנשא לו מחרם ר״ג ומצווי המלכות וזה ודאי אינו ויראה בשאלת הרב בגדי כהונה הנ״ל שהי׳ ג״כ הנדון שהבועל כבר נשא אשה ואפי׳ הכי לא עלה על דעתו לחשדה כמופקרת ועל מה שכתב מעכ״ת נ״י שפשוט בעיניו להתיר עד שכמעט אפי׳ היתה מינקת בעצמה הי׳ דעתו להתיר כיון שהבעל התחייב עצמו בדין המלכות הן ע״פ דתי המדינה והן ע״י כתב מיוחד לזון את הולד לענ״ד יש להשיב שאף שמטעם מה שמפקפק מעכ״ת בדינא דמלכותא דינא אין חשש בזה כיון שהתחייב עצמו בשטר גמור שמועיל חיוב כזה אפי׳ בדיני ישראל מכ״מ מה מועיל תקנה זאת שהרי בש״ס דיבמות בסוגיא דמינקת שם פריך ולתבעיניה ליורשי׳ ומשני אשה בושה לבא לב״ד והורגת את בנה וא״כ ודאי הסברא נותנת שיותר מזה אשה בושה לתבוע לבעל שלה אשר היא באמנה אתו ובביתו לבא לב״ד וגדולה מזו אמרו בכתובות (דף ק״ד ע״ב) דלכך אלמנה כל זמן שהי׳ בבית בעלה גובה כתובתה לעולם דאמרינן משום כיסופא מהיורשים כיון שמפרנסי׳ אותה אינה תובעת מהם כתובתה וכש״כ בנדון דידן שתבוש לתבוע בעלה במשפט לזון בן שודאי אינו בנו וכיון שהוא ודאי אינו אביו ואין לו רחמי אב על הבן יש לחוש שיחזור מדיבורו ולא יפרנסו ולכן אין לזה דמיון לנדון הרב בגדי כהונה ששם התחייב עצמו הבועל שהי׳ אביו של הבן ממש וגם שייך בזה שתבע אותו לדין אחר שאין לה התקשרות עמו משא״כ שנסמוך שתתבע בעלה לדין ודאי הסברא נותנת אם היא בושה לתבוע ליורשיו לדין כש״כ בעלה לכן קשה בעיני לסמוך על היתרים הללו במקום שנתייראו גדולי המורים אכן מכ״מ לא יחשב פסקי זה כפסק לאיסור אם יסכימו מגדולים שבמדינת מעכ״ת נר״ו או משאר מקומות להתיר כנלענ״ד, הקטן יעקב.\\\vspace{0pt}

\end{multicols}\newpage

\newchap{סימן קלד}
\begin{multicols}{2}
ב״ה אלטאנא, יום ו׳ כ״ד מרחשון תרכ״ד לפ״ק. לחתני הרה״ג וכו׳ מ״ה יוסף איזאקזאן נ״י אב״ד דק״ק ראטטערדאם יע״א.\\\vspace{0pt}

על דבר השאלה הבאה לפניך שמעון נשא רחל בת ראובן ולאחר שמתה נשא אשה אחרת ושבק חיים לכ״י ועתה בא ראובן לאחר שמתה אשתו לישא האשה הזאת שהיא אשת חתנו שנתרחק אם יש היתר בדבר.\\\vspace{0pt}

תשובה – הנה לא נעלם ממך שיש ב׳ דיעות בזה שהובאו בש״ע אהע״ז (סי׳ ט״ו ס׳ כ״ד) אלא שדעתך כיון דהש״ע כתב הדיעה הראשונה להתיר אשת חמיו ואשת חתנו בסתם והדיעה האוסרת בלשון ויש מי שאוסר הוא פוסק כדיעה הראשונה ע״פ הכלל שכתב הש״ך בי״ד (סי׳ רמ״ב) וצדקת בזה שכן כתב גם הברכי יוסף. והנה הפלוגתא באשת חמיו קדומה היא ונשתלשלה דור אחר דור בגמרא דילן יבמות (דף כ׳) מייתי ברייתא דמותר אדם באשת חמיו והתוספ׳ שם הביאו הירושלמי דאוסר מפני מראית העין והרי״ף והרמב״ם פסקו כגמרא דילן דמותר אכן התוספ׳ והרא״ש כתבו בשם רבינו חננאל ובשם הלכות גדולות ובשם ר״ת לאסור דשמא אחר כך אסרו והסכימו עמהם וכ״כ רבינו ירוחם וכ״כ הטור וכן מצאתי באור זרוע (שנדפס עתה) שאוסר אשת חמיו והנימוקי יוסף כתב בשם הריטב״א שהרמב״ן ורבים מרבותינו התירו ושכן עיקר ושכן המנהג בכל ספרד אמנם כל חכמי אשכנז פסקו כהאוסרים המהרי״ל והמהרש״ל והב״ח ושו״ת צ״צ והבית שמואל והבית הלל ובשו״ת נודע ביהודה מ״ק חאה״ע סי׳ כ״ו כתב שכן המנהג בכל אשכנז אלא ששדי בי׳ נרגא במה שכתבו התוספ׳ והרא״ש דאע״ג דע״פ הגמרא דילן מותרת אשת חמיו שמא אח״כ אסרו דאיך אפשר לומר כן דהאיסור דירושלמי הי׳ אחר גמרא דילן שהרי בירושלמי האוסר הוא רבי חנינא ובגמרא דילן נראה דרבא דבתרא הוא התיר אשת חמיו והניח בצ״ע אלא דמכ״מ לא מלאו לבו לחלוק על התוספ׳ והרא״ש ולענ״ד לק״מ דאין כוונתם דהאיסור דירושלמי הי׳ אחר גמרא דילן אלא שכתבו דמה דר״ח והלכות גדולות ור״ת אסרו וסמכו על הירושלמי נגד גמרא דילן מפני שאפשר דבימי רבנן סבוראי או הגאונים גזרו ואסרו וכ״כ בפי׳ המהרש״ל ביש״ש ומסיים שם ולפ״ד סברא ישרה היא שאי אפשר שר״ח וה״ג חלקו על התלמוד אם לא שנאסר במנין עכ״ל ובמכ״ה נעלם זה מהגאון נ״ב.\\\vspace{0pt}

והנה לענין אשת חתנו ראיתי פלוגתא גם בין האוסרים אשת חמיו דהלכות גדולות אוסר גם באשת חתנו וכן הטור והב״י כתב דאין זה מוכרח דגם באשת חתנו שייך גזירת מראית העין דהבו דלא לוסיף ומכ״מ בש״ע הביא היש מי שאוסר גם על אשת חתנו והמהרש״ל ביש״ש כתב ונראה דאשת חתנו בודאי מותרת דפשיטא דליכא למיחש מפני מראית העין ובודאי קלא אית להו וכולהו ידעי ומרגישי בין בתו לנכרית משא״כ באשת חמיו שנראת כחמותו וברור למבין גם בירושלמי לא נאמר והתוספ׳ והסמ״ג לא הזכירו כלל והרא״ש הזכיר בשם ה״ג אבל לא הכריע בינהם בזה רק שהטור כתב לאיסור ולא נהירא כלל כי אין כאן מראית העין ופשיטא שיותר נוטה להתיר מחורגין וכן נ״ל הלכה למעשה אפילו לכתחלה ותאב אני מתי יבא לידי ואתיר ואי יישר חילי אבטלנה עכ״ל ולא הבנתי אחר שהמהרש״ל בעצמו כתב לסמוך באשת חמיו על הגאונים נגד גמרא דילן שאי אפשר שהגאונים חלקו על התלמוד אם לא שנאסר במנין א״כ איך חולק על הלכות גדולות לענין אשת חתנו ואדרבא לדידן יש סברא יותר לאסור באשת חתנו מבאשת חמיו דהנ״ב המציא בשו״ת הנ״ל דאפשר מה שאסרו באשת חמיו היינו בעוד שאשתו בת חמיו קיימת אבל לדידן שאי אפשר לישא ב׳ נשים א״כ לא משכחת שישא אשת חמיו רק אחר מיתת אשתו אפשר דלא אסרו כיון דבזה קליש איסור חמותו דלקצת פוסקים אין כאן אפילו כרת רק איסור שוכב עם חותנתו ולכלם אין עוד איסור שריפה אפשר שבזה לא גזרו וכמעט שרצה לסמוך על זה להתיר אשת חמיו אלא שלא מלאו לבו אחר שלא נזכר בפוסקים והיתר זה לא שייך באשת חתנו שנאסרה משום מראית העין דנראת כבתו דהיא ודאי היא איסור שריפה גם בשו״ת מהרי״ל (סי׳ פ״ה) אוסר באשת חתנו וכתב נ״ו היינו טעמא דאשת חתנו דרגילין למקריא לי׳ חתן הן חמיו עצמו הן שאר אינשי רוב דעלמא קרי לי׳ חתן פלוני וכיון שכן הוא נקרא בפי רובא דעלמא ועל שם אשתו נקרא כן לא ידעי כולי עלמא איזה מן הנשים בתו של זה עכ״ל ולכן כיון שהמהרי״ל המיסד מנהגי אשכנז אוסר ואין מן האחרונים שמתיר אשת חתנו בפירוש חוץ מן המהרש״ל אין בידי להתיר נגד הטור והמהרי״ל ובפרט שאע״פ שכתבתי לעיל בשם ברכי יוסף שדעתו שהש״ע פסק כדעה הראשונה אין זה מוכרח דבשו״ת צמח צדק סי׳ מ׳ כתב שמשמע שנוטה דעת הרב ב״י להחמיר ממה שהביא שני הדעות וכתב דעת האוסרין באחרונה משמע שדעתו נוטה להחמיר ומהמהרש״ל עצמו נראית שהי׳ המנהג לאסור מדכתיב ואי יישר חילי אבטלינא גם מהנ״ב נראה שלא רצה להכריע בין הטעמים למען התיר אשת חתנו מכל הלין טעמי לא מלאני לבי להתיר להשואל לישא אלמנת חתנו. וראיתי להזכיר עוד שמה שכתב הרא״ש שהלכות גדולות אוסר באשת חמיו ובאשת חתנו לא מצאתי בהלכות גדולות שלנו כן דלא בלבד שלא הזכיר אשת חתנו אלא שגם באשת חמיו התיר אכן בלא״ה כבר מצאתי סתירות כאלה בין מה שהזכירו הראשונים בשם ה״ג ובין ה״ג שלפנינו והזכרתים בספרי ע״ל על נדה שהסמ״ג לאווין רמ״ז הביא בשם ה״ג דלאחר י״ב לנקבה וי״ג לזכר נדריהן נדר אע״פ שאין יודעין לשם מי נדרו ואע״פ שלא הביאו ב׳ שערות ובהלכות גדולות שלפנינו כתוב להיפך וכן הביא הראב״ד בשער הטבילה בשם הלכות גדולות דזבה בעי מים חיים ובהלכות גדולות שלנו מביא הברייתא דזבה לא בעי מים חיים ומזה נ״ל ברור שה״ג שלנו ושהזכירו הראשונים אינם שוים שכבר כתב הראב״ד בספר הקבלה שרבינו שמעון מקיירא חיבר הלכות גדולות שנת ד״א תק״א ואחריו חיבר הלכות פסוקות רבינו יהודא גאון בשנת תקכ״א ואחר שרש״י ושאר ראשונים כתבו שהמחבר של הלכות גדולות הוא רב יהודא גאון עיין בסוכה (דף ל״ח) ובהרבה מקומות מזה נראה דה״ג זה הי׳ לפני הראשונים והלכות גדולות שלנו הוא מר״ש קיירא ז״ל וכמדומה שנעלם זה מהגאון תבואת שור בי״ד סימן מ״ח. כנלענ״ד, הקטן יעקב.\\\vspace{0pt}

\end{multicols}\newpage

\newchap{סימן קלה}
\begin{multicols}{2}
ב״ה אלטאנא, יום ב׳ ב׳ כסליו תרכ״ו לפ״ק. להרה״ג וכו׳ מ״ה משה נפתלי יעקב נ״י הגאב״ד דק״ק נייאמעגען יע״א.\\\vspace{0pt}

על דבר שאלתו דמר נ״י אם מותר ליקח אשת אחי אמו מן האם שנולד בזנות דהיינו האם שהי׳ לה בן מבעלה נתאלמנה וילדה אחר שנתאלמנה בת בזנות ובנה נשא אשה ומת אם בן הבת שנולדה בזנות מותר לישא את אלמנתו שאפשר שעד כאן לא מבעי׳ בש״ס יבמות (דף כ״א) אלא באשת אחי אמו מן האם בנישואין אבל לא בנולד בזנות.\\\vspace{0pt}

תשובה – ע״פ המבואר בגמרא אסרו אשת אחי האם מן האם משום כל שבנקבה ערו׳ בזכר גזרו דכיון דאחות האם מן האם ערו׳ הלכך בזכר גזרו על אשתו ולענין אחות האם ודאי אין חילוק אם היא אחותה שנולדה בנישואין או בזנות שהרי ביבמות (דף כ״ב) אמרינן דאחיו לכל דבר חוץ מי שיש לו אח מן השפחה ומן הנכרית וזה ודאי דהוא הדין מי שיש לה אחות מכל מקום דנחשבה אחותה לכל דבר חוץ בנולדה מן השפחה ונכרית וכיון דבזכר אין חילוק בין נולדה בנישואין או בזנות ה״ה בנקבה. כנלענ״ד הקטן יעקב.\\\vspace{0pt}

\end{multicols}\newpage

\newchap{סימן קלו}
\begin{multicols}{2}
ב״ה אלטאנא, יום ו׳ שושן פורים תרי״ג לפ״ק. להרב וכו׳ מ״ה מאיר ליב לעבוואהל נ״י בק״ק קראקא יע״א.\\\vspace{0pt}

מר נ״י כתב אלי וז״ל בספרו ערוך לנר יבמות דף י׳ ע״ד ד״ה שם ברש״י הביא מעכ״ת נ״י ספק אם בת אשתו שנולדה מאיש אחר לאחר שגירשה אסורה לו משום בת אשתו כיון שכשהיתה אשתו לא היתה בתה ומשנעשה בתה לא נקראת עוד אשתו ונ״ל לפשוט ספק זה ממה דאמרינן במכות (דף ט״ז) דבאונס שגירש לא משכחת בטלו ע״ש ואי ס״ד דאין בזה משום בת אשתו שפיר משכחת בטלו כגון שגירש אנוסתו וזנתה לאחר גירושין וילדה בת וקדש הוא את הבת הרי ביטל שלא יכול להחזיר שוב אנוסתו משום חמותו אע״כ דנקראת בת אשתו גם הנולדת לאחר גירושין וא״כ אין קידושין תופסין לו בה. ומזה נ״ל להשיב ג״כ על מה שחקר בספרו הנ״ל ביבמות (דף מ״ד ע״ג) בעדים שהעידו על כהן אחד שאמר בפניהם שבן גרושה הוא ונגמר דינו להיות בן גרושה ואח״כ הוזמו אם נעשים בן גרושה תחתיו כיון דלא שייך בזה הטעם דאמרינן מכו׳ (דף ב׳) דאין אומרים יעשה בן גרושה משום לו ולא לזרעו שהרי בזה לא היו באים לפסול לזרעו דאין נאמן רק על עצמו ולא על זרעו כדאמרינן יבמות (דף מ״ז) ואי אתה נאמן לפסול בניך וגם לא שייך מה שכתבו התוספ׳ במכות לענין מצרי שני מלו ולא לאשתו דבזה גם אשתו לא באו לפסול כמש״כ התוספ׳ ביבמות (שם) דאם בא על בת כהן לא פסלה ע״ש ולפי דעתי יש לפשוט חקירה זאת ג״כ מסוגיא דמכות (דף ט״ו) הנ״ל דאם איתא דבזה נעשה בן גרושה משכחת בטלו באונס שגירש כגון לאחר קידושין העיד עם אחר על אדם אחד שהוא אמר על עצמו שהוא ממזר ונגמר דינו להיות ממזר ואח״כ הוזמו ונעשה הוא ממזר תחתיו והרי ביטל ע״י עדותו שלא יכול להחזיר אנוסתו כיון שנעשה ממזר אע״כ דלא אמרינן גם בכה״ג שיעשה ממזר תחתיו עכ״ד מעכ״ת נ״י.\\\vspace{0pt}

ועל זה אשיב: הנה ראי׳ שניי׳ על חקירתי לענין בן גרושה פשיטא שאינה ראי׳ שהרי בשעה שהעיד עדיין לא ביטל עד שהוזם ונגמר דינו להיות ממזר ואז הביטול ממילא קאתי אבל הראי׳ הראשונה לענין ספקי בבת אשתו שנולד׳ לאחר גירושין לכאור׳ ראי׳ שכלית היא ולא אאריך במה שהי׳ נלענ״ד להשיב עליו אבל יצאתי לדון בדבר חדש דרך כלל על הב׳ ראיות שבענינים הללו לא נקרא ביטל דאחר שעבר על הל״ת שניתק לע׳ והתורה נתנה לו תקנה שיקיים העשה לתקן הלאו אזי לא נענש על בטול העשה אלא אם ביטל העשה באותו דבר עצמו שעליו מצו׳ לקיימה בו אבל אם באותו דבר עצמו לא עשה שום דבר רק בדבר אחר עשה מעשה או ענין ועי״ז נתבטל שלא יכול לקיים העשה עוד זה לא נקרא בטלו. והנה כל עניני בטלו שהוזכרו במכות שם הם בהדברים עצמם שהתורה הטילה עליו לקיים העשה כגון בשלוח הקן שהתורה צותה שלח תשלח האם והוא הרגה או שבר כנפי׳ או באונס שהתורה צו׳ לו תהי׳ לאשה והוא הרגה או קדשה לאחר או הדירה על עצמו וכן במשכון או בגזילה ששרף ולא יכול להחזיר וכן בפיאה שהתורה צו׳ לעני תעזוב והוא אכלה או בנותר שהתורה צוה באש תשרפנו והוא השליכו לנהר דזה ג״כ מקרי בטלו לדעת הרמב״ן בספר הזכות ובכל הני הבטול הוא בהדבר שעליו נאמרה העשה אבל אם בטל העשה דלו תהי׳ לאשה ע״י שקדש את בתה או ע״י שעשה עצמו ממזר ע״י עדותו זה לא נקרא בטלו כיון שאין המעשה של הבטול באנוסה עצמה שעלי׳ נאמרה העשה רק בדבר אחר והבטול נעשה ממילא.\\\vspace{0pt}

ובזה מיושב ג״כ הקושיא ששמעתי דלמה לא משכחת בטלו באונס שגרש בשעשה עצמו כרות שפכה שאסור להחזירה או שסירס עצמו שלא יכול שוב להחזירה ועי״ז ביטל העשה. ואף שבספרי ערוך לנר למסכת מכות כתבתי יישוב לקושיא זו אמנם ע״פ כלל זה בלא״ה אתי שפיר טפי דזה לא מקרי בטלו כיון שלא נעשה מעשה הבטול בהאנוסה שעלי׳ נאמרה העשה.\\\vspace{0pt}

ולפ״ז יצא לנו בנטל האם מעל הבנים והמית הבנים או השליך הקן לנהר שאע״פ שעי״ז ביטל העשה דשלח תשלח שלא יכול לקיימה לעולם מכ״מ למ״ד בטלו אינו לוקה כיון שלא עשה מעשה הבטול בהאם שעלי׳ נאמרה שלח תשלח ועשה ממילא נתבטלה. והגם שלא מצאתי כלל זה מפורש לא בראשונים ולא באחרונים מכ״מ הוא ברור לפענ״ד.\\\vspace{0pt}

והנה בהספק שנסתפקתי אם הבת שנולדה מאשתו אחר שגרשה מקרי בת אשתו כבר כתבתי בספרי ע״ל ביבמות (דף ט׳ ע״ב) כנ״ל שמרש״י נראה שדעתו דאינה ערו׳ או עכ״פ שרש״י מסופק בזה. ולכאורה הי׳ אפשר לומר דתלי בשני גרסות ביבמות (דף נ״ה ע״ב) במה דאמרינן דסד״א הואיל לאחר מיתה נמי אקרי שארו וכו׳ שרש״י גרס כן ולפ״ז מסיק דלאחר מיתה נקרא עדיין שארו והיינו ע״כ מפני שהיתה שארו וא״כ ה״ה ג״כ דנקראה אשתו לאחר גירושין אבל מהתוספ׳ נראה שלא גרסו הואיל לאחר מיתה וכו׳ אלא לאחר מיתה וכו׳ ולכן כתבו דשמעינן מדרשה דממעט מתה דלא חייב עלי׳ על אשת איש דה״ה ג״כ דלא נקרא שארו לענין שאר ענינים וכמו שכתבתי בספרי ע״ל שם וכמו דלאחר מיתה לא נקראה שארו ואשתו י״ל דה״ה ג״כ לענין גירושין וא״כ לא נחשבה הבת שנולדה לאחר גירושין מאיש אחר בת אשתו להיות לו ערו׳ וכמו שכתבתי שהיא דעת רש״י ולפ״ז יהי׳ סתירה בדברי רש״י ולכן כתבתי לעיל דאפשר דרש״י מסופק בזה ועכ״פ יהי׳ לפי דברינו ספק קידושין במי שקדש בת אשתו שנולדה לאחר גירושין אבל י״ל דאע״ג דבמתניתן נקראה הערו׳ דאשה ובתה בת אשתו מכ״מ בלשון הכתוב נקראה ערות אשה ובתה לא תגלה וא״כ אין חילוק מתי נולדה הבת כיון דעכ״פ מגלה ערות אשה ובתה זו אחר זו שהרי גם כשאינה בת אשתו הרי היא בת אשה שגלה ערותה ולכן ודאי ערו׳ היא וקידושין לא תפסי בה ודברי רש״י ביבמות דף ט׳ צ״ע כנלענ״ד, הקטן יעקב.\\\vspace{0pt}

\end{multicols}\newpage

\newchap{סימן קלז}
\begin{multicols}{2}
ב״ה אלטאנא, יום ג׳ ז׳ טבת תרכ״ד לפ״ק. להרה״ג וכו׳ מ״ה אלכסנדר אראן אב״ד דק״ק פעגערסהיים נ״י.\\\vspace{0pt}

על דבר שאלתו דמר נ״י באשה שכל פעם שילדה קשתה לילד והיתה בסכנה גדולה ונמלטה בחסד הקב״ה אבל מעתה אמרו הרופאים להבעל שישמר לבל יקרב עוד אל אשתו כי אם תתעבר עוד ותלד אין מזור ורפואה ותמות בודאי ושאל האיש ברוח נשברה ובמר נפשו אם מותר לו לשמש עם אשתו במוך לבל תתעבר כי אף שקיים כבר מצות פרי׳ ורבי׳ בבן ובת עכ״ז יצרו תקפו לבלתי שבת גלמוד ויבא לידי איסור וגם כדי לקיים מצות עונה.\\\vspace{0pt}

תשובה – גרסינן יבמות (דף י״ב) ג׳ נשים משמשות במוך קטנה מעוברת וזקנה קטנה שמא תתעבר ושמא תמות מעוברת שמא תעשה עוברה סנדל מניקה שמא תגמול את בנה וימות וכו׳ דברי ר״מ וחכמים אומרים אחת זו ואחת זו משמשת כדרכה והולכת ומן השמים ירחמו משום שנאמר שומר פתאים ד׳ ופי׳ רש״י משמשות במוך מותרת לתת מוך במקום תשמיש כשהן משמשות כדי שלא יתעברו עכ״ל והוכיחו התוספ׳ מזה דשאר נשים אסורות משום השחתת זרע ור״ת אומר דלפני תשמיש ודאי אסור ליתן מוך דאין דרך תשמיש בכך והרי הוא כמטיל זרע על העצים ועל האבנים כשמטיל על המוך אבל אם נותנת אחר תשמיש אין נראה לאסור דהאי גברא כי אורחי׳ משמש והאשה שנותנה אח״כ מוך לא הוזהרה אהשחתת זרע כיון דלא מפקדה אפרי׳ ורבי׳ ומשמשות במוך דקתני הכא היינו צריכות לשמש במוך עכ״ל וכל זה לר״מ אבל לחכמים דאמרו דמשמשות כדרכה ומן השמים ירחמו בין לרש״י בין לר״ת אסורה ליתן מוך קודם תשמיש וכוותייהו פסקינן. אמנם אם מותר ליתן מוך אחר תשמיש לשאוב הזרע בזה יש פלוגתא בין הראשונים דלר״ת שרי כמו שכתבו התוספ׳ בשמו דאשה אינה מצו׳ על השחתת זרע אבל דעת רש״י נראה דגם זה אסור דאל״כ למה אמרו חכמים משמשת כדרכה והולכת נהי דס״ל דאסורה לשמש במוך הרי יש תקנה שתתן מוך אחר תשמיש ולא תצטרך לסמוך על רחמי שמים אלא ע״כ דלרש״י גם זה אסור וכן נראה מדברי הריטב״א הובאו בש״מ בכתובות (דף ל״ט) שכתב לשיטת רש״י דאע״ג דאתתא לא מפקדא אפרי׳ ורבי׳ אסורה להשחית זרע בעלה וכו׳ אבל להשחית זרע הראוי להוליד ולאבדו במוך אסור אפילו לדידה ואפילו הי׳ המוך מונח בשעת תשמיש עכ״ל ומדכתב ואפילו הי׳ המוך מונח בשעת תשמיש משמע דכל שכן אחר תשמיש דמאבדה הזרע בידים אכן מדעת הרא״ש שהביא בש״מ שם נראה שמסופק בזה לשיטת רש״י אם דעתו דאשה מצו׳ בהשחתת זרע דאז אסורה אפילו אחר תשמיש או אם דעתו לבד דאסורה להשחית זרע האיש דאז רק אסורה ליתן מוך קודם תשמיש אבל לשאוב הזרע לאחר תשמיש שרי.\\\vspace{0pt}

והנה כל הראשונים הרמב״ן והרא״ה ותלמיד הרשב״א והריטב״א נטו מפי׳ רש״י משום דקשיא להו דאי הפי׳ מותרת לשמש א״כ לחכמים דאמרו מן השמים ירחמו אסורה לשמש במוך משום השחתת זרע ואמאי כיון דמן השמים ירחמו שלא תתעבר א״כ ליכא השחתת זרע וכן קודם הזמן דאמרינן משמשת כדרכה ואמאי אסורה לשמש במוך כיון דלא תתעבר ולענ״ד יש ליישב זה דאטו השחתת זרע תלי בעיבור דא״כ עקר וסריס שאינו ראוי להוליד יהא מותר להשחית זרעו ועוד דזרע שאינו יורה כחץ דאמרינן בנדה שאינו מוליד יהי׳ מותר להשחיתו ועוד דלפי מה שכתבו התוספ׳ ביבמות שם ובסנהדרין (דף נ״ט) וכן בנדה (דף י״ג) תלי איסור השחתת זרע בציווי פרי׳ ורבי׳ וא״כ מי שקיים פרי׳ ורבי׳ כבר בבן ובת יהי׳ מותר להשחית זרע וח״ו לומר כן אלא ודאי איסור השחתת זרע הוא איסור לעצמו אפילו אם הזרע אין ראוי להוליד ואם אין צריך להוליד ורק שאינו במי שאינו מצו׳ על פו״ר והן אמת שלא מצאנו אזהרה מפורשת לאיסור הגדול דהשחתת זרע ובחידושי לנדה כתבתי שאולי האזהרה היא מלא תשחית כמו שיש בכלל אזהרה זו לחבול בעצמו או שהלכה למשה מסיני היא אבל עכ״פ לענ״ד לא תלי בזה שיהי׳ הזרע ראוי להתעבר ממנו ולא באשה זו דוקא ולכן ל״ק קושיא זו על רש״י אבל ביותר קשה מה שהקשו הראשונים עוד על רש״י דאיך אמרינן משום השמים ירחמו דלמה לא מותר לשמש במוך משום ספק סכנה ומה שכתב מר נ״י על זה כיון דרובא דרובא לא מתעברות לא חיישינן מלבד שאין זה רוב מבורר בלא״ה לא מהני דהרי כלל בידינו דאין הולכין בפקוח נפש אחר הרוב וכמובא בא״ח סי׳ שכ״ט ואפילו למיעוטא דמיעוטא חיישינן כדמוכח ממה דאמרינן שם נפל עליו גל ספק הוא שם וכו׳ אע״ג דאיכא כמה ספקות מכ״מ מפקחים ולכן קשה זה מאוד אמנם לא ידעתי למה הוקשה קושיא זו לפי׳ רש״י דוקא הלא גם לשיטת ר״ת קשה כן כיון דלר״מ מחוייבת לשמש במוך משום ספק סכנה איך התירו חכמים לשמש שלא במוך ולכנוס בספק סכנה ולא עוד אלא דלענ״ד לשיטת ר״ת קשה זה עוד יותר דזה ודאי שנתינת מוך אחר תשמיש מהני ג״כ שלא תתעבר לא בלבד לשיטת ר״ת דכל הפלוגתא היא בנתינה לאחר תשמיש אלא גם ע״כ רש״י מודה בזה כדמוכח ממה שכתב בכתובות (דף ל״ז) אמה דאמרינן אשה מזנה משמשת במוך כדי שלא תתעבר נותנת מוך לאחר בעילה ושואבת הזרע עכ״ל ומה שכתב רש״י דג׳ נשים משמשות במוך קודם תשמיש צ״ל כיון דמשום סכנה הוא חיישינן שמא תתעבר ע״י שיקלוט הזרע מיד קודם שתשאוב אותו אע״ג דלא שכיח הוא ולכן להך שיטה שכתב הרא״ש דגם לרש״י מותר ליתן מוך אחר תשמיש דאינה מצו׳ להשחית זרע האיש י״ל דכיון דיש לה תקנה ליתן המוך אחר תשמיש דמהני ג״כ ע״פ רובא דרובא לכן סמכו רבנן על מן השמים ירחמו כיון דאין בזה סכנה מבוררת עוד אבל לפי׳ ר״ת דכל הפלוגתא היא רק אלאחר תשמיש דלפני תשמיש גם לר״מ אסור וא״כ קאמרי רבנן דאפילו לאחר תשמיש ל״צ ליתן ובזה קשה ביותר איך תכנוס לספק פקוח נפש והנלענ״ד דטעמא דרבנן דאע״ג דכלל בידינו דאין לך דבר עומד בפני פקוח נפש ואין הולכין בפ״נ אחר הרוב זה דוקא ביש ודאי סכנת נפש לפנינו כגון בנפל עליו הגל דאז חוששין אפילו למיעוטא דמיעוטא אבל בשעתה אין כאן פקוח נפש רק שיש לחוש לסכנה הבאה בזה אזלינן בתר רובא כמו לענין איסורא דאל״כ איך מותר לירד לים ולצאת למדבר שהם מהדברים שצריכין להודות על שנצולו ואיך מותר לכתחלה לכנוס לסכנה ולעבור על ונשמרתם מאוד לנפשותיכם אע״כ כיון דבאותה שעה שהולך עדיין ליכא סכנה הולכין אחר הרוב ועוד ראי׳ לזה ממה דאמרינן ברכות (דף ל״ג) אפילו נחש כרוך על עקבו לא יפסיק אמר רב ששת ל״ש אלא נחש אבל עקרב פוסק ופי׳ הרמב״ם בפי׳ המשניות וכ״כ הברטנורא כיון דנחש אינו נושך ברוב הפעמים אבל עקרב שמנהגו לנשוך תמיד פוסק והלא יקשה מה בכך דהנחש אינו נושך ברוב הפעמים הרי אין הולכין בפ״נ אחר הרוב אע״כ דה״ט כיון דכן אין כאן סכנת נפשות וזה לענ״ד ג״כ טעמא דרבנן כיון דבעוד שלא נתעברה אין כאן סכנת נפשות לכן סמכינן אטעמא דמן השמים ירחמו שלא תבא לידי סכנה וכל שכן דא״ש למ״ד שמא תתעבר ושמא תמות דהוי ספק ספקא שמא לא תתעבר ואת״ל תתעבר שמא לא תמות ורק ר״מ דחייש למיעוטא גם בלא סכנה ס״ל דתשמש במוך דחייש למיעוטא שמא תבא לידי סכנה. והיוצא מזה בנדון השאלה דאפילו נאמין לדברי הרופאים (מה שיש לפקפק בלא״ה לדברי הטור י״ד סי׳ קפ״ז) עכ״פ לא נאמין להם שבודאי תמות אם תלד דמאין יכלו לשפוט כן בודאי אחרי שהאשה כבר ילדה איזה פעמים וחיתה וא״כ לא עדיף מהא דג׳ נשים דג״כ איכא ס״ס זה שמא תתעבר ושמא תמות ואעפ״כ ס״ל רבנן דקיי״ל כוותייהו דמן השמים ירחמו ולא תשמש במוך ולכן ליתן מוך לפני תשמיש בין לרש״י בין לר״ת אסור אכן גם ליתן לאחר תשמיש ולשאוב הזרע לא ראיתי היתר שאף שלר״ת הותר זה מכ״מ התוספ׳ והרמב״ן ריש פ׳ כל היד חולקין עליו וכפי הנראה גם הרשב״א והר״ן שם הסכימו עמהם ואפילו רש״י אף שהרא״ש מסופק בשיטתו בזה מכ״מ הריטב״א ביבמות כתב בפי׳ דגם אחר תשמיש אסור לרש״י וכנראה גם מדבריו בש״מ שהבאתי לעיל עד שכמעט דעת ר״ת היא דעת יחידית בזה ומי יקל ראשו להכריע כמוהו נגד כל שאר הראשונים.\\\vspace{0pt}

ומה שרצה מר נ״י להביא ראי׳ להתיר ממה דאמרינן כתובות (דף ע״ב) או שתהא ממלא ומערה לאשפה יוציא ויתן כתובה ומדברי הראשונים שם והאריך בדברים לענ״ד לא בלבד שאין ראי׳ משם להתיר אלא יש ראי׳ לאסור דמה דאמרינן שם ותיעבד אמר ריא״ש שתמלא ונופצת במתניתא תנא שתמלא עשרה כדי מים ותערה לאשפה כתוב בירושלמי בלשון אחר דאיתא שם תמן אמרין כגון מעשה ער ורבנן דהכא אמרין דברים של בטלה עכ״ל ופשוט שמה שאמר תמן אמרין כגון מעשה ער היינו מה שאמר רב יהודה אמר שמואל שתמלא ונופצת וקרא לי׳ תמן דהירושלמי נתחבר בא״י ור״י ושמואל היו בבבל ולכן הביא הריב״ש המובא בש״מ ראי׳ מהך דירושלמי לפרש מה דאמרינן בבבלי שתהא ממלא ונופצת רק שמפרש שמה שאמר כמעשה ער היינו שאיכא השחתת זרע ע״י שנופצת הזרע שקבלה בשעת תשמיש ולא כמעשה ער ממש שזה לא שייך בדידה וכתב הטעם שיוציא או משום דאמרה בעינא חוטרא לידה שאע״ג שאינה מצו׳ על פו״ר מכ״מ רוצה שיהי׳ לה בן או דאיכא איסור דאע״ג דלא מפקדה אפו״ר אסור לה להשחית זרע בעלה וב׳ הטעמים תלי בב׳ השיטות שהזכרנו דלשיטת ר״ת שמותר לה להשחית זרע לאחר תשמיש ע״כ הטעם משום דבעינן חוטרא לידה אבל לדעת שאר הראשונים דאיכא איסור בדבר להשחית זרע בעלה הטעם משום איסור ופשוט הוא דלהך שיטה אפילו הבעל מוחל אסור דהיאך יכול למחול לה להשחית זרעו מה שגם לו אסור והנה בשם רש״י מהדורא קמא הביא הש״מ הטעם דאמרה בעי חוטרא לידה ומזה משמע דדעת רש״י כשיטת ר״ת דלאחר תשמיש אין איסור השחתת זרע אצלה וכמו שרצה הרא״ש לומר לשיטתו אבל מכ״מ אין ראי׳ דאפשר דאע״ג דהשחתת זרע הבעל אסור מכ״מ זה דוקא כששואבת במוך שמאבדת בידים אבל כשתרוץ ברגלי׳ עד שתנפץ הזרע כמו שפי׳ רש״י על שתמלא ונופצת שהזרע מעצמו נופל ממנה אפשר שזה לכ״ע שרי ולכן כתב רש״י הטעם דבעי חוטרא אבל שתשחית בידים ע״י מוך אולי גם לרש״י אסור אכן אפילו נימא דכוונת רש״י במהדורא קמא כר״ת מכ״מ מדלא כתב במהדורא בתרא כן משמע שחזר מזה ועכ״פ הריב״ש כתב בפי׳ לחד טעמא דאסור אפילו שתהי׳ מתהפכת ותנפץ הזרע כש״כ להשחית בידים גם מלשון הירושלמי כמעשה ער משמע כן דהטעם משום איסור השחתת זרע וכל זה גם לאחר תשמיש דלשון ממלא ונופצת לא שייך רק לאחר תשמיש ויעיין מה שכתב המגן אברהם סי׳ תר״ו ס״ק ח׳ ומה שדייק מר נ״י שהטוש״ע כתבו בי״ד סי׳ רל״ה לאחר תשמיש ובאהע״ז סי׳ ע״ו כתב הטור שתתהפך בשעת תשמיש וחשב שיש סתירה בזה לענ״ד פשוט שמה שכתב הטור באהע״ז בשעת תשמיש היינו בעת שתעשה תשמיש אבל פשיטא דלא שייך מתהפכת אלא לאחר תשמיש וכנראה ג״כ מלשון ממלא ונופצת ונקט הטור בזה לשון הרי״ף שכתב א״ל בשעת תשמיש שתתהפך כדי שתנפץ ש״ז ולא תתעבר וראי׳ לזה שהרמ״א שהעתיק לשון הטור כתב שתתהפך לאחר התשמיש הרי דמפרש ג״כ דברי הטור הכי.\\\vspace{0pt}

עוד רצה מר נ״י להביא ראי׳ להתיר ממה דאמרינן מגילה (דף י״ג) שהיתה עומדת מחיקו של אחשורוש וטובלת ויושבת בחיקו של מרדכי וכתבו התוספ׳ וא״ת והא לא הי׳ שם הבחנה ג׳ חדשים שהרי בכל יום הי׳ אותו רשע מצוי אצלה וי״ל שהיתה משמשת במוך עכ״ל ופי׳ בטורי אבן דליכא למימר שעם אחשורוש היתה משמשת במוך שהרי איתא באגודה שדריוש בן אחשורוש שנבנה הבית בימיו הי׳ בנה אלא שעם מרדכי היתה משמשת במוך וא״כ מוכח דאין איסור בדבר ולענ״ד גם מזה אין ראי׳ כלל דאפילו נפרש כמו שכתב הט״א י״ל שהתוספ׳ כתבו כן בשיטת ר״ת ולאחר תשמיש שאבה במוך וכמו שפי׳ ר״ת הך דג׳ נשים משמשות במוך וא״ל דא״כ אכתי להנך שיטות דחולקים על ר״ת דגם זה אסור היאך יתורץ קושית התוספ׳ די״ל כיון דביבמות (דף ל״ז) איכא ב׳ טעמים להבחנה דג׳ חדשים דשמואל יליף מקרא דלהיות לך לאלקים ומשמע דמדאורייתא הוא ורבא אומר דהטעם משום גזירה ע״ש וא״כ י״ל דמ״ד שהיתה עומדת מחיקו של אחשורוש ויושבת בחיקו של מרדכי ס״ל כטעם דרבא ובימי מרדכי עוד לא נגזרה גזירה זו או י״ל דהנך שיטות ס״ל כהך דיעה שהביא הבית שמואל (סי׳ י״ג) דאפילו לשמואל הבחנה היא מדרבנן וא״כ לכ״ע ל״ק קושית התוספ׳ די״ל דאז עדיין לא גזרו הבחנה ולכן אפילו נפרש כפי׳ הט״א אין ראי׳ מזה להתיר אבל באמת פי׳ הט״א דחוק הוא בלשון התוספ׳ ולא ראה להמהרש״א שכבר דחה פי׳ זה וכתב שדוחק לפרש כששמשה עם מרדכי משמשת היתה במוך דלא התירו לשמש במוך אלא לג׳ נשים והקושיא שהביאה הט״א לפרש דעם מרדכי שמשה במוך שהרי הי׳ לה בן מאחשורוש כבר תירץ המהרש״א שנתעברה ממנו אחר שנאסרה למרדכי כשאמרה כאשר אבדתי וכו׳ כך אובד ממך ששוב לא היתה משמשה במוך ע״ש ולענ״ד י״ל עוד ביישוב קושיא זו דתי המהרש״א דחוק כמו שהשיב ביערות דבש דהרי בשנת י״ב הי׳ כשנכנסה ברצון ונולד דריוש בשנת י״ג וא״כ הי׳ בן שתים כשבנה ביהמ״ק לאחר מיתת אחשורוש וזה מהנמנע אבל הנלענ״ד בזה דודאי כשנלקחה אסתר אל בית המלך תוך שאר הבתולות לא הי׳ אפשר למרדכי לקרב אלי׳ שהיתה תחת ידי הגי שומר הנשים ובבקר היא שבה אל יד שעשגז שומר הפילגשים ואיך יקרב זר אל בית הנשים והכתוב אומר מרדכי מתהלך לפני חצר בית הנשים לדעת את שלו׳ אסתר הרי דאלי׳ לא בא ולכן כשנלקחה אסתר אל בית המלך בשנת שבע למלכותו לא היתה צריכה הבחנה ולא שמשה במוך ואפשר שמהקב״ה הי׳ שנתעברה מיד שיולד בן טהור מאמו שיבנה ביהמ״ק ונולד אז דריוש בשנת שבע לאחשורוש אכן כשנתן כתר מלכות בראשה וישבה בבית המלכות לבדה אז הי׳ באפשרי שיקרב מרדכי אלי׳ ולכן כתיב כאשר היתה באמנה אתו שממנו נדרש שהיתה יושבת בחיקו של מרדכי אחר שכבר מלכה ומאז והלאה שמשה במוך עם אחשורוש וזה כוונת התוספ׳ לפענ״ד אבל עם מרדכי לא שמשה במוך דלא הותר כמו שכתב המהרש״א מכל הנ״ל נראה שאין להתיר לשמש במוך. אכן מה שמצאנו באשת ר׳ חייא ביבמות (דף ס״ה) ששתתה כוס של עיקרין מפני צער לידה בזה אולי תמצא גם האשה הזאת תקנה לעצמה ואע״פ שיש מהפוסקים אחרונים שפקפקו גם בזה מפני חשש סירוס באשה מכ״מ אין זה רק חשש בעלמא וכה״ג במקום סכנה ודאי אין צריכה לחוש.\\\vspace{0pt}

שוב אחר כתבי הדברים האלה מצאתי בשו״ת חתם סופר חלק י״ד סי׳ קע״ב שנשאלה שאלה זו ממש לפניו והשיב שלהיות המוך בשעת תשמיש אין להתיר כלל אך אחר התשמיש אפשר שיש להקל ברצון בעלה גם הביא הראי׳ מתוספ׳ מגילה שאסתר שמשה במוך מפי׳ הט״א ודחה פירושו אכן הקושיא היאך ילדה לדריוש נדחק ביישובה. והנלפענ״ד כתבתי, הקטן יעקב:\\\vspace{0pt}

\end{multicols}\newpage

\newchap{סימן קלח}
\begin{multicols}{2}
ב״ה אלטאנא, בחדש אייר תרכ״ב לפ״ק.\\\vspace{0pt}

שאלה – מה שנקרא ייחוד עם העריות שאסור מן התורה כמבואר בטוש״ע אהע״ז (סי׳ כב) אם דוקא אסור להתייחד עם הערו׳ בחדר שנסגר הדלת או אסור גם כשהדלת מגופה ואינה סגורה.\\\vspace{0pt}

תשובה – כתב הרשב״א בשו״ת (סי׳ אלף רנ״א) והגפת דלתות שאמרת אין זה ייחוד עד שיהא בית נעול דתרעא טריק בירושלמי שער נעול במנעל משמע וכדמשמע התם בירושלמי בפ׳ המדיר דגרסינן התם תרעא טריק סוטה מוגף צריכה וטעמא כל שאינו במנעול ירא הוא שמא יכנוס אחר שלא ברשות וכעין האי עובדא דפרק בתרא דע״ז בי״נ עכ״ל ומזה נראה בפירוש שדעת הרשב״א דלא נקרא יחוד רק בדלת נעול אמנם שדא בי׳ נרגא הגאון בית מאיר הובא בשו״ת ר׳ עקיבא סי׳ ק׳ וז״ל אמת שם הוא לשון הרשב״א והגפת דלתות שאמרת אין זה יחוד עד שיהא הבית נעול במנעול אבל באמת זהו ליתא כדמוכח מהש״ס דדייקא פ״פ לרה״ר הוא דאין בה משום יחוד ואיכא למידק מיני׳ הא פתח סתום אפילו אינו נעול והוא פונה לרה״ר או פתוח לחצר והוא עמה ביחוד בהבית הוי יחוד וכן משמע מדאמר רב כהנא שם בקידושין (דף פ״א) אנשים מבפנים ונשים מבחוץ חיישינן משום יחוד ופי׳ רש״י דשמא יצא אחד מהן ויתיחד עם הנשים הרי אף אם נימא דבחשש שיצא לחדר האנשים שבתוכו יש חשש יחוד וכו׳ א״ו אפילו בלא סגירת מנעול הוי יחוד וה״נ מוכח מהרא״ש (פ׳ כיצד) שכ׳ ונראה דבסתירה לחוד וכו׳ ובהרשב״א טעות סופר ועיקר כגירסת הב״י וכן העתיקו הפירוש שעל הירושלמי פ׳ המדיר דהיינו והגפות הדלתות שאסרת אין זה יסוד ע״ש עכ״ל הבית מאיר והסכים עמו הגאון ר׳ עקיבא אכן בשו״ת מהר״מ מטראני (ח״א סי׳ רפ״ז) בשו״ת הרדב״ז (חלק א׳ סי׳ קכ״א) העתיק דברי הרשב״א כמו שהוא לפנינו אין זה יחוד ולענ״ד הלשון שאסרת אין זה יסוד כמו שהגיהו הבית מאיר קשה להבין דהכי הוי לי׳ למימר הגפות הדלתות אין זה יסוד לאסור ובאשר שקשה מאוד לפ״ד הגאונים ז״ל להנצל מאיסור יחוד לפעמים שלא בדעת כשתכנוס המשרתת בחדר כשהבעל בבית לבדו שם לתקן צרכי הבית ותגיף הדלת כדרך בפרט בימי החורף ואם יחשב זה יחוד כמעט שאין להנצל מזה לכן ראיתי לבאר הענין.\\\vspace{0pt}

והנה בתחלה אוסיף ראיות מהגמרא שגם בלא סגירת דלתות מקרי יחוד ותמהני על הגדולים שלא הביאו ראיות הללו דבמגילה (דף י״ד) אמרינן דבורה דכתיב ודבורה אשה נביאה אשת לפידות מאי אשת לפידות שהיתה עושה פתילות למקדש והיא יושבת תחת תומר מאי שנא תחת תומר אמר ר״ש בן אבשלום משום יחוד ופירש רש״י שהוא גבוה ואין לו צל ואין אדם יכול להתיחד שם עמה כמו בבית עכ״ל ודרך דרוש אמרתי למה הקדים שהיתה אשת לפידות דלכאורה קשה מה חשש יחוד היה דהרי אמרינן בקידושין (דף פ״א) בעלה בעיר אין שם יחוד דאשה בעלה משמרה אבל י״ל דאיתא בתנא דבי אליהו (הובא בילקוט) אשת לפידות שהיתה עושה לפידות והיתה שולחת את בעלה להביאם לביהמ״ק בשילה עכ״ל פירוש שעל ידי זה נקראת אשת לפידות שבעלה נקרא לפידות על שם שילוחו ולכן כיון דהרבה פעמים לא הי׳ הבעל בעיר שישבה שם הוצרכה לישב תחת תומר משום יחוד וזהו מה דאמר הפסוק ופירש הגמרא מפני שהיתה אשת לפידות שהלך לפעמים לשילה היא ישבה תחת תומר לדון מפני היחוד אכן אי אמרינן דרק בדלת נעול מקרי יחוד למה לה לישב תחת תומר הרי תחת שאר אילן שענפיו מרובים ויש לו צל ג״כ אין יחוד ועוד דאפילו בבית היתה יכולה לדון בלא סגירת דלת אלא ע״כ דכל שנתכסה מן העין אפילו אינו סגור עד שאי אפשר לשום אדם לבא לשם ג״כ מקרי יחוד ועוד ראי׳ ממה דאמרינן סוכה (דף כ״ה) וליעבדו חופה בסוכה אביי אמר משום יחוד ורבא אמר משום צער חתן ע״ש הרי דשייך יחוד גם בלא סגירת דלת דודאי אין חשש שיתיחד אדם עם הכלה ויסגור הדלת בעדו ועוד דא״כ מה חילוק בין חופה בסוכה ובין חופה בבית אבל באמת מזה מצאנו פשר דבר שרש״י כתב משום יחוד דסתם סוכה היו עושין בגגותיהן ואין דרך ביאה ויציאה שם לרבים תמיד מפני הטורח ושמא ירד החתן לעשות צרכיו ויתיחד אחר עם הכלה עכ״ל ויש להקשות מה בעי רש״י בזה שהיו עושין סוכה בגגותיהן במקום שאין דרך ביאה ויציאה שם לרבים אלא ודאי כוונתו שרצה לבאר בזה למה דוקא בחופה בסוכה יש חשש ייחוד דאין שייך יחוד במקום שאפשר לבוא שם אדם בכל רגע כיון דיחוד הוא משום חשש זנות כל שמתיראים שיבא אדם ויראה אין חוששין לזה וזה מה שכתב הרשב״א כל שאינו במנעול ירא הוא שמא יכנוס אחר שלא ברשות ולכן אם הוא מקום שאין דרך לבוא אחר לשם נחשב כבית נעול ולזה כתב רש״י שבסוכה כיון שהיא בגג ואין דרך רבים לבוא לשם נחשב ג״כ יחוד כמו בבית סגור מה שאין כן אם החופה בבית וכן בדבורה כיון דלפעמים בשעת הדין מוציאין כל אדם לחוץ ומשיירין רק אחד כדאיתא בסנהדרין כשבודקין את העדים יהיה זה יחוד כיון שאין אדם נכנס לשם עד שירצה הדיין אבל בכל מקום שאפשר בכל עת לבוא אדם לשם בלא נטילת רשות לא מקרי יחוד בלא דלת סגור ולכן אין ראי׳ מהא דאמרינן בקידושין (דף פ״א) אנשים מבפנים ונשים מבחוץ חיישינן משום יחוד דכיון דאנשים לבד ונשים לבד אין דרך לאנשים לבוא לחדר הנשים לכן כשיצא אחד מהן ויתיחד הוי יחוד גמור דאינו מתירא שיבא גם אחר לשם אבל בבית ובחדר שיוצאים ונכנסים תמיד כדרכם לא שייך יחוד בהגפת הדלת בלא סגירה ומזה יש תשובה ג״כ על הראי׳ שהביא הבית מאיר מהרא״ש ע״ש וגם ממה שהוכיח מדאמרינן בית הפתוח לר״ה אין יחוד דמשמע הא פתוח לחצר יש יחוד לפ״ז אין ראי׳ רק לדברינו שהכל תלוי אם מתיראים בכל עת שיבא אדם לשם דבפתח הפתוח לרה״ר לעולם מתיראים אבל בבית הפתוח לחצר פעמים מתיראים ופעמים לא כפי ענין הדיורים שם ולכן כללא כייל לנקוט פתח פתוח לרה״ר דבזה לעולם אין ייחוד וכמו כן ודאי שייך ייחוד אפילו בלא סגירת דלת בחדר בעלי׳ דומיא דסוכה כמו שכתב רש״י במקום שאין דיורים בבית וכדומה אבל בחדר שבכל רגע יכול לבא אדם לשם אין ייחוד בלא דלת סגור וא״צ להגי׳ בדברי הרשב״א שהרי החצר מטרה הי׳ פתוח ליוצאים ובאים אע״פ שהגיפו הדלתות כמש״כ הרשב״א שם שאינה נחבשת ביד הכותב אדרבא מתירא מן המלך שמא תזעק ויתפש עלי׳ עכ״ל הרי שהי׳ אפשר לאחרים לבא לשם אם תזעק לתפשו לכן כתב שאין יחוד בלא סגירת הדלת שמתירא בלא סגירה שמא יכנוס שם אדם אפילו שלא ברשות ואפילו לא תזעוק ועל כן כל כה״ג שמתיראים המתיחדים שיבא אדם לשם בלא נטילת רשות לא נקרא יחוד בלא סגירת הדלת, כנלענ״ד, הקטן יעקב.\\\vspace{0pt}

\end{multicols}\newpage

\newchap{סימן קלט}
\begin{multicols}{2}
ב״ה אלטאנא, יום ג׳ ז׳ טבת תרי״ט לפ״ק. לחתני הרה״ג וכו׳ מ״ה משלם זלמן הכהן נ״י אב״ד דק״ק מאסטריכט יע״א.\\\vspace{0pt}

אשר שאלת אודות המנהג שבמדינה שם לפרוס טלית על החתן והכלה בשעת ברכות אירוסין ונשואין להיות במקום החופה אם יש לנטלו מפני שלפעמים הכלה אינה טהורה.\\\vspace{0pt}

אשיב לך שמנהג זה ישן נושן הוא בכל מדינות אשכנז הנוהגין אחר מהרי״ל והוזכר במהרי״ל ה׳ נשואין וז״ל מנהג שנוטלין את הציפל מן המטרון של חתן ומניחין על ראש הכלה להיות להן לחופה אכן מהרי״ל בהילולת בתו נטל שפת צעיפין שקורין (ענד) והניח עליהן לחופה ואמר דזכור לו מקדם המנהג כך על שם ותקח הצעיף ותתכס ונשכח ולקחו הציפל ואם הכלה אינה טהורה לבעלה מסירין מעל ראשה מיד אחר ברכת אירוסין עכ״ל וכן ראיתי מנעורי בארץ מולדתי (כי פה שהוא מנהג פולין לא נוהגין כן) וכבר ידוע להשושבנות שאם הכלה אינה טהורה שמסירין מיד אחר שנתקדשה וממה שכ׳ מהרי״ל מיד אחר ברכת אירוסין משמע דאפילו בשעת קידושין לא יפרסו עוד כשאינה טהורה אבל לא ראיתי נוהגין כן אלא שאחר הקידושין מסירין ובאמת לא ידעתי מה תועיל הכיסוי קודם הקידושין כיון שהוא לחופה וחופה קודם קידושין ליכא ואולי גם כוונת מהרי״ל כן ומה שכ׳ אחר ברכת אירוסין אתי לאפוקי לבד שלא יהי׳ עלי׳ ג״כ בברכת נשואין כמו בטהורה. ואיתא במהרי״ל עוד וז״ל והי׳ תוחב לה הטבעת באצבע שאצל האגודל ואם אינה טהורה אינו מגיע בה אך יניח ליפול מעצמו לאצבעה עכ״ל וגם בזה לא ראיתי נוהגין כן ולענ״ד יש קצת ראי׳ נגד מנהג זה ממה שכ׳ רמ״א אהע״ז (סי׳ ס״א) על מה שכ׳ הש״ע כשר הדבר שלא תנשא עד שתטהר וז״ל ועכשיו המנהג שלא לדקדק ואין ממתינין ומכ״מ טוב להודיע לחתן תחלה שהיא נדה עכ״ל ולמה כתב טוב להודיע הא צריך להודיע לו למען לא יושיט הטבעת באצבעה ומזה משמע דא״צ להקפיד כיון דבשעה שמושיט עדיין אינה מקודשת ואינה אשתו וכשהושיט וקדשה מעצמו פורש ממנה ועדיין צ״ע כי לא ראינו אינה ראי׳ ולאנשי מעשה יש להקפיד לעשות כן. ומה שהורת שלא לרקד ולזמר כשהחתן אבל על או״א יפה הורת והמערער על פסקך ידו על התחתונה ואם תוכל לבטל הרקוד בכל החתונות ישר כחך וחילך כי אין זה שמחה של מצו׳ כשמרקדים בחורים ובתולות יחד ועוברין על ביזרא דעריות והשטן מרקד ביניהם ולשמחה מה זו עושה ונזכה לראות בנחמת ציון ואז ימלא שחוק פינו ולשוננו רנה. הקטן יעקב.\\\vspace{0pt}

\end{multicols}\newpage

\newchap{סימן קמ}
\begin{multicols}{2}
ב״ה אלטאנא, יום ב׳ א׳ דר״ח טבת תרכ״ו לפ״ק. לחתני הרה״ג וכו׳ מ״ה יוסף איזאקזאהן נ״י אב״ד דק״ק ראטטערדאם יע״א.\\\vspace{0pt}

על דבר השאלה הבאה לידך שהמנהג שם שהחתן קודם החופה מוסר טבעת קידושין להנאמן והוא מוסרו לרב המסדר קידושין להראותו לפני הקידושין אל העדים שיש בו שו׳ פרוטה כמנהג אשכנז ע״פ מהרי״ל ואירע מקרה ששני חתנים עשו חופה זה אחר זה ושניהם מסרו הטבעות שלהם להנאמן והרב מסר לכל א׳ הטבעת שלו קודם הקידושין כמנהג ויהי אחרי כלות החופות אמר האחד להרב ראיתי עתה שהנאמן החליף הטבעות ונתן לי הטבעת של החתן האחר והאחר ג״כ הכיר טעות זה וכששמעו שיהי׳ בזה חשש בהקידושין באשר שכל אחד מהם קידש בטבעת שאינו שלו אמרו הרי אנו רואין הדבר כאלו החלפנו הטבעות קודם הקידושין ושלי שלו ושלו שלי באשר שאין חילוף בשוין רק משונים מעט להכירן, מה הדין של קידושין הללו.\\\vspace{0pt}

תשובה – שני הקידושין ודאי אינם קידושין כלל באשר שכל אחד קידש בטבעת שאינו שלו ומה שאמרו אנו רואין כאלו החלפנו למפרע זה אינו מועיל דלא אמרינן הוברר הדבר למפרע שכל אחד קידש בשלו דקיימא לן בדאורייתא אין ברירה כדמסקינן ביצה (דף ל״ח) ועוד דהכא אפילו למ״ד יש ברירה אין מועיל להחזיק החלוף למפרע דברירה שייך בדבר שעד עתה לא הי׳ מבורר אלא סתם ולבסוף נתברר בזה שייך לומר יש ברירה למפרע שנברר שהסתם הי׳ כאשר נתפרש עתה וכן כתבו התוספ׳ תמורה (דף ל׳) וז״ל ואומר מורי הרמ״ר כלל גדול בדין זה דודאי כל דבר שהוברר האיסור מתחלה ואח״כ נתערב בהיתר לא סמכינן אברירה כיון שתערובתו הי׳ באיסור אבל הני תערובתו בהיתר כי האיסור לא הי׳ מבורר קודם תערובתו ולאחר תערובת נולד האיסור אז סמכינן אברירה עכ״ל ולכן הכא שהי׳ מבורר לכל אחד טבעת שלו איך שייך שיחול החלוף למפרע ובמה יצא טבעת שלו מרשותו ונכנס לרשות חבירו ועל כן הקידושין לא חלו וצריך כל אחד לקדש מחדש לפני ב׳ עדים אבל א״צ לברך עוד הפעם ברכות נשואין כמבואר באהע״ז (סי׳ ס״א ס״ב) ברמ״א אם קידש בטעות וכו׳ א״צ לברך שנית ז׳ ברכות וכן ברכת אירוסין א״צ לברך שנית קודם הקידושין שאע״פ שהרמ״א לא הזכיר רק שא״צ לברך שנית ז׳ ברכות שהם ברכות נישואין מכ״מ גם ברכת אירוסין ל״צ כמבואר בהגהת מרדכי סוף קידושין שכתב כן בשם רש״י שהורה שא״צ שנית ברכת אירוסין ונישואין ע״ש וכן הובא בכנסת הגדולה שם (סי׳ ל״ד) וע״כ א״צ ג״כ עשרה כשיקדשו שנית דעשרה אינם רק משום ברכות אירוסין ונישואין. כנלענ״ד. הקטן יעקב.\\\vspace{0pt}

\end{multicols}\newpage

\newchap{סימן קמא}
\begin{multicols}{2}
ב״ה אלטאנא, יום ו׳ כ״ט כסליו תרי״ד לפ״ק. להרב וכו׳ מ״ה פנחס שיפפער נ״י בק״ק לעמבערג יע״א.\\\vspace{0pt}

מעכ״ת נ״י תימה על מה שכתב הרמ״א באהע״ז (סי׳ ס״ד ס״ג) ונהגו שלא לישא נשים אלא בתחלת החדש בעוד שהלבנה במילואה שדבר הזה אין לו ביאור דהלא גם בתחלת החדש אין הלבנה במלואה ובאמת המעיין במקור הדברים יראה שיש שינוי לשון ממה שהביא רמ״א כאן דז״ל הנ״י ספ״ד מיתות ואע״ג גבי ישבי בנוב כו׳ שרי התם דלאו נחש הוא דלא סמכינן עלי׳ לגמרי אלא סימנא בעלמא הוא דנקיט לי׳ וכה״ג שרי כמו שנהגו לישא נשים במלואה של לבנה לסימן טוב עכ״ל הנ״י והנה כל דברי הנ״י שם מבוארים בתשובה המיוחסת להרמב״ן סי׳ רפ״ג וז״ל תשובה לרמב״ן על ענין המנחשים כו׳ מה שנוהגים שאין נושאין נשים עד מלוי הלבנה באלו הארצות אינו ניחוש אלא כשם שמושחין המלכים על המעין דתמשך מלכותו כן עושין במילוי ולא בחסרון וסימנא טבא הוא כדרך שמושכין יין בצינורות לפני חתנים ואין בו משום דרכי האמורי עכ״ל הרמב״ן והרואה דברים אלו יראה דאין לנטות ימין ושמאל מפשטות הדברים דהנוהגים לדקדק בענין הנישואין אין עושין הנישואין רק בחצי כ״ט י״ב תשצ״ג שאז הלבנה במלואה לא קודם ולא אח״כ שאז הלבנה הוא בחסרון דכן מבואר בלשון הרמב״ן במ״ש כן עושין במילוי ולא בחסרון ע״כ וזה דוחק גדול לומר דקודם חצי חודש גם המקפידים מתירים משום שהלבנה עתידה להתמלאות משא״כ אחר חצי חודש היא הולכת לחסרונה דמלבד שזה הוא דוחק גדול מבואר מלשון הרמב״ן ז״ל במ״ש וז״ל מה שנוהגים שאין נושאין נשים עד מילוי הלבנה כו׳ עכ״ל הפוך זה דהמדקדקים אין נושאין גם קודם חצי חודש לכן לא ידעתי אנה מצא הרמ״א ז״ל דברים אלו שכתב ונהגו כו׳ אלא בתחילת החודש כו׳ דהלא במקור הדברים שהבאתי מבואר דהנוהגים לדקדק אין נושאים גם בתחלת החודש דגם אז הלבנה בחסרון ולא במילוי וכיון שכן דחזינן בזה״ז דעושין נישואין בתחילת החודש דאז הלבנה בחסרון ואין מקפידין שוב אין להקפיד ג״כ מלעשות הנישואין אחר חצי חודש איזו ימים וכן נראה לי להביא ראי׳ מש״ס ערוכה מ״ק (דף ח׳) מתניתין אין נושאין נשים במועד ור״ל בחוה״מ ואמרינן עלה בגמרא כמה טעמים ואם איתא דאין לעשות נשואין אחר חצי כ״ט י״ב תשצ״ג משום דאז הלבנה מתחלת לחסור הלא כל חו״ה הוא עכ״פ ט״ז ימים לחודש ואם שבת סמוך ליו״ט לאחרי׳ אז חו״ה י״ז ימים לחודש וא״כ תיפוק לי׳ אפי׳ כל השנה כולה אין לעשות נשואין באותו זמן משום דהוי אחר חצי כ״ט י״ב תשצ״ג דאז הלבנה בחסרון א״ו דהיתר גמור הוא ואין להקפיד במה דמבואר בגמרא שלא הי׳ מקפידין בזה עכ״ד מר נ״י.\\\vspace{0pt}

על זה אשיב: מר נ״י מפרש מה שכתב הרמב״ן שנוהגים שאין נושאין עד מילוי הלבנה שהוא בחצי כ״ט י״ב תשצ״ג לא קודם זה ולא אחרי כן לענ״ד כמה מן הדוחק בזה לומר שכוונת הרמב״ן שמדקדקים את הרגע של מילוי הלבנה ועוד הרי המילוי אינו רק רגע אחת שמיד שנתמלא מתחיל להחסר ואיך יעשו נישואין במילוי ולכן ודאי כוונת הרמב״ן כמו שפי׳ הרמ״א שאין נושאין בחצי האחרון של החדש שאז הלבנה הולכת וחוסרת אלא ממתינין עד תחלת החדש עת מילוי הלבנה דהיינו כשהלבנה מתמלאת והולכת וזה נקרא מילוי הלבנה כאשר אבאר דהנה מקור המנהג שחושבין מילוי הלבנה לסימן טוב לא נתבאר אבל לענ״ד יצא ממה דאמרינן בסנהדרין (דף מ״ב) וליבריך הטוב והמטיב ופי׳ רש״י נהי נמי דמשבעה ואילך ליכא למימר מחדש חדשים ניבריך מיהו הטוב והמטיב שכל שעה הוא מטיב לנו במילואתה תמיד שמוספת להאיר לעולם עכ״ל (הרי בפי׳ שמילואתה של לבנה לא נקרא בלבד הרגע כשנתמלאה לגמרי אלא כשמוספת להאיר לעולם שהוא עד חצי החדש וכן ג״כ כוונת הרמב״ן) ומתרץ אטו כי חסר מי מברכינן דיין האמת ופי׳ רש״י אטו כי חסר מי חשבינן לה רעה לגבן לברוכי דיין האמת דנחשוב את מליאותה טיבותא לברוכי הטוב והמטיב עכ״ל וליברכינהו לתרווייהו כיון דהיינו אורחי׳ לא מברכינן ע״ש והנה אף דמסקינן דלא מברכינן הברכות על המילוי ועל החסרון כיון דמנהג העולם הוא עכ״ז נראה מזה שחצי ראשון של החדש שהלבנה מוספת אורה הוא סימן עת אורה ושמחה וטובה והחצי האחרון בהיפך רק דלא נחשב טובה ורעה ממש לברך עליהם הברכות.\\\vspace{0pt}

גם יש סמך לזה ממה דאיתא במדרש רבה פ׳ בא ד״א החדש הזה לכם הלבנה בראשון של ניסן מתחלת להאיר וכל שהיא הולכת מאירה עד ט״ו ימים וכו׳ ומט״ו עד ל׳ אור שלה חסר כך ישראל ט״ו דור מן אברהם ועד שלמה וכו׳ כיון שבא שלמה נתמלא דיסקוס של לבנה וכו׳ ומשם התחילו המלכים פוחתין והולכין וכו׳ ע״ש הרי ג״כ ששעת מילוי הלבנה הוא סימן לעליי׳ וחסרונה סימן ליריד׳ וכן מפורש בתקוני זוהר (תקון ס״ט) ע״ש ועפ״ז נתייסד המנהג לענ״ד במקום שנהגו שלא לישא רק עד ט״ו לחדש כדברי הרמ״א. והנה כאשר ראיתי במקומות שנוהגין מנהג אשכנז מקפידין שלא לישא רק עד מילוי הלבנה כמנהג שהביא הרמ״א ובמקומות שנוהגין מנהג פולין אין מקפידין ובמקומות אילו ודאי יפה דן מעכ״ת נ״י שאין להורות איסור בדבר אבל במקומות שנהגו לנהוג כמנהג שהביא הרמב״ן והרמ״א לענ״ד אין להורות לשנות מנהג שהרי אפילו לענין מנהג של רשות אמרינן בפ׳ הפועלים אל ישנה אדם מן המנהג כש״כ במנהג כזה שיסודתו בדברי רז״ל כאשר כתבנו ומאין נושאין נשים במועד ודאי אין ראי׳ שרצה התנא להשמיענו שאסור בחה״מ מן התורה ולא במה שנתלה במנהג שאפשר שבימי התנאים עוד לא נהגו. כנלענ״ד, הקטן יעקב\\\vspace{0pt}

\end{multicols}\newpage

\newchap{סימן קמב}
\begin{multicols}{2}
ב״ה אלטאנא, יום ב׳ ז״ך מנחם תרכ״ד לפ״ק. להרה״ג וכו׳ מ״ה שמואל נ״י ראש ב״ד בק״ק מעזריטש יע״א.\\\vspace{0pt}

מה ששאל מעכ״ת נ״י ממני לחוות דעתי אודות מעשה שאירע שזה כחמשה שנים עבר דרך קהלתו איש אחד מבישגראייע שולח ביד פשעו מאת הממשלה לארץ גזירה ובא לב״ד להגיד כי רצונו לשלוח גט לאשתו וסדרו לו ב״ד את הגט ומסר הוא בעצמו את הגט לשליח להולכה אשר עשה ובבא השליח עם הגט למקום האשה ראו כי נכתב שם אבי האשה בשינוי גמור ממש לשם אחר והשינוי הי׳ ע״פ הבעל כי הוא צוה לכתוב כן ובא השליח לחזרה עם הגט ולא ניתן הגט ליד האשה ומורה אחד רוצה להתיר ליתן הגט הזה עם עוד גט אחר שנכתב בו שם האב כראוי ליד האשה אם יש לסמוך עליו.\\\vspace{0pt}

תשובה – הגם שמקום עיגון הוא אין אני רואה שום היתר בזה שמה שכתב שהמורה רצה לסמוך על תשובת עבודת הגרשוני (סי׳ נה) שהכשיר גט שבא ממרחקים שנשתנה שם אבי האשה במקום עיגון אין אני רואה שום סמך בזה שכבר חלקו על היתר זה בשו״ת כנסת יחזקאל (סי׳ ע׳) ובשו״ת צמח צדק (סי׳ פ״ג) שכתבו ח״ו לסמוך על גט זה והצ״צ כתב לשרוף הגט שלא יבא אחד ויכשל בו להתיר ליתן אותו גם בבית מאיר סי׳ קי״ז חולק על עבודת הגרשוני גם בשו״ת נו״ב סי׳ פ״ט כתב דגט שנכתב בת פלוני הכהן והוא אינו כהן שהוא פסול וכן כתב בשו״ת ר״ע איגר ז״ל דגט שנכתב בו בת פלוני הלוי ולא הי׳ לוי שהוא פסול ואפילו במקום עיגון לא רצו כל הפוסקים הללו לסמוך על שו״ת עבודת הגרשוני ומי יערב לבו לסמוך על דעת יחידית נגד כל הגדולים החולקים עליו ובפרט אחר שכתב הכנסת יחזקאל שבעל עבודת הגרשוני עצמו לא כתב להתיר רק לצדד וסיים שלא יועיל עד שיסכים חותנו בעל צ״צ והרב צ״צ לא הסכים עמו אלא גזר לשרוף הגט ולא עוד אלא שבנדון זה גרע מנדון של עבה״ג ששם נכתב השם בטעות מהסופר אבל הכא שהבעל קרא שם אבי האשה בשקר יש חשש כמש״כ הצ״צ וז״ל יש לחוש דלא נתכוון לגרש אשתו כלל ולכך שינה את שם אבי׳ והרי אין כאן גירושין כלל לאשתו פלונית בת נתן שהיא אשתו באמת לכך ודאי דבכה״ג ליכא מאן דפליג דאין כאן גט כלל לאשה הזאת עכ״ל ומטעם זה אין מקום ג״כ להיתר אחר שהזכיר מעכ״ת נ״י בשם המורה ע״פ מה שכתב בשו״ת פני יהושע חלק שני בסי׳ פ״ב במעשה שנכתב בגט שם אבי האשה אלעזר ושמו באמת אליעזר והתיר ליתן להאשה גט זה עם עוד גט אחר שנכתב שם אבי האשה אליעזר וטעמו כיון דמנהגנו האידנא ע״פ תקנות ר״י בעל התוספ׳ לצוות להסופר לכתוב כ״כ גיטין אפילו עד מאה עד שיצא אחד שיוכשר בעיני הרב ולכל מי שיראה לו כמבואר אהע״ז (סי׳ קכ״ב) וא״כ בכה״ג הרמב״ם נמי מודה שיכולין לכתוב גט אחר עכ״ד וע״פ הנ״ל אין זה דמיון לנדון דכאן ששם נכתב בטעות מהסופר אבל הכא שהוא צו׳ לכתוב בשם שקר הרי יש לחוש דבכוונה צוה כן ולא הי׳ רוצה לגרש אשתו כלל וא״כ גם לסופר לא צו׳ לגרש אשתו באמת.\\\vspace{0pt}

אכן מלבד זה אין לסמוך על היתר זה כלל דלפי מה שהעתיק מעכ״ת נ״י דברי הפ״י (כי השו״ת חלק שני עוד לא ראיתי׳) נראה דעיקר סמיכתו הי׳ דשם זה אם אליעזר או אלעזר לא נתברר איזה שם הוא העיקר ועכ״פ אין חשש לעז בזה דהעולם אינם מבחינים בשינוי זה מה שלא שייך בנדון השאלה שהי׳ השינוי שינוי הניכר לכל וגם הרי כתב הפ״י שלא רצה לסמוך עד שיסכימו גדולי המורים ומי יודע אם הסכימו. ואמנם הגם שיודע אני בעצמי שאפילו מקטני המורים אינני לא אחריש מלהזכיר שפסק הפ״י לא זכיתי להבין שאע״פ שבשעת הדחק יכול לצוות לסופר לכתוב גט ולעדים לחתום מבלי שימסור בעצמו ליד השליח להולכה שמינה הרי עכ״פ פורט שם האשה עם שם אבי׳ ובזה ממנה הסופר לכתוב אפילו הרבה גטין עד שיהי׳ אחד מהם כשר אבל הרי לא הרשה להסופר לכתוב גט בשם אחר שלא הזכיר וא״כ אם מינה לסופר לכתוב מאה גטין בשם רחל בת ראובן ושמה רחל בת שמעון הרי לזה לא מינה אותו לשליח ומה יועיל אם יכתוב הסופר עתה גט בשם שמעון כיון שלא נתמנה מהבעל לזה ואם הגט הראשון שהבעל עצמו מסר ליד שליח אינו כשר כש״כ דגט זה לא יכשר דלזה לא נתמנה מהבעל לשליח לא הוא ולא הסופר ולא העדים כמבואר ברמ״א סי׳ קכ״ב בשם הרמב״ן והרשב״א דאם מסר הבעל בעצמו לשליח לא יכולים לכתוב גט אחר בנמצא פסול. אכן בנדון השאלה כל הדברים האלה בלא״ה הם למותר שכפי מה שכתב מעכ״ת נ״י בקהלתו אין המנהג לצוות להסופר לכתוב כ״כ גטין עד שיהי׳ כשר בעיני הרב וגם בהגט של נדון השאלה לא הגיד הבעל להסופר כן א״כ הרי נפל היתר זה לבירא ותמהתי איך יסמוך המורה על זה ואולי לא נודע לו זה. עכ״פ מכל הלין נלענ״ד שגט זה פסול הוא ואפילו ניתן כבר עם גט אחר ליד האשה אין לה היתר להנשא עד שתקבל גט כשר מבעלה. כנלענ״ד, הקטן יעקב.\\\vspace{0pt}

\end{multicols}\newpage

\newchap{סימן קמג}
\begin{multicols}{2}
ב״ה אלטאנא, יום ג׳ ט״ז שבט תרכ״ז לפ״ק. לחתני הרה״ג וכו׳ מ״ה ישראל מאיר פריימאן נ״י אב״ד דק״ק פילעהנע יע״א.\\\vspace{0pt}

כתבת אלי וז״ל – ראובן נתן גט פיטורין לפני רב אחד לאשתו ואמר לה בשעת נתינת הגט ע״מ שלא תנשאי לפלוני והלכה האשה ונשאת לאחר והיו לה בנים ואח״כ נתגרשה גם מזה וקבלה קידושין לפני עדים כשרים מאיש שנאסר עלי׳ והראשון חי עודנה אם הקידושין חלין וצריכה גט מהמקדש או לא.\\\vspace{0pt}

ואציע לפני א״ח נ״י מה שהעלתה במצודתי: על כזה כבר נשאלו גדולי האחרונים ה״ה הב״י בתשובה ומהר״ם מטראני (ח״א של״ג) וכתב שנראה לו פשוט שלא עברה על התנאי דאינו נקרא נישואין אלא בכניסה לחופה ושהיא ראוי׳ לביאה יע״ש וראייתו מדאמרינן פרק המגרש הרי הותרה אצלו בזנות משמע דנישואין דוקא והביא עוד שם ראי׳ מדברי הרמב״ם (פ״ח מהל׳ גירושין) בהתנה שתינשא לפלוני וכו׳ וכן דעת הר״ש בר צמח והרב בעל שער המלך (בפ״ח מהל׳ גירושין הל׳ י״ב) שדא נרגא בפסק זה והניח בצ״ע. והנה לדידן דגם בע״מ צריך ליטול הימנה ויחזור ויתננה לה כמבואר באהע״ז (סי׳ קל״ז) ועוד לדעת הרמב״ם (פ״ח מהל׳ גירושין הל׳ י׳) דצריך להתנות עד זמן פלוני לא עשה המסדר הגט כהוגן שסידר גט כזה ועל הבנים מן השני יש לדון כי לדעת הרמב״ם הם ממזרים אבל מה אעשה כבר אמרו חז״ל כל שאינו יודע בטיב גיטין וכו׳ ומה שעבר עבר וכעת הנדון אם האשה מקודשת לנאסרה לה או לא.\\\vspace{0pt}

ונראה לי להביא ראי׳ שהקידושין חלין באם אמר המגרש ע״מ שלא תנשאי לפלוני מגיטין (דף פ״ג ע״א) דאמר ר״ע הרי שהלכה זו ונשאת לאחד מן השוק לא נמצא גט בטל ובני׳ ממזרים ועמדו כל הראשונים התוספת והרמב״ן והרשב״א והר״ן ותמהו והא אין נישואין חלין שהרי אסורה עליו משום איסור א״א ותוספ׳ שם תירצו דמיירי לאחר מיתת המגרש ואין כן דעת בעה״ג דסובר לאחר מיתת המגרש אין הגט בטל ובנים מהשני לא הוי ממזרים א״כ יקשה קושיא הנ״ל והרמב״ן תירץ אף דהנשואין אינן עושין קנין מ״מ ע״י תנאי נאסרה במעשה הנשואין יע״ש והר״ן דרך דרך לעצמו ואמר דהוי כאלו אמר הרי את מותרת לכ״א אלא שאם תנשאי לפלוני הרי את אשתי עד שעת נישואין ומשעת נישואין ואילך אלא בשעת נישואין אין את אשתי ולא אמרינן כיון דפסקה פסקה כיון שלא פסק אלא לענין שאפשר לנשואי אותו פלוני לחול יע״ש ועיין בספר נתיבות המשפט לח״מ (סי׳ רמ״א ס״ק ח׳) מה שכתב בזה.\\\vspace{0pt}

ולכאורה דברי הר״ן מוכרחים דעמדתי בקושיא אחת שקשה אלי לדעת הפוסקים בע״מ שלא תנשאי לפלוני הוי גט אף בלא נתן זמן (עיין באהע״ז סי׳ קמ״ג סעיף ט״ז ובב״ש שם) איך יפרשו הסוגיא גיטין (דף פ״ה) חוץ מקידושי קטן מהו א״ל תניתוהו קטנה מתגרשת בקידושי אבי׳ וכו׳ הכא נמי אתיא לכלל הוי׳ ע״ש משם מבואר דלא מיקרי גט כריתות אם נאסרה להתקדש לפלוני וא״כ אם אמר חוץ מקידושי קטן לא הוי גט כיון דנאסרה בקידושי קטן וא״כ מ״ש אם אמר הרי זה גיטך ע״מ שלא תנשאי לפלוני אמאי הוי גט הא בעינן ויצאה והיתה ולזה שנאסרה אינה בת הוי׳ ואין לחלק בין אמר לשון חוץ לאמר לשון תנאי הא מ״מ ויצאה והיתה ליכא ועיין בדברי רש״י שם שכתב חוץ מקידושי קטן מפלוני קטן אלא ודאי כדעת הר״ן דאף לפלוני שנאסרה באם אמר ע״מ שלא תנשאי לפלוני הנישואין חלין ויש לקיים ויצאה והיתה וא״כ הוי גט ומוכרחים דברי הר״ן ובזה מסולקין דברי הרמב״ן.\\\vspace{0pt}

אמנם יש לתרץ קושיתי באופן אחר ואין מכאן ראי׳ לסברת הר״ן ויצדקו דברי הרמב״ן בתירוצו דאי אמרינן כדעת האחרונים שהבאתי דקידושין חלין באומר ע״מ שלא תנשאי לפלוני א״כ איכא ויצאה והיתה ע״י קידושין כיון דהוי׳ איכא ולא כן גבי חוץ דהוי׳ ע״י קידושין ג״כ ליכא ולפ״ז בנדון דידן הקידושין חלין וצריכה ממנו גט.\\\vspace{0pt}

והנה הרמב״ם בהל׳ גירושין (פ״ח) כתב דהמגרש ע״מ שלא תנשאי לפלוני צריך להגביל זמן ואי לא לא הוי תנאי והראשונים חלקו עליו ובה׳ אישות (פ׳ ט״ו הי״ג) כתב האומר לאשה הרי את מקודשת חוץ מפלוני וכו׳ יע״ש וכתב השער המלך (בפרק הנ״ל) יש לתמוה דהא רבינו ז״ל סובר דבע״מ שלא תנשאי לפלוני אינו גט משום דאכתי אגידה בי׳ ודוקא בע״מ שלא תנשאי לפלוני עד ג׳ שנים וכו׳ וה״ל לרבינו לפרש דוקא אם א״ל הרי את מקודשת ע״מ שתהי׳ מותרת לפלוני עד זמן פלוני ה״ז מקודשת דומיא דגירושין והניח בצ״ע ועיין מ״ש הלח״מ שם ולפמ״ש מוכרח הדין שהביא הרמב״ם כיון דגבי כריתות צריך האשה שלא תהי׳ אגידה בבעל כל ימי חיי׳ משום דכתיב ויצאה והיתה וכיון שאמר ע״מ שלא תנשאי לפלוני ליכא ויצאה והיתה כמובן אבל אם אמר לחמשים אתיא לכלל הוי׳ ולא אגידה בי׳ ודומה ממש למ״ש בגמרא גבי קדושי קטן כיון דאתיא לכלל הוי׳ וזהו ניחא לענין גירושין דרחמנא אמרה ויצאה והיתה אבל בקידושין ליתא לסברא זו ע״כ לא חילק הרמב״ם גבי קידושין בין זמן התנאי ומיושבים דברי הרמב״ם.\\\vspace{0pt}

מדברי הרמב״ם הנ״ל מבואר נגד דעת האחרונים באומר שלא תנשאי לפלוני עד חמשים שנה קידושין לא תופסין בה ממי שנאסרה עלי׳ דאל״כ נתסר ראייתי שהבאתי לדברי הרמב״ם כמובן וא״כ יסתרו דברי רבינו ממה שהכריח השעה״מ שהבאתי לעיל מדברי הרמב״ם שסובר דקידושין תופסין בה ממה שפסק בהתנה שתנשאי לפלוני לא קיים התנאי בקידושין יע״ש וע״ז יש לתרץ ולומר בהתנה שתנשא לפלוני משמע קידושין ונשואין וע״כ כתב בהא בהתנה שלא תנשאי לפלוני משמע קידושין נמי ולא תפסי בה קידושין ובהתנה שתנשאי לפלוני לא קיימה התנאי בקידושין לחודא.\\\vspace{0pt}

אמנם לאחר העיון נראה דגם הרמב״ם נמי מודה באומר ע״מ שלא תנשאי לפלוני וקבלה קידושין מפלוני תופסין בה קידושין מדרבנן אף דלא שייך גבה ויצאה והיתה דזה לשון הרמב״ם (פ״י ה״א מהל׳ גירושין) אמר לה הרי את מגורשת ממני ואין את מותרת לכל שאעפ״י שאין זה גט הרי זו פסולה לכהונה מדבריהם שנאמר ואשה גרושה מאישה וכו׳ וזהו ריח הגט שפוסל בכהונה וכתב עליו ה״ה הרי בפ׳ המגרש משמע שהיא מן התורה וכו׳ דטעמי׳ דר״א מהכא אפי׳ לא נתגרשה אלא מאישה וכו׳ מבואר שפיסול זה הוי מן התורה וסיים שם שקושיא זו צ״ע לדעת רבינו ולי נראה דהרמב״ם הכריח דעתו באם אמר הרי את מגורשת ממני ואי את מותרת לכל העולם הוי רק ריח הגט מדבריהם ועיין יבמות (דף צ״ד) כיון דקי״ל כר׳ נחמן (דף פ״ה) דבעינן אתיא לכלל הוי׳ ש״מ דלא הוי גט מן התורה באם נתגרשה רק מאישה וזהו ניחא לדעת חכמים שחולקים על ר״א גבי חוץ אבל ר״א דמתיר באומר לה חוץ מפלוני משום דילפינן מאשה גרושה מאישה לית לי׳ סברת ר״נ דבעינן ויצאה והיתה ומאי דמתרץ הגמרא כאן ורבנן איסור כהונה שאני היינו בס״ד דלא ידעינן מסברת ר״נ אבל לפי האמת דקיי״ל כר״נ דבעינן אתיא לכלל הוי׳ אין כאן קושיא על רבנן די״ל בפשיטות משום דליכא ויצאה והיתה באומר חוץ לפלוני וכן צריכים אנו לפרש אליבא דר׳ יוחנן לקמן (דף פ״ד) שמתרץ למתניתן אפי׳ כרבי ולכך צריך לטלו הימינה הואיל וקנאתו לפסול בו לכהונה דהקנין לא הוי רק מדרבנן ויעיין ברשב״א שם.\\\vspace{0pt}

ובענין זה ישבתי קושית הפני יהושע שהניח בצ״ע על מתניתן דחכמים סוברים כיצד יעשה יטלנה הימנה ויחזור ויתננה לה הא לפי דעת הפוסקים דסוברים בע״מ הוי גט גמור ורק בחוץ חולקים אמאי יאמר הרי את מותרת לכל אדם וזה שלא כרצון המגרש שרצה לאוסרה על פלוני אמאי לא קתני תקנתא טפי שיטלנו הימנה ויאמר ע״מ שלא תנשאי לפלוני ויתננה לה ע״ש אבל לפי מה שכתבתי ליכא תקנה זו כדעת הפ״י דאם נתן הגט מעיקרא ואמר חוץ לפלוני א״כ נאסרה לכהונה ואם יחזור ויטלנו הימנה ויאמר הרי את מותרת ע״מ שלא תנשאי לפלוני ואם האשה לא תתקיים התנאי לדעת הראשונים שתוכל לבטל התנאי בחיי הבעל כדברי הרמב״ן והר״ן שהבאתי לעיל אזי הגט בטל ואם ימות הבעל אחר הביטול יסברו העולם דהיא אלמנה ומותרת לכהונה ובאמת אשה זו כיון דכבר קבלה הגט באמירת הבעל חוץ מפלוני כבר נפסלה לכהונה ויש לחוש לקלקלה ע״כ תקנו חכמים בזה ואמרו אם נתגרשה האשה בגט הניתן לה באמירת הבעל חוץ מפלוני אין כאן תקנה רק בשיטלנה ממנה ויאמר הרי את מותרת לכל אדם ויחזור ויתננה לה.\\\vspace{0pt}

ונחזור לדברינו לעיל דדעת הרמב״ם דסובר כר״נ דבעינן ויצאה והיתה אתי׳ לכלל הוי׳ מ״מ מודה בנתגרשה האשה מאישה לבד דהוי גט דרבנן ואסורה לכהונה ותופסין קידושין מדרבנן עיין באע״ז (סי׳ ק״נ סעי׳ ג׳) ובב״ש שם וא״כ בנדון דידן קידושין תפסי מדרבנן ואף שלא נגבל זמן בהתנאי שלא תנשאי לפלוני לדעת הרמב״ם מדרבנן הוי גט ואף דמשמעות דברי הרמב״ם (פ״ח מה׳ גירושין הי״ב) שכתב אינו גט משמע דלא הוי גט כלל נ״ל דסמך עצמו על מה שכתב בפ״י מהל׳ אלו בנתגרשה מאישה לחוד הוי גט מדבריהם משום דלא בעינן ויצאה והיתה וכמו כן בשאר תנאים ויעין בב״ש אהע״ז סי׳ קמ״ז ס״ק ו׳.\\\vspace{0pt}

אבל אחרי בינותי ראיתי שדברי הב״ש צדקים מדכתב הרמב״ם בפ״י מה״ג ה״ו הרי שנתקדשה ונמצא הגט בטל וכו׳ אינה צריכה גט משני וכו׳ ע״ש מדסתם וכתב ונמצא גט בטל ולא הוציא בריש הפרק רק בנתגרשה מאישה לבד משמע בשאר דיני תנאים כ״מ שהגט בטל אף לכהונה אינה אסורה וא״כ קידושין אינן תופסין בה אף מדרבנן וכיון שכתבתי לעיל טעמא דהרמב״ם דפוסק במגרש ע״מ שלא תנשאי לפלוני משום דבעינן אתיא לכלל הוי׳ לכך אינה מגורשת רק בהגבלת הזמן מבואר מדבריו דחולק ע״ד זו שהביא הב״י ודעתו כדעת התוספת שהביא שער המלך שהבאתי לעיל דאין קידושין תופסין ודעת הרי״ף שלא הביא פשיטותא דגמרא גבי קידושי קטן וכבר נתעורר הר״ן שם אפשר דסובר כדעת הבעל הלכות גדולות דלא גרסינן תניתוהו וכיון שהביא ביבמות פרק האשה שהלך דברי רב דרשת ואשה גרושה מאישה משמע דהוי גט אף דלא הוי בת הוי׳ א״כ נשאר האבעי׳ בתיקו כמו שאר האבעיות שם.\\\vspace{0pt}

אמנם נ״ל להביא ראי׳ בתנאי שלא תנשאי לפלוני שקידושין תופסין מנאסר מדאמרינן בגיטין (ד׳ פ״ג) ר״י קאמר שאין משיבין את הארי לאחר מותו וכו׳ דתניא מקיש קודמי הוי׳ ראשונה להוי׳ שני׳ מה התם לא אגידה גבי׳ וכו׳ ויעיין מה שכתב הרשב״א שם וקושיתו קאי על ר״א דמתיר בחוץ אבל בע״מ מודה ר״י לפי דעת הפוסקים דבע״מ מודו חכמים א״כ קשה הא יש להקיש הוי׳ ראשונה להוי׳ שני׳ בזה הצד נמי מה הוי׳ ראשונה היתה יכולה להתקדש לכל אדם אף בהוי׳ שני׳ תהי׳ מותרת לכ״א ואף דבהוי׳ שני׳ נאסרת לקרובים כבר עמדו בזה התוספ׳ שם וכתבו שתהא מופקעת לגמרי יע״ש וא״כ בתנאי נמי אינה מופקעת לגמרי ואין להקשות משאר תנאים דעלמא כגון ע״מ שתתן לי מאתים זוז דשם מותרת להתקדש רק פלוגתת אמוראים אי צריכה קיום התנאי קודם נישואין שמא תאנס וזהו חשש דרבנן כמבואר ברשב״א ריש פרק המגרש אבל בתנאי שלא תנשאי לפלוני אינה יכולה להתקדש לפלוני וא״כ אמאי לא מקשינן הוי׳ ראשונה לשני׳ אף לזה א״ו דהקידושין תופסין מנאסר.\\\vspace{0pt}

ולענ״ד נראה ליישב קושית הראשונים במה שתמהו והלא אין נישואין חלין שהרי אסורה עליו משום א״א דבקידושין (דף ס״ז) איתא מניין שאין קידושין תופסין בא״א ומשני דילפינן דהקישא דר״י מאחות אשה ועתה נחזי אנן באשה שתופסין בה קידושין כגון שנתגרשה ע״מ שלא תנשאי לפלוני ונתקדשה לפלוני אז חופה קונה בה ועושה קנין נישואין כיון דכאן יקשה קושית הש״ס מניין שאין קידושין תופסין בא״א ואין חופה עושה קנין בא״א (אי חופה הוי קודם קידושין יעיין במשאת בנימין ובמ״ל פ״י מה״א ה״ב) ואין לתרץ דילפינן מהקישא דר״י ז״א דמה לעריות שאין קידושין תופסין בהן וכאן באשה הנ״ל קידושין תופסין אמרינן דחופה עושה קנין לפ״ז קאמר ר״ג שפיר הרי שהלכה היא ונשאת נמצא גט בטל ובני׳ ממזרים כיון דהחופה עושה קנין ובטל הגט ולא דמי כלל למה שהביאו התוספת מן אבא ואביך דשם קידושין נמי אינן תופסין.\\\vspace{0pt}

אבל מה שיש בזה מהעיון אי חופה קונה בכהאי גוונא כיון דאינה ראוי׳ לביאה אינה קונה יעיין ברמב״ם (פ״י מה״א) שכתב דחופת נדה אינה קונה והוי כארוסה ואף שפסקינן כרב לגבי שמואל ביבמות יש חופה לפסולות באלמנה לכהן גדול אע״פ דלא חזיא לביאה היינו דוקא לתרומה או דכנסה לבוא עלי׳ כפי דמחלק הרב בלח״מ שם אמנם התוספת דלא תירצו כן כמ״ש סוברים דחופה שאינה ראוי׳ לביאה אינה קונה א״כ הקשו שפיר והלא אין נישואין חלין שהרי אסורה משום א״א ולא שייך תירוצו.\\\vspace{0pt}

ובזה סלקתי תמיהת אדם גדול בענקים ה״ה מהרי״ט בתשובותיו ח״א מ״ט במה שהשיב לר׳ יעקב אבואלעפי׳ וז״ל שם וצריך אני להתלמד בדבריהם וקשיא לי טובא חדא וכו׳ אבל כשאמר ע״מ שלא תנשאי לפלוני הא קיי״ל כל האומר ע״מ כאומר מעכשיו דמי והרי עכשיו כפנוי׳ אצל כל אדם ואף אצל זה שנאסרה עליו וחלין בה קידושין אלא שאחר שהוא מקדשה חוזרין הקידושין ונפקעין מחמת התנאי שמתבטל הגט למפרע ונמצאת א״א שהרי אמר לה ע״מ שלא תנשאי לפלוני עכ״ל ולפענ״ד לא קשה מידי דקושית התוספ׳ היא הא אין נשואין חלין משום דאסורה עליו משום א״א ומאי בכך דהוי פנוי׳ לכל העולם הא הקנין חופה אינו קונה משום דאינה ראוי׳ לביאה כיון דהוי א״א. ומה שהקשה מהרי״ט עוד שם דאף דלא שייך נישואין וכו׳ דודאי לא לעביד כעין נישואין זהו תירוץ הרמב״ן על קושית התוספת. וכמו כן עמדתי ע״ד הב״ח באהע״ז (סי׳ קמ״ג ס״ק כ״ב) מ״ש לתרץ דברי הרמב״ם דמיירי שנכנסה לחופה ולא נבעלה ע״כ אם נשאת לא תצא הא חופה לא קני כיון שאינה ראוי׳ לביאה והרמב״ם לשיטתו סובר דחופה שאינה ראוי׳ לביאה אינה קונה ואפשר לחלק כיון דבידה לקיים תנאי ראוי׳ מקרי לא כן גבי חופת נדה דאינה בידה ועיין בתוספת יו״ט גיטין בד״ה עד אחד. ובזה מיושב נמי מה שעמד מהרי״ט בתשו׳ הנ״ל על דברי הרא״ש בתשו׳ שכתב באומר ע״מ שלא תנשאי לפלוני אין תקנה אלא שיקדשנה ויתן לה גט אחר בלי תנאי ותמה עליו וז״ל ועוד אני תמי׳ אפי׳ תימא דביטול לא מועיל למה הוצרך לקדשה כדי לגרשה יכתוב לה גט אחר ויתננו לה בלי שום תנאי ודיו בכך וכו׳ דמ״מ בשעה שהיא תנשא לאותו ראובן נתבטל הגט הראשון והרי היא א״א למפרע ויכול לחול שפיר הגט האחרון שהוא בלי שום תנאי עכ״ל והאריך עוד להביא ראיות שיש גט אחר גט. ולפמ״ש לא קשה מידי דנשואי הנאסר אינן עושין קנין כיון שאינה ראוי׳ לביאה א״כ התנאי שלא תנשאי בתוקפו עומד ואין לך מקום אחר להמציא רוח והצלה להנשא לנאסר רק באם שהראשון מקדש אותה פעם אחר ומגרשה בלי תנאי כדעת הרא״ש ואף שהרא״ש בפסקיו (והובא דבריו באהע״ז סי׳ ס״א) פוסק דחופת נדה קונה וחולק אדעת הרמב״ם אולי בתשו׳ החזיק בסברת הרמב״ם ואפשר שיש לחלק בין נידון זה שאינה ראוי לביאה מצד א״א לחופת נדה ויען שהענינים האלה בחופה שאינה ראוי׳ לביאה עמוקים המה לא ראיתי להאריך כעת בזה.\\\vspace{0pt}

ונחזור לנדון שאלתנו מכח שני תירוצים שכתבתי על השני קושיות של מהרי״ט על דברי התוספת וע״ד הרא״ש מבואר באומר הרי את מותרת ע״מ שלא תינשאי לפלוני ונתקדשה לנאסר עלי׳ שקידושין תופסין וצריכה גט דאל״כ בקידושין עברה על תנאי וא״כ קושיות מהרי״ט על מכונם מוצבים.\\\vspace{0pt}

בהא נחתינן וסלקינן בנידון שאלתי שצריכה גט מהמקדש כדעת הב״י. ע״כ דברי חתני נ״י.\\\vspace{0pt}

תשובה – חתני נ״י בקשת ממני להעיר על דבריך מה שיש לי להשיב. הנה כעת טרידנא מאוד ורק למען מלאות רצונך במקצת אעיר על איזה דברים. מה שהקשת לשיטת הפוסקים דע״מ שלא תנשא לפלוני הוי גט אפילו בלא נתן זמן איך יפרשו הסוגיא דגטין (דף פ״ה) חוץ מקידושי קטן דמסיק דבעינן ויצאה והיתה ומוכח משם דהיכי דנאסרה להתקדש לפלוני אין זה כריתות כמו אם אמר חוץ מקידושי קטן דלא הוי גט דלא קרינן בי׳ ויצאה והיתה – אין אני רואה קושיא בזה דדוקא אם ע״י גט לא הותרה להתקדש לשום אדם בזה לא שייך ויצאה והיתה ולכן כדמבעי׳ לי׳ רבא מר״נ אם דומה חוץ מקידושי קטן לחוץ לאבא ולאביך דלא מקרי שיור כיון דבלא״ה אינה יכולה להתקדש להם וה״נ קטן כיון דעתה לאו בר הוי׳ הוא או אי נימא דמקרי בר הוי׳ כיון דבא לכלל הוי׳ וא״כ מקרי שיור והגט פסול על זה השיב ר״נ ממה דתניא קטנה שקדשה אבי׳ ומת היא מקבלת גטה בקטנותה ואמאי הרי אינה יכולה לקדש את עצמה לשום אדם וא״כ לא קרינן בה ויצאה והיתה לאיש אחר כיון דע״י גט זה עתה לא הותרה לשום אדם אע״כ צ״ל כיון דגט זה מועיל שתוכל לבא לכלל הוי׳ כשתתגדל ויצאה והיתה קרינן בה א״כ ה״נ מקרי קטן בר הוי׳ כיון דאתיא לכלל הוי׳ ולפ״ז רק אם אינה בת הוי׳ לשום אדם שייך לומר דלא קרינן ויצאה והיתה וכמו דהערת שפיר דר״א דמכשיר גט באמר חוץ לפלוני דדרש אפילו לא נתגרשה רק מאישה ולא מותרת לשום אדם מקרי גט ע״כ לית לי׳ דרשה דר״נ דויצאה והיתה אבל במגרש ע״מ שלא תנשא לפלוני כיון דע״י גט הותרה לכל העולם שפיר קרינן בי׳ ויצאה והיתה לאיש אחר ולכן הגט כשר.\\\vspace{0pt}

גם מה שהביא חתני נ״י ראי׳ דמגרש על מנת שלא תנשא לפלוני הקדושין תופסים בנאסר ממה דפריך ר׳ יהושע לר״א מקיש קודמי הוי׳ שני׳ לקודמי הוי׳ ראשונה מה קודמי הוי׳ ראשונה דלא אגידא באינש אחרינא אף קודמי הוי׳ שני׳ דלא אגידה באינש אחרינא וא״כ איך תתגרש בע״מ שלא תנשא לפלוני הרי לא דומה קודמי הוי׳ ראשונה לקודמי הוי׳ שני׳ דקודם קידושין הראשונים היתה יכולה להתקדש לכל אדם מה שאין כן בהוי׳ שני׳ אע״כ דקידושין תופסין בנאסר – לענ״ד אינה ראי׳ דדוקא לר׳ אליעזר דס״ל דגם בחוץ מפלוני דשייר בגט הגט כשר פריך ר׳ יהושע שפיר כיון דמצד הגט עצמו דאתיא מויצאה מביתו לא דמי והיתה לאיש אחר דלאחר גירושין לכי יקח איש אשה דקודם גירושין אבל היכי דמצד הגט עצמו דאתי מויצאה לא אגידה באינש אחרינא שהוא גט גמור ורק ע״י תנאי שעם הגט אגידה שפיר דמי קודמי הוי׳ שני׳ לקודמי הוי׳ ראשונה דהא זהו הטעם דבחוץ לא מגורשת לחכמים דר״א ובעל מנת מגורשת כיון דבחוץ יש שיור בגט מה שאין כן בעל מנת דכיון דאין אוסרה אלא בלשון תנאי הוי כאלו הותרה לכל כמו שכתבו התוספ׳ שם דף פ״ב).\\\vspace{0pt}

אמנם להלכה אני מסכים עם חתני נ״י דקדושי הנאסר תופסין בה אבל לא ידעתי למה כתבת כדעת הב״י שהרי הב״י לא החליט שכתב דמיראי הוראה הוא ומשמע דרק להחמיר ס״ל דקדושין תופסין ולא להקל דלדבריו לאחר קידושי הנאסר צריכה מספק גט שני להתירה לעלמא דשמא עברה על התנאי וגם גט מן המקדש דשמא על הקידושין לא התנה וחלו וביותר הי׳ לך לכתוב כדברי המבי״ט ורשב״ץ שכתבו בפשיטות דקידושין אינם בכלל תנאי וכן נלענ״ד שהרי בקידושין (דף י׳) דאבעי׳ לי׳ ביאה נישואין עושה או אירוסין עושה פשיט אביי מדתנן האב זכאי וכו׳ קתני ביאה וקתני נשאת ע״ש הוי פשיטא להגמרא דאפילו בבא עלי׳ לשם קידושין אם ביאה לא נחשב רק קידושין לא שייך בה לשון נשאת א״כ למה נאמר דכשהתנה עמה שלא תנשאי לפלוני שדעתו הי׳ ג״כ שלא תתקדש לו. והנה השער המלך רצה להוכיח מדברי התוספ׳ יבמות (דף י׳) וגטין (דף פ״ב) שבכלל לא תנשא הוא ג״כ שלא תתקדש לו שהקשו למה לא תנא ביבמות ט״ז נשים פוטרות צרותיהן דמשכחת גם אשת איש שנופלת לייבום כגון שגרשה זה על מנת שלא תנשא לפלוני ונשאה אחיו של פלוני ומת בלא בנים ונופלת לפניו לייבום והיא לו אשת איש וכתבו ומעל מנת שלא תנשאי גרידא דמותרת בזנות או לא תיבעלי גרידא שיכולה להתקדש אין קשה דלא דמי לאחות אשה שאסורה גם בזנות ואינה יכולה להתקדש אבל מעל מנת שלא תנשאי ולא תיבעלי קשה עכ״ל ודייק השעה״מ למה הוצרכו התוספ׳ לומר דע״מ שלא תנשא לא דמי לאחות אשה דמותרת בזנות למה לא כתבו ג״כ כמו בלא תיבעלי שיכולה להתקדש וביותר קשה על מה שכתבו דאם אמר ע״מ שלא תנשאי ושלא תיבעלי קשה דזה דומה לאחות אשה הא אכתי אינו דומה שיכולה להתקדש לו משא״כ באחות אשה אע״כ שס״ל להתוספ׳ דבכלל שלא תנשא הוא ג״כ שאינה יכולה להתקדש לו ולענ״ד יש להשיב על ראי׳ זו דאע״ג שהתוספ׳ נקטו בלשונם שמשכחת בגרשה על מנת שלא תנשא לפלוני י״ל דנקטו כן מפני שהוא לשון הברייתא שם מכ״מ יקשה קושיתם דמשכחת אשת איש שנופלת לייבום בגרשה על מנת שלא תתקדש ולא תיבעל לפלוני דבזה דומה לאחות אשה לגמרי ולכן אין מזה ראי׳ גמורה שדעת התוספ׳ דבכלל לא תנשא הוא ג״כ שלא תתקדש דמנ״ל לומר כן והשער המלך עצמו כתב שמשו״ת הרשב״א מוכח ג״כ כדעת הרשב״ץ והר״מ מטראני שאם אמר שלא תנשא מותרת להתקדש לו. כנלענ״ד, הקטן יעקב.\\\vspace{0pt}

\end{multicols}\newpage

\newchap{סימן קמד}
\begin{multicols}{2}
ב״ה אלטאנא, יום ג׳ י״ב תמוז תרט״ז לפ״ק.\\\vspace{0pt}

שאלה: הי׳ מריבה וקטטה בין איש לאשתו וקבלה האשה אצל השררה ונפסק שם שנחשבים כנפרדים ומותר להאשה אשר יצאה זכאי להיות לאיש אחר ולענין התביעה בדברים שבממון שיש להם זה על זה נפסק שלכל אחד לברר ע״פ המשפט מה שיש לו לתבוע. והנה פשוט שע״פ דין תורה אין האשה מותרת להנשא עד שתקבל ממנו גט כדת משה וישראל וגם שניהם נתרצים ליתן ולקבל גט, אמנם בשביל שהאיש כעת אין לו מאומה אבל כפי הנראה יפול לו סך עצום בירושה במות אביו לכן האשה אינה נתרצה למחול על תביעות שיש לה על בעלה מכתובה ונדוני׳ והתוספת והבעל אומר לא תקבל ממני מאומה אחרי שע״פ דין המלכות לא יהי׳ לה גם אם יקנה נכסים מה שאין לה עתה, ונשאלתי אם מותר לסדר גט כזה בשהאשה אינה מוחלת על תביעתה.\\\vspace{0pt}

תשובה: לפי המבואר בסדר הגט סעיף פ״א בהג״ה צריכה האשה קודם נתינת הגט לחזור לבעל הכתובה כדי שלא יבואו אח״כ לידי קטט מחמת הכתובה ושיאמר הבעל ע״מ כן לא גירשה וכן נאמר שם בסימנים הקטנים בסדר הגט סי׳ רי״ט יזהירם הרב שלא יהיו קשורים ביחד בשום דבר, והדין נובע מסדר הגט של מהר״י מינץ שכתב שמצא כן בכתב מהור״ר איסרלן ז״ל וביאר שם הטעם מהא דתנן פרק השולח נשאת לאחר והיא תובעת כתובתה אמר רבי יהודה אומרים לה שתיקתך יפה מדיבורך ומפרש רש״י ותוס׳ דמצי אמר לה אדעתא דיהיבנא כתובתך לא גרשתיך וכן הלכה לכן דן הר״י מינץ שצריך להסתלק תביעת הכתובה קודם הגירושין שלא יבוא הבעל ויאמר על דעת דליתן כתובה לא גרשתי כי אמרתי שלא תתבע כתובתה או שאפטר לשלם מאיזה טעם שיהי׳.\\\vspace{0pt}

אכן קשה לכאורה על טעם דר״י מינץ שהרי בגטין (דף מ״ה) אהא דהמוציא את אשתו לא יחזיר אמרינן א״ר יוסף וכו׳ והוא שאמר לה וכו׳ אי אמר לה הכי מצי מקלקל וכו׳ א״ד וכו׳ ע״ש והיוצא משם דפליגי תרי לישנא אליבי׳ דר״נ אם אפילו בלא אמר בפי׳ מפני מה מוציאה מכ״מ לא יחזיר משום קנס אבל עכ״פ חשש קלקול ולעז לא שייך רק באמר בפי׳ שמטעם זה מוציאה ולפי האמור שם (דף מ״ו ע״ב) בסוגיא דאיילונית מבואר דלר״מ דס״ל דבעינן תנאי כפול לא יכול לקלקלה כי אם בכפלי׳ לתנאי׳ וא״כ במה שיש לה תביעה עליו מכתובה וכדומה איך יכול לקלקלה אם לא אמר בפי׳ שמגרשה ע״מ שלא לשלם לה כתובה ונדוני׳ והנה בזה אם בעינן שכפל לתנאו כבר יש פלוגתא בין הראשונים הביאה הטור באהע״ז (סי׳ י׳) והוא פסק כשיטה שצריך לכפול ובש״ע שם סעיף ג׳ הביא ג׳ שיטות בזה ולשיטה האחרונה שהיא שיטת הרמב״ם אפילו לא אמר מפני כך אני מוציאך לא יחזיר וכפי הנראה פסק הש״ע כן מדהביא שיטה זו לבסוף אכן גם לפי שיטה זו אין הטעם דלא יחזיר משום דיכול לקלקלה בלא אמירה ג״כ אלא שהרמב״ם פסק כלישנא בתרא דר״נ דמשום שלא יהיו פרוצים לא יחזיר ולפ״ז לכל השיטות בלא שאמר בשעה שמגרש שע״מ כן הוא מגרש לא יכול לקלקלה אח״כ ולומר אלו הייתי יודע וכו׳ ואיך שייך לעז בלא נתן כתובה והנלענ״ד שהר״י מינץ סמך על מה שכתב ה״ה הביאו גם הבית שמואל שם ס״ק ז׳ דגבי איילונית שכתב הרמב״ם ג״כ דבלא אמירה ג״כ לא יכול להחזירה והרי שם לא שייך קנס ופי׳ ה״ה כיון דאיילונית הוא מום גדול סתמו כפירושו ואע״פ שלא אמר לה יכול לקלקלה ולהוציא לעז עלי׳ ע״ש וא״כ ה״נ כיון שידוע לכל שהאשה תוכל לתבוע כתובה מהמגרש והיא קבלה הגט בלא כתובה נעשה כמי שאמר בפי׳ שלכך גירשתיך מפני שלא אצטרך לשלם לך כתובתיך ואם תתבע אח״כ יכול לקלקלה ולהוציא לעז. והנה בנדון שלפנינו שהיא הכריחה אותו ע״פ השררה לפירוד מה שמביא לידי גירושין והוא כבר אמר בפי׳ שלא תוכל לקבל ממנו מאומה לפ״ז נעשה ג״כ סתמא כפירושו שאם תכריחנו כשיבא לידי נכסים לשלם לה מה שחושב עתה שלא יצטרך לשלם לה ע״פ דין המלכות יתחרט על הגט ויאמר אם לא הייתי מגרשה היו נשארים לי נכסי ועתה הוצרכתי ליתן לה ולענ״ד אין חילוק להקל בין דין דהר״י מינץ לזה ואדרבא בזה שייך עוד קלקול יותר כיון שאמר בפי׳ שלא יצטרך ליתן לה א״כ הוי אומדנא דאם יראה שטעה יתחרט משא״כ בכתובה שלא אמר בפי׳ כן שמגרש ע״מ שלא לשלם כתובה.\\\vspace{0pt}

אמנם עדיין יש מקום עיון ע״פ מה שכתב הרשב״א בגטין (דף מ״ו) שהקשה כיון דאם אומרים לו שלא יכול להחזירה ליכא קלקול שהוא לא ישתדל להוציא לעז שאינו מצוי לקלקלה אלא כל זמן שהוא חושב שתחזור לו דא״כ ליכא קלקול לעולם דהא מכי נשואה לאחר אסורה לחזור לו ותירץ שהוא סבר כל שאין הגט בטל אלא מחמת שנמצאו הדברים בדאים אף כשנשאת אנוסה היתה כשסבורה שהיא מגורשת גמורה ואונס בישראל משרא שרי עכ״ל ולפ״ז אף שגם בתובעת כתובה שייך כן שיש לו תקו׳ שתשוב אליו כשיאמר שאדעתא דהכי לא גירש והגט בטל והיא שנשאת אנוסה היתה ומותרת לחזור לו מכ״מ יקשה דהתינח באשת ישראל אבל כשהמגרש כהן דאפילו אנוסה אסורה לו מה לעז שייך והרי בסדר הגט לא חילק בזה בין אם כהן הוא או ישראל וצ״ל דהר״י מינץ סובר כיון דלאו כ״ע דינא גמירא לעולם יהי׳ לעז אם הבעל צוח גרשתי בטעות ושהגט בטל ובני׳ ממזרים אף שהאמת אינו כן כיון שלא גרש על תנאי זה בתנאי כפול מכ״מ כיון שלפי מחשבתו הוא כן יוציא לעז זה כמש״כ הרשב״א בדבור הנ״ל.\\\vspace{0pt}

סוף דבר אני אומר אף שהראתי פנים להקל בדין זה עכ״ז לא מלאני לבי לחלוק על דין פשוט של סדר הגט המוסכם להלכה מכל ישראל. גם לא ראיתי שום חילוק בין נדון השאלה לנדון של סדר הגט. אמנם מכ״מ מי שלבו רחבה ורואה קלקולים בחדול הגט ורוצה לסמוך על דעתו להקל לענ״ד ימצא סמיכה קצת בהערות הנ״ל. כנלענ״ד, הקטן יעקב.\\\vspace{0pt}

\end{multicols}\newpage

\newchap{סימן קמה}
\begin{multicols}{2}
ב״ה אלטאנא, יום ה׳ י״ד למב״י תרכ״ד לפ״ק. להרב וכו׳ מ״ה משולם יעקב פרענקעל נ״י בק״ק אימאן יע״א במדינת רוסלאנד.\\\vspace{0pt}

לא אכחד ממעכ״ת נ״י כי המכתב אשר ערך עלי זה שני חדשים אודות הקטטה שבין איש לאשתו בא אלי בעתו אל נכון. ומה שלא השבתי באשר שאין מדרכי בדברים שבין אדם לחבירו להודיע דעתי לצד האחד מבלי ששמעתי גם טענות צד שכנגדו ומבלי שירצו שני הצדדים שאפסוק להם דין כפי ענ״ד. אכן אחר בוא אגרתו שנית בימי חג העבר אלי בהפצרת בקשת תשובתי לא אחר עוד מלהשיב בקוצר דברים כפי ענ״ד מפני הכבוד אך בתנאי שלא יחשבו דברי לפסק דין רק כפי הנראה להלכה.\\\vspace{0pt}

השאלה היא – איש שמו עובדיהו נשא אשה ודר עמה בשלו׳ איזה חדשים ושוב התקוטט עמה שיש בה מום שמשתנת במטה ואחר שקט הריב וישב עמה עוד איזה חדשים בשלו׳ ושלו׳ החל רוח קנאתו לפעמו שנית ועזב את אשתו זה שנתים בטענה שמאוסה עליו מפני מומה וקרובי האשה הביאו עדים שהעידו לפני ב״ד שהבעל ידע קודם החתונה שיש מום זה במשודכתו והבעל הביא עדים שנשאל קודם החתונה אם יש למשודכתו המום הזה שכבר נשמע עלי׳ והשיב שאין בה מום זה ועתה באשר שלא תרצה לקבל גט ממנו רוצה לישא אשה אחרת מבלי גירש את זאת אם יעשה כן בישראל.\\\vspace{0pt}

תשובה – אם כוונת האיש שמאחר שיש בה מום קידושיו קידושי טעות היו ולכן מותר לו לישא אחרת מבלי גירש את זאת שאינה אשתו כלל זה ודאי אינה טענה דאפילו אם המום הוא מום גמור כפי טענתו ואפילו הטעתו ולא ידע מהמום בשעת קידושין מכ״מ כיון שלא התנה בפירוש כשקדשה ע״מ שאין בה מומין הוי קידושי ספק כמבואר באהע״ז סי׳ ל״ט ס׳ ה׳ ועוד הרי לא התקוטט על המום עד אחר איזה חדשים אחר הנישואין ולפי מה שכתבו הח״מ והב״ש שם בשתק וצווח לבסוף הוי קידושין ודאי ולכן פשיטא שחל עליו חרם ר״ג שלא לישא אשה על אשתו ורק בזה יש לדון אם מותר לגרשה בעל כרחה או אם חל עליו גם בזה חרם דר״ג והנה לא נעלם ממר נ״י מה שכתב הרמ״א אהע״ז סי׳ קי״ז ס׳ ה׳ שהביא שם פלוגתת הפוסקים דלדעת הרשב״ץ משתנת במטה לא הוי מום ולדעת הב״י הוי מום ועמו הסכים הרמ״א שכתב וכן נ״ל עיקר וכ״נ מדבריו בדרכי משה בסי׳ ל״ט שהביא הב״י שם שו״ת הרשב״ץ (ומה שכתב בשו״ת באר שבע סי׳ ס״ח שהרמ״א הסכים עם התשב״ץ דלא הוי מום במכ״ה שלא בדיוק כ׳ כן) וכן הסכמת כל האחרונים ולכן אם באמת הי׳ מום זה בה קודם החתונה והוא לא ידע והטעתו הדין שכופין אותה לקבל גט ממנו שאע״פ שעל פי מה שכתב הרמ״א סי׳ קי״ז ס׳ י״א רק במום של נכפה יכול לגרשה בע״כ מפני שסכנה הוא וגם אלו הי׳ באיש היו כופין לגרש מדין התלמוד אבל משום שאר מומין אינו יכול לגרשה בע״כ שאין בזה סכנה (ובפרט לפי מה שכתב הבאר שבע בשו״ת הנ״ל דאם יש מום זה באיש אין כופין לגרש אף שגם זה אינו מוכרח שהלך בזה אחר דעתו שגם באשה אין זה מום ולפענ״ד זה אינו כמו שכתבתי) הרי כבר כתב הב״ש שם ס״ק כ״ד דדין זה דרמ״א איירי במומין שנולדו בה אחר החתונה או שידע ונתפייס אבל אם היו בה מומין והוא לא ידע והטעתו י״ל דכופין אותה כיון דעשתה שלא כהוגן וכמש״כ בשו״ת הרא״ש כלל מ״ב דגם איש שקידש ברמאות כופין אותו לגרש ולכן בנדון השאלה אם התעתו והוא לא ידע מהמום הי׳ יכול לגרשה בע״כ.\\\vspace{0pt}

אמנם מאחר שעל פי עדות גמור ידע הבעל מהמום קודם החתונה וגם מה שהעידו עדיו שנשאל קודם החתונה אם יש למשודכתו מום זה והשיב לא אינה הוכחה נגד עדות עדי האשה שאפשר שלא הקפיד ובשביל שלא לפרסם מומה אמר כן ועכ״פ אפילו סבר שבאמת אין בה המום או שהי׳ בה ונתרפא הי׳ לו לחקור על זה מאחר שיצא הקול והוי לי׳ לבדוק כמו ביש מרחץ בעיר שאינו יכול לטעון גם על מומין בסתר ויש לדון אותו כידע ממומה ונתפייס ולכן אין יכול לכופה לקבל גט וכל שכן שלא יכול לישא אשה אחרת בלי שגרשה וגם אין יכול למנוע ממנה שאר וכסות אבל יכול למנוע ממנה עונה מאחר שמאוסה עליו והוי שנואת הלב וכמבואר ברמ״א הנ״ל שאפילו במקום שאין יכול לגרשה בע״כ מכ״מ יכול למנוע ממנה עונה כן נלענ״ד להלכה ע״פ הפוסקים ורופא חולי עמו ישראל הוא ירפא את האשה האומללה ותשב עם בעלה באהבה ושלו׳ ויראו זרע ויאריכו ימים. הקטן יעקב.\\\vspace{0pt}

\end{multicols}\newpage

\newchap{סימן קמו}
\begin{multicols}{2}
ב״ה אלטאנא, יום ה׳ ט״ז שבט תר״י לפ״ק. לחתני הרה״ג וכו׳ מ״ה משלם זלמן הכהן נ״י אב״ד דק״ק אפפעלן יע״א.\\\vspace{0pt}

כתבת אלי וז״ל הרא״ש ביבמות (ד׳ ס״ה ע״ב) גבי עובדא דאתי לקמי׳ דר״נ כתב וז״ל וכן הלכתא ולא חיישינן שמה נתנה עיני׳ באחר כדפרישת דמיירי בשהה עמה עשר שנים וידענא בי׳ שהוא עקור ולכך כופין עכ״ל משמע מדבריו אף דידעינן בי׳ שהוא עקור אעפ״כ צריך לשהות עשר שנים דבתוך עשר חיישינן שמא נתנה עיני׳ באחר וכן כתב הב״ש בשם הרא״ש באהע״ז (סי׳ קנ״ד ס״ק ט״ז) והיינו דחיישינן דמה שאמרה בעינא חוטרא לידי שקר הוא דלא איכפת לה שיהי׳ לה בן רק שנתנה עיני׳ באחר והנה מלבד שסברתו לא הבנתי דמה מהני לזה עשר שנים דהרי גבי טמאה אני לך אינה נאמנת אפילו אחר י׳ שנים ומ״ש טמאה מידוע שהוא עקור שבזה ובזה אין ראי׳ שדברי׳ אמת אבל חוץ מזה קשה לענ״ד עליו קושיא עצומה דלפי שיטתו מה פריך הגמרא בנדרים (דף צ״א) השמים ביני לבינך דמשנה אחרונה תהוי תיובתא דרב המנונא והא הכא דהיא ידעה ובעלה ידע בה וקתני דלא מהמני ע״ש ומה קושיא נהי דמהמנינן לה דאינו י״כ וכי עדיף זה מידוע שהוא עקור דממתינן עשר שנים וא״כ שפיר קתני במתניתן (דאיירי תוך עשר כמו שכתבו הפוסקים) דאינה נאמנת שלא תהא אשה מקלקלת על בעלה ותתן עיני׳ באחר דאמרינן שהיא משקרת במה שאומרת דבעי חוטרא לידה ובמה שאומרת שאינו י״כ באמת נאמנת מחמת חזקה אבל זה שקר מה דבאה בטענה וכדעת הרא״ש וצל״ע עכ״ד.\\\vspace{0pt}

תשובה – הנה מלבד שהקושיא עצומה סברת הרא״ש מצד עצמה לכאורה קשה מאוד שמאחר שהוא עקר ואינו ראוי להוליד למה צריך ג״כ שהה עשר שנים להחזיק טענתה דבעינא חוטרא לידי ובפרט שהתוספ׳ והטור וש״ע לא כתבו כן אלא דבחדא סגי או בידוע שהוא עקר או בשהה י׳ שנים וכבר עלה ברעיוני להגיה בדברי הרא״ש תחת וידענא או בידענא וכמו שכתבו התוספ׳ אבל ראיתי שגם בשו״ת כלל מ״ג כתב כן שאחר שהביא דבאה מחמת טענה כופין אותו להוציא בשהתה עמו י׳ שנים כתב ואי תקשי לך אם שהתה י״ש בלא טענתה כופין אותו להוציא ולישא אחרת משום פו״ר לא קשיא דמיירי כגון שיש לו בנים מאשה אחרת או דידענא בי׳ שהוא עקר ומשום טענתה כופין אותו להוציא ולא חיישינן שמא עיני׳ נתנה באחר כיון דשהתה עמו עשר שנים עכ״ל הרי בפי׳ דס״ל להרא״ש דאף דידענו שהוא עקר מכ״מ דוקא משום דשהתה י״ש לא חיישינן שמא נתנה עיני׳ באחר אבל א״ע נלענ״ד דסברת הרא״ש נכונה מאוד ול״ק מידי עליו דתרתי בעינן דכיון דשהתה י״ש עמו לחוד לא סגי מפני הקושיא דא״כ בלא״ה כופין אותו להוציא משום פו״ר כקושית התוספ׳ והרא״ש וע״כ צריך לאוקמי׳ בידענא שהוא עקר א״כ יקשה אפכא איך נאמין לה בטענתה דבעי חוטרא לידה דא״כ מתחלה איך נשאה אותו אחר שידעה שהוא עקר ושלא יהיו לה בנים אע״כ יש ראי׳ שעתה נתנה עיני׳ באחר ולכן ס״ל להרא״ש דאיירי בגם שהתה עמו י״ש דאז יש לתלות שלא ידעה או שלא האמינה שהוא עקר אבל מששהתה י״ש ולא ילדה נתברר לה שלא תלד עוד ולכן טענה עתה וא״כ דוקא בבאה מחמת טענה השיהוי של י״ש מחזיק לטענה מדלא טענה עד עתה אחר י״ש דוקא אבל לענין טמאה אני לך השיהוי של עשר שנים אצלו לא מחזיק לטענתה ולכן גם בשהתה עמו י״ש לא נאמנת והנה כל זה לא שייך רק לענין ידוע שהוא עקר דשם בעינן ראי׳ לטענתה מדנשאה אותו מתחלה אבל לענין השמים ביני לבינך לא שייך זה כיון דזה אי אפשר שידעה קודם נישואין ולכן פריך שפיר לרב המנונא כיון שהיא ובעלה יודעין שאינו י״כ למה אינה נאמנת גם בלא שהתה י״ש דכיון דחזקה דלא משקרת באינו י״כ לא בעינן י״ש להחזיק טענתה דבעי חוטרא לידה כיון דליכא ריעותא כנלענ״ד. הקטן יעקב.\\\vspace{0pt}

\end{multicols}\newpage

\newchap{סימן קמז}
\begin{multicols}{2}
ב״ה אלטאנא, יום ו׳ שושן פורים תר״ט לפ״ק. להרה״ג וכו׳ מ״ה אברהם זוטרא נ״י הגאב״ד דק״ק מינסטער יע״א.\\\vspace{0pt}

מעכ״ת נ״י הורה במה שנסתפק המשנה למלך ובמה שפליגי השלטי גבורים והמהרש״ל אם ביאת יבמה צריכה עדים כדעת המהרש״ל דא״צ עדים ומטעם זה הורה להקל הלכה למעשה במי שבא על יבמתו שלא בעדים שיוציאה בגט ואינה צריכה חליצה – ולבני לא כן יחשוב ובתחלה אומר שכל ראיות של מעכ״ת נ״י לשיטת המהרש״ל לענ״ד אינן מקויימות שמה שהביא ראי׳ ממה דאמרינן הבא על יבמתו במזיד לשם זנות קנה ומי ייחד עדים לשם זנות עכ״ד לענ״ד תמו׳ הוא דמי בעינן שייחד עדים אפילו בקידושין והרי ממה דפליגי ב״ש וב״ה בגטין (דף פ״א) אי אמרינן הן הן עדי ייחוד ה״ה עדי ביאה מוכח דכל שבא עלי׳ לפני עדים קנאה ואפילו לא ייחד עדים לכך ואפילו לא בא עלי׳ בפני עדים אלא נתייחד עמה בפני עדים בלבד קנאה כמבואר באהע״ז (סי׳ ל״ג) ורק לחומרא בעלמא פסק הר״פ הובא בש״ך ח״מ (סי׳ ל״ו) שייחד עדים לקידושין אבל ודאי לכל הפוסקים גם בלא ייחד קידושיו קידושין ולכן כשבא על יבמתו בפני עדים או בנתיחד עמה בפני עדים ואומר שבא עלי׳ לשם זנות קמ״ל דקנה אבל אכתי לא שמענו דגם בלא עדים קנאה ולכן גם מה שהביא מעכ״ת נ״י עוד ראי׳ לשיטת המהרש״ל ממה דאמרינן יבמות (דף קי״א) היא אומרת נבעלתי והוא אומר לא בעלתי יוצאה בגט וא״צ חליצה והרי לא הי׳ עדים בבעילה אע״כ דגם בלא עדים קנה לענ״ד אינה ראי׳ שהרי אפשר בשנתיחד עמה בפני עדים שזה גם לענין קידושין נחשב עדות ושפיר שייך בזה שהוא אומר שלא בעל ומה שהביא מעכ״ת נ״י עוד מקושית הירושלמי שפריך על אשת אחיו יש לה קנס והרי זקוקה היא לו ודלמא איירי בבא עלי׳ בלא עדים פשיטא שאין זה ראי׳ שכבר הרגיש מעכ״ת נ״י עצמו דבלא עדים ליכא למימר שהרי א״כ אין כאן קנס דמודה בקנס פטור וע״כ מיירי שהי׳ עדים ומה שהשיב דלמא איירי שהם לא ראו את העדים לא שייך ע״פ דברינו הנ״ל שהרי אפילו ייחוד בפני עדים מהני לקנות כש״כ בשבא עלי׳ בפני עדים אף שלא ראו אותם ובלא״ה אין ראי׳ מהירושלמי שהרי מה שפשוט יותר לומר דאיירי שנתגרשה מאחיו לא תירץ הירושלמי מפני שמשני בלא״ה שפיר דאיירי בהי׳ בנים לאחיו וכמש״כ התוספ׳ שם כש״כ שלא הוצרך לתירוץ דחוק כזה.\\\vspace{0pt}

עוד ראיתי בתוספ׳ ר׳ עקיבא (ר״פ הבא על יבמתו) שרצה להביא ראי׳ לשיטת המהרש״ל ממה דפריך יבמות (דף נ״ח) שומרת יבם קרית לה אשתו מעליא היא ודלמא איירי שבא עלי׳ שלא בפני עדים אע״כ דגם שלא בפני עדים קנה אכן מלבד שסתר הראי׳ בעצמו דא״כ מנא ידעינן שקדמה שכיבתו לבועל לא ידעתי למה לא הזכיר דעכ״פ לא יהי׳ זה ראי׳ רק לחד פי׳ רש״י שפריך מטעם שקדמה שב״ל אבל לאידך פירושו שתפס עיקר דפריך דאם עדיין שומרת יבם היא לא שותה בלא״ה לא שייך ראיתו דפריך ממנ״פ שפיר דאם בעדים אשתו מעליא היא ובלא עדים עדיין שומרת יבם היא ולא שותת.\\\vspace{0pt}

ואחרי השבתי על כל הראיות לשיטת המהרש״ל אזכיר מה שנלענ״ד ראי׳ נגד שיטתו דהנה כבר הביא השער המלך ה׳ יבום (פ׳ ב׳) ראי׳ מדברי התוספ׳ קדושין (דף י״ב) דצריך עדים רק שמפלפל שם אם לא יש לפרש דברי התוספ׳ שיש ראי׳ להיפך ומעכ״ת נ״י הזכיר ג״כ ראי׳ זו והשיב עלי׳ דודאי לכתחלה בעינן עדים באופן שאין ראי׳ מדברי התוספ׳ הללו אבל באמת תמהני על הגאונים מהרש״ל ומ״ל ושעה״מ איך נעלם מעיני כבודם ראי׳ מפורשת דס״ל להתוספ׳ והרא״ש דבעינן עדים. דביבמות (דף נ״א ע״ב) ד״ה אי כתבו וז״ל ואר״י שהוא ט״ס ולא פירשו רש״י מעולם וכו׳ אבל בעידי טומאה נאסרה בלא קינוי ע״ש וצריך להבין מאי קושיא כ״כ עד שאמר ר״י שט״ס הוא דלמא איירי באמת שלא הי׳ רק עד א׳ בביאת השני אכן בתוספ׳ הרא״ש ביאר קושיתם יותר דע״כ איירי בבא עלי׳ בפני עדים דהא ביאה ביבמה צריכה עדים לקנות כמו ביאה דקדושין ע״ש הרי בפי׳ דפשיטא להו להתוספ׳ והרא״ש כ״כ דביאת יבמה צריכה עדים עד דניחא להו טפי לומר דט״ס הוא ברש״י מלומר דרש״י ס״ל דביאת יבמה ל״צ עדים וגם יש סמך להתוספ׳ דלא ס״ל לרש״י כן דבפ׳ האשה רבה פי׳ באמת כפירושם דאיירי שבא עלי׳ בשוגג אכן גם מהגמרא נלענ״ד להביא ראי׳ לשיטת הש״ג דצריך עדים ממה דאמרינן יבמות (דף קי״א ע״ב) מאן תנא דעד תלתין יומין וכו׳ אר״י ר״מ היא וכו׳ ע״ש והיאך דייק דר״מ היא דלמא פלוגתייהו בודאי איירי דר״מ סבר אפילו בודאי נסתרה בעינן ל׳ יום ור׳ יוסי סבר דודאי נסתרה לאלתר ובודאי לא נסתרה עד לאחר כמה שנין כמו דאיירי באמת הכא כן לחד תי׳ התוספ׳ אבל בספק י״ל דגם רבי יוסי מודה דבעינן ל׳ יום ודלמא במתניתן בספק נסתרה איירי ואתיא שפיר גם כר׳ יוסי אע״כ מוכח כשיטת הש״ג דלעולם בעינן עדי ביאה או עדי ייחוד בייבום כמו בקידושין ולכן דייק שפיר דע״כ מתניתן איירי בנסתרה ודאי דאל״כ אפילו נבעלה לא יצא ידי יבום ולא נפטרה וא״כ מכל זה מוכח לא בלבד דלא אמרינן כשיטת המהרש״ל דלא בעינן עדים כלל אלא מוכח ג״כ נגד מה שכתב הב״ש (סי׳ קס״ו) שמשמע מהרמ״א דאם עשה מאמר בעדים ל״צ ביאה בעדים ע״ש אלא לעולם לא נקנת רק בביאה בעדים וביבום בלא עדים לא נפטרה בלא חליצה. כנלענ״ד, הקטן יעקב.\\\vspace{0pt}

\end{multicols}\newpage

\newchap{סימן קמח}
\begin{multicols}{2}
ב״ה אלטאנא, יום ד׳ כ׳ טבת תרט״ו לפ״ק. להרה״ג וכו׳ מ״ה יעקב נפתלי לעמאן נ״י הגאב״ד דק״ק נימוועגען יע״א.\\\vspace{0pt}

טרם אחלה לחרוט בחרט אנוש מה שנלפענ״ד בענין השאלה אודות היבמה הנשואה לזר בלא חליצה אם יש לה תקנה שתקבל חליצה בעודה תחת בעלה ותהי׳ מותרת לו מאז והלאה – אודיע למעכ״ת נ״י כי עיקר המעשה אשר גרמה הרעה הזאת כמוה לא תעשה בישראל, ואשר הודיע מר נ״י לי רק ברמז, נודע לי באר היטב ע״י בעל המעשה בעצמו, הוא המורה רעה יושב ביהמ״ד בק״ק פלוני יע״א, כי האיש הזה כתב לי בקיץ העבר פעם אחר פעם, ובקש ללחצני שאכניס ראשי למעשה הרע אשר עשה, ולא טרם הוראתו אלא אחר שכבר סידר הקידושין להזר כמעשה עבדין בישין דבתר דעבדין מתמלכין ולא השבתי לו, ובאמת ראוי לקרוע כי כך עלתה בימינו שאנשים כאלה יחרף לבם להתיר דברים שבערו׳ החמורין על היתרים בנויים וארוגים מהבל וריק כאשר יראה כל קורא שיבושיו.\\\vspace{0pt}

והנה ממכתב מעכ״ת נ״י נראה שגם אצלו ברור שאין ממש בדברי ריק ותוהו של המתיר להיות אפילו סניף להתיר היבמה לבעלה, אכן כמדומה כי גודל הרעה לחצהו להמעיט מסיבותיו אשר על כן לפענ״ד נכנס קצת בדחוקים להמציא תרופה להזוג ולכן אף כי דברי מעכ״ת נ״י דברי חכמה ודעת הם לא יכולתי להסכים עם היתרו ואפרש שיחתי.\\\vspace{0pt}

בתחלה אזכיר כי שאלה מענין זה כבר הוזכרה בשו״ת כנסת יחזקאל סי׳ ס״ז רק ששם הי׳ הענין שהיבמה נשאת לזר ע״פ עדות שני עדים כשרים שהעידו בעדות גמורה שמת יבמה ואח״כ בא היבם ורצה להתיר היבמה לבעלה אחר חליצה, ועיקר היתרו הי׳ בנוי על שהעידו שני עדים כאשר יראה המעיין, ואף זה לא רצה רק באימה וביראה אם יסכימו עמו גדולי ישראל, ולא נודע אם רצו להסכים כי נפרד הזוג בלא״ה. ואם כי מאורי ישראל אפילו בענין עדיף מנדון שלפנינו לא רצו לסמוך על היתרם, אנו מה נענו אבתרייהו, ובפרט כי הענין ההוא מענין שלפנינו מוחלק בב׳ דברים – האחד – כי עיקר היתרו לא הי׳ רק מפני שנשאת על פי ב׳ עדים שהיתה אנוסה גמורה, ובב׳ עדים אפילו באשה לבעלה לא גזרו רוב הפוסקים וזה לא שייך כאן – והב׳ – כי הוא לא רצה להתיר החליצה רק אחר שנתגרשה מבעלה מטעם דכל שאינו עולה לייבום אינו עולה לחליצה והוציא כן מלשון הרמב״ם וש״ע במקדש יבמה ואף דמדבריו נראה דרק מטעם כיון שאפשר שתהא החליצה בהיתר כתב כן כמו שכתב רש״י לענין ספקות לענ״ד בנדון זה החליצה קודם גירושין אינה חליצה כלל דלפי המבואר בסוגיא דיבמות (דף כ׳) הא דחייבי כריתות פטורין מחליצה אתיא מואם לא יחפוץ וכו׳ כל העולה לייבום וכו׳ ורק חייבי לאווין צריכין חליצה כיון דמדאורייתא עולין לייבום וכן אלמנה לכה״ג אף שמדאורייתא אינה מתייבמת ילפינן דעולה לחליצה מיבמתו וכן העלה הרשב״א שם ולכן בכל מקום דאמרינן שעולה לחליצה אף שאינה עולה לייבום הוא רק היכי שמדאורייתא רמיא לייבום ואסורה מדרבנן בזה בת חליצה היא או אפילו היכי שמדאורייתא אינה בת ייבום אבל אינה אסורה רק בל״ת וע׳ דלענין זה רבי רחמנא יבמתו אכן כשהיא חייבי כריתות אז וודאי מן התורה אינה בת חליצה כלל מטעם דואם לא יחפוץ כל העולה וכו׳ וכמש״כ הרשב״א הנ״ל דאל״כ יקשה מנ״ל דט״ו עריות פטורין מחליצה שהרי מה שמיעטן התורה מייבום ע״י דרשה דעלי׳ מזה לא ידענו רק שלא תתיבמנה ומנ״ל שאינן צריכות חליצה וזה ע״כ אתי מואם לא יחפוץ וכדאמרינן (יבמות דף ג׳) והשתא הנשואה לזר שאשת איש היא היאך תהני לה חליצה ולכן כתבו הרמב״ם והש״ע שתתגרש תחלה ואח״כ תחלוץ ולא משום חליצה בהיתר לכתחלה כנראה קצת דעת כנסת יחזקאל אלא מפני שחליצת אשת איש לאו חליצה כלל. והנה ראיתי שגם מעכ״ת נ״י הרגיש בחשש זה דכל שאינו עולה לייבום ודחאו בקש שמסוגיא ביבמות (דף צ״ב) ממה דקאמר אם הי׳ יבמה כהן וכו׳ מוכח שאין זה חשש וכן מוכח שם בתוס׳ עכ״ד ולא ידעתי שום ראי׳ מגמרא דשם דהא יש לפרש חולץ לה אחר שגרשה בעלה והראי׳ מהתוספ׳ כבר דחאה בשו״ת כנ״י הנ״ל שוב אחר כתבתי זה ראיתי בשו״ת שב יעקב סי׳ מ׳ שעמד על שו״ת כנ״י ממה שכתב הרמ״א (סי׳ קנ״ט) דצריך לחזור ולקדש אותה ואם החליצה אינה רק אחר הגירושין מאי קמ״ל פשיטא עכ״ד ולענ״ד אין זה ראי׳ דהרי עכ״פ צריך להשמיענו אפילו אם בדיעבד חלץ לפני הגירושין צריך לחזור ולקדש אבל אכתי לא שמענו דלכתחלה ג״כ יכול לחלוץ לפני הגירושין נגד משמעות הרמב״ם וטוש״ע. גם הביא השב יעקב שם בעצמו דברי הלבוש שכבר חשש לחשש זה מטעם דכל שאינו עולה לייבום וכו׳ לכתחלה עכ״פ לגרש קודם חליצה וגם השב יעקב לא חלק לענין לכתחלה רק בנדון שם שהיו רק ספק יבמה לשוק שוב נאמר לי שבספר בית מאיר כתב ג״כ שאין לחוש לגרש לכתחלה והביא ראי׳ ממש״כ הרשב״א ריש יבמות ולענ״ד אדרבא מדברי הרשב״א האלה יש ראי׳ להיפך שהרי כתב שם שנקראה בת חליצה שאין איסורה עכשיו רק מחמת תנאו ולא מחמת ערוה שהרי גיטה גט גמור ע״ש וזה לא שייך רק שם במגרש על תנאי הרי עכ״פ נראה שהיכי שהיא אשת איש גמורה גם להרשב״א אינה בת חליצה (אכן גם ראית רשב״א דאם דבר אחר גרם לה בת חליצה היא מנודרת הנאה לא הבנתי ע״פ מה שכ׳ הרשב״א הנ״ל (דף כ׳) בעצמו דהיכא דאינה עולה לייבום מכח ל״ת וע׳ גלי רחמנא מיבמתו דאעפ״כ עולה לחליצה משא״כ חייבי כריתות וא״כ מה ראי׳ מנודרת הנאה דשם בת חליצה היא מן התורה משא״כ בשהיא אשת איש) ועוד דאפילו בכה״ג שדבר אחר גרם לא סמך הרשב״א על דעתו שהביא אח״כ דעת הרמב״ן דגם בהם אני קורא כל שאינה עולה לייבום וכו׳ וסיים ואין דעתינו מכרעת במקומו עכ״ל גם מדברי התוס׳ יבמות (דף י׳) נראה כן שאם היא אשת איש אפי׳ אם בשעת נפילה עדיין לא היתה מכ״מ אינה בת חליצה וגם אין לומר דממנ״פ תועיל החליצה קודם גירושין דאם אין קידושין תופסין ביבמה לשוק הרי אינה מחייבי כריתות ואם קידושין תופסין הרי פטורה מן החליצה ומן הייבום ופקע זיקה לגמרי דזה אינו דלפי מה שכתבו התוס׳ יבמות (דף ט״ז) ד״ה בני צרות לא פקע זיקה ע״י קידושין כיון שאפשר בגירושין להתייבם ע״ש ולכן תמהתי על מה שנראה מרמ״א ומלבוש ושאר אחרונים דעכ״פ בדיעבד מהני חליצה קודם גירושין דלענ״ד מהראשונים שהזכרתי נראה שאפי׳ מדאורייתא לא מהני וכמו שנראה גם מרמב״ם וש״ע בסי׳ קנ״ט שהצריכו גט קודם חליצה אכן מי שירצה לסמוך על דברי האחרונים עכ״פ צריך לכתחלה גירושין קודם כנראה מהלבוש וגם השב יעקב רק בנדון שלו שהיתה ספק יבמה לשוק לא חשש לזה וא״כ בנדון שלפנינו אפילו יהי׳ היתר ע״י חליצה עכ״פ יהי׳ חשש שלא יחלוץ לכתחלה קודם גירושין דלזה אפי׳ חוק המדינה שלא לגרש אינו מתנגד שהרי החוק אינו מוחה מלגרש רק שהגירושין אין להם כח להוציא אשה מתחת בעלה והנה כל מה שדברו הפוסקים הנ״ל הוא במקום שהותרה לו ע״י חליצה דהיינו בקידש לבד אבל בנשא לכל הפוסקים לא הותרה לו. אכן מעכ״ת נ״י חתר למצוא לזה היתר מחמת טעמים שפרט ולענ״ד אין לסמוך עליהם להתיר על ידם איסור המפורש במשנה בגמרא ובפוסקים.\\\vspace{0pt}

א – רצה מעכ״ת נ״י לומר אחר שבירושלמי נמצא אמורא דס״ל זה חולץ וזה מקיים א״כ נראה שהאיסור לא הי׳ נתפשט בימיהם ותמהתי איך אפשר לקרוא לזה לא נתפשט האיסור שפשוט במשנה בגמרא ובפוסקים מפני דיעה יחידה ועוד הרי כבר כתב הרי״ף דמסתברא דגם ר׳ ירמי׳ דס״ל זה חולץ וזה מקיים לא ס״ל כן אלא ביש לו בנים דמסתמא ר׳ ירמי׳ לא חולק על סתם משנה וא״כ ליכא חולק בזה כלל באין לה בנים כנדון דלפנינו וגם אפי׳ לא הי׳ נתפשט בימיהם עכ״פ בימינו נתפשט דסברת ר׳ ירמי׳ כבר נשתקע והאיך נאמר דאם הטעם בטל גם הגזירה בטלה גם מה שפשוט למעכ״ת שהטעם לא שייך רק במקום שמתייבמים ולא אצלינו לא נלענ״ד ולא ידעתי האיך תלי זה בייבום דמה שדומה יבמה לאשת איש אינו ע״י הייבום אלא על ידי דאגידא בהייבם ולא מותרת להנשא לאחר עד שתחלוץ זה עושה אותה דומיא דאשת איש שלא מותרת לשוק עד שתתגרש א״כ איך תלה זה בייבום ולכן לענ״ד אין בזה אפילו סניף להתיר.\\\vspace{0pt}

ב – הטעם שבנה מעכ״ת נ״י היתרו עליו מחמת שע״פ חוק המדינה לא נתן רשות לגרש עד אחר כמה שנים וטורח רב וכיון דהוא עדיין לא קיים פרי׳ ורבי׳ א״כ דומה זה למה שהתירו לשחרר חצי עבד משום מצוה רבה דפו״ר עכ״ד לפענ״ד אין לזה דמיון כלל מחמת כמה טעמים: – הא׳ – אין לדמות גזירה דרבנן לאיסור דאורייתא דהרבה פעמים עשו חיזוק לדבריהם יותר משל תורה כדאמרינן יבמות (דף כ״ה) ובכמה דוכתי׳ ובענין זה בעצמו מצינו כמה החמירו חכמים לחזק גזירותיהן שהרי לענין איסור אשת איש החמור סמכו אאשה דייקא ומנסבא להתירה ע״פ עדות עד א׳ מפי עבד ומפי שפחה משום עגונה ובהלכה צרתה למדינת הים צריכה להתעגן כל ימיה עד שתדע ע״י עדות ברור שלא ילדה צרתה ולא מתירין אותה ע״י חליצה משום גזרות הרחוקות דשמא יהי׳ הולד בן קיימא ונמצא אתה מצריכה כרוז לכהונה ודלמא איכא דהוה בחליצה ולא בהכרזה וקאמר חלוצה שריא לכהן כדאמרינן יבמות (דף קי״ט) הרי דהחמירו הרבה פעמים יותר מבאיסורי תורה והאיך נסמוך לדמות שלא לחוש לגזירה דרבנן במקום שהתירו איסור תורה. – ב׳ – האיך נסמוך על התוספ׳ דמשום פו״ר הותר לעבור על העשה דוהתנחלת׳ אחר שיש עוד תירוצים אחרים על קושיתם גטין (ד׳ מ״א) דהרשב״א תירץ דבחצי עבד ליכא עשה (ומה שקשה עליו מהא דחצי׳ שפחה דפ׳ השולח תרצתי במ״א) והר״ן תירץ שמה שעושה משום מצוה ליכא עבירת ע׳ דליכא רק משום לא תחנם א״כ האיך נדון דגם איסור גמור נתיר בקום ועשה משום ביטול פו״ר בשב ואל תעשה. – ג׳ – שם לא עובר על עשה דשחרור רק פעם א׳ דמיד כששחררו נפסקה העשה דוהתנחלתם אבל הכא בכל יום ויום ובכל ביאה וביאה עובר על גזירה דרבנן ומאן לימא לן דאיסורים כזה ג״כ יש להתיר משום פו״ר. – ד׳ – שם מתירין לו דווקא קודם שקיים פו״ר וא״כ יש טעמא להתיר משום קיום פו״ר אבל הכא שאמרו חכמים להוציא א״כ לעולם בעמוד והוציא קאי ואם נתיר לו הרי תשאר אצלו גם לאחר שהוליד בן ובת וקיים פו״ר – ה׳ – הרי לפי דברי מעכ״ת נ״י לא הוחלט ע״פ חוק המדינה שלא יגרש אלא שיש בזה טורח רב ושיהוי כמה שנים וא״כ אין בזה רק שיהוי מצוה דפו״ר ואיך נדמה זה להך דעבד דאם לא משחרר יש ביטול מצוה לגמרי לכן לענ״ד אין לסמוך בזה להתיר ומה שמזכיר מעכ״ת נ״י ממה שדן מר אביו הגאון זצ״ל מטעם חוק מלכות כדיעבד לא ידעתי איך מצא מקום בזה לסמוך על היתרו דהיכא שיש חילוק בין לכתחלה לדיעבד לא הוחלט האיסור אבל בנדון זה הרי כל הענין הוא דיעבד שגזרו בו י״ג דבר וא״כ הוחלט האיסור גם על הדיעבד ובלאו כל הנ״ל לענ״ד אין לסמך על היתר זה כאשר העיר כבר מעכ״ת נ״י בעצמו שהרי יכול לילך למדינה אחרת ומה שהשיב על זה דאין זה דרכי נועם לענ״ד לא שייך בזה דזה לא מצאנו רק כגון אם ע״י מצות התורה לא יהי׳ דרך נועם כמו בהך דיבמות (דף פ״ז) אם תהי׳ זקוקה אחר מיתת בנים או כדאמרינן בסוכה (דף ל״ב ע״א) לענין כופרא ושם (ע״ב) לענין הירדוף אבל הכא מצד מצות התורה יש דרך נועם שיכול לישא אשה אחרת במקומו ורק מצד חוק המלכות אינו דרך נועם שמזקיקו לעזוב מקומו וכן חלקו התוס׳ יבמות (דף ב׳) דומה לזה אם מצד מצוה אינו דרך נועם או מצד דבר אחר יע״ש גם מה שהביא מעכ״ת נ״י ראי׳ לזה ממה דכופין הרב לשחרר העבד ולא כופין לעבד לעלות לא״י לא ידעתי מה ראי׳ משם וכי ע״י שברח לא״י נפסק מלהיות עבד הרי רק משום לא תסגיר אין משיבין אותו ואם רבו ילך לשם נשאר עבדו והרי הרב גם שם צריך ליתן לו גט שיחרור וכותב לו שטר על דמיו כמו בח״ל וא״כ יש גם בזה ביטול מצות ע׳ ורק אם לא ירצה לשחררו ב״ד מפקיעין שעבודו וע״כ גם בזה יש ביטול עשה דלעולם בהם תעבודו דאל״כ למה בתחלה אומרים לאדון לשחררו ולעבור על ע׳ כיון דיש תקנה ע״י ב״ד בלא ביטול ע׳ רק שבא״י משום לא תסגיר צריך לבטל העשה כמו בח״ל משום ביטול פו״ר.\\\vspace{0pt}

ג – עוד סמך מעכ״ת היתרו על חוק המלכות כיון שאין יכול להוציאה מביתו מה יעשה הבן שלא יחטא ודומה למה דאמרינן שבת (דף ג׳) שהתירו לו לרדות קודם שיבוא לידי איסור סקילה ולמה שכתב המהר״ם מינץ סי׳ ק״ה בזקוקה ליבם מומר עכ״ד הנה לא ידעתי אם חוק המלכות שלא להוציאה מביתו כי כמדומה ששמעתי (בארץ מולדתי שגם שם חוק המלכות שלא לגרש רק ע״פ רשיון) שיכולים להיות מפורדים זה מזה רק שבמשך ג׳ שנים בכל רביע השנה יבואו לפני השופט לשמוע אם עוד עומדים בדעתם להיות נפרדים זה מזה ואז יהי׳ הפירוד לגמרי ע״פ חוק המלכות ויותר לאיש ואשה להנשא לאחרים ואם גם במדינתו הענין כן א״כ אין בזה חשש ולנדון המהר״ם מינץ פשיטא שאינו דומה ששם לא הי׳ תקנה לאשה כלל אפי׳ אחר גירושין כיון שהיתה זקוקה לעולם אבל כאן אם תתגרש תהי׳ מותרת לשוק לכל אדם אחר החליצה מיבמה ועוד כל זה לא יהי׳ טעם רק שתחלוץ למען הצילה מאיסור יבמה לשוק אבל מאיזה טעם נוכל להתירה אחר חליצה שתשאר תחתיו והנה מעכ״ת נ״י כבר בעצמו הרגיש בזה וכתב שאין אדם רוצה שתתבזה אשתו שתחלוץ ולא יהי׳ לה היתר ותמהתי מה לנו ברצונו שלא תתבזה אשתו אשר ע״פ הדין לא תהי׳ אשתו עוד שמחוייב לגרשה ואדרבא תתגנה ותתגנה כדאמרי׳ יבמות (דף צ׳) ועוד כתב מר נ״י שיש זילותא דבי דינא אם תחלוץ ולא יהי׳ לה היתר ולא ידעתי מה זילותא דבי דינא הוא אם פוסק כדין המשנה והש״ע שצריכה חליצה ואסורה לבעלה. עוד כתב ועוד שע״י מעשה ב״ד כזה שמקפידים וחשים כ״כ על דברי תורה יתנו יד לפושעים לזלזל על דברי חז״ל עכ״ל והם דברים תמוהים בעיני וכי למען לא יזלזלו הפושעים בדברי חז״ל נעבור אנחנו עליהם ולא נחוש לדין המשנה הלא על זה נאמר ישרים דרכי ד׳ צדיקים ילכו בם ופושעים יכשלו בם ולענ״ד בדור הפרוץ הזה שרבים מהמורים הרשעים מורים את העם שאין עוד זיקת ייבום באשר שע״פ חוק המלכות לא מותר ליבם ושעל כן ל״צ היבמות חליצה ביותר יש לחוש שלא להמציא היתר כזה להתיר היבמה הנשאת לזר ע״י חליצה בלא גירושין שאם יראו שיש תקנה כזה ירבו המתפרצות הנשאות לשוק בלא חליצה אם בקל לא ישיגו חליצה כגון שהיבם במרחק מה ששכיח בזה״ז או יבקש מהן ממון ומה שיש תקנה לזו הוא קלקול לאחרים ומה לנו לחוש לתקנת הזוג החוטאים בנפשותם שלא שמעו לקול מעכ״ת נ״י המורה להם איסור והלכו אחר פסק המורה רעה שכאשר שמעתי נתנו לו ממון הרבה בעד היתרו א״כ כבר ידעו שאין היתרם נכון ורק למראית העין בקשוהו ועל כן עכ״פ קרוב למזידים הם.\\\vspace{0pt}

סוף דבר לענ״ד לא יכולתי למצוא תקנה להזוג להתירם שהחליצה לא מהני קודם גירושין וצריך לגרשה ואח״כ תחלוץ ותהי׳ אסורה לו לעולם רק שבאם אי אפשר להביאו לכך שיגרש מקודם ואז תשאר תחתיו בלא חליצה לענ״ד טוב שיאמרו להם שטרם שיעיינו ב״ד בהיתר נישואיהן צריכה עכ״פ לחלוץ שהרי מטעין לחליצה ואח״כ נוכל לומר שלא מצאנו היתר ואם אעפ״י כן לא ירצה לגרשה ולא ישמעו לקול מורים עכ״פ יצאו מידי איסור יבמה לשוק לדעת הפוסקים הנ״ל דמהני חליצה קודם גירושין ויהא קולר עבירת איסור דרבנן תלוי בצואריהם ואנחנו נקיים. זה מה שנלפענ״ד, הקטן יעקב.\\\vspace{0pt}

\end{multicols}\newpage

\newchap{סימן קמט}
\begin{multicols}{2}
ב״ה אלטאנא, יום ד׳ כ״א אייר תרט״ו לפ״ק. עוד להרב הנ״ל נ״י.\\\vspace{0pt}

מכתב מעכ״ת מיום י״ג ניסן העבר קבלתי לנכון ולא מצאתי על ידו מקום לנטות מפסקי הראשון כי מה שחידש מעכ״ת נ״י בתשובתו שנית בענין זה יסוב על ב׳ דברים: – א׳ – שאחר שעתה נתברר לו כי ע״פ חקי המלכות לא יכול לגרש כלל א״כ לא יהי׳ תקנה עולמית להאיש לקיים פו״ר. – ב׳ – שאחרי שבמדינתו נהגו להקל באותם שנשאו נשים נכריות להתיר להם לקיימן לאחר גירותן אף שעברו בזדון לבם בתחלה כש״כ באילו שעשו ע״פ מורה שהורה להם כן. ואשיב על ראשון ראשון כי לענ״ד שאף אם ע״פ התחדשות הזה אזיל חדא מטעמים שכתבתי שאין בזה ביטול פו״ר אחר שיכול להמתין עד שיותן רשות לגרש מהשררה מכ״מ הלא לא על זה לבד סמכתי אלא על הטעם שיכול לילך למדינה אחרת וכי מפני שקשה בעיניו לעשות כן נתיר לו איסור דרבנן גמור ומלבד שאר הטעמים שהזכרתי בכתבי הראשון עוד נגד היתר דביטול פו״ר לענ״ד גם יש מקום עיון שאדרבא אפכא מסתברא אחר שכבר האשה הזאת היתה נשואה לשני אנשים כמה שנים ולא ילדה א״כ יש לחוש עכ״פ שלא תלד עוד ואף שע״פ הדין לא תחשב שלא תלד עוד עד ששהתה אצלו י׳ נשים מכ״מ יש לחשוב כריעותא שלא להקל באיסור דרבנן על מצות פו״ר שיתקויים על ידה ועוד גם אם עתה נתיר לו שתשהה אצלו משום פו״ר מה תהי׳ עלי׳ כששהתה עשר שנים ולא ילדה כיון דבזה מן הדין תחשב לחזקה שלא תלד אצלו עוד כמבואר באהע״ז (סי׳ א׳ וסי׳ קנ״ד) ומה שאין כופין עתה לגרשו הוא בלבד מפני שאין כופין עתה על פו״ר כמבואר שם (סי׳ א׳) אכן אז תהי׳ אסורה לו מצד נישואי׳ בלא חליצה כיון שעתה אזיל היתר דפו״ר אחר שנראה שלא תלד אצלו ולכן מצד פו״ר לענ״ד אין למצוא היתר.\\\vspace{0pt}

ומה שרצה מעכ״ת נ״י להוכיח שיש להקל על מה שמקילין בנשואה לו בנכריותה שתשאר אצלו לאחר גירות לענ״ד אין למצוא גם מזה היתר. בראשון, לו יהי כן שדומה לזה נימא ערבך ערבא צריך וכי מפני שמקילין בזה בלא טעמא נקל גם בענין אחר אדרבא הבו דלא לוסיף עלי׳. בשנית, אין זה דומה לשם דשם האיסור רק בשעת נישואין דמן הדין אין לישא אותה אבל אם נשא הרי כיון דפסקינן כמ״ד כלם גרים הם אמרינן ביבמות (דף כ״ד) דאם נשא אין מוציאין מידו ולכן המקילין לא מקילין רק באיסור הנעשה פעם א׳ אבל באשה הנשואה לשוק דמן הדין מוציאין מידו א״כ בכל יום שתשהה אצלו עוברין על איסור דרבנן כל ימי חייהם ומנ״ל להקל גם בזה ועוד דבלא״ה איסור דאינו רק לכתחלה לא חמור כ״כ כאיסור שאסרו חכמים גם בדיעבד שהרי מה שלא אסרו רק לכתחלה הוא מפני שאיסור קל הוא ועוד נלענ״ד שבנכרית שבא עלי׳ ודאי באיסור כגון שיש לה בנים ממנו ונתגיירה יש סברא להקל יותר מבנטען על הנכרית בספק ונתגיירה כיון ששם הטעם דלא תנשא לכתחלה משום לזות שפתים הרחק ממך שלא להחזיק הלעז והרי בזו הלעז כבר הוחזק וכיון שאפילו בא עלי׳ ודאי מכ״מ בעבר ונשא אין מוציאין מידו כמו שכ׳ הב״ש (סי׳ י״א) בשם הנימוקי יוסף יש ללמד זכות בזה על המקילין במדינתו דמר נ״י להתיר לו לישא אותה לכתחלה אף שדעתי לא נוחה בזה וכבר נשאלתי על זה ואסרתי ואע״פ שמכל מקום גם בזה יש איסור לקבלה לגיורת אחר שנראה לנו שלא נתגיירה רק משום נישואין מכ״מ גם בזה יש ללמד זכות על המקילין שסומכין עצמם על מה שכ׳ הב״י בי״ד (סי׳ רס״ט) הביא הש״ך שם דאם לפי ראות עיני ב״ד סופו שיהי׳ גר לשם שמים מקבלין אותו אף שעתה עושה משום דבר אחר והביאו ראי׳ מהלל שקבל הגר שבא להתגייר ע״מ שיעשה כהן הגדול ואע״פ שגם כזה לא נחה דעתי וכבר השבתי על ראית הב״י ואכ״מ מכ״מ יש מקום סמיכה להמקילין שהוא דבר התלוי בעיני הדיינים אבל באיסור יבמה שנשואה לשוק שנאסרה בדיעבד אין להביא ראי׳ משם להקל כנלענ״ד, הקטן יעקב.\\\vspace{0pt}

\end{multicols}\newpage

\newchap{סימן קנ}
\begin{multicols}{2}
ב״ה אלטאנא, יום ג׳ י״ג שבט תר״ח לפ״ק.\\\vspace{0pt}

ראיתי לחקור במה דאמרינן יבמות (דף כ״ד) מצוה בגדול לייבם ויליף מוהי׳ הבכור איך הדין בבכור שאינו גדול האחים וזה מצינו ע״פ מה שכתב האגור מגמרא דערכין (דף ל״א) ונפסק כן בא״ח סי׳ נ״ה דנער שנולד בכ״ט לאדר ראשון ואחד נולד בא׳ לאדר שני ונעשו גדולים בשנה פשוטה אז הנולד ראשון הוא קטן עד כ״ט והנולד אחרון נעשה גדול בא׳ אדר ולפ״ז בשני אחים תאומים שנולדו בכזה שהבכור יצא סוף אדר ראשון ואחיו יצא תחלת אדר שני ונעשו בני י״ג בשנה פשוטה אז כל חדש אדר הבכור עדיין קטן ואחיו הנולד אחריו כבר נעשה גדול ויש לספק אי נימא דמצו׳ על הגדול שנעשה גדול קודם כמו שמשמע מלשון מצו׳ בגדול לייבם או נימא כיון דיליף מוהי׳ הבכור לכן גזיה״כ הוא דהבכור קודמו וצריך להמתין עד שיגדל גם הבכור וייבם או יחלוץ ואפילו את״ל כיון דתנן מצוה בגדול לייבם הכל תלוי בגדול ואין משגיחין בבכור אכתי יש לספק אי נימא דרק באותו חדש שעדיין הבכור קטן הגדול קודמו או אי נימא כיון שנעשה הוא גדול קודם נשאר עליו שם הגדול בכל שנותיו שיקדים לייבום וחליצה והי׳ נלענ״ד לפשוט ספק זה ממה דפריך ביבמות (שם) ואלא למאי הלכתא כתבה רחמנא בכור ופי׳ רש״י דלכתוב גדול ע״ש ומאי קושיא דלמא לכך כתיב בכור ולא גדול להשמיענו שהבכור שנולד ראשון יקדים לגדול שנעשה גדול מקודם אע״כ דפשיטא להגמרא דגם אי הוי כתיב גדול לא הוי קרינן כן אלא למי שנולד ראשון ולא למי שנעשה רא... גדול בשנים ולכן גם במה דתנן מצו׳ בגדול לייבם אמרינן כן אכן כעת מצאתי בשו״ת הלכות קטנות (ח״ב סי׳ קע״ד) שעמד ג״כ בספק זה וכתב היפך מדברי וז״ל אע״פ שהראשון בכור לכהן אבל אינו בדין והי׳ הבכור אשר תלד כי אין לנו אלא דברי רבותינו דבגדול האחים הכתוב מדבר והצעיר הוא הגדול עכ״ל ולענ״ד צ״ע שמהראי׳ שכתבתי מוכח להיפך דהבכור הוא הגדול ואפילו לפי סברתו שהצעיר הוא הגדול מכ״מ מה שנראה לו פשוט שהצעיר הוא הגדול לעולם לפענ״ד אפילו מצד הסברא לא נראה כן כיון דבכל שנה מעוברת הבכור שנולד ראשון נכנס לשנה שאחריו קודם השני איך יקרא השני שנולד אחריו והצעיר גם בשנים אז לעולם גדול האחים אבל לפי מה שהוכחתי אפילו באותו החדש שעדיין הבכור לא נעשה גדול בשנים והנולד אחריו כבר נעשה גדול מכ״מ הבכור הוא הגדול ולדינא צ״ע. כנלענ״ד, הקטן יעקב.\\\vspace{0pt}

\end{multicols}\newpage

\newchap{סימן קנא}
\begin{multicols}{2}
ב״ה אלטאנא, יום ג׳ כ״ג א״ש תר״ח לפ״ק. להרה״ג וכו׳ מ״ה גבריאל אדלער הכהן נ״י הגאב״ד דק״ק אבערדארף יע״א.\\\vspace{0pt}

על דבר חקירתי בענין מצוה גדול לייבם שמחתי במה שהודיעני מעכ״ת נ״י שכוונתי לדעת השבות יעקב חלק ראשון סי׳ ט׳ שחולק ג״כ על הלכות קטנות ופוסק כפי אשר הי׳ נראה לענ״ד דלענין יבום הולכין לעולם אחר הגדול שנולד ראשון אכן מה שהשיב מעכ״ת נ״י על ראיתי ממה דפריך ואלא למאי הלכתא קרי׳ רחמנא בכור דאפשר לומר וליטעמיך דלמא אשמעינן קרא דהנולד ראשון הוא המיבם לא הבנתי שהרי זה הדבר אשר דברתי דע״כ פשיטא להש״ס דגם אי הוי כתיב גדול הוי דיינינן כן ולכן פריך דלא ה״ל לכתוב בכור דאפשר למטעי דדוקא בכור מייבם ולא פשוט.\\\vspace{0pt}

ועל דבר הספק של מעכ״ת נ״י במי שנולד ביום א׳ דר״ח אדר בש״פ ונעשה ב״מ בשנת העבור לענ״ד נראה פשוט דלא נעשה ב״מ עד א׳ דר״ח אדר שני וכמו שנוטה לזה גם דעת מעכ״ת נ״י שממה שכתב הש״ע (בסי׳ נ״ה) דינו במי שנולד בכ״ט אדר ראשון ואחר באחד לאדר שני ונעשו ב״מ בשנה פשוטה שאז מי שנולד בכ״ט צריך להמתין עד כ״ט ומי שנולד באחד לחדש נעשה ב״מ באחד לחדש משמע בפי׳ שאם הי׳ נולד הראשון בל׳ לאדר ראשון אז פשיטא דנעשה ב״מ בשנה פשוטה ביום ראשון דר״ח אדר ולא אמרינן דלא נעשה ב״מ עד ל׳ לאדר שהוא ר״ח ניסן אע״כ דשני ימי ר״ח לא חשבינן רק כיומא אריכתא וכמשכ׳ הח״מ וכן מוכח ג״כ ממה שכתב הר״י מינץ בתשובה (סי׳ ט׳) להוכיח שמי שנולד באדר בש״פ לא נעשה ב״מ עד אדר שני וז״ל כיון דאדר ראשון הוא ודאי חדש העבור מסברא יש לנו לומר אלו נולד בניסן ובשבת י״ג יהי׳ עבור הדבר פשוט דאמרינן דבניסן יהי׳ ב״מ ולא באדר הכי נמי נימא דבאדר השני יהי׳ ב״מ ולא באדר הראשון שהוא חדש העבור ולא נקרא אותו אדר אלא שבט ואלו קרא אותו שבט פשיטא שהיינו אומרים שהי׳ ב״מ באדר ה״נ ל״ש וכי בשביל קריאת השם שקראו לו אדר נכניס זה להיות בר עונשין ח״ו ולא נאמר נתעברה שנה נתעברה לו עכ״ל הרי שפשיטא לו דאדר שני הוא עיקר אדר ואדר ראשון דיינינן לי׳ לענין ב״מ כשבט א״כ גם אם נימא דמי שנולד ביום א׳ דר״ח אדר חשבינן לנולד בל׳ דשבט מכ״מ לא הגיע יום זה עד א׳ דר״ח אדר שני כנלענ״ד, הקטן יעקב.\\\vspace{0pt}

\end{multicols}\newpage

\newchap{סימן קנב}
\begin{multicols}{2}
ב״ה אלטאנא, יום ג׳ ה׳ ניסן תר״ח לפ״ק. להרה״ג וכו׳ מ״ה שמעון נאדאש נ״י הגאב״ד דק״ק ציפער יע״א.\\\vspace{0pt}

על דבר מה שכתב מעכ״ת נ״י בענין ספקתי אם הבכור או גדול האחים קודם וז״ל לא אוכל לצייר לי ספקתו של מעכ״ת נ״י על אותו חודש הלא ביבמות (דף כ״ד) מבואר דבכור דכתיב בקרא דוקא בכור לנחלה ולא בוכרא דאמא וכ׳ שם התוס׳ רא״ש וז״ל ובכור לאם לית לי׳ צד מעליותא ליבום א״כ אם הראשון בכור לאב לנחלה ועתה בעת היבום עדיין קטן הוא ע״כ האח שמת שנולד אח״כ קטן הי׳ בעת שמת וקטן שקידש אין קדושיו קידושין וא״צ יבום כלל ובבכור מאם אינו מעלה ואינו מוריד ואפילו לר״י ברזלי בתשו׳ מהרי״ק הביאו המשנה למלך פ״ו מה׳ גרושין דאם קיבל האב קדושין בעד בנו הקטן חוששי׳ לקדושין דאורייתא מ״מ אם מת בקטנותו נראה דא״צ יבום דהא לא הי׳ ראוי להוליד עדיין דקטן אינו מוליד ואם כוונת מעכ״ת נ״י מצד שהראשון נולד קודם הוה גדול האחין (כמו שמסופק באמת בתשובת שבות יעקב חלק א״ח סי׳ ט׳ באחד שהי׳ לו ב׳ נשים במקומות שלא נתפשט חדר״ג ואחת ילדה בסוף אדר ראשון ושני׳ בתחלת א״ש מי קודם אם גדול בשנים או גדול האחים ומסיק שם באמת דלא כשו״ת ה״ק דלא תלי׳ בגדול בשנים כ״א בגדול האחים) כונה זו לא נראה מכותלי מכתבו רק משום דראשון בכור הוה הי׳ לן לומר דבכור קודם ובכור מאם לא מעלה ולא מוריד ע״כ כונתו אבכור מאב שהוא קטן עדיין בחודש אדר ובכה״ג לא אוכל לצייר ספקתו עכ״ד.\\\vspace{0pt}

על זה אשיב ואבאר שאם קראתיו בכור לא כוונתי רק שהוא הי׳ היוצא ראשון מהתאומים שהאחד הוא הבכור דהיינו שנולד ראשון והשני הגדול שנעשה קודם לו גדול בשנים אבל שיהי׳ האחד בכור לנחלה או לפטר רחם הרי זה לא מעלה ולא מוריד לענין יבום ולכן הספק שביארתי הי׳ שכבר הי׳ להאחים התאומים אח גדול בשנים שנשא ומת אי נימא שהבכור שנולד ראשון מהתאומים ייבם אחר שאמר הכתוב והי׳ הבכור ודרשינן מה בכור בכורתו גרמה לו אף גדול גדולתו גרמה לו א״כ הכל תלוי במי שנולד ראשון שדומה לבכור או נימא דמי שנעשה גדול בשנים קודם הוא הנקרא גדול ומשום שהויי מצו׳ ג״כ לא שייך לזמן קצר כנראה מדברי הפוסקים (בסי׳ קס״א) שאפילו כשהיבם הגדול בעיר אחרת כל שאינו במדינה אחרת ממתינין עליו וכן ודאי בשהגדול חולה לשעתו ממתינין עליו כמו שמשמע ממה דנקט הי׳ חרש אין ממתינין עד שיבריא ולא נקט בסתם חולה משמע דוקא חולי כזה שמסתמא ימשך הרבה וא״כ כש״כ שלא שייך שהויי מצוה משום איזה ימים או שבועות כנלענ״ד, הקטן יעקב.\\\vspace{0pt}

\end{multicols}\newpage

\newchap{סימן קנג}
\begin{multicols}{2}
ב״ה אלטאנא, יום ג׳ כ״ז אייר תרכ״ב לפ״ק. להרה״ג וכו׳ מ״ה בירך אברהם אויסטערליטץ נ״י הגאב״ד דק״ק סקאליטץ יע״א.\\\vspace{0pt}

בהאי עובדא דאתי לקמי דמר נ״י באשה זקנה שנפלה לפני יבם ואינה רוצה לחלוץ ומענה בפי׳ כי אבי׳ הגאון ז״ל שהי׳ אב״ד באחת הקהלות אמר לה שאם תגיע לששים שנים אזי אינה מחוייבת עוד לחלוץ וגם היא בעלת מום וחלושה שבודאי לא תנשא עוד לאיש ונשאל מה יעשה בה אם לכופה על החליצה או לא.\\\vspace{0pt}

תשובה – בשו״ת הרשב״א (סי׳ י״ח) הביא תשובת רבי יוסף אבן פלט להראב״ד דאין מברכים על חליצה כיון שהיא משום יבום שהעיקר משום פרי׳ ורבי׳ והיא אינה מצו׳ על פו״ר ומזה נראה דס״ל דמצות יבום וחליצה אינה רק על היבם ולא על האשה וכן ביאר דבריו גם בברכי יוסף י״ד (סי׳ רמ״א) וגם הרא״ה בספר החינוך כתב דמצות יבום וחליצה היא על הזכרים דוקא והנה בספרי ע״ל יבמות (דף כ׳) הוכחתי קצת מדברי התוספ׳ והראשונים שדעתם אינה כן אלא שנוהגין מצות אילו גם באשה אכן אחר שהפוסקים הנ״ל כתבו בפי׳ שאין נוהגין באשה ודאי אין לעשות מעשה לכוף האשה לחליצה כשאינה רוצה להנשא לשוק.\\\vspace{0pt}

אמנם כבר העיר מעכ״ת נ״י לנכון על מה שכתב הב״ש (סי׳ קס״ה) בשם הזוהר פ׳ חקת דמצו׳ דוקא בחליצה וכפי הנראה מדברי הזוהר שם ענין החליצה הוא לתקון המת דאזיל מעלמא בלא בני למען יחוס עלי׳ הקב״ה ויקבל לנשמתו לעלמא אחרא והגוף ינוח בקבר ואין מזה סתירה למה שכתוב בזוהר חדש פ׳ כי תצא משום דהוי לי׳ למיעבד טיבו עם מיתא ואיהו לא בעי לפיכך עבדא לי׳ קלנא קמא כולא דירקה בפניו ע״ש ואם על ידי חליצה ג״כ נעשה תקון וטובה למת למה תבזה להיבם דז״א דלפי המבואר בזוהר בכמה דוכתי׳ ובפרט בסבא פ׳ משפטים תקון היבום מעולה יותר דבזה המת נולד מחדש לתקן את עצמו דבן הנולד הוא בעצמו האח המת ואשתו נעשית אמו ובזה ער ואונן נתקנו בפרץ וזרח וכן מחלון בעובד אבל ע״י חליצה אין תקון כ״כ למת רק שינוח בקבר ושיחוס עליו הקב״ה כנאמר בזוהר פ׳ חקת ובעי לבטשא לי׳ לההוא נעל בארעא לאחזאה דשכיך גופא דההוא מיתא וקב״ה לזימנא דא או לבתר זימנא חייס עלי׳ ויקבל לי׳ לעלמא אחרא עכ״ל וכיון שאין זה תקון ברור וודאי כ״כ לכן הקדימה התורה מצות ייבום לחליצה וצותה להכלים היבם בשביל שלא יעשה חסד הגדול עם המת להחזירו לימי עלומיו אבל מכ״מ גם בחליצה יש תקון למת כמבואר ויש זה בכלל ג״ח שעושין עם המתים. ולכן אף שאין כופין על ג״ח דדוקא על הצדקה כופין משום דאיכא לאו כמש״כ התוספ׳ בחולין (דף ק״י) לחד תירוצא מכ״מ נלענ״ד שאפילו הי׳ אבי האשה מצו׳ לה בפירוש שלא לחלוץ אחר ששים שנה לא צריכה לשמוע לו כדין המבואר י״ד (סי׳ ר״מ סט״ו) דאם מצו׳ האב לעבור אפילו על מצו׳ של דבריהם לא ישמע לו וכל שכן בנדון זה שגם ע״פ דבורה לא צו׳ לה אבי׳ אלא אמר לה שאינה מחוייבת לחלוץ וגם זה ודאי טעות הוא בידה דקשה להאמין שאמר לה אבי׳ כזה דלא מצאנו בשום מקום שיעור שנים זה ולכן לענ״ד יש לדבר על לב האשה שבל תמנע גמילות חסד מבעלה שמת ושתחלוץ לזכותו אמנם בדברים שהם בסידור חליצה רק מנהגים וחומרות יש להקל אצלה שלא תתבייש מפני מומה וחלישות ידי׳ ולכן לא תהי׳ החליצה בפרסום כשאינה רוצה בכך אלא בפני ב״ד של חמשה לבד וכדומה אבל אם לא תרצה לשמוע אין כופין אותה כנלענ״ד, הקטן יעקב.\\\vspace{0pt}

\end{multicols}\newpage

\newchap{סימן קנד}
\begin{multicols}{2}
ב״ה אלטאנא, יום ד׳ כ״ה אדר ראשון תרי״ט לפ״ק.\\\vspace{0pt}

להרה״ג וכו׳ מ״ה מענדל פרידלענדער נ״י הגאב״ד דק״ק געארגען במדינת אונגארן.\\\vspace{0pt}

שאלה – הנה זה ימים לא כביר בא מעשה לפני אשר כל השומע תצילנה שתי אזניו באחת הכפרים העומדים תחת דגלי דרים ב׳ יהודים ודרכם לילך על המסחר מהלך איזה ימים ונשותיהם לבדן בבית עם בניהם ובנותיהם ומשרתיהם ויהי היום כאשר יצא האיש כדרכו למסחר ואיש אחד בא ממדינת פולין ובגדיו קרועים ויבקש מהאשה מקום ללון והאשה אשר היתה צנועה ביותר כל ימי׳ אבל יראתה כסלתה רחמה עליו ונתנה לו מקום ללון וגם לאכול ולשתות אמנם האורח ההוא לא אכל אצלה שום דבר מן החי גם לא שתה משקה רק מים וכהנה עשה מעשה פרישות וסיגף עצמו בסיגופים קשים כל היום ישב בחדרו מסוגר וספר בידו גם מדי לילה בלילה עד חצות ואח״כ הי׳ מתאונן על חורבן בית אלקינו וכששכב לא שכב על מטה וספסל כי אם על הארץ ואבנים תחת ראשו ומדי יום ביומו טבל עצמו במים קרים של נהר פעמים בעת הקור כן נהג בבית האשה מיום א׳ פ׳ תרומה עד ש״ק פ׳ תצוה. \\\vspace{0pt}

אמנם בליל שבת אחר סעודתו התינוקים ומשרתי הבית עמדו כלם מעל השלחן והלכו לישן לחדר האחרת והאיש הרמאי עודנו ישב על השלחן עם האשה לבדו ונכנס אתה בדברים עד ששאלה אותו מי אתה ומאין תבא ואנה תלך וענה לה שלוחא דרחמנא אנא ושמי אליהו הנביא ואת אחי אנכי מבקש לקבץ אותם מארבע כנפות הארץ ואין מגלין הדבר אלא לצנועין והאשה לרוב סכלותה האמינה לו. היא הלכה לישן על מטתה בחדר הסמוך והעוכר הנ״ל עודנו יושב על מקומו ועיין בספר עד חצות לילה ולאחר חצות עמד והלך לאט על אצבעות רגליו אל המטה אשר האשה שכבה שמה והעיר אותה משנתה וידבר אלי׳ הנה הלכתי מקצה הארץ ועד קצה הארץ ולא מצאתי צדקת כמותך אשר היה ראוי לצאת ממנה משיח אך המניעה היא מצד בעלך שאינו הגון לכך לזאת נשלחתי מן השמים לשכב אותן וכעת חי׳ תלדי בן והוא יהי׳ משיח בן דוד ויגאל את ישראל וזה לך האות כי אליהו אנכי הנה ביום ג׳ הבע״ל לאחר הפרדי מאתך אם תפתח את פתח התיבה אשר עומד פה בחדר משכבך תמצא שם אוצר מטמון רב ארבע מאות דוקאטען של זהב אך בתנאי שלא תפתח התיבה קודם זמן המוגבל כה דבר אלי׳ הנואף עד שפתה וטמא אותה פעמים בליל שבת ומוצאי שבת וביום הראשון טרם עלות השחר ברח הנואף משם ולא נודע מקומו. \\\vspace{0pt}

ואשת כסלות הנ״ל חשב לכתוב לבעלה שישוב מהר לביתו באשר הצליח ד׳ את ביתו במטמון גדול והאיש שמע אלי׳ וישב ביום ג׳ ותפתח האשה את התיבה ולא מצאה מאומה מהמטמון אשר אמר הנואף ובראותה כי שקר בפיו צעקה ובכתה במר נפשה וספרה לבעלה את כל התועבה אשר עשה הרשע הזה ודברה על לבו הלא לא במרד ובמעל עשיתי זאת סהדי במרומים כוונתי הי׳ לשם שמים והלא הנואף הי׳ איש מאוס ומכוער מאוד ומה יסיתני לזנות עמו אמנם הבעל לא שקט בזה רק בא אלי וספר לי כל הדברים ושאל ממני כדת מה לעושת עם אשתו ושלחתי אחר האשה וחקרתי בחקירות שונות והיא ספרה גם לי ככל הדברים הנ״ל וצויתי לפרוש זה מזה עד שאציע הדבר לפני מעכ״ה נ״י: \\\vspace{0pt}

זה תוכן השאלה מהרב הגאב״ד הנ״ל נ״י:\\\vspace{0pt}

תשובה – חזרתי על כל צדדים וקשה מאוד למצוא תרופה ומזור למכת האולת שתהי׳ מותרת לבעלה כי מה שטוענת שוגגת היא וכונתה הי׳ לשם שמים אין זו טענה להתירה לפי משכ׳ המהרי״ק שרש קס״ח ונפסק ברמ״א אהע״ז (סי׳ קע״ח) שאם זינתה שסברה שמותר לזנות הוי כמזידה ואסורה לבעלה ישראל וכמו שהעיר גם מעלתו נ״י אמנם לקמן נדבר עוד מזה:\\\vspace{0pt}

והנה לכאורה הי׳ אפשר למצוא צד קולא כיון דאין עדים בדבר גם לא יצא עלי׳ קול אלא היא בלבד אמרה שזינתה וקיי״ל כמשנה אחרונה וכמבואר באהע״ז סי׳ קט״ו ס״ו אין עדים שזינתה אלא היא אומרת שזינתה אין חוששין לדבר זה לאוסרה דשמא עיני׳ נתנה באחר וא״כ גם באשה זו ניחוש כן.\\\vspace{0pt}

ואע״ג דהיא טוענת ששוגגת היא ורוצה להשאר תחת בעלה וא״כ איך נאמר שעיני׳ נתנה באחר ומשכ׳ בפסקי מהראי סי׳ רכ״ב בכעין זה וז״ל ולפ״ז נוכל לומר נמי דלעולם בשאומרה טמאה אני נתנה עיני׳ באחר ואח״כ זכרה היא בעצמו או הזכירו׳ ואלפוה מרוב בושת ופגם שיצאו לה ולמשפחתה נתחזקה נגד יצרה ולבה והעלימה עיני׳ מאחר שנתנה בו תחלה והפכה דבר עכ״ל לא שייך בזה כיון דבנדון דלפנינו לא הפכה דברי׳ אלא מיד כשהודה לבעלה שזינתה התנצלה ששוגגת היתה ושאל ידחנה א״כ הרי לא נתנה עיני׳ באחר מכ״מ י״ל דלמא איערומי קא מערמא שיודעת שאם תאמר שזינתה במזיד לא תשיג מבוקשה שישלחה ותנשא לאחר שעיני׳ נתנה בו שיחשוד אותה שעיני׳ נתנה באחר לכך מתנצלת לפניו למען יאמין לדברי׳ שזינתה וישלח אותה וסברא כזו הוזכרה גם בשו״ת נ״ב מ״ק חאה״ז סי׳ ע״א וגם מלשון הש״ע משמע כן מדכתב סתמא דאין חוששין לדברי׳ דשמא עיני׳ נתנה באחר ולא חילק בין אם היא מבקשת לצאת ממנו או להשאר אצל בעלה:\\\vspace{0pt}

גם מה שכתב מעכ״ת נ״י שספרו לו כי בבקר ביום ש״ק באו המשרתים אל החדר שהאשה שוכבת שם ומצאו הנואף שוכב שם על הארץ וא״כ איכא רגלים לדבר מפני שנתייחד עם האשה וביש רגלים לא אמרינן שמא עיני׳ נתנה באחר וכמשכ׳ הב״ש סי׳ קט״ו ס״ק כ״ג דאם ידוע שנתיחדה עם אחד ואמרה דזנתה נדחה דנאמנת.\\\vspace{0pt}

גם מטעם זה עדיין אין לאסרה ע״פ משכ׳ הח״מ שם מדברי הרא״ש דהיכא דאיכא טעמא להתירא כגון אירכוסי הוי מירכס אין אוסרין אותה אע״ג דיש רגלים וא״כ ה״נ הוי הך טעמא דאם איתא דנאף היאך שכב שם על הארץ בחדר משכב האשה עד שבאו המשרתים וראוהו ולא חזר להחדר שישב שם עד חצות לילה או חזר לחדר משכבו ואע״ג דהב״ש חולק על הח״מ והחמיר כתוספ׳ דפ׳ אע״פ דהיכא דאיכא רגלים לא מהני טעמא דהתירא עכ״פ תלי זה בב׳ שיטות התוספ׳ והנ״ב סי׳ ע׳ החזיק בדברי הח״מ נגד הב״ש וגם בלא״ה י״ל כמו שהעיר גם מעכ״ת דאין זה ייחוד להקרא רגלים לדבר כיון שלא יש ראי׳ שנתייחדו לשם זנות ומה גם שהי׳ פתח פתוח לכל בני בית לבא לשם:\\\vspace{0pt}

אמנם כל זה הי׳ אפשר לומר אם לא היו כאן רק דברי האשה לבדה אבל כפי הנראה מדברי השאלה ממה שהבעל הי׳ צועק ובוכה על המעשה ועל בושה שלו הוא מאמין לדברי׳ והרי מבואר בש״ע סי׳ קט״ו דאם הי׳ מאמינה ודעתו סומכת על דברי׳ ה״ז חייב להוציאה ואע״ג דבסי׳ קע״ח הביא הרמ״א בשם י״א דבזמן הזה שיש חרם דר״ג אינו נאמן לומר שמאמינה מכ״מ הרי לבסוף הביא י״א דס״ל להו דגם בזה״ז נאמן וכפי הנראה פסק כן מדהביא דעה זו לבסוף וגם מדלא הביא זו בסי׳ קט״ו על מה שכתב הש״ע שם דחייב להוציאה ומשמע מזה שהסכים עמו להלכה גם כל האחרונים נקטו בפשיטות כן דהיכא דהבעל מאמין אסורה לו ותמהני על מעכ״ת שלא העיר על זה.\\\vspace{0pt}

לכן מצד שמא עיני׳ נתנה באחר אין תרופה לה:\\\vspace{0pt}

אמנם אחראי רואי משכ׳ הפסקי מהראי סי׳ רכ״ב דטובא מקיל מור״ם שלא לאסור אשה לבעלה ואע״ג דרגיל הוא בכל מקום למעבד הכא לחומרא והכא לחומרא א״כ אין לנו אלא לצאת בעקבותיו וכו׳ ע״ש גם אנכי חפשתי למצוא היתר כפענ״ד ואדברה וירוח לי.\\\vspace{0pt}

דהנה בשו״ת מהרי״ק הנ״ל על שאלת מהרי״ל באשה שזנתה תחת בעלה ברצון והיא לא ידעה שיש איסור בדבר אם יחשב ששוגג השיב וז״ל לענ״ד נראה דאין לזו דין שוגגת להתירה לבעלה כיון שהיא מתכוונת למעול מעל באישה ומזנה תחתיו דהא לא כתיב איש איש וגו׳ ומעלה מעל בד׳ דלשתמע דוקא במכוונת לאיסור אלא ומעלה בו מעל כתיב עכ״ל ושוב כ׳ עוד נראה לענ״ד ראי׳ דע״כ אין הדבר תלוי בכוונת איסור דהא גרסינן במגילה פ״ק וכאשר אבדתי אבדתי כאשר אבדתי מבית אבא אבדתי ממך דעד השתא באונס ועכשיו ברצון ש״מ שמאותה שעה נאסרה על מרדכי והנה דבר פשוט שאסתר לא עשתה שום איסור ולא הי׳ בדבר אפילו נדנוד עבירה הלא מצו׳ רבה עשתה שהצילה את ישראל ותדע שכן הוא שהרי בבאה לפני המלך שרתה עלי׳ רוח הקדש וכו׳ ואפילו הכי נאסרה על מרדכי בעלה משום אותו מעשה שהי׳ ברצון והלא דברים ק״ו ומה התם דלא הי׳ בדבר שום נדנוד עבירה אלא אדרבא מצו׳ קעבדה ואפילו הכי נאסרה על מרדכי בעלה אשה שזנתה תחת בעלה לא כש״כ שהיא אסורה עליו ואע״פ שאינה יודעת שיש איסור בדבר מכ״מ עשתה היא עבירה וצריכה כפרה וחייבת קרבן עכ״ל מהרי״ק.\\\vspace{0pt}

ומטעם זה כ׳ ג״כ הב״ש סי׳ קע״ח ואם זינתה ברצון כדי להציל נפשות כעובדא דאסתר לאחשורוש אסורה לבעלה כיון שהביאה הי׳ ברצון עכ״ל.\\\vspace{0pt}

ולענ״ד יש להשיב על זה דאף דסברת מהרי״ק סברא גדולה היא דגם אם לא מעלה בד׳ רק שמעלה בבעלה אסורה לענ״ד לא שיך זה רק בזינתה ברצון ונתכוונה להנאתה אלא שלא ידעה שאיסור הוא שמכ״מ נתכוונה למעול בבעלה אבל בזינתה לשם מצו׳ שכוונתה רק לשם שמים היאך יקרא זה שמעלה בבעלה ודיותר יקשה גבי מרדכי שהוה בעצמו צו׳ לה לבא אל המלך נגד רצונה והיאך תחשב שמעלה בו מעל גם כבר הקשו המהרש״א והרי״ף בע״י סתירה בדברי אסתר דלמרדכי אמרה ועכשיו ברצון ושם במגילה אמרינן א״ר לוי כיון שהגיע לבית הצלמים נסתלקה הימנה שכינה אמרה אלי אלי למה עזבתני שמא אתה דן על שוגג כמזיד ואונס כרצון ופי׳ רש״י אע״פ שאני בא אליו מאלי אונס הוא הרי שהיא בעצמה קראה עצמה אונס.\\\vspace{0pt}

ולכן הי׳ נלענ״ד דאם בודאי הי׳ זה צורך להצלת ישראל אין לך אונס גדול מזה אבל כפי הנראה מדברי מרדכי הוא בעצמו הי׳ מסופק בזה שאמר אם החרש תחרישי בעת הזאת רוח והצלה יעמוד ליהודים ממקום אחר וגו׳ ומי יודע אם לעת כזאת הגעת למלכות וכוונת דבריו שהי׳ בוטח בד׳ שישלח הצלה לישראל רק הי׳ מסופק אם ממקום אחר אם ע״י אסתר כמו שאמר ומי יודע אם לא הגעת למלכות רק לבעבור הצלת ישראל כמו שפי׳ הראב״ע ולכן מצד איסור א״א אע״פ שהי׳ ספק מכ״מ הי׳ מותר דעל ספק נפשות ג״כ מחללים אפילו שבת אבל שהתי׳ אח״כ מותרת לבעלה זה הוי ספק איסור דשמא הי׳ אפשר להציל ממקום אחר וזינתה ברצון שלא לצורך וזה שאמרה אסתר וכאשר אבדתי אבדתי שמעכשיו ברצון ומספק נאסרתי לך ולכן כשהגיעה לביהצ״ל וסלקה ממנה שכינה אמרה למה עזבנתי וכי אתה דן שוגג כמזיד ואונס כרצון לא שהיתה אסתר מסופקת בזה דדבר המפורש בתורה הוא שאין הקב״ה דן אונס כרצון אלא כוונה לומר שמא בזה הראני שלא אלך ושאין זה אונס אצלי אלא תרצה להושיע לישראל על ידי ולכן כששבה השכינה אלי׳ ידעה באמת שמד׳ היתה זאת שרק על ידה רצה להושיע לישראל ולכן להצדקת באמת לא נחשב לזנות שתאסר על בעלה שאונס גמורה היתה:\\\vspace{0pt}

והיוצא הזה שאם נאמר כן אז בזינתה ברצון לשם שמים לא תקרא שמעלה בבעלה. \\\vspace{0pt}

אכן איני כדאי לחלוק על מהרי״ק וב״ש ולהתיר איסור נגדם אמנם ראיתי בשו״ת שבות יעקב ח״ב סי׳ קי״ז שנשאל באיש שהלך עם אשתו ועם אחרים ביער ובאו עליהם רוצחים ולא ידעו להציל נפשם כי אם ע״י שהפקידה האשה עצמה להם ברצון בעלה אם מותרת לבעלה והשיב ע״פ דברי מהרי״ק שהקשה מה הי׳ החילוק באסתר בין עד עכשיו שהי׳ באונס ובין עכשיו שנחשב כרצון כיון שהי׳ כ״ג להצלת ישראל ותירץ בסברא ישרה דאם האונס על הבעילה עצמה כמו שהי׳ כשנלקחה לאחשורוש זה נחשב זינתה באונס ומותרת אבל בשאין האונס על הבעילה רק מחמת אימה אחרת הולכת אליו והיא מתרצת מרצונה לבעילה משום הצלה אע״ג דשפיר עבדה להצלת עצמה והרבים ומקרי אנוסה מכ״מ כיון דבעילתה הי׳ ברצונה נאסרה על בעלה ובזה מתרץ ג״כ הסתירה במה שאסתר פעם חשבה עצמה כרצון ופעם כאונס ולכן חילק אם הבעילה לא הי׳ באונס רק להציל אסורה לבעלה אבל אם הבעילה בעצמה הי׳ באונס מותרת ע״ש.\\\vspace{0pt}

והשתא בנדון השאלה שאמר לה הנואף ימ״ש שהוא אליהו הנביא ושלזאת נשלח מן השמים לשכב עמה והאמינה בו האולת כ״כ עד שקראה לבעלה לקבל העשירות כאלו כבר הוא בידה א״כ לפי אולתה היתה מצו׳ מן השמים על הבעילה עצמה ואין לך אונס גדול מזה ולא נתכוונה בהבעילה למעול בבעלה כי אם כמו שאמרה שסהדה במרומים שהי׳ כוונתה לש״ש לזה יש לדון שגם ע״פ מהרי״ק והאחרונים נקראה אונס גמור ומותרת לבעלה כנלענ״ד אכן אין לסמוך על הוראתי אם לא יסכימו על זה עוד שנים מבעלי הוראה ואז אצטרף עמהם להתיר אשה לבעלה ובפרט שכפי שנאמר בהשאלה אשה כשרה היתה מאז ויש להם בנים כנלענ״ד\\\vspace{0pt}

הקטן יעקב: \\\vspace{0pt}

\end{multicols}\newpage

\newchap{סימן קנה}
\begin{multicols}{2}
ב״ה אלטאנא, יום ב׳ ז׳ ניסן תרי״ט לפ״ק. עוד להרב הנ״ל נ״י.\\\vspace{0pt}

ר״ד אקריב למעכ״ת נ״י תשואות חן על שירד לעמק השו׳ והסברא בענין הצד היתר שהמצאתי למצוא תרופה ומזור למחלת האולת שזנתה תחת בעלה בחשבה שמצו׳ על כן מן השמים והביא ראיות לסברא זו רק שסיים שיש לפקפק קצת עלי׳ משום דעובדא דידן לא עדיפא מטעה בדבר מצו׳ ולא עשה מצו׳ דחייב חטאת משום דטעה לא הוי אונס אלא שוגג ואשה זו לא עדיפא מטעה בדבר מצו׳ ואדרבא גרוע דהמל של מ״ש בשבת עכ״פ טעה בדבר שהוא מצו׳ באמת למול בשבת רק שבאותו יום לא הי׳ מצו׳ אבל לזנות אין כאן מצו׳ כלל וא״כ אינה עכ״פ רק שוגג ולא אונס ועוד דמה לי שלא ידעה שאסור לזנות דאסורה לבעלה כפסק המהרי״ק ומה לי שלא ידעה שאסור לשמוע לנביא השקר לעבור על ד״ת והביא ראיות לזה זה תוכן דברי מעכ״ת נ״י. אמנם לענ״ד אין מזה סתירה לסברתי דזה פשיטא דאשה דזנתה בשוגג אף שחייבת חטאת מפני שנקראת חוטאת שהי׳ לה לבדוק מכ״מ מותרת לבעלה דהתורה לא אסרה רק מזידה וכמו שכ׳ גם המהרי״ק וראי׳ לזה ממשנה יבמות (דף ל״ג) שנים שקדשו שתי נשים וכו׳ ומפרישין אותן ג׳ חדשים וכו׳ הרי דאם אינה מעוברת מותרת לבעלה מיד אחר ג״ח אף שצריכה קרבן חטאת כדמוכח מהא דתנא רבי חייא שם (דף ל״ד) הרי כאן ט״ז חטאת ולכן גם המזנה שסבורה שמותר לזנות מצד אומר מותר אף ששוגגת היא וחייבת קרבן היתה מותרת לבעלה כמש״כ הב״ש סי׳ קע״ח ולא אסורה רק מטעם שעכ״פ נתכוונה למעול מעל בבעלה דהיינו להפר בריתה עמו שתהי׳ מיוחדת לו לבדו ולזה הוסיף המהרי״ק ע״פ הראי׳ מאסתר דאפילו נתכוונה לדבר מצו׳ בבעילתה מכ״מ נקראת מועלת מעל בבעלה והשבות יעקב ביאר סברא זו שזה דוקא באם הבעילה עצמה היא ברצון רק מטעם אחר היא מצו׳ דהרי עכ״פ היא נבעלת ברצון ורוצה להפר ברית בעלה מפני תועלת מצו׳ זו אבל אם על הבעילה עצמה היא אנוסה לא נקראת מועלת מעל בבעלה דמה לה לעשות ועפ״ז דקדקתי בנדון המעשה שלפנינו שאף שלפי האמת האולת הזאת אין דינה כאנוסה אלא כאומר מותר בעלמא מכ״מ הרי לפי אולתה חשבה עצמה כאנוסה וכמצו׳ מן השמים לבעילה זו ולא היתה כוונה שלה להפר ברית עם בעלה ברצון מפני מצו׳ אחרת אלא חשבה עצמה כאנוסה בבעילה זו א״כ לא נתכוונה להפר בריתה עם בעלה ועל כן לא תקרא מועלת מעל באישה וגם אני אומר שע״פ הדין אינה אנוסה רק אומרת מותר שדינה כשוגגת ולכשיבנה ביהמ״ק תביא חטאת שמנה ומכ״מ ממנ״פ אינה אסורה לבעלה דמצד אומר מותר הוי שוגגת דמותרת לבעלה וע״פ אולתה אנוסה היתה על בעילה זו עצמה ולא נתכוונה בבעילות של הנואף להפר בריתה עם בעלה שע״פ אולתה היתה מוכרחת לכך בציווי מן השמים ומה הי׳ לה לעשות להקם בריתה עם בעלה זאת הנלענ״ד אבל כבר אמרתי שלא אסמוך על זה אם לא יסכימו עם זה גדולים ומפורסמים בהוראה. הקטן יעקב.\\\vspace{0pt}

בעזרת דלים עוזר, תם חלק אבן העזר.\\\vspace{0pt}

\end{multicols}\newpage

\newchap{סימן קנו}
\begin{multicols}{2}
ב״ה אלטאנא, יום ה׳ י״ג שבט תרכ״א לפ״ק. להרה״ג וכו׳ מ״ה יצחק דוב ב״ב הלוי נ״י הגאב״ד דק״ק ווירצבורג יע״א.\\\vspace{0pt}

בדיק לן מר נ״י במה שתימה על מה שפסק הרמ״א בח״מ (סי׳ כ״ה) ע״פ המהרי״ק דאם רבים פליגי עם יחיד הולכים אחר הרבים אפילו אין הרבים מסכימים מטעם א׳ אלא שלכל א׳ יש טעם בפני עצמו והש״ך שם ובי״ד סי׳ רמ״ב בכללי הוראות או״ה חולק עליו דאם ב׳ פוסקים בספרים מסכימים לדין א׳ מטעמים שונים לא נחשבו רק כיחידים אבל מכ״מ מודה הש״ך למהרי״ק ורמ״א דאם הרבים לפנינו ומסכימים על הדין מטעמים שונים שהם נדונים כרבים ועל זה הקשה מירושלמי ר״ה פ׳ ב׳ הלכה ה׳ גבי עיבור השנה דאיתא התם לא כן א״ר זעירה והן שיהו כולם מורים מטעם א׳ עכ״ל והרי מירושלמי לא בלבד דמוכח נגד המהרי״ק אלא גם נגד הש״ך שהרי במעשה דשם היו הרבים לפנינו ובזה גם הש״ך ס״ל דלא בעינן שיסכימו לטעם א׳ עכ״ד מעכ״ת נ״י ואני תמה שמר נ״י הביא ראי׳ מהמקשן שבירושלמי ושביק התרצן דאיתא שם כיון דאילין מודיי לאילין ואילין מודיי לאילין כמו שכולן מורים מטעם א׳ עכ״ל והרי זה בפי׳ כדעת מהרי״ק וסייעתו דהיכא דמודים זה לזה בפסק כמי שמורים מטעם א׳ דמי אמנם יש להבין דע״פ תי׳ זה דהיכא דזה מודה לזה וזה לזה בפסק כאילו הורו מטעם א׳ דמי א״כ לענין מה אמר ר׳ זעירה שצריכים שיהיו כולם מורים מטעם א׳ ולכן נלענ״ד שאין ראי׳ כלל לא מן המקשן ולא מן התרצן לפסק המהרי״ק או נגדו דהנה עיבור השנה חלוק מכל דינים שבתורה שבכולם אין קפידא רק שיהי׳ הפסק של ב״ד ע״פ טעמים נכונים אבל לא מסויים מאיזה טעם שיפסק הדין כנראה בסנהדרין (דף ל״ד) דלא כתבו טעם המזכין והמחייבין רק כדי שלא יאמרו שנים טעם א׳ מב׳ מקראות דממנ״פ אחד אינו אמת אבל אין קפידא מאיזה טעם שיפסקו רק שיהי׳ טעם אמת ונכון אמנם לענין עיבור השנה אין הדבר כן דשם הטעמים בעצמם מסויימים כדאמרינן בסנהדרין (דף י״א) ת״ר על ג׳ דברים מעברין השנה על האביב ועל פירות האילן ועל התקופה על שנים מהם מעברין ועל א׳ מהם אין מעברין עכ״ל ועוד (שם) קחשיב דברים שמעברין עליהם אם הי׳ צריכה מפני הדרכים ומפני הגשרים וכו׳ דבאילו גם על א׳ מעברין כמש״כ התוספ׳ שם והי׳ אפשר לחשוב דאם א׳ מג׳ סנהדרין שמעברין השנה אומר שראוי לעבר מפני האביב ואחד מפני התקופה וא׳ מפני פירות האילן כיון שכולם מסכימים שראוי לעבר דמעברין לזה משמיענו ר׳ זעירא דבעינן שיהי׳ כולם מורים מטעם א׳ דהיינו שיהיו כולם (או רובם דנחשבו ככלם) מסכימים לטעם א׳ בשנים מהם או אפילו בטעם באחד מהם באותם שמעברין עליהם בטעם יחידי אבל אם א׳ מהסנהדרין אומר טעם א׳ ואחד טעם אחר אין מעברין כיון דאין רוב על טעם אחד שעל ידו מעברין והנה המקשן הי׳ סובר דע״פ דברי ר׳ זעירא לא הי׳ אפשר לעבר על דברי רועי בקר כיון דאמר כל אחד מהם טעם אחר ועל זה מתרץ כיון דאילו מודים לאילו שעדיין העת קר יותר ממה שרגיל להיות באדר ומזה יש לידע שלא יהי׳ האביב בימי פסח ופירות האילן בימי עצרת הוי כמי שהורו מטעם א׳ אבל בשאר דינים שבתורה שאין הטעמים מסויימין אין ראי׳ מזה דבעינן שיהי׳ כולם מסכימים לטעם א׳ מן המקשן וגם לא להיפך מן התרצן בירושלמי, כנלענ״ד, הקטן יעקב.\\\vspace{0pt}

\end{multicols}\newpage

\newchap{סימן קנז}
\begin{multicols}{2}
ב״ה אלטאנא, יום ב׳ ה׳ תמוז תרכ״ז לפ״ק. לחתני הרה״ג וכו׳ מ״ה משה ארי׳ ב״ב הלוי נ״י אב״ד דק״ק קיססינגען יע״א.\\\vspace{0pt}

על דבר שאלתך באם יש חשש פסול בעדי קידושין הממונים מקהל אם יש תקנה לזה כבר הודעתיך שיש תקנה שיאמר המסדר קידושין קודם שיקדש החתן שכל מי שכשר להעיד יראה הקידושין ובזה הוציא הפסולים מכלל העדות ואע״פ שהמקדש לא ייחד העדים אין בזה חשש שמצד דין דגמרא לא צריך לייחד עדים כלל אלא כל שקדש בפני עדים כשרים מקודשת כדמוכח בכמה דוכתי׳ ומה שמייחדים עדים לקידושין בזה״ז הוא ע״פ מה שכתב הר״פ בהגהת הסמ״ק הביאו הש״ך בח״מ (סי׳ ל״ו ס״ק ח׳) וז״ל צריך לברר עדים בגטין וקדושין דאל״כ רגילות הוא ששם עומדים קרובים ועדותן בטלה וכן פירשו רבותינו וכן נהגו העולם עכ״ל ולכן כשמזמין המסדר קידושין לכל מי שכשר להעיד הרי הוציא הפסולים מן הכלל ואפילו יכוונו להעיד לא הוי כנמצא בהם קרוב או פסול לבטל עדות הכשרים כיון שהוא לא זמנם וכמו שכתב הש״ך (שם) ולענין מה שהקשו התוספ׳ והרא״ש היאך תמצא גטין וקידושין כשרים שנעשו בפומבי במעמד קרובים י״ל דלדידן פשיטא דכשר דהא מזמנים עדים כשרים מיוחדים לכך וא״כ פשיטא דהקרובים אינם מכוונים להעיד ואפילו יכוונו להעיד אינם פוסלים כיון דמזמנים עדים לכך הרי אנו מוציאין כל האחרים מכלל העדות ולאו כל כמינייהו דקרובים ופסולים להיות עדות בעל כרחנו עכ״ל ולכן כל שאומר בפי׳ שרק הכשרים יהיו עדים הרי הוציא הפסולים מכלל העדות אכן זה דוקא בלא ייחד הפסולים להיות עדים אבל בייחד אותם אז לא מועיל עדות הכשרים כנראה מדברי הש״ך הנ״ל וכ״כ גם בשו״ת חות יאיר סי׳ י״ט דהוי נמצא א׳ מהם קרוב או פסול דפוסל עדות הכשרים אפילו לא העיד בב״ד כמסקנת רוב הפוסקים. ומה שפקפקת על תקנה זו ממה שמבואר באהע״ז (סי׳ מ״ב ס׳ ד׳) שצריכים המקדש והמתקדשת לראות את העדים כבר השבת לנכון שזה דוקא לאפוקי אם יסברו שאין כאן עדים כלל שאז י״ל שהמתקדשת לא כוונה לקידושין שידעה שאין קידושין מועילים בלא עדים אבל בשיודעים שיש כאן עדים אין קפידא שידעו מי הם וכן נראה משו״ת הריב״ש (סי׳ רס״ו) שכתב אבל מכ״מ צריכה שהאשה תראה שיש שם עדים אבל כשאינה רואה שיש שם עדים אין ראי׳ שמקבלת אותם בתורת קידושין ואף אם שמעו מפי׳ שהיא מקבלת אותם בתורת קידושין אינה ראי׳ שנתרצית בקידושין אלא שאומרת כן מפני שסבורה שאין שם עדים ויודעת שהמקדש בלא קידושין אין חוששין לקידושיו עכ״ל הרי שכל קפידא שיראו את העדים הוא רק למען ידעו שיש שם עדים אבל לא בעינן שיכירו את העדים ויתנו עיניהם עליהם והנה הב״ש ס״ק י״א אחר שהביא הב׳ דיעות דלמהר״ם פאדוא אם כוונתם לשם קידושין אפילו לא ראו העדים מהני ולדעת מהרי״ט לא מהני כתב שיש נפקותא בין ב׳ הדעות אם קדש בפני עדים פסולים שלא נודע פסולם וב׳ עדים כשרים ראו מאחורי הגדר שלא ידעו המקדש והמתקדשת מהם דלדעת המהר״ם מקודשת ולדעת מהרי״ט אינה מקודשת ע״ש ולענ״ד לא בלבד לדעת המהר״ם הוי קידושין מעליא אם ידעו שיש כאן עדים כשרים ולא ראו אותם אלא אפילו למהרי״ט דדוקא בנדון הב״ש שסמכו המקדש והמתקדשת על העדים הפסולים ולא ידעו כלל שיש כאן עדים אחרים כתב שפיר שאין כאן קידושין אבל בשידעו שיש כאן עדים כשרים רק שלא ראו אותם בזה ודאי הוי קידושין ולכן לענ״ד אפילו עמדו העדים מן הצד של המקדש או המתקדשת אין קפידא דאל״כ הי׳ צריך להזהיר להמקדש והמתקדשת שיראו בשעת קידושין על העדים ומעולם לא ראיתי ולא שמעתי מרבותי שהקפידו על זה או שהראו להם העדים למען יכירו אותם אלא במה שאומרים בפניהם להעדים שיראו הקידושין די בזה שעי״ז יודעים המקדש והמתקדשת שיש כאן עדים שרואים הקידושין ולכן כשאומר המסדר קידושין שכל הכשר להעיד יראה הקידושין ושמעו המקוהמ״ת כן אין כאן חשש איסור לענ״ד ובפרט לעת הצורך.\\\vspace{0pt}

ועל דבר שאלתך אם המסדר קידושין יכול להיות אחד מן העדים באשר ששמעת מאביך הגאון נ״י שלא יעשה כן אם לא שמוחל על שכר ס״ק בלב שלם דאל״כ הוי נוגע הנה משו״ת חות יאיר סי׳ י״ט נראה שפשוט לו שהמסדר קידושין יכול להיות עד שכתב וז״ל תמה אני מה זו שאלה הלא המסדר קידושין כהר״ז קירוויילער הי׳ מקרא להחתן מלה במלה והוא צופה ומביט וכו׳ ע״ש ומה זו תימה אולי קבל המס״ק שכר כנהוג אע״כ דלא מקרי זה נוגע ולענ״ד עדים המס״ק לעד מאחר דבאחר יש חשש ליקח שכר עדות דאע״ג דלפי המבואר בח״מ סי׳ ל״ד ס׳ י״ח ברמ״א שמי שאינו מחוייב להעיד והולך שם לראות ולהעיד מותר ליטול שכר עדותו הרי כבר העיר הש״ך על מה שכתב הרמ״א באהע״ז סי׳ ק״ל סכ״א לענין עדי גט שלא יכולים לקבל יותר משכר בטלה אם לא שיש עוד טעמים אחרים מה שלא שייך בעדי קידושין והרי המס״ק מקבל שכרו בשביל הסידור קידושין ומעיד בחנם ומסתמא כוונת מר אביך שכר בעד ס״ק ובזה נוגע בעדותו ולענ״ד יש להשיב דמסתמא שכר בעד ס״ק ובזה ובזה נוגע בעדותו ולענ״ד יש להשיב דמסתמא גם בלא עדות המס״ק יהי׳ עוד אחר כשר כאן להכשיר הקידושין ואפילו נימא שכולם פסולים ורק ע״י עדותו נעשה הקידושין בהכשר מכ״מ יש לו לקבל שכר ס״ק דהועיל להזוג כשמקדש שנית בפני עדים כשרים שלא צריך להיות בחופה בעשרה ולא צריך שנית ברכות אירוסין ונשואין כמבואר באהע״ז (סי׳ ס״א ס׳ א׳) ברמ״א אם קידש בטעות והי׳ לו נישואין עמה אע״פ שחוזר ומקדשה א״צ לברך שנית ז׳ ברכות עכ״ל וכבר כתבתי במ״א (סי׳ ק״מ) שה״ה שא״צ ברכת אירוסין ולא בעשרה רק שיקדש בינו לבינה לפני ב׳ עדים ולכן דל עדות המס״ק מכ״מ עשה את שלו לסדר קידושין בפני עשרה ולברך ברכת אירוסין אשר על זה מקבל שכרו וא״כ כשמעיד לא הוי נוגע במה שמקבל שכרו, כנלענ״ד, הקטן יעקב.\\\vspace{0pt}

\end{multicols}\newpage

\newchap{סימן קנח}
\begin{multicols}{2}
מירושלם עיה״ק תוב״ב – י״ב מנחם תרט״ז לפ״ק.\\\vspace{0pt}

שאלה – משרתת יהודית היתה נשואה לבעל ומת בלא בנים והיו אומרים שלא הי׳ לו גבורת אנשים ונתאלמנה בתולה מן הנישואין ונכנסה בבית גביר א׳ להיות משרתת לבנותיו ואחותה נשואה לאופה באותה עיר והוא רע מעללים ויעש לו דרך ומבוא לבית הגביר ויהי מצחק עם גיסתו המשרתת ומדבר נבלות עמה תמיד וכן הולכת המשרתת לבית אחותה תמיד וגם יקרה שתשב שם חדש ימים ולעין כל משכיל האופה נואף עם המשרתת ונתעברה הארורה ואחר ששה חדשים עשתה עצמה חולה בבית הגביר וישלח אחר הרופאים ויבינו מגפתה כי הרתה ונתוודע להאופה ויתיעץ בערמה לקחת אותה לביתו ולהשליך תזנותיו על הגביר וילמדה לאמר לכל עם ועדה הגביר שכב עמי בעל כרחי שלא בטובתי באותה לילה ובאותו מקום באותות שקרים וסימנים מזוייפים גם לפי דברי׳ שהגידה שהוא ששה חדשים להריונה והנה נמצא שכל חדש עבורה היתה בבית האופה הנואף לא בבית הגביר אבל מצח אשה זונה העיזה פני׳ בפני כל לאמר הגביר שכב עמי בעל כרחי שלא ברצוני כלל ותהום העיר ורצו לעשות משפט להגביר ולקנסו בסך גדול גם אמרו שע״פ הדין חייב לישאנה תחת אשר ענה וגם חייב לזון את הולד ויחר מאוד להגביר ויבא לביה״כ בעת קה״ת ותפס בשתי ידיו את הס״ת ואמר בזה״ל: רבותי יצא עלי שם רע ששכבתי עם פלונית אני נשבע שבועה חמורה בזה הס״ת שזה שקר וכזב ומעולם לא עלתה על דעתי כלל ואני מבקש מכם רבותי אם תדינוני ע״פ התורה תדינוני לא אסור ממנה ימין ושמאל. ויענו אותו אין אנו יודעים דין תורה רק נשפוט אותך כפי דעתנו והוסיפו לביישו ברבים תוך ביה״כ גם הביאו את הזונה ואופה ויעיזו בפניו ויאמינו להזונה בלא עד ובלא ראי׳ ואדרבא האומדנא דמוכח לכל שהאופה גיסה הוא הנואף והגביר צועק אני רוצה דין תוה״ק ויתחזק נגדם להוציא משפטו ע״פ התורה ע״כ לשון השאלה.\\\vspace{0pt}

תשובה – הנה זה פשוט דהדין עם הגביר ושאין רשות ביד אדם לקנסו וכל שכן לבזותו אחרי שאין עדים ואין ראי׳ על האשמה שנעמסה עליו ואחרי שרוצה לציית דין בפני ב״ד ע״פ דין התורה ואפילו הי׳ האומדנא שחטא אין דנין ע״פ אומדנא כי אם ע״פ עדים כדמוכח בכמה דוכתי׳ והיא אינה נאמנת עליו כמבואר בשו״ת הריב״ש סי׳ מ״א ובשו״ת מהרי״ק סי׳ קפ״ט וכן פסק הרמ״א אהע״ז סי׳ כ״ב וכל המחרף והמבייש את הגביר על דברי׳ חטאו גדול ועתיד ליתן את הדין גם זה פשוט שאין רשות לשום אדם לתובעו להגביר לדין כי אם המשרתת הנבעלת דלכל שאר אדם יכול לומר לאו בעל דברים דידי את אם לא שנתמנה מהמשרתת למורשה בעבורה ע״פ התנאים המבוארים בח״מ (סי׳ קכ״ב ס׳ ד׳) וכיון דמדברי השאלה אינו יוצא שהיא תובעת אותו אין לשפוט בזה עד שנשמע מה תובעת ותבקש ממנו אמנם לעמוד על הדבר איך אפשר שיפול דבר הדין נחקור – להלכה ולא למעשה – שאפשר שתתבע ממנו ג׳ תביעות. דהיינו: – א – צער בשת ופגם וקנס, – ב – שישא אותה לאשה, – ג – שיספיק את הילד אשר ילדה, ועל זה אשיב.\\\vspace{0pt}

א – זה ודאי שאפילו נבעלה מהגביר רק לא כדברי׳ שלא ברצונה כי אם ברצונה אין לה עליו שום תביעה מקנס ופגם וצער ובושת דבמפותה פטור מכל דמחלה כמבואר בגמרא ופוסקים ולפ״ז אף שטוענת שנאנסה לכאורה אינה נאמנת ע״פ מה שכתב הרמב״ם ה׳ נערה בתולה (פ׳ א׳) כל הנבעלת בעיר היא בחזקת מפותה ע״ש וכן פסק הטור אהע״ז (סי׳ קע״ז) אמנם בזה יש פלוגתא בין הפוסקים דהמשנה למלך (שם) הביא בשם מהרח״ש דטוענת בעיר טענת אנוסה אינה נאמנת כלל אפילו להשביעו אבל המ״ל חולק עליו ומביא ראי׳ דמכ״מ משביעין אותו על טענותי׳ וגם בשו״ת חכם צבי (סי׳ קמ״ו) האריך בזה ומסיק שגם בעיר יש לה טענת אנוסה לענין שחייב להשבע ע״ש ולכן מצד זה יהי׳ הגביר חייב שבועה אבל מטעם אחר יש לדין לפוטרו אפילו משבועה שהרי פסק הרמב״ם שם (פ׳ ב׳) כל בת שיש לה קנס יש לה בושת ופגם ואם היא אנוסה יש לה צער וכל בת שאין לה קנס אין לה בושת ופגם וכו׳ ע״ש וא״כ בנדון זה שהנבעלת כבר היתה נשואה לבעל ואלמנה מן הנשואין אין לה טענת קנס כמו שכתב הרמב״ם שם דמגורשת מן הנישואין אין לה קנס וא״כ פשיטא דה״ה אלמנה מן הנישואין שהרי מה דמגורשת אין לה קנס אתיא ממה דתנן כתובות (דף י׳) בתולה אלמנה גרושה וחלוצה מן הנשואין כתובתן מנה ואין להן טענת בתולים כמש״כ הכס״מ שם טעם הרמב״ם והרי שם הושו׳ דין אלמכה לדין גרושה וגם מסברא אין חילוק ומה דנקט הרמב״ם מגורשת אפשר דקמ״ל אפילו יטעון המגרש לא בעלתי׳ מכ״מ אינה נאמנת אפילו לחייבו שבועה על הקנס דנשואה בחזקת בעולה היא וכיון דאין לה קנס ה״ה דגם צער בושת ופגם אין לה ע״פ דברי הרמב״ם הנ״ל והנה אמת שהטור אהע״ז (סי׳ קע״ז) כתב על הרמב״ם ואיני מבין דבריו דמה תלוי צער בשת ופגם בקנס דאע״פ שאין להן קנס למה לא יהיו להן שאר הדברים וגם הב״י כתב גם בעיני יפלא וצ״ע ע״ש עם כל זה אחר שלא חלקו בפי׳ להלכה על הרמב״ם אין לזוז מפסקו ובפרט שלא להוציא ממון מהמוחזק אמנם הבית שמואל (שם) כתב שמשמע מהרמב״ם דדוקא קנס אין לבעולה אבל בשת יש גם לבעולה והוציא כן ממה כתב הרמב״ם ה׳ נ״ב (פ׳ ב׳) באו עלי׳ שנים א׳ כדרכה והשני שלא כדרכה חייב השני בבשת אע״ג דכבר היא בעולה מראשון ע״ש אכן לפ״ז לכאורה דברי הרמב״ם סותרים זה את זה שהרי כתב שם (פרק ד׳ הלכה ט׳) עשרה בנות שאין להם קנס ובתוכן גם אותן שנפטר מטעם שהן בעולות כגון מגורשת והגיורת והשבוי׳ והמשוחררת ואעפ״כ כתב שם (פרק ד׳ ה׳ י׳) דכמו שנפטר מקנס נפטר ג״כ מבשת ופגם א״כ משמע דבעולה אין לה בשת ויש ליישב ע״פ מש״כ הרמב״ם שם (פרק ב׳ ה׳ ח׳) אבל אין בשת ופגם של בת שלא נבעלה כלל כבשת ופגם של זו שנבעלה שלא כדרכה עכ״ל וכמו כן י״ל דמה שכתב דמגורשת אין לה בשת היינו בשת של בתולה אין לה אבל מכ״מ קצת בשת יש גם לבעולה שנאנסה שהרי נתביישה עי״ז וזה חייב לשלם לה. היוצא מזה בנדון שלפנינו ע״פ טענתו פטור מכלום אבל ע״פ טענתה שנאנסה יהי׳ חייב לשלם לה בשת של בעילה וכיון שהוא כופר חייב לשבע שבועת היסת.\\\vspace{0pt}

ב – שישא אותה לאשה בזה ודאי אין לה טענה עליו שאפילו הוא כדברי׳ שאנוסה היא הרי מצות לו תהי׳ לאשה היא בלבד באונס את הבתולה שהיא נערה וכן נפסק גם בש״ע אהע״ז (סי׳ קע״ז) אבל לא באלמנה מן הנישואין ולא עוד אלא אפילו היתה נערה בתולה שטענה כן מכ״מ כיון שהוא כופר נאמן בלא שבועה שאין משביעין רק על תביעת ממון אבל לא על האיסור ואפילו נימא דגם תביעת ממון תלי בזה שהרי אם יצטרך לישא אותה צריך ליתן לה שאר וכסות מכ״מ כיון דעיקר התביעה היא על קיום המצו׳ שישאנה אף שתלי בה גם תביעת ממון מכ״מ אין משביעין על זה כמבואר בח״מ (סי׳ פ״ז סעיף כ״ה) דאם עיקר התביעה משום איסור אף שנ״מ גם לענין ממון מכ״מ אין משביעין ולכן גם בנדון זה אין משביעין לנתבע.\\\vspace{0pt}

ג – אמנם לענין התביעה שיפרנס את הולד אם נאמן בלא שבועה בזה יש לדון ולחקור דהריב״ש (סי׳ מ״א) כתב דאם תובעת אותו שיפרנס הולד והוא מכחיש נאמן אפילו בלא שבועה דהתורה האמינתו לאב לומר זה בני וזה אינו בני כדילפינן מיכיר יכירנו לאחרים ועוד דאפילו ודאי בא עלי׳ מכ״מ אמרינן שמא זנתה גם עם אחרים ונתעברה מאחד מהם ע״ש אמנם דברי הריב״ש אילו הביא הבית שמואל (סי׳ ד׳ ס״ק מ״א וסי׳ כ״ב ס״ק ה׳) אבל לא ראיתי לב״י וש״ע ורמ״א שהעתיקו דבריו כי מה שהביא הרמ״א סי׳ כ״ב ס״ב הוא שו״ת הריב״ש (סי׳ רס״ה) ששם מדבר שהיא אינה נאמנת עליו לביישו או לקנסו וכן איירי המהרי״ק שרש קפ״ט שהביא שם אבל פסק זה שנאמן בלא שבועה על תביעת פרנסת הולד שהביא הריב״ש סי׳ מ״א ומ״ב זה לא העתיק וביותר יש לתמו׳ שהרמ״א בח״מ (סי׳ פ״ז סכ״ה) העתיק מה שכתב הריב״ש בשו״ת הנ״ל שאם תובעת אותה שנדר לה דבר באתננה והוא כופר חייב לישבע ע״ש ועיקר הדין של אותה שו״ת שא״צ לישבע על טענתה אודות הולד לא העתיק ולא ראיתי לשום א׳ מן הפוסקים שפסק כן חוץ מהבית שמואל והבי״ה אלא השו״ת ר״י מינץ (סי׳ ה׳) שכתב סוף התשובה שם אע״ג דנאמנת לומר מפלוני נתעברתי ה״מ להכשיר הולד אבל להוציא ממנו שכרה או פרנסת הולד אינה נאמנת כיון דאיכא למימר כשם שזינתה עם זה כך זינתה מאחר עכ״ל ואע״פ שהוא לא הזכיר שהוא פטור משבועה מכ״מ ע״פ טעמו משמע שדעתו כן אכן כבר העיד השו״ת שב יעקב (ח״ב סי׳ ג׳) שבכמה קהלות פסקו שבועה על הנחשד אם תבעו אותו לזון את הולד רק שכתב שנפלא בעיניו מה מקום לשבועה זו נגד פסק הריב״ש ע״ש וגם פה בב״ד שלנו נפסק כן כמה פעמים ואחרי שגדולי אנשי השם ובעלי הוראה ישבו על כסא הוראה פה ומכללם הגאון כנסת יחזקאל ובעל התומים מסתמא ע״פ פסקם נהגו כן גם אמרו לי הדיינים דקהלתנו נ״י שטרם באתי לפה כתבו להגאון בעל חתם סופר זצ״ל על זה והסכים גם הוא לחייב הנתבע שבועה למיגדר מלתא. ועכ״ז הי׳ קשה בעיני לפסוק נגד הריב״ש עד שמצאתי מקור הפסק הזה בשו״ת תשב״ץ חלק ב׳ סי׳ י״ט שכתב שם נגד הריב״ש שאין לפטור הנתבע מן השבועה וכתב שם אמת הוא שהרב (הוא הריב״ש) ז״ל השיב כן בהיותו במלייאנה ובספרו הדברים האלה תמהתי על הוראתו ונשאתי ונתתי עמו בזה וכו׳ ע״ש שהאריך לבאר טעם הוראתו ותוכן דבריו דסברא כשם שזינתה ע״ז כך זינתה עם אחר לא נאמר רק לרב ביבמות ס״פ אלמנה לכה״ג אבל לא לשמואל דפסקינן כוותי׳ ואפילו לרב ג״כ רק ללישנא קמא ורק בבא על ארוסתו בבית חמיו לענין חשש ממזרת אבל לא באונס או מפתה אשה והביא ראי׳ ממה דאמרינן שם דכהן האונס ומפתה בת ישראל הולד מאכילה בתרומה ולא אמרינן שמא זינתה עם אחר אף שתרומה היא איסור מיתה לזרה ועל סברת הריב״ש שהתורה האמינתו לאב השיב דודאי הוא נאמן לומר שאינו בנו אבל רק לענין יוחסין ולא לענין תביעת ממון שיש לה עליו והוא כופר וסיים שם התשב״ץ ואל יקשה עליך היותי חולק על הרב ז״ל בחיוב שבועה זו ותחשבני יוצא משיטת המוסר בהיותי משיב את הארי לאחר מותו שאין לדיין אלא מה שעיניו רואות ואין לנו להיות נושאים פנים בתורה ובמלתא דתלי בסברא יתגלו לאחרונים דברים לא שערום ראשונים עכ״ל גם מתרומת הדשן בפסקיו (סי׳ ל״ז) נראה שדעתו דפנוי׳ שטוענת על איש שנתעברה ממנו ויזון הולד שחייב לישבע שכתב שם ואפילו למהר״ם דס״ל דנתקנה שבועת היסת אאיסורא וכו׳ ותו דמאן לימא לן דתקנו חכמים לזון בניו כשהן קטני קטנים וממזרים וכו׳ ע״ש ודבריו הביא גם הש״ך ח״מ (סי׳ פ״ז ס״ק נ״ז) ע״ש שנראה מדבריו דאם פנוי׳ תובעת שבועה על מזונות הולד ולא על האיסור שגם למהרא״י חייב שבועה וכפסק התשב״ץ ובזה מסולקת תמיהת השב יעקב על הקהלות שנפסק בהם שבועה על הנחשד שהם סמכו על התשב״ץ אחר שראה הריב״ש וחלק עליו והב״י והרמ״א אפשר שלא רצו להכריע בין הריב״ש לתשב״ץ ולכן השמיטו דין זה לגמרי. גם נ״ל שמי שע״פ דין חייב לשבע וקפץ ונשבע מעצמו קודם שנפסקה השבועה עליו שלא בפני בעל דינו שלא נפטר על ידי זה מהשבועה שאע״פ דלפי המבואר בח״מ (סי׳ פ״ז סעיף כ״ג) אם השביעוהו שלא בפני הלה שכנגדו נפטר מכ״מ הלא מפורש שם דוקא אם השביעוהו ואם נשבע כתקנו ועל דעת ב״ד נפטר אבל מי שנשבע מעצמו הרי לא השביעוהו על דעת ב״ד וזה אפילו בדיעבד לא נקרא שבועה משום קניא דרבא דאם לא אמרינן כן מה הועילו חכמים בתקונם להשביעו על דעת ב״ד שלא יהי׳ בלבו כוונה אחרת הלא כל אחד יערים ויקפוץ וישבע לדעתו ויהי׳ נפטר משבועת ב״ד אלא ודאי דזה אפילו בדיעבד לא מהני וכמשמעות הש״ע הנ״ל.\\\vspace{0pt}

והיוצא מזה שאם המשרתת תבעה להגביר לדין ומבקשת ממנו צער ובשת ופגם וקנס הנתבע חייב שבועה שלא בא עלי׳ באונס אבל ל״צ לשבע שלא בא עלי׳ כלל שאפילו בא עלי׳ רק שהי׳ לרצונה פטור מכלום אבל אם לא רוצה לשבע על טענת אונס אז חייב לשלם לה דמי בושה כפי שומת ב״ד לפי כבודה אבל מכל שאר תביעות נפטר גם בלא שבועה ואם תתבע ממנו שישאנה אזי פטור מתביעה זו אפילו בלא שבועה ואם תתבע אותו ליתן שכר מזונות הולד אשר ילדה אזי תלי זה במנהג הקהלה אם לחייבו שבועה וכפי אומדן דעת ב״ד אם לפסוק כריב״ש או כתשב״ץ ואם ב״ד יחייבו שבועה אז צריך לשבע שבועה זו שלא בא עלי׳ כלל אפילו לרצונה שלזון הולד יתחוייב לשיטת התשב״ץ אפילו היא מפותה והשבועה שנשבע הגביר לפני ס״ת שלא בא עלי׳ לא פטרתו משבועה שיטיל עליו הב״ד על תביעת בושת ועל תביעת מזונות הולד כיון שלא נשבע לדעת ב״ד אבל מכ״מ אפילו ישלם ולא ישבע אין רשות לשום אדם לקנסו או לחשדו וכל שכן לחרפו דאולי אפילו שבועת אמת לא רוצה לשבע אמנם גם את המשרתת אין לחרף ואפילו ישבע כנגדה שבועת ב״ד דשבועה זו לא מהני רק לפטור אותו מתביעת ממון שיש לה עלי׳ אבל לא להחזיקה לזינתה עם אחר וכל שכן שלא לקרותה ארורה ונואפת אחר שע״פ דברי׳ אנוסה היתה והמשפט לאלקים הוא אשר לו נגלו נסתרות ואין לדיין רק מה שעיניו רואות, כנלענ״ד, הקטן יעקב.\\\vspace{0pt}

\end{multicols}\newpage

\newchap{סימן קנט}
\begin{multicols}{2}
ב״ה אלטאנא, תמוז תרי״ב לפ״ק. לק״ק האלבערשטאדט.\\\vspace{0pt}

שאלה – עשיר א׳ הניח ברכה אחריו שמסר סך לקהל לקרן קיימת עלמין והפירות שיעלו לכל שנה יותנו ללומדי תורה בק״ק השד״ט ומינה למשגיח על הד״ט אחד מקרוביו ונתן לו רשות לחלק הרווחים בין לומדי תורה כפי ראות עיניו ושאחר העדרו יותנו הרווחים לאשר יפרט הוא או באי כחו ובל יעברו ב״כ את תפקידם אשר יפקוד להם המשגיח הזה בכתב יושר והנה המשגיח מת ולא נודע ממנו מי מינה לבאי כחו והרווחים נתנו מהקהל בכל שנה לאחד ממשפחתו הדר בק״ק הנ״ל וכשנעדר גם הוא בא בכל עת אחר ממשפחת המצוה והמשגיח הראשון תחתיו וכן נמשך זה שנים רבות וזה הרבה שנים שנתנו הרווחים בכל שנה ליד א׳ מהמשפחה אשר כידוע גם לאביו נתנו כל עוד אשר בחיים הי׳ ועתה קמו ראשי קהל דק״ק השד״ט בהטענה שאחר שהד״ט מיועד ללומדים בקהלתם והמשגיח לא מינה לב״כ לכן להם משפט החלוקה והאיש ממשפחתו אשר קבל עד עתה אומר כי לו משפט החלוקה כמקדם, והדין עם מי.\\\vspace{0pt}

תשובה – מבואר בח״מ (סי׳ קמ״ט ס׳ ל״א) דהמחזיק בהקדש עניים או ביה״כ אין לו חזקה ואם יש לו גזברים ממונים עליו יש לו חזקה וביאר הרמ״א שם דזה דוקא ביש לגזברים חלק באותו הקדש והיינו כיון דהטעם דהמחזיק בהקדש עניים אין לו חזקה הוא כמבואר בסמ״ע שם בשם הטור דמסתמא אין מוחה דכל א׳ יסבור שהחזיק וקבץ בשביל עניים ולכן אם הי׳ לגזברים חלק מסתמא הי׳ להם למחות ומדלא מיחו הוי חזקה וכן הדין בהי׳ ההקדש ע״י צואה שהיתה בשטר באופן שיש לה קול אם החזיק היורש ולא מיחו הגזברים הוי חזקה ואף שהיורש אין לו שטר מכ״מ נאמן לומר נאבד שטרי כמבואר בתשובת הרשב״א (סי׳ תרמ״ב) שממנה נובע דין הרמ״א וכפי המבואר ברמ״א (שם) המחזיק בשל קהל ג״כ דינו כמחזיק בשל הקדש וכן המחזיק במצוה.\\\vspace{0pt}

ולכן בנדון זה אף שלא ידוע שמינה המשגיח באי כחו מכ״מ כיון שעד עתה הי׳ א׳ מבני משפחתו והמקבל עתה ג״כ הוא מוחזק בהחלוקה זה שנים רבות הרי הם מוחזקים בהחלוקה ואמרינן מסתמא כן צוה המשגיח ומינה א׳ ממשפחתו בכל עת בק״ק השד״ט לב״כ ואף שאין כאן שטר לראי׳ הרי נאמן לומר נאבד השטר וא״כ ממנ״פ אם נחשב לראשי הקהל לגזברים שיש להם חלק בהד״ט הרי מדלא מיחו עד עתה נאמן המחזיק מטעם חזקה ואם לא נחשבם ליש להם חלק א״כ אין להם טענה גם עתה, כנלענ״ד, הקטן יעקב.\\\vspace{0pt}

\end{multicols}\newpage

\newchap{סימן קס}
\begin{multicols}{2}
ב״ה אלטאנא, יום ו׳ ז׳ מרחשון תרכ״ג לפ״ק. להרה״ג מחותני וכו׳ מ״ה דוד פרידמאן נ״י בק״ק מאהילעוו יע״א.\\\vspace{0pt}

הוקשה למר נ״י על פסק הרמ״א ח״מ (סי׳ רנ״ז) המקדיש קרקע מהיום ולאחר מותו או לאחר ל׳ יום לא קדוש כלל דהא אי אפשר לומר גוף מהיום ופירות לאחר זמן דהא קא אכיל פירות של הקדש עכ״ל והוא משו״ת הרשב״א סי׳ תקס״ג והרשב״א ביאר שם טעמו מהירושלמי פ׳ מי שאחזו גבי פלוגתא דרבי ורבנן באומר זה גטך מהיום ולאחר מיתה לפי שכל הנותן לחבירו מעכשיו ולאחר מיתה קיי״ל הגוף קני היום ופרי לאחר מיתה אבל לגבו׳ ולהקדש אי אפשר לומר כן לפי שאי אפשר שאילן של הדיוט יונק משדה הקדש אא״כ שייר מקום היניקה ואם תאמר מה ראית לבטל ההקדש מפני שלא שייר אדרבא נבטל השיור מפני ההקדש דזה אינו כיון דהקרקע בחזקת בעלי׳ ואתה צריך לבטל אחד מהם אוקי ממונא בחזקת מרי׳ וכדאמרינן בירושלמי הנ״ל לענין דאם הפקיר מעכשיו ולאחר ל׳ יום על דעתן דרבי מופקרת על דעתן דחכמים אינה מופקרת דאע״ג דס״ל לענין גט מגורשת ואינה מגורשת דמספקא להו אי תנאה או חזרה הוי דלענין הפקר אי אפשר שיהי׳ השדה מופקר לאחרים ופירותיו לבעלים ולכן אוקי ממון בחזקת מרי׳ ובטל ההפקר לגמרי והוא הדין לענין הקדש כיון דאי אפשר שהשדה יהי׳ הקדש והפירות חולין ולכן אוקי ממונא בחזקת מרי׳ ובטל ההקדש לגמרי עכ״ד ועל זה הוקשה למר נ״י שהרי מסיים הירושלמי שם אף בהקדש כן פי׳ ששואל אם דין הקדש כדין הפקר ומשיב כל עמא מודו אמירתו לגבו׳ כמסירתו להדיוט פי׳ דהקדש אינו דומה להפקר שדינו כמו מתנה שיכול להקדיש הגוף ולשייר לעצמו הפירות וחל ההקדש לאחר ל׳ לגמרי ובתוך הזמן אינו יכול לחזור בו ושכן מוכח ג״כ מירושלמי קידושין פ׳ האומר באומר שור זה עולה לאחר ל׳ יכול למוכרה בתוך ל׳ יום אבל באומר מעכשיו ולאחר ל׳ הוי כמסירה להדיוט וחל ההקדש ואינו יכול לחזור ושכן מוכח ג״כ ממה שפסק הרמב״ם ה׳ ערוכין פ׳ ו׳ ה׳ ט׳ וסיים מר נ״י שהתפלא מאוד על הרמ״א ונושאי כליו שהביאו דינו של הרשב״א בלא שום חולק ושודאי העיקר כהרמב״ם שיש לו ראי׳ מהירושלמי להלכה ולמעשה ומקום הניחו לו נושאי כליו של הש״ע להתגדר בזה.\\\vspace{0pt}

תשובה: ידע מעכ״ת נ״י שכבר הקדימו בספר קצות החושן סי׳ רנ״ז ורנ״ח שהתפלא ג״כ על פסק הרשב״א הנ״ל ממה שסיים הירושלמי דבהקדש כ״ע מודו דאמירה לגבו׳ כמסירתו להדיוט (וחלק ג״כ על פסק הרמ״א אמנם מטעם אחר שמפרש שהרשב״א איירי בחזר תוך ל׳ ועוד דהרשב״א איירי מהקדש בדק הבית אבל הרמ״א שע״כ איירי מהקדש לעניים אין הדין כן כמש״כ הרבינו ירוחם בנתיבות) ולענ״ד לק״מ שהרי הרשב״א ואחריו הרמ״א לא כתבו כן רק במקדיש קרקע הקדש אבל לא בהקדש דבר אחר ולכן אין קושיא ממה דמסיים הירושלמי דהקדש אינו כגט דהיינו בהקדש שלא שייך בו פירות כגון שהקדיש דבר שאינו צומח או בהקדיש שור לעולה שלא שייך בשור גיזה וולדות ששור אין לו גיזה ולא וולד או אפילו בדבר שיש לו פירות אבל הפירות מותרים לינק מהקדש כגון בהקדיש רחל לבדק הבית ושייר הגיזה לעצמו שיכול לעשות כן כדמוכח מחולין (דף קל״ה ע״א) בכל אילו מה שהקדיש מעכשיו ולאחר ל׳ הי׳ כוונתו שלא יחול ההקדש רק לאחר ל׳ ושיהי׳ ברשותו לחזור עד ל׳ בזה אמר הירושלמי שהקדש אינו כגט שיש ספק אם יכול לחזור אלא כיון דאמירתו לגבו׳ כמסירתו להדיוט אף שלא חל ההקדש עד אחר ל׳ מכ״מ אינו יכול לחזור בו ולכן לא מפרש הירושלמי לענין הקדש שדה זו הקדש כמו שמפרש לענין הפקר שדה זו מופקרת אבל במקדיש שדה מעכשיו ולאחר ל׳ כיון דבשדה יש פירות ואילנות שמשדה הקדש ינקי והדין בזה לענין הדיוט הגוף מהיום ופירות לאחר ל׳ וכיון דזה אינו יכול להיות כן דאי אפשר שיינקו פירות חולין משדה הקדש הוי כמו הפקר שג״כ אמרינן כיון שאי אפשר להיות השדה הפקר והפירות שלו ולכן לא חל ההפקר דאוקי שדה בחזקת בעלים ה״ה ג״כ בהקדש שדה אבל בשאר הקדש ודאי חל מיד שלא יכול לחזור בו. וכזה נראה מדברי הרשב״א בנדרים (דף כ״ט) שמה דאמר הירושלמי אמירתו לגבו׳ כמסירה להדיוט הוא לענין חזרה בלבד רק ששם נראה דס״ל דאפילו בדבר שיש לו פירות אחר ל׳ חל ההקדש ומזה חזר בתשובה וס״ל דההקדש לא חל כלל כיון דלענין הפירות לא יכול לחול אבל ההקדש דירושלמי איירי באין לו פירות ובזה גם הרשב״א מודה דחל ההקדש לאחר ל׳ שלא אמר דינו רק לענין שדה.\\\vspace{0pt}

ומה שתימה מר נ״י על התוספ׳ ביבמות (דף צ״ג ע״א) שמסופקים אם יש להשות מעכשיו ולאחר ל׳ למהיום ולאחר מיתה לענין גוף מהיום ופירות לאחר זמן מהירושלמי דגטין דנראה דאין חילוק לענ״ד אין קושיא דהתוספ׳ לא כתבו כן רק לר׳ יוחנן דס״ל כן והרי לענין מתנה גם ר׳ יוחנן בירושלמי שם נקט מהיום ולאחר מיתה ומה דמשו׳ סתם ירושלמי שם לענין גט שחרור והפקר גם לאחר ל׳ לזה אפשר דפליג בזה על ר׳ יוחנן ועוד יש לחלק ואין להאריך בדברים הנסעפים להרבה, כנלענ״ד, הקטן יעקב.\\\vspace{0pt}

\end{multicols}\newpage

\newchap{סימן קסא}
\begin{multicols}{2}
ב״ה אלטאנא, יום ב׳ ח׳ א״ש תרי״ט לפ״ק. להרה״ג וכו׳ מ״ה יצחק דוב הלוי נ״י הגאב״ד דק״ק ווירצבורג יע״א.\\\vspace{0pt}

מה דבדיק לן מר נ״י באחד שנתן שטר חח״ז לבנותיו וי״ח כנהוג ועתה רוצה להקנות קצת מנכסיו בתורת צוואת בריא לקצת מבנותיו שיש לו מאשתו שניי׳ חוץ מחלק ירושה שיפול להם וכן לאשתו אם רשאי לעשות כן.\\\vspace{0pt}

תשובה: לפי המבואר בח״מ (סי׳ רפ״א ס׳ ז׳) ברמ״א הכותב לבתו שתקח לאחר מותו כחצי ח״ז דינו כירושה בעלמא ובעל חוב וכתובה קודמין למתנה זו וכן עשור נכסי הבת וכל ימי חיי הנותן יכול למכור הנכסים אע״פ שכתב לה מהיום ולאחר מיתה עכ״ל מזה נראה שאין חשש איסור להקנות מנכסיו לאחר אכן המהרי״ל בשו״ת (סי׳ צ״ב) כתב שאין נכון ליתן נכסיו לאשתו כשיש לבנותיו שטרי ירושות דהוי מערים לאפקועי מהן את ששעבד כבר וכן אין להוסיף לאשה על נכסי מלוג שלה כדי להשליטה בתוקף על הנכסים דגם התם הוי כהערמה לאפקועי מן שטרי ירושות עכ״ל וכוותי׳ פסק הרמ״א אהע״ז (סי׳ ק״ח) ואע״ג דהח״מ שם ביאר דברי הרמ״א כגון שרוצה להערים וליתן במתנה אח״כ רוב נכסיו לבניו או שיערים ליתן לאשתו מתנה מרובה וכ״כ הב״ש שם מכ״מ ודאי רוב נכסיו או מתנה מרובה לאו דוקא דה״ה אם מערים על מיעוט נכסיו רק דאתי לאפוקי בנותן מעט שאין דרך להקפיד וכ״נ מדברי מהר״יל ורמ״א שלא חלקו בין רוב נכסיו למיעוט והנה המרדכי פ׳ יש נוחלין כ׳ בשם הר״מ באחד שנתן לחתנו שטר מתנה שיטול חלק וכו׳ כאחד מן היורשים שיכול ליתן מתנה כל ימי חייו שהרי לא כתב לו רק שיטול כאחד מן היורשים והמתנה שנתן לאחרים תמעט חלק כל אחד מהיורשים א״כ נתמעט נמי חלק חתנו ע״ש והם דברי רמ״א הנ״ל ולכאורה זה מתנגד לפסק המהרי״ל הנ״ל וא״כ יש סתירה ברמ״א שפסק כמהרי״ל וכמרדכי אבל לאחר עיון זה אינו דודאי אין כוונת הנותן שטר ח״ז שלא יהי׳ לו רשות לשלוט בנכסיו כל ימי חייו רק שמתנה שלא יערים ע״י ערמה לשנות החלקי ירושה לאחר מותו ולכן המרדכי איירי בנותן מתנה מיד בחייו ואין בזה ערמה אבל המהרי״ל איירי בנותן על אחר מותו ולכן כתב שאין ליתן לאשתו דבזה לא הוי רק על לאחר מותו דבחייו מה שקנתה אשה קנה בעלה ובמצו׳ על נכסיו באופן שלאחר מותו לא יטלו בעלי שטר ח״ז מה שנכבת להם זה הוי ערמה ואסור וכה״ג ראיתי א״ע גם בנחלת שבעה שמחלק בכך ועוד נראה לחלק דהמרדכי איירי בנותן שטר מתנה בעלמא שלא התנה בו שלא יערים בשום תחבולה אבל ע״פ נוסח שטר ח״ז שלנו שמקבל בח״ח ובשד״א שלא יערים לגרע כח השטר וכו׳ בזה ודאי אסור לעשות וראיתי בשו״ת רבי עקיבא איגר ז״ל (סי׳ קכ״ט) שהאריך אם בדיעבד מהני מה שנתן מתנה לאחר בערמה כיון שעבר על שבועתו ואי עבר על מימרא דרחמנא אי עביד לא מהני ע״ש ועכ״פ פשיטא לו שיש איסור בדבר ולכן לא מצאתי תקנה בזה להתיר לכתחלה לעשות כן אם לא בהסכמת בנותיו בעלות השטרות של ח״ז וצריך לזה גם הסכמות בעליהן שאע״פ שהשטרות נכתבו על שמן מכ״מ הרי מבואר באהע״ז (סי׳ צ״א) דאשה שהכניסה שטר חוב לבעלה וחזרה ומחלתו אינו מחול מפני שידו כידה ואין חילוק בין נכסי מלוג לנכסי צאן ברזל כמש״כ הח״מ וב״ש שם וא״כ ה״ה במחלה מקצת משטר ח״ז שאינו מחול בלא הסכמת בעלה, כנלענ״ד, הקטן יעקב.\\\vspace{0pt}

\end{multicols}\newpage

\newchap{סימן קסב}
\begin{multicols}{2}
ב״ה אלטאנא, י״ג תשרי תרי״ח לפ״ק. לחתני הרה״ג וכו׳ מ״ה זלמן כהן נ״י אב״ד דק״ק מאסטריכט יע״א.\\\vspace{0pt}

על דבר שאלתך בנטבע שנמצא מהדייגים וקברוהו בקבר של נכרים ועתה נודע שהי׳ ישראל ע״י שנמצאו אצלו טלית ותפילין ולכן רוצים ישראל להוציאו ולקברו בקבר של ישראל אבל השררה מונעת באמרה שטלית ותפילין אין ראי׳ שמא מצאם או גנבם והוא נכרי ולכן יביאו ראיות יותר, אם מותר לפתוח הקבר לראות אחר בריתו אם הוא מהול ואם נמצא מהול אם מותר להוציאו מקברו.\\\vspace{0pt}

הנה בעיקר הדבר כבר הראית לנכון שכבר התיר החכם צבי להוציא מקבר נכרי כדי לקבור בקבר ישראל רק נסתפקת אם מותר לעשות כן מטעם ניוול שצריך לפתוח הארון ולראות בניוולו והנה גם בזה הראית מקום לנכון ממה דאמרינן ב״ב (דף קנ״ד) אנן זוזי יהבינן לי׳ לינוול ולינוול רק שצדדת ששם עכ״פ הוא נתחייב בהמעות וא״כ יש טענה כנגדו משא״כ הכא וכן הראית מקום ממה דאמרינן חולין (דף י״א) וכ״ת משום אבוד נשמה וכו׳ וגם בזה נסתפקת אם יש ראי׳ משם.\\\vspace{0pt}

אמנם לפי מה דפסקינן בח״מ (סי׳ רל״ה ס׳ י״ג) אסור לפתוח הקבר של קטן שמכר לידע אם כבר הביא ב׳ שערות משום ניוול המת וגם בשו״ת נ״ב מה״ת י״ד סי׳ קס״ד נשאל בתנוק שמת קודם שמנה ימים ונקבר בערלתו אם יפתח הקבר למולו משום ניוול והשיב אם כבר עברו איזה ימים לא יפתח כיון דודאי יש ניוול אכן לענ״ד אין זה דומה לנדון זה דשם הניוול דערלה לקטן שלא הגיע למילה אינו ניוול גמור ועדיף שיתנוול בזה משאם יתנוול ע״י פתיחת הארון אבל הכא דודאי יש כאן הניוול הגדול שנקבר בין א״י שעתידים להשליך עצמותיו לחוץ כמנהגם ועוד שאר חששים דניוול ודאי ניחא לי׳ טפי בניוול זה דפתיחת הקבר מניוולים הללו ועוד דבשעת הפתיחה אין כאן רק ספק ניוול דדלמא ערל הוא ואם ימצא מהול אנו מצילין אותו מודאי ניוול ולכן נ״ל דמותר לפתוח הארון ואם נמצא מהול מותר להוציאו מקברו ולהביאו לקבר ישראל ובלע המות לנצח אכיה״ר, כנלענ״ד, הקטן יעקב.\\\vspace{0pt}

\end{multicols}\newpage

\newchap{סימן קסג}
\begin{multicols}{2}
ב״ה אלטאנא, יום ה׳ טו״ב סיון תרכ״ו לפ״ק. להאלופים היקרים פקידים ואמרכלים על נדבות ארץ הקדושה תוב״ב נ״י בק״ק אמשטרדם יע״א.\\\vspace{0pt}

בדבר השאלה אודות ר׳ אהרן ב״ר שלמה זלמן ז״ל אשר תקע אהלו בירושלם עיה״ק תוב״ב ובקש שיקבלו אותו הכולל הו״ד ה״י להיות כאחד מאנשי הכולל ולקבל חלוקה ביניהם בטענה שהרבה שנים כבר נסע מפולין וגר בק״ק אמשטרדם יע״א טרם נסע לארץ הקדושה ובזה יהי׳ נדון כאנשי האללאנד ורוב אנשי כולל הו״ד ה״י רוצים לקבלו אבל יש יחידים מערערים שאין לו לקבל חלק מהנדבות רק מאנשי כולל עיר מולדתו ולא יגרע חלק מאנשי הו״ד ה״י על ידו וכנגד מה שרוב הכולל רוצים לקבלו טוענים אין הולכין בממון אחר הרוב, הדין עם מי.\\\vspace{0pt}

תשובה – אם שאין לי בירור על איזה ענין ינהגו הכוללים בעיה״ק בקיבול היחידים ביניהם אם על פי ארץ מולדתם או אם על פי ארץ תושבותם יתחלקו וכמה שנים צריך היחיד להיות תושב במדינה שיחשב כבני המדינה עם כל זה נלענ״ד שאם ירצו רוב הכולל לקבל היחיד להיות כאחד מהם אין המיעוט יכול לעכב בטענה שאין הולכין בממון אחר הרוב דכלל זה לא נאמר אלא שאין הולכין אחר הרוב להוציא ממון מהמוחזק כמבואר בח״מ סי׳ רל״ב ס׳ כ״ג במוכר שור לחבירו ונמצא נגחן דלא אזלינן בתר רובא דקונין לחרישה להוציא ממון מיד המוכר וכן מבואר בש״ך סי׳ רצ״ב ס״ק כ״ז דרק נגד חזקת ממון אין הולכין אחר הרוב אבל היכא דליכא חזקה הולכין ובנדון זה אין להמערערים חזקה נגד הרוב כי החלוקה אשר עלי׳ יטענו שנגרעה מהם עדיין לא באה לידם ועוד מטעם אחר לא שייך בזה אין הולכין במא״ה דזה אמרינן רק ברובא דליתא קמן אבל ברובא דאיתא קמן הולכין אחר הרוב כמש״כ בשו״ת תרומת הדשן סי׳ שי״ד ואף שהמרדכי מוכיח דאפילו באיתא קמן אין הולכין אחר הרוב מכ״מ בהגה שם מוכיח מרשב״ם ומגליון תוספ׳ דברובא דאיתא קמן גם שמואל מודה דאזלינן בממון בתר רוב ולענ״ד מוכח כן דבסנהדרין (דף ג׳) הקשו התוספ׳ על שמואל דאמאי לא נילף בק״ו מדיני נפשות דהולכין בממון אחר הרוב וא״ל דשאני דנ״פ דהוי רובא דאיתא קמן שהרי גם ברובא דליתא קמן הולכין אחר הרוב בדנ״פ כדמוכח בפ׳ בן סורר ותרצו דצ״ל דרובא לרדיא זבני לא חשיב רוב גמור והב״ח סי׳ רל״ב מבאר דבריהם למה לא הרי רוב גמור ע״ש הרי מבואר מדאזלינן בד״מ בתר רובא כדתנן שנים אומרים חייב וא׳ אומר זכאי חייב ואמרינן חייב אפילו להוציא ממון כדמשמע מלשון חייב ע״כ צ״ל לשמואל או דמודה ברובא דאיתא קמן דאזלינן בתר רוב או דברוב גמור אפילו דליתא קמן אזלינן בתר רוב ולכן ברוב גמור דאיתא קמן ע״כ גם שמואל ס״ל דאזלינן בתרי׳ ואף דלפי דברי התוספ׳ ב״ק דף כ״ז יש חילוק בין רוב דיינים לשאר רוב דבריהם סתומים וכבר טרחו האחרונים לבארם ועכ״פ מדבריהם בסנהדרין נראה דלא ס״ל כן אלא דהטעם או משום דהוי רוב דאיתא קמן או משום דהוי רוב גמור ולענ״ד מטעם זה כתב בתשובת מהר״ם ופסק הרמ״א בח״מ סי׳ קס״ג כל צרכי צבור שאינן יכולים להשוות עצמם יש להושיב כל ב״ב הנותנים מס ויקבלו עליהם שכל אחד יאמר דעתו לשם שמים וילכו אחר הרוב ומבואר שם דהרוב יכול לכוף להמיעוט שיתנו חלקם וע״כ הטעם שזה רוב גמור דאיתא קמן ועל כן אזלינן בתרי׳ אפילו להוציא ממון מהמוחזק ועל כן בנדון השאלה דהוי רוב גמור דאיתא קמן וגם אין אחד מהם מוחזק ודאי רוב דיעות יכריעו וכמו בכל צרכי צבור ע״פ פסק הרמ״א ורק מה שנאמר שם דהב״ב שנותנים מס בלבד יתנו דעתם והטעם מפני שלהם נוגע הדבר בשכר או הפסד מטעם זה בעצמו בנדון זה גם המקבלי חלוקה יאמרו דעתם לשם שמים ואם הרוב ירצו לקבל את ר׳ אהרן הנ״ל להיות כאחד מאנשי כולל אין בכח המיעוט לעכב ויקבל חלקו כאחד מהם והי׳ זה שלו׳, כנלענ״ד, הקטן יעקב.\\\vspace{0pt}

\end{multicols}\newpage

\newchap{סימן קסד}
\begin{multicols}{2}
ב״ה: אלטאנא, יום ו׳ כ׳ תמוז תרכ״א לפ״ק. להרה״ג וכו׳ מ״ה בירך אברהם אויסטערליץ נ״י הגאב״ד דק״ק סקאליטץ יע״א.\\\vspace{0pt}

בדיק לן מר נ״י בשאלה דאתי לקמי׳ בסופר שכתב ס״ת לאחד בשכר ונמצא בו אזכרה אחת יתירה וכיון דיש אוסרים לקדור השם רצה הבעל הבית שיכתוב לו הסופר יריע׳ שלמה חדשה בחנם והסופר טוען קים לי בהוראת רבי שקבל מרבו לקלף השם וכשר ואין לו עלי כלום, הדין עם מי.\\\vspace{0pt}

הנה בשאלה לא נתבאר מי הוא התובע ומי הנתבע דהיינו אם הסופר כבר קבל מבעה״ב שכרו והבעל הבית תובע ממנו שיכתוב לו יריעה או יחזיר לו משכרו שקבל כדי כתיבת יריעה או אם הסופר הוא התובע שבעה״ב רצה לעכב משכרו עד שיכתוב לו יריעה חדשה אמנם ממה שהזכיר בשאלה דהסופר טוען קים לי משמע שהבעה״ב הוא התובע דקים לי לא יכול לטעון רק המוחזק כמבואר בח״מ סי׳ כ״ה ויע״ש בש״ך ס״ק י״ז ואם הדבר כן אז פשיטא שיש לדון שהדין עם הסופר שא״צ לכתוב יריעה אחרת ואע״ג שמה שרוצה לקלף את השם אין להורות לו דרבו האוסרי׳ לקלוף כמו שהיטב לראות מעכ״ת נ״י ממש״כ בבדק הבית י״ד סי׳ רע״ו בשם התשב״ץ ובש״ך שם סי׳ רע״ה ס״ק ג׳ ואף שהמג״א סי׳ ל״ב ס״ק כ״ו הביא בשם שו״ת מהר״י הלוי שדייק מתשובת הרא״ש שמותר לקלוף השם כבר השיב עליו בשו״ת פרח שושן חא״ח כלל ב׳ גם בשו״ת דבר שמואל סי׳ קס״ה אוסר עם כל זה אין למחות בידו אם רוצה לקדור השם עם עוד איזו תיבות ולכתוב התיבות על המטלית והשם על הקלף וכמו שכ׳ הט״ז י״ד סי׳ רע״ו דבזה שוו רוב הראשונים והאחרונים להתיר התשב״ץ והרשב״ש בנו ושו״ת חנוך ב״י סי׳ ע״ו ותשובת יד אלי׳ ובני יונה כתב בפשיטות להתיר ודחה דברי האוסרים ואף שהרא״ש והר״י בן הרא״ש אוסרים עכ״ז אחר שרוב הראשונים והאחרונים התירו לקדור בפשיטות אין למחות בהסופר בשאומר קים לי כוותייהו ועוד אפילו יעבור הסופר ויקלוף השם כמו שאמר שקבל מרבו אף שעשה שלא כדין עם כל זה אין לפסול היריעה על זה בדיעבד ואע״פ שהר״י בן הרא״ש בש״ת סי׳ ג׳ שנשאל על היריעה שנקדר השם ממנה מה שאסור לפי דעתו ודעת אביו ז״ל השיב א״א ז״ל כתב מוטב שיסלק היריעה ממה שיקדור השמות אולי אם יקדור לא הי׳ מסלק היריעה מכ״מ כיון שנעשה בה עבירה טוב להחליפה עכ״ל יש לומר דזה דוקא במה שנעשה עבירה לפי דעתו אבל בקילוף אין האיסור מוחלט כ״כ כנ״ל ועכ״פ בקידור ודאי אין לפסול היריעה וא״כ יכול הסופר לומר קים לי. אמנם כל זה אם הבעל הבית הוא התובע אבל אם הסופר הוא התובע שהבעה״ב מעכב לו שכרו עד שיכתוב לו יריעה חדשה לענ״ד הדין עם הבעל הבית דיכול לומר קים לי כדעת הרא״ש ור״י בנו שאסור לקדור לשם ואפילו בדיעבד טוב להחליפה כמש״כ הר״י בן הרא״ש בשו״ת הנ״ל ועוד שהבעה״ב יכול לומר אי אפשי בס״ת שנכתב בה על המטלית ששיווי שלה אינה כס״ת שלמה שמה שפשיטא למעכ״ת נ״י דמשום מטלית אחת אין שיווי שלה הוא פחות למוכרה לדברים שמותר למכור ס״ת לענ״ד אינו כן וגם בשו״ת חתם סופר סי׳ רנ״ט ור״ס כתב בפשיטות שאם ימצא טעות בשם ויקדור ויכתוב על המטלית שיפסד שיווי ס״ת שרצה להוכיח שדעת התוספ׳ שמותר לקלוף שם שנכתב בטעות ממה שהקשו בב״ב דף כ״א ע״ב ד״ה סופר מתא על פי׳ רש״י שכתב ס״ת בטעות דאין זה פסידא דלא הדר דיכול לתקנו והקשה אם ימצא טעות בשם מה יעשה הרי אם יקדור ויטלה מטלית נפסד שיווי ס״ת דאינו שו׳ כ״כ כמו שהי׳ שו׳ אם לא הי׳ מטלית אע״כ דמותר לקלוף עכ״ד (ואגב אזכיר שמה שרצה להוכיח מזה שדעת רש״י שאסור לקלוף ושזה כוונת רש״י לתרץ קושית התוספ׳ לענ״ד אינו ראי׳ דיש לפרש כוונת רש״י מה שכתב שהסופר כתב ס״ת בטעות שכתב בכל דף ד׳ טעיות שבזה צריך לגנוז ואין תקנה לה להגי׳ כדאמרינן מנחות [דף כ״ט] ולכן ל״ק קושית התוספ׳ עליו) הרי שפשיטא לו ג״כ דעי״ז נפסד שיווי הס״ת ושהסופר מקרי מותרה ועומד וצריך לשלם. גם מה שפשיטא למעכ״ת נ״י שטעות כזה הוא בכלל מה שכתב הרשב״א בשו״ת ופסקו הרמ״א בח״מ (סי׳ ש״ו) שבטעות שדרך סופרים לטעות אין הסופר חייב כלום לענ״ד אין זה פשוט די״ל דבכתיבת השם דרך הסופרים לשמור מטעיות בעבור שזה מעות שאין יכול לתקון וגם כיון דצריך כוונה בכתיבת השם שצריך לקדשו בפה טרם שיכתוב לא שכיח שיטעה ואם שגם לפענ״ד לא יצא מכלל ספק דיו שלא לחייב הסופר אם הוא המוחזק אבל לחייב בעה״ב המוחזק לשלם לסופר לא נ״ל שהרי הרשב״א כתב הטעם דטעיות הנמצאות בכתב שכיחי ואין לך סופר שידקדק בכתיבתו כ״כ שלא יטעה כלל וכל כיוצא בזה מן הסתם אין דעת בעלים להקפיד ואחולי מחלי׳ עכ״ל וטעות כזה להכפיל השם ודאי לא שכיח ומן הסתם ודעת הבעלים להקפיד כיון שאין תקנה בלי קדירה ומטלית.\\\vspace{0pt}

ועוד נלענ״ד דאפילו אין לבעה״ב טענה רק שנפגם יופי הספר והוא רוצה לעשות מצו׳ מן המובחר בספר נאה משום זה אלי ואנוהו ג״כ שומעין לו שבספרי בכורי יעקב על הל׳ סוכה ולולב הבאתי בסי׳ תרנ״ו ס״ק ג׳ הפלוגתא שבין הר״מ מינץ והמ״ל והחכם צבי במי שהפסיד לחבירו אתרוג יפה אם יכול לשלם אתרוג שיוצא בו ואף שיישבתי שם דעת הר״מ מהשגותיהם מכ״מ הערתי על השגה שכתב השעה״מ על פסק הר״מ ממה שכתב השיטה מקובצת בשם הראב״ד דמה דאמרינן בב״ק דגנב עולת שור גנב פוטר עצמו בכבש כיון דבעלים יוצאים בו דמזה הוכיח הר״מ מינץ דינו שמזה אין ראי׳ שהראב״ד כתב שמה שגנב פוטר עצמו בשה זה דוקא לענין כפל אבל קרן צריך לשלם להקדש ואם אמנם השבתי גם על השגה זו שהוכחתי שיש ב׳ שיטות בראב״ד מכ״מ סיימתי שם כיון שאין הכרע י״ל המוציא מחברו עליו הראי׳ ורק בתפס הניזק יכול לומר קים לי ולכן בנדון דידן שהבעה״ב הוא המוחזק יכול לומר המצו׳ מן המובחר שו׳ לי ממון ודומה למה דאמרינן בערכין (דף כ״ח) מחרים אדם את קדשיו אם נדבה נותן את טובתה כמש״כ שם ובפרט כיון שנותן ממונו להיות לו ספר נאה דבזה לכ״ע טובת הנאה ממון כמו שהוכחתי בתוספת בכורים בסי׳ הנ״ל ולכן בהא נחיתנא שאם קבל כבר הסופר שכרו והבעה״ב הוא התובע אז אין הסופר צריך להחליף היריעה אלא יקדור וישים מטלית אבל אם עדיין לא קבל שכרו משלם והסופר תובעו אז הבעה״ב יכול לעכב עד שיכתוב לו יריעה אחרת שלמה זהו מה שנלענ״ד.\\\vspace{0pt}

ומה שהקשה מעכ״ת נ״י בסוגיא דנדרים (דף ל״ו) דפירשו הרא״ש והר״ן קושית הגמרא על ר׳ יוחנן יביא חטאת חלב על חבירו משום דס״ל דנין אפשר מאי אפשר והרי ר׳ יוחנן ס״ל ביבמות (דף מ״ו) אינו גר עד שימול ויטבול ולא יליף מאמהות דטבילה לחוד מהני משום אין דנין אפשר מאי אפשר והניח בצע״ג.\\\vspace{0pt}

הנה אמת גם אנכי כתבתי בספרי ערוך לנר בסוכה (דף נ׳ ע״ב) דמאן דס״ל אין גר עד שימול ויטבול ע״כ ס״ל אין דנין אפשר מא״א ויישבתי בזה פסק הרמב״ם פ״ד מה׳ ביאת מקדש שהציץ דוקא עודהו על מצחו מרצה ממה שהקשה הכסף משנה שם עליו אכן מצאתי לאחד מגדולי הראשונים דעת אחרת בזה והוא הר״ש מקינון שכתב בספר כריתות לשון למודים שער ג׳ סי׳ ק״ט וז״ל ואע״ג דרבנן ס״ל דאינו גר עד שימול ויטבול לאו משום דלא ילפינן אפשר מאי אפשר אלא טעמא אחרינא איתא התם ור׳ יהושע אבות טבילה נמי הוי עכ״ל ויעיין ביבין שמועה כלל קי״ג מה שכתב בפי׳ דבריו עכ״פ נראה דס״ל דאפילו מ״ד דנין אפשר מאי אפשר ס״ל דאין גר עד שימול ויטבול וא״כ י״ל שגם דעת הרא״ש והר״ן כן ולבר מן דין י״ל לענ״ד דהך דנדרים לא דומה להך דיבמות דמה דאמר ר׳ יוחנן הכל צריכין דעת חוץ ממחוסר כפרה ויליף ממה שאדם מביא קרבן על בניו ובנותיו הקטנים אין זה ילפותא רק גילוי מלתא בעלמא שהרי שצריך דעת לא כתיב רק גבי קרבן שבא לכפרה כמו שכ׳ הר״ן שם וא״כ מה שיליף ממה שמביא קרבן על הקטנים לא מתנגד לשום ילפותא ולכן יליף בזה אפשר מאי אפשר לגילוי מלתא וכן שיביא חטאת ופסח על חבירו לא מתנגד לשום דבר דאכתי לא ידענו שלא יכול להביא אבל בגר שצריך מילה וטבילה ילפינן בכריתות מככם כגר וא״כ אי נילף מאמהות דגם טבילה לחוד מהני יהי׳ מתנגד להיקש זה בזה י״ל דגם ר׳ יוחנן מודה דאין דנין מא״א, כנלענ״ד, הקטן יעקב.\\\vspace{0pt}

\end{multicols}\newpage

\newchap{סימן קסה}
\begin{multicols}{2}
ב״ה אלטאנא, יום ו׳ כ׳ תמוז תר״א לפ״ק. להתורני יניק וחכים כמ״ה אהרן עהרליך נ״י בק״ק נאדאש יע״א.\\\vspace{0pt}

כתב אלי – בחולין (דף ק״מ) בתוספ׳ ד״ה למעוטי כתבו וז״ל וצ״ע בבהמת עיר הנדחת אם עבר והקדישה והקריבה אם הוא קרבן כשר כיון דלשריפה קאי עכ״ל והקשתי קו׳ עצומה לפ״מ דאמרינן בב״ק בשור הנסקל אם הקדישו אינו מוקדש מכרו אינו מכור כיון דהוא אסור בהנאה פקע מיני׳ שם בעלים והוי כמקדיש דבר שאינו שלו וא״כ בבהמת עיה״נ דאסורה בהנאה היאך יעלה על הדעת דיהי׳ הקרבן כשר דהרי לא חל עלה שם ההקדש כלל כיון שאינה שלו ובודאי מיירי שהקדישה לאחר שנעשה עיר הנדחת דאל״כ לא חל שם עיה״נ דאמרינן בגמרא בפ׳ חלק שללה ולא שלל שמים וכמו שכתב המהרש״א בתוספ׳ הנ״ל והיא קושיא חמורה לענ״ד וכאשר באתי במכתב לפני הרב הגאון החריף מ״ה שלמה קוועטש נ״י אב״ד דק״ק לייפניק יע״א ושאלתי ממנו מענה השיבני וז״ל הא דאמרינן בשור הנסקל אם הקדישו אינו מוקדש אין הטעם משום דפקע מיני׳ שם בעלים אלא הטעם הוא כיון דאמרה התורה לסקלו דמי׳ להך דאמרינן בתמורה (דף כ״ד) אם אמר על הבכור שיהי׳ עולה ביציאת הרוב דלא אמר כלום דדברי הרב ודברי התלמיד דברי מי שומעין וא״כ גם בשור הנסקל דהתורה צותה לסקלו אינו חל ההקדש משום דדברי הרב ודברי התלמיד וכו׳ דהרי מבטל מצות התורה לסקלו מה שאין כן בעיה״נ דהתורה אמרה לפי חרב וא״כ גם בשחיטה מקיים לפי חרב לא שייך לומר דברי הרב ודברי התלמיד וכו׳ וא״כ שפיר חייל ההקדש עכ״ד הגאון הנ״ל נ״י.\\\vspace{0pt}

תשובה – אם אמנם תקשה הקושיא לא הבנתי תירוץ הרב הגאב״ד דק״ק לייפניק נ״י שהרי בסנהדרין (דף קי״ב ע״א) בעי רב חסדא בהמת עיה״נ מהו דתתהני בה שחיטה וכו׳ וסלקא בתיקו א״כ לכאורה צריך להבין איך נסתפקו התוספ׳ אם הקרבן כשר כיון דעדיין לא נפשט ספק רב חסדא אם שחיטה לא מקרי לפי חרב ועושה אותה נבלה וצריך לומר דספק התוספ׳ הוא כיון דדרשינן שללה ולא שלל גבו׳ ולכן כל שהקדישה קודם שנעשה עיה״נ פשיטא שאין איסור עיה״נ חל עלי׳ אבל מסופקים אם נימא דההקדש חל גם אח״כ וממילא הוי שלל שמים ולא צריך לקיים בה דין עיה״נ או דלמא כיון דלשריפה קיימא לא חל שם הקדש עלי׳ אבל עכ״פ מבואר ע״י אבעי׳ דרב חסדא דאם נאמר דשחיטה מקרי לפי חרב אזי מקרי נבלה ומטמא וכמו שפי׳ רש״י בפירוש בסנהדרין (שם) ואיך רצה הרב נ״י לומר שהתוספ׳ מסופקים דלמא חל ההקדש כיון דע״י שחיטת הקרבן מקיים לפי חרב.\\\vspace{0pt}

אבל לענ״ד י״ל ביישוב קושיתו דהנה נסתפקתי לפי המבואר בח״מ (סי׳ רע״ה ס׳ כ״ד) המחזיק בנכסי הגר ובהפקר ואין דעתו לקנות אע״פ שגדר ובנה לא קנה עכ״ל ועוד נפסק שם (סי׳ שנ״ד ס׳ ו׳) גנב או גזל ולא נתייאשו הבעלים אינם יכולים להקדיש לא הגנב והגזלן ולא הבעלים וכתב הסמ״ע הגנב והגזלן לפי שאינו שלו והבעל לפי שאינו ברשותו וכתיב ואיש כי יקדיש את ביתו מה ביתו ברשותו אף כל דוקא שהוא ברשותו עכ״ל ונסתפקתי מי שלפניו הפקר שלא רצה לזכות בו והקדישו אם ההקדש חל עלי׳ אי נימא כיון שלא זכה בו לא מקרי שלו או נימא כיון שהי׳ יכול לזכות בו שאינו ברשות אחר יכול להקדישו ונ״ל לפשוט ספק זה ממה דאמרינן נדרים (דף ל״ד ע״ב) אמר רבא הי׳ לפניו ככר של הפקר ואמר ככר זו הקדש וכו׳ ולפי מה שכתב הרא״ש שם בשם הרבינו יצחק איירי שהי׳ הככר רחוק ממנו וקמ״ל שחל ההקדש על ההפקר אף שאינו ברשותו כיון שיכול לזכות ע״ש ולכן אם נימא דע״י הקדשו ראוי להקרא שלל שמים אף אחר שכבר נעשה בהמת עיה״נ א״כ ממילא הוי בידו לזכות וחל ההקדש אף שעתה הוא הפקר ואינו ברשותו.\\\vspace{0pt}

ובזה נלענ״ד ליישב מה שקשה על מה שכתב הריטב״א בסוכה (דף ל״ה) אמה דאמרינן שם דאתרוג של ערלה פסול משום דאין בו דין ממון שפירשו רש״י וראב״ד שכיון שאסור בהנאה לאו שלכם הוא וז״ל והאי פירושא ליתא וכו׳ ותו דודאי כל מידי דהוי דידי׳ וברשותי׳ דלית בי׳ זכות לאחרים לכם קרינן בי׳ עכ״ל וכבר הקשה עליו בפרי מגדים ממה דאמרינן דחמץ בפסח מדאסור בהנאה מקרי אינו ברשותו של אדם ע״ש וכן קשה עליו מגמרא דב״ק הנ״ל דאמרינן אינו מוקדש כיון דאינו ברשותו אבל לפי מה שחלקתי בין אינו שלו לשאינו ברשותו א״ש דנראה דשיטת הריטב״א היא כיון דאמרינן שם (דף ל׳) לכם משלכם להוציא את השאול ואת הגזול א״כ לא ממועט רק מה שהוא לאחרים אבל מה שאין לאחרים בו זכות ואיננו ברשות אחרים מקרי לכם להיות יוצא בו אף שאינו שלו וברשותו ומש״כ הריטב״א דהוי דידי׳ וברשותי׳ כוונתו השלילות לבד שאינו ברשות אחרים אבל רש״י וראב״ד סוברים דלכם לא מקרי רק שיהי׳ שלו וברשותו ולא די במה שאינו לאחרים ולכן איסור הנאה לא מקרי לכם (ויש נפקותא לדינא לענין הספק שכתבתי בספרי בכורי יעקב [בסי׳ תרמ״ט] מי שנטל אתרוג של הפקר ביום ראשון ע״מ שלא לזכות בו דלרש״י לא יצא ולריטב״א יצא) אבל מכ״מ גם לשיטת הריטב״א איסור הנאה לא מקרי ברשותו כיון שאינו שוה מידי ולכן בשור הנסקל אינו מוקדש וכן חמץ מקרי אינו ברשותו אבל בעיה״נ התוספ׳ מסופקים שפיר כיון דבשלל שמים אינו נוהג דין עיה״נ א״כ אפשר דאמרינן שיכול להקדישו אף דבשעת הקדש עדיין אינו ברשותו כיון דיכול לזכות בו בשביל ההקדש כמו הפקר שיכול להקדישו בשעה שעדיין אינו ברשותו כיון שיכול לזכות בו לדעת רבינו יצחק שהרי אם ההקדש חל נעשה שלל שמים ואינו נוהג בו עוד דין עיה״נ או אי נימא כיון דלשריפה קאי לא חל שם הקדש. כנלענ״ד, הקטן יעקב.\\\vspace{0pt}

\end{multicols}\newpage

\newchap{סימן קסו}
\begin{multicols}{2}
ב״ה אלטאנא, יום ד׳ כ׳ מנחם תר״ט לפ״ק. להרה״ג וכו׳ מ״ה משה שיק נ״י הגאב״ד דק״ק יערגן יע״א.\\\vspace{0pt}

כתב אלי מעכ״ת נ״י וז״ל – נ״ל פשוט דהפקר אין יכול להקדישו דלענין הקדש דרשינן מה ביתו שלו ואינו הפקר אבל דבר של הפקר אינו יכול להקדיש כדמוכח מחולין (קל״ט ע״א) דקאמר אי דראה קן ואקדשי׳ מי יכול להקדישו וכו׳ והש״ס דנקט שם מה ביתו ברשותו פשוט דאין הכוונה שאינו ברשותו דהרי משכחת לה ברשותו ואינו מזומן כדאיתא שם (קמ״א ע״ב) ביוני שובך ועלי׳ ועכ״ח משום דאינו שלו קרא לי׳ אינו ברשותו וכלשון זה איתא בב״ק (ס״ט ע״ב) מעשר ממון גבו׳ ואינו ברשותו וכו׳ והכוונה דאינו שלו וכן הוא לשון רש״י פסחים (דף ל׳ סע״ב) ד״ה כולי עלמא וכו׳ וכיוצא בזה נקט בע״ז (ס״ג ע״א) אינו ברשותו על אינו שלו והריטב״א שם כ׳ דלרבותא נקט ולכך נקט רש״י בב״ק (מ״ה ע״א) על הא דשור הנסקל דאינו יכול להקדישו משום שאינו ברשותו והיינו שאינו שלו ופשוט לכ״ע דהפקר אינו יכול להקדישו והרא״ש בנדרים (ל״ד ע״ב) דכ׳ הואיל ויכול לזכות בו יכול להקדיש היינו שיכול להקדישו לכשיבוא לידו דכל שבידו לא חשוב כדבר שלב״ל וכן מבואר בתוספ׳ שם אבל פשוט דלכ״ע אין יכול להקדיש דבר של הפקר ותמהני על כבוד מעכ״ת נ״י דכ׳ בפשיטות דיכול להקדיש הפקר וגם רצה לחלק בין אינו ברשותו לאינו שלו ולענ״ד פשוט ומוכרח מש״ס דחולין הנ״ל דשל הפקר אינו יכול להקדיש ואין חילוק בזה לאינו שלו לאינו ברשותו עכ״ד.\\\vspace{0pt}

על זה אשיב: הבאתי ראי׳ דאדם יכול להקדיש דבר שהוא הפקר ממה דאמרינן נדרים (דף ל״ד) אמר רבא היתה לפניו ככר של הפקר ופי׳ הרא״ש בשם רבינו יצחק שהככר הי׳ רחוק ממנו ואמר לכשאזכה בככר זה תהי׳ הקדש חל ההקדש כיון דבידו לזכות בו ומזה שפטתי שיכול להקדיש דבר בשעה שעדיין אינו ברשותו כיון שיכול לזכות בו ולכן לא ידעתי מה השיב מעכ״ת נ״י על זה כיון דיכול לזכות בו יכול להקדישו לכשיבוא לידו הרי לכשיבוא לידו אינו אומר כלום ובמה מקדישו ועוד הרי רבא אומר בפי׳ הי׳ לפניו ככר של הפקר ואמר ככר זו הקדש הרי שמקדישו בשעה שהוא עדיין הפקר ואעפ״כ חל ההקדש לכשיזכה בו ולא זו בלבד שחל ההקדש אחר שבא לידו אלא מיד בשעת ההקדש שעדיין אינו בידו חל דכן מוכח ממה דאמרינן נטלה לאכלה ופי׳ הרא״ש ובעי למיהדר בו מעל דאמירתו לגבו׳ כמסירתו להדיוט עכ״ל ואי ס״ד דההקדש לא חל עד שבא לידו וקנאו הרי אז לא רצה עוד להקדישו אלא לאכלו וא״כ לא חל ההקדש כלל ואיך שייך אמירתו לגבו׳ אע״כ דכל שרוצה לזכות בו בשעת ההקדש חל ההקדש מיד אפילו עדיין לא זכה בו והוא הפקר וכן נראה גם מדברי רש״י שם שכתב דלכך נקט ככר של הפקר דרבותא קמ״ל דמצי זכי בהפקר לחלה קדושה עלוי׳ עכ״ל ואי ס״ד דההקדש לא חל עד אחר שבא לידו וזכה בו מה רבותא קמל״ן פשיטא דכשזכה בהפקר הוי כשלו לכל דבר וגם להקדישו אע״כ דס״ל דחל ההקדש מיד בשעה שעדיין הוא הפקר ולא בא לידו ובזה רבותא קמ״ל דאם הקדיש בשעת הפקר ואח״כ החזיק בו חל ההקדש למפרע משעת דבורו ואילך וא״כ ה״נ אמרינן בבהמת עיה״נ שיכול להקדישה אף שבשעת הקדש עדיין אינה שלו מכ״מ חל ההקדש כיון דע״י דבורו נעשה שלל שמים ופקע דין עיר הנדחת ומה שחשב מעכ״ת נ״י להשיב על דברי ממה דאמרינן חולין (דף קל״ט) אלא דחזא קן בעלמא ואקדשי ומי קדוש וכו׳ דמזה מוכח דאין יכול להקדיש דבר של הפקר אינו ראי׳ כלל דהתם איכא תרתי דסתרי דלענין להקדיש צריך שיהי׳ שלו ולענין חיוב שלוח צריך שלא יהי׳ שלו דאם הוא שלו הוי מזומן ולכן פריך שפיר דממנ״פ לא שייך שלוח הקן במוקדשין דבעוד שאינו שלו לא הוי הקדש וכשהוא שלו לא חייב בשלוח עוד אבל לענין בהמת עיה״נ חל ההקדש שפיר שיעשה שלל שמים. כנלענ״ד, הקטן יעקב.\\\vspace{0pt}

\end{multicols}\newpage

\newchap{סימן קסז}
\begin{multicols}{2}
ב״ה: אלטאנא, יום ו׳ כ״ה תמוז תרי״א לפ״ק.\\\vspace{0pt}

כלל גדול בתורה שאין מצו׳ עומד בפני פקוח נפש חוץ מג׳ עבירות ע״ז ג״ע וש״ד אכן אם מותר להציל עצמו מפני פקוח נפש בממון חבירו דהיינו אם מותר לגזול ממון חבירו כדי להציל עצמו מסכנת נפש זה לא ראיתי מבואר בפוסקים ומצאנו בזה פלוגתא בין גדולי הראשונים והיא השאלה אשר שאל כבר דוד לסנהדרין כדאמרינן בב״ק (דף ס׳) ויתאו׳ דוד ויאמר מי ישקני מים מבור בית לחם אשר בשער מאי קמבעי׳ וכו׳ רב הונא אמר גדישים דשעורים דישראל הוו דהוו מטמרי פלשתים בהו וקא מבעיא לי׳ מהו להציל עצמו בממון חבירו שלחו לי׳ אסור להציל עצמו בממון חבירו אבל אתה מלך אתה ומלך פורץ לעשות לו דרך ואין מוחין בידו וכתבו שם התוספ׳ איבעי׳ לי׳ אי חייב לשלם כשהציל עצמו מפני פקוח נפש עכ״ל וכ״כ גם הרא״ש וז״ל הא לא מבעי׳ לי׳ אי שרי למקלינהו להצלת ישראלים דמלתא דפשיטא היא שאין לך דבר עומד בפני פקוח נפש אלא שלש עבירות אלא הכי מבעי׳ לי׳ מהו למיקלינהו אדעתא דליפטר מתשלומין ואמרו לי׳ אסור להציל עצמו בממון חבירו אדעתא דליפטר מתשלומין עכ״ל והנה ודאי פי׳ זה של התוספ׳ והרא״ש דחוק מאוד בלשון הגמרא דלא הוי לי׳ למימר מהו להציל עצמו בממון חבירו אלא הל״ל מי שהציל עצמו בממון חבירו חייב לשלם או לא רק שהוצרכו הראשונים לפרש כן כיון שאין לך דבר עומד בפני פ״נ והסנהדרין פסקו דאסור להציל עצמו אבל דעת רש״י אינה כן דאמה דאמרינן בב״ק (שם) בשלמא למ״ד למקלי היינו דכתיב ויתיצב בתוך החלקה ויצילה כתב רש״י ויצילה שלא ישרפוה הואיל ואסור להציל את עצמו בממון חבירו עכ״ל הרי דפשיטא לי׳ דאסור להציל עצמו לכתחלה דהורה שלא ישרפוה למען הציל הנפשות מיד פלשתים וראיתי בפרשת דרכים (דרוש י״ט) שהביא שם דברי רש״י אלה וכתב עליהם משמע דס״ל דמה שנסתפק דוד הוא אם מותר להציל עצמו בממון חבירו והדבר הוא תימה בעיני דאיך יתכן דבמקום פקוח נפש אסור להציל עצמו בממון חבירו עכ״ל ולענ״ד י״ל ליישוב שיטת רש״י שלא בלבד דלשון הגמרא מסייעו מה שדחוק מאוד לפי׳ התוספ׳ והרא״ש כמו שכתבתי אלא דגם מסברא יש לומר כן שהרי ביומא (דף פ״ה) יליף ר״ע דפקוח נפש דוחה שבת דעבודה דוחה שבת ופקוח נפש דוחה עבודה כש״כ שדוחה שבת ע״ש והרי זה פשיטא אף שהתירה התורה לחלל שבת מפני עבודת תמידין ומוספין מכ״מ לא התירה לגזול דאם לא מצא תמיד אלא מן הגזל לא התירה התורה להביאו דהא סתמא אמרינן להוציא את הגזול ולא מחלקינן בין מצא אחר ללא מצא אחר הרי אף שהתירה התורה שבת מפני עבודה מכ״מ לא התירה ממון של אחרים וכיון דפקוח נפש ילפינן מעבודה נאמר ג״כ כמו שעבודה לא דוחה את הגזל גם פקוח נפש לא דוחה את הגזל ואין לומר דא״כ דרק המצות התירה התורה מפני פקוח נפש ולא מה ששייך לחבירו ל״ל לרבא לפרש שם (דף פ״ב) הטעם דאסור להציל עצמו בנפש חבירו משום מאי חזית דדמך סומק טפי וכו׳ תיפוק לי׳ אפילו לא הוי נפש חבירו רק ממונו ג״כ לא מותר להציל עצמו בו דיש לומר דרבא קושטא קאמר דהא הוי נפש חבירו ועוד דיש נפקותא באם מוחל חבירו ורוצה שיציל עצמו בו דבממון כה״ג ודאי מותר אבל להציל עצמו בנפש חבירו אפילו חבירו רוצה בכך אסור מטעם דמאי חזית כיון דחביב לפני הקב״ה נפש ישראל מאי חזית דנפשך חביבה לו מנפשו כמו שפי׳ רש״י אבל אם לא מוחל חבירו על ממונו י״ל דגם בממונו אינו רשאי להציל זה לענ״ד שיטת רש״י.\\\vspace{0pt}

וראיתי סמך לזה שאפילו לענין פקוח נפש יש סברא לחלק בין איסור ובין ממון ממה דכתיב (שמואל א׳ כ״א) שאמר דוד לאחימלך ועתה מה יש תחת ידך חמשה לחם תנה בידי וגו׳ ויען הכהן את דוד ויאמר אין לחם חול אל תחת ידי כי אם לחם קדש וגו׳ ויען דוד את הכהן וגו׳ והוא דרך חול ואף כי היום יקדש בכלי ויתן לו הכהן קדש כי לא הי׳ שם לחם כי אם לחם הפנים המוסרים לפני ד׳ לשום לחם חום ביום הלקחו ואמרינן במנחות (דף צ״ה) מאי דרך חול דקאמר לי׳ הכי קאמר לי׳ ליכא לחם כי אם לחם חום המוסרים מלפני ד׳ א״ל לא מבעיא הא כיון דנפיק ממעילה דרך חול אלא אפילו היאך נמי דהיום יקדש בכלי הבו לי׳ דליכול דמסוכן הוא דאחזו בולמוס ופי׳ רש״י דנפיק ממעילה דכיון דסלקו מותר לכהנים וכל שיש לו שעת היתר לכהנים אין בו מעילה ולא מבעי׳ הך שכבר נסתלק אלא אף אם הם מסודרים על השולחן הבו לי׳ דמסוכן הוא עכ״ל וכתב הפרשת דרכים (שם) ויש לתמו׳ טובא דמאי קאמר ל״מ הך כיון דנפיק ממעילה הותר לזרים והלא קדשי קדשים הם וכ״ת משום דהי׳ מסוכן א״כ מה חדש לו דוד אפי׳ היאך נמי דהיום יקדש בכלי פשיטא כיון דהותר לו לאכול קדשי קדשים אף שהי׳ זר משום דהי׳ מסוכן פשיטא דהותרו לו כל האיסורים עכ״ל אכן ע״פ הנ״ל י״ל דזה ודאי אם אסור להציל את עצמו בממון חבירו כל שכן שאסור להציל עצמו בממון גבו׳ דלא חמיר הדיוט מגבו׳ ולכן הי׳ אפשר לחשוב דאף דהותר לזר לאכול ק״ק מפני הסכנה זה דוקא לאחר שיצאו מידי מעילה ונעשו ממון כהן דאז הכהן יכול להאכיל למסוכן שאין כאן גזל לא מבעי׳ לר׳ יהודה דס״ל בקידושין (דף נ״ב) דמקדש אשה בקדשי קדשים מקודשת דכתיב לך לכל צרכך אלא אפילו לר׳ יוסי דאינה מקודשת הרי הטעם דכהנים משלחן גבו׳ קא זכו רק לאכילה ולא לצורך אחר ע״ש וא״כ כיון דהכהן מאכיל מה שזכה בו לאכילה למי שמותר לאכול מפני הסכנה פשיטא שאין כאן איסור אבל קדשי קדשים שיש בהם מעילה שעדיין ממון גבו׳ הם יש לסבור שאסור להאכיל מפני הסכנה שאסור להציל עצמו בממון גבו׳ לכן שפיר חדש דוד אפילו אין כאן מה שכבר נעשה חול אלא מה שקדש היום בכלי שעדיין לא יצא מידי מעילה ג״כ מותר לאכול שאז סבר דוד שמותר להציל עצמו בממון חבירו וגם בממון גבו׳ אבל אחר שאינה ד׳ לידו לחם שיצא כבר מידי מעילה אפשר דעי״ז בעצמו נסתפק דוד במה שהי׳ פשיטא לו שמותר להציל עצמו במ״ח שאולי שמרו הקב״ה מלחטוא ולכן כששוב בא מעשה כזה לידו שאל לסנהדרין אם מותר להציל עצמו בממון חבירו (שמעשה זה הי׳ אחר מעשה דאחימלך כנראה מהכתובים וממה שהשיבו לו אתה מלך שאז כבר היה מלך) והשיבו לו שאסור וא״כ באמת י״ל שאסור להציל עצמו ג״כ בממון גבו׳.\\\vspace{0pt}

ובזה נלענ״ד לבאר דברי רש״י באבות (פ׳ ה׳) על מה דאמרינן שם לא הפילה אשה מריח בשר הקדש פי׳ רש״י מתאות בשר הקדש אי נמי מריח אברים של המערכה שאלו הריחה ובאת לטעום מהם אין שומעין לה להאכילה בשר קדש עכ״ל וכבר הקשה עליו בתוספ׳ חדשים שהרי אין לך דבר שעומד בפני פקוח נפש ועוד דגמרא ערוכה היא ביומא (דף פ״ב) דעוברה שהריחה מאכילין אותה בשר קדש שאין לך עומד בפני פ״נ והניח בקושיא ובשם הגאון הרב מ״ה שאול זצ״ל אב״ד דק״ק אמשטרדם יע״א כתוב שם בגליון המשניות יישוב לשיטת רש״י דממה דנקט לא הפילה אשה ולא קתני לא הפילה עוברה משמע דאיירי שלא הוכר עוברה ואז אין מאכילין לה עד שתשתנה פני׳ ובזה הי׳ אפשר שתבא לידי סכנה ואף שסיים שם כן נלענ״ד ליישב פי׳ רש״י על נכון ע״ש לענ״ד כמה מן הדוחק לומר שרש״י נתכוון לכך במה שאין רמיזה כלל בדבריו ועוד מה זה שכפל רש״י דבריו אחר שכתב סתם מתאות בשר קדש כלישנא דמתניתן שוב פרט אי נמי מריח אברים של המערכה הרי זה בכלל בשר קדש ולא עוד אלא שמה שמסיים רש״י בזה שאלו הריחה ובאת לטעום מהם אין שומעין לה משמע שזה לא קאי רק על אברים של מערכה דאם על בשר דרישא קאי הל״ל דאלו התאוה ובאת לטעום ממנו אבל לענ״ד הפי׳ בזה דרש״י לשיטתו קאי דלפמש״כ יש חילוק בזה בין ב׳ מיני קדשים אותם שיצאו מכדי מעילה שהם ממון כהן ודאי מאכילין למסוכן ומהם איירי ביומא במה דאמרינן שם שמאכילין למעוברת דקאמר שם דבתחלה מאכילין לה מרוטב ואם לא הספיק מאכילין שומן הרי דאיירי מקדשים שנתבשלו שהם חטאת ואשם של כהן ומהם איירי ג״כ פירוש ראשון של רש״י במה דכתב מתאות בשר קדש דתאו׳ וחימוד הם בלב כמש״כ ג״כ התוספת חדשים ונקט כן דלא הי׳ צריך לנס שלא הפילה משום קדשים שנאכלים לכהנים דהם מאכילים לה אבל לפעמים הי׳ אפשר דהתאוה בלב בלבד ולא פרסמה תאותה שיודע להאכילה ולזה הי׳ הנס שלא הפילה ע״י תאות בשר קדש אבל באי נמי פי׳ רש״י דמריח אברי מערכה איירי שהם ממון גבו׳ דמהם אין מאכילין לה אפילו הריחה ונסתכנה דאין מצילין בממון גבו׳ כמו בממון חבירו.\\\vspace{0pt}

ויצא לנו מזה דלרש״י אין מצילין מן הסכנה לא בממון חבירו ולא בממון גבו׳ אבל לתוספ׳ ורא״ש מצילין ומדברי שאר ראשונים לא ראיתי בזה הכרע שהם כתבו המציל עצמו בממון חבירו חייב וכ״כ גם הרמב״ם ה׳ חובל ומזיק (פ״ח) ואף שלא כתבו שאסור להציל מכ״מ אין הכרע ששיטתם כשיטת התוספ׳ שהרי לא כתבו ג״כ איפכא שאף שחייב מכ״מ מותר לכתחלה להציל עצמו אכן מלשון הראב״ד (שם) משמע ששיטתו כשיטת התוספ׳ ומכ״מ גם ממנו אין ראי׳ גמורה דהוא מחלק בין למסור עצמו כדי להציל ובין מציל בשנוטל ממון חבירו מעצמו ולכן כתב דודאי אינו מתחייב להניח להרוג את עצמו משום הצלת ממון חבירו אבל אם מותר לכתחלה ליטול ממון חבירו מעצמו להציל עצמו ממיתה עדיין אין הכרע מדבריו ועיין בנ״י בהגוזל בתרא. כנלענ״ד, הקטן יעקב.\\\vspace{0pt}

\end{multicols}\newpage

\newchap{סימן קסח}
\begin{multicols}{2}
ב״ה אלטאנא, יום ה׳ כ״ג מנחם תרי״א לפ״ק. להרה״ג וכו׳ מ״ה יוסף חיים קרא נ״י אב״ד דק״ק פיננע יע״א.\\\vspace{0pt}

כתב אלי מר נ״י – תמה אני על מה שכתב מעכ״ת נ״י דלדעת רש״י יהא אסור הגזל אפילו משום פיקוח נפש והלא ש״ס ערוך הוא בכתובות דף י״ט ע״א א״ל רבא השתא אילו אתי קמן לאמלוכי אמרינן להו זילו חתומו ולא תתקטלון דאמר מר אין לך דבר יע״ש הרי דאי אפשר לומר כן בדעת רש״י ומ״ש מעכ״ת נ״י דגם מסברא יש לומר כן כיון דפיקוח נפש ילפינן מעבודה כמו שעבודה אינו דוחה את הגזל כך פיק״נ אינו דוחה אותה אינו נראה לפענ״ד חדא דהא דעבודה אינו דוחה את הגזל היינו טעמא משום דאין מצותה בכך דקב״ה אמר אני ד׳ שונא גזל בעולה וכאלו לא הביא כלל דמי וכמו שלא נאמר דאם אין כאן זכר לעולת הבקר יביא נקבה לתמיד וידחה מצוה דזכר תמים יקריבנו כיון דאין מצותה בכך ועוד דאף אם נאמר דעבודה אינו דוחה גזל מאין הרגלים לומר דפיקוח נפש דחמיר מיני׳ דדוחה עבודה שאינו דוחה גזל ג״כ וא״ל מדשאלו התנאים אלו מניין לפיק״נ שדוחה שבת מכלל דצריך לימוד דוקא א״כ לגבי גזל דליכא ק״ו עכ״פ אין ראי׳ דידחה ז״א דעד כאן לא שאלו אלו התנאים רק מנ״ל דידחה שבת דכיון דאיסור שבת במיתה א״כ חזינן דקטלין גברא מפני מצוה דשבת א״כ מהיכא תיתי נאמר דמחללין שבת להצלה דגברא (ובאמת שאלתי כן בילדותי לאאמ״ו ז״ל) ע״ז הוכיח הש״ס אף דחילול שבת במיתה אעפ״כ לתקוני עבודה לא מיקרי חילול כלל מכ״ש לפיק״נ דחמיר דלא מקרי חילול וברשות קא עביד ואי בעית אימא כיון דאמר מר דשקולה שבת כע״ז הו״א דאינו דוחה אותו כע״ז לכך בעי ק״ו אבל גזל אע״ג דעבודה לא דחי לי׳ מ״מ הדרינן לכללא דאין לך כו׳ ואם שדברי כבוד מעכ״ת נ״י שם מלאים חן ושכל טוב ליישב בסברתו קושיות הפ״ד אין המונח אמת כדמוכח מש״ס דכתובות הנ״ל עכ״ד.\\\vspace{0pt}

תשובה – הנה מלבד מה שכתב מעכ״ת נ״י על חקירתי הנ״ל כתב לי ג״כ הרה״ג וכו׳ מ״ה גבריאל אדלער הכהן נ״י הגאב״ד דק״ק אבערדארף וז״ל אתמה במה שמסתפק אם מותר להציל עצמו מפני פ״נ בממון חבירו דהיינו אם מותר לגזול מ״ח כדי להציל עצמו מסכנת נפש כתב לא ראיתי מבואר בפוסקים וכו׳ הלא מפורש דין זה בטוש״ע ח״מ (סי׳ שנ״ט סעיף ד׳) אפילו הוא בסכנת נפש וצריך לגזול את חבירו כדי להציל נפשו צריך שלא יקחנו אלא על דעת לשלם אלמא דמותר וצריך לגזול וכמש״כ הרא״ש דאין לך דבר עומד בפני פ״נ ובודאי גם רש״י מודה להך דינא אך איננו פטור מתשלומין כשאפשר לו לשלם אח״כ אבל ספיקא דדינא אין בזה עכ״ל ועל זה השבתי מה שתימה מר נ״י שכתבתי שהדין לא מבואר בפוסקים והרי מפורש בש״ע לא כתבתי שהדין לא הוזכר בפוסקים אלא שלא מבואר בפוסקים שיש פלוגתא בו בין גדולי הראשונים והרי הטור לא כתב רק דעת אביו הרא״ש שהזכרתי ואחריו כ׳ הש״ע כן אבל לא הזכירו שדעת רש״י אינה כן ומה שכתב מעכ״ת נ״י שגם דעת רש״י כהרא״ש שמותר לא ידעתי היאך אפשר לומר כן שהרי רש״י כתב בפי׳ ויצילה שלא ישרפוה שאסור להציל עצמו בממון חבירו וכמו שדייק ג״כ הפרשת דרכים. ועתה אשיב על דברי מר נ״י שהשו׳ דעתו עם הרב הנ״ל נ״י להחליט מכח סברא שלא יעלה על הדעת לומר שפקוח נפש לא ידחה את הגזל ואחרי העמקתי בעיון הדבר נלענ״ד שלא בלבד שדעת רש״י כן שפליג על הרא״ש אלא שגם כבר פלוגתא דאמוראים ואפילו פלוגתא דתנאים מצאנו בזה. מר נ״י הביא ראי׳ דמותר להציל עצמו בממון חבירו לכתחלה ממה דאמרינן בכתובות (דף י״ט) אמרינן להו זילו חתומו ולא תתקטלון ומתוך כך רצה לדחוק בגמרא דב״ק במעשה דפלשתים שלא הי׳ פקוח נפש להם אלא ספק שמא באחר יבא לידי פ״נ מה שודאי לשון הגמרא מהו להציל עצמו בממון חבירו מתנגד לפי׳ כזה ולא ידעתי לו יהי כדברי מר נ״י שיהי׳ משם ראי׳ הרי רק רבא הוא דאמר כך לרב חסדא אבל רב חסדא הרי קאמר בפי׳ עדים שאמר להם חתמו שקר ואל תהרגו יהרגו ואל יחתמו שקר א״כ לרב חסדא אסור להציל עצמו בממון חבירו וכן כתב בספר הכתובה דס״ל לרב חסדא בהא אם ירצה הגזלן להרוג אותו והוא בא לפטור עצמו בחתימת שקר על חבירו דאין יכול להציל עצמו בממון חבירו וס״ל דאסור לגזול או להפסיד ממון חבירו לרפאות את עצמו עכ״ל ולפ״ז יש פלוגתא דאמוראי בדין זה בין רב חסדא לרבא הן אמת דהרמב״ן והרשב״א פרשו דרב חסדא רק מדרך חסידות קאמר דיהרגו ולא מן הדין ולכן תימה הריטב״א על רש״י שכתב דאם לא יהרגו משוו נפשייהו רשעים דנראה דדעתו דלר״מ דינא הוא שיהרגו והרי אין דבר עומד בפני פ״נ חוץ מג׳ דברים ע״ש אכן ברור שגם בזה רש״י לשיטתו קאי דס״ל דלרב חסדא הדין כן וכיון דגם בב״ק בהך מעשה דפלשתים יש פלוגתא דאמוראי מה הי׳ השאלה ורק רב הונא הוא דמפרש ששאלו מהו להציל עצמו בממון חבירו והשיבו לו שאסור יש לומר דרש״י ס״ל דרב הונא ס״ל כרב חסדא דמן הדין אסור להציל עצמו בממון חבירו ורבא דפליג על זה באמת מוקי הך דפלשתים בענין אחר כדאמרינן שם אמר רבא אר״נ טמון באש קמבעי׳ לי׳ וכו׳ ע״ש ולכן שפיר כתב רש״י ויצילה שלא ישרפוה. שוב ראיתי שפלוגתא דתנאי יש בדין זה דז״ל הרמב״ן הובא בש״מ בכתובות שם ואיכא דאמרי רב חסדא אמר קסבר ר״מ מדינא נמי יהרגו ואל יחתמו שקר לפי שנמצא בברייתא חיצונית שלשה דברים אין עומדים בפני פ״נ ואילו הן ע״ז וג״ע וש״ד ר״מ אומר אף הגזל ורבא אמר אנן ודאי קיי״ל שאין לך דבר עומד בפני פ״נ אלא שלשה אילו בלבד ודלמא אתא קמי׳ בי דינא דס״ל כרבנן ואורו לי׳ הכי ולאו מלתא היא עכ״ל הרי דפליגי בזה ר״מ ורבנן ולר״מ גזל עומד בפני פ״נ ומה שדחה הרמב״ן פי׳ זה נראה דס״ל דאף דר״מ ס״ל גזל דוחה פ״נ מכ״מ בהך דעדים לאו גזל הוא שהם אינם גוזלים בעדותן רק עדות שקר הוא וזה אינו עומד בפני פ״נ והאיכא דאמרי ס״ל כיון דלר״מ דנין דינא דגרמי ובזה פסקינן כוותי׳ כדאמרינן בב״ק (דף ק׳) ועדות שקר דמי לגרמא דדן את הדין וא״כ יש בזה הפסד ממון ודומה לגזל אבל רבא סובר דאין זה דומה לגזל דגזל דאורייתא הוא דוקא כשלוקח מידו ובזה בלבד ס״ל לר״מ דעומד בפני פ״נ עכ״פ הנך רואה דר״מ ס״ל דאסור להציל עצמו בממון חבירו אפילו משלם אח״כ ולכן שיטת רש״י א״ש דמפרש דלאוקימתא דרב הונא הסנהדרין הורו לדור כר״מ שלא ישרפוה ולא צריך לדחוק כמו שפי׳ התוס׳ והראש דהשאלה הי׳ אם חייב לשלם בלבד ובזה א״ש ג״כ מה שפירשתי דברי רש״י באבות שאסור להציל עצמו בהקדש דאברי המערכה דבזה ג״כ פי׳ רש״י בחד פירוש אליבי׳ דהך שיטה שהורו סנהדרין לדוד ואף דלדינא ודאי פסקינן כפסק הטור וש״ע דמותר להציל עצמו בממון חבירו מכ״מ הרוחנו במה שחקרנו לידע שלאו כ״ע ס״ל כן ועוד אכתוב אי״ה במה שיש נפקותא לדינא בזה. כנלענ״ד, הקטן יעקב.\\\vspace{0pt}

\end{multicols}\newpage

\newchap{סימן קסט}
\begin{multicols}{2}
ב״ה אלטאנא, יום ד׳ כ״ב סיון תרי״ז לפ״ק. להרה״ג וכו׳ מ״ה מתתי׳ מונק הכהן נ״י אב״ד דק״ק קריאנקע יע״א.\\\vspace{0pt}

העיר מעכ״ת נ״י על חקירתי אם מותר להציל עצמו בממון חבירו שממעשה דר׳ יהודה ור׳ יוסי דהובא ביומא (ד׳ פ״ג) מוכח שיכול להציל עצמו בממון חבירו דקפחי׳ ר׳ יהודה שאחזו בולמוס לרועה ואכלי׳ לריפתא ולא נזכר שם שהחזיר ר״י דמי ריפתא לרועה כלל – לא ידעתי אם פשיטא למעכ״ת נ״י שלא שילם ר׳ יהודה איך הקשה עלי כן הרי יקשה לפסק הלכה שנפסק בח״מ (סי׳ שנ״ט) שאסור להציל עצמו בממון חבירו אם לא ע״מ לשלם ולכן ודאי צ״ל שר׳ יהודה לא קפחו רק ע״מ לשלם ושילם לו וא״כ לשיטת התוספ׳ והרא״ש ודאי אין קושיא אבל יצדק הוכחת מר נ״י לכאורה נגד שיטת רש״י שהוכחתי דס״ל דאסור להציל עצמו בממון חבירו אפילו ישלם לבסוף אכן אחרי עיון לענ״ד אדרבא יש ראי׳ מזה לשיטה זו וכפי אשר ביארתי׳ דז״ל הגמרא (שם) רבי יהודה ור׳ יוסי הוו קא אזלי באורחא אחזי׳ בולמוס לר׳ יהודה קפחי׳ לרועה אכלי׳ לריפתא א״ל רבי יוסי קפחת את הרועה כי מטו למתא אחזי׳ בולמוס לר׳ יוסי אהדרוה בלגי וצעי א״ל ר׳ יהודה אני קפחתי את הרועה ואתה קפחת את העיר כולה עכ״ל וק׳ מה תרעומת הי׳ לר׳ יוסי על ר׳ יהודה שאמר לו קפחת את הרועה הרי במה שאחזו בולמוס אונס הי׳ ובמה שקפח לרועה מצו׳ קעביד להציל את נפשו וכן ק׳ מה השיב לו ר׳ יהודה ואתה קפחת את העיר כולה היאך יקרא זה קפוח הרי ברצון טוב הביאו ונתנו לו אבל יובן זה ע״פ מה שהבאתי במקום אחר (סי׳ הקדום) בשם הרמב״ן ברייתא שיש פלוגתא בזה דלת״ק רק ג׳ דברים עומדין בפני פקוח נפש ור״מ אומר אף הגזל ולפ״ז י״ל דגם רבי יהודה ור׳ יוסי פליגי בזה דר׳ יהודה ס״ל כת״ק דר״מ ובפרט שכפי הנראה מסוגיא דפסחים (דף י״א) ומסוגיא דסנהדרין (דף מ״ה) פשיטא להגמרא דסתם בר פלוגתא דר״מ הוא ר׳ יהודה וכן אפכא ע״ש וא״כ ר׳ יהודה לשיטתו אזיל דס״ל דגזל אינו עומד בפני פ״נ רק הג׳ דברים ע״ז וג״ע וש״ד ולכן הי׳ ס״ל דמותר לקפח את הרועה אבל רבי יוסי ס״ל כר״מ דאף הגזל עומד בפני פ״נ דאסור להציל עצמו בממון חבירו ולכן אמר לו קפחת את הרועה ונקרא זה קפוח וגזילה דאסור הי׳ לך לעשות כן ולכן כשאחזו גם לר׳ יוסי בולמוס והביאו לו בני העיר מעצמם אמר ר׳ יהודה אם אני קפחתי את הרועה גם אתה קפחת את בני העיר כלומר כמו שאצלך לא הי׳ קפוח שמרצונם נתנו כן גם אצלי לא יקרא קפוח אף שכפיתי אותו שמן הדין הי׳ מותר לי לעשות כן ולפ״ז א״ש דפסק רש״י דאסור להציל עצמו בממון חבירו שהרי בעירובין (דף מ״ו) אמרינן דר׳ יהודה ור׳ יוסי הלכה כר׳ יוסי וכש״כ כשגם ר״מ מסייע לו ולכן מוקי גם סתם גמרא בב״ק (דף ע׳) כן. כנלענ״ד, הקטן יעקב.\\\vspace{0pt}

\end{multicols}\newpage

\newchap{סימן קע}
\begin{multicols}{2}
ב״ה אלטאנא, יום ג׳ ט״ו כסליו תרי״ב לפ״ק.\\\vspace{0pt}

נשאלתי – חולה אחד נחלה בחולי נפלא ועסקו הרופאים ברפואתו ללא הועיל כי מת בחליו ויש שם עוד חולה שנחלה בחולי כזה ורצו הרופאים לפתוח את המת לראות ענין החולי למען מצוא תרופה לאשר עוד בחיים אם מותר לעשות כן לנוול המת או לא?\\\vspace{0pt}

תשובה – שאלה קרובה לנדון זה כתובה בשו״ת נודע ביהודה מ״ב חי״ד סי׳ ר״י רק דשם הי׳ הענין שלא הי׳ שם חולה כזה אלא שרצו הרופאים לנוול את המת ולפתחו למען ידעו למצוא תרופה אם יבא חולה כזה לידם והשואל שם רצה להתיר דאין לך דבר עומד בפני פקוח נפש ומה דאמרינן בב״ב (דף קנ״ד) במעשה דבני ברק שאמר ר״ע אי אתם רשאים לנוולו וכן נפסק בח״מ (סי׳ רל״ה ס׳ י״ג) שאני התם דבשביל ממון הוא אבל בשביל פ״נ ודאי מותר שהרי אפילו בשביל ממון אמרינן שם ואלא אי אמרת לקוחות לינוול ולינוול הרי מבואר דמשום פסידא דלקוחות לא משגחינן בניוול המת והגאון בעל נ״ב הסכים עמו בראי׳ זו רק שכתב דעכ״פ אם יש שם קרובים של המת אינם רשאים לנוולו ועוד הביא השואל ראי׳ להתיר ממה דאמרינן בחולין דף י״א וכ״ת משום אבוד נשמה דהאי ננוול הרי דפשיטא להגמרא דבשביל להציל הרוצח מותר לנוול הנרצח ועל ראי׳ זו השיב הגאון בעל נ״ב דעכ״פ שם התירה התורה לנוול דאם לא אזלינן בתר רובא כס״ד שם ואעפ״כ צותה התורה להרוג הרוצח ע״כ אי אפשר זה אלא בענין שיבדקו הנרצח דלאו טרפה הי׳ ועוד השיב דשאני התם כיון דהניוול הוא למען יעשה דין ברוצח א״כ זהו כבודו של נרצח וכל שהוא לכבודו לא שייך ניוול ושלכן אין ראי׳ מזה אבל העיקר כתב דאין מקרי ספק פקוח נפש כיון דאין שם פקוח נפש לפנינו ולשמא יבא לפנינו לא חיישינן ולא מקרי זה ספק פקוח נפש דא״כ התרת כל מלאכות שבת שמא יצטרך לחולה אע״כ דלהא לא חיישינן עכ״ד והנה ודאי דבריו נכונים בזה דלא מקרי פ״נ לשאם יבא כזה לידינו אבל לפ״ז בנדון השאלה הנ״ל שיש שם חולה כזה ויש פ״נ לפנינו יהי׳ מותר ע״פ סברת הגאון נ״ב לפתוח ולנתחו כשאין שם יורשים המוחים בדבר.\\\vspace{0pt}

אכן לפענ״ד אין הדבר כן ובתחלה אשיב על ראיות המתיר דמה שהוכיח ממה דאמרינן אי אמרת לקוחות לינוול ולינוול דמכיון דמשום פסידא דלקוחות לא משגחינן ה״ה לא מפני פ״נ תמהתי היאך הסכים עמו הגאון נ״ב שלענ״ד אדרבא משם ראי׳ להיפך שהרי אמרינן שם אמאי שתקי לימרו לי׳ אנן זוזי יהבינן לי׳ לינוול ולינוול ע״ש הרי בפי׳ דרק משום דבאו מכח טענה דהוא נשאר חייב להם שקבל זוזי מהם לינול ומטעם זה כתבו התוספ׳ שם דמשום ירושה דלא מידי יהבי אין רשאים לינוולו וכן פסק גם הרמב״ם ה׳ מכירה (פ׳ כ״ט) ונפסק כן בח״מ (סי׳ רכ״ה) ע״ש ברמ״א וא״כ המת דלא נתחייב מידי למה יתנוול הרי לא דמי ללקוחות רק לירושה וגם הראי׳ שהביא המתיר מחולין דאמרינן וכ״ת משום אבוד נשמה דהאי נינוולי׳ מלבד מה שהשיב כבר בנ״ב על זה בלא״ה לענ״ד אין ראי׳ משם דהנה בכל עניני התורה אם יארע התנגדות בין ב׳ מצות אמרינן שב ואל תעשה עדיף ולכן צריך ילפותא דעשה דוחה ל״ת דמסברא הוי אמרינן מוטב ידחה העשה בשב ולא תעשה ואל ידחה הל״ת בקום ועשה ולענין ע׳ ול״ת ולקצת שיטות גם לענין ל״ת שיש בה כרת דלית לן ילפותא דלידחי מפני ע׳ באמת אמרינן דידחה העשה מפניהם בשוא״ת ועל כן גם כאן אם ימצא התנגדות בין מצות פקוח נפש ובין מצו׳ שלא לנוול המת מסברא אמרינן שוא״ת עדיף ולא ינוול אבל התם דצוותה התורה להרוג הרוצח רק דהספק הוא אם בלא בדיקת הנרצח או אחר שיבדק קאמר שפיר דמשום אבוד נשמה דהאי נינוולי׳ דהרי הבדיקה היא לא לבד להציל הנרצח אלא להציל גם הב״ד והעדים שלא יעברו על עון רציחה שלא כדין בקום ועשה שאם הנרצח טרפה הי׳ מיתת הרוצח נחשבה כרציחה ושפיר אמרינן שיש סברא דמוטב לעבור בקום ועשה על עוון ניוול המת מלעבור בקום ועשה על ספק רציחה ולכן נקט שם הלשון משום איבוד נשמה דהאי ולא נקט הלשון משום פקוח נפש דלא משום פקוח נפש שיעבור עליו בשוא״ת אתינן עלי׳ אלא משום אבוד נשמה בקום ועשה וא״כ בנדון השאלה דאין על הצד של פקוח נפש רק שוא״ת מסברא נאמר תדחה משום פקוח נפש בשב ואל תעשה ואל ידחה איסור ניוול המת בקום ועשה.\\\vspace{0pt}

רק דעדיין יש לדון בזה שהרי כלל גדול בתורה דאין לך דבר עומד בפני פקוח נפש ואין חילוק בין ודאי לספק פ״נ אבל נלענ״ד דגם מטעם זה אין להתיר כאן דכבר הוכחתי במ״א (סי׳ קס״ז) שדעת רש״י ע״פ גמרא דב״ק (דף ס׳) דאמרינן שם שאסור להציל עצמו בממון חבירו שאסור לאדם לגזול ממון חבירו למען הציל עצמו ממיתה ונגד דעת התוספ׳ והרא״ש שפירשו הסוגיא שם דוקא לענין דצריך לשלם אבל לא שיהי׳ אסור לכתחלה להציל והנה לדעת רש״י כיון שאסור להציל עצמו בממון חבירו כש״כ דאסור להציל עצמו בקלון חבירו דכבודו חביב לו מממונו כדאמרינן בב״ק פ׳ החובל בהאשה שבאת לפני ר״ע ע״ש וא״כ האיך נאמר דמשום פ״נ דהחולה יהי׳ מותר לבזות ולנוול המת דמסתמא לא מוחל על בזיונו דעד כאן לא פליגי תנאי בשקלים (פ׳ ב׳) רק אי מחיל אינש זילותי׳ לגבי יורשים ומשמע דלגבי אחר מסתמא אינו מוחל אכן לא בלבד לשיטת רש״י נראה דאסור אלא גם לשיטת התוספ׳ והרא״ש דפסקוה הטור וש״ע בח״מ (סי׳ שנ״ט) נראה שהדין כן דז״ל הטור שם ואפילו הוא בסכנת מות ובא לגזול את חבירו ולהציל נפשו אסור לו לגזול אם לא על דעת לשלם ודאי אין לך דבר שעומד בפני פ״נ לכך הוא רשאי לנולו ולהציל נפשו אבל לא יקחנו אלא על דעת לשלם עכ״ל וכן נפסק גם בש״ע שם הרי בפירוש דאסור לגזול אפילו להציל נפש אם לא על מנת לשלם דאין חילוק בין שיטת רש״י לשיטת התוספ׳ והרא״ש אלא דלשיטת רש״י מה דאמרינן אסור להציל עצמו בממון חבירו הוא אפילו ע״מ לשלם לבסוף ולשיטת התוספ׳ והרא״ש לא אסור רק אם מציל עצמו ע״מ שלא ישלם לבסוף ולפ״ז בנדון זה דלא שייך שישלם לבסוף דאי אפשר לשלם למת את בזיונו גם לשיטת התוספ׳ והרא״ש אסור להציל נפש בבזיון אחר ואין לומר כיון דפסקינן כת״ק בשקלים (שם) דגבו לצורך המת והותיר יתנו ליורשיו דאדם מחיל זילותי׳ לגבי׳ יורשי׳ כמבואר בטוש״ע י״ד (סי׳ שנ״ו) א״כ גם הכא שייך שישלם לבסוף דישלמו דמי בזיון המת ליורשיו דז״א דבשלמא התם שכבר גבו יש סכום בזה לומר דבזיון זה מחל ליורשיו אבל הכא מי ישום כמה דמי הבזיון לשלם ליורשיו ועוד דרש״י כתב בסנהדרין (דף מ״ח) וז״ל תנא קמא סבר מחיים אחלי׳ לזילותא וניחא לי׳ שיתבזה לאחר מיתתו להנאת יורשים עכ״ל הרי שכתב דמחיים אחלי׳ לזילותי׳ והיינו דודאי לא שייך מחילה לאחר מיתה ואף דלא נגבה הממון רק לאחר מיתה אמרינן מסתמא ידע המת שאין לו צרכי קבורה ויצטרכו לגבות בשבילו לכן מחיים כבר מחל זילותא דהמותר ליורשי׳ וא״כ לא שייך מחילה רק לענין בזיון דאסיק המת אדעתי׳ בחייו מה שאין כן בנדון שלפנינו וא״כ לא שייך בזה השבה וגם לשיטת התוספת והרא״ש אסור להציל בבזיון חבירו ועוד דבשלמא לענין הצלת עצמו בממון חבירו יש סברא דמותר כיון דהנגזל בעצמו חייב להציל חבירו בממונו משום לא תעמוד על דם רעך כמבואר בח״מ (סי׳ תכ״ו) לכן מותר להציל עצמו בממונו דלא מידי גזלו כיון דהוא בעצמו הי׳ חייב בזה אבל בנדון השאלה דלא שייך לומר כן דאין על המת מוטל להציל את החי דכיון שאדם מת נעשה חפשי מן המצות כדאמרינן נדה (דף ס״א) לכן גם מטעם זה נלענ״ד דגם ע״פ שיטת התוספ׳ והרא״ש אסור להציל החי ע״י ביזוי המת וכל שכן כיון דהביזוי והניוול הוא ודאי ואם יגיע הצלה לחי עי״ז הוא ספק ושב ואל תעשה עדיף. כנלענ״ד, הקטן יעקב.\\\vspace{0pt}

\end{multicols}\newpage

\newchap{סימן קעא}
\begin{multicols}{2}
ב״ה אלטאנא, יום ה׳ כ״ז אדר תרי״ב לפ״ק. להרה״ג וכו׳ מ״ה משה שיק נ״י הגאב״ד דק״ק יערגן יע״א.\\\vspace{0pt}

מעכ״ת נ״י השיב על מה שכתבתי לאסור לנוול המת לצורך פקוח נפש והביא ראיות שעל פיהם לפי דעתו מבואר ומוכרח דבמקום פקוח נפש שרי ניוול המת ולענ״ד לא מבואר ולא מוכרח ואשיב על ראיותיו: מה שהביא ראי׳ ממה דאמרינן ערכין (דף ז׳) בישבה על המשבר ומחה מביאין סכין וקורעין בטנה כדי להציל הולד הרי דמשום הצלת עובר מותר לנוול המת לפענ״ד ג׳ תשובות בדבר.\\\vspace{0pt}

(א׳) ניוול המת לא יקרא רק מה שמשקץ ומתעב המת בעיני רואיו ובפרט מה שנקרא ניוול גם גבי החי אבל להוציא העובר ע״י חתיכת בטנה מאן לימא לן דזה ניוול יקרא אחרי שידוע שע״פ דרכי הרפואה גם בחי׳ לפעמים נעשה כן והרבה פעמים מצאנו בדברי רז״ל יוצא דופן הרי שהי׳ ידוע דרך רפואה זו אצלם ואין זה ניוול משא״כ לפתוח בטן המת ולנתח בני מיעיו כאשר צריך לעשות מי שרוצה ללמוד ענין החולי זה ודאי ניוול יקרא.\\\vspace{0pt}

(ב׳) כבר כתבתי בתשובתי הנ״ל דמסתמא אדם מוחל זילותא לגבי יורשים כת״ק פ״ק דשקלים ומעכ״ת נ״י כתב על זה דלא ידע מה אענה לש״ס ב״ב (דף קנ״ד) דמשם הוכיח הנב״י דאין אדם מוחל ניוול משום הנאת יורשים ותמהתי אני הבאתי ראי׳ ממה שפסקינן בטוש״ע י״ד (סי׳ שנ״ו) בפי׳ דאדם מוחל זילותי׳ לגבי יורשים והוא השיב מראית הנב״י מגמרא דב״ב ואדרבא אשאל מה יענה מעכ״ת נ״י לסתירה זו אכן לו מר נ״י שם עיניו על מה שכתבתי גם הי׳ יודע מה אענה על ראית הש״ס דב״ב כי כבר עניתי שם במה שהבאתי דברי רש״י בסנהדרין (דף מ״ח) שכתב דת״ק סבר דמחיים אחלי׳ לזילותא וניחא לי׳ שיתבזה לאחר מיתתו להנאת יורשים עכ״ל והיינו משום דלא שייך מחילה לאחר מיתה וא״כ לא נאמר דאדם מוחל זילותא לגבי יורשים אלא בדבר שעלה על דעתו מחיים שיבא לידי כך ומוחל זילותי׳ כגון דהך דשקלים שקבצו מעות לעני והותיר דזוכר שאפשר שימות ולא יצטרך ומוחל לגבי יורשים מה שאין כן בהך דב״ב שם דאיך שייך דעלה על דעתו מחיים שיצטרכו לפתוח קברו ולנוולו משום הנאת יורשים וא״כ לא אפשר שמחל מחיים ואין מחילה לאחר מיתה ולפ״ז ביושבת על המשבר בסכנת מות בהקשותה בלדתה מסתמא עולה על דעתה שאם תמות יפתחו בטנה להציל לולד בפרט שזה ידוע בדיני רפואה כנ״ל וא״כ שייך בזה כמו בהך דשקלים דמסתמא מוחלת מחיים זילותא לגבי יורשים ומה גם אצל פרי בטנה אשר נפשה קשורה בנפשו וכי יש דמיון לזה שנאמר דאדם מוחל זילותי׳ גם לגבי אחר במקום שלא עלה על דעתו מחיים שיבא לידי כך.\\\vspace{0pt}

(ג׳) אפילו נאמר דלהוציא הולד ע״י פתיחת בטנה ניוול מקרי ושאינה מוחלת על בזיונה עם כל זה אין ראי׳ משם דלפי מה שכתבו התוספ׳ בנדה (דף מ״ד) אף דביושבת על המשבר ומקשה לילד מותר להרוג הולד להצילה מכ״מ היכא דמתה האם מי שהורג הולד חייב דכמונח בקופסא דמי וכ״כ ג״כ הרמב״ן הביאו המג״א (סי׳ ש״ל) כיון שמתה האם נחשב הולד כחי ודלת נעולה בפניו ע״ש וכיון דאמרינן בב״ב דאפילו משום זוזי דלקוחות שנתנו יכולים לטעון שינוול כש״כ שנאמר דמי שתופס את החי בקופסא ונועל דלת בפניו שיתנוול כדי להציל את החי מה שאין כן לנוול את המת בשביל פקוח נפש של אחר שאין לו תביעה על המת.\\\vspace{0pt}

והנה על ראיתי מדגזל עומד בפני פקוח נפש ה״ה ג״כ בזיון חבירו השיב מר נ״י דשאני גזל דאמרינן בב״ק כל הגוזל חבירו שו׳ פרוטה כאלו נוטל את נפשו והוי כמו שפיכת דמים מה שאין כן בבזיון חבירו ולא ידעתי אם פשיטא מזה למעכ״ת נ״י דהוי גזל כמו שפיכת דמים למה לא דן ג״כ דהוי בזיון כמו שפיכת דמים דהרי אמרינן ג״כ ב״מ (דף נ״ח) כל המלבין פני חבירו ברבים כאלו שופך דמים דאזיל סומקא ואתי חוורא ולא זו בלבד אלא דלפי טעם זה נאמר ג״כ דחלול מועדות עומד בפני פקוח נפש שהרי אמרינן מכות (דף כ״ג) כל המבזה את המועדות כאילו עובד ע״ז והרי ע״ז עומד בפני פ״נ אכן על כיוצא בזה כבר אמר הרב ר׳ יונתן ז״ל אם אמרתי אספרה כמו הנה דור בניך בגדתי אם נפרש כל כמו שאמרו רז״ל בדרך היקש כל הכועס כאלו עובד ע״ז כל המספר לשה״ר כאלו כפר בעיקר וכדומה ח״ו יהיו רוב ישראל עע״ז וכופרים אלא ודאי כל כאלו שאמרו רז״ל אינו רק שבענין א׳ דומה לזה אבל לא שנאמר שהיקש הוא לדון שמה שנוהג בזה נוהג בזה ולכן לענ״ד אין לנו טעם לומר בגזל שעומד בפני פ״נ אלא או הטעם שכתבתי (סי׳ קס״ז) שהרי ר״ע למד פ״נ שדוחה ל״ת מעבודה שדוחה שבת וראינו גזל שחמור משבת שהרי עבודה דוחה שבת ואינו דוחה גזל ואף שזה לא שייך רק לטעם דר״ע מכ״מ מהיכי תיתי נעשה פלוגתא חדשה בין טעם דר״ע לשאר הטעמים או שנאמר גם אם ילפינן מוחי בהם ולא שימות בהם הרי שם כתוב ועשית׳ את כל חקותי ואת כל משפטי ובזה לא התיר הקב״ה שלא למות רק מה שנוגע אליו דהיינו חקותיו ומשפטיו אבל לא מה שהוא לאחרים ולהיות גם רע לבריות למען הציל עצמו ממיתה או שנאמר אחר דילפינן שע״ז עומד בפני פ״נ מואהבת את ה׳ אלקיך וגו׳ בכל נפשך כדאמרינן בסנהדרין (דף ע״ד) ואמרינן שם יש לך אדם שממונו חביב עליו מגופו לכך נאמר בכל מאודך ולכן כמו דילפינן שם דשפיכות דמים עומד בפני פ״נ ה״ה ג״כ גזל כיון דיש אדם שממונו חביב עליו מגופו ולכל הטעמים האלה גם ביזוי חבירו עומד בפני פ״נ.\\\vspace{0pt}

אכן מעכ״ת נ״י רצה לומר דגם להציל עצמו בממון חבירו מותר ומה דאמרינן בב״ק אסור הפי׳ דאסור לסבב ולעמוד עצמו במקום שיגרם על עצמו סכנה ויהי׳ צריך להציל עצמו בממון חבירו ואני תמה איך יעלה על הדעת לומר כן וכי מותר לעמוד במקום סכנה אפילו אין צריך להציל עצמו בממון חבירו הרי עובר בושמור נפשו מאוד ולכן אני אומר שכל מי שמפרש דברי רש״י שדעתו שמותר להציל עצמו בממון חבירו הוא יוצא מגדר האמת וכמו שהוכחתי ג״כ מדברי רש״י באבות (פ״ה) שכ׳ שאסור להציל עצמו באברי המערכה והיינו שלא גרע ממון גבו׳ מממון הדיוט וודאי לשון אסור להציל עצמו בממון חבירו אי אפשר לפרש אלא או כפי׳ רש״י דאסו׳ להציל כלל או כמו שפירשו התוספ׳ והרא״ש והטור שאסור להציל עצמו בממון חבירו אם לא על דעת לשלם לבסוף אבל בנדון זה שבא לגזול כבוד המת שאי אפשר שישלם לו לכל השיטות אסור להציל עצמו בשל חבירו כיון שיודע ודאי בשעת הצלה שאי אפשר להשיב.\\\vspace{0pt}

עוד רצה מעכ״ת נ״י להוכיח דניוול המת נדחה מפני פ״נ דאל״כ נילף בק״ו דקבורת מת מצו׳ דוחה שבת ולא אאריך בתשובות שיש לי להשיב על זה דלענ״ד מעיקרא דדינא ליתא ואין דמיונו עולה יפה שמדמה איסור ניוול המת לקבורת מת מצוה וז״א דלא משום כבוד המת נאמר דאסור לנוולו אלא משום איסור גזל דאסור לגזול כבודו דעדיף מממון החי אבל מי שאינו קובר מת מצו׳ אינו גוזל ממנו אלא שלא חש לכבודו וכבוד המת ודאי לא דוחה פ״נ דכי יעלה על הדעת מי שבא לידו קבורת מ״מ והצלת החי מסכנה דעדיף לקבור את המת מלהציל את החי ואני לא אמרתי אלא שאסור לגזול מן המת את כבודו ולבזותו כדי להציל בו את החי ואם הגאונים בעל נב״י ובעל חתם סופר זצ״ל לא פסקו כן אני אומר אלו שמעו הגאונים זצ״ל את טענותי וחלקו עלי ודאי הייתי מבטל דעתי מפני דעתם אבל אחר שהם לא נחתי לעיקר ראייתי ממה דאמרינן אסור להציל עצמו בממון חבירו כאשר כבר הזכרתי (סי׳ קס״ז) שלא ראיתי לאחד מן הפוסקים שביאר ענין זה שיש עוד דבר חוץ מג׳ דברים שעומד בפני פקוח נפש לא נסגתי אחור והבוחר יבחר. כנלענ״ד, הקטן יעקב.\\\vspace{0pt}

\end{multicols}\newpage

\newchap{סימן קעב}
\begin{multicols}{2}
ב״ה אלטאנא, יום ו׳ ד׳ תשרי תרי״ג לפ״ק.\\\vspace{0pt}

כבר הערתי על ענין גזל במקום פקוח נפש וכתבתי שיש פלוגתת ראשונים בזה ועתה באתי לחקור עוד בענין קרוב לזה אם מותר להציל עצמו ממיתה ע״י שמבייש לחבירו.\\\vspace{0pt}

והנה (בסימן הקדום) כבר הערתי לפי מה דאמרינן ב״מ (דף נ״ח) כל המלבין פני חבירו ברבים כאלו שופך דמים א״כ שו׳ הלבנת פני חבירו לרציחה וכמו שברציחה יהרג ואל יעבור ה״ה בהלבנת פנים אלא ששם כתבתי שמזה אין ראי׳ דא״כ גם ביזוי המועדות דאמרינן במכות (דף כ״ג) דהוי כעובד ע״ז לא יעמוד בפני פ״נ אלא ודאי דלשון כאילו שדברו בו רז״ל הרבה פעמים אינו שו׳ בכל אלא במקצת ולכן אין ללמוד מזה גם להלבנת פנים במקום פ״נ. אכן מטעם אחר יש לדון דהלבנת פני חבירו ברבים אינה מותרת במקום פקוח נפש שהרי אמרינן בברכות (דף מ״ג) ובשאר דוכתי׳ אר״י משום רשב״י נוח לו לאדם שימסור עצמו לתוך כבשן האש ואל ילבין פני חבירו ברבים מנ״ל מתמר דכתיב היא מוצת ע״ש והנה בכתובות (דף ס״ז) ד״ה דכתיב היא מוצת כתבו התוספ׳ ר״ח גריס מוצת בלא אלף לשון ויצת אש בציון עכ״ל ובפני יהושע וספר כתובה שם נדחקו מה בעי התוספ׳ בזה ומה דחק לר״ח לפרש כן ולענ״ד הפי׳ ע״פ מה שהרחיבו התוספ׳ דבור זה בב״מ (דף נ״ט) שכתבו שם מוצת בלא אלף כמו ויצת אש בציון שהיתה קרובה לאש כבר ולא שלחה לו אלא ברמז אם לא הי׳ מודה לא היתה מפרסמתו עכ״ל ונראה כוונתם דלפי מה שפי׳ רש״י שם היא מוצאת אע״פ שהיו מוציאין אותה לשריפה לא אמרה להם ליהודה נבעלתי וכו׳ ואם יודה הוא מעצמו יודה עכ״ל לכאורה אין ראי׳ שנוח לאדם להפיל עצמו לכבשן האש ולא ילבין פני חבירו שהרי גם ברודף אחר חבירו להורגו אמרינן בסנהדרין (דף ע״ד) שאם יכול להצילו באחד מאבריו אסור להורגו וא״כ ה״נ ודאי בתחלה היתה צריכה לרמז לו שמא יודה ואולי אם לא הי׳ מודה באמת היתה מצלת עצמה ע״י שפרסמה שממנו הרה אכן זה לא שייך רק לפי׳ רש״י שבעת שהיו מוציאין אותה שלחה שאז עדיין הי׳ פנאי לשלוח אם לא הי׳ מודה ולכן פי׳ ר״ח דמוצת גרסינן שהיתה קרובה לאש כבר שאם לא הי׳ מודה מיד היתה נשרפת ולא הי׳ פנאי עוד לפרסם והצדקת רמזה זה שלא תרצה לפרסם אם לא יודה במה שהמתינה לשלוח לו עד שהיתה קרובה לאש ולא מקודם ועכ״פ מוכח מזה שהלבנת פנים חמיר מפקוח נפש.\\\vspace{0pt}

שוב ראיתי שהדבר מפורש בתוספ׳ סוטה (דף י׳) שכתבו אהא דואל ילבין פני חבירו ברבים ונראה האי דלא חשיב לי׳ בהדי ג׳ עבירות שאין עומדים בפני פ״נ ע״א וג״ע וש״ך משום דעבירת הלבנת פנים אינה מפורשת בתורה ולא נקט אלא עבירות המפורשות עכ״ל והנה מה שכתבו התוספ׳ דהלבנת פנים אינה מפורשת בתורה קצת קשה שהרי הוא בכלל הלאו דלא תשא עליו חטא כמש״כ הרמב״ם ה׳ דעות (פ״ו) וכ״כ הסמ״ג והחנוך וצ״ל דכוונתם דהלשון דלא תשא אינו מורה כ״כ מפורש אהלבנת פנים כמו ג׳ עבירות אכן מדברי רבינו יונה למדנו תירוץ אחר למה שהקשו התוספ׳ למה לא קחשיב גם הלבנת פנים בהדי ג׳ עבירות דיהרג ולא יעבור שכתב בשערי תשובה שער שלישי אחר שהביא שם דבשלש עבירות יש ג״כ אבק עבירה דיהרג ואל יעבור כגון להתרפאות מעצי אשירה לענין ע״ז או שתספר עמו לענין ג״ע וז״ל והנה אבק הרציחה הלבנת פנים כי פניו יחורו ונס מראה האודם ודומה אל הרציחה וכן ארז״ל והשנית כי צער ההלבנה מר ממות על כן ארז״ל לעולם יפיל אדם עצמו לכבשן האש ואל ילבין פני חבירו ברבים ולא אמרו כן בשאר עבירות חמורות אכן דימו אבק הרציחה אל הרציחה וכמו שאמרו כי יהרג ולא ירצח ודומה לזה אמרו שיפיל עצמו לכבשן האש ולא ילבין פ״ח ברבים ולמדו זה מענין תמר שנאמר היא מוצאת וגו׳ הנה כי אף שהוציאוה לישרף לא גלתה כי היתה הרה מיהודה שלא להלבין פניו עכ״ל הרי שכתב ג״כ כמו התוספ׳ דעל הלבנת פני חבירו יהרג ואל יעבור רק שהוסיף שהוא בכלל רציחה דמקרי אבק רציחה ולכן אין קושיא למה לא קחשיב בכלל ג׳ דברים שהרי גם אבק ע״ז כגון להתרפאות בעצי אשרה ואבק ג״ע לעמוד לפניו ערומה ולדבר עמה ג״כ לא קחשיב באפי נפשייהו ולכן ק״ל על הפוסקים בי״ד (סי׳ קנ״ז) שלא הזכירו רק אבק ע״ז להתרפאות בעצי אשרה וכן אבק ג״ע לספר עמה ולא ג״כ אבק דשפיכות דמים ואולי נרמז זה במה שכתב הרמ״א (שם) וכל איסור ע״ז וג״ע וש״ד אע״פ שאין בו מיתה רק לאו בעלמא צריך ליהרג ולא לעבור עכ״ל ולא נתבאר היאך משכחת לאו בעלמא אצל שפיכות דמים אבל לפי הנ״ל א״ש דמלבין פני חבירו הוא לאו דלא תשא עליו חטא.\\\vspace{0pt}

ומצאתי בספר תיבת גומא (להרב המחבר ספר פרי מגדים חקירה ה׳) שהביא ג״כ דברי התוספ׳ ור״י וכתב ג״כ שי״ל דלא חשיב הך דהלבנת פנים שהוא בכלל אבזרייהו דרציחה אלא שהקשה שם על התוספ׳ היאך שייך למילף מתמר שיהודה רצה לשורפה דעל הלבנת פנים יהרג ואל יעבור הרי ע״פ מה שכתבו התוספ׳ בפסחים דאם אומרים לאדם הנח עצמך להשליך על התנוק ואם לאו תהרג מותר דאדרבא מאי חזית דדמא דחברך סומק טפי דלמא דמא דידך סומק טפי כל שאינו עושה מעשה ובתמר היתה רשאה להציל עצמה ולפנים משורת הדין עבדה והניח בצ״ע ותמהתי היאך שייך בזה שלפנים משורת הדין עבדה שהרי לכל הפוסקים הובאו (סי׳ קנ״ז) אם לא במתכוין לעבור על דת אסור למסור עצמו בעבירות שדינם דיעבור ואל יהרג ונקרא מתחייב בנפשו והיאך מסרה תמר אם לא היתה מתחייבת מן הדין גם מה שהקשה מדברי התוספ׳ לענ״ד ל״ק שהרי שם הטעם מפני שהוא אינו עושה מעשה אבל בהך דתמר הרי אם היתה בשב ואל תעשה לא היתה יכולה להציל עצמה רק במה שעשתה מעשה דהלבנת פנים והיאך ידמה זה למניח עצמו להפיל על התנוק וביותר הי׳ אפשר להקשות היאך ילפינן מיהודה כיון שהוא רצה לשורפה שלא כדין א״כ נחשב רודף והבא להרגך השכם להורגו אבל באמת גם זה לא קשה שהרי יהודה בעצמו לא הי׳ הורג אותה רק ע״פ ציוויו היתה נהרגת ולענין זה לא נקרא רודף להציל עצמו בדמו ולכן שפיר ילפינן מתמר שע״כ ע״פ הדין היתה מוסרת עצמה למיתה דמשום הלבנת פני חבירו ברבים יהרג ואל יעבור.\\\vspace{0pt}

ומה נקרא רבים לענין זה כתב הפ״מ שם וז״ל ומסתברא דרבים הוי כל שיש שלשה דהיינו הוא ושנים עמו ובייש לא׳ משוק הוי רבים כמו בטומאה דר״ה ספק טהור היינו שלשה ויש רבים עשרה מכ״מ כאן בשלשה נמי כדאמרן עכ״ל גם הביא שם דברי הרמב״ם ה׳ דעות שכתב שלא לבייש חבירו ברבים בין גדול בין קטן וכתב מה שכתב הרמב״ם קטן מסברא הוא דהורג קטן בן יומו חייב ה״ה לבייש דכרוצח הוא עכ״ד ולענ״ד ק׳ הרי מה שנחשב כרוצח הוא משום דאזיל סומקא ואתי חוורא וזה לא שייך בקטן בן יומו ולכן נ״ל שמה שכתב הרמב״ם קטן לא מסברא כתב כן אלא ע״פ הגמרא דב״ק (דף פ״ו) דאמרינן שם דקטן יש לו בושת ומסקינן דהיינו בשיש בו דעת דמכלמי לי׳ ומיכלם וכן נפסק ברמב״ם ה׳ חובל (פ״ג) ובטוש״ע ח״מ (סי׳ ת״ך) ולקטן כזה נתכוון הרמב״ם במש״כ בה׳ דעות שלא לביישו ברבים.\\\vspace{0pt}

ויצא לנו לפ״ז דאם יכול להציל עצמו ממיתה ע״י שיבייש גדול או קטן דמרגיש בבושה בפני שני ישראלים חייב למסור עצמו למיתה ולא יביישו אבל מפני קטן שאינו מרגיש בבושה אמרינן יעבור ואל יהרג וישן יש לו דין גדול דהמבייש את הישן חייב כמבואר בב״ק ובפוסקים שם וא״כ ה״ה לענין שיהרג ואל יעבור לביישו כיון שמכיר בבשתו לכשיקיץ ואזיל סומקא ומכל זה לא ראיתי מבואר בפוסקים. כנלענ״ד, הקטן יעקב.\\\vspace{0pt}

\end{multicols}\newpage

\newchap{סימן קעג}
\begin{multicols}{2}
ב״ה אלטאנא, יום ה׳ ה׳ טבת תרי״ג לפ״ק. להרה״ג וכו׳ מ״ה בער אפפענהיים נ״י הגאב״ד דק״ק אייבענשיץ יע״א.\\\vspace{0pt}

מר נ״י השיג על פסקי בדין למסור עצמו שלא לבייש ולענ״ד כל דבריו תמוהים כאשר אבאר: השיג על מה שהוצאתי פסק מהך מימרא דנוח למסור עצמו וכו׳ דאין למדין מדברי אגדה כמש״כ התה״ד בפסקיו (סי׳ ק״ח) לא ידעתי למה הוצרך להשיג מתה״ד שאין ראי׳ כלל מדבריו לנדון זה שהוא לא כתב שם רק שאין מחלקין במצו׳ מפני טעם שבאגדה כמו שהוכיח ממה שנתנו טעם למילה לשמנה שלא יהיו אביו ואמו עצבים והרי גם ביולדת בזוב ג״כ נמול לשמנה ע״ש אבל שלא נלמוד הלכה ממה שנאמר באגדה דרך הלכה בראי׳ מן הפסוק לא שמענו מזה ולכן ביותר ה״ל להביא מירושלמי פ״ב דפיאה הביאו גם התוספ׳ י״ט ברכות פ׳ ה׳ ר״ז בשם שמואל אין למדין לא מן ההלכות ולא מן ההגדות ולא מן התוספתות אלא מן הגמרא עכ״ל אכן גם מזה אין השגה דכבר כתב בקול הרמ״ז שם דמה שאין למדין ממדרש היינו אם שיש בגמרא היפך המדרש אבל כשאין בגמרא סתירה למה לא נלמוד מן המדרש שהרי הר״ת פסק שאין לאכול בין מנחה למעריב בשבת מטעם הנזכר במדרש וכמה דינים למדו הפוסקים ממדרש הזוהר ע״ש וכחילוק זה כתב ג״כ הפ״ח במים חיים וכן הביא בספר יד מלאכי בשם רדב״ז וכה״ג ולחם יהודה ולכן ודאי היטב כתבו התוספ׳ להלכה מהא דתמר דמשום הלבנת פנים יהרג ואל יעבור וכבר הזכרתי שגם הרבינו יונה כתב כן וכעת ראיתי שגם הפר״ח במים חיים כתב כן להלכה.\\\vspace{0pt}

ומה שהשיג מר נ״י על פסק זה בהסכימו עם מה שכתב הרב פרי מגדים שהרי ממה דאמרינן אסתר קרקע עולם הוי מוכח דהיכי דאינו עושה מעשה אמרינן דיעבור ואל יהרג והרי במבייש את חבירו אינו עושה מעשה למ״ד עקימת שפתיו לא הוי מעשה הלא כבר השבתי על זה (סי׳ ק״ע) ואבאר בזה יותר שאין הדבר תלוי בעושה מעשה אלא אם הוא נשאר בשב ואל תעשה כמו גבי אסתר קרקע עולם שמצדה לא נעשה התעוררות כלל וכמו בהך נדון דהתוספ׳ דהנח עצמך להפיל על התנוק מה שאין כן בהלבנת פנים שאם נשארה בלי התעוררות כלל כמו קרקע עולם לא נעשה ההלבנת פנים ובזה יהרג ואל יעבור ואף שאינו רק מדבר ודאי גם על דבור יהרג ואל יעבור שהרי פשיטא דלומר לע״ז אלי אתה יהרג ואל יעבור דאיכא חיוב מיתה על זה אף שאינו רק מדבר.\\\vspace{0pt}

על מה שהשגתי על דברי הפ״מ שכ׳ דתמר לפנים מש״ה עשתה שהרי אם אינה מצו׳ היתה אסורה למסור משום וחי בהם כתב מר נ״י שלא ידע איך שכחתי סוגיא דסנהדרין דב״נ אינו מצו׳ על ק״ה וא״כ וחי בהם לא נאמר בתמר דאפילו להנך שיטות דבני יעקב היו להם דין ישראל תמר לא היתה מבני יעקב ואך את דמכם לא נאמר רק במאבד עצמו לדעת ולא במוסר עצמו למיתה ולכן מסיק שדברי התוס׳ דסוטה תמוהים ולפענ״ד מר נ״י לא צדק בשתים.\\\vspace{0pt}

בראשון – מה שכתב דלכל השיטות רק בני יעקב היו להם דין ישראל הרי הרמב״ן הוא אב שיטה זו וכתב בפ׳ אמור במגדף דהי׳ לו דין ישראל שמעת שבא אברהם בברית הי׳ לו דין ישראל והביא ראי׳ ממה שאמרו בעשו ודלמא ישראל מומר שאני ע״ש הרי אפילו עשו הי׳ נדון כישראל לשיטה זו וא״כ כש״כ נשי האבות ובניהם שהרי כלם נקראו בני ובנות אברהם שהוא אב לגרים דכתיב כי אב המון גוים נתתיך ועוד אי ס״ד דהאבות הי׳ להם דין ישראל אבל נשיהם לא היאך בניהם מתיחסים אחריהם שהרי ולד נכרית כמותה וא״כ ודאי גם תמר שהיתה גיורת ואשת בני יהודה שהרי על זה הי׳ נדונת למיתה היתה כבני יעקב לדין ישראל.\\\vspace{0pt}

בשנית – מה שכתב דאם תמר הי׳ לה דין ב״נ הותרה למסור עצמה דאך את דמכם לא נאמר רק במאבד עצמו לדעת ולא במוסר עצמו הנה מלבד שמר נ״י לא הראה לנו חילוק בזה אלא שתמהתי אחרי שבעצמו הזכיר דברי הפרשת דרכים איך לא זכר היסוד שעליו בנה הפ״ד כל באוריו בג׳ דרשותיו הראשונות והוא מה שכתב (דרוש א׳) והנה המבחין לידע אם יצאו קודם מתן תורה מכלל ב״נ לגמרי או לא הוא הצלתו של א״א מאור כשדים שהרי ב״נ אינם מוזהרים על ק״ה וקיי״ל שכל מי שנאמר בו יעבור ואל יהרג אם נהרג ה״ז חובל בעצמו ועובר על מה שכתוב ואך את דמכם וכו׳ עכ״ל הרי בפי׳ שבמקום שא״צ למסור עצמו גם ב״נ עובר על ואך את דמכם אם מוסר עצמו ולכן שפיר כתבו התוספ׳ ור״י להלכה דמהך דתמר מוכח דמשום הלבנת פנים יהרג ואל יעבור. כנלענ״ד, הקטן יעקב.\\\vspace{0pt}

\end{multicols}\newpage

\newchap{סימן קעד}
\begin{multicols}{2}
ב״ה אלטאנא, יום ה׳ י״ז טבת תר״ט לפ״ק. להרה״ג וכו׳ מ״ה מתתיהו מונק הכהן נ״י אב״ד דק״ק קראינקע יע״א.\\\vspace{0pt}

כתב לי מעכ״ת נ״י קשה לי פליאה עצומה על הראשונים והאחרונים שנעלם מהם במחילת כבודם גמרא ערוכה בדין ש״ץ קבוע א״ח בסי׳ נ״ג סעיף ט״ו דאיתא שם בהמחבר וז״ל ש״ץ קבוע יורד לפני התיבה מעצמו ולא ימתין שיאמרו לו והטור הביא דין זה וביאר הבית יוסף זצ״ל דהטור דייק זה מהש״ס ברכות דף ל״ד דאיתא שם העובר לפני התיבה צריך לסרב וכו׳ ומפרש רבינו דהיינו במי שאינו ש״ץ קבוע דאלו ש״ץ קבוע אין לו להמתין שיאמרו לו כלום אלא הוא מעצמו יעלה וכו׳ יעוי״ש וחידוש גדול על הרב בית יוסף דלא הביא המקור לדין זה מגמ׳ ערכין דשם נאמר באר היטב ולא צריכא שום דיוק ולבד זאת דנאמר שם מנטילת רשות לש״צ נראה עוד דאפי׳ ש״צ קבוע צריך ליטול רשות דאיתא שם דף י״א בעי רבי אבין עולת נדבת צבור טעון שירה או לא כו׳ ופשט מקרא ויאמר חזקי׳ וגו׳ ומסיק הגמ׳ שם אפילו תימא עולת חובה מידי דהוה אשליח דציבורא דממליך הרי דבפירוש נאמר דש״ץ צריך נטילת רשות ואף שהוא ש״ץ קבוע דגמרא מדמה עולת חובה שבכל יום ויום לש״ץ וע״כ לש״ץ הקבוע להתפלל לפני התיבה תמיד ושם בברכות לא איתא כלל מנטילת רשות רק צריך לסרב מעט וזה מפני כבוד הצבור וצע״ג שלא הרגיש שום מחבר מהאחרונים דבר זה.\\\vspace{0pt}

תשובה: מר נ״י הקשה על הב״י דה״ל להביא מקור הדין מהא דערכין ועל הטור הקשה שיש סתירה משם ולענ״ד אין מקור ואין סתירה ממה דאמרינן שם מידי דהו׳ אשליחא דצבורא דנמלך שהרי פשוט הוא שאין לאדם להתפלל לפני הצבור אם לא שהצבור נתנו לו רשות לזה בפעם ראשון ועשאוהו שלוחם שהרי עי״ז נעשה ונקרא שליח צבור וזה בעצמו ילפינן מקרבן כמש״כ המהרי״ק שורש מ״ד הביאו הב״י שהתפלה היא של הקהל במקום התמידין שהיו באין משל צבור ואין ראוי שיהי׳ אדם שלוחם להקריב קרבנם שלא מדעתם ורצונם עכ״ל וכ״כ ג״כ הט״ז (סי׳ נ״ג) וזה בלבד הוא ממה דאיירי בערכין שהרי הך ויאמר חזקי׳ דמפרשינן שנטל רשות להקריב קרבנות נדבה (ולפי מה דמסיק אפילו חובה) כתיב בד״ה אחרי שטהר הקדש וחנכו והתחיל העבודה במקדש אחר שהושבתה שנים רבים בימי אחז ובזה הוצרך לטול רשות מב״ד בשם ישראל להיות מעתה העבודה שתתחיל בשליחותם כמו שליח צבור כשרוצה להתחיל בעבודתו צריך לטול רשות משולחיו דאל״כ אינו שלוחם אבל אין ראי׳ משם דש״ץ בכל תפלה ותפלה צריך לטול רשות שהרי הסברא נותנת דחזקי׳ לא בכל קרבן וקרבן שהוקרב מאז והלאה נטל רשות מב״ד ולא עוד אלא אפילו נדחוק דבכל יום הי׳ צריך נטילת רשות להקריב קרבנות חובה בשביל הצבור אכתי לא דמי לש״ץ קבוע שהרי בקרבנות צבור הי׳ כ״ד משמרות כהונה וכל משמר נתחלק לז׳ בתי אבות עד שלכל בית אב לא הגיע העבודה אלא יום א׳ בכ״ד שבועות וגם אז ע״פ הרוב עבדו כהנים חדשים שלא עבדו מקודם א״כ בכל פעם הי׳ כתפלה בעד הצבור בפעם ראשון אבל אכתי אין לפשוט מזה אם ש״ץ קבוע צריך לטול רשות בכל עת שמתפלל ולכן לא הי׳ הב״י יכול להביא ראי׳ או סתירה מהא דערכין אם הך דצריך לסרב הוא בש״ץ קבוע ג״כ או לא. כנלענ״ד, הקטן יעקב.\\\vspace{0pt}

\end{multicols}\newpage

\newchap{סימן קעה}
\begin{multicols}{2}
ב״ה אלטאנא, יום ו׳ כ׳ תמוז תר״ח לפ״ק. להרבני וכו׳ מ״ה יהודא עברי נ״י בק״ק מערגענטהיים יע״א.\\\vspace{0pt}

הקשה מר נ״י וז״ל גרסינן בב״מ דף ס״ב שנים שהיו מהלכים בדרך וביד א׳ קיתון של מים וכו׳ עד שבא ר״ע ולמוד וחי אחיך עמך חייך קודמין לחיי חברך והנה הרי״ף והרא״ש הביאו ברייתא זו בהלכותיהם ובלי ספק שהלכה כר״ע מחבירו אבל חפשתי בהרמב״ם בה׳ רוצח ובשאר מקומות הנוגעים לענין זה ולא מצאתי שהעתיק דין זה וצ״ע טובא שלא זו בלבד שדרכו להעתיק כל הדינים אף שלא נהגו כעת אף זו שהדין הזה יארע ג״כ בימינו לפעמים בעוברי ימים כידוע עכ״ל דמר נ״י.\\\vspace{0pt}

על זה אשיב דלפי המבואר בסוגיא דב״מ (דף ס״א ע״ב) פליגי ר״א ור׳ יוחנן בריבית קצוצה דלר״א יוצאה בדיינים ולר׳ יוחנן אינה יוצאה ומפרש טעמא דר״א מוחי אחיך עמך והקשה בגמרא ר״י האי קרא מאי עביד לי׳ ומשני דאתי להא דר״ע דחייך קודמין ומשמע מזה דלר״א דצריך קרא לרבית קצוצה לא דרשינן הא דר״ע ואין לתמו׳ דלפ״ז הא דר׳ אלעזר תנאי היא שהרי בלא״ה מסיק בסוגיא שם דתנאי פליגי בזה והשתא כיון דהרמב״ם ה׳ מלו׳ ולו׳ (פ״ה ופ״ו) פסק כר׳ אלעזר דרק״צ יוצאה בדיינים דע״כ אתיא מוחי אחיך לא יכול לפסוק כרע״ק כיון דלית לן קרא לדרשת ר״ע ומכ״מ כבן פטורא ג״כ לא רצה לפסוק כיון דב״פ רק מסברא קאמר ור״ע פליג ולכן נסתפק בהלכה והשמיט לגמרי. הן אמת שנראה מהרא״ש שכבר הרגיש בזה וכתב דאף דר״א צריך קרא לרק״צ מכ״מ אתיא דרשה דר״ע מעמך אכן ודאי פשטות דסוגיא לא משמע כן דא״כ דאפשר לדרוש שני הדרשות מוחי אחיך עמך אכתי יקשה דלמא כולי להכי הוא דאתי ומנ״ל לר״א ללמוד רק״צ מזה וראיתי בש״מ שהקשה כן בשם תלמידי ר״פ ותירץ בדוחק ולכן ודאי טעמא רבה איכא להרמב״ם דפסק ע״פ פשטות דסוגיא דר״א לא ס״ל הך דר״ע ולכן גם הטוש״ע ח״מ (סי׳ תכ״ו) השמיטו דין זה.\\\vspace{0pt}

והנה בחדושי ליבמות כתבתי שממה שכתבו התוספ׳ שם (דף נ״ג ע״ב) בד״ה אין אונס וז״ל ואין מצו׳ להציל חבירו בגופו דאדרבא חייו קודמין עכ״ל לכאורה נראה דאליבי׳ דר״ע כתבו כן ופסקו כוותי׳ דלדעת בן פטורא לא מותר להניח עצמו לזרוק על התנוק אבל א״ע נראה שי״ל דגם אליב׳ דב״פ כתבו התוספ׳ כן דהנה יש להסתפק לב״פ אם בעל המים עוד לא שתה כלל וחבירו כבר שתה אבל לא כל צרכו ואם ישתה עוד מעט יחי׳ חבירו ששתה עתה כל צרכו והוא ימות ואם ישתה הוא כל המים שבידו יספיק לו לחיות על ידם אבל חבירו ימות איך הדין בזה שבזה לא שייך סברת ב״פ דאל יראה א׳ במיתת חבירו דממנ״פ האחד רואה במיתת חבירו בכל אופן שיעשו ונ״ל דבזה גם בן פטורא מודה דישתה הוא דחייו קודמין ולכן בנדון דהתוספ׳ דג״כ ממנ״פ האחד ימות דאם נזרק התינוק מת והוא חי ואם לא מניח לזרוק הוא נהרג והתינוק חי בזה י״ל דגם ב״פ מודה דחייו קודמין ובזה א״ש דלא יקשה בן פטורא האי עמך מאי עביד לי׳ דלפ״ז י״ל דגם הוא דרש עמך בכה״ג רק דפליג אר״ע דדרש לי׳ גם לענין היכי שאפשר שאין א׳ רואה במיתת חבירו כן העלתי בחדושי וע״פ זה כל שכן שא״ש למה לא הביא הרמב״ם עכ״פ דעת ב״פ כיון דהוא ג״כ צריך דרשה דעמך וא״כ ר׳ אלעזר דדרש וחי אחיך עמך לרבית קצוצה לית לי׳ דרשה לא לדר״ע ולא לב״פ ואם שע״כ מסברא אמרינן כחד מינייהו מכ״מ אין הכרע מתוך הסוגיא והרי ידוע שהרמב״ם לא מביא רק מה שמבואר בגמרא. כנלענ״ד, הקטן יעקב.\\\vspace{0pt}

\end{multicols}\newpage

\newchap{סימן קעו}
\begin{multicols}{2}
ב״ה אלטאנא, יום ו׳ כ״ה סיון תרי״ג לפ״ק. להרה״ג וכו׳ מ״ה בירך אברהם פינטנער נ״י הגאב״ד דק״ק זקאליץ יע״א.\\\vspace{0pt}

מעכ״ת נ״י כתב אלי: בקידושין (דף מ״ג) ד״ה שלא מצינו הקשו התוספ׳ על מה דאמרינן שלא מצינו בכל התורה כולה זה נהנה וזה מתחייב וא״ת והא איכא מעילה שאם אמר הגזבר לשליח אכול ככר זה של הקדש והשליח לא ידע שהוא של הקדש דחייב המשלח ואמאי זה נהנה וזה מתחייב הוא וי״ל דמעילה לא מתחייב המשלח ע״י הנאת השליח אלא ע״י הגבהה דהשליח דמיד דאגבהי׳ שליח קניא והאי שעתא חייב המשלח עכ״ל וקשה עלי עד מאוד דהרי בכורות (דף י״ג) העלו התוספ׳ דבהקדש לא שייך קנין משיכה דכל היכא דאיתא בי גזא דרחמנא איתא עכ״ד.\\\vspace{0pt}

על זה אשיב דהנה יש פלוגתא בין הרמב״ם להראב״ד דלהרמב״ם בגנב הקדש והוציא מרשות הקדש לרשותו לא מעל אם לא נהנה כנראה מדבריו ממה שכתב הל׳ מעילה (פ״ו הל׳ ח׳) נטל אבן או קורה של הקדש לא מעל שהרי לא נהנה והראב״ד השיג שם עליו שזה דוקא בגזבר שהכל תחת ידו שנטילתו אינה ניכרת לגזל ההקדש אבל מי שגנב מבית הקדש ונתנו לחבירו מעל על ידי שהוציא מרשות הקדש לרשות אחר אע״פ שלא נהנה וכן נראה ג״כ דעת התוספ׳ בקידושין הנ״ל וכן כתבו בפי׳ גם התוספת מעילה (דף י״ח) ד״ה ואומר וז״ל ועוד יש שינוי אחר בלא שום הנאה כגון שינוי מרשות לרשות שמוציא מרשות הקדש ומכניס לרשות הדיוט כגון מכירה ונתינה והשאלה ומיד שנשתנה לרשות הדיוט דהיינו שעשה בה ההדיוט מעשה שחשוב זכיי׳ וקנין מהדיוט להדיוט אע״פ שלא נהנה וכו׳ עכ״ל הרי דס״ל ג״כ דהיכי שעשה שום קנין שמועיל בהדיוט מועיל ג״כ לענין מעילה וזה שיטתם ג״כ בקידושין ומה שכתבו בבכורות דבהקדש לא שייך קנין משיכה דכל היכי דאיתא בי גזא דרחמנא איתא מלבד שיש לומר שיש חילוק בין קנין דמשיכה דגרוע הוא שהרי אינו קונה בכל מקום ובין קנין דהגבהה דקונה בכל מקום כדאמרינן ב״ב (דף פ״ד) ופסק כן הרמב״ם ה׳ מכירה (פ׳ ד׳) ע״ש ולכן לא שייך לענין קנין דהגבהה דבי גזא דרחמנא איתא שהרי הגבהה קונה בכל מקום גם ברשות המוכר משא״כ במשיכה שאינה קונה רק בסמטא או בחצר של שניהם עוד יש חילוק אחר בין קנין שאיירי בבכורו׳ שרוצה לקנות מן ההקדש בתורת מכירה דבזה העמידו החכמים על דין תורה דרק מעות קונות ובין קנין דמעילה שקונה בתורת גניבה וגזילה דלא משכחת בזה קנין אחר רק ע״י שעושה מעשה קנין להוציא מרשות הקדש לרשותו דאל״כ לא משכחת מעילה לעולם לשיטת התוספ׳ והראב״ד דס״ל דגם בהוצאה מרשות לבד בלא הנאה מועלים. אכן כבר הזכרתי שדעת הרמב״ם אינה כן דבהוצאת מרשות הקדש לבד בלא הנאה אין מועלין ובזה נ״ל ליישב מה שתימה בתוספ׳ ר׳ עקיבא במעילה (פ׳ ה׳) דאמה דתנן שם נתנה לחבירו הוא מעל כתב הרמב״ם בנתינת המתנה בטובת הנאה הבאה לו וביאר בתוספ׳ י״ט דהכי איתא בגמרא ב״מ (דף צ״ט) וכתב על זה בתוספ׳ ר״ע אינו מובן לי דהתם בב״מ בקורדם שהשאיל לחבירו לבקע בו א״כ נתן לו רק הנאת בקוע אבל בנתן כולו מעל בכולו וכ״כ תוספ׳ שם להדיא עכ״ל ולענ״ד הרמב״ם לשיטתו אזיל דס״ל דבהוצאה מרשות הקדש לרשות הדיוט אינו מועל רק בנהנה וא״כ איך ימעול הנותן לחבירו לכן פירש דמועל בטובת הנאה שיש לו ע״י נתינה שזה הנאתו אבל מה ששו׳ יותר אף שהוציא מרשות הקדש מכ״מ לא מעל כיון שלא נהנה בזה אכן התוספ׳ לשיטתם אזלי דס״ל דבהוצאה מרשות הקדש לבד גם בלא הנאה מעל ולכן כתבו דבנתנו לחברו מעל בכולו. כנלענ״ד, הקטן יעקב.\\\vspace{0pt}

\end{multicols}\newpage

\newchap{סימן קעז}
\begin{multicols}{2}
ב״ה אלטאנא, יום ו׳ כ״ה סיון תרי״ג לפ״ק. להרב וכו׳ מ״ה מאיר ליב לעבוואהל נ״י בק״ק קראקא יע״א.\\\vspace{0pt}

כתב אלי מר נ״י וז״ל שמעתי ממעכ״ת נ״י שתירץ על קושית הגאון מ״ה נחום טרעביטש ז״ל היאך חייבה התורה ליתן מתנות לכהן דלמא הבהמה טרפה היא וא״ל דהולכין אחר הרוב דאכתי יקשה לשמואל דס״ל אין הולכין בממון אחר הרוב ופסקינן כוותי׳ בח״מ סי׳ רל״ב ותירץ מר נ״י דממון התלוי באיסור דנין כאיסור והולכין בו אחר הרוב ותירץ בזה גם קושית השער המלך שהקשה איך פודין בכור דלמא טרפה הוא ואמר לי שאח״כ מצא סברא זו גם בספר הכתובה ספ״ק דכתובות. אמנם מצאתי ראי׳ מפורשת נגד סברא זו במרדכי בחולין פ׳ הזרוע וז״ל ונראה לראבי׳ דהקונה הלחיים דכלפי הראש מן הכותים וגם הזרוע והראש של ישראל דפטור כיון דניהוג בזה שאם תמצא טרפה שמניח לכותי אימת מחייב משעת זביחה והאי שעתא לית ברירה דליהוי דידי׳ אע״ג דרוב בהמות כשרות הן ק״ל כשמואל דאמר ר״פ המוכר פירות כי אזלינן בתר רוב ה״מ לענין איסור אבל לענין המוציא מחבירו לא עכ״ל הרי בפי׳ אף דהוי ממון דתלי באיסור מכ״מ אין הולכין בממון אחר הרוב ותמהני על הגאונים איך לפי שעה נעלם מהם ראי׳ זו וז״ב עכ״ד.\\\vspace{0pt}

על זה אשיב: אף שיפה דיבר מר נ״י שראוי לעמוד על ראי׳ זו עכ״ז א״ע נלענ״ד שאין מזה סתירה להכלל שמוכח מכמה דוכתין דממון התלוי באיסור אזלינן לגבי׳ בתר רוב כמו לענין איסור אכן זה דוקא אם בשעה שנולד הספק לענין ממון כבר הלכנו אחר רוב לענין איסור דבזה אין לחלק בין ממון לאיסור אבל בנדון דהראבי׳ ששם ע״כ אנו דנין אחר שעת הזביחה שאז צריך להיות מבורר שהבהמה אינה טרפה ושהיא בהמת ישראל אם תתחייב במתנות דמה שהוברר אח״כ אין ראי׳ דאין ברירה כנראה מדבריו ממה שכתב אימת מחייב משעת זביחה וכמש״כ הש״ך בי״ד (ס׳ ס״א) והרי בשעת זביחה עדיין אין אנו הולכין אחר רוב בהמות לענין איסור דעדיין היא בחזקת איסור דאינו זבוח דמטעם זה לדעת הרבה פוסקים דפסק גם הרמ״א י״ד (סי׳ נ׳) כוותייהו תנולד ריעותא מחיים לא אמרינן נשחטה הותרה לילך בתר רוב בהמות דאינן טרפות ועוד שהרי לא דוקא אם תמצא הבהמה טרפה היא לכותי אלא גם אם תמצא נבלה בשחיטה ולכן בתחלת שחיטה עדיין אינו מבורר שהבהמה היא של ישראל ומה שכתב הראבי׳ דאין הולכין אחר רוב בהמות דכשרות הן משום דאין הולכין בממון אחר הרוב לרווחא דמלתא כתב כן דבלא״ה לא שייך בזה רוב בהמות כשרות דעד שתודע שנשחטה כראוי בחזקת איסור עומדת ומחזקינן לה בתחלת שחיטה עוד כבהמת נכרי ועוד שבאמת נראה שגם להראבי׳ לא הי׳ ברור לומר שאפילו בענין כזה אין הולכין בממון אחר הרוב שהרי סיים ועוד דטריפות דריאה שכיחי עכ״ל הרי שלא סמך על סברא לחוד דאפשר דג״כ מספקא לי׳ אם אפילו לענין כזה שעוד לא דנין רק על ממון אמרינן דאין הולכין בממון אחר הרוב אבל עכ״פ באותן הנדונים שכתבתי כן שכבר אנו דנין על רוב איסור נראה ברור שאין לחלק בין ממון לאיסור שיהי׳ כחוכא וטלולא שיהי׳ הישראל נאמן לומר לכהן אייתי ראי׳ שהבהמה אינה טרפה והוא אוכל אותה על סמך רוב בהמות אינן טרפות או שיהי׳ פטור מלפדות הבכור שמא טרפה הוא וההורגו נהרג עליו בחזקת שאינו טרפ׳ או שהממזר לא יירש את אביו שמא אינו בנו ואם מקללו נהרג עליו ולכן כתבתי דממון התלוי באיסור שאנו דנין אחר רוב איסור גם הממון נגרר אחריו שנלך בו אחר הרוב. כנלענ״ד, הקטן יעקב.\\\vspace{0pt}

\end{multicols}\newpage

\newchap{סימן קעח}
\begin{multicols}{2}
ב״ה אלטאנא, יום ה׳ ב׳ אלול תרט״ו לפ״ק. להרב וכו׳ מ״ה פנחס שיפפער נ״י בק״ק לעמבערג יע״א.\\\vspace{0pt}

על דברי מעכ״ת נ״י שכתב אלי וז״ל בחו״מ (סי׳ ל״ד סעי׳ כ״ח) ברמ״א ב׳ עדים שהעידו על אחד שהוא פסול כגון שאחד אומר שגנב והשני אומר שהלוה בריבית מצטרפים לפסלו עכ״ל הרמ״א וציין ע״ז בבאר הגולה מעובדא דבר בונתוס סנהדרין ד׳ כ״ה ודברי הבאר הגולה תמוהין דשם מבואר דשניהם העידו לפסול אותו דאוזיף בריבית אבל בשתי עבירות לא מיירי כלל שם אח״כ עיינתי בשו״ת מהר״ם מפדווא סי׳ ל״ז וראיתי שכתב שדין זה דאפילו בשתי עבירות ב׳ עדים יכולים לפסלו הוא מבואר ברי״ו נתיב ב׳ ח״ד וכתב דכן משמע בסנהדרין פ״ב ותמה עליו ג״כ במהר״ם פדווא כמו שתמהתי על הבאר הגולה אכן מ״מ לדינא הסכים שם לרבנו ירוחם הגם שאין לו ראי׳ יעיי״ש ולדידי הצעיר יש לי ראי׳ אחת להיפוך מדברי התוס׳ ב״ק (ד׳ ע״ב) ד״ה אין לך בו אלא משעת חידושו ואילך כו׳ וא״ת ולמה הוי חידוש כלל והלא מן הדין יש להאמין בתראי במגו דהוי פסלי לקמאי בגזלנותא כו׳ ואור״י דלא שייך מגו אלא באדם אחד אבל בשני בני אדם לא שייך מגו דאין דעת שניהם שוה שמה שירצה לטעון זה לא יטעון זה עכ״ל והנה פשוט כוונת התוס׳ דאין זה מגו משום הגם דשניהם רוצים לפוסלו מ״מ דלמא אחד יפסול אותו בגניבה ואחד בהלואת ריבית וכדומה ולא יהי׳ עדותן מכוונת ולא יועילו כלום בעדותן וא״כ מבואר להדיא מדברי התוס׳ דהיכי דשניהם מעידים על שתי עבירות אינם יכולים לפסלו דאין לומר דכוונת התוס׳ דלא יכוונו ואחד יפסול אותו בהזמה ז״א דהא השתא סברינן דאית להו מגו לפסלו בלא הזמה גם אין לומר דלמא לא יכוונו ואחד יפסול אותו והאחר ישתוק גם ז״א דהא שניהם באים לפסלו לכן דין זה צ״ע עכ״ד.\\\vspace{0pt}

אשיב: לענ״ד אין מדברי התוספת בב״ק ראי׳ נגד פסק הרמ״א דלפי מה שמפרש מעכ״ת נ״י דברי התוספ׳ דאין זה מגו דדלמא האחד יפסול בגניבה וא׳ ברבית מה זה הלשון שכתבו התוספ׳ דאין דעת שניהם שו׳ שמה שירצה לטעון זה לא יטעון זה עכ״ל מה לשון שמה שירצה ה״ל לכתוב שמה שיטעון זה לא יטעון זה ולכן נראה שכוונתם כיון שאנו אומרים דלמא בשקר הסכימו להזים בזה אין להאמינם במגו דאי בעי משקרי הוי פסלי להו בגזלנותא דדלמא באמת נשאו ונתנו בזה אבל לא הסכימו שמה שרצה זה לטעון לא רצה זה והם רצו להעיד בהסכמה אחת והסכימו בהזמה ולכן אף שהיו נאמנים אם היו מעידין כל א׳ בעבירה לעצמה דלמא הם לא ידעו זה או רצו להעיד שקר בהסכמה אחת כמו שעשו באמת כן נראה מלשון שמה שירצה לטעון זה אכן אפילו לא נדייק לשון שמה שירצה אז נראה כוונת התוספ׳ כמו שכתבו בכתובות (דף י״ט ע״ב) ד״ה ואם וז״ל ואור״י דבשני עדים לא שייך מגו דאין אחד יודע מה בלב חבירו עכ״ל וא״כ הפירוש דאין זה מגו דאי בעי פוסל בגזלנותא דכיון דאין א׳ יודע מה בלב חבירו כל א׳ הי׳ חושב דלמא חבירו יפסול בהזמה ואם הוא יפסול בגזלנותא אין כאן פסול לא משום הזמה ולא משום גזל ולכן הזים גם הוא אבל לעולם י״ל אם הי׳ יודע כל א׳ דגם חבירו יפסול בעבירה אפילו לא הסכים לעבירה דידי׳ הי׳ אומר כיון דמהני ועוד בלא כל זה אין ראי׳ מדברי התוספת דאפילו היכי דמהני עדות של אחד נמי מכ״מ לא אמרינן מגו בשני עדים דכל א׳ לא רוצה שחבירו יכחישו וכן תירץ התומים (סי׳ מ״ו ס״ק מ״ו) על מה שהקשה באנוסים היינו להמנו במגו דתנאי הי׳ ולא נתקיים התנאי דלא שייך בזה דהוי מגו דבי תרי דהא אחד נמי נאמן ולא צריך להסכמת השני ותירץ ממה שהעלה בכלל מגו דנוח לו לומר דבר שהשני מסכים עמו ממה שיאמר דבר במה שהשני מכחישו ע״ש ולכן אפילו אם כל א׳ הי׳ נאמן בעדותו של עבירה מכ״מ לא רצה שחבירו יכחישו ולכן ניחא להו להזים בהסכמה א׳ ועל כן אין הוכחה מדברי התוספ׳ נגד דין הרמ״א הנ״ל.\\\vspace{0pt}

אמנם נלענ״ד שהוכחת הרבינו ירוחם (שהוא מקור דין הזה) מגמרא דסנהדרין היא הוכחה גמורה שז״ל הרבינו ירוחם ראובן שהעידו עליו עד א׳ אומר בפני גנב דבר פלוני ועד אומר לי גנב דבר אחר ראובן פסול על פיהם וכן הדין אם עד אחד מעיד על עבירה א׳ והעד השני על עבירה אחרת וכן מוכח בסנהדרין עכ״ל ומזה מבואר דמה שמסיים וכן מוכח בסנהדרין קאי על מה שכתב בתחלה אחד אומר בפני גנב דבר פלוני וכו׳ שהוא עובדא דסנהדרין ומה שכתב וכן הדין אם עד אחד מעיד על עבירה א׳ וכו׳ זה הוציא על פי סברא משם ומה שהמהר״ם פאדווא לא פירש כן היינו שהוא העתיק דברי רי״ו שאחר תיבות ראובן פסול על פיהם כתוב זה הדין פשוט בגמרא דסנהדרין ובספר הרא״ש פ׳ שני וכו׳ ולפי גרסא זו באמת ע״כ מה שכתוב בסוף וכן מוכח בסנהדרין לא קאי רק על מה שהוסיף דכן הדין גם בעבירות חלוקות אכן ברי״ו שלפנינו זה אינו ונראה שנוסחא מוטעת נזדמן למהר״ם שהי׳ כתוב זה בתוספ׳ גליון וע״פ גרסא שלנו א״ש שהוציא כן מסברא ובאמת סברא ישרה היא כיון דשם אף שהעידו לפסול אותו בגניבה מכ״מ על כל מעשה גניבה לא הי׳ רק עד א׳ ואעפ״כ נאמנים לפסלו כיון שהסכימו לעשותו רשע א״כ מה לי אם העידו בעדות חלוקות דגניבה או בעדות חלוקות דשאר עבירות דבשניהם לא דנין רק על מה שיוצא מעדות שניהם דפסול הוא ועל זה נאמנים. כנלענ״ד, הקטן יעקב.\\\vspace{0pt}

\end{multicols}\newpage

\newchap{סימן קעט}
\begin{multicols}{2}
ב״ה אלטאנא, יום ד׳ ט׳ תמוז תרי״ד לפ״ק. להרה״ג וכו׳ מ״ה קאפל לעווענשטיין נ״י אב״ד דק״ק בישאפסהיים יע״א.\\\vspace{0pt}

כתב אלי מעכ״ת נ״י נתקשתי היום בעברי על פרק ט׳ דתרומות משנה ג׳ דלמה אסור להדש בתרומה לחסום פי הבהמה הא במעילה (דף י״ג) אמרינן דישו ולא דישו של הקדש ומאי שנא תרומה מהקדש ואדרבא איסור תרומה לזרים מפורש יותר ממעילה כדאמרינן שם (דף י״ח ע״ב) ובסנהדרין (דף פ״ד) וכן בחלה אמרינן בפסחים (דף מ״ו) לענין בל יראה שאינה קרוי׳ שלך וה״נ נימא בתרומה לענין לאו דחסימה ולענ״ד צ״ע.\\\vspace{0pt}

תשובה: על קושית מעכ״ת נ״י כתב לי חתני הרב מ״ה זלמן כהן נ״י אב״ד דק״ק מאאסטריכט יע״א וז״ל לפענ״ד ודאי אי אמרינן טובת הנאה ממון ובתרומה יש לישראל טובת הנאה שיכול ליתן לכל כהן שירצה (עיין רש״י בפסחים ד׳ מ״ו ד״ה טובת הנאה ובנדרים דף פ״ז) לק״מ דכיון שיש לישראל טובת הנאה וט״ה ממון הנה מאכיל הישראל את פרתו מדידי׳ ומש״ה עובר על לאו דלא תחסום דהא לר׳ אליעזר דס״ל טובת הנאה ממון עובר באמת בבל יראה על חלה ואף דמסיק בגמרא שם בפסחים כ״ע ס״ל טובת הנאה אינה ממון מ״מ יעו״ש דף מ״ח רש״י ד״ה אהדר לי׳ ותו הא בן בתירא ס״ל תטיל בצונן ע״כ טעמו משום טובת הנאה ממון עיי׳ בתוי״ט שם והנה רבי ס״ל טובת הנאה ממון שם בפסחים דף מ״ח דפסק כר״א (ועיי׳ נדרים פ״ח ע״א) וא״כ אוקי למתניתן דתרומות דאסר לחסום בדישת תרומה כרבי מה שא״כ בהקדש דמה טובת הנאה שייך גבי׳ אבל אפילו אמרינן ט״ה אינה ממון נמי לא קשיא דדלמא אתיא מתניתן דתרומות כמ״ד דאית לי׳ הואיל וכיון דבתרומה אי בעי לאתשולי׳ עלי׳ הוי דידי׳ (עיי׳ נדרים נ״ט) לפיכך עובר על לאו דלאו תחסום אבל גבי הקדש אחר שבא ליד גזבר לאו דידי׳ מקרי עיין תוספ׳ פסחים דף מ״ח ד״ה הואיל עכ״ל והנה צדק חתני נ״י במה שכתב דתרומה אקרי יותר דידי׳ מהקדש אי משום טובת הנאה אי משום הואיל עיי׳ קדושין (דף נ״ח) וכל שכן אם היא תרומה שבאה כבר ליד כהן דמקרי דישו אכן ליישוב הקושיא של מעכ״ת נ״י א״צ לזה דמעיקרא לק״מ דאמרינן בב״מ (דף צ׳) והדשות בתרומה ומעשר אינו עובר משום בל תחסום אבל משום מראית העין מביא בול מאותו מין ותולה לה בטרסקלין שבפיה ע״ש וכן פסק הטור ואחריו הרמ״א בח״מ (סי׳ של״ח) הרי שבאמת בתרומה לא עובר בבל תחסום מדאורייתא והטעם לפי רש״י שם דסתם דייש אינו בתרומה שהוא קודם מירוח ולפי התוספ׳ שם דממעטינן מדישו ותרומה אינו ראוי ליתן לבהמתו ע״ש ומה שלא פירש רש״י כהתוס׳ מטעם דישו ולא של תרומה י״ל משום דתרומה מקרי דישו שהיא שלו מטעם ט״ה או מטעם הואיל ועוד דבכלל הדשות בתרומה הוא ג״כ תרומה שבאה כבר ליד כהן דודאי מקרי שלו ורק מטעם שאינו יכול לתת אותה לבהמתו כתבו התוס׳ דלא מקרי שלו ועכ״פ מבואר דאין חילוק באמת בין תרומה להקדש לענין ל״ת דבל תחסום ומה דאמרינן במתניתן דתרומות (פ׳ ט׳) תולה כפיפות בצואר הבהמה זה אינו רק מדרבנן מפני מראית העין ומה שלא גזרו רבנן כן מפני מראית העין גם בדש תבואת הקדש י״ל דלא רצו לגזור מפני פסידא דהקדש ועוד דהקדש אית לי׳ קלא כדאמרינן בעלמא ולכן לא שייך גבי הקדש מראית העין דבאית לי׳ קלא לא חיישינן משום מראית העין כדאמרינן פסחים (דף ו׳) בהמה אית לי׳ קלא ע״ש וגם אמרינן (שם) הקדש בדילי אינשי מני׳ ע״ש הרי דסמכינן דכל א׳ ידע דהקדש הוא ולא חיישינן דלמא אתי למיכל מני׳ וא״כ גם בדש הקדש ליכא חשש מראית העין משא״כ גבי דש בתרומה. כנלענ״ד, הקטן יעקב.\\\vspace{0pt}

\end{multicols}\newpage

\newchap{סימן קפ}
\begin{multicols}{2}
ב״ה אלטאנא, יום ה׳ ח׳ אייר תרט״ו לפ״ק. להרה״ג וכו׳ מ״ה גבריאל אדלער הכהן נ״י הגאב״ד דק״ק אבערדארף יע״א.\\\vspace{0pt}

בדיק לן מר נ״י וז״ל נזיר דף מ״ב ע״א ראיתי סוגי׳ נשגב׳ ממני לא אוכל לה דתמן אמרינן אמר מר וכולם שגלחו שלא בתער או ששיירו שתי שערות לא עשו ולכ״ל אמר רב אחא ברי׳ דר״א זאת אומרת רובו ככולו מדאורייתא ממאי מדגלי רחמנא ביום השביעי יגלחנו הכא עד דאיכא כולו הא בעלמא רובו ככולו וכו׳ הפירוש הנקרא פרש״י (אכן לע״ד וכאשר שמעתי איננו יוצא מפי מאור עיני הגולה רש״י נבג״ע זצ״ל) וא״ת וליגמר מני׳ משום דהוי נזיר ומצורע שני כתובים הכ״א ואין מלמדים עכ״ל ותמהתי מה זאת אומרת איכא הכא במה דכתב רחמנא בהדי׳ אחרי רבים להטות והכא הוי רובא דאיתא קמן כיון דאיכא רוב שערות לפנינו ואין לומר דשתי שערות הוי קבוע דא״כ אמאי הוי רובא מדאורייתא דהא קבוע הוי כמחצה ע״מ אלא ע״כ דלא הוי קבוע וקיימא כיון דנתערבו בהדי אחרינהו ואינן נכרין במקומן כ״א לאחר גלוח שאר שערות ולכאורה י״ל דמן אחרי רבים להטות לחודא לא מוכח דאזלינן ב״ר כ״א בהדי דעות דסנהדרין דבטלה דעתו וכמש״כ תוס׳ רפ״ג דב״ק לשמואל דאין הולכין בממון א״ה אע״ג דבדיינים הולכין א״ה דחשיב מיעוט כמי שאינו דבטלה דעתו ובזה מתורצת קושית פרי מגדים ריש הלכות תערובת של רש״י דכ׳ בטול ברוב באיסור שנתערב בהיתר מן אחרי רבים להטות ע״ש חולין דף צ״ח הא לא דמי לסנהדרין ע״ש שהניח בקושי׳ ולפ״ז ניחא דמוכח מנזיר דאפילו באיסור בעין בטל אמנם לרש״י דלא מחלק בזה כדמוכח מסוגי׳ דחולין ק׳ מאי זאת אומרת וצ״ל דר״א בדר״א חדית לן דבכל מצוות עשה הכתובה בתורה כיון דנעשה רובא הוי ככולא כגון שחיטה דקמל״ן רובא ככולו ותוס׳ כ׳ ריש פ״ב דחולין דהוי הלל״מ מן כאשר צויתיך ורב אחא חדית לן דהוי מדאורייתא וכגון רוב ציצין המעכבין את המילה כיון דרובא מל ופרעו הוי ככולו ומן אחרי רבים להטות לא מוכח כ״א בסנהדרין או בביטול איסור ברובא דאיתא קמן ורב ד״א קמל״ן דאפילו עשה קיים ברובו וצ״ע עכ״ד.\\\vspace{0pt}

תשובה הקשה מר נ״י על מה דאמרינן נזיר (דף מ״ב) זאת אומרת רובו ככולו מה צריך ראי׳ לזה הרי בכ״מ אזלינן בתר רובא ולתרץ זה רצה לפרש דקמל״ן דבכל מצות עשה הכתובה בתורה כיון דנעשה רובא הוי ככולו וסיים מר נ״י וצ״ע ויפה עשה להניח בצ״ע דלענ״ד אי אפשר לומר כן דכי ס״ד היכי דצותה התורה שיאכל מצה וסתם אכילה בכזית שאם אכל רוב כזית יצא משום דרובו ככולו או שצותה התורה לפדות בכור בחמש סלעים שאם פדאו בשלשה ג״כ יצא משום דרובו ככולו אלא ודאי מה דאזלינן בתר רוב לא שייך אלא היכי דהמיעוט מתנגד להרוב כגון גבי סנהדרין שדעות המיעוט הם נגד הרוב וע״כ אתה צריך לתפוש דעת החלק האחד בזה אמרינן שדעות הרוב מכריעין דעות המיעוט וכן לענין ביטול המיעוט ברוב אמרינן דלא אזלינן לדון התערובות כהמיעוט לאסור אלא כהרוב להתיר וכן לענין מה דאזלינן בתר רובא בכל הני דמייתינן בחולין (דף י״א) שאם אנו מסופקין אם הוא מהמיעוט טרפות או מהרוב שאינן טרפות אזלינן בתר רובא וכן בסנהדרין (דף ס״ט) לענין מיעוט איילונית וביבמות (דף קט״ז) לענין רוב יולדות וכדומה דבכל הני יש מיעוט ורוב מתנגדין זה לזה ואנו דנין לילך בתר הרוב נגד המיעוט אבל היכי שצותה התורה שיעשה דבר ודאי בעינן כולו ולכן אמרינן בפ״ב דחולין (דף כ״ח) דרובו של כל א׳ כמוהו לענין שחיטה הוי הלכה למשה מסיני דמסברא לא הוי אמרינן כן וכן אמרינן סוכה (דף ו׳) דבר תורה רובו ומקפיד עליו חוצץ מיעוטו אינו חוצץ דזה ג״כ הל״מ כנראה שם דאי לאו הכי היינו אומרים דאפילו המיעוט חוצץ אחר שצוה הכתוב שיבא בשרו במים וכן לענין ציצין המעכבין את המילה שהרי מה שצריך לגלות העטרה לא נכתב בכתוב והוא הל״מ וא״כ הכי נמסר שסגי ברובו אבל אם הי׳ כתוב בתור׳ באמת בעינן כולו ולא סגי ברובו.\\\vspace{0pt}

הן אמת שלכאורה קשה על זה ממה דאמרינן בשבועות (דף י״ז) הנכנס לבית המנוגע דרך אחוריו ואפילו כולו חוץ מחוטמו טהור וכתבו התוספ׳ שאם נכנס דרך פניו משנכנס רובו טמא אלמא דרובו ככולו וכן אמרינן בנגעים (פ׳ י״ג) דטהור שהכניס ראשו ורובו לבית טמא נטמא ע״ש אלמא דמשנכנס רובו נקרא בא אל הבית וזה ע״כ משום דרובו ככולו אבל י״ל דשם גלי קרא כן שהתוספ׳ בשבועות (שם) הקשו ותימא דדרך אחוריו נמי כשנכנס רובו יהא טמא מטעם דרובו ככולו וכו׳ וי״ל דא״כ לא הי׳ חילוק בין דרך ביאה לדרך אחוריו ואמאי כתיב והבא אל הבית דמשמע דרך ביאה עכ״ל הרי דעל ידי דגלי קרא והבא אל הבית ידענו שיש חילוק בין דרך פניו לדרך אחוריו ובבא כולו ע״כ גם דרך אחוריו טמא דלא גרע מכלים שבבית וע״כ מה דמיעט קרא הוא בבא רובו לחוד וממילא מוכח דדרך פניו גם ביאת רובו מקרי ביאה ולכן טהור שבא לבית טמא נטמא בשבא רובו לחוד אחר דגלי קרא דזה דרך ביאה הוא ויש לומר דמזה ילפינן ג״כ מה דאמרינן סוכה (דף ג׳) דיושב ראשו ורובו בסוכה יצא מדאורייתא דמדגלי קרא דבא ראשו ורובו מקרי ביאה אל הבית א״כ נקרא ג״כ ישיבה בסוכה.\\\vspace{0pt}

כללו של דבר שבכל מקום שצותה התורה מצו׳ או הזהירה על מעשה שלא לעשות אין הציווי רק על כולו ולא גם על רובו אם לא במקום שגלתה התורה בפירוש כן או ע״י הל״מ שגם רובו ככולו ומה דאמרינן בנזיר זאת אומרת רובו ככולו דאורייתא היינו לענין גלוח שערות דוקא שכן כתב השיטה מקובצת שם בשם הר׳ עזריאל וז״ל זאת אומרת רובו ככולו דאורייתא פי׳ בפאת זקן דאמר רחמנא לא תשחית אפילו לא גלח אלא רוב הפיאות חייב כאלו גלח כולן אמאי מדגלי רחמנא ביום השביעי יגלחנו גבי נזיר דלא הוי צריך דכתיב כבר יגלח אלא לאשמעינן דבעי תגלחת אחרונה גלוח כל ראשו שלא ישייר שתי שערות דבשער א׳ לא מקרי שיור ומדאצטריך הכא לרבויי כל ראשו ולא סגי ברוב ש״מ דדוקא הכא בנזיר בעינן שלא ישייר כלום וכן במצורע דכתיב כל בשרו ובלויים נמי על כל בשרם אבל בעלמא כגון בפאת זקן רובו ככולו עכ״ל הרי דס״ל דרק לענין גלוח גלי קרא דרובו ככולו אבל בעלמא לא וכן מוכח מדאצטריך למילף מנזיר וממצורע ומלוים דבפאת זקן רובו ככולו מכלל דמסברא בלא ילפותא לא הוי אמרינן כן ואין לומר דא״כ דמסברא אמרינן בעינן כולו דוקא ולא רובו ל״ל קרא גבי נזיר ומצורע ולוים דבעינן כולו די״ל דאצטריך למעט גלוח בעלמא דלא בעינן כולו דאי לא הוי כתיבי הני רבויי הוי אמרינן דגם בפאת זקן בעינן כולו ולא חייב ארובו.\\\vspace{0pt}

והנה התוספ׳ בנזיר שם כתבו אהא דאמרינן הא בעלמא רובו ככולו אמר הר״מ דבעלמא קיימא מיהו פאת זקן אמר ס״פ המצניע (דף צ״ד) דמלא פי הזוג מחייב דמשמע פורתא עכ״ל הרי דהר״מ לא ס״ל כפי׳ ה״ר עזריאל דעל פאת זקן קאי אכן דברי הר״מ שהביא ראי׳ מפ׳ המצניע צ״ע וכבר העיר באורח מישור כן דשם לא הוזכר כלל מפאת הזקן רק איירי לענין שבת וקרחה ע״ש וא״כ פי׳ ה״ר עזריאל א״ש ומה שלא הזכירו מפאת הראש י״ל דשם יש שיעור כמש״כ הרמב״ם בשם זקניו דלא יניח פחות מד׳ או לחד גרסא ממ׳ שערות כמבואר בטור י״ד (סי׳ קפ״א) וא״כ י״ל דזה שיעור הפיאה ע״פ קבלה שקבלו כן ומכ״מ צל״ע דלענין פאת הזקן לא הוזכר בפוסקים שיעור כלל ומדבריהם נראה דאם מניח חוט של שיער בלבד לא נקרא משחית פאת זקן שכן כתב (שם) הטור בשם הרא״ש וירא שמים יצא ידי כולם ולא יעביר תער על כל זקנו כלל ולא כאותן שמניחין חוט כל שהוא על הפיאות כי לפעמים אין מכוונין על הפיאות עכ״ל וכן כתב הב״י בשם הסמ״ק וכ״כ ג״כ רבינו יונה באגרת תשובה וזה תימה דהרי מסוגיא זו דנזיר מוכח דאפילו במניח מיעוט הפיאה שודאי הוא יותר מחוט כל שהוא ג״כ נקרא משחית פאת זאן וחייב דרובו ככולו וצ״ע אכן בלא״ה כבר כתב הרא״ש כנ״ל ירא שמים לא יעביר תער על זקנו גם במניח חוט ויפלס דרכו במאזני צדק ומשפט. תם חלק חושן המשפט בעזרת מלך אוהב צדקה ומשפט. בו שם מבטחו, הקטן יעקב.\\\vspace{0pt}

למען תהי׳ החתימה מעין הפתיחה בעבודת בית המקדש אציג פה עוד ב׳ תשובות על קושיות חמורות בענין עבודת קדשים עדי נזכה לראות בהכין בציון מזבח חדש.\\\vspace{0pt}

\end{multicols}\newpage

\newchap{סימן קפא}
\begin{multicols}{2}
ב״ה אלטאנא, יום ב׳ ט׳ סיון תרכ״ד לפ״ק. להרב וכו׳ מ״ה יעקב משה עטטינגער נ״י בן הגאון המופלג מ״ה מרדכי זאב זצ״ל בק״ק לעמבערג יע״א.\\\vspace{0pt}

שמחני מעכ״ת נ״י בחבור המופלג שהניח אחריו ברכה מר אביו זצ״ל אשר שלח לי ולמען הראות לו רק באחת כי שמתי עיני עליו אזכיר לו מה שראיתי בהחבור מדודו זקנו הגאון המופלג ז״ל קושיא אחת שכתב עלי׳ המחבר זצ״ל שהיא קושיא שאין עלי׳ תשובה וגם במפרשי הים נקראת קושיא חזקה ונוראה והיא על מה דאמרינן ב״ק (דף ק״י) אימא לבעלי מומין טהורין שבאותו משמר והקשה הגאון זצ״ל הרי ע״כ איירי שאין כאן כהן טהור כלל וא״כ הקרבן נעשה בטומאה והא קיי״ל דק״צ הבא בטומאה אינו נאכל וא״כ איך אמרינן דעבודתה דהיינו האכילה לבעלי מומין הרי לא נאכל כלל וסיים וצע״ג ולית נגר וב״נ דיפרקיני׳ ובמפרשי הים שם חתרו הגאונים אל היבשה בסברות דקות אשר אף שנכונות בעצמותן מכ״מ לפענ״ד רחוקות מפשטות הגמרא אמנם לענ״ד יש ליישב הקושיא בדרך פשוט רק בתחלה אזכיר שממש כקושיא זו כבר הקשו הר״י קורקוס והכס״מ על הרמב״ם ה׳ מעשה הקרבנות (פ׳ י׳ ה׳ כ״ג) שכתב שם ק״צ הבא בטומאה אע״פ שהטמאים מקריבין אותו אין חולקין עם הטהורים לאכול לערב מפני שאינם ראויים לאכילה והקשה שם הר״י קורקוס ג״כ קושיא זו כיון שהטמאים הקריבו אותו איך נאכל ותירץ בדוחק שכוונת הרמב״ם מה שכתב שהטמאין מקריבין אותו שהיו יכולים להקריבו אם אין כאן טהורים וזה ודאי רחוק מדעת הרמב״ם וגם מה שכתב הכס״מ שם דהרמב״ם איירי בקרבן פסח ובמחצה טהורים ומחצה טמאים לענ״ד מלבד שרחוק לפרש דעת הרמב״ם שמה שאמר קרבן צבור כוונתו לפסח ולא אמר בפי׳ קרבן פסח גם הלשון אין חולקין עם הטהורים לא משמע כן דלענין פסח מצאנו בכל מקום לשון נמנין על הפסח אבל לשון חולקין שהוא מהכתוב חלק כחלק יאכלו נאמר רק על חלוקת הכהנים בקרבנות שמביאין בעד הצבור והיחיד וגם התירוץ של מפרשי הים לא שייך על הרמב״ם.\\\vspace{0pt}

אמנם לענ״ד י״ל שמה שפשוט להגאונים ז״ל שקרבן צבור הבא בטומאה אינו נאכל הוא ממה ששנינו בפסחים (דף ע״ו) חמשה דברים באין בטומאה ואינן נאכלין בטומאה והנה שם לא נאמר שאינן נאכלין כלל רק שאינם נאכלים בטומאה ואדרבה מדקאמר התנא ואינן נאכלין בטומאה ולא סתם ואינן נאכלין משמע שרק בטומאה אינם נאכלים אבל בטהרה נאכלים רק שלכאורה אין למצוא אופן שיהיו הכהנים המקריבים טמאים ויהא נאכל בטהרה שלא בלבד שע״כ אין כאן כהנים טהורים אם יקרב בטומאה למ״ד טומאה דחוי׳ בצבור אלא גם מפני שהקרבן נטמא בהקרבת הטמאים דאין לומר דמשכחת בלא הוכשר ואמרינן בפסחים (דף כ׳) שמשקה בית מטבחי׳ לא מטמאין ולא מכשירין שזה אינו שהרי אעפ״כ הוכשר בחבת הקדש כדאמרינן שם דחבת הקדש מכשיר וא״כ נטמא הקרבן אפילו לא הוכשר במשקה לא בלבד לשיטת הראב״ד שס״ל דחבת הקדש מכשיר מדאורייתא אלא אפילו לשיטת הרמב״ם ה׳ ט״א דאינו מכשיר רק מדרבנן הרי עכ״פ נטמא מדרבנן ולא נאכל אכן באמת שפיר משכחת שיהי׳ הקרבן נקרב בטומאה ואעפ״כ נאכל בטהרה שהרי הנגיעה בקרבן ע״י העובד ל״צ רק בשחיטה והפשט ונתוח להוצאת האימורים הנקרבים והרי שחיטה וכן הפשט ונתוח כשרים בזרים ובכהנים בעלי מומין ולפ״ז שפיר משכחת שאין כאן כהן טהור בכל המשמרות ושחט הזר או כהן בעל מום שהם טהורים את החטאת או זבחי שלמי צבור וקבל הכהן הטמא את הדם במזרק בלי שנגע בקרבן ואפילו יש חבור בקלוח הדם בין צואר הבהמה ובין הדם שבמזרק עם כל זה לא נטמא החטאת בכך דאמרינן טהרות (פ׳ ז׳) הנצוק אינו חבור לטומאה וכ״פ גם הרמב״ם ה׳ טומאת אוכלין (פ׳ ז׳) ואח״כ עשה הזר הטהור הפשט ונתוח והוציא האימורין ונתנם לכהן הטמא שהקטירם ואפילו ליתן האימורים במיגס עד שיבא הכהן המקטיר להקטירם כשר בזר כדאמרינן בפסחים (דף ס״ה) ע״ש וא״כ עשה הכהן הטמא כל העבודות דהיינו קבלת הולכת וזריקת הדם והקטרת האימורים כדינו מבלי שנגע בחטאת ונשאר טהור ואפילו בזבחי שלמי צבור שהכהן עושה תנופה בכבשים עם שתי לחם לא נטמאו הכבשים כיון דחיים הם ובהמה בחיי׳ לא מקבלת טומאה כדאמרינן בחולין (דף ע״א) ולכן אם יש כאן כהנים בעלי מומין טהורים יכולים לאכול הקרבן בטהרה שנקרב בטומאה ובכי האי גוונא איירי הסוגיא דב״ק (דף ק״י) וגם מה שפסק הרמב״ם ה׳ מעשה הקרבנות (פ׳ י׳) ומדוייק בזה ג״כ מה שנאמר במשנה ואינן נאכלין בטומאה לומר דאם אפשר להאכל בטהרה כגון שיש כאן כהנים בעלי מומין טהורים נאכלין הקרבנות דהיינו זבחי שלמי צבור ושעירי חטאת שאפשר שלא נטמאו.\\\vspace{0pt}

והנה כל זה כתבתי לשיטת הרמב״ם דפסק כמ״ד טומאה דחוי׳ בצבור וס״ל כל שיש כהן טהור בעולם אפילו במשמר אחר לא נעשה ק״צ בטומאה אכן דעת רש״י אינה כן שכתב ד״ה אי איכא טהורין בהאי משמר עכ״ל ומדכתב בהאי משמר משמע דס״ל אפילו איכא טהורים במשמר אחר מכ״מ נעשה בטומאה והמהרש״א העיר על זה למה כ׳ רש״י באותו משמר ותירץ בדוחק ובמהדורא בתרא כבר השיב דמגמרא דיומא לא מוכח רק דאי איכא טהורים באותו משמר ובית אב אחר בזה צריך לאהדורי על טהורים אבל אם כל בתי אבות של אותו משמר טמאים לא צריך לאהדורי על משמר אחר ואע״ג דדעת הרמב״ם אינה כן מכ״מ נלענ״ד להוכיח דרש״י ס״ל הכי דבזבחים (דף צ״ט) על מה דבעי רב אושעיא טמא בקרבן צבור מהו פי׳ רש״י דבהקריבוהו טהורים קא מבעי׳ לי׳ אי פלגי מני׳ להאי טמא הואיל וראוי לחיטוי דק״צ דוחה טומאה אבל בהקריבוהו טמאים לא מבעי׳ לי׳ שהרי אינו נאכל דקיי״ל דק״צ באין בטומאה ואינן נאכלין עכ״ל וקשה כיון דהקריבוהו טהורים ע״כ שיש כאן כהנים טהורים והיאך נקרא הטמא ראוי לחיטוי כיון שק״צ לא נקרב בטומאה רק כשאין כאן כהנים טהורים אע״כ מוכח דס״ל לרש״י דדוקא אם באותו משמר יש טהורים אז לא נקרב בטומאה אבל למשמר אחר ל״צ להחזיר אחר כהנים טהורים והכא איירי שהמשמר הטמא הניח למשמר אחר טהור להקריב הק״צ ולכן מקרי הטמא ראוי לחיטוי אע״פ שהי׳ כאן כהנים טהורים ולכן כתב רש״י בב״ק ג״כ אי איכא טהורים באותו משמר ובמכ״ה נעלם זה מהמהרש״א והוכיח רש״י כן דאי בהקריבוהו טמאים הלא לא נאכל כמש״כ בפי׳ אבל הרמב״ם מפרש סוגיא זו בהקריבוהו טמאים דלשיטתו אזיל דס״ל דאי איכא טהורים בעולם אפילו במשמר אחר מכ״מ אסור לטמאים להקריב ולשיטתו צריך לפרש דמכ״מ משכחת דנאכל אם הקריבוהו טמאים באופן שלא טמאו הבשר וכנ״ל והיוצא מזה דלרש״י הסוגיא דב״ק בפשטות א״ש דמשכחת דחולקין בעלי מומין שבאותו משמר בשהקריבוהו הטהורים של משמר אחר והמשמר של אותו שבוע הי׳ טמא ואעפ״כ הקרבן שלהם לחלקו לבעלי מומין שבמשמר שלהם שראויין לאכלו כיון שהקרבן נקרב בטהרה. כנלענ״ד, הקטן יעקב.\\\vspace{0pt}

\end{multicols}\newpage

\newchap{סימן קפב}
\begin{multicols}{2}
ב״ה קארלסרוהע, בחדש אדר שנת תקפ״ב לפ״ק. לכבוד אדמ״ו הגאון המופלא וכו׳ מ״ה אברהם בינג נ״י הגאב״ד ור״מ דק״ק ווירצבורג יע״א. שמעתי קושיא עצומה מאדמ״ו נ״י אמה דאמרינן בסנהדרין (דף י״ב) א״ר יהודה מעשה בחזקי׳ המלך שעבר את השנה מפני הטומאה ובקש רחמים על עצמו דכתיב כי מרבית העם רבת מאפרים ומנשה יששכר וזבולון לא הטהרו כי אכלו את הפסח בלא ככתוב כי התפלל חזקיהו עליהם לאמר ד׳ הטוב יכפר בעד רבי שמעון אומר אם מפני הטומאה עברוה מעוברת אלא מפני מה בקש רחמים על עצמו שאין מעברין אלא אדר והוא עיבר ניסן בניסן ר״ש בן יהודה אומר משום ר״ש מפני שהשיאן את ישראל לעשות פסח שני ופי׳ רש״י לא עברה אלא השיאן לפסח שני שלא כדין לפיכך בקש רחמים על עצמו עכ״ל ומפרש בגמרא דבתחלה סבר נשים בראשון חובה ועמהן היו הטהורים רוב ישראל ודחה מיעוט הטמאים לפסח שני ולבסוף סבר נשים בראשון רשות דהוו להו טמאים רובא ורובא לא מדחו לפסח שני ולכן בקש רחמים על שדחה ישראל לפסח שני שלא כדין ועל זה הקשה אדמ״ו נ״י הרי בדברי הימים ב׳ פסוק י״ג כתוב ויאספו ירושלם עם רב לעשות את חג המצות בחדש השני וכן כתיב פסוק כ״א ויעשו בני ישראל הנמצאים בירושלם את חג המצות שבעת ימים בשמחה גדולה וגו׳ הרי מוכח שעיבר חזקי׳ וכדעת ר״י ור״ש והיאך יפרש ר״ש בן יהודה את הפסוקים. והנה מצאתי בסמ״ג (עשין מ״ז) שכתב וז״ל ועיבור השנה למדנו מחזקי׳ מלך יהודה שכך כתוב ויועץ המלך ושריו וכל הקהל לעשות הפסח בחדש השני ואין לומר שפסח שני הי׳ שהרי פסח שני לא הוקבע לצבור לעשותו כי אם ליחידים כי לעולם הצבור עושין הראשון וכו׳ ועוד כי פסח שני אינו נוהג אלא יום א׳ שחיטתו בי״ד ואכילתו בליל ט״ו על מצות ומרורים לפי שפ״ש הוא תשלום לזביחת הפסח ואין תשלום לחג המצות והפסח שעשה חזקיהו בחדש השני כל הקהל עשו אותו וחגגוהו שבעה ימים בשמחה גדולה מצינו למדין שעיבר חזקיהו השנה מפני הטעם שכתוב שם לא יכלו לעשותו בעת ההיא כי הכהנים לא התקדשו למדי והעם לא נאספו לירושלם עכ״ל הסמ״ג הרי מבואר כדברי אדמ״ו נ״י דמוכרח מהפסוקים שעיבר חזקי׳ השנה ולא עשה פסח שני וא״כ ביותר יש לתמו׳ איך יתיושב זה לדעת ר״ש בן יהודה.\\\vspace{0pt}

ונלע״ד ליישב דודאי מדברי רשב״י מוכח דס״ל שלא עיבר וכפי׳ רש״י כיון דמפרש קרא דלעשות הפסח בחדש השני שהוא פסח שני ויש לומר דהוכיח כן מן הפסוקים דהנה יש להקשות למה התפלל חזקי׳ דוקא על אשר אכלו הפסח בלא ככתוב דהא אם עיבר שלא כדין נעשו עבירות רבות אכילת חמץ דבכרת ומלאכה בי״ט ובטול מצות מצה ומרור והשבתת חמץ וכדומה גם י״ל דבהועץ המלך ושריו ובשלוח שלוחיו אל ישראל (בפסוק ב׳ ופסוק ה׳) כתיב לעשות הפסח בחדש השני ולא הוזכר חג המצות ובאסיפת העם לירושלים (בפסוק י״ג) כתיב ויאספו ירושלם עם רב לעשות את חג המצות בחדש השני גם (פסוק כ״ב) וידבר חזקיהו על לב כל הלוים המשכילים שכל טוב לד׳ ויאכלו את המועד שבעת הימים מזבחים זבחי שלמים ומתודים לד׳ אלדי אבותיהם הוא סתום וחתום שלא ידענו מה דבר על לב כל הלוים והמפרש דחק מאוד בפירושו ובפרט מה שפירש ומתודים לד׳ שהביאו קרבן תודה תמו׳ שהרי אין מביאין תודה בחג המצות ואי הפי׳ וידוי כפשוטו וידוי בי״ט מה זו עושה וגם לשון ויאכלו את המועד נפלא ויחוגו את המועד הל״ל ולכן נ״ל דהענין הי׳ כך לדעת רשב״י משום ר״ש דסמוך לי״ד בניסן נתיעץ חזקי׳ לצרף הנשים להמנין באופן שהטמאים ואותם שהם בדרך רחוקה היו המיעוט ונדחו לפ״ש וזה הפי׳ (בפסוק ג׳) כי הכהנים לא התקדשו למדי פי׳ שהיו רק מעט כהנים טהורים ולכן הוצרכו אפילו מן ישראלים טהורים לעשות בטומאה וגם העם לא נאספו והיו בדרך רחוקה והם יעשו פ״ש וא״כ עשו חזקי׳ ורוב טהורים פסח ראשון וכל ישראל עשו חג המצות כדינו כל א׳ בעירו וכעבור החג שלח חזקיהו השלוחים לכל ישראל לעשות פסח סתם ככתוב (בפסוק ה׳) ולא הוזכר שם לא בחדש הראשון ולא בחדש השני ולכן במקומות אשר הגיע דבר המלך חשבו כי חזקי׳ עיבר את השנה והם עשו בטעות את חג המצות באדר שני כי אם לא עיבר הרי היו רובם טמאים והי׳ לו לקרוא להם לפסח ראשון כי הם לא ידעו דמצרפים נשים לטהורים כמו שחזקי׳ באמת חזר בו ג״כ ולכן כתוב שבאו לירושלים לעשות את חג המצות כי הם חשבו כי ניסן הוא ומה שכתוב לעשות את חג המצות בחדש השני לא שהם ידעו כן שחדש השני הוא אלא הכתוב קראו כן בחדש שהוא חדש השני באמת (והרי גם להתנאים שס״ל שעיבר חזקי׳ צריך לפרש כל החדש השני שבפרשה זו כן) ואח״כ חזר חזקי׳ שאין מצרפים הנשים והי׳ להם לעשות פסח ראשון בטומאה והנה בדין הפסח לא שינה זה דבר דמכ״מ היו צריכים לעשות פסח דלא גרע משגג ונאנס ולא עשה הראשון יעשה השני אכן ירא חזקי׳ פן יערבב לישראל את שמחתם בשמעם שלא עושין פסח כדינו ושעברו אפילו רק בשוגג על איסור כרת שלא לעשות פסח בזמנו ולכן נתיעץ להחריש עד אחר כלות שמחתם ולהניחם על הטעות שפסח ראשון הם עושים כי שמחתם בשמחה של מצוה היתה גדולה עד מאוד כאשר העיד הפסוק ולכן כתוב שעשו בני ישראל את חג המצות ז׳ ימים שחגגו כדינו כי הם חשבו שהראשון שעשו בחדש הקודם הי׳ בטעות וזה עיקר וחזקי׳ הניחם לעשות כן כי אין בזה איסור אכן הכהנים והלוים הם ודאי ידעו בדבר כי להם הוצרך חזקי׳ לגלות בעבור שלא יקריבו מוספי י״ט ושלא יקריבו לעם עולות ראי׳ ושלמי חגיגה לשם חובה רק לשם נדבה וזה הפירוש (בפסוק כ״ב) שדבר חזקי׳ על לב הלוים המשכילים פי׳ בסוד דבר אליהם שגם הם יחרישו ויאכלו את המועד פי׳ שחגגו גם הם את המועד אבל רק באכילה ובזביחת זבחי שלמים וזה דוקא לעיני העם אבל בינם לבין בוראם מתודים על חטאם כי הם ידעו שחול גמור הוא.\\\vspace{0pt}

ואגב אזכיר מה שק״ל אגמרא דערכין (דף י״ח ע״ב) דקאמר כבש הבא עם העומר קמבעי׳ לי וכו׳ והרי לת״ק ור׳ יהודה דפסקינן כוותייהו אותו החדש שקראו הפסוק החדש הראשון אותו עשה חזקי׳ לאדר שני וא״כ אין כאן לא עומר ולא פסח ולא מצה ודוחק לומר דהסוגיא דשם אליבי׳ דר״ש אזלא דס״ל שלא עיבר חזקי׳ שהרי אליבי׳ דהלכתא רצה לפשוט משם ולא פסקינן כר״ש ועוד דא״כ הי׳ להש״ס למימר היניחא לר״ש אלא לת״ק ור״י מא״ל ועלה ברעיוני לומר דאף שחזקי׳ עיבר מפני הטומאה בל׳ באדר מכ״מ גם אצלו מידי ספק לא נפיק כדכתיב ויועץ וכו׳ ועיבר רק מספק שמעברין מספק כדתנן בעדיות שמעברין על תנאי ועשה גם באותו חדש שקבע לאדר שני י״ט עם כל דיניו מספק אכן ביום ט״ו שעדיין לא נגמר טהרת הבית לא יכול להקריב מוספים אף שקרבן צבור בא בטומאה דשמא כדין עיבר וא״כ אין זה י״ט ולא בעי מוספים ולא שייך שיקריב מספק לנדבת צבור שהרי אין נדבת צבור בא בטומאה אבל ביום ט״ז שכילה טהרת הבית וקרב בטהרה אז קמבעי׳ לי׳ אכבש הבא עם העומר אם לקרבו על הספק שאם היום ט״ז יהי׳ לחובה ואם לאו לנדבה ואף שבמוספים יש גם חטאת שאין קרב בנדבה הרי מוספים אין מעכבים זא״ז ולכן קמבעי׳ דוקא אכבש ולא אעומר גופא דהוא אין בא בתנאי דאין מנחה נדבה לצבור ועוד דמנחה של שעורים אפילו ליחיד לא נקרב והנה פסח ודאי לא הקריב שהרי ג״כ אין בא על תנאי דבשאר ימות השנה לשמו פסול וגם הבית עדיין טמא ולכן ממנ״פ הי׳ עושה פסח בחדש השני אם כדין עיבר מטעם פ״ר ואי לאו מטעם פסח שני אכן אחר כלות המועד נתיעץ המלך שיהי׳ העבור עבור ושיהי׳ ניסן בחדש השני כמו שהי׳ דעתו בראשונה ולכן עשו חג המצות בחדש השני ונתחרט לבסוף על זה ואמר ד׳ הטוב יכפר בעד ומה דקאמר פסח היכי עביד אין הפי׳ קרבן פסח דהרי באמת לא עבדו ועוד דא״כ למה הוסיף עוד להקשות מצה היכי אכיל דלחשוב גם מרור ושביתת י״ט וכדומה אי כרוכלא ליחשב וליזל אבל נ״ל דכוונתו אי״ט של פסח דבלשון המשנה והגמרא נקרא הי״ט של פסח בשם פסח סתם כמו ואלו עוברין בפסח פרה הי׳ לנו זבחי שלמים ואכלנוהו בפסח ורבים כדומה ובזה יתורץ גם לת״ק ור״י מה שהקשינו לעיל למה התפלל חזקי׳ דוקא על אשר אכלו הפסח בלא ככתוב ולא על בטול שאר מצות.\\\vspace{0pt}

גם מתורץ בזה מה שהקשו התוספ׳ שם ד״ה אלא וז״ל טובא תימא מאי ס״ד דהא אתמול הקריבו תמידין ומוספין אמאי לא אמלך אלא ממוסף ט״ז בניסן עכ״ל ולפי דברינו א״ש דהא אתמול שהי׳ ספק י״ט ועדיין הבית לא נטהר לא הי׳ באפשר להקריב מוספים מספק אבל ביום ט״ז שהי׳ הבית טהור הי׳ יכול להקריב ספק חובה ספק נדבה. גם מתורץ בזה מה שהקשו התוספ׳ ד״ה איבעיא להו אי אקבע ר״ח וקשה לרש״י דבדברי הימים כתיב האי קרא מקמי׳ ויועץ המלך לעשות הפסח בחדש השני עכ״ל ולפי דברינו אתי שפיר דודאי הי׳ זה קודם ויועץ המלך לעשות הפסח בחדש השני שזה הי׳ בחדש הסמוך לאדר ואחר כלות המועד נתייעץ המלך שיהי׳ העבור עבור ושעשה י״ט בטעות באדר שני ועשה פסח עם כל חקותיו בחדש השני ואחר כלות המועד נתחרט על זה והתפלל ד׳ הטוב יכפר בעד. כנלענ״ד, ואם שגיתי כמוהו אשוע ד׳ הטוב יכפר בעד, הקטן יעקב.\\\vspace{0pt}

סליק תשובת בנין ציון על ארבע טורים. מוסדת על פסקי המורים. בעזרת יוצר הרים.\\\vspace{0pt}

\end{multicols}
\newpage

\addtocontents{toc}{\protect\end{multicols}}
\end{document}
