\documentclass[12pt, openany]{book}
\usepackage[
paperheight=9in,
paperwidth=6in,
top=0.5in,
bottom=0.5in,
inner=0.7in,
outer=0.5in,
marginparsep=0.1in,
headsep=16pt
]{geometry}

\newcommand{\texttitle}{מגילה}\usepackage{titlesec}
\newcommand{\partname}[1]{}
\usepackage{resources/unnumberedtotoc}

\usepackage{fancyhdr}
\pagestyle{fancy}
\fancyhf{}
\fancyhead[LO,RE]{\thepage}
\fancyhead[CO]{}
\fancyhead[CE]{\partname\chapname \space\textendash\space \sectname}

\usepackage{paracol}
\usepackage{anyfontsize}
\usepackage{ragged2e}
\usepackage{polyglossia}
\usepackage{multicol}
\usepackage{hyperref}
\usepackage[marginal]{footmisc}
\usepackage[titles]{tocloft}
\usepackage{xifthen}
\usepackage{graphicx}
\usepackage{dblfnote}\DFNalwaysdouble

\setdefaultlanguage{hebrew}
\setotherlanguage{english}
\usepackage{fontspec}
\setmainfont{Times New Roman}
\newfontfamily\englishfont{Times New Roman}
\setsansfont{Aharoni}

\newcommand{\sethebfont}{
\fontsize{10.5pt}{13.1pt} \selectfont
}

\newcommand{\hebeng}[2]{
	{\sethebfont #1\\}
	
	\begin{english}
		#2
	\end{english}
	\clearpage
}

\newcommand{\twocol}[1]{
	{\sethebfont \begin{multicols}{2}
			#1
	\end{multicols}}	
}

\newcommand{\textblock}[1]{
{\sethebfont #1\\}	
}

\setlength{\parskip}{6pt}
\setlength\parindent{0in}

\newcommand{\chapname}{}
\newcommand{\sectname}{}

\newcommand{\newchap}[1]{
	\addcontentsline{toc}{chapter}{#1}
	\renewcommand{\chapname}{#1}
		\begin{center}
			\textbf{%
\fontsize{16pt}{16pt}\selectfont
				#1}
		\end{center}
}

\let\footnoterule\relax

\setlength{\columnsep}{0.25in}

\newcommand{\newsection}[1]{
	%\addcontentsline{toc}{section}{#1}
	\renewcommand{\sectname}{#1}	
	\vspace{-\baselineskip}
	\begin{center}
		\textbf{%
\fontsize{16pt}{16pt}\selectfont
			#1}
	\end{center}
	\vspace{-\baselineskip}
	\nopagebreak
}

\newcommand{\footnotecomment}[1]{
	\renewcommand\thefootnote{}
	\footnote{\textsf{#1}}}

\newcommand{\parencomment}[1]{\footnotesize (#1)}

\newcommand{\blockcomment}[2]{ 
\newsection{#1}
\sethebfont	\textsf{#2}}

\newcommand{\commenta}[1]{\footnotecomment{#1}\hspace{0em}}

\newcommand{\vsnum}[1]{(\hebrewnumeral{#1})\space}
\newcommand{\vsnumeng}[1]{(#1)\space}

\begin{document}
\frontmatter
\pagenumbering{roman}

\newcommand{\oneline}[1]{%
	\newdimen{\namewidth}%
	\setlength{\namewidth}{\widthof{#1}}%
	\ifthenelse{\lengthtest{\namewidth < \textwidth}}%
	{#1}% do nothing if shorter than text width
	{\resizebox{\textwidth}{!}{#1}}% scale down
}

\title{\oneline{\hspace*{0.5in}\texttitle\hspace*{0.5in}}}

\author{}

\date{}

\maketitle

\begin{minipage}[b][\textheight][b]{\textwidth}\englishfont\footnotesize
	\begin{english}
		\vfill
		The following book includes:
\begin{itemize}
\item[$\bullet$] Tanach with Ta'amei Hamikra
\begin{itemize}
\item[$\bullet$] License: Public Domain
\item[$\bullet$] Source: \url{http://www.tanach.us/Tanach.xml}
\end{itemize}
\item[$\bullet$] The Metsudah Five Megillot, Lakewood, N.J., 2001
\begin{itemize}
\item[$\bullet$] License: CC-BY
\item[$\bullet$] Source: \url{http://primo.nli.org.il/primo_library/libweb/action/dlDisplay.do?vid=NLI&docId=NNL_ALEPH002162036}
\end{itemize}
\item[$\bullet$] Mishnah, ed. Romm, Vilna 1913
\begin{itemize}
\item[$\bullet$] License: Public Domain
\item[$\bullet$] Source: \url{http://primo.nli.org.il/primo_library/libweb/action/dlDisplay.do?vid=NLI&docId=NNL_ALEPH001741739}
\end{itemize}
\item[$\bullet$] On Your Way
\begin{itemize}
\item[$\bullet$] License: Public Domain
\item[$\bullet$] Source: \url{http://mobile.tora.ws/}
\end{itemize}
\item[$\bullet$] Mishnah Yomit by Dr. Joshua Kulp
\begin{itemize}
\item[$\bullet$] License: CC-BY
\item[$\bullet$] Source: \url{http://learn.conservativeyeshiva.org/mishnah/}
\end{itemize}
\item[$\bullet$] Senlake edition 2019 based on Ben Yehoyada, Jerusalem, 1897
\begin{itemize}
\item[$\bullet$] License: CC0
\item[$\bullet$] Source: \url{http://beta.nli.org.il/he/books/NNL_ALEPH001933802/NLIl}
\end{itemize}
\end{itemize}
		It was retrieved from Sefaria on \today\space \texthebrew{(\Hebrewtoday)}.  It was typeset and formatted by Ktavi.
		\clearpage
		
	\end{english}
\end{minipage}

\titleformat{\chapter}[hang]{\huge\bfseries}{\thechapter.}{1em}{}
\titlespacing*{\chapter}{0pt}{-3em}{1.1\parskip}
\titlelabel{\thetitle\quad}
%\addtocontents{toc}{\protect\vspace{-\baselineskip}}
\addtocontents{toc}{\protect\begin{multicols}{2}}
%\vspace*{-5\baselineskip}
{\small \tableofcontents}


\clearpage
\mainmatter
\pagenumbering{arabic}

\addpart{אסתר}\renewcommand{\partname}[1]{אסתר}
\fancyhead[CO]{\chapname}
\fancyhead[CE]{\partname}
\twocol{\clearpage}

\newchap{פרק א}
\twocol{\vsnum{1}וַיְהִ֖י בִּימֵ֣י אֲחַשְׁוֵר֑וֹשׁ ה֣וּא אֲחַשְׁוֵר֗וֹשׁ הַמֹּלֵךְ֙ מֵהֹ֣דּוּ וְעַד־כּ֔וּשׁ שֶׁ֛בַע וְעֶשְׂרִ֥ים וּמֵאָ֖ה מְדִינָֽה׃%
\commenta{\normalfont{וַיְהִי בִּימֵי אֲחַשְׁוֵרוֹשׁ.} מֶלֶךְ פָּרַס הָיָה, שֶׁמָּלַךְ תַּחַת כּוֹרֶשׁ לְסוֹף שִׁבְעִים שָׁנָה שֶׁל גָּלוּת בָּבֶל: }%endcomment
\vsnum{2}בַּיָּמִ֖ים הָהֵ֑ם כְּשֶׁ֣בֶת ׀ הַמֶּ֣לֶךְ אֲחַשְׁוֵר֗וֹשׁ עַ֚ל כִּסֵּ֣א מַלְכוּת֔וֹ אֲשֶׁ֖ר בְּשׁוּשַׁ֥ן הַבִּירָֽה׃%
\commenta{\normalfont{כְּשֶׁבֶת הַמֶּלֶךְ אֲחַשְׁוֵרוֹשׁ, וגו'.} כְּשֶׁנִּתְקַיֵּם הַמַּלְכוּת בְּיָדוֹ. וְרַבּוֹתֵינוּ פֵּרְשׁוּהוּ בְּעִנְיָן אַחֵר בְּמַסֶּכֶת מְגִילָּה:}%endcomment
\vsnum{3}בִּשְׁנַ֤ת שָׁלוֹשׁ֙ לְמָלְכ֔וֹ עָשָׂ֣ה מִשְׁתֶּ֔ה לְכָל־שָׂרָ֖יו וַעֲבָדָ֑יו חֵ֣יל ׀ פָּרַ֣ס וּמָדַ֗י הַֽפַּרְתְּמִ֛ים וְשָׂרֵ֥י הַמְּדִינ֖וֹת לְפָנָֽיו׃%
\commenta{\normalfont{הַפַּרְתְּמִים.} שִׁלְטוֹנִים בִּלְשׁוֹן פָּרַס:}%endcomment
\vsnum{4}בְּהַרְאֹת֗וֹ אֶת־עֹ֙שֶׁר֙ כְּב֣וֹד מַלְכוּת֔וֹ וְאֶ֨ת־יְקָ֔ר תִּפְאֶ֖רֶת גְּדוּלָּת֑וֹ יָמִ֣ים רַבִּ֔ים שְׁמוֹנִ֥ים וּמְאַ֖ת יֽוֹם׃%
\commenta{\normalfont{יָמִים רַבִּים.} עָשָׂה לָהֶם מִשְׁתֶּה:}%endcomment
\vsnum{5}וּבִמְל֣וֹאת ׀ הַיָּמִ֣ים הָאֵ֗לֶּה עָשָׂ֣ה הַמֶּ֡לֶךְ לְכָל־הָעָ֣ם הַנִּמְצְאִים֩ בְּשׁוּשַׁ֨ן הַבִּירָ֜ה לְמִגָּ֧דוֹל וְעַד־קָטָ֛ן מִשְׁתֶּ֖ה שִׁבְעַ֣ת יָמִ֑ים בַּחֲצַ֕ר גִּנַּ֥ת בִּיתַ֖ן הַמֶּֽלֶךְ׃%
\commenta{\normalfont{גִּנַּת.} מְקוֹם זֵרְעוֹנֵי יְרָקוֹת:}%endcomment
\vsnum{6}ח֣וּר ׀ כַּרְפַּ֣ס וּתְכֵ֗לֶת אָחוּז֙ בְּחַבְלֵי־ב֣וּץ וְאַרְגָּמָ֔ן עַל־גְּלִ֥ילֵי כֶ֖סֶף וְעַמּ֣וּדֵי שֵׁ֑שׁ מִטּ֣וֹת ׀ זָהָ֣ב וָכֶ֗סֶף עַ֛ל רִֽצְפַ֥ת בַּהַט־וָשֵׁ֖שׁ וְדַ֥ר וְסֹחָֽרֶת׃%
\commenta{\normalfont{חוּר כַּרְפַּס וּתְכֵלֶת.} מִינֵי בְגָדִים צִבְעוֹנִים פֵּרַס לָהֶם לְמַצָּעוֹת:}%endcomment
\vsnum{7}וְהַשְׁקוֹת֙ בִּכְלֵ֣י זָהָ֔ב וְכֵלִ֖ים מִכֵּלִ֣ים שׁוֹנִ֑ים וְיֵ֥ין מַלְכ֛וּת רָ֖ב כְּיַ֥ד הַמֶּֽלֶךְ׃%
\commenta{\normalfont{וְהַשְׁקוֹת בִּכְלֵי זָהָב.} כְּמוֹ וּלְהַשְׁקוֹת:}%endcomment
\vsnum{8}וְהַשְּׁתִיָּ֥ה כַדָּ֖ת אֵ֣ין אֹנֵ֑ס כִּי־כֵ֣ן ׀ יִסַּ֣ד הַמֶּ֗לֶךְ עַ֚ל כָּל־רַ֣ב בֵּית֔וֹ לַעֲשׂ֖וֹת כִּרְצ֥וֹן אִישׁ־וָאִֽישׁ׃%
\commenta{\normalfont{כַדָּת.} לְפִי שֶׁיֵּשׁ סְעוּדוֹת שֶׁכּוֹפִין אֶת הַמְּסֻבִּין לִשְׁתּוֹת כְּלִי גָדוֹל, וְיֵשׁ שֶׁאֵינוֹ יָכוֹל לִשְׁתּוֹתוֹ כִּי אִם בְּקֹשִׁי, אֲבָל כַּאן: "אֵין אוֹנֵס": }%endcomment
\vsnum{9}גַּ֚ם וַשְׁתִּ֣י הַמַּלְכָּ֔ה עָשְׂתָ֖ה מִשְׁתֵּ֣ה נָשִׁ֑ים בֵּ֚ית הַמַּלְכ֔וּת אֲשֶׁ֖ר לַמֶּ֥לֶךְ אֲחַשְׁוֵרֽוֹשׁ׃ (ס)
\vsnum{10}בַּיּוֹם֙ הַשְּׁבִיעִ֔י כְּט֥וֹב לֵב־הַמֶּ֖לֶךְ בַּיָּ֑יִן אָמַ֡ר לִ֠מְהוּמָן בִּזְּתָ֨א חַרְבוֹנָ֜א בִּגְתָ֤א וַאֲבַגְתָא֙ זֵתַ֣ר וְכַרְכַּ֔ס שִׁבְעַת֙ הַסָּ֣רִיסִ֔ים הַמְשָׁ֣רְתִ֔ים אֶת־פְּנֵ֖י הַמֶּ֥לֶךְ אֲחַשְׁוֵרֽוֹשׁ׃%
\commenta{\normalfont{בַּיּוֹם הַשְּׁבִיעִי.} רַבּוֹתֵינוּ אָמְרוּ: שַׁבָּת הָיָה:}%endcomment
\vsnum{11}לְ֠הָבִיא אֶת־וַשְׁתִּ֧י הַמַּלְכָּ֛ה לִפְנֵ֥י הַמֶּ֖לֶךְ בְּכֶ֣תֶר מַלְכ֑וּת לְהַרְא֨וֹת הָֽעַמִּ֤ים וְהַשָּׂרִים֙ אֶת־יָפְיָ֔הּ כִּֽי־טוֹבַ֥ת מַרְאֶ֖ה הִֽיא׃
\vsnum{12}וַתְּמָאֵ֞ן הַמַּלְכָּ֣ה וַשְׁתִּ֗י לָבוֹא֙ בִּדְבַ֣ר הַמֶּ֔לֶךְ אֲשֶׁ֖ר בְּיַ֣ד הַסָּרִיסִ֑ים וַיִּקְצֹ֤ף הַמֶּ֙לֶךְ֙ מְאֹ֔ד וַחֲמָת֖וֹ בָּעֲרָ֥ה בֽוֹ׃%
\commenta{\normalfont{וַתְּמָאֵן הַמַּלְכָּה וַשְׁתִּי.} רַבּוֹתֵינוּ אָמְרוּ: לְפִי שֶׁפָּרְחָה בָהּ צָרַעַת כְּדֵי שֶׁתְּמָאֵן וְתֵהָרֵג. לְפִי שֶׁהָיְתָה מַפְשֶׁטֶת בְּנוֹת יִשְׂרָאֵל עֲרֻמּוֹת וְעוֹשָׂה בָהֶן מְלָאכָה בַּשַּׁבָּת, נִגְזַר עָלֶיהָ שֶׁתִּפָּשֵׁט עֲרֻמָּה בַּשַּׁבָּת: }%endcomment
\vsnum{13}וַיֹּ֣אמֶר הַמֶּ֔לֶךְ לַחֲכָמִ֖ים יֹדְעֵ֣י הָֽעִתִּ֑ים כִּי־כֵן֙ דְּבַ֣ר הַמֶּ֔לֶךְ לִפְנֵ֕י כָּל־יֹדְעֵ֖י דָּ֥ת וָדִֽין׃%
\commenta{\normalfont{כִּי כֵן דְּבַר הַמֶּלֶךְ.} כִּי כֵן מִנְהַג הַמֶּלֶךְ בְּכָל מִשְׁפָּט לָשׂוּם אֶת הַדָּבָר "לִפְנֵי כָּל יוֹדְעֵי דָּת וָדִין": }%endcomment
\vsnum{14}וְהַקָּרֹ֣ב אֵלָ֗יו כַּרְשְׁנָ֤א שֵׁתָר֙ אַדְמָ֣תָא תַרְשִׁ֔ישׁ מֶ֥רֶס מַרְסְנָ֖א מְמוּכָ֑ן שִׁבְעַ֞ת שָׂרֵ֣י ׀ פָּרַ֣ס וּמָדַ֗י רֹאֵי֙ פְּנֵ֣י הַמֶּ֔לֶךְ הַיֹּשְׁבִ֥ים רִאשֹׁנָ֖ה בַּמַּלְכֽוּת׃%
\commenta{\normalfont{וְהַקָּרֹב אֵלָיו.} לַעֲרֹךְ דְּבָרָיו לִפְנֵיהֶם. אֵלּוּ הֵם: כַּרְשְׁנָא שֵׁתָר, וגו': }%endcomment
\vsnum{15}כְּדָת֙ מַֽה־לַּעֲשׂ֔וֹת בַּמַּלְכָּ֖ה וַשְׁתִּ֑י עַ֣ל ׀ אֲשֶׁ֣ר לֹֽא־עָשְׂתָ֗ה אֶֽת־מַאֲמַר֙ הַמֶּ֣לֶךְ אֲחַשְׁוֵר֔וֹשׁ בְּיַ֖ד הַסָּרִיסִֽים׃ (ס)%
\commenta{\normalfont{כְּדָת מַה לַּעֲשׂוֹת.} מוּסָב עַל "וַיֹּאמֶר הַמֶּלֶךְ לַחֲכָמִים": }%endcomment
\vsnum{16}וַיֹּ֣אמֶר מומכן [מְמוּכָ֗ן] לִפְנֵ֤י הַמֶּ֙לֶךְ֙ וְהַשָּׂרִ֔ים לֹ֤א עַל־הַמֶּ֙לֶךְ֙ לְבַדּ֔וֹ עָוְתָ֖ה וַשְׁתִּ֣י הַמַּלְכָּ֑ה כִּ֤י עַל־כָּל־הַשָּׂרִים֙ וְעַל־כָּל־הָ֣עַמִּ֔ים אֲשֶׁ֕ר בְּכָל־מְדִינ֖וֹת הַמֶּ֥לֶךְ אֲחַשְׁוֵרֽוֹשׁ׃%
\commenta{\normalfont{עָוְתָה.} לְשׁוֹן עָו‍ֹן:}%endcomment
\vsnum{17}כִּֽי־יֵצֵ֤א דְבַר־הַמַּלְכָּה֙ עַל־כָּל־הַנָּשִׁ֔ים לְהַבְז֥וֹת בַּעְלֵיהֶ֖ן בְּעֵינֵיהֶ֑ן בְּאָמְרָ֗ם הַמֶּ֣לֶךְ אֲחַשְׁוֵר֡וֹשׁ אָמַ֞ר לְהָבִ֨יא אֶת־וַשְׁתִּ֧י הַמַּלְכָּ֛ה לְפָנָ֖יו וְלֹא־בָֽאָה׃%
\commenta{\normalfont{כִּי יֵצֵא דְבַר הַמַּלְכָּה עַל כָּל הַנָּשִׁים.} זֶה שֶׁבִּזְּתָה אֶת הַמֶּלֶךְ עַל כָּל הַנָּשִׁים לְהַבְזוֹת אַף הֵן אֶת בַּעֲלֵיהֶן:}%endcomment
\vsnum{18}וְֽהַיּ֨וֹם הַזֶּ֜ה תֹּאמַ֣רְנָה ׀ שָׂר֣וֹת פָּֽרַס־וּמָדַ֗י אֲשֶׁ֤ר שָֽׁמְעוּ֙ אֶת־דְּבַ֣ר הַמַּלְכָּ֔ה לְכֹ֖ל שָׂרֵ֣י הַמֶּ֑לֶךְ וּכְדַ֖י בִּזָּי֥וֹן וָקָֽצֶף׃%
\commenta{\normalfont{תּאמַרְנָה שָׂרוֹת פָּרַס וּמָדַי וגו'.} לְכֹל שָׂרֵי הַמֶּלֶךְ אֶת הַדָּבָר הַזֶּה. וַהֲרֵי זֶה מִקְרָא קָצֵר:}%endcomment
\vsnum{19}אִם־עַל־הַמֶּ֣לֶךְ ט֗וֹב יֵצֵ֤א דְבַר־מַלְכוּת֙ מִלְּפָנָ֔יו וְיִכָּתֵ֛ב בְּדָתֵ֥י פָֽרַס־וּמָדַ֖י וְלֹ֣א יַעֲב֑וֹר אֲשֶׁ֨ר לֹֽא־תָב֜וֹא וַשְׁתִּ֗י לִפְנֵי֙ הַמֶּ֣לֶךְ אֲחַשְׁוֵר֔וֹשׁ וּמַלְכוּתָהּ֙ יִתֵּ֣ן הַמֶּ֔לֶךְ לִרְעוּתָ֖הּ הַטּוֹבָ֥ה מִמֶּֽנָּה׃%
\commenta{\normalfont{דְבַר מַלְכוּת.} גְּזֵרַת מַלְכוּת שֶׁל נְקָמָה שֶׁצִּוָּה לְהָרְגָהּ:}%endcomment
\vsnum{20}וְנִשְׁמַע֩ פִּתְגָ֨ם הַמֶּ֤לֶךְ אֲשֶֽׁר־יַעֲשֶׂה֙ בְּכָל־מַלְכוּת֔וֹ כִּ֥י רַבָּ֖ה הִ֑יא וְכָל־הַנָּשִׁ֗ים יִתְּנ֤וּ יְקָר֙ לְבַעְלֵיהֶ֔ן לְמִגָּד֖וֹל וְעַד־קָטָֽן׃
\vsnum{21}וַיִּיטַב֙ הַדָּבָ֔ר בְּעֵינֵ֥י הַמֶּ֖לֶךְ וְהַשָּׂרִ֑ים וַיַּ֥עַשׂ הַמֶּ֖לֶךְ כִּדְבַ֥ר מְמוּכָֽן׃
\vsnum{22}וַיִּשְׁלַ֤ח סְפָרִים֙ אֶל־כָּל־מְדִינ֣וֹת הַמֶּ֔לֶךְ אֶל־מְדִינָ֤ה וּמְדִינָה֙ כִּכְתָבָ֔הּ וְאֶל־עַ֥ם וָעָ֖ם כִּלְשׁוֹנ֑וֹ לִהְי֤וֹת כָּל־אִישׁ֙ שֹׂרֵ֣ר בְּבֵית֔וֹ וּמְדַבֵּ֖ר כִּלְשׁ֥וֹן עַמּֽוֹ׃ (פ)%
\commenta{\normalfont{וּמְדַבֵּר כִּלְשׁוֹן עַמּוֹ.} כּוֹפֶה אֶת אִשְׁתּוֹ לִלְמֹד אֶת לְשׁוֹנוֹ אִם הִיא בַּת לָשׁוֹן אַחֵר:}%endcomment
\clearpage}

\newchap{פרק ב}
\twocol{\vsnum{1}אַחַר֙ הַדְּבָרִ֣ים הָאֵ֔לֶּה כְּשֹׁ֕ךְ חֲמַ֖ת הַמֶּ֣לֶךְ אֲחַשְׁוֵר֑וֹשׁ זָכַ֤ר אֶת־וַשְׁתִּי֙ וְאֵ֣ת אֲשֶׁר־עָשָׂ֔תָה וְאֵ֥ת אֲשֶׁר־נִגְזַ֖ר עָלֶֽיהָ׃%
\commenta{\normalfont{זָכַר אֶת וַשְׁתִּי.} אֶת יָפְיָהּ וְנֶעֱצַב:}%endcomment
\vsnum{2}וַיֹּאמְר֥וּ נַעֲרֵֽי־הַמֶּ֖לֶךְ מְשָׁרְתָ֑יו יְבַקְשׁ֥וּ לַמֶּ֛לֶךְ נְעָר֥וֹת בְּתוּל֖וֹת טוֹב֥וֹת מַרְאֶֽה׃
\vsnum{3}וְיַפְקֵ֨ד הַמֶּ֣לֶךְ פְּקִידִים֮ בְּכָל־מְדִינ֣וֹת מַלְכוּתוֹ֒ וְיִקְבְּצ֣וּ אֶת־כָּל־נַעֲרָֽה־בְ֠תוּלָה טוֹבַ֨ת מַרְאֶ֜ה אֶל־שׁוּשַׁ֤ן הַבִּירָה֙ אֶל־בֵּ֣ית הַנָּשִׁ֔ים אֶל־יַ֥ד הֵגֶ֛א סְרִ֥יס הַמֶּ֖לֶךְ שֹׁמֵ֣ר הַנָּשִׁ֑ים וְנָת֖וֹן תַּמְרוּקֵיהֶֽן׃%
\commenta{\normalfont{וְיַפְקֵד הַמֶּלֶךְ פְּקִידִים.} לְפִי שֶׁכָּל פָּקִיד וּפָקִיד יְדוּעוֹת לוֹ נָשִׁים הַיָּפוֹת שֶׁבִּמְדִינָתוֹ:}%endcomment
\vsnum{4}וְהַֽנַּעֲרָ֗ה אֲשֶׁ֤ר תִּיטַב֙ בְּעֵינֵ֣י הַמֶּ֔לֶךְ תִּמְלֹ֖ךְ תַּ֣חַת וַשְׁתִּ֑י וַיִּיטַ֧ב הַדָּבָ֛ר בְּעֵינֵ֥י הַמֶּ֖לֶךְ וַיַּ֥עַשׂ כֵּֽן׃ (ס)
\vsnum{5}אִ֣ישׁ יְהוּדִ֔י הָיָ֖ה בְּשׁוּשַׁ֣ן הַבִּירָ֑ה וּשְׁמ֣וֹ מָרְדֳּכַ֗י בֶּ֣ן יָאִ֧יר בֶּן־שִׁמְעִ֛י בֶּן־קִ֖ישׁ אִ֥ישׁ יְמִינִֽי׃%
\commenta{\normalfont{אִישׁ יְהוּדִי.} עַל שֶׁגָּלָה עִם גָּלוּת יְהוּדָה. כָּל אוֹתָן שֶׁגָּלוּ עִם מַלְכֵי יְהוּדָה הָיוּ קְרוּיִים "יְהוּדִים" בֵּין הַגּוֹיִם, וַאֲפִילוּ מִשֵּׁבֶט אַחֵר הֵם: }%endcomment
\vsnum{6}אֲשֶׁ֤ר הָגְלָה֙ מִיר֣וּשָׁלַ֔יִם עִם־הַגֹּלָה֙ אֲשֶׁ֣ר הָגְלְתָ֔ה עִ֖ם יְכָנְיָ֣ה מֶֽלֶךְ־יְהוּדָ֑ה אֲשֶׁ֣ר הֶגְלָ֔ה נְבוּכַדְנֶאצַּ֖ר מֶ֥לֶךְ בָּבֶֽל׃
\vsnum{7}וַיְהִ֨י אֹמֵ֜ן אֶת־הֲדַסָּ֗ה הִ֤יא אֶסְתֵּר֙ בַּת־דֹּד֔וֹ כִּ֛י אֵ֥ין לָ֖הּ אָ֣ב וָאֵ֑ם וְהַנַּעֲרָ֤ה יְפַת־תֹּ֙אַר֙ וְטוֹבַ֣ת מַרְאֶ֔ה וּבְמ֤וֹת אָבִ֙יהָ֙ וְאִמָּ֔הּ לְקָחָ֧הּ מָרְדֳּכַ֛י ל֖וֹ לְבַֽת׃%
\commenta{\normalfont{לוֹ לְבַת.} רַבּוֹתֵינוּ פֵּרְשׁוּ "לְבַיִת", לְאִשָּׁה: }%endcomment
\vsnum{8}וַיְהִ֗י בְּהִשָּׁמַ֤ע דְּבַר־הַמֶּ֙לֶךְ֙ וְדָת֔וֹ וּֽבְהִקָּבֵ֞ץ נְעָר֥וֹת רַבּ֛וֹת אֶל־שׁוּשַׁ֥ן הַבִּירָ֖ה אֶל־יַ֣ד הֵגָ֑י וַתִּלָּקַ֤ח אֶסְתֵּר֙ אֶל־בֵּ֣ית הַמֶּ֔לֶךְ אֶל־יַ֥ד הֵגַ֖י שֹׁמֵ֥ר הַנָּשִֽׁים׃
\vsnum{9}וַתִּיטַ֨ב הַנַּעֲרָ֣ה בְעֵינָיו֮ וַתִּשָּׂ֣א חֶ֣סֶד לְפָנָיו֒ וַ֠יְבַהֵל אֶת־תַּמְרוּקֶ֤יהָ וְאֶת־מָנוֹתֶ֙הָ֙ לָתֵ֣ת לָ֔הּ וְאֵת֙ שֶׁ֣בַע הַנְּעָר֔וֹת הָרְאֻי֥וֹת לָֽתֶת־לָ֖הּ מִבֵּ֣ית הַמֶּ֑לֶךְ וַיְשַׁנֶּ֧הָ וְאֶת־נַעֲרוֹתֶ֛יהָ לְט֖וֹב בֵּ֥ית הַנָּשִֽׁים׃%
\commenta{\normalfont{וַיְבַהֵל אֶת תַּמְרוּקֶיהָ.} זָרִיז וּמְמַהֵר בְּשֶׁלָּהּ, מִשֶּׁל כֻּלָּן: }%endcomment
\vsnum{10}לֹא־הִגִּ֣ידָה אֶסְתֵּ֔ר אֶת־עַמָּ֖הּ וְאֶת־מֽוֹלַדְתָּ֑הּ כִּ֧י מָרְדֳּכַ֛י צִוָּ֥ה עָלֶ֖יהָ אֲשֶׁ֥ר לֹא־תַגִּֽיד׃%
\commenta{\normalfont{אֲשֶׁר לֹא תַגִּיד.} כְּדֵי שֶׁיֹּאמְרוּ שֶׁהִיא מִמִּשְׁפָּחָה בְזוּיָה וִישַׁלְּחוּהָ, שֶׁאִם יֵדְעוּ שֶׁהִיא מִמִּשְׁפַּחַת שָׁאוּל הַמֶּלֶךְ הָיוּ מַחֲזִיקִים בָּהּ: }%endcomment
\vsnum{11}וּבְכָל־י֣וֹם וָי֔וֹם מָרְדֳּכַי֙ מִתְהַלֵּ֔ךְ לִפְנֵ֖י חֲצַ֣ר בֵּית־הַנָּשִׁ֑ים לָדַ֙עַת֙ אֶת־שְׁל֣וֹם אֶסְתֵּ֔ר וּמַה־יֵּעָשֶׂ֖ה בָּֽהּ׃%
\commenta{\normalfont{וּמַה יֵּעָשֶׂה בָּהּ.} זֶה אֶחָד מִשְּׁנֵי צַדִּיקִים שֶׁנִּתַּן לָהֶם רֶמֶז יְשׁוּעָה: דָּוִד וּמָרְדְּכַי. דָּוִד, שֶׁנֶּאֱמַר "גַּם אֶת הָאֲרִי גַּם אֶת הַדּוֹב הִכָּה עַבְדֶּךָ". אָמַר: "לֹא בָא לְיָדִי דָבָר זֶה אֶלָּא לִסְמֹךְ עָלָיו לְהִלָּחֵם עִם זֶה". וְכֵן מָרְדְּכַי אָמַר: "לֹא אֵרַע לְצַדֶּקֶת זוּ שֶׁתִּלָּקַח לְמִשְׁכַּב נָכְרִי אֶלָּא שֶׁעֲתִידָה לָקוּם לְהוֹשִׁיעַ לְיִשְׂרָאֵל". לְפִיכָךְ, הָיָה מְחַזֵּר לָדַעַת מַה יְּהֵא בְסוֹפָהּ: }%endcomment
\vsnum{12}וּבְהַגִּ֡יעַ תֹּר֩ נַעֲרָ֨ה וְנַעֲרָ֜ה לָב֣וֹא ׀ אֶל־הַמֶּ֣לֶךְ אֲחַשְׁוֵר֗וֹשׁ מִקֵּץ֩ הֱי֨וֹת לָ֜הּ כְּדָ֤ת הַנָּשִׁים֙ שְׁנֵ֣ים עָשָׂ֣ר חֹ֔דֶשׁ כִּ֛י כֵּ֥ן יִמְלְא֖וּ יְמֵ֣י מְרוּקֵיהֶ֑ן שִׁשָּׁ֤ה חֳדָשִׁים֙ בְּשֶׁ֣מֶן הַמֹּ֔ר וְשִׁשָּׁ֤ה חֳדָשִׁים֙ בַּבְּשָׂמִ֔ים וּבְתַמְרוּקֵ֖י הַנָּשִֽׁים׃%
\commenta{\normalfont{תּר.} זְמַן:}%endcomment
\vsnum{13}וּבָזֶ֕ה הַֽנַּעֲרָ֖ה בָּאָ֣ה אֶל־הַמֶּ֑לֶךְ אֵת֩ כָּל־אֲשֶׁ֨ר תֹּאמַ֜ר יִנָּ֤תֵֽן לָהּ֙ לָב֣וֹא עִמָּ֔הּ מִבֵּ֥ית הַנָּשִׁ֖ים עַד־בֵּ֥ית הַמֶּֽלֶךְ׃%
\commenta{\normalfont{כָּל אֲשֶׁר תּאמַר.} כָּל שְׂחוֹק וּמִינֵי זֶמֶר:}%endcomment
\vsnum{14}בָּעֶ֣רֶב ׀ הִ֣יא בָאָ֗ה וּ֠בַבֹּקֶר הִ֣יא שָׁבָ֞ה אֶל־בֵּ֤ית הַנָּשִׁים֙ שֵׁנִ֔י אֶל־יַ֧ד שַֽׁעֲשְׁגַ֛ז סְרִ֥יס הַמֶּ֖לֶךְ שֹׁמֵ֣ר הַפִּֽילַגְשִׁ֑ים לֹא־תָב֥וֹא עוֹד֙ אֶל־הַמֶּ֔לֶךְ כִּ֣י אִם־חָפֵ֥ץ בָּ֛הּ הַמֶּ֖לֶךְ וְנִקְרְאָ֥ה בְשֵֽׁם׃%
\commenta{\normalfont{אֶל בֵּית הַנָּשִׁים שֵׁנִי.} הַשֵּׁנִי:}%endcomment
\vsnum{15}וּבְהַגִּ֣יעַ תֹּר־אֶסְתֵּ֣ר בַּת־אֲבִיחַ֣יִל דֹּ֣ד מָרְדֳּכַ֡י אֲשֶׁר֩ לָקַֽח־ל֨וֹ לְבַ֜ת לָב֣וֹא אֶל־הַמֶּ֗לֶךְ לֹ֤א בִקְשָׁה֙ דָּבָ֔ר כִּ֠י אִ֣ם אֶת־אֲשֶׁ֥ר יֹאמַ֛ר הֵגַ֥י סְרִיס־הַמֶּ֖לֶךְ שֹׁמֵ֣ר הַנָּשִׁ֑ים וַתְּהִ֤י אֶסְתֵּר֙ נֹשֵׂ֣את חֵ֔ן בְּעֵינֵ֖י כָּל־רֹאֶֽיהָ׃
\vsnum{16}וַתִּלָּקַ֨ח אֶסְתֵּ֜ר אֶל־הַמֶּ֤לֶךְ אֲחַשְׁוֵרוֹשׁ֙ אֶל־בֵּ֣ית מַלְכוּת֔וֹ בַּחֹ֥דֶשׁ הָעֲשִׂירִ֖י הוּא־חֹ֣דֶשׁ טֵבֵ֑ת בִּשְׁנַת־שֶׁ֖בַע לְמַלְכוּתֽוֹ׃%
\commenta{\normalfont{בַּחֹדֶשׁ הָעֲשִׂירִי.} עֵת צִנָּה שֶׁהַגּוּף נֶהֱנֶה מִן הַגּוּף. זִמֵּן הַקָּדוֹשׁ בָּרוּךְ הוּא אוֹתוֹ עֵת צִנָּה כְּדֵי לְחַבְּבָהּ עָלָיו:}%endcomment
\vsnum{17}וַיֶּאֱהַ֨ב הַמֶּ֤לֶךְ אֶת־אֶסְתֵּר֙ מִכָּל־הַנָּשִׁ֔ים וַתִּשָּׂא־חֵ֥ן וָחֶ֛סֶד לְפָנָ֖יו מִכָּל־הַבְּתוּלֹ֑ת וַיָּ֤שֶׂם כֶּֽתֶר־מַלְכוּת֙ בְּרֹאשָׁ֔הּ וַיַּמְלִיכֶ֖הָ תַּ֥חַת וַשְׁתִּֽי׃%
\commenta{\normalfont{מִכָּל הַנָּשִׁים.} הַבְּעוּלוֹת, שֶׁאַף נָשִׁים הַבְּעוּלוֹת קִבֵּץ: }%endcomment
\vsnum{18}וַיַּ֨עַשׂ הַמֶּ֜לֶךְ מִשְׁתֶּ֣ה גָד֗וֹל לְכָל־שָׂרָיו֙ וַעֲבָדָ֔יו אֵ֖ת מִשְׁתֵּ֣ה אֶסְתֵּ֑ר וַהֲנָחָ֤ה לַמְּדִינוֹת֙ עָשָׂ֔ה וַיִּתֵּ֥ן מַשְׂאֵ֖ת כְּיַ֥ד הַמֶּֽלֶךְ׃%
\commenta{\normalfont{וַהֲנָחָה לַמְּדִינוֹת עָשָׂה.} לִכְבוֹדָהּ הֵנִיחַ לָהֶם מִן הַמַּס שֶׁעֲלֵיהֶם:}%endcomment
\vsnum{19}וּבְהִקָּבֵ֥ץ בְּתוּל֖וֹת שֵׁנִ֑ית וּמָרְדֳּכַ֖י יֹשֵׁ֥ב בְּשַֽׁעַר־הַמֶּֽלֶךְ׃
\vsnum{20}אֵ֣ין אֶסְתֵּ֗ר מַגֶּ֤דֶת מֽוֹלַדְתָּהּ֙ וְאֶת־עַמָּ֔הּ כַּאֲשֶׁ֛ר צִוָּ֥ה עָלֶ֖יהָ מָרְדֳּכָ֑י וְאֶת־מַאֲמַ֤ר מָרְדֳּכַי֙ אֶסְתֵּ֣ר עֹשָׂ֔ה כַּאֲשֶׁ֛ר הָיְתָ֥ה בְאָמְנָ֖ה אִתּֽוֹ׃ (ס)%
\commenta{\normalfont{אֵין אֶסְתֵּר מַגֶּדֶת מוֹלַדְתָּהּ.} לְפִי שֶׁמָּרְדְּכַי יוֹשֵׁב בְּשַׁעַר הַמֶּלֶךְ, הַמְזָרְזָהּ וְהַמְרַמְּזָהּ עַל כָּךְ: }%endcomment
\vsnum{21}בַּיָּמִ֣ים הָהֵ֔ם וּמָרְדֳּכַ֖י יֹשֵׁ֣ב בְּשַֽׁעַר־הַמֶּ֑לֶךְ קָצַף֩ בִּגְתָ֨ן וָתֶ֜רֶשׁ שְׁנֵֽי־סָרִיסֵ֤י הַמֶּ֙לֶךְ֙ מִשֹּׁמְרֵ֣י הַסַּ֔ף וַיְבַקְשׁוּ֙ לִשְׁלֹ֣חַ יָ֔ד בַּמֶּ֖לֶךְ אֲחַשְׁוֵֽרֹשׁ׃%
\commenta{\normalfont{לִשְׁלֹחַ יָד.} לְהַשְׁקוֹתוֹ סַם הַמָּוֶת:}%endcomment
\vsnum{22}וַיִּוָּדַ֤ע הַדָּבָר֙ לְמָרְדֳּכַ֔י וַיַּגֵּ֖ד לְאֶסְתֵּ֣ר הַמַּלְכָּ֑ה וַתֹּ֧אמֶר אֶסְתֵּ֛ר לַמֶּ֖לֶךְ בְּשֵׁ֥ם מָרְדֳּכָֽי׃%
\commenta{\normalfont{וַיִּוָּדַע הַדָּבָר לְמָרְדְּכַי.} שֶׁהָיוּ מְסַפְּרִים דִּבְרֵיהֶם לְפָנָיו בְּלָשׁוֹן טוּרְסִי, וְאֵין יוֹדְעִים שֶׁהָיָה מָרְדְּכַי מַכִּיר בְּשִׁבְעִים לְשׁוֹנוֹת שֶׁהָיָה מִיּוֹשְׁבֵי לִשְׁכַּת הַגָּזִית: }%endcomment
\vsnum{23}וַיְבֻקַּ֤שׁ הַדָּבָר֙ וַיִּמָּצֵ֔א וַיִּתָּל֥וּ שְׁנֵיהֶ֖ם עַל־עֵ֑ץ וַיִּכָּתֵ֗ב בְּסֵ֛פֶר דִּבְרֵ֥י הַיָּמִ֖ים לִפְנֵ֥י הַמֶּֽלֶךְ׃ (פ)%
\commenta{\normalfont{וַיִּכָּתֵב בְּסֵפֶר דִּבְרֵי הַיָּמִים.} הַטּוֹבָה שֶׁעָשָׂה מָרְדְּכַי לַמֶּלֶךְ:}%endcomment
\clearpage}

\newchap{פרק ג}
\twocol{\vsnum{1}אַחַ֣ר ׀ הַדְּבָרִ֣ים הָאֵ֗לֶּה גִּדַּל֩ הַמֶּ֨לֶךְ אֲחַשְׁוֵר֜וֹשׁ אֶת־הָמָ֧ן בֶּֽן־הַמְּדָ֛תָא הָאֲגָגִ֖י וַֽיְנַשְּׂאֵ֑הוּ וַיָּ֙שֶׂם֙ אֶת־כִּסְא֔וֹ מֵעַ֕ל כָּל־הַשָּׂרִ֖ים אֲשֶׁ֥ר אִתּֽוֹ׃%
\commenta{\normalfont{אַחַר הַדְּבָרִים הָאֵלֶּה.} שֶׁנִבְרֵאת רְפוּאָה זוּ לִהְיוֹת לִתְשׁוּעָה לְיִשְׂרָאֵל: }%endcomment
\vsnum{2}וְכָל־עַבְדֵ֨י הַמֶּ֜לֶךְ אֲשֶׁר־בְּשַׁ֣עַר הַמֶּ֗לֶךְ כֹּרְעִ֤ים וּמִֽשְׁתַּחֲוִים֙ לְהָמָ֔ן כִּי־כֵ֖ן צִוָּה־ל֣וֹ הַמֶּ֑לֶךְ וּמָ֨רְדֳּכַ֔י לֹ֥א יִכְרַ֖ע וְלֹ֥א יִֽשְׁתַּחֲוֶֽה׃%
\commenta{\normalfont{כֹּרְעִים וּמִשְׁתַּחֲוִים.} שֶׁעָשָׂה עַצְמוֹ אֱלוֹהַּ, לְפִיכָךְ, וּמָרְדְּכַי לֹא יִכְרַע וְלֹא יִשְׁתַּחֲוֶה: }%endcomment
\vsnum{3}וַיֹּ֨אמְר֜וּ עַבְדֵ֥י הַמֶּ֛לֶךְ אֲשֶׁר־בְּשַׁ֥עַר הַמֶּ֖לֶךְ לְמָרְדֳּכָ֑י מַדּ֙וּעַ֙ אַתָּ֣ה עוֹבֵ֔ר אֵ֖ת מִצְוַ֥ת הַמֶּֽלֶךְ׃
\vsnum{4}וַיְהִ֗י באמרם [כְּאָמְרָ֤ם] אֵלָיו֙ י֣וֹם וָי֔וֹם וְלֹ֥א שָׁמַ֖ע אֲלֵיהֶ֑ם וַיַּגִּ֣ידוּ לְהָמָ֗ן לִרְאוֹת֙ הֲיַֽעַמְדוּ֙ דִּבְרֵ֣י מָרְדֳּכַ֔י כִּֽי־הִגִּ֥יד לָהֶ֖ם אֲשֶׁר־ה֥וּא יְהוּדִֽי׃%
\commenta{\normalfont{הֲיַעַמְדוּ דִּבְרֵי מָרְדְּכַי.} הָאוֹמֵר שֶׁלֹּא יִשְׁתַּחֲוֶה עוֹלָמִית, כִּי הוּא יְהוּדִי, וְהוּזְהַר עַל עֲבוֹדַת אֱלִילִים: }%endcomment
\vsnum{5}וַיַּ֣רְא הָמָ֔ן כִּי־אֵ֣ין מָרְדֳּכַ֔י כֹּרֵ֥עַ וּמִֽשְׁתַּחֲוֶ֖ה ל֑וֹ וַיִּמָּלֵ֥א הָמָ֖ן חֵמָֽה׃
\vsnum{6}וַיִּ֣בֶז בְּעֵינָ֗יו לִשְׁלֹ֤ח יָד֙ בְּמָרְדֳּכַ֣י לְבַדּ֔וֹ כִּֽי־הִגִּ֥ידוּ ל֖וֹ אֶת־עַ֣ם מָרְדֳּכָ֑י וַיְבַקֵּ֣שׁ הָמָ֗ן לְהַשְׁמִ֧יד אֶת־כָּל־הַיְּהוּדִ֛ים אֲשֶׁ֛ר בְּכָל־מַלְכ֥וּת אֲחַשְׁוֵר֖וֹשׁ עַ֥ם מָרְדֳּכָֽי׃
\vsnum{7}בַּחֹ֤דֶשׁ הָרִאשׁוֹן֙ הוּא־חֹ֣דֶשׁ נִיסָ֔ן בִּשְׁנַת֙ שְׁתֵּ֣ים עֶשְׂרֵ֔ה לַמֶּ֖לֶךְ אֲחַשְׁוֵר֑וֹשׁ הִפִּ֣יל פּוּר֩ ה֨וּא הַגּוֹרָ֜ל לִפְנֵ֣י הָמָ֗ן מִיּ֧וֹם ׀ לְי֛וֹם וּמֵחֹ֛דֶשׁ לְחֹ֥דֶשׁ שְׁנֵים־עָשָׂ֖ר הוּא־חֹ֥דֶשׁ אֲדָֽר׃ (ס)%
\commenta{\normalfont{הִפִּיל פּוּר.} הִפִּיל מִי שֶׁהִפִּיל, וְלֹא פִּירֵשׁ מִי. וּמִקְרָא קָצֵר הוּא: }%endcomment
\vsnum{8}וַיֹּ֤אמֶר הָמָן֙ לַמֶּ֣לֶךְ אֲחַשְׁוֵר֔וֹשׁ יֶשְׁנ֣וֹ עַם־אֶחָ֗ד מְפֻזָּ֤ר וּמְפֹרָד֙ בֵּ֣ין הָֽעַמִּ֔ים בְּכֹ֖ל מְדִינ֣וֹת מַלְכוּתֶ֑ךָ וְדָתֵיהֶ֞ם שֹׁנ֣וֹת מִכָּל־עָ֗ם וְאֶת־דָּתֵ֤י הַמֶּ֙לֶךְ֙ אֵינָ֣ם עֹשִׂ֔ים וְלַמֶּ֥לֶךְ אֵין־שֹׁוֶ֖ה לְהַנִּיחָֽם׃%
\commenta{\normalfont{וְאֶת דָּתֵי הַמֶּלֶךְ.} לָתֵת מַס לַעֲבוֹדַת הַמֶּלֶךְ:}%endcomment
\vsnum{9}אִם־עַל־הַמֶּ֣לֶךְ ט֔וֹב יִכָּתֵ֖ב לְאַבְּדָ֑ם וַעֲשֶׂ֨רֶת אֲלָפִ֜ים כִּכַּר־כֶּ֗סֶף אֶשְׁקוֹל֙ עַל־יְדֵי֙ עֹשֵׂ֣י הַמְּלָאכָ֔ה לְהָבִ֖יא אֶל־גִּנְזֵ֥י הַמֶּֽלֶךְ׃%
\commenta{\normalfont{יִכָּתֵב לְאַבְּדָם.} יִכָּתֵב סְפָרִים לִשְׁלֹחַ לְשָׂרֵי הַמְּדִינוֹת לְאַבְּדָם:}%endcomment
\vsnum{10}וַיָּ֧סַר הַמֶּ֛לֶךְ אֶת־טַבַּעְתּ֖וֹ מֵעַ֣ל יָד֑וֹ וַֽיִּתְּנָ֗הּ לְהָמָ֧ן בֶּֽן־הַמְּדָ֛תָא הָאֲגָגִ֖י צֹרֵ֥ר הַיְּהוּדִֽים׃%
\commenta{\normalfont{וַיָּסַר הַמֶּלֶךְ אֶת טַבַּעְתּוֹ.} הוּא מַתַּן כָּל דָּבָר גָּדוֹל שֶׁיִּשְׁאֲלוּ מֵאֵת הַמֶּלֶךְ, לִהְיוֹת מִי שֶׁהַטַּבַּעַת בְּיָדוֹ שַׁלִּיט בְּכָל דְּבַר הַמֶּלֶךְ: }%endcomment
\vsnum{11}וַיֹּ֤אמֶר הַמֶּ֙לֶךְ֙ לְהָמָ֔ן הַכֶּ֖סֶף נָת֣וּן לָ֑ךְ וְהָעָ֕ם לַעֲשׂ֥וֹת בּ֖וֹ כַּטּ֥וֹב בְּעֵינֶֽיךָ׃
\vsnum{12}וַיִּקָּרְאוּ֩ סֹפְרֵ֨י הַמֶּ֜לֶךְ בַּחֹ֣דֶשׁ הָרִאשׁ֗וֹן בִּשְׁלוֹשָׁ֨ה עָשָׂ֣ר יוֹם֮ בּוֹ֒ וַיִּכָּתֵ֣ב כְּֽכָל־אֲשֶׁר־צִוָּ֣ה הָמָ֡ן אֶ֣ל אֲחַשְׁדַּרְפְּנֵֽי־הַ֠מֶּלֶךְ וְֽאֶל־הַפַּח֞וֹת אֲשֶׁ֣ר ׀ עַל־מְדִינָ֣ה וּמְדִינָ֗ה וְאֶל־שָׂ֤רֵי עַם֙ וָעָ֔ם מְדִינָ֤ה וּמְדִינָה֙ כִּכְתָבָ֔הּ וְעַ֥ם וָעָ֖ם כִּלְשׁוֹנ֑וֹ בְּשֵׁ֨ם הַמֶּ֤לֶךְ אֲחַשְׁוֵרֹשׁ֙ נִכְתָּ֔ב וְנֶחְתָּ֖ם בְּטַבַּ֥עַת הַמֶּֽלֶךְ׃
\vsnum{13}וְנִשְׁל֨וֹחַ סְפָרִ֜ים בְּיַ֣ד הָרָצִים֮ אֶל־כָּל־מְדִינ֣וֹת הַמֶּלֶךְ֒ לְהַשְׁמִ֡יד לַהֲרֹ֣ג וּלְאַבֵּ֣ד אֶת־כָּל־הַ֠יְּהוּדִים מִנַּ֨עַר וְעַד־זָקֵ֜ן טַ֤ף וְנָשִׁים֙ בְּי֣וֹם אֶחָ֔ד בִּשְׁלוֹשָׁ֥ה עָשָׂ֛ר לְחֹ֥דֶשׁ שְׁנֵים־עָשָׂ֖ר הוּא־חֹ֣דֶשׁ אֲדָ֑ר וּשְׁלָלָ֖ם לָבֽוֹז׃%
\commenta{\normalfont{וְנִשְׁלוֹחַ סְפָרִים.} וְיִהְיוּ נִשְׁלָחִים אישטר"א טרמי"ש בלע"ז. וְהוּא מִגִּזְרַת "אִם נִלְחוֹם נִלְחַם" נִלְחוֹם, "הַנִגְלֹה נִגְלֵיתִי" נִגְלֹה, "נִדְמֹה נִדְמֵיתִי": נִדְמֹה: }%endcomment
\vsnum{14}פַּתְשֶׁ֣גֶן הַכְּתָ֗ב לְהִנָּ֤תֵֽן דָּת֙ בְּכָל־מְדִינָ֣ה וּמְדִינָ֔ה גָּל֖וּי לְכָל־הָֽעַמִּ֑ים לִהְי֥וֹת עֲתִדִ֖ים לַיּ֥וֹם הַזֶּֽה׃%
\commenta{\normalfont{פַּתְשֶׁגֶן.} לָשׁוֹן אֲרַמִּי פַּתְשֶׁגֶן סִפּוּר הַכְּתָב דרישמאנ"ט בְּלַעַ"ז: }%endcomment
\vsnum{15}הָֽרָצִ֞ים יָצְא֤וּ דְחוּפִים֙ בִּדְבַ֣ר הַמֶּ֔לֶךְ וְהַדָּ֥ת נִתְּנָ֖ה בְּשׁוּשַׁ֣ן הַבִּירָ֑ה וְהַמֶּ֤לֶךְ וְהָמָן֙ יָשְׁב֣וּ לִשְׁתּ֔וֹת וְהָעִ֥יר שׁוּשָׁ֖ן נָבֽוֹכָה׃ (פ)%
\commenta{\normalfont{וְהַדָּת נִתְּנָה בְּשׁוּשַׁן.} מָקוֹם שֶׁהָיָה הַמֶּלֶךְ שָׁם נִתַּן הַחֹק בּוֹ בַיּוֹם, לִהְיוֹת עֲתִידִים לְיוֹם י"ג לְחֹדֶשׁ אֲדָר. לְכַךְ, }%endcomment
\clearpage}

\newchap{פרק ד}
\twocol{\vsnum{1}וּמָרְדֳּכַ֗י יָדַע֙ אֶת־כָּל־אֲשֶׁ֣ר נַעֲשָׂ֔ה וַיִּקְרַ֤ע מָרְדֳּכַי֙ אֶת־בְּגָדָ֔יו וַיִּלְבַּ֥שׁ שַׂ֖ק וָאֵ֑פֶר וַיֵּצֵא֙ בְּת֣וֹךְ הָעִ֔יר וַיִּזְעַ֛ק זְעָקָ֥ה גְדֹלָ֖ה וּמָרָֽה׃%
\commenta{\normalfont{וּמָרְדְּכַי יָדַע.} בַּעַל הַחֲלוֹם אָמַר לוֹ שֶׁהִסְכִּימוּ הָעֶלְיוֹנִים עַל כָּךְ לְפִי שֶׁהִשְׁתַּחֲווּ לַצֶּלֶם בִּימֵי נְבוּכַדְנֶצַר וְשֶׁנֶּהֱנוּ מִסְּעוּדַת אֲחַשְׁוֵרוֹשׁ:}%endcomment
\vsnum{2}וַיָּב֕וֹא עַ֖ד לִפְנֵ֣י שַֽׁעַר־הַמֶּ֑לֶךְ כִּ֣י אֵ֥ין לָב֛וֹא אֶל־שַׁ֥עַר הַמֶּ֖לֶךְ בִּלְב֥וּשׁ שָֽׂק׃%
\commenta{\normalfont{כִּי אֵין לָבוֹא.} אֵין דֶּרֶךְ אֶרֶץ לָבוֹא אֶל שַׁעַר הַמֶּלֶךְ בִּלְבוּשׁ שָׂק:}%endcomment
\vsnum{3}וּבְכָל־מְדִינָ֣ה וּמְדִינָ֗ה מְקוֹם֙ אֲשֶׁ֨ר דְּבַר־הַמֶּ֤לֶךְ וְדָתוֹ֙ מַגִּ֔יעַ אֵ֤בֶל גָּדוֹל֙ לַיְּהוּדִ֔ים וְצ֥וֹם וּבְכִ֖י וּמִסְפֵּ֑ד שַׂ֣ק וָאֵ֔פֶר יֻצַּ֖ע לָֽרַבִּֽים׃%
\commenta{\normalfont{דְּבַר הַמֶּלֶךְ וְדָתוֹ.} כְּשֶׁהַשְּׁלוּחִים נוֹשְׂאֵי הַסְּפָרִים עוֹבְרִים שָׁם נִתְּנָה הַדָּת בָּעִיר:}%endcomment
\vsnum{4}וַ֠תָּבוֹאינָה נַעֲר֨וֹת אֶסְתֵּ֤ר וְסָרִיסֶ֙יהָ֙ וַיַּגִּ֣ידוּ לָ֔הּ וַתִּתְחַלְחַ֥ל הַמַּלְכָּ֖ה מְאֹ֑ד וַתִּשְׁלַ֨ח בְּגָדִ֜ים לְהַלְבִּ֣ישׁ אֶֽת־מָרְדֳּכַ֗י וּלְהָסִ֥יר שַׂקּ֛וֹ מֵעָלָ֖יו וְלֹ֥א קִבֵּֽל׃
\vsnum{5}וַתִּקְרָא֩ אֶסְתֵּ֨ר לַהֲתָ֜ךְ מִסָּרִיסֵ֤י הַמֶּ֙לֶךְ֙ אֲשֶׁ֣ר הֶעֱמִ֣יד לְפָנֶ֔יהָ וַתְּצַוֵּ֖הוּ עַֽל־מָרְדֳּכָ֑י לָדַ֥עַת מַה־זֶּ֖ה וְעַל־מַה־זֶּֽה׃
\vsnum{6}וַיֵּצֵ֥א הֲתָ֖ךְ אֶֽל־מָרְדֳּכָ֑י אֶל־רְח֣וֹב הָעִ֔יר אֲשֶׁ֖ר לִפְנֵ֥י שַֽׁעַר־הַמֶּֽלֶךְ׃
\vsnum{7}וַיַּגֶּד־ל֣וֹ מָרְדֳּכַ֔י אֵ֖ת כָּל־אֲשֶׁ֣ר קָרָ֑הוּ וְאֵ֣ת ׀ פָּרָשַׁ֣ת הַכֶּ֗סֶף אֲשֶׁ֨ר אָמַ֤ר הָמָן֙ לִ֠שְׁקוֹל עַל־גִּנְזֵ֥י הַמֶּ֛לֶךְ ביהודיים [בַּיְּהוּדִ֖ים] לְאַבְּדָֽם׃%
\commenta{\normalfont{פָּרָשַׁת הַכֶּסֶף.} פֵּרוּשׁ הַכֶּסֶף:}%endcomment
\vsnum{8}וְאֶת־פַּתְשֶׁ֣גֶן כְּתָֽב־הַ֠דָּת אֲשֶׁר־נִתַּ֨ן בְּשׁוּשָׁ֤ן לְהַשְׁמִידָם֙ נָ֣תַן ל֔וֹ לְהַרְא֥וֹת אֶת־אֶסְתֵּ֖ר וּלְהַגִּ֣יד לָ֑הּ וּלְצַוּ֣וֹת עָלֶ֗יהָ לָב֨וֹא אֶל־הַמֶּ֧לֶךְ לְהִֽתְחַנֶּן־ל֛וֹ וּלְבַקֵּ֥שׁ מִלְּפָנָ֖יו עַל־עַמָּֽהּ׃
\vsnum{9}וַיָּב֖וֹא הֲתָ֑ךְ וַיַּגֵּ֣ד לְאֶסְתֵּ֔ר אֵ֖ת דִּבְרֵ֥י מָרְדֳּכָֽי׃
\vsnum{10}וַתֹּ֤אמֶר אֶסְתֵּר֙ לַהֲתָ֔ךְ וַתְּצַוֵּ֖הוּ אֶֽל־מָרְדֳּכָֽי׃
\vsnum{11}כָּל־עַבְדֵ֣י הַמֶּ֡לֶךְ וְעַם־מְדִינ֨וֹת הַמֶּ֜לֶךְ יֽוֹדְעִ֗ים אֲשֶׁ֣ר כָּל־אִ֣ישׁ וְאִשָּׁ֡ה אֲשֶׁ֣ר יָבֽוֹא־אֶל־הַמֶּלֶךְ֩ אֶל־הֶחָצֵ֨ר הַפְּנִימִ֜ית אֲשֶׁ֣ר לֹֽא־יִקָּרֵ֗א אַחַ֤ת דָּתוֹ֙ לְהָמִ֔ית לְ֠בַד מֵאֲשֶׁ֨ר יֽוֹשִׁיט־ל֥וֹ הַמֶּ֛לֶךְ אֶת־שַׁרְבִ֥יט הַזָּהָ֖ב וְחָיָ֑ה וַאֲנִ֗י לֹ֤א נִקְרֵ֙אתי֙ לָב֣וֹא אֶל־הַמֶּ֔לֶךְ זֶ֖ה שְׁלוֹשִׁ֥ים יֽוֹם׃
\vsnum{12}וַיַּגִּ֣ידוּ לְמָרְדֳּכָ֔י אֵ֖ת דִּבְרֵ֥י אֶסְתֵּֽר׃ (פ)
\vsnum{13}וַיֹּ֥אמֶר מָרְדֳּכַ֖י לְהָשִׁ֣יב אֶל־אֶסְתֵּ֑ר אַל־תְּדַמִּ֣י בְנַפְשֵׁ֔ךְ לְהִמָּלֵ֥ט בֵּית־הַמֶּ֖לֶךְ מִכָּל־הַיְּהוּדִֽים׃%
\commenta{\normalfont{אַל תְּדַמִּי בְנַפְשֵׁךְ.} אַל תַּחְשְׁבִי, כְּמוֹ "וְהָיָה כַּאֲשֶׁר דִּמִּיתִי". "אַל תְּדַמִּי בְנַפְשֵׁךְ": "אַל תְּהִי סְבוּרָה לְהִמָּלֵט בְּיוֹם הַהֲרֵגָה בְּבֵית הַמֶּלֶךְ, שֶׁאֵין אַתְּ רוֹצָה לְסַכֵּן אֶת עַצְמֵךְ עַכְשָׁיו עַל הַסָּפֵק לָבֹא אֶל הַמֶּלֶךְ שֶׁלֹּא בִרְשׁוּת": }%endcomment
\vsnum{14}כִּ֣י אִם־הַחֲרֵ֣שׁ תַּחֲרִישִׁי֮ בָּעֵ֣ת הַזֹּאת֒ רֶ֣וַח וְהַצָּלָ֞ה יַעֲמ֤וֹד לַיְּהוּדִים֙ מִמָּק֣וֹם אַחֵ֔ר וְאַ֥תְּ וּבֵית־אָבִ֖יךְ תֹּאבֵ֑דוּ וּמִ֣י יוֹדֵ֔עַ אִם־לְעֵ֣ת כָּזֹ֔את הִגַּ֖עַתְּ לַמַּלְכֽוּת׃%
\commenta{\normalfont{וּמִי יוֹדֵעַ אִם לְעֵת כָּזֹאת הִגַּעַתְּ לַמַּלְכוּת.} וּמִי יוֹדֵעַ אִם יַחְפֹּץ בָּךְ הַמֶּלֶךְ לַשָּׁנָה הַבָּאָה שֶׁהוּא זְמַן הַהֲרֵגָה:}%endcomment
\vsnum{15}וַתֹּ֥אמֶר אֶסְתֵּ֖ר לְהָשִׁ֥יב אֶֽל־מָרְדֳּכָֽי׃
\vsnum{16}לֵךְ֩ כְּנ֨וֹס אֶת־כָּל־הַיְּהוּדִ֜ים הַֽנִּמְצְאִ֣ים בְּשׁוּשָׁ֗ן וְצ֣וּמוּ עָ֠לַי וְאַל־תֹּאכְל֨וּ וְאַל־תִּשְׁתּ֜וּ שְׁלֹ֤שֶׁת יָמִים֙ לַ֣יְלָה וָי֔וֹם גַּם־אֲנִ֥י וְנַעֲרֹתַ֖י אָצ֣וּם כֵּ֑ן וּבְכֵ֞ן אָב֤וֹא אֶל־הַמֶּ֙לֶךְ֙ אֲשֶׁ֣ר לֹֽא־כַדָּ֔ת וְכַאֲשֶׁ֥ר אָבַ֖דְתִּי אָבָֽדְתִּי׃%
\commenta{\normalfont{אֲשֶׁר לֹא כַדָּת.} שֶׁאֵין דָּת לִכָּנֵס אֲשֶׁר לֹא יִקָּרֵא. וּמִדְרַשׁ אַגָּדָה: "אֲשֶׁר לֹא כַדָּת" שֶׁעַד עַתָּה בְאֹנֶס וְעַכְשָׁיו בְּרָצוֹן: }%endcomment
\vsnum{17}וַֽיַּעֲבֹ֖ר מָרְדֳּכָ֑י וַיַּ֕עַשׂ כְּכֹ֛ל אֲשֶׁר־צִוְּתָ֥ה עָלָ֖יו אֶסְתֵּֽר׃ (ס)%
\commenta{\normalfont{וַיַּעֲבֹר מָרְדְּכָי.} עַל דָּת, לְהִתְעַנּוֹת בְּיוֹם טוֹב רִאשׁוֹן שֶׁל פֶּסַח, שֶׁהִתְעַנָּה י"ד בְּנִיסָן וְט"ו וְט"ז, שֶׁהֲרֵי בְּיוֹם י"ג נִכְתְּבוּ הַסְּפָרִים: }%endcomment
\clearpage}

\newchap{פרק ה}
\twocol{\vsnum{1}וַיְהִ֣י ׀ בַּיּ֣וֹם הַשְּׁלִישִׁ֗י וַתִּלְבַּ֤שׁ אֶסְתֵּר֙ מַלְכ֔וּת וַֽתַּעֲמֹ֞ד בַּחֲצַ֤ר בֵּית־הַמֶּ֙לֶךְ֙ הַפְּנִימִ֔ית נֹ֖כַח בֵּ֣ית הַמֶּ֑לֶךְ וְ֠הַמֶּלֶךְ יוֹשֵׁ֞ב עַל־כִּסֵּ֤א מַלְכוּתוֹ֙ בְּבֵ֣ית הַמַּלְכ֔וּת נֹ֖כַח פֶּ֥תַח הַבָּֽיִת׃%
\commenta{\normalfont{מַלְכוּת.} בִּגְדֵי מַלְכוּת. וְרַבּוֹתֵינוּ אָמְרוּ: שֶׁלְּבָשַׁתָּה רוּחַ הַקֹּדֶשׁ, כְּמָה דְּאַתְּ אָמַר "וְרוּחַ לָבְשָׁה אֶת עֲמָשַׂי": }%endcomment
\vsnum{2}וַיְהִי֩ כִרְא֨וֹת הַמֶּ֜לֶךְ אֶת־אֶסְתֵּ֣ר הַמַּלְכָּ֗ה עֹמֶ֙דֶת֙ בֶּֽחָצֵ֔ר נָשְׂאָ֥ה חֵ֖ן בְּעֵינָ֑יו וַיּ֨וֹשֶׁט הַמֶּ֜לֶךְ לְאֶסְתֵּ֗ר אֶת־שַׁרְבִ֤יט הַזָּהָב֙ אֲשֶׁ֣ר בְּיָד֔וֹ וַתִּקְרַ֣ב אֶסְתֵּ֔ר וַתִּגַּ֖ע בְּרֹ֥אשׁ הַשַּׁרְבִֽיט׃ (ס)
\vsnum{3}וַיֹּ֤אמֶר לָהּ֙ הַמֶּ֔לֶךְ מַה־לָּ֖ךְ אֶסְתֵּ֣ר הַמַּלְכָּ֑ה וּמַה־בַּקָּשָׁתֵ֛ךְ עַד־חֲצִ֥י הַמַּלְכ֖וּת וְיִנָּ֥תֵֽן לָֽךְ׃%
\commenta{\normalfont{עַד חֲצִי הַמַּלְכוּת.} דָּבָר שֶׁהוּא בְאֶמְצַע וּבַחֲצִי הַמַּלְכוּת. הוּא בֵית הַמִּקְדָּשׁ, שֶׁהִתְחִילוּ לִבְנוֹתוֹ בִּימֵי כֹרֶשׁ, וְחָזַר בּוֹ וְצִוָּה לְבַטֵּל הַמְּלָאכָה. וַאֲחַשְׁוֵרוֹשׁ שֶׁעָמַד אַחֲרָיו גַּם הוּא בִּטֵּל הַמְּלָאכָה. וּפְשׁוּטוֹ שֶׁל מִקְרָא: אַף אִם תִּשְׁאֲלִי מִמֶּנִּי חֲצִי הַמַּלְכוּת, אֶתֵּן לָךְ: }%endcomment
\vsnum{4}וַתֹּ֣אמֶר אֶסְתֵּ֔ר אִם־עַל־הַמֶּ֖לֶךְ ט֑וֹב יָב֨וֹא הַמֶּ֤לֶךְ וְהָמָן֙ הַיּ֔וֹם אֶל־הַמִּשְׁתֶּ֖ה אֲשֶׁר־עָשִׂ֥יתִי לֽוֹ׃%
\commenta{\normalfont{יָבוֹא הַמֶּלֶךְ וְהָמָן.} רַבּוֹתֵינוּ אָמְרוּ טְעָמִים הַרְבֵּה בַדָּבָר: מָה רָאֲתָה אֶסְתֵּר שֶׁזִּמְּנָה אֶת הָמָן כְּדֵי לְקַנְאוֹ בַּמֶּלֶךְ וּבַשָּׂרִים, שֶׁהַמֶּלֶךְ יַחְשֹׁב שֶׁהוּא חוֹשֵׁק אֵלֶיהָ וְיַהַרְגֶנּוּ, וְעוֹד טְעָמִים רַבִּים: }%endcomment
\vsnum{5}וַיֹּ֣אמֶר הַמֶּ֔לֶךְ מַהֲרוּ֙ אֶת־הָמָ֔ן לַעֲשׂ֖וֹת אֶת־דְּבַ֣ר אֶסְתֵּ֑ר וַיָּבֹ֤א הַמֶּ֙לֶךְ֙ וְהָמָ֔ן אֶל־הַמִּשְׁתֶּ֖ה אֲשֶׁר־עָשְׂתָ֥ה אֶסְתֵּֽר׃
\vsnum{6}וַיֹּ֨אמֶר הַמֶּ֤לֶךְ לְאֶסְתֵּר֙ בְּמִשְׁתֵּ֣ה הַיַּ֔יִן מַה־שְּׁאֵלָתֵ֖ךְ וְיִנָּ֣תֵֽן לָ֑ךְ וּמַה־בַּקָּשָׁתֵ֛ךְ עַד־חֲצִ֥י הַמַּלְכ֖וּת וְתֵעָֽשׂ׃
\vsnum{7}וַתַּ֥עַן אֶסְתֵּ֖ר וַתֹּאמַ֑ר שְׁאֵלָתִ֖י וּבַקָּשָׁתִֽי׃
\vsnum{8}אִם־מָצָ֨אתִי חֵ֜ן בְּעֵינֵ֣י הַמֶּ֗לֶךְ וְאִם־עַל־הַמֶּ֙לֶךְ֙ ט֔וֹב לָתֵת֙ אֶת־שְׁאֵ֣לָתִ֔י וְלַעֲשׂ֖וֹת אֶת־בַּקָּשָׁתִ֑י יָב֧וֹא הַמֶּ֣לֶךְ וְהָמָ֗ן אֶל־הַמִּשְׁתֶּה֙ אֲשֶׁ֣ר אֶֽעֱשֶׂ֣ה לָהֶ֔ם וּמָחָ֥ר אֶֽעֱשֶׂ֖ה כִּדְבַ֥ר הַמֶּֽלֶךְ׃%
\commenta{\normalfont{וּמָחָר אֶעֱשֶׂה כִּדְבַר הַמֶּלֶךְ.} מַה שֶּׁבִּקַּשְׁתָּ מִמֶּנִּי כָּל הַיָּמִים, לְגַלּוֹת לְךָ אֶת עַמִּי וְאֶת מוֹלַדְתִּי: }%endcomment
\vsnum{9}וַיֵּצֵ֤א הָמָן֙ בַּיּ֣וֹם הַה֔וּא שָׂמֵ֖חַ וְט֣וֹב לֵ֑ב וְכִרְאוֹת֩ הָמָ֨ן אֶֽת־מָרְדֳּכַ֜י בְּשַׁ֣עַר הַמֶּ֗לֶךְ וְלֹא־קָם֙ וְלֹא־זָ֣ע מִמֶּ֔נּוּ וַיִּמָּלֵ֥א הָמָ֛ן עַֽל־מָרְדֳּכַ֖י חֵמָֽה׃
\vsnum{10}וַיִּתְאַפַּ֣ק הָמָ֔ן וַיָּב֖וֹא אֶל־בֵּית֑וֹ וַיִּשְׁלַ֛ח וַיָּבֵ֥א אֶת־אֹהֲבָ֖יו וְאֶת־זֶ֥רֶשׁ אִשְׁתּֽוֹ׃%
\commenta{\normalfont{וַיִּתְאַפַּק.} נִתְחַזֵּק לַעֲמֹד עַל כַּעְסוֹ, כִּי הָיָה יָרֵא לְהִנָּקֵם בְּלֹא רְשׁוּת. וַיִּתְאַפַּק אישטנט"ר בְּלַעַ"ז: }%endcomment
\vsnum{11}וַיְסַפֵּ֨ר לָהֶ֥ם הָמָ֛ן אֶת־כְּב֥וֹד עָשְׁר֖וֹ וְרֹ֣ב בָּנָ֑יו וְאֵת֩ כָּל־אֲשֶׁ֨ר גִּדְּל֤וֹ הַמֶּ֙לֶךְ֙ וְאֵ֣ת אֲשֶׁ֣ר נִשְּׂא֔וֹ עַל־הַשָּׂרִ֖ים וְעַבְדֵ֥י הַמֶּֽלֶךְ׃
\vsnum{12}וַיֹּאמֶר֮ הָמָן֒ אַ֣ף לֹא־הֵבִיאָה֩ אֶסְתֵּ֨ר הַמַּלְכָּ֧ה עִם־הַמֶּ֛לֶךְ אֶל־הַמִּשְׁתֶּ֥ה אֲשֶׁר־עָשָׂ֖תָה כִּ֣י אִם־אוֹתִ֑י וְגַם־לְמָחָ֛ר אֲנִ֥י קָֽרוּא־לָ֖הּ עִם־הַמֶּֽלֶךְ׃
\vsnum{13}וְכָל־זֶ֕ה אֵינֶ֥נּוּ שֹׁוֶ֖ה לִ֑י בְּכָל־עֵ֗ת אֲשֶׁ֨ר אֲנִ֤י רֹאֶה֙ אֶת־מָרְדֳּכַ֣י הַיְּהוּדִ֔י יוֹשֵׁ֖ב בְּשַׁ֥עַר הַמֶּֽלֶךְ׃%
\commenta{\normalfont{אֵינֶנּוּ שֹׁוֶה לִי.} אֵינִי חָשׁ לְכָל הַכָּבוֹד אֲשֶׁר לִי:}%endcomment
\vsnum{14}וַתֹּ֣אמֶר לוֹ֩ זֶ֨רֶשׁ אִשְׁתּ֜וֹ וְכָל־אֹֽהֲבָ֗יו יַֽעֲשׂוּ־עֵץ֮ גָּבֹ֣הַּ חֲמִשִּׁ֣ים אַמָּה֒ וּבַבֹּ֣קֶר ׀ אֱמֹ֣ר לַמֶּ֗לֶךְ וְיִתְל֤וּ אֶֽת־מָרְדֳּכַי֙ עָלָ֔יו וּבֹֽא־עִם־הַמֶּ֥לֶךְ אֶל הַמִּשְׁתֶּ֖ה שָׂמֵ֑חַ וַיִּיטַ֧ב הַדָּבָ֛ר לִפְנֵ֥י הָמָ֖ן וַיַּ֥עַשׂ הָעֵֽץ׃ (פ)
\clearpage}

\newchap{פרק ו}
\twocol{\vsnum{1}בַּלַּ֣יְלָה הַה֔וּא נָדְדָ֖ה שְׁנַ֣ת הַמֶּ֑לֶךְ וַיֹּ֗אמֶר לְהָבִ֞יא אֶת־סֵ֤פֶר הַזִּכְרֹנוֹת֙ דִּבְרֵ֣י הַיָּמִ֔ים וַיִּהְי֥וּ נִקְרָאִ֖ים לִפְנֵ֥י הַמֶּֽלֶךְ׃%
\commenta{\normalfont{נָדְדָה שְׁנַת הַמֶּלֶךְ.} נֵס הָיָה. וְיֵשׁ אוֹמְרִים שָׂם אֶת לִבּוֹ עַל שֶׁזִּמְּנָה אֶסְתֵּר אֶת הָמָן שֶׁמָּא נָתְנָה עֵינֶיהָ בּוֹ וְיַהַרְגֵהוּ:}%endcomment
\vsnum{2}וַיִּמָּצֵ֣א כָת֗וּב אֲשֶׁר֩ הִגִּ֨יד מָרְדֳּכַ֜י עַל־בִּגְתָ֣נָא וָתֶ֗רֶשׁ שְׁנֵי֙ סָרִיסֵ֣י הַמֶּ֔לֶךְ מִשֹּׁמְרֵ֖י הַסַּ֑ף אֲשֶׁ֤ר בִּקְשׁוּ֙ לִשְׁלֹ֣חַ יָ֔ד בַּמֶּ֖לֶךְ אֲחַשְׁוֵרֽוֹשׁ׃
\vsnum{3}וַיֹּ֣אמֶר הַמֶּ֔לֶךְ מַֽה־נַּעֲשָׂ֞ה יְקָ֧ר וּגְדוּלָּ֛ה לְמָרְדֳּכַ֖י עַל־זֶ֑ה וַיֹּ֨אמְר֜וּ נַעֲרֵ֤י הַמֶּ֙לֶךְ֙ מְשָׁ֣רְתָ֔יו לֹא־נַעֲשָׂ֥ה עִמּ֖וֹ דָּבָֽר׃
\vsnum{4}וַיֹּ֥אמֶר הַמֶּ֖לֶךְ מִ֣י בֶחָצֵ֑ר וְהָמָ֣ן בָּ֗א לַחֲצַ֤ר בֵּית־הַמֶּ֙לֶךְ֙ הַחִ֣יצוֹנָ֔ה לֵאמֹ֣ר לַמֶּ֔לֶךְ לִתְלוֹת֙ אֶֽת־מָרְדֳּכַ֔י עַל־הָעֵ֖ץ אֲשֶׁר־הֵכִ֥ין לֽוֹ׃
\vsnum{5}וַיֹּ֨אמְר֜וּ נַעֲרֵ֤י הַמֶּ֙לֶךְ֙ אֵלָ֔יו הִנֵּ֥ה הָמָ֖ן עֹמֵ֣ד בֶּחָצֵ֑ר וַיֹּ֥אמֶר הַמֶּ֖לֶךְ יָבֽוֹא׃
\vsnum{6}וַיָּבוֹא֮ הָמָן֒ וַיֹּ֤אמֶר לוֹ֙ הַמֶּ֔לֶךְ מַה־לַעֲשׂ֕וֹת בָּאִ֕ישׁ אֲשֶׁ֥ר הַמֶּ֖לֶךְ חָפֵ֣ץ בִּיקָר֑וֹ וַיֹּ֤אמֶר הָמָן֙ בְּלִבּ֔וֹ לְמִ֞י יַחְפֹּ֥ץ הַמֶּ֛לֶךְ לַעֲשׂ֥וֹת יְקָ֖ר יוֹתֵ֥ר מִמֶּֽנִּי׃
\vsnum{7}וַיֹּ֥אמֶר הָמָ֖ן אֶל־הַמֶּ֑לֶךְ אִ֕ישׁ אֲשֶׁ֥ר הַמֶּ֖לֶךְ חָפֵ֥ץ בִּיקָרֽוֹ׃
\vsnum{8}יָבִ֙יאוּ֙ לְב֣וּשׁ מַלְכ֔וּת אֲשֶׁ֥ר לָֽבַשׁ־בּ֖וֹ הַמֶּ֑לֶךְ וְס֗וּס אֲשֶׁ֨ר רָכַ֤ב עָלָיו֙ הַמֶּ֔לֶךְ וַאֲשֶׁ֥ר נִתַּ֛ן כֶּ֥תֶר מַלְכ֖וּת בְּרֹאשֽׁוֹ׃
\vsnum{9}וְנָת֨וֹן הַלְּב֜וּשׁ וְהַסּ֗וּס עַל־יַד־אִ֞ישׁ מִשָּׂרֵ֤י הַמֶּ֙לֶךְ֙ הַֽפַּרְתְּמִ֔ים וְהִלְבִּ֙ישׁוּ֙ אֶת־הָאִ֔ישׁ אֲשֶׁ֥ר הַמֶּ֖לֶךְ חָפֵ֣ץ בִּֽיקָר֑וֹ וְהִרְכִּיבֻ֤הוּ עַל־הַסּוּס֙ בִּרְח֣וֹב הָעִ֔יר וְקָרְא֣וּ לְפָנָ֔יו כָּ֚כָה יֵעָשֶׂ֣ה לָאִ֔ישׁ אֲשֶׁ֥ר הַמֶּ֖לֶךְ חָפֵ֥ץ בִּיקָרֽוֹ׃%
\commenta{\normalfont{וְנָתוֹן הַלְּבוּשׁ וְהַסּוּס עַל יַד אִישׁ.} וְאֶת הַכֶּתֶר לֹא הִזְכִּיר, שֶׁרָאָה עֵינוֹ שֶׁל מֶלֶךְ רָעָה עַל שֶׁאָמַר שֶׁיִּתְּנוּ הַכֶּתֶר בְּרֹאשׁ אָדָם: }%endcomment
\vsnum{10}וַיֹּ֨אמֶר הַמֶּ֜לֶךְ לְהָמָ֗ן מַ֠הֵר קַ֣ח אֶת־הַלְּב֤וּשׁ וְאֶת־הַסּוּס֙ כַּאֲשֶׁ֣ר דִּבַּ֔רְתָּ וַֽעֲשֵׂה־כֵן֙ לְמָרְדֳּכַ֣י הַיְּהוּדִ֔י הַיּוֹשֵׁ֖ב בְּשַׁ֣עַר הַמֶּ֑לֶךְ אַל־תַּפֵּ֣ל דָּבָ֔ר מִכֹּ֖ל אֲשֶׁ֥ר דִּבַּֽרְתָּ׃
\vsnum{11}וַיִּקַּ֤ח הָמָן֙ אֶת־הַלְּב֣וּשׁ וְאֶת־הַסּ֔וּס וַיַּלְבֵּ֖שׁ אֶֽת־מָרְדֳּכָ֑י וַיַּרְכִּיבֵ֙הוּ֙ בִּרְח֣וֹב הָעִ֔יר וַיִּקְרָ֣א לְפָנָ֔יו כָּ֚כָה יֵעָשֶׂ֣ה לָאִ֔ישׁ אֲשֶׁ֥ר הַמֶּ֖לֶךְ חָפֵ֥ץ בִּיקָרֽוֹ׃
\vsnum{12}וַיָּ֥שָׁב מָרְדֳּכַ֖י אֶל־שַׁ֣עַר הַמֶּ֑לֶךְ וְהָמָן֙ נִדְחַ֣ף אֶל־בֵּית֔וֹ אָבֵ֖ל וַחֲפ֥וּי רֹֽאשׁ׃%
\commenta{\normalfont{וַיָּשָׁב מָרְדְּכַי.} לְשַׂקּוֹ וּלְתַעֲנִיתוֹ:}%endcomment
\vsnum{13}וַיְסַפֵּ֨ר הָמָ֜ן לְזֶ֤רֶשׁ אִשְׁתּוֹ֙ וּלְכָל־אֹ֣הֲבָ֔יו אֵ֖ת כָּל־אֲשֶׁ֣ר קָרָ֑הוּ וַיֹּ֩אמְרוּ֩ ל֨וֹ חֲכָמָ֜יו וְזֶ֣רֶשׁ אִשְׁתּ֗וֹ אִ֣ם מִזֶּ֣רַע הַיְּהוּדִ֡ים מָרְדֳּכַ֞י אֲשֶׁר֩ הַחִלּ֨וֹתָ לִנְפֹּ֤ל לְפָנָיו֙ לֹא־תוּכַ֣ל ל֔וֹ כִּֽי־נָפ֥וֹל תִּפּ֖וֹל לְפָנָֽיו׃%
\commenta{\normalfont{אֲשֶׁר הַחִלּוֹתָ לִנְפֹּל וגו'.} אָמְרָה: אֻמָּה זוּ נִמְשְׁלוּ לְכוֹכָבִים וּלְעָפָר. כְּשֶׁהֵם יוֹרְדִים יוֹרְדִים עַד לֶעָפָר, וּכְשֶׁהֵם עוֹלִים עוֹלִים עַד לָרָקִיעַ וְעַד הַכּוֹכָבִים: }%endcomment
\vsnum{14}עוֹדָם֙ מְדַבְּרִ֣ים עִמּ֔וֹ וְסָרִיסֵ֥י הַמֶּ֖לֶךְ הִגִּ֑יעוּ וַיַּבְהִ֙לוּ֙ לְהָבִ֣יא אֶת־הָמָ֔ן אֶל־הַמִּשְׁתֶּ֖ה אֲשֶׁר־עָשְׂתָ֥ה אֶסְתֵּֽר׃
\clearpage}

\newchap{פרק ז}
\twocol{\vsnum{1}וַיָּבֹ֤א הַמֶּ֙לֶךְ֙ וְהָמָ֔ן לִשְׁתּ֖וֹת עִם־אֶסְתֵּ֥ר הַמַּלְכָּֽה׃
\vsnum{2}וַיֹּאמֶר֩ הַמֶּ֨לֶךְ לְאֶסְתֵּ֜ר גַּ֣ם בַּיּ֤וֹם הַשֵּׁנִי֙ בְּמִשְׁתֵּ֣ה הַיַּ֔יִן מַה־שְּׁאֵלָתֵ֛ךְ אֶסְתֵּ֥ר הַמַּלְכָּ֖ה וְתִנָּ֣תֵֽן לָ֑ךְ וּמַה־בַּקָּשָׁתֵ֛ךְ עַד־חֲצִ֥י הַמַּלְכ֖וּת וְתֵעָֽשׂ׃
\vsnum{3}וַתַּ֨עַן אֶסְתֵּ֤ר הַמַּלְכָּה֙ וַתֹּאמַ֔ר אִם־מָצָ֨אתִי חֵ֤ן בְּעֵינֶ֙יךָ֙ הַמֶּ֔לֶךְ וְאִם־עַל־הַמֶּ֖לֶךְ ט֑וֹב תִּנָּֽתֶן־לִ֤י נַפְשִׁי֙ בִּשְׁאֵ֣לָתִ֔י וְעַמִּ֖י בְּבַקָּשָׁתִֽי׃%
\commenta{\normalfont{תִּנָּתֶן לִי נַפְשִׁי.} שֶׁלֹּא אֵהָרֵג בִּשְׁלשָׁה עָשָׂר בַּאֲדָר שֶׁגָּזַרְתָּ גְזֵרַת הֲרֵגָה עַל עַמִּי וּמוֹלַדְתִּי:}%endcomment
\vsnum{4}כִּ֤י נִמְכַּ֙רְנוּ֙ אֲנִ֣י וְעַמִּ֔י לְהַשְׁמִ֖יד לַהֲר֣וֹג וּלְאַבֵּ֑ד וְ֠אִלּוּ לַעֲבָדִ֨ים וְלִשְׁפָח֤וֹת נִמְכַּ֙רְנוּ֙ הֶחֱרַ֔שְׁתִּי כִּ֣י אֵ֥ין הַצָּ֛ר שֹׁוֶ֖ה בְּנֵ֥זֶק הַמֶּֽלֶךְ׃ (ס)%
\commenta{\normalfont{כִּי אֵין הַצָּר שֹׁוֶה בְּנֵזֶק הַמֶּלֶךְ.} אֵינֶנּוּ חוֹשֵׁשׁ בְּנֵזֶק הַמֶּלֶךְ שֶׁאִלּוּ רָדַף אַחַר הֲנָאָתְךָ, הָיָה לוֹ לוֹמַר: מְכֹר אוֹתָם לַעֲבָדִים וְלִשְׁפָחוֹת וְקַבֵּל הַמָּמוֹן אוֹ הַחֲיֵה אוֹתָם לִהְיוֹת לְךָ לַעֲבָדִים הֵם וְזַרְעָם: }%endcomment
\vsnum{5}וַיֹּ֙אמֶר֙ הַמֶּ֣לֶךְ אֲחַשְׁוֵר֔וֹשׁ וַיֹּ֖אמֶר לְאֶסְתֵּ֣ר הַמַּלְכָּ֑ה מִ֣י ה֥וּא זֶה֙ וְאֵֽי־זֶ֣ה ה֔וּא אֲשֶׁר־מְלָא֥וֹ לִבּ֖וֹ לַעֲשׂ֥וֹת כֵּֽן׃%
\commenta{\normalfont{וַיֹּאמֶר הַמֶּלֶךְ אֲחַשְׁוֵרוֹשׁ וַיֹּאמֶר לְאֶסְתֵּר הַמַּלְכָּה.} כָּל מָקוֹם שֶׁנֶּאֱמַר: "וַיֹּאמֶר, וַיֹּאמֶר" שְׁנֵי פְעָמִים, אֵינוֹ אֶלָּא לְמִדְרָשׁ, וּמִדְרָשׁוֹ שֶׁל זֶה: בַּתְּחִלָּה הָיָה מְדַבֵּר עִמָּהּ עַל יְדֵי שָׁלִיחַ. עַכְשָׁיו שֶׁיָּדַע שֶׁמִּמִּשְׁפַּחַת מְלָכִים הִיא, דִּבֵּר עִמָּהּ הוּא בְעַצְמוֹ: }%endcomment
\vsnum{6}וַתֹּ֣אמֶר־אֶסְתֵּ֔ר אִ֚ישׁ צַ֣ר וְאוֹיֵ֔ב הָמָ֥ן הָרָ֖ע הַזֶּ֑ה וְהָמָ֣ן נִבְעַ֔ת מִלִּפְנֵ֥י הַמֶּ֖לֶךְ וְהַמַּלְכָּֽה׃
\vsnum{7}וְהַמֶּ֜לֶךְ קָ֤ם בַּחֲמָתוֹ֙ מִמִּשְׁתֵּ֣ה הַיַּ֔יִן אֶל־גִּנַּ֖ת הַבִּיתָ֑ן וְהָמָ֣ן עָמַ֗ד לְבַקֵּ֤שׁ עַל־נַפְשׁוֹ֙ מֵֽאֶסְתֵּ֣ר הַמַּלְכָּ֔ה כִּ֣י רָאָ֔ה כִּֽי־כָלְתָ֥ה אֵלָ֛יו הָרָעָ֖ה מֵאֵ֥ת הַמֶּֽלֶךְ׃%
\commenta{\normalfont{כִּי כָלְתָה.} נִגְמְרָה הָרָעָה וְהַשִּׂנְאָה וְהַנְּקָמָה:}%endcomment
\vsnum{8}וְהַמֶּ֡לֶךְ שָׁב֩ מִגִּנַּ֨ת הַבִּיתָ֜ן אֶל־בֵּ֣ית ׀ מִשְׁתֵּ֣ה הַיַּ֗יִן וְהָמָן֙ נֹפֵ֔ל עַל־הַמִּטָּה֙ אֲשֶׁ֣ר אֶסְתֵּ֣ר עָלֶ֔יהָ וַיֹּ֣אמֶר הַמֶּ֔לֶךְ הֲ֠גַם לִכְבּ֧וֹשׁ אֶת־הַמַּלְכָּ֛ה עִמִּ֖י בַּבָּ֑יִת הַדָּבָ֗ר יָצָא֙ מִפִּ֣י הַמֶּ֔לֶךְ וּפְנֵ֥י הָמָ֖ן חָפֽוּ׃ (ס)%
\commenta{\normalfont{וְהָמָן נֹפֵל.} הַמַּלְאָךְ דְּחָפוֹ:}%endcomment
\vsnum{9}וַיֹּ֣אמֶר חַ֠רְבוֹנָה אֶחָ֨ד מִן־הַסָּרִיסִ֜ים לִפְנֵ֣י הַמֶּ֗לֶךְ גַּ֣ם הִנֵּה־הָעֵ֣ץ אֲשֶׁר־עָשָׂ֪ה הָמָ֟ן לְֽמָרְדֳּכַ֞י אֲשֶׁ֧ר דִּבֶּר־ט֣וֹב עַל־הַמֶּ֗לֶךְ עֹמֵד֙ בְּבֵ֣ית הָמָ֔ן גָּבֹ֖הַּ חֲמִשִּׁ֣ים אַמָּ֑ה וַיֹּ֥אמֶר הַמֶּ֖לֶךְ תְּלֻ֥הוּ עָלָֽיו׃%
\commenta{\normalfont{גַּם הִנֵּה הָעֵץ.} "גַּם" רָעָה אַחֶרֶת עָשָׂה שֶׁהֵכִין הָעֵץ לִתְלוֹת אוֹהֲבוֹ שֶׁל מֶלֶךְ שֶׁהִצִּיל הַמֶּלֶךְ מִסַּם הַמָּוֶת: }%endcomment
\vsnum{10}וַיִּתְלוּ֙ אֶת־הָמָ֔ן עַל־הָעֵ֖ץ אֲשֶׁר־הֵכִ֣ין לְמָרְדֳּכָ֑י וַחֲמַ֥ת הַמֶּ֖לֶךְ שָׁכָֽכָה׃ (פ)
\clearpage}

\newchap{פרק ח}
\twocol{\vsnum{1}בַּיּ֣וֹם הַה֗וּא נָתַ֞ן הַמֶּ֤לֶךְ אֲחַשְׁוֵרוֹשׁ֙ לְאֶסְתֵּ֣ר הַמַּלְכָּ֔ה אֶת־בֵּ֥ית הָמָ֖ן צֹרֵ֣ר היהודיים [הַיְּהוּדִ֑ים] וּמָרְדֳּכַ֗י בָּ֚א לִפְנֵ֣י הַמֶּ֔לֶךְ כִּֽי־הִגִּ֥ידָה אֶסְתֵּ֖ר מַ֥ה הוּא־לָֽהּ׃%
\commenta{\normalfont{מָה הוּא לָהּ.} אֵיךְ הוּא קָרוֹב לָהּ:}%endcomment
\vsnum{2}וַיָּ֨סַר הַמֶּ֜לֶךְ אֶת־טַבַּעְתּ֗וֹ אֲשֶׁ֤ר הֶֽעֱבִיר֙ מֵֽהָמָ֔ן וַֽיִּתְּנָ֖הּ לְמָרְדֳּכָ֑י וַתָּ֧שֶׂם אֶסְתֵּ֛ר אֶֽת־מָרְדֳּכַ֖י עַל־בֵּ֥ית הָמָֽן׃ (פ)
\vsnum{3}וַתּ֣וֹסֶף אֶסְתֵּ֗ר וַתְּדַבֵּר֙ לִפְנֵ֣י הַמֶּ֔לֶךְ וַתִּפֹּ֖ל לִפְנֵ֣י רַגְלָ֑יו וַתֵּ֣בְךְּ וַתִּתְחַנֶּן־ל֗וֹ לְהַֽעֲבִיר֙ אֶת־רָעַת֙ הָמָ֣ן הָֽאֲגָגִ֔י וְאֵת֙ מַֽחֲשַׁבְתּ֔וֹ אֲשֶׁ֥ר חָשַׁ֖ב עַל־הַיְּהוּדִֽים׃%
\commenta{\normalfont{לְהַעֲבִיר אֶת רָעַת הָמָן.} שֶׁלֹּא תִתְקַיֵּם עֲצָתוֹ הָרָעָה:}%endcomment
\vsnum{4}וַיּ֤וֹשֶׁט הַמֶּ֙לֶךְ֙ לְאֶסְתֵּ֔ר אֵ֖ת שַׁרְבִ֣ט הַזָּהָ֑ב וַתָּ֣קָם אֶסְתֵּ֔ר וַֽתַּעֲמֹ֖ד לִפְנֵ֥י הַמֶּֽלֶךְ׃
\vsnum{5}וַ֠תֹּאמֶר אִם־עַל־הַמֶּ֨לֶךְ ט֜וֹב וְאִם־מָצָ֧אתִי חֵ֣ן לְפָנָ֗יו וְכָשֵׁ֤ר הַדָּבָר֙ לִפְנֵ֣י הַמֶּ֔לֶךְ וְטוֹבָ֥ה אֲנִ֖י בְּעֵינָ֑יו יִכָּתֵ֞ב לְהָשִׁ֣יב אֶת־הַסְּפָרִ֗ים מַחֲשֶׁ֜בֶת הָמָ֤ן בֶּֽן־הַמְּדָ֙תָא֙ הָאֲגָגִ֔י אֲשֶׁ֣ר כָּתַ֗ב לְאַבֵּד֙ אֶת־הַיְּהוּדִ֔ים אֲשֶׁ֖ר בְּכָל־מְדִינ֥וֹת הַמֶּֽלֶךְ׃
\vsnum{6}כִּ֠י אֵיכָכָ֤ה אוּכַל֙ וְֽרָאִ֔יתִי בָּרָעָ֖ה אֲשֶׁר־יִמְצָ֣א אֶת־עַמִּ֑י וְאֵֽיכָכָ֤ה אוּכַל֙ וְֽרָאִ֔יתִי בְּאָבְדַ֖ן מוֹלַדְתִּֽי׃ (ס)
\vsnum{7}וַיֹּ֨אמֶר הַמֶּ֤לֶךְ אֲחַשְׁוֵרֹשׁ֙ לְאֶסְתֵּ֣ר הַמַּלְכָּ֔ה וּֽלְמָרְדֳּכַ֖י הַיְּהוּדִ֑י הִנֵּ֨ה בֵית־הָמָ֜ן נָתַ֣תִּי לְאֶסְתֵּ֗ר וְאֹתוֹ֙ תָּל֣וּ עַל־הָעֵ֔ץ עַ֛ל אֲשֶׁר־שָׁלַ֥ח יָד֖וֹ ביהודיים [בַּיְּהוּדִֽים׃]%
\commenta{\normalfont{הִנֵּה בֵית הָמָן וגו'.} וּמֵעַתָּה הַכֹּל רוֹאִים שֶׁאֲנִי חָפֵץ בָּכֶם, וְכָל מַה שֶּׁתּאמְרוּ יַאֲמִינוּ הַכֹּל שֶׁמֵּאִתִּי הוּא. לְפִיכָךְ, אֵין צְרִיכִין אַתֶּם לַהֲשִׁיבָם, אֶלָּא "כִּתְבוּ" סְפָרִים אֲחֵרִים "כַּטּוֹב בְּעֵינֵיכֶם": }%endcomment
\vsnum{8}וְ֠אַתֶּם כִּתְב֨וּ עַל־הַיְּהוּדִ֜ים כַּטּ֤וֹב בְּעֵֽינֵיכֶם֙ בְּשֵׁ֣ם הַמֶּ֔לֶךְ וְחִתְמ֖וּ בְּטַבַּ֣עַת הַמֶּ֑לֶךְ כִּֽי־כְתָ֞ב אֲשֶׁר־נִכְתָּ֣ב בְּשֵׁם־הַמֶּ֗לֶךְ וְנַחְתּ֛וֹם בְּטַבַּ֥עַת הַמֶּ֖לֶךְ אֵ֥ין לְהָשִֽׁיב׃%
\commenta{\normalfont{אֵין לְהָשִׁיב.} אֵין נָאֶה לַהֲשִׁיבוֹ וְלַעֲשׂוֹת כְּתַב הַמֶּלֶךְ בְּזִיּוּף:}%endcomment
\vsnum{9}וַיִּקָּרְא֣וּ סֹפְרֵֽי־הַמֶּ֣לֶךְ בָּֽעֵת־הַ֠הִיא בַּחֹ֨דֶשׁ הַשְּׁלִישִׁ֜י הוּא־חֹ֣דֶשׁ סִיוָ֗ן בִּשְׁלוֹשָׁ֣ה וְעֶשְׂרִים֮ בּוֹ֒ וַיִּכָּתֵ֣ב כְּֽכָל־אֲשֶׁר־צִוָּ֣ה מָרְדֳּכַ֣י אֶל־הַיְּהוּדִ֡ים וְאֶ֣ל הָאֲחַשְׁדַּרְפְּנִֽים־וְהַפַּחוֹת֩ וְשָׂרֵ֨י הַמְּדִינ֜וֹת אֲשֶׁ֣ר ׀ מֵהֹ֣דּוּ וְעַד־כּ֗וּשׁ שֶׁ֣בַע וְעֶשְׂרִ֤ים וּמֵאָה֙ מְדִינָ֔ה מְדִינָ֤ה וּמְדִינָה֙ כִּכְתָבָ֔הּ וְעַ֥ם וָעָ֖ם כִּלְשֹׁנ֑וֹ וְאֶ֨ל־הַיְּהוּדִ֔ים כִּכְתָבָ֖ם וְכִלְשׁוֹנָֽם׃%
\commenta{\normalfont{כִּכְתָבָהּ.} בָּאוֹתִיּוֹת שֶׁלָּהּ:}%endcomment
\vsnum{10}וַיִּכְתֹּ֗ב בְּשֵׁם֙ הַמֶּ֣לֶךְ אֲחַשְׁוֵרֹ֔שׁ וַיַּחְתֹּ֖ם בְּטַבַּ֣עַת הַמֶּ֑לֶךְ וַיִּשְׁלַ֣ח סְפָרִ֡ים בְּיַד֩ הָרָצִ֨ים בַּסּוּסִ֜ים רֹכְבֵ֤י הָרֶ֙כֶשׁ֙ הָֽאֲחַשְׁתְּרָנִ֔ים בְּנֵ֖י הָֽרַמָּכִֽים׃%
\commenta{\normalfont{בְּיַד הָרָצִים.} רוֹכְבֵי סוּסִים שֶׁצִּוָּה לָהֶם לָרוּץ:}%endcomment
\vsnum{11}אֲשֶׁר֩ נָתַ֨ן הַמֶּ֜לֶךְ לַיְּהוּדִ֣ים ׀ אֲשֶׁ֣ר בְּכָל־עִיר־וָעִ֗יר לְהִקָּהֵל֮ וְלַעֲמֹ֣ד עַל־נַפְשָׁם֒ לְהַשְׁמִיד֩ וְלַהֲרֹ֨ג וּלְאַבֵּ֜ד אֶת־כָּל־חֵ֨יל עַ֧ם וּמְדִינָ֛ה הַצָּרִ֥ים אֹתָ֖ם טַ֣ף וְנָשִׁ֑ים וּשְׁלָלָ֖ם לָבֽוֹז׃%
\commenta{\normalfont{וּשְׁלָלָם לָבוֹז.} כַּאֲשֶׁר נִכְתַּב בָּרִאשׁוֹנוֹת וְהֵם בַּבִּזָּה לֹא שָׁלְחוּ אֶת יָדָם, שֶׁהֶרְאוּ לַכֹּל שֶׁלֹּא נַעֲשָׂה לְשֵׁם מָמוֹן: }%endcomment
\vsnum{12}בְּי֣וֹם אֶחָ֔ד בְּכָל־מְדִינ֖וֹת הַמֶּ֣לֶךְ אֲחַשְׁוֵר֑וֹשׁ בִּשְׁלוֹשָׁ֥ה עָשָׂ֛ר לְחֹ֥דֶשׁ שְׁנֵים־עָשָׂ֖ר הוּא־חֹ֥דֶשׁ אֲדָֽר׃
\vsnum{13}פַּתְשֶׁ֣גֶן הַכְּתָ֗ב לְהִנָּ֤תֵֽן דָּת֙ בְּכָל־מְדִינָ֣ה וּמְדִינָ֔ה גָּל֖וּי לְכָל־הָעַמִּ֑ים וְלִהְי֨וֹת היהודיים [הַיְּהוּדִ֤ים] עתודים [עֲתִידִים֙] לַיּ֣וֹם הַזֶּ֔ה לְהִנָּקֵ֖ם מֵאֹיְבֵיהֶֽם׃%
\commenta{\normalfont{פַּתְשֶׁגֶן.} אִגֶּרֶת מְפֹרָשׁ:}%endcomment
\vsnum{14}הָרָצִ֞ים רֹכְבֵ֤י הָרֶ֙כֶשׁ֙ הָֽאֲחַשְׁתְּרָנִ֔ים יָֽצְא֛וּ מְבֹהָלִ֥ים וּדְחוּפִ֖ים בִּדְבַ֣ר הַמֶּ֑לֶךְ וְהַדָּ֥ת נִתְּנָ֖ה בְּשׁוּשַׁ֥ן הַבִּירָֽה׃ (פ)%
\commenta{\normalfont{מְבֹהָלִים.} מְמַהֲרִים אוֹתָם לַעֲשׂוֹת מְהֵרָה, לְפִי שֶׁלֹּא הָיָה לָהֶם פְּנַאי, שֶׁהָיָה לָהֶם לְהַקְדִּים רָצִים הָרִאשׁוֹנִים לְהַעֲבִירָם: }%endcomment
\vsnum{15}וּמָרְדֳּכַ֞י יָצָ֣א ׀ מִלִּפְנֵ֣י הַמֶּ֗לֶךְ בִּלְב֤וּשׁ מַלְכוּת֙ תְּכֵ֣לֶת וָח֔וּר וַעֲטֶ֤רֶת זָהָב֙ גְּדוֹלָ֔ה וְתַכְרִ֥יךְ בּ֖וּץ וְאַרְגָּמָ֑ן וְהָעִ֣יר שׁוּשָׁ֔ן צָהֲלָ֖ה וְשָׂמֵֽחָה׃%
\commenta{\normalfont{וְתַכְרִיךְ בּוּץ.} מַעֲטֵה בוּץ, טַלִּית הֶעָשׂוּי לְהִתְעַטֵּף: }%endcomment
\vsnum{16}לַיְּהוּדִ֕ים הָֽיְתָ֥ה אוֹרָ֖ה וְשִׂמְחָ֑ה וְשָׂשֹׂ֖ן וִיקָֽר׃
\vsnum{17}וּבְכָל־מְדִינָ֨ה וּמְדִינָ֜ה וּבְכָל־עִ֣יר וָעִ֗יר מְקוֹם֙ אֲשֶׁ֨ר דְּבַר־הַמֶּ֤לֶךְ וְדָתוֹ֙ מַגִּ֔יעַ שִׂמְחָ֤ה וְשָׂשׂוֹן֙ לַיְּהוּדִ֔ים מִשְׁתֶּ֖ה וְי֣וֹם ט֑וֹב וְרַבִּ֞ים מֵֽעַמֵּ֤י הָאָ֙רֶץ֙ מִֽתְיַהֲדִ֔ים כִּֽי־נָפַ֥ל פַּֽחַד־הַיְּהוּדִ֖ים עֲלֵיהֶֽם׃%
\commenta{\normalfont{מִתְיַהֲדִים.} מִתְגַּיְּרִים:}%endcomment
\clearpage}

\newchap{פרק ט}
\twocol{\vsnum{1}וּבִשְׁנֵים֩ עָשָׂ֨ר חֹ֜דֶשׁ הוּא־חֹ֣דֶשׁ אֲדָ֗ר בִּשְׁלוֹשָׁ֨ה עָשָׂ֥ר יוֹם֙ בּ֔וֹ אֲשֶׁ֨ר הִגִּ֧יעַ דְּבַר־הַמֶּ֛לֶךְ וְדָת֖וֹ לְהֵעָשׂ֑וֹת בַּיּ֗וֹם אֲשֶׁ֨ר שִׂבְּר֜וּ אֹיְבֵ֤י הַיְּהוּדִים֙ לִשְׁל֣וֹט בָּהֶ֔ם וְנַהֲפ֣וֹךְ ה֔וּא אֲשֶׁ֨ר יִשְׁלְט֧וּ הַיְּהוּדִ֛ים הֵ֖מָּה בְּשֹׂנְאֵיהֶֽם׃
\vsnum{2}נִקְהֲל֨וּ הַיְּהוּדִ֜ים בְּעָרֵיהֶ֗ם בְּכָל־מְדִינוֹת֙ הַמֶּ֣לֶךְ אֳחַשְׁוֵר֔וֹשׁ לִשְׁלֹ֣חַ יָ֔ד בִּמְבַקְשֵׁ֖י רָֽעָתָ֑ם וְאִישׁ֙ לֹא־עָמַ֣ד לִפְנֵיהֶ֔ם כִּֽי־נָפַ֥ל פַּחְדָּ֖ם עַל־כָּל־הָעַמִּֽים׃
\vsnum{3}וְכָל־שָׂרֵ֨י הַמְּדִינ֜וֹת וְהָאֲחַשְׁדַּרְפְּנִ֣ים וְהַפַּח֗וֹת וְעֹשֵׂ֤י הַמְּלָאכָה֙ אֲשֶׁ֣ר לַמֶּ֔לֶךְ מְנַשְּׂאִ֖ים אֶת־הַיְּהוּדִ֑ים כִּֽי־נָפַ֥ל פַּֽחַד־מָרְדֳּכַ֖י עֲלֵיהֶֽם׃%
\commenta{\normalfont{וְעֹשֵׂי הַמְּלָאכָה.} אוֹתָם שֶׁהָיוּ מְמֻנִּים לַעֲשׂוֹת צָרְכֵי הַמֶּלֶךְ:}%endcomment
\vsnum{4}כִּֽי־גָ֤דוֹל מָרְדֳּכַי֙ בְּבֵ֣ית הַמֶּ֔לֶךְ וְשָׁמְע֖וֹ הוֹלֵ֣ךְ בְּכָל־הַמְּדִינ֑וֹת כִּֽי־הָאִ֥ישׁ מָרְדֳּכַ֖י הוֹלֵ֥ךְ וְגָדֽוֹל׃ (פ)
\vsnum{5}וַיַּכּ֤וּ הַיְּהוּדִים֙ בְּכָל־אֹ֣יְבֵיהֶ֔ם מַכַּת־חֶ֥רֶב וְהֶ֖רֶג וְאַבְדָ֑ן וַיַּֽעֲשׂ֥וּ בְשֹׂנְאֵיהֶ֖ם כִּרְצוֹנָֽם׃
\vsnum{6}וּבְשׁוּשַׁ֣ן הַבִּירָ֗ה הָרְג֤וּ הַיְּהוּדִים֙ וְאַבֵּ֔ד חֲמֵ֥שׁ מֵא֖וֹת אִֽישׁ׃
\vsnum{7}וְאֵ֧ת ׀ פַּרְשַׁנְדָּ֛תָא וְאֵ֥ת ׀ דַּֽלְפ֖וֹן וְאֵ֥ת ׀ אַסְפָּֽתָא׃
\vsnum{8}וְאֵ֧ת ׀ פּוֹרָ֛תָא וְאֵ֥ת ׀ אֲדַלְיָ֖א וְאֵ֥ת ׀ אֲרִידָֽתָא׃
\vsnum{9}וְאֵ֤ת ׀ פַּרְמַ֙שְׁתָּא֙ וְאֵ֣ת ׀ אֲרִיסַ֔י וְאֵ֥ת ׀ אֲרִדַ֖י וְאֵ֥ת ׀ וַיְזָֽתָא׃
\vsnum{10}עֲ֠שֶׂרֶת בְּנֵ֨י הָמָ֧ן בֶּֽן־הַמְּדָ֛תָא צֹרֵ֥ר הַיְּהוּדִ֖ים הָרָ֑גוּ וּבַ֨בִּזָּ֔ה לֹ֥א שָׁלְח֖וּ אֶת־יָדָֽם׃%
\commenta{\normalfont{עֲשֶׂרֶת בְּנֵי הָמָן.} רָאִיתִי בְסֵדֶר עוֹלָם אֵלּוּ י' שֶׁכָּתְבוּ שִׂטְנָה עַל יְהוּדָה וִירוּשָׁלָיִם כְּמוֹ שֶׁכָּתוּב בְּסֵפֶר עֶזְרָא "וּבְמַלְכוּת אֲחַשְׁוֵרוֹשׁ בִּתְחִלַּת מַלְכוּתוֹ כָּתְבוּ שִׂטְנָה עַל ישְׁבֵי יְהוּדָה וִירוּשָׁלָיִם". וּמָה הִיא הַשִׂטְנָה? לְבַטֵּל הָעוֹלִים מִן הַגּוֹלָה בִּימֵי כוֹרֶשׁ, שֶׁהִתְחִילוּ לִבְנוֹת אֶת הַבַּיִת וְהִלְשִׁינוּ עֲלֵיהֶם הַכּוּתִים וְהֶחֱדִילוּם וּכְשֶׁמֵּת כּוֹרֶשׁ וּמָלַךְ אֲחַשְׁוֵרוֹשׁ וְהִתְנַשֵׂא הָמָן, דָּאַג שֶׁלֹּא יַעַסְקוּ אוֹתָן שֶׁבִּירוּשָׁלַיִם בַּבִּנְיָן וְשָׁלְחוּ בְשֵׁם אֲחַשְׁוֵרוֹשׁ לְשָׂרֵי עֵבֶר הַנָּהָר לְבַטְּלָן: }%endcomment
\vsnum{11}בַּיּ֣וֹם הַה֗וּא בָּ֣א מִסְפַּ֧ר הַֽהֲרוּגִ֛ים בְּשׁוּשַׁ֥ן הַבִּירָ֖ה לִפְנֵ֥י הַמֶּֽלֶךְ׃ (ס)
\vsnum{12}וַיֹּ֨אמֶר הַמֶּ֜לֶךְ לְאֶסְתֵּ֣ר הַמַּלְכָּ֗ה בְּשׁוּשַׁ֣ן הַבִּירָ֡ה הָרְגוּ֩ הַיְּהוּדִ֨ים וְאַבֵּ֜ד חֲמֵ֧שׁ מֵא֣וֹת אִ֗ישׁ וְאֵת֙ עֲשֶׂ֣רֶת בְּנֵֽי־הָמָ֔ן בִּשְׁאָ֛ר מְדִינ֥וֹת הַמֶּ֖לֶךְ מֶ֣ה עָשׂ֑וּ וּמַה־שְּׁאֵֽלָתֵךְ֙ וְיִנָּ֣תֵֽן לָ֔ךְ וּמַה־בַּקָּשָׁתֵ֥ךְ ע֖וֹד וְתֵעָֽשׂ׃
\vsnum{13}וַתֹּ֤אמֶר אֶסְתֵּר֙ אִם־עַל־הַמֶּ֣לֶךְ ט֔וֹב יִנָּתֵ֣ן גַּם־מָחָ֗ר לַיְּהוּדִים֙ אֲשֶׁ֣ר בְּשׁוּשָׁ֔ן לַעֲשׂ֖וֹת כְּדָ֣ת הַיּ֑וֹם וְאֵ֛ת עֲשֶׂ֥רֶת בְּנֵֽי־הָמָ֖ן יִתְל֥וּ עַל־הָעֵֽץ׃%
\commenta{\normalfont{וְאֵת עֲשֶׂרֶת בְּנֵי הָמָן יִתְלוּ עַל הָעֵץ.} אוֹתָן שֶׁנֶּהֶרְגוּ:}%endcomment
\vsnum{14}וַיֹּ֤אמֶר הַמֶּ֙לֶךְ֙ לְהֵֽעָשׂ֣וֹת כֵּ֔ן וַתִּנָּתֵ֥ן דָּ֖ת בְּשׁוּשָׁ֑ן וְאֵ֛ת עֲשֶׂ֥רֶת בְּנֵֽי־הָמָ֖ן תָּלֽוּ׃%
\commenta{\normalfont{וַתִּנָּתֵן דָּת.} נִגְזַר חֹק מֵאֵת הַמֶּלֶךְ:}%endcomment
\vsnum{15}וַיִּֽקָּהֲל֞וּ היהודיים [הַיְּהוּדִ֣ים] אֲשֶׁר־בְּשׁוּשָׁ֗ן גַּ֠ם בְּי֣וֹם אַרְבָּעָ֤ה עָשָׂר֙ לְחֹ֣דֶשׁ אֲדָ֔ר וַיַּֽהַרְג֣וּ בְשׁוּשָׁ֔ן שְׁלֹ֥שׁ מֵא֖וֹת אִ֑ישׁ וּבַ֨בִּזָּ֔ה לֹ֥א שָׁלְח֖וּ אֶת־יָדָֽם׃
\vsnum{16}וּשְׁאָ֣ר הַיְּהוּדִ֡ים אֲשֶׁר֩ בִּמְדִינ֨וֹת הַמֶּ֜לֶךְ נִקְהֲל֣וּ ׀ וְעָמֹ֣ד עַל־נַפְשָׁ֗ם וְנ֙וֹחַ֙ מֵאֹ֣יְבֵיהֶ֔ם וְהָרֹג֙ בְּשֹׂ֣נְאֵיהֶ֔ם חֲמִשָּׁ֥ה וְשִׁבְעִ֖ים אָ֑לֶף וּבַ֨בִּזָּ֔ה לֹ֥א שָֽׁלְח֖וּ אֶת־יָדָֽם׃
\vsnum{17}בְּיוֹם־שְׁלֹשָׁ֥ה עָשָׂ֖ר לְחֹ֣דֶשׁ אֲדָ֑ר וְנ֗וֹחַ בְּאַרְבָּעָ֤ה עָשָׂר֙ בּ֔וֹ וְעָשֹׂ֣ה אֹת֔וֹ י֖וֹם מִשְׁתֶּ֥ה וְשִׂמְחָֽה׃
\vsnum{18}והיהודיים [וְהַיְּהוּדִ֣ים] אֲשֶׁר־בְּשׁוּשָׁ֗ן נִקְהֲלוּ֙ בִּשְׁלֹשָׁ֤ה עָשָׂר֙ בּ֔וֹ וּבְאַרְבָּעָ֥ה עָשָׂ֖ר בּ֑וֹ וְנ֗וֹחַ בַּחֲמִשָּׁ֤ה עָשָׂר֙ בּ֔וֹ וְעָשֹׂ֣ה אֹת֔וֹ י֖וֹם מִשְׁתֶּ֥ה וְשִׂמְחָֽה׃
\vsnum{19}עַל־כֵּ֞ן הַיְּהוּדִ֣ים הפרוזים [הַפְּרָזִ֗ים] הַיֹּשְׁבִים֮ בְּעָרֵ֣י הַפְּרָזוֹת֒ עֹשִׂ֗ים אֵ֠ת י֣וֹם אַרְבָּעָ֤ה עָשָׂר֙ לְחֹ֣דֶשׁ אֲדָ֔ר שִׂמְחָ֥ה וּמִשְׁתֶּ֖ה וְי֣וֹם ט֑וֹב וּמִשְׁל֥וֹחַ מָנ֖וֹת אִ֥ישׁ לְרֵעֵֽהוּ׃ (פ)%
\commenta{\normalfont{הַפְּרָזִים.} שֶׁאֵינָם יוֹשְׁבִים בְּעָרֵי חוֹמָה בְּאַרְבָּעָה עָשָׂר, וּמֻקָּפִין חוֹמָה בְּט"ו כְּשׁוּשַׁן. וְהֶקֵּף זֶה צָרִיךְ שֶׁיִּהְיֶה מִימוֹת יְהוֹשֻׁעַ בִּן נוּן. כַּךְ דָּרְשׁוּ וְלָמְדוּ רַבּוֹתֵינוּ: }%endcomment
\vsnum{20}וַיִּכְתֹּ֣ב מָרְדֳּכַ֔י אֶת־הַדְּבָרִ֖ים הָאֵ֑לֶּה וַיִּשְׁלַ֨ח סְפָרִ֜ים אֶל־כָּל־הַיְּהוּדִ֗ים אֲשֶׁר֙ בְּכָל־מְדִינוֹת֙ הַמֶּ֣לֶךְ אֲחַשְׁוֵר֔וֹשׁ הַקְּרוֹבִ֖ים וְהָרְחוֹקִֽים׃%
\commenta{\normalfont{וַיִּכְתּב מָרְדְּכַי.} הִיא הַמְּגִלָּה הַזֹּאת כְּמוֹת שֶׁהִיא:}%endcomment
\vsnum{21}לְקַיֵּם֮ עֲלֵיהֶם֒ לִהְי֣וֹת עֹשִׂ֗ים אֵ֠ת י֣וֹם אַרְבָּעָ֤ה עָשָׂר֙ לְחֹ֣דֶשׁ אֲדָ֔ר וְאֵ֛ת יוֹם־חֲמִשָּׁ֥ה עָשָׂ֖ר בּ֑וֹ בְּכָל־שָׁנָ֖ה וְשָׁנָֽה׃
\vsnum{22}כַּיָּמִ֗ים אֲשֶׁר־נָ֨חוּ בָהֶ֤ם הַיְּהוּדִים֙ מֵא֣וֹיְבֵיהֶ֔ם וְהַחֹ֗דֶשׁ אֲשֶׁר֩ נֶהְפַּ֨ךְ לָהֶ֤ם מִיָּגוֹן֙ לְשִׂמְחָ֔ה וּמֵאֵ֖בֶל לְי֣וֹם ט֑וֹב לַעֲשׂ֣וֹת אוֹתָ֗ם יְמֵי֙ מִשְׁתֶּ֣ה וְשִׂמְחָ֔ה וּמִשְׁל֤וֹחַ מָנוֹת֙ אִ֣ישׁ לְרֵעֵ֔הוּ וּמַתָּנ֖וֹת לָֽאֶבְיוֹנִֽים׃
\vsnum{23}וְקִבֵּל֙ הַיְּהוּדִ֔ים אֵ֥ת אֲשֶׁר־הֵחֵ֖לּוּ לַעֲשׂ֑וֹת וְאֵ֛ת אֲשֶׁר־כָּתַ֥ב מָרְדֳּכַ֖י אֲלֵיהֶֽם׃
\vsnum{24}כִּי֩ הָמָ֨ן בֶּֽן־הַמְּדָ֜תָא הָֽאֲגָגִ֗י צֹרֵר֙ כָּל־הַיְּהוּדִ֔ים חָשַׁ֥ב עַל־הַיְּהוּדִ֖ים לְאַבְּדָ֑ם וְהִפִּ֥יל פּוּר֙ ה֣וּא הַגּוֹרָ֔ל לְהֻמָּ֖ם וּֽלְאַבְּדָֽם׃%
\commenta{\normalfont{כִּי הָמָן בֶּן הַמְּדָתָא.} חָשַׁב לְהֻמָּם וּלְאַבְּדָם:}%endcomment
\vsnum{25}וּבְבֹאָהּ֮ לִפְנֵ֣י הַמֶּלֶךְ֒ אָמַ֣ר עִם־הַסֵּ֔פֶר יָשׁ֞וּב מַחֲשַׁבְתּ֧וֹ הָרָעָ֛ה אֲשֶׁר־חָשַׁ֥ב עַל־הַיְּהוּדִ֖ים עַל־רֹאשׁ֑וֹ וְתָל֥וּ אֹת֛וֹ וְאֶת־בָּנָ֖יו עַל־הָעֵֽץ׃%
\commenta{\normalfont{וּבְבֹאָהּ.} אֶסְתֵּר אֶל הַמֶּלֶךְ לְהִתְחַנֶּן לוֹ:}%endcomment
\vsnum{26}עַל־כֵּ֡ן קָֽרְאוּ֩ לַיָּמִ֨ים הָאֵ֤לֶּה פוּרִים֙ עַל־שֵׁ֣ם הַפּ֔וּר עַל־כֵּ֕ן עַל־כָּל־דִּבְרֵ֖י הָאִגֶּ֣רֶת הַזֹּ֑את וּמָֽה־רָא֣וּ עַל־כָּ֔כָה וּמָ֥ה הִגִּ֖יעַ אֲלֵיהֶֽם׃%
\commenta{\normalfont{עַל כֵּן עַל כָּל דִּבְרֵי הָאִגֶּרֶת הַזֹּאת.} נִקְבְּעוּ הַיָּמִים הָאֵלֶּה וּלְכַךְ נִכְתְּבָה לָדַעַת דּוֹרוֹת הַבָּאִים:}%endcomment
\vsnum{27}קִיְּמ֣וּ וקבל [וְקִבְּל֣וּ] הַיְּהוּדִים֩ ׀ עֲלֵיהֶ֨ם ׀ וְעַל־זַרְעָ֜ם וְעַ֨ל כָּל־הַנִּלְוִ֤ים עֲלֵיהֶם֙ וְלֹ֣א יַעֲב֔וֹר לִהְי֣וֹת עֹשִׂ֗ים אֵ֣ת שְׁנֵ֤י הַיָּמִים֙ הָאֵ֔לֶּה כִּכְתָבָ֖ם וְכִזְמַנָּ֑ם בְּכָל־שָׁנָ֖ה וְשָׁנָֽה׃%
\commenta{\normalfont{הַנִּלְוִים עֲלֵיהֶם.} גֵּרִים הָעֲתִידִים לְהִתְגַּיֵּר:}%endcomment
\vsnum{28}וְהַיָּמִ֣ים הָ֠אֵלֶּה נִזְכָּרִ֨ים וְנַעֲשִׂ֜ים בְּכָל־דּ֣וֹר וָד֗וֹר מִשְׁפָּחָה֙ וּמִשְׁפָּחָ֔ה מְדִינָ֥ה וּמְדִינָ֖ה וְעִ֣יר וָעִ֑יר וִימֵ֞י הַפּוּרִ֣ים הָאֵ֗לֶּה לֹ֤א יַֽעַבְרוּ֙ מִתּ֣וֹךְ הַיְּהוּדִ֔ים וְזִכְרָ֖ם לֹא־יָס֥וּף מִזַּרְעָֽם׃ (ס)%
\commenta{\normalfont{נִזְכָּרִים.} בִּקְרִיאַת הַמְּגִלָּה:}%endcomment
\vsnum{29}וַ֠תִּכְתֹּב אֶסְתֵּ֨ר הַמַּלְכָּ֧ה בַת־אֲבִיחַ֛יִל וּמָרְדֳּכַ֥י הַיְּהוּדִ֖י אֶת־כָּל־תֹּ֑קֶף לְקַיֵּ֗ם אֵ֣ת אִגֶּ֧רֶת הַפּוּרִ֛ים הַזֹּ֖את הַשֵּׁנִֽית׃%
\commenta{\normalfont{אֶת כָּל תּקֶף.} תָּקְפּוֹ שֶׁל נֵס, שֶׁל אֲחַשְׁוֵרוֹשׁ וְשֶׁל הָמָן וְשֶׁל מָרְדְּכַי וְשֶׁל אֶסְתֵּר: }%endcomment
\vsnum{30}וַיִּשְׁלַ֨ח סְפָרִ֜ים אֶל־כָּל־הַיְּהוּדִ֗ים אֶל־שֶׁ֨בַע וְעֶשְׂרִ֤ים וּמֵאָה֙ מְדִינָ֔ה מַלְכ֖וּת אֲחַשְׁוֵר֑וֹשׁ דִּבְרֵ֥י שָׁל֖וֹם וֶאֱמֶֽת׃
\vsnum{31}לְקַיֵּ֡ם אֵת־יְמֵי֩ הַפֻּרִ֨ים הָאֵ֜לֶּה בִּזְמַנֵּיהֶ֗ם כַּאֲשֶׁר֩ קִיַּ֨ם עֲלֵיהֶ֜ם מָרְדֳּכַ֤י הַיְּהוּדִי֙ וְאֶסְתֵּ֣ר הַמַּלְכָּ֔ה וְכַאֲשֶׁ֛ר קִיְּמ֥וּ עַל־נַפְשָׁ֖ם וְעַל־זַרְעָ֑ם דִּבְרֵ֥י הַצֹּמ֖וֹת וְזַעֲקָתָֽם׃
\vsnum{32}וּמַאֲמַ֣ר אֶסְתֵּ֔ר קִיַּ֕ם דִּבְרֵ֥י הַפֻּרִ֖ים הָאֵ֑לֶּה וְנִכְתָּ֖ב בַּסֵּֽפֶר׃ (פ)%
\commenta{\normalfont{וּמַאֲמַר אֶסְתֵּר קִיַּם וגו'.} אֶסְתֵּר בִּקְשָׁה מֵאֵת חַכְמֵי הַדּוֹר לְקָבְעָהּ וְלִכְתּב סֵפֶר זֶה עִם שְׁאָר הַכְּתוּבִים. וְזֶהוּ "וְנִכְתָּב בַּסֵּפֶר": }%endcomment
\clearpage}

\newchap{פרק י}
\twocol{\vsnum{1}וַיָּשֶׂם֩ הַמֶּ֨לֶךְ אחשרש [אֲחַשְׁוֵר֧וֹשׁ ׀] מַ֛ס עַל־הָאָ֖רֶץ וְאִיֵּ֥י הַיָּֽם׃
\vsnum{2}וְכָל־מַעֲשֵׂ֤ה תָקְפּוֹ֙ וּגְב֣וּרָת֔וֹ וּפָרָשַׁת֙ גְּדֻלַּ֣ת מָרְדֳּכַ֔י אֲשֶׁ֥ר גִּדְּל֖וֹ הַמֶּ֑לֶךְ הֲלוֹא־הֵ֣ם כְּתוּבִ֗ים עַל־סֵ֙פֶר֙ דִּבְרֵ֣י הַיָּמִ֔ים לְמַלְכֵ֖י מָדַ֥י וּפָרָֽס׃
\vsnum{3}כִּ֣י ׀ מָרְדֳּכַ֣י הַיְּהוּדִ֗י מִשְׁנֶה֙ לַמֶּ֣לֶךְ אֲחַשְׁוֵר֔וֹשׁ וְגָדוֹל֙ לַיְּהוּדִ֔ים וְרָצ֖וּי לְרֹ֣ב אֶחָ֑יו דֹּרֵ֥שׁ טוֹב֙ לְעַמּ֔וֹ וְדֹבֵ֥ר שָׁל֖וֹם לְכָל־זַרְעֽוֹ׃%
\commenta{\normalfont{לְרֹב אֶחָיו.} וְלֹא לְכָל אֶחָיו, מְלַמֵּד שֶׁפֵּרְשׁוּ מִמֶּנּוּ מִקְצַת סַנְהֶדְרִין, לְפִי שֶׁנַּעֲשָׂה קָרוֹב לַמַּלְכוּת וְהָיָה בָטֵל מִתַּלְמוּדוֹ: }%endcomment
}
\addpart{משנה מגילה}\renewcommand{\partname}[1]{משנה מגילה}
\fancyhead[CO]{\chapname}
\fancyhead[CE]{\partname}
\clearpage
\newchap{פרק א}
\hebeng{\vsnum{1}מגילה נקראת באחד עשר בשנים עשר. בשלשה עשר. בארבעה עשר. בחמשה עשר. לא פחות ולא יותר. כרכין המוקפין חומה מימות יהושע בן נון. קורין בחמשה עשר. כפרים ועיירות גדולות. קורין בארבעה עשר. אלא שהכפרים מקדימין ליום הכניסה: 
}{\vsnumeng{1}The Megillah is read on the eleventh, the twelfth, the thirteenth, the fourteenth, and the fifteenth {[of Adar]}, never earlier and never later. Cities which have been walled since the days of Joshua ben Nun read on the fifteenth; villages and large towns read on the fourteenth, Except that villages move the reading up to the day of gathering.}%

\hebeng{\vsnum{2}כיצד. חל להיות יום ארבעה עשר בשני. כפרים ועיירות גדולות קורין בו ביום. ומוקפות חומה למחר. חל להיות בשלישי או ברביעי כפרים מקדימין ליום הכניסה. ועיירות גדולות קורין בו ביום. ומוקפות חומה למחר. חל להיות בחמישי. כפרים ועיירות גדולות קורין בו ביום. ומוקפות חומה למחר. חל להיות ערב שבת. כפרים מקדימין ליום הכניסה. ועיירות גדולות ומוקפות חומה קורין בו ביום. חל להיות בשבת. כפרים ועיירות גדולות מקדימין וקורין ליום הכניסה. ומוקפות חומה למחר. חל להיות אחר השבת כפרים מקדימין ליום הכניסה. ועיירות גדולות קורין בו ביום. ומוקפות חומה למחר: 
}{\vsnumeng{2}How so? If the fourteenth {[of Adar]} falls on Monday, the villages and large towns read on that day and the walled places on the next day. If it falls on Tuesday or on Wednesday, the villages move the reading up to the day of gathering, the large towns read on that day, and the walled places on the next day. If it falls on Thursday, the villages and large towns read on that day and the walled places on the next day. If it falls on Friday, the villages move the reading up to the day of gathering and the large towns and walled places read on that day. If it falls on Shabbat, the villages and large towns move the reading up to the day of gathering, and the walled places read on the next day. If it falls on Sunday, the villages move the reading up to the day of gathering, the large towns read on that day, and the walled cities on the day following.}%

\hebeng{\vsnum{3}איזו היא עיר גדולה כל שיש בה עשרה בטלנים. פחות מכאן הרי זה כפר. באלו אמרו מקדימין ולא מאחרין. אבל זמן עצי כהנים. ותשעה באב. חגיגה. והקהל. מאחרין ולא מקדימין. אף על פי שאמרו מקדימין ולא מאחרין. מותרין בהספד ובתעניות ומתנות לאביונים. אמר רבי יהודה אימתי מקום שנכנסין בשני ובחמישי. אבל מקום שאין נכנסין לא בשני ולא בחמישי. אין קורין אותה אלא בזמנה: 
}{\vsnumeng{3}What is considered a large town? One which has in it ten idle men. One that has fewer is considered a village. In respect of these they said that they should be moved up but not postponed. But with regard to the bringing the wood for the priests, the {[fast of]} Tisha B’Av, the hagigah, and assembling the people they postpone {[until after Shabbat]} and they do not move them up. Although they said that they should be moved up but not postponed, it is permissible to mourn, to fast, and to distribute gifts to the poor {[on these earlier days]}. Rabbi Judah said: When is this so? In a place where people gather on Mondays and Thursdays, but in places where people do not gather on Mondays and Thursdays, the Megillah is read only on its proper day.}%

\hebeng{\vsnum{4}קראו את המגילה באדר הראשון. ונתעברה השנה. קורין אותה באדר שני. אין בין אדר הראשון. לאדר השני. אלא קריאת המגילה ומתנות לאביונים: 
}{\vsnumeng{4}If they read the Megillah during the first Adar and the year was intercalated (a month was added), it is read again in the second Adar. There is no difference between the first Adar and the second Adar except the reading of the Megillah and the giving of gifts to the poor.}%

\hebeng{\vsnum{5}אין בין יום טוב לשבת. אלא אוכל נפש בלבד. אין בין שבת ליום הכפורים. אלא שזה זדונו בידי אדם. וזה זדונו בכרת: 
}{\vsnumeng{5}There is no difference between Shabbat and Yom Tov except the preparation of food. There is no difference between Shabbat and Yom HaKippurim except that the deliberate violation of the one is punished by a human court and the deliberate violation of the other by karet.}%

\hebeng{\vsnum{6}אין בין המודר הנאה מחבירו למודר ממנו מאכל. אלא דריסת הרגל. וכלים שאין עושין בהן אוכל נפש. אין בין נדרים לנדבות. אלא שהנדרים חייב באחריותן. ונדבות. אינו חייב באחריותן: 
}{\vsnumeng{6}There is no difference between one who is prohibited by vow from benefiting from his fellow and one who is prohibited by vow from {[benefiting from]} his food, except in the matter of setting foot {[on his property]} and of vessels which are not used for {[preparing]} food. There is no difference between vowed offerings and freewill-offerings except that he is responsible for vowed offering but not responsible for freewill-offerings.}%

\hebeng{\vsnum{7}אין בין זב הרואה שתי ראיות. לרואה שלש. אלא קרבן. אין בין מצורע מוסגר למצורע מוחלט. אלא פריעה ופרימה. אין בין טהור מתוך הסגר. לטהור מתוך החלט. אלא תגלחת וצפרים: 
}{\vsnumeng{7}There is no difference between a zav who sees {[genital discharge]} twice and one who sees three, except the sacrifice. There is no difference between a metzora who is under observation and one declared to be a definite metzora except the disheveling of hair and tearing the clothes. There is no difference between a metzora who has been declared clean after being under observation and one who has been declared clean after having been a definite metzorah except shaving and {[sacrificing]} the birds.}%

\hebeng{\vsnum{8}אין בין ספרים לתפילין ומזוזות. אלא שהספרים נכתבין בכל לשון. ותפילין ומזוזות אינן נכתבות אלא אשורית. רבן שמעון בן גמליאל אומר. אף בספרים לא התירו שיכתבו אלא יונית: 
}{\vsnumeng{8}There is no difference between scrolls {[of the Tanakh]} and tefillin and mezuzahs except that scrolls may be written in any language whereas tefillin and mezuzahs may be written only in Assyrian. Rabban Shimon ben Gamaliel says that scrolls {[of the Tanakh]} were permitted {[by the sages]} to be written only in Greek.}%

\hebeng{\vsnum{9}אין בין כהן משוח בשמן המשחה. למרובה בגדים. אלא פר הבא על כל המצות. אין בין כהן משמש לכהן שעבר. אלא פר יום הכפורים ועשירית האיפה: 
}{\vsnumeng{9}There is no difference between a priest anointed with the oil of anointment and one who {[only]} wears the additional garments except for the bull which is offered for the {[unwitting transgression of]} any of the commandments. There is no difference between a serving {[high]} priest and one whose time has passed except the bull of Yom HaKippurim and the tenth of the ephah.}%

\hebeng{\vsnum{10}אין בין במה גדולה לבמה קטנה. אלא פסחים. זה הכלל כל שהוא נידר ונידב. קרב בבמה. וכל שאינו לא נידר ולא נידב. אינו קרב בבמה: 
}{\vsnumeng{10}There is no difference between a great altar and a small altar except for the pesach offering. This is the general principle: any animal which can be brought as a vow-offering or a freewill offering may be brought on a {[small]} altar, any animal which is not the object of a vow or a freewill-offering may not be brought on a {[small]} altar.}%

\hebeng{\vsnum{11}אין בין שילה לירושלם. אלא שבשילה אוכלים קדשים קלים. ומעשר שני. בכל הרואה. ובירושלים לפנים מן החומה. וכאן וכאן קדשי קדשים נאכלים לפנים מן הקלעים. קדושת שילה יש אחריה היתר. וקדושת ירושלים אין אחריה היתר: 
}{\vsnumeng{11}There is no difference between Shiloh and Jerusalem except that in Shiloh sacrifices of lesser sanctity and second tithe could be eaten anywhere within sight {[of the town]}, whereas in Jerusalem {[they had to be eaten]} within the walls. In both places the most holy sacrifices were eaten within the curtains. After the sanctification of Shiloh there is permission {[for altars]}, but after the sanctification of Jerusalem there is no such permission.}%

\clearpage
\blockcomment{ברטנורא על משנה מגילה}{\normalfont{מגילה נקראת באחד עשר בשנים עשר.} פעמים בזה ופעמים בזה כדמפרש ואזיל:\\\normalfont{חל להיות.} י״ד. בע״ש. עיירות ומוקפין קורין בו ביום, שאין קריאת מגילה בשבת, גזירה שמא יטלנה בידו ויעבירנה ד׳ אמות ברשות הרבים, ואם יאחרנה עד אחד בשבת הוה ליה י״ו, ואמר קרא ולא יעבור. ואע״פ שבני הכרכים קורין המגילה בי״ד כשחל ט״ו להיות בשבת, מ״מ אין קורין ויבא עמלק אלא בשבת שהוא יום ט״ו, ומפטירים פקדתי, ושואלין ודורשין בהלכות פורים כל אותה שבת. וסעודת פורים אית דאמרי דעבדי לה ביום י״ד שבו קורין את המגילה, ואית דאמרי דמאחרין אותה לאחר השבת. והכי משמע בירושלמי דסעודת פורים שחל להיות בשבת מאחרין ולא מקדימין. ולכ״ע אין עושים אותה בשבת:\\\normalfont{עשרה בטלנין.} של בית הכנסת, שבטלים ממלאכתן ונזונים משל צבור כדי להיות מצויין תמיד בשעת התפילה בבהכ״נ:\\\normalfont{אין בין אדר ראשון לאדר שני וכו׳} הכי קאמר אין בין ארבעה עשר וחמשה עשר של אדר ראשון לארבעה עשר וחמשה עשר של אדר שני, אלא מקרא מגילה ומתנות לאביונים, הא לענין הספד ותענית זה וזה שוין: \\\normalfont{אין בין יו״ט לשבת אלא אוכל נפש בלבד.} מתניתין ב״ש היא [ביצה יב.] דאמרי אין מוציאין את הקטן ולא את הלולב ולא את הס״ת לרשות הרבים כיון שאין בהם צורך אוכל נפש. ואין כן הלכה אלא כדברי בית הלל דאמרי מתוך שהותרה הוצאה לצורך אכילה הותרה נמי שלא לצורך אכילה. ואיכא נמי מילי אחרינא שאסורים בשבת ומותרים ביו״ט אע״פ שאינם צורך אוכל נפש, כגון משילין פירות דרך ארובה ביו״ט, אבל לא בשבת:\\\normalfont{אין בין המודר הנאה.} אין מודר הנאה חמור ממודר מאכל אלא דריסת הרגל, שמודר הנאה אסור לו ליכנס בתוך שלו ומודר מאכל מותר:\\\normalfont{שתי ראיות.} בין ביום א׳ בין בשני ימים רצופים. וכן ג׳ ראיות בין ביום א׳ בין בג׳ ימים רצופים, או ב׳ ביום א׳ וא׳ למחר:\\\normalfont{נכתבין בכל לשון.} בכתב של כל אומה ובלשון של כל אומה:\\\normalfont{למרובה בבגדים.} כהנים ששמשו בבית שני, ואף בבית ראשון מן יאשיהו ואילך שנגנזה צלוחית של שמן המשחה בימיו, ולא היו כהנים גדולים אלא בלבישת הבגדים בלבד:\\\normalfont{אין בין במה גדולה.} בשעת היתר הבמות מיירי. במה גדולה היא במת צבור שהיתה בנוב וגבעון:\\\normalfont{בכל הרואה.} בכל מקום שיכול לראות משם את שילה:\\\n}\clearpage %endcomment
\newchap{פרק ב}
\hebeng{\vsnum{1}הקורא את המגילה למפרע לא יצא. קראה על פה. קראה תרגום. בכל לשון. לא יצא. אבל קורין אותה ללועזות בלעז. והלועז ששמע אשורית יצא: }{\vsnumeng{1}If one reads the Megillah out of order, he has not fulfilled his obligation. If he reads it by heart, if he reads it in a translation {[targum]}, or in any other language, he has not fulfilled his obligation. But they may read it to those who do not understand Hebrew in a language other than Hebrew. One who doesn’t understand Hebrew who heard it in Assyrian {[Hebrew]}, has fulfilled his obligation.}%

\hebeng{\vsnum{2}קראה סירוגין ומתנמנם יצא. היה כותבה. דורשה ומגיהה. אם כיון לבו יצא. ואם לאו לא יצא. היתה כתובה בסם. ובסיקרא. ובקומוס ובקנקנתום. על הנייר. ועל הדפתרא. לא יצא. עד שתהא כתובה אשורית על הספר ובדיו: }{\vsnumeng{2}If one reads it with breaks, or naps {[in between readings]}, he has fulfilled his obligation. If he was copying it, explaining it or correcting {[a scroll of Esther]}, if he directed his heart, he has fulfilled his obligation, but if not, he has not fulfilled his obligation. If it was written with arsenic, with red chalk, with gum or with sulfate of copper, or on paper or on scratch paper, he has not fulfilled his obligation, unless it is written in Assyrian on parchment and in ink.}%

\hebeng{\vsnum{3}בן עיר שהלך לכרך. ובן כרך שהלך לעיר. אם עתיד לחזור למקומו. קורא כמקומו. ואם לאו קורא עמהן. ומהיכן קורא אדם את המגילה ויוצא בה ידי חובתו. רבי מאיר אומר כולה. רבי יהודה אומר מאיש יהודי. רבי יוסי אומר מאחר הדברים האלה: }{\vsnumeng{3}A resident of a town who has gone to a walled city or a resident of a walled city who has gone to a town, if he is to return to his own place he reads according to the rule of his own place, and if not reads with them. From where does a man read the Megillah and thereby fulfill his obligation? Rabbi Meir says: all of it. Rabbi Judah says: from “There was a Jew” (Esther 2:5). Rabbi Yose says: from “After these things” (Esther 3:1).}%

\hebeng{\vsnum{4}הכל כשרין לקרות את המגילה חוץ מחרש שוטה וקטן. רבי יהודה מכשיר בקטן. אין קורין את המגילה. ולא מלין. ולא טובלין. ולא מזין. וכן שומרת יום כנגד יום לא תטבול עד שתנץ החמה וכולן שעשו משעלה עמוד השחר. כשר: }{\vsnumeng{4}All are qualified to read the Megillah except a deaf person, an idiot and a minor. Rabbi Judah qualifies a minor. They do not read the Megillah, nor circumcise, nor go to the mikveh, nor sprinkling {[purificatory waters]}, and similarly a woman keeping day for day should not take a ritual bath until the sun has risen. But if any of these things is done after dawn, it is valid.}%

\hebeng{\vsnum{5}כל היום כשר לקריאת המגילה. ולקריאת ההלל. ולתקיעת שופר. ולנטילת לולב. ולתפלת המוספין. ולמוספין. ולוידוי הפרים. ולוידוי המעשר. ולוידוי יום הכפורים. לסמיכה. לשחיטה. לתנופה. להגשה. לקמיצה. ולהקטרה. למליקה. ולקבלה. ולהזייה. ולהשקיית סוטה. ולעריפת העגלה. ולטהרת המצורע. }{\vsnumeng{5}The whole day is a valid time for reading the Megillah; reciting Hallel; for the blowing of the shofar; for taking up the lulav; for the Musaf prayer; for Musaf sacrifices; for confession over the oxen; for the confession over the tithe; for the confession of sins on Yom HaKippurim; for laying on of hands; for slaughtering {[the sacrifices]}; for waving {[them]}; for bringing near {[the vessel with the minhah-offering to the altar]}; for taking a handful; for placing it on the fire; for pinching off {[the head of a bird-offering]}; for receiving the blood {[in a vessel]}; for sprinkling {[the blood on the altar]}; for making the sotah drink {[the bitter waters]}; for breaking the neck of the heifer; and for purifying the metzora.}%

\hebeng{\vsnum{6}כל הלילה כשר לקצירת העומר. ולהקטר חלבים ואיברים. זה הכלל. דבר שמצותו ביום. כשר כל היום. דבר שמצותו בלילה. כשר כל הלילה: }{\vsnumeng{6}The whole night is valid for reaping the Omer and for burning fat and limbs {[on the altar]}. This is the general principle: any matter whose commandment is during the day, is valid all day and any matter whose commandment is at night is valid all night.}%

\clearpage
\blockcomment{ברטנורא על משנה מגילה}{\normalfont{הקורא את המגלה למפרע לא יצא.} דכתיב (אסתר ט׳:כ״ח) והימים האלה נזכרים ונעשים, מה עשיית הימים א״א למפרע, דא״א שיהא חמשה עשר קודם ארבעה עשר, אף זכירה שהיא קריאת המגילה, למפרע לא:\\\normalfont{סירוגין.} שקרא מעט ושהא, וחזר וקרא מעט ושהא, אפילו שהא יותר מכדי לגמור את כולה, יצא:\\\normalfont{בן עיר.} שזמנו בי״ד:\\\normalfont{הכל כשרים לקרות את המגילה.} הכל לאתויי נשים:\\\normalfont{ולוידוי הפרים.} פר כהן משיח, ופר העלם דבר של צבור, שמתודים עליהם חטאם שהביאום עליו:\\\normalfont{ולהקטר חלבים ואברים.} מותרי תמיד של בין הערבים, דכתיב בהו (ויקרא ו׳:ב׳) היא העולה על מוקדה על המזבח כל הלילה:\\\n}\clearpage %endcomment
\newchap{פרק ג}
\hebeng{\vsnum{1}בני העיר שמכרו רחובה של עיר לוקחין בדמיו בית הכנסת. בית הכנסת לוקחין תיבה. תיבה לוקחין מטפחות. מטפחות לוקחין ספרים. ספרים לוקחים תורה. אבל אם מכרו תורה לא יקחו ספרים. ספרים לא יקחו מטפחות. מטפחות לא יקחו תיבה. תיבה לא יקחו בית הכנסת. בית הכנסת לא יקחו את הרחוב. וכן במותריהן. אין מוכרין את של רבים ליחיד. מפני שמורידין אותו מקדושתו. דברי רבי יהודה. אמרו לו אם כן אף לא מעיר גדולה לעיר קטנה: }{\vsnumeng{1}Townspeople who sold the town square, they may buy with the proceeds a synagogue. {[If they sold]} a synagogue, they may buy with the proceeds an ark. {[If they sold]} an ark they may buy covers {[for scrolls]}. {[If they sold]} covers, they may buy scrolls {[of the Tanakh]}. {[If they sold]} scrolls they may buy a Torah. But if they sold a Torah they may not buy with the proceeds scrolls {[of the Tanakh]}. If {[they sold]} scrolls they may not buy covers. If {[they sold]} covers they may not buy an ark. If {[they sold]} an ark they may not buy a synagogue. If {[they sold]} a synagogue they may not buy a town square. The same applies to any money left over. They may not sell {[something]} belonging to a community because this lowers its sanctity, the words of Rabbi Yehuda. They said to him: if so, it should not be allowed to sell from a larger town to a smaller one.}%

\hebeng{\vsnum{2}אין מוכרין בית הכנסת אלא על תנאי. שאם ירצו יחזירוהו. דברי רבי מאיר. וחכמים אומרים מוכרים אותו ממכר עולם. חוץ מארבעה דברים למרחץ. ולבורסקי. ולטבילה. ולבית המים. רבי יהודה אומר מוכרין אותו לשם חצר והלוקח מה שירצה יעשה: }{\vsnumeng{2}They may not sell a synagogue except with the stipulation that it may be bought back whenever they want, the words of Rabbi Meir. But the sages say: they may sell it in perpetuity, except for four purposes for it to become one of four things: a bathhouse, a tannery, a ritual bath, or a urinal. Rabbi Judah says: they may sell it to be a courtyard, and the purchaser may do what he likes with it.}%

\hebeng{\vsnum{3}ועוד אמר רבי יהודה בית הכנסת שחרב אין מספידין בתוכו. ואין מפשילין בתוכו חבלים. ואין פורשין לתוכו מצודות. ואין שוטחין על גגו פירות. ואין עושין אותו קפנדריא שנאמר (ויקרא כו, לא) והשימותי את מקדשיכם. קדושתן אף כשהן שוממין. עלו בו עשבים לא יתלוש. מפני עגמת נפש: }{\vsnumeng{3}Rabbi Judah said further: a synagogue that has fallen into ruins, they may not eulogize in it, nor twist ropes, nor to spread nets {[to trap animals]}, nor to lay out produce on its roof {[to dry]}, nor to use it as a short cut, as it says, “And I will desolate your holy places” (Leviticus 26:3 their holiness remains even when they are desolate. If grass comes up in it, it should not be plucked, {[in order to elicit]} melancholy.}%

\hebeng{\vsnum{4}ראש חדש אדר שחל להיות בשבת. קורין בפרשת שקלים. חל להיות בתוך השבת. מקדימין לשעבר. ומפסיקין לשבת אחרת. בשניה זכור. בשלישית פרה אדומה. ברביעית החדש הזה לכם. בחמישית חוזרין לכסדרן. לכל מפסיקין. בראשי חדשים. בחנוכה. ובפורים. בתעניות ובמעמדות. וביום הכפורים: }{\vsnumeng{4}If Rosh Hodesh Adar falls on Shabbat the portion of shekalim is read {[on that day]}. If it falls in the middle of the week, it is read on the Shabbat before, and on the next Shabbat there is a break. On the second {[of the special Shabbatot]} they read “Zakhor;” On the third the portion of the red heifer; On the fourth “This month shall be for you;” On the fifth the regular order is resumed. They interrupt {[the regular order]} for anything: for Rosh Hodesh, for Hanukkah, for Purim, for fasts, for Ma’amadot, and for Yom HaKippurim.}%

\hebeng{\vsnum{5}בפסח קורין בפרשת מועדות של תורת כהנים. בעצרת שבעה שבועות. בראש השנה. בחדש השביעי באחד לחדש. ביום הכפורים אחרי מות. ביום טוב הראשון של חג. קורין בפרשת מועדות שבתורת כהנים. ובשאר כל ימות החג. בקרבנות החג: }{\vsnumeng{5}On Pesah we read from the portion of the festivals in Leviticus (Torat Kohanim) (Leviticus 23:4). On Shavuot, “Seven weeks” (Deuteronomy 16:9). On Rosh Hashanah “On the seventh day on the first of the month” (Leviticus 23:2. On Yom Hakippurim, “After the death” (Leviticus. On the first day of the Festival {[of Sukkot]} they read from the portion of the festivals in Leviticus, and on the other days of the Festival {[of Sukkot]} the {[sections]} on the offerings of the Festival.}%

\hebeng{\vsnum{6}בחנוכה בנשיאים. בפורים ויבא עמלק. בראשי חדשים. ובראשי חדשיכם. במעמדות. במעשה בראשית. בתעניות. ברכות וקללות. אין מפסיקין בקללות. אלא אחד קורא את כולן. בשני ובחמישי ובשבת במנחה קורין כסדרן. ואין עולין להם מן החשבון. שנאמר (ויקרא כג, מד) וידבר משה את מועדי ה׳ אל בני ישראל. מצותן שיהו קורין כל אחד ואחד בזמנו: }{\vsnumeng{6}On Hanukkah they read the section of the princes (Numbers. On Purim, “And Amalek came” (Exodus 17:8). On Rosh Hodesh, “And on the first of your months” (Numbers 28:11). On Maamadot, the account of the creation (Genesis 1:1-2:3). On fast days, the blessings and curses (Leviticus 26:3 ff and Deuteronomy. They do not interrupt while reading the curses, but rather one reads them all. On Monday and Thursday and on Shabbat at minhah they read according to the regular order and this does not count as part of the reading {[for the succeeding Shabbat]}. As it says, “And Moshe declared to the children of Israel the appointed seasons of the Lord” (Leviticus 23:44) it is their mitzvah that each should be read in its appropriate time.}%

\clearpage
\blockcomment{ברטנורא על משנה מגילה}{\normalfont{בני העיר.} רחובה של עיר. יש בה קדושה, שמתפללין בה בתעניות. וחכמים פליגי על סתם מתניתין ואמרי רחובה של עיר אין בה משום קדושה, הואיל ואין מתפללין בה אלא באקראי בעלמא. והלכה כחכמים:\\\normalfont{אלא על תנאי.} ואפילו משל רבים לרבים אסור למכור מכירה חלוטה, דדרך בזיון הוא, כלומר אינו בעינינו כלום. ואין הלכה כר״מ:\\\normalfont{ואין מפשילין.} גודלים ופותלים. וה״ה לכל שאר מלאכות, אלא שהפשלת חבלים צריך מקום רחב ידים, ובהכ״נ בית גדול הוא וראוי לכך והוי אורחיה:\\\normalfont{קורין פרשת שקלים.} כי תשא, להודיע שיביאו שקליהם באדר, כדי שיקריבו בא׳ בניסן מתרומה חדשה:\\\normalfont{בפרשת מועדות שבתורת כהנים.} שור או כשב או עז. וביומא קמא מיירי. והאידנא נהוג עלמא שקורין ביום ראשון משכו וקחו לכם, ומפטירין בפסח גלגל. בשני שור או כשב, ומפטירין בפסח יאשיהו. בשלישי קדש לי כל בכור. ברביעי אם כסף תלוה. בחמישי פסל לך. בששי ויעשו בני ישראל את הפסח במועדו. בשביעי שירת הים, ומפטירין וידבר דוד. בח׳ שהוא יו״ט אחרון של גליות קורין כל הבכור, ומפטירין עוד היום בנוב לעמוד. בעצרת, ביו״ט ראשון, בחודש השלישי, ומפטירין במרכבה של יחזקאל. ביו״ט שני של גליות קורין כל הבכור, ומפטירין בחבקוק. בראש השנה וה׳ פקד את שרה, דבר״ה נפקדה שרה, ומפטירין בחנה שגם היא נפקדה בר״ה. ביו״ט שני בעקידה, ומפטירין הבן יקיר לי אפרים. ביוה״כ שחרית קורין באחרי מות, ומפטירין כה אמר רם ונשא. במנחה קורין בעריות, ומפטירין ביונה. בשני ימים טובים של חג קורין שור או כשב או עז ומפטירין ביו״ט ראשון הנה יום בא לה׳. וביו״ט שני ויקהלו אל המלך. ושאר כל ימות החג קורין בקרבנות החג, כיצד, יום ג׳ שהוא יום ראשון של חוה״מ, כהן קורא וביום השני, לוי קורא וביום השלישי, ישראל קורא וביום הרביעי, והרביעי חוזר וקורא וביום השני וביום השלישי. וביום הד׳ כהן קורא וביום השלישי, לוי קורא וביום הרביעי, ישראל קורא וביום החמישי, והרביעי חוזר וקורא וביום השלישי וביום הרביעי. וכן כולם. ביו״ט אחרון של חג, כל הבכור, ומפטירין ויהי ככלות שלמה. ולמחר קורין וזאת הברכה, ומפטירין ויהי אחרי מות משה. ושבת שחל להיות בחולו של מועד, בין בפסח בין בסוכות, קורין ראה אתה אומר אלי. ומפטירין, בפסח, העצמות היבשות. ובסוכות ביום בא גוג. שמסורת בידינו דתחיית המתים עתידה להיות בפסח, ומלחמות גוג ומגוג בסוכות: \\\normalfont{במעמדות במעשה בראשית.} שבשביל הקרבנות נתקיימו שמים וארץ. וסדר קריאתן מפורש במסכת תענית בפרק אחרון:\\\n}\clearpage %endcomment
\newchap{פרק ד}
\hebeng{\vsnum{1}הקורא את המגילה. עומד. ויושב. קראה אחד. קראוה שנים יצאו. מקום שנהגו לברך יברך. ושלא לברך לא יברך. בשני ובחמישי ובשבת במנחה קורין שלשה. אין פוחתין ואין מוסיפין עליהן. ואין מפטירין בנביא. הפותח והחותם בתורה. מברך לפניה. ולאחריה: }{\vsnumeng{1}He who reads the Megillah may either stand or sit. Whether one read it or two read it {[together]} they {[those listening]} have fulfilled their obligation. In places where it is the custom to say a blessing, they say the blessing, and where it is not the custom they do not say the blessing. On Mondays and Thursdays and on Shabbat at minhah, three read from the torah, they do not add {[to this number]} nor decrease {[from it]}, nor do they conclude with {[a haftarah]} from the Prophets. The one who begins the Torah reading and the one who concludes the Torah reading blesses before it and after it.}%

\hebeng{\vsnum{2}בראשי החדשים. ובחולו של מועד. קורין ארבעה. אין פוחתין מהן. ואין מוסיפין עליהן. ואין מפטירין בנביא. הפותח והחותם בתורה מברך. לפניה. ולאחריה. זה הכלל. כל שיש בו מוסף ואינו יום טוב. קורין ארבעה. ביום טוב חמשה. ביום הכפורים ששה. בשבת שבעה. אין פוחתין מהן. אבל מוסיפין עליהן. ומפטירין בנביא. הפותח והחותם בתורה. מברך לפניה ולאחריה: }{\vsnumeng{2}On Rosh Hodesh and on the intermediate days of festivals four read. They do not add {[to this number]} nor decrease {[from it]}, nor do they conclude with {[a haftarah]} from the Prophets. The one who begins the Torah reading and the one who concludes the Torah reading blesses before it and after it. This is the general rule: on any day which has a musaf and is not a festival four read. On a festival five. On Yom Hakippurim six. On Shabbat seven; they may not decrease {[from this number]} but they may add {[to it]}, and they conclude with {[a haftarah]} from the Prophets. The one who begins the Torah reading and the one who concludes the Torah reading blesses before it and after it.}%

\hebeng{\vsnum{3}אין פורסין את שמע. ואין עוברין לפני התיבה. ואין נושאין את כפיהם. ואין קורין בתורה. ואין מפטירין בנביא. ואין עושין מעמד ומושב. ואין אומרים ברכת אבלים ותנחומי אבלים. וברכת חתנים. ואין מזמנין בשם. פחות מעשרה. ובקרקעות תשעה וכהן ואדם כיוצא בהן: }{\vsnumeng{3}They do not recite the Shema responsively, And they do not pass before the ark; And they {[the priests]} do not lift up their hands; And they do not read the Torah {[publicly]}; And they do not conclude with a haftarah from the prophets; And they do not make stops {[at funeral]} processions; And they do not say the blessing for mourners, or the comfort of mourners, or the blessing of bridegrooms; And they do not mention God’s name in the invitation {[to say Birkat Hamazon]}; Except in the presence of ten. {[For redeeming sanctified]} land nine and a priest {[are sufficient]}, and similarly with human beings.}%

\hebeng{\vsnum{4}הקורא בתורה לא יפחות משלשה פסוקים. לא יקרא למתורגמן יותר מפסוק אחד. ובנביא שלשה. היו שלשתן שלש פרשיות. קורין אחד אחד. מדלגין בנביא. ואין מדלגין בתורה. ועד כמה הוא מדלג. עד כדי שלא יפסוק המתורגמן: }{\vsnumeng{4}One who reads the Torah {[in public]} may not read less than three verses. And he should not read to the translator more than one verse {[at a time]}, but {[if reading from the book of a]} prophet {[he may read to him]} three at a time. If the three verses constitute three separate paragraphs, he must read them {[to the translator]} one by one. They may skip {[from place to place]} in a prophet but not in the Torah. How far may he skip {[in the prophet]}? {[Only]} so far that the translator will not have stopped {[before he finds his place]}.}%

\hebeng{\vsnum{5}המפטיר בנביא הוא פורס על שמע. והוא עובר לפני התיבה. והוא נושא את כפיו. ואם היה קטן. אביו או רבו עוברין על ידו: }{\vsnumeng{5}The one who concludes with the haftarah also leads the responsive reading of the Shema and he passes before the ark and he lifts up his hands. If he is a child, his father or his teacher passes before the ark in his place.}%

\hebeng{\vsnum{6}קטן קורא בתורה ומתרגם. אבל אינו פורס על שמע. ואינו עובר לפני התיבה. ואינו נושא את כפיו. פוחח פורס את שמע ומתרגם. אבל אינו קורא בתורה. ואינו עובר לפני התיבה. ואינו נושא את כפיו. סומא פורס את שמע ומתרגם. רבי יהודה אומר כל שלא ראה מאורות מימיו. אינו פורס על שמע: }{\vsnumeng{6}A child may read in the Torah and translate, but he may not pass before the ark or lift up his hands. A person in rags may lead the responsive reading of the Shema and translate, but he may not read in the Torah, pass before the ark, or lift up his hands. A blind man may lead the responsive reading of the Shema and translate. Rabbi Judah says: one who has never seen the light from his birth may not lead the responsive reading of the Shema.}%

\hebeng{\vsnum{7}כהן שיש בידיו מומין. לא ישא את כפיו. רבי יהודה אומר אף מי שהיו ידיו צבועות אסטיס. ופואה. לא ישא את כפיו. מפני שהעם מסתכלין בו: }{\vsnumeng{7}A priest whose hands are deformed should not lift up his hands {[to say the priestly blessing]}. Rabbi Judah says: also one whose hands are colored with woad or madder should not lift up his hands, because {[this makes]} the congregation look at him.}%

\hebeng{\vsnum{8}האומר איני עובר לפני התיבה בצבועין אף בלבנים לא יעבור. בסנדל איני עובר. אף יחף לא יעבור. העושה תפלתו עגולה. סכנה ואין בה מצוה. נתנה על מצחו. או על פס ידו. הרי זו דרך המינות. ציפן זהב ונתנה על בית אונקלי שלו. הרי זו דרך החיצונים: }{\vsnumeng{8}If one says, “I will not pass before the ark in colored clothes,” even in white clothes he may not pass before it. {[If one says]}, “I will not pass before it in shoes,” even barefoot he may not pass before it. One who makes his tefillin {[for the head]} round, it is dangerous and has no religious value. If he put them on his forehead or on the palm of his hand, behold this is the way of heresy. If he overlaid them with gold or put {[the one for the hand]} on his sleeve, behold this is the manner of the outsiders.}%

\hebeng{\vsnum{9}האומר יברכוך טובים. הרי זו דרך המינות. על קן צפור יגיעו רחמיך ועל טוב יזכר שמך. מודים. מודים. משתקין אותו. המכנה בעריות. משתקין אותו. האומר מזרעך לא תתן להעביר למולך (ויקרא יח, כא). ומזרעך לא תתן לאעברא בארמיותא. משתקין אותו בנזיפה: }{\vsnumeng{9}If one says “May the good bless you,” this is the way of heresy. {[If one says]}, “May Your mercy reach the nest of a bird,” “May Your name be mentioned for the good,” “We give thanks, we give thanks,” they silence him. One who uses euphemisms in the portion dealing with forbidden marriages, he is silenced. If he says, {[instead of]} “And you shall not give any of your seed to be passed to Moloch,” (Leviticus 18:21) “You shall not give {[your seed]} to pass to a Gentile woman,” he silenced with a rebuke.}%

\hebeng{\vsnum{10}מעשה ראובן נקרא ולא מיתרגם. מעשה תמר נקרא ומיתרגם. מעשה עגל הראשון נקרא ומיתרגם. והשני נקרא ולא מיתרגם. ברכת כהנים. מעשה דוד ואמנון. לא נקראין ולא מיתרגמין. אין מפטירין במרכבה. ורבי יהודה מתיר. רבי אליעזר אומר. אין מפטירין בהודע את ירושלים:  }{\vsnumeng{10}The incident of Reuven is read but not translated. The story of Tamar is read and translated. The first part of the incident of the golden calf is both read and translated, but the second is read but not translated. The blessing of the priests, the stories of David and Amnon are not read or translated. They do not conclude with the portion of the chariot as a haftarah. But Rabbi Judah permits this. R. Eliezar says: they do not conclude with “Proclaim Jerusalem’s {[abominations]}” (Ezekiel as a haftarah.}%

\blockcomment{ברטנורא על משנה מגילה}{\normalfont{הקורא את המגילה עומד ויושב.} רצה עומד רצה יושב:\\\normalfont{ואין מוסיפין עליהן.} דבראשי חדשים ובחולו של מועד נמי איכא בטול מלאכה, דמלאכת דבר האבד מותרת:\\\normalfont{אין פורסין על שמע.} עשרה שבאו לבית הכנסת אחר שקראו צבור את שמע, עומד אחד ואומר קדיש וברכו וברכה ראשונה שלפני קריאת שמע. פורסין, לשון פרוסה, כלומר חצי דבר, משתי ברכות שלפני ק״ש אומר ברכה אחת:\\\normalfont{ולא יקרא למתורגמן יותר מפסוק אחד.} שלא יטעה המתורגמן שמתרגם על פה:\\\normalfont{המפטיר בנביא.} מי שרגיל להפטיר בנביא, תקנו חכמים שיהא הוא פורס על שמע ברבים:\\\normalfont{קטן קורא בתורה.} ויש מן הגאונים שאמרו דוקא משלישי ואילך:\\\normalfont{כהן שיש בידו מומין.} וכן בפניו או ברגליו:\\\normalfont{אף בלבנים לא יעבור.} חיישינן שמא מינות נזרקה בו, דעובדי ע״ז מקפידין בכך:\\\normalfont{יברכוך טובים הרי זו דרך מינות.} שצריכין ישראל לצרף עמהם פושעי ישראל באגודת תעניותיהם. שהרי חלבנה ריחה רע ומנאה הכתוב עם סממני הקטורת:\\\normalfont{מעשה אמנון ותמר נקרא ומיתרגם.} ולא חיישינן ליקריה דדוד. והוא, דלא כתיב אמנון בן דוד, כדבעינן למימר לקמן:\\\n}\clearpage %endcomment
\addpart{בן יהוידע על מגילה}\renewcommand{\partname}[1]{בן יהוידע על מגילה}
\fancyhead[CO]{ \partname\space\textendash\space \chapname}
\fancyhead[CE]{\partname}
\newchap{פרק \hebrewnumeral{2} הקורא למפרע}
\textblock{}
\textblock{\textbf{מִנַּיִן שֶׁאוֹמְרִים ׳אָבוֹת׳? שֶׁנֶּאֱמַר {\small (תהלים כט, א)}׃ ״הָבוּ לַה׳ בְּנֵי אֵלִים״}. נראה לי בס״ד מה שנקראים האבות בְּנֵי אֵלִים על פי מה שכתב רבינו האר״י ז״ל בדרוש ראש השנה דרוש וא״ו, דשלשה שמות אֵ־ל אֵ־ל אֵ־ל שהם גימטריא מָגֵן {\small [31×3=93]} נרמזים בג׳ פעמים ׳אֱלֹקֵי׳ שיש בתפילת העמידה כמו שאומרים ׳אֱלקֵי אַבְרָהָם אֱלקֵי יִצְחָק וֵאלקֵי יַעֲקב׳ יעוין שם. נמצא ג׳ פעמים אֵ־ל שיש בג׳ פעמים אֱלֹקֵי האמורים על שם האבות הם רומזים לבחינת שלשה שמות אֵ־ל בפני עצמו ולכן נקראים האבות בְּנֵי אֵלִים על שלשה שמות אֵ־ל הנזכרים על שמם בתפילת העמידה.\par או יובן בס״ד כי האבות הם רגלי הכסא ו׳אֵלִים׳ גימטריא כִּסֵּא {\small [81]} ולכן נקראים בְּנֵי אֵלִים כלומר בְּנֵי הַכִּסֵּא.}
\textblock{\textbf{וּמָה רָאוּ לוֹמַר בִּרְכַּת ׳הַשָּׁנִים׳ בַּתְּשִׁיעִית? אָמַר רַבִּי אַלֶכְּסַנְדְּרִי: כְּנֶגֶד מַפְקִיעֵי שְׁעָרִים, דִּכְתִיב {\small (תהלים י, טו)}׃ ״שְׁבֹר זְרוֹעַ רָשָׁע״}. הקשה מהרש״א ז״ל כיון דפירש רש״י ז״ל דכל הפרשה נאמרה על מפקיעי שערים, אמאי מייתי רק האי קרא ד׳שְׁבֹר זְרוֹעַ רָשָׁע׳? יעוין שם.\par ונראה לי בס״ד משום דבהאי קרא רומז העונש שיעשה לו שלא יוכל להפקיע עוד מכאן ולהבא. והא דאותיות העומדים בראש וסוף התיבה נקראים זרועות, וכאן אמר ׳זְרוֹעַ רָשָׁע׳ משמע חד והוא אות עי״ן שבסוף שמו של רָשָׁע דאם מסיר אותו ישאר ׳רָשׁ׳ ורמז בזה שיעני אותו וישאר רָשׁ, ואז לא יוכל לקבץ התבואה תחת ידו כדי ליקר השער. וזהו ׳שְׁבֹר זְרוֹעַ רָשָׁע׳ הוא אות ע׳ כדי שיהיה רש ועני, וקראו בשם \textbf{רָשָׁע} שהוא בהפוך אתוון \textbf{שַׁעַר} , כלומר רשעתו אינה בעבירה של זנות וכיוצא אלא היא בענין יוקר השער של הדגן.}
\clearpage
\newsection{דף יח}
\textblock{\textbf{שְׁכָחוּם וְחָזַר וְסִדְּרָם}. אין הכונה ששכחו סדר הברכות, דזה לא יתכן, דאפילו המון העם מתפללים כל שמונה עשרה ואיך ישכחום? אך הכונה ששכח הטעמים של הסדר, למה זו ראשונה וזו שניה וזו שלישית וכיוצא וכמו שסידרם הש״ס כאן בטעמייהו וגם שכחו טעמי הסדר שיש בהם על פי הסוד.}
\textblock{\textbf{{\small (תהלים קו, ב)} לְמִי נָאֶה לְמַלֵּל גְּבוּרוֹת ה׳? לְמִי שֶׁיָּכוֹל לְהַשְׁמִיעַ כָּל תְּהִלָּתוֹ}. מקשים וכי אנשי כנסת הגדולה היו יכולים להשמיע כל תהלתו, ואיך תיקנו כל זה?\par ונראה לי בס״ד דאנשי כנסת הגדולה תיקנו דברים כנגד ספירות העליונים שיש לכל ברכה ושבח ספירה פרטית כנגדה והרי זה דומה למלך שיש לו כמה אלפים ורבבות ארמונים עד אין מספר ובא אחד וסיפר שיש בארמון אחד מן הארמונים כך וכך חדרים וכך וכך עליות אין זה גורע דאינו סופר כל החדרים ועליות שיש למלך בכל הארמונים אלא רק מדבר בארמון פרטי, אבל מי שבא לומר יש למלך כך וכך חדרים בסתם וכך וכך עליות בסתם שאינו מתכוין על ארמון פרטי הרי זה גורע, דבאמת אין מספר לחדרים ולעליות של המלך! וכן הענין כאן, והמניעה היא למי שמתכוין לקבץ שבחים הרבה מן הכלל ומן הפרט לזה אומרים סיימתנהו לשבחי דמרך.}
\textblock{\textbf{הַמְסַפֵּר בְּשִׁבְחוֹ שֶׁל הַקָּדוֹשׁ בָּרוּךְ הוּא יוֹתֵר מִדַּאי, נֶעֱקָר מִן הָעוֹלָם}. נראה לי בס״ד דאיתא באותיות דרבי עקיבא, בעבור השיר שמקלסין להקב״ה נברא העולם עיין שם, ועל כן שזה שהוא מרבה בשבח שחושב שמחזק העולם שנברא בעבור השיר לכך מדה כנגד מדה נעקר מן העולם.}
\textblock{\textbf{מִלָּה בְּסֶלַע, מַשְׁתּוּקָא בִּתְרֵין}. נראה לי בס״ד בתוך אותיות ׳סֶלַע׳ במילואם יש ׳מַיִם׳ כזה: ס\textbf{מ} ״ך ל\textbf{מ} ״ד ע\textbf{י} ״ן, וידוע מה שאמר רבי עקיבא לחכמים שנכנסו לפרדס כשתגיעו לאבני שיש טהור, אל תאמרו ׳מים מים׳ {\small (חגיגה יד:)} , נמצא יש מקום שאם יאמרו בו ב׳ פעמים מים צרוך להשתיקם, וזהו שאמר יש \textbf{מִלָּה בְּסֶלַע} שהיא תיבת מַיִם המונחת בתוך סֶלַע ראוי לעשות בה \textbf{מַשְׁתּוּקָא} , אם יאמרו אותה \textbf{בִּתְרֵין} וסוד הנזכר שמנע מהם הכפל של ׳מים מים׳ מפורש בדברי מהרח״ו {\small [מורנו הרב רבי חיים ויטאל]} ז״ל בביאור {\small (משנה אבות ו, א)} ׳כָּל הָעוֹסֵק בַּתּוֹרָה לִשְׁמָהּ׳ יעוין שם.\par ועוד נראה לי דרך הלצה, אם יש שטר צוואה או שטר חוב או מתנה שכתוב בו תנו סלע לפלוני או חייב אני סלע לפלוני, ורוצה בעל השטר לזייף להוסיף אותיות ים על סלע, כדי שתהיה נקראת ׳סלעים׳ שבדעתו לתבוע מאות ואלפים, לא תועיל לו ערמתו כי יכול בעל דינו להשתיקו בתרין סלעים, כי יאמר מעוט רבים שנים. וזהו שאמר מלה בסלע, רוצה לומר אם תרצה להוסיף \textbf{מִלָּה} אחת של ים במלת \textbf{בְּסֶלַע} , כדי שתהיה נקראת סלעים לשון רבים, עושים לך \textbf{מַשְׁתּוּקָא} לתביעתך, \textbf{בִּתְרֵין} סלעים שיפטרו אותך בהם דיאמרו לך מיעוט רבים שנים.\par מיהו צריך להבין למה נקטו שיעור ׳סֶלַע׳ לדבור? ונראה לי בס״ד כי תוכו של אדם כמו בור עמוק והפתח הוא הפה שיוצא ממנו הדיבור אך הדברים נחלקים לארבעה חלקים: שמדבר על עצמו, ועל זולתו, על העבר, ומדבר על עצמו ועל זולתו על העתיד, נמצא נחלק קרבו לארבע בורות לארבעה חלקים של דבור, לכך נקרא דִּבּוּר - ד׳ בּוֹר, ולכן עשה ערך לדבור ׳סֶלַע׳ שהוא ארבעה דנרים.\par \textbf{ובני ידידי} כה״ר יעקב נר״ו פירש דידוע המלאכים דברו תביעתם על התורה בעבור חלק הסוד דוקא דשייך להם, וזהו שאמר מִלָּה בְּ׳סֶלַע׳ ראשי תיבות \textbf{ס} וד \textbf{ל} אוין \textbf{ע} שין, על זה היתה מלה של תביעה אך נשתתקו בתרין בשביל שנתן הקב״ה לישראל שתים חלקים ביחד חלק הסוד וחלק הנגלה דלא שייך במלאכים.\par ועוד פירש מלה בסלע, דאמרו רבותינו ז״ל {\small (קידושין לא.)} כשאמר{\small (שמות כ, ב)} ׳אָנֹכִי וְלֹא יִהְיֶה לְךָ׳ הרהרו אומות העולם על זה ואמרו לכבוד עצמו דורש! וזהו \textbf{מִלָּה בְּסֶלַע} הוא דבור ראשון שהוא גבוה שבעשרת הדברות ולהכי קרי ליה סלע שהוא גבוה אך נעשה להם \textbf{מַשְׁתּוּקָא בִּתְרֵין} , כאשר שמעו דבור ׳כַּבֵּד אֶת אָבִיךָ וְאֶת אִמֶּךָ׳, שצוה על כבוד האב והאם שהם שנים, דאז חזרו והודו לדברות ראשונות עד כאן דבריו נר״ו.}
\textblock{\textbf{מִנַּיִן שֶׁקְּרָאוֹ הַקָּדוֹשׁ בָּרוּךְ הוּא לְיַעֲקֹב ׳אֵ־ל׳?}. יש להבין איך יצוייר זה בדעת? ונראה לי דלא קראו אֵ־ל בסתם אלא קראו יִשְׂרָאֶל בנקוד סגול תחת האלף כזה ׳יִשְׂרָאֶל׳ דאז יוצא בשמו שם אֵ־ל ולא קראו כמו שם ׳יִשְׁמָעֵאל׳ דאין נקוד תחת האלף דעל כן אינו יוצא במבטא אות האלף אלא העין נדבק עם הלמד, וכאשר הוא כן בשם יחזקאל הנביא ע״ה ופירש דכתיב {\small (בראשית לג, כ)} ׳וַיַּצֶּב שָׁם מִזְבֵּחַ וַיִּקְרָא לוֹ: אֵ־ל אֱלֹקֵי יִשְׂרָאֵל׳ רוצה לומר אֱלֹקֵי יִשְׂרָאֵל קרא שם אֶל כשקראו ישראל, ולכן אברהם שמו כפול, הרי יו״ד אותיות, ושם יצחק אינו כפול הרי י״ד אותיות, ושם יעקב כפול הרי כ״ב ושם ישרן בלא ווי״ן הרי כ״ו, ונמצא שם ישראל משלים ל״א אותיות כמנין שם אל.}
\clearpage
\newsection{דף יט}
\textblock{\textbf{מַאן דְּאָמַר {\small (אסתר ט, כט)}׃ ׳כֻּלָּה׳, תָּקְפּוֹ שֶׁל אֲחַשְׁוֵרוֹשׁ}. דכתיב {\small (אסתר א, ד)} ׳בְּהַרְאֹתוֹ אֶת עֹשֶׁר כְּבוֹד מַלְכוּתוֹ׳ דנמצא שהיה עשיר גדול ואיך יצוייר שימכור אומה שלימה בעשרת אלפים ככר כסף? אשר אלו לא היו נחשבים אצלו אפילו בערך אבנים! ואם כן מוכח דהמן זייף באומרו שקנה אותם בעשרת אלפים ככר כסף, וכיון דזייף בזה זייף בכל ובטלו האגרות שכתב!\par \textbf{וּמַאן דְּאָמַר}׃ \textbf{מֵ} ׳\textbf{אִישׁ יְהוּדִי}׳, \textbf{תָּקְפּוֹ שֶׁל מָרְדֳּכַי} שהיה לו כח מן יהודה ומן בנימין ובזה הכניע את המן שהיה בידו קטרוג מכח תמנע שלא קבלו אותה ומכח השבטים שהשתחוו לעשו וכמו שכתב מהר״י ז״ל דבכח אִישׁ יְהוּדִי דאתי מיהודה ביטל קטרוג תמנע, ומכח אִישׁ יְמִינִי דאתי מבנימין שלא השתחוה לעשו ביטל קטרוג ההשתחויה שהשתחוו השבטים לעשו.\par \textbf{וּמַאן דְּאָמַר}׃ \textbf{מֵ} ׳\textbf{אַחַר הַדְּבָרִים הָאֵלֶּה}׳, \textbf{תָּקְפּוֹ שֶׁל הָמָן} כי בגאותו ותוקפו נתן עיניו לכלות הכל ובזה ממילא נעשה נס ההצלה מה שאין כן אם היה נותן עיניו במקצת היה עולה בידו. \textbf{וּמַאן דְּאָמַר}׃ \textbf{מִ} ׳\textbf{בַּלַּיְלָה הַהוּא}׳, \textbf{תָּקְפּוֹ שֶׁל נֵס} כי נזדמן אותה הלילה שהיתה מוחזקת בנסים גדולים מדורות הראשונים ולכך הצליחו בה במעשה נסים.}
\textblock{\textbf{׳מָה רָאָה׳ אֲחַשְׁוֵרוֹשׁ שֶׁנִּשְׁתַּמֵּשׁ בַּכֵּלִים שֶׁל בֵּית הַמִּקְדָּשׁ? ׳עַל כָּכָה׳, מִשּׁוּם דְּחָשִׁיב שִׁבְעִים שְׁנִין וְלָא אִיפְרוּק}. נראה לי בס״ד נרמזו השבעים בזה דאות כף של כָּכָה כפול ואם תסיר הכפל יהיו ׳עַל כָּכָה׳ אותיות ׳ע׳ כלה׳ פירוש ראה שכלה זמן של ע׳ שנה ועל ידי כך ׳וּמָה הִגִּיעַ אֲלֵיהֶם׳? דְּקָטַל וַשְׁתִּי! שנעשה להם מהומה גדולה שנוסף אותיות ומה על אותיות מה ונעשה צירוף מְהוּמָה.}
\textblock{\textbf{׳מָה רָאָה׳ מָרְדֳּכַי דְּאִיקְנִי בְּהָמָן? ׳עַל כָּכָה׳, דְּשַׁוִּי נַפְשֵׁיהּ עֲבוֹדָה זָרָה}. נראה לי נרמז זה בתיבת כָּכָה שהוא מספר לוט {\small [45]} שהוא תרגום של ארירה, דאומרים ארור המן על אשר עשה עצמו עבודה זרה. ׳וּמָה הִגִּיעַ אֲלֵיהֶם׳ רוצה לומר אותיות מה של גְּאוּלָּה {\small [45]} , שמספרם גְּאוּלָּה שנעשה בה הנס.\par ואומרו ׳\textbf{מָה רָאָה} ׳ \textbf{הָמָן שֶׁנִּתְקַנֵּא בְּכָל הַיְּהוּדִים} ׳\textbf{עַל כָּכָה} ׳ \textbf{מִשּׁוּם דְּמָרְדֳּכַי לֹא יִכְרַע וְלֹא יִשְׁתַּחֲוֶה} והיינו כָּכָה גימטריא ׳לב אחד׳ {\small [45]} , פירוש שהיה לו לב אחד, ׳וּמָה הִגִּיעַ אֲלֵיהֶם׳? ׳וְתָלוּ אוֹתוֹ וְאֶת בָּנָיו עַל הָעֵץ׳! שיצאה נשמת בניו ברגע אחד בעבור זכות מרדכי שהיה לו לב אחד.\par ואמר עוד ׳\textbf{מָה רָאָה} ׳ \textbf{אֲחַשְׁוֵרוֹשׁ לְהָבִיא אֶת סֵפֶר הַזִּכְרוֹנוֹת?} ׳\textbf{עַל כָּכָה}׳, \textbf{דְּזַמִּינְתֵּיהּ אֶסְתֵּר לְהָמָן בַּהֲדֵיהּ} נראה לי דדריש כָּכָה על סעודה לשון דוק בבכי, ׳וּמָה הִגִּיעַ אֲלֵיהֶם׳? כי מן הסעודה נולד מעשה הנס.}
\clearpage
\newsection{דף כא}
\textblock{\textbf{אִלְמָלֵא מִקְרָא כָּתוּב, אִי־אֶפְשָׁר לְאָמְרוֹ, כִּבְיָכוֹל, שֶׁאֲפִלּוּ הַקָּדוֹשׁ בָּרוּךְ הוּא בַּעֲמִידָה}. הכונה כי הנה קרא כתיב {\small (הושע יב, יא)} ׳וּבְיַד הַנְּבִיאִים אֲדַמֶּה׳ דאי אפשר לשום נביא לראות פני שכינה, דכתיב על ידי משה רבינו ע״ה בעצמו {\small (שמות לג, כ)} ׳כִּי לֹא יִרְאַנִי הָאָדָם וָחָי׳, אך יצוייר לעיני הנביאים איזה דמיון כדי שיוכלו להשיג הנבואה כל אחד לפי ערכו, וכמו שכתב הרמב״ם ז״ל בהלכות יסודי התורה, דכל נביא יהיה איזה דמיון לנגד עיניו לפי שעה, והדמיון הזה אינו קבוע לאותו נביא, אלא יהיה לפי אותו הענין של הנבואה דאפילו משה רבינו ע״ה לא ראה הדמיון שוה כי בים סוף נדמה לו כגיבור ובסיני נדמה לו כזקן ובאמת אין לו יתברך דמות וצורה אלא הכל יצוייר לעיני הנביא כמראה הנבואה וכמחזה לפי שעה ואין דעתו של אדם מבין אמיתות הדבר, יעוין שם. וכן הענין כאן, שהיה נדמה לו כעומד ולא יושב ואף על פי שבאמת הציור הזה שהיה מצטער לעיני משה רבינו ע״ה בסיני אין זה אמיתות הדבר אלא הוא דרך דמיון, עם כל זה אמר רבי אבהו אלמלא מקרא שכתוב לא היינו יכולים לאומרו בפינו, וכונת בעל המאמר לומר כי נצטייר ונדמה לעיני משה רבינו ע״ה כעומד, ללמדינו דעת בזה שיהיו הרב והתלמיד התחתונים שוים, שניהם בעמידה או שניהם בישיבה או שניהם על גבי קרקע.\par ועוד נראה לי בס״ד דלעולם לא היה רואה משה רבינו ע״ה בהר בעת לימודו שום ציור ודמיון בעמידה אלא רק היה שומע הקול מדבר אליו ומלמדו, ומה שאמר לו {\small (דברים ה, כז)} ׳עֲמֹד עִמָּדִי׳ אמר לו בלשון זה המורה כאלו היה הקב״ה בעמידה, כדי ללמד לתחתונים שתהיה קריאתם בעומד. וגם ללמדם שלא יהיה הרב ע על גבי מטה ותלמידיו על גבי קרקע.}
\textblock{\textbf{מִימוֹת מֹשֶׁה עַד רַבָּן גַּמְלִיאֵל, הָיוּ לְמֵדִין תּוֹרָה מְעֻמָּד}. נראה לי בס״ד הטעם כדי להטריח עצמן בתורה, כמו שאמרו {\small (מגילה ו:)} ׳יָגַעְתִּי וּמָצָאתִי תַּאֲמֵן׳.\par או יובן הרגלים רומזים לנצח והוד, והתורה ניתנה על ידי לוחות שהוא סוד נצח והוד, ולמדוה מפי משה ואהרן שהם נצח והוד.\par או יובן הגוף שהוא עולם קטן עומד על שני רגלים וכן העולם עומד על התורה דכתיב {\small (ירמיה לג, כה)} ׳אִם לֹא בְרִיתִי יוֹמָם וָלָיְלָה חֻקּוֹת שָׁמַיִם וָאָרֶץ לֹא שָׂמְתִּי׳ והתורה כוללת שני עמודים שהוא עסק התורה וגם עמוד העבודה ד׳כל הקורא בפרשת עולה כאלו הקריב עולה׳ {\small (מנחות קי.)} .\par אי נמי לימוד התורה מעומד יורה דראוי לאדם ללמוד תורה מן הרב שהוא דומה למלאך, דכתיב {\small (מלאכי ב, ז)} כִּי שִׂפְתֵי כֹהֵן יִשְׁמְרוּ דַעַת וְהַתּוֹרָה יְבַקְשׁוּ מִפִּיהוּ כִּי מַלְאַךְ הֳ׳ צְבָקוֹת הוּא, וידוע כי המלאכים אין להם ישיבה למעלה אלא הם עומדים דכתיב {\small (זכריה ג, ז)} ׳בֵּין הָעֹמְדִים הָאֵלֶּה׳.}
\textblock{\textbf{רַכּוֹת מְעֻמָּד, קָשׁוֹת מְיֻשָּׁב}. קשה יתן לו כח שיוכל ללמוד גם הקשות מעומד מאחר שהוא בנס היה שם שהיה בלא אכילה ובלא שתיה ובלא שינה וכיון שהיה עומד שם בנס בכח אלקי אם כן מה הפרש יש לו בין ישיבה לעמידה כי באמת אינו עומד שם בכח טבעי?\par ונראה לי דודאי לא יש אצלו הפרש ורק עשה לו כן שילמד רכות מעומד קשות מיושב כדי ללמד לתחתונים שהנהגתם וכוחן הוא טבעי שככה יעשו רכות מעומד וקשות מיושב ומאן דאמר שהיה ׳שׁוֹחֶה׳ היינו כדי להראות הכנעה על דרך שאמרו גבי מלך בתפילת העמידה.}

\addtocontents{toc}{\protect\end{multicols}}
\end{document}
