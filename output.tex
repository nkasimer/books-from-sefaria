\documentclass[12pt, openany]{book}
\usepackage[
paperheight=8.5in,
paperwidth=5.5in,
top=0.5in,
bottom=0.5in,
inner=0.7in,
outer=0.5in,
marginparsep=0.1in,
headsep=16pt
]{geometry}

\newcommand{\texttitle}{מסכת נידה עם פירושי רש״י ורמב״ן}\usepackage{titlesec}
\usepackage{resources/unnumberedtotoc}

\usepackage{fancyhdr}
\pagestyle{fancy}
\fancyhf{}
\fancyhead[LO,RE]{\thepage}
\fancyhead[CO]{\chapname}
\fancyhead[CE]{\texttitle}

\usepackage{paracol}
\usepackage{anyfontsize}
\usepackage{ragged2e}
\usepackage{polyglossia}
\usepackage{multicol}
\usepackage{hyperref}
\usepackage[marginal]{footmisc}
\usepackage[titles]{tocloft}
\usepackage{xifthen}
\usepackage{graphicx}

\setdefaultlanguage{hebrew}
\setotherlanguage{english}
\usepackage{fontspec}
\setmainfont{Times New Roman}
\newfontfamily\englishfont{Times New Roman}

\newcommand{\sethebfont}{
\fontsize{10.5pt}{21.0pt} \selectfont
}

\newcommand{\hebeng}[2]{
	{\sethebfont #1\\}
	
	\begin{english}
		#2
	\end{english}
	\clearpage
}

\newcommand{\twocol}[1]{
	{\sethebfont \begin{multicols}{2}
			#1
	\end{multicols}}	
}

\newcommand{\textblock}[1]{
{\sethebfont #1\\}	
}

\setlength{\parskip}{8pt}
\setlength\parindent{0in}

\newcommand{\chapname}{}
\newcommand{\sectname}{}

\newcommand{\newchap}[1]{
	\addcontentsline{toc}{chapter}{#1}
	\renewcommand{\chapname}{#1}
		\begin{center}
			\textbf{%
\fontsize{16pt}{16pt}\selectfont
				#1}
		\end{center}
}

\let\footnoterule\relax

\setlength{\columnsep}{0.25in}

\newcommand{\newsection}[1]{
	%\addcontentsline{toc}{section}{#1}
	\renewcommand{\sectname}{#1}	
	\vspace{-\baselineskip}
	\begin{center}
		\textbf{%
\fontsize{16pt}{16pt}\selectfont
			#1}
	\end{center}
	\vspace{-\baselineskip}
	\nopagebreak
}

\newcommand{\footnotecomment}[1]{
	\renewcommand\thefootnote{}
	\footnote{#1}}

\newcommand{\parencomment}[1]{\footnotesize (#1)}

\newcommand{\commenta}[1]{\footnotecomment{#1}}

\begin{document}
\frontmatter
\pagenumbering{roman}

\newcommand{\oneline}[1]{%
	\newdimen{\namewidth}%
	\setlength{\namewidth}{\widthof{#1}}%
	\ifthenelse{\lengthtest{\namewidth < \textwidth}}%
	{#1}% do nothing if shorter than text width
	{\resizebox{\textwidth}{!}{#1}}% scale down
}

\title{\oneline{\hspace*{0.5in}\texttitle\hspace*{0.5in}}}

\author{}

\date{}

\maketitle

\begin{minipage}[b][\textheight][b]{\textwidth}\englishfont	
	\begin{english}
		\vfill
		The following book includes:
\begin{itemize}
\item[$\bullet$] Wikisource Talmud Bavli
\begin{itemize}
\item[$\bullet$] License: CC-BY
\item[$\bullet$] Source: \url{http://he.wikisource.org/wiki/%D7%AA%D7%9C%D7%9E%D7%95%D7%93_%D7%91%D7%91%D7%9C%D7%99}
\end{itemize}
\item[$\bullet$] Vilna Edition
\begin{itemize}
\item[$\bullet$] License: CC-BY
\item[$\bullet$] Source: \url{http://primo.nli.org.il/primo_library/libweb/action/dlDisplay.do?vid=NLI&docId=NNL_ALEPH001300957}
\end{itemize}
\item[$\bullet$] Chiddushei HaRamban, Jerusalem 1928\textendash 29
\begin{itemize}
\item[$\bullet$] License: Public Domain
\item[$\bullet$] Source: \url{http://primo.nli.org.il/primo_library/libweb/action/dlDisplay.do?vid=NLI&docId=NNL_ALEPH001294828}
\end{itemize}
\end{itemize}
		It was retrieved from Sefaria on \today\space \texthebrew{(\Hebrewtoday)}.  It was typeset and formatted by Ktavi.
		\clearpage
		
	\end{english}
\end{minipage}

\titleformat{\chapter}[hang]{\huge\bfseries}{\thechapter.}{1em}{}
\titlespacing*{\chapter}{0pt}{-3em}{1.1\parskip}
\titlelabel{\thetitle\quad}
%\addtocontents{toc}{\protect\vspace{-\baselineskip}}
\addtocontents{toc}{\protect\begin{multicols}{2}}
%\vspace*{-5\baselineskip}
{\small \tableofcontents}


\clearpage
\mainmatter
\pagenumbering{arabic}

\addpart{נדה}\newchap{פרק \hebrewnumeral{2} כל היד}
\twocol{\clearpage}

\newsection{דף יג}
\twocol{
\commenta{גמ' הרגשה - שמזדעזעין אבריו כשמתחמם ורואה קרי:}
מתני׳ {\large\emph{כל}} היד המרבה לבדוק בנשים משובחת ובאנשים תקצץ
\commenta{אנשים - משובחת דקתני קאי מרבה אבל אתקצץ דבאנשים אפילו חדא זימנא:}
{\large\emph{גמ׳}} מ"ש נשים ומאי שנא אנשים נשים לאו בנות הרגשה נינהו משובחות אנשים דבני הרגשה נינהו תקצץ 
\commenta{לענין זיבה - שיצא זוב ממנו ורוצה לבדוק כדי למנות ראיותיו שתים לטומאה ושלש לקרבן. זוב דומה למי בצק של שעורים ובא מבשר המת ושכבת זרע בא מבשר החי וקשורה כלובן ביצה שאינה מוזרת:}
אי הכי מאי איריא מרבה כי לא מרבה נמי כי קתני מרבה אנשים 
\commenta{בצרור - דבר קשה אינו מחמם:}
תנא בד"א לענין שכבת זרע אבל לענין זוב אף הוא משובח כנשים 
\commenta{במטלית עבה - קשה היא ואינה מחממת:}
ואפי' לענין שכבת זרע אם בא לבדוק בצרור או בחרס בודק 
\commenta{שנזדעזעו אבריו - שנעקר זרע מגופו:}
ובמטלית לא והתניא בודק עצמו במטלית ובכל דבר שרוצה כדאמר אביי במטלית עבה הכא נמי במטלית עבה 
\commenta{כאילו מביא מבול - שעבירה זו היתה בידם דכתיב (בראשית ו׳:י״ב) כי השחית כל בשר ואמרו ברותחין קלקלו:}
והיכא איתמר דאביי אהא דתנן היה אוכל בתרומה והרגיש שנזדעזעו איבריו אוחז באמתו ובולע את התרומה 
\commenta{כיון דעקר עקר - ומשום חמום דהשתא לא נפיק מידי:}
אוחז והתניא רבי אליעזר אומר כל האוחז באמתו ומשתין כאילו מביא מבול לעולם אמר אביי במטלית עבה 
רבא אמר אפילו תימא במטלית רכה כיון דעקר עקר ואביי חייש דלמא אתי לאוסופי ורבא לא חייש דלמא אתי לאוסופי
\commenta{לא שכיח - וכי קתני למה זה דומה דתחילתו לא יצא אלא ע"י משמוש האבר שאינו נעקר יחד אלא כשהוא ממשמש יוצא מעט מעט והולך אבל היכא דיצא מהרגשה ונעקר כולו כאחד תו לא מוסיף:}
 ולא והתניא הא למה זה דומה לנותן אצבע בעין שכל זמן שאצבע בעין עין מדמעת וחוזרת ומדמעת 
\commenta{ניצוצות ניתזין - שאינו אוחז ואינו מטיל למרחוק:}
ורבא כל אחמומי והדר אחמומי בשעתיה לא שכיח 
גופא ר"א אומר כל האוחז באמה ומשתין כאילו מביא מבול לעולם אמרו לו לרבי אליעזר והלא נצוצות נתזין על רגליו ונראה ככרות שפכה ונמצא מוציא לעז על בניו שהן ממזרים 
אמר להן מוטב שיוציא לעז על בניו שהן ממזרים ואל יעשה עצמו רשע שעה אחת לפני המקום 
\commenta{איסורא - אסור לאחוז ואפילו אי אפשר לו בענין אחר אע"פ שמוציא לעז כדקתני מוטב שיוציא לעז כו':}
תניא אידך אמר להן רבי אליעזר לחכמים אפשר יעמוד אדם במקום גבוה וישתין או ישתין בעפר תיחוח ואל יעשה עצמו רשע שעה אחת לפני המקום 
הי אמר להו ברישא אילימא קמייתא אמר להו ברישא בתר דאמר להו איסורא הדר אמר להו תקנתא 
אלא הא אמר להו ברישא ואמרו ליה אין לו מקום גבוה ועפר תיחוח מאי אמר להן מוטב שיוציא לעז על בניו ואל יעשה עצמו רשע שעה אחת לפני המקום
\commenta{הנחמים באלים - שמתחממים בעצי אילנות במדורות גדולים כלומר בעלי הנאות:}
וכל כך למה מפני שמוציא שכבת זרע לבטלה דא"ר יוחנן כל המוציא שכבת זרע לבטלה חייב מיתה שנאמר (בראשית לח, י) וירע בעיני ה' (את) אשר עשה וימת גם אותו 
רבי יצחק ורבי אמי אמרי כאילו שופך דמים שנאמר (ישעיהו נז, ה) הנחמים באלים תחת כל עץ רענן שוחטי הילדים בנחלים תחת סעיפי הסלעים אל תקרי שוחטי אלא סוחטי 
\commenta{דשף ויתיב - שם מקום המובלע במלכות נהרדעא מפי מורי הזקן:}
רב אסי אמר כאילו עובד עבודת כוכבים כתיב הכא תחת כל עץ רענן וכתיב התם (דברים יב, ב) על ההרים הרמים ותחת כל עץ רענן 
רב יהודה ושמואל הוו קיימי אאיגרא דבי כנישתא דשף ויתיב בנהרדעא אמר ליה רב יהודה לשמואל צריך אני להשתין א"ל שיננא אחוז באמתך והשתן לחוץ 
\commenta{עשאוהו - להא דרב יהודה:}
היכי עביד הכי והתניא ר"א אומר כל האוחז באמתו ומשתין כאילו מביא מבול לעולם 
\commenta{דליליא - ומתיירא שמא יפול:}
אמר אביי עשאו כבולשת דתנן בולשת שנכנס לעיר בשעת שלום חביות פתוחות אסורות סתומות מותרות בשעת מלחמה אלו ואלו מותרות לפי שאין להן פנאי לנסך אלמא דכיון דבעיתי לא אתי לנסוכי הכא נמי כיון דבעיתי לא אתי להרהורי 
והכא מאי ביעתותא איכא איבעית אימא ביעתותא דליליא ודאיגרא ואיבעית אימא ביעתותא דרביה ואב"א ביעתותא דשכינה ואיבעית אימא אימתא דמריה עליה דקרי שמואל עליה אין זה ילוד אשה 
\commenta{עטרה - שפה גבוהה המקפת את ראש הגיד:}
ואיבעית אימא נשוי הוה דאמר רב נחמן אם היה נשוי מותר 
ואיבעית אימא כי הא אורי ליה דתני אבא בריה דרבי בנימין בר חייא אבל מסייע בביצים מלמטה ואיבעית אימא כי הא אורי ליה דאמר רבי אבהו אמר רבי יוחנן גבול יש לו מעטרה ולמטה מותר
מעטרה ולמעלה אסור 
\commenta{וירע - אלמא מיקרי רע:}
אמר רב המקשה עצמו לדעת יהא בנדוי ולימא אסור דקמגרי יצה"ר אנפשיה ורבי אמי אמר נקרא עבריין שכך אומנתו של יצר הרע היום אומר לו עשה כך ולמחר אומר לו עשה כך ולמחר אומר לו לך עבוד עבודת כוכבים והולך ועובד 
\commenta{ביד - מוציא זרע לבטלה:}
איכא דאמרי אמר רבי אמי כל המביא עצמו לידי הרהור אין מכניסין אותו במחיצתו של הקב"ה כתיב הכא (בראשית לח, י) וירע בעיני ה' וכתיב התם (תהלים ה, ה) כי לא אל חפץ רשע אתה לא יגורך רע 
\commenta{כספחת - דכתיב בגרים (ישעיהו יד) ונלוה הגר עליהם ונספחו על בית יעקב לשון ספחת שקשים לישראל כספחת שאין בקיאים במצות ומביאין פורענות ועוד שמא למדים ישראל ממעשיהן מפי מורי הזקן. ויש אומר שכל ישראל ערבים זה בזה ולאו מילתא היא שלא נתערבו בשביל הגרים דאמרינן במס' סוטה (דף לז:) נמצא לכל אחד מישראל שש מאות אלף ושלשת אלפים ותק"ן בריתות שכולן נתערבו זה בזה אלמא לא נתערבו על הגרים:}
ואמר ר' אלעזר מאי דכתיב (ישעיהו א, טו) ידיכם דמים מלאו אלו המנאפים ביד תנא דבי רבי ישמעאל (שמות כ, יג) לא תנאף לא תהא בך ניאוף בין ביד בין ברגל 
\commenta{בני סקילה נינהו - ואת אמרת מעכבים את המשיח ותו לא הלא במיתה הן נידונים:}
ת"ר הגרים והמשחקין בתינוקות מעכבין את המשיח בשלמא גרים כדר' חלבו דא"ר חלבו קשין גרים לישראל כספחת אלא משחקין בתנוקות מאי היא 
\commenta{דנסיבי קטנות - והוא ראוי להוליד נמצא בטל מפריה ורביה כל ימי קטנותה:}
אילימא משכב זכור בני סקילה נינהו אלא דרך אברים בני מבול נינהו 
\commenta{קץ ידא - דאמרינן בסנהדרין כל המרים ידו על חבירו תקצץ דכתיב וזרוע רמה תשבר. רב הונא קץ ידא אלמא תקצץ דינא הוא:}
אלא דנסיבי קטנות דלאו בנות אולודי נינהו דא"ר יוסי אין בן דוד בא עד שיכלו כל הנשמות שבגוף שנאמר (ישעיהו נז, טז) כי רוח מלפני יעטוף ונשמות אני עשיתי
\commenta{תקצץ ידו על טבורו - קס"ד טבורו ממש והיינו דאמרי ליה והלא כריסו נבקעת כשקוצצין שם את ידו:}
באנשים תקצץ איבעיא להו דינא תנן או לטותא תנן דינא תנן כי הא דרב הונא קץ ידא או לטותא תנן 
\commenta{והלא כריסו נבקעת - מפני הקוץ:}
ת"ש דתניא רבי טרפון אומר יד לאמה תקצץ ידו על טבורו אמרו לו ישב לו קוץ בכריסו לא יטלנו א"ל לא אמר להן מוטב תבקע כריסו ואל ירד לבאר שחת 
אי אמרת בשלמא דינא תנן היינו דאמרי והלא כריסו נבקעת אלא אי אמרת לטותא תנן מאי כריסו נבקעת אלא מאי דינא תנן לא סגי דלאו על טבורו 
\commenta{מתני' מתקנות אותן - בודקות אותו ומטבילין אותן:}
אלא ה"ק רבי טרפון כל המכניס ידו למטה מטבורו תקצץ אמרו לו לרבי טרפון ישב לו קוץ בכריסו לא יטלנו אמר להן לא והלא כריסו נבקעת אמר להן מוטב תבקע כריסו ואל ירד לבאר שחת
\commenta{גמ' ומראות לה - שהיתה בקיאה במראה דם טמא ודם טהור:}
{\large\emph{מתני׳}} החרשת והשוטה והסומא ושנטרפה דעתה אם יש להן פקחות מתקנות אותן והן אוכלות בתרומה
\commenta{מדברת ואינה שומעת - כבר שמעה ולמדה לדבר ואחר כן נתחרשה:}
{\large\emph{גמ׳}} חרשת איהי תבדוק לנפשה דתניא אמר רבי חרשת היתה בשכונתינו לא דיה שבודקת לעצמה אלא שחברותיה רואות ומראות לה 
התם במדברת ואינה שומעת הכא בשאינה מדברת ואינה שומעת כדתנן חרש שדברו חכמים בכל מקום אינו שומע ואינו מדבר
הסומא איהי תבדוק לנפשה ותיחזי לחבירתה א"ר יוסי ברבי חנינא סומא אינה משנה
\commenta{שלא יישן - כדי שלא יתחמם ויראה:}
ושנטרפה דעתה היינו שוטה שנטרפה דעתה מחמת חולי 
\commenta{כיס של עור - וכשירצו להאכילו יבדקו בכיס:}
תנו רבנן כהן שוטה מטבילין אותו ומאכילין אותו תרומה לערב ומשמרין אותו שלא יישן ישן טמא לא ישן טהור 
\commenta{ותבלע בכיס - וקאכיל תרומה בטומאת הגוף והוי במיתה דכתיב (ויקרא כ״ב:ט׳) ומתו בו כי יחללוהו:}
רבי אליעזר ברבי צדוק אומר עושין לו כיס של עור אמרו לו כל שכן שמביא לידי חימום אמר להן לדבריכם שוטה אין לו תקנה 
אמרו לו לדברינו ישן טמא לא ישן טהור לדבריך שמא יראה טפה כחרדל ותבלע בכיס 
\commenta{רואין אותן - לענין מי חטאת קאמר לטמא במשא דבעי' עד דדרי כשיעור הזאה כדאמר בפ"ק (לעיל נדה ט.) כמה יהו במים ויהיה בהם כדי הזאה כדי שיטבול ראשי גבעולין ויזה כלומר כדי שיהא בהם כדי להזות לבד מה שהאזוב בולע:}
תנא משום רבי אלעזר אמרו עושין לו כיס של מתכת 
\commenta{שמע מינה - מדקתני כל שכן שאתה מביאו לידי חימום אלמא מכנסים אסורין:}
אמר אביי ושל נחשת כדתניא רבי יהודה אומר רואין אותן גבעולין של אזוב כאילו הן של נחשת 
\commenta{פמלניא - סינר פורצינ"ט. חללו תלוי למטה ברוחב שלא יתעטף האבר ויתחמם:}
אמר רב פפא שמע מינה מכנסים אסורים והכתיב (שמות כח, מב) ועשה להם מכנסי בד לכסות בשר ערוה 
ההוא כדתניא מכנסי כהנים למה הן דומין כמין פמלניא של פרשים למעלה עד מתנים למטה עד ירכים ויש להם שנצים ואין להם לא בית הנקב ולא בית הערוה 
אמר אביי
\clearpage}

\newsection{דף יד}
\twocol{רוכבי גמלים אסורין לאכול בתרומה תניא נמי הכי רוכבי גמלים כולם רשעים הספנים כולם צדיקים
\commenta{דמכף - לשון אוכף של חמור. אם יש לו אוכף האבר אינו מתחמם שהעץ קשה הוא ואם אין לו אוכף מתחמם בבשר החמור אבל דרך גמלים לרכוב בלא אוכף אלא במרדעת ומתחמם:}
החמרים מהן רשעים מהן צדיקים איכא דאמרי הא דמכף הא דלא מכף ואיכא דאמרי הא דמטרטין הא דלא מטרטין 
\commenta{אפרקיד - פניו למעלה וגנות הוא שפעמים שיתקשה אבר תוך שינתו ויתגלה ועוד שידיו מונחות לו על אברו ומתחמם:}
ריב"ל לייט אמאן דגני אפרקיד איני והאמר רב יוסף פרקדן לא יקרא קרית שמע קרית שמע הוא דלא יקרא הא מגנא שפיר דמי 
\commenta{מצלי - מוטה מעט על צדו: }
לענין מגנא כי מצלי שפיר דמי לענין ק"ש כי מצלי אסור והא ר' יוחנן מצלי וקרי ק"ש שאני רבי יוחנן דבעל בשר הוה
\commenta{מתני' אחד לו ואחד לה - לקנח לאחר תשמיש:}
{\large\emph{מתני׳}} דרך בנות ישראל משמשות בשני עדים אחד לו ואחד לה והצנועות מתקנות שלישי לתקן את הבית 
\commenta{נמצא דם על שלו - ואפילו לאחר זמן ששהה לאחר בעילה זמן ארוך קודם קינוח בידוע שהיה דם בשעת תשמיש: }
נמצא על שלו טמאין וחייבין קרבן נמצא על שלה אותיום טמאין וחייבין בקרבן נמצא על שלה לאחר זמן טמאין מספק ופטורים מן הקרבן 
\commenta{את פניה - פניה שלמטה:}
איזהו אחר זמן כדי שתרד מן המטה ותדיח פניה ואח"כ מטמאה מעת לעת ואינה מטמאה את בועלה ר"ע אומר אף מטמאה את בועלה 
מודים חכמים לרבי עקיבא ברואה כתם שמטמאה את בועלה
\commenta{גמ' מאכולת - כינה. ואמאי טמאין ודאי לשרוף תרומה ולהתחייב בקרבן ודאי להוי ספק לתלות תרומה ולהביא אשם תלוי ולא חטאת דדלמא דם מאכולת הוא שהיה באותו מקום באשה וכשבעל נדבק בו:}
{\large\emph{גמ׳}} וניחוש דלמא דם מאכולת הוא אמר רבי זירא אותו מקום בדוק הוא אצל מאכולת ואיכא דאמרי דחוק הוא אצל מאכולת 
\commenta{רצופה - מעוכה על העד רחוק מן הדם קצת. ללישנא דבדוק כיון דקים להו לרבנן דאין שם מאכולת ודאי דם מגופה הוא שהרי העד היה בדוק לה קודם הקינוח ומאכולת (מעלמא הואי התם בעד שהיא רצופה) וללישנא דדחוק איכא לספוקי שמא האי דם ממאכולת ומתוך שדחוק הוא מעכה השמש בשעת תשמיש:}
מאי בינייהו איכא בינייהו דאשתכח מאכולת רצופה להך לישנא דאמר בדוק הוא הא מעלמא אתאי להך לישנא דאמר דחוק הוא אימא שמש רצפה 
\commenta{ולמחר מצאה עליה דם - על הירך:}
אתמר בדקה בעד הבדוק לה וטחתו בירכה ולמחר מצאה עליה דם אמר רב טמאה נדה א"ל רב שימי בר חייא והא חוששת אמרת לן 
\commenta{טמאה נדה - טומאה ודאית דכיון דהעד בדוק לה שלא היה בו דם קודם בדיקה ודאי מגופה אתאי על העד ומשם הוטח בירכה ולא אמרינן דלמא מעלמא אתא על ירכה וספק טומאה היא:}
איתמר נמי אמר שמואל טמאה נדה וכן מורין בי מדרשא טמאה נדה 
\commenta{ולמחר מצאה עליו דם - על העד ואיכא לספוקי דלמא מקמי הכי הוה ביה. להכי לא נקט הכא טחתו בירכה דא"כ הוו להו תרי ספיקי לקולא חדא דלמא לאו מן העד הוטח על ירכה ואפילו את"ל מן העד אתא דלמא כיון דאינו בדוק מקמי הכי הוה:}
אתמר בדקה בעד שאינו בדוק לה והניחתו בקופסא ולמחר מצאה עליו דם א"ר יוסף כל ימיו של ר' חייא טימא ולעת זקנתו טיהר 
\commenta{משום נדה - טומאה ודאית דכיון דחזקת דמים שם לא מספקינן במקמי הכי:}
איבעיא להו היכי קאמר כל ימיו טימא משום נדה ולעת זקנתו טיהר משום נדה וטימא משום כתם 
או דלמא כל ימיו טימא משום כתם ולעת זקנתו טיהר מולא כלום 
תא שמע דתניא בדקה בעד שאינו בדוק לה והניחתו בקופסא ולמחר מצאה עליו דם רבי אומר טמאה משום נדה ורבי חייא אמר טמאה משום כתם
אמר לו ר' חייא אי אתה מודה שצריכה כגריס ועוד א"ל אבל אמר לו א"כ (אתה) אף אתה עשיתו כתם 
\commenta{לאפוקי מדם מאכולת - דכיון דאינו בדוק לה כל כמה דליכא כגריס ועוד הוה לה לספוקי בדם מאכולת קודם בדיקה:}
ורבי סבר בעינן כגריס ועוד לאפוקי מדם מאכולת וכיון דנפק לה מדם מאכולת ודאי מגופה אתא 
\commenta{מאי לאו בזקנותו קאי - מדפליג עליה דרבי רביה:}
מאי לאו בזקנותו קאי הא בילדותו טימא משום נדה שמע מינה 
משתבח ליה רבי לרבי ישמעאל ברבי יוסי ברבי חמא בר ביסא דאדם גדול הוא אמר לו לכשיבא לידך הביאהו לידי 
\commenta{אמר ליה - רבי ישמעאל לרבי חמא בעי מינאי מילתא:}
כי אתא א"ל בעי מינאי מילתא בעא מיניה בדקה בעד שאינו בדוק לה והניחתו בקופסא ולמחר מצאה עליו דם מהו 
\commenta{כדברי אבא - רבי יוסי או כדברי רבי ולקמן תני פלוגתייהו בשמעתין:}
אמר לו כדברי אבא אימא לך או כדברי רבי אימא לך א"ל כדברי רבי אימא לי 
אמר רבי ישמעאל זהו שאומרין עליו דאדם גדול הוא היאך מניחין דברי הרב ושומעין דברי התלמיד 
ור' חמא בר ביסא סבר רבי ריש מתיבתא הוא ושכיחי רבנן קמיה ומחדדי שמעתתיה 
מאי רבי ומאי רבי יוסי אמר רב אדא בר מתנא תנא רבי מטמא ורבי יוסי מטהר 
\commenta{לעצמו טיהר - לטעמיה אזיל דאמר נמי הכי בדוכתא אחריתי (לקמן נדה דף נט:):}
ואמר רבי זירא כשטימא רבי כר"מ וכשטיהר רבי יוסי לעצמו טיהר 
\commenta{צרכיה - מטילה מים:}
דתניא האשה שהיתה עושה צרכיה וראתה דם ר"מ אומר אם עומדת טמאה אם יושבת טהורה 
רבי יוסי אומר בין כך ובין כך טהורה 
\commenta{משום כתם - טומאת ספק לתלות ואילו רבי לעיל טימא משום נדה דקתני רבי אומר טמאה נדה:}
א"ל רב אחא בריה דרבא לרב אשי והא א"ר יוסי בר' חנינא כשטימא ר"מ לא טימא אלא משום כתם ואילו רבי משום נדה קאמר א"ל אנן הכי קאמרינן כי איתמר ההיא משום נדה איתמר
נמצא על שלה אותיום טמאין וכו' ת"ר איזהו שיעור וסת משל לשמש ועד שעומדין בצד המשקוף ביציאת שמש נכנס עד
\commenta{הוי וסת שאמרו - לחייבה בחטאת:}
הוי וסת שאמרו לקינוח אבל לא לבדיקה
\commenta{תנא וחייבין באשם תלוי - אנמצא לאחר אותיום קאי דקתני במתניתין ופטורין מן הקרבן:}
נמצא על שלה לאחר זמן וכו' תנא וחייבין אשם תלוי ותנא דידן מ"ט 
\commenta{בעינן חתיכה משתי חתיכות - דכתיב באשם תלוי (ויקרא ה׳:י״ז) אחת מכל מצות ה' וקסבר יש אם למקרא מצוות קרינן הילכך בעינן שתים אחת של איסור ואחת של היתר ואכל אחת ואינו יודע איזהו אכל אשתו ואחותו עמו בבית ובא על אחת מהן כסבור אשתו היא ונודע ששתיהן היו במטה ואינו יודע על איזהו מהן בא אבל הכא חדא חתיכה היא ספק שריא ספק אסורה ותנא ברא סבר יש אם למסורת והכי פליגי במסכת כריתות (דף יז:):}
בעינן חתיכה משתי חתיכות
\commenta{שתושיט - ועודה במטה הא כדי שתרד שיעורא רבה ואינה מטמאה את בועלה אלא כשאר מגע מעת לעת:}
איזהו אחר זמן וכו' ורמינהי איזהו אחר זמן פירש ר' אליעזר ברבי צדוק כדי שתושיט ידה תחת הכר או תחת הכסת ותטול עד ותבדוק בו 
\commenta{מאי אחר - דקתני במתני' איזהו אחר זמן כדי שתרד לאו אאחר אותיום קאי דטמאים מספק אלא איזהו אחר זמן שהוא אחר אותו שלאחר אותיום שאינה מטמאה את בועלה לרבנן אלא לר"ע כדי שתרד כו':}
אמר רב חסדא מאי אחר אחר אחר 
\commenta{
והא עלה קתני כו' - אלמא אטמאין מספק קאי משום שמטמאה את בועלה אפי' לרבנן:}
והא קתני עלה נמצא על שלה לאחר זמן טמאין מספק ופטורין מן הקרבן איזהו אחר זמן כדי שתרד מן המטה ותדיח פניה 
\commenta{הכי קאמר - חסורי מחסרא והכי קתני איזהו אחר זמן שהן טמאין מספק כדי שתושיט אבל שהתה כדי שתרד מטמאה מעת לעת ואינה מטמאה את בועלה רבי עקיבא אומר מטמאה את בועלה ואפילו כל מעת לעת:
}
ה"ק איזהו אחר זמן כדי שתושיט ידה לתחת הכר או לתחת הכסת ותטול עד ותבדוק בו וכדי שתרד מן המטה ותדיח את פניה מחלוקת ר"ע וחכמים 
\commenta{והא אח"כ קתני - דקתני במתני' ואח"כ מטמאה מעת לעת ואינה מטמאה כו' אלמא כששהתה יותר מכדי שתרד הוא דפליגי אבל בכדי שתרד טומאה אפילו לרבה וקשיא ברייתא דכדי שתושיט:}
והא אח"כ קתני ה"ק וזהו אח"כ שנחלקו ר"ע וחכמים 
\commenta{עד בידה - שאינה צריכה להושיט ידה תחת הכר הוי שיעורא בכדי שתרד ותדיח באותו עד:}
רב אשי אמר אידי ואידי חד שיעורא הוא עד בידה כדי שתרד מן המטה ותדיח את פניה אין עד בידה כדי שתושיט ידה לתחת הכר או לתחת הכסת ותטול עד ותבדוק בו 
מיתיבי איזהו אחר זמן דבר זה שאל רבי אלעזר ברבי צדוק לפני חכמים באושא ואמר להם
\clearpage}

\newsection{דף טו}
\twocol{שמא כרבי עקיבא אתם אומרים שמטמאה את בועלה אמרו לו לא שמענו 
\commenta{לא שהתה כדי שתרד - אלא כדי שתשב ותבדוק ועודה במטה טמאין מספק דהיינו אחר אות יום כרב חסדא:}
אמר להם כך פרשו חכמים ביבנה לא שהתה כדי שתרד מן המטה ותדיח את פניה תוך זמן הוא זה וטמאין מספק ופטורין מקרבן וחייבין באשם תלוי 
\commenta{שהתה כדי שתרד - דנפיש שיעוריה: }
שהתה כדי שתרד מן המטה ותדיח את פניה אחר הזמן הוא זה 
\commenta{וכן מעת לעת - כלומר כשיעור הזה ומכאן ולהלן כל מעת לעת של בעילה אם בדקה ומצאתה טמאה בועלה מטמאה משום נוגע במעת לעת בנדה טומאת ערב וספק מדרבנן לתלות ואינו מטמא משום בועל נדה טומאת שבעה ואפילו לתלות:}
וכן כששהתה מעת לעת ומפקידה לפקידה בועלה מטמא משום מגע ואינו מטמא משום בועל רבי עקיבא אומר אף מטמא משום בועל רבי יהודה בנו של רבן יוחנן בן זכאי אומר בעלה נכנס להיכל ומקטיר קטורת 
\commenta{בשלמא לרב חסדא - דאמר כדי שתרד אחר אחר הוא היינו דמטהרי רבנן:}
בשלמא לרב חסדא היינו דמטהרי רבנן 
אלא לרב אשי אמאי מטהרי רבנן 
\commenta{וכי תימא דאין עד בידה - דאחר שתרד הוצרכה להושיט דהוה להו תרי שיעורי א"כ איבעי ליה למיתני ולפלוגי בין עד בידה לאין עד בידה דלא תיפוק חורבה מיניה דהשתא משמע דאפילו עד בידה דלא שהתה אלא כדי שתרד מטהרי רבנן ומדלא מפליג ש"מ דודאי מטהרי רבנן בכדי שתרד לחודיה:}
וכי תימא דאין עד בידה האי עד בידה ואין עד בידה מיבעי ליה קשיא
רבי יהודה בנו של רבן יוחנן בן זכאי אומר בעלה נכנס להיכל ומקטיר קטורת ותיפוק ליה דהוה נוגע במעת לעת שבנדה 
הוא דאמר כשמאי דאמר כל הנשים דיין שעתן 
\commenta{בשלא גמר ביאתו - ואפילו הכי לר"ע מטמאה את בועלה דהעראה בנדה כגמר ביאה דכתיב (ויקרא כ׳:י״ח) את מקורה הערה אבל לענין קרי עד שיזריע:}
ותיפוק ליה דהוה בעל קרי בשלא גמר ביאתו
\commenta{ור"מ היא - דמחמיר בכתמים כדאמר בפ"ק (לעיל נדה ה.) ולקמן בפרק בא סימן (נדה דף נב:):}
ומודים חכמים לרבי עקיבא ברואה כתם אמר רב למפרע ורבי מאיר היא 
\commenta{פשיטא - מאי מודי הא בלמפרע פליגי לגבי מעת לעת ולא במכאן ולהבא:}
ושמואל אמר מכאן ולהבא ורבנן היא מכאן ולהבא פשיטא 
\commenta{וכתמים דרבנן - אפי' מכאן ולהבא:}
מהו דתימא הואיל ומעת לעת דרבנן וכתמים דרבנן מה מעת לעת לא מטמאה את בועלה אף כתמים לא מטמאה את בועלה קא משמע לן 
\commenta{אין שור שחוט - דאתמול לא חזאי אלא סייג בעלמא הוא:}
ואימא הכי נמי התם אין שור שחוט לפניך הכא יש שור שחוט לפניך 
וכן אמר ריש לקיש למפרע ורבי מאיר היא רבי יוחנן אמר מכאן ולהבא ורבנן היא
\commenta{מתני' כל הנשים בחזקת טהרה לבעליהן - בלא בדיקה שלא הוזכרה בדיקה אלא לעסוקה בטהרות:}
{\large\emph{מתני׳}} כל הנשים בחזקת טהרה לבעליהן הבאין מן הדרך נשיהן להן בחזקת טהרה
{\large\emph{גמ׳}} למה ליה למתני הבאין מן הדרך סד"א הני מילי היכא דאיתיה במתא דרמיא אנפשה ובדקה אבל היכא דליתא במתא דלא רמיא אנפשה לא קא משמע לן 
\commenta{גמ' בתוך ימי עונתה - ל' יום לראייה אבל לאחר ל' בעיא בדיקה הואיל וסתם נשים חזיין לסוף עונה:}
אמר ריש לקיש משום רבי יהודה נשיאה והוא שבא ומצאה בתוך ימי עונתה 
אמר רב הונא ל"ש אלא שאין לה וסת אבל יש לה וסת אסור לשמש 
כלפי לייא אדרבה איפכא מסתברא אין לה וסת אימא חזאי יש לה וסת וסת קביע לה 
\commenta{לא שנו - דכי מצאה תוך ימי עונתה לא בעיא בדיקה:}
אלא אי איתמר הכי איתמר אמר רב הונא ל"ש אלא שלא הגיע שעת וסתה אבל הגיע שעת וסתה אסורה קסבר וסתות דאורייתא 
\commenta{וסתות דרבנן - הצריכוה חכמים לבדוק ביום וסתה שמא תראה ומיהו היכא דלא הוה בעיר ולא ידעינן אי בדקה אי לא בדקה לא מספקינן לה בטומאה:}
רבה בר בר חנה אמר אפילו הגיע שעת וסתה נמי מותרת קסבר וסתות דרבנן 
רב אשי מתני הכי אמר רב הונא
לא שנו אלא שאין לה וסת לימים אלא יש לה וסת לימים ולקפיצות כיון דבמעשה תליא מילתא אימא לא קפיץ ולא חזאי אבל יש לה וסת לימים אסורה לשמש
קסבר וסתות דאורייתא 
רבה בר בר חנה אמר אפילו יש לה וסת לימים מותרת קסבר וסתות דרבנן 
\commenta{מחשב ימי וסתה - אם שהה בדרך שבעה ימים אחר וסתה דעכשיו עומדת בימים שיכולה להטהר אפילו הגיע וסתה אמרינן טבלה ואין צריך לשואלה:}
אמר רב שמואל משמיה דרבי יוחנן אשה שיש לה וסת בעלה מחשב ימי וסתה ובא עליה 
\commenta{דבזיזא - בושה היא לטבול עד שיפייסנה:}
אמר ליה רב שמואל בר ייבא לרבי אבא אמר רבי יוחנן אפילו ילדה דבזיזא למטבל 
\commenta{ודאי ראתה מי א"ר יוחנן - שלא יהא צריך לשואלה ואפילו היא זקנה:}
אמר ליה אטו ודאי ראתה מי אמר רבי יוחנן אימר דאמר רבי יוחנן ספק ראתה ספק לא ראתה ואם תמצא לומר ראתה אימא טבלה
\commenta{הוי ספק - להיתר:}
אבל ודאי ראתה מי יימר דטבלה הוה ליה ספק וודאי ואין ספק מוציא מידי ודאי 
\commenta{והא הכא דודאי טבל - דהא חזינן שנגמרה מלאכתן וממורחין הן:}
ולא והתניא חבר שמת והניח מגורה מלאה פירות אפילו הן בני יומן הרי הן בחזקת מתוקנין והא הכא ודאי טבל ספק מעושר ספק אינו מעושר וקאתי ספק ומוציא מידי ודאי 
התם ודאי וודאי הוא כדרב חנינא חוזאה דאמר רב חנינא חוזאה חזקה על חבר שאינו מוציא מתחת ידו דבר שאינו מתוקן 
\commenta{איבעית אימא ספק וספק הוא - דכי היכי דמספקא לן אי מעשרי אי לא ה"נ מספקא לן דלמא לא טביל ולא איחייב במעשר דאימא כרבי אושעיא עיילה ומכניסה במוץ שלה דאין טבל מתחייב במעשר אלא בראיית פני הבית כשרואה את הפתח וכיון דבשעת ראיית פני הבית לא נגמרה מלאכתו למעשר תו לא מיחייבא אע"פ שהוא דשה וממרחה לאחר שהכניסה מידי דהוה אמכניס פירות דרך גגו וקרפיפו לבית אפילו היא ממורחת דקיימא לן דפטורה מן המעשר הואיל ולא ראתה את הפתח:}
ואיבעית אימא ספק וספק הוא וכדרבי אושעיא דא"ר אושעיא מערים אדם על תבואתו ומכניסה במוץ שלה כדי שתהא בהמתו אוכלת ופטורה מן המעשר 
\commenta{אם זכר הוא כו' - כהן חכם היה ומורה הוראות ומתכוין להורות לה ימי טומאה וטהרה. אי נמי שגורה אצלו בבית ונוגעת בתרומתו ורוצה להזהר ממנה כל ימי טומאתה. וי"א לידע אימתי יהא זמן הבאת קרבנה אם יארע במשמרתו אם לאו. ולא סבירא לי דבשביל תור עולה אחת לכולה משמרה לא עייל האי כהן נפשיה להכי ועוד שהרי היא יכולה לאחר קרבנה כל זמן שתרצה וגם לשון ראשון אינו נראה דאי חכם הוא כלום הוא בא להאהיל ולעבור על לאו דטומאת מת אלא כהן עם הארץ היה או קטן ושגרתו אשה להציץ ולראות:}
ואכתי אין ספק מוציא מידי ודאי והתניא מעשה בשפחתו של מסיק אחד ברימון שהטילה נפל לבור ובא כהן והציץ בו לידע אם זכר אם נקבה
\commenta{וטהרוהו - מאהל המת:}
ובא מעשה לפני חכמים וטהרוהו מפני שחולדה וברדלס מצויים שם 
והא הכא דודאי הטילה נפל ספק גררוהו ספק לא גררוהו וקאתי ספק ומוציא מידי ודאי 
לא תימא הטילה נפל לבור אלא אימא
\clearpage}

\newsection{דף טז}
\twocol{כמין נפל 
והא לידע אם זכר אם נקבה קתני 
ה"ק ובא כהן והציץ בו לידע אם נפל הפילה אם רוח הפילה ואת"ל נפל הפילה לידע אם זכר אם נקבה 
\commenta{ודאי גררוהו - הלכך ודאי וודאי הוא:}
ואיבעית אימא כיון דחולדה וברדלס מצויים שם ודאי גררוהו 
\commenta{וסתות דאורייתא או דרבנן - הא דקיימא לן דבעי בדיקה בשעת וסתה דאורייתא היא דאם לא בדקה טמאה ואפילו בדקה אחר זמן ומצאתה טהורה דאורח בזמנו בא ודם חזאי ונפל לקרקע או דרבנן הוא דאצרכוה ואם לא בדקה טהורה:}
בעו מיניה מרב נחמן וסתות דאורייתא או דרבנן 
\commenta{ולבסוף ראתה - בבדיקה ראשונה שבדקה עצמה לאחר זמן ומצאתה טמאה:}
אמר להו מדאמר הונא חברין משמיה דרב אשה שיש לה וסת והגיע שעת וסתה ולא בדקה ולבסוף ראתה חוששת לוסתה וחוששת לראייתה אלמא וסתות דאורייתא 
\commenta{טעמא דראתה - דכיון דבבדיקה ראשונה מצאתה טמאה אם בדקה נמי בשעת וסתה הוה משכחא:}
איכא דאמרי הכי קא"ל טעמא דראתה הא לא ראתה אין חוששין אלמא וסתות דרבנן 
איתמר אשה שיש לה וסת והגיע שעת וסתה ולא בדקה ולבסוף בדקה אמר רב בדקה ומצאת טמאה טמאה טהורה טהורה ושמואל אמר אפילו בדקה ומצאת טהורה נמי טמאה מפני שאורח בזמנו בא 
לימא בוסתות קמיפלגי דמ"ס דאורייתא ומ"ס דרבנן 
\commenta{דכ"ע - אפי' רב מודה דוסתות דאורייתא והא דקאמר דבמצאתה טהורה טהורה כשבדקה בשיעור וסת של וסתה דודאי אי הוה דם הוה משכחא ושמואל סבר אפי' הכי טמאה דאורח בזמנו בא הואיל ולא בדקה בשעת וסת ממש:}
אמר ר' זירא דכ"ע וסתות דאורייתא כאן שבדקה עצמה כשיעור וסת כאן שלא בדקה עצמה כשיעור וסת 
ר"נ בר יצחק אמר בוסתות גופייהו קמיפלגי דמ"ס וסתות דאורייתא ומר סבר וסתות דרבנן 
\commenta{טמאה נדה - אם הגיע וסתה ולא בדקה. קסבר וסתות דאורייתא:}
אמר רב ששת כתנאי ר' אליעזר אומר טמאה נדה
\commenta{תבדק - לאחר וסתה מצאתה טמאה טמאה משעת וסתה טהורה טהורה:}
ורבי יהושע אומר תבדק והני תנאי כי הני תנאי דתניא רבי מאיר אומר טמאה נדה וחכ"א תבדק 
\commenta{אף אנן נמי תנינא - דלרבי מאיר וסתות דאורייתא:}
אמר אביי אף אנן נמי תנינא דתנן ר"מ אומר אם היתה במחבא והגיע שעת וסתה ולא בדקה טהורה שחרדה מסלקת את הדמים טעמא דאיכא חרדה הא ליכא חרדה טמאה אלמא וסתות דאורייתא 
\commenta{הרואה דם מחמת מכה - שיש לה במעיה:}
לימא הני תנאי בהא נמי פליגי דתניא הרואה דם מחמת מכה אפילו בתוך ימי נדתה טהורה דברי רשב"ג 
\commenta{אם יש לה וסת חוששת לוסתה - כלומר כשאין לה וסת קבוע מודינא לך שאף על פי שעברה עונתה תולה במכה אבל אם יש לה וסת קבוע וראתה ביום וסתה אפי' מחמת מכה חיישינן שמא טיפת נדה מעורבת בו:}
רבי אומר אם יש לה וסת חוששת לוסתה 
מאי לאו בהא קמיפלגי דמר סבר וסתות דאורייתא ומר סבר וסתות דרבנן 
\commenta{מקומו טמא - וכל דם הבא דרך שם ואפילו אינו דם נדה הוי אב הטומאה לטמא אדם טומאת ערב ואם נגעה בטהרות טמאים. וקאמר ליה רבי כו' ולהקל על דבריו בא וה"ק אם יש לה וסת כלומר אם אתה אומר לטמאה מפני וסתה חוששת לוסתה כלומר יש לך לטמאה טומאת שבעה והואיל ואינך חושש לכך שהרי אתה אומר טהורה משום נדה שוב אין לה טומאה:}
אמר רבינא לא דכ"ע וסתות דרבנן והכא במקור מקומו טמא קמיפלגי 
רשב"ג סבר אשה טהורה ודם טמא דקאתי דרך מקור 
ואמר ליה רבי אי חיישת לוסת אשה נמי טמאה ואי לא חיישת לוסת מקור מקומו טהור הוא
\commenta{מתני' שני עדים - חדשים על כל תשמיש אחד לפניו ואחד לאחריו ולמחר תעיין בהם:}
{\large\emph{מתני׳}} בית שמאי אומרים צריכה ב' עדים על כל תשמיש ותשמיש או תשמש לאור הנר בית הלל אומרים דיה בשני עדים כל הלילה:
{\large\emph{גמ׳}} ת"ר אע"פ שאמרו המשמש מטתו לאור הנר הרי זה מגונה בש"א צריכה שני עדים על כל תשמיש או תשמש לאור הנר ובה"א דיה בשני עדים כל הלילה 
\commenta{שמא תראה - בביאה ראשונה ויעמידוהו כותלי בית הרחם וכיון דלא מצריכיתו בדיקה בין תשמיש לתשמיש כי הדר משמש תחפנה שכבת זרע בביאה שניה וכי בדקה לאחר תשמיש אחרון לא מינכר למחר ונהי נמי דלדידן משמשא לאחר קינוח ואע"ג דלא ידעינן עד למחר מיהו בעינן מוכיחה קיים דאי לאו הכי מה הועילו חכמים בתקנתם:}
תניא אמרו להם ב"ש לב"ה לדבריכם ליחוש שמא תראה טיפת דם כחרדל בביאה ראשונה ותחפנה שכבת זרע בביאה שניה 
\commenta{עד שהרוק בתוך הפה כו' - רוק דם. פה אותו מקום. כלומר אפילו כשהיתה בודקת לאחר תשמיש שמא ראתה טפה כחרדל וחפתה שכבת זרע באותה ביאה עצמה:}
א"ל ב"ה אף לדבריכם ליחוש עד שהרוק בתוך הפה שמא נימוק והולך לו 
אמרו להם לפי שאינו דומה נימוק פעם אחת לנימוק שתי פעמים 
\commenta{הארכת - החמרת:}
תניא א"ר יהושע רואה אני את דברי ב"ש אמרו לו תלמידיו רבי כמה הארכת עלינו אמר להם מוטב שאאריך עליכם בעוה"ז כדי שיאריכו ימיכם לעוה"ב 
\commenta{בעל נפש - חסיד:}
אמר ר' זירא מדברי כולם נלמד בעל נפש לא יבעול וישנה 
רבא אמר בועל ושונה כי תניא ההיא לטהרות 
\commenta{בד"א - דבעיא בדיקה:}
תניא נמי הכי בד"א לטהרות אבל לבעלה מותרת ובד"א שהניחה בחזקת טהרה אבל הניחה בחזקת טמאה לעולם היא בחזקתה עד שתאמר לו טהורה אני 
\commenta{בדקה בעד - בלילה לפני תשמיש:}
א"ר אבא א"ר חייא בר אשי אמר רב בדקה בעד ואבד אסורה לשמש עד שתבדוק מתקיף לה ר' אילא אילו איתא מי לא משמשה ואע"ג דלא ידעה השתא נמי תשמש 
\commenta{זו מוכיחה קיים - הרי העד לפנינו ולמחר תראה בו ומהני לה לטהרות דכי אצרכוה רבנן לשמש בעדים משום טהרות הוא וגם אם ימצא דם באותו שלפני תשמיש תביא חטאת כפרתה:}
א"ל רבא זו מוכיחה קיים וזו אין מוכיחה קיים
\commenta{בוזה דרכיו - תשמיש כמו (משלי ל׳:י״ט) דרך גבר בעלמה. והמשמש ביום בא לידי בזיון שמא יראה בה דבר מגונה ותתגנה עליו:}
א"ר יוחנן  אסור לאדם שישמש מטתו ביום אמר רב המנונא מאי קרא שנאמר (איוב ג, ג) יאבד יום אולד בו והלילה אמר הורה גבר לילה ניתן להריון ויום לא ניתן להריון ריש לקיש אמר מהכא (משלי יט, טז) בוזה דרכיו ימות 
ור"ל האי קרא דר' יוחנן מאי דריש ביה מבעי ליה לכדדריש רבי חנינא בר פפא דדריש ר' חנינא בר פפא אותו מלאך הממונה על ההריון לילה שמו ונוטל טפה ומעמידה לפני הקב"ה ואומר לפניו רבש"ע טפה זו מה תהא עליה גבור או חלש חכם או טיפש עשיר או עני 
\commenta{הכל בידי שמים - כל מדותיו וקורותיו של אדם באין לו בגזרת מלך חוץ מזו:}
ואילו רשע או צדיק לא קאמר כדר' חנינא דא"ר חנינא הכל בידי שמים חוץ מיראת שמים שנאמר (דברים י, יב) ועתה ישראל מה ה' אלהיך שואל מעמך כי אם ליראה וגו' 
\commenta{מאי הורה גבר - מדסמך הריון ללילה ש"מ למדרש הכי לילה ניתן להריון ולא יום מדאסמכיה קרא שמעינן לדר' יוחנן ומגופה דקרא שמעינן להא דר' חנינא בר פפא:}
ור' יוחנן א"כ נכתוב קרא גבר הורה מאי הורה גבר לילה ניתן להריון ויום לא ניתן להריון 
\commenta{מיבעי ליה כו' - והיינו בוזה דרכיו שמגנה עצמו שמלעיזין עליו בני המדינה במנהגו:}
ור' יוחנן האי קרא דר"ל מאי דריש ביה מבעי לי' לכדכתיב בספר בן סירא שלשה שנאתי וארבעה לא אהבתי שר הנרגל בבית המשתאות ואמרי לה שר הנרגן ואמרי לה שר הנרגז
\commenta{והמושיב שבת - לתלמידים במרומי קרת אחת שנראה מגסי הרוח ועוד שמפסיקין אותן עוברי דרכים:}
והמושיב שבת במרומי קרת והאוחז באמה ומשתין מים והנכנס לבית חבירו פתאום אמר רבי יוחנן ואפילו לביתו 
אמר רבי שמעון בן יוחאי ארבעה דברים הקב"ה שונאן ואני איני אוהבן הנכנס לביתו פתאום ואצ"ל לבית חבירו והאוחז באמה ומשתין מים
\clearpage}

\newsection{דף יז}
\twocol{ומשתין מים ערום לפני מטתו והמשמש מטתו בפני כל חי אמר ליה רב יהודה לשמואל ואפי' לפני עכברים א"ל שיננא לא אלא כגון של בית פלוני שמשמשין מטותיהן בפני עבדיהם ושפחותיהם 
ואינהו מאי דרוש (בראשית כב, ה) שבו לכם פה עם החמור עם הדומה לחמור 
\commenta{מקרקש זגי - פעמונים התלויים בכילה שסביב מטתו מקשקשן בעת תשמיש לסור בני ביתו משם:}
רבה בר רב הונא מקרקש זגי דכילתא אביי באלי דידבי רבא באלי פרוחי 
\commenta{דמו בראשו - נענש על עצמו וחובת דמו נדרשת ממנו לאחר מיתתו וראיה לדבר ודמו בראשנו דמרגלים (יהושע ב׳:י״ט) חובת דמו יהא מוטל בראשנו:}
אמר ר"ש בן יוחי ה' דברים הן שהעושה אותן מתחייב בנפשו ודמו בראשו האוכל שום קלוף ובצל קלוף וביצה קלופה והשותה משקין מזוגין שעבר עליהן הלילה והלן בבית הקברות והנוטל צפרניו וזורקן לרה"ר והמקיז דם ומשמש מטתו
\commenta{בסילתא - סל:}
האוכל שום קלוף כו' ואע"ג דמנחי בסילתא ומציירי וחתימי רוח רעה שורה עליהן ולא אמרן אלא דלא שייר בהן עיקרן או קליפתן אבל שייר בהן עיקרן או קליפתן לית לן בה
והשותה משקין מזוגין שעבר עליהן הלילה אמר רב יהודה אמר שמואל והוא שלנו בכלי מתכות אמר רב פפא וכלי נתר ככלי מתכות דמו וכן אמר רבי יוחנן והוא שלנו בכלי מתכות וכלי נתר ככלי מתכות דמו
\commenta{רוח טומאה - כמדת הלצים שקורין גרמנטי"ר האוחזים את העיניים:}
והלן בבית הקברות כדי שתשרה עליו רוח טומאה זימנין דמסכנין ליה
\commenta{בגנוסטרי - מספרים:}
והנוטל צפרניו וזורקן לרשות הרבים מפני שאשה מעוברת עוברת עליהן ומפלת ולא אמרן אלא דשקיל בגנוסטרי ולא אמרן אלא דשקיל דידיה ודכרעיה ולא אמרן אלא דלא גז מידי בתרייהו אבל גז מידי בתרייהו לית לן בה ולא היא לכולה מילתא חיישינן 
\commenta{חסיד - עדיף מצדיק. כשקוברן איכא למיחש דהדרי ומגלו:}
ת"ר ג' דברים נאמרו בצפרנים שורפן חסיד קוברן צדיק זורקן רשע
\commenta{ויתקין - חלשים:}
והמקיז דם ומשמש מטתו דאמר מר מקיז דם ומשמש מטתו הויין לו בנים ויתקין הקיזו שניהם ושמשו הויין לו בנים בעלי ראתן אמר רב ולא אמרן אלא דלא טעים מידי אבל טעים מידי לית לן בה 
אמר רב חסדא אסור לו לאדם שישמש מטתו ביום שנאמר (ויקרא יט, יח) ואהבת לרעך כמוך מאי משמע אמר אביי שמא יראה בה דבר מגונה ותתגנה עליו אמר רב הונא ישראל קדושים הם ואין משמשין מטותיהן ביום 
\commenta{תלמיד חכם - דידע לאצטנועי נפשיה:}
אמר רבא ואם היה בית אפל מותר ות"ח מאפיל בכסותו ומשמש 
תנן או תשמש לאור הנר אימא תבדוק לאור הנר 
\commenta{אף על פי שאמרו המשמש מטתו כו' - כדאיתא בברייתא דלעיל בריש שמעתין ומסקנא או תשמש לאור הנר אלמא מגונה הוא דהוי אבל איסורא ליכא מדקתני או תשמש והשתא ליכא למימר תבדוק דא"כ מאי אע"פ שאמרו כו':}
ת"ש אע"פ שאמרו המשמש מטתו לאור הנר הרי זה מגונה אימא הבודק מטתו לאור הנר הרי זה מגונה 
\commenta{מילא פרהבא - קוטו"ן. מתוך שהוא לבן נראית בו טיפה כחרדל. לישנא אחרינא צמר נקי ורך:}
תא שמע ושל בית מונבז המלך היו עושין ג' דברים ומזכירין אותן לשבח היו משמשין מטותיהם ביום ובודקין מטותיהם במילא פרהבא ונוהגין טומאה וטהרה בשלגים קתני מיהא משמשין מטותיהן ביום 
\commenta{ומזכירין חכמים אותן לשבח - בתמיה. נהי אי הוה מותר בעלמא הוה תמיה לן וכ"ש שמזכירין לשבח ומה הוא שבחן:}
אימא בודקין מטותיהם ביום הכי נמי מסתברא דאי ס"ד משמשין מזכירין אותן לשבח אין ה"נ דאגב דאיכא אונס שינה מגניא באפיה 
\commenta{פקולין - קוטו"ן:}
ובודקין מטותיהן במילא פרהבא מסייע ליה לשמואל דאמר שמואל אין בודקין את המטה אלא בפקולין או בצמר נקי ורך אמר רב היינו דכי הואי התם בערבי שבתות הוו אמרי מאן בעי פקולי בנהמא ולא ידענא מאי קאמרי 
\commenta{שחקי דכיתנא - בגדים בלויין של פשתן:}
אמר רבא הני שחקי דכיתנא מעלי לבדיקה איני והא תנא דבי מנשה אין בודקין את המטה לא בעד אדום ולא בעד שחור ולא בפשתן אלא בפקולין או בצמר נקי ורך 
לא קשיא הא בכיתנא הא במאני דכיתנא ואיבעית אימא הא והא במאני דכיתנא הא בחדתי הא בשחקי 
\commenta{חישב עליו וכו' - דקרוב הוא להיות משקה יותר מאוכל:}
נוהגין טומאה וטהרה בשלגין תנן התם שלג אינו לא אוכל ולא משקה חישב עליו לאכילה אינו מטמא טומאת אוכלין למשקה מטמא טומאת משקין 
\commenta{לא נטמא כו' - דאינו חבור אלא כל קורט בפני עצמו חשוב ואינו טמא אלא מקום מגע טומאה:}
נטמא מקצתו לא נטמא כולו נטהר מקצתו נטהר כולו 
הא גופא קשיא אמרת נטמא מקצתו לא נטמא כולו והדר תני נטהר מקצתו נטהר כולו למימרא דנטמא כולו 
\commenta{שהעבירו - לשלג:}
אמר אביי כגון שהעבירו על אויר תנור דהתורה העידה על כלי חרס
אפילו מלא חרדל 
\commenta{מתני' החדר והפרוזדור - בגמ' מפרש היכי קיימי:}
{\large\emph{מתני׳}} משל משלו חכמים באשה החדר והפרוזדור והעלייה 
\commenta{דם החדר טמא - והיינו מקור:}
דם החדר טמא דם העלייה טהור נמצא בפרוזדור ספקו טמא לפי שחזקתו מן המקור
\commenta{גמ' חדר מבפנים ופרוזדור מבחוץ - שניהם זה אצל זה בעובי גופה חדר לצד אחוריה ופרוזדור לפניה וכותלי רחם למטה באמצע פרוזדור ודרך שם דמים יוצאים:}
{\large\emph{גמ׳}} רמי בר שמואל ורב יצחק בריה דרב יהודה תנו נדה בי רב הונא אשכחינהו רבה בר רב הונא דיתבי וקאמרי החדר מבפנים והפרוזדור מבחוץ ועלייה בנויה על שתיהן ולול פתוח בין עלייה לפרוזדור
\commenta{ספקו טמא - דאי מעלייה אתא מן הלול ולחוץ איבעי ליה לאשתכוחי דהא כי נחית מן הלול לפרוזדור כלפי חוץ הוא יורד ויוצא דרך יציאה ואינו חוזר לצד אחוריה. ולקמן פריך ודאי טמא הוי ואמאי קתני ספקו טמא:}
 נמצא מן הלול ולפנים ספקו טמא מן הלול ולחוץ ספקו טהור 
\commenta{ספקו טמא - ולא ודאי מדקתני מן הלול ולחוץ ספקו טהור:}
אתא ואמר ליה לאבוה ספקו טמא אמרת לן מר והא אנן שחזקתו מן המקור תנן 
\commenta{מן הלול ולפנים ודאי טמא - ומתניתין מן הלול ולפנים:}
א"ל אנא הכי קאמינא מן הלול ולפנים ודאי טמא מן הלול ולחוץ ספקו טמא 
\commenta{דספקו טמא - ולא אמרינן ודאי טהור:}
אמר אביי מאי שנא מן הלול ולחוץ דספקו טמא דדלמא שחתה ומחדר אתא מן הלול ולפנים נמי אימא אזדקרה ומעלייה אתא 
\commenta{אי בתר חששא אזלת - דלא מחזקת ליה לדם כמנהגו והילוכו אלא באת לחוש לקורות הנולדות כגון שחתה:}
אלא אמר אביי אי בתר חששא אזלת אידי ואידי ספק הוא ואי בתר חזקה אזלת מן הלול ולפנים ודאי טמא מן הלול ולחוץ ודאי טהור 
\commenta{חייבין עליו - אם נכנסה למקדש דודאי טמאה. ולקמן מפרש אי מן הלול ולפנים אי לחוץ:}
תני רבי חייא דם הנמצא בפרוזדור חייבין עליו על ביאת מקדש ושורפין עליו את התרומה ורב קטינא אמר אין חייבין עליו על ביאת מקדש ואין שורפין עליו את התרומה 
\commenta{להך לישנא - דאי בתר חששא אידי ואידי ספקא הוא:}
להך לישנא דאמר אביי אי בתר חששא אזלת מסייע ליה לרב קטינא ופליגא דרבי חייא
\commenta{להך לישנא - דבתר חזקה מסייע ליה לר' חייא ומתוקמא במן הלול ולפנים:}
להך לישנא דאמרת אי בתר חזקה אזלת מסייע ליה לרבי חייא
\clearpage}

\newsection{דף יח}
\twocol{ופליגא דרב קטינא 
\commenta{לרב הונא - דאמר ליה לבריה מן הלול ולפנים ודאי מן הלול ולחוץ ספקא:}
לרב הונא לא פליגי כאן מן הלול ולפנים כאן מן הלול ולחוץ 
\commenta{אלא לרמי בר שמואל ולרב יצחק - דאמרי מן הלול ולחוץ ספק טהור - על כרחך הני מתני' דר' חייא ודרב קטינא במאי מיתוקמי במן הלול ולפנים דהא מן הלול ולחוץ אפילו ספקא ליכא דהא טהור קאמרינן ודרב קטינא קיימא כוותייהו אלא רבי חייא דאמר טומאה ודאית לימא פליגא עלייהו דלדידהו ליכא טומאה ודאית: }
אלא לרמי בר שמואל ולרב יצחק בריה דרב יהודה דאמרי מן הלול ולחוץ ספקו טהור מן הלול ולפנים ספקו טמא הני במאי מתוקמא מן הלול ולפנים
לימא פליגא דרבי חייא 
\commenta{ומשנינן לא פליגי הא דרבי חייא דאמר ודאי במן הלול ולפנים בקרקע פרוזדור דהיינו כדרך יציאתו מן החדר ודרמי ורב יצחק דאמרי ספק טמא בגג פרוזדור שקרוב לעלייה ולעולם בתר חזקה אזלי ומודו בקרקע פרוזדור במתני' וכדרבי חייא אבל בגגו במן הלול ולפנים על כרחך יצא זה ממנהג הילוכו דהבא מן החדר היה לו לימצא בקרקע פרוזדור שהחדר נמוך ושוה לפרוזדור ואם מן העלייה בא הוה לו לימצא מן הלול ולחוץ הלכך איכא למיתליה בתרי ספיקי דילמא מעלייה אתא לפרוזדור ואזדקרה או מחדר אתא לפרוזדור וע"י שנזדקרה עלה הדם מלמטה למעלה כיון דרמי ורב יצחק לא יהבי טעמא למילתייהו איכא לאוקמי בשינוי אבל לאביי ללישנא דבתר חששא ליכא לשנויי הכי ופליג דר' חייא עליה דכיון דאביי תלי טעמא בחששא וכל היכא דאיכא למיחש להאי ולהאי משוי ליה ספק אפילו בקרקע פרוזדור איכא לספוקי בחדר ולעלייה הילכך דרבי חייא פליג עליה:}
לא קשיא כאן כשנמצא בקרקע פרוזדור וכאן שנמצא בגג פרוזדור 
\commenta{ועשאום כודאי - משום דאזלינן בתר רובא:}
אמר רבי יוחנן בשלשה מקומות הלכו בו חכמים אחר הרוב ועשאום כודאי מקור שליא חתיכה מקור הא דאמרן
\commenta{הבית טמא - אף אם אין ולד נראה הבית טמא משום אהל המת:}
שליא דתנן שליא בבית הבית טמא ולא שהשליא ולד אלא שאין שליא בלא ולד ר"ש אומר נמוק הולד עד שלא יצא 
\commenta{יד חתוכה - מחותכת בצורתה בחיתוך אצבעות:}
חתיכה דתנן המפלת יד חתוכה ורגל חתוכה אמו טמאה לידה ואין חוששין שמא מגוף אטום באת 
ותו ליכא והאיכא תשע חנויות 
\commenta{ספקו אסור - דכל קבוע כמחצה על מחצה דמיא וכיון שלקחה ממקום הקביעות מחזקינן לה כמחצה על מחצה:}
דתניא תשע חנויות כולן מוכרות בשר שחוטה ואחת מוכרת בשר נבלה ולקח מאחת מהן ואינו יודע מאיזה מהן לקח ספקו אסור
\commenta{ובנמצא - בקרקע דה"ל מופרש מהן ולא לקחו ממקום הקביעות:}
ובנמצא הלך אחר הרוב 
טומאה קאמרינן איסור לא קאמרינן 
\commenta{
ברשות היחיד כו' - דהוה ליה ספק ואע"ג דרובא צפרדעים ואין להם טומאה כיון דשרץ קבוע ביניהם הוה ליה מחצה על מחצה:}
והאיכא תשע צפרדעין ושרץ אחד ביניהם ונגע באחד מהן ואינו יודע באיזה מהן נגע ברה"י ספקו טמא ברה"ר ספקו טהור
\commenta{ובנמצא - שפירש אחד מהן למקום אחר ונגע בו אדם שם טהור דהשתא לאו קבוע הוי ואמרינן כל דפריש מרובא פריש וצפרדע הוא:}
ובנמצא הלך אחר הרוב 
טומאה דאשה קאמרינן טומאה בעלמא לא קאמרינן 
והאיכא הא דאמר רבי יהושע בן לוי עברה בנהר
והפילה מביאה קרבן ונאכל
הלך אחר רוב נשים ורוב נשים ולד מעליא ילדן 
\commenta{שמעתתא - ר' יהושע בן לוי אמורא הוא ושלשה מקומות דאמר ר' יוחנן משנה או ברייתא הן:}
מתניתין קאמרינן שמעתתא לא קאמרינן 
\commenta{והא כי אתא רבין אמר - בפ' המפלת (לקמן נדה דף כט.):}
והא כי אתא רבין אמר מתיב רבי יוסי בר רבי חנינא טועה ולא ידענא מאי תיובתיה
\commenta{מאי לאו - ה"ק רבין דהא מתני' לאו תיובתא היא דרבי יהושע אלא סייעתא אלמא מתני' אמרה כרבי יהושע ואשכחן ארבעה מקומות בטומאת נשים במתני' שהלכו אחר הרוב וקשיא לר' יוחנן דלא קחשיב אלא תלת:}
מאי לאו לא תיובתא אלא סייעתא 
לא דלמא לא תיובתא ולא סייעתא 
\commenta{למעוטי מאי - ג' מקומות דקאמר רבי יוחנן דמשמע הני ותו לא למעוטי מאי דמשום רובא לא נשוויה כודאי:}
למעוטי מאי 
\commenta{אילימא למעוטי רובא - בטומאת אשה:}
אילימא למעוטי רובא דאיכא חזקה בהדיה דלא שרפינן עליה את התרומה והא אמרה ר' יוחנן חדא זימנא 
\commenta{לטפח - בשרצים:}
דתנן תינוק הנמצא בצד העיסה ובצק בידו רבי מאיר מטהר וחכמים מטמאין שדרכו של תינוק לטפח 
\commenta{סמוך מיעוטא לחזקה - והוה להו לטהר תרי ורובא לחודיה לטמא ואין א' עומד במקום שנים:}
ואמרינן מאי טעמא דר"מ קסבר רוב תינוקות מטפחין ומיעוט אין מטפחין ועיסה זו בחזקת טהורה עומדת סמוך מיעוטא לחזקה ואיתרע ליה רובא 
ורבנן מיעוטא כמאן דליתיה דמי ורובא וחזקה רובא עדיף 
\commenta{זו היא חזקה כו' - חזקה זו שאנו מחזיקין סתם רוב תינוקות מטפחין חזקה גמורה החזיקוה לעשותה כודאי ולשרוף עליה תרומה:}
ואמר ריש לקיש משום רבי אושעיא זו היא חזקה ששורפין עליה את התרומה ורבי יוחנן אמר אין זו חזקה ששורפין עליה את התרומה 
\commenta{אלא למעוטי רובא דרבי יהודה - דאע"ג דאמר זיל בתר רובא הכי גמר לה ר' יוחנן דטמא דקאמר רבי יהודה לתלות קאמר ולא לשרוף:}
אלא למעוטי רובא דרבי יהודה דתנן המפלת חתיכה אם יש עמה דם טמאה ואם לאו טהורה רבי יהודה אומר בין כך ובין כך טמאה 
\commenta{של ארבע כו' - שהחתיכה מארבע מראות דם טמא השנויין במתני' ואע"ג דחמשה תנן שחור אדום הוא אלא שלקה דכיון דמראה דם טמא יש בה אמרינן כולו דם הוא וטמאה נדה ורבנן סברי אין זה דם אלא בשר:}
ואמר רב יהודה אמר שמואל לא טימא רבי יהודה אלא בחתיכה של ארבע מיני דמים אבל שאר מיני דמים טהורה ורבי יוחנן אמר של ארבע מיני דמים דברי הכל טמאה ושל שאר דמים דברי הכל טהורה לא נחלקו אלא כשהפילה
\clearpage}

\newsection{דף יט}
\twocol{ואינה יודעת מה הפילה רבי יהודה סבר זיל בתר רוב חתיכות ורוב חתיכות של ארבע מיני דמים הויין ורבנן סברי זיל בתר רוב חתיכות לא אמרינן
\commenta{מתני' וכקרן כרכום - כזוית גן שכרכום גדל בו:}
{\large\emph{מתני׳}} חמשה דמים טמאים באשה האדום והשחור וכקרן כרכום וכמימי אדמה וכמזוג בש"א אף כמימי תלתן וכמימי בשר צלי וב"ה מטהרים הירוק עקביא בן מהללאל מטמא וחכמים מטהרין 
\commenta{אם אינו מטמא משום כתם כו' - מפרש בגמרא:}
אמר רבי מאיר אם אינו מטמא משום כתם מטמא משום משקה רבי יוסי אומר לא כך ולא כך 
\commenta{כדם המכה - מפרש בגמ':}
איזהו אדום כדם המכה שחור כחרת עמוק מכן טמא דיהה מכן טהור וכקרן כרכום כברור שבו 
\commenta{בית כרם - מקום ומשם מביא אדמה ונותנה בכלי:}
וכמימי אדמה מבקעת בית כרם ומיצף מים וכמזוג שני חלקים מים ואחד יין מן היין השרוני
{\large\emph{גמ׳}} מנלן דאיכא דם טהור באשה דלמא כל דם דאתי מינה טמא 
אמר רבי חמא בר יוסף אמר רבי אושעיא אמר קרא (דברים יז, ח) כי יפלא ממך דבר למשפט בין דם לדם בין דם טהור לדם טמא 
\commenta{\textbf{גמ' אלא מעתה כו'} עד \textbf{וכולן טמאין} לשון קושיא הוא:}
אלא מעתה בין נגע לנגע הכי נמי בין נגע טמא לנגע טהור וכי תימא ה"נ נגע טהור מי איכא וכי תימא (ויקרא יג, יג) כולו הפך לבן טהור הוא ההוא בוהק מקרי 
אלא בין נגעי אדם לנגעי בתים ולנגעי בגדים וכולן טמאין הכא נמי בין דם נדה לדם זיבה וכולן טמאין 
\commenta{בשלמא התם - גבי נגעים אע"ג דכולהו לטמא איכא לאוקומי קרא דאיירי בפלוגתא דזקן ממרא וסנהדרין שבדורו דכתיב (דברים י״ז:י״ב) והאיש אשר יעשה בזדון כגון דמפלגי בנגעי אדם ובפלוגתא דרבי יהושע כו' דקאמר זקן ממרא כחד מינייהו וסנהדרין כחד מינייהו:}
האי מאי בשלמא התם איכא לאפלוגי בנגעי אדם ובפלוגתא דרבי יהושע ורבנן
\commenta{אם בהרת קודמת לשער לבן טמא - דשער לבן מסימני טומאה וכגון שקדמתו בהרת דכתיב (ויקרא י״ג:כ״ה) והנה נהפך שער לבן בבהרת:}
דתנן אם בהרת קודם לשער לבן טמא ואם שער לבן קודם לבהרת טהור ספק טמא ורבי יהושע אומר כהה ואמר רבה כהה וטהור 
\commenta{שיעור נגע כגריס לאורך ולרוחב:}
בנגעי בתים כי הא פלוגתא דרבי אלעזר ברבי שמעון ורבנן דתנן ר"א בר"ש אומר לעולם אין הבית טמא עד שיראה כשני גריסין על שני אבנים בשני כותלים בקרן זוית ארכו כשני גריסין ורחבו כגריס 
\commenta{כתיב קיר - ומראיהן שפל מן הקיר (שם):}
מ"ט דר"א בר"ש כתיב (ויקרא יד) קיר וכתיב קירות איזהו קיר שהוא כשני קירות הוי אומר זה קרן זוית 
\commenta{פריחה - פרחה צרעת בכולהו:}
בנגעי בגדים בפלוגתא דר' יונתן בן אבטולמוס ורבנן דתניא ר' יונתן בן אבטולמוס אומר מנין לפריחת בגדים שהיא טהורה
נאמר {ויקרא י״ג:ל״ט } קרחת וגבחת בבגדים ונאמר קרחת וגבחת באדם
מה להלן פרח בכולו טהור אף כאן נמי פרח בכולו טהור 
אלא הכא אי דם טהור ליכא במאי פליגי 
\commenta{למימרא דדם אדום הוא - ואלו כולם אדומין הן:}
וממאי דהני טהורין והני טמאין אמר רבי אבהו דאמר קרא (מלכים ב ג:כב) ויראו מואב את המים אדומים כדם למימרא דדם אדום הוא אימא אדום ותו לא 
\commenta{דמיה דמיה - והיא גלתה את מקור דמיה (שם כ) וטהרה ממקור דמיה (שם יב) הרי ד' מיני אדמומית טמאה בה אבל שלמטה מהן לאו דם הוא:}
א"ר אבהו אמר קרא {ויקרא יב, ז} דמיה {ויקרא כ, יח} דמיה הרי כאן ארבעה 
והא אנן חמשה תנן אמר רבי חנינא שחור אדום הוא אלא שלקה 
\commenta{דיהה מכן אפילו - הוא שחור ככחול טהור הואיל ואינו שחור כחרת:}
תניא נמי הכי שחור כחרת עמוק מכן טמא דיהה אפי' ככחול טהור ושחור זה לא מתחלתו הוא משחיר אלא כשנעקר הוא משחיר משל לדם מכה לכשנעקר הוא משחיר
בש"א אף כמימי תלתן ולית להו לב"ש דמיה דמיה הרי כאן ארבעה 
אב"א לית להו ואב"א אית להו מי לא א"ר חנינא שחור אדום הוא אלא שלקה ה"נ מלקא הוא דלקי
וב"ה מטהרין היינו תנא קמא 
איכא בינייהו
לתלות 
הירוק עקביא בן מהללאל מטמא ולית ליה לעקביא דמיה דמיה הרי כאן ארבעה 
אב"א לית ליה ואב"א אית ליה מי לא א"ר חנינא שחור אדום הוא אלא שלקה הכא נמי מלקא הוא דלקי
\commenta{היינו ת"ק - דאמר ה' דמים ותו לא:}
וחכמים מטהרין היינו ת"ק איכא בינייהו לתלות
א"ר מאיר אם אינו מטמא משום כתם כו'
\commenta{ירד ר' מאיר כו' - ומטמא דם ירוק משום נדה:}
א"ר יוחנן ירד ר"מ לשיטת עקביא בן מהללאל וטימא וה"ק להו לרבנן נהי דהיכא דקא משכחת כתם ירוק אמנא לא מטמאיתו היכא דקחזיא דם ירוק מגופה תטמא 
\commenta{תטמא משום משקה - כרוקה וכמי רגליה אם נוגעין בו אדם וכלים וטהרות באותו דם ירוק שראתה בימי נדותה שהיתה נדה מדם אדום ואמאי מטהריתו לגמרי:}
אי הכי אם אינו מטמא משום כתם מטמא משום משקה משום רואה מבעיא ליה 
אלא ה"ק להו נהי היכא דקא חזיא דם ירוק מעיקרא לא מטמאיתו היכא דחזיא דם אדום והדר חזיא דם ירוק תטמא מידי דהוה אמשקה זב וזבה 
\commenta{שמתעגל ויוצא - שמתאסף תחלה ואח"כ יוצא לאפוקי דם בין טמא בין טהור כשהוא בא נוטף ויוצא ראשון ראשון:}
ורבנן דומיא דרוק מה רוק שמתעגל ויוצא אף כל שמתעגל ויוצא לאפוקי האי דאין מתעגל ויוצא אי הכי שפיר קאמרי ליה רבנן לר' מאיר 
\commenta{להוי כמשקה להכשיר זרעים - שאם יגע שרץ בהן יטמא ואמאי מטהריתו לגמרי:}
אלא ה"ק להו להוי כמשקה להכשיר את הזרעים ורבנן בעי (במדבר כג:כד) דם חללים וליכא אי הכי שפיר קאמרי ליה רבנן לר' מאיר 
\commenta{אלפוה בג"ש - להכשיר את הזרעים:}
אלא הכי קאמר להו אלפוה בג"ש כתיב הכא (שיר השירים ד׳:י״ג) שלחיך פרדס רמונים וכתיב התם (איוב ה, י) ושולח מים על פני חוצות 
\commenta{אין אדם דן ג"ש מעצמו - אא"כ קבלה מרבו הל"מ דדלמא קרא למילתא אחריתי איצטריך:}
ורבנן אדם דן ק"ו מעצמו ואין אדם דן ג"ש מעצמו
רבי יוסי אומר לא כך וכו' היינו ת"ק הא קמ"ל מאן ת"ק רבי יוסי וכל האומר דבר בשם אומרו מביא גאולה לעולם
\commenta{שור - דמו אדום כדאמר בסדר יומא (דף נו:) האי חיור והאי סומק: }
איזהו אדום כדם המכה מאי כדם המכה אמר רב יהודה אמר שמואל כדם שור שחוט 
\commenta{דם שחיטה משתנה והולך:}
ולימא כדם שחיטה אי אמר כדם שחיטה הוה אמינא ככולה שחיטה קמ"ל כדם המכה כתחילת הכאה של סכין 
\commenta{צפור חיה - דם היוצא מצפור חי ע"י מכה:}
עולא אמר כדם צפור חיה איבעיא להו חיה לאפוקי שחוט או דלמא לאפוקי כחוש תיקו 
\commenta{הרגה מאכולת - אשה הרואה כתם בבגדה אם הרגה מאכולת תולה בה. ואי קשיא הא שיעור כתם כגריס ועוד ומאכולת כולי האי לא הוי כדאמר בפירקין בעינן כגריס ועוד לאפוקי מדם מאכולת לא קשיא ההיא לר' חנינא בן אנטיגנוס הא לרבנן בפרק הרואה כתם (לקמן נדה דף נח:) דלא בעי האי שיעור:}
זעירי אמר רבי חנינא כדם מאכולת של ראש מיתיבי הרגה מאכולת הרי זה תולה בה מאי לאו דכוליה גופה לא דראשה 
\commenta{אמי ורדינאה - נאה כורד ובגיטין (דף מא.) קרי ליה אמי שפיר נאה. ל"א ורדינאה מקום הוא ששמו ורדינא בעירובין (דף מט.):}
אמי ורדינאה א"ר אבהו כדם אצבע קטנה של יד שנגפה וחייתה וחזרה ונגפה ולא של כל אדם אלא של בחור שלא נשא אשה ועד כמה עד בן עשרים 
\commenta{תולה - כתמה:}
מיתיבי תולה בבנה ובבעלה בשלמא בבנה משכחת לה אלא בעלה היכי משכחת לה 
\commenta{רב נחמן - אמתני' קאי:}
אמר ר"נ בר יצחק כגון שנכנסה לחופה ולא נבעלה 
\commenta{ותלה רבי מאיר - כתם:}
ר"נ אמר כדם הקזה מיתיבי מעשה ותלה ר"מ
\clearpage}

\newsection{דף כ}
\twocol{בקילור ורבי תלה בשרף שקמה מאי לאו אאדום 
לא אשאר דמים 
\commenta{אומנא - מקיז דם בקרן:}
אמימר ומר זוטרא ורב אשי הוו יתבי קמיה אומנא שקלי ליה קרנא קמייתא לאמימר חזייה אמר להו אדום דתנן כי האי שקלי ליה אחריתי אמר להו אשתני אמר רב אשי כגון אנא דלא ידענא בין האי להאי לא מבעי לי למחזי דמא
\commenta{חרת שאמרו דיו - כלומר לא כשחרורית חרת הניתך במים אלא שחרורית חרת הניתך בדיו:}
שחור כחרת אמר רבה בר רב הונא חרת שאמרו דיו תניא נמי הכי שחור כחרת ושחור שאמרו דיו ולימא דיו אי אמר דיו הוה אמינא כי פכחותא דדיותא קמ"ל כי חרותא דדיותא 
\commenta{בלחה - בדיו לחה כדמפרש:}
איבעיא להו בלחה או ביבשתא תא שמע דרבי אמי פלי קורטא דדיותא ובדיק בה 
\commenta{כקיר - שעוה שחורה:}
אמר רב יהודה אמר שמואל כקיר כדיו וכענב טמאה וזוהי ששנינו עמוק מכן טמאה אמר רבי אלעזר כזית כזפת וכעורב טהור וזוהי ששנינו דיהה מכן טהור 
\commenta{עולא אשחור דמתניתין קאי דקתני טמא:}
עולא אמר כלבושא סיואה עולא אקלע לפומבדיתא חזייה לההוא טייעא דלבוש לבושא אוכמא אמר להו שחור דתנן כי האי מרטו מיניה פורתא פורתא יהבו ביה ארבע מאה זוזי 
\commenta{אוליירין - בלנים מחממי מרחצאות. לשון אחר שם מקום:}
רבי יוחנן אמר אלו כלים האוליירין הבאים ממדינת הים למימרא דאוכמי נינהו והאמר להו רבי ינאי לבניו בני אל תקברוני לא בכלים שחורים ולא בכלים לבנים שחורים שמא אזכה ואהיה כאבל בין החתנים לבנים שמא לא אזכה ואהיה כחתן בין האבלים אלא בכלים האוליירין הבאים ממדינת הים 
אלמא לאו אוכמי נינהו לא קשיא הא בגלימא הא בפתורא 
\commenta{וכולן - חמשה דמים:}
אמר רב יהודה אמר שמואל וכולם אין בודקין אלא על גבי מטלית לבנה אמר רב יצחק בר אבודימי ושחור על גבי אדום 
אמר רב ירמיה מדפתי ולא פליגי הא בשחור הא בשאר דמים מתקיף לה רב אשי אי הכי לימא שמואל חוץ משחור אלא אמר רב אשי בשחור גופיה קמיפלגי 
\commenta{כולם - הדמים:}
אמר עולא כולן עמוק מכן טמא דיהה מכן טהור כשחור 
\commenta{
אפי' דיהה מכן ליטמא - דהאי אדום הוה וע"י לקותא הושחר:}
ואלא מאי שנא שחור דנקט סד"א הואיל ואמר רבי חנינא שחור אדום הוא אלא שלקה הילכך אפילו דיהה מכן נמי ליטמא קמשמע לן 
\commenta{דיהה דדיהה - שנדחה ממראיתו יותר מדאי:}
רבי אמי בר אבא אמר וכולן עמוק מכן טמא דיהה מכן נמי טמא חוץ משחור אלא מאי אהני שיעוריה דרבנן לאפוקי דיהה דדיהה 
ואיכא דאמרי רמי בר אבא אמר וכולן עמוק מכן טהור דיהה מכן טהור חוץ משחור ולהכי מהני שיעוריה דרבנן 
\commenta{עמוק מכן טהור - אם מאותו מראה הוא אבל אם האדים הרבה עד שבא לכלל אחד משאר המראות של מעלה הימנו הרי בא לחדא מהנך דמתני':}
בר קפרא אמר וכולן עמוק מכן טמא דיהה מכן טהור חוץ ממזג שעמוק מכן טהור דיהה מכן טהור בר קפרא אדיהו ליה ודכי אעמיקו ליה ודכי אמר רבי חנינא כמה נפיש גברא דלביה כמשמעתיה
וכקרן כרכום תנא לח ולא יבש 
\commenta{כתחתון - עלה תחתון:}
תני חדא כתחתון ולא כעליון ותניא אידך כעליון ולא כתחתון ותניא אידך כעליון וכל שכן כתחתון ותניא אידך כתחתון וכל שכן כעליון 
\commenta{תלתא דרי - דעלין הוו זו למעלה מזו:}
אמר אביי תלתא דרי ותלתא טרפן הויין
\commenta{נקוט דרא מציעא - לבדוק בה את הדם:}
נקוט דרא מציעאה וטרפא מציעתא בידך 
\commenta{בגושייהו שנינו - לבדוק בהן ולא לאחר תלישתן:}
כי אתו לקמיה דרבי אבהו אמר להו בגושייהו שנינו
וכמימי אדמה תנו רבנן כמימי אדמה מביא אדמה שמנה מבקעת בית כרם ומציף עליה מים דברי רבי מאיר רבי עקיבא אומר מבקעת יודפת רבי יוסי אומר מבקעת סכני רבי שמעון אומר אף מבקעת גנוסר וכיוצא בהן 
\commenta{ואין שיעור למים - משום דאין שיעור לעפר - אלא יביא עפר כמו שירצה אם רב אם מעט ויציף על מים כעובי קליפת השום למעלה מן העפר:}
תניא אידך וכמימי אדמה מביא אדמה שמנה מבקעת בית כרם ומציף עליה מים כקליפת השום ואין שיעור למים משום דאין שיעור לעפר ואין בודקין אותן צלולין אלא עכורין צללו חוזר ועוכרן וכשהוא עוכרן אין עוכרן ביד אלא בכלי 
\commenta{לא לרמיה בידיה - לא יתן העפר אל תוך ידו ויעכור המים בעפר שבידו אע"פ שחוזר ונותנו בכלי לאחר עכירה:}
איבעיא להו אין עוכרין אותן ביד אלא בכלי דלא לרמיה בידיה ולעכרינהו אבל במנא כי עכר ליה בידיה שפיר דמי או דלמא דלא לעכרינהו בידיה אלא במנא 
\commenta{אלא בכוס - של זכוכית. אלמא לא לירמי עפר במים בכפו ויבדוק:}
ת"ש כשהוא בודקן אין בודקן אלא בכוס ועדיין תבעי לך בדיקה בכוס עכירה במאי תיקו 
\commenta{אמר להו במקומה שנינו - שאין האדמה דומה לדם אלא במקומה אבל הביאה במקום אחר נשתנה מראיתה:}
כי אתו לקמיה דרבה בר אבוה אמר להו במקומה שנינו רבי חנינא פלי קורטא דגרגשתא ובדיק ביה לייט עליה רבי ישמעאל ברבי יוסי באסכרה
רבי חנינא הוא דחכים כולי עלמא לאו חכימי הכי 
אמר רבי יוחנן חכמתא דרבי חנינא גרמא לי דלא אחזי דמא מטמינא מטהר מטהרנא מטמא אמר רבי אלעזר ענוותנותא דרבי חנינא גרמא לי דחזאי דמא ומה רבי חנינא דענותן הוא מחית נפשיה לספק וחזי אנא לא אחזי 
אמר רבי זירא טבעא דבבל גרמא לי דלא חזאי דמא דאמינא בטבעא לא ידענא בדמא ידענא 
למימרא דבטבעא תליא מלתא והא רבה הוא דידע בטבעא ולא ידע בדמא כל שכן קאמר ומה רבה דידע בטבעא לא חזא דמא ואנא אחזי 
\commenta{דמרא דארעא דישראל - בקי במראות דם מכל חכמי ארץ ישראל כדמפרש לקמן: פומבדיתא אתריה דרב יהודה בשילהי פ"ק דסנהדרין (דף יז:) סבי דפומבדיתא רב יהודה ורב עינא:}
עולא אקלע לפומבדיתא אייתו לקמיה דמא ולא חזא אמר ומה רבי אלעזר דמרא דארעא דישראל הוה כי מקלע לאתרא דר' יהודה לא חזי דמא אנא אחזי 
\commenta{ארחיה - הריח בו:}
ואמאי קרו ליה מרא דארעא דישראל דההיא אתתא דאייתא דמא לקמיה דרבי אלעזר הוה יתיב רבי אמי קמיה ארחיה אמר לה האי דם חימוד הוא בתר דנפקה אטפל לה רבי אמי אמרה ליה בעלי היה בדרך וחמדתיו קרי עליה (תהלים כה, יד) סוד ה' ליראיו 
\commenta{איפרא הורמיז - נכרית היתה וכן שמה ואעפ"כ היתה משמרת עצמה מנדות וקרובה להתגייר ואף קרבנות היתה שולחת. ולשון יוני איפרא חן כמו אפריון נמטייה לרבי שמעון (ב"מ דף קיט.):}
אפרא הורמיז אמיה דשבור מלכא שדרה דמא לקמיה דרבא הוה יתיב רב עובדיה קמיה ארחיה אמר לה האי דם חימוד הוא אמרה ליה לבריה תא חזי כמה חכימי יהודאי א"ל דלמא כסומא בארובה 
\commenta{שלח לה - משום דורון:}
הדר שדרה ליה שתין מיני דמא וכולהו אמרינהו ההוא בתרא דם כנים הוה ולא ידע אסתייע מילתא ושדר לה סריקותא דמקטלא כלמי אמרה יהודאי בתווני דלבא יתביתו 
\commenta{לא חזינא - דדילמא אשתני ואע"ג דמראה דם טהור יש לה דלמא אי אייתא טיפתא קמייתא הוה מיחזי טמא:}
אמר רב יהודה מרישא הוה חזינא דמא כיון דאמרה לי אמיה דיצחק ברי האי טיפתא קמייתא לא מייתינן לה קמייהו דרבנן משום דזהימא לא חזינא
\commenta{בין טמאה לטהורה - כשכלין ימי טוהר של זכר או של נקבה וראתה ביום ארבעים ואחד או ביום שמונים ואחד:}
בין טמאה לטהורה ודאי חזינא 
ילתא אייתא דמא לקמיה דרבה בר בר חנה וטמי לה הדר אייתא לקמיה דרב יצחק בריה דרב יהודה ודכי לה 
והיכי עביד הכי והתניא חכם שטימא אין חברו רשאי לטהר אסר אין חבירו רשאי להתיר 
\commenta{מעיקרא טמויי טמי לה - רב יצחק משום כבודו דרבה בר בר חנה:}
מעיקרא טמויי הוה מטמי לה כיון דא"ל דכל יומא הוה מדכי לי כי האי גונא והאידנא הוא דחש בעיניה דכי לה 
\commenta{כזה ראיתי - מביאה דם אחר או שלה או של חברתה ואומרת כזה ראיתי ואבדתיו אם דם זה טהור לא מספקינן ליה בטומאה משום דם האבוד דמהימנא למימר כזה ראיתי: }
ומי מהימני אין והתניא נאמנת אשה לומר כזה ראיתי ואבדתיו 
\commenta{כזה טיהר לי פלוני כו' מהו - שתסמוך עליה חברתה שהראתה דמה לזו ואמרה לה כדם זה שלך הראתי גם אני לפלוני חכם וטיהר:}
איבעיא להו כזה טיהר איש פלוני חכם מהו 
\commenta{נאמנת אשה כו' - אלמא קים לה במראה דמים:}
תא שמע נאמנת אשה לומר כזה ראיתי ואבדתיו שאני התם דליתיה לקמה 
תא שמע דילתא אייתא דמא לקמיה דרבה בר בר חנה וטמי לה לקמיה דרב יצחק בריה דרב יהודה ודכי לה והיכי עביד הכי והתניא חכם שטימא אין חבירו רשאי לטהר וכו' 
ואמרינן טמויי הוה מטמי לה כיון דאמרה ליה דכל יומא מדכי לה כי האי גונא והאידנא הוא דחש בעיניה הדר דכי לה אלמא מהימנא לה 
\commenta{אגמריה סמך - דפשיטא ליה דטהור והאי דהוה מטמא מעיקרא משום כבודו דרבה בר בר חנה אבל אי הוה מספקא ליה לא הוה סמיך עלה:}
רב יצחק בר יהודה אגמריה סמך 
\commenta{ראה ביום - לאותו דם עצמו:}
רבי ראה דם בלילה וטימא ראה ביום וטיהר המתין שעה אחת חזר וטימא אמר אוי לי שמא טעיתי 
שמא טעיתי ודאי טעה דתניא לא יאמר חכם אילו היה לח היה ודאי טמא
\commenta{מעיקרא אחזקיה בטומאה - כלומר אמש לא מספק טימא אלא שפיר חזייה ואחזיק בטומאה דאמרי' לקמן רבי בדק לאור הנר:}
אלא אמר אין לו לדיין אלא מה שעיניו רואות מעיקרא אחזקיה בטמא כיון דחזא לצפרא דאשתני אמר (ליה) ודאי טהור הוה ובלילה הוא דלא אתחזי שפיר כיון דחזא דהדר אשתני אמר האי טמא הוא ומפכח הוא דקא מפכח ואזיל 
\commenta{בין העמודים - היכא דגרסי ואע"ג דליכא נהורא כולי האי:}
רבי בדיק לאור הנר רבי ישמעאל ברבי יוסף בדיק ביום המעונן ביני עמודי אמר רב אמי בר שמואל וכולן אין בודקין אותן אלא בין חמה לצל רב נחמן אמר רבה בר אבוה בחמה ובצל ידו
וכמזוג שני חלקים כו' תנא
\clearpage}

\newsection{דף כא}
\twocol{השרוני נידון ככרמלי חי ולא מזוג חדש ולא ישן 
\commenta{בכוס טבריא של זכוכית פשוט - טנב"ש ואיידי דקליש נראה בו אדמומית היין:}
אמר רב יצחק בר אבודימי וכולן אין בודקין אותן אלא בכוס טבריא פשוט מאי טעמא אמר אביי של כל העולם כולו מחזיק לוג עושין אותו ממנה שני לוגין עושין אותו ממאתים כוס טבריא פשוט אפי' מחזיק שני לוגין עושין אותו ממנה ואיידי דקליש ידיע ביה טפי
\par \par {\large\emph{הדרן עלך כל היד}}\par \par 
\commenta{מתני' המפלת חתיכה אם יש עמה דם טמאה - משום נדה:}

\commenta{כמין שערה - שער:}
מתני׳ {\large\emph{המפלת}} חתיכה אם יש עמה דם טמאה ואם לאו טהורה ר' יהודה אומר בין כך ובין כך טמאה 
\commenta{מין דגים - נמי לאו ולד הוא וטעמא מפרש בגמרא הלכך אי איכא דם טמאה נדה ואם לאו טהורה:}
המפלת כמין קליפה כמין שערה כמין עפר כמין יבחושין אדומים תטיל למים אם נמוחו טמאה ואם לאו טהורה 
\commenta{תשב לזכר - ימי טומאת שבעה וימי טהרה ל"ג וכל דמים שתראה בהן טהורין:}
המפלת כמין דגים חגבים שקצים ורמשים אם יש עמהם דם טמאה ואם לאו טהורה 
\commenta{לזכר ולנקבה - לחומרא ימי טומאה דנקבה שבועים וימי טוהר כלים לסוף ארבעים יום כזכר ולא פ' כנקבה. ובגמרא מפרש מאי שנא דלחיה ועוף קרי ולד יותר מדגים:}
המפלת מין בהמה חיה ועוף בין טמאין בין טהורין אם זכר תשב לזכר ואם נקבה תשב לנקבה
\commenta{גמ' של ד' מיני דמים - דקסבר היא גופה דם הלכך טמאה נדה אבל ולד לא הוי:}
ואם אין ידוע תשב לזכר ולנקבה דברי רבי מאיר וחכמים אומרים כל שאין בו מצורת אדם אינו ולד
{\large\emph{גמ׳}} אמר רב יהודה אמר שמואל לא טימא רבי יהודה אלא בחתיכה של ארבעת מיני דמים אבל של שאר מיני דמים טהורה 
\commenta{לא אמרינן כו' - דקסברי אין רוב חתיכות של ארבע מיני דמים דלא אתחזק לן האי רובא:}
ור' יוחנן אמר של ארבעת מיני דמים דברי הכל טמאה של שאר מיני דמים דברי הכל טהורה 
\commenta{ירוקה ולבנה - לאו מארבע מיני דמים הן אלא אדום וכמימי אדמה וכקרן כרכום וכמזוג. ושחור לא קא חשיב דהיינו אדום אלא שלקה:}
לא נחלקו אלא שהפילה ואינה יודעת מה הפילה רבי יהודה סבר זיל בתר רוב חתיכות ורוב חתיכות של (מיני) ארבעת מיני דמים הויין ורבנן סברי לא אמרינן רוב חתיכות של ארבעת מיני דמים 
איני והא כי אתא רב הושעיא מנהרדעא אתא ואייתי מתניתא בידיה המפלת חתיכה אדומה שחורה ירוקה ולבנה אם יש עמה דם טמאה ואם לאו טהורה רבי יהודה אומר בין כך ובין כך טמאה קשיא לשמואל בחדא ולרבי יוחנן בתרתי 
\commenta{למאן קתני לה - כלומר להודיע דבריו של מי בא:}
לשמואל בחדא דאמר שמואל לא טימא רבי יהודה אלא בחתיכה של ארבעת מיני דמים והא קתני ירוקה ולבנה ופליג רבי יהודה 
\commenta{אלא לר' יהודה - להודיע דאירוקה ולבנה נמי פליג:}
וכי תימא כי פליג רבי יהודה אאדומה ושחורה ואירוקה ולבנה לא אלא ירוקה ולבנה למאן קתני לה 
\commenta{ותו לרבי יוחנן וכו' - כלומר הך קושיא דירוקה ולבנה לא איבעי לן לפרושי דכי היכי דקשיא ליה לשמואל קשיא ליה לרבי יוחנן דהא רבי יוחנן נמי אמר על שאר מינין לרבי יהודה טהורה והכא קתני ירוקה ולבנה ופליג רבי יהודה ותו קשיא אחריתי לר' יוחנן דאמר של ארבעת מיני דמים דברי הכל כו':}
אילימא רבנן השתא אדומה ושחורה מטהרי רבנן ירוקה ולבנה מיבעיא אלא לאו לרבי יהודה ופליג 
\commenta{הכי גרסינן \textbf{וכי תימא כי פליגי רבנן אירוקה ולבנה אלא אדומה ושחורה למאן קתני לה כו' אלא לאו רבנן ופליגי} :}
ותו לרבי יוחנן דאמר של ארבעת מיני דמים דברי הכל טמאה הא קתני אדומה ושחורה ופליגי רבנן 
וכי תימא כי פליגי רבנן אירוקה ולבנה אבל אאדומה ושחורה לא אלא אדומה ושחורה למאן קתני לה 
\commenta{פתיחת הקבר - הרחם:}
אילימא רבי יהודה השתא ירוקה ולבנה טמאה אדומה ושחורה מיבעיא אלא לאו רבנן ופליגי 
אלא אמר רב נחמן בר יצחק באפשר לפתיחת הקבר בלא דם קמיפלגי ובפלוגתא דהני תנאי דתניא קשתה שנים ולשלישי הפילה ואינה יודעת מה הפילה}

\newchap{פרק \hebrewnumeral{3} המפלת חתיכה}
\twocol{
\commenta{שאי אפשר - הלכך מה נפשך אם ולד הוא הרי כאן לידה ואם אינו ולד הרי כאן זיבה דבשלישי חזאי:}
הרי זו ספק לידה ספק זיבה מביאה קרבן ואינו נאכל 
רבי יהושע אומר מביאה קרבן ונאכל שאי אפשר לפתיחת הקבר בלא דם 
לישנא אחרינא אמרי לה אמר רב יהודה אמר שמואל לא טימא רבי יהודה אלא בחתיכה של ארבעה מיני דמים אבל של שאר מיני דמים טהורה 
איני והא כי אתא רב הושעיא מנהרדעא אתא ואייתי מתניתא בידיה המפלת חתיכה אדומה ושחורה ירוקה ולבנה אם יש עמה דם טמאה ואם לאו טהורה ורבי יהודה אומר בין כך ובין כך טמאה 
קתני אדומה ושחורה ירוקה ולבנה ופליג ר' יהודה 
וכי תימא כי פליג ר' יהודה אאדומה ושחורה אבל ירוקה ולבנה לא אלא ירוקה ולבנה מאן קתני לה 
אילימא לרבנן השתא אדומה ושחורה קא מטהרי רבנן ירוקה ולבנה מיבעיא אלא לאו לר' יהודה ופליג
אלא אמר רב יהודה באפשר לפתיחת הקבר בלא דם קמיפלגי ובפלוגתא דהני תנאי דתניא קשתה שנים ולשלישי הפילה ואינה יודעת מה הפילה הרי זו ספק לידה ספק זיבה מביאה קרבן ואינו נאכל 
רבי יהושע אומר מביאה קרבן ונאכל לפי שאי אפשר לפתיחת הקבר בלא דם 
\commenta{ועדיפא - חמירא:}
ת"ר המפלת חתיכה סומכוס אומר משום רבי מאיר וכן היה רבי שמעון בן מנסיא אומר כדבריו קורעה אם יש דם בתוכה טמאה ואם לאו טהורה 
\commenta{אם תוכה מאדים - הבשר מאדים ואפי' אין שם דם ממש:}
כרבנן ועדיפא מדרבנן כרבנן דאמרי אפשר לפתיחת הקבר בלא דם ועדיפא מדרבנן דאינהו סברי עמה אין בתוכה לא וסומכוס סבר אפילו בתוכה 
\commenta{כסומכוס - דאמר תוכה כעמה ועדיפא מדסומכוס דאיהו בעי דם ממש:}
ותניא אידך המפלת חתיכה ר' אחא אומר קורעה אם תוכה מאדים טמאה ואם לאו טהורה 
\commenta{טמאה לידה - ויש לה ימי טוהר ותשב לזכר ולנקבה ימי טומאה דנקבה וימי טוהר דזכר:}
כסומכוס ועדיפא מסומכוס 
\commenta{אגור - דם הרבה:}
ותניא אידך המפלת חתיכה רבי בנימין אומר קורעה אם יש בה עצם אמו טמאה לידה אמר רב חסדא ובחתיכה לבנה וכן כי אתא זוגא דמן חדייב אתא ואייתי מתניתא בידיה המפלת חתיכה לבנה קורעה אם יש בה עצם אמו טמאה לידה 
\commenta{בשפופרת - הכניסה קנה חלול באותו מקום ונמצא בתוכו דם:}
אמר רבי יוחנן משום רבי שמעון בן יוחי המפלת חתיכה קורעה אם יש בה דם אגור טמאה ואם לאו טהורה כסומכוס וקילא מכולהו 
בעא מיניה רבי ירמיה מרבי זירא הרואה דם בשפופרת מהו (ויקרא טו:ז) בבשרה אמר רחמנא ולא בשפופרת או דלמא האי בבשרה מיבעי ליה שמטמאה מבפנים כבחוץ 
\commenta{אם יש בה דם אגור כו' - ומאי שנא משפופרת:}
אמר ליה בבשרה אמר רחמנא ולא בשפופרת דאי בבשרה מבעי ליה שמטמאה מבפנים כבחוץ א"כ נימא קרא (בבשר) מאי בבשרה שמע מינה תרתי 
\commenta{דרכה של אשה לראות דם בחתיכה - הלכך בבשרה קרינא ביה דמין במינו לא חייץ:}
והא"ר יוחנן משום רבי שמעון בן יוחי המפלת חתיכה קורעה אם יש בה דם אגור טמאה ואם לאו טהורה 
\commenta{שפיר - עור הולד קודם שנבראו בתוכו גידים ועצמות ובשר:}
הכי השתא התם דרכה של אשה לראות דם בחתיכה הכא אין דרכה של אשה לראות דם בשפופרת 
לימא שפופרת תנאי היא דתניא המפלת חתיכה אף על פי שמלאה דם אם יש עמה דם טמאה ואם לאו טהורה רבי אליעזר אומר בבשרה ולא בשפיר ולא בחתיכה 
\commenta{דפלי פלויי - שיש בה שורות שורות ביקועים מבחוץ והדם שם:}
ר' אליעזר היינו תנא קמא אימא שרבי אליעזר אומר בבשרה ולא בשפיר ולא בחתיכה
\commenta{דת"ק - לא מטהר לה אלא מגזרת הכתוב סבר האי דם טהור הוא ומשום חציצה היא טהורה ומשום גזרת הכתוב וה"ה לשפופרת כיון דטעמא משום חציצה הוא:}
וחכמים אומרים אין זה דם נדה אלא דם חתיכה תנא קמא נמי טהורי מטהר אלא דפלי פלויי איכא בינייהו
תנא קמא סבר בבשרה ולא בשפיר ולא בחתיכה והוא הדין לשפופרת והני מילי היכא דשיעא אבל פלי פלויי טמאה מאי טעמיה בבשרה קרינא ביה 
\commenta{כולי עלמא לא פליגי (דטהורה. דאין דרך הרואה בכך):}
ואתו רבנן למימר אף על גב דפלי פלויי אין זה דם נדה אלא דם חתיכה הא דם נדה ודאי טמא ואפילו בשפופרת נמי 
אמר אביי בשפופרת כולי עלמא לא פליגי דטהורה
\clearpage}

\newsection{דף כב}
\twocol{כי פליגי בחתיכה מר סבר דרכה של אשה לראות דם בחתיכה ומר סבר אין דרכה של אשה לראות דם בחתיכה 
רבא אמר דכולי עלמא אין דרכה של אשה לראות דם בחתיכה
\commenta{אשה טהורה - דלאו ראייה היא ומיהו טיפת דם מטמאה טהרות אם נגעה בהן ואת האשה טומאת ערב משום דנגעה במקור דקרא מתורת ראייה מעטיה ולאו מטומאה:}
והכא באשה טהורה ומקור מקומו טמא קמיפלגי דר' אליעזר סבר אשה טהורה ודם טמא דהא אתי דרך מקור ורבנן סברי אשה טהורה ומקור מקומו טהור 
\commenta{הרואה קרי בקיסם - הכניס קיסם בפי אמה והוציא בו זרעו:}
בעא מיניה רבה מרב הונא הרואה קרי בקיסם מהו {ויקרא טו } ממנו אמר רחמנא עד דנפיק מבשרו ולא בקיסם או דלמא האי ממנו עד שתצא טומאתו לחוץ ואפי' בקיסם נמי 
\commenta{הוא עצמו - ואפילו בלא קיסם:}
אמר ליה תיפוק ליה דהוא עצמו אינו מטמא אלא בחתימת פי האמה
\commenta{למימרא דנוגע הוי - מדיהיב שיעורא מכלל דטומאת קרי לאו משום ראייה היא אלא משום דבשר האבר נוגע בשכבת זרע היוצא ממנו ומשום הכי בעי שיעורא כטומאת כל מגעות דבעי שיעורא כגון שרץ ונבלה ומת דאי ראייה חשיבא ליה בטומאת ראייה לא בעיא שיעורא דהא נדה רואה היא ולא בעיא שיעורא:}
למימרא דנוגע הוי אלא מעתה אל יסתור בזיבה 
\commenta{אלמה תניא זאת תורת הזב - הקיש הכתוב שכבת זרע לזיבה:}
אלמה תניא (ויקרא טו, לב) זאת תורת הזב ואשר תצא ממנו שכבת זרע מה זיבה סותרת אף שכבת זרע נמי סותר 
\commenta{לפי שאי אפשר - לראות שכבת זרע הבאה מן הזב בלא טיפת צחצוחי זיבה דהשתא קצת זיבה דאית בה קא סתר:}
אמר ליה סתירה היינו טעמא דסותר לפי שאי אפשר לה בלא צחצוחי זיבה 
אלא מעתה תסתור כל שבעה אלמה תניא זאת תורת הזב וגו' מה זיבה סותרת אף שכבת זרע סותר
\commenta{כל ז' - אם ראה בשביעי סותר כל אותן ימים שספר:}
אי מה זיבה סותרת כל ז' אף שכבת זרע נמי סותר כל ז' ת"ל (ויקרא טו, לב) לטמאה בה אין לך בה אלא מה שאמור בה סותרת יום אחד 
אמר ליה גזירת הכתוב היא זיבה גמורה דלא ערבה בה שכבת זרע סותרת כל שבעה צחצוחי זיבה דערבה בה שכבת זרע לא סותרת אלא יום אחד 
\commenta{דם יבש - כגון שיצא מגופה חתיכה של דם יבש:}
בעא מיניה ר' יוסי ברבי חנינא מרבי אלעזר דם יבש מהו (ויקרא טו, כה) כי יזוב זוב דמה אמר רחמנא עד דמידב דייב ליה לח אין יבש לא או דלמא האי כי יזוב זוב דמה אורחא דמילתא היא ולעולם אפילו יבש נמי 
\commenta{לח ונעשה יבש - דהא קתני מטמאין יבשין בנפל דם לח על הבגד ויבש ואחר כך נגע הדם בטהרות דומיא דיבש בשר המת דיבש מעיקרא בשר המת ליכא:}
א"ל תניתוה דם הנדה ובשר המת מטמאין לחים ויבשים אמר ליה לח ונעשה יבש לא קא מיבעיא לי כי מיבעיא לי יבש מעיקרא 
הא נמי תניתוה המפלת כמין קליפה כמין שערה כמין עפר כמין יבחושין אדומין תטיל למים
אם נמוחו טמאה אי הכי בלא נמוחו נמי אמר רבה כי לא נמוחו בריה בפני עצמה היא 
\commenta{כי האי גוונא - מפלת כמין שערות וקליפות:}
ומי איכא כי האי גוונא אין והתניא א"ר אלעזר בר' צדוק שני מעשים העלה אבא מטבעין ליבנה 
\commenta{קליפות - גלדי המכה: דרך שומא להיות בה שער:}
מעשה באשה שהיתה מפלת כמין קליפות אדומות ובאו ושאלו את אבא ואבא שאל לחכמים וחכמים שאלו לרופאים ואמרו להם אשה זו מכה יש לה בתוך מעיה שממנה מפלת כמין קליפות תטיל למים אם נמוחו טמאה 
ושוב מעשה באשה שהיתה מפלת כמין שערות אדומות ובאה ושאלה את אבא ואבא שאל לחכמים וחכמים לרופאים ואמרו להם שומא יש לה בתוך מעיה שממנה מפלת כמין שערות אדומות תטיל למים אם נמוחו טמאה 
\commenta{ובפושרין - דאי בדק בצונן ולא נימוח לא מטהרינן לה בהכי דדילמא אי היה בפושרין הוה נימוח:}
אמר ריש לקיש ובפושרין תניא נמי הכי תטיל למים ובפושרין רשב"ג אומר ממעכתו ברוק על גבי הצפורן מאי בינייהו אמר רבינא מעוך על ידי הדחק איכא בינייהו 
\commenta{כמה היא שרייתן - בפ' דם הנדה השרץ והנבלה מטמאין לחין ואין מטמאין יבשין ואם יכולין להשרות ולחזור לכמות שהוא טמאין וכמה היא שרייתן כו':}
התם תנן כמה היא שרייתן בפושרין מעת לעת הכא מאי מי בעינא מעת לעת או לא 
\commenta{דאקושי - קשים:}
שרץ ונבלה דאקושי בעינן מעת לעת אבל דם דרכיך לא או דלמא לא שנא תיקו
\commenta{וליפלוג נמי ר' יהודה בהא - ונימא בין כך ובין כך טמאה דהא טעמא דר' יהודה אוקמינן דאי אפשר לפתיחת הקבר בלא דם:}
המפלת כמין דגים וליפלוג נמי רבי יהודה בהא 
\commenta{במחלוקת שנויה - משנה זו שנויה במחלוקת דלאו סתמא ודברי הכל היא אלא ר' יהודה פליג:}
אמר ריש לקיש במחלוקת שנויה ורבנן היא ורבי יוחנן אמר אפילו תימא רבי יהודה עד כאן לא קאמר רבי יהודה התם אלא גבי חתיכה דעביד דם דקריש והוי חתיכה אבל בריה לא הוי 
\commenta{ולהך לישנא כו' - כדאמרן לעיל בשמעתין:}
ולהך לישנא דא"ר יוחנן באי אפשר לפתיחת הקבר בלא דם קמיפלגי לפלוג נמי ר' יהודה בהא 
מאן דמתני הך לישנא מתני הכי רבי יוחנן וריש לקיש דאמרי תרוייהו במחלוקת שנויה ורבנן היא
המפלת כמין בהמה [וכו']
\commenta{יצירה - ויצר ה' אלהים (את חית השדה ואת) עוף וגו' וכתיב באדם וייצר ה' אלהים את האדם עפר מן האדמה ורבנן לית להו גזירה שוה:}
אמר רב יהודה אמר שמואל מ"ט דר' מאיר הואיל ונאמרה בו יצירה כאדם 
\commenta{תנין - דג גדול. ואמאי תנן המפלת מין דגים טהורה:}
אלא מעתה המפלת דמות תנין תהא אמו טמאה לידה הואיל ונאמר בו יצירה כאדם שנאמר (בראשית א, כא) ויברא אלהים את התנינים הגדולים 
אמרי דנין יצירה מיצירה ואין דנין בריאה מיצירה 
\commenta{זו היא שיבה זו היא ביאה - כלומר ילפינן גזרה שוה משיבה לביאה כמו דהוו תרוייהו שיבה או תרוייהו ביאה מה שיבה חולץ וקוצה וטח כו':}
מאי נפקא מינה הא תנא דבי רבי ישמעאל (ויקרא יד:לט) ושב הכהן (ויקרא יד, מד) ובא הכהן זו היא שיבה זו היא ביאה 
\commenta{ועוד - גבי אדם נמי כתיב בריאה דכתיב ויברא אלהים את האדם בצלמו ונילף בריאת דגים מבריאת אדם:}
ועוד נגמר בריאה מבריאה דכתיב (בראשית א:כז) ויברא אלהים את האדם בצלמו 
\commenta{וייצר לאפנויי - לגזירה שוה ודנין ג"ש לחיה ולעוף דכתיב בהו יצירה דדמי מדדמי:}
אמרי ויברא לגופיה וייצר לאפנויי ודנין יצירה מיצירה 
אדרבה וייצר לגופיה ויברא לאפנויי ודנין בריאה מבריאה 
\commenta{מופנה - גבי אדם דהא כתיב ויברא:}
אלא וייצר מופנה משני צדדין מופנה גבי אדם ומופנה גבי בהמה ויברא גבי אדם מופנה גבי תנינים אינו מופנה 
\commenta{ופרכינן גבי תנין נמי מופנה הוא - דכתיב בקרא דויעש ואת כל רמש האדמה:}
מאי מופנה גבי בהמה אילימא מדכתיב (בראשית א:כה) ויעש אלהים את חית הארץ וכתיב {בראשית ב } ויצר [ה'] אלהים מן האדמה כל חית השדה גבי תנין נמי אפנויי מופנה דכתיב (בראשית א:כה) ואת כל רמש האדמה וכתיב (בראשית א:כא) ויברא אלהים את התנינים הגדולים 
\commenta{ומאי נפקא מינה - דניחא לך למילף משני צדדין טפי:}
רמש דכתיב התם דיבשה הוא ומאי נפקא מינה בין מופנה מצד אחד למופנה משני צדדין 
\commenta{כל ג"ש שאינה מופנה כל עיקר אין למדין הימנה - ואפילו אין להשיב:}
נפקא מינה דאמר רב יהודה אמר שמואל משום רבי ישמעאל כל גזרה שוה שאינה מופנה כל עיקר אין למדין הימנה מופנה מצד אחד לרבי ישמעאל למדין ואין מושיבין לרבנן למדין ומשיבין מופנה משני צדדין דברי הכל למדין ואין משיבין 
ורבי ישמעאל מאי איכא בין מופנה מצד אחד למופנה משני צדדין נפקא מינה דהיכא דאיכא מופנה מצד אחד ומופנה משני צדדין שבקינן מופנה מצד אחד
\clearpage}

\newsection{דף כג}
\twocol{וילפינן מופנה משני צדדין ולהכי אפניה רחמנא לבהמה משני צדדין כי היכי דלא נגמר מן מופנה מצד אחד 
\commenta{לקולא - דלא בעינן מופנה כולי האי:}
רב אחא בריה דרבא מתני לה משמיה דרבי אלעזר לקולא כל גזרה שוה שאינה מופנה כל עיקר למדין ומשיבין מופנה מצד אחד לרבי ישמעאל למדין ואין משיבין לרבנן למדין ומשיבין מופנה משני צדדין דברי הכל למדין ואין משיבין 
ולרבנן מאי איכא בין מופנה מצד אחד לשאינה מופנה כל עיקר 
נ"מ היכא דמשכחת לה מופנה מצד אחד ושאינה מופנה כל עיקר ולאו להאי אית ליה פירכא ולאו להאי אית ליה פירכא שבקינן שאינה מופנה כל עיקר וגמרינן ממופנה מצד אחד 
\commenta{והכא מאי פירכא איכא - כלומר והכא בג"ש דתני אאדם מה תשובה איכא דקאמרת דניחא לך למיגמר מחיה דהוי מופנה משני צדדין משום דאין משיבין ולא נגמר בריאה בריאה מתנין משום דמופנה מצד אחד הוא ומשיבין ומאי אית לך לאותובי:}
והכא מאי פירכא איכא משום דאיכא למיפרך מה לאדם שכן מטמא מחיים 
\commenta{וכן אמר רבי חייא כו' - אדרב יהודה אמר שמואל קאי:}
וכן א"ר חייא בר אבא א"ר יוחנן היינו טעמא דר"מ הואיל ונאמרה בו יצירה כאדם 
\commenta{הר מי קמפלת - מי מפלת נפל גדול כהר:}
א"ל רבי אמי אלא מעתה המפלת דמות הר אמו טמאה לידה שנאמר (עמוס ד:יג) כי הנה יוצר הרים ובורא רוח אמר ליה הר מי קא מפלת אבן היא דקא מפלת ההוא גוש איקרי
אלא מעתה המפלת רוח תהא אמו טמאה לידה הואיל ונאמרה בו בריאה כאדם דכתיב {עמוס ד } ובורא רוח וכי תימא לא מופנה מדהוה ליה למכתב יוצר הרים ורוח וכתיב ובורא רוח ש"מ לאפנויי 
א"ל דנין דברי תורה מדברי תורה ואין דנין דברי תורה מדברי קבלה 
\commenta{הואיל ועיניהן - של חיה ובהמה:}
(אמר) רבה בר בר חנה אמר רבי יוחנן היינו טעמא דר"מ הואיל ועיניהם דומות כשל אדם 
אלא מעתה המפלת דמות נחש תהא אמו טמאה לידה הואיל וגלגל עינו עגולה כשל אדם וכי תימא הכי נמי ליתני נחש 
אי תנא נחש הוה אמינא בנחש הוא דפליגי רבנן עליה דר"מ דלא כתיב ביה יצירה אבל בהמה וחיה לא פליגי דכתיבא ביה יצירה 
\commenta{והא גבי מומין תני לה - כלומר היכי אמרת דעין בהמה דומה לאדם והא גבי מומי הבכור קתני לה בבכורות (דף מ.) ושגלגל עינו הוי מום ומדחשיב ליה מום נמצא שאין דרך בהמה להיות עיניה כאדם:}
והא גבי מומין קתני לה את שגלגל עינו עגול כשל אדם לא קשיא הא באוכמא הא בציריא 
\commenta{לפניהם - לאפוקי דגים דעיניהם בצדיהם וכן נחש:}
רבי ינאי אמר היינו טעמא דר"מ הואיל ועיניהם הולכות לפניהם כשל אדם והרי עוף דאין עיניו הולכות לפניו וקאמר ר"מ דטמא אמר אביי בקריא וקיפופא ובשאר עופות לא 
מיתיבי ר' חנינא בן (אנטיגנוס) אומר נראין דברי ר"מ בבהמה וחיה ודברי חכמים בעופות
\commenta{קריא וקיפופא - עוף הצועק בלילה ופניו דומה לחתול ועיניו לפניו:}
מאי עופות אילימא בקריא וקיפופא מ"ש בהמה וחיה דעיניהן הולכות לפניהן כשל אדם קריא וקיפופא נמי 
אלא פשיטא בשאר עופות מכלל דר"מ פליג בשאר עופות 
\commenta{חסורי מיחסרא והכי קתני נראין דברי ר"מ בבהמה ובחיה והוא הדין לקריא וקיפופא - דעיניו הולכו' לפניו ואע"פ דפליגי רבנן עליה:}
חסורי מיחסרא והכי קתני ר' חנינא בן אנטיגנוס אומר נראין דברי ר"מ בבהמה וחיה והוא הדין לקריא וקיפופא ודברי חכמים בשאר עופות שאף ר"מ לא נחלק עמהם אלא בקריא וקיפופא אבל בשאר עופות מודי להו 
\commenta{והתניא - בניחותא:}
והתניא א"ר אלעזר בר' צדוק המפלת מין בהמה וחיה לדברי ר"מ ולד ולדברי חכמים אינו ולד ובעופות תיבדק 
למאן תיבדק לאו לדברי ר"מ דאמר קריא וקיפופא אין שאר עופות לא 
אמר רב אחא בריה דרב איקא לא תיבדק לרבנן דאמרי קריא וקיפופא אין שאר עופות לא 
\commenta{ומאי שנא קריא וקיפופא - דקאמר לרבנן דהוי ולד טפי משאר עופות אי משום דעיניהם הולכות לפניהם הרי חיה ובהמה נמי דעיניהן הולכות לפניהם ופליגי רבנן:}
ומ"ש קריא וקיפופא מבהמה וחיה הואיל ויש להן לסתות כאדם 
בעא מיניה רבי ירמיה מר' זירא לר"מ דאמר בהמה במעי אשה ולד מעליא הוא קבל בה אביה קידושין מהו למאי נפקא מינה לאיתסורי באחותה 
\commenta{למימרא דחיי - הא אחותה לא מיתסרא אלא בחייה דאין איסור אחות אשה אלא בחייה:}
למימרא דחיי והאמר רב יהודה אמר רב לא אמרה ר"מ אלא הואיל ובמינו מתקיים אמר רב אחא בר יעקב עד כאן הביאו רבי ירמיה לר' זירא לידי גיחוך ולא גחיך 
גופא אמר רב יהודה אמר רב לא אמרה רבי מאיר אלא הואיל ובמינו מתקיים אמר רב ירמיה מדפתי
אף אנן נמי תנינא המפלת כמין בהמה חיה ועוף (ולד מעליא הוא) דברי ר"מ וחכ"א עד שיהא בו מצורת אדם 
\commenta{אין בכור לכהן - שהרי זה פטר את הרחם:}
והמפלת סנדל או שליא או שפיר מרוקם והיוצא מחותך הבא אחריו בכור לנחלה ואינו בכור לכהן ואי ס"ד דחיי הבא אחריו בכור לנחלה מי הוי 
\commenta{מי שלבו דוה עליו - לב אביו מתאבל על מותו הוא דחשיב לענין נחלה:}
אמר רבא לעולם דחיי ושאני התם דאמר קרא {דברים כח } ראשית אונו מי שלבו דוה עליו יצא זה שאין לבו דוה עליו 
\commenta{אדם במעי בהמה - שחטה ומצא בה דמות אדם מהו מי אמרינן כי היכי דאזיל ר"מ בתר אימיה הכי נמי זיל בתר בהמה ושרי לאכילה:}
בעא מיניה רב אדא בר אהבה מאביי לרבי מאיר דאמר בהמה במעי אשה ולד מעליא הוא אדם במעי בהמה מאי למאי נפקא מיניה לאשתרויי באכילה 
ותפשוט ליה מהא דר' יוחנן דא"ר יוחנן השוחט את הבהמה ומצא בה דמות יונה אסורה באכילה 
\commenta{עוף אין לו פרסה - אבל אדם נהי דפרסות ליכא שאין פרסותיו סדוקות שאין לו שני עקבים כמו לבהמה:}
הכי השתא התם לא פרסות איכא ולא פרסה איכא הכא נהי דפרסות ליכא פרסה מיהא איכא
\commenta{גופו תייש ופניו אדם אדם הוא - דבתר צורת פנים אזלינן:}
וחכ"א כל שאין בו כו' אמר רב ירמיה בר אבא אמר רב הכל מודים גופו תייש ופניו אדם אדם גופו אדם ופניו תייש ולא כלום
\commenta{בעין אחד כבהמה - אחת מעיניו דומה לבהמה: }
לא נחלקו אלא שפניו אדם ונברא בעין אחת כבהמה שרבי מאיר אומר מצורת אדם וחכ"א כל צורת אדם 
\commenta{והא איפכא תניא - שר"מ אומר כל צורת והאי כל צורת דר"מ הכי קאמר כל שהוא צורת אדם שבו הוי ולד ואפילו כל פניו תייש אלא שנברא בעין אחת או בלסת אחת הדומה לאדם:}
אמר לו לרב ירמיה בר אבא והא איפכא תניא ר"מ אומר כל צורת וחכ"א מצורת אמר להו אי תניא תניא 
\commenta{הגבינים - גבות עינים שורצי"ל. וגבות הזקן סנטר שקורין מנטו"ן:}
אמר ר' ירמיה בר אבא אמר רבי יוחנן מצח והגבינים והעינים והלסתות וגבות הזקן עד שיהו כולם כאחד רבא אמר חסא מצח והגבן והעין והלסת וגבת הזקן עד שיהו כולם כאחת 
ולא פליגי הא כמ"ד כל צורת הא כמ"ד מצורת 
\commenta{אפילו פרצוף אחד - אבר אחד כגון עינו או מצחו או לסת אחד דדומה לאדם הוי ולד:}
מיתיבי צורת פנים שאמרו אפילו פרצוף אחד מן הפרצופין חוץ מן האוזן למימרא דמחד נמי סגי 
\commenta{אמר אביי כי תניא ההיא לעכב תניא - כלומר צורת פנים שאמרו חכמים דמעכבים אפילו אבר אחד כבהמה מעכבא חוץ מן האוזן דלא מעכבא וכמ"ד לרבנן כל צורת מתוקמא ואע"ג דמצי לאוקמי כדאיתא וכדר"מ דאמר כל דהו צורת ניחא ליה לאוקומי כרבנן:
}
אמר אביי כי תניא ההיא לעכב תניא וכמ"ד כל צורת ואיבעית אימא לעולם כמ"ד מצורת ומאי אחד אחד אחד 
\commenta{מן הצד - שעינו בצדו וירכו בצדו כשאר בני אדם:}
אמר רבא נברא בעין אחת ובירך אחד מן הצד אמו טמאה באמצע אמו טהורה 
\commenta{ושטו נקוב אמו טמאה - קסבר טרפה חיה:}
אמר רבא ושטו נקוב אמו טמאה ושטו אטום אמו טהורה 
ת"ר המפלת גוף אטום אין אמו טמאה לידה ואיזהו גוף אטום רבי אומר כדי שינטל מן החי וימות 
וכמה ינטל מן החי וימות רבי זכאי אומר
\clearpage}

\newsection{דף כד}
\twocol{עד הארכובה רבי ינאי אומר עד לנקביו ר' יוחנן אומר משום רבי יוסי בן יהושע עד מקום טבורו 
בין רבי זכאי לרבי ינאי איכא בינייהו טרפה חיה מר סבר טרפה חיה ומר סבר טרפה אינה חיה 
\commenta{בין ר' ינאי לרבי יוחנן איכא בינייהו דר' אלעזר - ותרוייהו אית להו טרפה חיה:}
בין ר' ינאי לר' יוחנן איכא בינייהו דר"א דאמר רבי אלעזר ניטל ירך וחלל שלה נבלה 
\commenta{מלמעלה - שנחתך מגולגלתו:}
אמר רב פפא מחלוקת מלמטה למעלה אבל מלמעלה למטה אפי' כל דהו טהורה וכן אמר רב גידל אמר רבי יוחנן המפלת את שגולגלתו אטומה אמו טהורה 
\commenta{כמין אפקתא דדיקלא - שהדקל למטה יחיד ומתפצל מלמעלה כך היו ידיו ורגליו על כתפו ומלמטה הוא בלא צורה:}
ואמר רב גידל אמר רבי יוחנן המפלת כמין אפקתא דדיקלא אמו טהורה 
איתמר המפלת מי שפניו מוסמסים רבי יוחנן אמר אמו טמאה ר"ל אמר אמו טהורה 
\commenta{מוסמסין - מעוכין קצת ולא כל כך כפניו טוחות:}
איתיביה ר' יוחנן לריש לקיש המפלת יד חתוכה ורגל חתוכה אמו טמאה לידה ואין חוששין שמא מגוף אטום באתה ואם איתא ליתני שמא מגוף אטום או ממי שפניו מוסמסין 
\commenta{כי פליגי בפניו טוחות - שאין צורה בהן ניכרת לגמרי:}
אמר רב פפי בפניו מוסמסין כולי עלמא לא פליגי דטמאה כי פליגי בפניו טוחות ואיפכא איתמר רבי יוחנן אמר אמו טהורה וריש לקיש אמר אמו טמאה 
\commenta{לותביה ריש לקיש לר' יוחנן מהא - דאם איתא ליתני או ממי שפניו טוחות:}
ולותביה ר"ל לרבי יוחנן מהא משום דשני ליה היינו גוף אטום היינו מי שפניו טוחות 
\commenta{היינו פניו טוחות - אבל בפניו מוסמסין ליכא לשנויי הכי דהא ניכרת בהן הצורה ולאו אטום הוא דאטום היינו חסר לגמרי:}
בני רבי חייא נפיק לקרייתא אתו לקמיה דאבוהון אמר להם  כלום בא מעשה לידכם אמרו לו פנים טוחות בא לידינו וטימאנוה 
\commenta{לחומרא - שטמאתם אותה שבועים משום ספק נקבה:}
אמר להם צאו וטהרו מה שטמאתם מאי דעתייכו לחומרא חומרא דאתיא לידי קולא היא דקיהביתו לה ימי טוהר 
\commenta{בבהמה אסור באכילה - אם נשחטה אמו ונמצא במעיה וכל שכן יצא לאויר העולם דנפל הוא ונבלה:}
איתמר המפלת בריה שיש לה ב' גבים וב' שדראות אמר רב באשה אינו ולד בבהמה אסור באכילה ושמואל אמר באשה ולד בבהמה מותר באכילה 
\commenta{השסועה - רחמנא אסרה דכתיב (דברים יד) את זה לא תאכלו וגו' השסועה:}
במאי קמיפלגי בדרב חנין בר אבא דאמר רב חנין בר אבא השסועה בריה שיש לה ב' גבין וב' שדראות 
\commenta{רב סבר - האי דאסר רחמנא כשנמצאת במעי בהמה אסורה דהא בריה בעלמא בפני עצמה ליתא:}
רב אמר בריה בעלמא ליתא וכי אגמריה רחמנא למשה במעי אמה אגמריה ושמואל אמר בריה בעלמא איתא וכי אגמריה רחמנא למשה בעלמא אגמריה אבל במעי אמה שריא 
\commenta{ר' חנינא בן אנטיגנוס - בבכור קאי וקאמר דהוי מום ומדפסל ליה לקרבן מכלל דלהדיוט שרי וקשיא לתרוייהו דהא אפי' יצא לאויר העולם שרי ושמואל לא שרי אלא במעי אמו. והא דאקשי רב שימי טפי לרב מדשמואל משום דגביה הוה קאי ובר בריה הוה כדאמר בפרק כל הבשר (חולין קיא:) רב אקלע לבי רב שימי בר חייא בר בריה:}
איתיביה רב שימי בר חייא לרב רבי חנינא בן אנטיגנוס אומר כל שיש לו ב' גבין ושני שדראות פסול לעבודה אלמא דחיי (וקשיא לרב) א"ל שימי את ששדרתו עקומה 
\commenta{יש בעוברין - שיצאו לאויר העולם שאסורין משום נבלה:}
מיתיבי יש בעוברין שהן אסורין בן ארבעה לדקה בן שמנה לגסה הימנו ולמטה אסור יצא מי שיש לו שני גבין ושני שדראות 
\commenta{לאו יצא מכלל עוברין - דהני אע"ג דלא כלו חדשיהן כי נמצאו במעי אמן שרו אבל האי אפי' במעי אמן אסור וקשיא לשמואל:}
מאי יצא לאו יצא מכלל עוברין שאפילו במעי אמן אסורין 
\commenta{רב מתרץ לטעמיה - דרב לא צריך לתרוצי אלא משום היא גופה דלא משתמע שפיר דהא רישא כשיצאו לאויר העולם מיירי וסיפא דקתני יצא משמע דעלה קאי וכשיצא לאויר העולם משמע דמותר ואי אפשר לומר כן הלכך צריך להוסיף עליה אבל במעי אמו שרי דכתיב כל בבהמה תאכלו בפרק בהמה המקשה (חולין סט.):}
רב מתרץ לטעמיה ושמואל מתרץ לטעמיה רב מתרץ לטעמיה בן ארבעה לדקה בן ח' לגסה הימנו ולמטה אסור 
\commenta{יצא מי שיש לו שני גבין וכו' - דכי כתב רחמנא השסועה למיסר במעי אמו אסרה דהא מין בפני עצמה ליתא:}
במה דברים אמורים כשיצא לאויר העולם אבל במעי אמו שרי יצא מי שיש לו שני גבין ושני שדראות דאפילו במעי אמו נמי אסור
ושמואל מתרץ לטעמיה בן ארבעה לדקה בן שמנה לגסה הימנו ולמטה אסור במה דברים אמורים בשלא כלו לו חדשיו אבל כלו לו חדשיו מותר יצא מי שיש לו ב' גבין וב' שדראות דאע"ג דכלו לו חדשיו אם יצא לאויר העולם אסור במעי אמו שרי 
\commenta{שאינו חתוך - כולו שלם כעין דף ואין בו חיתוך אברים וצורה אלא חלק כעין עיגול:}
תני תנא קמיה דרב המפלת בריית גוף שאינו חתוך ובריית ראש שאינו חתוך יכול תהא אמו טמאה לידה ת"ל (ויקרא יב, ב) אשה כי תזריע וילדה זכר וגו' וביום השמיני ימול וגו'
\commenta{א"ל רב - לתנא סיים בה נמי הכי ואימא יצאו אלו מי שיש לו שני גבין וכו':}
מי שראוי לברית שמנה יצאו אלו שאינן ראויין לברית שמנה א"ל רב וסיים בה הכי ושיש לו שני גבין ושני שדראות 
\commenta{כוותיה דשמואל - בשני גבין דולד הוא וקאמר טמאה אפילו בלידה יבישתא דולד הוא:}
רבי ירמיה בר אבא סבר למעבד עובדא כוותיה דשמואל אמר ליה רב הונא מאי דעתיך לחומרא חומרא דאתי לידי קולא הוא דקיהבת לה דמי טוהר עביד מיהא כותיה דרב דקיימא לן הלכתא כרב באיסורי בין לקולא בין לחומרא 
\commenta{הרי אמרו אשה יולדת לט' ויולדת לז' - ואנא נמי אמינא כי היכי דבהמה לט' כדאמרי' לעיל בן ח' לגסה נפל הכא נמי יולדת לז' נפל או לאו נפל הוא ומותר:}
אמר רבא הרי אמרו אשה יולדת לתשעה ויולדת לשבעה בהמה גסה יולדת לתשעה יולדת לשבעה או לא ילדה 
\commenta{הימנו ולמטה - בברייתא דלעיל תניא:}
אמר רב נחמן בר יצחק ת"ש הימנו ולמטה אסור מאי לאו אגסה לא אדקה 
האי מאי אי אמרת בשלמא אגסה אצטריך סלקא דעתך אמינא הואיל ובאשה חיי בבהמה נמי חיי קמ"ל דלא חיי 
\commenta{אלא אי אמרת אדקה פשיטא - הואיל ובר תלתא הוא לא חיי:}
אלא אי אמרת אדקה איתמר פשיטא בת תלתא ירחי מי קא חיי 
\commenta{כל בציר תרי ירחי - מכדרכו חיי כי היכי דבאשה ובבהמה גסה בר ז' חיי דהוי תרי ירחי בציר מכי אורחיה הכי נמי בר תלתא לדקה חיי אע"ג דבר ד' לא חיי:}
אצטריך סד"א כל בציר תרי ירחי חיי קמ"ל 
\commenta{דמות לילית - שידה. שד יש לו פרצוף אדם ויש לו כנפים:}
אמר רב יהודה אמר שמואל  המפלת דמות לילית אמו טמאה לידה ולד הוא אלא שיש לו כנפים תנ"ה א"ר יוסי מעשה בסימוני באחת שהפילה דמות לילית ובא מעשה לפני חכמים ואמרו ולד הוא אלא שיש לו כנפים 
\commenta{הנהג בן אחיך ובא - כלומר הבא בן אחיך ובא עמו:}
המפלת דמות נחש הורה חנינא בן אחיו של רבי יהושע אמו טמאה לידה הלך ר' יוסף וספר דברים לפני ר"ג שלח לו רבי יהושע הנהג בן אחיך ובא 
\commenta{ ה"ג בהליכתן יצתה כלת חנינא לקראת רבי יהושע:}
בהליכתן יצתה כלת (ר') חנינא לקראתו אמרה לו רבי המפלת כמין נחש מהו אמר לה אמו טהורה אמרה לו והלא משמך אמרה לי חמותי אמו טמאה ואמר לה מאיזה טעם הואיל וגלגל עינו עגול כשל אדם מתוך דבריה נזכר רבי יהושע שלח לו לרבן גמליאל מפי הורה חנינא 
אמר אביי ש"מ צורבא מרבנן דאמר מילתא לימא בה טעמא דכי מדכרו ליה מדכר
\commenta{מתני' מלא גנונים - גוונים:}
{\large\emph{מתני׳}} המפלת שפיר מלא מים מלא דם מלא גנונים אינה חוששת לולד ואם היה מרוקם תשב לזכר ולנקבה המפלת סנדל או שליא תשב לזכר ולנקבה
\commenta{גמ' שמא ולד היה ונימוח - כיון דלאו דם הוא ולאו מים נינהו איכא למיחש לולד:}
{\large\emph{גמ׳}} בשלמא דם ומים לא כלום היא אלא גנונים ניחוש שמא ולד הוה ונימוח אמר אביי כמה יין חי שתת אמו של זה שנמוח עוברה בתוך מעיה 
\commenta{ה"ג רבא אמר מלא תנן:}
רבא אמר מלא תנן ואם איתא דאתמוחי אתמח מחסר חסר רב אדא בר אהבה אמר גוונים תנן ואם איתא דאתמוחי אתמח כולה בחד גוונא הוי קאי 
\commenta{סכויין - שחורין. שמעתי ל"א סכויין בלא מוח ומשיחה אלא יבשה:}
תניא אבא שאול אומר קובר מתים הייתי והייתי מסתכל בעצמות של מתים השותה יין חי עצמותיו שרופין מזוג עצמותיו סכויין כראוי עצמותיו משוחין וכל מי ששתייתו מרובה מאכילתו עצמותיו שרופין אכילתו מרובה משתייתו עצמותיו סכויין כראוי עצמותיו משוחין 
\commenta{קולית - עצם הירך:}
תניא אבא שאול אומר ואיתימא רבי יוחנן קובר מתים הייתי פעם אחת רצתי אחר צבי ונכנסתי בקולית של מת ורצתי אחריו שלש פרסאות וצבי לא הגעתי וקולית לא כלתה כשחזרתי לאחורי אמרו לי של עוג מלך הבשן היתה 
תניא אבא שאול אומר קובר מתים הייתי פעם אחת נפתחה מערה תחתי ועמדתי בגלגל עינו של מת עד חוטמי כשחזרתי לאחורי אמרו עין של אבשלום היתה 
ושמא תאמר אבא שאול ננס הוה אבא שאול ארוך בדורו הוה ורבי טרפון מגיע לכתפו ור' טרפון ארוך בדורו הוה ור"מ מגיע לכתפו רבי מאיר ארוך בדורו הוה ורבי מגיע לכתפו רבי ארוך בדורו הוה
\commenta{דיילא - שמש דרבנן ואדא שמו:}
ורבי חייא מגיע לכתפו ורבי חייא ארוך בדורו הוה ורב מגיע לכתפו רב ארוך בדורו הוה ורב יהודה מגיע לכתפו ורב יהודה ארוך בדורו הוה ואדא דיילא מגיע לכתפו
\clearpage}

\newsection{דף כה}
\twocol{פרשתבינא דפומבדיתא קאי ליה לאדא דיילא עד פלגיה וכולי עלמא קאי לפרשתבינא דפומבדיתא עד חרציה
שאלו לפני רבי המפלת שפיר מלא בשר מהו אמר להם לא שמעתי 
אמר לפניו ר' ישמעאל בר' יוסי כך אמר אבא מלא דם טמאה נדה מלא בשר טמאה לידה 
א"ל אילמלי דבר חדש אמרת לנו משום אביך שמענוך עכשיו
\commenta{כסומכוס - בריש פירקין דם שבתוך החתיכה מטמא משום נדה ויחידאה היא:}
מדהא קמייתא כיחידאה קאמר כסומכוס שאמר משום ר"מ הא נמי שמא כרבי יהושע אמרה ואין הלכה כר' יהושע 
דתניא המפלת שפיר שאינו מרוקם ר' יהושע אומר ולד וחכ"א אינו ולד 
\commenta{בעכור - מה שבתוכו עכור דאיכא למימר ולד היה ונימוח:}
אמר ר"ש בן לקיש משום ר' אושעיא מחלוקת בעכור אבל בצלול דברי הכל אינו ולד ור' יהושע בן לוי אמר בצלול מחלוקת 
איבעיא להו בצלול מחלוקת אבל בעכור דברי הכל ולד או דלמא בין בזה ובין בזה מחלוקת תיקו 
מיתיבי את זו דרש ר' יהושע בן חנניא (בראשית ג, כא) ויעש ה' אלהים לאדם ולאשתו כתנות עור וילבישם מלמד שאין הקב"ה עושה עור לאדם אלא א"כ נוצר 
אלמא בעור תליא מילתא לא שנא עכור ול"ש צלול 
\commenta{היינו דאיצטריך קרא - לרבי יהושע דמשום עור לחודיה הוי ולד:}
אי אמרת בשלמא בצלול מחלוקת היינו דאיצטריך קרא אלא אי אמרת בעכור מחלוקת למה לי קרא סברא בעלמא הוא אלא שמע מינה בצלול מחלוקת שמע מינה 
וכן אמר ר"נ אמר רבה בר אבוה מחלוקת בעכור אבל בצלול דברי הכל אינו ולד 
\commenta{אלא אמרו - רישא בבכורות פרק הלוקח בהמה מן העובד כוכבים (דף יט:) לענין לפטור את שבא אחריו מתורת בכור:}
איתיביה רבא לר"נ אלא אמרו סימן ולד בבהמה דקה טינוף בגסה שליא באשה שפיר ושליא
ואילו שפיר בבהמה לא פטר אי אמרת בשלמא בצלול מחלוקת משום הכי 
אשה דרבי בה קרא פטר בה שפיר בבהמה דלא רבי קרא לא פטר בה שפיר 
אלא אי אמרת בעכור מחלוקת מכדי סברא הוא מאי שנא אשה ומאי שנא בהמה 
\commenta{ספוקי מספקא ליה - בעכור גופיה אי ולד הוא אם לאו:}
מי סברת רבי יהושע מפשט פשיט ליה רבי יהושע ספוקי מספקא ליה ואזיל הכא לחומרא והכא לחומרא 
גבי אשה דממונא הוא ספק ממונא לקולא 
גבי בהמה דאיסורא הוא דאיכא לגבי גיזה ועבודה ספק איסורא לחומרא ה"נ גבי אשה ספק טומאה לחומרא 
ומי מספקא ליה והא קרא קאמר מדרבנן וקרא אסמכתא בעלמא הוא 
\commenta{הא רבי - דאמר לעיל לא שמעתי:}
א"ל רב חנינא בר שלמיא לרב הא רבי הא ר' ישמעאל בר' יוסי והא רבי אושעיא והא רבי יהושע בן לוי מר כמאן ס"ל 
\commenta{אחד זה ואחד זה - אחד עכור ואחד צלול:}
א"ל אני אומר אחד זה ואחד זה אינה חוששת 
\commenta{לא דכו שפיר - לא טיהרו את המפלת שפיר משבועים ואפי' בלידה יבישתא:}
ושמואל אמר אחד זה ואחד זה חוששת ואזדא שמואל לטעמיה דכי אתא רב דימי אמר מעולם לא דכו שפיר בנהרדעא לבר מההוא שפירא דאתא לקמיה דשמואל דמנח עליה חוט השערה מהאי גיסא וחזיא מהאי גיסא אמר אם איתא דולד הואי לא הוה זיג כולי האי
\commenta{טיפין של זבוב - עיני זבוב:}
ואם היה מרוקם וכו' תנו רבנן איזהו שפיר מרוקם אבא שאול אומר תחלת ברייתו מראשו ושתי עיניו כשתי טיפין של זבוב תני רבי חייא מרוחקין זה מזה שני חוטמין כשתי טיפים של זבוב 
\commenta{ופיו מתוח כחוט השערה - סדק קטן ונראה כאילו חוט השערה מתוח שם:}
תני רבי חייא ומקורבין זה לזה ופיו מתוח כחוט השערה וגויתו כעדשה ואם היתה נקבה נדונה כשעורה לארכה 
\commenta{וחיתוך ידים ורגלים - עדיין אין נראין כדאמרינן תחלת ברייתו מראשו:}
וחתוך ידים ורגלים אין לו ועליו מפורש בקבלה (איוב י, י) הלא כחלב תתיכני וכגבינה תקפיאני עור ובשר תלבישני ועצמות וגידים תסוככני חיים וחסד עשית עמדי ופקודתך שמרה רוחי 
ואין בודקין אותו במים שהמים עזין
וטורדין אותו אלא בודקין אותו בשמן שהשמן רך ומצחצחו ואין רואין אלא בחמה 
כיצד בודקין אותו כיצד בודקין אותו כדאמרינן אלא במה בודקין אותו לידע אם זכר הוא אם נקבה היא 
\commenta{בר רמש:}
אבא שאול בר נש ואמרי לה אבא שאול בר רמש אומר מביא קיסם שראשו חלק ומנענע באותו מקום אם מסכסך בידוע שזכר הוא ואם לאו בידוע שנקבה היא 
\commenta{מלמטה למעלה - שהרי בסדק השעורה אין עיכוב לארכה:}
א"ר נחמן אמר רבה בר אבוה ל"ש אלא מלמטה למעלה אבל מן הצדדין אימא כותלי בית הרחם נינהו 
\commenta{תנא אם היתה נקבה נידונת כשעורה סדוקה - לאוסופי ולפרושי אתא דלעיל תני נידונת כשעורה ולא תני סדוקה:}
א"ר אדא בר אהבה תנא אם היתה נקבה נדונה כשעורה סדוקה מתקיף לה ר"נ ודילמא חוט של ביצים נינהו אמר אביי השתא ביצים גופייהו לא ידיעי חוט של ביצים ידיע 
\commenta{שתי ירכותיו - כשהוא מתחיל להבראות בחיתוך ידים ורגלים:}
א"ר עמרם תנא ב' ירכותיו כב' חוטין של זהורית וא"ר עמרם עלה כשל ערב ושני זרועותיו כב' חוטין של זהורית וא"ר עמרם עלה כשל שתי 
\commenta{לא תעביד עובדא - לעשותו ולד עד שיהא לו שערות בראשו:}
א"ל שמואל לרב יהודה שיננא לא תעביד עובדא עד שישעיר ומי אמר שמואל הכי והאמר שמואל אחת זו ואחת זו חוששת 
\commenta{לחוש חוששת - להחמיר ולישב שבועים:}
אמר רב אמי בר שמואל לדידי מפרשא לי מיניה דמר שמואל לחוש חוששת ימי טהרה לא יהבינן לה עד שישעיר 
למימרא דמספקא ליה לשמואל והא ההוא שפירא דאתאי לקמיה דמר שמואל אמר הא בר ארבעין וחד יומא וחשיב מיומא דאזלא לטבילה עד ההוא יומא ולא הוה אלא ארבעין יומין
\commenta{כפתיה ואודי - אלמא היה בקי ביצירת הולד:}
ואמר להו האי בנדה בעל כפתיה ואודי שאני שמואל דרב גובריה
\commenta{דג של ים - דג יש בים ושמו סנדל:}
המפלת סנדל וכו' ת"ר סנדל דומה לדג של ים מתחלתו ולד הוא אלא שנרצף רשב"ג אומר סנדל דומה ללשון של שור הגדול משום רבותינו העידו סנדל צריך צורת פנים 
\commenta{שסטר - הכהו על לחיו:}
א"ר יהודה אמר שמואל הלכה סנדל צריך צורת פנים ... א"ר אדא א"ר יוסף א"ר יצחק סנדל צריך צורת פנים ואפי' מאחוריו משל לאדם שסטר את חבירו והחזיר פניו לאחוריו 
בימי רבי ינאי בקשו לטהר את הסנדל שאין לו צורת פנים אמר להם ר' ינאי טיהרתם את הוולדות 
\commenta{משנה זו - דקתני צריך צורת פנים מעדותו של רבי נחוניא נשנית הוא העיד עליה בבית המדרש ויחידאה היא:}
והתניא משום רבותינו העידו סנדל צריך צורת פנים אמר רב ביבי בר אביי אמר רבי יוחנן מעדותו של רבי נחוניא נשנית משנה זו אמר רבי זעירא זכה בה רב ביבי בשמעתיה דאנא והוא הוינא יתבינן קמיה דרבי יוחנן כי אמרה להא שמעתא וקדם איהו ואמר וזכה בה 
\commenta{והלא אין סנדל שאין עמו ולד - ובלא סנדל נמי הויא טמאה לידה:}
למה הזכירו סנדל והלא אין סנדל שאין עמו ולד 
\commenta{אי דאיתיליד נקבה בהדיה ה"נ - דלא היה צריך להזכירו דיש לה ימי טומאה וטהרה דנקבה שאפילו הוא זכר גמור כי איכא נקבה בהדיה יושבת לנקבה ימי טומאה וטהרה ושל זכר מובלעים בתוכן:}
אי דאתילידא נקבה בהדיה ה"נ הכא במאי עסקינן דאתיליד זכר בהדיה 
\commenta{האי נמי זכר - ולא תשב אלא לזכר קמ"ל כו':}
מהו דתימא הואיל ואמר רב יצחק בר אמי אשה מזרעת תחילה יולדת זכר איש מזריע תחלה יולדת נקבה מדהא זכר הא נמי זכר
קמ"ל אימא שניהם הזריעו בבת אחת האי זכר והאי נקבה 
\commenta{ד"א - אפי' ילדה עמו נקבה צריך להזכיר ספק סנדל: }
דבר אחר שאם תלד נקבה לפני שקיעת החמה וסנדל לאחר שקיעת החמה
מונה תחלת נדה לראשון ותחלת נדה לאחרון 
סנדל דתנן
\clearpage}

\newsection{דף כו}
\twocol{גבי בכורות למאי הלכתא 
לבא אחריו בכור לנחלה ואין בכור לכהן 
\commenta{גבי כריתות - בפ"ק (כריתות דף ז:) לענין חיוב קרבן:}
סנדל דתנן גבי כריתות למאי הלכתא 
\commenta{ולד יוצא דרך דופן - ע"י סם לא מיחייבה קרבן דכתיב גבי קרבן (ויקרא יב) אשה כי תזריע וילדה זכר עד שתלד ממקום שמזרעת:
}
שאם תלד ולד דרך דופן וסנדל דרך רחם דמייתא קרבן אסנדל 
\commenta{ולר' שמעון כו' - לקמן בפ' יוצא דופן (נדה דף מ.):}
ולרבי שמעון דאמר יוצא דרך דופן ולד מעליא הוא מאי איכא למימר 
אמר רבי ירמיה שאם תלד ולד בהיותה עובדת כוכבים וסנדל לאחר שנתגיירה דמייתא קרבן אסנדל 
\commenta{כשהן יוצאין - סנדל וולד שעמו:}
אמרוה רבנן קמיה דרב פפא ומי איתנהו להני שינויי והא תניא כשהן יוצאין אין יוצאין אלא כרוכין 
\commenta{שמע מינה - מדהוזכר סנדל גבי כריתות משום גירות וכגון שיצא ולד תחלה ואחר כן סנדל וגבי בכורות כגון שיצא סנדל תחלה כדאוקמינן לעיל דפוטר ולד מחמש סלעים ותניא דבבת אחת יוצאין שמע מינה אין שוכבין זה אצל זה דא"כ לא משכחת לה שיצא זה קודם זה אלא האי כרוכין דקתני הכי הוא מכרך כריך ליה ולד לסנדל אפלגיה דראש הולד תחוב כנגד טיבורו של סנדל:}
אמר רב פפא שמע מינה מכרך כריך ליה ולד לסנדל אפלגיה ומשלחיף ליה כלפי רישיה גבי בכורות כגון שיצאו דרך ראשיהם דסנדל קדים ונפיק גבי כריתות שיצאו דרך מרגלותיהם דולד קדים ונפיק 
\commenta{מצומצמין - שוכבין זה אצל זה:}
רב הונא בר תחליפא משמיה דרבא אמר אפילו תימא מצומצמין ואיפוך שמעתתא גבי בכורות שיצאו דרך מרגלותיהם ולד דאית ביה חיותא סריך ולא נפיק סנדל דלית ביה חיותא שריק ונפיק גבי כריתות שיצאו דרך ראשיהן ולד דאית ביה חיותא מדנפיק רישיה הויא לידה סנדל עד דנפיק רוביה
\commenta{מתני' הבית טמא - משום אהל המת דולד הוה בה ומת:}
{\large\emph{מתני׳}} שליא בבית הבית טמא לא שהשליא ולד אלא שאין שליא בלא ולד 
\commenta{נימוק הולד - ונעשה דם ונתערב בדם הלידה ובטל ברוב:}
רבי שמעון אומר נימוק הולד עד שלא יצא
\commenta{גמ' וסופה - כשהיא הולכת ורווחת הויא רחב כתורמוס. לישנא אחרינא גרסינן בתוספתא וראשה דומה כתורמוס:}
{\large\emph{גמ׳}} תנו רבנן שליא תחלתה דומה לחוט של ערב וסופה דומה כתורמוס וחלולה כחצוצרת ואין שליא פחותה מטפח רבי שמעון בן גמליאל אומר שליא דומה לקורקבן של תרנגולין שהדקין יוצאין ממנה 
תניא רבי אושעיא זעירא דמן חבריא חמשה שיעורן טפח ואלו הן שליא שופר שדרה דופן סוכה והאזוב 
\commenta{שיאחזנו בידו - אחיזה אית בה טפח שהטפח ארבע אצבעות בגודל:}
שליא הא דאמרן שופר דתניא כמה יהא שיעור שופר פירש רבי שמעון בן גמליאל כדי שיאחזנו בידו ויראה לכאן ולכאן טפח 
\commenta{שדרו של לולב צריך שיהא יוצא מן ההדס טפח - לבד אורך העלין. שדרה קרי כל זמן שהעלין הולכין ומתחברין ומוסיפין ואחר שכלתה השדרה גבוהין אורך העלין למעלה:}
שדרה מה היא דא"ר פרנך אמר רבי יוחנן שדרו של לולב צריך שיהא יוצא מן ההדס טפח דופן סוכה דתניא שתים כהלכתן שלישית אפילו טפח אזוב דתני רבי חייא אזוב טפח 
\commenta{חדא היא - דליכא שמעתתא אלא האי דשדרה דאמר ר' יוחנן:}
אמר רבי חנינא בר פפא דריש שילא איש כפר תמרתא תלת מתניתא ותרתי שמעתתא שיעורא טפח תרתי חדא היא אמר אביי אימא אמר רבי חייא אזוב טפח 
\commenta{טפח על טפח על רום טפח - אהל קטן שהוא טפח אורך וטפח רוחב ורום חללו טפח:}
ותו ליכא והאיכא טפח על טפח על רום טפח מרובע מביא את הטומאה וחוצץ בפני הטומאה 
טפח קאמרינן טפח על טפח לא קאמרינן 
\commenta{תנור - מקום שפיתת שתי קדרות. כירה מקום שפיתת קדרה אחת:}
והא איכא אבן היוצא מן התנור טפח ומן הכירה שלש אצבעות חבור 
כי קאמרינן היכא דבציר מטפח לא חזי אבל הכא כ"ש דבציר מטפח יד תנור הוא 
והאיכא
תנורי בנות טפח דתניא תנור תחלתו ארבעה ושיריו ד' דברי רבי מאיר 
\commenta{ה"ג תחלתו כל שהוא משתגמר מלאכתו ושיריו ברובו ומקשה בהעור והרוטב רובו דטפח למאי חזי ומשני האי ושיריו אגדול קאי וה"ק שירי גדול ברובו ומותבינן בגדול והא קאמר שיריו ד' ומשנינן ההוא בתנור בר ז' דד' הוי רובו והא בתנור בר ט':}
וחכמים אומרים במה דברים אמורים בגדול אבל בקטן תחלתו כל שהוא משתגמר מלאכתו ושיריו ברובו 
\commenta{בפלוגתא - כגון הכא דלר"מ לא הוי שיעורא בטפח:}
וכמה כל שהוא אמר רבי ינאי טפח שכן עושין תנורי בנות טפח בפלוגתא לא קמיירי 
\commenta{לא אמרו טפח אלא בין תנור ולכותל - דהתם יותר מטפח עומדת ליקצץ מפני שמעכב את התנור מלקרבו לכותל ודוחקת את הכותל הילכך יותר מטפח עומדת לינטל היא אבל לצד הבית אפי' גדולה הרבה יד הוא לתנור:}
השתא דאתית להכי הא נמי פלוגתא היא דקתני סיפא אמר ר' יהודה לא אמרו טפח אלא מן התנור ולכותל 
\commenta{בדכתיבן - ששיעורו מפורש מן התורה:}
והאיכא מסגרת טפח בדכתיבן לא קא מיירי והאיכא כפורת טפח בקדשים לא קמיירי 
\commenta{דיה לקורה - לגבי מבוי:}
והאיכא דיה לקורה שהיא רחבה טפח בדרבנן לא קמיירי אלא בדכתיבן ולא מפרשי שיעורייהו 
\commenta{כל ג' ימים - כגון יצא ולד תחלה ושליא אחריו בתוך ג' ימים:}
יתיב רב יצחק בר שמואל בר מרתא קמיה דרב כהנא ויתיב וקאמר אמר רב יהודה אמר רב כל שלשה ימים הראשונים תולין את השליא בולד מכאן ואילך חוששין לולד אחר 
\commenta{בנפלים - מתעכב:}
אמר ליה ומי אמר רב הכי והאמר רב אין הולד מתעכב אחר חבירו כלום אישתיק אמר ליה דלמא כאן בנפל כאן בבן קיימא 
\commenta{את אמרת לשמעתתיה דרב - בתמיה. כלומר את אתית לפרושי מדידך:}
אמר ליה את אמרת לשמעתתיה דרב בפירוש אמר רב הכי הפילה נפל ואחר כך הפילה שליא כל שלשה ימים תולין את השליא בולד מכאן ואילך חוששין לולד אחר ילדה ואח"כ הפילה שליא אפילו מכאן ועד עשרה ימים אין חוששין לולד אחר 
\commenta{באלי ואתי - שהיה ממהר לבא כמו (לעיל נדה דף יז.) באלי דידבי פי' מגרש:}
שמואל ותלמידי דרב ורב יהודה הוו יתבי חליף ואזיל רב יוסף בריה דרב מנשיא מדויל לאפייהו באלי ואתי אמר אתי לן גברא דרמינן ליה בגילא דחטתא ומרמי ומדחי 
\commenta{אלא בדבר של קיימא - שכיוצא בו מתקיים אם היום היו חדשיו כלו למעוטי שאם הפילה דבר שאין ראוי לבריית נשמה כגון בירך אחד באמצע או גוף אטום או מפלת מין בהמה וחיה ועוף ואח"כ הפילה שליא אפי' בתוך ג' חוששת לולד אחר:}
אדהכי אתא אמר ליה שמואל מאי אמר רב בשליא א"ל הכי אמר רב אין תולין את השליא אלא בדבר של קיימא שיילינהו שמואל לכל תלמידי דרב ואמרי ליה הכי הדר חזייה לרב יהודה בישות 
\commenta{המפלת עורב ושליא מהו - מי תלינן שליא בעורב ולא ניחוש לולד אחר או לא תלינן:}
בעא מיניה רבי יוסי בן שאול מרבי המפלת דמות עורב ושליא מהו אמר ליה אין תולין אלא בדבר שיש במינו שליא 
קשורה בו מהו אמר ליה דבר שאינו שאלת איתיביה המפלת מין בהמה חיה ועוף ושליא עמהן בזמן שהשליא קשורה עמהן אין חוששין לולד אחר אין שליא קשורה עמהן חוששין לולד אחר הריני מטיל עליהן
\clearpage}

\newsection{דף כז}
\twocol{חומר שני ולדות שאני אומר שמא נמוח שפיר של שליא ונמוח שליא של שפיר תיובתא 
\commenta{אחר הולד - אבל יצתה שליא קודם חוששין לאחר:}
אמר רבה בר שילא אמר רב מתנה אמר שמואל מעשה ותלו את השליא בולד עד עשרה ימים ולא אמרו תולין אלא בשליא הבאה אחר הולד 
אמר רבה בר בר חנה אמר רבי יוחנן מעשה ותלו את השליא בולד עד כ"ג ימים אמר ליה רב יוסף עד כ"ד אמרת לן 
אמר רב אחא בריה דרב עוירא א"ר יצחק מעשה ונשתהה הולד אחר חבירו ל"ג יום א"ל רב יוסף ל"ד אמרת לן 
\commenta{הניחא למ"ד כו' - פלוגתייהו לקמן בפ' בנות כותים (נדה דף לח:):}
הניחא למאן דאמר יולדת לתשעה יולדת למקוטעין משכחת לה אחד נגמרה צורתו לסוף שבעה ואחד נגמרה צורתו לתחלת תשעה אלא למ"ד יולדת לתשעה אינה יולדת למקוטעין מאי איכא למימר 
\commenta{איפוך שמעתא - דר' יוחנן:}
איפוך שמעתתא ל"ג לשליא כ"ג לולד 
א"ר אבין בר רב אדא אמר רב מנחם איש כפר שערים ואמרי לה בית שערים מעשה ונשתהה ולד אחד אחר חבירו ג' חדשים והרי הם יושבים לפנינו בבית המדרש ומאן נינהו יהודה וחזקיה בני רבי חייא 
\commenta{האמר מר כו' - קס"ד בג' חדשים לא משכחת לה אא"כ נתעברה זה אחר זה:}
והא אמר מר אין אשה מתעברת וחוזרת ומתעברת אמר אביי טיפה אחת היתה ונתחלקה לשתים אחד נגמרה צורתו בתחלת ז' ואחד בסוף ט'
שליא בבית הבית טמא תנו רבנן שליא בבית הבית טמא לא שהשליא ולד אלא שאין שליא שאין ולד עמה דברי רבי מאיר רבי יוסי ורבי יהודה ורבי שמעון מטהרין 
\commenta{לבית החיצון - לבית אחר שהבית טהור:}
אמרו לו לרבי מאיר אי אתה מודה שאם הוציאוהו בספל לבית החיצון שהוא טהור אמר להן אבל ולמה לפי שאינו 
\commenta{אמרו לו למה - מאי שנא בית ראשון דמטמית ליה אמר להו לפי שאינו הואיל וטלטלוהו נמוק:}
אמרו לו כשם שאינו בבית החיצון כך אינו בבית הפנימי אמר להן אינו דומה נמוק פעם אחת לנמוק ב' פעמים 
\commenta{מ"ט דר"ש - נהי דנמוק מ"מ כל גופו של מת כאן הוה וה"ל כרקב וכנצל [בשר המת שנימוח ונעשה ליחה סרוחה]:}
יתיב רב פפא אחורי דרב ביבי קמיה דרב המנונא ויתיב וקאמר מאי טעמא דרבי שמעון קסבר כל טומאה שנתערב בה ממין אחר בטלה 
\commenta{אחיכו עליה - אמרו ליה פשיטא מאי שנא טעמייהו מטעמיה כולה חדא מילתא אמרי:}
אמר להו רב פפא היינו נמי טעמייהו דרבי יהודה ורבי יוסי אחיכו עליה מאי שנא פשיטא 
\commenta{אפי' כי האי מילתא - דאתיא לידי חוכא:}
אמר רב פפא אפילו כי הא מילתא לימא איניש ולא נשתוק קמיה רביה משום שנאמר (משלי ל, לב) אם נבלת בהתנשא ואם זמות יד לפה 
\commenta{מלא תרוד - כף:}
ואזדא רבי שמעון לטעמיה דתניא מלא תרוד רקב שנפל לתוכו עפר כל שהו טמא ורבי שמעון מטהר 
מאי טעמא דרבי שמעון אמר רבה אשכחתינהו לרבנן דבי רב דיתבי וקאמרי אי אפשר שלא ירבו שתי פרידות עפר על פרידה אחת של רקב וחסיר ליה 
\commenta{א"א - אף על גב דרובא רקב אי אפשר למקום שנפל העפר שלא תהא גרגר של רקב בין שני גרגרים של עפר ובטל הרקב ובציר לו שיעורא דתרוד:}
ואמינא להו אדרבה א"א שלא ירבו שתי פרידות רקב על
פרידה אחת עפר (ונפיל) ליה שיעורא 
\commenta{סופו - של רקב כתחלתו:}
אלא אמר רבה היינו טעמא דרבי שמעון סופו כתחלתו מה תחלתו נעשה לו דבר אחר גנגילון אף סופו נעשה לו דבר אחר גנגילון 
\commenta{מאי היא - דאמר תחלתו ד"א נעשה לו גנגילון:}
מאי היא דתניא איזהו מת שיש לו רקב ואיזהו מת שאין לו רקב נקבר ערום בארון של שיש או ע"ג רצפה של אבנים זהו מת שיש לו רקב
\commenta{נקבר בכסותו - דאיכא רקבובית או אפילו ערום בארון של עץ דאיכא רקבובית:}
ואיזהו מת שאין לו רקב נקבר בכסותו או בארון של עץ או ע"ג רצפה של לבנים זהו מת שאין לו רקב ולא אמרו רקב אלא למת בלבד למעוטי הרוג דלא 
\commenta{שנתפזר בתוך הבית [הבית] טמא - משום אהל:}
גופא מלא תרוד רקב שנפל לתוכו עפר כל שהוא טמא ור' שמעון מטהר מלא תרוד רקב שנתפזר בבית הבית טמא ורבי שמעון מטהר 
\commenta{דאין מאהיל וחוזר ומאהיל - גג שכנגד זה מאהיל על חצי שיעור ושכנגד זה מאהיל על חצי שיעור ושני אהלות הן ואין מצטרפין:}
וצריכא דאי אשמעינן קמייתא בההיא קאמרי רבנן משום דמכניף אבל נתפזר אימא מודו לו לרבי שמעון דאין מאהיל וחוזר ומאהיל 
ואי אשמעינן בהא בהא אמר רבי שמעון דאין מאהיל וחוזר ומאהיל אבל בהא אימא מודה להו לרבנן צריכא 
\commenta{מלא תרוד ועוד - יותר ממלא תרוד עפר בית הקברות. ל"א רקב ממש של מת אלא כגון שנקבר בכסותו או בקרקע בלא ארון של שיש ויש כאן מלא תרוד ועוד מאותו עפר דהוו מעורבין רקב ועפר:}
תניא אידך מלא תרוד ועוד עפר בית הקברות טמא ורבי שמעון מטהר מאי טעמייהו דרבנן לפי שא"א למלא תרוד ועוד עפר בית הקברות שאין בו מלא תרוד רקב 
\commenta{בטל ברוב - שדם הלידה נתרבה על מיחוי הולד ומבטלו:}
השתא דאמרת טעמא דרבי שמעון משום סופו כתחלתו גבי שליא מאי טעמא אמר רבי יוחנן משום בטול ברוב נגעו בה 
\commenta{אמרו דבר אחד - ולקמן מוקמינן דר"א בן יעקב משום בטול ברוב:}
ואזדא רבי יוחנן לטעמיה דאמר רבי יוחנן רבי שמעון ור"א בן יעקב אמרו דבר אחד רבי שמעון הא דאמרן רבי אליעזר דתניא רבי אליעזר בן יעקב אומר בהמה גסה ששפעה חררת דם הרי זו תקבר ופטורה מן הבכורה 
\commenta{אינה מטמאה - אותה חררה:
}
ותני רבי חייא עלה אינה מטמאה לא במגע ולא במשא ומאחר שאינה מטמאה לא במגע ולא במשא אמאי תקבר כדי לפרסמה שהיא פטורה מן הבכורה 
\commenta{ופרכינן מדפטורה מן הבכורה אלמא ולד מעליא היא אמאי אינה מטמאה:}
אלמא ולד מעליא הוא ואמאי תני רבי חייא אינה מטמאה לא במגע ולא במשא אמר רבי יוחנן משום בטול ברוב נגעו בה 
\commenta{ומודה ר"ש - במתני' בשליא אע"ג שהבית טהור אמו טמאה לידה:}
א"ר אמי אמר רבי יוחנן ומודה רבי שמעון שאמו טמאה לידה 
\commenta{כעין שהזריעה - כלומר דנימוח כזרע:}
אמר ההוא סבא לרבי אמי אסברא לך טעמא דרבי יוחנן דאמר קרא (ויקרא יב, ב) אשה כי תזריע וילדה זכר וגו' אפילו לא ילדה אלא כעין שהזריעה טמאה לידה 
\commenta{שטרפוהו - כמו (חולין סד.) ביצים טרופות בקערה. שמחקוהו ובלבלוהו:}
ריש לקיש אמר שפיר שטרפוהו במימיו נעשה כמת שנתבלבלה צורתו 
\commenta{שנתבלבלה - נמחק כולו כמו שרוף שנתפזר ואין שלדו קיימת:}
אמר ליה רבי יוחנן לריש לקיש מת שנתבלבלה צורתו מנלן דטהור אילימא מהא דאמר רבי שבתאי אמר ר' יצחק מגדלאה ואמרי לה א"ר יצחק מגדלאה א"ר שבתאי מת שנשרף ושלדו קיימת טמא מעשה היה וטמאו לו פתחים גדולים
\clearpage}

\newsection{דף כח}
\twocol{וטהרו לו פתחים קטנים וקא דייקת מינה טעמא דשלדו קיימת הא לאו הכי טהור 
אדרבה דוק מינה להאי גיסא שלדו קיימת הוא דטהרו לו פתחים קטנים הא לאו הכי פתחים קטנים נמי טמאין דכל חד וחד חזי לאפוקי חד חד אבר 
\commenta{ר' יוחנן - דפליג עליה דריש לקיש דבעי למימר דמת שנתבלבלה צורתו טמא דאמר כמאן כר' אליעזר:}
א"ל רבינא לרב אשי ר' יוחנן דאמר כמאן כר' אליעזר דתנן אפר שרופין ר"א אומר שיעור' ברובע 
\commenta{קטבלא - עור שלוק וקשה ואינה נשרפת עם המת:}
היכי דמי מת שנשרף ושלדו קיימת אמר אביי כגון ששרפו על גבי קטבלא רבא אמר כגון ששרפו על גבי אפודרים רבינא אמר כגון דאיחרכי אחרוכי 
ת"ר המפלת יד חתוכה ורגל חתוכה אמו טמאה לידה ואין חוששין שמא מגוף אטום באו 
\commenta{אין נותנין לה ימי טוהר - דאע"ג דפשיטא לן דולד הוא:}
רב חסדא ורבה בר רב הונא דאמרי תרוייהו אין נותנין לה ימי טוהר מ"ט אימא הרחיקה לידתה 
\commenta{ניתני ולנדה - שכל דמים שתראה יהיו טמאים שלא יהיו לה ימי טוהר והכי איבעי ליה למיתני תשב לזכר ולנקבה ולנדה כלומר שבועיים דנקבה מטמאה בלא שום ראייה דדלמא נקבה הואי וימי טוהר לא יהיה לה שהרי כמה ימים שהתחילה לילד ויצא הרוב והיינו לנדה. ולקמן בהאי פירקא (נדה דף ל.) מפרש למה הוזכר זכר שאם תראה לסוף ל"ד ותחזור ותראה ליום מ"א תהא מקולקלת עד יום מ"ח:}
מתיב רב יוסף המפלת ואין ידוע מה הפילה תשב לזכר ולנקבה ואי ס"ד כל כהאי גוונא אימא הרחיקה לידתה לתני ולנדה 
\commenta{הוה אמינא מביאה קרבן ואינו נאכל - דכיון דתני לנדה משמע דמספקא ליה אי ולד הוא אי לא קמ"ל מדלא תני לנדה אלא לזכר ולנקבה דודאי פשיטא לן דולד הוא ומביאה קרבן ונאכל ומיהו ימי טוהר לית לה אימר הרחיקה לידתה:}
אמר אביי אי תנא לנדה הוה אמינא מביאה קרבן ואינו נאכל קמ"ל דנאכל 
\commenta{ויתן יד - אלמא לידת יד קרויה לידה:}
אמר רב הונא הוציא עובר את ידו והחזירה אמו טמאה לידה שנאמר (בראשית לח, כח) ויהי בלדתה ויתן יד 
מתיב רב יהודה הוציא עובר את ידו אין אמו חוששת לכל דבר אמר רב נחמן לדידי מיפרשא לי מיניה דרב הונא לחוש חוששת ימי טוהר לא יהבינן לה עד דנפיק רוביה 
\commenta{והא קרא קאמר - רב הונא אלמא לידה דאורייתא קרי ליה:}
והא אין אמו חוששת לכל דבר קאמר אמר אביי אינה חוששת לכל דבר מדאורייתא אבל מדרבנן חוששת והא קרא קאמר מדרבנן וקרא אסמכתא בעלמא
\commenta{מתני' לזכר ולנקבה - ימי טוהר לזכר וימי טומאה דנקבה:}
{\large\emph{מתני׳}} המפלת טומטום ואנדרוגינוס תשב לזכר ולנקבה
\commenta{תשב לנקבה בלבד - אפי' הוי טומטום זכר בתר נקבות אזלינן דכל ימי זכר בין לטומאה בין לטהרה מובלעים תוך של נקבה:}
טומטום וזכר אנדרוגינוס וזכר תשב לזכר ולנקבה טומטום ונקבה אנדרוגינוס ונקבה תשב לנקבה בלבד 
\commenta{מסורס - דרך מרגלותיו ולשון היפוך הוא כמו סרס המקרא ודרשהו:}
יצא מחותך או מסורס משיצא רובו הרי הוא כילוד יצא כדרכו עד שיצא רוב ראשו ואיזהו רוב ראשו משיצא פדחתו
{\large\emph{גמ׳}} השתא טומטום לחודיה ואנדרוגינוס לחודיה אמר תשב לזכר ולנקבה טומטום וזכר אנדרוגינוס וזכר מיבעיא
\commenta{גמ' איצטריך - אע"ג דטומטום לחודיה מספקינן בזכר ובנקבה היכא דאיתיליד זכר בהדיה נימא זכר הוא דמוכחא מילתא דהיא הזריעה תחילה קמ"ל:}
איצטריך מהו דתימ' הואיל וא"ר יצחק אשה מזרעת תחל' יולדת זכר איש מזריע תחלה יולדת נקבה אימא מדהאי זכר האי נמי זכר קמ"ל אימא שניהם הזריעו בבת אחת זו זכר וזה נקבה 
\commenta{לובן - דומה לקרי:}
אמר ר"נ אמר רב טומטום ואנדרוגינוס שראו לובן או אודם אין חייבין על ביאת מקדש ואין שורפין עליהם את התרומה 
\commenta{ראו אודם ולובן כאחד - דממ"נ טמא אפ"ה לענין טומאת ביאת מקדש פטירי מקרבן כדמפרש טעמא:}
ראו לובן ואודם כאחד אין חייבין על ביאת מקדש אבל שורפין עליהם את התרומה שנאמר {במדבר ה׳:ג׳ } מזכר ועד נקבה
תשלחו זכר ודאי נקבה ודאית ולא טומטום ואנדרוגינוס .
לימא מסייע ליה טומטום ואנדרוגינוס שראו לובן או אודם אין חייבין על ביאת מקדש ואין שורפין עליהם את התרומה ראו לובן ואודם כאחת אין חייבין על ביאת מקדש אבל שורפין עליהם את התרומה 
\commenta{תשלחו - היינו שילוח טמאים ממקדש:}
מ"ט לאו משום שנאמר (במדבר ה, ג) מזכר ועד נקבה תשלחו זכר ודאי נקבה ודאית ולא טומטום ואנדרוגינוס אמר עולא לא הא מני ר' אליעזר היא 
\commenta{השרץ ונעלם - או בנבלת שרץ ונעלם ממנו:}
דתנן רבי אליעזר אומר השרץ (ויקרא ה, ב) ונעלם ממנו על העלם שרץ הוא חייב ואינו חייב על העלם מקדש 
רבי עקיבא אומר ונעלם ממנו והוא טמא על העלם טומאה הוא חייב ואינו חייב על העלם מקדש 
\commenta{שרץ ונבלה - לפניו וידע שנגע באחד מהם ואינו יודע באיזה מהם נגע וכשבא למקדש שכח אותו ומשיצא נזכר:}
ואמרינן מאי בינייהו ואמר חזקיה שרץ ונבלה איכא בינייהו דרבי אליעזר סבר בעינן עד דידע אי בשרץ איטמי אי בנבילה איטמי ור' עקיבא סבר לא בעינן 
\commenta{ר' אליעזר - דאזכר שם שרץ סבר דבעינן עד דידע אם בשרץ איטמי כו':}
לאו אמר רבי אליעזר התם בעינן דידע אי בשרץ איטמי אי בנבלה איטמי הכא נמי בעינן דידע אי בלובן איטמי אי באודם איטמי 
אבל לרבי עקיבא דאמר משום טומאה מיחייב הכא נמי משום טומאה מיחייב 
ורב מאי שנא ביאת מקדש דלא דכתיב מזכר ועד נקבה תשלחו זכר ודאי נקבה ודאית ולא טומטום ואנדרוגינוס 
\commenta{הזב את זובו לזכר ולנקבה - ולדרוש נמי זכר ודאי נקבה ודאית מטמא בזיבה ולא טומטום:}
אי הכי תרומה נמי לא נשרוף דכתיב (ויקרא טו, לג) והזב את זובו לזכר ולנקבה זכר ודאי נקבה ודאית ולא טומטום ואנדרוגינוס 
\commenta{לזכר לרבות מצורע - שהוא מוקש לזב להיות מעינותיו אב הטומאה לטמא אדם וכלים כזב דכתיב ביה (ויקרא ט״ו:ח׳) וכי ירוק הזב וגו':}
ההוא מבעי ליה לכדרבי יצחק דאמר רבי יצחק לזכר לרבות את המצורע למעינותיו ולנקבה לרבות את המצורעת למעינותיה 
\commenta{פרט לכלי חרס - טמא שאין טעון שילוח מן העזרה:}
האי נמי מבעי ליה במי שיש לו טהרה במקוה פרט לכלי חרס דברי רבי יוסי 
אם כן נכתוב רחמנא אדם
\commenta{הוה אמינא כלי מתכות לא - טעון שילוח ולהכי שני בדיבוריה למדרש כל שיש לו דין הנוהג בזכר ובנקבה דהיינו טהרה במקוה:}
וכי תימא אי כתב רחמנא אדם הוה אמינא כלי מתכות לא מכל טמא לנפש נפקא זכר ונקבה למה לי לכדרב 
\commenta{עד נקבה - משמע כל מי שיש לו דין שהוא מזכר עד נקבה שיש להן טהרה:}
ואימא כוליה לכדרב הוא דאתא אם כן נכתוב זכר ונקבה מאי מזכר ועד נקבה עד כל דבר שיש לו טהרה במקוה 
\commenta{אי הכי - לעיל קאי. כיון דאמר רב לענין שילוח זכר ודאי ולא טומטום בשאר טומאות נמי כגון טומטום שנטמא במת או בשרץ ונכנס למקדש יפטר:}
אי הכי כי איטמי בשאר טומאות לא לישלחו אמר קרא מזכר מטומאה הפורשת מן הזכר 
וכל היכא דכתיב מזכר עד נקבה למעוטי טומטום ואנדרוגינוס הוא דאתא והא גבי ערכין דכתיב {ויקרא כז } הזכר
\commenta{תלמוד לומר הזכר - ה"א דריש:}
ותניא הזכר ולא טומטום ואנדרוגינוס יכול לא יהא בערך איש אבל יהא בערך אשה תלמוד לומר הזכר ואם נקבה זכר ודאי נקבה ודאית ולא טומטום ואנדרוגינוס 
\commenta{ההוא מיבעי ליה - למכתב זכר ונקבה:}
טעמא דכתיב הזכר ואם נקבה הא מזכר ונקבה לא ממעט ההוא מבעי ליה
\clearpage}

\newsection{דף כט}
\twocol{לחלק בין ערך איש לערך אשה
\commenta{ואפילו הראש עמהם - אמחותך קאי דהיכא דיצא מחותך אפי' יצא הראש אין חשוב ולד עד שיצא רובו:}
יצא מחותך או מסורס וכו' א"ר אלעזר אפילו הראש עמהן 
ור' יוחנן אמר לא שנו אלא שאין הראש עמהן אבל הראש עמהן הראש פוטר 
\commenta{אין הראש פוטר בנפלים - יצא ראש הנפל כגון יולדת תאומים לשבעה אחד נגמרה צורתו ואחד לא נגמרה צורתו ויצא ראש הנפל והחזירו ואח"כ יצא אחיו הוי בכור לכהן שאין ראש הנפל פוטר את הבא אחריו מן הבכורה ורבי אלעזר אית ליה דשמואל:}
לימא בדשמואל קמיפלגי דאמר שמואל אין הראש פוטר בנפלים 
בשלם דכולי עלמא לא פליגי כי פליגי במחותך דמר סבר בשלם הוא דקחשיב במחותך לא קחשיב ומר סבר במחותך נמי חשיב 
לישנא אחרינא טעמא דיצא מחותך או מסורס הא כתקנו הראש פוטר תרוייהו לית להו דשמואל דאמר שמואל אין הראש פוטר בנפלים 
\commenta{באנפי נפשיה - ולא אמתניתין. ואיכא בינייהו מאן דמתני לה אמתניתין משמע דאמחותך קא פליגי ואיכא למימר כדאמרן דבשלם אפי' ר' אלעזר מודה דלית ליה דשמואל ומאן דמתני לה באנפי נפשיה משמע דאשלם נמי פליגי ובדשמואל:}
איכא דמתני לה להא שמעתתא באפי נפשה א"ר אלעזר אין הראש כרוב אברים ורבי יוחנן אמר הראש כרוב אברים וקמיפלגי בדשמואל 
תנן יצא מחותך או מסורס משיצא רובו הרי הוא כילוד מדקאמר מסורס מכלל דמחותך כתקנו וקאמר משיצא רובו הרי זה כילוד קשיא לרבי יוחנן 
אמר לך רבי יוחנן אימא יצא מחותך ומסורס 
והא או קתני הכי קאמר יצא מחותך או שלם וזה וזה מסורס משיצא רובו הרי זה כילוד 
\commenta{כתקנו - ראש יצא תחלה וקאמר דבעינן רובא וקשיא לר' יוחנן:
}
אמר רב פפא כתנאי יצא מחותך או מסורס משיצא רובו הרי הוא כילוד רבי יוסי אומר משיצא כתקנו מאי קאמר 
\commenta{מחותך ומסורס - מחותך ודרך מרגלותיו:}
אמר רב פפא הכי קאמר יצא מחותך ומסורס משיצא רובו הרי הוא כילוד הא כתקנו הראש פוטר רבי יוסי אומר משיצא רובו כתקנו 
\commenta{מתקיף לה רב זביד - לתירוצא דמתרצינן אליבא דרבי יוסי:}
מתקיף לה רב זביד מכלל דבמסורס רובו נמי לא פוטר הא קי"ל דרובו ככולו 
\commenta{אלא אמר רב זביד - רבי יוסי אדיוקא דת"ק קאי דאמר הא כתקנו הראש פוטר ואע"ג דמחותך וקאמר ר' יוסי אין הראש פוטר אלא שיצא כתקנו וכשאר ולדות שיוצאין לחיים אבל מחותך אפילו כתקנו אין הראש פוטר:}
אלא אמר רב זביד הכי קאמר יצא מחותך ומסורס משיצא רובו הרי זה כילוד הא כתקנו הראש פוטר רבי יוסי אומר משיצא כתקנו לחיים 
\commenta{מי שיצא כתקנו לחיים - הוא דראש פוטר אבל מחותך אפילו כתקנו בעינן רובו:}
תניא נמי הכי יצא מחותך (או) מסורס משיצא רובו הרי זה כילוד הא כתקנו הראש פוטר ר' יוסי אומר משיצא כתקנו לחיים 
\commenta{צדעיו - טנפל"ש:}
ואיזהו כתקנו לחיים משיצא רוב ראשו ואיזהו רוב ראשו ר' יוסי אומר משיצאו צדעיו אבא חנן משום ר' יהושע אומר משיצא פדחתו וי"א משיראו קרני ראשו
\commenta{מתני' ואין ידוע מהו - אם זכר אם נקבה אבל יודעת היא שהוא ולד:}
{\large\emph{מתני׳}} המפלת ואין ידוע מהו תשב לזכר ולנקבה אין ידוע אם ולד היה אם לאו תשב לזכר ולנקבה ולנדה
\commenta{גמ' והפילה - ואינה יודעת מה אם נפל אם רוח שהרי נפל למים:}
{\large\emph{גמ׳}} א"ר יהושע בן לוי עברה נהר והפילה מביאה קרבן ונאכל הלך אחר רוב נשים ורוב נשים ולד מעליא ילדן 
\commenta{אמאי תשב לנדה - ניתיב לה מיהא ימי טוהר דזכר:}
תנן אין ידוע אם ולד היה תשב לזכר ולנקבה ולנדה אמאי תשב לנדה לימא הלך אחר רוב נשים ורוב נשים ולד מעליא ילדן 
מתני' בשלא הוחזקה עוברה וכי קאמר ריב"ל כשהוחזקה עוברה 
\commenta{שיצאה מלאה - כשהיתה מעוברת יצתה לאפר:}
ת"ש בהמה שיצאה מלאה ובאה ריקנית הבא אחריו בכור מספק 
\commenta{והאי פשוט הוא - ויאכל בלא מום:}
ואמאי הלך אחר רוב בהמות ורוב בהמות ולד מעליא ילדן והאי פשוט הוא 
\commenta{מטנפות - יום אחד קודם לידה כעין גלייר"א:}
אמר רבינא משום דאיכא למימר רוב בהמות יולדות דבר הפוטר מבכורה ומעוטן יולדות דבר שאינו פוטר מבכורה וכל היולדות מטנפות וזו הואיל ולא טנפה אתרע לה רובא 
אי כל היולדות מטנפות הא מדלא מטנפה בכור מעליא הוא אלא אימא רוב יולדות מטנפות וזו הואיל ולא טנפה אתרע לה רובא 
\commenta{מתיב רבי יוסי - אדר' יהושע בן לוי:}
כי אתא רבין אמר מתיב רבי יוסי ברבי חנינא טועה ולא ידענא מאי תיובתא מאי היא דתניא
אשה שיצתה מלאה ובאה ריקנית והביאה לפנינו שלשה שבועין טהורין ועשרה שבועות אחד טמא ואחד טהור
\commenta{משמשת לאור שלשים וחמש - שהוא ליל סוף שבוע חמישי והוא ליל כניסת שלשים וחמש ויציאת יום ל"ד ושוב אינה משמשת. וטעמא דכולן מפרש לקמן ואמאי נקט הכי י"ג שבועים:}
משמשת לאור שלשים וחמש ומטבילין אותה תשעים וחמש טבילות דברי ב"ש וב"ה אומרים שלשים וחמש רבי יוסי בר' יהודה אומר דיה לטבילה שתהא באחרונה 
\commenta{בשלמא שבוע ראשון כו' - השתא מפרש תיובתא:}
בשלמא שבוע ראשון לא משמשת אימר יולדת זכר היא שבוע שני אימר יולדת נקבה היא
\commenta{בזוב - שמא מתוך י"א יום דימי זיבה ילדה וראתה ג' רצופין ובלא צער דהוה שופי סמוך ללידה ובעיא שבעה נקיים לזיבתה. ואע"ג דקא יתבא י"ד נקיים הואיל ואיכא לספוקינהו בימי לידה לא סלקא לה דקסבר ימי לידה שאינה רואה בהן אין עולין לה לספירת זיבתה:}
שבוע שלישי אימר יולדת נקבה בזוב היא
\commenta{אלא שבוע רביעי - אי אמרינן רוב נשים ולד מעליא ילדן האי ולד הוא שהרי הוחזקה עוברה:}
אלא שבוע רביעי אע"ג דקא חזיא דם תשמש דהא דם טהור הוא לאו משום דלא אזלינן בתר רובא 
\commenta{אימר הרחיקה לידתה - האי דלא יהבינן לה ימי טוהר לאו משום ספק דלאו ולד הוא אלא אימר הרי ימים רבים קודם שבאת לפנינו ילדה וכלו לה ימי טהרתה אבל לענין קרבן מביאה קרבן ונאכל בין קרבה ובין הרחיקה בת קרבן היא מאחר דקפדינן ארובא:
}
אלא מאי לא ידענא מאי תיובתא אימר הרחיקה לידתה 
\commenta{שבוע חמישי דטהור הוא תשמש כו' - ואמאי לא משמשת עד ליל שביעי:}
הך שבוע חמישי דטהור הוא תשמש 
\commenta{ומשנינן שבוע רביעי - שראתה בכל יום כל יומא איכא לספוקי לסוף לידה סוף ימי טוהר כלומר אתמול כלו ימי טוהר והיום היא תחלת נדה הילכך יום כ"ח שהוא סוף שבוע רביעי מספקינן לה בתחלת נדה ובעיא לממני משבוע חמישי ששה ימים להשלים שבעת ימי נדות ובליל שביעי טובלת ומשמשת שהוא אור לשלשים וחמש משבאת לפנינו וליל שמיני שהוא תחלת שבוע ששי נמי לא משמשת שהשבוע ששי טמא הוא דמוקמינן לה לקמן שהיא רואה מבערב בכל ימי ראייתה מכאן ואילך לא משמשת לעולם דדילמא שבוע רביעי תחלת נדה היא והששי בימי זיבה והרי היא זבה ושבוע שביעי שהוא טהור תספור לנקיים ואם לא שרואה מבערב היתה משמשת ליל תחלת שבוע שמינית ואע"ג דיכולה לטבול ביום שביעי שמקצת היום ככולו לענין ספירה אפ"ה לא משמשת ביום דמוקמינן לה כר"ש דאסר וכן לעולם י"ל בכל שבועים טמאים שלה זהו בימי זיבה וצריכה ז' נקיים וכל שבוע טהור תספור לנקיים אבל שבוע רביעית ממ"נ אין אתה יכול לספקו בזיבה שאין זיבה אלא לאחר נדות והרי זו ישבה ג' שבועים טהורים והרביעי הוא דראתה הילכך בתחלת נדה הוא דאיכא לספוקי ולא בעיא למיתב בשבוע חמישי אלא ששה ימים משום ספק יום אחרון של רביעי דמספקינן לה בתחלת נדה:}
הך שבוע רביעי כל יומא ויומא מספקין בסוף לידה ובתחלת נדה ועשרין ותמניא גופיה אימר תחלת נדה היא ובעיא למיתב שבעה לנדתה 
\commenta{בעשרים וחד - שהוא סוף שבוע שלישית תשמש דהא אפי' קרבה לידתה לילד בו ביום שבאת לפנינו כבר כלו ימי טומאה והיום שביעי לנקיים של ספירה וקי"ל דטבילת זב וזבה ביום בפ"ק דיומא (דף ו.) דמקצת היום ככולו הילכך ביום כ"א תשמש:}
בעשרים וחד תשמש 
\commenta{ר"ש היא דאמר - בת"כ אסור לעשות כן לשמש זבה ביום טבילתה:}
רבי שמעון היא דאמר אסור לעשות כן שמא תבא לידי ספק לאורתא תשמש כשראתה בערב 
\commenta{בלילותא - בכל לילה ולילה דבכל לילה מספקינן לה [שמא] השתא סוף שבעה ללידתה הוא וטבילה בזמנה מצוה:
}
ומטבילין אותה תשעים וחמש טבילות שבוע קמא מטבילין אותה בלילותא אימר יולדת זכר היא 
\commenta{בלילותא אימור יולדת נקבה הואי - וכל לילה איכא לספוקי האידנא סוף שבועים וליל אחרון מספקינן שמא ביום שבאתה ילדה והשתא כלין:}
שבוע שני מטבילין אותה בלילותא אימר יולדת נקבה היא ביממא אימר יולדת זכר בזוב היא 
\commenta{שבוע שלישי מטבלינן לה ביממא אימר יולדת נקבה בזוב היא - וכל יומא איכא לספוקי השתא כלין ימי ספירה יום ראשון דלמא ששה ימים קודם שבאת לפנינו ילדה ושבועים הרי עשרים יום והיום הוי כ"א שסוף ספירה הוא ולמחר אמרי' ה' ימים קודם ביאתה ילדה והיום כלים ויום אחרון אימר ביום שבאתה ילדה והיום כלים:}
שבוע שלישי מטבילין לה ביממא אימר יולדת נקבה בזוב היא
\commenta{בלילותא - דשבוע שלישי מטבילין לה ואף ע"ג דליכא לספוקי בשבוע שלישי בסוף לידת נקבה:}
בלילותא ב"ש לטעמייהו דאמרי טבולת יום ארוך בעי טבילה
\clearpage}

\newsection{דף ל}
\twocol{מכדי ימי טהרה כמה הוו שתין ושיתא דל שבוע ג' דאטבלינן לה פשו להו שתין נכי חדא שתין נכי חדא ותלתין וה' תשעין וד' הויין תשעין וחמש מאי עבידתייהו 
\commenta{טבילה יתירתא - עד השתא הוה ס"ד דבאת לפנינו ביום כגון ביום ר"ח ניסן ואטבלוה לאורתא ומשכו פ' דילה ל' מניסן וכ"ט דאייר וכ"א דסיון כלים פ' יום הרי פ' טבילות בלילות וי"ד בימים של ב' שבועות הראשונות הרי צ"ד והשתא מוקמינן שבאת לפנינו בין השמשות לילה שהוא ר"ח ואטבלה בההיא ליליא משום ספק סוף טומאה ומשכי נמי פ' שלה עד כ"א בסיון כאילו באתה ביום ראש חדש דאיכא לספוקי דילמא באותו בין השמשות ילדה והוא מר"ח וההיא טבילה דליל ראש חדש איתוספא לה נמצאו לה ח' טבילות בשבוע ראשונה וכ"ח דב' וג' הרי ל"ו. וקשיא לי אמאי נקט בין השמשות ה"ל לאוקמא שבאת לפנינו בלילה ומסתברא דלהכי נקט בין השמשות דדק בלישנא דמתני' דנקט ג' שבועים טהורים הילכך כי איכא בין השמשות איכא למיקרי יום זה טהור דשמא בין השמשות יממא הוא ומיום שעבר הוא והיום הזה כולו טהור ומיהו טבילותיה משכי עד שמונים לבד מהך משום דבין השמשות ספק ליליא:}
אמר רב ירמיה מדפתי כגון שבאת לפנינו בין השמשות דיהבינן לה טבילה יתירתא 
\commenta{ל"ה - דקאמרי ב"ה מאי עבידתייהו. ומשני עשרין ותמניא הויין:}
ולב"ה דאמרי טבולת יום ארוך לא בעי טבילה ל"ה מאי עבידתייהו 
\commenta{כדאמר - י"ד בימים וי"ד בלילות וכולן משום ספק סוף ימי טומאה. והוא הדין דלב"ה הוו להו עשרין ותשע דכיון דאוקמינן שבאה בין השמשות איכא ט"ו טבילות בלילה משום ספק בין השמשות והאי דלא מותבינן ליה לקמיה משום דאותבינן אחריתי ומשנינן חד בשבוע לא קמיירי והך נמי חדא בשבוע היא:}
עשרים ותמניא כדאמרן הך שבוע ה' מטבלינן כל ליליא וליליא אימר סוף נדה היא 
\commenta{בתמניא ופלגא סגיא - לבד הג' שבועות דטהורה דהא כלין ימי טבילה בסוף שמונים והוא באמצע שבוע תשיעי של י' שבועים:}
י' שבועין למה לי בתמניא ופלגא סגי 
\commenta{שבוע - תשיעי טמא הוא:}
איידי דתנא פלגא דשבוע מסיק ליה ואיידי דתנא שבוע טמא תנא נמי שבוע טהור 
\commenta{טבילת זבה - דכל שבוע טמא איכא לספוקי בזיבה לבד משבוע רביעי שאי אתה יכול לספקו בכך כדפרישית לעיל שאין זבה אלא לאחר נדות וזו לא ראתה כלום עד הרביעי וכל שבוע טהור איכא לספוקי בימי ספירת זיבתה ובעיא למטבל ביום ז' של כל שבוע טהור:}
והאיכא טבילת זבה 
דלפני תשמיש קחשיב דלאחר תשמיש לא קחשיב 
ולב"ש דחשיב דלאחר תשמיש ניחשוב נמי טבילת זבה בלידה קמיירי בזיבה לא קמיירי 
והאיכא יולדת בזוב יולדת בזוב קחשיב זיבה גרידתא לא קחשיב 
שבועתא קמא דאתיא לקמן ליטבלה ביומא דילמא כל יומא ויומא שלימו לה ספורים דידה 
\commenta{ר"ע - דפרק אחרון (לקמן נדה דף סט.):}
הא מני ר"ע היא דאמר בעינן ספורים בפנינו 
\commenta{סוף שבוע ראשון - ביום השביעי ליטבלה ביום דילמא בין השמשות יום הוא ויום ראשון שבאתה לפנינו טהור ואיכא ז' ספורים לפנינו:}
סוף שבוע קמא ליטבלה חד בשבוע לא קמיירי 
יומא קמא דאתיא לקמן ליטבלה דילמא שומרת יום כנגד יום היא בזבה גדולה קמיירי בזבה קטנה לא קמיירי 
ש"מ תלת ש"מ ר"ע היא דאמר בעינן ספורים בפנינו 
וש"מ ר"ש היא דאמר אבל אמרו חכמים אסור לעשות כן שמא תבא לידי ספק 
וש"מ טבילה בזמנה מצוה ורבי יוסי בר' יהודה אומר דיה לטבילה באחרונה ולא אמרינן טבילה בזמנה מצוה
\commenta{מתני' המפלת - שליא:}
{\large\emph{מתני׳}} המפלת ליום מ' אינה חוששת לולד ליום מ"א תשב לזכר ולנקבה ולנדה
\commenta{תשב לזכר - ז' ימי טומאה ואפילו בלידה יבשתא אבל שבועים דנקבה ליכא לספוקי כדקתני טעמא שיצירת נקבה לפ"א:}
רבי ישמעאל אומר יום מ"א תשב לזכר ולנדה יום פ"א תשב לזכר ולנקבה ולנדה שהזכר נגמר למ"א והנקבה לפ"א וחכ"א אחד בריית הזכר ואחד בריית הנקבה זה וזה מ"א
{\large\emph{גמ׳}} למה הוזכר זכר
\commenta{גמ' הא קתני נקבה - דה"ל שבועים ויש בכלל מאתים מנה:}
אי לימי טומאה הא קתני נקבה ואי לימי טהרה
הא קתני נדה 
שאם תראה יום ל"ד ותחזור ותראה יום מ' ואחד תהא מקולקלת עד מ"ח 
\commenta{וכן לענין נקבה - כלומר וספקא דנקבה נמי מקלקלא לה האי קלקולא:}
וכן לענין נקבה שאם תראה יום ע"ד ותחזור ותראה יום פ"א תהא מקולקלת עד פ"ח
\commenta{טימא וטיהר בזכר - מ' יום בין שתיהן ז' לטומאה ול"ג יום לימי טוהר:}
רבי ישמעאל אומר יום מ"א תשב לזכר ולנדה כו' תניא רבי ישמעאל אומר טימא וטיהר בזכר וטימא וטיהר בנקבה
מה כשטימא וטיהר בזכר יצירתו כיוצא בו אף כשטימא וטיהר בנקבה יצירתה כיוצא בה אמרו לו אין למדין יצירה מטומאה 
\commenta{נתחייבו הריגה - ומתוך שעומדות ליהרג ניסו בהם ויחדום לביאה וקרעו שתיהם לסוף מ' ונמצא אחד זכר ואחד נקבה:}
אמרו לו לר' ישמעאל מעשה בקליאופטרא מלכת אלכסנדרוס שנתחייבו שפחותיה הריגה למלכות ובדקן ומצאן זה וזה למ"א אמר להן אני מביא לכם ראייה מן התורה ואתם מביאין לי ראייה מן השוטים 
מאי ראיה מן התורה אילימא טימא וטיהר בזכר וטימא וטיהר בנקבה כו' הא קאמרי ליה אין דנין יצירה מטומאה 
\commenta{תלד - דאם נקבה תלד (ויקרא י״ב:ה׳) קרא יתירא הוא דמצי למכתב ואם נקבה וטמאה שבועים דהא כתיב לעיל וילדה זכר:}
אמר קרא תלד הוסיף לה הכתוב לידה אחרת בנקבה 
\commenta{מקמי זכר - ואף ע"ג שיחדום לבעול ביום אחד שמא מעוברת היתה קודם לכן:}
ומאי ראיה מן השוטים אימר נקבה קדים ואיעבור ארבעין יומין קמי זכר 
\commenta{סמא דנפצא - סם שמפלת בו ומנפצא כל זרע שבמעים קודם לכן:}
ורבנן סמא דנפצא אשקינהו ור' ישמעאל איכא גופא דלא מקבל סמא
אמר להם ר' ישמעאל מעשה בקלפטרא מלכת יוונית שנתחייבו שפחותיה הריגה למלכות ובדקן ומצאן זכר לארבעים ואחד ונקבה לפ"א אמרו לו אין מביאין ראיה מן השוטים 
מאי טעמא הך דנקבה אייתרה ארבעין יומין והדר איעבר 
\commenta{לשומר מסרינהו - שלא בא אדם עליהן אלא אותו היום:}
ורבי ישמעאל לשומר מסרינהו ורבנן אין אפוטרופוס לעריות אימא שומר גופיה בא עליה 
\commenta{ודילמא אי קרעוה כו' - לרבי ישמעאל פריך נהי נמי דלשומר מסרינהו אפ"ה מנלן שלא נוצרה נקבה עד פ"א דילמא אי קרעוה להך ליום מ"א הוה משתכחא:}
ודילמא אי קרעוהו להך דנקבה בארבעין וחד הוה משתכחא כזכר אמר אביי בסימניהון שוין
\commenta{היינו ת"ק - דתנא לעיל ליום מ"א תשב לזכר ולנקבה ולנדה אלמא בריית נקבה ליום מ"א:}
וחכ"א אחד בריית זכר ואחד בריית נקבה וכו' חכמים היינו ת"ק 
\commenta{וכי תימא - רישא להכי תנא ליה לאשמועינן סתמא דמתניתין כרבנן ונשמע מינה דהלכתא כוותייהו ולא נהירא לי גירסא ופירושה דה"ל סתם ואח"כ מחלוקת ואין הלכה כסתם. ונ"ל דהכי גרס וכי תימא למסתמא כרבנן ויחיד ורבים הלכה כרבים פשיטא מהו דתימא כו'. והכי פירושה וכי תימא תנא הך סיפא לאשמעינן דסתמא דלעיל סתמא דרבנן היא ויחיד ורבים הלכה כרבים פשיטא דסתמא דרבים היא הואיל וסתמא תנייה רבי ולא אשכחן יחידאה דאמר הכי דנימא פלוני היא:}
וכי תימא למסתמא רישא כרבנן ויחיד ורבים הלכה כרבים פשיטא 
\commenta{מהו דתימא כו' - קמשמע לן מדמהדר רבי ושנאה בלשון חכמים שמע מינה דהלכה כסתם ראשון ואף על פי שמחלוקת ר' ישמעאל בצדו:}
מהו דתימא מסתברא טעמא דרבי ישמעאל דקמסייע ליה קראי קמ"ל
\commenta{פנקס - לוחין שכותבין בהן הגלחים:}
דרש רבי שמלאי למה הולד דומה במעי אמו לפנקס שמקופל ומונח ידיו על שתי צדעיו שתי אציליו על ב' ארכובותיו וב' עקביו על ב' עגבותיו וראשו מונח לו בין ברכיו ופיו סתום וטבורו פתוח ואוכל ממה שאמו אוכלת ושותה ממה שאמו שותה ואינו מוציא רעי שמא יהרוג את אמו וכיון שיצא לאויר העולם נפתח הסתום ונסתם הפתוח שאלמלא כן אינו יכול לחיות אפילו שעה אחת 
ונר דלוק לו על ראשו וצופה ומביט מסוף העולם ועד סופו שנאמר (איוב כט, ג) בהלו נרו עלי ראשי לאורו אלך חשך ואל תתמה שהרי אדם ישן כאן ורואה חלום באספמיא 
ואין לך ימים שאדם שרוי בטובה יותר מאותן הימים שנאמר (איוב כט, ב) מי יתנני כירחי קדם כימי אלוה ישמרני ואיזהו ימים שיש בהם ירחים ואין בהם שנים הוי אומר אלו ירחי לידה 
\commenta{ת"ש בסוד אלוה עלי אהלי - והא אוקמיה בירחי לידה מדכתיב כירחי קדם:}
ומלמדין אותו כל התורה כולה שנאמר (משלי ד ד) ויורני ויאמר לי יתמך דברי לבך שמור מצותי וחיה ואומר (איוב כט, ד) בסוד אלוה עלי אהלי 
מאי ואומר וכי תימא נביא הוא דקאמר ת"ש בסוד אלוה עלי אהלי 
וכיון שבא לאויר העולם בא מלאך וסטרו על פיו ומשכחו כל התורה כולה שנאמר (בראשית ד, ז) לפתח חטאת רובץ 
\commenta{ולא נשבע למרמה - שקיים שבועתו לשמור את התורה דאי בשבועה בעלמא אטו משום דלא נשבע למרמה חשיב להו בעולין בהר ה':}
ואינו יוצא משם עד שמשביעין אותו שנאמר (ישעיהו מה, כג) כי לי תכרע כל ברך תשבע כל לשון כי לי תכרע כל ברך זה יום המיתה שנאמר (תהלים כב, ל) לפניו יכרעו כל יורדי עפר תשבע כל לשון זה יום הלידה שנאמר (תהלים כד, ד) נקי כפים ובר לבב אשר לא נשא לשוא נפשו ולא נשבע למרמה
ומה היא השבועה שמשביעין אותו תהי צדיק ואל תהי רשע ואפילו כל העולם כולו אומרים לך צדיק אתה היה בעיניך כרשע והוי יודע שהקב"ה טהור ומשרתיו טהורים ונשמה שנתן בך טהורה היא אם אתה משמרה בטהרה מוטב ואם לאו הריני נוטלה ממך 
תנא דבי ר' ישמעאל משל לכהן שמסר תרומה לעם הארץ ואמר לו אם אתה משמרה בטהרה מוטב ואם לאו הריני שורפה לפניך 
א"ר אלעזר
\clearpage}

\newsection{דף לא}
\twocol{מאי קרא (תהלים עא, ו) ממעי אמי אתה גוזי מאי משמע דהאי גוזי לישנא דאשתבועי הוא דכתיב (ירמיהו ז, כט) גזי נזרך והשליכי 
\commenta{ למה ולד דומה:}
ואמר רבי אלעזר למה ולד דומה במעי אמו לאגוז מונח בספל של מים אדם נותן אצבעו עליו שוקע לכאן ולכאן 
תנו רבנן שלשה חדשים הראשונים ולד דר במדור התחתון אמצעיים ולד דר במדור האמצעי אחרונים ולד דר במדור העליון וכיון שהגיע זמנו לצאת מתהפך ויוצא וזהו חבלי אשה 
והיינו דתנן חבלי של נקבה מרובין משל זכר 
\commenta{רקמתי - היינו יצירה ראשונה וכתיב בתחתיות דהיינו מדור תחתון:}
ואמר רבי אלעזר מאי קרא (תהלים קלט, טו) אשר עשיתי בסתר רקמתי בתחתיות ארץ דרתי לא נאמר אלא רקמתי
\commenta{חבלי - צערה:}
מאי שנא חבלי נקבה מרובין משל זכר זה בא כדרך תשמישו וזה בא כדרך תשמישו זו הופכת פניה וזה אין הופך פניו 
\commenta{קשה לאשה וקשה לולד - מפני שדר במדור התחתון. קשה לאשה לא ידענא למאי: זרע מלבן את הולד מגיעוליו:}
תנו רבנן שלשה חדשים הראשונים תשמיש קשה לאשה וגם קשה לולד אמצעיים קשה לאשה ויפה לולד אחרונים יפה לאשה ויפה לולד שמתוך כך נמצא הולד מלובן ומזורז 
\commenta{ליום תשעים - לשליש ימים הוי חיותו:}
תנא המשמש מטתו ליום תשעים כאילו שופך דמים מנא ידע אלא אמר אביי משמש והולך (תהלים קטז, ו) ושומר פתאים ה' 
\commenta{קלסתר - זיו:}
תנו רבנן שלשה שותפין יש באדם הקב"ה ואביו ואמו אביו מזריע הלובן שממנו עצמות וגידים וצפרנים ומוח שבראשו ולובן שבעין אמו מזרעת אודם שממנו עור ובשר ושערות ושחור שבעין והקב"ה נותן בו רוח ונשמה וקלסתר פנים וראיית העין ושמיעת האוזן ודבור פה והלוך רגלים ובינה והשכל 
\commenta{פוץ מילחא - השלך המלח מן הבשר ושוב אינו ראוי אלא לכלבים כך הנשמה היא מלח לגוף לקיימו כיון שהלכה אז מסריח הגוף:}
וכיון שהגיע זמנו להפטר מן העולם הקב"ה נוטל חלקו וחלק אביו ואמו מניח לפניהם אמר רב פפא היינו דאמרי אינשי פוץ מלחא ושדי בשרא לכלבא 
דרש רב חיננא בר פפא מאי דכתיב (איוב ט, י) עושה גדולות עד אין חקר ונפלאות עד אין מספר בא וראה שלא כמדת הקב"ה מדת בשר ודם מדת בשר ודם נותן חפץ בחמת צרורה ופיה למעלה ספק משתמר ספק אין משתמר ואילו הקב"ה צר העובר במעי אשה פתוחה ופיה למטה ומשתמר 
\commenta{עולה למעלה - כדתני לעיל אחרונים דר במדור העליון:}
דבר אחר אדם נותן חפציו לכף מאזנים כל זמן שמכביד יורד למטה ואילו הקב"ה כל זמן שמכביד הולד עולה למעלה 
\commenta{כולין עולין למין אחד - זרע האב והאם נעשין בריה אחת:}
דרש רבי יוסי הגלילי מאי דכתיב {תהילים קל״ט:י״ד } אודך (ה') על כי נוראות נפליתי נפלאים מעשיך ונפשי יודעת מאד בא וראה שלא כמדת הקב"ה מדת בשר ודם מדת בשר ודם אדם נותן זרעונים בערוגה כל אחת ואחת עולה במינו ואילו הקב"ה צר העובר במעי אשה וכולם עולין למין אחד 
\commenta{צבע - כמה סממנין צריך לצבע שחור קליפת עץ וטחינת ריחיים של נפחים וכמה דברים וכולן נעשין שחור ואינו יכול לצובעו ביורה אחת משנים ושלשה גוונים בשנים ושלשה מקומות:}
דבר אחר צבע נותן סמנין ליורה כולן עולין לצבע אחד ואילו הקב"ה צר העובר במעי אשה כל אחת ואחת עולה למינו 
\commenta{כי אנפת בי - מפני שכעסת עלי אני מודה לך שלטובתי היה:}
דרש רב יוסף מאי דכתיב (ישעיהו יב, א) אודך ה' כי אנפת בי ישוב אפך ותנחמני במה הכתוב מדבר 
\commenta{ישב לו קוץ - ברגלו לאחד מהם ולא יכול לצאת:}
בשני בני אדם שיצאו לסחורה ישב לו קוץ לאחד מהן התחיל מחרף ומגדף לימים שמע שטבעה ספינתו של חבירו בים התחיל מודה ומשבח לכך נאמר ישוב אפך ותנחמני 
\commenta{לבדו - הוא לבדו יודע שהוא נס אבל בעל הנס אינו מכירו:}
והיינו דאמר רבי אלעזר מאי דכתיב (תהלים עב, יח) עושה נפלאות (גדולות) לבדו וברוך שם כבודו לעולם אפילו בעל הנס אינו מכיר בנסו 
\commenta{ארחי - תשמיש כמו דרך גבר בעלמה (משלי ל׳:י״ט):}
דריש רבי חנינא בר פפא מאי דכתיב (תהלים קלט, ג) ארחי ורבעי זרית וכל דרכי הסכנת מלמד שלא נוצר אדם מן כל הטפה אלא מן הברור שבה תנא דבי רבי ישמעאל משל לאדם שזורה בבית הגרנות נוטל את האוכל ומניח את הפסולת 
\commenta{ותזרני - חסר א' לשון מזרה:}
כדרבי אבהו דרבי אבהו רמי כתיב (שמואל ב כב, מ) ותזרני חיל וכתיב (תהלים יח, לג) האל המאזרני חיל אמר דוד לפני הקב"ה רבש"ע זיריתני וזרזתני 
\commenta{רביעיותיהם - עונת תשמישן:}
דרש רבי אבהו מאי דכתיב (במדבר כג, י) מי מנה עפר יעקב ומספר את רובע ישראל מלמד שהקב"ה יושב וסופר את רביעיותיהם של ישראל מתי תבא טיפה שהצדיק נוצר הימנה 
ועל דבר זה נסמית עינו של בלעם הרשע אמר מי שהוא טהור וקדוש ומשרתיו טהורים וקדושים יציץ בדבר זה מיד נסמית עינו דכתיב (במדבר כד, ג) נאם הגבר שתום העין 
\commenta{הוא - קודשא בריך הוא סייעה שנטה חמורו של יעקב לאהל לאה ואותו הלילה של האחרות היה:}
והיינו דאמר רבי יוחנן מאי דכתיב (בראשית ל, טז) וישכב עמה בלילה הוא מלמד שהקב"ה סייע באותו מעשה שנאמר (בראשית מט, יד) יששכר חמור גרם חמור גרם לו ליששכר 
אמר רבי יצחק אמר רבי אמי אשה מזרעת תחילה יולדת זכר איש מזריע תחילה יולדת נקבה שנאמר (ויקרא יג, כט) אשה כי תזריע וילדה זכר 
\commenta{ולא פירשו - את הדבר מנלן:}
תנו רבנן בראשונה היו אומרים אשה מזרעת תחילה יולדת זכר איש מזריע תחלה יולדת נקבה ולא פירשו חכמים את הדבר עד שבא רבי צדוק ופירשו (בראשית מו, טו) אלה בני לאה אשר ילדה ליעקב בפדן ארם ואת דינה בתו תלה הזכרים בנקבות ונקבות בזכרים 
(דברי הימים א ח, מ) ויהיו בני אולם אנשים גבורי חיל דורכי קשת ומרבים בנים ובני בנים וכי בידו של אדם להרבות בנים ובני בנים אלא מתוך
שמשהין עצמן בבטן כדי שיזריעו נשותיהן תחלה שיהו בניהם זכרים מעלה עליהן הכתוב כאילו הם מרבים בנים ובני בנים והיינו דאמר רב קטינא יכולני לעשות כל בני זכרים אמר רבא הרוצה לעשות כל בניו זכרים יבעול וישנה 
\commenta{בעון - על ידי דם נדה שהוא סימן להריון כדקיימא לן (כתובות דף י:) כל אשה שדמיה מרובין בניה מרובין והוא מביאו לידי עון וקדייק ר' יצחק בעון שסמוך לעון נדות שהוא מוזהר לפרוש הימנה:}
ואמר רבי יצחק אמר רבי אמי אין אשה מתעברת אלא סמוך לוסתה שנאמר (תהלים נא, ז) הן בעון חוללתי 
\commenta{ובחטא - לשון טבילה וטהרה כדמפרש. וה"ק על ידי דם נדות שהוא מביא לידי עון חוללתי שהוא גרם לי שנבראתי וכשנתחטאת אמי ממנו יחמתני:}
ורבי יוחנן אמר סמוך לטבילה שנאמר (תהלים נא, ז) ובחטא יחמתני אמי 
מאי משמע דהאי חטא לישנא דדכויי הוא דכתיב (ויקרא יד, מט) וחטא את הבית ומתרגמינן וידכי ית ביתא ואי בעית אימא מהכא (תהלים נא, ט) תחטאני באזוב ואטהר 
\commenta{כר - מנחה דהיינו סימן שלום:}
ואמר רבי יצחק אמר רבי אמי כיון שבא זכר בעולם בא שלום בעולם שנאמר (ישעיהו טז, א) שלחו כר מושל ארץ זכר זה כר 
ואמר ר' יצחק דבי רבי אמי בא זכר בעולם בא ככרו בידו זכר זה כר דכתיב (מלכים ב ו, כג) ויכרה להם כירה גדולה 
\commenta{נקיה - חסרה כמו (ב"ק דף מא.) יצא פלוני נקי מנכסיו:}
נקבה אין עמה כלום נקבה נקייה באה עד דאמרה מזוני לא יהבי לה דכתיב (בראשית ל, כח) נקבה שכרך עלי ואתנה 
\commenta{לפיכך תביא קרבן - שבועת ביטוי חטאת:}
שאלו תלמידיו את רבי שמעון בן יוחי מפני מה אמרה תורה יולדת מביאה קרבן אמר להן בשעה שכורעת לילד קופצת ונשבעת שלא תזקק לבעלה לפיכך אמרה תורה תביא קרבן 
\commenta{מזידה היא - ואין כאן קרבן דהא ונעלם כתיב בשבועת ביטוי:}
מתקיף לה רב יוסף והא מזידה היא ובחרטה תליא מילתא ועוד קרבן שבועה בעי איתויי 
\commenta{מתחרטת לשבעה - ומתאוה לתשמיש בעלה כדי שתחזור ותתעבר זכר:}
ומפני מה אמרה תורה זכר לשבעה ונקבה לארבעה עשר זכר שהכל שמחים בו מתחרטת לשבעה נקבה שהכל עצבים בה מתחרטת לארבעה עשר 
\commenta{מפני מה מילה לשמונה - ולא לשבעה:}
ומפני מה אמרה תורה מילה לשמונה שלא יהו כולם שמחים ואביו ואמו עצבים 
\commenta{נדה שבעה - לאו ביולדת קאי:}
תניא היה ר"מ אומר מפני מה אמרה תורה נדה לשבעה מפני שרגיל בה וקץ בה אמרה תורה תהא טמאה שבעה ימים כדי שתהא חביבה על בעלה כשעת כניסתה לחופה 
\commenta{מחזר על האשה - אדם פנוי מבקש ומחזר עד שנושא:}
שאלו תלמידיו את רבי דוסתאי ברבי ינאי מפני מה איש מחזר על אשה ואין אשה מחזרת על איש משל לאדם שאבד לו אבידה מי מחזר על מי בעל אבידה מחזיר על אבידתו 
\commenta{פניו למטה - בשעת תשמיש:}
ומפני מה איש פניו למטה ואשה פניה למעלה כלפי האיש זה ממקום שנברא וזו ממקום שנבראת 
\commenta{מקבל פיוס - נוח לרצות:}
ומפני מה האיש מקבל פיוס ואין אשה מקבלת פיוס זה ממקום שנברא וזו ממקום שנבראת 
\commenta{עצם - כשמכין בו קולו נשמע אבל קרקע כשמכין בו אין קולו נשמע:}
מפני מה אשה קולה ערב ואין איש קולו ערב זה ממקום שנברא וזו ממקום שנבראת שנאמר {שיר השירים ב } כי קולך ערב ומראך נאוה
\par \par {\large\emph{הדרן עלך המפלת חתיכה}}\par \par 
\commenta{מתני' בנות כותים נדות מעריסתן - מקטנותן. ובגמרא מפרש טעמא:}

\commenta{והן יושבות על כל דם ודם - כלומר להכי הוו בועלי נדות משום דכל דם שרואות בין אדום בין ירוק יושבות עליו ימי נדות וזו תקלה היא להם שאם רואה דם ירוק היום מתחלת למנות ששה והוא אם תראה באותם ימי נדות דם אדום אינה מונה אלא מיום ראייה ראשונה ואותו דם ירוק טהור היה ומראייה שניה היא צריכה למנות:}
מתני׳ {\large\emph{בנות}} כותים נדות מעריסתן והכותים מטמאים משכב תחתון כעליון מפני שהן בועלי נדות
\commenta{אין חייבין עליה - הלובש או מתכסה באותן בגדים ונכנס למקדש פטור מקרבן או אם נגעו אותן בגדים של משכב בתרומה תולין דלא ידעינן אי נדה היא אי לא והא דאמר דמסתמא טמאות גזירה דרבנן היא:}
והן יושבות על כל דם ודם 
\commenta{גמ' אפילו דידן נמי - אי חזיא בעריסתן טמאה דתניא לקמן (נדה דף לב.) בת יום אחד לנדה מנין תלמוד לומר ואשה:}
ואין חייבין עליהן על ביאת מקדש ואין שורפין עליהם את התרומה מפני שטומאתן ספק 
\commenta{דאיכא מיעוטא דחזיין - בקטנותן:}
{\large\emph{גמ׳}} ה"ד אי דקא חזיין אפילו דידן נמי ואי דלא קחזיין דידהו נמי לא
אמר רבא בריה דרב אחא בר רב הונא אמר רב ששת הכא במאי עסקינן בסתמא דכיון דאיכא מיעוטא דחזיין חיישינן ומאן תנא דחייש למיעוטא}

\newchap{פרק \hebrewnumeral{4} בנות כותים}
\twocol{\clearpage}

\newsection{דף לב}
\twocol{
\commenta{איש כתוב בפרשה - (דברים כ״ה:ז׳) ואם לא יחפוץ האיש איש פוטרה בחליצה ולא קטן. אלא מה טעם אין מיבמין ויגדלו יחד:}
ר"מ היא דתניא קטן וקטנה לא חולצין ולא מיבמין דברי ר"מ 
\commenta{סריס - לאו בר הקמת שם הוא ואינו מצווה ליבם וכן אילונית וכיון דמייבם בלא מצוה נושא אשת אחיו בחנם וערוה היא ובכרת:}
אמרו לו לר"מ יפה אמרת שאין חולצין איש כתוב בפרשה ומקשינן אשה לאיש ומה טעם אין מיבמין 
אמר להן קטן שמא ימצא סריס קטנה שמא תמצא אילונית ונמצאו פוגעין בערוה שלא במקום מצוה 
ורבנן זיל בתר רובא דקטנים ורוב קטנים לאו סריסים נינהו זיל בתר רובא דקטנות ורוב קטנות לאו אילונית נינהו 
\commenta{בעין בול - מקום:}
אימר דשמעת ליה לר"מ מיעוטא דשכיח אבל מיעוטא דלא שכיח מי שמעת ליה 
\commenta{משום תרומת א"י - דמיפסלא בנגיעה:}
הא נמי מיעוטא דשכיח הוא דתניא א"ר יוסי מעשה בעין בול והטבילוה קודם לאמה ואמר רבי מעשה בבית שערים והטבילוה קודם לאמה וא"ר יוסף מעשה בפומבדיתא והטבילוה קודם לאמה 
\commenta{לסוכה שמן של תרומה - דסיכה כאכילה ובאכילה קאמרינן דתרומת חו"ל אסורה לנדה:}
בשלמא דר' יוסי ודרבי משום תרומת א"י אלא דרב יוסף למה לי והא אמר שמואל אין תרומת חו"ל אסורה אלא במי שטומאה יוצאה מגופו והני מילי באכילה אבל בנגיעה לא 
\commenta{ותבא כמים וגו' - בקללה גבי רשע קאי:}
אמר מר זוטרא לא נצרכה אלא לסוכה שמן של תרומה דתניא (ויקרא כב, טו) ולא יחללו את קדשי בני ישראל אשר ירימו לה' לרבות את הסך ואת השותה
\commenta{א"ה - דחייש למיעוטא:}
שותה למה לי קרא שתיה בכלל אכילה אלא לרבות את הסך כשותה ואיבעית אימא מהכא (תהלים קט, יח) ותבא כמים בקרבו וכשמן בעצמותיו 
\commenta{אשה ואשה - לרבות קטנה כדמפרש:}
אי הכי דידן נמי 
\commenta{ואשה כי תהיה זבה וגו' - בנדה משתעי קרא דכתיב (ויקרא ט״ו:י״ט) שבעת ימים תהיה בנדתה:}
אנן דדרשינן אשה ואשה וכי חזיין מפרשי להו לא גזרו בהו רבנן אינהו דלא דרשי אשה ואשה וכי חזיין לא מפרשי להו גזרו בהו רבנן 
\commenta{ורמינהו אשה - ואשה אשר ישכב איש אותה שכבת זרע:}
מאי אשה ואשה דתניא אשה אין לי אלא אשה תינוקת בת יום אחד לנדה מנין ת"ל ואשה 
\commenta{הלכתא נינהו - לענין נדה בת יום אחד ולענין ביאה בת ג' דברים הללו הלכה למשה מסיני הם כדמפרש ואזיל דחדא מינייהו הלכתא:}
אלמא כי מרבי קרא בת יום אחד מרבי ורמינהו אשה אין לי אלא אשה תינוקת בת ג' שנים ויום אחד לביאה מנין ת"ל ואשה 
\commenta{אלא בת ג' - לענין ביאה הלכתא דמקרא לא משתמע ורבנן אסמכתא בעלמא אסמכוה ובת יום א' לנדה משתמעא מקרא:}
אמר רבא הלכתא נינהו ואסמכינהו רבנן אקראי הי קרא והי הלכתא אילימא בת יום אחד הלכתא בת שלש שנים ויום אחד קרא קרא סתמא כתיב 
אלא בת ג' שנים ויום אחד הלכתא בת יום אחד קרא ומאחר דהלכתא קרא ל"ל
למעוטי איש מאודם 
\commenta{והא דתניא אשה - לענין זבה כתיב ואשה כי יזוב זוב דמה ימים רבים ימים ב' רבים ג':}
והא דתניא אשה אין לי אלא אשה בת י' ימים לזיבה מנין ת"ל ואשה למה לי ליגמר מנדה 
צריכא דאי כתב רחמנא בנדה הוה אמינא נדה משום דכי חזאי חד יומא בעיא למיתב ז' אבל זבה דאי חזאי חד יומא בשומרת יום כנגד יום סגי לה אימא לא צריכא 
\commenta{ואלא קרא - ואשה דכתיב גבי נדה למה לי:}
וליכתוב רחמנא בזבה ולא בעי בנדה ואנא ידענא דאין זבה בלא נדה אין ה"נ ואלא קרא למה לי למעוטי איש מאודם 
\commenta{והא מיעטתיה - לעיל:}
הא מיעטתיה חדא זימנא חד למעוטי משכבת זרע וחד למעוטי מדם 
\commenta{וכן לענין זכרים - מרבי בן יום אחד:}
וכן לענין זכרים דתניא (ויקרא טו, ב) איש איש מה ת"ל איש איש לרבות תינוק בן יום אחד שהוא מטמא בזיבה דברי רבי יהודה 
רבי ישמעאל בנו של ר' יוחנן בן ברוקה אומר אין צריך הרי הוא אומר (ויקרא טו, לג) לזכר ולנקבה לזכר כל שהוא זכר בין שהוא גדול בין שהוא קטן ולנקבה כל שהיא נקבה בין גדולה בין קטנה א"כ מה ת"ל איש איש דברה תורה כלשון בני אדם 
\commenta{איש אין לי אלא איש - ואיש כי תצא ממנו שכבת זרע:}
אלמא כי מרבי קרא בן יום אחד מרבי ורמינהו איש אין לי אלא איש בן תשע שנים ויום אחד מנין ת"ל {ויקרא טו } ואיש 
אמר רבא הלכתא נינהו ואסמכינהו רבנן אקראי הי הלכתא והי קרא אילימא בן יום אחד הלכתא ובן ט' שנים ויום אחד קרא קרא סתמא כתיב 
\commenta{ ה"נ גרסינן\textbf{ ומאחר דהלכתא קרא למה לי} - ואבן תשעה לשכבת זרע קמהדר:}
אלא בן ט' שנים ויום אחד הלכתא ובן יום א' קרא וכי מאחר דהלכתא היא קרא למה לי למעוטי אשה מלובן
\commenta{למה לי למיכתב בזכרים - איש איש לרבות קטן בן יומו ולמה לי למיכתב בנקבות ואשה לרבות בת יום אחד לנדה ובת י' לזיבה:}
למה לי למכתב בזכרים ולמה לי למכתב בנקבות
\commenta{בראיות כבימים - בין שרואה ג' ביום אחד בין שרואה שלש בשלשה ימים רצופין הוי זב כדילפינן בב"ק בפרק כיצד (דף כד.) לזכר ולנקבה מה נקבה בימים אף זכר בימים וראיות בגופיה כתיב מנה הכתוב שלש וקראו טמא אבל זבה שראתה שלש ביום אחד לא הויא זבה:}
צריכי דאי כתב רחמנא בזכרים משום דמטמאו בראיות כבימים אבל נקבות דלא מטמאו בראיות כבימים אימא לא 
ואי כתב רחמנא בנקבות משום דקמטמו באונס אבל זכרים דלא מטמאו באונס אימא לא צריכא
הכותים מטמאין משכב תחתון כעליון מאי משכב תחתון כעליון אילימא דאי איכא י' מצעות ויתיב עלייהו מטמו להו פשיטא דהא דרס להו 
\commenta{זב לא מיטמא באונס דכתיב מבשרו ולא מחמת אונסו. זבה מיטמאה באונס בהאי פירקא (לקמן נדה דף לו:) כשהוא אומר כי יזוב זוב דמה הרי אונס אמור:}
אלא שיהא תחתונו של בועל נדה כעליונו של זב מה עליונו של זב אינו מטמא אלא אוכלין ומשקין אף תחתונו של בועל נדה אינו מטמא אלא אוכלין ומשקין 
עליונו של זב מנלן דכתיב (ויקרא טו, י) וכל הנוגע בכל אשר יהיה תחתיו יטמא מאי תחתיו
\clearpage}

\newsection{דף לג}
\twocol{אילימא תחתיו דזב {ויקרא ט״ו:י׳ } מואיש אשר יגע במשכבו נפקא אלא הנוגע בכל אשר יהיה הזב תחתיו ומאי ניהו עליון של זב
\commenta{ומאי ניהו נישא - בגד הנישא על הזב. מאי טעמא והנושא והנישא כתיב לא ידענא מאי היא ונראה בעיני שהוא פי' משובש:}
והנושא נמי יטמא ומאי ניהו נישא מ"ט והנשא כתיב 
\commenta{נתקו הכתוב מטומאה חמורה - מדכתיב וכל הנוגע בכל אשר יהיה תחתיו יטמא והנושא אותם יכבס בגדיו ומדלא ערבינהו ונכתוב וכל הנוגע בכל אשר יהיה תחתיו והנושא אותם יכבס בגדיו ואפסקינהו ביטמא מכלל דהאי יטמא לאו באדם ובגדים קמיירי אלא באוכלין ומשקין והנושא אותם נדרש בתורת כהנים לנושא משכבו ומושבו של זב:}
נתקו הכתוב מטומאה חמורה והביאו לידי טומאה קלה לומר לך שאינו מטמא אלא אוכלין ומשקין 
\commenta{אימא - להכי אפסקינהו קרא דלא יטמא אדם לטמא בגדים שעליו להכי כתב יטמא דאדם לחודיה טמא או אם נגעו בגדים יהו בגדים טמאים ולא הוא:}
אימר נתקו הכתוב מטומאה חמורה דלא מטמא אדם לטמא בגדים אבל אדם או בגדים ליטמא אמר קרא יטמא טומאה קלה משמע 
ותחתונו של בועל נדה מנלן דתניא (ויקרא טו, כד) ותהי נדתה עליו
\commenta{יעלה לרגלה - יעלה לסופה. כגון בא עליה בששי שלה יטבול בליל המחרת כמותה:}
יכול יעלה לרגלה ת"ל יטמא ז' ימים 
\commenta{מה היא מטמאה אדם - לטמא בגדים שעליו דאתי בקל וחומר ממשכבה אף הוא מטמא אדם לטמא בגדים שעליו:}
ומה ת"ל ותהי נדתה עליו שיכול לא יטמא אדם וכלי חרס ת"ל ותהי נדתה עליו מה היא מטמאה אדם וכלי חרס אף הוא מטמא אדם וכלי חרס 
אי מה היא עושה משכב ומושב לטמא אדם לטמא בגדים אף הוא עושה משכב ומושב לטמא אדם לטמא בגדים ת"ל וכל המשכב אשר ישכב עליו יטמא 
\commenta{שאין ת"ל וכל המשכב אשר ישכב עליו יטמא - שהרי כבר נאמר ותהי נדתה עליו ובדידה כתיב וכל הנוגע במשכבה יכבס בגדיו: }
שאין ת"ל וכל המשכב אשר ישכב עליו יטמא ומה ת"ל וכל המשכב אשר וגו' נתקו הכתוב מטומאה חמורה והביאו לידי טומאה קלה לומר לך שאינו מטמא אלא אוכלין ומשקין 
\commenta{אבל אדם ובגדים - או אדם אם תגע בו או בגדים אם תגע בהם:}
פריך רב אחאי אימא נתקו הכתוב מטומאה חמורה והביאו לטומאה קלה דלא ליטמא אדם לטמויי בגדים אבל אדם ובגדים ליטמא אמר רב אסי יטמא טומאה קלה משמע 
\commenta{מידי אחרינא לא - כגון אדם וכלי חרס:}
אימא ותהי נדתה עליו כלל וכל המשכב פרט כלל ופרט אין בכלל אלא מה שבפרט משכב ומושב אין מידי אחרינא לא 
אמר אביי יטמא ז' ימים מפסיק הענין הוי כלל ופרט המרוחקין זה מזה וכל כלל ופרט המרוחקין זה מזה אין דנין אותו בכלל ופרט 
\commenta{וכל המשכב ריבויא הוא ולאו פרטא:}
רבא אמר לעולם דנין וכל ריבויא הוא 
\commenta{אימא כהיא - בועלה כמותה דהא אקשינהו קרא:}
מתקיף לה רבי יעקב אימא כהיא מה היא לא חלקת בה בין מגעה למשכבה לטמא אדם ולטמא בגדים לחומרא אף הוא לא תחלוק בו בין מגעו למשכבו לטמא אדם ולטמא בגדים לקולא 
\commenta{עליו - להחמיר משמע ולא להקל:}
אמר רבא עליו להטעינו משמע
\commenta{אטו כולהו בועלי נדות נינהו - והא איכא פנויים:}
מפני שהן בועלי נדות וכו' אטו כולהו בועלי נדות נינהו א"ר יצחק מגדלאה בנשואות שנו
\commenta{אם הן יושבות על כל דם ודם - שהן רואות בין טמא בין טהור יושבת עליו ז' ימי נדתה שהתחיל מנינה משנשתנה מראה דם ותמנה ז' ימים:}
והן יושבות על דם וכו' תניא אר"מ אם הן יושבות על כל דם ודם תקנה גדולה היא להן
אלא שרואות דם אדום ומשלימות אותו לדם ירוק 
\commenta{יום שפוסקת בו - כשהיא רואה ג' רצופים לאחר ז' ימי הנדה והויא זבה ופוסקת באמצע היום סופרת אותו קצת היום שעד הערב למנין שבעה נקיים ואנן ז' נקיים שלמים בעינן לזבה:}
דבר אחר יום שפוסקת בו סופרתו למנין שבעה 
מתקיף לה רמי בר חמא ותספרנו ואנן נמי ניספריה דקיימא לן מקצת היום ככולו 
\commenta{א"כ - דלענין נקיים מקצת היום ככולו זב שפסק והתחיל לספור וראה קרי וקיי"ל דסותר אותו היום ותו לא בהמפלת בשמעתא קמייתא (לעיל נדה דף כב.) ואי מקצת היום ככולו מאי סותר הא איכא קצת היום לבא:}
אמר רבא אם כן שכבת זרע דסתר בזיבה היכי משכחת לה והא מקצת היום ככולו 
\commenta{אי דחזאי בפלגא דיומא כו' - קושיא היא. ומשני ולימא ליה לקרא כו' בתמיה:}
אי דחזאי בפלגא דיומא ה"נ הכא במאי עסקינן דחזאי סמוך לשקיעת החמה 
\commenta{וקא מהדר לאוקמי אתקפתיה \textbf{אין על כרחיך שבקיה לקרא כו'} - דכיון דקיי"ל בכל דוכתי מקצת היום ככולו איהו דחיק ומוקים אנפשיה בסמוך לשקיעת החמה:}
וליקום ולימא ליה לקרא כי כתיבא סמוך לשקיעת החמה כתיבא אין על כרחך שבקיה לקרא דאיהו דחיק ומוקי אנפשיה 
\commenta{פולטת - כגון ששמשה בזוב ופסקה והתחילה למנות ופלטה מהו שתסתור מניינה הא דקי"ל דפולטת טמאה משום רואה היא וסתרה:}
בעי רמי בר חמא פולטת שכבת זרע מהו שתסתור בזיבה רואה היתה וסותרת
או דילמא נוגעת היתה ולא סתרה 
\commenta{דיה כבועלה - זב המונה וראה קרי אינו סותר אלא יום אחד:}
אמר רבא לפום חורפא שבשתא נהי נמי דסתרה כמה תסתור תסתור שבעה דיה כבועלה 
\commenta{תסתור יום אחד - אם כן לא הוו רצופין ורחמנא אמר ואחר תטהר אחר אחר לכולן כלומר זמן אחד לכולן שיהו רצופין טהרתן:}
תסתור יום אחד (ויקרא טו, כח) ואחר תטהר אמר רחמנא אחר אחר לכולן שלא תהא טומאה מפסקת ביניהם 
\commenta{לטהרתו אמר רחמנא - שבעת ימים לטהרתו משמע טהרה אחת שלא תהא הפסקה ביניהם והיכי סתר קרי יום א' ולא הוו רצופין:}
וליטעמיך זב גופיה היכי סתר לטהרתו אמר רחמנא שלא תהא טומאה מפסקת ביניהן 
\commenta{אלא מאי אית לך למימר שלא תהא טומאת זיבה מפסקת ביניהן - שאם רואה זוב סותר את כל מה שמנה אבל קרי לא הוה הפסקה:}
אלא מאי אית לך למימר שלא תהא טומאת זיבה מפסקת ביניהן הכא נמי שלא תהא טומאת זיבה מפסקת ביניהן
\commenta{לתואך - מקום:}
ואין חייבין עליהן על ביאת מקדש וכו' רב פפא איקלע לתואך אמר אי איכא צורבא מרבנן הכא איזיל אקבל אפיה אמרה ליה ההיא סבתא איכא הכא צורבא מרבנן ורב שמואל שמיה ותני מתניתא יהא רעוא דתהוי כוותיה 
\commenta{רמא ליה תורא - רב שמואל שחט שור לכבוד רב פפא:}
אמר מדקמברכי לי בגוויה ש"מ ירא שמים הוא אזל לגביה רמא ליה תורא רמא ליה מתני' אהדדי תנן אין חייבין עליהן על ביאת מקדש ואין שורפין עליהן את התרומה מפני שטומאתה ספק אלמא מספיקא לא שרפינן תרומה
\commenta{על ספק בגדי עם הארץ - כלומר על בגדי עם הארץ אם נגעו בה נשרפת משום שמא טמא היה והאי כותי נמי תיפוק ליה אם נגע משכבו בתרומה דתהא נשרפת משום דעם הארץ הוא והוה ליה בגדי עם הארץ:}
ורמינהי על ששה ספקות שורפין את התרומה על ספק בגדי עם הארץ 
אמר רב פפא יהא רעוא דלתאכיל האי תורא לשלמא הכא במאי עסקינן בכותי חבר 
כותי חבר בועל נדה משוית ליה 
\commenta{שבקיה - רב פפא לאושפיזיה משום דכספיה ואזל לגבי רב שימי:}
שבקיה ואתא לקמיה דרב שימי בר אשי אמר ליה מאי טעמא לא משנית ליה בכותי שטבל ועלה ודרס על בגדי חבר ואזלו בגדי חבר ונגעו בתרומה 
\commenta{לא בעל בקרוב - כבר עברו ימי טומאתו וסלקא ליה טבילה:}
דאי משום טומאת עם הארץ הא טביל ליה ואי משום בועל נדה ספק בעל בקרוב ספק לא בעל בקרוב 
ואם תמצי לומר בעל בקרוב ספק השלימתו ירוק ספק לא השלימתו והוי ספק ספיקא ואספק ספיקא לא שרפינן תרומה 
\commenta{משום בגדי עם הארץ מדרס לפרושים - כלומר כשם שמדרס מטמא אדם ובגדים כך בגדי עם הארץ מדרס לפרושים:}
ותיפוק ליה משום בגדי עם הארץ דאמר מר בגדי עם הארץ מדרס לפרושין אמר ליה בכותי ערום
{\large\emph{מתני׳}} בנות צדוקין בזמן שנהגו ללכת בדרכי אבותיהן הרי הן ככותיות פרשו ללכת בדרכי ישראל הרי הן כישראלית רבי יוסי אומר לעולם הן כישראלית עד שיפרשו ללכת בדרכי אבותיהן
\commenta{סתמא מאי - לתנא קמא דאמר נהגו ללכת בדרכי אבותיהן הרי אלו ככותיות פרשו ללכת בדרכי ישראלית הרי אלו כישראלית סתמא מאי:}
{\large\emph{גמ׳}} איבעיא להו סתמא מאי ת"ש בנות צדוקין בזמן שנוהגות ללכת בדרכי אבותיהן הרי הן ככותיות הא סתמא כישראלית אימא סיפא פרשו ללכת בדרכי ישראל הרי הן כישראלית הא סתמא ככותיות אלא מהא ליכא למשמע מיניה 
\commenta{עד שיפרשו - אבל סתמא כישראלית מכלל דלת"ק סתמא ככותיות:}
ת"ש דתנן ר' יוסי אומר לעולם הן כישראלית עד שיפרשו ללכת בדרכי אבותיהן מכלל דת"ק סבר סתמא ככותיות ש"מ
\commenta{והוריקו פניו - חרה לו שנטמאו בגדיו וקדם כהן גדול אצל אשתו של צדוקי לשואלה אם השלימה לדם ירוק או לא:}
תנו רבנן מעשה בצדוקי אחד שספר עם כהן גדול בשוק ונתזה צנורא מפיו ונפלה לכהן גדול על בגדיו והוריקו פניו של כהן גדול וקדם אצל אשתו 
אמרה לו אף על פי שנשי צדוקים הן מתיראות מן הפרושים ומראות דם לחכמים 
\commenta{בקיאין אנו בהן - אני מכיר בנשי צדוקים ששכן היה להן:}
אמר רבי יוסי בקיאין אנו בהן יותר מן הכל והן מראות דם לחכמים חוץ מאשה אחת שהיתה בשכונתינו שלא הראת דם לחכמים ומתה 
\commenta{ותיפוק ליה - דאפילו עם הארץ ישראל דאינו בועל נדה קי"ל צינורא דע"ה מטמאה כל זמן שהיא לחה:}
ותיפוק ליה משום צנורא דעם הארץ אמר אביי בצדוקי חבר אמר רבא צדוקי חבר בועל נדה משוית ליה אלא אמר רבא
\clearpage}

\newsection{דף לד}
\twocol{רגל הוה וטומאת עם הארץ ברגל כטהרה שוינהו רבנן דכתיב (שופטים כ, יא) ויאסף כל איש ישראל אל העיר כאיש אחד חברים הכתוב עשאן כולן חברים
\commenta{מתני' בית שמאי מטהרין - בדם עובדת כוכבים אע"ג דברוקה ובמימי רגליה מודו דחכמים גזרו עליהם להיות כזבין לכל דבריהם אפי' הכי דמה טהור. וטעמא מפרש בגמרא דשיירו בה רבנן למיהוי היכרא דטומאתה מדרבנן:}
{\large\emph{מתני׳}} דם עובדת כוכבים ודם טהרה של מצורעת ב"ש מטהרים ובית הלל אומרים כרוקה וכמימי רגליה 
\commenta{דם היולדת - שעברו ימי טומאת לידה ולא טבלה:}
דם היולדת שלא טבלה ב"ש אומרים כרוקה וכמימי רגליה וב"ה אומרים מטמא לח ויבש 
ומודים ביולדת בזוב שהיא מטמאה לח ויבש
\commenta{גמ' גזרו עליהן שיהו כזבין לכל דבריהן - משום שלא יהא תינוק ישראל רגיל אצלו במשכב זכור:}
{\large\emph{גמ׳}} ולית להו לב"ש (ויקרא טו, ב) דבר אל בני ישראל ואמרת אליהם איש איש כי יהיה זב בני ישראל מטמאין בזיבה ואין העובדי כוכבים מטמאין בזיבה אבל גזרו עליהן שיהו כזבין לכל דבריהם
\commenta{עשיתו כשל תורה - דכיון דליכא היכרא אתי למשרף עליה תרומה וקדשים:}
אמרי לך ב"ש (ההוא בזכרים איתמר דאי בנקבות) היכי לעביד ליטמא לח ויבש עשיתו כשל תורה ליטמי לח ולא ליטמי יבש חלקת בשל תורה 
\commenta{אי הכי רוקה ומימי רגליה נמי - דלגבי ישראלית לח ולא יבש כי מטמית להו בעובדת כוכבים עשיתו כשל תורה דקיי"ל בפ' דם הנדה (לקמן נדה דף נד:) רוקו וניעו מטמאין לחין ולא יבשין ונעביד היכרא ברוקה דעובדת כוכבים ומימי רגליה לטהרן לגמרי:}
אי הכי רוקה ומימי רגליה נמי כיון דעבדינן היכרא בדמה מידע ידיע דרוקה ומימי רגליה דרבנן 
\commenta{דם לגבי רוק לא שכיח חשיב ליה:}
ולעביד היכרא ברוקה ומימי רגליה ולטמויי לדמה רוקה ומימי רגליה דשכיחי גזרו בהו רבנן דמה דלא שכיחא לא גזרו ביה רבנן 
\commenta{זובו - של עובד כוכבים:}
אמר רבא זובו טמא אפילו לב"ש קריו טהור אפילו לב"ה 
זובו טמא אפילו לב"ש דהא איכא למעבד היכרא בקריו
קריו טהור אפי' לב"ה עבוד ביה רבנן היכרא כי היכי דלא לשרוף עליה תרומה וקדשים 
\commenta{קריו תלוי במעשה - אין קרי בא אלא ע"י חימום:}
ולעביד היכרא בזובו ולטמויי לקריו זובו דלא תלי במעשה גזרו ביה רבנן קריו דתלי במעשה לא גזרו ביה רבנן 
\commenta{עובדת כוכבים שפלטה כו' טמאה - השכבת זרע אבל בעובדת כוכבים לא שייכא בה טומאה כבהמה דעלמא:}
לימא מסייע ליה עובדת כוכבים שפלטה שכבת זרע מישראל טמאה ובת ישראל שפלטה שכבת זרע מן העובד כוכבי' טהורה מאי לאו טהורה גמורה לא טהורה מדאורייתא טמאה מדרבנן 
ת"ש נמצאת אומר שכבת זרע של ישראל טמאה בכל מקום
ואפי' במעי עובדת כוכבים ושל עובד כוכבים טהור' בכל מקום ואפי' במעי ישראלית חוץ ממי רגלים שבה 
וכי תימא ה"נ טהור' מדאוריית' אבל טמאה מדרבנן אטו מי רגליה מדאורייתא מי מטמאו אלא ש"מ טהורה אפילו מדרבנן ש"מ
אמר מר שכבת זרע של ישראל טמאה בכ"מ אפי' במעי עובדת כוכבים תפשוט דבעי רב פפא דבעי רב פפא שכבת זרע של ישראל במעי עובדת כוכבים מהו 
\commenta{כי מיבעיא ליה לאחר ג' - דלגבי ישראלית טהורה כדאמר במסכת שבת בפרק ר"ע:}
בתוך ג' לא קמיבעיא ליה לרב פפא כי קמיבעיא ליה לאחר ג' מאי 
\commenta{דדייגי - יראים וחרדים במצות ומתוך דאגתן מתחממין:}
ישראל דדייגי במצות חביל גופייהו ומסריח עובדי כוכבים דלא דייגי במצות לא חביל גופייהו ולא מסריח או דילמא כיון דאכלי שקצים ורמשים חביל גופייהו ומסריח תיקו
דם טהרה של מצורעת ב"ש כו' מאי טעמא דב"ה אמר ר' יצחק לזכר לרבות מצורע למעינותיו ולנקבה לרבות מצורעת למעינותיה 
\commenta{שאר מעינותיה - כגון רוקה ומימי רגליה דאתו מזב דכתיב ביה (ויקרא ט״ו:ח׳) וכי ירוק הזב ומיניה נפקי שאר מעיינות:}
מאי מעינותיה אילימא שאר מעינותיה מזכר נפקא אלא לדמה לטמא דם טהרה שלה 
\commenta{ואסור בתשמיש המטה - בשבעת ימי ספירו דכתיב וישב מחוץ לאהלו ואין אהלו אלא אשתו שנאמר (דברים ה׳:כ״ז) שובו לכם לאהליכם: }
וב"ש נקבה מזכר לא אתיא דאיכא למיפרך מה לזכר שכן טעון פריעה ופרימה ואסור בתשמיש המטה תאמר בנקבה דלא 
וב"ה לכתוב רחמנא בנקבה ולא בעי זכר ואנא אמינא ומה נקבה שאינה טעונה פריעה ופרימה ואינה אסורה בתשמיש המטה רבי רחמנא מעינותיה זכר לא כ"ש 
\commenta{ואם אינו ענין לשאר מעינותיה - דנפקי מלנקבה:}
אם אינו ענין לזכר תנהו ענין לנקבה ואם אינו ענין למעינותיה תנהו ענין לדמה לטמא דם טהרה שלה 
\commenta{שכן מטמאה באונס - לגבי זיבה:}
וב"ש זכר מנקבה לא אתיא דאיכא למיפרך מה לנקבה שכן מטמאה מאונס תאמר בזכר דלא 
\commenta{ופרכי מילי דזב - בתמיה: }
וב"ה קיימי במצורע ופרכי מילי דזב וב"ש שום טומאה פרכי 
\commenta{ה"ג ואי בעית אימא אמרי לך ב"ש האי לזכר מיבעי ליה כל שהוא זכר כו' - ול"ג נקבה כל שהיא נקבה כו' דא"כ מעינות למצורע מנא להו אלא לנקבה לרבות מצורעת למעינותיה ויליף זכר מנקבה למעינות והאי לזכר דמייתר להו לב"ה דרשי ליה ב"ש לזב קטן שמטמא בזיבה בן יומו ולא נפקא להו מאיש איש דסבירא להו דברה תורה כלשון בני אדם ותינוקת קטנה נפקא להו מאשה ואשה כדילפינן בריש פירקא ואייתר להו לנקבה דהכא לרבות מצורעת למעינותיה וב"ה נמי לא דרשי איש איש מיהו נפקא להו מזאת תורת הזב בין גדול בין קטן מטמא בזיבה:}
ואיבעית אימא אמרי לך ב"ש האי לזכר מיבעי ליה לזכר כל שהוא זכר (האי) בין גדול בין קטן ובית הלל נפקא להו מזאת תורת הזב בין גדול בין קטן 
\commenta{כי פשיט - כי דריש:}
אמר רב יוסף כי פשיט רבי שמעון בן לקיש בזב בעי הכי ראייה ראשונה של זב קטן מהו שתטמא במגע (ויקרא טו, לב) זאת תורת הזב ואשר תצא ממנו שכבת זרע אמר רחמנא
\commenta{מטמא - נמי כשל גדול טומאת ערב בקרי:}
כל ששכבת זרע שלו מטמא ראייה ראשונה שלו מטמאה והאי כיון דשכבת זרע שלו לא מטמאה ראייה ראשונה נמי לא תטמא או דילמא כיון דאילו איהו חזי תרתי מצטרפא מטמיא 
אמר רבא ת"ש זאת תורת הזב בין גדול בין קטן מה גדול ראייה ראשונה שלו מטמא אף קטן ראייה ראשונה נמי מטמא 
\commenta{ראייה ראשונה של מצורע מהו שתטמא במשא - של איש טהור לא תבעי לך דודאי לא מטמא במשא דאיתקש לקרי וקרי לא מטמא במשא כדתנן בפ"ק דמסכת כלים ושניה נמי לא תבעי לך דאפילו באדם טהור מטמיא במשא כדאמר לקמן בפרק דם הנדה (נדה דף נה.) זוב מטמא במשא ויליף לה מזובו טמא הוא וההוא בראייה שניה כתיב דמנה הכתוב שתים וזב בעל שתי ראיות זב גמור הוא לטומאה אלא שאין טעון קרבן כי תיבעי לן ראשונה דבאדם דעלמא לא מטמיא אלא במגע הכא במצורע מאי מקום זיבה מעיין הוא ואיתרבי מלזכר לרבות מצורע למעיינותיו ומעיינות מטמאו במשא כדאמר בפרק דם הנדה (לקמן נדה דף נה:) וכי ירוק הזב בטהור במה שביד טהור:}
בעי רב יוסף ראייה ראשונה של מצורע מהו שתטמא במשא מקום זיבה מעין הוא ומטמא או דילמא לאו מעין הוא 
\commenta{זובו טמא - היינו ראייה שניה שהרי מנה הכתוב שתים זב מבשרו זובו טמא:}
אמר רבא ת"ש (ויקרא טו, ב) זובו טמא הוא לימד על הזוב שהוא טמא במאי אילימא בזב גרידא
\clearpage}

\newsection{דף לה}
\twocol{לאחרים גורם טומאה לעצמו לא כל שכן אלא פשיטא בזב מצורע
\commenta{ומדאיצטריך קרא לרבויי לשניה ש"מ - בראשונה דלאו מעיין הוא דאי מעיין הוא ראשונה מטמאה שניה לא כל שכן:}
ומדאיצטריך קרא לרבויי בראייה שניה שמע מינה מקום זיבה לאו מעין הוא 
\commenta{לעולם אימא לך בזב גרידא - דאי במצורע לא איצטריך קרא לרבויי דאפילו ראשונה מטמא משום מעיין:}
אמר ליה רב יהודה מדסקרתא לרבא ממאי דילמא לעולם אימא לך בזב גרידא ודקאמרת לאחרים גורם טומאה לעצמו לא כל שכן שעיר המשתלח יוכיח שגורם טומאה לאחרים והוא עצמו טהור 
\commenta{מאי תיבעי ליה - לרב יוסף:}
אמר אביי מאי תבעי ליה והא הוא דאמר זאת תורת הזב בין גדול בין קטן וכיון דנפקא ליה מהתם אייתר ליה לזכר לרבות מצורע למעינותיו נקבה לרבות מצורעת למעינותיה
\commenta{ואיתקש מצורע לזב גמור - בעל שתי ראיות דהא בהאי קרא דהיקשא דמצורע תרתי כתיבי והזב את זובו הרי שתים וכתיב ביה לזכר לרבות מצורע כדין זב גמור מה זב גמור זובו מטמא במשא אף מצורע זובו מטמא במשא ואפילו ראשונה דבשלמא אי מפיק ליה האי לזכר לזב קטן ויליף מצורע זכר ממצורעת נקבה איכא למבעי ראייה ראשונה מהו מי הוי מתעגל ויוצא ומתרבי עם שאר מעיינות או לאו דבנקבה ליכא זוב לובן למילפיה מהיקשא דזב גמור אי לאו מעיין הוי אבל השתא דמצורע זכר איתקש לזב גמור לא צריך תו למבעי: ואית דמפרשי בעיא דרב יוסף הכי ראייה ראשונה של מצורע מהו שתטמא במשא בזב גרידא לא תבעי לך דודאי מטמא כדאמר לקמן לימד על הזוב שהוא טמא ומוקמינן בדם הנדה דקרא למשא אתא כי תבעי לך במצורע דאיכא למימר בזב גרידא הוא דמטמא משום דגורמת לו טומאה אבל במצורע כבר טמא הוא ואינה גורמת לו טומאה לא מטמיא הטיפה אי לא אמרינן מעיין הוא כרוקו או מימי רגליו מאי מעיין הוא או לא. לימד על הזוב ומוקמינן לה בראייה ראשונה והכי משמע כי יהיה זב מבשרו אותו הזב טמא. לאחרים לבעלה. ומדאיצטריך לרבויי לראייה ראשונה ש"מ לאו מעיין הוא. ושיבוש הוא מפני כמה תשובות חדא דקא פשיט ליה מקום זיבה לאו מעיין הוא והא למשא רבייה קרא כדאמר בדם הנדה מה לי מעיין ומה לי לאו מעיין ועוד מי איכא לאוקמי להאי קרא בראייה ראשונה והא מהכא נפקא לן בכל דוכתי מנה הכתוב שתים וקראו טמא ועוד מי איכא למימר בראייה ראשונה בזב גרידא לאחרים גורם טומאת משא הא זב בעל ראייה אחת אינו אלא כרואה קרי כדאמר בכמה מקומות שתים לטומאת זיבה ושלש לקרבן ועוד ראייה ראשונה הא אקשיה לעיל לקרי וקרי לא מטמא במשא ומשום שכתוב בספרים מדאיצטריך קרא בראייה ראשונה מפרשי ליה הכי ולאו מילתא היא דהאי בראייה ראשונה פירוש היה בספרים ולקמיה קאי והכי קאמר מדאיצטריך קרא לרבויי לשניה ש"מ ראשונה מקום זיבה לאו מעיין הוא וה"ג בשלהי שמעתתא מה זב גמור מטמא זובו במשא אף מצורע כו' אלמא זב גמור הוא דמטמא טיפת זובו במשא אבל זב בעל ראייה אחת לא:}
ואקשיה רחמנא מצורע לזב גמור מה זב גמור מטמא במשא אף ראייה ראשונה של מצורע מטמא במשא 
\commenta{מטמא באונס - אותה טומאה קלה האמורה בה מגע וטומאת ערב בקרי ולהצטרף לשניה לטמא כזב גמור והא דכתיב מבשרו ולא מחמת אונסו על כרחך אראייה שניה דכתיב בתרה קאי ולא אראשונה דהא איתקש לשכבת זרע ושכבת זרע כל עצמו מחמת אונס חימום בא:}
א"ר הונא ראייה ראשונה של זב מטמאה באונס שנאמר (ויקרא טו, לב) זאת תורת הזב ואשר תצא ממנו שכבת זרע מה שכבת זרע מטמא באונס אף ראייה ראשונה של זב מטמאה באונס 
\commenta{בודקין אותו - בשבעה דרכים במאכל במשתה בקפיצה ובמשאוי כו':}
תא שמע ראה ראייה ראשונה בודקין אותו מאי לאו לטומאה לא לקרבן 
\commenta{אילימא לקרבן אבל לטומאה - האמורה בו דהיינו טומאה חמורה לא בעי בדיקה:}
ת"ש בשניה בודקין אותו למאי אילימא לקרבן אבל לטומאה לא אקרי כאן מבשרו ולא מחמת אונסו אלא לאו לטומאה ומדסיפא לטומאה רישא נמי לטומאה 
מידי איריא הא כדאיתא והא כדאיתא 
תא שמע רבי אליעזר אומר אף בשלישי בודקין אותו מפני הקרבן מכלל דתנא קמא מפני הטומאה קאמר 
\commenta{אתים - והזב את זובו לזכר ולנקבה (ויקרא ט״ו:ל״ג):}
לא דכולי עלמא לקרבן והכא באתים קא מיפלגי רבנן לא דרשי אתים ורבי אליעזר דריש אתים 
\commenta{זב חדא זובו תרתי לזכר - כלומר עד כאן יש לו דין זכר שאינו מטמא באונס מכאן ואילך דהיינו שלישית לנקבה כתיב כלומר יש לו דין נקבה שמטמאה באונס כדילפינן בפירקין (לקמן נדה דף לו:) מכי יזוב זוב ואיהי נמי מייתא ליה לידי קרבן:}
רבנן לא דרשי אתים הזב חדא זובו תרתי לזכר בשלישי אקשיה רחמנא לנקבה 
\commenta{ברביעית - אז אקשיה רחמנא לנקבה דאפילו באונס. ולא לענין טומאה דהא מכי חזיא תרתי איטמי ליה ולקרבן נמי לא דמשלשה איחייב ליה אלא לסתירה כגון דפסק והתחיל לספור שבעה נקיים וראה ראייה רביעית אפילו באונס סותר:}
ורבי אליעזר דריש אתים הזב חדא את תרתי זובו תלת ברביעי אקשיה רחמנא לנקבה 
\commenta{והלא זב בכלל בעל קרי היה - כדמפרש לקמיה ואזיל דאיתקוש לשכבת זרע:}
תא שמע רבי יצחק אומר והלא זב בכלל בעל קרי היה ולמה יצא להקל עליו ולהחמיר עליו להקל עליו שאין מטמא באונס ולהחמיר עליו
שהוא עושה משכב ומושב 
\commenta{אלא פשיטא בראייה ראשונה - דהוי בכלל קרי כדאקשינן לעיל הזב ואשר תצא ממנו וכו':}
אימת אילימא בראייה שניה היכא הוה בכלל בעל קרי אלא פשיטא בראייה ראשונה וקתני להקל עליו שאינו מטמא באונס 
ותסברא להחמיר עליו שהוא עושה משכב ומושב בראייה ראשונה בר משכב ומושב הוא 
אלא הכי קאמר רבי יצחק אומר והלא זב בכלל בעל קרי היה בראייה ראשונה ולמה יצא בראייה שנייה להקל עליו ולהחמיר עליו להקל עליו שאינו מטמא באונס ולהחמיר עליו שהוא עושה משכב ומושב 
\commenta{דיהה - כלומר מתמקמק ונפרד ואינו קשור:}
אמר רב הונא זוב דומה למי בצק של שעורים זוב בא מבשר המת שכבת זרע בא מבשר החי זוב דיהה ודומה ללובן ביצה המוזרת שכבת זרע קשורה ודומה ללובן ביצה שאינה מוזרת
דם היולדת שלא טבלה וכו'
תניא אמרו להן בית הלל לבית שמאי אי אתם מודים בנדה שלא טבלה וראתה דם שהיא טמאה אמרו להם בית שמאי לא אם אמרתם בנדה שאפילו טבלה וראתה טמאה תאמרו ביולדת שאם טבלה וראתה שהיא טהורה 
\commenta{יולדת בזוב תוכיח שאם טבלה - לאחר שמנתה ז' נקיים לבד ימי טומאת לידה:}
אמרו להם יולדת בזוב תוכיח שאם טבלה וראתה לאחר ימי ספירה טהורה לא טבלה וראתה טמאה 
\commenta{אמרו להם בית שמאי הוא הדין כו' - כלומר כי היכי דאמרינן ביולדת גמורה גרידא שלא טבלה וראתה טהורה כך יולדת בזוב שעברו עליה ימי לידה ושבעה נקיים ולא טבלה וראתה טהורה:}
אמרו להם הוא הדין והיא התשובה 
למימרא דפליגי והתנן ומודים ביולדת בזוב שהיא מטמאה לח ויבש 
\commenta{כאן שספרה - בראיותה בספרה ז' נקיים. מתניתין דמודו בית שמאי כשעברו ימי לידה ועדיין לא ספרה אחריהם שבעה נקיים והכי קאמר אף על גב דאמרי בית שמאי ביולדת גרידתא כרוקה וכמימי רגליה מודים ביולדת בזוב שלא ספרה דכל זמן שלא ספרה זבה היא ואם ראתה לא הוי כרוקה וכמימי רגליה דאילו רוק מטמא לח ולא יבש כדאמר בדם הנדה (לקמן נדה נו.) וכי ירוק כעין רקיקה דהיינו לח אבל דמה מטמא לח ויבש כשאר דם הנדה כדילפינן בדם הנדה (שם דף נד:):}
לא קשיא כאן שספרה כאן שלא ספרה 
\commenta{והתניא - בניחותא:}
והתניא יולדת בזוב שספרה ולא טבלה וראתה הלכו בית שמאי לשיטתן וב"ה לשיטתן 
\commenta{מעין אחד הוא - דם הבא בימי לידה ודם הבא בימי טוהר ממעין אחד הן באין:}
איתמר רב אמר מעין אחד הוא התורה טמאתו והתורה טהרתו 
\commenta{נסתם הטמא - לאחר שבועים ונפתח הטהור:}
ולוי אמר שני מעינות הם נסתם הטמא נפתח הטהור נסתם הטהור נפתח הטמא 
מאי בינייהו איכא בינייהו שופעת מתוך שבעה לאחר שבעה ומתוך ארבעה עשר לאחר ארבעה עשר ומתוך ארבעים לאחר ארבעים ומתוך שמנים לאחר שמנים
\commenta{לרב רישא - דשופעת מתוך ימי טומאה לימי טהרה לקולא וטהורה ואף על גב דלא פסק דהא התורה טהרתו:}
לרב רישא לקולא וסיפא לחומרא
\commenta{ללוי רישא לחומרא - כיון דלא פסקה לא נסתם מעין טמא:}
ללוי רישא לחומרא וסיפא לקולא 
מיתיבי דם היולדת שלא טבלה בית שמאי אומרים כרוקה וכמימי רגליה וב"ה אומרים מטמא לח ויבש 
\commenta{קס"ד דפסקה - יום או שנים לאחר שבועים ומשום הכי מטמא לח ויבש דממעין טמא בא והתורה לא טהרתו לב"ה אלא ביומי וטבילה:}
קא ס"ד דפסקה בשלמא לרב דאמר מעין אחד הוא משום הכי מטמא לח ויבש אלא ללוי דאמר שני מעינות הן אמאי מטמא לח ויבש 
\commenta{בשופעת - מתוך ימי טומאה לימי טהרה ולא פסקה:}
אמר לך לוי הכא במאי עסקינן בשופעת אי בשופעת מ"ט דב"ש קסברי ב"ש מעין אחד הוא 
\commenta{בשלמא ללוי - דאמר לב"ה ב' מעינות הן:}
בשלמא ללוי היינו דאיכא בין ב"ש וב"ה אלא לרב מאי בינייהו 
\commenta{ביומי וטבילה - לא שנא שופעת לא שנא פוסקת כל זמן שלא טבלה הוי דמה טמא בימי לידתה:}
איכא בינייהו יומי וטבילה דבית שמאי סברי ביומי תלה רחמנא וב"ה סברי ביומי וטבילה 
\commenta{ומודים ביולדת בזוב שמטמאה - ואוקימנא כשעברו ימי לידתה ולא ספרה אחריהן ז' נקיים וקס"ד הכא נמי דפסקה אחר שבועים ומשום הכי מטמא דממעיין טמא אתו והתורה לא טהרתו עד שתשב ז' נקיים כדין זבה:}
ת"ש ומודים ביולדת בזוב שהיא מטמאה לח ויבש ס"ד הכא נמי דפסקה
\commenta{אמאי מטמאה - דהא ממקום טהור הוא בא והרי הוא כדם מגפתה וכרוקה שמטמא לח משום מעין הזב ולא יבש:}
בשלמא לרב דאמר מעין אחד הוא משום הכי מטמא לח ויבש אלא ללוי דאמר שני מעינות הן אמאי מטמא לח ויבש 
אמר לך הכא נמי בשופעת אי בשופעת למאי איצטריך 
\commenta{לב"ש אצטריך - דקאמר מעין אחד הוא ומטהרי לה ביולדת גרידתא אחר ימי לידתה מיד דביומי תלא רחמנא אצטריך ביולדת בזוב לאשמועינן דמטמאה:}
לב"ש איצטריך אף על גב דקאמרי בית שמאי מעין אחד הוא וביומי תלה רחמנא הני מילי ביולדת גרידתא דשלימו להו יומי אבל יולדת בזוב דבעי ספירה לא 
\commenta{דותה תטמא - קרא יתירא הוא דה"ל למכתב וטמאה ז' ימים כימי נדה ולשתוק:}
תא שמע (ויקרא יב, ב) דותה תטמא לרבות את בועלה
\commenta{משום הכי בעיא - הפסקה שבעה נקיים דממעין דמים טמאים אתי והתורה לא טהרתו עד שתספור שבעה נקיים כדכתיב דותה תטמא:}
דותה תטמא לרבות הלילות דותה תטמא לרבות היולדת בזוב שצריכה שתשב שבעה ימים נקיים בשלמא לרב דאמר מעין אחד הוא משום הכי בעיא שבעה ימים נקיים
\clearpage}

\newsection{דף לו}
\twocol{אלא ללוי דאמר שני מעינות הן למה לי שבעה במשהו סגיא 
\commenta{שתפסוק משהו - לאחר שבועים דנקבה דאי בתוך שבועים לאו הפסק היא דכיון דחוזרת ורואה בתוך שבועים חזר הטמא להפתח:}
הכי קאמר צריכה שתפסוק משהו שיעלו לה לשבעה נקיים 
\commenta{ימי עיבורה עולין לה - בפ"ק (לעיל נדה דף י:) אמרי' מעוברת ומניקה שעברו עליה ג' עונות וראתה דיה שעתה:}
ת"ש ימי עיבורה עולים לה לימי מניקותה וימי מניקותה עולים לה לימי עיבורה 
\commenta{עולין לה - כלומר מצטרפים לימי מניקותה כגון היתה מניקה והתחילה לפסוק ונתעברה והשלימה ג' עונות מצטרפין ימי עיבורה לימי מניקותה:}
כיצד הפסיקה שתים בימי עיבורה ואחת בימי מניקותה שתים בימי מניקותה ואחת בימי עיבורה אחת ומחצה בימי עיבורה ואחת ומחצה בימי מניקותה עולין לה לג' עונות 
\commenta{בשלמא לרב דאמר מעין אחד הוא משום הכי - מעוברת שהתחילה להפסיק וילדה צריכה להשלים ג' עונות שלא תראה אחר לידה כלום ודם לידה לא מפסיק לשלש עונות כדתרצינן בפ"ק דם נדה לחוד ודם לידה לחוד אבל אחר לידה אי חזאי לא השלימה ג' עונות ומטמאה מעת לעת שאין דמיה מסולקין:}
בשלמא לרב דאמר מעין אחד הוא משום הכי בעי הפסק שלש עונות אלא ללוי דאמר שני מעינות הן למה לי הפסק שלש עונות במשהו סגי 
\commenta{שיעלו לה - ימי טהרה:}
הכי קאמר צריכה שתפסוק משהו כדי שיעלו לה לשלש עונות 
\commenta{ושוין - שמאי והלל דפליגי במעת לעת:}
ת"ש ושוין ברואה אחר דם טוהר שדיה שעתה 
\commenta{בשלמא ללוי דאמר שני מעינות הן משום הכי דיה שעתה - שהרי דמיה מסולקין שכמה ימים שעברו ולא ראתה דם ממעין זה והויא לה ראייה ראשונה ודיה שעתה ומניקה מארבע נשים היא דדיין שעתן בראייה ראשונה:}
בשלמא ללוי דאמר שני מעינות הן משום הכי דיה שעתה אלא לרב דאמר מעין אחד הוא אמאי דיה שעתה תטמא מעת לעת 
\commenta{דליכא שהות - כגון שראתה ביום שמונים ואחד מיד דליכא למימר מעת לעת דאפי' הוה מאתמול כי האידנא בבית החיצון דם טהור הוא:}
דליכא שהות 
\commenta{ותטמא מפקידה לפקידה - כגון אם בדקה שחרית ומצאתה טמאה תטמא טהרות של ערבית והכא ליכא למימר דליכא שהות דאי אפילו מפקידה לפקידה ליכא צריכא למימר דדיה שעתה:}
ותטמא מפקידה לפקידה כיון דמעת לעת ליכא מפקידה לפקידה נמי לא גזרו בה רבנן 
\commenta{וב"ה לשיטתן - דכל זמן שלא טבלה מטמא לח ויבש והא כיון דספרה אחר ימי לידה ודאי פסק וקשיא ללוי:}
תא שמע יולדת בזוב שספרה ולא טבלה וראתה הלכו ב"ש לשיטתן ובית הלל לשיטתן 
בשלמא לרב דאמר מעין אחד הוא משום הכי מטמא לח ויבש אלא ללוי דאמר שני מעינות הן אמאי מטמא לח ויבש 
\commenta{כתנא דשוין - דאמר לעיל ושוין ברואה אחר דם טוהר שדיה שעתה ואנא אוקימנא בדאיכא שהות ואפ"ה דיה שעתה אלמא שני מעינות הן:}
אמר לך לוי אנא דאמרי כתנא דשוין 
\commenta{והא ספרה קתני - ואי לא פסק היכי ספרה. ומשני לעולם בשופעת מימי טומאה לימי טוהר וספירה הכי הואי כגון שהיתה יולדת נקבה בזוב וימי טומאת לידה שני שבועים ופסקה שבוע קמא וזו היא ספירתה דקסבר ימי לידתה שאינה רואה בהן עולין לה לספירת זיבתה ושבוע שני לא פסקה וחזר ונפתח מעין טמא ושופעת מתוכו בתוך שבוע שלישי ומש"ה מטמא לח ויבש לב"ה דהא ממעין טמא אתא ותורה לא טהרתו אלא ביומי וטבילה ולב"ש טהור דביומי תליא רחמנא והא שלמו להו ולהכי נקט יולדת נקבה דאי ילדה זכר אי אפשר לה לספור שבעה שלא תפסוק משהו אחר ימי לידה דלא הוו אלא שבעה:}
ואיבעית אימא בשופעת והא ספרה קתני 
הכא ביולדת נקבה בזוב עסקינן דשבוע קמא פסקה שבוע בתרא לא פסקה וקסבר ימי לידתה שאין רואה בהן עולין לה לספירת זיבתה 
\commenta{כוותיה דרב לחומרא - בשופעת מתוך שמונים לאחר שמונים דמטמאה ולא אמרי' ממעין טהור אתא:}
אמר ליה רבינא לרב אשי אמר לן רב שמן מסכרא אקלע מר זוטרא לאתרין ודרש הילכתא כוותיה דרב לחומרא והלכתא כוותיה דלוי לחומרא 
\commenta{בין לקולא - כגון שופעת מתוך י"ד לאחר י"ד דהוה ליה רב לקולא דאמר מעין אחד הוא והתורה טהרתו ולא בעי הפסקה:}
רב אשי אמר הלכתא כוותיה דרב בין לקולא בין לחומרא דריש מרימר הלכתא כוותיה דרב בין לקולא בין לחומרא והלכתא כוותיה דרב בין לקולא בין לחומרא
{\large\emph{מתני׳}} המקשה נדה קשתה שלשה ימים בתוך י"א יום ושפתה מעת לעת וילדה הרי זו יולדת בזוב דברי רבי אליעזר 
\commenta{לילה ויום - דאם שפתה בחצי יום שלישי וילדה יום רביעי דשפתה מעת לעת אין זו שופי אלא אם כן שפתה לילה ויום שלאחר הלילה כלילי שבת ויומו:}
רבי יהושע אומר לילה ויום כלילי שבת ויומו ששפתה מן הצער ולא מן הדם 
\commenta{כמה הוא קישויה - דלא אתיא לידי זיבה בכל דמים שתראה:}
כמה היא קישויה ר' מאיר אומר אפילו ארבעים וחמשים יום רבי יהודה אומר דיה חדשה ר' יוסי ור' שמעון אומרים אין קישוי יותר משתי שבתות
\commenta{גמ' אטו כל המקשה נדה היא - אילו קשתה בימי זיבה מי הויא נדה והא אין נדות בימי זוב:}
{\large\emph{גמ׳}} אטו כל המקשה נדה היא
\commenta{אמר רב נדה ליומא - וה"ק המקשה ורואה דם בימי זיבה אע"ג דלא אתיא ביה לידי זיבה מיהו באותו היום אסורה לשמש אבל לערב טובלת ואינה צריכה לשמור יום כנגד יום דרחמנא טהריה לדם קושי:}
אמר רב נדה ליומא ושמואל אמר חיישינן שמא תשפה 
\commenta{אינה כלום - ואפילו באותו היום משמשת:}
ור' יצחק אמר המקשה אינה כלום והקתני המקשה נדה 
\commenta{בימי נדה - בימים הראוים לנדות:}
אמר רבא בימי נדה נדה בימי זיבה טהורה והתניא המקשה בימי נדה נדה בימי זיבה טהורה 
\commenta{ושפתה - שנים מן הצער ולא מן הדם:}
כיצד קשתה יום אחד ושפתה שנים או שקשתה שנים ושפתה יום אחד או ששפתה וקשתה וחזרה ושפתה הרי זו יולדת בזוב 
אבל שפתה יום אחד וקשתה שנים או ששפתה שנים וקשתה יום אחד או שקשתה ושפתה וחזרה וקשתה אין זו יולדת בזוב כללו של דבר קושי סמוך ללידה אין זו יולדת בזוב שופי סמוך ללידה הרי זו יולדת בזוב 
\commenta{כל שחל קישויה - לאו לשון התחלה הוא אלא הכי קאמר כל שקשתה בשלישי אפי' שעה אחת בליל כניסת שלישי אפילו כל היום כולו בשופי ושפתה מליל ד' שעה אחת להשלמת מעת לעת אין זו יולדת בזוב דבעינן שופי כל יום ג' המביאה לידי זבה:}
חנניא בן אחי ר' יהושע אומר כל שחל קישויה בשלישי שלה אפילו כל היום כולו בשופי אין זו יולדת בזוב 
כללו של דבר לאתויי מאי לאתויי דחנניא 
מה"מ דת"ר דמה דמה מחמת עצמה ולא מחמת ולד 
\commenta{וכי יזוב זוב - זוב משמע זוב מ"מ ואפי' באונס:}
אתה אומר מחמת ולד או אינו אלא מחמת אונס כשהוא אומר (ויקרא טו, כה) ואשה כי יזוב זוב דמה הרי אונס אמור הא מה אני מקיים דמה דמה מחמת עצמה ולא מחמת ולד 
\commenta{שיש טהרה אחריו - דם טוהר:}
ומה ראית לטהר את הולד ולטמא באונס מטהר אני בולד שיש טהרה אחריו ומטמא אני באונס שאין טהרה אחריו 
\commenta{ואונס באשה לא אשכחן - דטהור:}
אדרבה מטהר אני באונס שכן אונס בזב טהור השתא מיהא באשה קיימינן ואונס באשה לא אשכחן 
ואיבעית אימא מאי דעתיך לטהורי באונס ולטמויי בולד אין לך אונס גדול מזה 
\commenta{זובה - דם יהיה זובה ובנדה משתעי דפרשה ראשונה נאמרה בנדה ושניה בזבה כדכתיב בשניה בלא עת נדתה:}
אי הכי נדה נמי נימא זובה זובה מחמת עצמה ולא מחמת ולד 
אתה אומר ולד או אינו אלא אונס כשהוא אומר (ויקרא טו, יט) ואשה כי תהיה זבה הרי אונס אמור הא מה אני מקיים זובה זובה מחמת עצמה ולא מחמת ולד 
\commenta{תשב על דמי טהרה - דהוה ליה למכתב וששים יום וששת ימים דמי טהרה:}
אמר ר"ל אמר קרא תשב יש לך ישיבה אחרת שהיא כזו ואיזו זו זו קושי בימי זיבה ואימא זו קושי בימי נדה 
אלא אמר אבוה דשמואל אמר קרא (ויקרא יב, ה) וטמאה שבועים כנדתה ולא כזיבתה מכלל דזיבתה טהור ואיזו זו זו קושי בימי זיבה 
\commenta{ואפי' בשופי - ואפי' הפסיק שופי בין קושי ללידה נימא טהורה הואיל ודם בקושי בא:}
והשתא דכתיב וטמאה שבועים כנדתה דמה למה לי אי לאו דמה הוה אמינא כנדתה ולא כזיבתה ואפילו בשופי קמ"ל 
\commenta{כוותיה דרב - נדה ליומא:}
שילא בר אבינא עבד עובדא כוותיה דרב כי קא נח נפשיה דרב א"ל לרב אסי זיל צנעיה ואי לא ציית גרייה הוא סבר גדייה א"ל 
\commenta{לדידי הוה אמר לי - דתלמידיה אנא:}
בתר דנח נפשיה דרב א"ל הדר בך דהדר ביה רב א"ל אם איתא דהדר ביה לדידי הוה אמר לי לא ציית גדייה א"ל ולא מסתפי מר מדליקתא 
\commenta{אסיתא - משום דשמיה הכי דריש לשמיה הכי:}
א"ל אנא איסי בן יהודה דהוא איסי בן גור אריה דהוא איסי בן גמליאל דהוא איסי בן מהללאל אסיתא דנחשא דלא שליט ביה רקבא א"ל ואנא שילא בר אבינא בוכנא דפרזלא דמתבר אסיתא דנחשא 
\commenta{עיילוה בחמימי - חולי חם:}
חלש רב אסי עיילוה בחמימי אפקוה מקרירי עיילוה בקרירי אפקוה מחמימי נח נפשיה דרב אסי
\clearpage}

\newsection{דף לז}
\twocol{אזל שילא אמר לדביתהו צבית לי זוודתא דלא ליזיל ולימא ליה לרב מילי עילואי צביתה ליה זוודתא נח נפשיה דשילא חזו דפרחא אסא מהאי פוריא להאי פוריא אמרי ש"מ עבדו רבנן פייסא 
\commenta{קושי מהו שתסתור בזיבה - כל מה שספרה:}
בעי רבא קושי מהו שתסתור בזיבה 
\commenta{דבר הגורם - לידי זיבה אם ראתה שלש והאי לאו גורם הוא דלא אתיא בה לידי זיבה. ואי קשיא והרי קרי שאינו גורם כדקי"ל וסותר ההיא לאו סתירה היא דחד יומא הוא דסותר וכי קא בעי רבא לסתור הכל:}
דבר המטמא סותר והאי נמי מטמא כימי נדה הוא או דילמא דבר הגורם סותר והאי לאו גורם הוא 
\commenta{שאינו גורם - כדקי"ל ולא מחמת אונסו:}
א"ל אביי אונס בזיבה יוכיח שאינו גורם וסותר 
\commenta{לאיי - באמת אונס נמי גורם הוא לידי זוב שאם ראה שתים שלא באונס ושלישית באונס מצטרפין לקרבן דתנן שלישית אין בודקין אותו:}
אמר ליה לאיי האי נמי גורם הוא דתנן ראה ראייה ראשונה בודקין אותו שניה בודקין אותו שלישית אין בודקין אותו 
ולרבי אליעזר דאמר אף בשלישי' בודקין אותו ה"נ כיון דלא גרים לא סתר אמר ליה לרבי אליעזר ה"נ 
ת"ש רבי אליעזר אומר אף בשלישית בודקין אותו ברביעית אין בודקין אותו מאי לאו לסתירה 
לא לטמויה לההיא טיפה במשא 
\commenta{לקרבן אמרתי - דבעי בדיקה ולא לסתירה:}
ת"ש בשלישית רבי אליעזר אומר בודקין אותו ברביעית אין בודקין אותו לקרבן אמרתי ולא לסתירה 
\commenta{אלא - לא תוקי כדתרצה לטמויי במשא אלא ודאי לסתירה ובעיין לא תפשוט דאנן לרבנן בעינן ואת פשטת מדר"א:}
אלא לר"א תפשוט דדבר שאינו גורם סותר לרבנן מאי 
ת"ש דתני אבוה דרבי אבין מה גרם לו זובו שבעה לפיכך סותר שבעה מה גרם לו קריו יום אחד לפיכך סותר יום אחד 
\commenta{ש"מ - מדנקט לשון גרמא מכלל דקיימא ליה דבר הגורם סותר:}
מאי שבעה אילימא דמטמא שבעה האי מה זובו טמא שבעה מבעי ליה אלא לאו דבר הגורם סותר דבר שאינו גורם אינו סותר ש"מ 
\commenta{רבי אליעזר היא - דאמר אונס סותר ואינו גורם:}
אמר אביי נקטינן אין קושי סותר בזיבה ואי משכחת תנא דאמר סותר ההוא ר"א היא 
\commenta{מהו שתעלה - אם ילדה בזוב ובימי לידתה לא ראתה מהו שיעלה לה לנקיים ולא תצטרך לספור אחריהן:}
תניא רבי מרינוס אומר אין לידה סותרת בזיבה איבעיא להו מהו שתעלה אביי אמר אינה סותרת ואינה עולה רבא אמר אינה סותרת ועולה 
אמר רבא מנא אמינא לה דתניא (ויקרא טו, כח) ואחר תטהר אחר אחר לכולן שלא תהא טומאה מפסקת ביניהם 
אי אמרת בשלמא עולה היינו דלא מפסקת טומאה אלא אי אמרת אינה עולה אפסיק ליה לידה ואביי אמר לך שלא תהא טומאת זיבה מפסקת ביניהם 
\commenta{מזובה - וספרה לה מזובה:}
אמר רבא מנא אמינא לה דתניא מזובה מזובה ולא מנגעה מזובה ולא מלידתה ואביי אמר לך תני חדא מזובה ולא מנגעה ולא תתני ולא מלידתה 
ורבא האי מאי אי אמרת בשלמא מזובה ולא מלידתה איידי דאצטריך ליה לידה תנא נגעה אטו לידה אלא אי אמרת מזובה ולא מנגעה האי {ויקרא ט״ו:י״ג } מוכי יטהר הזב מזובו נפקא מזובו ולא מנגעו 
ואביי חד בזב וחד בזבה וצריכי דאי כתב רחמנא
בזב משום דלא מטמא באונס אבל זבה דמטמיא באונס אימא לא צריכא 
\commenta{זב מטמא בין בשלש ראיות ביום אחד בין בשלש ראיות בג' ימים רצופין אבל זבה אינה מטמאה אלא בשלשה ימים כדכתיב בה ימים. בב"ק (דף כד.) מרבינן זב לימים: }
ואי כתב רחמנא בזבה משום דלא מטמיא בראיות כבימים אבל זב דמטמא בראיות כבימים אימא לא צריכא 
אמר אביי מנא אמינא לה דתניא (ויקרא יב, ב) דותה תטמא לרבות את בועלה
דותה תטמא לרבות את הלילות דותה תטמא לרבות את היולדת בזוב שצריכה שתשב שבעה נקיים 
\commenta{נקיים מלידה - שלא תהא לידה ביניהם דאע"ג דלא חזיא כמאן דחזיא דמי דהא ימי לידה טמאין בלא דם:}
מאי לאו נקיים מלידה לא מדם 
\commenta{מה ימי נדתה אין ראויין לזיבה - דאין זבה אלא אחר נדה ואין ספירת שבעה לזיבה עולה מהן דהא מכי הויא זבה לא חיילא ימי נדה לעולם עד שתשב שבעה נקיים וכל דם שרואה אינה אלא סותרת אבל נדה לא הויא הילכך לא משכחת ימי ספירה בימי נדה ולא ימי נדה בימי ספירה:}
ואמר אביי מנא אמינא לה דתניא כימי נדתה כך ימי לידתה מה ימי נדתה אין ראוין לזיבה ואין ספירת שבעה עולה מהן אף ימי לידתה שאין ראוין לזיבה אין ספירת שבעה עולה מהן 
\commenta{מסתר נמי סתרה - וכ"ש דלא סלקא כי היכי דס"ל באונס בזב שאינו גורם וסותר ימי לידה נמי אע"ג דלא גרמי סתר וכי קאמינא אנא לרבנן דאמרי דבר שאינו גורם אינו סותר אינה סותרת ועולה:}
ורבא הא מני רבי אליעזר היא דאמר מסתר נמי סתרה 
\commenta{וכי דנין אפשר - כגון לידה דאפשר לה ללדת בימי ספירה:}
וכי דנין אפשר משאי אפשר 
\commenta{ר' אליעזר היא - במסכת מנחות בפ' התודה דיליף פסח דורות מפסח מצרים שלא יביא אלא מן החולין אמר לו ר"ע וכי דנין אפשר משאי אפשר במצרים אכתי לא הוה מעשר אבל לדורות דאיכא מעות מעשר שני נייתי מנייהו אמר לו אף ע"פ שאי אפשר ראיה גדולה היא ונלמד הימנה:}
אמר רב אחדבוי בר אמי ר' אליעזר היא דאמר דנין אפשר משאי אפשר ורב ששת אמר על כרחך הקישן הכתוב איכא דאמרי אמר רב אחדבוי בר אמי אמר רב ששת רבי אליעזר היא דאמר דנין אפשר משאי אפשר ורב פפא אמר על כרחך הקישן הכתוב
קשתה שלשה ימים וכו'
\commenta{רב חסדא אמר טמאה - דהא איכא שופי:}
איבעיא להו שפתה מזה ומזה מהו רב חסדא אמר טמאה רבי חנינא אמר טהורה 
\commenta{לפניו - יום אחד או שנים קודם ביאתו:}
א"ר חנינא משל למלך שיצא וחיילותיו לפניו בידוע שחיילותיו של מלך הן 
\commenta{דמפיש חיילות טפי - סמוך לביאתו:}
ורב חסדא אמר כל שכן דבעי נפיש חיילות טפי 
תנן רבי יהושע אומר לילה ויום כלילי שבת ויומו ששפתה מן הצער ולא מן הדם טעמא דמן הצער ולא מן הדם הא מזה ומזה טהורה תיובתא דרב חסדא 
\commenta{דפסקו להו חיילות לגמרי - דהשתא ודאי קמאי לא מחמתיה אתו דאי מחמתיה אתו כל שכן דמפיש חיילות:}
אמר לך רב חסדא לא מבעיא מזה ומזה דטמאה דפסקי להו חיילות לגמרי אבל מן הצער ולא מן הדם אימר כי היכי דמדם לא פסקה מקושי נמי לא פסקה והא תונבא בעלמא הוא דנקט לה קמ"ל 
תנן קשתה שלשה ימים בתוך אחד עשר יום ושפתה מעת לעת וילדה הרי זו יולדת בזוב 
\commenta{אי נמי כי קתני - שלשה בקושי וא' בשופי ובכולן ראתה למה לי כולי האי:}
היכי דמי אילימא כדקתני למה לי שלש בתרי בקושי וחד בשופי סגי 
\commenta{או שקשתה שנים ושפתה מעת לעת - בשלישי מן הצער ולא מן הדם: }
אלא לאו הכי קאמר קשתה שלשה ושפתה מזה ומזה או שקשתה שנים ושפתה מעת לעת הרי זו יולדת בזוב ותיובתא דר' חנינא 
\commenta{כדקתני - שלשה בקושי ומקצת שלישי ומקצת רביעי בשופי:}
אמר לך רבי חנינא לא לעולם כדקתני והא קא משמע לן דאע"ג דמתחיל קישוי בשלישי ושפתה מעת לעת טמאה לאפוקי מרבי חנינא
\commenta{בריאה - ארבעים:}
כמה היא קשויה ר"מ אומר וכו' השתא חמשים מקשיא ארבעים מיבעיא אמר רב חסדא ל"ק כאן לחולה כאן לבריאה 
\commenta{אין הולד מטהר אלא ימים הראויין להיות בה זבה - דהיינו י"א יום שבין נדה לנדה אבל ראתה אחריהן טמאה נדה כדאמר לעיל בימי נדה נדה:}
א"ר לוי אין הולד מטהר אלא ימים הראויין להיות בהן זבה ורב אמר אפי' בימים הראויין לספירת זבה אמר רב אדא בר אהבה ולטעמיה דרב
\clearpage}

\newsection{דף לח}
\twocol{אפי' ימים הראויין לספירת סתירת זבה 
\commenta{ועוד ארבעים וחמשים יום - קס"ד לגמרי מטהר לה ר"מ מזוב ומנדות:}
תנן כמה הוא קשויה ר"מ אומר ארבעים וחמשים יום 
בשלמא לרב משכחת לה כרב אדא בר אהבה אלא ללוי קשיא
\commenta{בימי נדה נדה - וה"ק ר"מ כמה הוא קשויה שלא תבא להתחייב בקרבן ולספירת נקיים ארבעים וחמשים יום ומיהו בכל הני ארבעים כי שלמי ימי זוב ומטו ימי נדות טמאה נדה וכי הדרי שלמי ז' ימי נדות ומטו ימי זוב טהורה:}
אמר לך לוי מי קתני טהורה בכולן בימי נדה נדה בימי זיבה טהורה 
\commenta{זבה גדולה - ראתה שלשה רצופין בתוך י"א יום טהורה מספירת נקיים ומקרבן אבל לא ראתה אלא יום אחד צריכה לשמור יום כנגד יום דכי כתיב דמה דדרשינן ולא מחמת ולד בזבה גדולה כתיב דכתיב בההוא קרא ימים רבים ואם משך קשויה יותר מאחד עשר יום שנכנסה לימים שראויים לנדה טמאה:}
לישנא אחרינא אמרי א"ר לוי אין הולד מטהר אלא ימים הראויין להיות בהן זבה גדולה מ"ט (ויקרא טו, כה) דמה ימים רבים כתיב 
\commenta{ימי וכל ימי זוב טומאתה וגו' - ומינייהו נפקא לן זבה קטנה בשילהי מכילתין בשמעתא בתרייתא:}
אבא שאול משמיה דרב אמר אפילו ימים הראויין להיות בהן זבה קטנה מ"ט ימי וכל ימי התם כתיבי 
\commenta{קשיא לתרוייהו - דאמרי זבה גדולה וקטנה הוא דמטהר ולד אבל אם יאריך אחריהן טמא:}
תנן כמה הוא קשויה ר"מ אומר אפי' ארבעים וחמשים יום קשיא לתרוייהו מי קתני טהורה בכולן קשתה בימי נדתה נדה בימי זיבתה טהורה 
\commenta{יש מקשה - שהיא רואה:}
תניא היה ר"מ אומר יש מקשה ק"נ יום ואין זיבה עולה בהן כיצד שנים בלא עת
\commenta{וחמשים שהולד מטהר - הנך חמשים של קושי לפני השמונים:}
ושבעה נדה ושנים של אחר הנדה וחמשים שהולד מטהר
ושמונים של נקבה ושבעה נדה ושנים של אחר הנדה 
\commenta{א"כ - דקחשבת שמונים של נקבה:}
אמרו לו א"כ יש מקשה כל ימיה ואין זיבה עולה בהן 
\commenta{אמר להן - ר"מ מאי דעתייכו משום נפלים כדפרשינן:}
אמר להן מאי דעתייכו משום נפלים אין קושי לנפלים 
\commenta{יש רואה - בלא קושי:}
ת"ר יש רואה מאה יום ואין זיבה עולה בהן כיצד שנים בלא עת ושבעה נדה ושנים של אחר הנדה ושמונים של נקבה ושבעה נדה ושנים של אחר הנדה 
\commenta{קמ"ל דאפשר לפתיחת הקבר בלא דם - דאי לא אפשר הויא לה זבה כשילדה אחר שנים שלאחר הנדה דהוה ליה יום לידה שלישי ואי הויא חזיא קודם לידה בפתיחת הקבר הרי היא זבה הואיל ובלא קושי ראתה:}
מאי קמ"ל לאפוקי ממ"ד אי אפשר לפתיחת הקבר בלא דם קמ"ל דאפשר לפתיחת הקבר בלא דם
\commenta{להקל ולהחמיר - פעמים שאינו מטהר כלום פעמים שהוא מטהר חדש ויום אחד:}
ר' יהודה אומר דיה וכו' תניא רבי יהודה אומר משום רבי טרפון דיה חדשה ויש בדבר להקל ולהחמיר 
\commenta{הרי זו יולדת בזוב - הואיל ורוב זיבה בשמיני הואי שאין הולד מטהר שדינן חד יומא בתר תרין והרי היא זבה:}
כיצד קשתה שנים בסוף שמיני ואחד בתחלת תשיעי ואפילו בתחלת תשיעי ילדה הרי זו יולדת בזוב
\commenta{ואפי' בסוף תשיעי ילדה - ואע"ג דתשיעי כולו בשופי חוץ משני ימים אין זו יולדת בזוב בלבד שלא תראה דם בימי השופי ולא אמרינן כיון דשופי סמוך ללידה נטמאנה בדם הקושי דקסבר ר' יהודה שיפורא גרים כלומר החדש של תשיעי גורמת לידת הולד דמכי עייל הוה זמן לידה הילכך כל הדמים שרואה בו בקושי אינו אלא מחמת הולד אע"ג דאיכא שופי בתרייהו ואפי' בתחלת תשיעי הוי מחמת ולד דקסבר יולדת לט' יולדת למקוטעין:}
אבל קשתה יום אחד בסוף שמיני ושתים בתחלת תשיעי ואפילו בסוף תשיעי ילדה אין זו יולדת בזוב 
\commenta{שיפורא - שופר שתוקעין בו ב"ד בקדוש החדש:}
אמר רב אדא בר אהבה ש"מ קסבר רבי יהודה שיפורא גרים איני והא אמר שמואל אין אשה מתעברת ויולדת אלא למאתים ושבעים ואחד יום או למאתים ושבעים ושנים יום או למאתים ושבעים ושלשה 
\commenta{[הוא] - שמואל דאמר כחסידים:}
הוא דאמר כחסידים הראשונים דתניא חסידים הראשונים לא היו משמשין מטותיהן אלא ברביעי בשבת שלא יבואו נשותיהן
לידי חלול שבת ברביעי ותו לא אימא מרביעי ואילך 
אמר מר זוטרא מאי טעמייהו דחסידים הראשונים דכתיב {רות ד׳:י״ג } ויתן [ה'] לה הריון הריון בגימטריא מאתן ושבעים וחד הוו 
\commenta{לתקופות הימים - כתיב:}
אמר מר זוטרא אפי' למ"ד יולדת לתשעה אינה יולדת למקוטעים יולדת לשבעה יולדת למקוטעים שנאמר (שמואל א א, כ) ויהי לתקופות הימים ותהר חנה ותלד בן מיעוט תקופות שנים מיעוט ימים שנים
\commenta{מכלל דזיבתה טהורה - ומהכא גמרינן לעיל קושי בימי זיבה:}
רבי יוסי ור"ש אומרים אין קושי יותר מב' שבתות אמר שמואל מאי טעמייהו דרבנן דכתיב (ויקרא יב, ה) וטמאה שבועים כנדתה כנדתה ולא כזיבתה מכלל דזיבתה טהורה וכמה שבועים 
\commenta{יש מקשה - שרואה כ"ה יום בין קושי ובין שופי:}
ת"ר יש מקשה עשרים וחמשה יום ואין זיבה עולה בהן כיצד שנים בלא עת ושבעה נדה וב' שלאחר נדה וארבעה עשר שהולד מטהר 
\commenta{בלא ולד - לקמן פריך משמע שאין שם ולד א"כ לאו קושי הוא ולמה לי כ"ו בתלתא לאחר הנדה סגי:}
ואי אפשר שתתקשה עשרים וששה יום בלא ולד ולא תהא יולדת בזוב 
בלא ולד בתלתא נמי סגי אמר רב ששת אימא במקום שיש ולד אמר ליה רבא והא בלא ולד קתני 
אלא אמר רבא הכי קאמר אי אפשר שתתקשה עשרים וששה יום במקום שיש ולד ולא תהא יולדת בזוב ובמקום שאין ולד אלא נפל בתלתא נמי הויא זבה מ"ט אין קושי לנפלים
\commenta{מתני' המקשה תוך שמונים - שנתעברה בימי טוהר או שנשתהה ולד אחר חבירו ב' חדשים וחצי אחד נגמרה צורתו באמצע שביעי ואחד נגמרה צורתו לסוף תשיעי כמעשה דיהודה וחזקיה בני ר' חייא (לעיל נדה דף כז.):}
{\large\emph{מתני׳}} המקשה בתוך שמונים של נקבה כל דמים שהיא רואה טהורין עד שיצא הולד ורבי אליעזר מטמא
\commenta{שהחמיר בדם השופי - כגון שאר יולדת שרואה ג' בשופי וילדה הרי זו יולדת בזוב:}
אמרו לו לרבי אליעזר ומה במקום שהחמיר בדם השופי היקל בדם הקושי מקום שהיקל בדם השופי אינו דין שנקל בדם הקושי 
\commenta{דיו לבא מן הדין - כגון קושי של תוך מלאת שאתם למדין מקושי דעלמא:}
אמר להן דיו לבא מן הדין להיות כנדון ממה היקל עליה מטומאת זיבה אבל טמאה טומאת נדה
\commenta{גמ' תשב - על דמי טהרה:}
{\large\emph{גמ׳}} תנו רבנן תשב לרבות המקשה בתוך שמונים של נקבה שכל דמים שהיא רואה טהורין עד שיצא הולד ור"א מטמא 
אמרו לו לר"א ומה במקום שהחמיר בשופי שלפני הולד היקל בשופי שלאחר הולד מקום שהיקל בקושי שלפני הולד אינו דין שנקל בקושי שלאחר הולד 
אמר להם דיו לבא מן הדין להיות כנדון ממה היקל עליה מטומאת זיבה אבל מטמאה טומאת נדה 
\commenta{הרי אנו משיבין אותך לשון אחר - שלא נלמד קושי שלאחר מקושי שלפני כדאמרי' לעיל מקום שהיקל בקושי שלפני הולד אינו דין שנקל בקושי שלאחריו דהכא הוא דאיכא למימר דיו לקושי שלאחריו להיות כקושי שלפניו שהרי ממנו אנו למדים אלא הכי גמרינן מקום שהיקל בשופי שלאחריו אינו דין שנקל בקושי שעמו שעם השופי דהשתא יליף קושי של אחריו דטהור לגמרי וליכא למימר ממה היקל מטומאת זיבה כדקאמר:}
אמרו לו הרי אנו משיבין לך לשון אחר ומה במקום שהחמיר בשופי שלפני הולד היקל בקושי שעמו מקום שהיקל בשופי שלאחר הולד אינו דין שנקל בקושי שעמו 
\commenta{דיו לבא כו' - שהרי ק"ו אינו נידון אלא מכח קושי שלפניו כדקתני ומה במקום שהחמיר בשופי שלפני הולד היקל בקושי שעמו:}
אמר להם אפילו אתם משיבין כל היום כולו דיו לבא מן הדין להיות כנדון ממה היקל עליה מטומאת זיבה אבל מטמאה טומאת נדה 
\commenta{זכינהו - תשובה נצחת יכול להשיבן:}
אמר רבא בהא זכינהו ר"א לרבנן לאו אמריתו דמה דמה מחמת עצמה ולא מחמת ולד ה"נ (ויקרא יב, ז) וטהרה ממקור דמיה דמיה מחמת עצמה ולא מחמת ולד 
\commenta{אימא בימי נדה נדה - לר"א פריך כיון דאית ליה דיו לודי מיהא דבימי זיבה טהורה כקושי שלפניו כגון ראתה בקושי שבעה ימים ראשונים תהא טמאה ואם ראתה יותר יהא הדם טהור ותטבול לאחר שבעה ראשונים של קושי:}
אימא בימי נדה נדה בימי זיבה טהורה אמר קרא תשב ישיבה אחת לכולן
\commenta{מתני' כל אחד עשר יום - אחר הנדות אשה עומדת בחזקת טהרה. ובגמרא מפרש למאי הלכתא:}
{\large\emph{מתני׳}} כל אחד עשר יום בחזקת טהרה
\clearpage}

\newsection{דף לט}
\twocol{ישבה לה ולא בדקה שגגה נאנסה הזידה ולא בדקה טהורה 
\commenta{טמאה - דאורח בזמנו בא:}
הגיע שעת וסתה ולא בדקה הרי זו טמאה ר"מ אומר אם היתה במחבא והגיע שעת וסתה ולא בדקה הרי זו טהורה מפני שחרדה מסלקת את הדמים 
\commenta{ימי הזב - ימי ספירו:}
אבל ימי הזב והזבה ושומרת יום כנגד יום הרי אלו בחזקת טומאה
\commenta{גמ' לומר שאינה צריכה בדיקה - שחרית וערבית ומשעברו צריכה לבדוק כדאיתא בפ"ק (לעיל נדה דף יא.):}
{\large\emph{גמ׳}} למאי הלכתא אמר רב יהודה לומר שאינה צריכה בדיקה והא מדקתני סיפא ישבה ולא בדקה מכלל דלכתחלה בעיא בדיקה 
\commenta{לימי נדה - לימים שראויין לנדה כגון אחר י"א שבין נדה לנדה:}
סיפא אתאן לימי נדה וה"ק כל י"א בחזקת טהרה ולא בעיא בדיקה אבל בימי נדתה בעיא בדיקה ישבה ולא בדקה שגגה נאנסה הזידה ולא בדקה טהורה 
\commenta{לא נצרכא - הא דקתני כל י"א בחזקת טהרה אלא לר"מ בסוף פרק ראשון בגמרא:}
רב חסדא אמר לא צריכא אלא לר"מ דאמר אשה שאין לה וסת אסורה לשמש ה"מ בימי נדתה אבל בימי זיבתה בחזקת טהרה קיימא 
\commenta{אמאי יוציא - תשמש כל י"א:}
א"ה אמאי א"ר מאיר יוציא ולא יחזיר עולמית דלמא אתיא לקלקולא בימי נדה 
הא מדקתני סיפא הגיע שעת וסתה ולא בדקה מכלל דבאשה שיש לה וסת עסקינן חסורי מחסרא והכי קתני כל י"א בחזקת טהרה ושריא לבעלה ובימי נדה אסורה 
בד"א באשה שאין לה וסת אבל יש לה וסת מותרת וצריכה בדיקה ישבה ולא בדקה שגגה נאנסה הזידה ולא בדקה טהורה הגיע שעת וסתה ולא בדקה טמאה 
הא מדסיפא ר"מ רישא לאו ר"מ כולה ר"מ היא וה"ק אם לא היתה במחבא והגיע שעת וסתה ולא בדקה טמאה שר"מ אומר אם היתה במחבא והגיע שעת וסתה ולא בדקה טהורה שחרדה מסלקת את הדמים 
\commenta{רבא אמר - מתני' דקתני כל י"א בחזקת טהרה:}
רבא אמר לומר שאינה מטמאה מעת לעת 
\commenta{הנדה - כלומר אשה שרואה דם מטמאה מעת לעת בראיית תחלת נדותה:}
מיתיבי הנדה והזבה והשומרת יום כנגד יום והיולדת כולן מטמאות מעת לעת תיובתא 
\commenta{חייא בר רב הונא - פי' אמתני' מהדר:}
רב הונא בר חייא אמר שמואל לומר שאינה קובעת לה וסת בתוך ימי זיבתה אמר רב יוסף לא שמיע לי הא שמעתתא 
\commenta{את אמרת ניהלן - קודם חולייך:}
א"ל אביי את אמרת ניהלן ואהא אמרת לן היתה למודה להיות רואה יום ט"ו (יום) ושינתה ליום כ' זה וזה אסורין לשמש שינתה פעמים ליום כ' זה וזה אסורין 
\commenta{לא שנו - דליום ט"ו הוי וסת קבוע דליבעי תלתא זימני למיעקריה:}
ואמרת לן עלה אמר רב יהודה אמר שמואל ל"ש אלא ט"ו לטבילתה שהן כ"ב לראיתה דהתם בימי נדתה קאי לה אבל ט"ו לראיתה דבימי זיבתה קאי לא קבעה 
\commenta{מקבע לא קבעה - בתוך ימי זיבה דתבעי ג' זימני למיעקריה דבחדא זימנא הוא דעקר לה ואם ישבה ולא בדקה טהורה:}
אמר רב פפא אמריתא לשמעתא קמיה רב יהודה מדסקרתא מקבע לא קבעה מיחש מהו דניחוש לה 
אישתיק ולא א"ל ולא מידי אמר רב פפא נחזי אנן היתה למודה להיות רואה ליום ט"ו ושינתה ליום כ' זה וזה אסורין
ואמר רב יהודה אמר שמואל ל"ש אלא ט"ו לטבילתה שהן כ"ב לראייתה
\commenta{וקתני אסורה - דכי הדר אתו כ"ב לחשבון יום שהיתה ראויה לראות בו כשדילגה בה קיימא לה בימי זיבה לחשבון ראייתה כדפרישית. וקסבר רב פפא יום עשרין ותרתין דאסרינן עלה מיום כ"ב מנינן כלומר לאו כ"ב לראיית דילוגה שהיה בכ"ז אלא כ"ב מיום שהיתה ראויה לראות לימודה. ומדקרי להו רב פפא ימי זיבה קסבר נדה ופתחה ימי נדותה וזובה מעשרין וז' מנינן דהיינו מיום ראיית דילוגה ומיום ראיית כ"ז עד השתא לא הוו לה י"ז דאי ליום שראויה לראות כשדילגה מנינן ימי נדות וזוב לא הוה קרי להו ימי זובה דהא חלפו להו כ"ב מההוא יומא:}
ושינתה ליום כ"ז דכי הדרי ואתו עשרין ותרתי קיימא לה בתוך ימי זיבתה וקתני זה וזה אסורין אלמא דחיישינן לה 
\commenta{פתחה - קרי חשבון הנדה לימי הנדה וזוב:}
וקסבר רב פפא עשרין ותרתין מעשרין ותרתין מנינן נדה ופתחה מעשרין וז' מנינן 
\commenta{ממאי - דהאי יום כ"ב דאסרינן ליה מנינן מיום כ"ב שהיתה ראויה לראות כשדילגה דמטי ליה בימי זיבה לחשבון ראייה שראתה בכ"ז:}
א"ל רב הונא בריה דרב יהושע לרב פפא ממאי דלמא עשרין ותרתין נמי מעשרין וז' מנינן דכי הדרי ואתו עשרין ותרתין קיימא לה בתוך ימי נדותה 
\commenta{והכי מסתברא - דהא משום דחיישינן שמא תחזור לקדמותה לראות מכ"ב לכ"ב אסרינן לה דהשתא הוי כ"ב לראייה אבל ליום כ"ב לחשבון יום שהיתה ראויה לראות לא מצי אסרינן משום הא חששא דאפי' חוזרת לקדמותה לא תראה היום שהוא י"ז לראייתה דההיא לא היתה למודה לראות אלא לכ"ב לראייה ואי משום שהוא כ"ב ליום הראוי לה אין זה דרך החוזרת לוסתה לקצר ימי טהרתה ולמהר ימי ראייתה כדי שיבא לפי חשבון של תחלה אלא חוזרת וקובעת כ"כ ליום שראתה: }
וה"נ מסתברא דאי לא תימא הכי האי תרנגולתא דרמיא יומא וכבשה יומא ורמיא יומא וכבשה יומא וכבשה תרי יומי ורמיא חד יומא
\commenta{כדלקמיה נקטה - מעכשיו היא חוזרת למנות כגון שעומדת יום ז' שאחר יום ו' שהטילה ומטלת יום שמיני:}
כי הדרה נקטה כדלקמיה נקטה או כדמעיקרא נקטה על כרחך כדלקמיה נקטה 
\commenta{א"ל רב פפא אלא הא דאמר כו' אין אשה קובעת לה וסת תוך ימי נדותה - בימים שהיא נדה כדבעינן לפרושי מילתא דרבי יוחנן דחזאי ריש ירחא וה' בירחא וריש ירחא וה' בירחא והשתא חזאי בה' בירחא דהוו להו ג' זימני בה' בירחא ומעכשיו יהיה לה וסת בה' בחדש. לריש לקיש לא הוה וסת דהא תרי זימני קמאי דה' בירחא הוי תוך ימי נדותה ממש דהא נדה הויא מריש ירחא ופעם ג' הוה ימי נדה לפי חשבון הראוי שהיתה ראויה לראות ריש ירחא ומדקרי לפעם ג' תוך ימי נדה אלמא לפי חשבון הראוי מנינן:}
א"ל רב פפא אלא הא דאמר ר"ל אשה קובעת לה וסת בתוך ימי זיבתה ואין אשה קובעת לה וסת בתוך ימי נדותה ורבי יוחנן אמר אשה קובעת לה וסת בתוך ימי נדותה ה"ד 
לאו כגון דחזאי ריש ירחא וחמשא בירחא וריש ירחא וחמשא בירחא והשתא חזאי בחמשא בירחא ובריש ירחא לא חזאי
וקאמר אשה קובעת לה וסת בתוך ימי נדותה אלמא מריש ירחא מנינא 
\commenta{א"ל לא אלא הכי א"ר יוחנן כגון דחזיא - תרי זימני בריש ירחא ופעם שלישית בכ"ה ורביעית בר"ח דה"ל הך ראייתה בתרייתא שהיא שלישית לראייה של ר"ח בתוך ימי נדותה ממש שהיא נדה מיום כ"ה ומש"ה קרי להו תוך ימי נדותה דהא נדה היא. וטעמא מאי הוי קביעות הא אורחה בהכי שנפתח מעיינה משום דאמר הך ראייה דראש חדש ראיית וסתה היא והיום זמנה והאי דאקדים וחזאי בכ"ה דמים יתירי איתוספו בה:}
א"ל לא הכי א"ר יוחנן כגון דחזאי ריש ירחא וריש ירחא ועשרין וחמשה בירחא וריש ירחא דאמרינן דמי יתירי הוא דאתוספו בה 
\commenta{אמרוה - להא דרבי יוחנן:}
וכן כי אתא רבין וכל נחותי ימא אמרוה כרב הונא בריה דרב יהושע
\par \par {\large\emph{הדרן עלך בנות כותים}}\par \par }

\addpart{חידושי רמב"ן על נדה}\newchap{פרק \hebrewnumeral{1} שמאי אומר}
\twocol{\clearpage}

\newsection{דף ב}
\twocol{מתניתין \textbf{שמאי אומר כל הנשים דיין שעתן.} פי' לשון דיין בכל מקום לומר דיין בכך. ואע"פ שיש עדיין להחמיר יותר אין מחמירין עליו כאותה שאמרו בפ' הזהב (דף נג ע"ב) דיו שיאמר הוא וחומשו מחולל על מעות הראשונות ואתמר עליה אא"ב אין בחומשו היינו דקתני דיו אלא למ"ד אין בו מאי דיו קשיא. ובמסכת סוטה בפ' היה מביא (דף יד ע"ב) מגישה בקרן דרומית מערבית כנגד חודה של קרן ודיו והוינן בה מאי ודיו ואמר רב אשי אצטריך סד"א תיבעי הגשת מנחה גופה קמ"ל אף כאן סד"א תיבעי סייג כדברי הלל וחכמים קמ"ל דיין שעתן, אבל לא דייקינן בשום דוכתין בלשון דיין אם יכולין להקל יותר אלא כיון שהיה בדין להחמיר ואין מחמירין היינו לשון דיין.\par וי"מ דיין שתחמיר עליהם לטמאן שעתן למפרע (מהו) [מיהא] שיעור וסת שהכל מודים שטמאה ואף לשרוף כדלקמן (דף יב, א) ואינו נכון. 
\textbf{הלל אומר מפקידה לפקידה ואפילו לימים הרבה.} פי' אפילו מפקידה לפקידה חוששין להם דמפקידה לפקידה חששא דרבנן היא וספק טומאה גזרו עליו לתלות בתרומה וקדשים כדאיתא בגמרא ואמרינן נמי לקמן בדבר שיש בו דעת לישאל. וממילא נמי שמעינן דדוקא ברשות היחיד אבל ברשות הרבים טהור דלא גזרו אלא דלהוי כספק טומאה דברשות היחיד ובדבר שאין בו דעת לישאל ואפילו ברה"י ספקו טהור וברה"י ודבר שיש בו דעת לישאל נמי לא גזרו אלא לתלות אבל לא טומאה ודאי דחששא בעלמא הוא ולחומרא כטעמא דמפרש בגמרא.\par וי"מ דמעת לעת שבנדה נוהג בין ברה"י בין ברה"ר גזרו לתלות בתרומה וקדשים וכן אמרו בירוש' דמעת לעת שבנדה נוהג בין ברשות היחיד בין ברשות הרבים והיינו נמי דאקשינן הכא בשמעתין להלל קשיא טומאה ודאי דאלו מעת לעת שבנדה תולין ולא אקשינן נמי ברשות הרבים טהור. ואלו הכא קתני בין ברה"י בין ברה"ר טמא והטעם לזה שלא הלכו בגזרה דלמפרע על דרך טומאה דסוטה. דטומאת סוטה מכאן ולהבא היא הילכך השוו רשויות לתלות ולא לשרוף. ואין זה מחוור כלל. 
גמרא \textbf{מאי טעמא דשמאי.} פרושי קא מפרש מתניתין ואזיל דאלו טעמיה דשמאי דכולי עלמא אית להו אלא שהחמירו לתלות בתרומה וקדשים כדפרישית, ואוקים משום דהעמד אשה על חזקתה ובחזקת טהורה עומדת שהרי טבלה לנדתה ובדוקה היא משעה שפסקה לנדתה הראשון.\par והלל כי אמרי העמד דבר על חזקתו ואפילו לתרומה וקדשים היכא דלית ליה ריעותא מגופיה הך אשה כיון דמגופה קא חזיא לא אמרי' אוקמה אחזקה אלא חיישינן הילכך בחולין אף על גב דאיכא למיחש במילתא כיון דלא מכרעא מילתא דטומאה מוקמינן טהרות אחזקתייהו ג) ד"א מאפישי טומאה לא מפשינן וגבי תרומה וקדשים עבד בהו רבנן מעלה וכיון דליכא חזקה גמורה תולין.\par וי"מ דאף על גב דליכא חזקה מיהו ה"ל ספק טומאה ומסוטה גמרינן מה התם מכאן ולהבא ולא למפרע אף כל ספק טומאה לא מטמינן למפרע אלא משום מעלה דקדשים דתולין וברשות הרבים טהור לגמרי דגמרינן מסוטה בק"ו. 
ואקשינן \textbf{מ"ש ממקוה לשמאי קשיא למפרע ולהלל קשיא טומאת ודאי דאלו מעת לעת שבנדה תולין וכו'.} פי' וכיון דתולין אלמא ספיקא בעלמא הוא וה"ה דקשיא ברשות הרבים ודבר שאין בו דעת לישאל נמי אלא חדא מספיקא נקט משום דתניא לקמן בהדיא וזה וזה תולין.\par ואי קשיא לך מ"ש אשה מהא דתנן לקמן נגע בא' בלילה וכו' שחכמים מטמאים טומאה ודאי שכל הטומאות כשעת מציאתן וכאן נמי הרי דם לפניך. לא קשיא דשאני אשה דבחזקת טהרה עומדת שהרי בדוקה היא ואף על גב דשכיחי בה דמים מכל מקום כל שהפסיקה וטהרה בחזקתה זו היא אבל אדם זה אינו עומד בחזקת חי תדע דתניא בתוספתא ומודים חכמים לר"מ כשראוהו חי אלמא דבכה"ג בחזקת חי הוא לא מפקינן ליה מחזקתיה אף על פי שנמצא מת בשעת מציאתן ובמקום מציאתן אבל א"ל גבי אשה כיון דספק ביאה הוא ספק הוה ספק לא הוה תולין להקל. ולאו מילתא היא דהא קופה באותה זוית עצמה טהרות הראשונו' (טהורות) [טמאות] לדברי הכל. ואע"פ שהוא דומה לאשה בזה דספק הוה ספק לא הוה הוא אלא משום חזקה ראשונה היא דליתא בקופה וכדבעינא למימר קמן. מ"ש ממקוה לשמאי קשיא למפרע ולהלל קשיא טומאת ודאי דאלו מעת לעת שבנדה תולין וכו'. פי' וכיון דתולין אלמא ספיקא בעלמא הוא וה"ה דקשיא ברשות הרבים ודבר שאין בו דעת לישאל נמי אלא חדא מספיקא נקט משום דתניא לקמן בהדיא וזה וזה תולין.\par ואי קשיא לך מ"ש אשה מהא דתנן לקמן נגע בא' בלילה וכו' שחכמים מטמאים טומאה ודאי שכל הטומאות כשעת מציאתן וכאן נמי הרי דם לפניך. לא קשיא דשאני אשה דבחזקת טהרה עומדת שהרי בדוקה היא ואף על גב דשכיחי בה דמים מכל מקום כל שהפסיקה וטהרה בחזקתה זו היא אבל אדם זה אינו עומד בחזקת חי תדע דתניא בתוספתא ומודים חכמים לר"מ כשראוהו חי אלמא דבכה"ג בחזקת חי הוא לא מפקינן ליה מחזקתיה אף על פי שנמצא מת בשעת מציאתן ובמקום מציאתן אבל א"ל גבי אשה כיון דספק ביאה הוא ספק הוה ספק לא הוה תולין להקל. ולאו מילתא היא דהא קופה באותה זוית עצמה טהרות הראשונו' (טהורות) [טמאות] לדברי הכל. ואע"פ שהוא דומה לאשה בזה דספק הוה ספק לא הוה הוא אלא משום חזקה ראשונה היא דליתא בקופה וכדבעינא למימר קמן. 
\clearpage}

\newsection{דף ג}
\twocol{\textbf{ושניהם לא למדו אלא מסוטה.} פי' רבינו תם ז"ל לא למדוה אלא ממה שהן מחלקין בין זו לסוטה דלתנא קמא התם ברשות היחיד ובשיש בו דעת לישאול והכא אפילו ברשות הרבים דאיכא תרתי לריעותא כודאי טומאה הוא ולר"ש התם איכא רגלים הכא ליכא רגלים ואפילו ברשות היחיד תולין כלומר רבנן מחלקים להחמיר ור"ש להקל אבל למגמר מסוטה ממש ליכא דגמר דהא לא דמו ויפה פירש.\par ועדיין קשה לעיקר שמועה עצמה אי סוטה משום רגלים לדבר הוא כל ספק טומאה ברשות היחיד דטמא נימא שאני סוטה דאיכא רגלים לדבר ובתוס' אמרו דלר"ש כל ספק טומאה ברשות היחיד ספיקא משוי ליה ותולין, ואינו כלום דאטו נימא כולי תנויין דלא כר"ש בכל מקום ספק טומאה ברשות היחיד ספיקו טמא לגמרי קאמרינן, ושמא יש לומר דלר"ש כל שקדמה בדיקה לספק טהור דאמרינן בדיקה ראשונה מהניא עד שעת מציאה דספק. אבל ספקות דעלמא ספק נגע ספק לא נגע ספק הוה ספק לא הוה ספיקו טמא ולישנא דגמרא דקאמר רגלים לדבר לומר דאף על גב דאשה כמי שקדם לה בדיק' היא שאין אשה מזנה והיא אומרת טהורה אני והיה לנו להאמין אותה כמו שמאמינן אותה בטבילה לנדתה וכן כיוצא בהן אפילו הכי משום רגלים לדבר טמאה הכתוב וכל שכן בשאר ספיקות אבל במקום שקדם בדיקה אין מטמאין אלא ברגלים לדבר. 
והא דאמרינן בטעמיה דר"ש \textbf{דגמר סוף טומאה מתחלת טומאה} ולא גמר רשות היחיד דסוף טומאה מתחלת טומאה משום דסבירא ליה דרשות היחיד גמרינן מסוטה בתחלת טומאה וגזרת הכתוב הוא רשות הרבים דינא הוא וגמרינן מיניה דלא אורועי ולטמויי מידי מספיקא. 
\textbf{ואב"א היינו טעמיה דשמאי הואיל ומרגשת בעצמה.} הקשו בתוספות רבותינו הצרפתים ז"ל הואיל אם בדקה עצמה עכשיו ומצאה טמא היאך תאמר מרגשת היתה והלא לא הרגישה עכשיו. ואם תאמר נתלה להקל ונאמר בבדיקה אירע לה אורח וכסבורה הרגשת עד הוא וכפירש"י דא"כ אם שמשה מאתמול נמי נימא הא דלא ארגישה מאתמול כסבורה הרגשי שמש הוא כדאיתמר נמי התם.\par והם פי' דטעמא דשמאי משום דכיון דרוב פעמים מרגש' לא חיישינן למיעוטא ולא גזרו חכמים מחמתן כלל דהא ללישנ' בתרא דאמרי' משום דאם איתא דהוה דם מעיקרא אתא שוכבת במטה ולא נתהפכה מא"ל אלא משום דבר שאין דרכו בכך לא החמירו לאסור מעת לעת ומיהו שוטה ומשמשת במוך כיון דלעולם כך דינן של אלו לחוש להן לפי שאין בהם הרגשה לעולם לפיכך שמאי מודה בהן ופי' טעמא דשמאי באשה מרגשת לומר דהואיל ואשה מרגשת בעצמה לכך לא חשש שמאי כלל ולא עשה סיג לדבריו. והלל אומר הרגשת מי רגלים היא אלא העמד האשה על ספיק' ודינה לתלות כדפרישי' ללישנא קמא. 
\textbf{והאיכא שוטה מודה שמאי בשוטה.} פירש למאן מודה להלל דהוא בר פלוגתיה והיינו לתלות בתרומה וקדשים ואטעמא דלישנא קמא סמכינן בדהלל דהא ליכא למימר דטעמא דהלל משום דהרגשת מי רגלים חששא היא בעלמא ולפיכך אמר תולין דאם כן אף הלל מודה בשוטה דשורפין ואפילו ברשות הרבים וכן ללישנא בתרא במוך ואנן לא אשכחן במעת לעת אלא תולין, אלא הני לישני בטעמא דשמאי נינהו אבל הלל טעמיה משום דגריעי חזקה דאשה וכיון דדם לפניך עבוד רבנן מעלה בקדשים כדפרישית לעיל, וכן הא דאמרינן נימא תנן כתמים דלא כשמאי, אי הוו להו כתמים כמעת לעת דלא כהלל נמי הוו דהא כתמים לבעלה ולחולין ומעת לעת דוקא לקדשים אלא חזקה באשה הוא דמטהר מעת לעת בחולין להלל כלישנא קמא דגמרא והא דאמרינן נמי להך לישנא ליכא למירמי חבית ומקוה ומבוי הכי נמי פירושי' לשמאי ליכא לאקשויי למפרע אבל להלל בטומאת ודאי לא קושיא היא שהדין נותן כיון דאיכא תרתי לריעותא טומאה ודאי במקום דאיכא חדא ריעותא תולין. וכן שוטה ומשמשת במוך לשמאי אינן אלא תולות כשאר נשים להלל ולא קשיא להו טומאה ודאי דחבית ומקוה ומבוי משום האי טעמא דפרישית ולמה שפירש לעיל דכל ספק טומאה למפרע טהור מסוטה איכא לפרושי דבין שמאי ובין להלל בהאי טעמא בלחוד פליגי מר סבר אשה מרגשת לעצמה ואפילו לתלות אין תולין, ומר סבר אימר הרגשת מי רגלים הוא. והוה ליה ספיקא וכל ספק למפרע טהור מן התורה ומדבריהם החמירו בקדשים לתלות והחמירו בכתמים אפילו לחולין. 
 הא דאמרינן \textbf{מודה שמאי בשוטה.} הוקשה בתוספות דבר שאין בו דעת לישאל היא ואין במעת לעת טומאה כשאין בו דעת לישאל ואומרים שיש לומר שאם נגע בה אדם תולין בו דהא יש בו דעת לישאל בנוגע אעפ"י שאין דעת לישאל בטומאת'.\par עי"ל דאיכא שוטה שיודעת לישאל אם נגעה אם לאו ואין לה הרגש' בראית דמים ואנן שוטה סתם פרכינן לעשותה כשאר הטמאות כשיש בענין דעת לישאל וכשאין בו. 
\textbf{והאיכא כתמים לימא תנן כתמים דלא כשמאי.} פירש"י ז"ל האיכא כתמים דקי"ל דמטמא' למפרע וכו' ולאו דוקא פירכא משום למפרע דהאיכא ר"ש בן אלעזר דאמר בסוף בא סימן דכתם אינו מטמא למפרע כלל שלא יהא כתמה חמור מראיתה אלא פירכא משום מכאן ולהבא היא דלא אשכחן תנא דמטהר בכתמים להבא ולשמאי דאמר אשה מרגש' ולא מחמירין עליה אפילו בדרבנן טהורה היא בכתמי' לגמרי דלא מגופה הוא הואיל ולא ארגישה ומפרקינן מודה שמאי בכתמים אפילו בלמפרע כרבנן מאי טעמא וכו'. 
\textbf{משמשת במוך מא"ל.} פירש"י ז"ל ג' נשים ולא דוקא דהא ארבע נשים דיין שעתן במתניתין אלא כל אשה שמשמשת במוך. 
\textbf{מאי איכא בין הנך לישנא להנך לישני.} איכא למידק והאיכא שוטה ומשמשת במוך ללישנא קמא כולן לשמאי דיין שעתן ולא ללשונות הללו. וא"ל אה"נ אלא האי טעמא עדיפא ליה.\par ואפשר לפרש מאי איכא בין האי לישנא להנך לישני כדאמר מ"ט משנינן הני לישני בתראי ולא תפסינן לישנא קמא דאתא במתני' כפשטא כל הנשים כולן דיין שעתן ואפילו שוטה ומוך ופריק משום דאיכא למירמי ללישנא קמא חבית מקוה ומבוי ולהני לישנא ליכא למירמי ותו בעי ומאן דתני האי לישנא בתרא מ"ט לא אמר לישנא קמא ופריק משום דאיכא מוך שהדין נותן דשמאי מודה במוך. 
\textbf{אי אתה מודה בקופה שנשתמשו בה טהרות וכו'.} איכא למידק להלל גופיה קשיא טומאה ודאי דאלו קופה טומאה ודאי דקתני טמאות וכדמוכחא נמי שמעתין לקמן ואלו מעל"ע תולין. א"ל להלל גופיה זו יש לה שולים וזו אין לה שולים אלא ה"ק ליה כיון דכשיש לה שולי' טמאות ודאי דין הוא לתלות באשה מפני שהיא כמי שיש לה אוגנים. ואפילו לחזקיה דאמר התם טהורות שאני פירי דלא שרקי וקפיד עלייהו.\par וכן אתה מפרש ללשון שאמרו בקופה שאינה בדוקה א"נ שהיא מכוסה שלא בא הלל להשוות אשה לשאינה בדוקה ולשאילה מכוסה אלא שמאחר שבאלו טמאות ודאי באשה היה לנו לתלות מפני שהיא דומה במקצת לשאינה בדוקה ואינה מכוסה משום דשכיחי בה דמים. 
\clearpage}

\newsection{דף ד}
\twocol{והא דאמרינן \textbf{ואב"א כי מודו שמאי והלל בזויות דקופה.} ה"פ: לעולם בשאינה בדוקה מודו ובזויות קופה דאיכא תרתי לריעותא אבל בבדוקה אע"פ שבזויות קופה נמצא אינן טמאות דה"ל כאותה שאמרו בתוספתא ומודים חכמים לר"מ בשראוהו חי ועד כאן נמי לא מטמינן בשרץ שנמצא במבוי אלא משום דאיכא שרצים דיליה ושרצים דאתו ליה מעלמא דה"ל כתרתי לריעותא הא לאו הכי לא מטמינן ביה למפרע כלל ל"ש אותה זויות ול"ש זויות אחרת ובקופה לא שכיחי ביה שרצים כלל אלא ודאי בקופה שאינה בדוקה מודו שמאי והלל לכולהו לישני דאי לא א) אמרינן אוקמה אחזקיה ואפי' לתלות ב) וכ"ש דלא מודו בטומאה ודאי וה"נ מוכחא רישא דשמעתין. 
\textbf{שאני אומר אדם טהור נכנס לשם ונטלה.} איכא למידק וליחוש נמי לאדם טמא כדאמרינן בפסחים פ"ק קרדום שאבד בבית או שהניחו בזויות זו ונמצא בזויות אחרת הבית טמא שאני אומר אדם טמא נכנס לשם ונטלו וכ"ש הכא דאיכא למיתלי בתרתי לריעותא באדם טמא ונפלה על גבי מדף. ובפ"ק דשחיטת חולין נמי אמרינן צלוחי' שהניחה מגולה ובא ומצאה מכוסה טמאה שאני אומר אדם טמא נכנס לשם וכסה.\par ויש מתרצים דכיון דאיכא טומאת מדף תחתיה נראין הדברים שאין אדם נוטלה מלמעלה והניחה למטה אלא כדי שלא תפול ותטמא עשה כן ואלמלא שהוא טהור איך הוא מטמא בידים, ואין זה לשון מחוור.\par אבל ר"ת ז"ל פירש בספר הישר דהתם כלים נינהו וספק כלים הנמצאים טמאים כדאמרינן בפ"ק דשבת על ששה ספקות שורפין את התרומה על ספק כלים הנמצאים דשוינהו רבנן לרובא דעלמא טמאים לגבי כלים הנמצאים אבל ספק אוכלים לא קא חשיב אלמא לא גזרו בהו והוו להו רובא דעלמא טהורים לגבייהו הילכך ליכא למיחש גבי מדף אלא לאדם טהור ולנפילה והוה לו ספק טומאה בדבר שאין בו דעת לישאל וספיקו טהור.\par ואיני יודע למה גזרו על כלים ולא גזרו על אוכלים הנמצאים שאם נאמר מפני שחששו לפסידתן מפני שאין להם טהרה במקוה אף כלי חרס כגון צלוחית דחולין אין להם טהרה במקוה, ושמא שכלים נמצאים בדרך נפילה ובכל מקום ואין אוכלים נמצאים בדרכים לפי שהן נמאסין מ"ה גזרו על הכלים הנופלים אפילו בבית כל שלא ידענו דרך הנחתו וטימאוה אבל אוכלין שאין מצויין אלא בבית ורוב מציאה דבית דרך הנחה היא ובטהורין הוי לא גזרו עליהם שאלו פירות הנמצאין בשוק דרך נפילה לא הוצרכו לגזור עליהן דרובן נפסלין הן מתורת אוכלים. 
\clearpage}

\newsection{דף ה}
\twocol{\textbf{מתוך שמהומה לביתה אין מכניסתו לחורין ולסדקין.} מכאן למד הראב"ד ז"ל שכל לבעלה אינו צריך בדיקת חורין וסדקין שהרי בדיקה זו אינה מועילה לטהרות מפני שאין בה בדיקת חורין וסדקין ואע"פ כן מועלת לגבי בעלה. ועוד הביא ראיה ממה שאמרו ג) דתביעה הרי היא כבדיקה לגבי בעלה.\par והוא ז"ל כתב שיש מי שחולק ואמר דבדיקת זבה בין ביום ההפסקה בין בבדיקת השבעה בעינן בדיקה מעולה כשל טהרות ולא הקלו לגבי בעל אלא בבדיקת המעלות כגון זו שאשה העסוקה בטהרות הצריכו בדיקה זו אף לבעלה מתוך חומר שהחמרת עליה בטהרות אבל בדיקה מעולה הצריכה להן מחמת הבעל עצמו צריכה להיות בדיקה מעולה ועוד שיש להחמיר אפילו בבדיקת מעלות ומה שאמר כאן אינה מכניסתו לא שלא הוצרכה אלא מתוך שמהומה לביתה חוששין לטהרות שמא לא עשתה כהוגן ולא הכניסתן לחורין ולסדקין והוא כעין מה שאמרו בסמוך ניחוש שמא תראה טפת דם ותחפנו שכבת זרע וכו' וכן נראה מדברי רש"י ז"ל דכל בדיקה היינו לחורין וסדקין כשילהי פירקין.\par ולפי הסברא יש להכריע שבדיקת ההפסקה שהיא מעלה אותה מטומאה לטהרה ומוציא אותה מחזקה לחזקה צריכה להיות בדיקה מעולה שאין אחריה עליו ספק, אבל בדיקת השבעה כיון שכבר פסקה להעמידה בחזקתה בבדיקה כל דהו סגיא דהא אפילו לא בדקה כלל אלא בשעת ההפסקה ובסוף שבעה טהורה לקמן בפרק בתרא הילכך לקולא כדברי הראב"ד ז"ל ובעל נפש לא יקל בכך. 
\textbf{השתא מעת לעת ממעטה מפקידה לפקידה מיבעיא.} פי' לדברי חכמים לעולם מפקידה לפקידה זמנה מועט מעת לעת שכבר מיעט מעל"ע ע"י הפקידה וכיון שעד זה ממעט על יד מעת לעת אם לא בדקה מאתמול כ"ש שדינו למעט על יד הפקידה אם בדקה עצמה היום בשעה ראשונה ולרביעית שמשה דכיון שהזמן ביניהם מועט אין לחוש כ"כ בשעת פקידה לומר עם סלוק ידיה ראתה. ופריק סד"א עד זה לא תמעט אלא על יד מעל"ע אע"פ שהחששא שלו יותר קרובה מפני הפסד טהרות הקל. אבל על יד פקידה לא ימעט דליחוש שמא תחפנו שכבת זרע קמ"ל. 
\textbf{טעמא אמטו דדיה שעתה דמטה לא מיטמיא הא מעת לעת מטה נמי מיטמיא.} פי' מיטמיא לעשות אב הטומאה כדין משכב הנדה לטמא אדם ולטמא בגדים והיינו מסייעי ליה לזעירי דאי ס"ד טעמא אמטו דיה שעת' טהורה לגמרי הא מעת לעת טמאה כדין מגעה פשיטא היינו טהרות למה לי למיתנא מטה. וכ"ת הא קמ"ל דאפילו אדם וכלים נמי מיטמו במעת לעת ולא תימא אוכלין ומשקיל בלחוד הוא דמטמא, א"כ ליתני היתה עסוקה בטהרות ונוגעות בכלים מטה למה לי ש"מ לרבותא לדין משכב נדה הוא דקתני, ועוד דהא לא ס"ד דאוכלין ומשקין מטמי' ולא אדם וכלי דהשתא הני דהפסידן מרובה דלית להו טהרה במקוה גזרו בהו רבנן הני דאין הפסד להן מיבעיא אלא ודאי מתני' מסייעי ליה לזעירי.\par ופי' לטמא אדם לטמא בגדים היינו בגדים שהוא לבוש או שהוא תפוס בהן בשעה שהוא נוגע לטומאה דתניא בת"כ מניין לעשות שאר כלים כבגדים ת"ל טמא יכול יטמא אדם וכלי חרס ת"ל בגד בגד הוא מטמא ולא אדם ולא כלי חרס מדקא ממעטינן כלי חרס ומרבינן שאר כלים כבגדים ש"מ דאפילו מה שאינו לבוש בהן הוא מטמא.\par ומיהו דוקא שהוא נוגע בהן בשעת נגיעתו לטומאה דדומיא דבגדים ריבה אותן הכתוב אבל לאחר שפירש מן הטומאה אם יגע בבגדים ואפילו לבשן אינן מטמאן והיינו דאמרינן בפרק קמא דבתרא טומאה בחבורן שאני ותנן באהלות פרק קמא אדם ובגדים מטמאי בזב חומר באדם מבגדים ובגדים מבאדם שהאדם הנוגע בזב מטמא בגדים ואין בגדים הנוגעין בזב מטמאין וכל זה דוקא בחבורין כדפרישית. 
\textbf{מכדי האי מטה דבר שאין בו דעת לישאל הוא.} פירש מדקא מקשינן הכי אלמא ברשות היחיד בלחוד הוא דגזור רבנן במעת לעת בספק טומאה שאלו ברשות הרבים אינו חלוק בין בדבר שיש בו דעת לשאין בו דעת דלעולם ספיקו טהור וכמו שפירשתי במשנה.\par ולפיכך הקשו חכמי הצרפתים היכי תרגימנא בשחברותיה נושאו' אותה אם כן הויין לה אינהו תרתי ואיהי חדא הא תלתא ה"ל רשות הרבים וספיקו טהור דהכי אמרינן בגמרא בריש פרק שני נזירים.\par ויש מי שתירץ אין רשות הרבים אלא בשלשה אנשים אבל נשים אפילו מאה נמי כאחד דמיין ורשות היחיד הוא מאי טעמא דגמרינן מסוטה מה סוטה אין סתירתה אלא באיש אחד אבל ב' אנשים והיא לא סתירה היא שהרי אשה אחת מתייחדת עם ב' אנשים א) אף רשות היחיד בלא שלשה אנשים אבל נשים אפילו עשר נשים אין אדם מתיחד הילכך הויא לה סתירה ורשות היחיד היא, וזה אינו כלום.\par ואחרים העמידוה לזו כשהיא ישנה בכילה במטה וחלקה רשות לעצמה ולי נראה שזו היא טמאה ודאי ואין הספק בה אלא שהיא מטמאה אינה נחשבת בכלל המנין אלא הרי היא כגוף השרץ. 
\textbf{היה מתעטף בטליתו וטהרות וטמאות בצדו טהרות וטמאות למעלה מראשו.} יש מפרשים כגון שהוא וטליתו טהורים וטמאות בצדן שראויין לטמא בגדים כגון משכב ומושב ושאר אבות הטומאות ספק נגע טליתו בטמאות ונטמא ונגע בטהרות ונטמאו או ספק לא נגע הטלית לא בזה ולא בזה ספיקו טהור בין בטהרות שהן שתי ספיקות בין בטלית שאינן אלא ספיקא חד. ומסקנא ברה"י ספיקו טמא בשתיהן שהרי שנינו כל שאתה יכול לרבות ספיקו' וספק ספקו' ברה"י ספקו טמא אבל ברה"ר ספקו טהור אפילו הטלית שאין אלא ספק אחד.\par ואחרים פירשו דאו או קתני היה הוא טהור וטמאות בצדו ולמעלה מראשו או שהיה הוא טמא וטהרות בצדו ולמעלה מראשו וכך פי' רש"י ז"ל. 
והא דקתני \textbf{ואם א"א לו אלא א"כ נגע טמא.} לאו דוקא א"א שא"כ האיך אמר רשב"ג אומרים לו שנה והלא א"א וא"ת רשב"ג ארישא פליג, א"כ ה"ל רבנן לקולא ואיהו לחומרא ואנן איפכא אמרינן לקמן במכילתין דכי אמרי רבנן אין שונין בטהרות לחומרא אבל לקולא שונין אלא ודאי רשב"ג אסיפא פליג דה"ל רבנן לחומרא ולפיכך אמרו אין שונין וא"א לאו דוקא אלא שהדבר קרוב הרבה ליגע ורחוק שלא ליגע ובכיוצא בזה א"א דלאו דוקא לגמרי בפרק כיצד העדים.\par ומה שכתב רש"י ז"ל אין שונין חוששין שמא עכשו נגע ובתחלה לא או חלוף לאו דוקא דא"כ אפילו לחומר' אלא חוששין שמא עכשיו לא נגע ובתחלה נגע ולא חלוף. 
\textbf{ומה כלי חרס המוקף צמיד פתיל וכו'.} הקשו בתוספות ונימא דיו לבא מן הדין להיות כנדון מהיכא מייתית ליה מכלי חרס מה כלי חרס אינו מטמא אדם לטמא בגדים אף משכב ומושב לא יטמא אדם לטמא בגדים. ולאו קושיא היא דאנן הכי קאמרינן ומה כלי חרס שטומאתו מועט' שהוא ניצל באה' המת גזרו על מעת לעת שלו כנד' עצמה משכבות ומושבות שטומאתן מרובה לכ"ש שנעשו מעת לעת שבנדה.\par ועוד הקשו דנימא פכים קטנים יוכיח שטמאים במת ואין מטמאים במעל"ע שבנדה כדאמרינן בבבא קמא ופירש רש"י ז"ל שהוא של חרס ואי אפשר ליגע בתוכן ואעפ"י שאפשר בהיסט להכי אפקיה רחמנא להיסט בלשון נגיעה לומר שכל שאי אפשר להטמאות בנגיעה אינו מטמא בהיסט, גם זו אינה קושיא דמה לפכין קטנים שהן טהורין בנדה עצמה תאמר במשכבו' ומושבות דכיון שמטמאין בנדה עצמה עשו מעת לעת כמוה דאשכח' בכלי חרס מוקף צמיד פתיל כ"ש לדעת הגאונים שהן מפרשים פכין קטנים שאינן ראויין לישיבה וטהורין במדרס הזב אבל מן ההיסט אין לך ניצל מהן ולא ממגע תוך כגון בשערו רוקו ומשקה הזב והזבה. 
\clearpage}

\newsection{דף ו}
\twocol{הא דאקשינן \textbf{א"ה ליתנייה גבי מעלות.} ומפרקינן כי קתני היכא דאית ליה דררא דטומאה היכא דלית ליה דררא דטומאה לא קתני. קשיא עלה והא קתני התם דלית בה דררא דטומאה כדאמר התם בגמרא פרק חומר בקודש (דף כ"א ע"ב) חמש קמייתא דאית להו דררא דטומאה דאורייתא גזרו בהו רבנן בין לקדש בין לחולין שנעשו על טהרות קודש חמש בתרייתא דלית בהו דררא דטומאה מדאורייתא לקדש גזרו בהו רבנן לחולין שנעשו על טהרות הקדש לא גזרו בהו רבנן.\par וי"ל דהתם דאורייתא לית להו אבל דררא דטומאה דרבנן אית להו הכא אפילו דררא דעלמא מדרבנן ליכא דכל שהוא חששא בעלמא למפרע לאו דררא היא כלל אלא כענין קנסא משום דלא בדקה הפסידוה עונה.\par וי"מ דהכא הכי פרכינן ליתנייה גבי מעלות קמייתא דאינון בין לקדש בין לחולין שנעשו על טהרת הקדש דאלו בהדי בתריית' כיון דליתנהו אלא לקדש לא מצי למיתנייה דהא מעת לעת שבנדה לחולין שנעשו על טהרת הקדש נמי איתא כדאמר בשילהי שמעתין ולהכי מפרקינן כי קתני בהנהו היכא דאית ליה דררא דטומאה אבל היכא דלי ליה דררא דטומאה ואפ"ה החמירו בהו לא קתני.\par והלשון משובש הוא לדעתי דההוא דאמרינן בשלהי שמעתין לחולין שנעשו על טהרת הקדש לתרוצא לברייתא דקתני לקדש אבל לא לתרומה איתמר אבל השתא להאי לישנא לקדש ולא לתרומה ולא לחולין שנעשו על על טהרת הקדש קאמרינן דאי לת"ה לא הוו צריכין בדר' לתרוצא כדעולא דהא איכא אוכלין חוליהן בטהרת הקדש בימיו ממש אלא ודאי צ"ל להאי לישנא דאף לחולין שנעשו על טהרת קודש לא גזרו במעל"ע שבנדה לכך צריך לשנוי כדעולא כיון שהיו עושין על טהרת הקרש על מנת שהיו מתנסכין ממש על גבי מזבח היינו קדשי מזבח גמורין. 
הא ד\textbf{אמר רבי חנינא בן גמליאל מתפלל י"ח מפני שצריך לומר הבדלה בחונן הדעת.} איכא דמקשו עלה וניכללה מכלל כדמקשינן בגמרא במסכת ברכות (דף כ"ט) דאמרינן כל השנה כולה מתפלל אדם הביננו חוץ ממוצאי שבתות ומוצאי י"ט שצריך לומר הבדלה בחונן הדעת והוינן בה וניכללה מכלל ואסיקנא בקשיא והכא נמי תיקשי וניכללה מכלל.\par ומתרצין התם כיון דאמר כל השנה כולה מתפלל הביננו מפקע פקיע ליה למכלל כמה דבעי אבל הכא כיון דלא רגיל בהביננו אלא במוצאי יום כפורים בלבד הוא דמתפלל ליה מפני הטורח אי אתי למיכלל ביה מידי אתי למיטעא. עוד י"ל התם והכא קשיא לגמרא ומיהו לא מידחי מימרא בקשיא ואפשר הוה התם למימר ולטעמי' הא דתניא תיקשי לך אלא איכא כמה דוכתי דיכול למימר ולטעמיך ולא אמר. 
 הא דאמר רב ששת בריה דרב אידי \textbf{כי קתני מידי דתלי במעשה.} פירש אליבא דר"ש קסבר האי תנא בוגרת מותרת לכהן גדול א"נ נפקא מינה לכתובה ולא לכהן תניא ומוכת עץ מילתא דתלי במעשה הוא והאי דקתני כל זמן שלא נבעלה לאו דוקא אלא שלא נטלו בתוליה בין בעץ בין באדם. א"נ קסבר מוכת עץ מותרת לכהן גדול וכתובתה מאתים ומחלוקת היא ביבמות ובכתובות. וכן הא דאמר נ"מ לנחל איתן סבר לה כר' יאשיה אשר לא יעבד בו לשעבר ואיתא בפלוגתא בפ' עגלה ערופה. 
\clearpage}

\newsection{דף ט}
\twocol{\textbf{ראתה ואח"כ הוכר עוברה מהו.} יש להקשות והלא כל מדות חכמים כך הם במ' סאה הוא טובל בחסר קרטוב אינו יכול לטבול אף כאן מכיון שנתנו שיעור לדבר בהכרת העובר ראתה ולא הוכר פשיטא שמטמאה מעת לעת אע"פ שאח"כ הוכר בסמוך.\par ויש לפרש ראתה ואח"כ הוכר עוברה בו ביום ששלמו לה שלשה חדשים מי אמרינן כי מצפרא נמי הויא היכירא ואנן הוא דלא בקאינן או דילמא גבול יש לה וא"ל מידי הוא טעמא אלא משום דראשה כבד עליה בעידנא דחזאי אין ראשה כבד עליה בתמיה הילכך דיה שעתא.\par וזה הלשון אינו נכון מפני שהיה להם לפרש ובו ביום הוכר עוברה ולימא נמי כיון דבו ביום חזיא לא מטמיא.\par ויש לפרש אותה כפשוטה ור' ירמיה הכי בעי מיני' גבול שנתנו לה חכמים משיהא ראשה ואיבריה כבדין והיינו משעת הכרת העובר וקרוב לו מלפניו כשהיא מרגשת בעצמה או דילמא גבול שנתנו לה הכרת העובר ממש הוא וקודם לכן אפילו בסמוך אינו מטהרתה וא"ל אף איבריה אינן כבידין עליה אלא משעת הכרת העובר ממש שאין הולד חי ומכביד עליה אלא מזמן זה ואילך וזה הלשון עיקר. 
 גרסת ר"ח ז"ל \textbf{רבא אמר רב חסדא שלשה ועשרים יום ולא פליגי מר וכו'.} ופשוטה היא. והספרים גורסין כגרסת רש"י ז"ל כ' יום ופירש דלא פליגי מר חשיב ימי טומאה ז' ימי נידה וג' ימי זיבה ומכאן ואילך כיון דבעינן נקיים ימי טהרה הם ולפיכך לא מנה אלא עשרים שדבר בסתם נשים שהן טמאות לנדה ולזיבה. 
\textbf{ועוד עברו עליה ג' עונות וראתה דיה שעתה.} י"מ דהא מני רבי היא דאמר בתרי זימנא הויא חזקה והא דבעינן הכא עד תלתא משום שכל שעברו עליה ג' עונות סמוך לזקנתה הוחזקה בזקנה שדמיה מסולקין ואין ראיה זו מוציאה מכלל זקנו' שכן דרך סלוק דמיהן של זקנות מפסקת ורואה ופוסקת ושוב אינה רואה וכשהיא רואה פעם לסוף ג' עונות עכשיו הוא שמתחלת להחזיק עצמה ברוא' ואפילו בזקנתה לפיכך דיה שעתה בראיה זו שהיא תחלת למניין ובשנייה שהיא שלישית הוחזקה ומטמאה מעת לעת ומתני' דקתני במה שאמרו דיה שעתה בראיה ראשונה אבל בראיה שניה מטמאה מעל"ע התם כשקרבה ראיתה בשנייה דאיגלא מילתא דג' עונות קמייתא לאו משום סלוק דמים הוו ולא הגיעה זו לכלל זקנה. אבל רחקה אף ראיה שניה זקנה היא אלא שרואה וצריכה חזקה, וזה לשון נכון.\par ול"נ דזקנה צריכה להחזיק עצמה בראיות והא דקתני מתני' אבל בראיה שנייה מטמאה מעת לעת אבתולה קאי והוא דאיתמר בגמרא עלה רב אמר אכולהו לומר שכולן ישנן בדין הזה שאם החזיקו עצמן בדמים מטמאות מעת לעת, ושמואל אמר ל"ש אלא בתולה וזקנה שישנן בדין הזה שמחזקות עצמן בראיות ומטמאות אח"כ מעת לעת, אבל מעוברת ומניקה דיין שעתן כל ימי מניקותיהן ועוברן ואפילו הן שופעות פעמים הרבה. ומיהו מתני' דקתני דבראיה אחת הוחזקה לדמים ודאי הכל מודים דלאו אזקנה קאי דזקנה הא קתני לה בבתרייתא בתרי זימני אלא בדין הזה להחזיק עצמה כשאר הנשים מעת לעת וזה פירש מדוקדק.\par וי"מ אותה לרשב"ג דבתלתא זימני הוי חזקה ומטמאה מעת לעת כשור המועד מה שור המועד בשלשה זימני אתחזק, ואידך כי נגח משלם אף זו בג' פעמים הוחזקה הילכך מטמאה בשלישי עצמה מעת לעת ולהאי פי' אמרינן דמתני' דקתני אבל בראיה שני' מטמאה מעת לעת סתמא כרבי דהא זקנה בכללה לדברי הכל ואינו מחוור. ועוד דשטתא דוסתו' לא מוקמינן להו כרבי דלקמן תנן סתמא כרשב"ג וביבמות אמרינן וסתות ושור המועד כרשב"ג.\par ולכל הלשונות נמי קשיא כיון שהוחזקה זו ולבסוף דלאו סלוק דמים הוה בה כלל נטמא למפרע מעת לעת שבכל ראיותיה והשיב רש"י מעת לעת דרבנן הוא וכל שבשעת ראיתה בחזקת טהרה אין מחמירין לטמא אותה למפרע. ואם תשאל הרי הקשו למעלה גבי היתה בחזקת מעוברת וראתה אמאי [אין] מטמאין אותה למפרע לכשהפילה רוח וזו אינה קושיא דהאיכא רב פפא דתריץ הכי הנח מעת לעת לרבנן ולרב פפא שאני התם דאיגלאי מילתא דלאו עובר הוא אבל הכא אכתי איכא למיתלי מעיקרא לא שכיחי בה דמים והשתא הוא דאכחיש' ואיתרעי א"נ התם בסמוך לראיה הפילה דאפשר לטמויה אבל לאחר עונות ליכא למימר הכי. 
 הא דתניא \textbf{תנוקות שלא הגיע זמנה לראות וכו'.} יפה פי' רש"י ז"ל דהיינו טעמא דלא מטמיא מעת לעת אלא בשלישית לרבי וברביעית לרשב"ג מפני שכל אשה שלא הוחזקה כבר ברואה אינה מטמאה מעת לעת דטעמא דמעת לעת כעין קנסא דרבנן הוא כדאמרו חכמים תקנו להן לבנות ישראל שיהיו בודקות עצמן שחרית וערבית וזו הואיל ולא בדקה הפסידה עונה יתירה הילכך כל שאינה צריכה לבדוק עצמה כלל אינה בכלל מעת לעת שבנדה ותנוקות שלא הגיעה זמנה לראות הרי הן בחזקת טהרה כדתניא לקמן ואין הנשים בודקות אותן הילכך פעם ראשונה ושניה שעדיין לא הוחזקה לראות ולא היתה בכלל תקנה לבדוק שחרית וערבית דיה שעתה. וכיון שראתה בשניה הוחזק ברואה לדברי רבי הילכך בג' מטמאה מעת לעת שהרי היתה צריכה לבדוק שחרית וערבית, ואע"פ שעדיין לא הגיע לכלל שנותיה, וכיון שלא בדקה הפסידה עונה יתירה ושהגיע זמנה לראות כיון שצריכה בדין היה לגזור עליה אפילו בראשונה אלא שהיא קולא לדבריהם. עברו עליה ג' עונות חזרה לכלל תנוקות שלא הגיע זמנה עד שתראה שתים ותהא מוחזקת לראות לדברי רבי דשוב צריכה בדיקה ומטמאה מעת לעת.\par והא דאמרינן לקמן (דף י' ע"א) בין שניה לשלישית כיון דלא אתחזק בדם כתמה נמי לא מטמינן לאו אליבא דהך ברייתא דרבי אלא אליבא דהילכתא כרשב"ג. וכן פסק הר"ם ז"ל דקטנה כתמה טהור עד שתראה דם ג' וסתות.\par וחזקיה סבר כיון (דחזיא) [דאלו חזיא] הרי היא כשאר כל הנשים אח"כ כתמה מחזיקה וטמא דבכל (מראיה שניה) [מראות משניה] ואילך הוחזקה.\par אבל רש"י ז"ל פי' אליבא דברייתא [דרבי] ומאי כיון דלא איתחזק בדם שעדיין לא הוחזק' בה לטמא מעת לעת ולפי דבריו ז"ל ולדידן דקי"ל כרשב"ג אין כתמה טמא עד שיעברו עליה ד' וסתות. לראיה פעם ראשונה [דרב גידל] פירש רש"י ז"ל דהיינו [ראשונה שאחר ההפסקה ושניה היינו] ראיה שניה של דלוג שהיא ראשונה לראיה דעונות. ולפי שהיא עומדת בה כשרואה עכשיו בעונו' קרי לה הכי.\par ויש לפרש "הדר קחזיא בעונות" [דקאי גם על] פעם אחרת בין דלוג ראשון ושני ראתה פעם אחת בעונה ובין שני לשלישי חזרה וראתה עוד בעונה ותרתי בעיי אהדדי איתמר ורב אשי בעא [לאפסןקי ולמפשט חדא חדא] מיניה דסד"א כיון דראתה שתים בדילוגו ושתים בעונות סלוק דמים הוא דקא מנע מינה עונות ולא תהא מוחזקת לא לדלוג ולא לעונות דכיון דשנתה כ"כ אונס בעלמא הוא. ואמר רב גידל פעם ראשונה של עונות ט) דיה שעתא כדאמרן שניה של עונות י) כיון דראיה שלישית הוא יא) מכי חזיא בעונות ואילך לעולם מטמאה מעת לעת. ומיהו בראיה (ג') [ראשונה] שלה לא מטמיא מעל"ת משום לסוף ג' עונות חזיא ואכתי לא הוחזקה ג' פעמים להפסקה דהא אנן לר' אלעזר קאמרינן ולישנא דהדרא קא חזיא דייקא כדאמרן. 
\clearpage}

\newsection{דף יא}
\twocol{\textbf{אלא לקפיצות והתניא.} פי' רש"י ז"ל דכל יום שתקפוץ מחזיקין לה רואה ואפילו בשאר ימות השנה ואין פירו' זה נכון דאי מעיקרא קס"ד דלקפיצות לחודייהו תקבע וסת הכי הוה לן לתרוצי בברייתא לא קבעה לה וסת לימים אבל קבעה לה וסת לקפיצות לחודייהו דמאי דקא אמרינן מעיקרא משנינן ועוד כי מקשינן לימים לחודייהו פשיטא לימא ליה לקפיצות לחודייהו קמ"ל.\par אלא ה"פ אלא לקפיצות בימים קבעה והתניא אינה קובעת ומתרץ אינה קובעת לימים ולא לקפיצות לחודייהו פשיטא היא כדפרישית. 
 הכי אשכחן בנוסחי: \textbf{לימים לחודייהו פשיטא אמר רב אשי כגון דקפץ בחד בשבת וחזאי וקפץ בחד בשבא וחזא ולשבתא נמי קפצה ולא חזאי מהו דתימא איגלי מילתא דיומא הוא דגרים קמ"ל דקפיצה נמי גרמא ומשום דאכתי לא מטאי זמן קפיצה.} וק"ל כיון דלימים לחודייהו פשיטא ליה וה"ה לקפיצות לחודייהו נמי דפשיטא ליה דלא קבעה כדפרישית קפצה בשבא ולא חזאי מאי מהני לן פשיטא ודאי דלא תיחזי אלא בימים וקפיצה ועוד מאי קא מקשי' ומאי קא אתי רב אשי לחדותי הא מימר קאמרינן בברייתא דלא קבעה וסתות לקפיצות לחודייהו. ואי תקפוץ בשבת לא תיחזי והלכך איצטריכא ליה לאשמועינן ימים לפום מאי דקא משנינן ואדרבא פשיטא דבעיא ימים וקפיצה.\par ונראה שרש"י ז"ל גורס ולשבתא קפצא ולא חזיא ולמחר חזאי בלא קפיצה ומהו דתימא איגלאי מילתא דיומא הוא דגרים ולא קפיצה דהא בקפיצה בלא יום לא חזאי וביום בלא קפיצה חזאי קמ"ל דקפיצה דאתמול גרמא לראיה דהאידנא ומשום דאכתי לא מטאי זמן קפיצה לא חזא מאתמול וזהו הנכון.\par ומיהו לישנא אחרינא אמרי לה להא דרב הונא ולא ידעינן אי פליגן לישני ולמדחי' לקמא איתמר בתרא או דילמא אע"ג דלאו הכי איתמר אלא האי תרווייהו איתנהו לענין מעשה ומסתברא כיון דוסתו' דרבנן לקולא נקטינן בהו והלכתא כתרי לישני ולקולא, ואחר שכתבתי זה מצאתי להרמב"ם פאסי ז"ל שהחמיר ובטלה דעתינו מפני דעתו. 
 מתניתין \textbf{צריכה להיות בודקת וכו' ומשמשת בעדים וכו'.} פירש מתני' פרושי קא מפרש לה ואזיל וכיצד קתני כיצד צריכה להיות בודקת פעמים ביום שחרית וערבית ואע"פ שלא שמשה כלל וכיצד משמשת בעדים בודקת נמי בשעה שהיא עוברת משאר עסקיה לשמש את ביתה ומשמשת בעדים וע"כ מדקתני בשעה שהיא עוברת לשמש היינו עד שלפני תשמיש וש"מ דצריכה בדיקה לפני תשמיש והעד (הג') [הב'] לפני תשמיש אי אפשר אלא לאחר תשמיש הוא וכדתנן אחד לו ואחד לה אלמא צריכה בדיקה בין לפני תשמיש בין לאחר תשמיש.\par והיינו דאמרינן לעיל [דף ה' ע"א] שתי בדיקות אצרכוה רבנן חדא לפני תשמיש וחדא לאחר תשמיש ורמינ' למתני' דקתני והמשמשת בעדים הרי זו כפקידה דהיינו עדים דקתני דאינון לפני תשמיש ולאחר תשמיש דומיא דמשמשת בעדים דהך סיפא [וכי תריץ] נמי לעיל גבי רישא דמתני' מעיקרא [אידי ואידי לאחר תשמיש] משום דקשיא להו קס"ד לפרושי ההיא דשני עדים דבסוף קא חשיב אבל בהך סיפא דכ"ע עד שלפני תשמיש קתני וכדמפרש עלה לקמן בגמרא. 
\textbf{מדאמרינן הכא מימי טהרה לימי טומאה לא קבעא.} נ"ל דאשה קובעת וסת בימי מניקתה דלא ממעטינן הכא אלא ימי טוהר דידה, וה"ר אברהם ז"ל היה אומר שאין קובעת לא בימי עוברה ולא בימי מניקתה ואי מוקמת שמעתין במפלת שאין לה חלב אכתי קשיא דמידי הוא טעמא אלא משום דראשה כבד עליה וכו' כל היולדות בכלל הן. 
\textbf{דמגו דבעיא בדיקה לטהרות בעיא נמי בדיקה לבעלה.} פירש אע"פ שתקנו חכמים לבנות ישראל לבדוק שחרית וערבית להכשיר הטהרות עוד החמירו עליהן שאם שמשו מטתן יהו צריכות בדיקה לטהרות חוששין שמא ראתה מחמת תשמיש ובדיקה שניה שהצרכוה סמוך לתשמיש מאחריו בתוך שיעור כדי שתרד מן המטה א"נ באחר אחר כפירקן דכל היד קודם שתלך או שתקנח וכן הצריכו לאיש עצמו לקנח בעד ולבדוק והכל משום חומר הטהרות שרגילה לעסוק בהן ומתוך חומר הטהרות החמירו לבעלה שתהא צריכה לבדוק לפני התשמיש וכ"ש לאחר תשמיש ואע"פ שאין דעתה לעסוק בטהרות עכשיו מאחר שהורגלה לעסוק בהן והוחזקה להיות בודקת לטהרות לאחר תשמיש.\par נמצא שאין כאן בדיקה מן הדין אלא שלאחר תשמיש ולטהרות והשאר מדין מגו וכיון שאינו אלא משום טהרות אינה צריכה אלא עדותו של עד כלומר שמקנחת בעד לפני תשמיש ולמחר בודקת בו אם מצאה עליו דם טמאה לטהרות ומחייבת בעלה בחטאת. אבל מותר הוא לבעול משבדקה בעד ואע"פ שאינו מועיל לו עכשיו שהרי אינן יודעין אם ראתה אם לאו זהו דרכו של פי' רש"י ז"ל.\par וי"ל דכל גבי בעלה צריכה להיות בודקת ורואה ואח"כ תשמש דאין בדיקה סתם בכל מקום אלא במקנחת ורואה מה העד מעיד בדבר דאי לא תימא הכי כל הנשים בחזקת טהרה לבעליהן הן ואפילו בעסוקה בטהרות אלא משמע דכל לפני תשמיש בודקת ורואה לגבי בעלה והכל ודאי משום מגו דטהרות והיינו דאמרי (לעיל) [לקמן ע"ב] אימר שמש עכרו, ז) וכיון שאינה צריכה בדיקה לאחר תשמיש לטהרות אף לפני תשמיש לבעלה אינה צריכה דהא ליכא מגו כך פי' הרב ז"ל. 
\textbf{חדא מכלל דחברתא איתמר.} נראה דהא דרב יהודה איתמר מכללא דההוא דכיון דשמעיה רבה בר ירמיה לשמואל דאמר אשה אין לה וסת אסורה לשמש עד שתבדוק בעסוקה בטהרות וש"מ נמי דכל לבעלה לא בעיא בדיקה כלל מדלא מתרץ לה לר' זירא אין לה וסת אפילו לבעלה בעיא בדיקה מ"ה דייק רב יהודה דשמעא מיניה ואמרה משמיה דשמואל דמתני' דוקא בעסוקה בטהרות היא דמכלל היא דמתני' ליכא למימר [דוקא ביש לה וסת ולא] אין לה וסת ודוקא עירה כדאיתמר לקמן.\par וי"מ הא דאמרינן חדא מכלל דחברתא איתמר לאו אעיקר מימרא דשמואל אלא ה"ק שמואל שמעתא דהכא לא שאנו אלא בעסוקה בטהרות, ואמ' תו במימרא דלקמן דרבא בר' ירמיה ומכללא דהך אוקימנא לההוא שעסוקה בטהרות ולאשמעינן עירה וישנה אתמר ואי לא איתמר הך לא הוה ידעינן ההיא דבעסוקה בטהרות היא כדאמר רבא לקמן וכי אמרינן חדא מכלל חברתה איתמר אאוקמתין דעסוקה בטהרות קאמר וכי אמרינן [ואוקימנא אאוקימתא דרבה בר ירמיה] קאמרינן דאוקמתין קשי' ואוקמתין מתרצין. 
הא דאמרינן \textbf{תנ"ה בד"א לטהרות אבל לבעלה מותרת וכו'.} היינו טעמא דמשמע לן אפילו כשאין לה וסת משום דמתניתין בין שיש לה וסת וכו' ועלה קתני בד"א לטהרות אבל לבעלה מותר בין בזו בין בזו משמע ועוד דכל עיקר לא הוצרכה ברייתא זו לשנותה אלא בשאין לה וסת שאלו בשיש לה מתני' היא כל הנשים בחזקת טהורות לבעליהן. 
\clearpage}

\newsection{דף יב}
\twocol{ והא ד\textbf{בעיא מיניה ר' זירא מר' יהודה מהו שתבדוק ותבדוק ומה בכך.} לפני תשמיש קאמר ולהחמיר על עצמו היה רוצה שאע"פ שאינה עסוקה בטהרות יהא נוהג חומר כעסוקה בהן ואמר לאו שא"כ לבו נוקפו ופורש, ור' אבא בעיא מרב הונא לאחר תשמיש ולהחמיר שלא מן הדין בשאינה עסוקה ואיהו נמי [אלא דלא] מבעיא שלא יהא לבו נוקפו ופורש.\par ופיר' הענין שהיו השואלין סבורין שבדיקות הללו של טהרות יש לחוש לספיקן וזה שלא הצריכוה חכמים אלא לטהרות קולא היא לגבי הבעל שלא להאריך עליהן את הדרך ומי שמחמיר על עצמו נקרא צנוע וכשר ופשטו להן שבדיקות של טהרות חומר הוא שתקנו בהן חכמים כדי שתהא היד מרבה לבדוק בנשים ומשובחת ולא שיהא מקום ספק לחוש להן ואף הרוצה להחמיר אין רוח חכמים נוחה הימנו מפני שלבו נוקפו ופורש ומבטל פריה ורביה בישראל.\par ולדברי רש"י ז"ל לבו נוקפו בבדיקה שלפני תשמיש לפי שהיא בודקת ואינן יודעין מה מצאו עד למחר ונמצא בועל על הספק ולבדוק ולראות לא עלה על דעתו שאפילו בטהרות לא אמרו כן ושלאחר תשמיש נמי לבו נוקפו בו ופורש.\par ולפי דברינו [במתניתין] אפי' (בבדיקות) [בבודקות] ורואה לפני תשמיש כיון שאין אתה מחזיקה בטהורה לבו נוקפו בחששות ובחומר בדיקת חורין וסדקין ואינו סומך בבדיקת אור הנר וכל זה וכיוצא בו גורמין פרישה הן.\par ובשם ר"ח ז"ל מצאתי שאומרים דמ"ה לבו נוקפו דסבור אלמלא שלא הרגישה לא בדקה כלומר דהא קים לה דכולה לבעלה לא בעי בדיקה, וזה אינו סבור שמחמרת אלא שבודקת משום הרגשה וחוששין כיון שהרגישה ודאי בא אורח וטפה כחרדל היתה ואבדה בעד. 
\textbf{א"ר אמי א"ר ינאי וזהו עדן של צנועות.} פי' ודאי משנתינו עד שלפני תשמיש ולאחר תשמיש קתני ובעסוקה בטהרות וא"ר ינאי עד זה שלפני תשמיש זהו עדן של צנועות דקתני מתני' בפרק כל היד והקשו לרבי אמי הא מתניתין צריכות קתני כדתנן צריכה להיות בודקת ומשמשות בעדים ופריק שאני אומר כל המקיים דברי חכמים נקרא צנוע.\par ולדבריו של ר' אמי פיר' משנתינו שבפרק כל היד כך הוא דרך בנות ישראל שתקנו להן חכמים להיות משמשות בשני עדים אחד לו ואחד לה לאחר תשמיש ואם לא בדקו או שאבדו עידיהן אסורות לשמש עד שיבדקוהו שמא מחמת תשמיש ראתה והצנועות שמקיימות דברי חכמים מתקינות שלישי אחר לתקן את הבית לבעליהן שכך הצריכו אותן חכמים אלא שלא אסרו להם לשמש אם אבד עד זה או (שאפשר) [שאי אפשר] להן לבדוק לאור הנר דכיון שאין עדות בכל מקום אלא עד של מגו לא החמירו בו כשנאחר תשמיש לשהוחזקה בו לטהרות מן הדין.\par ואקשיה ליה רבא לרב אמי ופרכיה ופריש רבא הא דקאמר ר' ינאי וזהו עדן של צנועות לומר שבעד זה הוא צניעות הצנועות ששנינו במשנתינו שעד שבודקין בו לפני תשמיש זה אין בודקות בו לפני תשמיש אחר, ולדברי רבא פי' משנתינו כך הוא דרך בנות ישראל שתקנו חכמים משמשות בשני עדים אחד לו ואחד לה שלו מקנח בו לאחר תשמיש שהרי אין לו בדיקה אחרת ושלה בודקת בו עצמה כל זמן שהיא צריכה לבדוק דהיינו לפני תשמיש ולאחר תשמיש כדתנן הכא והצנועות מתקנות להן שלישי אחר לתקן את הבית כלומר חדש ולבן, והיינו דקתני תיקון והיינו נמי לשון שלישי לומר שאם רצו לשמש פעם אחרת למחר מתקינות להן שלישי שלא נשתמש בו כלל אפילו לפני תשמיש.\par ורש"י ז"ל מפרש שני עדים אחד לו ואחד לה לאחר תשמיש, ולפי דבריו הכי מתרצא מתני' והצנועות מתקנות להן השלישי שהן צריכות א' לתקן הבית ומהו תקונן שלא נשתמשו בו ואפילו לפני תשמיש.\par ולדברי הכל משנתינו בעסוקה בטהרו' וכדאוקי שמואל להא מתניתין דפירקן דתרווייהו בני חד ביקתא אינון וכדקתני רישא דההיא ואוכל' בתרומה ועלה קתני דרך בנות ישראל וכו', ולפום הכי קתני סיפא כל הנשים בחזקת טהרה לבעליהן כלומר אע"פ שהצריכוה בדיקה לא אמרן אלא לטהרות אבל לבעלה בחזקת טהרה הן. 
\textbf{והא שמואל במאי מוקים לה.} וא"ת כשאין עסוקה בטהרות ולבעלה ואפילו אין לה וסת א"ב מאי איריא חמרין ופועלין אפילו עומדין בעיר נמי ועוד מאי קמ"ל הא קתני לה אידך לעיל בד"א לטהרות וכל זה אינו מחוור דאיכא למימר ישנות קמ"ל ואכתי נמי לא קים לן דאיכא חלוק בין בא מן הדרך לשוהה בעיר אלא א"ל מדקתני סתמא ש"מ אפילו בעסוקה בטהרות הוא דכולהי סתמי לטהרות ולבעלה פרושי מפרש לה בבמה ד"א א"נ דאינהו בטויי מיבעיא ליה וגמר' מתרץ דעדיף מיניה דקאמר דדוקא נקט חמרין ופועלין ואוקימנא בשיש לה וסת. 
\textbf{וכיון שתבעוה אין לך בדיקה גדולה מזו.} פירש רש"י ז"ל דסתם הבא מן הדרך דרכו לפייס ולרצות ולתבוע וכי מרצו קמה ותבע רמיא אנפשה ואי הוה חזיא מרגשה. וכי אמרינן דבעינן בדיקה בשוהה עמה שאינו צריך ריצוי כ"כ ומיהו הניח בחזקת טומאה אף ריצוי לא מהני ליה עד שישמע מפיה טהורה אני והא דשאל רב כהנא אינשי דביתהו דרבנן לומר אם מחמירן על עצמן לבדוק בשאינן עסוקות בטהרות דומיא דבעיא דר' זירא דלעיל והאי דנקט כי אתו מבי רב אורחיה דמילתיה נקט שיוצאין ובאין מערב שבת לע"ש.\par ויש לפרש דישינות בעיא מנייהו אם מחמירין בהן בבאין מן הדרך משום דכיון דאין בעלה עמה לא קפדה אנפשה ואמרו להן לאו, נמצא כלל השמועה הלכה למעש' שכל לבעלה לא בעי בדיקה לא לפני תשמיש ולא לאחר תשמיש ואפילו כשאין לה וסת לפי פירושו של רש"י ז"ל בדברי רבי חנינא בן אנטיגנוס.\par אבל מדברי הרמב"ם הספרדי ז"ל למדנו שיש לו דרך אחרת בשמועה זו שהוא מפרש זו ששנינו דרך בנות ישראל לבעלה בשאין לה עסק בטהרות והצנועות בודקות אף לפני תשמיש לבעליהן וכל מה שהקל ר' יהודה ור' זירא משמיה דשמואל אינו אלא בבדיקה זו שלפני תשמיש שכשהן עסוקות בטהרות אפילו שאינן צנועות צריכות. וכשאין עסוקות בה הרי הן בחזקת טהרה לבעליהן לפני תשמיש.\par וטעם לדבריו מפני שלפני תשמיש אשה מרגשת בעצמה ואפילו בישינה נמי הקלו מפני שבחזקת טהרה הן ולאחר תשמיש חוששין שמא ראתה מחמת שמש ואינה מרגשת.\par וההיא דבעיא מיניה ר' אבא מרב הונא צריך הוא לפרש שלא מנעו אלא מלבדוק בשיעור וסת ואח"כ כדי שלא תתחייבנו באשם תלוי ויהא לבו נוקפו אבל לאחר אחר בודקת קודם שתלך ותקנח ואף על פי שהוא בודק בשלו לחטאת שאני התם דא"א בבדיקה שלא תחייבנו חטאת והוא צריך בדיקה מ"מ שמא ראתה מחמת תשמיש אבל בשלה אפשר לבדיקה זו לאחר זמן של אשם תלוי שלא יהא לבו נוקפו בביאה זו ותהא מתוקנת בביאה אחרת, וכן דברי ר' זירא לר' יהודה כך הן מתפרשין לי מהו שתבדוק עצמה לבעלה לחייב בעלה שאם לפני תשמיש והלא בצנועות (הוא) [במתני' קתני] לה וזהו שאמר רבי ינאי זו עדן של צנועות לא צנועות שנשנו בפרק כל היד אלא ר' אמי פירש לומר שכל העושה כן נקרא צנועה, ורבא פירש לומר דבעד זה נכרת אם צנועה היא אם לא אבל לדברי הכל מתני' צריכות קתני ובעסוקה בטהרות ואלו בשאינה עסוקה כבר שנינו והצנועות מתקינות וכו'.\par ושאר השמועה פשוטה היא לפי דרכו לפיכך כתב אינה צריכה עד שלפני תשמיש אלא משום צנועות אבל לאחר תשמיש הכל צריכים שני עדים אחד לו ואחד לה אפילו מעוברת ומניקה זקנה וקטנה האריך עלינו את הדרך.\par אבל דברי רש"י ז"ל יותר נכונים ומוכרעים בכמה מקומות בשמועה והחכם יבור לעצמו, ודברי רבינו יצחק אלפסי ז"ל שנראין נמי כדברי רש"י ז"ל שהוא כתב בהלכות ברייתא זו דתניא החמרין והפועלין והתיר בין עירות בין ישינות ולא הזכיר משניו' הללו של שני עדים בשאין לו וסת אלמא אין לנו עדים אלא לטהרות. 
הא דאמר \textbf{ר' חנינא בן אנטיגנוס משמשת בשני עדים והן עיוותיה ותיקוניה.} לפי פי' רש"י ז"ל שני עדים א' לפני תשמיש וא' לאחר תשמיש ואין עוות ותיקון אלא לחייב בעלה בחטאת ואשם תלוי או לטמא מעת לעת ולתקן אותם כדתנן בפירקן דכל היד והיינו דמקשינן אי בעסוקה בטהרות הא אמרה שמואל חדא זימנא אי בשאינה עסוקה למה לה דאמר ר' זירא וכו' ומהתם ש"מ תרתי בעסוקה שצריכה ובשאינה עסוקה שאינה צריכה ופריק מאן דמתני הא לא מתני הא, וא"ת הא ר' יהודה אמרה לעיל א"ל ההיא מכללא איתמר כדאמרי' לעיל כלומר ששמע השומע לשמואל שאמר הלכה כר' חנינא בעסוקות בטהרות ופי' בשמו שאין משנתינו אלא בעסוקה בטהרות ור' זירא שנאה מימרא לעצמו בלשון אחר, ורש"י ז"ל תירץ את זה בע"א.\par ומדברי הר"ם ז"ל שהוא מפרש מאן דמתני הא לא מתני הא לימא לעולם בשאינה עסוקה בטהרות ואפ"ה כשאין לה וסת צריכה בדיקה ואמוראי נינהו ואליבא דרב יהודה והא דלא מקשינן דידיה אדידיה משום דלא מתפרש להו בהדיא ההיא באשה שאין לה וסת.\par ופסק הרב ז"ל כמאן דמתני הלכה כר' חנינא בן אנטיגנוס שכל אשה שאין לה וסת משמשת לעולם בשני עדים א' לפני תשמיש ואחד לאחר תשמיש בין לו בין לה.\par ורבינו הגדול ז"ל הצריך לה בדיקה ג' פעמים עד שתהא מתוקנת והוחזקה שלא תראה מחמת תשמיש ומשם ואילך הרי היא כשאר כל הנשים ואינה צריכה כלום ועל כן יאמר בספר מלחמות ה' ג) דמשמשת בעדים היינו אחד לו ואחד לה ושלה בודקת בו לפני תשמיש ולאחר תשמיש כענין משנתינו וכמו שפי' למעלה והי עוותיה כשהורעא לה ראיה בתוך תשמיש ג' פעמים ותקוניה בשלא אירע לה כלום.\par ואקשינן למה ליה לשמואל הלכה כר' חנינא בן אנטיגנוס אי בעסוקה בטהרות מוקי לה ומשמשת בעדים לעולם משום הטהרות והם מעותין אותה בג' פעמים של ראיה להוציא ומתקנין אותה כשהוחזקה שלא לראות ויצתה מחששא הא אמרה שמואל חדא זימנא דרב יהודה גופיה אמרה לעיל ואי בשאינה עסוקה בטהרות ואפ"ה בעיא תיקון ג' פעמים הא אמר כל לבעלה לא בעיא בדיקה ואפילו פעם אחד דאמר ר' זירא אשה שאין לה וסת אסורה לשמש עד שתבדוק ואוקי' בעסוקה בטהרות ומדלא אמר עד שתבדוק ג' פעמים לבעלה ש"מ דכל לבעלה לא בעיא כלום כדדייקינן לעיל.\par ומפרקינן לעולם בשאינה עסוקה ואמוראי נינהו אליבא דשמואל הא רב יהודה והא ר' זירא ואידך דאמר א) לעיל נבעלה מותרת קמ"ל דמשהוחזקה ואילך אינה צריכה כלום, ב) זהו פי' השמועה לדעת רבינו הגדול ז"ל וראוי הוא לסמוך עליו. 
\textbf{לא פירות ולא מזונות ולא בלאות.} פירש שאינה יכולה להוציא ממנו פירות שאכל דמחילה בטעות שמה מחילה והכי מפרש בגמרא פרק איזהו נשך, ולא מזונות שאינו משלם מה שלותה ואכלה, ולא בלאות של נכסי צאן ברזל ודוקא שאינן קיימין.\par ועל הני הוא דמפרש בגמרא מ"ט תנאי כתובה ככתובה דמי וכיון שכתובה קבל עליו נכסים הללו כצאן ברזל וקבל עליו מזונות ואין לה כתובה פטור הוא מכולם אבל הקיימים נוטלת ויוצאה שאפילו זנתה נוטלת מה שבפניה ויוצאה ודינה של זו כדין איילנות שנוטלת הקיימים בשל ברזל ומפסדת שאינן קיימים, ובשל מלוג דינה כשאר הנשים לדעת רבינו ז"ל בכתובות, וכן היא נוטלת לדעתי ודעת הגאונים תוס' כתובתה וכבר פי' בפרק אלמנה נזונות ומה שכתב רש"י ז"ל בכאן אינו נכון 
 והא דאמר רבי מאיר\textbf{יוציא ולא יחזיר עולמית.} משום קלקולא נ"ל והוא שאמר לה משום שאין לך וסת אני מוציאך ואם לא מפני כן לא הייתי מוציאך ואם לא אמר כן אין כאן חשש לקלקו' כדתנן המוציא אשתו משום איילנות רבי יהודה אומר לא יחזיר וחכמים אומרים יחזיר ואוקימנא מאן חכמים ר"מ ומשום דלא כפליה לתנאיה והא נמי לההיא דמיא ולדידן נמי לא בעיא כפילא כרבנן והוא שאמר סתם משום כך אני מוציאך ואף על פי שלא כפל הא גירש סתם יחזיר כדאמרינן התם גבי איילנות ומוציא משום נדר ומשום שם רע ויש לומר שאפילו לא התנה ולא אמר כלום יש לחוש לקלקל דבשלמא התם אם לא התנה כלום י"ל עילה הוא רוצה לגרש שכמה אנשים נשואים לאיילנות וע"י שיש בהם נחת רוח מהן מקיימין אותם וכך אמרו שם בירושלמי אבל זו שאסורה היא לשמש כלל בידוע שאין בעלה מגרשה אלא מחמת פיסול זה וכן במוציא משום שם רע י"ל מכיון שלא שהה לראו' אם הדברים נראין עילה מצא וגורש לפיכך אין חוששין לקלקול אלא שאמר משום שם רע אני מוציאך. }

\newchap{פרק \hebrewnumeral{2} כל היד}
\twocol{\clearpage}

\newsection{דף יג}
\twocol{\textbf{נשים דלאו בנות הרגשה נינהו משובחת.} ק"ל לר"ת ז"ל למה ליה האי טעמא תיפוק לי' משום דלא מיפקדי אפריה ורביה ושמעתי שהיה מפרש דלאו בנות הרגשה נינהו שאינן בדין הרגשה כלל לומר שאינן מרגישות ולא מצוות. וכן פי' זו ששנינו בברייתא שלשה נשים משמשת במוך חייבות לשמש דאלו מותרות כל הנשים מותרת כן. ואין פי' נכון בכאן דבנות הרגשה לאו בדין הרגשה משמע.\par ואפשר לפרש דמשום משובחת קאמר שאלו היו בנות הרגשה אע"פ שאינן מצוות כאנשים ולא דינן ליקצף מ"מ לא היתה יד המרבה לבדוק יותר מדאי משובח' לפי שהיא משחיתה ואין שבח בהשחת' אפילו לנשים ועוד דהא מביאה עצמה לידי הרהור ואלו היתה בה הרגשה בת נדוי היא כדלקמן ולפיכך הוצרכו בגמ' לפרש דלאו בנות הרגשה נינהו.\par אבל כל עיקר אין דינו של הרב ז"ל נראה לי שאע"פ שאינן מצוות על פריה ורביה ורשאי מן התורה לבטל איסור הוא בהשחתה. ואע"פ שהאשה מותרת לעקור את עצמה מה שאין כן באיש ואע"פ שקיים מצות פריה ורביה התם מצוה אחריתי היא שנצטוו על הסירוס ואפילו מסרס אחר מסרס חייב אבל בהשחתה כל בשר כתיב. 
הא דאקשי' בכולה שמעתין מדר"א דאמר \textbf{כל האוחז באמה וכו'.} י"ל דהכי אקשינן וע"כ לא פליגי רבנן דאמרו לו עליה דר"א אלא בדליכא עפר תיחוח ולא מקום גבוה ומשום חשש פסול המשפחות אבל במקום אחר מודו ליה. הילכך גבי תרומ' ה"ל למימר שיפלוט ואע"פ שמפסיד' וכן בדרב יהודה שיטה וירד חוץ לכנישתא וישתין. וי"ל דא"ל לאו פלוגתא היא אלא בשואלין לפרש להן היו. וכן נמי משמע במס' ברכות פ' כיצד מברכין (מ, א). 
הא דאמרינן ב\textbf{מקשה עצמו יהא בנדוי.} פי' בתוספות לא שהוא מנודה בעצמו בנדוי דרבנן של רבותינו אלא שב"ד מצווין לנדותו ועד שנדוהו אינו מנודה, וראיה לדבר דקאמרינן הקורא לחבירו עבד שיהא בנידוי ואמרי' עלה בקדושין באומר לו עבד אתה ההוא שמותי משמתינן ליה דתניא הקורא לחבירו עבד יהא בנדוי. 
\clearpage}

\newsection{דף יד}
\twocol{הא דתניא \textbf{רוכבי גמלים כולן רשעים.} ובשלהי מס' קדושין אמרינן הגמלים כולן כשרים התם במילי דעלמ' ולבן לשמים הכא רשעים בדבר הזה, א"נ הכא רוכבי גמלים ורשעתם משום חמום זה התם גמלים שמחמרים אחריהם. והא דמדכר הכא הספנין ואע"ג דליתנהו ברכיבה לומר שהם כשרים וצדיקי' גמורים בכל דבר בחמום ובשאר דברים ואגב אחריני נקט להו וכן פי' החמרים יש מהן צדיקים ויש מהן רשעים בדבר הזה קאמר הא דמכף הא דלא מכף ואלו בשאר דברים רובן רשעים וליסטים. 
\textbf{וליחוש דילמא דם מאכולת הוא.} פי' רש"י ז"ל דעל עד שלו פריך ומיהו ה"נ קשיא לעד שלה.\par ולפיכך הקשו מקצת המפרשים א"כ אין לך אשה שנטמאה בנדה בבדיקה. וי"ל בעלמא ודאי לא חיישינן משום דלא שכיח אבל מתוך שבדקה מיד לפני תשמיש ומצאה טהור יש לספק ולא מחוור.\par וי"ל שלא בשעת בעילה ממש ודאי אותו מקום בדוק הוא אצל מאכולת שהוא סתום מלכנוס ואם נכנסה מתה היא ואין דמה יוצא ממנה אבל כאן יש לחוש שמא בשע' בעילה דחקה ונכנסה והשמש הרגה ושפך דמה עד עקבו א"נ שמא על השמש היתה מאכולת ועמו נכנסה ופריק דחוק הוא ואין מאכולת שעל השמש נכנסת עמו וכ"ש בפני עצמה.\par ובתוספת אמרו דעל עד שלו דוקא פריך משום דכיון שבדקה היא בשיעור וסת ומצאה טהור והוא מצא יש לחוש למאכולת שאם היה דם נדה בשלו היה נמצא בשלה נמי. אבל בשלה בבדיקה דעדים לא חיישינן למאכולת דרוב דמים מצויין בנשים ואין קנוח העד נמי ממעך מאכולת והורגה (ובמעורה) [ובמעוכא] דשמש אין לתלותה כיון שלא נמצא על שלה.\par אבל עדיין אני תמה על עד שלו דלא יהבו ליה רבנן שיעורא אלא אע"פ שיהא אחר בעילה זמן מרובה קודם קנוח טמאין ואמאי ליחוש שמא מאכול' הוא שבאה עליה אחר שבעל שהרי אינו אלא ככתם בעלמא ויש לדחוק ולומר שכל קודם קנוח כיון שעדיין שכבת זרע לחה אין המאכול' באה עליו ואין לחוש שמא נרצפה על הסדין וממנו נתקנח בו שדם מאכול' מועט הוא ואינו מתקלח מן הסדין אלא נבלע הוא בו מהכ"ש שא"א לו להתקנח (במקום) [ממקום] ששוכבת עליו לזה. 
\textbf{בדקה בעד הבדוק לה וטחתו ביריכה ולמחר מצאת עליו דם.} גרסינן בכולהי נוסחי וכן מצינו בשם ר"ח. ופירושו שמצאת עליו על העד דליכא למיתני אלא דילמא דם בירך הוה ואי הוה בירך נמי כתם הוא וטמא וכיון דאיכא ספיקא דירך גופיה ועוד דלא שכיח קרוב הוא יותר לתלות בעד לטומאת נדה ולא בירך הילכך בין שנמצא נמי על הירך בין שלא נמצא אלא על העד טמאה נדה.\par ורש"י ז"ל גרס ונמצ' עליה דם ופירש על יריכ' של אשה ואין דבריו מחוורים שאם העד נמצא ואין עליו כלום נראה ודאי שדם יבש על ירכה היה ואין לטמא אותה נדה ואם דוקא כשנמצא אף על העד למה לי מציאת הירך וכי מפני שהוטח ממנה דם נקל ויש להעמידה בשאבד עדה. וי"ל לעולם כשנמצא אף על העד וכשלא נמצא על הירך כלום פשיטא בעד הוה שאלמלא על הירך הוה לא נתקנח לגמרי ממנה אלא אפילו נמצא על הירך נמי טמאה נדה כיון שנמצא אף בעד.\par ולדברי הכל אף בזה צריכה כגריס ועוד שאלמלא כן חוששין שמא דם מאכולת הוא שאפילו בקנוח של שיעור וסת ואח"כ שהדבר קרוב ואפילו לחטאת ואשם הקשו למעלה וליחוש דילמא מאכולת הוא. ומיהו כיון דאיכא שיעורא דנפקא ליה מחשש מאכולת טמאה נדה ואפילו לר' חייא ואם היה משוך טמא בכל שהוא דלא גרע מהניחתו תחת הכר וכן בפרק הרואה כתם אליבא דהלכתא וזה דעת הראב"ד ז"ל.\par ולפי גרסת ר"ח ז"ל י"ל כמו שנמצא על העד ולא על הירך כלום שאם מאכולת נרצפה שם בתחלה על הירך נמי היה נמצא ומשהוטח העד על הירך מקום (דחוק) א) הוא אצל מאכולת אלא מקנוח היה דם ונבלע בעד ולפיכך לא הוטח ממנו על הירך כלום. וכן נראה מלישנא דגמ' דקא מקשה ר' חייא בסמוך אף אתה עשיתו כתם אלמא אין לך שצריכה שיעור כתם וטמאה נדה אלא זו לדברי רבי. 
\textbf{בדקה בעד שאינו בדוק לה.} פירש בתוספות כגון שהזמינה פקולין או צמר נקי ולבנים אלא שלא חזרה וראתה בהן סמוך לבדיקתה אם יש עליהם טיפי דמים מן מאכולת או משאר דברים הא בבגד שאינו בדוק כלל לא אמר ר' טמאה נדה אטו לקחה בגד מן האשפ' וקנחה בו מי מטמא רבי נדה.\par (ואי) [ועוד אני] אומר כיון שהצריכוה כגריס ועוד הרידינו כסדין וחלוק וכל שלא היה בדוק כלל אפילו לרבי טהורה לגמרי אפילו לקחתו מן השוק סתם טהורה לבעלה. והראב"ד ז"ל סובר דבעד אפילו אינו בדוק כלל טמאה דלא דמי לחלוק דכיון שבדקה בו ממש רגלים לדבר דרוב דמים מצויין בו. 
הא דתנן \textbf{כדי שתרד מן המטה ותדיח את פניה.} לישנא מעליא הוא ומאי פניה שלמטה ומאי הדחה בדיקה כלומר כדי שתרד מן המטה ותבדוק והכי מוכחא שמעתא והיינו דאקשי' מדתניא כדי שתושיט ידה מתחת הכר או לתחת הכסת ותטול עד ותבדוק בו דאלמא שיעור אשם תלוי אינו אלא שתקח העד ותבדוק דקס"ד סתמא כשאין עד בידה אלא תחת הכר או תחת הכסת. 
 ופריק רב חסדא דה"ק \textbf{איזו אחר זמן וכו'} פי' רש"י ז"ל וחסורי מחסרא וה"ק ול"נ אלא רב חסדא פרושי מפרש לה למתניתין הכי תנן נמצא על שלה לאחר זמן טמאים מן הספק ופטורין מן הקרבן ולא פי' שיעור אחר זמן מה הוה. והדר תני איזו אחר של שיעור זה שאינן טמאין מן הספק כדי שתרד מן המטה ותבדוק שזהו אחר כך שמטמאה מעת לעת ואינה מטמאה את בועלה והאי דפריש תנא האי שיעורא לא ללמד על דין עצמו שהרי כל מעת לעת כך הוא נדון בין לרבנן בין לר' עקיבא אלא כך אמר אם ירדה מן המטה ובדקה אין בועלה טמא שזהו אחר זמן הא כל זמן שלא שהת' כשיעור הזה אלא בדקה עצמה על המטה טמאין מספק בא זה ולמד על זה.\par אבל לדברי רש"י ז"ל שאומר חסורי מחסרא וה"ק בהדיא איזו אחר זמן כדי שתושיט ידה ותטול עד ותבדוק לא הוה תו למיתני כדי שתרד מן המטה ואפשר לתרץ לו שבא לפרש שלא תעלה על דעת שבדיקת ראשונה שלאחר זמן ראשונה כשירדה מן המט' היא לפיכך פירש שתיהן ואמר שאלו ירדה אחר אחר הוא וכ"ש לדברינו דמחוור טפי לומר שפי' אחר אחר ללמד על אחר הזמן הראשון ושלא ליתן בו מקום לטעות.\par וברייתא נמי דייקא כדידן, דתניא איזהו אחר זמן דבר זה שאל ר' אלעזר בר צדוק לפני חכמי' באושא שמא כר"ע אתם אומרים שמטמאה את בועלה מעת לעת פירש ולפיכך אין אתם חוששין לפרש אחר זמן שהרי כל מעת לעת נמי כך הוא דינן ואע"פ שהיו צריכין לפרש לדבריו דר' אליעזר בר צדוק משום אשם תלוי יודע היה בהם דבעי חתיכה משתי חתיכות ולא תמה עליהם בזה אלא אם כדברי ר"ע שהוא יחיד הם אומרים אמרו לו לא שמענו לפי' אין אנו מפרשין אבל לא בדברי היחיד אנו אומרים ואמר להם כך פרשו חכמים ביבנה לא שהתה כדי שתרד מן המטה ותדיח פניה תוך זמן זה כלומר כל ששהתה ובדקה ולא שהתה שיעור שתרד מן המטה ותבדוק אלא על המטה בדקה אע"פ שהושיטה ידה לעד תוך זמן זה אבל ירדה מן המטה ובדקה או שהתה כשיעור הזה לעולם טהור ואע"פ שעד בידה ש"מ שבכל מקום שפי' אחר לא פירש באחר זמן דבר אחר אלא כל שלא שהתה כשיעור אחר וכן דרך משנתינו ללשון שפירשנו.\par ואקשינן לרב אשי אמאי קא מטהרי רבנן בברייתא ביורדת מן המטה ובודק' הא במתני' מטמו כשיעור הזה וכ"ת דאין עד בידה ה"ל לפרושי שהרי אין משמעו' הלשון זה אלא כל ששהתה כדי שתרד לעולם טהור ואפילו עד בידה וכדקתני מתניתין נמי כדי שתרד ומוקמת לה בעד [בידה] דהיכי אפשר דהכא והכא חד שיעורא קתני והכא טהור והכא טמא ה"ל לפרושי במתני' עד בידה ובברייתא אין עד בידה אלא ש"מ כרב חסדא ותרווייהו שיעור לטהר וזהו דרך פירש רש"י ז"ל בשמועה כולה ויש לשונות אחרים ואין בהם ממש. 
\clearpage}

\newsection{דף טו}
\twocol{\textbf{והוא שבא ומצאה בתוך ימי עונתה.} פי' רש"י ז"ל שלשים יום לראיה ואמרו שכך נמצא במס' נדה בירושלמי ואמר רב הונא דכי מצאה תוך ימי עונתה לא בעיא בדיקה אנא שלא הגיע ימי וסת אבל הגיע קודם ביאתו מן הדרך אסורה וסתות דאורייתא הלכך אסורה עד שתאמר לו בדקתי בשעת הוסת עצמו וטהורה אני ורבה בר בר חנא אמר אפילו הגיע עת וסתה מותרו' וסתות דרבנן שהם הצריכוה לבדוק בימי וסתה שמא תראה.\par ומיהו היכא דבעלה לא היה בעיר ולא ידעינן אי בדקה אי לא בדקה לא מחזקינן לה בטומאה. ואע"פ דאמרינן לקמן תבדוק ומשמע דאסורה עד שתבדוק וכיון דכי לא בדקה מחזיקן לה בטומאה עד שתבדוק כי לא ידעינן ודאי בההיא דחזקה נמי קיימא עד שתאמר בדקתי וטהורה אני א"ל הכא בבא מן הדרך הקל כיון דאיכא תביעה אין לך בדיקה גדולה מזו ומהניא לחששא דוסתו' כדמהניא לבדיקה דטהרות בפירקין קמא דתרווייהו מדרבנן. א"נ כיון דלא ידעינן תולין שמא בדקה ומצאה טהור או שלא ראתה והיתה טהורה מ"ה מותרת. והא דאמר ר' יוחנן בעלה מחשב ימי וסתה ובא עליה תפתר בשוהה בעיר ולשהות עמה בין עירה בין ישינה ואף ע"פ שאינה אומרת לו כלום ולא הוא תובעה כנ"ל לפי פי' רש"י ז"ל והוא נכון.\par אלא בזו שפי' ימי עונה לאשה שיש לה וסת אינו מחוור שכיון שלא הגיע עת וסתה היאך נחוש לעונה והלא כל שיש לה וסת קבוע דמיה מסולקין ממנה עד זמן וסת [לכן נראה] דרב הונא אמתני' קאי ולדידיה יש לה וסת חוששת לוסתה ולא לעונה אין לה וסת חוששת לעונה ולרבה בב"ח אפילו יש לה וסת אינו חושש לוסתה כדאמרן ומיהו חוששת לעונה אחר הוסת כלומר שאם עבר עליה עת וסתה כיון שאין אנו חוששין לוסת נחוש לעונה אבל ודאי הגיעו ימי עונה ולא הגיע ימי הוסת אינה אסורה דאפילו למ"ד וסתו' דרבנן מסולקת דמים היא אפילו מעונות עד הוסת תדע שהרי אמרו דיה שעתה בוסתות ולא אמרו כן בעונות.\par נמצא עכשיו לדברינו כל אשה שאין לה וסת בעלה חושש לימי עונתה ושיש לה וסת חוששין לימי עונה שאחר הוסת ואפילו בבא מן הדרך ובכולן מחשב ימיה ובא עליה דהא מ"מ תרי ספיקי נינהו ואע"פ שאינו בעיר נמי סופרת היא בעצמה רוב פעמים הילכך מותרת ומסתברא נמי שאם היה הוסת רחוק שאפשר שטבלה והעונה קרובה תולין להקל שמא בשעת הוסת ראתה וטבלה ושוב אין לה עונה עכשיו דכולהי ספיקי נינהו ולקולא ולא החמירו בעונה יותר מן הוסת מפני שהיא חמורה אלא מפני שא"א לנו לומר שהאשה לא תראה לעולם לפיכך תולין בעונות וחוששין להן אבל אם בא לתלות נמי בוסת תולין. זהו מה שנ"ל.\par והראב"ד ז"ל כתב דהא דרבה פליגא אדר' יוחנן דלר' יוחנן אע"ג דוסתו' דרבנן (בפי') [צריכה] חשוב ימים לטבילה ואפילו לבא מן הדרך שאין הוסת יוצא מחזקת טומאה (נהי) [עד] שתבדוק כדלקמן ופסק הלכה כרבי יוחנן וזו דרך טובה להחמיר לענין מעשה ונמצא חוששת לוסת וחושש' לעונו' כשאין לה וסת אבל לחוש לשניהם כאחד אין לנו.\par אבל תמהוני על רבינו הגדול ז"ל שכת' זו ששנינו חמרין ופועלין וכו' נשיהן להן בחזק' טהרה ולא חלק בין הגיעו ימי עונה ועת הוסת לשלא הגיעו. והר"ם תלמידו ז"ל כתב הלך בעלה למדינה אחרת והניחה טהורה כשיבא אינו צריך לשאול ואפילו מצאה ישינה הרי זה מותר לבא עליה שלא בעונת וסתה ואינו חושש שמא נדה היא. אף הוא לא הפריש בכלום.\par ונראה שהם ז"ל מפרשים תוך ימי עונתה היינו עונת וסתה לאפוקי יום עונה עצמו ודלא שתעלה על דעתך שבבא מן הדרך לא חששו אף לעונת הוסת או שיתלו להקל לומר שמא עקרתו וקמ"ל דבעי מיחשב דלא בעונת וסת קיימא האידנא.\par והם עוד סבורין דר' יוחנן דאמר מחשב קסבר וסתו' דאורייתא והלכתא כרבה דאמר מותרת דסוגיין וסתו' דרבנן ומסתייעי מאינשי דביתיה דרב פפא ורב הונא בריה דר' יהושע דפ"ק דכי אתו מבי רב לבתר וסת ועונה אתו ולא מיחשבי ולא בדקי בין עירות בין ישינות והא דתניא ר' יהושע אומר תבדוק תפתר לטהרות וזו קולא גדולה טוב לפני האלהים ימלט ממנו. 
הא ד\textbf{אמר רבי אושעיא וכו'} במסכת ע"א שמעתיה (מא, ב) ושם פירשתי. 
\clearpage}

\newsection{דף טז}
\twocol{הא דתניא \textbf{הרואה דם מחמת המכה אפילו בתוך ימי נדותיה טהורה.} טעמא דמילתא דאע"ג דאמרינן תבדוק הכא כיון דלא אפשר לה למיבדק טהורה ואין דנין אפש' ממי שאי אפשר וסתו' דרבנן ובאפשר תקנו וכן זו שאמרו אף אנן נמי תנינא לר"נו דוסתות דאורייתא מדקתני שחרדה מסלק' את הדמים טעמ' דאיכא חרדה הא ליכא חרדה טמאה אלמא וסתות דאורייתא ולא אמרי' כי איכא חרדה טהורה כי ליכא חרדה תבדוק היינו טעמא משום דלמ"ד וסתו' דרבנן אע"ג דליכא חרדה טהורה לגמרי כיון שלא היה יכולה לבדוק בשעת הוסת לא הטריחו עליה לבדוק אחר כן כלל.\par ובתוספות מפרשים דדייק לה מלישנא דקתני מסלקת את הדמים משמע דמים הבאים בזמנן ואינו נכון (לא) [ועוד] אמרו דסיפא נקט דקתני לה בדר"מ וארישא סמיך דקתני הניע שעת וסתה ולא בדקה טמאה ועלה א"ר מאיר שאם היתה במחבא טהורה הא לא היתה במחבא מודה דטמאה. 
מתני': \textbf{ב"ש אומרים צריכה שני עדים על כל תשמיש ותשמיש.} פי' רש"י ז"ל א' לפני תשמיש וא' לאחר תשמיש ולמחר בודק בשניהם ולא עכשיו כמו שפירשתי בפ"ק לדעת הרב ז"ל, ולשון צריכה נראה כן מדלא קתני צריכים ומיהו יכולה היא לבדוק באותו שלפני תשמיש זה לפני תשמיש אחר חוץ מן הצנועו' ששנינו למעלה אבל עד שלאחר כל תשמיש ותשמיש צריך לבן וכיון שזה צריך לכל הנשים קתני נמי שלפני תשמיש שדרך הצנועו'.\par ואינו מחוור. ועוד ק"ל אמאי לא תנא צריכי' שלשה עדים וליחשוב נמי א' שלו אבל נראה שכל מקום ששנינו שני עדים א' לו וא' לה.\par וכך פי' משנתינו לפי דעתי צריכה שני עדים א' לו וא' לה שלה בודק בו לפני תשמיש ראשון ורואה טהרה ומשמשת ואח"כ מקנח' בו לאחר תשמי' וכשבאה לשמש פעם אחרת אינה צריכה כלום אלא משמשת ולאתר תשמיש מקנחין היא ובעלה בשני עדים אחרים ומניחין אותן עד למחר שמא מחמת תשמיש ראתה וזו הבדיקה אינו מועלת להם [אלא] לטהרות חוץ מבדיקה שלפני תשמיש ראשון שהיא אף לבעל לפיכך בודקת ורואה מיד לאור היום בין השמשות או לאור הנר אם התחילה משחשיכה לגמרי ואפילו לב"ש עד שלפני תשמיש ושלאחר תשמיש עד א' הוא שאפילו הצנועות עצמן לא נהגו צניעותן אלא בעד שלפני תשמיש מתוך שהוא נקי וטיפה כל שהיא ניכרת בו אינן רוצות שיהי' בו לכלוך אפילו שלפני תשמיש אחר כדי שתהא בדיקתן מעולה לגמרי אבל של אחר תשמיש ששכבת זרע רבה עליו [אין בין עד חדש לעד] מבדיקה ראשונה כלום.\par והיינו נמי דאמר או תשמש לאור הנר ותבדוק בו דבעד שלאתר תשמיש לא בעינן לבן אלא שצריך שידע אם דם שעליו מתשמיש ראשון או מאחרון היה כדי לחייב עצמה ובועל' לידע אם טמא משום בועל נדה ולעולם לפני תשמיש [א"צ בדיקה] שלא בעי ב"ש בדיקה אלא לפני תשמיש ראשון בלבד שהרי דיה בדיקה אחת לעונה אחת ולא הוצרכו הללו שבין תשמיש לתשמיש אפילו לטהרו' [אלא] מפני חשש רואה מחמת תשמיש הילכך דיה בבדיקה שלאחר כל תשמיש ותשמיש.\par ועוד שכיון שהיא משמשת והולכת בודק' שלאחר תשמיש זה שהיא לפני תשמיש זה דסמוכין הן וב"ה אומרים דיה שני עדים כל הלילה אף אלו א' לו וא' לה שלה עולה לפני תשמיש ראשון שבודק' ורואה ולאחר תשמיש אחרון ושלו לאחר תשמיש אחרון שכיון שאף לדברי ב"ש אין בדיקו' הללו אלא לטהרו' ולאחר תשמיש דיה בסוף.\par ולא יקשה עליך צריכה [{\small פי' ולא תנן צריכים} ] לשון שאמרנו שהרי כך שנינו דרך בנות ישראל משמשות בב' עדים ואע"פ שא' לו ולה אמרו דרך בנות ישראל ובני ישראל משמשין.\par והר"ם הספרדי פי' שאף ב"ה מצריכים לבדוק לאחר כל תשמיש אלא שדיין באותן שני עדים כל הלילה. 
גמרא \textbf{אמרו להם ב"ש לדבריכם.} כיון שאתם מודים בבדיקה שלאחר תשמיש משום שמא ראתה מחמת תשמיש הוה נמי בבדיקה בין תשמיש לתשמיש שמא תראה טפת דם כחרדל בביאה ראשונה שבא אורח מחמת תשמיש ותחפנה שכבת זרע בביאה שנייה ושוב לא תמצא בעד שלאחר כל התשמישן אמרו להם ב"ה א"כ אף מתחלת ביאה לסוף ביאה ניחוש כן ויכולין היו ב"ש לומר שאין דנין אפשר מא"א אלא גדולה מזו אמרו שאינו דומה וכו'.\par ולדברי הר"מ ז"ל צריכין לפרש תראה טיפה דם כחרדל בעד של אחר ביאה ראשונה ותחפנו שכבת זרע לאחר ביאה שנייה ואמרו להם ב"ה אף לדבריכם שמא בקנוח ראשון עצמו נמוקה הטפה ובטלה בשכבת זרע.\par ומצאתי בתוספו' שפי' כדבריו בשמו של ר' שמואל רומרוגי ז"ל והם הקשו ללשון רש"י ממה ששנינו צריכין ב' עדים על כל תשמיש ותשמיש או תשמש לאור הנר ומשמע דמשמשת לאור הנר אינה צריכה אלא כדברי ב"ה ולפירושו עדיין המשמשת לאור הנר צריכה לבדוק ולאחר כל תשמיש ותשמיש ולראו' בעד כדברי ב"ש וב"ה אינה צריכה בדיקה כלל עד סוף כל הלילה מ"מ לשון תראה מתפרש לנו יפה. 
הא דאמר ר' זירא \textbf{בעל נפש לא יבעול וישנה.} לא שהוא סבור שמשנתינו לבעלה שהוא עצמו אומר בפרק קמא כל לבעלה לא בעיא בדיקה ושם הביאו ברייתא זו בד"א לטהרות וכו' אלא ה"ק כיון שאע"פ שבדקה לפני תשמיש חששו כולם שמא ראתה מחמת תשמיש ומניחים העדים עד למחר הילכך בעל נפש לא יכניס עצמו לבית הספק אפילו לבדוק ולהניח דזה לא יועיל לגבי הבעל ורבא אמר שאין בחששות הללו שום ספק אלא חומר של טהרות הוא. 
והא דאמרי' \textbf{תניא נמי הכי} לא שהוא לבעל נפש אלא שהוא דוקא לטהרות ולא לגבי הבעל וכיוצא בזה תניא נמי הכי שאינו לראיה ממש בפרק קמא דמגילה מפני שעיניהם של עניים נשואות למקרא מגילה וכו' ובתוספות מפרשים לא יבעול וישנה בלא בדיקה הא בבדיקה מותר אפילו לחסיד שבחסידים מדאמרינן במס' שבת אמר ר' יוסי ה' בעילות בעלתי ושניתי, ואין סוגיא מתחוורת בפי' הזה. 
\textbf{בדקה בעד ואבד אסורה לשמש עד שתבדוק.} לדברי רש"י ז"ל בעד שלפני תשמיש ופשוט' היא, ולדברינו כגון שהחמירה ובדקה לאחר תשמיש ראשון או שהיתה דעתה שלא לשמש כל הלילה ומכיון שאין מוכיחה קיים ולא תדע אם ראתה כלום מחמת תשמיש זה אסורה לשמש עד שתבדוק לפני תשמיש לאור הנר בכל בדיקה של בעל.\par ואקשינן אלו קנחה בו ואיתי' מי לא משמשה אע"ג דלא ידעה הא אפילו ב"ש דמחמרי שרו לבדוק ולהניח ולשמש והאי דאקשינן הכי ולא אקשינן מדב"ה דלא מצרכי בדיקה זו דמי מדמי מקשינן דאלו התם דיומא משום קולא דבעל הוא שלא להטריח עליו בין תשמיש לתשמיש אבל מכיון שבדקה דנמלך צריכה היא בדיקה ממש לאור הנר.\par ומפרק זו מוכיחה קיים לטהרות וזו אין מוכיחה קיים לטהרות שמא ראתה לאחר תשמיש זה ולא תדע למחר הילכך אף לבעלה אסור' משום מגו של טהרות וכן נמי לב"ה דלא מצרכי הך בדיקה מוכיחה בעד שלאחר תשמי' אחרון ואין חוששין לנימוק אבל זו כבר נתקנח הדם בזה ואבד הילכך אסורה עד שתחזור ותבדוק עכשיו לפני תשמיש דבהכי ודאי מותר' שא"א להחמי' עליה יותר מכאן לא יהא זה חמור מן הוסת שאם בדקה ומצאה טהור טהור. ואפילו לטהרות עצמן אם בטלה בדיקה של עונה אחת כגון של שחרית או של ערבית בודקת עכשיו ועוסקת בהן וכן בבדיקה של אחר תשמיש בין לטהרות בין לבעלה. 
\clearpage}

\newsection{דף יז}
\twocol{הא דאמרינן \textbf{רבה בר רב הונא מקרקש זגי דכלתי.} פ' רש"י פעמונים תלויין בכילה שסביב מטתו ומקרקשן בעת תשמיש לסור בני ביתו ואין זו דרך צניעות אפילו לקלים אלא כדי שיבריחו העכברים והתרנגולים היה עושה וכעין עובדי דדידכי ופרוחי היא.\par ובספר הישר מפורש דזגי מין דבורים הם כדאמרי בבכורות דבש הגזין וצירים או גזין וצרעין, ועוד שאף לשון קרקוש אינו על הפעמונים שפי' רש"י ז"ל. 
 הא דאמרינן לצפרנים\textbf{ולא אמרן וכו'.} ואסיקנא ולא היא לכולי מילתא חיישינן מדגרסינן פ' ואלו מגלחין ר' יוחנן שקל טופריה בשיניה וזרקינהו בי מדרשא ואקשינן עליה היכי עביד הכי והתניא זורקן רשע ופריק אשה בבי מדרשא לא שכיחי וכ"ת דילמא כנשי להו לבראי הואיל ואישתני אישתני אלמא שאפי' בלא גנוסטרי ודידה בלחו' אסור. 
 הא \textbf{דאמר אביי כגון שהעבירה על אויר התנור.} חדא מתרי טעמיה נקט דה"נ אפשר לאוקומה כגון שנטמא באהל המת א"נ בהיסט הזב וזבה וכל המטמאים במשא. 
\textbf{תינוק הנמצא בצד העיסה.} פירש רש"י ז"ל בתינוק שאינו ודאי טמא אלא שרוב תינוקות מטפחין באשפה ושרצים מצויין שם ובצק בידו שנגע בעיסה.\par וזה הפי' אינו נכון שאם ודאי נגע בעיסה למה אין שורפין תרומה הרי רוב זה כרוב שליא שעשאוהו כודאי דהתם נמי א"ל דרוב שליא בולד ומיעוט בלא ולד ובית זה בחזקת טהור עומד סמוך מיעוטא לחזקה ואיתרע ליה רובא דלא שרפינן עליה תרומה אלא כיון דשליא בבית הוא אין חזקתו של בית הוא כלום אלא הרי הבית כשליא ורובא ומיעוטא ובתר רובא אזלינן הוא נמי בתר רובא אזלינן.\par אבל הפי' הנכון שבתינוק ודאי טמא נחלקו דכיון שנמצא בצד העיסה והבצק בידו אמרינן רוב תינוקות מטפחין בעיסה ומיעוטן אין מטפחין ועיסה זו בחזקת טהורה עומדת סמוך מיעוטה לחזקה שהרי שתי המדות הללו מחזיקין טהרה לעיסה לומר שלא נגעה בה תינוק וממקום אחר היה לו הבצק או אדם אחר נתנו לו ומעולם לא טפחו לבצק הזה בעיסה מה שאין כן בשליא שאי אפשר לצרף חזקת הבית למיעוט שליא שאין חזקה זו מלמדת שאין עם השליא ולד אלא הבית כשליא עצמה ואין חזקתה כלום שהרי נגעה ונעשית כמוה אבל בעיסה ותינוק שאנן מחזיקין שלא נגע תינוק בעיסה.\par ואיפשר שאף לדברי רש"י ז"ל לא בשנגע אלא שהבצק בידו ואף על פי שחזקה הוא שנגע עדיין חזקת עיסה במקומה כיון שלא נגעו ודאי בפנינו ואין זה נכון ומפורש הוא בירושלמי בסוף קדושין רוב תינוקות מטפחין בעיסה וכבר פירשתי זה בפרק כיסוי הדם והאי דלא מטמינן ברשות היחיד לעיסה ואפילו מחצה על מחצה כשאר כל הספקות משום דבר שאין בו דעת לשאול הוא ואלמלא הרוב טהור הויא וכן שליא אין דינה לשרוף אלא משום רוב דאפילו ברשות הרבים מטמינן מדקתני שאין שליא עמה ולד הא בבית ממש טמא אפילו בספיקא דהוה ליה ספק שרץ ספק צפרדע. 
הא דאמרינן \textbf{מאי לאו לא תיובתי' אלא סייעתיה וכו'.} ה"פ: דקסלקא דעתך מדקתני מתניתין דטועה מטבילין אותה צ"ה טבילות שמחמירין עליה כל חומרי טבילות הללו ש"מ בודאי ילדה (מצוי) א) שאלמלא יש ספק בולד ה"ל ספק ספיקא ספק זמן טבילה היום ספק אינו זמן ואת"ל הוא ספק אינו ולד ודיה טבילה אחת באחרונה כרבי יוסי לר' יהודה דלא מחמירין כולי האי בטבילה בזמנה מצוה אפילו דתרי ספיקי אי נמי בחד ספיקא בספק לידה כלל.\par ופריק דילמא לא תיובתיה ולא סייעתיה דכיון דאיכא רוב הולכין אחריו אפילו בתרי ספיקא להחמיר ואפילו לכתחלה אבל להקל לא עשאוהו בודאי.\par ויש אומרים מפני שהצריכוה למיטבל בשבוע ג' משום יולדת דזוב וספיקי טובא נינהו שמא ילדה שמא לא ילדה ואם תמצא לומר ילדה בזוב אימר עלו לה ימי שבוע שני לספירה אם הרחיקה לידתה ואינו נכון דהשתא נמי דהויא ודאי הויין ספיקות טובא אלא סלקא דעתך דבספק לידה לא מחמרינן בטבילה בזמנה. 
\textbf{אלא למעוטי רובא דרבי יהודה.} פירש דאפילו לרבי יהודה לאו רוב גמור הוא לשרוף אלא לתלות, וק"ל דהא איתותב ההוא לישנא ואסיקנא דבאי אפשר לפתיחת קבר בלא דם קמפלגי ובפיר' רש"י ז"ל אפילו ללישנא בתרא דר' יוחנן דאמר טעמיה משום דאי אפשר לפתיחת קבר בלא דם אפילו הכי טעמיה דר' יהודה משום רוב דברוב פתיחת קבר איכא דם ולא עשאו ברוב זה כודאי דכיון דאין עמה דם איתרע ליה.\par וק"ל דהא אוקימנא לר' יהודה כרבי יהושע דמשוה ליה חדא ב) דאמר מביא קרבן ונאכל דאי אפשר לפתיחת קבר בלא דם וא"ל סבר לה כרבי יהושע לתלות ולא לשרוף, ויש לומר דמאן דמתני בשלשה מקומות מתני ההוא לישנא קמא\par דר' יוחנן דמוקי פלוגתייהו דר' יהודה ורבנן בשאינה יודעת מה הפילה. וכן עיקר שאלמלא כן לא הזכירו בגמרא כאן לשון ראשון שהוא טעות במקום עיקרו. 
\clearpage}

\newsection{דף יט}
\twocol{הא דתנן \textbf{עמוק מכן טמא דיהה מכן טהור.} פירש רש"י ז"ל עמוק יותר שחור דיהה שנדחית מראיתו ואינו שחור כל כך.\par ושמעתי שה"ר שמואל ז"ל פירש בהפוך עמוק שאינו שחור כל כך דיהה שחור יותר מכן.\par והכניסו במחלוקת הזה מה שאמרו לענין נגעים בהרת עמוקה כמראה חמה עמוקה מן הצל אלמא לובן הוא העמוק, ועוד מזה מש"כ כזית ובזפת וכעורב טהור וזהו דיהה וכדיו וכענבה טמא וזהו עמוק והוא ז"ל שיער בדעתו שהעורב שחור יותר מהענבה ועוד שהדיו שהוא עמוק זהו הדיו עצמו והשחור השנוי במשנתינו הוא חרותא דדיותא והוא שחור יותר מן הדיו עצמו.\par וכל אלו דברים בטלין הם שהמראה חזק באותו גוון הוא נקרא עמוק בכולה גוונים והחלוש באותו צבע הוא הדיהה ממנו כדתנן עבר או שדיהה הרי זה כתם והזית והעורב שחרירותן מבהיק ושל ענבה משחיר ובוהק ג) וכן הדיו הלחה שכותבין בו ונשתהא כוהה יותר (מקום לו) מהחרת השנוי במשנה וכי היאך יעלה על דעת במזוג דשנינו (שלשה) [שני] חלקים מים ואחד יין שהעמוק מכן והוא הלבן שיש בו שלשה חלקים מים ואחד יין טמא והדיהה או דיהה דדיהה שאין בו אלא חלק אחד מים וחלק אחד יין טהור והלא אדום גמור הוא וקרוב למראה דם יותר מן הראשון כפלי כפלים וכל שכן באדום עצמו שהוא כדם המכה שהמלבין בו טהור והמתאדם ביותר טמא ודאי בלא ספק ולא ידע מר בטיבעא כלום.\par וראיתי בתוספות שהביאו מן הירושלמי בענין עמוק דיהה הוו בעיין מימר מאן דאמר טהור במצחצחו מאן דאמר טמא בשאינו מצחצח ושמע מינה מן הדה מעשה וכו' א"ל רביח לבו כן אמר רב הונא בשם ר' שחור מקדיר (טמא) [טהור] מצחצח טמא לא אמר אלא שחור כולהון אפילו מצחצחין טהורין אלמא המצחצח קרוב לטומאה יותר מן המקדיר וזה הפך ו) שהרי מתחלה סלקא דעתך המצחצח טהור ז) שהוא המשחיר ביותר הוא הטמא ח) ומ"ש אפילו מצחצחין הכי קאמר אפילו היה במראה הטומאה הואיל ומצחצחין טהורין. 
 הא דאמרינן \textbf{איכא בינייהו לתלות} דתני חמשה דמים טמאים והשאר יש מהם ספק, כגון ירוק ומימי תלתן ומימי בשר צלי נראה היה לפסוק כדברי תנא קמא דהוא סתם משנה ולא כדברי בית הלל וחכמים דהיינו רבי יוסי אלא שיש לסמוך הלכה כר' יוסי מפני ששנו אותה בלשון חכמים ואף על פי שפי' דמאן חכמים ר' יוסי לכך שנו אותה תחלה לומר כן שהיא הלכה וחזרו והזכירו דבר בשם אומרו להביא גאולה לעולם.\par ועוד יש לי סמך וראיה לדבר מפני שהיא שנויה בבחירתא דם הירוק עקביא בן מהללאל מטמא וחכמים מטהרין וכיון ששנינו שם דברי רבי יוסי בלשון חכמים לא שנו שתולין שמע מינה שהלכה כרבי יוסי כדאתמר בעלמא (ברכות כז, א) הלכתא כר' יהודה ותנן בבחירתא כוותיה והתם משמע דעקביא גופיה הדר ביה דתנן ובשעת מיתתו אמר לבנו חזור בך בד' דברים שהייתי אומר א"ל ואתה למה לא חזרת בך וכו' במתני' ומדלא קתני התם בפלוגתא אלא סבריה דר' יוסי הדר ביה עקביא לטהר לגמרי ועוד דמתניתא דרב דימי מנהרדעא וסוגיין דעלה בריש פ' המפלת כולהו כר' יוסי אמרינן.\par ואנו עכשיו שבטלו רואי דמים כל שיש בו מראה אדמימות ואפילו כמימי בשר צלי ודיהה ממנו יושבת עליו שבעה נקיים וכן בכל מראה שחור ואפילו כזית ודיהה ממנו ואפילו דיהה מן הדיהה אבל בדם הירוק ואצ"ל לבן ושאר מראות ודאי אינה יושבת כלום שלא גזרו אלא על דם ותולדותיו אבל אלו אינן בכלל לדברינו שפסקנו הלכה כר' יוסי בירוק, וכן כתב הרמב"ם ז"ל שהאשה שראתה לובן או אודם ירוק הרי היא טהורה אף בזמן הזה אלא שהוא סומך על דברי ב"ה לכתוב בחבורו שאין לך דמים טמאים כלל אלא ה' ולא חשש לסתם. 
\textbf{ירד ר"מ לשיטתו של עקביא וטימא וה"ק להו וכו'.} י"מ כל שמועה זו אחר דבריו של ר' יוחנן דהוא אמר ירד ר"מ לשיטתו של עקביא וטימא משום דקאמר אם אינו מטמא משום כתם משום דדם ירוק באשה לא שכיח א"נ לרבנן דחולין קמ"ל נהי דספיקא משויתו ליה גבי כתם דלקולא גבי רואה ממש תהא טמאה נדה לשרוף והיינו כעקביא.\par ואקשינן עליה דר' יוחנן א"ה אם אינו מטמא משום כתם וכו' משום רואה מיבעי לי' אלא ה"ק להו וכו' קושיא היא ומהדרינן לקיימא לדר' יוחנן א"ה שפיר קאמרי ליה אלא ה"ק להו וליהוי כמשקה להכשיר את הזרעים דדם נדה ודאי מכשיר את הזרעים ומטמאי' אותם כדתניא בתוספתא מסכת שבת פ"ח מנין לדם נדה שהוא משקה שנאמר ממקור דמיה ונאמר ביום ההוא יהי' מקור נפתח וכו' ואמאי לא שויתי' ליה מיהא כדם נדה להכשיר שהרי אף בשאר דמים דעלמא יש מכשירין והיאך אתם מקילין על זה שלא לעשותו כדם נדה אפילו בדבר שמצינו שאר דברים דינן כך ורבנן בעינא דם חללים וכל שאינו דם חללים טהור הילכך זה שאינו דם חללים טהור הילכך זה שאינו דם נדה לדברינו כשאר דמים שאין בהם חלל נדון אותו.\par ואקשינן לר' יוחנן נמי שפיר קאמרי ליה ה"ק להו אלפוה בג"ש כתיב הכא שלחיך פרדס רמונים כלומר כיון שבדם זה אסורה לבעלה וקרינן בה גן נעול יכשיר דעלה כתיב שלחיך ואמרי ליה רבנן אין אדם ג"ש מעצמו ואפילו לדברינו שתולין ואוסרין אותה לבעלה גבי הכשר לקולא ואין פי' זה אלא דברי נביאות.\par ואחרים פירשו השמועה כפשטה אלא שאמרו שר' יוחנן הוא דפריש וה"ק נהי דלא מטמא משום כתם תטמא משום רואה וכי אפרי' לההיא לישנא איפריך ליה דר' יוחנן, ול"נ שלא אמר ר' יוחנן ירד ר"מ לשיטתו של עקביא אלא מפני שאמר במשנתינו אם אינו מטמא דמשמע דלדבריהם אמר להו כלומר נהי דמקילתו בזה אודו לי מיהת בזו הא איהו כעקביא ס"ל ומטמא בזו ובזו הילכך לכולהו לישני איתא לדר' יוחנן וכולהו נמי קאמר להו כלומר לדבריכם כדפרישית וגמרא הוא דפריש ואזיל וה"ק להו וכו'.\par ומיהו הא ק"ל היכי א"ר יוחנן ירד ר"מ לשיטתו של עקביא וטמא והא לקמן בפ' בנות כותים א"ר מאיר אם יושבת הן על כל דם תקנה גדולה היא להן אלא שרואה דם אדום ומשלימתו לדם ירוק אלמא כרבנן ס"ל ואפשר דהתם לרבנן קאמר להו ונקט מראה טהור לדבריהם ולאו דוקא. 
 והא דאמרינן \textbf{אלפוה בג"ש דכתיב הכא שלחיך פרדס רמונים.} ודאי קשיא דר"מ לאו בכל מה שאשה משלחת מטמא משום הכשר דא"כ מאי איריא דם ירוק דנקט אלא בדם נדה בלחוד הוא דקא מכשיר מדכתיב גן נעול וכתיב פרדס מה פרדס הזה נעול להשתמר כך בנות ישראל נועלות פתחיהן לבעליהן וא"כ ה"ק נהי דלא מטמיתו משום דם נדה מכשיר והלא אין לך מכשיר אלא דם נדה והוא דם טהור הוא לדבריהם.\par וזו שאלה רבז"ל והשיב לר"מ כל היכא דחזיא דם אדום והדר קא חזיא כל דם מכשיר שכל זמן שהיא כפרד' שלחיך קרינן ביה וה"ק להו לדידי דם טמא הוא מכשיר מתחלתו אלא לדידכו אודו לי איהי מיהת דהיכא דחזיא דם אדום מעיקרא והויא פרדס אפילו ירוק יכשיר דקרינן ביה שלחיך ואמרו ליה ג"ש לא אמרינן הילכך אפילו דם אדום עצמו אינו מכשיר. 
לשמואל דאמר \textbf{כדם שור שחוט ולעולא דאמר של צפור חיה ולרב נחמן דאמר של הקזה.} ל"ק הא דתנן הרגה מאכולות תולה בה ובבנה ובבעלה דא"ל חד שיעורא הוא אלא לזעירי בלחוד דפריש הוא דמקשינן מינה הילכך אית לן לפרושי דכולהו לא פליגי אלא מר בקי בהאי חזותא ומר בקי באידך ודכולהו ודברייתא נמי חד הוא כדאמרן, והא דאקשינן הרגה מאכולת הרי זה תולה בה מאי לאו דכוליה גופא ואקשינן נמי תולה בבנה ובבעלה בשלמא בנה משכחת לה משמע דלא תלינן בכתמים אלא בדדמי ומקיפין ורואין ומכאן החמירו.\par ונמצא במקצת גליוני ראשונים דעכשיו בזמן הזה כיון שבטלו ראיית דמים ואין בקיאות במראיהן של ד' דמים אין תולין בכתמים ולא בבן ולא בבעל ולא במאכולת ושוק של טבחים ושאר כל מה ששנו חכמים בכתמים לתלות אלא בכולן אסורות לבעלה.\par ורבינו בעל התוספות השיב דכי אמרינן עברה בשוק של טבחים תולה וכן במאכולת בכולן מעצמה קאמרינן דתולה ולא מגופא אתא אלא מעלמא אתא, אלא מיהו היכא דידעינן ודאי דלא דמו ליכא למיתלי ומ"ה קאמרינן הכא בשמעתין למ"ד כדם מאכולת של ראש הרגה ודאי מאכולת של גוף היאך תולה בשאינו דומה ודאי וכן קושיא דבנה ובעלה אבל מסתמא תולין כתמים בכל מה שאמרו חכמים.\par וכענין זה כתב הראב"ד ז"ל דתולין מן הסתם כל מיני אדום באדום וכל מיני שחור בשחור עד שיתברר לה ודאי שאין אודם הצבע דומה לאודם הכתם דהתם אינו תולה כדתניא לקמן נתעסקה באדום אין תולה בו שחור ובשמעתין הכי אקשינן היאך אפשר לתלותו במכת בעלה ובדם מאכולת הגוף והלא דבר ברור הוא שאינו דומה, אבל מי שאינה יודעת בדמיונות או שהלך הצבע מנגד פניה ואינה יכולה לדמות תולה מן הסתם כדאמרינן באשה שבאת לפני ר' עקיבא ואמרה לו ראיתי כתם שמא מכה יש בידך וטהרה והרי אשה זו לא הביאה הכתם בידה לפניו וטהרה מיד ולא הקיף ולא ידע כל זה שכתב הרב ז"ל.\par וכן יש לפרש זו שהקשו בשמעתין ממאכולת ובעלה דה"ק ודאי מדבעין תליה בהני אלמא בכתם טמא עסקינן ואלו הנך דדמו כתמים טהורים הם ומה נפשך אי לא דמו לא תליא ואי דמו לא צריכי תלייה כלל, ולא בעי לתרוצי כשראתה ואבד ואינה יודעת מה ראתה דמסתמא ברואה ובאה לפני חכם תנן וידע דאיכא טובא מתירין הלכך בזמן שאין בקיאין תולה במאכולת הגוף ובבעלה ובכל מיני אדמות עד שיתברר לפי הדעת שאינן דומין וכן נהגו ואין לחוש.\par ושוב ראינו בכתמים שכתב הראב"ד ז"ל עיינתי בכל מילי דרבוואתא ולא אשכחית בהון דינא דכתמים אי נהיגי האידנא או לא, ואף על גב דאשכחן דמטמא את בועלה בפרק קמא דנדה דלמא לטהרות היא.\par וזה אינו נכון, וכבר השיב הרב חתנו ז"ל מדאמרינן בכתמים פעמים שהן מביאים לידי זיבה ואיתמר עלה מהו דתימא כל כהאי גוונא מביאה קרבן ונאכל קמשמע לן מביאה קרבן לאסרו לבעלה דאי לטהרות בלבד קרבן מאי עבידתיה דאפילו להכשיר בקדשים ליכא למיחש כיון דליתיה אלא דרבנן בעלמא לחוש בטהרות ולבעלה מתירין אותה לכתחלה.\par ועוד השיב מדאמרינן בהדיא בפרק הרואה (דף נח ע"ב) לדברי אין קץ שאין לך אשה שטהורה לבעלה שאין לך כל מטה ומטה שאין עליה כמה טיפי דם מאכולת לדברי חבירי אין סוף שאין לך אשה שאינה טהורה לבעלה וכו' אלמא כתמים לבעל נאמרו ואין צריך להאריך שהרי הוחזקו בנות ישראל שנהגו איסור בכתמים וקי"ל דמנהגא מילתא היא כרבי זירא, כל אלו דברי הרב ז"ל. 
\textbf{ת"ש דילתא אייתא דמא לקמיה דרבה בב"ח וכו'.} אי קשיא אדרבה איפכא מסתברא דאם כן דנאמנת אשה לומר כזה טיהר לי פלוני חכם ולטהר אפילו לחברתה, למה הביאתו ילתא קמיה דרבה תטהר לנפשה דהרו כל יומא נמי מטהר לה, א"ל קס"ד השתא דאיהי סברא דלא מהימנה לנפשה, אי נמי מחמירה על נפשה, אי נמי משום כבודן של חכמים הללו אינה רואה במקומן כדאמרן לעיל. 
\clearpage}

\newsection{דף כא}
\twocol{ה"ג \textbf{וכ"ת כי פליגי רבנן אירוקה ולבנה. הא אמר כל של שאר מיני דמים דברי הכל טהורה אלא אמר רב נחמן בר יצחק וכו'.} וה"פ: וכי תימא הכי קאמר ברייתא, המפלת חתיכה אדומה ושחורה או ירוקה ולבנה אינה טמאה לידה אלא אם יש עמה דם טמאה נדה. הלכך אדומה ושחורה הרי היא עצמה דם וירוקה ולבנה צריכה דם טמא עמה ואם לאו טהורה הא לר' יוחנן א"א לפרשה כן שהרי הוא אמר שלשאר דמים ד"ה טהורה ובודאי מצי לתרוצי כדאמרן למימר דלא תיקשי ליה אלא בחדא מיהו אנן ה"ק לר' יוחנן דודאי לא מוקי לה אלא כפשוטה קשיא בתרתי.\par ורש"י ז"ל גורס וכ"ת כ"פ רבנן אירוקה ולבנה אלא אדומה ושחורה למאן קתני לה וכו' ולא נהירא לי דכי אמרינן דרבנן אירוקה ולבנה בלחוד פליגי, ע"כ ברייתא ה"ק אדומה ושחורה טמאה ירוקה ולבנה אם יש עמה דם טמאה ואם לאו טהורה ורבנן גופייהו קתנו לה. 
\textbf{באפשר לפתיחת קבר בלא דם פליגי ובפלוגתא דהני תנאי וכו'.} ואי קשיא הא לת"ק דברייתא גופיה ספיקא משוי ליה ואלו ת"ק דמתני' קאמר ואי לאו טהורה א"ל התם משום דילדה ואינה יודעת מה ילדה וחוששין ללידה וחוששין נמי לזיבה שמא עם הנפל יצא דם אבל שילדה לידה יבישתא העמד אשה על חזקתה וטהורה היא ועוד שהרי בדקו ולא מצאו דם וכן ללשון הראשון שאמר רבנן סברי לא אמרי' רוב חתיכות מד' מיני דמים הן קשיא ותהוי נמי מחצה על מחצה תהא טמאה מספק אלא משום האי טעמא הוא דהעמד אשה על חזקתה.\par וי"מ דהתם ה"ק לא אמרינן רוב חתיכות מד' מינין הן ולפיכך טמאה גמורה אלא שאין שורפין כדאיתא בפ"ק, ואין פירוש זה נכון. }

\newchap{פרק \hebrewnumeral{3} המפלת חתיכה}
\twocol{ הא ד\textbf{אמר ר' יוחנן אם יש בה דם אגור טהורה.} איכא דמקשו ללישנא קמא דר' יוחנן ליחזי אם חתיכה זו מד' מינין טמאה שדרכה של אשה לראות דם נדה בחתיכה וא"ל אתיא כלישנא בתרא דר' יוחנן מאן דמתני לישנא קמא מתרץ הא דידיה הא דרביה, והאי דקאמר ר' יוחנן ד"ה טהורה לרבנן דמתני' ולר' יהודה מיירי. 
 הא דמקשינן לר' זירא \textbf{והא אמר ר' יוחנן משום רשב"י המפלת וכו'.} וא"ל אדמקשי ליה מיניה ליסייעיה ממתני' דקתני אם יש עמה דם אין בתוכה לא. וא"ל קס"ד השתא דמתני' לאו עמה לאפוקי תוכה אלא עמה לאפוקי חתיכה גופה דאפילו היא מארבע מינין טהורה ומדר' יוחנן מיפרשא ליה קושיא וממתני' לית ליה סייעתא בהדייהו.\par ואע"ג דאמרן לעיל בגמרא דאלו רבנן סברי עמה אין תוכה לא ההיא מימרא דגמרא היא ולא קס"ד השתא ומ"ה פריק אפילו בדר' יוחנן דהתם משום דדרכה של אשה לראות דם בחתיכה ומין במינו הוא ואינו חוצץ כדפרישית לעיל, א"נ א"ל דמתניתין לא סייעתא היא דקס"ד עמה לאפוקי תוכה משום שאין זה דם נדה אלא דם חתיכה. 
 הא דאקשינן \textbf{ת"ק נמי טהורי מטהר.} ומפרקי' אלא לאו דפלאי פלויי איכא בנייהו. לאו דצריך להכי דהא מצי למימר אלא שפופרת א"ב דת"ק סבר בבשרה ולא בשפיר ולא בחתיכה וכ"ש בשפופרת ואתי רבנן למימר אין זה דם נדה אלא של חתיכה הא דם נדה טמאה ואפילו בשפופרת אלא הא דאמרינן דפלי פלויי איכא בנייהו משום דקים ליה דבהא נמי פליגי לפום טעמייהו דכיון דר' אלעזר סבר דם אגור הוא א"א לטהר אלא בדלא אפלאי וכיון דרבנן סברי דם חתיכה גופה הוא אפילו איפלאי נמי ודאי טהור' הילכך מפרש ואזיל כולה פלוגתייהו. 
 ומהדר אביי \textbf{בשפופרת דכ"ע לא פליגי כי פליגי בחתיכה מר סבר דרכה של אשה לראות דם נדה בחתיכה.} פירש רש"י ז"ל דבפלאי פלויי פליגי מר דהו רבי אלעזר סבר דרכה של אשה לראות דם נדה בחתיכה וכיון דאיפלאי וליכא חציצה טמאה וכי לא אפלאי רחמנא מיעטה מבשרה ולא בשפיר ולא בחתיכה ורבנן סברי אין דרכה של אשה לראות דם נדה בחתיכה אלא האי דם חתיכה עצמה הוא.\par ולא מחוור דא"ה לא דמי האי דרכה של אשה וכו' לאותו שאמרו למעלה בדר' יוחנן דהתם קאמרינן דכיון דדרכה אע"ג דלא אפלאי נמי לא חוצה דהיינו אורחא ולישנא דגמרא נמי לא משמע הכי כלל.\par אלא ה"פ מרדאינהו רבנן סברי דרכה של אשה לראות דם בחתיכה ולא טהרו כאן אלא משום שאין זה דם נדה אלא דם חתיכה והיכא דהוי ודאי דם (חתיכה) [נדה] כגון מצא בה דם אגור טמאה והיינו לר' יוחנן ומר דהוא ר"א סבר אין דרכה של אשה הילכך הוי ליה כשפופרת ורחמנא אמר בבשרה.\par והאי דמדכרי' סברא דרבנן מקמי דר' אלעזרא"ל משום דאמרן לעיל דרכה של אשה כסברייהו א"נ לאו דוקא וכן בכמה דוכתי בתלמודא דלא קפדי.\par ובודאי דה"מ אביי לתרוצי כדרבנן הא דם נדה ודאי טמאה בדאיפלאי ובשפופרת דכו"ע לא פליגי דטהורה אלא ניחא ליה לתרוצי בדידה ולא לעיולי בה פילי דהשתא לא מוספינן בפלוגתייהו איפלאי פלויי כלל אלא בדם אגור בחתיכה פליגי כדפרישית, ועוד לאוקמה כדר' יוחנן דלעיל דלא לתקום דלא כחד כנ"ל.\par ויש מפרשים דלא ניחא ליה לאביי לאוקומה פלוגתא דרבנן אפלאי פלויי בלחוד דהא לא מדכרא בהדיא במילתיה דר' אלעזר דאנן בגמרא לאו חסורי מחסרא לברייתא כלל אלא מימר קאמרינן דלר"א בודאי פלאי טמאה ולא ניחא ליה לאוקומא פלוגתייהו אמאי דלא מתפרש בברייתא בהדיא.\par ולאו מילתא היא ור"א ורבנן תרווייהו מטהרין מר נסיב לה טעמא מבשרה ולא בחתיכה ולפום טעמיה איפלאי פלויי טמאה ומר נסיב לה טעמא אין זה דם נדה לטהורי נמי אפלאי פלויי אלא כטעמא דפרישית עיקר. 
\clearpage}

\newsection{דף כב}
\twocol{\textbf{א"ד האי ממנו עד שתצא טומאתו לחוץ.} ואי קשיא הכא איצטרך קרא לומר שלא יטמא בפנים דאלמא דינא הוא דמטמא והכא איצטרך בבשרה לומר שמטמאה בפנים אלמא אין דינה לטמא. לא קשיא דגלי רחמנא בחד ואיצטרך חבריה שלא תאמר זב וזבה הוקשו א"נ באיש דינו לטמאות משעקר שסופו לצאת מיד ולא האשה שהרי עומד בבית החיצון הרבה. 
\textbf{והלא עצמו הוא אינו מטמא אלא בחתימת פי האמה, למימרא דנוגע הוי.} פירש רב הונא אליבא דנפשיה פשיט ליה דס"ל כר' נתן דאמר זב אינו מטמא אלא בחתימת פי האמה ואל תתמה [דהא שמואל] רביה (דרבה) [דרב הונא] הוא דאמר נמי כר' נתן כדאיתא בפרק יוצא דופן ולפום הכי גמר רב הונא בעל קרי מיניה דזב וסבר לה נמי כר' שמעון דאמר בפ' ואלו דברים בפסחים דס"ל בזב כר' נתן דבעי פי האמה ואיתקש בעל קרי לזב ובעי נמי חתימת פי האמה כזב דהא קרא בבעל קרי לא כתיב ובזב כתיב או החתים בשרו.\par והא דדאיק מינייהו למימרא דנוגע הוי דלהכי בעינן חתימת פי האמה דליהוי נגיעת חוץ כדפי' רש"י ז"ל, ק"ל אי הכי זב נמי נוגע הוי ולמה לא יספור בזיבה, א"ל בזב ודאי אע"ג שנוגע הוי לענין שיעוריה מיהו הוי רואה לענין טומאה דיליה דאלו נוגע בזב טומאת ערב ואלו רואה טומאת שבעה והאי דאחמיר ד) עליה רחמנא בחתימת פי האמה דליהוי נמי נוגע גזירת הכתוב הוא שלא יהא טמא טומאת שבעה עד שיראה זוב ונגע בו מגע חוץ דה"ל רואה ונוגע ומ"ה אקשי' אא"ב בעל קרי רואה הוי ואפילו במקום שאינו טמא משום נוגע טמא הוא משום רואה הרי דומה לזב מצד אחד שאף הוא יש לו טומאה בראיה שאינו מדין מגע אא"א אינו טמא אלא בנוגע וטומאתו נמי טומאת מגע היא א"כ מה הנוגע בקרי אינו סותר בזיבה אף הרואה לא יסתור שהרי שניהן טומאה אחת להן בכל ענינן ומדין מגע טימאן הכתוב, ומפרקינן התם בשביל שא"א לה בלא צחצוחי זיבה ואם תאמר והלא אין בהם חתימת פי האמה ואין הזוב מטמא אלא כן י"ל כיון שיוצא עם שכבת זרע שהוא חותם פי האמה הרי הוא כנוגע ממש שמין במינו הוא ואינו חוצץ.\par והיינו דלא אקשינן יטמא טומא' שבעה אלא תסתור ז' דטומאה בזוב גמור לית ליה כיון דאינו רואה בשיעורו טומאת מגע זוב אית ליה וכיון דזוב הוא מיהא ובראיה דין הוא לסתור הכל שאין כאן ז' נקיים דהא הוה ליה כאלו ראה זוב בנתיים שאין אחר אחר לכולן.\par ומפרקי' גזרת הכתוב כך הוא מאחר שאין הזוב הזה כדי ראיה אין לו טומאת שבעה ואפילו לסתור ז' אלא סתירתו כטומאתו והא נמי רב הונא אליבא דנפשיה פשט ליה דהא בפרק כיצד הרגל בב"ק איכא ר' אליעזר דסבר אפשר בלא צחצוחי זיבה כלל, ומיהו בהא כרבנן פריק ליה ורבים נינהו.\par ואי קשיא לך לרב הונא דאמר בעל קרי נוגע הוי תרי קראי למה לי דהא כתיב רואה וכתיב נוגע וכדדרשינן בפרק יוצא דופן מדכתיב או איש, א"ל אע"ג דרואה דוקא בנוגע הוא דמטמא ה"א ה"מ ברואה דאיכא תרתי מגע וראיה אבל בנוגע לחודיה לא קמ"ל. 
הא דתנן \textbf{המפלת מין דגים וחגבים ושרצים אם יש עמהן דם טמאה.} אוקי' במחלוקת שנויה ורבנן היא ומדלא פרישנן הבי ברישא דקתני כמין קליפה כמין שערה כמין עפר וכמין יבחושין ש"מ דההיא ד"ה היא דכיון דמילי זוטרי נינהו אפשר לפתיחת קבר קטנה בלא דם ודמיא הא לההיא דמפרקינן בכריתות (דף ט') [דף י' {\small ואין הגירסא שם כן} ] כי אמרינן א"א לפתיחת הקבר בלא דם היכא דגמר הולד דמיפתח טפי ב) ואפשר לפתיחת הקבר בלא דם וכ"ש בכמין קליפה ויבחושין.\par ואם תאמר מ"ש ממפלת רוח דתנא באינה יודעת מה הפילה ר' יהושע אומר א"א לפתיחת הקבר בלא דם א"ל התם דהפילה שפיר יש בו פתיחת הקבר אבל יבחושין ועפר אינה לידה אלא כמקור שהפיל טיפין הוא וכרואה דם יבש בעלמא הוא והיינו נמי טעמא דלא אמרינן בדגים וחגבים ושרצים שאין עמהם דם תטיל למים ואם נמוקו טמאה דהתם בריה נינהו ודאי ונולדת היא אלא שאינה טמאה וכן בחתיכה דבשר גמור הוא אינה נמוחה אבל כאן רואה היא ומכה יש לה שממנה מפלת כן ואע"פ שהיתה בחזקת מעוברת והפילה לאו מעוברה הוא אלא של מכה הוא. 
הא דאמרינן מעיקרא \textbf{דנין יצירה מיצירה ואין דנין בריאה מיצירה.} לאו למימרא דסתרי אהדדי דהא אפשר לי' למיגמרינא לתרווייהו אלא ה"ק זו אינה ג"ש כלל, והיינו דאקשינן מאי נ"מ הא תנא דבי רבי ישמעאל ולא מפרק הני מילי היכא דליכא דדמי ליה אבל היכא דאיכא דדמי ליה מדדמי ליה ילפינן כדאתמר בעלמא אלא ודאי משום דלא אמרינן אלא היכא דסתרי אהדדי והכא תרווייהו דגמר.\par והדר אקשי' ועוד נגמר בריאה מבריאה [ומשני ויברא לגופיה] וייצר לאפנויי ודנין יצירה מיצירה דהשתא ודאי ליכא למיגמר אלא חד במופנה הילכך מדדמי ילפינן דאף על גב דוייצר מופנה גבי אדם ליכא למיגמר בריאה דתנין מיני' בדין מופנה מצד אחד דאפנויי דויצר דאדם לאו להך ג"ש הוא והוה ליה כשאינו מופנה כל עיקר אי נמי השתא לא מסיק טעמיה אלא מפרש ואזיל הוא ואמסקנא ניחא דליכא למיגמר דבריאה כלל כדבעי למימר קמן. 
 והא דאמרינן \textbf{ויברא גבי תנינים לאו מופנה.} אי קשיא הא כתיב נמי ישרצו המים ההוא אין כתוב בעשייה אלא בצוויי, ופי' רש"י ז"ל דכיון שאין מופנה משני צדדין ומשיבין ה"נ יש להשיב מה לאדם שכן מטמא מחיים.\par ול"נ דהאי לישנא קמא לא צריך פירכא דלכ"ע מופנה משני צדדין עדיף ממופנה מצד א' וכיון דע"כ יצירה יצירה גמרינן ה"ל בריאה דאדם לגופיה וגבי תנינים נמי לגופיה ואין מופנה כל עיקר וכל ג"ש שאינו מופנה כל עיקר אין למידן הימנה.\par וא"ת וייצר האדם לגופי' ודבהמה מופנה ודגמרינן ויברא דתנין לגופיה ודאדם מופנה וגמרינן היינו דקאמרי ומאי נ"מ זה כלומר אמאי ניחא לך לאפנויי לחדא לגמרי ומיגמר מינה ולא לאפנויי תרווייהו ומיגמר מנייהו ופריק לרבנן הא עדיפא דהא אין משיבין ולר' ישמעאל נמי הא עדיפא דהיכא דאיכא מופנה משני צדדין איהי עדיף ולהכי אפנויי רחמנא לבהמה משני צדדין דשדינן מופנ' דכולהו בגוה כי היכי דלא נימא באידך מופנה מצד אחד הוא דכל היכא דאיכא למישדי שני צדדין דמופנ' בדידיה שדינן ומיניה גמרינן בין לרבי ישמעאל בין לרבנן אבל ללישנא דרב אחא הויא דבעי' והאי מאי פירכא משום דאפילו כשאנו גומרין יצירה יצירה יכולין אנו לגמור בריאה בריאה אע"פ שאינה מופנה כל עיקר אלא שמשיבין ולפום הכי בעי' מאי פירכא ורבנן דפליגי עליה דר"מ במתני' לא גמירי כדאשכחן בפרק כל היד שאין אדם ג"ש מעצמו, וכן פי' רש"י ז"ל.\par ואי קשיא לך לר"מ מאי פירכא ליהדר דינא ותיתי מכאן דכיון דגמר יצורה ואתו בהמה חיה ועוף כי פרכת גבי תנין מה לאדם שכן מטמאו מחיים נימא בהמה תוכיח א"נ נגמר מוייצר דבהמה למד מלמד א"ל מה לשניהם שכן מטמאין במגע ובמשא תאמר בדגים שאינן מטמאין ואע"פ שמקבלין טומאה טומאת עצמן אין להם. 
\clearpage}

\newsection{דף כג}
\twocol{\textbf{למימרא דחיי.} פי' רש"י ז"ל דהא אחותה לא מיתסרא אלא בחייה דאין איסור אחות אשה לאחר מיתה ותמהני א"כ יפה שאל ר' ירמיה ונימא נפקא מינה לענין אתסורי באמה ואם אמה דאמות אפילו לאחר מיתה מן התורה ועוד נפקא מיניה לאתסורי באחותה שנים וג' ימים.\par אלא כך פירשו למימרא דחיי שאם א"א לו לחיות כלל אין קדושין תופסין בנפל גמור כגון בת שמונה והלא הרי הוא כאבן לכל דבר אלא ודאי סבר ר' ירמיה דחיי והאמר ר' יהודה אמר שמואל לא אמרה ר' מאיר אלא הואיל ובמינו מתקיים, פי' הואיל לאו דוקא דהא לא טעמא הוא לר"מ אלא משום יצירה יצירה או שגלגל עיניו כשל אדם או בשיש בו מצורת אדם אלא ה"ק לא אמר ר"מ שהוא ולד שיעלה על דעתו שהוא חי אלא ולד הוא לענין טומאה שבמינו מתקיים כנפל גמור שהוא אינו מתקיים ובמינו מתקיים. 
\textbf{א"ר ירמיה בר אבא אמר רב הכל מודים וכו'.} פי' ר' ירמיה משמיה דרב פליג אדשמואל ור' יוחנן דפרשי לעיל טעמיה דר"מ משום יצירה יצירה או משום גלגול עין שלדבריהם אפילו תייש גמור במעי אשה ולד מעליא הוא לטומאות לידה וכ"ש גופו אדם ופניו תייש דאיכא מקצת אדם.\par וה"נ משמע דס"ל לרב יהודה משמיה דרב כותיה מדקאמר הואיל ובמינו מתקיים ולרב ירמיה משמיה דרב לית ליה הנהו טעמי אלא ר"מ ורבנן בסברא בעלמא פליגי בשפניו אדם ונברא בעין א' כבהמה שר"מ אומר מצור' אדם בעינן והא איכא וחכמים אומרים כל צורת ממש.\par וה"ה לר' מאיר' דבמצח ועין וגבן העין ולסת וגבת הזקן סגי אלא להכי נקט פניו אדם ועין אחד כבהמה להודיעך כחן דרבנן והא דאמרי ליה רבנן והא איפכא תניא לאו איפכא תניא דוקא דמתהפכי תרתי סברי דהא לא (מתסרא) [מתהפכא] סברא דרבנן לר"מ אלא איפכא בלישנא לכולהו ואיפכא בסברא דר"מ דמפכא לה ברייתא לרבנן וכדפירש רש"י ז"ל.\par ולסבריה דרב הא דתני' לקמן המפל' דמות נחש אמו טמאה לידה ר' יהושע יחידאה היא ולית ליה דר"מ וכ"ש דרבנן וההיא דתניא לעיל נראין דברי ר"מ בבהמה וחיה בהכי נמי מתוקמ' בבהמה וחיה ומקצ' סימנין דאדם דכיון דהיא עצמה עיניה הולכות כשל אדם במקצת סימנין נעשית כאדם גמור מה שאין כן בעופות שאפילו כל פניו כאדם ועיניו לצדדין אינו כלום.\par והאי דדחי רב אחא בריה דרבא לעיל תבדוק לרבנן דמודו רבנן בקריא וקפוף הואיל ויש להם לסתות כאדם דחייה בעלמא היא דדחי בסברת דר' אלעזר בר צדוק אבל לפום מסקנא לרבנן לסתו' חדא מצורות דפנים נינהו וצריך נמי גבין וגבת זקן דאדם ועין נמי דאדם ואף ע"פ שהולכות לפניהם צריך צורת דאדם באוכמא.\par וי"מ דלרב ירמיה גופיה אית ליה אליבא דר"מ יצירה יצירה ואי כולה תייש בפניו וגופו אמו טמאה לידה הואיל ובמינו מתקיים והא דאמה גופו אדם ופניו תיש ולא כלום משום שאין זה לא מין בהמה ולא מין אדם והואיל ואין לו מין שמתקיים דברי הכל ולא כלום.\par ולפי דבריהם קשיא, א"כ מנא ליה לר' ירמיה א"ר דר"מ בפניו אדם ונברא בעין א' כבהמה פליג דילמא בההוא כרבנן ס"ל דלאו אדם הוא ומין בהמה נמי אינו שאין לך בבהמה כמותו והם אומרים קסבר רב דר"מ ורבנן בתרתי נמי פליגי ממאי מדאמרי ליה רבנן כל שאין בו מצורת אדם ולא קתני וחכמים אומרים אמו טהורה א"נ וחכ"א אינו ולד שמעי' דר"מ דמטמא נמי במקצת צורה ואמרי ליה לא כל צורה בעי למעוטי צורה בהמה גמורה ולמעוטי נמי מקצת צורת אדם והא דאמרי ליה רבנן והא איפכא תניא איפכא לגמרי הוא שר"מ אמר כל צורת לגמרי מ"ט או כולו אדם או כולו בהמה וחכ"א מצורת אדם ולא פניו בהמה אבל במקצת צורת אדם ולד הוא והיינו דקתני מתני' כל שאין בו מצורת לאפוקי כולו בהמה ומדלא קתני אמו טהורה סתם משמע ליה דבתרתי פליגי וברייתא נמי דמסייעא להו כך מפורש בספר הישר.\par ודברי רש"י ז"ל יותר נראין והוא הלשון הראשון שכתבנו ואע"ג דקשיא נחש דר' יהושע כדפרישית.\par והא דתניא המפלת דמות לילית אמו טמאה לידה ולד הוא אלא שיש לו כנפים ולא אמרינן משום דגופו תיש ופניו אדם היינו נמי טעמא משום דודאי פניו אדם אע"פ שגופו תיש בתר צורת פנים אזלינן אבל בדמות לילית ס"ד אין כאן צורת אדם כלל אלא צורת לילית היא זו בין בגוף בין בפנים קמשמע לן דלילית גופה ולד הוא אלא שיש לו כנפים. 
\clearpage}

\newsection{דף כד}
\twocol{\textbf{אמר רבא ושטו נקוב אמו טמאה.} פירש"י ז"ל קסבר טרפה חיה. וקשה להעמיד דברי הרב רבא שלא כהלכה ועוד אני תמה וכי מפני שאין טרפה חיה י"ב חדש נטהר אמו של זה והלא הנפל שהוא כאבן ואינו יכול לחיות או שיצא מת ומחותך אמו טמאה זה שיצא נקוב הושט וכלו לו חדשיו לא כ"ש ואפילו הולד שנימוק בשליא אמו טמאה והאיך יטהרוה שאלו היה דבר שמתחלת ברייתו הוא אפשר לומר שאינו ראוי לבריית נשמה וטהור אבל זה שמא עכשיו נקב ומה בינו למחותך ויצא איברים אברים.\par לפיכך נ"ל שכל הנולד בטריפות ודאי אמו טמאה היא ואפילו היה טרפותו בתחלת ברייתו שהרי ראוי הוא לחיות י"ב חדש וכ"ש בטרפות נקב וחתך. והא דאמר רבא ושטו נקב טמאה לד"ה קאמר ולא בא אלא להשמיעינו שושטו אטום אמו טהורה שאין זה בכלל אדם הואיל ונברא שלא כדרך החיים. 
 והא דאמרי' \textbf{קא מפלגי בטרפה חיה} שהזיקיקו לרש"י ז"ל לטהר ולד טרפה נ"ל שלא הקפידו אלא על לשון הברייתא שאמרו וכמה כדי שינטל מן החי וימות דקסבר האי תנא דכל שנברא אטום בלא חיתוך איברים עד מקום שאלו ינטל מן החי וימות אינו בכלל ולד ולא שיהא זה נקרא טרפה אלא זה אינו נולד הואיל ונברא אטום אבל נחלקו האמוראין כמה הוא כדי שינטל מן החי וימות ופי' ר' זכאי עד לארכובה ודקדקו ממנו שהוא סובר טרפה חיה דהא קאמר שבכך החי מת (לרש"י נטל) [לכשינטל] ממנו ור' ינאי אמר עד לנקובה שבכך נעשה נבלה אבל טרפה אינה מתה ר' יהושע דאמר עד טבורו קסבר בין זו בין זו חיות הן וכל זה אינו אלא בשיעור כמה כדי שינטל מן החי אבל בולד שנטל ממנו לא נחלקו (במינו) [בו] ולא אמרו כאן אלא הולד כשהוא אטום.\par ומה שפי'רש"י ז"ל אטום חסר אינו נראה אלא אטום כמשמעי שאין לו חיתוך איברים ואין לו חלק שבהן אלא כמין גולם אטום ודמיא להא דתניא לקמן בריית גוף שאינו חתוך וכו'. 
 הא דאקשי' \textbf{ואם איתא ליתני שמא מגוף אטום (ופניו) [או ממי שפניו] המוסמסין באתה.} א"ל איבעי למיתני טובא ליתני שמא באת מפניו תיש או אפילו פרצוף אחר או שיש לו שני גבין ושתי שדראות וכמין אפיקותא דדיקלא וכן כיוצא בהן א"ל הנהו לא שכיחי ולא ה"ל למיתני. אבל פניו ממוסמסין ה"ל למיתני משום דשכיח נמי טפי מגוף אטום. 
הא דאמרינן \textbf{ושמואל סבר בריה בעלמא איתא וכי אגמריה רחמנא למשה בעלמא.} פירש"י ז"ל אותו המין אסר לו וק"ל א"כ לשמואל אפילו יוצא לאויר העול' נמי לישתרי דה"ל כקלוט בן פרה דשרי ונראה מדבריו דבין לרב בין לשמואל במעי טהורה לא חיי הלכך יצא לאויר העולם משום נפל אסור אפילו לשמואל והא דפריך רב שימי ממתניתין ר' חנינא בן אנטיגנוס אומר וכו' לרב ה"ה לשמואל אלא גביה הוה קאי דבר בריה הוה.\par ולא נהירא ועוד דהתם בפ' ואלו מומין (דף מג ע"ב) תנן לה למתני' גבי מומי כהן איזהו גבן ר' חנינא בן אנטיגנוס אומר כל שיש לו שני גבין ושדראות והוי ביה למימר' דחיי והאמר רב באשה אינו לד בבהמה אסור באכילה ולא מדכרין התם דשמואל בכלום בעולם.\par אלא הכי משמע פירושא לכ"ע מינא בעלמא ליכא כי פליגי בבריה רב סבר אפילו בריה בעולם ליכא דלא חי הילכך כי אגמריה רחמנא למשה במעי בהמה אגמריה דבחוץ לא צריך נפל הוא. ושמואל סבר בריה בעלמא איתא דחיי וכי אגמריה רחמנא בשיצא לאויר העולם דלא תימא כקלוט בן פרה הוא אבל במעי בהמה דאפילו נפל שריא איהי נמי שרי. 
\clearpage}

\newsection{דף כה}
\twocol{\textbf{המפלת שפיר מלא בשר נימוח מהו.} פי' קא מיבעיא להו לרבנן דפליגי עליה דר' יהושע מיפלגי נמי בבשר נימוח או לא אמר להם לא שמעתי אמר לפניו ר' ישמעל בר' יוסי משום אביו כך אמר אבא מלא דם טמאה נדה מלא בשר טמא' לידה שהיה ר' ישמעאל סבור שלא אמר אביו כדברי היחיד ולפיכך דחה רבי ואמר שמא כדברי ר' יהושע אמרה וזה שאמר מלא בשר לאו דוקא אלא ה"ה למחוי עכור ולא בשר אלא להוציא צלול אפילו לר' יהושע.\par ויש שגורסין בה כמאן כסומכוס מדהא כיחידאה הא נמי דילמא כר' יהושע אמרה ואינו בספרים.\par וא"ת כיון שרבי לא קבלה אפילו בבשר נימוח שיהא ולד ריב"ל מנין לו דקאמרי בצלול מחלוקת אבל בעכור ד"ה ולד זה אינה שאלה דריב"ל כר' יוסי ס"ל וקסבר ר' יוסי לרבנן אמרה כדסבר נמי ר' ישמעאל בר' יוסי ועוד דכיון דרבי לא שמעתי אמר אינה תשובה לדברי ריב"ל שאם ר' לא שמע ריב"ל שמע לא ראינוה אינה ראיה.\par וי"א אין אומרים בדברים אלו זו דומה לזו שאפשר למימר עכור ולד ובשר נימוח שמא אינו ולד. ואיכא למימר נמי איפכא ולפיכך נחלקו בכולן. 
\clearpage}

\newsection{דף כו}
\twocol{הא ד\textbf{אמר רב הונא בר תחליפא משמיה דרבא ולד מדנפיק קבא דרישיה הויא ליה לידה סנדל עד דנפיק רוביה.} פי' רש"י ז"ל משום שאין הראש פוטר בנפלים. ושמואל דאמר הכי איתותב במס' בכורות אלא איכא לפרושי דאפילו למ"ד הראש פוטר בנפלים ה"מ בולד שלם או אפילו במחותך שנגמר' צורתו אבל סנדל שלא נגמרה לו צורה לא חשיב רישיה למהוי ביה לידה. 
 הא דאמרי' \textbf{תלת מתני' ותרתי שמעתא שיעורן טפח.} ואקשי' תרתי חדא היא היינו טעמא דלא מקשינן תלת ד' הוויין משום דהוה איכא למימר רבי שילא לית ליה דר' חייא דשיעור אזוב טפח ומ"ה מקשינן אם כן [תרתי חדא היא, ועוד א"ל] תרווייהו כי הדדי נינהו וחדא נקט. 
והא דאמרינן \textbf{השתא דאתית להכי הך נמי פלוגתא היא דקתני סיפא א"ר יהודה לא אמרו טפח אלא מן התנור לכותל.} ק"ל וניחשוב מן התנור לכותל דמודה ר' יהודה וה"מ למיחשבי' בדברי הכל וא"נ לא חשיב חד דדברי הכל ה"מ למימר סתמא אבן היוצא מן התנו' טפח דהא קא חשיב בכה"ג טפח סוכה למר בדופן ג' ולמר בדופן ד' כיון דכולהו אית להו דופן טפח קחשיב ליה ה"נ כולהו אית להו אבן היוצא מן התנור טפח מכאן או מכאן.\par ואיכא למימר התם הכל מודים ששיעור דופן א' בסוכה טפח והשאר כהלכתן. וכי אמר דופן סוכה טפח ליכא למטעי במידי דפלוגתא. אבל הכא אי אמר סתמא אבן היוצא מן התנור הוה משמע מכל צדדין ואתיא כרבנן ואי פריש נמי אבן היוצא בין התנור לכותל טפח הוה משמע הא בין תנור לבית אינו טפח כר' יהודה ואיהו במילתא דפלוגתא לא איירי לא כמר ולא כמר. ולא בעי לפרושי תרווייהו ובודאי משמע לכאורה דהשתא דאתינן להכי ומפרקינן דהך נמי פלוגתא היא הדר ביה מתירוציה דקאמר כי קאמרי היכא דבצר מטפח לא חזי וכו'.\par והשתא קשיא לי טובא וליחשוב הני דתנן במס' כלים פי"ח גדד לשתי כרעים טפח על טפח לוכסן או שמעטה פחות מטפח טהורה רישא ל"ק דטפח על טפח לא קאמרינן סיפא ליחשוב. ועוד שם בפרק (בתרא) [כ"ט] חוט מאזני' של חנוני ושל בעלי בתים טפח יד קרדום מלפניו טפח שירי הפרגל טפח יד מקבת ושל מפתחי אבנים טפח. וי"ל כי קאמרינן היכא דבציר מטפח לא חזי והני כ"ש דבציר מטפ' (דידהו) [דידות] הוי וכי קאמרינן השתא דאתית להכי לאו למימרא דליתי' לשנויה דשנינן אלא למימר' דמההי' לא תיתי תיובתא בבי מדרשא כלל. 
 הא דאמר רב \textbf{אין הולד מתעכב אחר חבירו כלום.} ראיתי מקצת בעלי פירושין שכתבו דלית ליה לרב הא דאמרינן לקמן מעשה ונשתהא ולד אחרונה אחר חבירו שלשים יום ולית ליה נמי מעשה דיהודה וחזקיה ולית ליה נמי האי דאמרינן בכתובות וביבמות ג' נשים משמשות במוך קטנה מעוברת מניקה מעוברת שמה תעשה עוברה סנדל אלא קסבר אין אשה מתעברת וחוזרת ומתעברת ולפיכך אין ולד מתעכב אחר חבירו כלום.\par ואין דבריהם נראין דג' נשים מתניתא הוא ונימאתהוי תיובתיה דרב. ועוד מעשה דיהודה וחזקיה בני חביביה דהוא יושב לפניו והן יושבין עמו בבה"מ היכי אפשר דלא חזי ליה ואם איהו אומר דלא היו דברים מעולם מאן מהימן לאסהודי עלייהו.\par אלא היינו טעמא דרב דקסבר אין אשה מתעברת וחוזרת ומתעברת בין נפל בין של קיימא אלא א"כ נעשה א' מהם סנדל וסנדל כרוך עם הולד הוא יוצא שחבור אתו ונדבק בו והיינו טעמא דמוך אבל כשהאשה מתעברת תאומים טפה אחת היא שמתחלקת וכשהן נגמרין לז' או לט' אין הולד מתעכב אחר חבירו כלום אלא א"כ הפילה א' נפל וא' שליא אבל פעמים שנתחלקו לשתים וא' מהן נגמר לט' וא' לז' ובזה מודה רב שהול' משתהא אחר חבירו כדי שתגמור צורתו בזמנו. והיינו מעשה דיהודה וחזקיה ומיהו לית ליה אפוכי שמעתא דלקמן (כ"ד) [כ"ג] לולד דבחד ירחא לא משתהא אלא ל"ג אית ליה.\par והא דאמרינן לעיל סנדל מהו דתימא הואיל וא"ר יצחק עד קמ"ל שניהם הזריעו בבת אחת ודאי קשיא דהא איכא סנדל דמתעברת וחוזרת ומתעברת וזה זכר וזה נקבה כדפרישי' במשמשת במוך. ואיכא למימר אין ודאי דמצי למימר הכי אלא שמא תאמר היכא דבעל ופירש מדהאי זכר האי נמי זכר קמ"ל אפי' בכה"ג חיישינן שמא שניהם הזריעו כא' והאי זכר (נמי) והאי נקבה כנ"ל. 
\textbf{אין תולין את השליח אלא בדבר של קיימא.} פי' רש"י ז"ל שכיוצא בו מתקיים אם היו חדשיו כלין למעוטי שאם הפילה דבר שאינו ראוי לבריית נשמה כגון נברא בירך אחת או גוף אטו' ואח"כ הפילה שליא (פי') [אפי'] בתוך ג' חוששין לולד אחר, ופירוש חזייה לרב יהודה בישות משום דשמעה מרב ולא אמרה.\par ואינו מחוור דבן קיימ' לאפוקי כל נפל משמע וכדאמרן דילמ' כאן בנפלי כאן בבן קיימא ולא ידעתי מי הזקיקו לשנות פירושו אלא הא דתלמיד' דרב פליגא אדרב יהוד' דאמר לעיל משמיה דרב הפילה נפל ואח"כ הפיל' שליא כל שלשה ימים תולין אותה בולד ושאר תלמידים דרב אומרי' משמי' דאין תולין את השליא בנפל אפילו יום א' אלא א"כ יצאה עמו אבל בבן קיימא תולין אותה אפילו מכאן ועד י' ימים.\par ושמעתי שפירשו בירושלמי במס' זו (ג, ד) לפי שאין השליא פורשת עד שיגמר לפיכך אין תולין אותה בנפל.\par ובשאלתות דרב אחא משבחא ז"ל כתב לכך תולין אותה בבן קיימא דאמרי' אגב חיותא דולד בזעא לשליא ונפיק. אבל נפל דלית ביה חיותא לא. ומ"ה חזייה שמואל לרב יהודה בישות דשמעיה דאמר משמיה דרב דכל ג' תליא שליא בנפל וכיון דשמעינהו לכולהו תלמידי דרב דאמרי אין תולין כלל אמר ודאי רב יהודה טעי. 
\clearpage}

\newsection{דף כז}
\twocol{\textbf{מ"ט דר' שמעון וכו'.} פירש"י ז"ל נהי נמי דנימוק מ"מ גופו של מת כאן הויא וה"ל כרקב וכנצל. וק"ל הא דאמר רשב"ל לקמן בשמעתין שפיר שטרפוהו במימיו טהור להוי כרקב וכנצל. ועוד לר"מ נמי בבי' החיצון אמאי טהור ליהוי כרקב וכנצל. וא"ת איהו נמי סבר כל טומאה שנתערב בה מין אחר בטלה אלא מאן תנא דפליג עליה דקאמרת קסבר ר' שמעון והא דתניא מלא תרוד רקב שנפל לתוכו עפר כל שהוא טמא ור' שמעון מטהר אמאי תרמייה הא דכ"ע טומאה כיון שנתערב בה מין אחר טהורה.\par ובתוספ' הקשו לה מדאמרינן ואזדא ר"ש לטעמיה מדאמר א"א שלא ירבו שתי פרידות עפר על פרידה אחת של רקב ויבטלנו ואמאי נהי נמי דבשיעור מצומצם כגון מלא תרוד רקב איכא למימר הכי גבי שליא מ"ט דאפילו הוה בה תרי שיעורי דמלא תרוד מטהר ר"ש דסתמא תנן ואע"ג דליכא למימר א"א שלא ירבו וכו', והם מפרשים הסוגיא כולה בענין אחר ברם נראין דברי מקצת ראשונים שפירשו מ"ט דר"ש דמטהר לגמרי והרי אנו מוצאים בכל יום ולדות חתוכים בשליא והאיך אפשר שלא תהא בכולה כזית ג) שלא נמוק לגמרי קודם שתצא שאפי' נחתך כולו לחצאי זתים מצטרפים הן בתוך השליא לטמא באהל ואמאי מטהר ליה לגמרי, ומפרקי' קסבר ר"ש כל טומאה שהיא כשיעורה ולא יותר שנתערב בה מין אחר בטלה דאמרינן כיון דהיא צריכה שיעור א"א שלא ירבה מין אחר על מקצתה ומבטלה ואף כאן כיון שנמוק הולד אע"פ שנשתייר ממנו (כחצאי) ד) זתים א"א שלא ירבו שתי טיפי מים ודם על מקצת בשר שלא נמוק ומבטלו והיינו דאמרינן ואזדא ר"ש לטעמי' דאמר א"א שלא ירבו שתי פרידו' עפר על פרידה אחת של רקב ומבטלו ובצר לה שיעורא ור"מ סבר לא מבטל אלא א"כ הוציאו לבית החיצון שנטרפו מימיו לגמרי וטהור. והא דאמר לו לר"מ כשם שאינו בבית החיצון כך אינו בבית פנימי לומר שאף בבית החיצון היה לנו לחוש שמא יש בו כזית בשר (שנמוק) ה) אנא משום בטול ואמר להו אינו דומה שזה נמוק לגמרי וה"ל כמים בטריפת בני אדם. אבל דרך לידה אינה נמוק לגמרי וביטול אינו מועיל. ואקשי להו רבא לרבנן דבי רב אדרבה כיון דרקב יותר הוא מן העפר היאך יאמ' ר"ש שהמועט רבה על מקצת המרובה ומבטלו ומגרע שיעורו אדרבה יש לנו לומר שהמרובה עומ' לבד על הממועט ומבטלו לגמרי. והיינו דאמרי' לקמן והשתא דאמרת טעמיה דר"ש סופו כתחלתו גבי שליא מ"ט דקס"ד שהמים והדם שבשליא מועטין הן אלא שרבין על מקצת בשר ומבטלין אותו כדפרישית וא"ר יוחנן משום ביטול ברוב נגעו בה שאפילו היו שם שני חצאי זתים או כזית שלם. יש במי שליא ודם שבה לבטל את כולה ואין אנו צריכין לומר שרבין על מקצת ומבטלו ומגרע שיעורו אלא על כולו הם רבים ומבטלין אותו ובהא פליגי דר"מ סבר אין טומאה בטלה ברוב מלטמא במשא ואהל דהא (קאמר) ו) מאהיל על כולה ומיהו בבי' החיצון טהור שנמוק לגמרי וה"ל אפר שרופין ופחות ממנו. ור"ש סבר בטלה היא לגמרי. 
\textbf{מלא תרווד ועוד עפר בית הקברות טמא.} פי' רש"י ז"ל לאו רקב של מת ממש אלא כגון שנקבר בכסותו או בקרקע בלא ארון ויש כאן מלא תרוד ועוד מאותו עפר דהיו מעורבין עפר ורקב.\par ואין פי' זה נכון דהא אמרין כל שתחלתו דבר א' נעשה גנגילון ואע"פ שיש בו שיעור מן הרקב דומיא דסיפא ומדאמרינן סופו כתחלתו (מאי) [מה] תחלתו דבר א' נעשה גנגילון אלמא פשיטא מילתא דכ"ע כל דבר א' עושה גנגילון בתחלתו. ועוד מדתניא איזהו מת שאין לו רקב נקבר בכסותו אלמא אין לו רקב כלל אפילו מלא סאה דאלת"ה ליתני שאין לו תרוד רקב. ועוד מדסוגיא במסכ' נזיר פרק כהן גדול (דף נ"א) דאמרי' שני מתים שנקברים זה עם זה נעשה גנגילון זה לזה. ואם קברן זה בפני עצמו וזה בפני עצמו והרקיבו ועמדו על מלא תרוד רקב טמא אלמא אין הדבר תלוי בשיעור מן הרקב אלא כל דבר שנקבר עמו נעשה לו גנגילון. וכן בכולה סוגיא דהתם הכי משמע דכי גמירי למלא תרוד רקב דוקא דנרקב בעיניה.\par אלא הא דתניא מלא תרוד ועוד עפר קברות פירושו כגון שנקבר המת ערום על גבי רצפה של אבנים והרקיב ונפחתה מערה ונתערב עפרה ברקב דקסבר ת"ק רוב מתים יש בהן רקב מלא תרוד שבכאן של מת והמותר הוא עפר בית הקברות ואלמלא שנמצא שם ועוד על כרחינו מת זה לא היה בו מלא תרוד דא"כ עפר בית הקברות להיכן הלך אבל מאחר שנמצא כאן ועוד זה א"א בלא מלא תרוד של מת ור"ש מטהר וחזרו לטעמייהו דהיינו מלא תרוד רקב שנפל לתוכו עפר כל שהוא. 
וא"ר יוחנן ד\textbf{ר"ש וראב"י אמרו דבר א'.} ואליבא דר' חייא דפריש למתניתן דתקבר לומר שנפטרה מן הבכורה ולא משום טומאה. והך סוגיא דר' יוחנן הוא דהתם בדוכתא במס' בכורות (כג, א) מסיק טעמיה דר' חייא משום דה"ל טומאה סרוחה. וצ"ע. 
\textbf{שפיר שטרפו במימיו.} גרסי' וכן בפר"ח ז"ל, ופי' שטרף השפיר ונמוקה צורתו אבל עדיין הוא קיים נעשה כמת שנתבלבלה צורתו באור וטהורים דכיון שאין באבריו צורת בשר ולא צורת עצם נפק ליה מדין כזית ועצם כשעורה וטהורין לגמרי. 
 והא דא"ר יוחנן \textbf{מת שנתבלבלה צורתו מנ"ל דטהור.} לאו דוקא דלא כרבנן אלא מדר' אליעזר שמע ליה ר' יוחנן דקסבר מודו ליה רבנן בשלא נעשה אפר כדפרישי'.\par ויש לפרש דקסב' רבינא דר"ילא מודה לי' לר"ל בשפיר שטרפו מימיו דמדלא א"ל בשלמא שפיר שנטרפו מימיו דקאמר טהור לחיי אלא נתבלבלה צורתו שלמה מנלן אלמא ה"ק מנלן דטהור דגמרת מינה לשפיר לא הא ולא הא איתנהו. ועלה קאמר רבינא דר"י דמטמא שפיר שנטרפו מימיו לגמרי כר' אליעזר אמרה דהאי כאפר שרופין הוא ומיהו במת שנתבלבל' צורתו דקא מתמה מנלן לד"ה אתיא.\par וזה הלשון לדברי מי שגורס שפיר שנטרפו מימיו דמשמע שנטרף לגמרי וחזר למים, אבל לפי גר"ח ז"ל שנטרפו במימיו. נראה דהיינו נתבלבלה צורתו בלחוד.\par ויש לי עוד לומר דר' יוחנן הלכה קא מיבעי ליה, וה"ק ליה מנלן דטהור כרבנן דילמא טמא כר' אליעזר דמסתברא טעמיה. אילימא מדרבי שבתאי קא גמרת הלכה דהוא אמורא וקא פסיק הלכה כרבנן. 
\clearpage}

\newsection{דף כח}
\twocol{\textbf{מעשה היה וטהרו לו פתחים קטנים.} פרש"י ז"ל טמאו לו פתחים גדולים של ד' טפחים וטהרו לו קטני' הפחותים מד' כשאר מתים גדולים שהפתחים הגדולים מצילין על הקטנים דקי"ל פתחו בד'. וה"מ להציל על הפחות מד' טפחים והכי אמרינן במס' אהלות המת פתחו בד' בד"א להציל על הפתחים אבל להוציא את הטומאה בפותח טפח. זה כתב הרב ז"ל.\par ואין הדין הזה אלא כשהפתחים כולן סתומים או מגופין שבהן שנינו המת בבית ולו פתחים הרבה כולן נעולין כולן טמאין, פי' משום דסוף טומאה לצאת נפתח א' מהם אע"פ שלא חשב עליו טיהר את כולן פירש כיון דנעולין הן וה"ה למגופין אבל בפתוחין כל פותח טפח מוציא טומא' לצד ב' ואין לו הצלה כלל. 
\textbf{המפלת יד חתוכה.} פי' רש"י ז"ל חתוכה שיש לה חיתוך אצבעות. וק"ל בלאו הכי נמי ליחוש ללידה שהרי אפילו השפיר שאין לו אפילו חתוך ידים עצמן אמו טמאה לידה.\par ואיכא למימר הכי ספיקא הוא ואם הפילה יד גמורה שאינה חתוכה אומרי' מגוף אטום באת כשם שהיא משונה כך באת מגוף משונה ושמא לאו מגוף באת אלא חתיכה של בשר שנעשית כמין פיסת היד היא הילכך אמו טהורה תולין להקל שרגלים לדבר.\par והרב ר' אברהם בר דוד ז"ל מפרש שלא אמרו חתוכה אלא לענין מביאה קרבן ונאכל דמדקתני ואין חוששין כלל במשמע ואלו בשאינה חתוכה אינו נאכל. (אלא) א) לענין האם טמאה מ"מ. ואין זה לשון הגון מדקתני ברייתא אמו טמא' ואין חוששין ולא קתני מביאה קרבן ונאכל ואין חוששין. 
\clearpage}

\newsection{דף כט}
\twocol{והא ד\textbf{אמ' רב פפא כתנאי יצא מחותך או מסורס.} אלישנא קמא דר"א ור' יוחנן קאי דפליגי במחותך.\par והא דאקשי ליה רב זביד למאי דמוקי ר' יוסי אומר משיצא רובו כתקנו מכלל דמסורס רובו נמי לא פטר קשיא ולימ' ר' יוסי אכתקנו פליג דלא מיפטר בראש עד שיצא רובו ומחמיר הוא.\par וי"ל מדקתני ברייתא בדר' יוסי משיצא כתקנו משמע דלאיפלוגי אמסורס אתא דה"ל למימר ר' יוסי אומר כתקנו משיצא רובו וליפלוג אכתקנו והשתא פלוגתא אמסור' משמע ואוקומא רב זביד משיצא לתקנו בחיים ופלוגתא בדלחיים היא.\par והיינו נמי תנאי דת"ק סבר מחותך ומסורס משיצא רובו הרי זה כילוד כתקנו אע"פ שמחותך הראש פוטר ר' יוסי אומר משיצא כתקנו לחיים. כלומר אין הראש פוטר במחותך אלא בשלם שכיוצא בו יוצא כתקנו לחיים ולכ"ע לית להו דשמואג אלא ס"ל בשלם הראש פוטר והא דקתני איזהו כתקנו לחיים משיצא ראשו ה"ק איזהו כתקנו שיוציא ראשו תחלה ויוציא כדרך שהחיים מוציאין שר' יוסי אומר רוב ראשו וזהו שיעור אבל משיצא ראשו לאו לשיעור קתני אלא לדרך לידה קתני.\par ואם באנו לפרש משיצא כתקנו לחיים לאפוקי נפל אפילו שלם כדשמואל ומאי כתנאי אפי' ללישנא בתרא דר"א ור"י תיקשי לן במס' בכורות פ' יש בכור לנחל' סלקא דשמואל בתיובתא ולא איתוקמא התם כתנאי. אלא שיש כיוצא בה בתלמו' תיובתא בחד דוכתא ותנאי בדוכתא אחריתי במס' תמורה. ועוד יש במסכת פסחים תיובתא בדריש שמעתא ותניא כותיה בשלהי דידה בפ' ערבי פסחים. אלא שאנו קיימנו שתיהן תיובתא. ותניא כותיה בשתי שמועות של ר' יוחנן וכבר כתבנו זה בספר המלחמות. 
 הא ד\textbf{אמר ריב"ל עברה בנהר והפילה וכו'.} בדין הוא דנירמי עליה מהא דתניא בריש פירקין ולשלישי הפילה ואינה יודעת מה הפילה מביאה קרבן ואינו נאכל אלמא לכ"ע הלך אחר רוב נשים לא אמרינן אלא מתוקמא ההיא כדתרצינן למתני' בשלא הוחזקה עוברה לפנינו. ודמתני' עדיפא לן למירמי. 
הא דתניא ב\textbf{אשה שיצאה מלאה ובאת ריקנית} דמחזקי' לה ביולדת בזוב וברואה נמי לאחר לידה כדמקשי' לקמן יומא קמא דאתיא לקמן ליטבלה דילמא שומרת יום כנגד יום היא אלמא ברואה השתא בימי זיבה מחזקין לה דוקא כגון שבאת ריקנית ואמרה ראיתי שלא בשעת לידה ואינה יודעת כמה ראיתי ואלו לא אמרה כן אינן מחזקינן אותה לא ביולדת בזוב ולא בשומרת יום אע"פ שלא בדקה כל אותן הימים שאין חוששין לראיה כל זמן שלא ידעה אלא בימי הוסת.\par וה"נ משמע בפ' בתרא דתניא בטועה ראיתי ואינה יודעת כמה ראיתי אלמא איני יודע אם ראיתי לא כלום הוא שאם אין אתה אומר כן השוטה והחרשת והקטנה נמי שראתה אסורות לשמש לעולם שמא ראו והן אינן מרגישות ולא יודעות. 
 והא דפריך מינה ר' יוסי בר חנינא ורבין לא ידע מאי תיובתיה משום \textbf{אימר הרחיקה לידתה.} דמשמע דלית ליה לר' יוסי בר חנינא הרחיקה לידתה קשיא טובא וכי היאך סלקה על דעתו כן. והא אי לאו משום הך תשש לא היו מבטלין אותה בשבוע ג' בליליותא דמשום טבולת יום ארוך מטבילין אות' שמא כבר עברו לה ימי טוהר וכ"ש בשבו' ד' דאיכא למימר כבר עברו וכן טבילות דב"ה נמי משום יולדת והרחיק' לידתה ז' או שבועים הן ועוד שבוע דטהור הוא תשמש דאי ילדה ולד מעליא אפי' בשבוע ד' נמי טהורה היא דדם טוהר הוא וכ"ש בה' דטהור' ואם לאו ולד מעליא הוה לספיקה דר' יוסי בר חנינא תחלת שבוע רביעי ה"ל תחלת ונדה ושבוע דטהור הוא מותרת אלא ע"כ משום הרחיקה לידת' וחוששין שמא כלו ימי טוהר בסוף שבוע רביעי ויום אחרון שבו היה לה התחלת נדה כדמפר' ואזיל בגמ'.\par אלא ע"כ ר' יוסי בר חנינא אגב חורפיה לא עיין בה ובגמ' ה"ל למימר ולטעמיך מ"ט דכל הני אלא אשכחן כמה דוכתי בתלמודא דהוה מצי למיפרך וליטעמיך ולא פריך ביה כלל. 
\textbf{שבוע קמא מטבילין לה בלילותא משום יולדת זכר ונקבה.} עיינו בתוספות שאין השבועין הנמצאין כאן בטבילות הלילו' שוים עם השבועין הנמנים כאן בטביל' הימים דהא למאי דס"ד מעיקרא שבאת לפנינו ביום וכן למאי דמתרצינן כגון שבאת לפנינו בין השמשות טבילות דלילותא מושכות עד לילה של שבוע שלאחריו כגון שבאת לפנינו בין השמשות של מוצאי שבת וכן שבאת לפנינו באחד בשבת ביום וטבילה ראשונה של לילה בליל שני בשבת ואחרונה במוצאי שבת וכן בשבוע שני ואלו טבילו' דימים דמשום זיבה ראשונה באחד בשבת ואחרונה בשבת. וליכא למימר דברייתא הכי קתני שהביאה לפנינו ג' שבועין טהורין חוץ מיום שבאת לפנינו שהרי אותו היום עילה הוא למנין שבועים ונמצאת זאת מותרת לשמש בלילי עשרין וחד שהרי אינה רואה כל אותה הליל' ולא יום שלאחריו אלא ע"כ יום שבאת לפנינו הוא ממנין שלשה שבועים טהורין. כל זה עיינו בתוספות.\par ודבר ברור הוא אלא כיון דמנין לידה וזיבה מיום א' בשבת הוא וכל טבילו' דעלמא הן דכל נדה ויולדת טבילתן בלילה של שבוע שני וטבילות דזיבה ביום בסוף שבוע שלהן לא חיישי בגמרא לפרושי הכא מידי. 
\clearpage}

\newsection{דף ל}
\twocol{\textbf{כגון שבאת לפנינו בין השמשות.} פי' רש"י הוא הדין דה"ל לאוקמ' בבאה לפנינו בלילה והוה ניחא טפי דתו לא הוה קשיא לקמן לסוף שבוע לטבלה ביממא דהא לא הוו ז' ספורים שהרי לא הפסיקה טהרה בתחלת היום ואין אותו יום שבאה לפנינו עולה לה לספירת נקיים אלא מדקתני ברייתא ג' שבועין טהורים משמע דכולן טהורים ובבין השמשות משכחת בהו מיהא פסיקת טהרה ואפי' ליום ראשון. כך פי' רש"י ז"ל. 
 ל"ה טבילות דקאמרי ב"ה קשיא לן כיון דאוקים ב\textbf{באה לפנינו בין השמשות דיהיבנא לה טבילה בתריהן.} תלתין ושש הווין. וראיתי בפירושים דכיון דתדא בשבוע היא לא קחשיב ואינו יודע מהו שאם בא לומר דטבילה דסוף שבוע רביעי הויא חדא בשבוע לא משמע הכי דהא טבלה נמי בימים הסמוכים לה ששה עד סוף ז' ובאור שביעי של שבוע חמישי גמרה טבילותיה וטהורה ואפשר שאותה טבילה ראשונה חדא בשבוע חשיבי לה ב"ה מפני שהיא נמנת לסוף שבוע שעבר ואותו היום עצמו נמנה לנו תחל' שבוע ללידה וטבילות שאח"כ ואט"ג דב"ש מנו לה ולא חשבי חדא בשבוע אינהי דמפשי טבילו' מנו לה כיון דמצטרפא בטבילות דלילות דשבוע א' אבל ב"ה לא מנו לה.\par וה"ר אב"ד ז"ל כתב דאיכא לתרוצי דכיון דלאו פסיקא להו דאי אתאי ביממי להא טבילה לא חשיבי ב"ה כי היכי דתרצינן בטועה בפ' בתרא. וזה הלשון נכון בעיני דב"ש דקא מפשי טבילות מהדרי לאפושי בהו טובא וב"ה דלא מפשי בהו טפי לא חשיב' לה.\par ובשם הרב חתנו ז"ל תירץ דבין השמשות דר' יהודה אפליגי ב"ש וב"ה ב"ש סברי כר' יהודה דספיקא הוא וב"ה סברי כר' יוסי דבין השמשות דר' יהודה יממא הוא. ולכך ליתא לטבילה יתירתי' ועומק גדול הוא אלא תימה גדול הוא היכי שתיק מיניה תלמודא. זה לשון הרב ז"ל. 
\textbf{איידי דפתח בשבוע מסיק לה איידי דתנא טמא תנא טהור.} פי' וה"ה דלענין צ"ה טבילות דב"ש ה"נ הוה קמ"ל בעשרה שבועים כולם טמאים או טהורים אלא משום דבעי למיתנא משמשת לאור ל"ה לא קודם לכן ולא לאחר כן משום חששות דאמרן תנא הכי.\par והק' בתוספ' כיון דמנינו עשרה שבועי נפישי להו טבילות שהרי שבוע ט' דטמא הוא ג' ימים ראשונים שבו איכא לספוקינהו בסוף לידה ותחלת נדה ויום ד' תחלת נדה ונמצאת צריכה טבילה לג' ימים בשבוע עשירי. ותירצו עד סוף שמוני' קחשיב לאחר פ' לא תשיב דהא לא תננהו אלא אגב גררא. וכ"ש למאי דפרקינן בסמוך דלא מיירי ב"ש אלא בלידה. 
 והא דאקשי\textbf{יומא קמא דאתיא לקמן לטבילה דילמא שומרת יום כנג' יום היא} לאו לב"ש מקשינן דהא אינהו לא זיבה גדולה ולא זבה קטנה קחשיבי אלא יולדת בזוב בלחוד כדאמרן. אלא לב"ה בעינן דהא דמחרצינן זיבה גרידתא לא קחשיב לב"ה לא צרכינן למימר הכי אלא דמקמי תשמיש קחשיבי כולהו דלבתר תשמיש לא קחשיבי. א"נ השתא דאתית להכי ליומא דחדא בשבוע לא קחשיב הדרי' מההוא טעמא דטבילת זבה חדא בשבוע נינהו ולפום הכי אקשי' ליחשוב דשומרת יום וה"ל ג' טבילות בשבוע זו ביום כיון שבאה בין השמשות וליחשוב ומפרקינן זבה גדולה קחשיב כלומר יולדת בזוב. א"נ זבה הוא קחשיב אי מיתרמיא ליה לפני תשמיש אבל זבה קטנה לא חשיב.\par וק"ל ולימא דילמא כשילדה ראתה יום א' בימי זיבה וצריכה לשמור יום כנגד יום ואין ספירת ימי לידתה עולין לה ודאי כשם שאינן עולין לםפירת זבה גדולה וליטבי' כל שבוע קמא ביממי משום שומרת יום כנג' יום זיבה שלפני לידתה ואמאי פריך יומא קמא בלחוד.\par ואיכא למימר דקסבר האומר שהימי לידה עולין לשמור דזיבה קטנה ויולדת בזוב קטן דמקש' דלטבילה יומא קמא דילמא יולד' בזוב קטן היא והרי ספרה יום זה לפנינו בין לב"ש בין לב"ה מקשי' וכן נראה לי עיקר דימי לידה אין עולין גמירי לה לקמן בפ' בנות כותיים מדכתיב כימי נדת דותה תטמא מה ימי נידתה אין ראויין לזיבה ואין ספירת ז' עולה בהן אף ימי לידתה כן, והא ליתא אלא לספירתן דזבה גדולה אבל שימור דזבה קטנה אף בימי נדה עולה דאפשר הוא כדאיתא בשלהי בא סימן (נג, א). 
והא דאמרינן \textbf{ש"מ תלתא.} איכא למידק ולימא נמי ש"מ ד' דהא ש"מ ימי לידה שאינה רואה בהן אין עולין לה לימי זיבת' ואיכא למימר דההיא פלוגתא דאביי ורבא היא ורבה דאמר עולין קסבר הא מני ר' אליעזר הוא דאמר מסתר נמי סתרא. ולפום הכי נמי לא אמרי' ש"מ ר' אלעזר היא כדאמרינן ש"מ ר' עקיבא היא ור' שמעון היא. משום דלאביי דברי הכל אינן עולין הלכך לא פסיקא ליה. 
מתניתין \textbf{בנות כותיים נדות מעריסתן.} אוקמינן בגמרא לר"מ דחייש למיעוטא וקסבר ר"מ כותיים גירי אמת הן דהכי אסיקנא בב"ק (דף לח) לדידיה וכיון שהן גירי אמת והן מטמאות בנדה מן התורה יש לחוש לספיקן והיינו נמי דקתני אין חייבין עליהן על ביאת מקדש מפני שטומאתן בספק.\par וא"ת ולמה העמידו משנתינו לר"מ לחוד דחייש למיעוטא. והא אפי' לר' יוסי נמי אית ליה בנות הכותיים נדו' מעריסתן כדאמרינן בפ"ק דשבת (דף טז ע"ב) גבי י"ח דבר לר' יוסי בצרי להו ואמר ר' נחמן בר יצחק בנות כותים נדו' מעריסתן בו ביום גזרו כלומר גזירה בעלמא כדי שלא יטמעו בהן או גזירה משום מיעוט שהן טמאות.\par י"ל כיון דמתני' קסבר כותיים גירי אמת הן דהיינו סבריה דר"מ ור' יוסי שמעינן ליה דפליג עליה וסבר גירי אריות הן כדאיתא במנחות בפרק ר' ישמעאל (דף סו) ובמקומות אחרים הילכך ניחא לן לאוקמא לדידיה ומדינא ועוד דאיהו סתם מתני' ולא למשקל תנאי מעלמא ומשו' גזרת י"ח דבר. ועוד דקתני לה דומיא דסיפא דכותיים עצמן והתם לאו נזיר' אלא דינא הוא לחוש לספיקן.\par וזה שכתבנו לפי גרסת מקצת הספרים אבל מהרבה מהן מספרי הגאונים שלא נמצא שם במס' שבת אותה הגרס' כלל ואעפ"כ חשבון י"ח דבר עונה להן יפה. }

\newchap{פרק \hebrewnumeral{4} בנות כותים}
\twocol{\clearpage}

\newsection{דף לב}
\twocol{\textbf{שמא תמצא איילנות ונמצמו פוגעין בערוה.} פירשתיה בתחל' מסכ' יבמות.
\textbf{הא נמי מיעוטא דשכיח הוא דתניא מעשה והטבילוה קודם לאמה.} וא"ת שמא משום נגיעות אמה בה הטבילוה לסוכה בתרומה. י"ל שיודעין היה בגמרא שלא בא ר' יוסי אלא להעיד על טומאת עצמה והכי קתני מעשה היה שפרשה נדה והטבילוה קודם לאמה. ולא אטבילה בלחוד אסהוד אלא אפרשה אסהיד דאי לאו הכי פשיטא ותא חזי מאן גברא רבה מסהיד עליה. א"נ אין דרכן של בני אדם להפרישה מאמה אלא לכך הטבילוה שלא תטמא את הנשים שגוגעות בה ומגפפות אותה ויחזרו ויטמאו הן תרומה שבא"י אבל בנגיעת אמה בה אין להקפיד לטומאות הנוגעים בה שהרי היא ראשון ואין מטמאה אדם. ואותה שבפומדיתא נמי משטבלה לטומא' גופה אינה צריכה להפרישה מאמה כדמפר' ואזיל. 
\textbf{ולא יחללו את קדשי בני ישראל לרבות את הסך ואת השותה.} י"מ שהיא אסמכתא דרבנן דהא קי"ל גבי יום הכפורים דאכילה ושתיה דאוריית' ובכרת ואין סיכה בכלל שתיה.\par ויש לפרש אע"פ שנתרבה סיכה כשתייה לענין תרומה מריבוי הכתוב לשאר כל התורה כולה אינה כשתיה ואי משום וכשמן בעצמותיו דשייך נמי בכל התורה ההיא ודאי אסמכת' בעלמא היא מדברי קבלה ומיהו ודאי מכיון דאמרינן גבי תרומה גופה מולא יחללו ואיבעית אימא מוכשמן בעצמותיו משמע דכולה דרבנן היא.\par ומאחר שכתבתי סברות הללו מצאתי בפ"ב ממסכת מעשר שני שאמרו בירושלמי לענין מעש' יצהרך זו סיכה והתור' קראתו אכילה ואינו מחוור וא"ת מחוו' ולקו עליו חוץ לחומ' וכו'. ומייתי נמי התם והתני שוה סיכה לשתי' לאסור ולתשלומין לא לעונש יום הכפורים ומקשי והתני לא יחללו מ"מ להביא הסך והשותה. 
\textbf{א"ר יוחנן לית כאן לאסר וכו'.} משמע דסיכה כשתיה דרבנן ואינה מחוורת מן התורה. 
תמיהה לן לר' ישמעאל בנו של רבי יוחנן בן ברוקא דדריש \textbf{לזכר כל שהוא זכר לנקבה כל שהוא נקבה} וא"ו דגבי זבה למה לי. וכי תימא לא דריש וא"ו א"כ אין איש מטמא בדם ובאודם מנא ליה. 
\textbf{למעוטי אשה מלובן.} מצאתי בתוספות שמקשים למה לי מיעוטא והרי מצינו ה' דמים טמאי' באשה ותו לא. וי"ל דנהי דאינו דם הוה אתי בק"ו ומה איש שאינו מטמא באודם מטמא בלובן אשה שמטמאה באודם אינו דין שתטמא בלובן ודם טהור באשה מיתוקם בירוק ודיהה כך השיב ר"ש לר' יהודה חתנו ז"ל.\par והם הקשו בתוספות והא ק"ו פירכא הוא מה לנקבה שכן אינה מטמא בראיות כבימי' כדאמרינן בסמוך ואמרו גלוי מילתא בענמא היא דאחד איש ואשה מטמאין בלובן כיון דמתוקם דם טהור שפיר. ולא מחוור לי דאם לובן טמא משום נדה כ"ש ירוק ודיהה שכולן לא נטהרו אלא משום שאינן אלא לובן.\par ונ"ל דמש"ה איצטריך יתורא דאי משום הא דה' דמי' לא הוה ממעטי' אלא מטומאת נדה ועדיין היינו מטמאי' באשה מדין שכבת זרע או זוב של איש היינו מטמאין לטהרות ולא לבעלה ומיניה ממעט ליה. ומ"מ יפה הרב ז"ל מלמדנו דאי לא ילפי מהדדי מנא תיתי לובן באשה ואודם ודם באיש דאצטריכו קראי למעוטינהו.\par ואמרו בתוספות שעוד שאלו בכל מקום ואוי"ן לרבות וכאן למעט השיב לו כ"ש כיון דלא אצטריכו לרבויא מפרשין להו לקרא דייתר ואיש ואשה לומר אשה דוקא אמרתי ולא איש איש דוק' אמרתי ולא אשה. 
\textbf{פשיטה דהא קא דרס להו.} פי' לאו פשיטא מגופא דמילתא אלא פשיטא דכל דקא דרים להו רחמנא רבינהו למדרס דתניא בת"כ אשר ישב עליו הזב אין לי אלא יושב ומגע מניין לעשרה מושבות זה על גב זה ואפילו על גבי אבן מוסמה ת"ל והיושב על הכלי אשר ישב וכו'. ומשום דמילתה רגילה היא בתלמודא הוא קאמרינן פשיטא דלא ה"ל הכא למיתני אלא שמטמ' מדרס.\par ובפי' עליונו של זב שמעתי דברים רבים והנכון מהם מה שאמרו משם ר"ש ז"ל שהוא דבר הנשא עליו כגון הוא בכף מאזנים ומשכבות ומושבות בכף שניה וכרעו הן טמאין מדרס כרע הזב זהו עליונו של זב ומטמאין אוכלין ומשקין. ואתינן למיבעי מנלן דתניא ובל הנוגע בכל אשר' יהיה תחתיו מאי תחתיו אלימא תחתיו דזב דהיינו משכב ומושב מאיש אשר יגע במשכבו נפקא. ואי קשיא לך אדרבא הוא מטמא בגדים דכתיב ביה יכבס בגדיו והכא ליכא כבוס אה"נ אלא גמרא לא איצטרך למיחת לה כולי האי. וקאמר סתם כל טומאה דמדרס מהתם היא כדכתיבנא ביה ולא מהכא ועוד דאי הוה נקיט טעמא מהך קושיא דכבוס בגדים דילמא הוה אמרינן דכי כתיב והנושא אותם יכבס בגדיו ארישא דקרא נמי קאי ולא בעי עיוליה נפשיה בספיקא דקושיי. 
\clearpage}

\newsection{דף לג}
\twocol{\textbf{אלא וכל הנוגע בכל אשר יהיה זב תחתיו ומאי נינהו נישא יטמא נתקו הכתוב וכו'.} זו היא גרסתו של רש"י ז"ל. ולשון יתר שבספרים מ"ט הנושא והנישא כתב נתקו וכו' כתב שהוא פי' משובש. אלא ה"ק מדכתיב וכל הנוגע בכל אשר יהיה תחתיו יטמא והנושא אותם יכב' בגדיו ולא ערבינהו ונכתוב וכל הנוגע בכל אשר יהיה תחתיו והנושא אותם יכבס בגדיו ואפסקינהו ביטמא מכלל דהאי יטמא לאו באדם ובגדים קא מיירי אלא באוכלין ומשקין מאי והנושא אותם לא לעליונו של זב דסמיך ליה אלא נדרש הוא בת"כ לנושא משכבו ומושבו של זב. אלו דברי הרב ז"ל.\par ואין פי' זה נכון דהא טמא עד הערב כחיב ואם אינו מטמא אלא אוכלין ומשקין מאי עד הערב.\par ואיכא למימר דהכי דריש דכתיב לעיל מינה וכל המרכב אשר ירכב עליו הזב יטמא וסמיך ליה וכל הנוגע בכל אשר יהיה תחתיו יטמא עד הערב ומפני שנתקו הכתוב מקרא נדרש לפניו דכתיב יטמא וכל הנוגע בכל אשר יהיה תחתיו נמי והוא עצמו כלומ' מה שהזב תחתיו יטמא עד הערב לומר שהוא מטמא כלים שיש בהן טומאת ערב לפי שיש להן טהרה במקוה.\par והרב אב"ד ז"ל מפרש דהאי יטמא עד הערב אמרכב דלעיל ולא מסתברא דהא לא כתיב במשכב ומושב עד הערב.\par ואיכא למידק, עליונו של זב מאי נינהו הסיטו הא מהכא נפקא מהתם נפקא וכלי חרס אשר יגע בו הזב ישבר אי זהו מגעו שהוא ככולו הוי אומר זה היסט ולקמן במכילתן בפ' יוצא דופץ אמרינן וכל אשר יגע בו הזב וידיו לא שטף במים זה היסטו של זב שלא מצינו לו חבר בכל התורה כולה. ואיכא למימר אי מהתם הוה אמינא עליונו של זב כתחתונו מטמא אדם ובגדים ואי מהכח ה"א כלי חרס שנטמא מאוירו לא קמ"ל וכלי חרס אשר יגע בו וכו' וכל אשר יגע מיבעי ליה למימרא דהיסט ונגיעה כידיו כדמפורש בפ' יוצא דופן.\par וא"ת אי לא כתב עליונו של זב מנא לך לעשותו כתחתונו להחמירו הא ל"ק דה"א נושא ונישא כי הדדי נינהו דהא בכולה שמעתין להחמיר עליו ולעשות כיוצא בתחתונו אנו טורחין ויש שמתרצין אי מהתם ה"א ה"מ היסטו כולן א) בא הכתוב הזה וכל הנוגע בכל אשר יהיה זב תחתיו לטמא אף לטהור וזב שהסיטו.\par וגרסת הספרים יש להעמידה אלא וכל הנוגע בכל אשר יהיה תחתיו ומאי נינהו נישא יטמא מ"ט כלומר מ"ט משמע לך יטמא טומאה קלה הנושא והנישא כתיבי בהדדי ולא ערבינהו רחמנא ואפסקינהו לומר לך נתקו כלשון רש"י ז"ל עצמו.\par ובתוספות ראיתי שפי' רבינו שלמה ז"ל בתשובה ה"ג אלא וכל הנוגע בכל אתר יהיה תחתיו יטמא והנושא נמי יטמא ומאי נינהו נשא מ"ט והנישא כתיב נתקו הכתוב וה"פ וכל הנוגע בכל אשר יהיה תחתיו בא הכתוב ולימד על המרכב שיטמא במגע דהאי קרא אחר מרכב כתיב וחלק מגעו ממשאו שמגע מטמא אדם ולא בגדי' ומשאו מטמ' אדם לטמא בגדים והכי מוקי לה בת"כ דמרכב חלק הכתוב מגעו ממשאו ובמשכב לא חלק בין מגעו למשאו ומהאי קרא נפקא לן מגע מרכב. והאי דדרשינן דהא כתיב והנישא חסר וא"ו למדרש אנישא של זב דקריי ארישא דקרא דכתיב יטמא עד הערב טומאה קלה ונתקו הכתוב מטומאה חמורה של אחריו, וה"ק והנוגע במרכב הזב יטמא טומאה קלה לטמא אוכלין ומשקין וכן הנישא על הזב דע"כ דרשא דנישא אטומאה דרישא קאי ולא אטומאה דסיפא דטומאה דסיפא אותם כתיב ביה וגבי נישא לא שייך אותם אלא אותם אמקרא קאי דקרינן נושא ודרשא דמסורת ארישא קאי ע"כ תשובתו של ר"ש ז"ל.\par והוקשה לו על גרסת הספרים שכתוב בהן בכל אשר יהי' תחתיו. ב) דהאי קרא במרכב מוקי לה בת"כ ועוד היכי מוקי לה באוכלין ומשקין הנוגעין בעליונו הא אין להם טהרה במקו' והכא כתיב וטמא עד הערב. כל זה הענין כתוב בתוספות. והעלו השמועה בשבוש ועמעום ועוד שאין זה נושא כתוב חסר בכל הספרים ובמסורת מלא ואין כאן יתור לדרוש בו מסורת ולשון ראשון של פי' רש"י ז"ל יותר נכוו הוא וכמו שכתבתיו למעלה.\par ולי נראה ענין אחר שפירוש עליונו של זב זהו שהיו עשר מצעות עליו ונשאת התחתונה על ראשו כולן טמאי' כמו שעושה בתחתונה את הנוגעת בזב מדרס מרבוי הכתוב כך עושה בעליונו את כולן עליונו של זב ואלו מדין היסט תחתונה טמאה ועליונוח שהן נשאות על עליונו של זה היסט דהיסט הוא וטהורין ומ"מ כי"ו דין היסט הן ולא מצינו להן חבר בשאר טומאות ופי' נתקו הכתוב מטומאה חמורה שאמרו כאן משום דה"ל למיכתב וכל המרכב אשר ירכב עניו הזב יטמא וכל אשר יהיה תחתיו דזב יטמא למיכתב בטומאת עצמן והדר ליכתוב בתרווייהו וכל הנוגע וגו' והנושא אותם יכבס בגדיו כדכתיב במשכב ובמושב עצמן כל המשכב יטמא וכל הכלי אשר ישב עליו הזב יטמא והדר בנגיעה דידהו ואיש אשר יגע במשכבו והיושב על הכלי. מדפלגינהו רחמנא ש"מ שאין עליונו של זב שוה למרכב הסמוך לו ולא למשכב ומושב דלעיל נתקו הכתוב מכולן שהן מטמאין אדם וזה אינו מטמא אלא אוכלין ומשקין ולהכי רהיט קרא ונסיב נוגע לפי שאינו נעשה אב הטומאה אלא שיש בו שם טומאה לנוגע בו וקרי ביה וכל אשר יהיה זב תחתיו יטמא עד הערב כלומר עליונו של זב עצמו טמא טומאת ערב ודרש בת"כ דחמור עליונו של זב מתחתונו דאוכלין ומשקין אינן נעשין תחתיו מדף ונעשין על גביו מדף כלומר טומאה קלה דעליונו של זב. וכן מפורש במשניות דתנן האוכלין והמשקין והמדף מלמטה טהורין והאוכלין והמשקין והמשכב והמושב והמדף מלמעלה מטמאין א' ופוסלין אחד. כלומר שעושין ראשון ושני באוכלין.\par ואפשר לפי זה הפירוש שתתקיים גרסת הספרים וה"ק מ"ט הנושא והנישא כתיבי כלומר מ"ט משתמע קרא הכי נימא דהיינו משכב ומרכב וכל שתחתיו ונישא דהיינו עליונו תרווייהו כתיבי הכא ונתקו הכתוב לנישא מנושא ופי' מעלמא הוא ולא מגופה דברייתא כדפרישו נמי מאי תחתיו וכו' דהאי לישנא דגמרא ואקשי' אימא נתקו הכתוב מטומאה חמורה שבמרכב דלא לטמא אדם ובגדים אפילו במשא ומפרקי' יטמא טומאה קלה משמע פי' משום דכל דלא מתפרש ביה מגע אחר לא משמע אלא שהוא טמא בעלמא דכל דמטמא אדם כתיב בהו ורחץ בשרו או יכבס בגדיו או והנוגע בהם יטמא כדכתיב במשכב ומושב הילכך הכא דלא כתיב אלא יטמא לחודיה טומאה קלה דלית ליה טהרה במקוה משמע מדלא כתיב ורחץ כדכתיב בכולהו נוגעים והיכא דלא כתיב ביה כגון בנדה דמפרש במשכב דידיה כתביה רחמנא סתם וכן לענין משכב דבועל נדה יטמא הוא עצמו משמע לומר שאינו מטמא אחרי' כשאר משכב ומושב דזב דמפרש בהו ואיש אשר יגע במשכבו ולדברי הרב אב"ד ז"ל חזר ופי' בו טומאת ערב במרכב. 
\textbf{מתקיף לה רמי בר חמא ותספרנו ואנן נמי ניספריה וכו'.} פי' רמי בר חמא טעמא הוה בעי אבל ודאי ליכא דסליק אדעתיה דדינא הכי הני ספרה אנן כדמקשינן בפ' בתרא דמכילתן א"ל רב ששת לרב ירמיה רב ככותאי אמרה לשמעתיה דאמרי' יום שפוסקת בו סופרת למנין ז'.\par ואי קשיא ההיא דגרסינן בפסחים פ' כיצד צולין (דף כא) ר' יוסי אומר שומרת יום כנגד יום ששחטו וזרקו עליה בשני שלה ואח"כ ראתה אינה אוכלת ופטורה מלעשות פסח שני ומפרשינן טעמיה דקסבר מכאן ולהבא מיטמיא דמקצת היום ככולו ובעינן עלה אלא לר' יוסי זבה גמורה היכי משכחת לה בשופעת ואיבעית אימא בגון שראתה שני בין השמשו' אלמא אמרינן מקצת היום ככולו.\par לאו מילתא היא דבסוף מנין אית ליה לר' יוסי מקצת תחלת היום ככולו בין זבה גדולה ובין בקטנה דשני שלה סוף מנין הוא דהא אנן נמי בזבה גדולה קי"ל כר"ש דאמר אחר מעשה תטהר אלא לדידן סותרת בכל היום ולר' יוסי לית ליה סתירה לאחר מקצת יום דהא שלימה היא טהרתה אבל בסוף יום ותחלת מנין דכ"ע לית להו מקצ' היום ככולו אלא לכותאי.\par וראיתי מי שמקשה כאן מאותה שאמרו בפ"ק דר"ה (דף י) אמר רבא ק"ו ומה נדה שאין תחלת היום עולה לה בסופה סוף היום עולה לה בתחלת שנה שיום א, עולה לה בתחלתה. וא"כ לר' יוסי נימא ק"ו ויהיה סוף היום עולה לה בתחלתה. וזה המקשה יכול להקשות כן בזבה גדולה לרבנן (ובין א) בקטנה דליכא בינייהו אלא סתירה ולפי דעתי שאין זו הקושיא דמקצת יום טמא ככולו טמא ומקצת יום טהור סוף היום כתחלתו בין בתחלתה בין בסופה הילכך לענין זיבה ביום נקי ליכא למיספריה אבל לענין נדה אפילו כולו נמי כימא סופרתו כנ"ל.\par ומיהו מקצת היום שעולה בספירה דזבה דוקא ביום אבל לילה אינה עולה לספירה כלל כדאמרינן בפ' בתרא דמכילתן ושוין בטבילות לילה לזבה שאינה טבילה ותנן נמי במס' מגילה דאינה טובלת עד הנץ החמה.\par וההיא דאמרינן מפ' כיצד צולין דמוקי לדר' יוסי לרואה בין השמשות וכן נמי איתא במס' נזיר (דף כז) וגרסי' בה הכי בנוסחי לדר' יוסי מכדי קסבר מקצת היום ככולו זבה גמורה דמייתי קרבן היכי משכח' לה כגון דחזאי פלגיה דיומא אידך פלגא דלמפרע סליק ליה שימור פי' דלמפרע היינו פלגא דיומא בתחלתו שעבר עליה בטהרה ומתרצי איבעית אימא דקא שפעא ג' יומי בהדי הדדי ואיבעית אימא דחזאי תלתא יום סמוך לשקיעת התמה דלא הוה שהות סליק ליה למנינא. ההיא לרוחא דמילתא איתמר דלא בעי לאתויי עלה התם קרא דמגלה דאמרינן כיון דבעי ספירה ספירה ביממא היא ואוקמוה בסוף היום ותחלתו דליכא שהות דספירה בין ראיה לראיה כלל.\par וי"מ דלא בעיא לאוקמי זבה גדולה בלילואתא דוקא משום דקראי ביממא כתיבי דכתיב ימים רבים כל ימי זובה ולקמן בשלהי מכילתין ואימא ביממי תהוי זיבה בלילואתא תהי נדה ובפ"ק דהוריות נמי אמרינן גבי צדוקין דאמר דזבה לא הויא אלא ביממא דכתיב כל ימי זובה הילכך אע"פ דמפקינן מקראי אפילו לילותא לא מפקינן קרא מימים. 
 והא דאסיקנא \textbf{שלא תהא טומאת זיבה מפסקת ביניהם.} לאו דוקא אלא שלא תהא טומאת ז' מפסקת ביניהם דהא טומאת לידה נמי אמר רבא לקמן בפירקן דדינה למיפסק וכדבעי' למימר קמן, א"נ אוקמתין דלא כרבא אלא כאביי דאמר טומאת זיבה דוקא, ומיהו בטומאת ערב ליכא למימר דפליג רבא דהא לא פריק הא דאמרי ולטעמיך זב גופיה היכי סתר וכו', אלא ע"כ קבולי מקבל דטומאת ערב מיהא לא סתרא כדפרישית, ולענין בעיין דפולטת בעינן למיכתב קמן טפי בפרק יוצא דופן (מב, א). 
\textbf{אלמא אספיקא לא שרפינן תרומה.} פי' לאו אכל ספיקא קאמר דהא למסקנא נמי אספיקא ודאי שרפינן אלא ה"ק אלמא אהך ספיקא דעם הארץ ואפילו בכותי לא שרפינן. 
\textbf{ורמינהו על ספק בגדי עם הארץ.} פי' שחכמים גזרו עליהם שיהיו זבים לכל דבריהם ובגדיהם יהיו מדרס לפרושין, והא דאמרינן בפרק השוחט מדרסות קאמרת שאני מדרסות גזירה שמא תשב עליהם אשתו נדה אבגדי אוכלי תרומה מדרס לקודש קאמר והיינו נמי דמיטמי' צינורא דידהו מדבריהם משום משקה הזב והזבה.\par ואי קשיא לך האי דאמרינן בפרק הניזקין (דף סא ע"ב) וליחוש שמא תסטנו אשתו נדה ולא חיישי' להיסט שלו, ועוד אמרו שם גבי חלה מניחה בכפישה או באנחותא וכשיבא עם הארץ ליטול נוטל את שתיהן ואינו חושש משום דלא נגע בהו ולא מטמאין בפשוטי כלי עץ ולא חשש להסיטו וכן נמי בפרק אין דורשין משמע גבי חמרין ופועלין שהן טוענין טהרות דלא מטמאין משום הסיטן.\par ותירץ ר"ת ז"ל שלא גזרו על עמי הארץ היםט שא"כ אין לך אדם מעביר לחבירו חבית ממקום למקום.\par ועוד הביאו ראיה ממשנת מסכת טהרות שאין עמי הארץ מטמאין בהסיטו ולא עושין נמי משכב ומושב דתנן בפרק קמא דטהרות הגנבים שנכנסו לבית אין טמא אלא מקום רגלי הגנבים ומה הן מטמאין אוכלין ומשקין וכלי חרס הפתוחין אבל משכבות ומושבות וכלי חרס המוקפין צמיד פתיל טהורי' ואם יש עמהם נכרי או אשה הכל טמא.\par ויש לי לדחות דהתם כיון דלא נגעו ודאי הקלו באלו שטומאתן רחוקה וספק.\par ועוד הביא ראיה מתוך שמעתן גופה שאין ע"ה מטמאין משכב ומושב דתנן משכב התחתון כעליון ואם היו ע"ה עושין משכב ומושב מטמאין הן התחתון כתחתונו של זב שהרי עשאום כזבים לכל דבריהם.\par ואע"פ שיש לי לפקפק אף בראיה זו נקבל אותם מפני שלא הזהירו חכמים על ע"ה שיהיו כזבים ממש אלמא דין חדש יש להם אבל מ"מ תמה הוא אם גזרו עליהם שיהיו כזבים למקצת היאך הטילו עליהם למחצה וטיהרו משכבות והיסט א"כ הקלת בשל תורה.\par וי"מ שאין ע"ה טמא טומאת עצמן כלל אלא חשש שמא נגעו בנשותיהן ובמדרסן שהן אבות הטומאה והוא נעשה ראשון.\par והא דמטמאינן צינורא דע"ה בשמעתין וכן במסכת חגיגה לאו משום משקה הזב והזבה אלא כיון שהחזיקו אותם חכמים בטמאי' משום משא מדרס טמאו המשקין בשפתי' דכלים מטמאין משקים מדבריהם בפ"ק דשבת אבל לא שיהיו כזבים מדבריהם ובגדיהם שהן מדרס לפרושין נמי משום תשש מדרס אשתו נדה הוא והא דא"ל מ"ט לא תשני ליה בכותי שטבל ועלה ה"ג לה שטבל ועלה (ואכל) [ונגע] בתרומה וכן כתבו בתוספות בנוסחאות ולא כגרסת רש"י שהוא גורס ודרס על בגדי חבר ונגעו בתרומה שאפילו לא טבל אין לו מדרס ואפילו נגע בהן ממש ואצ"ל בעשר מצעות שהרי לא עשיתו אלא ראשון משום טומאת ע"ה דהיא משום נושא מדרס ואינו מטמא כלים לאחר שפירש מן המדרס כלל.\par ואי קשיא ולימא ליה מאי ואין שורפין עליה את התרומה על התחתונות דעליונו קתני דאי משום טומאת ע"ה לא מטמא משכב ואי משום נדה תרי ספיקי נינהו, איכא למימר מפני שטומאתן ספק אפילו אגופייהו משמע ליה.\par תו קשיא לי ולימא ליה ברגל דטומאת ע"ה ברגל כטהרה שוויה רבנן מאי איכא בכותי משום בועל נדה תרי ספיקי נינהו, ול"ק דבשלמא ע"ה שוויה בטהרה שלא להרחיקן אבל בכותי לאו חברים קרינן ביה ואין זה דומה לצדוקי שהם בכלל ישראל הם. 
והא דמפרקינן ב\textbf{כותי ערום} ולא אמרינן בשנגע ביד וברגל בלא בגד מפני ששנינו הנוגע במשכב ובמושב מטמא שנים ופוסל אחד פירש מטמא אחד ופוסל אחד בפרק בתרא דזבים והילכך מטמא את התרומה אלא בכותי ערום קודם שיגע בבגדיו עסקינן,\par ובמסכת חגיגה מצאתי בירושלמי סוגיא גדולה בענין זה ובתוס' נמי הזכירוה ומשמע מינה שלא גזרו על עם הארץ שיהיו כזבין וכך הסוגיא שם על מתניתין בגדי עם הארץ מדרס לפרושין וכו'. 
\textbf{מתני רבי יוסי בשם רבי יוחנן במגעות שנינו,} פירוש אלו השנויין כאן אינן עושין מדרס בלא נגיעה אלא שאם נגעו בבגד עשאוהו כמדרס מדבריהם, רבי זעירא בעי קומי רבי יוסי מהיכן נטמא הבגד הזה מדרס א"ל תפתר שהיתה אשתו של עם הארץ יושבת עליה ערומה, פירוש ר' זעירא מקשי על רבי יוסי לדבריך מהיכן נטמא מדרס לא היה לנו לטמאן אלא מגע הזב פריק שאם נגעה בו אשתו בישיבה עשאוהו כמדרס אבל ישבה עליו בבגדיה לא גזרו עליו, שמואל בר בא בעי קומי ר' זעירא כמה דתימר תמן אין היסט בחולין ויש היסט בחולין על ידי מגע ודכותה אין משא בחולין ויש משא בחולין על ידי מגע וכו' גופו של פרוש מהו שיעשה כזב אצל תרומה מתיב ר' תנן והתנן המניח עם הארץ בתוך ביתו בזמן שהוא רואה את הנכנסים ואת היוצאים האוכלין והמשקין וכלי חרס הפתוחין טמאין אבל המשכבות ומושבות וכלי חרס מוקפין צמיד פתיל טהורים אין תימר עשו גופו כזב אצל תרומה אפילו מוקפין צמיד פתיל יהיו טמאין אמר רב ר' יהודה בר פזי תפתר בעם הארץ אצל הפרוש לא עשאוהו כזב אלא שגזרו על בגדיו מדרס במגע אשתו כדאמרן ואקשי' אמר ר' מונא כן אמר ר' יוסי רבי כל מה דאנן קיימין הכא בתרומה אנן קיימן תדע לך שהוא כן דתנינן אפילו מובל ואפילו כפות הכל טמא כלום אמרו יהו הן טמאין אלא משום היסט לא כן אמר ר' יוחנן לאו חציצות ולא הסיטו ולא רשות היחיד ולא רשות עם הארץ אצל תרומה, ע"כ גמר'.\par וה"פ דקא מקשי ליה רבי מונא לר"י בן פזי דאוקמא למתני' בחולין דודאי בתרומה קיימי' מדקתני סיפא הכל טמא ואי בחולין מדרסות והיסטות טהורין הן דאמר רבי יוחנן שלא אמרו שיהא דבר חוצץ במדרסות ולא טהרו היסטות ולא חלקו בספק רשות היחיד ולא רשות עם הארץ אצל תרומה הא אצל חולין הכל טהורין.\par וברייתא היא אצל זו ששנויה בתוספתא דחגיגה דקתני ספק רשות עם הארץ מדרסו וחצירו והיסטו טהורין לחולין וטמאין לתרומה, אלמא מתניתין דקתני הכל טמא בתרומה היא ושמע מינה שלא עשו גופו של פרוש ולא של עם הארץ כזב לטמא משכבות ומושבות והיסט אלא שחששו בזמן שאינו רואה את הנכנסים לאשה או לכותי לתרומה ולחולין הכל טהור ואפילו ספק רשותו עד שיתברר לך שנכנסה אשתו לשם אי נמי בגדים שלו שאי אפשר שלא נגעה בהם אשתו במדרס, ע"כ הארכתי לכתוב מן התוספת והן מגיהין ב) ולא רשות עם הארץ לחולין אלא אצל תרומה ודבריהם הללו כולן כתבתים מפני שדברים ברורים הם וצריכין הן לכמה סוגיות שבגמרא. 
ה"ג רש"י ז"ל: \textbf{אמר לך בית שמאי האי לזכר מיבעי ליה כל שהוא זכר בין גדול בין קטן} ולא גריס נקבה אלא לנקבה מיבעי ליה למעיינות ואתי זכר בקל וחומר ולא פרכי' במצורע מילי דזב.\par ובתוספות מעמידין הספרים ומפרשים ואי בעית אימא בית שמאי לית הך דרשא דר' יצחק כלל אלא לזכר כל שהוא זכר לנקבה כל שהוא נקבה ודקא קשיא לך מעיינות מנא להו לבית שמאי ולטעמיך לר' ישמעאל בנו של ר' יוחנן בן ברוקא דריש לעיל תרווייהו לקטן וקטנה מעיינות מנא ליה אלא נפקא להו משום דרשא בעלמא דלא מתפרש במכילתין ואדרבה משמע דבית שמאי כר' ישמעאל בנו של ר' יוחנן בן ברוקה אמרי לגמרי.\par ולי נראה דבין לבית שמאי בין לר' ישמעאל לזכר כל שהוא זכר ולנקבה למעיינות אתא ונקבה קטנה לרבי ישמעאל נפקא ליה מוא"ו יתירה דגבי זבה והא דנקט לעיל דנקבה כל שהיא נקבה סירכא נקט ולא דנפיק מינה אנא משום דהכי הוא אמר קרא ולא מהאי קרא מתרבן, וכבר כתבתי מכיוצא בזו הרבה בפרק קמא דקדושין ועכשיו תירצנו הקושיא שהקשינו למעלה דלכולי עלמא הנך ואו"י מדרשי בזב ובזבה דתרי נינהו. 
גרסת הספרים כך היא וכן בפירושי ר"ח ז"ל: \textbf{תא שמע זובו טמא לימד על הזוב שהוא טמא במאי אלימא בזב גרידא לאחרים גורם טומאה לעצמו לא כל שכן אלא פשיטא בזב ומצורע ומדאיצטריך לרבויי בראיה ראשונה שמע מינה מקום זיבה לאו מעין הוא.} ובודאי יפה פירש רש"י ז"ל שראיה ראשונה של אדם אחר אינו מטמא במשא אלא במגע (בקרי) [כקרי] וראיה שנייה מטמא אפילו במשא לקמן בפרק דם הנדה, והא דקאמר לאחרים גורם טומאה הכא קאמר לאחרים גורם שיהיו מטמאין במשא לעצמו לא כל שכן שיטמא במשא ומדין משא למשא פריך דאי גרס טומאה בעלמא קאמר אף בזב מצורע גורס טומאה דמשכב ומושב לטמא אדם ולטמא בגדים ולטמא נמי בהיסט שאין מצורע עושה כן אלא מדין משא גופיה פריך כדפרישית.\par ומיהו צריכין אנו לישב גרסת הספרים, ור"ש אומר פירוש שהיא בספרים והכי קאמר ומדאיצטריך לרבויי בשנייה שמע מינה דבראשונה לא מטמא במשא דלאו מעין הוא ואין וה הלשון גמרא.\par אבל יש לפרש שכך היא הצעה זו דאמר רבא תא שמע זובו טמא לימד על הזוב שהוא טמא במאי אלימא בזב גרידא ובראיה שניה דאלו בראשונה ולמגע ודאי לא צריכא קרא דלא גרע משכבת זרע ולא עדיף מיניה אלא פשיטא בשנייה ואכתי למה לי קרא לאחרים גורס טומאה ואפילו למשא עצמו לא כ"ש אלא פשיטא בזב ומצורע ואי בשניה מי גרע מזב גרידא אלא בראשונה ולטמויי במשא וש"מ תרתי ש"מ ראיה ראשונה של מצורע מטמא במשא וש"מ לאו משום דמעיין הוא כדרב יוסף דאי הכי לא איצטרך רחמנא לרבויי הכא דממעינות נפקא אלא דרחמנא רבייה לראיה ראשונה של מצורע כראיה שניה של זב גרידא.\par ואי קשיא נימא קרא לראיה שניה והא קמ"ל דוקא בשניה אבל בראשונה אינה מטמא דלאו מעיין הוא, זו אינה תורה דמי איכא ספיקא קמי שמיא במקום זיבה אי מעין הוא או לא ואיצטרך ליתורי קרא למיגמר מיניה דלאו מעין הוא, ועוד דכל היכא דקרא מרבה כגון זה דכתיב זובו טמא דרשינן ליה לרבויי כגון לרבו' ראי' ראשונה למשא ולא מוקמינן ליה ליתורא למימר בשניה כתיב ולמעוטי ראשונה איצטרך כנ"ל.\par ומה שהקשה רש"י ז"ל מי איכא לאוקומי להאי קרא בראיה ראשונה והא מהכא נפקא לן בכל דוכתא מנה הכתוב שתים וקרא טמא, אינה קושיא דהאמרינן בפרק יוצא דופן דלמאן דאית ליה מנה הכתוב שתים וקרא טמא לית ליה זובו טמא לימד על הזוב שיהא טמא ותנאי היא.\par וכן זה שאמר הרב ז"ל דגבי מצורע איצטרך לרבוייה לטיפ' עצמה דלא אתי בק"ו משום דלא גרמה ליה טומא' שהרי מחמת נגעו הוא מטמא אין זה מחוור דכיון דאי לאו מצורע הוא נמי הות מטמי' איתא לק"ו מ"מ וכ"ש דאיכא לפרושי גרס טומאה בהיסט ומדרסות כדאמרן לעיל ומהסט למשא גמרינן ודאי דחד אורח הוא למשאות, ובמסקנא פשט אביי דמטמא במשא דהא אקשייה רחמנא למצורע אזב גמור, ולא פשט במעיין כלום משום דלא מרבוייא דקרא יתירא אתי דנימא למאי איצטרך אלא דמ"מ מטמא במשא הוא. 
\clearpage}

\newsection{דף לה}
\twocol{הא ד\textbf{אמר רבא לאחרים גורם טומאה.} ק"ל א"כ שכבת זרע דכתב רחמנא מלטמא לימא לאחרים גורם טומאה לעצמו לא כ"ש ולמה לי הא דתניא מנין לנוגע בשכבת זרע ת"ל או איש וכו' כדאיתא בפ' יוצא דופן, וכן (נמי קושיא) [דם עושה] משכב ומושב לאחרים והוא עצמו אינו עושה משכב ומושב כדאמרינן בפרק דם הנדה וכל המשכב אשר תשכב עליו נדה ולא דמה ובהא איכא למימר התם מיעטיה רחמנא דלאו בר משכב ומושב הוא ולית ביה אלא נגיעה בעלמא.\par תו קשיא והאיכא נמי היסט שהזוב גורם טומאת היסט והוא עצמו אינו מטמא בהיסט.\par ואיכא למימר נמי התם מיעטיה רחמנא מדכתיב והנושא אותם מיעוטא הוא בפרק דם הנדה (נה, א) א"נ בכל היינו פרכיה דעדיף מינייהו אמר ליה שעיר המשתלח יוכיח שאין לו טומאה כלל וגורם טומאה חמורה והך פירכא ופירוקא דרב יהודה מדסקרתא בברייתא תניא בהו בפרק דם הנדה ומדלא מייתו לה אינהו ש"מ לא שמיעא לה ובגמרא לא שמיע' להו ובגמרא לא אמרינן תניא נמי הכי דבשקלא וטריא בעלמא לא מיתמר הכי. 
\textbf{ולוי אמר שני מעיינות הן.} מצאתי בשם חכמי הצרפתים שהם מקשים והתניא בפרק יש בכור (דף מו ע"ב) גיורת שיצאת פדחת ולדה בהיותה נכרית ואח"כ נתגיירה אין נותנין לה ימי טומאה וימי טהרה ואמאי ללוי דאמר נסתם הטמא נפתח הטהור אמאי אין לה ימי טהרה הא ממעיין טהור אתו, ומתרצים אין לה ימי טהרה לאו דוקא שאין לה כלום אלא לומר שאם ראתה ביום ז' לזכר וביום שבועיים לנקבה טמאה נדה ומונה בתוך ימי טוהר ימי נדות.\par ויש שמעמידין אותה אליבא דלוי כב"ש דאמרי מעין אחד הוא, עוד פירשו ללוי אף על פי ששני מעיינות הן לא טהרה התורה מעיין זה אלא בנולדות וזו כיון שאין לה דין לידה אף אותו מעיין טמא הוא ורואין את דמה אם מחמשה דמים הוא וזה פשוט ונכון. 
\textbf{בשלמא לרב דאמר מעין אח' הוא מ"ה מטמא לח ויבש.} פי' לבית הלל פשיטא ולבית שמאי נמי כיון שראיה זו מטמאתה מלספור נקיים נמצא שהיא גורמת טומאה וכדם הנדה הוא שמטמא לח ויבש ולא דמי לרואה בתוך ימי טוהר בלא זוב שאין ראייתה כלום אלא ללוי אמאי מטמא לח ויבש בין לבית שמאי בין לבית הלל, ופריק בשופעת.\par אי בשופעת למאי איצטרך וק"ל ולרב גופיה אמאי איצטרך ודאי לבית שמאי ללוי נמי לבית שמאי ואיכא למימר בשלמא לרב טעמיה לבית שמאי קמשמע לן דלא תימא טעמייהו משום דשני מעיינות הן ובהא פליגי קמשמע לן ומודים ואי שני מעיינות הן ביולדת בזוב נמי מטהרו בית שמאי אלא ללוי אמאי איצטרך האי טעמא בתרווייהו מכל מקום שמע מינה דהא לית ליה לאוקמינהו אלא בהך פלוגתא ופריק אפילו הכי איכא למיטעי בה לבית שמאי דסד"א אף על פי שופעות לא תטמא קמשמע לן.\par כיון דמפורש בשמעתין דלרב ימי טוהר שרואה בהן אין עולין לה לספירת זיבה, וקיימא לן בנות ישראל החמירו על עצמן שאפילו רואות טיפת דם כחרדל יושבות עליה שבעה נקיים וקיימא לן אי אפשר לפתיחת הקבר בלא דם אם כן היולדות צריכות שבעה נקיים בתוך ימי טוהר שלהן אלא שימי לידה נמי אם אינה רואה בהן עולין לספירת זיבתה כדלקמן וכן כחב הרמב"ם ז"ל שהיולדת בזמן הזה הרי היא כיולדת בזוב וצריכה שבעה נקיים. 
\clearpage}

\newsection{דף לו}
\twocol{הא דתניא \textbf{ושוין ברואה אחר דם טוהר שדיה שעתה} הקשו בתוספות אם כן מניקה שאמרו צריכה שתפסוק שלשה עונות והלא אחר דם טוהר היא רואה לעולם וכל שכן בזו שאמרו בברייתא שתים בימי עוברה ואחת בימי מניקתה וכו' למה לי הפסקה דעונות והא אחר ימי טוהר היא רואה ומוקמי לה כרבי מאיר וכשאינה מניקה וכגון שמת בנה ומשום רואה אחר ימי טוהר דיה שעתא.\par ולי נראה דה"ק: קיימא לן דלא אמרו דיין שעתן אלא בראיה ראשונה וקאמר השתא שאם ראתה אחר דם טוהר אינה כראיה שנייה אלא כראיה ראשונה ודיה שעתה והוא שהפסיקה כדינה בעונות, וכן נראה פירש"י ז"ל ונכון הוא לומר דשויין אפלוגתא דר' מאיר ור' יוסי בדין כל ימי עבורן וימי מניקתן קאי ומתרץ לה רב בדליכא שהות ובין שראתה בימי טוהר בין שלא ראתה ליכא לטמוייה כלל, ואף על גב דלא הפסיקה נמי כדמפרש לה ואזיל. 
\textbf{אמר רב נדה ליומא ושמואל אמר חיישינן שמא תשפה.} איכא דקשיא ליה לרב ור' יצחק אמאי לא חיישינן שמא תשפה והא חיישינן שמא יבקע הנאד, ותנן הרי זה גיטך שעה אחת קודם מיתתי אסורה לאכול בתרומה מיד אלמא חיישינן שמא ימות.\par ולאו מילתא היא, דהתם סופו למות שהכל למיתה הן עומדין אבל כאן אדרבה סופה להקשות כשמתקרבת ללידה והא דהדר ביה רב לקמן נ"ל דלגבי שמואל הדר בי' וחייש שמא תשפה ואף על גב דמתניתין כר' יצחק מתוקמא דתניא הכי מיהו דשמואל לא מיעקרא בהכי דאכתי איכא למימר מסבר' דחיישינן שמא תשפה דאי ס"ד רב לקולא הדר ביה אמאי גדייה רב אסי ומאי האי דאמרינן עבד עובדא כותיה הא חומרא בעלמא הוא דעבד. 
\textbf{כל שחל קישויה להיות בג' שלה וכו'.} פירש רש"י ז"ל כל שקשתה אפילו שעה א' בליל כניסת ג' אפילו כל היום כולו בשופי ושעה א' מליל ד' להשלמה מעת לעת וילדה אין זו יולדת בזוב דבעינן שופי כל יום ג' המביא לידי זיבה.\par פי' לפי' לאו משום (דכפי) [דבעי] חנניא לילה ויום כלילי שבת ויומו דאם כן היינו דר' יהושע אלא משום דבעי כל יום ג' בשופי ואם קשתה בג' אפילו שפת בד' כלילי שבת ויומו אינה זבה שאין קושי שבג' קובע אותה זבה ובכה"ג לא הוה זבה עד דחזיא ג' אח"כ בשופי דקושי שבשלישי אינו קובע בזיבהולא מצטרף (עד) [עם] יום ד' לקבעה בזיבה והיינו דאמרי לקמן לעולם כדקתני והא קמשמע לן אף ע"ג דאתחיל קושי בג' אם שפתה מעת לעת טמאה לאפוקי מדחנניא בן אחי ר' יהושע ואלו לאפוקי מדר' יהושע בהדיא קתני לה אם שפתה מעת לעת ר' יהושע אומר כלילי שבת ויומו. 
\clearpage}

\newsection{דף לז}
\twocol{\textbf{או דילמא דבר הגורם לטומאה סותר והא לאו גורם הוא.} כתב רש"י ז"ל הרי קרי דבר שאינו גורם וסותר ההיא לאו סתירה היא דחד יומא הוא דסתר וכי בעי רבא לסתור את הכל ופי' דבר הגורם לידי זיבה.\par ולא מחוור לי, חדא דקאמרינן דבר הגורם לטומאת שבעה סותר ולא אמרינן דבר הגורם לידי זיבה, ועוד דתניא הכא קריו יום א' לפיכך סותר יום אחד.\par אלא זהו טעמו של דבר קרי שמטמא יומו סותר יומו ולא מפני שאין אותו היום עולה לשבעה נקיים דהא מעין א' הוא ודינו שיעלה אלא האי דאינו עולה לספירה דיומיה הוא דכיון דרואה הוא אמר רחמנא יסתור ולפיכך הוצרכו לומר שסתירו כטומאתו אבל קושי אם דינו לסתור משום טומאת נדה סותר הוא שבעה אבל משום טומאתו לא היה דינו לסתור ואפילו יומו שהרי עכשיו אינו מטמא בימי זיבה כלל ואף על פי שמטמא בימי נדה אין דינו לסתור ואפילו יומו דדבר הגורם עכשיו סותר ולא דבר שטהור עכשיו וגורס בזמן אחר ומיהו ודאי כיון דמעיין אחד אתי אינו עולה דלא גרע מימי טוהר דבתר לידה דאמרן לעיל לרב דאמר מעין אחד הוא אינן עולין והא לא סתירה מקריא אלא דבעי' נקיים מכל דם של אותו מעין, והא דאמרת הגורם לטומאת שבעה ודאי משום דבעייה לסתירת שבעה אמר שבעה. 
והא דתניא \textbf{ר' מרינוס אומר אין לידה סותרת בזיבה,} אפילו ברואה קאמר לפי' שאין דם ראיה זו גורם כלום ואי קסבר נמי דאי אפשר לפתיחת קבר בלא דם ההוא בקושי חזיתיה ואין קושי סותר כר' מרינוס וכיון שאין דם הקושי שלפני הלידה ולא שלאחר הלידה ראוי להיות סותר אף הלידה אינה סותרת שאין סתירה אלא בראיה דהויא לה לטומאה זו כנגיעה וכנוגעת בטומאתה שאין לה סתירה כלל, ומיהו אם ראתה בלידה אין יומא עולה לדברי הכל אלא שאין זה נקרא סתירה כדפיר' רש"י זכרונו לברכה, והכי נמי מפרשינן בפלוגתא דאמוראי והכי נמי מתוקמא למר אינה סותרת לעולם ואינה עולה לעולם ולמר אינה סותרת לעונם ועולה כשהימים הם ראויים לעלות כגון שאינה רואה דלכולי עלמא לעלות נקיים בעינן ואפילו בתוך ימי טוהר כדאמרן לעיל, והיינו דאמרן מ"ל נקיים מלידה כלומר נקיים אף מלידה ורבא לא נקיים מדם לומר אעפ"י שאינו גורם נקיים בעינן ואף בימי טוהר וכדרב. 
והא דאמר רבא \textbf{א"א בשלמא עולה היינו דלא מפסקת טומאה.} ה"ק א"א בשלמא דינה לעלות בשאינה רואה אפילו ברואה נמי אינה סותרת שאין כאן טומאת ז' מפסקת אלא יומו הוא דלא חזי לעלות דומיא דרואה קרי שסותר יומו ואינו מפסיק כדאמרן בריש פירקן וכ"ש הכא דהאי דם לאו כגורם הוא טומאה כלל ולא מוסיף ביה טומאה דכלום אלא א"א אינו עולה האיכא טומאת ז' ושבועיים דמפסקא, כן נ"ל לפי' שמוע' זו ובתוספת מאריכין בה בענינים הרבה שאינן עולין.\par ואיכא למידק אשמעתין דהא בשילהי בא סימן (דף כ"ד) איבעי להו ימי לידתה שאינה רואה בהן מהו שיעלה לה לספירת זיבתה, ואמר רב כהנא ת"ש ומסקנא ש"מ עולין ש"מ, וי"מ דהכא אליבא דר' מרינוס איירינן דאביי דאיק מדקאמר אינה סותרת מכלל דאינה עולה דה"ל למיתנא רבותא דעולה ורבא אמר אפילו לר' מרינוס עולה והא דאמר רבא מנא אמינא לה מואח' תטהר לומר דכיון דקרא קא דרשינן אפילו לר' מרינוס אית ליה, וכן הא דתניא מזובה ולא מנגעה מזובה ולא מלידתה קרא קא דייק והיינו דקאמר ליה אביי תני חדא כלומר לר' מרינוס דוקא חדא אבל ברייתא תרתי קתני והא דאמר אביי מנא אמינא לה לומר דכיון דמשכחת תנא דאמר אינה עולה ר' מרינוס היא דהא לרבנן עולה, וזה הפי' שמעתי ולא נתקבל לי.\par ועכשיו מצאתי בתוספות בשמו של ר"ש ז"ל שכתבו בתשובותיו ואמר הרב ז"ל תדע דהא בעי לה לקמן בפרק בא סימן איבעי להו וכו', ואין דרך התלמוד לשאול בעיא אחת שתי פעמים ולא מצינו כן בשום מקום תלאוהו באילן גדול, ועדיין אינו מחוור לפי שאם היה אביי מודה לרבנן דעולין לא הוה משוי ליה לר' מרינוס טועה וחולק דליכא למידק מלישנא דידיה הכי כלל כדפרישית, ועוד הא דפרישו בדרבא דאמר מנא אמינא לה דמקרא דייק ולא מצי רבי מרינוס למפלג עלה הא לאו מילתא היא דאי איכא למידק מקרא תיקשי לר' אלעזר דאמר דאינה עולה אלא היינו טעמא דאביי דאיהו סבר לתרוצא לההיא ברייתא דבשלהי בא סימן כדמתרץ לה רב פפא א' שאני התם וכו'.\par ומסקנא דעולין ודאי כרבא אתיא דקי"ל כוותיה ולא כדברי הרב ר' יעקב ז"ל שפי' שהלמ"ד לידה כהכא אלא כפי קבלת הגאונים שהוא לחי במסכת עירובין (דף ט"ו) דאיתותב מיניה התם בגמרא ת"ש מעובדא דרב וההיא דתניא בפרק המפלת אינן עולין לרבא אתיא כר' אליעזר דאמר מסתר נמי סתרא והא דלא מייתינן לכולהו בשמעתין כמה איכא בתלמודא דכוותייהו שדברי תורה עניים במקומן ועשירים במקום אחר ומה שאמרו שלא מצאו בתלמוד בעיא א' בשני מקומות כאן מצינו.\par ועי"ל לו דהתם אמוראי בחראי אתו למיפשט אי כאביי או כרבא והלכה או אין הלכה מיבעיא להו וכן מצינו בפרק המגרש (דף פה ע"ב) דאיבעי להו מי בעינן ודן או לא עביד בעיא סתם בפלוגתא ברבי יהודה ורבנן [ועוד בעיא בפלוגתא דמתני'] דמתני' בפרק המקבל (דף קי"ד) מהו שיסדרו בבעל חוב וכן בפרק חזקת הבתים (דף מ ע"ב(איבעיא להו סתמא מאי קא מיבעי ליה מתרי לישני דרב יוסף דלעיל הי מינייהו הלכה וזו כן ופשטו מברייתא דעולין ואידתי ליה דאביי דסבר לכ"ע בין לרבנן בין לר' אליעזר אין עולין. 
 והא דתניא \textbf{הכא מה ימי נדתה אין ראויין לזיבה ואין ספירת ז' עולה בהן.} לקמן בשילהי בא סימן אמרינן זאת אומרת ימי נדתה שאינה רואה בהן עולין לה לימי זיבתה אלא שהן חלוקין בפירושיהן דהכא קאמרינן אין ספירת ז' של זבה גדולה עולה בימי נדה לפי שאינה נעשית תחלת נדה משראת נ' בזיבה עד שתספור שבעה נקיים והתם בזבה קטנה קאמרינן שאע"פ שראתה שנים בימי זיבה ונעשת נדה מונה יום אחד טהור מאותן ז' של נדה לזיבתה ודיה, ולשון אחר פירש שם רש"י ז"ל והכל שוין בדבר זה שמשעה שנעשת זבה גדולה כל ראיות שתראה אינה עולה בהן אלא סותרת ואינן ראויין למנות מהן ימי נדה וזיבה. 
\clearpage}

\newsection{דף לח}
\twocol{והא דאמרינן \textbf{הא קמשמע לן דאפשר לפתיחת קבר בלא דם.} אלמא אי הוה דם בפתיחת הקבר הויא זבה ליכא לאוקמא אלא בנפלים דלית להו קושי דאי בולד מעליא כי הוה דם בפתיחת הקבר נמי לא הויא זיבה דהא בקושי חזיתיה שאין לך קושי גדול מפתיחת הקבר אלא בנפלים הוא דמתוקמא ליה דמאה יום בלא קושי קתני, ואין זה נכון לומר דאין פתיחת הקבר בלידה קושי דלא מסתבר הכי ועוד דהא מכל מקום דמה מחמת עצמה ולא מחמת ולד קרינן ביה, וה"ה דמצו בגמרא למימר דא"א לפתיחת הקבר בלא דם ויש קושי לנפלים קמ"ל אלא רואה בלא קושי קתני דאלו בקושי יש רואה כל ימיה. 
הא דתנן \textbf{אמר להן דיו לבא מן הדין וכו'.} לא דסבר רבי אליעזר בימי נדה נדה בימי זיבה טהורה דהא רבי אליעזר מטמא תנן בכל עת משמע, אלא לומר כיון שיש לנו לדרוש דיו על כרחנו ונטמא ימי נדתה והכתוב אומר ישיבה אח' לכולן נמצא שמטמאה בין בימי נדה בין בימי זיבה, ואם תאמר אם כן מיפרך קל וחומר והיכא דמיפרך קל וחומר לא אמרינן דיו כדאיתא בבבא קמא, איכא למימר הכא משום דיו לחודיה לא מפרך דהא ב) טהורה בימי זיבה אלא ודאי בא דיו וטמא (יהיה) ימי נדתה ובא הכתוב וטמא אף ימי זיבה וכה"ג אמרינן דיו והיינו דמקשינן לקמן רבי אליעזר ואימא בימי נדה נדה בימי זיבה טהורה ופריק אמר קרא תשב וכו'.\par ויש מפרשים דלר' אליעזר אית ליה הכי ודאי דבימי נדה נדה וביומי זיבה טהורין כדין דיו, וה"ג לקמן דמיה מחמת עצמה ולא מחמת ולד ואימא בימי נדה נדה בימי זיבה טהורה בניחותא כלומר זכותא דקל וחומר לא ליפרך ומקיים קרא וקל וחומר וגרסי בתמיה ורבנן אמר קרא תשב ישיבה אחת לכולן, וזו היא גרסתו של ר"ח ז"ל וקבלת הנוסחא. 
 והא ד\textbf{אמר רבא בהא זכנהו ר' אליעזר.} טעמייהו קא מפרש דודאי ר' אליעזר אפילו פרכיה לק"ו אינו בדין עד שיטמא ימי טוהר שהתורה טהרתם סתם, ועוד שאין לך טעם להחמיר עליו יותר מן השופי ורבנן נמי לא צריכי לק"ו אלא למיפרך טעמיה דר"א. 
\clearpage}

\newsection{דף לט}
\twocol{\textbf{לומר שאין אשה קובע לה וסת בתוך ימי זיבתה.} פירש זיבתה ממש וכל שכן בימי נדתה שאם ראתה נדה בריש ירחא אינה קובעת וסת עד שמונה עשר יום שבעה ימי נדה ואחד עשר של ימי זיבה, והא דקתני אחד עשר משום דבאחד עשר לא קבעה לעולם אלא ביומי הראוין לנדה קבעה אבל בימי נדתה עצמן לא קבעה כלל והכי נמי איתמר בפרק קמא דמכילתין דתנן חוץ מן הנדה וכו', כדאיתא התם.\par ואי קשיא דשמואל אדריש לקיש ור' יוחנן דאמרי אשה קובעת לה וסת בימי זיבתה התם אי לרב פפא כגון דחזיא בריש ירחא ובתמניא בירחא וחזיא בריש ירחא ובתמניא בירחא והשתא חזאי בתמניא בירחא ובריש ירחא לא חזיא דכיון דעקרא דריש ירחא וקבעה דתמניא אמרינן הך דתמניא עיקר ואינך תוספו' דמים הוו ואי לרב הונא בריה דר' יהושע בהכי לא קבעה אלא כגון דחזאי תרתי קמייתא ממעין סתום ושלשה בימי זיבתה דאמרינן הך הקדמה תוספו' דמים הוות וכדפרשינן בגמרא גופא למימרא דריש לקיש ור' יוחנן דימי נדה הכי נמי מתפרש רישא דימי זיבה אבל חזיינוהו בימי זיבה ממש דכ"ע אין אשה קובעת בהן וסת כדשמואל נמצא שאין האשה קובעת לה וסת אלא במפלגת אחד עשר יום ורואה בי"ט וביותר מכאן, אבל רבה ראיותיה בפחות מכאן אינה קובעת.\par אלא שהרב ר' אברהם בר דוד ז"ל כתב אם רגילה לראות מט"ו לט"ו לראיתה דכל חדא וחדא קיימא לחברתה בתוך ימי זיבה רואין את האמצעיות כאלו אינן והשאר קובעת לה וסת מכ"ט לכ"ט דקיימי אהדדי בימי נדה ואי חזיא בהון ארבע ראיות מכ"ט לכ"ט קבעה לה וסת להפלגת כ"ט ובעיא עקירה תלתא זימני ובדיקה אבל אמצעית דקיימי להו בימי זיבה לא בעיא בדיקה ועקרא להו בחדא זימנא, אלו דברי הרב ז"ל, והם צ"ע, ואין אני מוצא בהם טעם שהרי אשה זו לא הפלינה כ"ט כדי שתקבע לה וסת להפלגת כ"ט.\par ועוד מצא הרב ז"ל קביעת וסת במקרבת ראותיה ואמר שלא אמר שמואל אין אשה קובעת וסת בימי זיבתה אלא בתוך אחד עשר כדקתני מתניתין אבל בימים הראויים לספירת אחד עשר קובעת הילכך אם ראתה שלשה ימים בתוך אחד עשר ויום א' בתחלת [ימי נדה] אף על גב דלא מתחלת נדה עד דספרה שבעה נקיים קובעת דלעניין וסת ימי נדה חשיבי שלא טהרו אלא אחד עשר וה"ט משום דקים להו לרבנן דבשבעה ימי נדה היא מתמרקת מדמיה ולא הדרי עד אחד עשר יום וכי מטיא לשמונה עשר יום בין שהיא ראוייה לנדה בין שהיא מספירת זיבה ראויין הן לקביעת וסת, וכבר הורה זקן ז"ל.\par אלא שעכשיו אין הנשים משמרות פתחי נדה וזיבות וכטועות משוינן להו ומכל מקום אם ראתה שלשה פעמים כל א' וא' בתוך י"ט יום זה דבר [ברור] הוא שעדיין לא קבעה וסת שהרי אי אפשר שלא תהא אח' מהן בתוך אחד עשר ממש אבל אם ראתה על הסדר הזה שמונה פעמים קבעה לה וסת מפני האמצעיות כמו שכתבנו לדעת הרב דהאיכא ארבע ראיות וקמייתא לאו בהפלגה חזיתא ולא הוו תלתא הפלגות עד דאיכא ארבע ראיות וחוששת היא פעם אחת לכולן.\par ולפי דעתי מכאן שהחמירו בנות ישראל על עצמן להיות כטועות, ורבי נמי דאתקן להן הכי אין מונין לעולם ואין משגיחין על שמונה עשר ולא על ימי נדה והאשה קובעת וסת בכל זמן שתראה בין במקרבת ראיותיה בין במפלגת אותן ולא נצטרך ללמד לבנות ישראל ימי נדה וי"א שכבר נשתכח מהן לגמרי וגזרו עליו ולא נחלק נמי בין רואה בהן שלש פעמים לרואה שמונה שלא יבואו לטעות בקביעת הוסת והרבה קלקולין שיהיו קרובין לבא בדבר.\par וחכמי צרפתים זכרונם לברכה כתבו בתוספות דרבי יוחנן וריש לקיש פליגי אדשמואל והלכה כמוהם ואשה קובעת לה וסת בין בימי נדה בין בימי זיבה (ודין) [לדין] התלמוד ומכל מקום בתוך שבעה ימים שראתה בהן נדה אינן יכולין להחמיר לדין הגמרא כדחנן בפרק קמא חוץ מן הנדה וכו'. ועכשיו לעולם קובעת. 
\textbf{מקבע לא קבעה.} פרש"י ז"ל דתיבעי ג"פ לעקרן, מיחש מהו דניחוש לה אם היתה רנילה מט"ו לט"ו דהיינו ימי זוב מיבעי' למיחש ולא תשמ' ליום ט"ו קודם ראיה שמא תראה ואינו יודע היכי אתינן למיפשט הא מילתא מדשמואל החס לר' פפא נמי ראית עשרין ותרין קמייאתא בימי נדה הוו וראית עשרין וז' דהאידנ' בימי נדה הויא, ולהאי פירושא אמאי לא תיחוש להו בכל זמן שיבא לה וסתה כיון שהוקבע הוסת כראוי בזמן נדה.\par אלא אפשר לפרש מקבע לא קבעה מיחש מהו דתיחוש לה אם היה לה וסת בין קבוע בין שאינו קבוע בימים הראוים לוסת ואירע לה אותו היום בימי זיבה מהו שתחוש שמא בימים הללו תראה ולא תשמש או דילמא כשם שאינה קובעת וסת בתוך י"א כך אינה חוששת לוסת הראשון שלה בימי י"א והא מילתא מיפשטא בהדיא מדשמואל לפירושיה דרב פפא.\par ולענין גמר' ודאי מסתברא דלית הלכתא כרב פפא אלא כרב הונא ברי' דר' יהושע ואיהו כיון דאידחיא לראיה דרב פפא מדשמואל ודאי מיפלג פליג עלי' וכיון דוסתות דרבנן אע"ג דלא איפשטא ליה לרב הונא לקולא אנן לקול' נקיטו בה ואע"ג דאמר רב פפא בשלהי האשה דלק' דחיישא, רב פפא לא מהימן בה דאיהו מדשמואל אמרה והא אידחי, אבל הרב רבי אברהם בר דוד ז"ל פסק בספרו כר' פפא ואף אנו עליו נסמוך וכ"ש מאחר שכתבנו שאין הנשים יודעת פתחי נדה שכל ראיה שהן רואות חוששין לה בכל זמן שהוא בתוך נדה הן. 
 גרסת רש"י ז"ל: \textbf{א"ל רב פפא אלא הא דאמר ר"ל וכו' אלמא מריש ירחא מנינן.} כלומר מדקרי ליה לפעם ג' בתוך ימי נדה לפי חשבון הראוי מנינן.\par וק"ל והא ר' פפא גופיה נמי אית ליה נדה ופתחה מכ"ז מנינן ולא לפי חשבון הראוי ועוד דילמא משום תרי זימני קמא דקביעינהו בימי נדה קרי ליה הכי.\par ואפשר דהכי קאמר ליה והא הכא (דמעשרין וחמשה) [דמחמישה בירחא] לריש ירחא קמייאתא [{\small פי' לריש ירחא ב' וכונתו הפסקה קמייתא לריש ירחא} ] לא הפליגה אלא כ"ה יום ואע"פ כן כיון שאנו רואין לה עכשיו שהפליגה למ"ד מנינן מריש ירחא ולא מנינן מחמשא בירחא לרישי ירחא ונאמר עיקר הוסת בראשון ו) ירחא לריש ירחא הוא דאיכא למ"ד ובשני הויא ליה עיקרו של וסת מחמשה לחמשה דהא ודאי השתא בהפלגת למ"ד חזאי וקבעה לה וסת להפלגות למ"ד וראיות דחמשה בירחא קמא לא מפסקא, ולא מנינן מינייהו אלמא דמנין כ"ב מעיקרא נקטא והתם נמי כיון שכבר קבעה לה וסת ואנו צריכין לחוש לו מכ"ב לכ"ב מעיקרא וסת ראשון מנינן ואין ראיה שבאמצע מפסקת, ופריק ר"ה בריה דר"י דלא אמרי' הכי אלא ברואה מעיקרא (ממנין) [ממעין] סתום ועכשיו קרבה בתוך כך דאפילו בימי נדה אמרי תוספת דמים הוא, וכ"ש בימי זיבה דכיון דקמייאתא (ממנין) [ממעין] סתום והפליגה שתיים ועכשיו נמי ראתה יש לנו לומר תוספות הואי אבל לחוש אינה חוששת אלא למנין ראיות.\par וה"ה ודאי דה"ל לתרוצי שאני התם דכיון דחזאי [בפעם ג' בחמישה בירחא] אמרי' דחמשה בירחא עיקר ודרישי ירחי (עיקר) [מיקרי] תוספות ואע"פ כן לחוש אין חוששין אלא בהפלגות ולא למנין הראוי. אלא מסתברא ליה דקביעות ראיות קמייאתא לא אמרי תוספת דמים הוו. וגם זה הפי' אינו מחוור.\par ומ"מ היינו ריש ירחי דנקט לאו דוקא אלא הפלגו' שוות בהן דאלו בוסת החדש ודאי למנין הראוי חוששת דהא לאו בהפלגות שוות חזיא מעיקרא אלא בימי דחדש הוא דהשוות ראיותיה וכדבעינן לברורי קמן בפרק האשה (סד, א).\par ושמע מינה לרב הונא דאינה מונה כ"ב אלא מכ"ז ואי חזאי בכ"ב חזר הוסת הראשון למקומו ונעקר החדש לגמרי אבל אם לא ראתה בעשרין ותרין שחששת לו חוששת לסוף חמשה ימים דה"ל כ"ז מכ"ז שראתה בו תחלה והיינו דתרנגולת'. [{\small ע' ביאור דברי הרמב"ן בריטב"א} ] וכן דעת ה"ר משה ולא כן פסק הרב ר' אברהם בספרו ז"ל. א"ל רב פפא אלא הא דאמר ר"ל וכו' אלמא מריש ירחא מנינן. כלומר מדקרי ליה לפעם ג' בתוך ימי נדה לפי חשבון הראוי מנינן.\par וק"ל והא ר' פפא גופיה נמי אית ליה נדה ופתחה מכ"ז מנינן ולא לפי חשבון הראוי ועוד דילמא משום תרי זימני קמא דקביעינהו בימי נדה קרי ליה הכי.\par ואפשר דהכי קאמר ליה והא הכא (דמעשרין וחמשה) [דמחמישה בירחא] לריש ירחא קמייאתא [{\small פי' לריש ירחא ב' וכונתו הפסקה קמייתא לריש ירחא} ] לא הפליגה אלא כ"ה יום ואע"פ כן כיון שאנו רואין לה עכשיו שהפליגה למ"ד מנינן מריש ירחא ולא מנינן מחמשא בירחא לרישי ירחא ונאמר עיקר הוסת בראשון ו) ירחא לריש ירחא הוא דאיכא למ"ד ובשני הויא ליה עיקרו של וסת מחמשה לחמשה דהא ודאי השתא בהפלגת למ"ד חזאי וקבעה לה וסת להפלגות למ"ד וראיות דחמשה בירחא קמא לא מפסקא, ולא מנינן מינייהו אלמא דמנין כ"ב מעיקרא נקטא והתם נמי כיון שכבר קבעה לה וסת ואנו צריכין לחוש לו מכ"ב לכ"ב מעיקרא וסת ראשון מנינן ואין ראיה שבאמצע מפסקת, ופריק ר"ה בריה דר"י דלא אמרי' הכי אלא ברואה מעיקרא (ממנין) [ממעין] סתום ועכשיו קרבה בתוך כך דאפילו בימי נדה אמרי תוספת דמים הוא, וכ"ש בימי זיבה דכיון דקמייאתא (ממנין) [ממעין] סתום והפליגה שתיים ועכשיו נמי ראתה יש לנו לומר תוספות הואי אבל לחוש אינה חוששת אלא למנין ראיות.\par וה"ה ודאי דה"ל לתרוצי שאני התם דכיון דחזאי [בפעם ג' בחמישה בירחא] אמרי' דחמשה בירחא עיקר ודרישי ירחי (עיקר) [מיקרי] תוספות ואע"פ כן לחוש אין חוששין אלא בהפלגות ולא למנין הראוי. אלא מסתברא ליה דקביעות ראיות קמייאתא לא אמרי תוספת דמים הוו. וגם זה הפי' אינו מחוור.\par ומ"מ היינו ריש ירחי דנקט לאו דוקא אלא הפלגו' שוות בהן דאלו בוסת החדש ודאי למנין הראוי חוששת דהא לאו בהפלגות שוות חזיא מעיקרא אלא בימי דחדש הוא דהשוות ראיותיה וכדבעינן לברורי קמן בפרק האשה (סד, א).\par ושמע מינה לרב הונא דאינה מונה כ"ב אלא מכ"ז ואי חזאי בכ"ב חזר הוסת הראשון למקומו ונעקר החדש לגמרי אבל אם לא ראתה בעשרין ותרין שחששת לו חוששת לסוף חמשה ימים דה"ל כ"ז מכ"ז שראתה בו תחלה והיינו דתרנגולת'. [{\small ע' ביאור דברי הרמב"ן בריטב"א} ] וכן דעת ה"ר משה ולא כן פסק הרב ר' אברהם בספרו ז"ל. }

\newchap{פרק \hebrewnumeral{5} יוצא דופן}
\twocol{\clearpage}

\newsection{דף מ}
\twocol{\textbf{ורבי שמעון ההוא דאפילו לא ילדה אלא כעין שהזריע' אמו טמאה לידה.} וא"ת לרבנן נמי מבעי להו להכי כיון דאינהו אמרי הבית טמא ולית להו נימוק הולד עד שלא יצא כדאיתא בפרק המפלת פשיטא דאמו טמאה לידה זה כתב רש"י ז"ל.\par וק"ל נהי נמי דלית להו במפלת שליא נימוק הולד היכא דחזינן ודאי דנימוק מנא להו דאמו טמאה לידה ואפשר דלר"ש דאית ליה ביטול ברוב לענין טומאה איצטרך קרא לענין לידה אבל לרבנן לית להו ביטול בכה"ג, ואיכא למימר נמי לרבנן ממילא ש"מ דא"כ לימא קרא כי תבעל וילדה מאי כי תזריע דאפילו לא ילדה אלא כעין שהזריעה טמאה לידה. 
\textbf{מ"ט גמר לידה מבכור.} פי' דאי לאו ג"ש כיון שנתרבה יוצא דופן בכלל לידה גבי אדם א"א למעטו מכי יולד ולהכי צריך ג"ש דלידה לידה והא דאמרינן שכן אמו מאמו לא דהיא ג"ש אלא מסתברא לידה לידה דאדם גמר כן משום דדמי מדמי הוא דהכא כתיב אמו והכא כתיב אמו והא דאמרינן לקמן מאמו אמו נפקא לאו דוקא אלא חדא מטעמיה דג"ש נקט : 
\textbf{זאת תורת העולה היא העולה הרי אלו ג' מעוטין פרט לנשחטה בלילה וכו'.} פירש לר' יהודה כיון דממעט הני אף על פי שפסולן בקדש ומכשר הלן והיוצ' ושארא כדבעינן למימר צריכי מעוטא לכל חד וחד אבל לר' שמעון כיון דקרא א' מרבה וקרא א' ממעט הריבוי ריבה הכל והמיעוט מיעט הכל, ול"ק כאן שפיסולו בקדש כאן שאין פיסולו בקדש.\par והא דאמרינן הרי אלו [ג'] מיעוטין ולא אמרינן אין מיעוט אחר מיעוט אלא לרבות משום דכיון דכל אחד ממעט את שלו אין כאן מיעוט אחר מיעוט שאין מיעוט אתר מיעוט לרבו' אלא כשהן ממעט' דבר אחד כגון שאמרו (סנהדרין טו, א) עשרה כהנים כתובים בפרשה כהן ולא ישראל ואין מיעוט אחר מיעוט אלא לרבות אבל כאן כל מיעוט הוא צריך למעט את שלו.\par וכיוצא בזו בב"ק (מד, ב) שור שור שור ז' פעמים להוציא שור האשה ושור היתומין ושור האפוטרופסין וכו' ולא היו מיעוט אתר מיעוט והני תלתא מיעוטי נמי ממעטי הני תלתא פסולי כדפרישית.\par אבל במס' הוריות מצאתי בפרק ראשון בירושלמי (ה"א) גבי נפש כי תחטא אחת תחטא בעשותה [תחטא] הרי אלו מיעוטין דמקשי בכל אתר את אמרת מיעוט אחר מיעוט לרבו' וכאן את אמרת מיעוט אתר מיעוט למעט א"ר מתניא שניה היא דכתיב מיעוט אחר מיעוט לאחר מיעוטי ולדעת זו ההיא דאמרי' בסנהדרין כהן ולא ישראל מפני שכולן צריכין לכתב לומר דעשרה בעינן ד) אבל בשאר דוכתי ג' מיעוטין או יותר נדרשין הן כולן למיעוט כמשמען ולא נאמרה מדה זו בתורה אלא בשני מיעוטן מיעוט אחר מיעוט.\par מ"מ כל הנך פסולי דמכשיר ר' שמעון מודה בהו ר' יהודה וטעמא מפרש במסכת זבחים (דף פ"ד) מפני מה אמרו לן בדם כשר שהרי לן כשר באימורין לן באימורין כשר שהרי לן כשר בבשר, יוצא הואיל וכשר בבמה, טמא הואיל ואשתרי לגבי צבור, ונשחט חוץ לזמנו הואיל ומרצה לפיגול, חוץ למקומו הואיל ואתקו' לחוץ לזמנו ושקבלו פסולין וזרקו את דמו בהנך פסולין דחזו לעבוד' צבור וכי דנין דבר שלא בהכשרו מדבר שהוא בהכשרו תנא אזאת תורת העולה ריבה קא סמיך הדין גמרא דהתם (ובתכפה) [וכתבנוה] מפני שהיא (תמה) [סתומה] ויש לדקדק בה דא"כ נשחטה בלילה נמי כשרה שהרי כשר בבמת יחיד כדאי' בזבחים (דף ק"כ), איכא למימר אתיא כמ"ד התם שחיטת לילה פסולה בבמת יחיד.\par אלא הא קשיא יצא דמה חוץ לקלעים נימא דכשר שהרי כשר בבמה שאין יוצא בבמה לא בבשר ולא בדם וכן חוץ למקומו נמי דקאמר הואיל ואתקוש לחוץ לזמנו לימא הואיל וכשר בבמה, ועוד דקאמר ושקבלו פסולין וזרקו את דמו בהנך פסולין דחזו לעבודת צבור אפילו זר גמור נמי יהא כשר שהרי כשר בבמת יחיד.\par ואיכא למימר נשחטה בלילה ויוצא דמה ליכא לאכשורי (בשרן) דאי הכי מיעוטין מאי אהנו לי ומסתבר' ליה לאוקמי בהני דבעיקר הכשרן נפסלו מאינך והא דאוקי בהנך פסולי דחזו לצבור ולא אמר משום דכשרין בבמה דבהא פשיטא ליה דליכא למילף מינה שזה ודאי דבר שעקר הכשרו כן הוא ולית להו הכשר בכשרין טפי מפסולין לעולם ודקאמרת חוץ למקומו הואיל ואתקוש משום דכל היכא דמשכח הואיל בפנים לא מייתי ליה מבמה, כנ"ל.\par ובתוספות מאריכין בע"א, וניתנין למעלה שנתנן למטה וכן בחוץ ובפנים ופסח וחטאת כולהו כיון דבפנים נמי אשכחן בהו הכשרא לא צריכא ליה למימר' דודאי לא ירדו. 
\clearpage}

\newsection{דף מא}
\twocol{והא דאמרינן \textbf{חד לבהמת קדשים.} פי' דאע"ג דהאי וזאת לא מיתר ביוצא דופן שאם עלה ירד מיהו אין במשמע תורת העולה לרבות כל העולי' לגמרי כיון שמקרא א' מרבה ומקרא אחר ממעט וא"א להעמי' המיעוט בבהמת תולין (דכולהו מעלמא) [דחולין מאמו] נפקי אפילו לענין שאם עלו ירדו הילכך ע"כ מוקמינן מיעוטא בבהמת קדשים וכיון דשקולים הם יבואו כולם וכל שהוא פסול בבהמת חולין מיעט בבהמת קדשים לפסול ולומר שאפילו אם עלה ירד דמיעוטא הכי משמע שאם עלו נמי ירדו. 
הא דאמרינן \textbf{מקור מקומו טמא.} לא דמקור בנגיעה דידיה מטמא דם דהא בית הסתרים הוא ודם גופה אינו לא אוכל ולא משקה אלא גזרת הכתוב שטומאה נבראת שם מתחלה וכל שנברא בו טמא הוא ומטמא (ביציאותיו) [בנגיעתו] טומאת ערב במשהו כדם הנדה דבמקומו הוא נעשה דם הנדה, וה"נ מוכח בפ' בתרא דמכילתן דקאמרינן באשה שמתה ויצאה ממנה דם במשהו מטמא משום כתם משום דמקור מקומו טמא ולא מטמא ליה נגיעה דמת. 
והא דאמרינן \textbf{ואזדא ר' יוחנן לטעמיה שאמר משום רשב"י וכו'.} אף ע"ג דר"י להא משום רשב"י נמי אמרה ור"ל נמי לר"ש הוא מודה דאשה טהורה היא משום דנסיב ליה קרא מפורש קאמרינן הכי לימא דאין אשה טמאה קרא מפורש הוא ובאי דם עצמו טמא פליגי מר גמר לדם מאשה ומר לא גמר. 
הא דאקשי' לר' שמעון \textbf{פולטת תיפוק לי' דהא שמשה.} ה"ק למה לי טומאה פולטת בחוץ הא שמשה ונטמאת אפילו בפנים, ואלו לרבנן י"ל שלא הצריכה התורה טבילת משמשת אלא מפני פליטתה שאלו מפני שמוש' טהורה היא דהא מגע הוא ואותו מקום בית הסתרים הוא למגע אבל כשהיא חוזרת ופולטת עשאה הכתוב כרואה מדכתיב יהיה וההוא גלי אורחצו במים דמשום פליטה הוא דטמאים אלא לר"ש קשיא. ומהדרינן בטבלה לשמושה. 
ואקשינן \textbf{למימרא דמשמשת בטומאת ערב סגי לה ולהכי טבלה והאמרת רבא וכו'.} ואע"ג דרבא משום פליטה קאמר ואפשר לך לומר שבזה בא ר"ש ללמד שאינה טמאה עד שתצא טומאתה לחוץ והך קושיא לרבנן נמי היא אלא כיון דאיירי בדר"ש מפרש ואזיל בהדי' ומהדרי' בשהטבילוה במטה כלומר וטהורה לשמושה ואח"כ פלטה טומאת' לבית החיצון ולא יצא לחוץ שלא הלכה ולא נתהפכ'. 
ואקשינן \textbf{מכלל וכו'.} וכי תימא דילמא אשתייר ומספיק' אסרינן לה בטומא' אי הכי חיישינן שמא נשתיי' מיבעי ליה אלא ודאי מדלא קאמר הכי ש"מ דכל היכא דאזלא בכרעא מותרת בתרומה דודאי שדתיה לכוליה ודרבא בשלא הלכה הוא דקאמר והתם לא הוה צריך למימר חיישינן אלא א"א הוא. 
\clearpage}

\newsection{דף מב}
\twocol{\textbf{אלא לרבא נמי כשהטבילוה במטה.} ורבא אקרא קאי ופריש דכי כתיב עד הערב לאו במהלכת ולאו במכבדת את הבית דא"כ תיפוק לי משום נגיעת חוץ אלא בשוכבת על מטתה ואינה מתהפכת אבל מתהפכת כל ג' ימים נמי אסור' ולאו מגזרת הכתוב דמשמשת אלא משום פליטה וי"ל דאפילו במהלכת חיישינן שמא נשתייר ולהכי לא אוקמוה לקרא (בהדי) [בהכי] וכי אקשינן אי הכי חיישינן שמא נשתייר מיבעי ליה לאו למימרא דלא חיישינן אלא לומר דרבא לא בכי האי גוונא מיירי.\par ולאו מילתא היא דא"כ לישמועינן רבא דעדיפא ולימא כי כתיב בשאינה מתהפכת אבל במתהפכת אפילו מהלכת כל ג' ימים אסורה לאכול בתרומה שמא תפליט ועוד אי משום חששא דשיור לא מוקי קרא במהלכת לוקמיה משמשת במכבדת שהיא טהורה משיור לגמרי אלא הקרא לא מתוקס בהנך גווני משום דאיכא נגיעת חוץ בפליטה כדפי' רש"י ז"ל.\par והא דתנן במסכת מקואות (ח, ד) האשה ששמשה את ביתה ירדה וטבלה ולא כבדה את הבית כאלו לא טבלה תפתר בשלא הלכה ואעפ"כ אם כבדה את הבית מותר' דודאי נפק כוליה. א"נ כשהלכה דפולטתו בבית החיצון וכותלי בית הרחם העמידוהו ומ"ה צריכה כבוד לטהרות משום נגיעה דשכבת זרע אלא דלית ליה דין פולטת לראיה ולטמא בפנים אלא ראשון דמגע שכבת זרע הוי. 
הא ד\textbf{בעא מיניה רב שמואל בר ביזנא מאביי.} פולטת ש"ז אי רואה הויא או נוגעת. ואסיקנא דלרבנן רואה הויא לכל מילי ואפילו לר"ש נמי לסתור ולטמא במה שהוא רואה הויא משמע לי דפשטין ודאי דלא כרב הונא דאמר בפ' המפלת דאפילו בעל קרי לא סתר אלא משום דא"א לו משום צחצוחי זיבה דמדבעי חתימת פי האמה אלמא נוגע הוי וכ"ש בפולטת דלא מגופ' הויא ואינה מטמאה אלא בחוץ דאית לן למימר למימר נוגעת הויא ולא סתרה דהא אין בפליטה דאשה צחצוחי דם כלל.\par וא"ת אי הכי קשיא לאביי היכי פשיטא לר"ש לסתור ולטמא במשהו הויא דהא שמעינן ליה לר"ש דס"ל כר' נתן דאמר זב צריך חתימת פי האמה ואיתקוש ב"ק לזב וצריך נמי חתימת פי האמה כדאיתא בפ' אלו דברים. וי"ל קסבר אביי דבעל קרי אע"פ דצריך חתימת פי האמה רואה הוי ושעורא בעלמא הוא דאצרכיה רחמנא דומיא דזב וכדפרישית בפ' המפלת. ומיהו גבי אשה דליכא למימר תחימת פי הרחם דשיעורא אחרינא הויא במשהו כרוא' דם בזיבה ולא אמרינן בהא דייה כבעלה משום דלא אפשר.\par וי"מ דבמשהו דקאמרינן כעדשה דשרץ קאמרינן מ"מ מסקי השתא דרואה הויא.\par ויש מחכמי הצרפתים ז"ל שחדשו בה ואמרו כיון שהיא רואה וסותרת יומה בזמן הזה שכל הנשים ספק זבות משוינן להו אשה ששמשה בלילי שבת וראתה בשב' ופסקה טהרה בו ביום או למחרתו אינ' מתחלת ספירתה עד יום ד' בשבת שהוא ג' ימים לאחר שמושה מ"ט דכל ג' ימים פולטת היא וסותר' וקי"ל נמי דשש עונות שלימות מטמאה בפולטת והלכה למעשה הורו והנהיגו הדור בחומרא זו אלא שמתירין בכבוד הבית יפה.\par ויש לדקדק אחר דבריהם שאפילו הדבר כן שהפולטת סותרת לבעלה אין לחוש כן במהלכ' שכבר נתפרשה לנו שאם הלכה מותרת בתרומה לערב ואין חוששין שמא נשתייר אבל אם האשה הרגישה בעצמה שפלטה בשני שלה או בג' או בעומדת על מטתה ומתהפכת בכאן יש מקום להורא' זו.\par והרב ר' אברהם בר דוד ז"ל נשאל בהוראה זו ואמר שלא אמרו פולטת סותרת אלא בענין תרומה וקדשים ולענין טהרות אבל לבעלה אינה סותר' שהרי אמרו דבר הגורם סותר דבר שאינו גורם אינו סותר. וכדתניא מה גורם לו זובו ז' לפיכך סותר ז' מה גרם לו קריו יום א' לפיכך סותר יום א' ואפילו קושי אינו סותר בזיבה מפני שאינו גורם עכשיו כל שכן פולטת שאינה גורמת טומאה לבעלה שאינה סותרת לעולם ועולה. ומצאתי בתוס' שהוזכרה ביניהם סברא זו ודחוה בשתי ידים ולא קבלו אותה כלל.\par והרב ז"ל הביא ראיה לדבריו מדתניא ד"ש אומר ואחר תטהר אחר מעשה תטהר אבל אמרו חכמים אסור לעשות כן שמא תבא לידי הספק דאלמא אי לאו חששה דראיה משמשת והולכת ואמאי והא יש לחוש לפולטת שסותרת.\par וזו אינה ראיה, די"ל התורה לא חששה לפולטת שאפשר לה שלא להתהפך, וכן זו שהקשו בפ' המפלת בעשרין וחד תשמש לרוחא דמילתא מתרצי משום דמשמע דאסר לה לשמש בכל ענין ואע"פ שלא תתהפך כל היום וכ"ש למאן דסבר נוגעת הויא דצרכינין לאוקומא כר' שמעון וחכמים אף לזה חששו ואסרוה לשמש, א"נ תטהר לתרומה וקדשים ואסרו חכמים לעשות כן שמא תבא לםפק ראיה. והכי תניא בהדיא בספרי אחר מעש' תטהר כיון שטבלה טהורה להתעסק בטהרו' אבל אמרו חכמים וכו'.\par אלא מהא דאמרינן בשלהי בא סימן יום א' טמא ויום א' טהור משמשות שמיני ולילו וד' לילות מתוך י"ח ימים והא הכא דמשמשת ליל ט' והתשיעי טמא וסופרת עשירי ואין חוששין לה ג' ימים משום פליטה. גם זו אינה ראיה למה שכתבנו דמהלכת אינה חוששת לשייר. א"נ במקנחת ומכבדת את הבית.\par ובודאי שדברי הרב ר' אברהם ז"ל מכריען בטעמן שכל שאינו גורם אינו סותר ולא מצינו לובן באשה לבעלה. אלא שיש לבעל דין לחלוק ולומר שלא נתנה דבריה לשעורן וכיון דסתרה לטהרת סותרת לבעלה דנקיים מכל ראיה דטומאה בעינן וקראי נמי דייקי דכתיב ואתר תטהר וביום השמיני תקח לה וכו'. ולפי מדה זו יש שטהורה בשביעי לביתה ובח' (עראי) [היא] סופרת ודברי הרמב"ם ז"ל מטין כן שהפולטת סותרת יומא לכל דבר אבל רבינו הגדול והגאונים ז"ל לא תששו לכתוב הדבר ובעל נפש יחוש לעצמו. 
\textbf{יולדת שירדה לטבול מטומאה לטהרה ונעקר ממנה דם.} פי' רש"י ז"ל לבית החיצון. וק"ל דא"כ היכי אקשי אמאי טומאה בלועה היא דהא אמר רבא לקמן דטומאת בית הסתרים הוה ומטמא במשא שהרי דם נדה מטמא במגע ובמשא.\par ואיכא לפרושי דנעקר לאו לבית החיצון משמע אלא רגישה בעלמא שנעקר מן המקור אע"פ שהוא בין השניים ולפנים טמאה ולכך אקשינן טומאה בלועה הוא דע"כ לא אמר רבא בית הסתרים הוי אלא בבית החיצון דהיינו בין השינים לר' יוחנן אבל לפנים בלוע הוי.\par וא"ת כי מוקמינן לדר' זירא בשיצא לחוץ נמי אמאי קשיא לן אלא יולדת אי בימי נדה נדה אי בימי זיבה זיבה נתריץ נמי ברייתא כגון שנעקר דם לבית החיצון ומטמ' התם במשא.\par לאו מילתא היא דהאמרינן דאפילו נעקר קצת מטמא לכשיוצא ולא מהני ליה טבילה ואי לאשמועינן דמעכשיו נמי היא טמאה בבית החיצון כמו שיצא הוה ליה למיתני עקירה ממקומו טומאה בבית החיצון ולא קתני ברייתא הכי אלא אי ברייתא כדר' זירא משמע מינה דלא הויא עקירה דלא להני לה טבילה עד דהוי במקום טומאה דהיינו בבית החיצון. אי נמי ניחא לן לתרוצה אליבא דכ"ע דלא תיקשי לן הניחא לרבא אלא לאביי מאי איכא למימר. ומיהו כי מתרצינן מעיקרא ברייתא דקתני כולן מטמאו' בפני' כבחוץ כי הא דר' זירא מצינן למידק עלה והא דומיא דנדה וזבה קתני והתם בבית החיצון והכא אפילו בפנים אלא אעיקר מילתא דייקינן למיקם אפירושה א"נ דהוה ליה למימר מידי איריא לטמויי בפנים כבחוץ הן שוין. אבל בשיעור מקומן הא כדאיתא והא כדאיתא.\par וי"מ פירכא אליבא דאביי ופירוקה נמי לאביי דאמרינן עשאוה כנבלת עוף טהור שמטמאה בגדים בבית הבליעה ומדמי לה אלמא התם נמי טומא' בלועה הוי דאי התם בית הסתרים הוי מאי דומיא הכא בבית הסתרים ודאי מטמיא במשא ובטומא' בלועה לא מטמיא כלל דלא דמיא נמי לנבלת עוף טהור. וזה הדרך ראיתי בתוספות.\par ולשון שלי הראשון מחוור ממנו לפי שנבלת עוף טהור מטמא בכל מקום מבית הבליעה ואפילו פנים דהוי טומאה בלועה וכי פליגי לקמי בתחלת בית הבליעה דלאביי ליכא מקום דתטמא בית הבליעה בנבלת בהמה במשא ועוף במגע ולרבא [איכא] היכא נמי דמטמא עוף בבית הבליעה מטמאה נמי בהמה משום משא הא לעוף אין לך מקום בושט ואפילו בלועה דלא ליטמיה ביה דלא יאכל לטמאה אמר רחמנ' כל דאכיל מטמא וברייתא דמסתייע אביי מינ' משום דמשמע דלעולם אין לבהמה טומאה בבית הבליעה ולא משמע להו לדחוקה בסוף הבליעה בלחוד דכתיב בה ולא באחרת. 
\textbf{עשאוה כנבלת עוף טהור.} ויש לפרש דה"ק חכמים גזרו על טומאה זו מפני שדרכה לצאת ועשאו' כנבל' עוף טהור ולמיסמך גזירה דרבנן אכעין דאורייתא קאמר שאלמלא שמצינו טומאה בלועה מטמאה בדאורייתא לא הוו גוזרין בהו שמטמא. 
\clearpage}

\newsection{דף מג}
\twocol{הא דאמרינן \textbf{כל שכבת זרע שאינו יורה כחץ אינה מטמאה.} ק"ל סריס חמה אינה מטמא בביאה ואין שכבת זרעו מטמא דהא אינו יורה כחץ ומשמע לי שהסריס יש לו הרגשה בין בעקירה בין ביציאה כדאמרינן ביבמות (סה, א) איהי קים לה כיורה כחץ איהו לא קים לי' כיורה כחץ. ואם אינו מרגי' הא קים ליה דלא מרגיש.\par אלא הכי קתני כל שאין בו הרגשה בעקירה וביציאה כיורה כחץ אינה מטמאה הא הרגיש בה ואין בה כח בירידתה כיורה חץ טמאה אע"פ שאינה מתעברת בה דההוא חולי בעלמא הוא דשכבת זרעו דיהה.\par והיינו דקאמרינן מאי איכא בין האי לישנא וכו'. דלאו למעוטי סריס ומי ששהה עם אשתו ואינו יורה בה כחץ אתינן ובי דרשינן נמי שכבת זרע בראויה להזריעה לומר שנעקרה ויוצאה בכח וראויה להזריע דההיא שעתא הוא דמטמיא אע"פ שחוזרת להיות דיהה ואינו יורה כחץ באשה. כל שכבת זרע שאינו יורה כחץ אינה מטמאה. ק"ל סריס חמה אינה מטמא בביאה ואין שכבת זרעו מטמא דהא אינו יורה כחץ ומשמע לי שהסריס יש לו הרגשה בין בעקירה בין ביציאה כדאמרינן ביבמות (סה, א) איהי קים לה כיורה כחץ איהו לא קים לי' כיורה כחץ. ואם אינו מרגי' הא קים ליה דלא מרגיש.\par אלא הכי קתני כל שאין בו הרגשה בעקירה וביציאה כיורה כחץ אינה מטמאה הא הרגיש בה ואין בה כח בירידתה כיורה חץ טמאה אע"פ שאינה מתעברת בה דההוא חולי בעלמא הוא דשכבת זרעו דיהה.\par והיינו דקאמרינן מאי איכא בין האי לישנא וכו'. דלאו למעוטי סריס ומי ששהה עם אשתו ואינו יורה בה כחץ אתינן ובי דרשינן נמי שכבת זרע בראויה להזריעה לומר שנעקרה ויוצאה בכח וראויה להזריע דההיא שעתא הוא דמטמיא אע"פ שחוזרת להיות דיהה ואינו יורה כחץ באשה. 
הא דאמרינן \textbf{את"ל בתר עקירה אזלינן לחומרא.} ק"ל אמאי לא פשטה להא מילתא מהא דאמרן ונעקר ממנה דם בירידה טמאה בשלמא בעייא דרבא ל"ק דאיהו לא משום ספיקא דבתר עקירה אזלינן בלחוד מספקא ליה אלא משום דשמואל דכיון דאינו יורה כחץ בשעת טומאתו אינו מטמא אבל הא דאמרינן גבי ירד וטבל את"ל בתר עקירה אזלינן קשיא.\par ואיכא למימר משום דכיון שטבל בנתיים א"א שלא בטלה הרגשתו ומיהו זבה שנעקרו מימי רגליה דמיא לההיא אלא דהתם מצי נקיט להו והא דקאמרינן ביה את"ל סרכא נקט. 
\clearpage}

\newsection{דף מד}
\twocol{הא ד\textbf{אמר רב ששת הב"ע בכהן שיש לו שתי נשים וכו'.} פי' רש"י ז"ל ודוקא בן יום א' אבל עובר לא משום דאין זכייה לעובר. וכן דעת רבינו אלפסי ז"ל וכתוב בהלכתא בפ' אלמנה לכ"ג (יבמות סז, א).\par וק"ל א"כ למה ליה לאוקומא בכהן שיש לו שתי נשים א' גרושה וכו' לימא שפוסל בעבדי אביו מלאכול בשביל המשפחה ועוד דר' ששת דקא מתרץ לה איהו הוא דאמר בפרק מי שמת (דף קמ"ב) המזכה לעובר קנה וכ"ש בירושה הבאה מאליה ומקשינן עלה התם מסוף דמתני' דקתני נוחל ומנחיל ומתרץ כי הכא משו' דאיהו מאית ברישא.\par אלא ה"פ: דרב ששת לטעמיה דאמר יש זכייה לעובר ומיהו בכהן שיש לו שתי נשים ויש לו בנים משאינ' גרושה כיון דאי עובר שבמעי הגרוש' נקבה היא לא פסלה ואי נפל הוא לא פסול הוו להו זכרים מיעוט' ולמיעוטא לא חיישינן ולאפוקי מדר' יוסי דאמר חוששין למיעוט ואפילו כולן זכרים לא יאכלו בשביל חלקו של עובר קמ"ל בן יום אחד אין עובר לא ואפילו כלו חדשיו נמי כיון שלא נולד י"ל שמא נפל הוא שאינו ראוי לברית נשמה.\par והך פלוגתא מפרש בגמרא ביבמות פרק אלמנה לכהן גדול וכבר פירשתי שם בפ' מי שמת הארכתי בכל הסוגיא הזו והרוצה לסמוך על העיקר ע"ש. 
\textbf{מ"ט דאיהו מיית ברישא.} פיר' ודוקא מתה דקאמרינן בערכין פ"ק (דף ז) אשה היוצאת ליהרג מכין אותה כנגד בית הריון שלה כדי שימות הולד תחלה והוינן בה למימרא דאיהי מתה ברישא והא קי"ל דאיהו מאית ברישא דתנן בן יום א' וכו'. ומפרקינן ה"מ מתה דאגב דולד זוטר חיותיה עולה ביה טיפה דמלאך המות ותתיך לה לסימנים אבל נהרגה איהי מתה ברישא ואע"ג דהאי תירוצא לאוקומי יש זכייה לעובר אתמר בפ' מי שמת מיהו קושטא הוא ולהכי הוינן מינה ומפרקי לה התם. א"נ למ"ד הכי מקשי מברייתא דקתני מכין אותה כנגד בית הריון ולאו למימרא דהכי הוא בודאי. 
והא דתנן \textbf{וההורגו חייב.} ודוקא בן יום א' אבל עובר לא דלא קרינא ביה נפש אדם וה"נ אמרינן בסנהדרין (עב, ב) האשה שמקשה לילד מביאין סכין ומחתכין אותו אבר אבר יצא ראשו אין נוגעין בו שאין דוחין נפש מפני נפש אלמא מעיקרא ליכא משום הצלת נפש וקרא נמי כתיב דמשלם דמי ולדות.\par ואיכא דקשיא ליה מההיא דגרסינן בערכין (ז, א) האשה שהיא יושבת על המשבר ומתה בשבת מביאין סכין וקורעין אותה ומוציאין ממנה הולד ואמאי מחללין שבת כיון שאינו קרוי' נפש.\par וליכא למימר התם ביושבת על המשבר דוקא משום דכיון דעקר גופא אחרינא הוא כדאיתמר התם בערכין במקשה לילד לא בעינן יושבת על המשבר ג) ועוד דהכא בן יום א' תנן וקרא דגבי דמי ולדות אפילו ביושבת על המשבר היא ולא אמרינן התם דכילוד הוא אלא גופא אחרינא הוא קאמרינן לומר שממתינן לה עד שתלד ואח"כ ממיתין אותה ולא מיתרבי מגם שניהם דאפילו קודם שתש' על המשבר כלל אי לאו קרא דגם לא הוה קטלינן לולד כדמפור' התם אבל לענין לידה דבר ברור הוא שאינו בכלל נפש אדם עד שנולד כדאמרינן.\par ולאו קושיא היא התם אמרה תורה חלל עליו שבת אחת כדי שיקיים שבתות הרבה והאי דאמרי' במס' שבת (קנא, ב) תינוק בן יום א' מחללין עליו את השבת לאו לאפוקי עובר אלא גוזמא היא כדי לומר דוד מלך ישראל מת אין מחללין עליו. 
\clearpage}

\newsection{דף מה}
\twocol{\textbf{למאי נפקא מינה כגון שבעל בתוך ג' ומצא דם לאחר ג' ולא מצא דם.} פי' שאלו בעל בתוך ג' ולא מצא דם ולאחר ג' נמי לא מצא ודאי ליכא ספיקא דאחר בא עליה דהא מכיון שלא בא דם בתוך ג' ודאי לא איתצדו הילכך לאחר ג' בשלא מצא היינו טענת בתולים דאיכא למיחש אחר בא עליה ובאשת כהן כדאיתא בכתובת פ"ק (ט, א).\par והא דאמרינן ללישנא בתרא כגון שבעל בתוך ג' ומצא דם ולאחר ג' ומצא דם לאו דוקא דאפילו לא בעל לאחר שלשה נמי היינו בעיין אלא סירכא דלישנא קמא נקט כשבעל וחזר ובעל כשמצא דם שאלו לא מצא הא אמרן דאחר בא עליה ואסורה. 
מדתנן \textbf{בן ט' שנים ויום א' שבא על יבמתו קנאה.} משמע קנאה לגמרי ליורשה וליטמא לה אלא שאינו נותן גט עד שיגדיל וכן נמי מדהוינן בה ולכשיגדיל בגט סגי לה והתניא עשו ביאת בן ט' כמאמר משמע דמדאורייתא קנאה לגמרי ובגט סגי לה אלא שהם גרעו כת ביאתו ועשאוה כמאמר וכן בדין שהרי ביאתו ביאה לכל דבר ואע"פ שאין לו דעת הא רבי רחמנא ביבמה ביאת שוגג כדמזיד וכבר פירשתי בפ' האשה רבה (יבמות צו, ב). 
\clearpage}

\newsection{דף מו}
\twocol{\textbf{[הא תוך זמן כלפני זמן תיובתא].} הא דאסיקנא תיובתא למ"ד תוך זמן כלאחר זמן ואתי רב נחמן לאוקמא בתר הכי כתנאי ולא קמא ק"ל דהא לקמן בפר' בא סימן אשכחן בה פלוגתא דר' יהודה ור"ש דר"י סבר תוך זמן כלאחר זמן ור"ש סבר כלפני זמן והיכי אסיקנא בתיובתא ולא אוקמינהו כתנאי.\par אלא שמצינו כיוצא בה בתלמוד והרי בזו המסכתא עצמה בפ' ג' (דף כ"ה ע"א) אלמא בעור תליא מילתא ל"ש עכור ול"ש צלול ואסיקנא תיובתא והדר אמר ר' נחמן מחלוק' בעכו' והוינן עלה ממתני' ולא אשגח אתיובתא קמייתא ובפ' אין צדין (דף לה) מדברי רבינו נלמד חיה שקנה בפרדס אינה צריבה זימון ואסיקנא בתיובתא והדר מתרצין ברייתא אחריתי כאן בסמוכה לעיר כאן בשאינו סמוכה ומצי לפרוקי תיובתין וכולהו שקלא דבתראי בתר מסקנא הוא והא נמי דכוותייהו. 
\textbf{וכי קאמר רבא חזקה למיאון.} אי קשי' למיאון למה לי חזקה בחששא בעלמא סגי והוה ליה למימר קטנה שהגיע לכלל שנותי' אינה ממאנת שמא הביאה שתי שערות. איכא למימר כי קאמר רבא חזקה לומר שאין ב"ד מטריחין עצמן לבדוק שלא תמאן ואם חששו היינו בודקין. א"נ אע"ג דאמרינן כי קאמר רבא חזקה למיאון לאו דחזקה אצטריכא ליה להכי אלא לומר דלא ממאנה וחזקה נמי היא ומהניא חזקה לנשים בודקת אותן כדלקמן בפ' בא סימן.\par והראשונים שאלו א"כ הא דאמרינן התם בודקין לה לחליצה ולמיאונין היכי משכחת לה ורבי' הגדול השיב נפקא מינה להגיעה לכלל שנותיה וקדש בתוך זמן ולא בעל אתר זמן דהוה דרבנן.\par ואלמלא שזה דבר ברור יכולין אנו לפטור עצמינו משאלה זו במה שאמרו מקצת המחברים בודקין למיאונין היינו כדאיתמר עלה לאפוקי מדר' יהודה דאמר עד שירבה שחור על הלבן קמ"ל בודקין ומכי אתיא שערות לא ממאנה ולאו למימרא דבדיקה צריך אלא בדיקה זו היינו שערות לומר דמשהביאה אותן בין בבדיקה בין בחזקה אינה ממאנת וקטנה שלא נודע אם הגיעה לכלל שנותיה והביאה סימנין לא מצינו בגמ' דינה מפורש ויש שכוללין אף בזו בכלל בודקין למיאונין ואם הביאה סימנין אינ' ממאנת ואע"פ שלא בעל אלא קוד' זמן ולא תלינן בשומא לקולא וה"נ לשאר הדברים מטילין אותה כחומרא כדין הספקות. 
\textbf{והא אין הבעל מפר בקודמין כדר' פנחס וכו'.} פי' והא נמי על דעת כן נדרה שאם הקפי' הבעל לא יחול נדרה ומיהו משלא בעל משהגדילה אינו מיפר שכיון שלא קנה קנין גמור אינו מפר שמתחלה מתלא תלי נדרה אם נתקיימו קידושיה יפר נדרה שעל דעת כן נדרה ואם לא נתקיימו הקדושין אף הוא אינו מפר שלא נדרה על דעתו שמא סבורה היא לצאת ומיהו האי תירוצא לא אתיא כרב הונא דאמר הקדיש ואכל לוקה שא"כ היאך היא אוכלת תחלה בהפרתו הרי עדיין קידושי' תלויין והגדר גדר גמור.\par ויש שגורסין אלא כדר' פנחסולא וכו'. כלומר לעולם בשלא בעל ולא מפר אלא שעה ראשונה ואעפ"כ אתיא הפרה דידיה ומבטל נדרה דאורייתא שעל דעתו נדרה מכיון שהוא חיי' במזונתיה ועומדת תחתיו ומשמשתו והאי פירושא עיקר ואפילו למאן דלא גריס אלא ה"ג מתפרש דבכמה דוכתא בתלמודא דהדר ביה מתירוציה קמאי ולא אתמר בהו "אלא". 
\clearpage}

\newsection{דף מז}
\twocol{הא דתנן \textbf{איזהו סימניה ר' יוסי הגלילי אומ' משיעלה קמט תחת הדד וכו'.} פי' רש"י ז"ל דאצמל קאי. וא"כ צריכין אנו לומר דשיעורא דמשתקיף העטרה ושיעלה הקמט תחת הדד חד שיעורא הוא או לומר דתרי תנאי אליבא דר' יוסי ועדיין אין הדברים נראין שנשנו שיעורין במשנה ואחרים בברייתא ולא הוזכרו של זה בזה כלל ועוד בוחל זה שאמרו בידוע שהביא שתי שערות והלא לא פי' אותו כלל לא במשנה ולא בברייתא וכן בגמ' לא הזכירו בפרק מהן. אלא איזה סימניה אבוחל קאי. ופי' מתני' סימן נערות דעליה קאי ברישא דמתני' ובוגרת לא קתני משום דממילא ידוע שאין בין נערות לבגרות אלא ו' חדשים בלבד ומאי שלא פי' במתניתין פי' בברייתא אלו הן סימניה בגרות וכו'. ועלה דהא מתני' קתני באידך פירקין בא סימן התחתון עד שלא בא העליון כלומר העליון השנוי שאלו לדברי רש"י סימן עליון אצמל דסליק מיניה משמע וליתא אלא אבוחל. 
הא דתניא \textbf{שנה האמורה בקדשים בבתי ערי חומה וכו'.} כולן מעת לעת מיום ליום קאמר וקראי דמיום אל יום נסיב בגמרא והני כולהו דמיום אל יום שוין הן אבל מעת לעת ממש בעינן בקדשים כדאמרינן בפ' שני דזבחים (דף כה ע"ב) זאת אומרת שעות פוסלות בקדשים. וכן בבתי ערי חומה בעינן מעת לעת ממש כדאמרי' בפ' בתרא דערכין. אבל שבבן ושבבת דפירקין דיוצא דופן לא בעינן מעת לעת כדאיתמר לעיל (דף מד ע"ב) ערב ראש השנה דג' איכא בנייהו וכן לענין עבד עברי לא שמענו. כך מפורש בתוספות. 
\clearpage}

\newsection{דף מח}
\twocol{הא דתנן \textbf{ר' מאיר אומר לא חולצת ולא מתיבמת.} ר"מ לטעמיה דאמר (לעיל לב, א) קטן וקטנה לא חולצין ולא מיבמין. והא דקתני סיפא ותכ"א או חולצת או מתיבמת מפני שאמרו לאו דוקא מתיבמת דהא לרבנן קטנה נמי מתייבמת אלא איידי דקתני בדר' מאיר חליצה ויבום וקתני להו בכל דוכתא נקט נמי הכא חולצת או מתייבמת. }

\newchap{פרק \hebrewnumeral{6} בא סימן}
\twocol{גמרא: הא דתניא \textbf{כל הנבדקות נבדקות על פי נשים.} היה נראה דלת"ק בין להקל בין להחמיר נשים נאמנות בין לפני הפרק בין תוכו בין לאחריו וטעמא דמילתא משום דבמילתא דאיתא קמן ומצינן לגלויה מהימנן ודמיא הא מילתא לההיא דאמרינן בפ' החולץ (דף לט ע"ב) ואשתמודענוהו לקמן דאחוה דמיתנא דמן אבוה הוא וקי"ל אפילו בקרוב ואפילו אשה דגלויי מילתא הוא.\par והוסיף רבינו הגדול ז"ל בפי' דמילתא ואמר טעמא משום דלאו אמילתא דאסורא קא מסהדי ולא אממונא קא מסהדי אלא מילתא הוא דמגלו דהדין הוא גבר פלן והא ניהי איתתיה וכו' כדכתיבא בהלכות והא נמי לההיא דמיא ועדיפא מינה משום דמילתא קמן היא לגלוייה הילכך נשים מהימני בין להקל בין להחמיר. כך נראה פי' דבר זה.\par והלכה כת"ק משום דהוא סתם ברייתא ור"י ור"ש יחידאי נינהו ועוד מעשה רב וכן היה ר' אליעזר מוסר לאשתו ור' ישמעאל מוסר לאמו. אבל רבינו הגדול ז"'ל פי' טעמא {\small [ברי"ף לפנינו ליתא שם טעמא דת"ק]} דת"ק משום דקסבר לפני פרק בדקן נשים דאי משתכחן לאחר הפרק שומא נינהו וכדר' יהודה תוך הפרק משום דכלפני הפרק דמי וכדר' שמעון ולאחר הפרק משום דאיכא חזקה דרבא וסמכינן אנשי' וכדר' יהודה ופםק הלכה כן כדאיתא בהלכות בפ' ב"ש ביבמות.\par ופי' לאחר הפרק בכל השמועה היינו זמן הנעורי'. ותוכו היינו שנת י"ב שהוא עונות הנדרים וכך פי' ר"ש ז"ל וכן היא קבל' רבינו וכל הגאוני' ז"ל ולפי שהזמן הזה שנוי במשנ' שנו בה סתם הפרק כלומ' הפר' שהזכירו א"נ ב"מ ומי שחולק בזה אין שומעין לו. 
\textbf{בשלמא לפני הפרק בעיא בדיק' דאי משתכחי לאחר הפרק שומא נינהו.} שאלמלא שבדקו הנשים בתוך הזמן היינו אומרים לאחר זמן הביאו ולא קודם לכן שאורח בזמנו בא. וה"ה אפי' לתוך זמן שאין חוששין שמא לפני זמן הביא' אותן למ"ד כלאחר זמן דמי וחולצת היא ועכשיו הנשים נאמנות וקטנה היא שלא תחלוץ ואפילו לומר קטנה היא שתמאן אינו נאמנות מ"ט כיון דקידש בתוך זמן ובעל לאחר זמן ה"ל ספיקא דאורייתא אפילו היו שם ק' עדים ששערות הללו שומא הן חוששין שמא הביאה שערות לאתר זמן ונשרו ולעולם אינה ממאנת.\par ואי קשיא היכי ניחא לן השתא בדיקה דלפני הפרק משום דאי משתכחי לאחר הפרק שומא נינהו והא למאי דקס"ד השתא דתזקה דרבא בין למיאון בין לחליצה הוא לאחר הפרק אפילו תאמר שאלו שומא הן מ"מ גדולה היא דחזקה הביאה שתי שערות.\par לאו קושיא היא ואנן מחזא חזינא בהדיא בבריית דאית ליה בדיק' בלאחר הפרק ולפום הכי קאמרינן בשלמא בדיקה דלפני הפרק לפום מאי דקאמרת בברייתא מהניא ודאי דלא עבדין עובדא בהנך שערות אלא בדיקה דלאחר הפרק גופיה קשיא למה לי ומשום דלא בעינן לאקשויי אדיוקא מאי דקתני בהדיא קאמרינן הכי והשתא לא נחתינן למידק בנשרו כלום דקתני וסתמא פרכינן.\par ומתרצינן לכ"ע כדמתרץ במסקנא לעיל באידך פירקין ולפרושא לברייתא בעלמא אתינן השתא דאי דייקי בנשירה הוה יכול למימר בדיקה דלאחר הפרק להחזיקה בקטנה כשלא נמצאו בה שערות אלא פירש ברייתא כמסקנא דלעיל הכא ולישנא דקאמר בעיא בדיקה לאו דוקא דאנן לא צרכינן למיבדק קודם זמן שומא הן. אלא מהניא בדיקה דנשים להכי. וכן ברייתא דקתני נשים בודקות לאו דוקא אלא לומר שהן נאמנות אם בדקו וה"נ משמע בכל מקום בתלמוד שאין חוששין לשערות שנמצא לאחר זמן שמא הביאו אותן קודם זמן ושומא הן כדאמרי' בפ' מי שמת (דף קנד) דמעשה דבני ברק בתינוקות וכו' ובאו ושאלי לר"ע מהו שיבדקו וכו'. 
\textbf{וסיפא דקתני ונאמנת אשה להחמיר.} אוקימנא אב"א ר' יהודה ואתוך הפרק. וק"ל בשלמא נאמנת לומר גדולה היא שלא תמאן ואינה נאמנת לומר גדולה היא שתחלוץ ניחא אלא קטנה שלא תחלוץ פשיטא דנאמנות דאפילו שתקא א"נ אמרה גדולה היא אינה חולצת וקטנה היא שתמאן אמאי אינה נאמנת ואפילו נאמנת אמאי צריכה והא אמרת צריכה לומר גדולה היא שלא תמאן.\par ואיכא למימר כולה ברייתא נאמנת ואינה נאמנ' במקום שהוצרכנו לעדותה היא והכי קתני נאמנת לומר גדולה היא שלא תמאן במקום שאנו צריכין לעדות (גדול) שלה שאלמלא עדות אשה זו ממאנת היא שבחזקת קטנה עומדת ואפילו בדקו עדים עכשיו ולא ראו בה שערו' נאמנות אשה זו לומר הביאה אותן ואינה ממאנת שמא נשרו.\par וכן נאמנ' לומר קטנה היא שלא תחלוץ במקום שאנו צריכין לעדות קטנותה כגון שבדקנו אותה ומצינו בה שערות אם אמרה אשה לפני זמן הביאתן נאמנת והיינו בדיקה דלפני הפרק ואגב אחרינא נקט להא. א"נ שלא תאמר כשהיא עדיין לפני הפרק נאמנת דעדיין בחזקת קטנה היא אבל לאחר שהביאה שערות והיא בזמנה והוחזקה גדולה בפנינו שמא תאמר אינה נאמנת לומר קטנה הוא שתוך זמן היו בה. וקמ"ל.\par אבל אין אשה נאמנת לומר תוך זמן קטנה היא שתמאן אם הוצרכנו לעדות זו כגו' שנמצאו בה שערות והוא שבעל בתוך זמן דה"ל ספק דאורייתא ואע"ג דהכא ליכא למימר שמא נשרו דהא אכתי תוך זמן זה הוא מיהו כיון שהיא גדולה בפנינו אין האשה נאמנת להקל בשל תורה לומר שומא הן וכן אינה נאמנת לומר גדולה היא שתחלוץ.\par ולהך לישנא דאמרינן ואב"א ר' שמעון ולאחר הפרק ולית ליה חזקה דרבא וה"נ קתני נאמנת לומר גדול' היא שלא תמאן ואפילו אין בה עכשיו שערות וקטנה היא שלא תחלוץ אפילו היו בה אבל אין נאמנת לומר קטנה היא שתמאן כשהיו בה ובעל כדפרישית ולא לומר גדולה היא שתחלוץ.\par ומצינו נוסחא אחרת. "ונאמנת אשה להחמיר אבל לא להקל כיצד גדולה היא שתמאן גדולה היא שתחלוץ". וכן גרסת רבינו הגדול ז"ל בהלכות. ונוסתא ישרה היא ופירושא נאמנת להחמיר לומר גדולה היא לענין מיאון ואינה נאמנת להקל לומר גדולה היא לענין חליצה כלומר מיאון וחליצה היינו להחמיר ולהקל מיאון היינו להחמיר חליצה היינו להקל. 
\clearpage}

\newsection{דף מט}
\twocol{\textbf{מוציא כשר למי חטאת ופסול משום גסטרא.} יש מקשים, ואפילו בלא מכניס ומוציא היאך יהיה כלי חרס כשר למי חטאת והלא שנינו במס' פרה כל מעשיה אינן נעשין אלא בכלי אבנים ובכלי גללים ובכלי אדמה. וי"ל ההיא מעלה בעלמא היא משום שמטמאין היו הכהן השורף אותה להוציא מלבן של צדוקים שהיו אומרים במעורבי שמש היא נעשית אבל מכניס פסול דין תורה הוא.\par וי"מ שעד שלא נתנו אפר במים היו מעשיה בכלים הללו שאין מקבלין טומאה אבל משנתנו אפר לתוך המים שוב אינה מקבלת טומא' כדאמרינן בעלמא א) מי תטא' שנגע בהם שרץ טהורין וכיון שכן בכל כלים היו נותנים ואפילו בשל חרם ולכך פסלו מכניס. 
הא דאקשי' \textbf{תנינא חדא זימנא הכל כשרין לדון דיני ממונו' וכו'.} ק"ל דהא כולהו תננהו כל חדא וחדא בדוכתא כדפריש בגמ' (לקמן נ, א) כל שחייב בפאה חייב במעשרות ממתני' דהתם וכן כל שיש לו ביעור יש לו שביעית והכא ודאי אגב גררא דכיוצא בו קתני להו וא"ת לרב יהודה פרכי' הא לאו מילתא היא דהא רב יהודה פרושי קא מפרש להו דהכא והתם חדא קתני ועוד דהא עיקר מתני' מצרכי' חדא לגר וחדא לממזר.\par וי"ל למימרא דרב יהודה ודאי ק"ל למה לי לפרושי תרתי זימנא ומתרצי' דלא איתמר ממזר אלא גר איתמר כך פי' בתוספת והביאו דומה לה משבת פ' ח"ר עקיבא דקאמר וצריכא על מימרא דרב יהודה אמר שמואל דאמר הלכה כר' עקיבא דכל מלאכה שאפשר לעשותה מע"ש אינו דוחה את השבת וכו' ואינו מחוור לי שאם אין המשניות מיותרת מנין לו לרב יהודה לפרושי חדא לגר וחדא לממזר. וי"ל דרב יהודה סברא דלפשיה קאמר דכולהו פסולין שהן ישראל כשרין לדיני ממונות ולא מיפסלי יוחסין אנא לדיני נפשות.\par וה"ג וכן בנוסח' אי' בסנהדרין (לו, ב) וצריכא דכי אשמועינן ממזר משום דאתי מטפה כשרה ואי אשמועינן גר משו' דראוין לבא בקהל וכשירין לדיני ממונו' מצרכינן דרב יהודה אהכל כשרין דרישא קאי. וקאמר לאתויי ממזר דכשר אבל אין הכל כשרין לדון דיני נפשות אלא כהנים וכו' פשיטא לנר ולממזר ואפילו חלל כולן פסולין הן וכדקתני סיפח התם ואין הכל כשרין לדון דיני נפשות אלא כהנים ולוים וישראלי' המשיאין לכהונה וחלל אינו מן המשיאין לכהונה.\par וי"א דאסיפא קיימינן לומר דממזר וגר פסולין לדון נפשות וכן כתוב כאן בנוסחאות וצריכא דאי אשמועינן גר משום דאתי מטפה פסולה וכו' ומפרשי' מדקאמרינן לאתויי גר וממזר ש"מ דפסולי כהונה כגון חלל כשרין לדיני נפשות דלאשמועינן דאפילו גר וממזר כשרין לדיני ממונות לא קאמרינן דהא בגמר' דיני נפשות מצרכינן כדאמרינן דאי אשמועינן גר משום דקאתי מטפה פסולה ולהכי פסול לדיני נפשות אבל ממזר דאתי מטפה כשרה אימא לא וכו' וכיון דלא אשמועינן אלא גר וממזר ש"מ דחלל כשר אפילו לדיני נפשות. והא דקתני בסנהדרין ואין הכל כשרין לדון דיני נפשות אלא כהנים לוים וישראלים המשיאין לכהונה הא פריש רב יהודה דלמעוטי ממזר וגר אתי אבל חלל אינו בכלל.\par ואינו נכון כלל דחלל נמי בהדיא ממעיט מהתם ועוד דתנן אין בודקין מן הסנהדרין ולמעלה ומפקינן לה מדכתיב ונשאו אתך בדומין לך. והם דוחין לזו דהתם בממונין סנהדרין קבועים אבל כשר הוא להושיבו בדיני נפשות ולמנות עמהן אע"פ שאינו משיאו לכהונה.\par וכן ב) פירש"י ז"ל הא שאמרו ביבמות (קב, א) שאם היתה אמו מישראל דן ואפילו ישראל [דהוא דיני נפשות] ורבינו יצחק בעל הלכות ז"ל פי' לזו [דמתני'] בשאמו מישראל במס' סנהדרין. וכבר פרשתיה שם ביבמות (מה, ב). 
\textbf{נבלת בהמה טהורה בכל מקום ונבלת העוף הטהור והחלב בכרכים אינם צריכים לא מחשבה ולא הכשר.} פי' נבלת בהמה ונבלת העוף אינם צריכים לא הכשר מים ולא הכשר שרץ מפני שסופן לטמא טומאה חמורה. והחלב אינו צריך הכשר מים שכבר הוכשר בדם שחיטה אבל הכשר שרץ צריך לצדדין קתני ואין אתה יכול לאומר' בחלב נבלה דא"כ צריך הוא הכשר מים ושרץ שהוא אין סופו לטמא טומאה חמורה דכתיב וחלב נבלה וחלב טרפה יעשה לכל מלאכה וא"א נמי בחלב של טמאה שהיא צריך מחשבה שהרי נבלת טמאה בכל מקום צריכין מחשבה ואין חלבה חלוק מבשרה אלא בחלב של שחיטה הוא ואינו צריך הכשר מים עכשו מפני שכבר הוכשר בדם שחיטה אבל הכשר שרץ צריך ולצדדין קתני הכשר כדפרישי'.\par ובסיפא גרס' נבלת בהמה טמאה בכל מקום ונבלת עוף טהור בכפרים צריכה מחשבה ואין צריכין הכשר ולא גרם בה חלב כיון דחלב בכפרים צריך מחשבה הכשר נמי צריך שלא הוכשר בשחיטה מפני שקדם הכשר למחשב' והבשר קודם למחשבה לא הוי הבשר כדאיתא בהעור והרוטב. ובנוסח המשניות נמי אין בהם חלב בסיפא. 
\textbf{והאי שבת מדחייבא בפאה מחייבה במעשר.} איכא למידק למה ליה לאתויי מכללא משנה שלימה היא שהשבת חייבת במעשרות דתנן במס' מעשרות ומייתינן לה בגמ' בפ"ק דע"א (ז, ב) ר"א אומר השבת מתעשרת זרע וירק וזירין וחכ"א אינו מתעשר ירק וזרע אלא השתלים והגרגיר בלבד הא הכל מודים שהוא מתעשר.\par ואיכא למימר אי מהתם ה"א דה"ק אינו מתעשר זרע וירק אלא השתלים והגרגיר אבל השבת אע"פ שהזרע והירק ממנו שוין אינו בכלל לפי שהוא פטור לגמרי. א"ג איידי דבעי מכללא ומדאיחייבה במעשר מטמא טומאת אוכלין דייק נמי מדמחייבא בפאה מחייב' במעשר וה"ה ודאי דמצי למיפשט תרווייהו בהדיא מההיא דמייתי בשלהי שמעתין השבת משנתנה טעם בקדרה וכ' אלא יגדיל תורה ויאדיר מתרץ לה ומסייעא ליה מההיא והכי אורחא דתלמוד' בכמה דוכתי'. 
 הא דאמרי' \textbf{לבני מערבא דמברכין בתר דסליקו תפילייהו לשמור חקיו.} פי' ר"ת ז"ל בספר הישר שלו שלא אמרו אלא בתפילין אבל בציצית ושאר מצות אין מברכין לאחר עשייתן.\par והביא ראיה ממה שאמרו בירושלמי בפ' היה קורא בתורה כיצד הוא מברך עליהן ר' זירקן בשם ר' יעקב בר אידי כשהוא נותן של יד מהו אומר בא"י אמ"ה על מצות תפילין וכשהוא נותן לראש מהו אומר אקב"ו על הנחת תפילין. וכשהוא חולצן מהו אומר ברוך וכו' לשמור חקיו ואתיא כמ"ד בחוקת תפילין הכתוב מדבר ברם כמ"ד בחוקת הפסח הכתוב מדבר לא כר"א [{\small לפנינו שם } לא בדא {\small ואם הגירסא נכונה } כר"א קאי על למטה ע"ש] והטעם לזה מפני שמניח תפילין לאחר שקיעת התמה עובר בעשה הילכך מברך בשעת סילוקן בלילה שהוא מקיים עשה ואין לך כן בכל המצוות, כך פי' חכמי הצרפתים בשמו ז"ל.\par ועדיין אינו מחוור, דא"כ הא דאמרינן בשמעתין לאתויי מצות ומקשי ולבני מערבא דמברכין בתר דמסלקי תפילייהו מאי איכא למימר מאי קושי' מתני' לאתויי כל שאר המצות. ועוד יש נסחאות שכחוב בהן ולבני מערבא דמברכי אמצות וכו'.\par אלא נראה לבני מערבא ה"ה לכל מצות שטעונו' ברכה לאחריהן וז"ש בירושלמי אתיא כמ"ד בחוקי התפילין לא הקפידו אלא על הלשון דלשמור חקיו אבל שאר כל המצות אין מברכין אלא לשמור מצותיו ובודאי נראה לומר שאין בני מערבא מברכין אלא כשהן מסלקין אותן בזמן ערבית ולא משום עשה שבהן אלא משום שכבר נגמרה מצותן דקסברי לילה לאו זמן תפילין הוא וא"כ סלקו אותן בע"ש ובערבי י"ט לד"ה מברכין היו אבל אם היו מסלקין ביום היאך יברך הלא מצוה להניחן ולא לסלקן. ולפיכך אמרו בירושלמי דאתיא כמ"ד בחוקת תפילין הכתוב מדבר וכתיב מימים ימימה ולא לילות וכך סמכו שם בירושלמי ר' אבהו בשם ר' אלעזר הנותן תפילין בלילה עובר\par בעשה מה טעם ושמרת וכו'. אבל נאמר לפי"ז הענין ולפ"ז הפי' שאין מברכין על כל מצוה שאין סילוקה גמר עשייתה כגון פושט ציצית ביום והיוצא מן הסוכה אבל בלילה מברכין על ציצית וכן לאחר שופר ולולב וכל כיוצא בהן שעשייתן גמר מלאכתן מברכין וזה שלא העמידו משנתינו דיש טעון במצות כיוצא באלו שאינן טעונות ברכה מפני שאין לשין לאחריו אלא לאחר שנגמר המעשה.\par וזה הלשון נכון הוא שאין הדין נותן לברך לאחרי' במצוה שעדיין הוא חייב בה והוא מסלקה ממנו שא"כ מצינו חוטא ומברך ואין לך כן אלא בקורא בתורה ובצבור מפני שהוא מצוה לגמור כדי שיהיו ג' או ז' קוראים כתקנת חכמים. אבל בגמר מצוה בכל מצוה נגמרת מברכין היו ודמיא להו להלל ומגלה ותורה בצבור וראינו לרבינו האי גאון ז"ל שכתב בהא דבני מערבא לא נהגינן הכי במתיבתא ומיהו אי בעי אינש למיעבד כבני מערבא שפיר דמי.\par ולשון הירושלמי שכתבנו נראה שמכריע כדברי בעל הלכות ז"ל שהצריך לברך א' של יד וא' של ראש אף על פי שלא שח. וכן החזירו שם הענין הזה בפרק הרואה ואמרו העושה תפילין לעצמו אומר בא"י אמ"ה לעשות תפילין לשמו כשהוא לובשן אומר בא"י אמ"ה על מצות תפילין וכשהוא מניחן אומר אקב"ו על הנחת תפילין בכל מקום מזכירין כן אע"פ שלא שח ולא כדברי רבי' הגדול ז"ל שפירש לא שח מברך א' בלבד על שתיהן.\par אלא שיש לנו פתחון פה לומר דגמרא ירושלמי ס"ל כדקס"ד מעיקרא בגמרא דילן אבל במסקנא אסיקו אביי ורבא לא שח מברך א' ואנן כמסקנא דגמ' דילן עבדינן או שענין הירושלמי במניח א' מהן ולא במניח שתיהן. 
\clearpage}

\newsection{דף נב}
\twocol{הא דאמרינן \textbf{ומודה ר' יהודה שאם נבעלה לאחר שהביאה שתי שערות שוב אינה יכולה למאן.} משום דודאי מודה ר' יהודה שהיא גדולה משהביאה שתי שערות והילכך אמרינן אדם יודע שאין קידושי קטנה כלום וגמר ובעל לשם קדושין ומאן דבעי למעבד עובדא כר' יהודה ואע"ג דנבעלה ליכא לפרושי משום דקסבר לר' יהודה קטנה היא ואין קידושין תופסין בה עד שירבה השחור דהא קתני בהדיא תינוקת שהביאה שתי שערות חייבת בכל מצות האמורות בתורה ולא פליג ר' יהודה. וכן בתינוק לא מצינו לר' יהודה מחלוקת בגדלות שלו אלא הכל מודים דשתי שערות גומרות בו ועוד דא"כ לר' יוסי דאמר במיאון עד שתקיף העטרה והוא זמן גדלות לדבריו כדאית' בפירקין דיוצא דופן א"כ עקרת זמן נערות מכל התורה כולה.\par אלא שהכל מודים דזמן גדלות היינו שתי שערות וטעמיה דרבי יהודה משום דיהיב ליה זמן מעתה משהיא בת דעת דהא קידושיה דרבנן הן דלא ס"ל כמ"ד כי גדלי גדלה קדושי בהדה ומשרבה השחור אע"פ שמדאורייתא אינה מקודשת תקינו לה רבנן נשואין ומאן דקסבר אפילו נבעלה סבר לא אמרינן אדם יודע שאין קדושי קטנה כלום אלא אמרינן לשום קדושין ראשונים בעל. והילכך אפילו גדולה ממש מן הדין אינה מקודשת אלא שתקנו חכמים זמן למיאוניה עד שירבה השחור תוך הזמן הזה אפי' בעל ממאנת לאחר הזמן הזה אפילו לא בעל אינה ממאנת. וזה הלשון נכון ועיקר הוא. 
והא ד\textbf{אמר ר' ישמעאל ויש לך אחרת שאפילו לא נתפסה מותרת ואיזו זו שקידושי' קידושי טעות.} פירש רש"י ז"ל כגון ע"מ שאני כהן והרי הוא ישראל וכגון קטנה שאין מעשיה כלום והכי ודאי פשטה דשמעתא דאמרי' דבר שאמר אותו צדיק יכשל בו זרעו. וא"כ לר' ישמעאל מצינו חמות ממאנת ושמואל אי סבר ליה כרביה ההוא דאיתמר בכתובות (דף ע"ג) קטנה שלא מיאנה והגדילה ועמדה ונשאת רב אמר אינה צריכה גט משני ושמואל אמר צריכה גט משני ולא מראשון קאמר.\par וההיא דאמרינן בפרק מי שמת (דף קנ"ו ע"א) אמר ר' נחמן אמר שמואל בודקין לקידושין ולגיטין ולחליצ' ולמיאונין ואיתמר עלה למיאוניו לאפוקי מדר' יהודה וכו' דאלמא משהביאה שתי שערות אינה ממאנת ההיא דלא כר' ישמעאל. אי משום דהא דידיה והא דרביה. אי משום דאמוראי נינהו אליבא דשמואל.\par וי"מ דוקא קידושי טעות אבל בממאנת בקדושי קטנות לא א"ר ישמעאל ופי' ממאנת והולכת לה לומר שאם לא רצתה בבעל הולכת לה ולא שתהא צריכה גט מיאון אלא מעשה כשהיה בבתו שנכנסה למאן היו סבורין לומר כשם שלר' ישמעאל ממאנת בקדושי תנאי ולא אמרינן אין אדם עושה בעילתו זנות ולשום קדושין בעל כך בקטנה שהגדילה ונמנו וגמרו שאפילו לר' ישמעאל בקדושי קטנות אינה ממאנת אלא עד שתביא שתי שערות ומיהו דרבנן הוא מפני שנראית כגדולה שנתקדשה או מפני שלא מיחת בקידושי דרבנן שעה ראשונה שוב אינה יכולה למחות מדבריהם. וכן משמע בפרק נושאין על האנוסה (יבמות ק, ב) כלשון הזה וכבר פירשתיה ביבמות בפרק ב"ש. 
אע"ג דקיימא לן כרבנן \textbf{עד שיהו שתי שערות במקום אחד.} מיהו שתים על גבי קשרי אצבעותי' של יד ושתים ע"ג קשרי אצבעותיה של רגל גדולה היא דלא פליגי רבנן עליה דר"ש בהאי.\par ותמהני על הרב רמב"ם פאסי ז"ל שכתב ב' שערות אלו צריכים שיהיו במקום הערוה ובשמעתין משמע אפילו על יד ורגל או בגבה. וי"מ גבה וכריסה במקום ערוה וכריסה למעלה עד מקום ערוה וכן שמעתי בשם ר"ת מיהו ביד ורגל סגי. 
 הא דאמרינן \textbf{מ"ד כל כה"ג מביאה קרבן ונאכל קמ"ל.} נראה לי שאין לפרש "קמ"ל" דאינו נאכל אבל מביאה קרבן כמו שכתבו רבים. דהא כתמים דרבנן הם ואפילו בידוע שמגופ' חזאי דבר תורה טהורה ואינן מביאין לא לידי זיבה ולא לידי נדה כדאית' לקמן בפרק הרואה. אלא ע"כ נפרש קמשמע לן דאינה מביאה קרבן. ולא נאכל שני ימים וחלוק דומיא דג' חלוקין.\par ויש לדחוק שכיון שראתה שנים וצריכה שימור בשלישיאף על פי שאין דין הכתם לטמא בתחלה כיון שדבר ברור הוא דמגופה חזאי בדין הוא שתעשה זבה גמור' שהרי לא עלתה לה שימור. ואין זה נכון. 
\textbf{מסמאה עצמה וקדשים למפרע.} פרש"י ז"ל עצמה לטהרות ובודאי שיש בכלל עצמה קדשים דאיהו נמי בכלל טהרות הן אלא משום פלוגתא דרשב"א נקט הכי.\par ואי קשי' לרשב"א הא מצינו כתמה תמור מראיתה דאלו כתמה מקולקלת למנינה ומטמא בועלה ואלו ראיתה אינה מקלקל' מנינה ואינה מטמא את בועלה כדאיתא בפרק קמא.\par ויש לומר דלרשב"א ל"ק ליה אלא שלא נאריך זמן טומאה בכתם למפרע יותר מזמן טומא' דראיה לפי שאינו בדין אבל אם נחמיר במנין דכתם יותר במנין דמעת לעת דראיה לענין קלקול ל"ק ליה לפי שזמן ראיה כיון דשעת ראיה דאוריית' מתחילין ממנו שאין זמן ספק מוציא מידי זמן ודאי אבל בכתם אף שעת מציאה ספק לפיכך לא התחילו למנות ממנו וכיון שאף בזמן דמע' לעת יש קלקול במנינה גבי כתם האריכו לענין מניין מקולקל' משעת הכבוס למניינה. ואע"פ שאינה מטמא טהרות ולא קדשים אלא עד מעת לעת של מציאה. כנ"ל. והא דנקט עצמה משמע לי מפני שמטמאה נמי את בועלה נקט הכי ולא קתני בהדיא טהרו' וקדשים.\par ובתוספ' מפרשי' היינו למניין שלה לומר שהיא מקלקל' למניינה. ורשב"א עצמה אינה מטמאה למפרע ומודה במעל"ע. ואין פירושם נכון לפי כוונתם שהרי בזמן מעל"ע מיהת חמור כתמה מראי' שהרי מודה רשב"א שמקולקל' למעלה בכתמים כדתניא בסמוך ואלו בראיה אינה מקולקל' כלל ' אלא טעמא כדפרישית. 
\textbf{שהוא מתקן הלכותיה לידי זיבה.} פ' רש"י ז"ל לענין זיבה הוא מיקל לדידיה היכא דלא חזאי ביום לא תלינן כתמה בראיתה ומונה ימי נדה מיום ראיתה ואין ימי זיבה מתחילין עד יום ח' לראיתה ולרבי מונה מיום מציאת כתמה ואף להקל ולטבול לילי ז' לכתמה אם פסקה ומיום ח' לכתמה אמרינן יום זוב הוא ונמצא רבי מחמיר לענין זיבה דכי חזיא בח' לכתמה אמרינן יום זיבה הוא וצריכה לשמור יום כנגד יום ולרשב"א סוף נדה הוא ואין צריך שימור ולא נראה דהא רשב"א כיון דלא תלינן כתמה בראיתה מקולקל' היא לכתמה אמרינן.\par ול"א פי' בה שהוא מתקן הלכותיה לידי זיבה שהוא מחמיר וחושש לכתם משום זוב בג' גריסין ועוד אי נמי שאם ראתה שנים והוא צריכה שימור. וכן בכל שלשה רצופים שתראה חוששת לזיבה וצריכה נקיים נמצא שהוא מתקנה ומוציאה מכל ספק זיבה ואני מעותה שאיני מוציאה מידי ספק כלומר נראין ומטין כדברי המחמיר.\par ואף לשון זה אינו עולה דלמה לידי זיבה לכל דבר הוא מחמי' שהרי רבי מטהר' ליום ששי לראי' ולרשב"א ליום שביעי. ועוד ק"ל כיון דקי"ל (נט, א) כתמים דרבנן ובראית כתמה אינה מטמאה היאך רבי מונה לה משעת כתמה והלא ביום ראיתה היא תחלת נדה וממנו ראוי למנו' דבר תורה. וכדאמרינן בפרק קמא (ו, א) ברואה כתם ומקולקל' למנינה ואינה מונה אלא משעת שראתה.\par לפיכך נ"ל שלא תלה רבי אלא כתמה בראיתה אבל ראיתה בכתמה לא, כתמה בראיתה לומר שאינה מטמאה עצמה וקדשים למפרע ואינה מקלקלת למנינה מיום לבישת החלוק אבל מכל מקום עיקר מנין נדה וזיבה מיום ראיה בדין תורה וחוששת נמי ליום. מציאת כתמה כדין דבריהם.\par לפיכך אמרו שהוא מתקן הלכותיה לידי זיבה כלומר שאינה תולה כתם בראיה אלא במקום שאין חילוק ספיר' זיבה ביניהם כך דהיינו אותו יום שמנין ימי נדה וזבה אחד הוא בין לכתם בין לראיה ונמצאו כל הספירו' ראויו' כדין תורה משעת ראיה וכשהוא מעת לעת הוא רואה אינו תולה ונמצא' מקולקל' לכתם ומונה משעת ראיה נמצא כשהוא אומר תולה מתוקנת לגמרי. וכשהוא אומר אינו תולה היא מקולקל' לגמרי.\par אבל רבי אפי' בשעה שהוא תולה כתמה בראיתה הוא מעותה לידי זיבה שהרי אסורה לשמש עד יום ז' לראיה שהוא ח' לכתמה. ואם ראתה בו ביום חוששת לזיבה בודאי נמצא לרבי שאפילו בשעת תקונה כלומר שהוא תולה הוא מעותה שתולה כתמה בראיתה ואינו תולה ראיתה בכתמה ולא השוה מדותיו כנ"ל וסליק שפיר. 
\clearpage}

\newsection{דף נד}
\twocol{הא דאמרינן \textbf{ותשמש נמי בתשעה עשר.} ק"ל והיכי ס"ד דעשירי לא בעי שימור והאמר רבי יוחנן בפרק תינוקו' עשירי כתשיעי ובעי שימור. ועוד רב ששת דמתרץ לה משום גרגרן שביק ר' יוחנן ואמר כריש לקיש והא קיימא לן הלכה כרבי יוחנן בר מהלי תלת ועוד אמאי נטר לה לסיפא והא מרישא ש"מ דיום א' טמא ויום א' טהור ביום העשירי ראתה ויום אחד עשר היא שומר' כנגדו כדקתני שאינה משמשת אלא שמיני ולילו וד' לילות מתוך שמונה עשר יום אלמא רואה בשבעה עשר ושומר' בשמונה עשר.\par וי"ל קסבר מקשה דעשירי יש לו שימור בתוך ימי זיבה דהיינו ביום אחד עשר והיינו רישא אבל זו כיון דיש בה עשירי וי"א וראתה בהן אין ספירה לעשיר' מעתה שאין ספירה אלא בימי זיבה וסוף [י"ח] ימי נדה היא [ולא בעי ספירה] דומיא ספיר' ר'א והיינו נמי דרב ששת. ורב אשי מתרץ שלא בטלה שמירת י"א את של עשירי אלא נהי די"א לא בעי שימור עשירי מיהת בעי לעולם.\par ואי קשיא ותיקשי ברייתא לר"ל דאמר אף לעשירי לא בעי שימור. ויש לומר סיפא מתרץ לה משום גרגרן אםור כר"ש וקסבר דרואה בעשירי ומשמש' לי"א כל שכן דהוי גרגרן ואע"ג דלא תנן.\par ויש מפרשים דלא אמר ר"ל עשירי לא בעי שימור אלא למ"ד הלכות י"א דהיינו ר' אלעזר בן עזריה אבל לר"ע דאמר קראי נינהו עשירי נמי בעי שימור חוץ מי"א דכיון שאין שימור שלו בימי זיבה אין לו שימור וכדאוקמא לפלוותא דר"ל ור"י בפרק בתרא, ואין זה מחוור. 
מדאמרינן \textbf{משמש' רביע ימיה מתוך שמונה ועשרים יום.} דלפי זה פתחה של זו מכ"ח לכ"ח. שמע מינה שאין האשה נעשי' תחלת נדה משנעשי' זבה גדולה עד שתספור נקיים שלה שהרי בשבוע שלישי שהוא טמא כל שבעה אין בו מימי זיבה אלא ארבע ימים הראשונים ואם תאמר בג' האחרונים נעשי' תחלת נדה נמצאת בשבוע חמישי' שהוא טמא זבה גדולה וצריכה שבעה נקיים ואם כן היאך פתחה של זו בתחלת כ"ח והרי כל אותו שבוע ה' בימי זיבה הוא ונעשת בו זבה גדולה ובשבוע ו' משמרת נקיים ובשבעי שהוא טמא ששה ימים שבו ימי זיבה הן. ואינה משמש' בשמיני נמצאת שלא שמשה בכ"ח שניים כלום.\par וכן נמי מדקתני סיפא משמשת חמשה עשר יום מתוך מ"ח ש"מ כה"ג דאי לא תימא הכי הרי שמנה חמישי' ארבעה האחרונים מתתל' ימי נדה נמצאו ימי זיבתן כלין בשבעה של שמוגה ימים השביעיים ואנו אומרים תחל' נדתה של זו שחוזרת חלילה.\par וכן סיפא דקתני וכן למאה וכן לאלף כלומר דמאה טהורין שבעה הראשונים תחלת נדה והשאר כולן ימי זיבה הן ומאה טהורין ז' לספירה וכולן לתשמיש והיינו ימי שמושה כימי זיבתה אלמא כולן ימי זיבה הן שמשנעש' זבה גדולה עד שתספור שבעה נקיים איו ראייתה אלא סתירה לספירתה ואינה מונה מהם ימי נדה.\par וז"ש בפרק בנות כותיים מה ימי נדתה אין ראויין לזיבה ואין ספירת שבעה עולה בהן כלומר לפי שא"א ושם אמרו וכי דנין אפשר משא"א לומר שא"א לספירת זיבה בימי נדה למ"ש וכן פי' שם רש"י ז"ל. 
והא דאקשינן \textbf{הני ארביס' הוו.} מפורש בדברי הר"ר אב"ד ז"ל דהכי מקשה בשמנה ימים הרביעיי' למה תשמש שבעה והלא צריכה היא לשמור יום א, לספיר' עשירי ואחד עשר של ימי זיבה שראתה בהן בשמונה השלישיים וכדאמרן ברישא דהיינו שימור בעו ופריק רב אדא זאת אומרת ימי נדה שאינה רואה בהן עולה לה לימי זיבתה כלומר של זוב קטן. ולפיכך יום א' של ח' רביעיים שהשלימה בהן ימי נדתה עולה לה לספיר' שמיר' של יום עשירי שאמרנו. ואין דברי רש"י ז"ל נוחין בזה.\par אבל דבר שהכל מודים בו שאין ימי נדה מתחילין עד שתספור נקיים.\par ובואו ונצווח על הרמב"ם פאסי ז"ל שכתב בחבורו שהאשה שראתה תחלה מונה שבעה לנידתה וסמוך להן אחד עשר ואח"כ מונה ז' לנדות אעפ"י שאינה רואה בהן ואחריהן אחד עשר ואם ראתה בהן הרי היא זבה וכן כל ימיה ואם קבעה לה וסת תחל' הוס' הוא יום נדו' וממנו מונה שמונה עשר ומונה שבעה לנדותה אף על פי שלא ראתה ואם ראתה אחריהן באחד עשר זבה היא.\par עוד שבש וכתב שאפילו ראתה ט' וי' ואחד עשר ושנים עשר הרי זו זבה ותחלת נדה וכל אלו דברי הבאי שלדבריו לא תמצא לרואה שבעה טמאים ושבעה טהורים שתשמש אלא שבוע שני ולסוף תשעה שבועות משמשת ששה ימים בשבוע העשירי וחמשה ימים בשבוע שנים עשר ופתחה של זו לסוף אחד עשר שבועות ובגמרא אמרו רביע ימיה ולא קיים אלא בתוך כ"ח הא'.\par וכן לדבריו בשמונה ימים טמאים ושמונה טהורים אינה משמשת תמשה עשר יום אלא מתוך שמונה וארבעים ראשונים אבל בשמונה וארבעים שניים אינה משמשת אלא שלושה ימים וכיון שלא אמרו משמשת ארבע עשר יום מתוך שנים ושלשים או משמשת שמונה עשר מתוך ל"ו וכן כיוצא במנינן הללו ש"מ שפתחה של זו מ"ח ומכאן ואילך חוזרת חלילה.\par וכן האשה שראתה עשרה ימים טמאים ועשרה ימים טהורים אין זיבתה ושימושה שוים אלא פעם אחת בלבד לפי דברי הרב ז"ל שהרי כשהיא חוזרת ורואה כן בשניה בשמונה ימים טהורים נשלמו ימי זיבה ראשונה והתחילו ימי נדה נמצא שבעשרה ימים טמאים השניים חמשה ימים מימי זיבה ואין ימי חמישה בטהורים אלא שלשה וכן למאה וכן לאלף למה מנה חכמים שבעה לנדה והשאר לזיבות והלא נעשה , היא נדה אף על פי שלא ספרה לזיבה. ועוד לדבריו מצינו אשה רואה יום אחד מסוף ימי נדה יושבת עליו ששה ימים מימי הזיבה ואין לנדה ספירה אלא בימיה.\par וכן שנויה בכמה מקומות במסכתא זו שהרואה יום מ"א לזכר ופ"א לנקבה הרי היא תחלת נדה ואין מונין לימים שמקודם לכן והטעם לפי שכבר נשלם המניין.\par והרב ז"ל הורה ביולדת שמפסק' ומתחלת למנות מתחל' ראיה שלאחר מלאת ולדבריו צריך הוא להביא ראיה מן התורה לשנוי זה שהוא משנה היולדת משאר נשים שאפילו כשאינן רואות הן מונות ימי נדה וזיבה כאלו הן רואות.\par ועוד דהא בפ' בנות כותיים אמרי' דלכולי עלמא נדה ופתחה מכ"ז מנינן ואם היינו מונין משעת ראיה ראשונה כ"ז בימי זיבה קאי לה.\par ועוד מהא דתנן היתה למודה לראות יום ט"ו ואוקמה שמואל ט"ו לטבילתה שהן כ"ב לראייתה וכו' ואם אתה מונה כל ימי נדת זובם לתחלת ראיה ראשונה שראתה זו כי הדרי אותו כ"ב תליתאי בימי זיבה קיימי והיאך קבעה וסת בכך שאין האשה קובעת וסת בי"א כדאיתה התם בשלהי בנות כותיים ואין הוסת נקבע אלא בשלשה הפלגו' כדבעינן לפרושי קמן וכל שכן לרב הונא בריה דר' יהושע דקשיא דאמר אינה חוששת בתוך אחד עשר וכל זה במס' זו.\par ותמהיני עליו אם העביר עיניו בפתחי נדה במס' ערכין דתנא רבנן טועה שאמרה יום אחד טמא ראיתי פתחה שבעה עשר פירש שאפילו היו תחלת ימי נדות הרי השלימה עליו ששה ועוד י"א אחריהן נמצאת חוזרת לתחלת נדה וכל שכן אם היה בימי זיבה שכבר עברו ימי זיבתה וימים שהיתה ראוייה להיות נדה ואלו לדברי הרב ז"ל א"א דהא איכא למימר שאותו יום בתוך אחד עשר היום וכשעמדה אחריו שבעה עשר נמצא עומדת בימי הזיבה למנין הראוי וכן כל השמועה ומדאמרינן התם נמי חמשה וארבעים ימים טמאים ראיתי וכן כולם אשתמע בהדיא דמשעה שנעשית זבה גדולה אינה נעשית נדה לעולם עד שתספור שבעה נקיים שלה. ואין לי להאריך.\par וכן יש שבושין בחבורי הראשונים בקצתם כגון רב סעדיה שכתב שכל אחד עשר יום שבין נדה לנדה בשלשה ראיית בשלשה ימים נעשית זבה גדולה בין ברצופין בין במפוזרין. וזה טעות מתפרש כאן ובכמה מקומות דרצופין בעינן ולא מפוזרין ועל כיוצא בדברים הללו ידוו כל הימים שהתורה משתכחת מלומדיה ואין אדם מוציא הלכה ברורה במקום אחד. }

\newchap{פרק \hebrewnumeral{7} דם הנדה}
\twocol{\clearpage}

\newsection{דף נה}
\twocol{\textbf{אבן מושמא.} כבר פירש במסכת שבת פרק ר' עקיבא (דף פ"ב ע"ב) בתוספות בשמו של רבינו תם ז"ל שהיא אבן גדולה שמושמת על משכב הזב ומושבו ואדם טהור יושב על האבן ונטמא מדין נישא שהרי משכב הזב נושא אותו וזהו היסטו של זב שלא מצינו לו חבר בכל התורה כולה שכל טומאות המסיטות טהורות חוץ מן הזב וריבה הכתוב אף משכב שלו ומושבו הנושאים תחת האבן כדאיתא בתורת כהנים וכן אם היו בכף מאזנים וכרעו הן מטמאים בהיסט.\par וכן דם נדה שהוא רוצה כאן לטמא תחת אבן מסמא לומר שיהא לו היסט כזב בין תחת האבן בין בכף מאזנים וממעטינן ליה מדכתיב והנושא אותם דכמשכב ומושב הוא דמטמא באבן מסמא להכלים שעליה כדמפרש בתורת כהנים וכתב אותם למעוטי דמה. ולשון רש"י הוא והלשון שפי' בתוס' כתבתי שם במסכת שבת. 
\textbf{שמא יעשה עור אביו ואמו שטיחין.} מפורש במסכת חולין בפרק העור והרוטב (קכב, א). 
\textbf{אמר ר' יהודה מדסקרתא סלקא דעתך אמינא שעיר המשתלח יוכיח וכו'.} תימא הוא למה חזר והזכיר הטעם שדוחה סברייתא קל וחומר שלו ולמה הוצרך לומר כן לפי שאלתנו זובו טמא למה לי.\par ויש לומר שזו הברייתא השגויה למעלה שעיר המשתלח יוכיח לא היתה שנויה בבה"מ ורבא לא היה יודע אותה כמ"ש בפרק בנות כותיים וכן ר' יהודה מדסקרתא לא שמע אותה והשיב לתרץ דאיצטרך זובו טמא והיה קשה עליו בק"ו והוצרך לומר שמדין ק"ו נמי לא אתי בך מפרש בתוספת. 
\clearpage}

\newsection{דף נז}
\twocol{הא דאקשינן\textbf{ודילמא כהן טמא הוא.} לאו למימרא דכהן טמא מותר ליטמא שלא אמרו אלא בחבור אדם במת כדאמרינן במסכת נזיר היה עומד בבית הקברות והושיטו לו מתו ומת אחר ונגע בו יכול יהא חייב תלמוד לומר לא יחלל יצא זה שמחולל ועומד ומוקים לה התם בחבורי אדם במת הא לאו הכי חייב אלא הכי קאמרינן ודילמא כהן טמא וקסברי כותיים שאין טמא מוזהר על הטומאה אי נמי כהן טמא דיומי הוא וקסברי כרבי עקיבא דאמר במסכת שמחות נטמא בו ביום ר' טרפון מחייב ור' עקיבא פוטר. וכבר פירשנו בארוכה בפרק ידיעת הטומאה (שבועות יז, א).\par ובענין הכותיים בטומאת כתמים אף על פי שכתמים מדבריהם הוחזקו בהן שהן סבורין שטומאתן תורה דאינהו לא דרשי בבשרה עד שתרגיש. אי נמי בכתמים שמרגשת בהן הן נזהרו' ושל ספקות לא חששו להם. וכן בנפלים זהירין הן בטומאה כטומאתן אף על פי שאין קוברין אותן לדבריו דר' יהודה שאם לא היו זהירין בהם נמצא כולן בחזקת טמא מתים ואנן לא תנינן אלא בועלי נדות וכשטבל לאותה טומאה טהור כדאיתא בפרק בנות כותיים. 
\textbf{הסככות.} פירש רש"י ז"ל אילן המיסך על הארץ והוא סמוך לדרך בית הקברות וזימנין דמיתרמי בין השמשות וקברי התם והיינו ספיקייהו. 
\textbf{הפרעות.} אבנים גדולות ובולטות מן הגדר וקבר תחת אחת מהן ואינן יודעין תחת אזו מהן.\par ותימה הוא, אם כן ספק טומאה הוא, ואם הוא ברשות היחיד ספקו טמא ואם היה ברשות הרבים וגזרו עליהן מפני שהוחזקה שם טומאה למה אמרו בכותי מהלך על פני כולה דנאמן שמא תולה הוא בספק טומאה ברשות הרבים דספיקו טהור וגזרו דרבנן לית להו עוד השיב הרב ר' אברהם בר דוד ז"ל דאם כן כל זיזין וגזוטראות נמי ומאי שנא אבנים דנקט.\par ופירש הרב ז"ל שהסככות אילן שענפיו אחת למעלה ואחת למטה ואין שם אוהל אלא שרואין את העליונה כאלו הן למטה והתחתונות כאלו הן למעלה ואף על פי שאין העליונות כדין התחתונות אלא שעדין נשאר שם אויר מועט נעשה כולו (אויר) [אהל] שלם וכן הפרעות אבנים שיוצאות מן הגדר ואינן נוגעות זו בזו אלא שראויו' לקבל מעזיבה נעשה אהל שלם ומביא טומאה מדבריהם. ועשאום כספק טומאה וזהו ספקן שאין אהל שלהם שלם ובשקברו שם בודאי מיירי.\par ואף על גב דאמרינן בסוכה שאין אומרים גוד אחית וגוד אסיק אלא בתוך שלשה משום לבוד וגבי קורות הבית תנן שאפילו אין ביניהם טפח טומאה תחתיהן ביניהם טהור אלמא לא כסתום דמי. שאני הבא דכיון שהכל מאילן אחד ומכותל אחד הוי חבור ומשלים אהל שלהן. כך כתב הרב הנז' ז"ל וכענין הזה שנוי בתוספות טהרות ומיהו דוקא שאין בין הענפים של סככות פותח טפח שאם היה ביניהם בודאי פותח טפח מפסיקין. 
\textbf{אמר ר' יוחנן במהלך ובא על פני כולה.} ה"נ איכא למיחש דלמא טמא הוא אלא באוכל (טומאה) [תרומה] הוא דמתוקמא דומיא דרישא דמתניתין ומשום דרישא מקצר ועולה.\par ותמה הרב רבי שמואל ז"ל אלא מתניתין דקתני נאמנין לומר קברנו שם את הנפלים ואינן נאמנין על הסככות הא ודאי כשם שנאמנין על הנפלים בכהן שלהם אוכל תרומה שם כך נמי נאמנים על הסככות במהלך שם ואוכל.\par וזו אינה קושיא שהמהלך ובא על פני כולה היכי שעובר בכל השדה שאפילו לא היה הכותי חושש לאהל הסככות נטמא בקבר עצמו אם היה שם ונאמן על גופו של קבר הא על טומאות הסככות כגון שמיסך תחת אחד מן האילנות אינו נאמן עליו דלית להו דין אהל בסככות ופרעות אבל בנפלים נאמנין הן ואף על גב דאיכא למיחש לבקיאות דיצירה כן נראה לי.\par וליכא לפרושי מהלך ובא על פני כולה שהולך תחת הסככות אורך ורוחב שהרי פירשנו שאהל הסככות עצמו מדבריהם והם אינן גוזרין כן והלכך ודאי אינן נאמנים עליהם אפילו עושין בהם מעשה. }

\newchap{פרק \hebrewnumeral{9} האשה שהיא עושה}
\twocol{\textbf{ורבי יוסי בחד ספיקא מטהר בספק ספיקא מיבעיא.} פירוש, לר' יוחנן פרכי' דקאמר דר' מאיר נמי מטהר דאלו לר"ל איכא למימר לא תנן ר' יוסי אלא להודיעך דבריו של ר"מ לומר לך שאין כאן מטהר אף בתרי ספיקי אלא ר' יוסי שהוא מטהר אף בחדא ומדר"ש היה לבר זוגיה ש"מ דר"מ מטמא כרישא אלא לר' יוחנן לישתוק גמי מדר' יוסי ופריק לאשמועינן דחד ספיקא דומיא דתרי ספיקי ואפילו לכתחלה וממשנה יתירה גמרינן. 
 הא דאיבעי לן \textbf{יושבת מה לי א"ר שמעון.} קשיא ותיפשוט ליה ממתני' כדאמרי' בסמוך כיון דאמר ר"ש חזקת דמים מן האשה ל"ש עומדת ול"ש יושבת. ואיכא למימר מעיקרא קס"ד שאין חזקת דמים שוים מי רגלים מן האשה אלא בעומדת. והשתא דאשמועינן ברייתא דר"ש אפילו ביושבת אפשר דאתי דם ממקור א"כ הלך אחר חזקתך שחזקת דמים מן האשה ולא מן האיש דל"ש עומדת ול"ש יושבת. א"נ איכא למימר דפשטה דברייתא משמע ליה טפי ועדיף מדיוקא דמתניתין. 
\clearpage}

\newsection{דף ס}
\twocol{הא דתניא \textbf{ושוין שתולה בשומר' יום כנג' יום בראשון שלה וכו'.} פי' משום דלא מקלקלא לה מידי דהא טומאתה תחלה הוא וכתמים דרבנן הם וכל כה"ג תולין קלקלה במקולקל דומיא דההיא דאמרינן בפסחים דבדרבנן אמרינן שאני אומר חולין לתוך חולין נפלו ותרומה לתוך תרומה נפלה.\par והא דאמר ר' חסדא טמא וטהור שהלכו בשני שבילין א' טמא וא' טהור באנו למחלוק' רבי ורשב"ג. בספק טומאה ברה"ר דספקו טהורה היא ובבאין לישאל בבת אחת א"נ שבא לישאל עליו ועל חבירו שא"נ היו טהורים היינו מטמאים שניהם.\par וראיתי למקצת המפרשים שהעמידו מימרא דרב חסדא בטמא שמנה (ימים) [ב'] וג' ימים מימי ספרו שאלו היה בראשון ד"ה תולין בטמא כדק' ושוין שתולה בשומרת יום כנגד יום בראשון שלה וכו'. ואם היה הטמא בז' שלו שהשלים ימי טומאתו היכי קאמרינן הכי מאי נפקא ליה מינה הא ודאי היינו פלוגתייהו דהכא והתם טבילה בעי דמים הן מחוסרים.\par וזה הפיר' אינו מחוור לי שא"כ היה ר' חסדא מפרש בשני שלו או בג' שלו. ועוד דרב אדא בר אהבה דפרכיה דילמא בטמא בז' שלו קאמר ר' חסדא דתרווייהו נמי כי הדדי נינהו וכן זו ששאל ר' יהודה בר' ליואי מר' יוחנן מהו לתלות כתם בכתם אינה מתחוורת לי לדברי המפרשים שאם באנו לומר כן דכתם בכתם בראשון שלה לרבי ובז' לרשב"ג היה לו לישאל. והיכי קאמר אליבא דרבי לא תיבעי לך ואם נאמר דבראשון שלה אפילו לר' תולין בלא טעם. [דמגופא קא חזיא] (אי נמי) [א"כ] ברייתא מיתרנא (לר' יוחנן) [לר' יהודה בר ליואי] שפיר לרבי כאן בראשון שלה והיכי קאמרינן מכל מקום קשיא.\par לפיכך נ"ל דהא דקתני ושוין שתולה כשומרת יום כנגד יום בראשון שלה לרב חסדא לא מפני שתולין קלקול במקולקל לר' דלית ליה האי סברא כלל אלא מפני שהוחזקה להיות רואה. וכן בתולה מפני ששירפה מצוי וכן היושבת על דם טוהר והנכרית מתוך שאינן מקפידות אינן מרגישות ויודעות לפיכך תולין בהן שאלו ראתה זו הרגישה.\par הילכך להאי טעמא טמא וטהור שהלכו בשני שבילין אפילו בראשון שלו באנו למחלוקת ורב אדא פליג אהך סברא ומוקי פלוגתייהו משום דכי הדדי נינהו ור"י בר ליואי נמי כרב חסדא סבר לה הילכך בבעלת הכתם אפילו בראשון אינה תולה לרבי. וכל השמועה על דרך זו תישב אותה. ושוין שתולה בשומר' יום כנג' יום בראשון שלה וכו'. פי' משום דלא מקלקלא לה מידי דהא טומאתה תחלה הוא וכתמים דרבנן הם וכל כה"ג תולין קלקלה במקולקל דומיא דההיא דאמרינן בפסחים דבדרבנן אמרינן שאני אומר חולין לתוך חולין נפלו ותרומה לתוך תרומה נפלה.\par והא דאמר ר' חסדא טמא וטהור שהלכו בשני שבילין א' טמא וא' טהור באנו למחלוק' רבי ורשב"ג. בספק טומאה ברה"ר דספקו טהורה היא ובבאין לישאל בבת אחת א"נ שבא לישאל עליו ועל חבירו שא"נ היו טהורים היינו מטמאים שניהם.\par וראיתי למקצת המפרשים שהעמידו מימרא דרב חסדא בטמא שמנה (ימים) [ב'] וג' ימים מימי ספרו שאלו היה בראשון ד"ה תולין בטמא כדק' ושוין שתולה בשומרת יום כנגד יום בראשון שלה וכו'. ואם היה הטמא בז' שלו שהשלים ימי טומאתו היכי קאמרינן הכי מאי נפקא ליה מינה הא ודאי היינו פלוגתייהו דהכא והתם טבילה בעי דמים הן מחוסרים.\par וזה הפיר' אינו מחוור לי שא"כ היה ר' חסדא מפרש בשני שלו או בג' שלו. ועוד דרב אדא בר אהבה דפרכיה דילמא בטמא בז' שלו קאמר ר' חסדא דתרווייהו נמי כי הדדי נינהו וכן זו ששאל ר' יהודה בר' ליואי מר' יוחנן מהו לתלות כתם בכתם אינה מתחוורת לי לדברי המפרשים שאם באנו לומר כן דכתם בכתם בראשון שלה לרבי ובז' לרשב"ג היה לו לישאל. והיכי קאמר אליבא דרבי לא תיבעי לך ואם נאמר דבראשון שלה אפילו לר' תולין בלא טעם. [דמגופא קא חזיא] (אי נמי) [א"כ] ברייתא מיתרנא (לר' יוחנן) [לר' יהודה בר ליואי] שפיר לרבי כאן בראשון שלה והיכי קאמרינן מכל מקום קשיא.\par לפיכך נ"ל דהא דקתני ושוין שתולה כשומרת יום כנגד יום בראשון שלה לרב חסדא לא מפני שתולין קלקול במקולקל לר' דלית ליה האי סברא כלל אלא מפני שהוחזקה להיות רואה. וכן בתולה מפני ששירפה מצוי וכן היושבת על דם טוהר והנכרית מתוך שאינן מקפידות אינן מרגישות ויודעות לפיכך תולין בהן שאלו ראתה זו הרגישה.\par הילכך להאי טעמא טמא וטהור שהלכו בשני שבילין אפילו בראשון שלו באנו למחלוקת ורב אדא פליג אהך סברא ומוקי פלוגתייהו משום דכי הדדי נינהו ור"י בר ליואי נמי כרב חסדא סבר לה הילכך בבעלת הכתם אפילו בראשון אינה תולה לרבי. וכל השמועה על דרך זו תישב אותה. 
 הא דאמרינן \textbf{רב אשי אמר הא והא רשב"ג. ול"ק כאן למפרע כאן להבא.} כך פירש שאם לבשו הן שתיהן החלוק הזה ואחר שפשטו אותו מצאה אחת מהן כתם א' בחלוק שלה אין תולין כתם בכתם אבל היתה אחת מהן כבר בעל' כתם ולבשו חלוק זה ונמצא בו כתם תולין בבעלת הכתם שהיתה כבר וזה הפי' נכון ולשון הגמרא מוכיח אבל הפי' שפירש ר"ש אינו נכון כלל. 
 כיון ד\textbf{דרש רב חייא בר רב מתנה משמיה דרב} כר' נחמיה ותנא ר' יעקב מטמא ור' נחמיה מטהר והורו חכמים כר' נחמיה שמע מינה דהלכתא כותיה. ועוד דהא רב הונא ורב חנינא ואביי דהוא בתרא מתרצי אליביה דאמרינן התם מדקמתרץ ר' יוחנן אליבא דר' יהודה ש"מ הילכתא כותיה. וההיא איתתא דבפרק הרואה דאשתכח לה דם במשתיתא משתיתא דבר המקבל טומאה הוא דהיינו טווי. וכן פסק הרמב"ם ז"ל כר' נחמיה בכתמים. 
\textbf{שוע טווי ונוז כתיב.} פירש רש"י ז"ל שוע חלק כדמתרגמינן שעיט כלומ' שיהיו חלוקין יחד במסרק וכן פי' טווי שיהיו טווין יחד ונוז לשון אריגה לומר שיהו ארוגין יחד.\par ורבינו תם ז"ל השיב אם כן בכלאים בציצית דשרא רחמנא היכי משכחת לה לא שוע ולא טווי ולא נוז איכא אלא שתי תכיפות בעלמא איכא כדאמרינן במנחו' ש"מ קשר עליון דאורייתא דאי סלקא דעתך לאו דאורייתא כלאים דשרא רחמנא גבי ציצית למה לי והא קיימא לן התוכף תכיפה אחת אינו חבור אלא ש"מ דאורייתא והוו להו שתי תכיפות אלמא מדאורייתא ואפילו (בבגדי כהונה) [בב' תכיפות] הוי חבור. וכן בפרק קמא דיבמות (דף ה') מפקינן שתי תכיפות מדאורייתא ואפילו בבגדי כהונה נמי שחוטן כפול ששה דנוז איכא שוע מיהת ליכא ולא הוי חלוקין במסרק יחד ואפ"ה אסורין משום כלאים כדאיתא במסכ' יומא פרק בא לו (סט, ט).\par אלא כך פירש רבינו תם ז"ל: שוע שיהא כל אחד חלוק במסרק לעצמו ושיהא כל אחד טווי לעצמו ושיהא כל א' שזור לעצמו ומכיון שהן כך אם תכף בהן שתי תכיפות חבור הוא דכתיב יחדיו והכי דייק לישנא דקרא לא תלבש שוע טווי ונוז צמר ופשתים יחדיו כלו' מחוברין ונוז לשון שזירה הוא ולא לשון אריגה ולא לשון חבור כמו שפי' אחרים מדקאמרינן הכא ואימא או שוע או טווי או נוז ומפרקי' כמר זוטרא מדאפקינהו רחמנא בחד לישנא אלמא כי היכי דשוע טווי קאי אכל חד באפי נפשה ה"נ נוז אכל חד באפי נפשה קאי שיהו שזורין ולאו לשון חבור הוא.\par והאי חוטא דכיתנא דקאמרינן דמדרבנן הוא כשאינו שזור דקאמר שסתם חוטין אינן שזורין וכדרומרינן בפרק בתרא דערובין (דף צו ע"ב) המוציא תכלת בשוק לשונות פסולה חוטין כשרה מאי שנא לשונות דפסולה דאמרינן אדעתא דגלימא צבעינהו חוטין נמי נימא אדעתא דגלימא טוינהו בשזורין שזורין נמי נימא אדעתא דשיפתא דגלימא עייפינהו וכו'. וש"מ דסתם חוטין היינו פשוטין ולא כפולין ושזורין והאי חוטא דכיתנא דאבד בנלימא דעמרא בשאינו שזור וגלימא גופא אינה שזורה שאין דרכן בשזורין אלא בשיפתא דגלימא מ"ה הוו כלאים דרבנן. ואפילו תפרש בחוטין בין שזורין בין שאינן שזורין במשמע מכל מקום כאן בחוט פשוט דאינו נוז דהיינו שזור קאמרינן.\par וזו ששנינו במס' כלאים (ט, ח) אין אסור משום כלאים אלא טווי ואריג שנאמר לא תלבש שעטנז דבר שהוא שוע טווי ונוז רשב"א אומר נלוז ומליז הוא אביו שבשמים עליו אלא לאו למימרא דמשוע טווי ונוז נפיק אריג אלא טווי נפיק משעטנז ואריג משום חבור הוא ומלשון יחדיו נפיק. והתם קתני הלבדין אסורין מפני שהן שוע אע"פ שאינו טוי ואריג ומדרבנן קתני או שוע או טווי או נוז ואסורין והיינו דרבנן וכן עיקר המשנה מוכית והיינו דקתני נלוז ומליז אביו שבשמים עליו אלמא משמע לשון נוז היינו ענין פתלתולות ועקש כדרך השזורין שפותלין אותן והיינו דשרא רחמנא כלאים בציצית משום דלגבי ציצית שזורין בעינן כדאמרינן בסיפרי פתיל תכלת טווי ושזור אין לי אלא תכלת לבן מנין וכו'.\par מכאן תשובה לאומרים שאין לשזור חוטי ציצית זימנין דמפרקי והוו להו י"ו חוטין ואין לחוש אי מפרקי דכיון שתחלתן שזורין תו לא מיפסלי דהוו להו כגרדומי ציצית דכשרין אלו דברי רבינו תם ז"ל. וצ"ע בגרדומא דקאמר רבי דהתם בעי שיעור כדי לעניבן ואם כן ה"נ בעינן שנשתייר בשיזור שבהן כדי עניבה. מכאן אתה למד שהתופר בגדי צמר בחוטין של פשתן ושניהן נצמדין שהוא כלאים של תורה. 
הא דתנן \textbf{שבעה סממנין מעבירין על הכתם.} לטהרות קאמר ותני והדר מפרש הטבילו ועשה על גבי טהרו' העביר עליו שבעה סממנין ולא עבר הרי זה צבע. כלומר תולין אותו להקל ונאמר שהוא צבע שכן דרך הצבע שלא לעבור בסימנין ואף על פי שאפשר שהוא דם כיון דבלוע כ"כ שאינו יכול לצאת על ידי סמנין הללו טומאה בלועה היא ואינה מטמאה.\par ומיהו אם לא הטבילו תחלה טהרותיו תלויות שהרי יש לו לחוש לדם ואף על פי שאין סופו לצאת מכל מקום הבגד טמא שכבר נטמא בשעת נפילה ומטמא אותן והיינו דקתני הטבילו ומיהו אם דם נדה ודאי הוא אף על פי שלא עבר טמא לפי שדרך בני אדם להקפיד בו ולהעביר עליו סימנין אלו הילכך לא עלתה לו טבילה ראשונה עד שיעבירם ויבטלנו. 
\clearpage}

\newsection{דף סב}
\twocol{\textbf{עבר או שדיהה הרי זה ודאי כתם וטהרותיו טמאות.} לפי שסוף טומאה זו לצאת וכל שיוציא על ידי סממנין אדם עשוי להוציאו והיינו דפריך מיניה ר"י לריש לקיש דקאמר טומאה שסופה לצאת חמירה אף על פי שלא יצתה ואף על פי שאין אדם עשוי להוציאה א) רק על ידי צפון (אין) תולין בה להקל וטהורה שהכל היו מודים דמתניתין הכי קתני לא עבר שמא צבע ותולין להקל עבר ודאי דם ובטומאה ודאי בעינן בטול סממנין מפני שדרך בני אדם להקפיד עליהן ולבטלן כדפרישית. והאי דלא פרכיה ממתניתין דקתני או שדיהה וצריך להטביל דאלמא סלקא ליה טבילה משום דאיכא למימר צריך להטביל מאחר שיעביר הכתם לגמרי על ידי צפון אי נמי בההיא נמי קולא היא משום דכתמים דרבנן זהו תורף של רש"י ז"ל. ואין במשנתינו בדיקה לגבי בעלה אם הכתם טמא או טהור אלא אם יש מקום לתלות כגון שצבעה תולה ואם לאו אינו תולה.\par והרמב"ם ז"ל פירש רישא דמתניתא לגבי בעלה שאין הכתם מטמא עד שיודע שהוא דם לפיכך בודקין אותו בסממנין הללו אם עבר או שדיהה טמאה ואם לאו טהורה ולדבריו מעשה שתלו בקלור ובשרף שקמה אינו אלא שיהו צריכין סממנין. וק' אם כן מאי פרכיה דר"י לריש לקיש ממתני' כשם שבדיקת סמנין מטהרת לבעלה למה לא תטהר לטהרות. ועוד למה לי למיתני והטבילו ואיכא למימר טהרות חמירי והא דקתני הטבילו משום סיפא ולא דייק. 
הא דאמרינן \textbf{ל"ש אלא טהרות שנעשו בין תכבוסת ראשונה וכו'.} פי' רש"י ז"ל תכבוסת העברת סממנין שהרי הקפיד עליו כשהחזירן והעבירן עליו וגלה דעתו שמקפיד עליו בספק דם ועבר ע"י העברה זו ונעשה בו מעשה דם שכן דרך דם לעבור ע"י סמנין ואין פי' מחוור לי שאין קפידה זו דומה להא דתני ר' חייא.\par אלא כך נראה פי' דכי מטהרינן כתם בטבילה ראשונה כשלא עבר בסמנין מפני שאין סופו לצאת בדרך כבוסו וכדפרישית וזה כיון שגלה דעתו שהוא רוצה להוציאו מ"מ אין זו טומאה בלועה אלא סופו לצאת היא וצריך טבילה לאחר שתצא לגמרי. 
הא ד\textbf{אמר שמואל הרי אמרו לימים שנים.} "אמרו" קאמר וליה לא סבירא דהא לקמן (סד, ב) בוסת דילוג אמרי' דשמואל כרשב"ג בוסתו' דיומי ס"ל והכי קי"ל.\par ומיהו בוסתות דגופה ק"ל היכי אשכח בהו פלוגתא דרשב"ג ורבנן הא לא אשכחן חזקה אלא לר' בתרי זימני ולרשב"ג בתלתא אלמא היכי דבעי חזקה לרבי נמי תרי בעינן ואיכא למימר שמואל גמרא גמיר דלרבנן בחד ואשכח מתני' דקתני בוסתו' דגופה וכל שתקבע לה וסת ג' פעמים הרי זה וסת. ואמר אמאן תרמיה ודאי לרשב"ג דאשכחן דמיקל (בחזקת חששו) [בוסתות] וכיון דסיפא ודאי ביומי רשב"ג היא רישא נמי לדידיה מוקמינן ופלוגתא אחריתי היה מ"ס בעי חזקה דהיינו בתלתא זימני ואפילו בדגופה. ומ"ס אפילו תרי לא בעיא דהיינו אורחאי. ואע"ג דלא אשכחן פסקא בוסתות דגופה כרשב"ג כיון דסתם מתני' הוא ומחלוקת בדשמואל לא עדיף ממחלוקת דבריית' והלכה כסתם מתני' ודאמר ר' יהודה אמר שמואל זו דברי ר"ג הוא וס"ל היא דהא ודאי ס"ל כוותיה ביומי כדפרישית. 
\clearpage}

\newsection{דף סד}
\twocol{\textbf{ראתה יום ט"ו בחדש זה וכו'.} אם באנו לחשב חדשים מלאים בכולן אין כאן דילוג אלא הוסת שוה להפלגת ל"ב ואם באנו לפרש אותן בכסדרן חסר ומלא אין סדר לדילוג הזה שהראשונה מט"ו בניסן לט"ז באייר שוה להפלגת ל"ב והשנייה שהיא מט"ז באייר לי"ז בסיון להפלג' ל"א הוא נמצא שלא דילגה אלא קרבה ראיתה וכשהיא משלשת בדילוג מי"ז בסיון לי"ח בתמוז חזרה להפלגה שוה לל"ב ואין כאן דלוג אלא א"כ דילגה עד י"ט בחדש. ולקמן מתרץ שמואל לברייתא כגון דרגילה למחזי ליום כ' וקתני כ"ג בחדש זה ולפי חשבינך ה"ל למיתני כ"ה.\par לפיכך פ' הראב"ד ז"ל שכשם שהאשה קובעת וסת להפלגת שוות כך קובעת וסת בימי החדש שאם ראתה ריש ירחא וריש ירחא וריש ירחא קבעה לה וסת ול"ח אע"פ שאין ההפלגות שוות שאחד מלא וא' חסר. הילכך בזו שדילגה כיון שאין בראיותיה צד השוה לה לא בהפלגות ולא בדילוגין אומרים לימי החדש היא קובעת ולא בהשואה אלא בדילוג.\par ופסק ר"ח ז"ל כרב באיסורי ומיהו דוקא בזו שהיא קובעת בימי החדש אבל בימי' שקבעה להפלגות אין הראשונה מן המנין דהא לאו בהפלגה חזיתא. וכן כתב ה"ר אברהם.\par וכזה מורין חכמי הצרפתים בתוספות וראיה נתנו לדבריהם דהא מתניתין דהיתה למודה לראות יום ט"ו ושנתה פעמים ליום כ' הרי לה ג' ראיות ואינה קובעת עד ששינתה ג' פעמים ליום כ' כדי שיהיו לה שלש הפלגות של כ'. וא"ת למודה שאני א"כ דקארי ליה לפירכיה דרב מברייתא אמאי קא מייתי לה הא מתנ' היא דלמודה שאני ועוד מדקתני עלה שאין האשה קובעת וסת וכו' ומשמע להו שאין חלוק בין קביעותיה של זו לקביע' אחרת שאינ' למודה והיינו פלוגתייהו דרב סבר כיון דזו לימי החדש היא קובעת אף הראשונה מונין לה שהרי היתה ביום ידוע מן החדש ומתחלת' לדילוג כונה. ושמואל סבר אם השות' ליום החדש מונין לה הראשונה אבל כיון שדילגה צריכה ג' דילוגין.\par וק"ל לרב דאמר למודה שאני מתני' אמאי קבעה לה וסת בשינוי של ג' פעמים קמייתא דט"ו שדי לה בתר ראיות ראשונות של ט"ו וליכא אלא שתי הפלגות. ואיכא למימר לרב לא בעינן אלא שוה מחד צד וכיון שנודעו הפלגותיה של זו בג' פעמים כבר הוקבע. וכי קאמר רב למודה שאני משום דהתם ליכא למימר כטעמיה דמשעה ראשונה כוונה לדילוג דהא לשמואל לא אמרינן למודה שאני כדמתריץ כגון דרגילה למיחזי ליום כ' וכו' אלמא קמייתא ממנינא כיון דאיכא ג' דילוגין מ"מ קבעה לה דהא איכא הכירה דהפלנה בראשונה דהיינו שלש.\par והראשונים הקשו על מה שאמרו שהכל מודים דבהפלגות שלש הפלגות שהן ד' ראיות בעינן מההיא דתניא בב"ק (דף לז) ראה שור נגח שור לא נגח שור נגח שור לא נגח שור נגח שור לא נגח נעשה מועד לסירוגין לשוורים ואע"ג דקמייתא לאו בסירוגין הוות מצטרפי. ומתרצין שאני הכא דכיון דאין הפלגה ידועה אלא בשתי ראיות ג' הפלגות בעינן דהיינו ד' ראיות אבל התם האיכא ג' נגיחות.\par ומיהו לדידי קשיא לי הא דאמ' בפ"ק (יא, א)קפצה וראתה קפצה וראתה וקבעה לה וסת לימים ואוקמה רב אשי כגון דקפצה בחד בשבאי וחזי וקפץ בחד בשבאי וחזאי ולחד בשבאי אחרינא חזאי בלא קפיצה ובודאי ימי שבוע לא קבעי וסת אלא בהפלגות שוות כגון דקפץ בחד בשבא ולאחר כ"ב קפץ נמי הכי ולאשנוי ראיות בהפלגה נקט חד בשבא וקא מני ג' ראיות וקאמר דקבעה.\par ואיכא למימ' התם לאו לאשמועינן בכמה הוסת נקבע אתא ומ"ה לא דק ונקט תלת ראיות בלחוד דאי לדברי רבי קבעה לה וסת אפילו להפלגות ואי לרשב"ג אפיך סדרא מחד בשבא לחד בירחי.\par וחכמי הצרפתים מוסיפין שאף היום גורם וסת כדאשכחן בשור המועד שבת ושבת ושבת נעשה מועד לשבתות הלכך חד בשבא וחד בשבא וחד בשבא קבעה וסת לימי השבוע נמצאו לדבריהם ג' דרכים בוסתות של ימים קביעו' היום והחדש וההפלג' ולדברי הרב ר' אברהם שנים הן וקורא אני עצמי מקרא זה פליאה דעת ממני נשגבה לא אוכל לה. וכי הוסת מזל יום גורם או מזל שעה גורם שיהא תלוי ביום השבוע או ביום החדש שהרי להפלגות שוות הדין נותן כן שכבר נתמלא' סאתה של זו וכן דרכן של נשים כולן וכן של אנשים בחלאים של הפסקות שהן באין בהפסקות שוות או בשעת המולד של לבנה ובמלואה אלא שיהא שיפורא גורם תמה הוא.\par ולפי דעתי בעניותי לא יפה כוונו הראשונים בחלוקי הוסתות שאני אומר אין וסת אלא להפלגה שוה לפי שהאורח בזמנו הוא בא מתמלא ונופצ' לזמן הקבוע שכיון שטבען של בני אדם וזו שאמרו בט"ו בחדש זה וי"ו בחדש זה וי"ז בחד' זה בחדשי' השוים הוא או בחסרי' וההפלגה שלה שוה היא ליום ל"ב. והדילוג שהורו בו אינו אלא דילוג הימים לומר שהיא קובעת וסת לדלג לי"ח לי"ט ולכ', וכן לעולם.\par ולעיקר המחלוקת של רב ושמואל לפי שהוסת הקבוע להפלגה שוה ולימים השוים כגון מט"ו לט"ו בג' ראיות הוא נקבע כשור המועד שבשלש נגיחות נעשה מועד ואלו נגח בט"ו לחדש ג' פעמים וכן בהפלגה שאין בה חדש מכ' לכ' בג' נגיחו' הוא מתיעד דאפי' בסירוגין נעשה מועד בג' נגיחות כמו שפרשתי וכשנגח בט"ו בחדש זה וי"ו בזה וי"ז בזה אינו נעשה מועד לשמואל עד שהוא בדילוג. וכן בוסתות (בהפלגה שוה) בג' ראיות קבעה אותו ואע"ג דקמייתא לאו בהפלגה חזיתה שהרי אף וסת הדילוג וסת של הפלגה הוא לדברינו ואין יום החדש גורמת קבעותו כלל ב) וטעמא דאיכא דילוג הא לאו הכי קבע' בג' ראיות וכדאמ' רב אשי בפ"ק ואע"ג דראיה קמייתא נמי לאו בהפלגה הות אפ"ה מודה שמואל שקובעת אותו בג' ראיות שהרי מתחלה כיון שראתה [ב' פעמים] בט"ו בחדש זה נכרת הפלגת ווסתה מעתה וכן בכל הפלגה [שוה] שתפליג בשתי ראיות ראשונות נודע וסתה לפיכך ראשונה מן המנין אבל וסת הדילוג כשראתה בט"ו בחדש זה וי"ו בחדש זה לא יודע וסתה של זו שהרי הפליגה ולא דילגה כלל ומנין לנו שלדילוג היא מכוונ' שמא תראה בחדש הבאה בי"ו והוםת שוה ה) וכשרא' למחרתו והוסת של דילוג הוא היינו שלישית {\small [לפי שראייה שלישית היא שביררה הדילוג]} ואינה ראויה לקביע בתחלתה והיינו דקאמר קמייתא לא בדילוג חזיתה כלומר בהפלגה ראשונה עדיין לא היה לה כלל וסת של דילוג אבל בהפלגה שוה מתחלתה כשחזרה וראתה [ניכרת ההפלגה] ויודע וסתה.\par [והא דאמרינן] שינתה למיום עשרים דבעי ג' פעמים למודה שאני לכ"ע דלט"ו היתה למודה לראות בקביעות הך ראיה בתרייתא בתר ראשונו' שדינן לה ועכשיו ששנתה ליום אחר לגמרי כמי שמתחלת לראות דמיא. וכי קס"ד לרב דלא אמרינן למודה שאני ולשמואל נמי לית ליה במדלגת בלחוד היא משום דעכשיו גמי אינו שינוי גמור אלא על הוסת ראשון עצמו מדלגת והולכת מט"ו לי"ו ומראיה דוסת ראשון ניכר וסת ב' של דילוג לרב כמו שפי'. וראיה לדבר ההיא דאמרינן בב"ק (דף לז) נגח שור שור שור וחמר וגמל מהו האי שור קמא בתר שורים שדינן ליה ואכתי לשוורים הוא דאיעד לשאר מיני לא איעד או דילמא בתר חמור וגמל שדינן להו זהו ואיעד לכולהו מיני. וה"נ אי חזיא ט"ו וט"ו וט"ו תלת למני והדר דילגה י"ו וי"ז היינו בעיין אבל למודה לראות ט"ו וט"ו ד' זימני כיון דאי צמי שדית בתרייתא בתר הני דדילוג אכתי קבעה לט"ו שדינן כולהו בתר מעיקר' דסרכא נקט וכן נמי בשור שור שור ושור (ושור) וחמו' וגמל שור בתריית' בתר שוורים שדינן ליה דהא איעד להו והשתא הוא דקא מיעד נפשיה לשאר מיני זהו הדרך שנראה לי בדברים הללו.\par ועדיין לבי מהסס, מה טעם אמרו בשור המועד נגח בט"ו בחדש זה וי"ו בחדש זה וכו' דאלמא מתיעד הוא בדילוג ובהפלגה. וכי מזל יום גורם נגיחות שאפילו בהפלגה שוה אין הטעם מתחוור בכך כמו שהוא מתחוויר בוסתו' אלא י"ל שראו חכמים בכל נגיחות שהן לזמן שוה כגון בין בסירוגין בין בדילוג שאין מתיעד אלא לאותו ענין שעשה לנגחותן שכך וסתו של זה ליגח ושמא מלמ"ד ללמ"ד מוסיף כח ונוגח וכן כיוצא בדבר זה לעולם למה שהשוה הוא מתיעד ולא לדבר אחר.\par ומדברי הרב ר' משה הספרדי ז"ל משמע שאין לו קביעות וסת אלא בהפלגה שוה שאם היה סובר כדברי הרב ר' אברהם או כדעת חכמי התוס' ז"ל היה לו לפרש (בחסדו) [בספרו] והוא ה"ר משה ז"ל פסק כשמואל בדילוג משום דאמרינן דיקא נמי וש"מ ומסתברא כותיה וה"ר אברהם ז"ל דן בה להחמיר. ובעל נפש יחוש להחמיר בענין הוסתות בין בדברינו בין לדברי ראשונים עד יערה עלינו או על אחרים רוח ממרום להכריע איזו היא הדרך הישר' שיבור לו האדם ואם ימצא בחבורי הגאונים או בתשובותיהם ענין מורה על א' מאלו הדרכים בה ראוי ללכת ולצאת בעקבותיהם. 
מהא דקתני בברייתא \textbf{שינתה לי"ז הותר י"ו ונאסר ט"ו וי"ז.} ולא מיתסר נמי י"ח דנימא זו כבר דילגה ונחוש בפעם א' לוסת של דילוג ש"מ שאין חוששין לוסת של דילוג כלל עד שתקבענו לגמרי וזה כתוב בתוספות. וכן הורה ה"ר אברהם ז"ל. 
\textbf{היתה למודת להיות רואה יום כ' ושנתה ליום ל'.} מדקתני האי לישנא ש"מ דוסת הפלגה הוא דקבעה מכ' לכ'. והשתא ק"ל כיון דקי"ל מראיה לראיה מנינן ולא לפי מנין הראוי כדאיתא בשלהי בנות כותיים כשהגיע יום כ' ולא ראתה ומנו עשרה לתשלום ולא ראתה הגיע יום כ' וראתה דקתני כי אורח בזמנו בא מאי נינהו הא ליכא הפלגה דעשרים השתא.\par ואיכא למימר הכא מנינן למנין הראוי ויום מ' לראיה אחרונה זו היא יום כ' דקתני שאם ראתה בעונות הראשונו' ביום זה תראה ואפילו הרחיקה יותר מונין לראיה אחרונה שפסקה בו עכשיו ולא שאלו ראתה מאותה ראיה ואילך בעונות של כ' יארע לה ראיתה ביום עשרי' אורח בזמנו בא דהכא רגלים לדבר שלמנין הראוי חוזרת אלא ש"ל זמן שרואה בוסת השינוי מונין להן מאות' ראיה אבל מכיון שהפסיקתו וחזרה לראות ביום (א') [אחר] אם למנין הראוי חזרה מונין לוסת הראשון לפי אותו מנין ואין אומרין הפלגה של מ' היא זו שרגלים לדבר.\par אבל הרב ר' אברהם בר דוד ז"ל פי' לזו בוסת החדש לפי דעתו ולמודה ליום כ' בחדש ושנתה ליום ל' בחדש קתני ולפי פי' בוסת של הפלגה אין אומרים חזר הוסת למקומו עד שתראה עכשיו ותחזור ותראה לסוף כ' שחזר האורח בזמנו. }

\newchap{פרק \hebrewnumeral{10} תינוקת}
\twocol{\clearpage}

\newsection{דף סה}
\twocol{ מנימון סקסנאה דעבר \textbf{מיעבד כרב ואפילו ראתה} לית ליה אידך דרב דאמר בועל בעילת מצוה ופורש משום דההיא אתיא כרבותינו דחזדו ונמנו ואיהו דאמר כסתם מתני' ושמואל לא קבילי ליה דמנימון (מידי) א) מ"ה דמאן דעבד כסתם מתני וכמעש' דר' לאו בר עונשין הוא אלא משום דבעי למיעב' דלא כחד ולמיתל' ברב מ"ה איענש דלא יאונה לצדיק כל און. 
\textbf{כולן צריכות לבדוק את עצמן.} פי' רש"י ז"ל שמא נשתנו מראה דמים שלה ולא סמכינן למימר הואיל ושופעת הכל דם א' הוא וטהורות.\par ואי קשיא מ"ש לאחר ד' לילות והא בתוך ד' לילות נמי אמרינן בפ"ק נשתנו מראה דמים שלה טמאה והתם נמי אמרינן ותבדוק בעדים דילמא נשתנו מראה דמים שלה.\par י"ל הכא כיון ששופעת אין צריך לבדוק כל זמנן אבל מתוך זמנן לאחר זמנן צריכות לבדוק. א"נ התם לטהרות הכא אפילו לבעלה צריכות לבדוק הא אם נשתנו ודאי בין לאחר זמנן בין בתוך זמנן טמאה והך רישא ד"ה היא דודאי נשתנה מראה דמים שלה טמאה. וסיפא פלוגתא דר"מ ורבנן פלוגתא אחריתי היא שהיה ר"מ אומר דם בתולים אינו אדום וזהום לפיכך בודקת בזה אם מצאתו אדום וזהום בידוע שהוא דם הנדה ואע"פ שאינו יודע תחלתו מה היה שאם מצאתו מתחלתו אדום שאעפ"כ היה טהור אע"פ שאין רגילתו בכך וחכמים אומרים כל מראה דמים א' הוא הילכך לעולם תולין בדם בתולים עד שיודע לך שנשתנה ממה שהיה בתחלה.\par ויש לפרש ברייתא כולה דר"מ היא וה"ק כולן שהיו שופעות ובאות מתוך ד' לילות ולילה א' לאחר זמנן אינן טהורות אלא בבדיקה שבכולן ר"מ מחמי' כדברי ב"ש מן הסתם ומיקל כדברי ב"ה עם הבדיק' ומה היא הבדיקה הזאת שיהא מראהו שלא כדם הנדה אינו אדום ואינו זהום הא בתוך זמנן כב"ש אינן בודקות בכך אבל אם ראתה שנשתנה מראה הדמים טמאה. 
\clearpage}

\newsection{דף סו}
\twocol{ה"ג וכן גרסת רש"י ז"ל \textbf{משמשת פעם ראשונה ושנייה ושלישית} וכן בכולהו גרסינן ושלישית. ותתגרש ותנשא לאחר וכולה רשב"ג היא דאמר בתלת זימני הוי חזקה וכי קתני דברי רבי אסיפא דמתני' דנאמנת אשה בלחוד קתני אבל ברישא כולה רשב"ג היא.\par ורבינו ז"ל כתב בהלכות עד כמה מותרת לינשא עד ג' יותר מכאן לא תנשא עד שתבדוק את עצמה ואיני יודע אם הוא לשון גרסת הגמרא מ"מ כרשב"ג אתיא. 
 והא דמקשינן \textbf{ותבדוק בביאה ג' של בעל ראשון.} נראה ודאי דה"ק היאך מותרת לינשא לב' ולשמש בלא בדיקה והלא כבר הוחזקה זו לראות מחמת תשמיש בג' ביאות של בעל ראשון הילכך תבדוק ולא תתגרש ופריק מותר לשני בלא בדיקה מפני שאין כל האצבעות שוות ואין מחזיקין אותה בבעל מום ורואה מחמת תשמיש אלא לאצבע זה שהוחזקה לו. לפיכך תתגרש ממנו אבל לשאר אצבעות [לא הוחזקה] עד שתהא מוחזקת לכל בג' אצבעות.\par והדר אקשי' כיון שהוחזקה בג' אצבעות למה לה ג' פעמים באצבע אחרון. ולמאי דפריך השתא דחזקה באצבעו' בלבד תהא חזקה אפילו שמשה פעם א' וראת ונתגרשה ושמשה עם השני וראת ומת ושמשה פעם אחרת עם ג' זה וראתה הוחזקה זו לכל.\par ופריק לפי שאין כל הכוחות שוות. ואיפשר שראוי' לאצבע בנחת אבל בג' אצבעות וג' כוחות בכל אצבע ואצבע הוחזקה לכל אבל אם רצתה להכניס עצמה בספק ולבדוק בבעל ראשון ושלא תתגרש ודאי מותרת שאין אחר בדיקה של חכמים בית מיחוש שנאסרות אותה לא' ונתיר לג'. וכי קתני תתגרש ותנשא רבותא קמ"ל דלא מיחזקה אלא בג'. ומי שסובר להחמיר בבעלי' הראשונים אין ממש בדבריו דאם יש לחוש בראשונים כ"ש לאחרון שהוחזקה ולא התירו ספק דבר שזדונו כרת בשביל שתנשא זו.\par אבל לענין מעשה עכשיו נראין דברי הרב ר' אברהם בר דוד ז"ל שאמר אין אנו בקיאין בבדיקה זו. ועוד שאפילו שידוע שהוא מן הצדדין הרי גזרו בנות ישראל בכל רואה טיפת דם כחרדל שיושבת עליה ז' נקיים בין מן המקור בין מן העליה בין מן הצדדין ודעת רבינו ז"ל שכתבה להנהיג בה הלכה למעשה. ואף ה"ר אברהם אמר שאין מוציאין אותה לאחר בדיקה. 
הא דא"ר יוחנן \textbf{לכי והבעלי לו על גב הנהר.} משום שאין וסת זה אוסר יומו לפי שלא היה וסתה אלא בהכנסתה לעיר שהיו חברותי' מרגישו' בה ושמא אף היא היתה מתביישת מהן מפני שמדברות בה ומתחלחלת וה"ל כוסת של קפיצות ושל אכילות שום שכל זמן שאינה אוכלת אינה חוששת. א"נ דימה לא קבעה וסת דאקראי בעלמא היא וה"ק לה לאו טבילה גרמא ליך דתיתסרי אלא דימת עירך גרמא ליך ומותרת את על גב הנהר. 
הא דאמר שמואל \textbf{ממלא ונופצת היא} ואין לה תקנה. ק"ל א"כ עקרת בדיקה הראשונ' ששנויה בבריית' דקא אמרת דאי אפילא נתרפאת ואי לא אפילא לית לה תקנה אי הכי מכחול למה ושמה אותה שהפילה חררה בידוע שנתרפאת וזו שלא הפילה כלום ממלא ונופצת היא. אבל אם ראתה מחמת בעיתותא ולא הפיל' חררה זו היא שצריכה לבדיקה של ברייתא וכן נמי שלא ביעתוה נבדקת בכך ואינו מחוור ומרבינו הגדול ז"ל לא כתב מעשים הללו. 
ה"ג וכן בנוסחאות \textbf{יום א' תשב ו' והוא ב' תשב ו' והן.} שהרי בני מקום זה שאין בני תורה בידוע שאין רואים דם ויש לחוש שמא יום א' דם טהור ויום ב' טמא וצריכה ו' ועוד שהרי אין נשיהן בקיאין בימי נדה וזבה שלכך תקן להם לג' ז' נקיים בכל זמן הלכך בשנים נמי יש לחוש שמא ראשון י"א לזיבה הוא ושני תחלת נדה הילכך צריכות ו' והן. ואין זה צריך לפנים אלא שבהלכות רבינו הגדול ז"ל דמחה והן נראה כטעות סופר.\par והלשון שכתב רש"י ז"ל תשב ו' והוא כדין תורה לומר שדין תורה כך הוא למנות ו' לאחד אבל אינו כדין תורה לכך שזו יושבת ו' נקיים שאם תראה צריכה יותר הילכך צריכה הפרשה בטהרה ובדיקה והאי דקאמר ז' נקיים ולא קאמר נמי ו' נקיים לישנא בעלמא נקט דשגירי למימר ז' ימים נקיים.\par וא"ת לדברי האומר ימי נדה שאינו רואה בהן אין עולין לה לספירת זיבתה עדיין היה לו לר' בית מיחש לומר שמא יום א' י"א הוא ובעי שימור והז' הן התחלת נדה אין עולה לו וצריך ז' והן לשני ימים. לאו מילתא שכבר פי' בפ' בנות כותיים שימי נדה ולידה שאינה רואה בהן עולין לספירת זיבה קטנה לדברי כל אדם ואין לחוש כלום. וכתב רבי' בעל הלכות ז"ל שאפילו בימי טוהר נמי אם ראתה סופרת דהאידנא יולדות בזוב הן לפי שא"א לפתיח' הקבר בלא דם ובנות ישראל סופרת ז' לכל טיפה דם. ושמענו כדבריו בזה שהגאונים החרימו בדם טוהר.\par והדברים נראין אף לדין הגמר' שכשם שחששו לטועות בפתחיהן ולמשלימות דם טמא לדם טהור ועשו נמי הרחקה יתירה בדבר כך יש לחוש שמא יבואו לטעות באותן שיושב' עליהן לזכר ולנקבה ולנדה שמא ינהגו בהן קולא שסוברות כל שיש לו טומאת לידה יש לו טוהר שלה וכ"ש שברוב נפלים אין בני אדם בקיאין. ואין עליהם לדון בהם אלא כך תשב לזכר ולנקבה ולנדה וקרוב הדבר לטעות בו הילכך אין ימי טוהר יוצאין מכלל ר' זירא שאף בהן החמירו בנות ישראל לישב ז' נקיים. וההיא דאמרינן דרש מרימר הילכתא כותיה דרב וכו'. דינא קאמרי כדאמרי בשמעתי' אמינא לך האי איסור ואת אמר' לי חומרא היכא דאחמור אחמור היכא דלא אחמור לא אחמור. והמקומות שבועלין עכשיו על דם טוהר הם יחושו לעצמן. 
\textbf{חפיפה.} פי' רש"י ז"ל חפיפת שערה ופי' לפי' חפיפת שער בכל מקום שבגוף בית השחי ובית הערוה ואצ"ל ראשה. ובודאי דלשון חפיפה לא שייך אלא בשער כדתנן נזיר חופף ומפספס אבל לא מסרק ואיתמר בעלמא הוה חייף רישיה.\par וה"נ משמע בפ' מרובה (דף סב) גבי עשר תקנות שתקן עזרא ושתהא אשה חופפת וטובלת ואקשינן דאורייתא הוא דכתיב את כל בשרו את הטפל לבשר ומאי ניהו שערו. ופריק מדאורייתא עיוני בעלמ' דילמא מיקטר א"נ מיאוס מידי משום חציצה אתא איהו ותיקן חפיפה. מדמקשינן את הטפל לבשרו ומאי ניהו שערו ש"מ שאפילו בכל הגוף צריכה לעיין דברי תורה משום דילמא מיאוס במידי משמע דלגבי הכי מקום השאר ושאר מקומות שבגוף שווין אלא דאתא עזרא וחייש דילמא אתיא למיטעי בעיוני דשער משום דשכיח ביה קטרי ואחמיר ביה חפיפה הילכך בעי עיוני בכוליה גופיה דאורייתא ובעי נמי חפיפה למקום שער מתקנתא.\par והיינו דאמרינן בשמעתין (לקמן סז, א) נתנה תבשיל לבנה וטבלה לא עלתה לה טבילה כלומר אם לא חזרה ועיינה בנפשה בשעת טבילה ממש אבל ודאי עיינה בעצמה אע"פ שלא חזרה לחוף נראה שעלתה לה טביל' דהא אע"פ שנתנה בנתיי' מעט תבשיל לבנה סמוך לחפיפה טבילה היא ומזיא לא מיקטרי בתבשיל של בנה.\par והא דאמרינן ונמצא עליה דבר חוצץ אם סמוך לחפיפה טבלה וכו'. דמשמע עליה על גופה לאו למימרא דחפיפה בכולי גופא היא אלא משום דודאי עיינה בנפשה בשעת חפיפה ואם לא טבלה סמוך לחפיפה ולא חזרה ועיינה בשעת טבילה אע"פ שהיתה משמר' נפשה מליתן תבשיל לבנה וכיוצא בו כיון שמצא עליה דבר חוצץ חוששין דילמא נגעה ולאו אדעתה.\par וההיא דאמרינן לקמן (סז, ב) בחופפת בע"ש וטובלת למ"ש וכולהו הרחקות דחפיפה כשחזרה ועיינה בעצמה בשעת טבילה שלא הקלו בשל תורה אלא בתקנת עזרא דהא לא אפשר מע"ש למ"ש דלא נחנה תבשיל לבנה וכיוצא בזה ואמרן דלא עלתה לה טבילה אלא התם בשלא עיינה בעצמה וכאן כשחזרה ועיינה בשעת טבילה.\par ומיהו מנהגא דנהגן נשי למשטף כולה גופה בחמימי בשעת חפיפה משום מקומות השער שבגוף הוא דקרירי ממשרו להו. וה"נ משמע בהא דאמרינן לקמן עבדי חסרת דודי חסרת טכטקי חסרת משמע דשטיפת כל גופא עבדא מדצריכה עבדי ודודי וטכטקי וכן החמירו בנות ישראל על עצמן והמעביר מנהג זה ימתח על העמוד. 
\clearpage}

\newsection{דף סז}
\twocol{ הא דאמרינן \textbf{ולית הלכתא ככל הני שמעתא. אלא כי הא דאמר ר"ל אשה לא תטבול אלא דרך גדילתה וראוי.} לאו למימרא דהנך פלגינן אדר"ל דהא אפשר דתרווייהו איתנהו אלא גמ' קאמר דלית הלכתא בכל הני שמעתת' דלא מחמרינן כולי האי בביאת מים בדברים שדרכן להיות באותו מקום אלא כך מחמרי' בביאת מים בדר"ל מחמירים ובעי טבילה דרך גדילתה כדי שיבואו מים קצת בבית השחי ובבית הסתרים.\par וכה"ג איכא טובא בתלמודא דקאמר לית הלכתא בכל הני שמעתתא אלא כי הא ולא פליגן אהדדי. כי ההיא דאמרינן ככתובות דלית הלכתא ככל הני שמעתא דזינתה וכחלה ופירכסה ותבעוה לינשא ונתפייסה אבדה מזונותיה אלא כי הא דא"ר יהודה תובעת כתובתה בב"ד אין לה מזונות והא (דא"א) [נמי אף דאפשר] דאיתנהו לכולהו. וכן במסכ' ברכות (דף מב) סלק אסור לאכול ומר אמר גמר ומר אמר משחא מעכב לן ואמר ולית הלכתא ככל הני שמעתא אלא כי הא דא"ר יהודה תכף לנטילת ידים ברכה. וכן בפרק שלשה שאכלו (דף מז ע"ב) ולית הלכתא ככל הני שמעתתא אלא כי הא דא"ר נחמן קטן היודע למי מברכין מזמנין עליו וההיא ודאי לא פליגא אשמעתתא דלעיל דט' ונראין כעשרה וט' וארון ושנים ושבת ושני ת"ח המחדדין זה את זה כ"ש קטן ועבד נעשה אותן סניפין לעשרה. והרבה מהן כך.\par אבל כי איתמר בתלמודו לית הלכתא ככל הני שמעתתא מהא דאמרינן ההיא ודאי פליגן ומחדא מידחיא חברתה ממש.\par ורבינו הגדול ז"ל גורס ולית הלכתא ככל הני שמעתתא אלא כי איתמר הני לענין טהרות איתמר אבל לבעלה שפיר דמי כי הא דאמר ריש לקיש וכו'. וק"ל מנלן מדריש לקיש דלבעלה מותרת דהא אפשר דכולהו איתנהו. ואיכא למימר כי מייתי דר"ל משום פתחה עיניה ביותר ועצמה עיניה ביותר דלא מעכב בטבילה דרך גדילתה נמי הוא שאין אדם נמנע מלפתוח ולעצום עיניו פעמים הרבה כדרכו ואין המים מתעכבים מליכנס שם כדרך שנכנסין בבית השחי ובבית הסתרים בטובלת דרך גדלתה ולא מחוור.\par תו קשה לי ההיא דגרסינן ביבמות (מז, ב) וכל דבר שחוצץ בטבילה של טהרות חוצץ בגר ובעבד משוחרר ובנדה לבעלה. זו היא גרסתו של רבינו עצמו ז"ל ומשמע דכל דבר שחוצץ בטבילה של טהרות חוצץ בגר ובעבד משוחרר ובנדה לבעלה. והא איכא הני דחצצי לטהרות ולא חצצי לנדה.\par ואיכא למימר דה"ק: כל דבר שחוצץ בטבילה אחרת חוצץ בגר ובעבד ובנדה ואף על פי שאין טבילתן מפורשת בתורה שהרי טבילת נדה מן הכתוב מפורש לא למדנו אלא מבנין אב אתיא דכתיב ורחצו במים בנין אב לכל הטמאין שיהיו בטומאתן עד שיבואו במים. ומה שאמרו שם במקום שנדה טובלת שם גר ועבד משוחרר טובל לא מפני שטבילת נדה מפורש' יותר אלא לומר דלא בעינן מעין כזב וא"נ דבעינן טבילה בבת אחת משום דסמכי לה אבמי נדה יתחטא מים שהנדה טובלת.\par והרב ר' אברהם בר דוד ז"ל פירש ולית הלכתא ככל הני שמעתתא לפלוף וכוחלי אלא כי הא דאמרן כדר' יוחנן פתחה עיניה ביותר דאמר ר"ל אשה לא תטבול אלא דרך גדלתה ומי שפתחה עיניה ביותר אינו דרך גדלתה. וזה הפירש מוקצה מן הדעת מפני שהוא מקלקל עלינו שיטת התלמוד שאמרו בכמה מקומות ולית הלכתא ככל הני שמעתתא אלא כי הא ובכולן פירושן ידוע שאין הלכה בכל הנזכרות אלא כי הא דבעינן למימר קמן.\par וקשה לן מרייהו דהני שמעתח היכי אמרינהו והאנן תנן (מקוואות ט, ב)אלו הן שחוצצין לפלוף שחוץ לעין וגלד שהוא חוץ למכה ואלו שאין חוצצין לפלוף שבעין וגלד שעל המכה.\par ואיכא למימר לפלוף שבעין מוקים לה רב עוקבא בלח. והיינו טעמא דלא חייץ משום דלא קפיד עליה וה"ל מיעוטו שאינו מקפיד אבל יבש ודאי מקפיד. ושחוץ לעין דקפיד עליה אפילו לח חוצץ וגלד שעל המכה נמי משום האי טעמא הוא דלא עביד אינש לקלף גלד מכתו משום דקשה למכה עד דיביש ומקליף מנפשיה וקסבר דהוא דוקא של מכה אבל ריבדא דכוסילתא עד תלתא יומין דלא קפיד איניש עלה לא חייצה מכאן ואילך חייצה. אי נמי עד תלתא יומין לחה ולא מעכבה מיהו מכאן ואילך חייצה דיבישה וקפיד עלה.\par ואי קשיא לך מאי שנא פתחה עיניה ביותר או שעצמה דלא מעכבי בטבילה ומ"ט קרצה שפתותיה כדתנן כאלו לא טבלה. לא תיקשי דודאי קרצה שפתותיה מעכבת ביאת מים במקום הגלויי אבל פתחה עיניה אינה מעכבת כלום אלא קמטין בעלמא הוא דעבדה במקום שדרכן בכך ואפשר נמי שאינן מעכבין כלל מלבא בהם מים ממש. 
הא דאמר ליה רב פפי לרבא \textbf{מכדי האידנא כולהו ספק זבות שותינהו ליטבלן ביממא דשבעה.} רש"י ז"ל מפרש כמשמעה לומר שאין לחוש להטבל בלילה אלא יכולין לטבול אותן ביום כדין הזבה דאי לנדה יותר משמיני הוי ואי לזבה טבילתה ביום הז' הוא.\par ומתרצים משום דר' שמעון דאמר אחר מעשה של ספירה תטהר מיד ומקצת היום ככולו אבל אמרו חכמים אסור לה לטבול שמא תבא לידי הספק שמא תבא לשמש כיון שטבלה ותראה.\par והגאונים כך סוברים שהאשה בזמן הזה אינה טובלת אלא בלילה משום סרך בתה שלא תטבול בז' ותראה ותסתור.\par ואחרים פרשו דרב פפי לאו אטבילה בלחוד פריך אלא ליטבלו ביממא דז' ויהיו מותרת לבעליהן קאמר שהרי מקצת היום בספירת הזבה כולו הוא. ומתרץ לה משום דר"ש דאמר אסור לעשותה כן להחזיק עצמה בטהורה לאחר ספירה מיד כלומר אסור שתשמש ותעסוק בטהרות שמא תראה ותסתור. וראיה לפירש זה מה שאמרו בסוף המפלת (דף כט ע"ב) בכ"א תשמש ר' שמעון היא דאמר אבל אמרו חכמים אסור לעשות כן שמא תבא לידי הספק הא טובלת אפילו לר' שמעון ואסורה לשמש הא לרבנן מותרת אפילו לשמש אלמא איסורא דר"ש בתשמיש בלחוד היא.\par ור"ש עצמו ז"ל כך כתב שם ר"ש היא דאמר בת"כ אסור לעשות כן לשמש זבה ביום טבילתה. ובודאי דהתם בת"כ מוכח כן דקתני כיון שטבלה טהורה להתעסק בטהרות אבל אמרו חכמים לא תעשה כן שלא תבא לידי הספק. וש"מ דאסור לעשות כן אעסק טהרות קאי וה"ה לתשמיש ולא אטבילה קאי ולפי הפירש הזה מקילין ואומרים דהאידנא טובלות הן ביום וליכא סרך בתה כלל.\par ובודאי שזה הפירש הוא הנכון דר"ש לא אסר אלא להחזיק עצמה בטהורה לטהרות אי נמי לתשמיש אבל שנקל לומר שיהו טובלו' ביום אינ' נראה דהא רבא דשרא במחוזא משום אבולאי הא לאו הכי אסור בתר חומרא דר' זירא הוה דקא"ל ר' פפא מכדי האידנא כולהו ספק זיבות שויתינהו ומשמע נמי דאהדא דתקון תקנתא ושרא משום אבולאי פריך ליה למה ליה אבולאי כולהו נמי ליטבלן ביומא דז'. אלמא לרבא אית ליה משום סרך ואפילו לבתר חומרא והכי פירכיה אפילו לטבול ולהתירן לבעלן יהיו מותרת, ומתרצין להתירן א"א משו' דר"ש וכיון דאסורות לטהרות ולבעלה אין טובלות אלא בלילה ואפילו בח' משום סרך בתה שמא תטבול ותטהר כאמה דודאי כשם שהיא נסרכ' אחר אמה בטבילה דנדה נסרכת אחריה בתשמיש גופה, ואם תאמר תשמיש גופיה גזירה דרבנן ואנן ניקום ונגזור גזירה לגזירה, כיון דבא לידי איסור דכרת גזרינן.\par ואי קשיא דרבנן היכי שרו לבעלה בז' והתנן טבלה ביום שלאחריו ושמשה הרי זה תרבות רעה התם בשומרת יום רגילה היא לבא לידי זיבה גדולה הכא כיון שספרה שבעה הוחזקה במעין סתום ואין חוששין לה אלא מדבריהם לר' שמעון, ולפי דעתי שלא נחלקו חביריו עליו כלל מדלא פרכינן תינח לר"ש לרבנן ליטבלן, והא דאמרינן בהמפלת, הא מני ר"ש היא משום דלדידיה שמעינן לה וכיוצא בה בתלמוד הרבה.\par ואחרים השיבו אי ר"ש לאו אטבילה קאי היכי קאמר שמא תבא לידי ספק ודאי לידי ספק באה ואין אנו גורסים אלא שלא תבוא לידי ספק, וכן בהלכות גדולות וכן בת"כ. ויש נוסחא שכתוב בה יבואו לידי ספק. 
\clearpage}

\newsection{דף סח}
\twocol{הא דתנן \textbf{וחכמים אומרים אפילו בשנים לנדתה בדקה וכו'.} דוקא בשנים אבל בראשון הואיל והוחזק מעין פתוח לא דתחלת נדה אין דרכה לפסוק ביומה ואפילו בדקה ומצאה טהור חוששין לה שמא חזרה וראתה ואין צריך לומר כשלא בדקה כלל אלא שראתה תחלה דלעולם היא בחזקת טומאה עד שתפרש בטהרה כיון שראתה נדה וליכא מאן דפליגי, ומיהו בשני אפילו ראתה בשחרי' ופסקה טהרה באמצע יום ובין השמשו' לא הפרישה ואחר הימים מצאה טמא הרי זו בחזקת טהרה, ובברייתא תניא דרבי מטהר אפילו במצאה טהור בראשון.\par ואין לפרש דלא פליגי אלא שני וה"ה לראשון דמכדי רבנן בתראי לטפויי מילתא אתו דתנא קמא רישא שביעי קאמר ואוסיפו אינהו אפילו בשנים אם בן לימרו ראשון וכן פי' רש"י ז"ל ומסתברא אפילו מצאה טהור כשבדקה אח"כ הרי זו חוששת דכיון שאין הפרישה של ראשון לנדה הפרשה גמורה אין בדיקה של עכשיו מחזיקה בטהרה למפרע ויש לדון בדבר אלא שהוא חומרא. 
\clearpage}

\newsection{דף סט}
\twocol{הא דמקשי' \textbf{טועה דר' יוסי בר חנינא לרב דאמר סופן אף על פי שאין תתלתן.} אף על גב דרב מכל מקום בדיקה דהפסקה בעי, שאני התם דכיון דילדה ודאי פסיקא אלא טועה דיום א' לא ידענא מאי סייעתא דאדרבא קשיא דהא התם לא הפסיקה טהרה שהרי אנן חוששין שמא עכשיו כשבאת לפנינו ראתה ולמה כל הני טבילות תפריש ותספור ותטבול לזבה, ואיכא למימר דהות משמע ליה מעיקרא דהכי קתני יום אחד טמא ראיתי והפסקתי ואיני יודעת כמה ראיתי והפסקתי דבלא הפסקה לא אפשר. 
והא דקתני ברייתא \textbf{בין השמשות טמא, ראיתי ואמר רב ירמיה מדיפתי שבאת לפני' בין השמשות.} נראה לי דהכי אמר ברייתא דקתני בין השמשות לאו לראיה אלא לביאה והכי קתני באה בין השמשות ואמרה טמא ראיתי, ואמר רב ירמיה מטבילין אותה אחד עשר טבילו' שהרי כל שבאה בין השמשות אפילו אמרה סתם יום אחד טמא ראיתי אחד עשר טבילות הן, אי נמי בין השמשו' טמא ראיתי דקתני שאם אמרה מבעוד יום ראיתי אין כאן י"א ולאו מילתא היא. 
 הא דתנן \textbf{בית שמאי אומרים כל הנשים מתות נדות.} אוקימנא בגמרא טעמייהו דבי' שמאי כדתני' בראשונ' היו מטבילין על גבי נדות מתות והיו נדות חיות מתביישות התקינו שיהו מטבילין על הכל, ולא למימרא דבית הלל פליגו אהך תקנתא אלא בית שמאי סברי גזרינן בכולהו לעשותן כנדות בין בחיים בין במיתה לטמא באבן מסמא ובית הלל סברי לענין הטבלת כלים עשאום כנדות, אבל לא לשאר טומאת דחיים ולא לאבן מסמא במיתה.\par ויש מפרשים שחזרו בית הלל ותקנו כב"ש ואינו נכון כלל ואחרים העמידו ברייתא זו כב"ש וגם זה שבוש. 
הא דתנן \textbf{ומקפת וקורא לה שם.} פי' רש"י ז"ל קוצה לה חלה עד שלא תקרא שם ומנחתה בכלי ומקפת מקרבת הכלי אצל העיסה, ולא צריך לישך (אלא מנחתה בכלי ומקפת מקרבת הכל אצל העיסה).\par ולא נראה לי שאין נקרא מוקף אלא בנשוך, אי נמי לרבי אליעזר בכלי אחד שהכלי מצרפן כדתנן שתי עיסות אחת טהורה ואחת טמאה נוטל כדי חלה מעיסה שלא הורמה חלתה ונותנין פחות מכביצה באמצע כדי ליטול מן המוקף שמע מינה נוגעין ממש בעינן למוקף. 
והא דקתני \textbf{בית שמאי אומרים צריכה טבילה.} לתרומה מפני שטבולת יום ארוך הוא והסיחה דעתה מן התרומה ואם ישראלית היא טובלת לביאת מקדש. פירש לפיכך טובלת כדי שתכנס לאתר כפרתה למקדש שהרי מחוסר כפורים שנכנס למקדש ענוש כרת כדאמרונן במס' מכות (דף ח ע"ב) טמא יהיה לרבות טבול יום עוד טמאתו בו לרבות מחוסר כפורים, וכן היא צריכה לטבול לנגיע' דתרומה, וב"ה אומרי' אינה צריכה אבל לקדשים מודי ב"ה דקיי"ל (חגיגה כא, א) האונן והמחוסר כפורים צריכין טבילה לקדש דמעלות דרבנן נינהו למדנו לדברי רש"י ז"ל שהחמירו באכילות קדשים יותר מביאת המקדש שיבנה במהרה בימינו אמן וכן יהי רצון. 
}

\addtocontents{toc}{\protect\end{multicols}}
\end{document}
