\documentclass[12pt, openany]{book}
\usepackage[
paperheight=8.5in,
paperwidth=5.5in,
top=0.5in,
bottom=0.5in,
inner=0.7in,
outer=0.5in,
marginparsep=0.1in,
headsep=16pt
]{geometry}

\newcommand{\texttitle}{מסכת נידה}\usepackage{titlesec}
\usepackage{resources/unnumberedtotoc}

\usepackage{fancyhdr}
\pagestyle{fancy}
\fancyhf{}
\fancyhead[LO,RE]{\thepage}
\fancyhead[CO]{\chapname}
\fancyhead[CE]{\texttitle}

\usepackage{paracol}
\usepackage{anyfontsize}
\usepackage{ragged2e}
\usepackage{polyglossia}
\usepackage{multicol}
\usepackage{hyperref}
\usepackage[marginal]{footmisc}
\usepackage[titles]{tocloft}
\usepackage{xifthen}
\usepackage{graphicx}

\setdefaultlanguage{hebrew}
\setotherlanguage{english}
\usepackage{fontspec}
\setmainfont{Times New Roman}
\newfontfamily\englishfont{Times New Roman}

\newcommand{\sethebfont}{
\fontsize{10.5pt}{21.0pt} \selectfont
}

\newcommand{\hebeng}[2]{
	{\sethebfont #1\\}
	
	\begin{english}
		#2
	\end{english}
	\clearpage
}

\newcommand{\twocol}[1]{
	{\sethebfont \begin{multicols}{2}
			#1
	\end{multicols}}	
}

\newcommand{\textblock}[1]{
{\sethebfont #1\\}	
}

\setlength{\parskip}{8pt}
\setlength\parindent{0in}

\newcommand{\chapname}{}
\newcommand{\sectname}{}

\newcommand{\newchap}[1]{
	\addcontentsline{toc}{chapter}{#1}
	\renewcommand{\chapname}{#1}
		\begin{center}
			\textbf{%
\fontsize{16pt}{16pt}\selectfont
				#1}
		\end{center}
}

\let\footnoterule\relax

\setlength{\columnsep}{0.25in}

\newcommand{\newsection}[1]{
	%\addcontentsline{toc}{section}{#1}
	\renewcommand{\sectname}{#1}	
	\vspace{-\baselineskip}
	\begin{center}
		\textbf{%
\fontsize{16pt}{16pt}\selectfont
			#1}
	\end{center}
	\vspace{-\baselineskip}
	\nopagebreak
}

\newcommand{\footnotecomment}[1]{
	\renewcommand\thefootnote{}
	\footnote{#1}}

\newcommand{\parencomment}[1]{\footnotesize (#1)}

\newcommand{\blockcomment}[2]{ 
\newsection{#1}
\sethebfont	#2}

\newcommand{\commenta}[1]{\footnotecomment{#1}}

\begin{document}
\frontmatter
\pagenumbering{roman}

\newcommand{\oneline}[1]{%
	\newdimen{\namewidth}%
	\setlength{\namewidth}{\widthof{#1}}%
	\ifthenelse{\lengthtest{\namewidth < \textwidth}}%
	{#1}% do nothing if shorter than text width
	{\resizebox{\textwidth}{!}{#1}}% scale down
}

\title{\oneline{\hspace*{0.5in}\texttitle\hspace*{0.5in}}}

\author{}

\date{}

\maketitle

\begin{minipage}[b][\textheight][b]{\textwidth}\englishfont\footnotesize
	\begin{english}
		\vfill
		The following book includes:
\begin{itemize}
\item[$\bullet$] Mishnah, ed. Romm, Vilna 1913
\begin{itemize}
\item[$\bullet$] License: Public Domain
\item[$\bullet$] Source: \url{http://primo.nli.org.il/primo_library/libweb/action/dlDisplay.do?vid=NLI&docId=NNL_ALEPH001741739}
\end{itemize}
\item[$\bullet$] On Your Way
\begin{itemize}
\item[$\bullet$] License: CC-BY
\item[$\bullet$] Source: \url{http://mobile.tora.ws/}
\end{itemize}
\item[$\bullet$] Wikisource Talmud Bavli
\begin{itemize}
\item[$\bullet$] License: CC-BY
\item[$\bullet$] Source: \url{http://he.wikisource.org/wiki/%D7%AA%D7%9C%D7%9E%D7%95%D7%93_%D7%91%D7%91%D7%9C%D7%99}
\end{itemize}
\item[$\bullet$] Vilna Edition
\begin{itemize}
\item[$\bullet$] License: CC-BY
\item[$\bullet$] Source: \url{http://primo.nli.org.il/primo_library/libweb/action/dlDisplay.do?vid=NLI&docId=NNL_ALEPH001300957}
\end{itemize}
\item[$\bullet$] Chiddushei HaRamban, Jerusalem 1928\textendash 29
\begin{itemize}
\item[$\bullet$] License: Public Domain
\item[$\bullet$] Source: \url{http://primo.nli.org.il/primo_library/libweb/action/dlDisplay.do?vid=NLI&docId=NNL_ALEPH001294828}
\end{itemize}
\end{itemize}
		It was retrieved from Sefaria on \today\space \texthebrew{(\Hebrewtoday)}.  It was typeset and formatted by Ktavi.
		\clearpage
		
	\end{english}
\end{minipage}

\titleformat{\chapter}[hang]{\huge\bfseries}{\thechapter.}{1em}{}
\titlespacing*{\chapter}{0pt}{-3em}{1.1\parskip}
\titlelabel{\thetitle\quad}
%\addtocontents{toc}{\protect\vspace{-\baselineskip}}
\addtocontents{toc}{\protect\begin{multicols}{2}}
%\vspace*{-5\baselineskip}
{\small \tableofcontents}


\clearpage
\mainmatter
\pagenumbering{arabic}

\addpart{משנה נדה}
\newchap{פרק ב}
\twocol{כל היד המרבה לבדוק בנשים משובחת. ובאנשים תקצץ. החרשת. והשוטה. והסומא. ושנטרפה דעתה. אם יש להן פקחות. מתקנות אותן והן אוכלות בתרומה. דרך בנות ישראל משמשות בשני עדים אחד לו ואחד לה. הצנועות מתקנות להן שלישי לתקן את הבית: 

נמצא על שלו טמאין. וחייבין קרבן. נמצא על שלה אותיום. טמאין וחייבין קרבן. נמצא על שלה לאחר זמן. טמאים מספק ופטורים מן הקרבן: 

איזהו אחר זמן כדי שתרד מן המטה ותדיח פניה. ואחר כך מטמאה מעת לעת ואינה מטמאה את בועלה. ר' עקיבא אומר אף מטמאה את בועלה. ומודים חכמים לר' עקיבא ברואה כתם שמטמאה את בועלה: 

כל הנשים בחזקת טהרה לבעליהן. הבאין מן הדרך. נשיהן להן בחזקת טהרה. בית שמאי אומרים צריכה שני עדים על כל תשמיש ותשמיש. או תשמש לאור הנר. בית הלל אומרים דיה בשני עדים כל הלילה: 

משל משלו חכמים באשה. החדר. והפרוזדור. והעלייה. דם החדר. טמא. נמצא בפרוזדור ספקו טמא. לפי שחזקתו מן המקור: 

חמשה דמים טמאים באשה. האדום. והשחור. וכקרן כרכום. וכמימי אדמה. וכמזוג. בית שמאי אומרים אף כמימי תלתן וכמימי בשר צלי. ובית הלל מטהרין. הירוק. עקביא בן מהללאל מטמא. וחכמים מטהרים. אמר רבי מאיר אם אינו מטמא משום כתם מטמא משום משקה. רבי יוסי אומר לא כך ולא כך: 

איזהו אדום. כדם המכה. שחור כחרת. עמוק מכן. טמא. דיהה מכן. טהור. וכקרן כרכום כברור שבו. וכמימי אדמה. מבקעת בית כרם. ומציף מים. וכמזוג. שני חלקים מים. ואחד יין. מן היין השרוני: 
\clearpage}

\blockcomment{ברטנורא על משנה נדה}{\textbf{נמצא דם על שלו.} אפילו שהה זמן ארוך לאחר בעילה קודם קינוח, בידוע שהיה דם בשעת תשמיש:\\\textbf{ותדיח את פניה.} קינוח שלמטה:\\\textbf{כל הנשים בחזקת טהרה לבעליהן.} ואין צריך לשאול לה אם היא טהורה. והני מילי, כשהניחה בחזקת טהורה:\\\textbf{החדר והעליה והפרוזדור.} חדר מבפנים, ופרוזדור מבחוץ, שניהם זה אצל זה, חדר לצד אחוריה, ופרוזדור לפניה, ועלייה בנויה על שתיהן, ונקב יש בין עליה לפרוזדור ונקרא לול. ופעמים יורד דם מן העליה לפרוזדור דרך הלול. וכותלי רחם, למטה באמצע פרוזדור, ודרך שם דמים יוצאים:\\\textbf{חמשה דמים טמאים.} דשתי פעמים כתיב דמיה בקרא בטומאת דמים שבאשה, (ויקרא כ׳:י״ח,) והיא גילתה את מקור דמיה, (שם י״ב) וטהרה ממקור דמיה, ומיעוט דמיה שנים, הרי ארבעה דמים. אודם, וקרן כרכום, ומימי אדמה, ומזוג. והשחור בכלל אדום הוא, והכי אמרינן בגמרא, האי שחור אדום הוא, אלא שלקה:\\\textbf{כדם המכה.} כדם של שור שחוט שהוא אדום הרבה, ובתחילת הכאתה של סכין, ולא כדם של כולה שחיטה, שדם שחיטה משתנה והולך. ואית דאמרי, כדם המכה של הקזה:\\}
\newchap{פרק ג}
\twocol{המפלת חתיכה. אם יש עמה דם טמאה. ואם לאו. טהורה. רבי יהודה אומר. בין כך ובין כך. טמאה: 

המפלת כמין קליפה. כמין שערה. כמין עפר. כמין יבחושים אדומים. תטיל למים. אם נמוחו. טמאה. ואם לאו. טהורה. המפלת כמין דגים. חגבים. שקצים ורמשים. אם יש עמהם דם. טמאה. ואם לאו. טהורה. המפלת מין בהמה חיה ועוף. בין טמאין. בין טהורים. אם זכר. תשב לזכר. ואם נקבה. תשב לנקבה. ואם אין ידוע תשב לזכר ולנקבה דברי רבי מאיר. וחכמים אומרים. כל שאין בו מצורת אדם. אינו ולד: 

המפלת שפיר. מלא מים. מלא דם. מלא גנונים. אינה חוששת לולד. ואם היה מרוקם. תשב לזכר ולנקבה: 

המפלת סנדל. או שליא. תשב לזכר ולנקבה. שליא בבית הבית טמא. לא שהשליא ולד. אלא שאין שליא בלא ולד. רבי שמעון אומר נימוק הולד עד שלא יצא: 

המפלת טומטום ואנדרוגינוס. תשב לזכר ולנקבה. טומטום וזכר אנדרוגינוס וזכר. תשב לזכר ולנקבה. טומטום ונקבה. אנדרוגינוס ונקבה. תשב לנקבה בלבד. יצא מחותך או מסורס. משיצא רובו הרי הוא כילוד. יצא כדרכו. עד שיצא רוב ראשו. ואיזהו רוב ראשו. משיצא פדחתו: 

המפלת ואין ידוע מה הוא. תשב לזכר ולנקבה. אין ידוע אם ולד היה. אם לאו. תשב לזכר. ולנקבה. ולנדה: 

המפלת ליום ארבעים. אינה חוששת לולד. ליום ארבעים ואחד. תשב לזכר. ולנקבה. ולנדה. ר' ישמעאל אומר. יום ארבעים ואחד. תשב לזכר ולנדה. יום שמונים ואחד. תשב לזכר ולנקבה. ולנדה. שהזכר נגמר לארבעים ואחד. והנקבה לשמונים ואחד. וחכמים אומרים. אחד בריית הזכר. ואחד בריית הנקבה. זה וזה לארבעים ואחד: 
\clearpage}

\blockcomment{ברטנורא על משנה נדה}{\textbf{כמין שערה.} שער:\\\textbf{שפיר.} חתיכה של בשר. ואם יש בה צורת אדם נקרא שפיר מרוקם. ואני שמעתי שהוא כשפופרת של ביצה, לפיכך נקרא שפיר:\\\textbf{סנדל.} חתיכת בשר עשויה כדמות לשון של שור. ומפני שיש לה צורת סנדל קורין לה סנדל. והוא רגיל לבוא עם ולד. ויש מפרשים, סנדל, שנאוי ולד [צ״ל ודל]:\\\textbf{תשב לזכר ולנקבה.} טמאה שבועים כנקבה, וימי טוהר שלה שלשים ושלשה יום בלבד כזכר:\\\textbf{ואין ידוע מה הוא.} אם זכר אם נקבה:\\\textbf{המפלת ליום ארבעים.} לטבילתה:\\}
\newchap{פרק ד}
\twocol{בנות כותים. נדות מעריסתן. והכותים מטמאים משכב תחתון כעליון. מפני שהן בועלי נדות. והן יושבות על כל דם ודם. ואין חייבין עליה על ביאת מקדש. ואין שורפין עליהם את התרומה. מפני שטומאתן ספק: 


בנות צדוקין. בזמן שנהגו ללכת בדרכי אבותיהן. הרי הן ככותיות. פרשו ללכת בדרכי ישראל. הרי הן כישראלית. ר' יוסי אומר. לעולם הן כישראל. עד שיפרשו ללכת בדרכי אבותיהן: 


דם עובדת כוכבים. ודם טהרה של מצורעת. בית שמאי מטהרים. ובית הלל אומרים כרוקה. וכמימי רגליה. דם יולדת שלא טבלה. בית שמאי אומרים כרוקה. וכמימי רגליה. ובית הלל אומרים מטמא לח ויבש. ומודים ביולדת בזוב. שהיא מטמאה לח ויבש: 


המקשה נדה. קשתה שלשה ימים בתוך אחד עשר יום. ושפתה מעת לעת. וילדה. הרי זו יולדת בזוב. דברי ר' אליעזר. ר' יהושע אומר לילה ויום כלילי שבת ויומו. ששפתה מן הצער. ולא מן הדם: 


כמה הוא קשויה. ר' מאיר אומר אפילו ארבעים וחמשים יום. ר' יהודה אומר דיה חדשה. ר' יוסי ור' שמעון אומרים אין קישוי יותר משתי שבתות: 


המקשה בתוך שמונים של נקבה כל דמים שהיא רואה טהורים עד שיצא הולד. ור' אליעזר מטמא. אמרו לו לר' אליעזר ומה במקום שהחמיר בדם השופי. הקל בדם הקושי. מקום שהקל בדם השופי. אינו דין שנקל בדם הקושי. אמר להן דיו לבא מן הדין להיות כנדון ממה הקל עליה מטומאת זיבה אבל טמאה טומאת נדה: 


כל אחד עשר יום בחזקת טהרה. ישבה לה ולא בדקה. שגגה. נאנסה. הזידה ולא בדקה. טהורה. הגיע שעת וסתה ולא בדקה. הרי זו טמאה. ר' מאיר אומר. אם היתה במחבא והגיע שעת וסתה ולא בדקה. הרי זו טהורה. מפני שחרדה מסלקת את הדמים. אבל ימי הזב והזבה. ושומרת יום כנגד יום. הרי אלו בחזקת טומאה: 

\clearpage}

\blockcomment{ברטנורא על משנה נדה}{\textbf{פירשו ללכת בדרכי ישראל הרי הן כישראליות.} אבל סתמא, הרי הן ככותיות:\\\textbf{דם נכרית בית שמאי מטהרין.} ואע״ג דברוקה ובמימי רגליה מודו בית שמאי דרבנן גזרו עליהן להיות כזבין לכל דבריהן, אפילו הכי דמה טהור, דשיירו בה רבנן למהוי הכירא דטומאתן דרבנן, כי היכי דלא לשרפו עלייהו תרומה וקדשים:\\\textbf{המקשה נדה.} הכי קאמר, המקשה בלדתה בימי נדתה וראתה דם בהקשותה, הרי היא נדה, דלא טיהרה תורה לדם קושי אלא באחד עשר יום של ימי זיבה, אבל בימים של נדה לא טיהריה רחמנא:\\\textbf{כמה הוא קשויה.} דלא אתיא לידי זיבה בכל דמים שתראה:\\\textbf{המקשה בתוך שמונים של נקבה.} כגון ששמשה לאחר ארבעה עשר של ימי לידה ונתעברה והפילה בתוך שמונים:\\\textbf{כל אחד עשר יום.} שאחר שבעת ימי נדה:\\}
\addpart{נדה}
\newsection{דף מב}
\twocol{וכי תימא דילמא אשתייר אי הכי חיישינן שמא נשתייר מבעי ליה 
\commenta{אלא לרבא נמי - דלא אזלה הילכך כל שעה פלטה והא דר' שמעון דיכולה לטבול לשמושה כגון שלא נתהפכה לאחר הטבילה ודרבא במתהפכת:}%endcomment
אלא לרבא נמי שהטבילוה במטה ולא קשיא כאן במתהפכת כאן בשאינה מתהפכת 
\commenta{ורבא - לפרושי לן קרא אתא דכי כתיב ורחצו במים דמשמע דיכולה לטבול בו ביום כשאינה מתהפכת. ל"א מתהפכת בשעת תשמיש ואין הזרע נקלט ברחם יחד הילכך פולטת והולכת מעט מעט אינה מתהפכת בשעת תשמיש נקלט הזרע יחד ויוצא בבת אחת:}%endcomment
ורבא אקרא קאי והכי קאמר כי כתב רחמנא ורחצו במים וטמאו עד הערב בשאינה מתהפכת אבל במתהפכת כל שלשה ימים אסורה לאכול בתרומה שאי אפשר לה שלא תפלוט 
\commenta{רואה הויא - משום ראייה טמייה קרא ומשום נוגעת בשכבת זרע דעד השתא הויא נגיעת בית הסתרים ולא הויא טמאה אלא ביום תשמיש ומגזרת הכתוב וכי פלטה נגעה לבראי ומטמיא:}%endcomment
בעא מיניה רב שמואל בר ביסנא מאביי פולטת שכבת זרע רואה הויא או נוגעת הויא 
\commenta{לסתור - בזיבה בימי ספירה אי רואה הויא סותרת אבל נוגעת לא שאם נגעה בשום טומאה אינה סותרת דהא וספרה לה כתיב והאיכא:}%endcomment
נפקא מינה לסתור ולטמא במשהו ולטמא בפנים כבחוץ 
\commenta{אי שמיע ליה - לרב שמואל:}%endcomment
מה נפשך אי שמיע ליה מתניתין לרבנן רואה הויא ולר' שמעון נוגעת הויא 
\commenta{ואי לא שמיע ליה - אמאי מספקא ליה מסברא נוגעת הויא דהא לא מגופה חזיא אלא מה שקבלה היא פולטת:}%endcomment
ואי לא שמיע ליה מתניתין מסתברא נוגעת הויא 
\commenta{ואליבא דרבנן לא קמיבעיא ליה - דכיון דלרבנן מטמיא בפנים כבחוץ אלמא רואה הויא וכ"ש דמטמאה במשהו ולענין סתירה נמי סתרה דהא בתורת זיבה מרבי ליה מיהיה זובו:}%endcomment
לעולם שמיע ליה מתניתין ואליבא דרבנן לא קמיבעיא ליה כי קא מיבעיא ליה אליבא דר"ש 
\commenta{כי קא מבעיא ליה לסתור ולטמא במשהו - מהו מי אמרי' כי אמר ר' שמעון דיה כבועלה לטמא בפנים כבחוץ מה בועלה לא מטמא איהי נמי לא מטמאה אבל לסתור ולטמא במשהו לא משוי לה כנוגעת דהא א"נ אמר דיה כבועלה סתרה ומטמאה במשהו כבועלה או דילמא לא שנא דכי היכי דלענין טומאת פנים משוי לה נוגעת לענין משהו וסתירה נמי משוי לה נוגעת:}%endcomment
ולטמא בפנים כבחוץ לא קמיבעיא ליה כי קמיבעיא ליה לסתור ולטמא בכל שהוא מאי 
כי קאמר רבי שמעון דיה כבועלה הני מילי לטמויי בפנים כבחוץ אבל לסתור ולטמא בכל שהוא רואה הויא או דילמא לא שנא 
\commenta{לעולם לא שמיע ליה מתני' - ודקאמרינן מסברא נוגעת הויא לא היא מסברא לא מיפשט לן דאיכא למימר מדאחמיר רחמנא בבעלי קריין בסיני דכתיב (שמות י״ט:ט״ו) אל תגשו אל אשה ולא הזהיר על טמא שרץ רואה הויא:}%endcomment
איכא דאמרי לעולם לא שמיע ליה מתניתא והכי קמיבעיא ליה מדאחמיר רחמנא אבעלי קריין בסיני רואה הויא 
או דילמא לא גמרינן מסיני דחדוש הוא דהא זבין ומצורעים דחמירי ולא אחמיר בהו רחמנא 
\commenta{כולכו ברוקא חדא תפיתו - כולכם שמעתם שמועה אחת. תפיתו לשון רוק וחבירו בכתובות תוף שדי בפרק אף על פי (כתובות דף סא:):}%endcomment
א"ל רואה הויא אתא שייליה לרבא א"ל רואה הויא אתא לקמיה דרב יוסף א"ל רואה הויא הדר אתא לקמיה דאביי א"ל כולכו ברוקא חדא תפיתו 
\commenta{אמר ליה - אביי שפיר אמר לך:}%endcomment
אמר ליה שפיר אמרי לך עד כאן לא קאמר ר"ש דיה כבועלה אלא לטמא בפנים כבחוץ אבל לסתור ולטמא בכל שהוא רואה הויא
\commenta{הזבה ושומרת יום - הכי קאמר בין שראתה ג' ימים בין שראתה יום אחד מטמאה בבית החיצון בראייה ראשונה:}%endcomment
ת"ר הנדה והזבה והשומרת יום כנגד יום והיולדת כולן מטמאות בפנים כבחוץ 
\commenta{אי בימי נדה - ילדה וראתה הא תני ליה ואי בימי זיבה הא תני ליה לזיבה:}%endcomment
בשלמא כולהו לחיי אלא יולדת אי בימי נדה נדה אי בימי זיבה זיבה 
\commenta{מטומאה לטהרה - לאחר שבועים וקיימא לן כבית הלל דאמרי בבנות כותים (לעיל נדה דף לה:) ביומי וטבילה תלא רחמנא דלא הוי דם טוהר אלא לאחר טבילה:}%endcomment
לא צריכא שירדה לטבול מטומאה לטהרה 
\commenta{ונעקר דם - לבית החיצון ועמד שם יום או יומים:}%endcomment
וכי הא דאמר רבי זירא א"ר חייא בר אשי אמר רב יולדת שירדה לטבול מטומאה לטהרה ונעקר ממנה דם בירידה טמאה בעלייה טהורה 
\commenta{טומאה בלועה היא - וקיימא לן בפרק בהמה המקשה (חולין דף עא.) טומאה בלועה אינה מטמאה לא במגע ולא במשא והכא הא טומאה בלועה היא ולא דמיא לכל הנשים שמטמאות בבית החיצון דהתם משום ראייה דמאתמול היא וטומאת שבעה ורחמנא אמר בבשרה מכי אתא בבית החיצון הויא ראייה אבל הכא משום משא ונגיעה היא וטומאת ערב דאי משום ראייה מאתמול איעקר ומאתמול הויא ראייה וראייה דמאתמול לא מהניא לה האידנא דמאתמול הוה להו ימי טומאה והשתא הוו להו ימי טוהר וטומאה בלועה לא מטמיא במגע ובמשא:}%endcomment
א"ל רבי ירמיה לר' זירא בירידה אמאי טמאה טומאה בלועה היא א"ל זיל שייליה לרבי אבין דאסברית ניהליה וכרכיש לי ברישיה בי מדרשא 
\commenta{בגדים בבית הבליעה - בגדים שהוא לבוש בהן בשעה שנוגע בה אינן טמאין שאין להן טומאת מגע דאמר רחמנא לא יאכל לטמאה בה (ויקרא כ״ב:ח׳) אין לך אלא מה שאמור בה אכילה ולא מגע ומה היא מטמאה בגדים שהוא לבוש בשעת בליעה כדכתיב (שם יז) [וכל] נפש אשר תאכל נבלה וטריפה באזרח ובגר וכבס בגדיו וטהר ובנבלת עוף טהור כתיב דנבלת בהמה לא מטמאה בבית הבליעה כדאמר בשמעתין ונבלת עוף טמא נמי לא מטמאה כדאמר בתורת כהנים מי שאיסורו משום בל תאכל נבלה יצא עוף טמא שאין איסורו משום בל תאכל נבלה אלא משום בל תאכל טמא:}%endcomment
אזל שייליה א"ל עשאוה כנבלת עוף טהור שמטמאה בגדים בבית הבליעה מי דמי
התם אין לה טומאה בחוץ הכא כי נפיק לבראי ליטמי הכא נמי כשיצא לחוץ 
\commenta{מאי למימרא - הא דם טמא הוא דקודם טבילה נעקר:}%endcomment
אי יצא לחוץ מאי למימרא מהו דתימא מגו דמהני טבילה לדם דאיכא גואי תהני נמי להאי קמ"ל 
\commenta{שמעתין - דרבי זירא דהוה קשיא לן טומאה בלועה היא:}%endcomment
שמעתין איפריק אלא יולדת אי בימי נדה נדה אי בימי זיבה זיבה 
\commenta{בלידה יבישתא - דליכא לא נדה ולא זיבה:}%endcomment
הכא במאי עסקינן בלידה יבשתא לידה יבשתא מאי מטמא בפנים כבחוץ איכא 
\commenta{כגון שהוציא ולד ראשו חוץ לפרוזדור - ואוליד וקאמר דאע"ג דאכתי ראשו בפנים הוא דהיינו בבית החיצון הוי כילוד וטמאה לידה וכרב אושעיא דאמר חוץ לפרוזדור ולד הוי:}%endcomment
כגון שהוציא ולד ראשו חוץ לפרוזדור וכדרב אושעיא דאמר רב אושעיא גזרה שמא יוציא הולד ראשו חוץ לפרוזדור 
\commenta{וכי ההוא דאתא כו' - כלומר מדרבא נמי שמעינן דראשו חוץ לפרוזדור הוי כילוד:}%endcomment
וכי ההוא דאתא לקמיה דרבא אמר ליה מהו לממהל בשבתא אמר ליה שפיר דמי בתר דנפק אמר רבא ס"ד דההוא גברא לא ידע דשרי לממהל בשבתא אזל בתריה אמר ליה אימא לי איזי גופא דעובדא היכי הוה 
\commenta{דצויץ - שמעתי את הולד צועק בתוך המעיים מבעוד יום של ערב שבת:}%endcomment
אמר ליה שמעית ולד דצויץ אפניא דמעלי שבתא ולא אתיליד עד שבתא אמר ליה האי הוציא ראשו חוץ לפרוזדור הוא והוי מילה שלא בזמנה וכל מילה שלא בזמנה אין מחללין עליה את השבת 
איבעיא להו אותו מקום של אשה בלוע הוי או בית הסתרים הוי 
\commenta{טומאה בלועה לא מטמאה - דהרי היא כמי שאינה ואין בה לא מגע ולא משא וטהורה אשה זו:}%endcomment
למאי נפקא מינה כגון שתחבה לה חבירתה כזית נבלה באותו מקום אי אמרת בלוע הוי טומאה בלועה לא מטמאה ואי אמרת בית הסתרים הוי נהי דבמגע לא מטמיא במשא מיהא מטמיא 
אביי אמר בלוע הוי רבא אמר בית הסתרים הוי אמר רבא מנא אמינא לה דתניא אלא מפני שטומאת בית הסתרים היא
וטומאת בית הסתרים לא מטמאה אלא שגזרת הכתוב היא 
ואביי חדא ועוד קאמר חדא דטומאה בלועה היא ועוד אפילו אם תמצי לומר טומאת בית הסתרים היא אינה מטמאה אלא שגזרת הכתוב היא 
\commenta{מקום נבלת עוף טהור - בית הבליעה של אדם שאין לנבלת עוף טהור טומאה אלא שם:}%endcomment
איבעיא להו מקום נבלת עוף טהור בלוע הוי או בית הסתרים הוי 
\commenta{שתחב לו חבירו - דאי תחב הוא לעצמו איטמי ליה במגע:}%endcomment
למאי נפקא מינה כגון שתחב לו חבירו כזית נבלה לתוך פיו אי אמרת בלוע הוי טומאה בלועה לא מטמיא (אלא אי) אמרת בית הסתרים הוי נהי נמי דבמגע לא מטמא במשא מיהא מטמא
\commenta{בה - בגדים טמאים בבית הבליעה ולא באחרת:}%endcomment
אביי אמר בלוע הוי ורבא אמר בית הסתרים הוי אמר אביי מנא אמינא לה דתניא יכול תהא נבלת בהמה מטמאה בגדים אבית הבליעה ת"ל (ויקרא כב, ח) נבלה וטרפה לא יאכל לטמאה בה
מי שאין לה טומאה אלא אכילתה יצתה זו שטמאה קודם שיאכלנה 
ותיתי בק"ו מנבלת עוף טהור ומה נבלת עוף טהור שאין לה טומאה בחוץ יש לה טומאה בפנים זו שיש לה טומאה בחוץ אינו דין שיש לה טומאה בפנים 
אמר קרא בה בה ולא באחרת 
אם כן מה תלמוד לומר {ויקרא יא } והאוכל 
ליתן שיעור לנוגע ולנושא כאוכל מה אוכל בכזית אף נוגע ונושא בכזית 
\commenta{קומטו - כגון בין אציליו או שאר קמטים פרונצ"ש בלע"ז:}%endcomment
אמר רבא שרץ בקומטו טהור נבלה בקומטו טמא 
שרץ בקומטו טהור שרץ בנגיעה הוא דמטמא ובית הסתרים לאו בר מגע הוא נבלה בקומטו טמא נהי דבמגע לא מטמא במשא מיהא מטמא 
שרץ בקומטו והכניסו לאויר התנור טמא פשיטא מהו דתימא תוכו אמר רחמנא
\clearpage}

\newsection{דף מג}
\twocol{ולא תוך תוכו קמ"ל 
\commenta{והסיט בו את הטהור טהור - כדמפרש טעמא לקמן:}%endcomment
אמר ר"ל קנה בקומטו של זב והסיט בו את הטהור טהור קנה בקומטו של טהור והסיט בו את הזב טמא 
\commenta{מאי טעמא - טהור הניסט בקנה של קומטו של זב טהור:}%endcomment
מאי טעמא דאמר קרא (ויקרא טו, יא) וכל אשר יגע בו הזב וידיו לא שטף במים זהו הסיטו של זב שלא מצינו לו טומאה בכל התורה כולה 
ואפקיה רחמנא בלשון נגיעה למימרא דהיסט ונגיעה כידיו מה התם מאבראי אף הכא מאבראי
אבל הזב ובעל קרי אינן מטמאין וכו' זב דכתיב (ויקרא טו, ב) כי יהיה זב מבשרו עד שיצא זובו מבשרו בעל קרי דכתיב (ויקרא טו, טז) ואיש כי תצא ממנו שכבת זרע
\commenta{כאילו מביא מבול - שמחמם את אבריו ומביא לידי קרי וזהו קלקול דור המבול שנאמר (בראשית ו׳:י״ב) כי השחית כל בשר:}%endcomment
היה אוכל בתרומה והרגיש וכו' אוחז והתניא ר"א אומר כל האוחז באמה ומשתין כאילו מביא מבול לעולם 
\commenta{במטלית עבה - שאינה מחממת:}%endcomment
אמר אביי במטלית עבה רבא אמר אפילו תימא במטלית רכה כיון דעקר עקר ואביי חייש דילמא אתי לאוסופי ורבא לאוסופי לא חייש 
והתניא למה זה דומה לנותן אצבע בעין שכל זמן שאצבע בעין מדמעת וחוזרת ומדמעת 
\commenta{לא שכיח - והא דתניא כל זמן שאצבע בעין מדמעת ה"מ היכא דמתחלה ממשמש וחימם את עצמו ויוצא כל שעה מעט אבל זה שנזדעזעו אבריו נעקר כל הזרע ביחד ושוב אינו חוזר ומתחמם לעקור עוד מיד בשעה אחת:}%endcomment
ורבא כל אחמומי והדר אחמומי בשעתא לא שכיח 
אמר שמואל כל שכבת זרע שאין כל גופו מרגיש בה אינה מטמאה מ"ט שכבת זרע אמר רחמנא בראויה להזריע 
\commenta{בשרו חם - אברו:}%endcomment
מיתיבי היה מהרהר בלילה ועמד ומצא בשרו חם טמא תרגמא רב הונא במשמש מטתו בחלומו דאי אפשר לשמש בלא הרגשה 
לישנא אחרינא אמר שמואל כל שכבת זרע שאינו יורה כחץ אינה מטמאה מאי איכא בין האי לישנא להאי לישנא איכא בינייהו נעקרה בהרגשה ויצאה שלא בהרגשה 
\commenta{ויצתה בלא הרגשה - דאינו יורה כחץ:}%endcomment
מילתא דפשיטא ליה לשמואל מיבעיא ליה לרבא דבעי רבא נעקרה בהרגשה ויצתה שלא בהרגשה מהו 
\commenta{לכשיטיל מים טמא - דנשתייר מן הקרי ויוצא עכשיו ואע"ג דלא נפיק בהרגשה הואיל ומעיקרא איעקר בהרגשה:}%endcomment
ת"ש בעל קרי שטבל ולא הטיל מים לכשיטיל מים טמא שאני התם דרובה בהרגשה נפק 
לישנא אחרינא אמרי לה אמר שמואל כל שכבת זרע שאינו יורה כחץ אינה מזרעת אזרועי הוא דלא מזרעא הא טמויי מטמיא שנאמר (דברים כג, יא) כי יהיה בך איש אשר לא יהיה טהור מקרה אפילו קרי בעולם 
\commenta{עובד כוכבים שהרהר - ונעקר ולא יצא:}%endcomment
בעי רבא עובד כוכבים שהרהר וירד וטבל מהו 
\commenta{או דילמא לא שנא - לחומרא כגון לגבי עקירה דהרגשה דישראל ולא שנא לקולא כי הכא אזלינן בתר עקירה:}%endcomment
אם תמצי לומר בתר עקירה אזלינן הני מילי לחומרא אבל הכא דלקולא לא אמרינן או דילמא לא שנא תיקו 
\commenta{שנעקרו מימי רגליה - דהוי מעין אב הטומאה:}%endcomment
בעי רבא זבה שנעקרו מימי רגליה וירדה וטבלה מהו 
\commenta{דלא מצי נקיט לה - מלצאת הילכך מכי עקר חשיב ליה צורך יציאה:}%endcomment
אם תמצא לומר בתר עקירה אזלינן הני מילי שכבת זרע דלא מצי נקיט לה אבל מימי רגליה דמצי נקיט לה לא או דילמא לא שנא תיקו 
\commenta{[עובדת כוכבים] - סתם עובדת כוכבים מי רגליה מטמאין דחכמים גזרו עליהן שיהו כזבין לכל דבריהם:}%endcomment
בעי רבא עובדת כוכבים זבה שנעקרו מימי רגליה
וירדה וטבלה מהו 
אם תמצי לומר בתר עקירה אזלינן אע"ג דמצי נקיט להו ה"מ ישראלית דטמאה דאורייתא אבל עובדת כוכבים זבה דטמאה דרבנן לא או דילמא לא שנא תיקו
ומטמאין בכל שהן אמר שמואל זב צריך כחתימת פי האמה שנאמר (ויקרא טו, ג) או החתים בשרו מזובו 
והאנן תנן מטמאין בכל שהן הוא דאמר כרבי נתן דתניא רבי נתן אומר משום רבי ישמעאל זב צריך כחתימת פי האמה ולא הודו לו 
מ"ט דרבי ישמעאל דאמר קרא או החתים בשרו מזובו 
\commenta{לח מטמא - דהחתים משמע לח שהיבש אינו חותם פי האמה דנפיל:}%endcomment
ורבנן ההוא מבעי ליה לח מטמא ואינו מטמא יבש 
ורבי ישמעאל ההוא מרר נפקא 
\commenta{רר - משמע לח כמו (שמוא"ל א כא) ויורד רירו:}%endcomment
ורבנן ההוא למנינא הוא דאתא זובו חדא רר בשרו תרי את זובו תלת לימד על זב בעל שלש ראיות שחייב בקרבן 
\commenta{או החתים - לח:}%endcomment
או החתים בשרו מזובו טמא מקצת זובו טמא לימד על זב בעל שתי ראיות שמטמא משכב ומושב ורבי ישמעאל מנינא מנא ליה נפקא ליה מדרבי סימאי
\commenta{שתים וקראו טמא - כי יהיה זב מבשרו חדא זובו טמא תרי טמא:}%endcomment
דתניא רבי סימאי אומר מנה הכתוב שתים וקראו טמא שלש וקראו טמא הא כיצד שתים לטומאה ושלש לקרבן 
ולמאן דנפקא ליה תרוייהו {ויקרא טו } מזאת תהיה טומאתו בזובו (ויקרא טו, ב) איש איש כי יהיה זב מבשרו מאי עביד ליה מבעי ליה עד שיצא מבשרו 
\commenta{על הזוב - שהטיפה עצמה טמאה:}%endcomment
זובו טמא למה לי לימד על הזוב שהוא טמא 
\commenta{לנוגע בכעדשה - דמנגיעה דשרץ נפקא לן נגיעה דש"ז לקמן מאו איש אשר יגע וגו':}%endcomment
אמר רב חנילאי משום ר"א בר"ש שכבת זרע לרואה במשהו לנוגע בכעדשה והאנן מטמאין בכל שהן תנן מאי לאו לנוגע לא לרואה 
\commenta{אין לו חלוקת טומאה - דשרץ קטן מטמא כגדול אבל קטן פחות מבן ט' אין לו קרי:}%endcomment
ת"ש חומר בשכבת זרע מבשרץ וחומר בשרץ מבשכבת זרע חומר בשרץ שהשרץ אין חלוקה טומאתו מה שאין כן בשכבת זרע חומר בשכבת זרע שהשכבת זרע מטמא בכל שהוא מה שאין כן בשרץ 
מאי לאו לנוגע  לא לרואה 
\commenta{שום שרץ קתני - אין לך שרץ מטמא במשהו אבל ש"ז יש מטמא במשהו כגון לרואה:}%endcomment
והא דומיא דשרץ קתני מה שרץ בנגיעה אף שכבת זרע בנגיעה אמר רב אדא בר אהבה שום שרץ קתני ושום שכבת זרע קתני 
ושרץ לא מטמא במשהו והא אנן תנן האברים אין להם שיעור פחות מכזית בשר המת ופחות מכזית בשר נבלה ופחות מכעדשה מן השרץ 
\commenta{דכוליה במקום כעדשה קאי - כלומר מתוך חשיבותיה שהוא אבר הוי חשוב במקום כעדשה:}%endcomment
שאני אבר דכוליה במקום עדשה קאי דהא אילו חסר פורתא אבר מי קמטמיא 
\commenta{עכבר דים - אינו מטמא כדתניא בהעור והרוטב (חולין דף קכו:) דעל הארץ כתיב:}%endcomment
שכבת זרע דחלוקה טומאתו מאי היא אילימא בין ישראל לדנכרים ה"נ איכא עכבר דים ועכבר דיבשה 
אלא בין קטן לגדול 
\commenta{כתנאי - נוגע אי במשהו אי בכעדשה:}%endcomment
אמר רב פפא כתנאי מנין לרבות נוגע בש"ז ת"ל (ויקרא כב, ד) או איש 
\commenta{ושמעי' בעלמא תנאי דפליגי בכמה דוכתי בדון מינה ומינה ובדון מינה ואוקי באתרא הילכך הנהו תנאי על כרחך פליגי בנוגע בקרי למ"ד דון מינה ומינה דון ש"ז משרץ דהא מקרא דשרץ ילפינן לה דמה שרץ מטמא בנגיעה אף ש"ז בנגיעה ומינה מה שרץ בכעדשה כו':}%endcomment
ופליגי תנאי בעלמא דאיכא דאמרי דון מינה ומינה ואיכא דאמרי דון מינה ואוקי באתרא 
למ"ד דון מינה ומינה מה שרץ בנגיעה אף שכבת זרע בנגיעה ומינה מה שרץ בכעדשה אף ש"ז בכעדשה 
ולמ"ד דון מינה ואוקי באתרא מה שרץ בנגיעה אף ש"ז בנגיעה ואוקי באתרא מה ש"ז לרואה במשהו אף לנוגע במשהו 
\commenta{דילמא מאו איש אשר תצא ממנו כו' ילפינן לה - ואפי' למ"ד דון מינה ומינה הוה במשהו דהא נגיעה מראיה יליף ודכ"ע דון מינה ומינה הוה דהכא ליכא פלוגתא דאוקי באתרא דהא מיניה וביה יליף:}%endcomment
א"ל רב הונא בריה דרב נתן לרב פפא ממאי דמאו איש דשרץ קמרבי ליה דילמא מאו איש אשר תצא ממנו שכבת זרע קמרבי ליה ודכ"ע דון מינה ומינה 
\commenta{איכא דתני כרב פפא - מאו איש אשר יגע בכל שרץ:}%endcomment
שיילינהו לתנאי איכא דתני כרב פפא ואיכא דתני כרב הונא בריה דרב נתן
{\large\emph{מתני׳}} תנוקת בת יום אחד מטמאה בנדה בת י' ימים מטמאה בזיבה
\commenta{מתני' וזוקק ליבום - אם נולד יום אחד קודם מיתת אחיו אבל נולד לאחר מיתת אחיו אינו זוקק דכתיב (דברים כ״ה:ה׳) כי ישבו אחים יחדו:}%endcomment
תנוק בן יום אחד מטמא בזיבה ומטמא בנגעים ומטמא בטמא מת וזוקק ליבום ופוטר מן היבום ומאכיל בתרומה ופוסל (את) [מן] התרומה
\clearpage}

\newsection{דף מד}
\twocol{ונוחל ומנחיל וההורגו חייב והרי הוא לאביו ולאמו ולכל קרוביו כחתן שלם 
\commenta{גמ' אשה - ואשה כי תהיה זבה וגו' והאי קרא בנדה משתעי:}%endcomment
{\large\emph{גמ׳}} מנהני מילי דת"ר אשה אין לי אלא אשה בת יום אחד לנדה מנין ת"ל ואשה
\commenta{בת עשרה ימים כו' - ואשה כי תזוב זוב דמה קא דריש:}%endcomment
בת י' ימים לזיבה מנא ה"מ דת"ר אשה אין לי אלא אשה בת י' ימים לזיבה מנין ת"ל ואשה
תינוק בן יום אחד כו' מנא הני מילי דת"ר (ויקרא טו, ב) איש איש מה ת"ל איש איש לרבות בן יום אחד שמטמא בזיבה דברי רבי יהודה 
רבי ישמעאל בנו של רבי יוחנן בן ברוקא אומר אינו צריך הרי הוא אומר (ויקרא טו, לג) והזב את זובו לזכר ולנקבה לזכר כל שהוא בין גדול בין קטן לנקבה כל שהיא בין גדולה בין קטנה אם כן מה ת"ל איש איש דברה תורה כלשון בני אדם
\commenta{אדם - משמע כל שהוא מין אדם אפילו קטן כדכתיב (במדבר ל״א:ל״ה) ונפש אדם מן הנשים וגו' ובקטנות משתעי קרא:}%endcomment
ומטמא בנגעים דכתיב (ויקרא יג, ב) אדם כי יהיה בעור בשרו אדם כל שהו
\commenta{נפשות - כל שיש בו נפש:}%endcomment
ומטמא בטמא מת דכתיב (במדבר יט, יח) ועל הנפשות אשר היו שם נפש כל דהו
וזוקק ליבום דכתיב (דברים כה, ה) כי ישבו אחים יחדיו אחים שהיה להם ישיבה אחת בעולם
ופוטר מן היבום (דברים כה, ה) ובן אין לו אמר רחמנא והא אית ליה
ומאכיל בתרומה דכתיב (ויקרא כב, יא) ויליד ביתו הם יאכלו בלחמו קרי ביה יאכילו בלחמו
ופוסל מן התרומה (ויקרא כב, יג) וזרע אין לה אמר רחמנא והא אית לה 
\commenta{כנעוריה - ושבה אל בית אביה כנעוריה כלומר אם דומה לימי נעורים ושבה ואם לאו לא תשוב:}%endcomment
מאי איריא זרע אפילו עובר נמי דכתיב כנעוריה פרט למעוברת 
צריכי דאי כתב רחמנא וזרע אין לה משום דמעיקרא חד גופא והשתא תרי גופי אבל הכא דמעיקרא חד גופא והשתא חד גופא אימא תיכול כתב רחמנא כנעוריה 
ואי כתב רחמנא כנעוריה משום דמעיקרא גופה סריקא והשתא גופה מליא אבל הכא דמעיקרא גופה סריקא והשתא גופה סריקא אימא תיכול צריכא 
\commenta{קראי אתרוץ - דתרוייהו צריכי:}%endcomment
קראי אתרוץ אלא מתניתין מאי אריא בן יום אחד אפי' עובר נמי אמר רב ששת הב"ע בכהן שיש לו שתי נשים אחת גרושה ואחת שאינה גרושה ויש לו בנים משאינה גרושה ויש לו בן יום אחד מן הגרושה
\commenta{ולאפוקי מדר' יוסי - דאמר במסכת יבמות בפרק אלמנה לכהן גדול (יבמות דף סז.) יש לו זכייה ואפילו הוא במעי אשה זרה הכשרה לכהונה פוסלה דקסבר כל זמן שהוא במעי זרה זר הוא וכל שכן האי דבמעי גרושה הוא דאפילו כשיולד לא יהא ראוי:}%endcomment
דפוסל בעבדי אביו מלאכול בתרומה ולאפוקי מדר' יוסי דאמר עובר נמי פוסל קמ"ל בן יום אחד אין עובר לא
\commenta{לאחיו מן האב - דהא אין ירושה אלא לקרובי האב:}%endcomment
נוחל ומנחיל נוחל ממאן מאביו ומנחיל למאן לאחיו מאביו אי בעי מאבוה לירתי ואי בעי מיניה לירתי 
\commenta{נוחל בנכסי האם - אם מתה אמו ביום שנולד הרי הוא יורשה וכשמת הוא באין אחיו מאביו ויורשין הימנו נכסי אמו דאי לאו איהו הוו שקלי ליה קרובי האם דבן נוחל את אמו. אבל האם אינה נוחלת את בנה וכל שכן קרובי האם שאין נוחלין אותו דכתיב (במדבר א׳:ב׳) למשפחותם לבית אבותם משפחת האם אינה קרויה משפחה:}%endcomment
אמר רב ששת נוחל בנכסי האם להנחיל לאחיו מן האב ודוקא בן יום אחד אבל עובר לא מ"ט דהוא מיית ברישא ואין הבן יורש את אמו
בקבר להנחיל לאחיו מן האב 
\commenta{ופרכס - עובר שבמעי אמו אחר מיתתה תלתא פרכוסי. אלמא היא מייתה ברישא:}%endcomment
איני והא הוה עובדא ופרכס עד תלת פרכוסי אמר מר בריה דרב אשי מידי דהוה אזנב הלטאה דמפרכסת 
\commenta{לומר שממעט כו' - מתניתין דקתני נוחל ומנחיל לומר שממעט בחלק בכורה כגון אם היה בן יום אחד כשמת אביו והיה לו אח בכור ואח פשוט ומת הוא אחר אביו נוחל הוא מאביו את חלק רביעי שבנכסיו ומנחילו לשני אחיו וחולקין הבכור והפשוט את חלקו של זה בשוה ונמצא ממעט בחלק הבכור דאי לאו איהו הוה שקיל בכור תרי תילתי כגון מששה זהובים ארבעה ופשוט נוטל ב' זהובים ועכשיו נוטל הבכור ג' זהובים כנגד שני אחין ומחלקו של זה ג' רבעי זהוב נמצאו ביד בכור ד' זהובים חסר רביע וביד פשוט ב' זהובים ורביע. ודוקא בן יום א' אבל עובר אינו ממעט כדאמר לקמן בן שנולד לאחר מיתת אביו אינו ממעט בחלק הבכורה ונוטל הבכור שליש בכורתו ובין שלשתן יחלקו ב' שלישים:}%endcomment
מר בריה דרב יוסף משמיה דרבא אמר לומר שממעט בחלק בכורה ואמר מר בריה דרב יוסף משמיה דרבא בן שנולד אחר מיתת האב אינו ממעט בחלק בכורה מאי טעמא (דברים כא, טו) וילדו לו בעינן 
\commenta{בכור שנולד אחר מיתת אביו - כגון מת אביו והאשה מעוברת תאומים אין היוצא ראשון נוטל פי שנים. בלאו תאומים לא משכחת לה דכיון דבכור לא נולד עד מות אביו פשוט מהיכן בא. אי נמי שהניח ב' נשים מעוברות:}%endcomment
בסורא מתנו הכי בפומבדיתא מתנו הכי אמר מר בריה דרב יוסף משמיה דרבא בכור שנולד לאחר מיתת אביו אינו נוטל פי שנים מאי טעמא {דברים כ״א:י״ז } יכיר בעינן והא ליכא
והלכתא ככל הני לישני דמר בריה דרב יוסף משמיה דרבא
וההורגו חייב דכתיב (ויקרא כד, יז) ואיש כי יכה כל נפש מ"מ
והרי הוא לאביו ולאמו ולכל קרוביו כחתן שלם למאי הלכתא אמר רב פפא לענין אבלות 
\commenta{כמאן דלא כר' שמעון - אכולה מתני' קאי דקא חשיב בן יום אחד בן קיימא:}%endcomment
כמאן דלא כרשב"ג דאמר כל ששהה שלשים יום באדם אינו נפל הא לא שהה ספק הוי הכא במאי עסקינן דקים ליה שכלו לו חדשיו
\commenta{מתני' בת ג' שנים - ביאתה ביאה לכל דבר:}%endcomment
{\large\emph{מתני׳}} בת שלש שנים ויום אחד מתקדשת בביאה ואם בא עליה יבם קנאה וחייבין עליה משום אשת איש
\commenta{ומטמאה את בועלה - אם היא נדה. אבל פחות מכן אע"ג שמטמאה במגע אינה מטמאה את בועלה משום בועל נדה לטומאת ז' אלא משום נוגע ולטומאת ערב:}%endcomment
ומטמאה את בועלה לטמא משכב תחתון כעליון 
\commenta{מן הפסולין - כגון חלל נתין וממזר ועבד ועובד כוכבים:}%endcomment
נשאת לכהן תאכל בתרומה בא עליה אחד מן הפסולין פסלה מן הכהונה בא עליה אחד מכל העריות האמורות בתורה מומתין עליה והיא פטורה 
פחות מכן כנותן אצבע בעין
\commenta{גמ' ערב ראש השנה - של שנה רביעית לרבי מאיר גופיה לא הוי בת ביאה עד יום אחרון של שלישית דהוא ערב ראש השנה ולרבנן עד למחר דמיקלע יום ראשון ברביעית:}%endcomment
{\large\emph{גמ׳}} ת"ר בת ג' שנים מתקדשת בביאה דברי רבי מאיר וחכ"א בת ג' שנים ויום אחד מאי בינייהו אמרי דבי רבי ינאי ערב ראש השנה איכא בינייהו 
\commenta{שלשים יום איכא בינייהו - לרבי מאיר מכי איקלע שלשים יום בשנה שלישית הויא בת ביאה ולרבנן עד דמיקלע שנה רביעית ואיכא בינייהו טובא:}%endcomment
ור' יוחנן אמר ל' יום בשנה חשובין שנה איכא בינייהו 
מיתיבי בת ג' שנים ואפי' בת שתי שנים ויום אחד מתקדשת בביאה דברי רבי מאיר וחכמים אומרים בת שלשה שנים ויום אחד
\clearpage}

\newsection{דף מה}
\twocol{בשלמא לר' יוחנן כי היכי דאיכא תנא דקאמר יום אחד בשנה חשוב שנה הכי נמי איכא תנא דאמר ל' יום בשנה חשובין שנה 
\commenta{אלא לרבי ינאי - דאמר לר"מ עד סוף שנה שלישית לאו בת ביאה היא קשיא:}%endcomment
אלא לר' ינאי קשיא קשיא
\commenta{הני בתולין - בתוך ג' שנים:}%endcomment
פחות מכאן כנותן אצבע בעין איבעיא להו הני בתולין מיזל אזלי ואתו או דלמא אתצודי הוא דלא מתצדי עד לאחר ג' 
\commenta{אי אמרת מיזל אזלי - בביאה בתוך ג' לגמרי והדר אתו לא מחזקינן לה כזונה דהא דלא הדור שהות הוא דלא הויא להו מתוך בעילה שבעל תדיר:}%endcomment
למאי נפקא מינה כגון שבעל בתוך ג' ומצא דם ובעל לאחר שלש ולא מצא דם אי אמרת מיזל אזלי ואתו שהות הוא דלא הויא להו
\commenta{אלא אי אמרת - לא אזלי בביאה תוך ג' הוה ליה למיתי דם לאחר ג' וכיון דלא חזיא אחר בא עליה לאחר ג' וזונה היא ופסולה לכהונה:}%endcomment
אלא אי אמרת אתצודי הוא דלא מתצדי עד לאחר ג' הא אחר בא עליה מאי 
\commenta{ומאן לימא לן - כלומר אפי' אם תמצי לומר מיזל אזלי בביאה תוך ג' מי איכא למימר שהות לא הוה להו מאן לימא לן דלא הדרי לאלתר:}%endcomment
מתקיף לה רב חייא בריה דרב איקא ומאן לימא לן דמכה שבתוך ג' אינה חוזרת לאלתר שמא חוזרת לאלתר והא אחר בא עליה 
\commenta{האי דם נדה הוא - האי דתוך ג': }%endcomment
אלא נפקא מינה כגון שבעל בתוך ג' ומצא דם ובעל לאחר ג' ומצא דם אי אמרת מיזל אזלי ואתו האי דם בתולין הוא אלא אי אמרת אתצודי הוא דלא מתצדי אלא עד לאחר ג' האי דם נדה הוא מאי 
אמר רב חסדא ת"ש פחות מכאן כנותן אצבע בעין למה לי למתני כנותן אצבע בעין לתני פחות מכאן ולא כלום מאי לאו הא קמ"ל מה עין מדמעת וחוזרת ומדמעת אף בתולין מיזל אזלי ואתי 
\commenta{יוסטני:}%endcomment
ת"ר מעשה ביוסטני בתו של אסוירוס בן אנטנינוס שבאת לפני רבי אמרה לו רבי אשה בכמה ניסת אמר לה בת ג' שנים ויום אחד 
\commenta{אוי לשלש שנים שאבדתי - משנראיתי לביאה:}%endcomment
ובכמה מתעברת אמר לה בת י"ב שנה ויום אחד אמרה לו אני נשאתי בשש וילדתי בשבע אוי לשלש שנים שאבדתי בבית אבא 
\commenta{משמשות במוך - תקנתן הוה לשמש במוך שלא יתעברו:}%endcomment
ומי מעברה והתני רב ביבי קמיה דרב נחמן ג' נשים משמשות במוך קטנה מעוברת ומניקה 
\commenta{סנדל - כשמתעברת תאומים פעמים שאחד דוחק את חבירו ופוחת את צורתו כדאמרינן במכילתין אין סנדל שאין עמו ולד:}%endcomment
קטנה שמא תתעבר ותמות מעוברת שמא תעשה עוברה סנדל מניקה שמא תגמול את בנה וימות 
\commenta{פחות מכן - אינה צריכה לשמש במוך דודאי לא תתעבר:}%endcomment
ואיזוהי קטנה מבת י"א שנה ויום אחד ועד י"ב שנה ויום אחד פחות מכאן או יתר על כן משמשת והולכת דברי ר"מ 
וחכ"א אחת זו ואחת זו משמשת כדרכה והולכת ומן השמים ירחמו שנאמר (תהלים קטז, ו) שומר פתאים ה' 
איבעית אימא (יחזקאל כג, כ) אשר בשר חמורים בשרם ואיבעית אימא (תהלים קמד, ח) אשר פיהם דבר שוא וימינם ימין שקר 
ת"ר מעשה באשה אחת שבאת לפני ר"ע אמרה לו ר' נבעלתי בתוך שלש שנים מה אני לכהונה אמר לה כשרה את לכהונה 
\commenta{מצצה - כלומר אף אני טעמתי טעם זנות:}%endcomment
אמרה לו רבי אמשול לך משל למה הדבר דומה לתינוק שטמנו לו אצבעו בדבש פעם ראשונה ושניה גוער בה שלישית מצצה אמר לה אם כן פסולה את לכהונה 
\commenta{לא אמרה - הא דקאמר פסולה לכהונה אלא לחדד:}%endcomment
ראה התלמידים מסתכלים זה בזה אמר להם למה הדבר קשה בעיניכם [אמרו ליה] כשם שכל התורה הלכה למשה מסיני כך פחותה מבת שלש שנים כשרה לכהונה הלכה למשה מסיני ואף רבי עקיבא לא אמרה אלא לחדד בה את התלמידים
\commenta{מתני' קנאה - וזכה בנכסי אחיו ואע"ג שאין קנין לקטן הרי קנויה לו ועומדת:}%endcomment
{\large\emph{מתני׳}} בן תשע שנים ויום אחד שבא על יבמתו קנאה ואין נותן גט עד שיגדיל
\commenta{ומטמא בנדה - אם בא עליה דביאתו ביאה לכל דבר:}%endcomment
ומטמא בנדה לטמא משכב תחתון כעליון 
\commenta{ופוסל - כגון בן ט' שנים כותי או ממזר שבא על הכהנת פסלה מתרומת אביה אבל בבן תשע ישראל שבא על בת כהן ליכא למימר דפוסל מן התרומה דאי משום ביאת זנות לא פסלה זנות דכשר מן התרומה ואי משום קדושין למהוי בת כהן לישראל לאו קדושין נינהו דקטן דקדיש אין קדושיו קדושין:}%endcomment
ופוסל ואינו מאכיל בתרומה ופוסל את הבהמה מע"ג המזבח ונסקלת על ידו ואם בא על אחת מכל העריות האמורות בתורה מומתין על ידו והוא פטור
\commenta{גמ' בגט סגיא לה - בתמיה אם בא לגרשה:}%endcomment
{\large\emph{גמ׳}} ולכשיגדיל בגט סגי לה והתניא עשו ביאת בן ט' כמאמר בגדול
\commenta{מאמר - קידושי יבמה:}%endcomment
מה מאמר בגדול צריך גט למאמרו וחליצה לזיקתו אף ביאת בן ט' צריך גט למאמרו וחליצה לזיקתו 
אמר רב הכי קאמר
לכשיגדיל יבעול ויתן גט
\commenta{מתני' נבדקין - אם ידעה לשם מי נדרה ולשם מי הקדישה:
}%endcomment
{\large\emph{מתני׳}} בת אחת עשרה שנה ויום א' נדריה נבדקין בת שתים עשרה שנה ויום א' נדריה קיימין ובודקין כל שתים עשרה
בן שתים עשרה שנה ויום אחד נדריו נבדקין בן י"ג שנה ויום אחד נדריו קיימין ובודקין כל שלש עשרה 
\commenta{קודם הזמן הזה - כגון בתינוקת קודם שיבא ראש השנה של שתים עשרה ובתינוק קודם ראש השנה של שלש עשרה:}%endcomment
קודם לזמן הזה אע"פ שאמרו יודעין אנו לשם מי נדרנו לשם מי הקדשנו אין נדריהם נדר ואין הקדשן הקדש לאחר הזמן הזה אע"פ שאמרו אין אנו יודעין לשם מי נדרנו לשם מי הקדשנו נדרן נדר והקדשן הקדש
\commenta{גמ' למה לי - ממילא ידענא דמכאן ואילך לא בעי בדיקה:}%endcomment
{\large\emph{גמ׳}} וכיון דתנא בת אחת עשרה שנה ויום א' נדריה נבדקין בת י"ב שנה ויום א' נדריה קיימין למה לי סד"א בודקין לעולם קמ"ל 
\commenta{בודקין כל י"ב למה לי - הא תנא דמיום א' בשתים עשרה מתחילין לבדוק ואינן קיימין בלא בדיקה עד יום אחד בי"ג אלמא כל י"ב הוי זמן בדיקה:}%endcomment
וכיון דתני בת י"ב שנה ויום אחד נדריה קיימין בודקין כל שתים עשרה למה לי סלקא דעתך אמינא הואיל ואמר מר ל' יום בשנה חשובים שנה היכא דבדקנא ל' ולא ידעה להפלות אימא תו לא ליבדוק קמ"ל 
\commenta{בת י"א כו' למה לי - מדתנן ליה בודקין כל י"ב שמעינן דמקמי הכי לא:}%endcomment
ולתני הני תרתי בבי בת י"ב שנה ויום א' נדריה קיימין ובודקין כל י"ב בת אחת עשרה ויום א' נדריה נבדקין למה לי 
איצטריך סד"א סתמא בשתים עשרה בעיא בדיקה באחת עשרה לא בעיא בדיקה והיכא דחזינן לה דחריפא טפי מיבדקה באחת עשרה קמ"ל 
\commenta{הני מילי - דמקמי הכי מוחזקין קטנים ולאחר זמן מוחזקין גדולים:}%endcomment
קודם הזמן הזה ואחר הזמן הזה למה לי סד"א הנ"מ היכא דלא קאמרי אינהו אבל היכא דקאמרי אינהו נסמוך עלייהו קמ"ל 
\commenta{דברים האמורים כו' - דתינוק ממהר להתחכם יותר מתינוקת:}%endcomment
ת"ר אלו דברי רבי ר"ש בן אלעזר אומר דברים האמורים בתינוקת בתינוק אמורים דברים האמורים בתנוק בתנוקת אמורים 
א"ר חסדא מ"ט דרבי דכתיב {בראשית ב׳:כ״ב } ויבן ה' [אלהים] את הצלע מלמד שנתן הקב"ה בינה יתירה באשה יותר מבאיש 
ואידך ההוא מבעי ליה לכדריש לקיש דאמר ריש לקיש משום ר"ש בן מנסיא ויבן ה' [אלהים] את הצלע אשר לקח מן האדם לאשה ויביאה אל האדם מלמד שקלעה הקב"ה לחוה והביאה אצל אדם הראשון שכן בכרכי הים קורין לקלעיתא בנייתא 
ור"ש בן אלעזר מ"ט אמר רב שמואל בר רב יצחק מתוך שהתינוק מצוי בבית רבו נכנסת בו ערמומית תחלה 
איבעיא להו תוך זמן כלפני זמן או כלאחר זמן 
למאי הלכתא אי לנדרים לאו כלפני זמן דמיא ולאו כלאחר זמן דמיא
\commenta{לעונשין - וכגון דאייתי שתי שערות תוך הזמן דאי לא אייתי אפילו לאחר זמן לאו בר עונשין הוא ולאחר זמן אי אייתי ב' שערות הוי גדול ובר עונשין:}%endcomment
אלא לעונשין מאי רב ור' חנינא דאמרי תרווייהו תוך זמן כלפני זמן ר' יוחנן ור' יהושע בן לוי דאמרי תרווייהו תוך זמן כלאחר זמן 
\commenta{וזאת לפנים - וזאת לשון נקבה כלומר רב חנינא דשמיה לשון נקבה אמר לפנים כלומר לפני זמן ולא תטעה להחליף דבריהם. ל"א שמו של רב דאמר כלפני הזמן מובלע בתיבת בישראל:}%endcomment
אמר רב נחמן בר יצחק וסימניך (רות ד, ז) וזאת לפנים בישראל 
\commenta{מתיב רב המנונא - קס"ד כדאמר לקמן דרב המנונא משנה יתירה דייק דלא איצטריך למתנייה כדפרכינן לעיל והא דתני ליה תנא דמתני' למידק מינה לענין עונשין הא תוך הזמן כלפני הזמן דאי לענין נדרים לא כלפני זמן ולא כלאחר זמן:}%endcomment
מתיב רב המנונא אחר זמן הזה אע"פ שאמרו אין אנו יודעים לשם מי נדרנו לשם מי הקדשנו נדריהם נדר והקדשן הקדש הא תוך זמן כלפני זמן 
אמר ליה רבא אימא רישא קודם הזמן הזה אע"פ שאמרו יודעים אנו לשם מי נדרנו לשם מי הקדשנו אין נדריהם נדר ואין הקדשן הקדש הא תוך זמן כלאחר זמן 
ולא היא רבא קטעי הוא סבר רב המנונא ממשנה יתירה קדייק ואדדייק מסיפא לידוק מרישא 
\commenta{מגופא דמתני' דייק - דלא מתוקמא אלא כשהביא שתי שערות ואפ"ה לא חשיב ליה גדול עד לאחר זמן. והשתא ליכא למיפרך לידוק מרישא דהא לא דייק מידי דבין רישא ובין סיפא לגופא איצטריך לענין נדרים דלא תימא הני מילי היכא דלא קאמרי אינהו אבל היכא דאמרי אינהו כו' כדאמר לעיל ומיהו מגופא דסיפא שמעינן דעד דמטו לאחר זמן בעינן יודע להפלות ואם אינו יודע להפלות ואפילו הביא ב' שערות לאו איש הוא דניהוי נדרו נדר מחמת אישות בלא הפלאה:}%endcomment
ולא היא רב המנונא מגופא דמתניתין קא דייק הא לאחר זמן היכי דמי אי דלא אייתי שתי שערות קטן הוא אלא לאו דאייתי שתי שערות
\clearpage}

\addpart{חידושי רמב"ן על נדה}
\newchap{פרק \hebrewnumeral{3} המפלת חתיכה}
\twocol{ הא ד\textbf{אמר ר' יוחנן אם יש בה דם אגור טהורה.} איכא דמקשו ללישנא קמא דר' יוחנן ליחזי אם חתיכה זו מד' מינין טמאה שדרכה של אשה לראות דם נדה בחתיכה וא"ל אתיא כלישנא בתרא דר' יוחנן מאן דמתני לישנא קמא מתרץ הא דידיה הא דרביה, והאי דקאמר ר' יוחנן ד"ה טהורה לרבנן דמתני' ולר' יהודה מיירי. 
 הא דמקשינן לר' זירא \textbf{והא אמר ר' יוחנן משום רשב"י המפלת וכו'.} וא"ל אדמקשי ליה מיניה ליסייעיה ממתני' דקתני אם יש עמה דם אין בתוכה לא. וא"ל קס"ד השתא דמתני' לאו עמה לאפוקי תוכה אלא עמה לאפוקי חתיכה גופה דאפילו היא מארבע מינין טהורה ומדר' יוחנן מיפרשא ליה קושיא וממתני' לית ליה סייעתא בהדייהו.\par ואע"ג דאמרן לעיל בגמרא דאלו רבנן סברי עמה אין תוכה לא ההיא מימרא דגמרא היא ולא קס"ד השתא ומ"ה פריק אפילו בדר' יוחנן דהתם משום דדרכה של אשה לראות דם בחתיכה ומין במינו הוא ואינו חוצץ כדפרישית לעיל, א"נ א"ל דמתניתין לא סייעתא היא דקס"ד עמה לאפוקי תוכה משום שאין זה דם נדה אלא דם חתיכה. 
 הא דאקשינן \textbf{ת"ק נמי טהורי מטהר.} ומפרקי' אלא לאו דפלאי פלויי איכא בנייהו. לאו דצריך להכי דהא מצי למימר אלא שפופרת א"ב דת"ק סבר בבשרה ולא בשפיר ולא בחתיכה וכ"ש בשפופרת ואתי רבנן למימר אין זה דם נדה אלא של חתיכה הא דם נדה טמאה ואפילו בשפופרת אלא הא דאמרינן דפלי פלויי איכא בנייהו משום דקים ליה דבהא נמי פליגי לפום טעמייהו דכיון דר' אלעזר סבר דם אגור הוא א"א לטהר אלא בדלא אפלאי וכיון דרבנן סברי דם חתיכה גופה הוא אפילו איפלאי נמי ודאי טהור' הילכך מפרש ואזיל כולה פלוגתייהו. 
 ומהדר אביי \textbf{בשפופרת דכ"ע לא פליגי כי פליגי בחתיכה מר סבר דרכה של אשה לראות דם נדה בחתיכה.} פירש רש"י ז"ל דבפלאי פלויי פליגי מר דהו רבי אלעזר סבר דרכה של אשה לראות דם נדה בחתיכה וכיון דאיפלאי וליכא חציצה טמאה וכי לא אפלאי רחמנא מיעטה מבשרה ולא בשפיר ולא בחתיכה ורבנן סברי אין דרכה של אשה לראות דם נדה בחתיכה אלא האי דם חתיכה עצמה הוא.\par ולא מחוור דא"ה לא דמי האי דרכה של אשה וכו' לאותו שאמרו למעלה בדר' יוחנן דהתם קאמרינן דכיון דדרכה אע"ג דלא אפלאי נמי לא חוצה דהיינו אורחא ולישנא דגמרא נמי לא משמע הכי כלל.\par אלא ה"פ מרדאינהו רבנן סברי דרכה של אשה לראות דם בחתיכה ולא טהרו כאן אלא משום שאין זה דם נדה אלא דם חתיכה והיכא דהוי ודאי דם (חתיכה) [נדה] כגון מצא בה דם אגור טמאה והיינו לר' יוחנן ומר דהוא ר"א סבר אין דרכה של אשה הילכך הוי ליה כשפופרת ורחמנא אמר בבשרה.\par והאי דמדכרי' סברא דרבנן מקמי דר' אלעזרא"ל משום דאמרן לעיל דרכה של אשה כסברייהו א"נ לאו דוקא וכן בכמה דוכתי בתלמודא דלא קפדי.\par ובודאי דה"מ אביי לתרוצי כדרבנן הא דם נדה ודאי טמאה בדאיפלאי ובשפופרת דכו"ע לא פליגי דטהורה אלא ניחא ליה לתרוצי בדידה ולא לעיולי בה פילי דהשתא לא מוספינן בפלוגתייהו איפלאי פלויי כלל אלא בדם אגור בחתיכה פליגי כדפרישית, ועוד לאוקמה כדר' יוחנן דלעיל דלא לתקום דלא כחד כנ"ל.\par ויש מפרשים דלא ניחא ליה לאביי לאוקומה פלוגתא דרבנן אפלאי פלויי בלחוד דהא לא מדכרא בהדיא במילתיה דר' אלעזר דאנן בגמרא לאו חסורי מחסרא לברייתא כלל אלא מימר קאמרינן דלר"א בודאי פלאי טמאה ולא ניחא ליה לאוקומא פלוגתייהו אמאי דלא מתפרש בברייתא בהדיא.\par ולאו מילתא היא ור"א ורבנן תרווייהו מטהרין מר נסיב לה טעמא מבשרה ולא בחתיכה ולפום טעמיה איפלאי פלויי טמאה ומר נסיב לה טעמא אין זה דם נדה לטהורי נמי אפלאי פלויי אלא כטעמא דפרישית עיקר. 
\clearpage}

\newsection{דף כב}
\twocol{\textbf{א"ד האי ממנו עד שתצא טומאתו לחוץ.} ואי קשיא הכא איצטרך קרא לומר שלא יטמא בפנים דאלמא דינא הוא דמטמא והכא איצטרך בבשרה לומר שמטמאה בפנים אלמא אין דינה לטמא. לא קשיא דגלי רחמנא בחד ואיצטרך חבריה שלא תאמר זב וזבה הוקשו א"נ באיש דינו לטמאות משעקר שסופו לצאת מיד ולא האשה שהרי עומד בבית החיצון הרבה. 
\textbf{והלא עצמו הוא אינו מטמא אלא בחתימת פי האמה, למימרא דנוגע הוי.} פירש רב הונא אליבא דנפשיה פשיט ליה דס"ל כר' נתן דאמר זב אינו מטמא אלא בחתימת פי האמה ואל תתמה [דהא שמואל] רביה (דרבה) [דרב הונא] הוא דאמר נמי כר' נתן כדאיתא בפרק יוצא דופן ולפום הכי גמר רב הונא בעל קרי מיניה דזב וסבר לה נמי כר' שמעון דאמר בפ' ואלו דברים בפסחים דס"ל בזב כר' נתן דבעי פי האמה ואיתקש בעל קרי לזב ובעי נמי חתימת פי האמה כזב דהא קרא בבעל קרי לא כתיב ובזב כתיב או החתים בשרו.\par והא דדאיק מינייהו למימרא דנוגע הוי דלהכי בעינן חתימת פי האמה דליהוי נגיעת חוץ כדפי' רש"י ז"ל, ק"ל אי הכי זב נמי נוגע הוי ולמה לא יספור בזיבה, א"ל בזב ודאי אע"ג שנוגע הוי לענין שיעוריה מיהו הוי רואה לענין טומאה דיליה דאלו נוגע בזב טומאת ערב ואלו רואה טומאת שבעה והאי דאחמיר ד) עליה רחמנא בחתימת פי האמה דליהוי נמי נוגע גזירת הכתוב הוא שלא יהא טמא טומאת שבעה עד שיראה זוב ונגע בו מגע חוץ דה"ל רואה ונוגע ומ"ה אקשי' אא"ב בעל קרי רואה הוי ואפילו במקום שאינו טמא משום נוגע טמא הוא משום רואה הרי דומה לזב מצד אחד שאף הוא יש לו טומאה בראיה שאינו מדין מגע אא"א אינו טמא אלא בנוגע וטומאתו נמי טומאת מגע היא א"כ מה הנוגע בקרי אינו סותר בזיבה אף הרואה לא יסתור שהרי שניהן טומאה אחת להן בכל ענינן ומדין מגע טימאן הכתוב, ומפרקינן התם בשביל שא"א לה בלא צחצוחי זיבה ואם תאמר והלא אין בהם חתימת פי האמה ואין הזוב מטמא אלא כן י"ל כיון שיוצא עם שכבת זרע שהוא חותם פי האמה הרי הוא כנוגע ממש שמין במינו הוא ואינו חוצץ.\par והיינו דלא אקשינן יטמא טומא' שבעה אלא תסתור ז' דטומאה בזוב גמור לית ליה כיון דאינו רואה בשיעורו טומאת מגע זוב אית ליה וכיון דזוב הוא מיהא ובראיה דין הוא לסתור הכל שאין כאן ז' נקיים דהא הוה ליה כאלו ראה זוב בנתיים שאין אחר אחר לכולן.\par ומפרקי' גזרת הכתוב כך הוא מאחר שאין הזוב הזה כדי ראיה אין לו טומאת שבעה ואפילו לסתור ז' אלא סתירתו כטומאתו והא נמי רב הונא אליבא דנפשיה פשט ליה דהא בפרק כיצד הרגל בב"ק איכא ר' אליעזר דסבר אפשר בלא צחצוחי זיבה כלל, ומיהו בהא כרבנן פריק ליה ורבים נינהו.\par ואי קשיא לך לרב הונא דאמר בעל קרי נוגע הוי תרי קראי למה לי דהא כתיב רואה וכתיב נוגע וכדדרשינן בפרק יוצא דופן מדכתיב או איש, א"ל אע"ג דרואה דוקא בנוגע הוא דמטמא ה"א ה"מ ברואה דאיכא תרתי מגע וראיה אבל בנוגע לחודיה לא קמ"ל. 
הא דתנן \textbf{המפלת מין דגים וחגבים ושרצים אם יש עמהן דם טמאה.} אוקי' במחלוקת שנויה ורבנן היא ומדלא פרישנן הבי ברישא דקתני כמין קליפה כמין שערה כמין עפר וכמין יבחושין ש"מ דההיא ד"ה היא דכיון דמילי זוטרי נינהו אפשר לפתיחת קבר קטנה בלא דם ודמיא הא לההיא דמפרקינן בכריתות (דף ט') [דף י' {\small ואין הגירסא שם כן} ] כי אמרינן א"א לפתיחת הקבר בלא דם היכא דגמר הולד דמיפתח טפי ב) ואפשר לפתיחת הקבר בלא דם וכ"ש בכמין קליפה ויבחושין.\par ואם תאמר מ"ש ממפלת רוח דתנא באינה יודעת מה הפילה ר' יהושע אומר א"א לפתיחת הקבר בלא דם א"ל התם דהפילה שפיר יש בו פתיחת הקבר אבל יבחושין ועפר אינה לידה אלא כמקור שהפיל טיפין הוא וכרואה דם יבש בעלמא הוא והיינו נמי טעמא דלא אמרינן בדגים וחגבים ושרצים שאין עמהם דם תטיל למים ואם נמוקו טמאה דהתם בריה נינהו ודאי ונולדת היא אלא שאינה טמאה וכן בחתיכה דבשר גמור הוא אינה נמוחה אבל כאן רואה היא ומכה יש לה שממנה מפלת כן ואע"פ שהיתה בחזקת מעוברת והפילה לאו מעוברה הוא אלא של מכה הוא. 
הא דאמרינן מעיקרא \textbf{דנין יצירה מיצירה ואין דנין בריאה מיצירה.} לאו למימרא דסתרי אהדדי דהא אפשר לי' למיגמרינא לתרווייהו אלא ה"ק זו אינה ג"ש כלל, והיינו דאקשינן מאי נ"מ הא תנא דבי רבי ישמעאל ולא מפרק הני מילי היכא דליכא דדמי ליה אבל היכא דאיכא דדמי ליה מדדמי ליה ילפינן כדאתמר בעלמא אלא ודאי משום דלא אמרינן אלא היכא דסתרי אהדדי והכא תרווייהו דגמר.\par והדר אקשי' ועוד נגמר בריאה מבריאה [ומשני ויברא לגופיה] וייצר לאפנויי ודנין יצירה מיצירה דהשתא ודאי ליכא למיגמר אלא חד במופנה הילכך מדדמי ילפינן דאף על גב דוייצר מופנה גבי אדם ליכא למיגמר בריאה דתנין מיני' בדין מופנה מצד אחד דאפנויי דויצר דאדם לאו להך ג"ש הוא והוה ליה כשאינו מופנה כל עיקר אי נמי השתא לא מסיק טעמיה אלא מפרש ואזיל הוא ואמסקנא ניחא דליכא למיגמר דבריאה כלל כדבעי למימר קמן. 
 והא דאמרינן \textbf{ויברא גבי תנינים לאו מופנה.} אי קשיא הא כתיב נמי ישרצו המים ההוא אין כתוב בעשייה אלא בצוויי, ופי' רש"י ז"ל דכיון שאין מופנה משני צדדין ומשיבין ה"נ יש להשיב מה לאדם שכן מטמא מחיים.\par ול"נ דהאי לישנא קמא לא צריך פירכא דלכ"ע מופנה משני צדדין עדיף ממופנה מצד א' וכיון דע"כ יצירה יצירה גמרינן ה"ל בריאה דאדם לגופיה וגבי תנינים נמי לגופיה ואין מופנה כל עיקר וכל ג"ש שאינו מופנה כל עיקר אין למידן הימנה.\par וא"ת וייצר האדם לגופי' ודבהמה מופנה ודגמרינן ויברא דתנין לגופיה ודאדם מופנה וגמרינן היינו דקאמרי ומאי נ"מ זה כלומר אמאי ניחא לך לאפנויי לחדא לגמרי ומיגמר מינה ולא לאפנויי תרווייהו ומיגמר מנייהו ופריק לרבנן הא עדיפא דהא אין משיבין ולר' ישמעאל נמי הא עדיפא דהיכא דאיכא מופנה משני צדדין איהי עדיף ולהכי אפנויי רחמנא לבהמה משני צדדין דשדינן מופנ' דכולהו בגוה כי היכי דלא נימא באידך מופנה מצד אחד הוא דכל היכא דאיכא למישדי שני צדדין דמופנ' בדידיה שדינן ומיניה גמרינן בין לרבי ישמעאל בין לרבנן אבל ללישנא דרב אחא הויא דבעי' והאי מאי פירכא משום דאפילו כשאנו גומרין יצירה יצירה יכולין אנו לגמור בריאה בריאה אע"פ שאינה מופנה כל עיקר אלא שמשיבין ולפום הכי בעי' מאי פירכא ורבנן דפליגי עליה דר"מ במתני' לא גמירי כדאשכחן בפרק כל היד שאין אדם ג"ש מעצמו, וכן פי' רש"י ז"ל.\par ואי קשיא לך לר"מ מאי פירכא ליהדר דינא ותיתי מכאן דכיון דגמר יצורה ואתו בהמה חיה ועוף כי פרכת גבי תנין מה לאדם שכן מטמאו מחיים נימא בהמה תוכיח א"נ נגמר מוייצר דבהמה למד מלמד א"ל מה לשניהם שכן מטמאין במגע ובמשא תאמר בדגים שאינן מטמאין ואע"פ שמקבלין טומאה טומאת עצמן אין להם. 
\clearpage}

\newsection{דף כג}
\twocol{\textbf{למימרא דחיי.} פי' רש"י ז"ל דהא אחותה לא מיתסרא אלא בחייה דאין איסור אחות אשה לאחר מיתה ותמהני א"כ יפה שאל ר' ירמיה ונימא נפקא מינה לענין אתסורי באמה ואם אמה דאמות אפילו לאחר מיתה מן התורה ועוד נפקא מיניה לאתסורי באחותה שנים וג' ימים.\par אלא כך פירשו למימרא דחיי שאם א"א לו לחיות כלל אין קדושין תופסין בנפל גמור כגון בת שמונה והלא הרי הוא כאבן לכל דבר אלא ודאי סבר ר' ירמיה דחיי והאמר ר' יהודה אמר שמואל לא אמרה ר' מאיר אלא הואיל ובמינו מתקיים, פי' הואיל לאו דוקא דהא לא טעמא הוא לר"מ אלא משום יצירה יצירה או שגלגל עיניו כשל אדם או בשיש בו מצורת אדם אלא ה"ק לא אמר ר"מ שהוא ולד שיעלה על דעתו שהוא חי אלא ולד הוא לענין טומאה שבמינו מתקיים כנפל גמור שהוא אינו מתקיים ובמינו מתקיים. 
\textbf{א"ר ירמיה בר אבא אמר רב הכל מודים וכו'.} פי' ר' ירמיה משמיה דרב פליג אדשמואל ור' יוחנן דפרשי לעיל טעמיה דר"מ משום יצירה יצירה או משום גלגול עין שלדבריהם אפילו תייש גמור במעי אשה ולד מעליא הוא לטומאות לידה וכ"ש גופו אדם ופניו תייש דאיכא מקצת אדם.\par וה"נ משמע דס"ל לרב יהודה משמיה דרב כותיה מדקאמר הואיל ובמינו מתקיים ולרב ירמיה משמיה דרב לית ליה הנהו טעמי אלא ר"מ ורבנן בסברא בעלמא פליגי בשפניו אדם ונברא בעין א' כבהמה שר"מ אומר מצור' אדם בעינן והא איכא וחכמים אומרים כל צורת ממש.\par וה"ה לר' מאיר' דבמצח ועין וגבן העין ולסת וגבת הזקן סגי אלא להכי נקט פניו אדם ועין אחד כבהמה להודיעך כחן דרבנן והא דאמרי ליה רבנן והא איפכא תניא לאו איפכא תניא דוקא דמתהפכי תרתי סברי דהא לא (מתסרא) [מתהפכא] סברא דרבנן לר"מ אלא איפכא בלישנא לכולהו ואיפכא בסברא דר"מ דמפכא לה ברייתא לרבנן וכדפירש רש"י ז"ל.\par ולסבריה דרב הא דתני' לקמן המפל' דמות נחש אמו טמאה לידה ר' יהושע יחידאה היא ולית ליה דר"מ וכ"ש דרבנן וההיא דתניא לעיל נראין דברי ר"מ בבהמה וחיה בהכי נמי מתוקמ' בבהמה וחיה ומקצ' סימנין דאדם דכיון דהיא עצמה עיניה הולכות כשל אדם במקצת סימנין נעשית כאדם גמור מה שאין כן בעופות שאפילו כל פניו כאדם ועיניו לצדדין אינו כלום.\par והאי דדחי רב אחא בריה דרבא לעיל תבדוק לרבנן דמודו רבנן בקריא וקפוף הואיל ויש להם לסתות כאדם דחייה בעלמא היא דדחי בסברת דר' אלעזר בר צדוק אבל לפום מסקנא לרבנן לסתו' חדא מצורות דפנים נינהו וצריך נמי גבין וגבת זקן דאדם ועין נמי דאדם ואף ע"פ שהולכות לפניהם צריך צורת דאדם באוכמא.\par וי"מ דלרב ירמיה גופיה אית ליה אליבא דר"מ יצירה יצירה ואי כולה תייש בפניו וגופו אמו טמאה לידה הואיל ובמינו מתקיים והא דאמה גופו אדם ופניו תיש ולא כלום משום שאין זה לא מין בהמה ולא מין אדם והואיל ואין לו מין שמתקיים דברי הכל ולא כלום.\par ולפי דבריהם קשיא, א"כ מנא ליה לר' ירמיה א"ר דר"מ בפניו אדם ונברא בעין א' כבהמה פליג דילמא בההוא כרבנן ס"ל דלאו אדם הוא ומין בהמה נמי אינו שאין לך בבהמה כמותו והם אומרים קסבר רב דר"מ ורבנן בתרתי נמי פליגי ממאי מדאמרי ליה רבנן כל שאין בו מצורת אדם ולא קתני וחכמים אומרים אמו טהורה א"נ וחכ"א אינו ולד שמעי' דר"מ דמטמא נמי במקצת צורה ואמרי ליה לא כל צורה בעי למעוטי צורה בהמה גמורה ולמעוטי נמי מקצת צורת אדם והא דאמרי ליה רבנן והא איפכא תניא איפכא לגמרי הוא שר"מ אמר כל צורת לגמרי מ"ט או כולו אדם או כולו בהמה וחכ"א מצורת אדם ולא פניו בהמה אבל במקצת צורת אדם ולד הוא והיינו דקתני מתני' כל שאין בו מצורת לאפוקי כולו בהמה ומדלא קתני אמו טהורה סתם משמע ליה דבתרתי פליגי וברייתא נמי דמסייעא להו כך מפורש בספר הישר.\par ודברי רש"י ז"ל יותר נראין והוא הלשון הראשון שכתבנו ואע"ג דקשיא נחש דר' יהושע כדפרישית.\par והא דתניא המפלת דמות לילית אמו טמאה לידה ולד הוא אלא שיש לו כנפים ולא אמרינן משום דגופו תיש ופניו אדם היינו נמי טעמא משום דודאי פניו אדם אע"פ שגופו תיש בתר צורת פנים אזלינן אבל בדמות לילית ס"ד אין כאן צורת אדם כלל אלא צורת לילית היא זו בין בגוף בין בפנים קמשמע לן דלילית גופה ולד הוא אלא שיש לו כנפים. 
\clearpage}

\newsection{דף כד}
\twocol{\textbf{אמר רבא ושטו נקוב אמו טמאה.} פירש"י ז"ל קסבר טרפה חיה. וקשה להעמיד דברי הרב רבא שלא כהלכה ועוד אני תמה וכי מפני שאין טרפה חיה י"ב חדש נטהר אמו של זה והלא הנפל שהוא כאבן ואינו יכול לחיות או שיצא מת ומחותך אמו טמאה זה שיצא נקוב הושט וכלו לו חדשיו לא כ"ש ואפילו הולד שנימוק בשליא אמו טמאה והאיך יטהרוה שאלו היה דבר שמתחלת ברייתו הוא אפשר לומר שאינו ראוי לבריית נשמה וטהור אבל זה שמא עכשיו נקב ומה בינו למחותך ויצא איברים אברים.\par לפיכך נ"ל שכל הנולד בטריפות ודאי אמו טמאה היא ואפילו היה טרפותו בתחלת ברייתו שהרי ראוי הוא לחיות י"ב חדש וכ"ש בטרפות נקב וחתך. והא דאמר רבא ושטו נקב טמאה לד"ה קאמר ולא בא אלא להשמיעינו שושטו אטום אמו טהורה שאין זה בכלל אדם הואיל ונברא שלא כדרך החיים. 
 והא דאמרי' \textbf{קא מפלגי בטרפה חיה} שהזיקיקו לרש"י ז"ל לטהר ולד טרפה נ"ל שלא הקפידו אלא על לשון הברייתא שאמרו וכמה כדי שינטל מן החי וימות דקסבר האי תנא דכל שנברא אטום בלא חיתוך איברים עד מקום שאלו ינטל מן החי וימות אינו בכלל ולד ולא שיהא זה נקרא טרפה אלא זה אינו נולד הואיל ונברא אטום אבל נחלקו האמוראין כמה הוא כדי שינטל מן החי וימות ופי' ר' זכאי עד לארכובה ודקדקו ממנו שהוא סובר טרפה חיה דהא קאמר שבכך החי מת (לרש"י נטל) [לכשינטל] ממנו ור' ינאי אמר עד לנקובה שבכך נעשה נבלה אבל טרפה אינה מתה ר' יהושע דאמר עד טבורו קסבר בין זו בין זו חיות הן וכל זה אינו אלא בשיעור כמה כדי שינטל מן החי אבל בולד שנטל ממנו לא נחלקו (במינו) [בו] ולא אמרו כאן אלא הולד כשהוא אטום.\par ומה שפי'רש"י ז"ל אטום חסר אינו נראה אלא אטום כמשמעי שאין לו חיתוך איברים ואין לו חלק שבהן אלא כמין גולם אטום ודמיא להא דתניא לקמן בריית גוף שאינו חתוך וכו'. 
 הא דאקשי' \textbf{ואם איתא ליתני שמא מגוף אטום (ופניו) [או ממי שפניו] המוסמסין באתה.} א"ל איבעי למיתני טובא ליתני שמא באת מפניו תיש או אפילו פרצוף אחר או שיש לו שני גבין ושתי שדראות וכמין אפיקותא דדיקלא וכן כיוצא בהן א"ל הנהו לא שכיחי ולא ה"ל למיתני. אבל פניו ממוסמסין ה"ל למיתני משום דשכיח נמי טפי מגוף אטום. 
הא דאמרינן \textbf{ושמואל סבר בריה בעלמא איתא וכי אגמריה רחמנא למשה בעלמא.} פירש"י ז"ל אותו המין אסר לו וק"ל א"כ לשמואל אפילו יוצא לאויר העול' נמי לישתרי דה"ל כקלוט בן פרה דשרי ונראה מדבריו דבין לרב בין לשמואל במעי טהורה לא חיי הלכך יצא לאויר העולם משום נפל אסור אפילו לשמואל והא דפריך רב שימי ממתניתין ר' חנינא בן אנטיגנוס אומר וכו' לרב ה"ה לשמואל אלא גביה הוה קאי דבר בריה הוה.\par ולא נהירא ועוד דהתם בפ' ואלו מומין (דף מג ע"ב) תנן לה למתני' גבי מומי כהן איזהו גבן ר' חנינא בן אנטיגנוס אומר כל שיש לו שני גבין ושדראות והוי ביה למימר' דחיי והאמר רב באשה אינו לד בבהמה אסור באכילה ולא מדכרין התם דשמואל בכלום בעולם.\par אלא הכי משמע פירושא לכ"ע מינא בעלמא ליכא כי פליגי בבריה רב סבר אפילו בריה בעולם ליכא דלא חי הילכך כי אגמריה רחמנא למשה במעי בהמה אגמריה דבחוץ לא צריך נפל הוא. ושמואל סבר בריה בעלמא איתא דחיי וכי אגמריה רחמנא בשיצא לאויר העולם דלא תימא כקלוט בן פרה הוא אבל במעי בהמה דאפילו נפל שריא איהי נמי שרי. 
\clearpage}

\newsection{דף כה}
\twocol{\textbf{המפלת שפיר מלא בשר נימוח מהו.} פי' קא מיבעיא להו לרבנן דפליגי עליה דר' יהושע מיפלגי נמי בבשר נימוח או לא אמר להם לא שמעתי אמר לפניו ר' ישמעל בר' יוסי משום אביו כך אמר אבא מלא דם טמאה נדה מלא בשר טמא' לידה שהיה ר' ישמעאל סבור שלא אמר אביו כדברי היחיד ולפיכך דחה רבי ואמר שמא כדברי ר' יהושע אמרה וזה שאמר מלא בשר לאו דוקא אלא ה"ה למחוי עכור ולא בשר אלא להוציא צלול אפילו לר' יהושע.\par ויש שגורסין בה כמאן כסומכוס מדהא כיחידאה הא נמי דילמא כר' יהושע אמרה ואינו בספרים.\par וא"ת כיון שרבי לא קבלה אפילו בבשר נימוח שיהא ולד ריב"ל מנין לו דקאמרי בצלול מחלוקת אבל בעכור ד"ה ולד זה אינה שאלה דריב"ל כר' יוסי ס"ל וקסבר ר' יוסי לרבנן אמרה כדסבר נמי ר' ישמעאל בר' יוסי ועוד דכיון דרבי לא שמעתי אמר אינה תשובה לדברי ריב"ל שאם ר' לא שמע ריב"ל שמע לא ראינוה אינה ראיה.\par וי"א אין אומרים בדברים אלו זו דומה לזו שאפשר למימר עכור ולד ובשר נימוח שמא אינו ולד. ואיכא למימר נמי איפכא ולפיכך נחלקו בכולן. 
\clearpage}

\newsection{דף כו}
\twocol{הא ד\textbf{אמר רב הונא בר תחליפא משמיה דרבא ולד מדנפיק קבא דרישיה הויא ליה לידה סנדל עד דנפיק רוביה.} פי' רש"י ז"ל משום שאין הראש פוטר בנפלים. ושמואל דאמר הכי איתותב במס' בכורות אלא איכא לפרושי דאפילו למ"ד הראש פוטר בנפלים ה"מ בולד שלם או אפילו במחותך שנגמר' צורתו אבל סנדל שלא נגמרה לו צורה לא חשיב רישיה למהוי ביה לידה. 
 הא דאמרי' \textbf{תלת מתני' ותרתי שמעתא שיעורן טפח.} ואקשי' תרתי חדא היא היינו טעמא דלא מקשינן תלת ד' הוויין משום דהוה איכא למימר רבי שילא לית ליה דר' חייא דשיעור אזוב טפח ומ"ה מקשינן אם כן [תרתי חדא היא, ועוד א"ל] תרווייהו כי הדדי נינהו וחדא נקט. 
והא דאמרינן \textbf{השתא דאתית להכי הך נמי פלוגתא היא דקתני סיפא א"ר יהודה לא אמרו טפח אלא מן התנור לכותל.} ק"ל וניחשוב מן התנור לכותל דמודה ר' יהודה וה"מ למיחשבי' בדברי הכל וא"נ לא חשיב חד דדברי הכל ה"מ למימר סתמא אבן היוצא מן התנו' טפח דהא קא חשיב בכה"ג טפח סוכה למר בדופן ג' ולמר בדופן ד' כיון דכולהו אית להו דופן טפח קחשיב ליה ה"נ כולהו אית להו אבן היוצא מן התנור טפח מכאן או מכאן.\par ואיכא למימר התם הכל מודים ששיעור דופן א' בסוכה טפח והשאר כהלכתן. וכי אמר דופן סוכה טפח ליכא למטעי במידי דפלוגתא. אבל הכא אי אמר סתמא אבן היוצא מן התנור הוה משמע מכל צדדין ואתיא כרבנן ואי פריש נמי אבן היוצא בין התנור לכותל טפח הוה משמע הא בין תנור לבית אינו טפח כר' יהודה ואיהו במילתא דפלוגתא לא איירי לא כמר ולא כמר. ולא בעי לפרושי תרווייהו ובודאי משמע לכאורה דהשתא דאתינן להכי ומפרקינן דהך נמי פלוגתא היא הדר ביה מתירוציה דקאמר כי קאמרי היכא דבצר מטפח לא חזי וכו'.\par והשתא קשיא לי טובא וליחשוב הני דתנן במס' כלים פי"ח גדד לשתי כרעים טפח על טפח לוכסן או שמעטה פחות מטפח טהורה רישא ל"ק דטפח על טפח לא קאמרינן סיפא ליחשוב. ועוד שם בפרק (בתרא) [כ"ט] חוט מאזני' של חנוני ושל בעלי בתים טפח יד קרדום מלפניו טפח שירי הפרגל טפח יד מקבת ושל מפתחי אבנים טפח. וי"ל כי קאמרינן היכא דבציר מטפח לא חזי והני כ"ש דבציר מטפ' (דידהו) [דידות] הוי וכי קאמרינן השתא דאתית להכי לאו למימרא דליתי' לשנויה דשנינן אלא למימר' דמההי' לא תיתי תיובתא בבי מדרשא כלל. 
 הא דאמר רב \textbf{אין הולד מתעכב אחר חבירו כלום.} ראיתי מקצת בעלי פירושין שכתבו דלית ליה לרב הא דאמרינן לקמן מעשה ונשתהא ולד אחרונה אחר חבירו שלשים יום ולית ליה נמי מעשה דיהודה וחזקיה ולית ליה נמי האי דאמרינן בכתובות וביבמות ג' נשים משמשות במוך קטנה מעוברת מניקה מעוברת שמה תעשה עוברה סנדל אלא קסבר אין אשה מתעברת וחוזרת ומתעברת ולפיכך אין ולד מתעכב אחר חבירו כלום.\par ואין דבריהם נראין דג' נשים מתניתא הוא ונימאתהוי תיובתיה דרב. ועוד מעשה דיהודה וחזקיה בני חביביה דהוא יושב לפניו והן יושבין עמו בבה"מ היכי אפשר דלא חזי ליה ואם איהו אומר דלא היו דברים מעולם מאן מהימן לאסהודי עלייהו.\par אלא היינו טעמא דרב דקסבר אין אשה מתעברת וחוזרת ומתעברת בין נפל בין של קיימא אלא א"כ נעשה א' מהם סנדל וסנדל כרוך עם הולד הוא יוצא שחבור אתו ונדבק בו והיינו טעמא דמוך אבל כשהאשה מתעברת תאומים טפה אחת היא שמתחלקת וכשהן נגמרין לז' או לט' אין הולד מתעכב אחר חבירו כלום אלא א"כ הפילה א' נפל וא' שליא אבל פעמים שנתחלקו לשתים וא' מהן נגמר לט' וא' לז' ובזה מודה רב שהול' משתהא אחר חבירו כדי שתגמור צורתו בזמנו. והיינו מעשה דיהודה וחזקיה ומיהו לית ליה אפוכי שמעתא דלקמן (כ"ד) [כ"ג] לולד דבחד ירחא לא משתהא אלא ל"ג אית ליה.\par והא דאמרינן לעיל סנדל מהו דתימא הואיל וא"ר יצחק עד קמ"ל שניהם הזריעו בבת אחת ודאי קשיא דהא איכא סנדל דמתעברת וחוזרת ומתעברת וזה זכר וזה נקבה כדפרישי' במשמשת במוך. ואיכא למימר אין ודאי דמצי למימר הכי אלא שמא תאמר היכא דבעל ופירש מדהאי זכר האי נמי זכר קמ"ל אפי' בכה"ג חיישינן שמא שניהם הזריעו כא' והאי זכר (נמי) והאי נקבה כנ"ל. 
\textbf{אין תולין את השליח אלא בדבר של קיימא.} פי' רש"י ז"ל שכיוצא בו מתקיים אם היו חדשיו כלין למעוטי שאם הפילה דבר שאינו ראוי לבריית נשמה כגון נברא בירך אחת או גוף אטו' ואח"כ הפילה שליא (פי') [אפי'] בתוך ג' חוששין לולד אחר, ופירוש חזייה לרב יהודה בישות משום דשמעה מרב ולא אמרה.\par ואינו מחוור דבן קיימ' לאפוקי כל נפל משמע וכדאמרן דילמ' כאן בנפלי כאן בבן קיימא ולא ידעתי מי הזקיקו לשנות פירושו אלא הא דתלמיד' דרב פליגא אדרב יהוד' דאמר לעיל משמיה דרב הפילה נפל ואח"כ הפיל' שליא כל שלשה ימים תולין אותה בולד ושאר תלמידים דרב אומרי' משמי' דאין תולין את השליא בנפל אפילו יום א' אלא א"כ יצאה עמו אבל בבן קיימא תולין אותה אפילו מכאן ועד י' ימים.\par ושמעתי שפירשו בירושלמי במס' זו (ג, ד) לפי שאין השליא פורשת עד שיגמר לפיכך אין תולין אותה בנפל.\par ובשאלתות דרב אחא משבחא ז"ל כתב לכך תולין אותה בבן קיימא דאמרי' אגב חיותא דולד בזעא לשליא ונפיק. אבל נפל דלית ביה חיותא לא. ומ"ה חזייה שמואל לרב יהודה בישות דשמעיה דאמר משמיה דרב דכל ג' תליא שליא בנפל וכיון דשמעינהו לכולהו תלמידי דרב דאמרי אין תולין כלל אמר ודאי רב יהודה טעי. 
\clearpage}

\newsection{דף כז}
\twocol{\textbf{מ"ט דר' שמעון וכו'.} פירש"י ז"ל נהי נמי דנימוק מ"מ גופו של מת כאן הויא וה"ל כרקב וכנצל. וק"ל הא דאמר רשב"ל לקמן בשמעתין שפיר שטרפוהו במימיו טהור להוי כרקב וכנצל. ועוד לר"מ נמי בבי' החיצון אמאי טהור ליהוי כרקב וכנצל. וא"ת איהו נמי סבר כל טומאה שנתערב בה מין אחר בטלה אלא מאן תנא דפליג עליה דקאמרת קסבר ר' שמעון והא דתניא מלא תרוד רקב שנפל לתוכו עפר כל שהוא טמא ור' שמעון מטהר אמאי תרמייה הא דכ"ע טומאה כיון שנתערב בה מין אחר טהורה.\par ובתוספ' הקשו לה מדאמרינן ואזדא ר"ש לטעמיה מדאמר א"א שלא ירבו שתי פרידות עפר על פרידה אחת של רקב ויבטלנו ואמאי נהי נמי דבשיעור מצומצם כגון מלא תרוד רקב איכא למימר הכי גבי שליא מ"ט דאפילו הוה בה תרי שיעורי דמלא תרוד מטהר ר"ש דסתמא תנן ואע"ג דליכא למימר א"א שלא ירבו וכו', והם מפרשים הסוגיא כולה בענין אחר ברם נראין דברי מקצת ראשונים שפירשו מ"ט דר"ש דמטהר לגמרי והרי אנו מוצאים בכל יום ולדות חתוכים בשליא והאיך אפשר שלא תהא בכולה כזית ג) שלא נמוק לגמרי קודם שתצא שאפי' נחתך כולו לחצאי זתים מצטרפים הן בתוך השליא לטמא באהל ואמאי מטהר ליה לגמרי, ומפרקי' קסבר ר"ש כל טומאה שהיא כשיעורה ולא יותר שנתערב בה מין אחר בטלה דאמרינן כיון דהיא צריכה שיעור א"א שלא ירבה מין אחר על מקצתה ומבטלה ואף כאן כיון שנמוק הולד אע"פ שנשתייר ממנו (כחצאי) ד) זתים א"א שלא ירבו שתי טיפי מים ודם על מקצת בשר שלא נמוק ומבטלו והיינו דאמרינן ואזדא ר"ש לטעמי' דאמר א"א שלא ירבו שתי פרידו' עפר על פרידה אחת של רקב ומבטלו ובצר לה שיעורא ור"מ סבר לא מבטל אלא א"כ הוציאו לבית החיצון שנטרפו מימיו לגמרי וטהור. והא דאמר לו לר"מ כשם שאינו בבית החיצון כך אינו בבית פנימי לומר שאף בבית החיצון היה לנו לחוש שמא יש בו כזית בשר (שנמוק) ה) אנא משום בטול ואמר להו אינו דומה שזה נמוק לגמרי וה"ל כמים בטריפת בני אדם. אבל דרך לידה אינה נמוק לגמרי וביטול אינו מועיל. ואקשי להו רבא לרבנן דבי רב אדרבה כיון דרקב יותר הוא מן העפר היאך יאמ' ר"ש שהמועט רבה על מקצת המרובה ומבטלו ומגרע שיעורו אדרבה יש לנו לומר שהמרובה עומ' לבד על הממועט ומבטלו לגמרי. והיינו דאמרי' לקמן והשתא דאמרת טעמיה דר"ש סופו כתחלתו גבי שליא מ"ט דקס"ד שהמים והדם שבשליא מועטין הן אלא שרבין על מקצת בשר ומבטלין אותו כדפרישית וא"ר יוחנן משום ביטול ברוב נגעו בה שאפילו היו שם שני חצאי זתים או כזית שלם. יש במי שליא ודם שבה לבטל את כולה ואין אנו צריכין לומר שרבין על מקצת ומבטלו ומגרע שיעורו אלא על כולו הם רבים ומבטלין אותו ובהא פליגי דר"מ סבר אין טומאה בטלה ברוב מלטמא במשא ואהל דהא (קאמר) ו) מאהיל על כולה ומיהו בבי' החיצון טהור שנמוק לגמרי וה"ל אפר שרופין ופחות ממנו. ור"ש סבר בטלה היא לגמרי. 
\textbf{מלא תרווד ועוד עפר בית הקברות טמא.} פי' רש"י ז"ל לאו רקב של מת ממש אלא כגון שנקבר בכסותו או בקרקע בלא ארון ויש כאן מלא תרוד ועוד מאותו עפר דהיו מעורבין עפר ורקב.\par ואין פי' זה נכון דהא אמרין כל שתחלתו דבר א' נעשה גנגילון ואע"פ שיש בו שיעור מן הרקב דומיא דסיפא ומדאמרינן סופו כתחלתו (מאי) [מה] תחלתו דבר א' נעשה גנגילון אלמא פשיטא מילתא דכ"ע כל דבר א' עושה גנגילון בתחלתו. ועוד מדתניא איזהו מת שאין לו רקב נקבר בכסותו אלמא אין לו רקב כלל אפילו מלא סאה דאלת"ה ליתני שאין לו תרוד רקב. ועוד מדסוגיא במסכ' נזיר פרק כהן גדול (דף נ"א) דאמרי' שני מתים שנקברים זה עם זה נעשה גנגילון זה לזה. ואם קברן זה בפני עצמו וזה בפני עצמו והרקיבו ועמדו על מלא תרוד רקב טמא אלמא אין הדבר תלוי בשיעור מן הרקב אלא כל דבר שנקבר עמו נעשה לו גנגילון. וכן בכולה סוגיא דהתם הכי משמע דכי גמירי למלא תרוד רקב דוקא דנרקב בעיניה.\par אלא הא דתניא מלא תרוד ועוד עפר קברות פירושו כגון שנקבר המת ערום על גבי רצפה של אבנים והרקיב ונפחתה מערה ונתערב עפרה ברקב דקסבר ת"ק רוב מתים יש בהן רקב מלא תרוד שבכאן של מת והמותר הוא עפר בית הקברות ואלמלא שנמצא שם ועוד על כרחינו מת זה לא היה בו מלא תרוד דא"כ עפר בית הקברות להיכן הלך אבל מאחר שנמצא כאן ועוד זה א"א בלא מלא תרוד של מת ור"ש מטהר וחזרו לטעמייהו דהיינו מלא תרוד רקב שנפל לתוכו עפר כל שהוא. 
וא"ר יוחנן ד\textbf{ר"ש וראב"י אמרו דבר א'.} ואליבא דר' חייא דפריש למתניתן דתקבר לומר שנפטרה מן הבכורה ולא משום טומאה. והך סוגיא דר' יוחנן הוא דהתם בדוכתא במס' בכורות (כג, א) מסיק טעמיה דר' חייא משום דה"ל טומאה סרוחה. וצ"ע. 
\textbf{שפיר שטרפו במימיו.} גרסי' וכן בפר"ח ז"ל, ופי' שטרף השפיר ונמוקה צורתו אבל עדיין הוא קיים נעשה כמת שנתבלבלה צורתו באור וטהורים דכיון שאין באבריו צורת בשר ולא צורת עצם נפק ליה מדין כזית ועצם כשעורה וטהורין לגמרי. 
 והא דא"ר יוחנן \textbf{מת שנתבלבלה צורתו מנ"ל דטהור.} לאו דוקא דלא כרבנן אלא מדר' אליעזר שמע ליה ר' יוחנן דקסבר מודו ליה רבנן בשלא נעשה אפר כדפרישי'.\par ויש לפרש דקסב' רבינא דר"ילא מודה לי' לר"ל בשפיר שטרפו מימיו דמדלא א"ל בשלמא שפיר שנטרפו מימיו דקאמר טהור לחיי אלא נתבלבלה צורתו שלמה מנלן אלמא ה"ק מנלן דטהור דגמרת מינה לשפיר לא הא ולא הא איתנהו. ועלה קאמר רבינא דר"י דמטמא שפיר שנטרפו מימיו לגמרי כר' אליעזר אמרה דהאי כאפר שרופין הוא ומיהו במת שנתבלבל' צורתו דקא מתמה מנלן לד"ה אתיא.\par וזה הלשון לדברי מי שגורס שפיר שנטרפו מימיו דמשמע שנטרף לגמרי וחזר למים, אבל לפי גר"ח ז"ל שנטרפו במימיו. נראה דהיינו נתבלבלה צורתו בלחוד.\par ויש לי עוד לומר דר' יוחנן הלכה קא מיבעי ליה, וה"ק ליה מנלן דטהור כרבנן דילמא טמא כר' אליעזר דמסתברא טעמיה. אילימא מדרבי שבתאי קא גמרת הלכה דהוא אמורא וקא פסיק הלכה כרבנן. 
\clearpage}

\newsection{דף כח}
\twocol{\textbf{מעשה היה וטהרו לו פתחים קטנים.} פרש"י ז"ל טמאו לו פתחים גדולים של ד' טפחים וטהרו לו קטני' הפחותים מד' כשאר מתים גדולים שהפתחים הגדולים מצילין על הקטנים דקי"ל פתחו בד'. וה"מ להציל על הפחות מד' טפחים והכי אמרינן במס' אהלות המת פתחו בד' בד"א להציל על הפתחים אבל להוציא את הטומאה בפותח טפח. זה כתב הרב ז"ל.\par ואין הדין הזה אלא כשהפתחים כולן סתומים או מגופין שבהן שנינו המת בבית ולו פתחים הרבה כולן נעולין כולן טמאין, פי' משום דסוף טומאה לצאת נפתח א' מהם אע"פ שלא חשב עליו טיהר את כולן פירש כיון דנעולין הן וה"ה למגופין אבל בפתוחין כל פותח טפח מוציא טומא' לצד ב' ואין לו הצלה כלל. 
\textbf{המפלת יד חתוכה.} פי' רש"י ז"ל חתוכה שיש לה חיתוך אצבעות. וק"ל בלאו הכי נמי ליחוש ללידה שהרי אפילו השפיר שאין לו אפילו חתוך ידים עצמן אמו טמאה לידה.\par ואיכא למימר הכי ספיקא הוא ואם הפילה יד גמורה שאינה חתוכה אומרי' מגוף אטום באת כשם שהיא משונה כך באת מגוף משונה ושמא לאו מגוף באת אלא חתיכה של בשר שנעשית כמין פיסת היד היא הילכך אמו טהורה תולין להקל שרגלים לדבר.\par והרב ר' אברהם בר דוד ז"ל מפרש שלא אמרו חתוכה אלא לענין מביאה קרבן ונאכל דמדקתני ואין חוששין כלל במשמע ואלו בשאינה חתוכה אינו נאכל. (אלא) א) לענין האם טמאה מ"מ. ואין זה לשון הגון מדקתני ברייתא אמו טמא' ואין חוששין ולא קתני מביאה קרבן ונאכל ואין חוששין. 
\clearpage}

\newsection{דף כט}
\twocol{והא ד\textbf{אמ' רב פפא כתנאי יצא מחותך או מסורס.} אלישנא קמא דר"א ור' יוחנן קאי דפליגי במחותך.\par והא דאקשי ליה רב זביד למאי דמוקי ר' יוסי אומר משיצא רובו כתקנו מכלל דמסורס רובו נמי לא פטר קשיא ולימ' ר' יוסי אכתקנו פליג דלא מיפטר בראש עד שיצא רובו ומחמיר הוא.\par וי"ל מדקתני ברייתא בדר' יוסי משיצא כתקנו משמע דלאיפלוגי אמסורס אתא דה"ל למימר ר' יוסי אומר כתקנו משיצא רובו וליפלוג אכתקנו והשתא פלוגתא אמסור' משמע ואוקומא רב זביד משיצא לתקנו בחיים ופלוגתא בדלחיים היא.\par והיינו נמי תנאי דת"ק סבר מחותך ומסורס משיצא רובו הרי זה כילוד כתקנו אע"פ שמחותך הראש פוטר ר' יוסי אומר משיצא כתקנו לחיים. כלומר אין הראש פוטר במחותך אלא בשלם שכיוצא בו יוצא כתקנו לחיים ולכ"ע לית להו דשמואג אלא ס"ל בשלם הראש פוטר והא דקתני איזהו כתקנו לחיים משיצא ראשו ה"ק איזהו כתקנו שיוציא ראשו תחלה ויוציא כדרך שהחיים מוציאין שר' יוסי אומר רוב ראשו וזהו שיעור אבל משיצא ראשו לאו לשיעור קתני אלא לדרך לידה קתני.\par ואם באנו לפרש משיצא כתקנו לחיים לאפוקי נפל אפילו שלם כדשמואל ומאי כתנאי אפי' ללישנא בתרא דר"א ור"י תיקשי לן במס' בכורות פ' יש בכור לנחל' סלקא דשמואל בתיובתא ולא איתוקמא התם כתנאי. אלא שיש כיוצא בה בתלמו' תיובתא בחד דוכתא ותנאי בדוכתא אחריתי במס' תמורה. ועוד יש במסכת פסחים תיובתא בדריש שמעתא ותניא כותיה בשלהי דידה בפ' ערבי פסחים. אלא שאנו קיימנו שתיהן תיובתא. ותניא כותיה בשתי שמועות של ר' יוחנן וכבר כתבנו זה בספר המלחמות. 
 הא ד\textbf{אמר ריב"ל עברה בנהר והפילה וכו'.} בדין הוא דנירמי עליה מהא דתניא בריש פירקין ולשלישי הפילה ואינה יודעת מה הפילה מביאה קרבן ואינו נאכל אלמא לכ"ע הלך אחר רוב נשים לא אמרינן אלא מתוקמא ההיא כדתרצינן למתני' בשלא הוחזקה עוברה לפנינו. ודמתני' עדיפא לן למירמי. 
הא דתניא ב\textbf{אשה שיצאה מלאה ובאת ריקנית} דמחזקי' לה ביולדת בזוב וברואה נמי לאחר לידה כדמקשי' לקמן יומא קמא דאתיא לקמן ליטבלה דילמא שומרת יום כנגד יום היא אלמא ברואה השתא בימי זיבה מחזקין לה דוקא כגון שבאת ריקנית ואמרה ראיתי שלא בשעת לידה ואינה יודעת כמה ראיתי ואלו לא אמרה כן אינן מחזקינן אותה לא ביולדת בזוב ולא בשומרת יום אע"פ שלא בדקה כל אותן הימים שאין חוששין לראיה כל זמן שלא ידעה אלא בימי הוסת.\par וה"נ משמע בפ' בתרא דתניא בטועה ראיתי ואינה יודעת כמה ראיתי אלמא איני יודע אם ראיתי לא כלום הוא שאם אין אתה אומר כן השוטה והחרשת והקטנה נמי שראתה אסורות לשמש לעולם שמא ראו והן אינן מרגישות ולא יודעות. 
 והא דפריך מינה ר' יוסי בר חנינא ורבין לא ידע מאי תיובתיה משום \textbf{אימר הרחיקה לידתה.} דמשמע דלית ליה לר' יוסי בר חנינא הרחיקה לידתה קשיא טובא וכי היאך סלקה על דעתו כן. והא אי לאו משום הך תשש לא היו מבטלין אותה בשבוע ג' בליליותא דמשום טבולת יום ארוך מטבילין אות' שמא כבר עברו לה ימי טוהר וכ"ש בשבו' ד' דאיכא למימר כבר עברו וכן טבילות דב"ה נמי משום יולדת והרחיק' לידתה ז' או שבועים הן ועוד שבוע דטהור הוא תשמש דאי ילדה ולד מעליא אפי' בשבוע ד' נמי טהורה היא דדם טוהר הוא וכ"ש בה' דטהור' ואם לאו ולד מעליא הוה לספיקה דר' יוסי בר חנינא תחלת שבוע רביעי ה"ל תחלת ונדה ושבוע דטהור הוא מותרת אלא ע"כ משום הרחיקה לידת' וחוששין שמא כלו ימי טוהר בסוף שבוע רביעי ויום אחרון שבו היה לה התחלת נדה כדמפר' ואזיל בגמ'.\par אלא ע"כ ר' יוסי בר חנינא אגב חורפיה לא עיין בה ובגמ' ה"ל למימר ולטעמיך מ"ט דכל הני אלא אשכחן כמה דוכתי בתלמודא דהוה מצי למיפרך וליטעמיך ולא פריך ביה כלל. 
\textbf{שבוע קמא מטבילין לה בלילותא משום יולדת זכר ונקבה.} עיינו בתוספות שאין השבועין הנמצאין כאן בטבילות הלילו' שוים עם השבועין הנמנים כאן בטביל' הימים דהא למאי דס"ד מעיקרא שבאת לפנינו ביום וכן למאי דמתרצינן כגון שבאת לפנינו בין השמשות טבילות דלילותא מושכות עד לילה של שבוע שלאחריו כגון שבאת לפנינו בין השמשות של מוצאי שבת וכן שבאת לפנינו באחד בשבת ביום וטבילה ראשונה של לילה בליל שני בשבת ואחרונה במוצאי שבת וכן בשבוע שני ואלו טבילו' דימים דמשום זיבה ראשונה באחד בשבת ואחרונה בשבת. וליכא למימר דברייתא הכי קתני שהביאה לפנינו ג' שבועין טהורין חוץ מיום שבאת לפנינו שהרי אותו היום עילה הוא למנין שבועים ונמצאת זאת מותרת לשמש בלילי עשרין וחד שהרי אינה רואה כל אותה הליל' ולא יום שלאחריו אלא ע"כ יום שבאת לפנינו הוא ממנין שלשה שבועים טהורין. כל זה עיינו בתוספות.\par ודבר ברור הוא אלא כיון דמנין לידה וזיבה מיום א' בשבת הוא וכל טבילו' דעלמא הן דכל נדה ויולדת טבילתן בלילה של שבוע שני וטבילות דזיבה ביום בסוף שבוע שלהן לא חיישי בגמרא לפרושי הכא מידי. 
\clearpage}

\newsection{דף ל}
\twocol{\textbf{כגון שבאת לפנינו בין השמשות.} פי' רש"י הוא הדין דה"ל לאוקמ' בבאה לפנינו בלילה והוה ניחא טפי דתו לא הוה קשיא לקמן לסוף שבוע לטבלה ביממא דהא לא הוו ז' ספורים שהרי לא הפסיקה טהרה בתחלת היום ואין אותו יום שבאה לפנינו עולה לה לספירת נקיים אלא מדקתני ברייתא ג' שבועין טהורים משמע דכולן טהורים ובבין השמשות משכחת בהו מיהא פסיקת טהרה ואפי' ליום ראשון. כך פי' רש"י ז"ל. 
 ל"ה טבילות דקאמרי ב"ה קשיא לן כיון דאוקים ב\textbf{באה לפנינו בין השמשות דיהיבנא לה טבילה בתריהן.} תלתין ושש הווין. וראיתי בפירושים דכיון דתדא בשבוע היא לא קחשיב ואינו יודע מהו שאם בא לומר דטבילה דסוף שבוע רביעי הויא חדא בשבוע לא משמע הכי דהא טבלה נמי בימים הסמוכים לה ששה עד סוף ז' ובאור שביעי של שבוע חמישי גמרה טבילותיה וטהורה ואפשר שאותה טבילה ראשונה חדא בשבוע חשיבי לה ב"ה מפני שהיא נמנת לסוף שבוע שעבר ואותו היום עצמו נמנה לנו תחל' שבוע ללידה וטבילות שאח"כ ואט"ג דב"ש מנו לה ולא חשבי חדא בשבוע אינהי דמפשי טבילו' מנו לה כיון דמצטרפא בטבילות דלילות דשבוע א' אבל ב"ה לא מנו לה.\par וה"ר אב"ד ז"ל כתב דאיכא לתרוצי דכיון דלאו פסיקא להו דאי אתאי ביממי להא טבילה לא חשיבי ב"ה כי היכי דתרצינן בטועה בפ' בתרא. וזה הלשון נכון בעיני דב"ש דקא מפשי טבילות מהדרי לאפושי בהו טובא וב"ה דלא מפשי בהו טפי לא חשיב' לה.\par ובשם הרב חתנו ז"ל תירץ דבין השמשות דר' יהודה אפליגי ב"ש וב"ה ב"ש סברי כר' יהודה דספיקא הוא וב"ה סברי כר' יוסי דבין השמשות דר' יהודה יממא הוא. ולכך ליתא לטבילה יתירתי' ועומק גדול הוא אלא תימה גדול הוא היכי שתיק מיניה תלמודא. זה לשון הרב ז"ל. 
\textbf{איידי דפתח בשבוע מסיק לה איידי דתנא טמא תנא טהור.} פי' וה"ה דלענין צ"ה טבילות דב"ש ה"נ הוה קמ"ל בעשרה שבועים כולם טמאים או טהורים אלא משום דבעי למיתנא משמשת לאור ל"ה לא קודם לכן ולא לאחר כן משום חששות דאמרן תנא הכי.\par והק' בתוספ' כיון דמנינו עשרה שבועי נפישי להו טבילות שהרי שבוע ט' דטמא הוא ג' ימים ראשונים שבו איכא לספוקינהו בסוף לידה ותחלת נדה ויום ד' תחלת נדה ונמצאת צריכה טבילה לג' ימים בשבוע עשירי. ותירצו עד סוף שמוני' קחשיב לאחר פ' לא תשיב דהא לא תננהו אלא אגב גררא. וכ"ש למאי דפרקינן בסמוך דלא מיירי ב"ש אלא בלידה. 
 והא דאקשי\textbf{יומא קמא דאתיא לקמן לטבילה דילמא שומרת יום כנג' יום היא} לאו לב"ש מקשינן דהא אינהו לא זיבה גדולה ולא זבה קטנה קחשיבי אלא יולדת בזוב בלחוד כדאמרן. אלא לב"ה בעינן דהא דמחרצינן זיבה גרידתא לא קחשיב לב"ה לא צרכינן למימר הכי אלא דמקמי תשמיש קחשיבי כולהו דלבתר תשמיש לא קחשיבי. א"נ השתא דאתית להכי ליומא דחדא בשבוע לא קחשיב הדרי' מההוא טעמא דטבילת זבה חדא בשבוע נינהו ולפום הכי אקשי' ליחשוב דשומרת יום וה"ל ג' טבילות בשבוע זו ביום כיון שבאה בין השמשות וליחשוב ומפרקינן זבה גדולה קחשיב כלומר יולדת בזוב. א"נ זבה הוא קחשיב אי מיתרמיא ליה לפני תשמיש אבל זבה קטנה לא חשיב.\par וק"ל ולימא דילמא כשילדה ראתה יום א' בימי זיבה וצריכה לשמור יום כנגד יום ואין ספירת ימי לידתה עולין לה ודאי כשם שאינן עולין לםפירת זבה גדולה וליטבי' כל שבוע קמא ביממי משום שומרת יום כנג' יום זיבה שלפני לידתה ואמאי פריך יומא קמא בלחוד.\par ואיכא למימר דקסבר האומר שהימי לידה עולין לשמור דזיבה קטנה ויולדת בזוב קטן דמקש' דלטבילה יומא קמא דילמא יולד' בזוב קטן היא והרי ספרה יום זה לפנינו בין לב"ש בין לב"ה מקשי' וכן נראה לי עיקר דימי לידה אין עולין גמירי לה לקמן בפ' בנות כותיים מדכתיב כימי נדת דותה תטמא מה ימי נידתה אין ראויין לזיבה ואין ספירת ז' עולה בהן אף ימי לידתה כן, והא ליתא אלא לספירתן דזבה גדולה אבל שימור דזבה קטנה אף בימי נדה עולה דאפשר הוא כדאיתא בשלהי בא סימן (נג, א). 
והא דאמרינן \textbf{ש"מ תלתא.} איכא למידק ולימא נמי ש"מ ד' דהא ש"מ ימי לידה שאינה רואה בהן אין עולין לה לימי זיבת' ואיכא למימר דההיא פלוגתא דאביי ורבא היא ורבה דאמר עולין קסבר הא מני ר' אליעזר הוא דאמר מסתר נמי סתרא. ולפום הכי נמי לא אמרי' ש"מ ר' אלעזר היא כדאמרינן ש"מ ר' עקיבא היא ור' שמעון היא. משום דלאביי דברי הכל אינן עולין הלכך לא פסיקא ליה. 
מתניתין \textbf{בנות כותיים נדות מעריסתן.} אוקמינן בגמרא לר"מ דחייש למיעוטא וקסבר ר"מ כותיים גירי אמת הן דהכי אסיקנא בב"ק (דף לח) לדידיה וכיון שהן גירי אמת והן מטמאות בנדה מן התורה יש לחוש לספיקן והיינו נמי דקתני אין חייבין עליהן על ביאת מקדש מפני שטומאתן בספק.\par וא"ת ולמה העמידו משנתינו לר"מ לחוד דחייש למיעוטא. והא אפי' לר' יוסי נמי אית ליה בנות הכותיים נדו' מעריסתן כדאמרינן בפ"ק דשבת (דף טז ע"ב) גבי י"ח דבר לר' יוסי בצרי להו ואמר ר' נחמן בר יצחק בנות כותים נדו' מעריסתן בו ביום גזרו כלומר גזירה בעלמא כדי שלא יטמעו בהן או גזירה משום מיעוט שהן טמאות.\par י"ל כיון דמתני' קסבר כותיים גירי אמת הן דהיינו סבריה דר"מ ור' יוסי שמעינן ליה דפליג עליה וסבר גירי אריות הן כדאיתא במנחות בפרק ר' ישמעאל (דף סו) ובמקומות אחרים הילכך ניחא לן לאוקמא לדידיה ומדינא ועוד דאיהו סתם מתני' ולא למשקל תנאי מעלמא ומשו' גזרת י"ח דבר. ועוד דקתני לה דומיא דסיפא דכותיים עצמן והתם לאו נזיר' אלא דינא הוא לחוש לספיקן.\par וזה שכתבנו לפי גרסת מקצת הספרים אבל מהרבה מהן מספרי הגאונים שלא נמצא שם במס' שבת אותה הגרס' כלל ואעפ"כ חשבון י"ח דבר עונה להן יפה. }


\addtocontents{toc}{\protect\end{multicols}}
\end{document}
