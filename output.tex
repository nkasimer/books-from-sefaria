\documentclass[12pt, openany]{book}
\usepackage[
paperheight=9in,
paperwidth=6in,
top=0.5in,
bottom=0.5in,
inner=0.7in,
outer=0.5in,
marginparsep=0.1in,
headsep=16pt
]{geometry}

\newcommand{\texttitle}{שבת}\usepackage{titlesec}
\renewcommand{\partname}[1]{}
\usepackage{resources/unnumberedtotoc}

\usepackage{fancyhdr}
\pagestyle{fancy}
\fancyhf{}
\fancyhead[LO,RE]{\thepage}
\fancyhead[CO]{}
\fancyhead[CE]{\partname\chapname \space\textendash\space \sectname}

\usepackage{paracol}
\usepackage{anyfontsize}
\usepackage{ragged2e}
\usepackage{polyglossia}
\usepackage{multicol}
\usepackage{hyperref}
\usepackage[marginal]{footmisc}
\usepackage[titles]{tocloft}
\usepackage{xifthen}
\usepackage{graphicx}
\usepackage{dblfnote}\DFNalwaysdouble

\setdefaultlanguage{hebrew}
\setotherlanguage{english}
\usepackage{fontspec}
\setmainfont{Times New Roman}
\newfontfamily\englishfont{Times New Roman}
\setsansfont{Aharoni}

\newcommand{\sethebfont}{
\fontsize{10.5pt}{13.1pt} \selectfont
}

\newcommand{\hebeng}[2]{
	\parbox{\textwidth}{\sethebfont #1}
	
	%\vspace{0.5\baselineskip}
	\parbox{\textwidth}{\raggedright\beginL\englishfont #2 \endL}
	
	\vspace{\baselineskip}
}

\newcommand{\twocol}[1]{
	{\sethebfont \begin{multicols}{2}
			#1
	\end{multicols}}	
}

\newcommand{\textblock}[1]{
{\sethebfont #1\\}	
}

\setlength{\parskip}{6pt}
\setlength\parindent{0in}

\newcommand{\chapname}{}
\newcommand{\sectname}{}

\newcommand{\newchap}[1]{
	\addcontentsline{toc}{chapter}{#1}
	\renewcommand{\chapname}{#1}
		\begin{center}
			\textbf{%
\fontsize{16pt}{16pt}\selectfont
				#1}
		\end{center}
}

\let\footnoterule\relax
\setlength\premulticols{10\baselineskip}
\setlength{\columnsep}{0.25in}

\newcommand{\newsection}[1]{
	%\addcontentsline{toc}{section}{#1}
	\renewcommand{\sectname}{#1}	
	\vspace{-\baselineskip}
	\begin{center}
		\textbf{%
\fontsize{16pt}{16pt}\selectfont
			#1}
	\end{center}
	\vspace{-\baselineskip}
	\nopagebreak
}

\newcommand{\footnotecomment}[1]{
	\renewcommand\thefootnote{}
	\footnote{\textsf{#1}}}

\newcommand{\parencomment}[1]{\footnotesize (#1)}

\newcommand{\blockcomment}[2]{ 
\vspace{\baselineskip}
\newsection{#1}
\sethebfont	\textsf{#2}
\vspace{\baselineskip}}

\newcommand{\commenta}[1]{\footnotecomment{#1}\hspace{0em}}

\newcommand{\vsnum}[1]{(\hebrewnumeral{#1})\space}
\newcommand{\vsnumeng}[1]{(#1)\space}

\begin{document}
\frontmatter
\pagenumbering{roman}

\newcommand{\oneline}[1]{%
	\newdimen{\namewidth}%
	\setlength{\namewidth}{\widthof{#1}}%
	\ifthenelse{\lengthtest{\namewidth < \textwidth}}%
	{#1}% do nothing if shorter than text width
	{\resizebox{\textwidth}{!}{#1}}% scale down
}

\title{\oneline{\hspace*{0.5in}\texttitle\hspace*{0.5in}}}

\author{}

\date{}

\maketitle

\begin{minipage}[b][\textheight][b]{\textwidth}\englishfont\footnotesize
	\begin{english}
		\vfill
		The following book includes:
\begin{itemize}
\item[$\bullet$] Wikisource Talmud Bavli
\begin{itemize}
\item[$\bullet$] License: CC-BY-SA
\item[$\bullet$] Source: \url{http://he.wikisource.org/wiki/%D7%AA%D7%9C%D7%9E%D7%95%D7%93_%D7%91%D7%91%D7%9C%D7%99}
\end{itemize}
\item[$\bullet$] Vilna Edition
\begin{itemize}
\item[$\bullet$] License: Public Domain
\item[$\bullet$] Source: \url{http://primo.nli.org.il/primo_library/libweb/action/dlDisplay.do?vid=NLI&docId=NNL_ALEPH001300957}
\end{itemize}
\item[$\bullet$] Chiddushei HaRamban, Jerusalem 1928\textendash 29
\begin{itemize}
\item[$\bullet$] License: Public Domain
\item[$\bullet$] Source: \url{http://primo.nli.org.il/primo_library/libweb/action/dlDisplay.do?vid=NLI&docId=NNL_ALEPH001294828}
\end{itemize}
\item[$\bullet$] Gerlitz edition, published by Oraita
\begin{itemize}
\item[$\bullet$] License: Public Domain
\item[$\bullet$] Source: \url{https://www.sefaria.org}
\end{itemize}
\end{itemize}
		It was retrieved from Sefaria on \today\space \texthebrew{(\Hebrewtoday)}.  It was typeset and formatted by Ktavi.
		\clearpage
		
	\end{english}
\end{minipage}

\titleformat{\chapter}[hang]{\huge\bfseries}{\thechapter.}{1em}{}
\titlespacing*{\chapter}{0pt}{-3em}{1.1\parskip}
\titlelabel{\thetitle\quad}
%\addtocontents{toc}{\protect\vspace{-\baselineskip}}
\addtocontents{toc}{\protect\begin{multicols}{2}}
%\vspace*{-5\baselineskip}
{\small \tableofcontents}


\clearpage
\mainmatter
\pagenumbering{arabic}

\addpart{שבת}\renewcommand{\partname}[1]{שבת}
\fancyhead[CO]{\partname}
\fancyhead[CE]{\chapname}
\renewcommand{\sethebfont}{\fontsize{14pt}{21.0pt} \selectfont}\sethebfont
\newchap{פרק \hebrewnumeral{3} כירה}
\end{multicols}\newpage

\newsection{דף לז}
\begin{multicols}{2}
לעולם אימא לך להחזיר תנן וחסורי מיחסרא והכי קתני כירה שהסיקוה בקש ובגבבא מחזירין עליה תבשיל בגפת ובעצים לא יחזיר עד שיגרוף או עד שיתן את האפר אבל לשהות משהין אע״פ שאינו גרוף ואינו קטום ומה הן משהין בית שמאי אומרים חמין אבל לא תבשיל ובית הלל אומרים חמין ותבשיל והך חזרה דאמרי לך לאו דברי הכל היא אלא מחלוקת בית שמאי ובית הלל שבית שמאי אומרים נוטלין ולא מחזירין ובית הלל אומרים אף מחזירין
ת״ש דאמר ר׳ חלבו א״ר חמא בר גוריא אמר רב לא שנו אלא על גבה אבל לתוכה אסור אי אמרת בשלמא להחזיר תנן היינו דשני בין תוכה לעל גבה אלא אי אמרת לשהות תנן מה לי תוכה מה לי על גבה מי סברת ר׳ חלבו ארישא קאי אסיפא קאי ובית הלל אומרים אף מחזירין ואמר ר׳ חלבו אמר רב חמא בר גוריא אמר רב לא שנו אלא על גבה אבל תוכה אסור
תא שמע ב׳ כירות המתאימות אחת גרופה וקטומה ואחת שאינה גרופה ואינה קטומה משהין על גבי גרופה וקטומה ואין משהין על שאינה גרופה ואינה קטומה ומה הן משהין בית שמאי אומרים ולא כלום ובית הלל אומרים חמין אבל לא תבשיל עקר דברי הכל לא יחזיר דברי רבי מאיר רבי יהודה אומר בית שמאי אומרים חמין אבל לא תבשיל ובית הלל אומרים חמין ותבשיל בית שמאי אומרים נוטלין אבל לא מחזירין ובית הלל אומרים אף מחזירין
אי אמרת בשלמא לשהות תנן מתני׳ מני רבי יהודה היא אלא אי אמרת להחזיר תנן מתניתין מני לא רבי יהודה ולא ר׳ מאיר אי רבי מאיר קשיא לב״ש בחדא ולבית הלל בתרתי אי רבי יהודה קשיא גרופה וקטומה
לעולם אימא לך להחזיר תנן ותנא דידן סבר לה כרבי יהודה בחדא ופליג עליה בחדא סבר לה כרבי יהודה בחדא בחמין ותבשיל ונוטלין ומחזירין ופליג עליה בחדא דאילו תנא דידן סבר לשהות ואף על פי שאינו גרוף וקטום ורבי יהודה סבר בלשהות נמי גרוף וקטום אין אי לא לא
איבעיא להו מהו לסמוך בה תוכה וגבה אסור אבל לסמוך בה שפיר דמי או דילמא לא שנא
תא שמע שתי כירות המתאימות אחת גרופה וקטומה ואחת שאינה גרופה וקטומה משהין על גבי גרופה וקטומה ואף על גב דקא סליק ליה הבלא מאידך דילמא שאני התם דכיון דמידליא שליט בה אוירא
תא שמע דאמר רב ספרא אמר רב חייא קטמה ונתלבתה סומכין לה ומקיימין עליה ונוטלין ממנה ומחזירין לה שמע מינה לסמוך נמי קטמה אין לא קטמה לא ולטעמיך נוטלין ממנה דקתני קטמה אין לא קטמה לא אלא תנא נוטלין משום מחזירין הכא נמי תנא סומכין משום מקיימין
הכי השתא התם נוטלין ומחזירין בחד מקום הוא תנא נוטלין משום מחזירין אלא הכא סומכין בחד מקום הוא ומקיימין בחד מקום הוא
מאי הוי עלה ת״ש כירה שהסיקוה בגפת ובעצים סומכין לה ואין מקיימין אא״כ גרופה וקטומה גחלים שעממו או שנתן עליה נעורת של פשתן דקה הרי היא כקטומה
אמר ר׳ יצחק בר נחמני א״ר אושעיא קטמה והובערה משהין עליה חמין שהוחמו כל צורכן ותבשיל שבישל כל צורכו
שמע מינה מצטמק ויפה לו מותר שאני הכא דקטמה אי הכי מאי למימרא הובערה איצטריכא ליה מהו דתימא כיון דהובערה הדרא לה למילתא קמייתא קמשמע לן
אמר רבה בר בר חנה אמר רבי יוחנן קטמה והובערה משהין עליה חמין שהוחמו כל צורכן ותבשיל שבישל כל צורכו ואפי׳ גחלים של רותם ש״מ מצטמק ויפה לו מותר שאני הכא דקטמה אי הכי מאי למימרא הובערה אצטריכא ליה היינו הך גחלים של רותם אצטריכא ליה
אמר רב ששת אמר ר׳ יוחנן כירה שהסיקוה בגפת ובעצים משהין עליה חמין שלא הוחמו כל צורכן ותבשיל שלא בישל כל צורכו עקר לא יחזיר עד שיגרוף או עד שיתן אפר קסבר מתניתין להחזיר תנן אבל לשהות משהין אע״פ שאינו גרוף ואינו קטום
אמר רבא תרווייהו תננהי לשהות תנינא אין נותנין את הפת בתוך התנור עם חשיכה ולא חררה על גבי גחלים אלא כדי שיקרמו פניה הא קרמו פניה שרי להחזיר נמי תנינא בית הלל אומרים אף מחזירין ועד כאן לא קשרו בית הלל אלא בגרופה וקטומה אבל בשאינה גרופה וקטומה לא ורב ששת נמי דיוקא דמתני׳ קמ״ל
אמר רב שמואל בר יהודה אמר רבי יוחנן כירה שהסיקוה בגפת ובעצים משהין עליה תבשיל שבישל כל צורכו וחמין שהוחמו כל צורכן ואפי׳ מצטמק ויפה לו אמר ליה ההוא מדרבנן לרב שמואל בר יהודה הא רב ושמואל דאמרי תרווייהו מצטמק ויפה לו אסור
אמר ליה אטו לית אנא ידע דאמר רב יוסף אמר רב יהודה אמר שמואל מצטמק ויפה לו אסור כי קאמינא לך לרבי יוחנן קאמינא אמר ליה רב עוקבא ממישן לרב אשי אתון דמקרביתו לרב ושמואל עבידו כרב ושמואל אנן נעביד כרבי יוחנן
אמר ליה אביי לרב יוסף מהו לשהות אמר ליה הא רב יהודה משהו ליה ואכיל א״ל בר מיניה דרב יהודה דכיון דמסוכן הוא אפילו בשבת נמי שרי למעבד ליה לי ולך מאי אמר ליה בסורא משהו דהא רב נחמן בר יצחק מרי דעובדא הוה ומשהו ליה ואכיל אמר רב אשי קאימנא קמיה דרב הונא ושהין ליה כסא דהרסנא ואכל ולא ידענא אי משום דקסבר מצטמק ויפה לו מותר אי משום דכיון דאית ביה מיחא מצטמק ורע לו הוא
אמר רב נחמן מצטמק ויפה לו אסור מצטמק ורע לו מותר כללא דמלתא כל דאית ביה מיחא מצטמק ורע לו לבר מתבשיל דליפתא דאף על גב דאית ביה מיחא מצטמק ויפה לו הוא והני מילי דאית ביה בשרא אבל לית ביה בשרא מצטמק ורע לו הוא וכי אית ביה בשרא נמי לא אמרן אלא דלא קבעי לה לאורחין אבל קבעי לה לאורחין מצטמק ורע לו לפדא דייסא ותמרי מצטמק ורע להן
בעו מיניה מרבי חייא בר אבא
\end{multicols}\newpage

\newsection{דף לח}
\begin{multicols}{2}
שכח קדירה על גבי כירה ובשלה בשבת מהו אישתיק ולא א״ל ולא מידי למחר נפק דרש להו המבשל בשבת בשוגג יאכל במזיד לא יאכל ולא שנא
מאי ול״ש רבה ורב יוסף דאמרי תרווייהו להיתירא מבשל הוא דקא עביד מעשה במזיד לא יאכל אבל האי דלא קא עביד מעשה במזיד נמי יאכל רב נחמן בר יצחק אמר לאיסורא מבשל הוא דלא אתי לאיערומי בשוגג יאכל אבל האי דאתי לאיערומי בשוגג נמי לא יאכל
מיתיבי שכח קדירה על גבי כירה ובישלה בשבת בשוגג יאכל במזיד לא יאכל בד״א בחמין שלא הוחמו כל צורכן ותבשיל שלא בישל כל צורכו אבל חמין שהוחמו כל צורכן ותבשיל שבישל כל צורכו בין בשוגג בין במזיד יאכל דברי ר״מ
ר׳ יהודה אומר חמין שהוחמו כל צורכן מותרין מפני שמצטמק ורע לו ותבשיל שבישל כל צורכו אסור מפני שמצטמק ויפה לו וכל המצטמק ויפה לו כגון כרוב ופולים ובשר טרוף אסור וכל המצטמק ורע לו מותר
קתני מיהא תבשיל שלא בישל כל צורכו בשלמא לרב נחמן בר יצחק לא קשיא כאן קודם גזרה כאן לאחר גזרה אלא רבה ורב יוסף דאמרי להיתירא אי קודם גזרה קשיא מזיד אי לאחר גזרה קשיא נמי שוגג קשיא
מאי גזירתא דאמר רב יהודה בר שמואל א״ר אבא אמר רב כהנא אמר רב בתחילה היו אומרים המבשל בשבת בשוגג יאכל במזיד לא יאכל וה״ה לשוכח משרבו משהין במזיד ואומרים שכחים אנו חזרו וקנסו על השוכח
קשיא דר׳ מאיר אדר׳ מאיר קשיא דר׳ יהודה אדר׳ יהודה דר׳ מאיר אדר״מ לא קשיא הא לכתחילה הא דיעבד דר׳ יהודה אדר׳ יהודה נמי לא קשיא כאן בגרופה וקטומה כאן בשאינה גרופה וקטומה
איבעיא להו עבר ושהה מאי מי קנסוהו רבנן או לא ת״ש דאמר שמואל בר נתן א״ר חנינא כשהלך רבי יוסי לציפורי מצא חמין שנשתהו על גבי כירה ולא אסר להן ביצים מצומקות שנשתהו על גבי כירה ואסר להן מאי לאו לאותו שבת לא לשבת הבאה
מכלל דביצים מצומקות מצטמקות ויפה להן נינהו אין דאמר רב חמא בר חנינא פעם אחת נתארחתי אני ורבי למקום אחד והביאו לפנינו ביצים מצומקות כעוזרדין ואכלנו מהן הרבה:
ב״ה אומרים אף מחזירין: אמר רב ששת לדברי האומר
מחזירין אפילו בשבת ואף ר׳ אושעיא סבר אף מחזירין אפי׳ בשבת דא״ר אושעיא פעם אחת היינו עומדים לעילא מר׳ חייא רבה והעלנו לו קומקמוס של חמין מדיוטא התחתונה לדיוטא העליונה ומזגנו לו את הכוס והחזרנוהו למקומו ולא אמר לנו דבר
א״ר זריקא א״ר אבא א״ר תדאי לא שנו אלא שעודן בידו אבל הניחן ע״ג קרקע אסור א״ר אמי ר׳ תדאי דעבד לגרמיה הוא דעבד אלא הכי א״ר חייא א״ר יוחנן אפילו הניחה על גבי קרקע מותר
פליגי בה רב דימי ורב שמואל בר יהודה ותרוייהו משמיה דרבי אלעזר אמרי חד אמר עודן בידו מותר ע״ג קרקע אסור וחד אמר הניחן על גבי קרקע נמי מותר אמר חזקיה משמיה דאביי הא דאמרת עודן בידו מותר לא אמרן אלא שדעתו להחזיר אבל אין דעתו להחזיר אסור מכלל דעל גבי קרקע אע״פ שדעתו להחזיר אסור
איכא דאמרי אמר חזקיה משמיה דאביי הא דאמרת על גבי קרקע אסור לא אמרן אלא שאין דעתו להחזיר אבל דעתו להחזיר מותר מכלל שבעודן בידו אע״פ שאין דעתו להחזיר מותר בעי ר׳ ירמיה תלאן במקל מהו הניחן על גבי מטה מהו בעי רב אשי פינן ממיחם למיחם מהו תיקו:
{\large\emph{מתני׳}} תנור שהסיקוהו בקש ובגבבא לא יתן בין מתוכו בין מעל גביו כופח שהסיקוהו בקש ובגבבא ה״ז ככיריים בגפת ובעצים הרי הוא כתנור:
{\large\emph{גמ׳}} תנור שהסיקוהו סבר רב יוסף למימר תוכו תוכו ממש על גביו על גביו ממש אבל לסמוך שפיר דמי איתיביה אביי כופח שהסיקוהו בקש ובגבבא הרי הוא ככיריים בגפת ובעצים הרי הוא כתנור ואסור הא ככירה שרי במאי עסקינן אילימא על גביו ובמאי אילימא כשאינו גרוף וקטום אלא כירה כי אינה גרופה וקטומה על גביו מי שרי אלא לאו לסמוך וקתני הרי הוא כתנור ואסיר
אמר רב אדא בר אהבה הכא בכופח גרוף וקטום ותנור גרופה וקטומה עסקינן הרי הוא כתנור דאע״ג דגרוף וקטום על גביו אסור דאי ככירה כי גרופה וקטומה שפיר דמי: תניא כוותיה דאביי תנור שהסיקוהו בקש ובגבבא אין סומכין לו ואין צריך לומר על גביו ואין צריך לומר לתוכו ואין צריך לומר בגפת ובעצים כופח שהסיקוהו בקש ובגבבא סומכין לו ואין נותנין על גביו בגפת ובעצים אין סומכין לו
אמר ליה רב אחא בריה דרבא לרב אשי האי כופח היכי דמי אי ככירה דמי אפילו בגפת ובעצים נמי אי כתנור דמי אפילו בקש ובגבבא נמי לא אמר ליה נפיש הבליה מדכירה וזוטר הבליה מדתנור
היכי דמי כופח היכי דמי כירה א״ר יוסי בר חנינא כופח מקום שפיתת קדרה אחת כירה מקום שפיתת שתי קדרות אמר אביי ואיתימא רבי ירמיה אף אנן נמי תנינא כירה שנחלקה לאורכה טהורה לרחבה טמאה כופח בין לאורכו בין לרוחבו טהור:
{\large\emph{מתני׳}} אין נותנין ביצה בצד המיחם בשביל שתתגלגל ולא יפקיענה בסודרין ור׳ יוסי מתיר ולא יטמיננה בחול ובאבק דרכים בשביל שתצלה
מעשה שעשו אנשי טבריא והביאו סילון של צונן לתוך אמה של חמין אמרו להם חכמים אם בשבת כחמין שהוחמו בשבת ואסורין ברחיצה ובשתיה אם ביום טוב כחמין שהוחמו ביום טוב ואסורין ברחיצה ומותרין בשתיה:
{\large\emph{גמ׳}} איבעיא להו גלגל מאי אמר רב יוסף גלגל חייב חטאת אמר מר בריה דרבינא אף אנן נמי תנינא
\end{multicols}\newpage

\newsection{דף לט}
\begin{multicols}{2}
כל שבא בחמין מלפני השבת שורין אותו בחמין בשבת וכל שלא בא בחמין מלפני השבת מדיחין אותו בחמין בשבת חוץ מן המליח ישן וקולייס האיספנין שהדחתן זו היא גמר מלאכתן ש״מ:
ולא יפקיענה בסודרין: והא דתנן נותנין תבשיל לתוך הבור בשביל שיהא שמור ואת המים היפים ברעים בשביל שיצננו ואת הצונן בחמה בשביל שיחמו לימא רבי יוסי היא ולא רבנן
אמר רב נחמן בחמה דכ״ע לא פליגי דשרי בתולדות האור כ״ע לא פליגי דאסיר כי פליגי בתולדות החמה מר סבר גזרינן תולדות החמה אטו תולדות האור ומר סבר לא גזרינן:
ולא יטמיננה בחול: וליפלוג נמי ר׳ יוסי בהא רבה אמר גזרה שמא יטמין ברמץ רב יוסף אמר מפני שמזיז עפר ממקומו מאי בינייהו איכא בינייהו עפר תיחוח
מיתיבי רשב״ג אומר מגלגלין ביצה על גבי גג רותח ואין מגלגלין ביצה על גבי סיד רותח בשלמא למאן דאמר גזרה שמא יטמין ברמץ ליכא למיגזר אלא למאן דאמר מפני שמזיז עפר ממקומו ליגזר סתם גג לית ביה עפר
ת״ש מעשה שעשו אנשי טבריא והביאו סילון של צונן לתוך אמה של חמין וכו׳ בשלמא למאן דאמר גזרה שמא יטמין ברמץ היינו דדמיא להטמנה אלא למאן דאמר מפני שמזיז עפר ממקומו מאי איכא למימר
מי סברת מעשה טבריא אסיפא קאי ארישא קאי לא יפקיענה בסודרין ור׳ יוסי מתיר והכי קאמרי ליה רבנן לר׳ יוסי הא מעשה דאנשי טבריא דתולדות חמה הוא ואסרי להו רבנן אמר להו ההוא תולדות אור הוא דחלפי אפיתחא דגיהנם
אמר רב חסדא
ממעשה שעשו אנשי טבריא ואסרי להו רבנן בטלה הטמנה בדבר המוסיף הבל ואפילו מבעוד יום אמר עולא הלכה כאנשי טבריא א״ל רב נחמן כבר תברינהו אנשי טבריא לסילונייהו:
מעשה שעשו אנשי טבריא: מאי רחיצה אילימא רחיצת כל גופו אלא חמין שהוחמו בשבת הוא דאסורין הא חמין שהוחמו מע״ש מותרין והתניא חמין שהוחמו מע״ש למחר רוחץ בהן פניו ידיו ורגליו אבל לא כל גופו אלא פניו ידיו ורגליו
אימא סיפא בי״ט כחמין שהוחמו בי״ט ואסורין ברחיצה ומותרין בשתיה לימא תנן סתמא כבית שמאי דתנן בית שמאי אומרים לא יחם אדם חמין לרגליו אא״כ ראויין לשתיה וב״ה מתירין
א״ר איקא בר חנניא להשתטף בהן כל גופו עסקינן והאי תנא הוא דתניא לא ישתטף אדם כל גופו בין בחמין ובין בצונן דברי ר״מ ר״ש מתיר ר׳ יהודה אומר בחמין אסור בצונן מותר
אמר רב חסדא מחלוקת בכלי אבל בקרקע דברי הכל מותר והא מעשה דאנשי טבריא בקרקע הוה ואסרי להו רבנן אלא אי איתמר הכי איתמר מחלוקת בקרקע אבל בכלי דברי הכל אסור
אמר רבה בר בר חנה אמר רבי יוחנן הלכה כרבי יהודה א״ל רב יוסף בפירוש שמיע לך או מכללא שמיע לך מאי כללא דאמר רב תנחום א״ר יוחנן א״ר ינאי אמר (רב) כל מקום שאתה מוצא שנים חלוקין ואחד מכריע הלכה כדברי המכריע חוץ מקולי מטלניות שאף על פי שרבי אליעזר מחמיר ורבי יהושע מיקל ור׳ עקיבא מכריע אין הלכה כדברי המכריע חדא דרבי עקיבא תלמיד הוא ועוד הא
\end{multicols}\newpage

\newsection{דף מ}
\begin{multicols}{2}
. הדר ביה ר״ע לגביה דרבי יהושע. ואי מכללא מאי דילמא ה״מ במתניתין אבל בברייתא לא א״ל אנא בפירוש שמיע לי
אתמר חמין שהוחמו מע״ש רב אמר למחר רוחץ בהן כל גופו אבר אבר ושמואל אמר. לא התירו לרחוץ אלא פניו ידיו ורגליו מיתיבי חמין שהוחמו מע״ש למחר רוחץ בהן פניו ידיו ורגליו אבל לא כל גופו תיובתא דרב אמר לך רב לא כל גופו בבת אחת אלא אבר אבר והא פניו ידיו ורגליו קתני כעין פניו ידיו ורגליו
תא שמע לא התירו לרחוץ בחמין שהוחמו מע״ש אלא פניו ידיו ורגליו ה״נ כעין פניו ידיו ורגליו
תניא כוותיה דשמואל חמין שהוחמו מע״ש למחר רוחץ בהן פניו ידיו ורגליו אבל לא כל גופו אבר אבר ואצ״ל חמין שהוחמו בי״ט רבה מתני לה להא שמעתא דרב בהאי לישנא חמין שהוחמו מע״ש למחר אמר רב רוחץ בהן כל גופו ומשייר אבר אחד איתיביה כל הני תיובתא תיובתא
א״ל רב יוסף לאביי רבה מי קא עביד כשמעתיה דרב א״ל לא ידענא מאי תיבעי ליה פשיטא דלא עביד דהא איתותב. דילמא) לא שמיעא ליה
ואי לא שמיעא ׳ ודאי עביד. דאמר אביי כל מילי דמר עביד כרב בר מהני תלת דעביד כשמואל מטילין מבגד לבגד ומדליקין מנר לנר והלכה כר״ש בגרירה כחומרי דרב עביד כקולי דרב לא עביד
ת״ר מרחץ שפקקו נקביו מע״ש למוצ״ש רוחץ בו מיד פקקו נקביו מעי״ט למחר נכנס ומזיע ויוצא ומשתטף בבית החיצון
אמר רב יהודה מעשה במרחץ של בני ברק שפקקו נקביו מעי״ט למחר נכנס ראב״ע ור״ע והזיעו בו ויצאו ונשתטפו בבית החיצון אלא שחמין שלו מחופין בנסרים כשבא הדבר לפני חכמים אמרו אף על פי שאין חמין שלו מחופין בנסרין ומשרבו עוברי עבירה התחילו לאסור אמבטיאות של כרכין מטייל בהן ואינו חושש
מאי עוברי עבירה דא״ר שמעון בן פזי אמר ריב״ל משום בר קפרא בתחלה היו רוחצין בחמין שהוחמו מע״ש התחילו הבלנים להחם בשבת ואומרים מערב שבת הוחמו אסרו את החמין והתירו את הזיעה ועדיין היו רוחצין בחמין ואומרים מזיעין אנחנו אסרו להן את הזיעה והתירו חמי טבריה ועדיין היו רוחצין בחמי האור ואומרים בחמי טבריה רחצנו אסרו להן חמי טבריה והתירו להן את הצונן ראו שאין הדבר עומד להן התירו להן חמי טבריה וזיעה במקומה עומדת
אמר רבא האי מאן דעבר אדרבנן שרי למיקרי ליה עבריינא כמאן
כי האי תנא
אמבטיאות של כרכים מטייל בהן ואינו חושש אמר רבא דוקא כרכין אבל דכפרים לא מ״ט כיון דזוטרין נפיש הבלייהו
ת״ר מתחמם אדם כנגד המדורה ויוצא ומשתטף בצונן ובלבד שלא ישתטף בצונן ויתחמם כנגד המדורה מפני שמפשיר מים שעליו ת״ר מיחם אדם אלונטית ומניחה על בני מעים בשבת ובלבד שלא יביא קומקומוס של מים חמין ויניחנו על בני מעים בשבת ודבר זה אפי׳ בחול אסור מפני הסכנה
ת״ר מביא אדם קיתון מים ומניחו כנגד המדורה לא בשביל שיחמו אלא בשביל שתפיג צינתן ר׳ יהודה אומר מביאה אשה פך של שמן ומניחתו כנגד המדורה לא בשביל שיבשל אלא בשביל שיפשר רשב״ג אומר אשה סכה ידה שמן ומחממתה כנגד המדורה וסכה לבנה קטן ואינה חוששת
איבעיא להו שמן מה הוא לתנא קמא רבה ורב יוסף דאמרי תרוייהו להתירא רב נחמן בר יצחק אמר לאיסורא רבה ורב יוסף דאמרי תרוייהו להתירא שמן אע״פ שהיד סולדת בו מותר קסבר ת״ק שמן אין בו משום בשול ואתא רבי יהודה למימר שמן יש בו משום בשול והפשרו לא זה הוא בשולו ואתא ר׳ שמעון בן גמליאל למימר שמן יש בו משום בשול והפשרו זהו בשולו
רב נחמן בר יצחק אמר לאיסורא שמן אע״פ שאין היד סולדת בו אסור קסבר שמן יש בו משום בשול והפשרו זהו בשולו ואתא ר׳ יהודה למימר הפשרו לא זהו בשולו ואתא רשב״ג למימר שמן יש בו משום בשול והפשרו זהו בשולו רשב״ג היינו ת״ק איכא בינייהו כלאחר יד
א״ר יהודה אמר שמואל אחד שמן ואחד מים יד סולדת בו אסור אין יד סולדת בו מותר והיכי דמי יד סולדת בו אמר רחבא כל שכריסו של תינוק נכוית
א״ר יצחק בר אבדימי פעם אחת נכנסתי אחר רבי לבית המרחץ ובקשתי להניח לו פך של שמן באמבטי ואמר לי טול בכלי שני ותן שמע מינה תלת שמע מינה שמן יש בו משום בשול וש״מ כלי שני אינו מבשל וש״מ הפשרו זהו בשולו
היכי עביד הכי והאמר רבה בר בר חנה א״ר יוחנן בכל מקום מותר להרהר חוץ מבית המרחץ ובית הכסא וכ״ת בלשון חול א״ל והאמר אביי דברים של חול מותר לאומרן בלשון קודש של קודש אסור לאומרן בלשון חול אפרושי מאיסורא שאני
תדע דאמר רב יהודה אמר שמואל מעשה בתלמידו של ר׳ מאיר שנכנס אחריו לבית המרחץ ובקש להדיח קרקע ואמר לו אין מדיחין לסוך לו קרקע אמר לו אין סכין אלמא אפרושי מאיסורא שאני הכא נמי לאפרושי מאיסורא שאני
אמר רבינא שמע מינה המבשל בחמי טבריה בשבת חייב דהא מעשה דר׳ לאחר גזירה הוה ואמר ליה טול בכלי שני ותן איני והאמר רב חסדא המבשל בחמי טבריה בשבת פטור מאי חייב נמי דקאמר מכת מרדות
א״ר זירא אנא חזיתיה לר׳ אבהו דשט באמבטי ולא ידענא אי עקר אי לא עקר פשיטא דלא עקר דתניא לא ישוט אדם בבריכה מלאה מים ואפי׳ עומדת בחצר לא קשיא הא
\end{multicols}\newpage

\newsection{דף מא}
\begin{multicols}{2}
דלית ליה גידודי הא דאית ליה גידודי:
וא״ר זירא אנא חזיתיה לר׳ אבהו שהניח ידיו כנגד פניו של מטה ולא ידענא אי נגע אי לא נגע פשיטא דלא נגע דתניא ר׳ אליעזר אומר כל האוחז באמה ומשתין כאילו מביא מבול לעולם
אמר אביי עשאוה כבולשת דתנן בולשת שנכנסה לעיר בשעת שלום חביות פתוחות אסורות סתומות מותרות בשעת מלחמה אלו ואלו מותרות לפי שאין פנאי לנסך אלמא כיון דבעיתי לא מנסכי ה״נ כיון דבעית לא אתי להרהורי והכא מאי ביעתותא ביעתותא דנהרא
איני והאמר ר׳ אבא אמר רב הונא אמר רב כל המניח ידיו כנגד פניו של מטה כאילו כופר בבריתו של אברהם אבינו לא קשיא הא כי נחית הא כי סליק כי הא דרבא שחי ר׳ זירא זקיף רבנן דבי רב אשי כי קא נחתי זקפי כי קא סלקי שחי
ר׳ זירא הוה קא משתמיט מדרב יהודה דבעי למיסק לארעא דישראל דאמר רב יהודה כל העולה מבבל לא״י עובר בעשה שנאמר (ירמיהו כז, כב) בבלה יובאו ושמה יהיו אמר איזיל ואשמע מיניה מילתא ואיתי ואיסק אזל אשכחיה דקאי בי באני וקאמר ליה לשמעיה הביאו לי נתר הביאו לי מסרק פתחו פומייכו ואפיקו הבלא ואשתו ממיא דבי באני אמר אילמלא (לא) באתי אלא לשמוע דבר זה דיי
בשלמא הביאו נתר הביאו מסרק קמ״ל דברים של חול מותר לאומרם בלשון קדש פתחו פומייכו ואפיקו הבלא נמי כדשמואל דאמר שמואל הבלא מפיק הבלא אלא אשתו מיא דבי באני מאי מעליותא דתניא אכל ולא שתה אכילתו דם וזהו תחילת חולי מעיים אכל ולא הלך ד׳ אמות אכילתו מרקבת וזהו תחילת ריח רע הנצרך. לנקביו ואכל דומה לתנור שהסיקוהו ע״ג אפרו וזהו תחילת ריח זוהמא רחץ בחמין ולא שתה מהן דומה לתנור שהסיקוהו מבחוץ ולא הסיקוהו מבפנים רחץ בחמין ולא נשתטף בצונן דומה לברזל שהכניסוהו לאור ולא הכניסוהו לצונן רחץ ולא סך דומה למים ע״ג חבית:
{\large\emph{מתני׳}} מוליאר הגרוף שותין הימנו בשבת אנטיכי אע״פ שגרופה אין שותין הימנה:
{\large\emph{גמ׳}} היכי דמי מוליאר הגרוף תנא מים מבפנים וגחלים מבחוץ אנטיכי רבה אמר בי כירי רב נחמן בר יצחק אמר בי דודי מאן דאמר בי דודי כ״ש בי כירי ומאן דאמר בי כירי אבל בי דודי. לא תניא כוותיה דרב נחמן אנטיכי אע״פ שגרופה וקטומה אין שותין הימנה מפני שנחושתה מחממתה:
{\large\emph{מתני׳}} המיחם שפינהו לא יתן לתוכו צונן בשביל שיחמו אבל נותן הוא לתוכו או לתוך הכוס כדי להפשירן:
{\large\emph{גמ׳}} מאי קאמר אמר רב אדא בר מתנא הכי קאמר המיחם שפינה ממנו מים חמין לא יתן לתוכן מים מועטים כדי שיחמו אבל נותן לתוכו מים מרובים כדי להפשירן
והלא מצרף ר׳ שמעון היא דאמר. דבר שאין מתכוין מותר מתקיף לה אביי מידי מיחם שפינה ממנו מים קתני מיחם שפינהו קתני
אלא אמר אביי הכי קאמר המיחם שפינהו ויש בו מים חמין לא יתן לתוכו מים מועטין בשביל שיחומו אבל נותן לתוכו מים מרובים כדי להפשירן ומיחם שפינה ממנו מים לא יתן לתוכו מים כל עיקר מפני שמצרף ור׳ יהודה היא דאמר דבר שאין מתכוין אסור
אמר רב ל״ש אלא להפשיר אבל לצרף אסור ושמואל אמר אפי׳ לצרף נמי מותר לצרף לכתחילה מי שרי אלא אי איתמר הכי איתמר אמר רב לא שנו אלא שיעור להפשיר אבל שיעור לצרף אסור ושמואל אמר אפ׳ שיעור לצרף
\end{multicols}\newpage

\newsection{דף מב}
\begin{multicols}{2}
מותר
למימרא דשמואל כרבי שמעון סבירא ליה והאמר שמואל מכבין גחלת של מתכת ברה״ר בשביל שלא יזוקו בה רבים אבל לא גחלת של עץ ואי ס״ד סבר לה כרבי שמעון אפילו של עץ נמי
בדבר שאין מתכוין סבר לה כרבי שמעון במלאכה שאינה צריכה לגופה סבר לה כרבי יהודה אמר רבינא הלכך קוץ ברשות הרבים מוליכו פחות פחות מד׳ אמות ובכרמלית אפילו טובא:
אבל נותן כו׳: ת״ר נותן אדם חמין לתוך הצונן ולא הצונן לתוך החמין דברי בית שמאי ובית הלל אומרים בין חמין לתוך הצונן ובין צונן לתוך החמין מותר בד״א בכוס אבל באמבטי חמין לתוך הצונן ולא צונן לתוך החמין ורבי שמעון בן מנסיא אוסר אמר רב נחמן הלכה כר״ש בן מנסיא
סבר רב יוסף למימר ספל הרי הוא כאמבטי א״ל אביי תני ר׳ חייא ספל אינו כאמבטי ולמאי דסליק אדעתא מעיקרא דספל הרי הוא כאמבטי ואמר רב נחמן הלכה כרבי שמעון בן מנסיא אלא בשבת רחיצה בחמין ליכא
מי סברת רבי שמעון אסיפא קאי ארישא קאי ובית הלל מתירין בין חמין לתוך צונן ובין צונן לתוך החמין ורבי שמעון בן מנסיא אוסר צונן לתוך חמין לימא רבי שמעון בן מנסיא דאמר כב״ש הכי קאמר לא נחלקו ב״ש וב״ה בדבר זה
אמר רב הונא בריה דרב יהושע חזינא ליה לרבא דלא קפיד אמנא מדתני רבי חייא נותן אדם קיתון של מים לתוך ספל של מים בין חמין לתוך צונן ובין צונן לתוך חמין אמר ליה רב הונא לרב אשי דילמא שאני התם דמיפסק כלי אמר ליה מערה איתמר מערה אדם קיתון של מים לתוך ספל של מים בין חמין לתוך צונן בין צונן לתוך חמין:
{\large\emph{מתני׳}} האילפס והקדרה שהעבירן מרותחין לא יתן לתוכן תבלין
אבל נותן הוא לתוך הקערה או לתוך התמחוי רבי יהודה אומר לכל הוא נותן חוץ מדבר שיש בו חומץ וציר:
{\large\emph{גמ׳}} איבעיא להו רבי יהודה ארישא קאי ולקולא או דילמא אסיפא קאי ולחומרא
ת״ש דתניא רבי יהודה אומר לכל אילפסין הוא נותן לכל הקדירות רותחות הוא נותן חוץ מדבר שיש בו חומץ וציר
סבר רב יוסף למימר מלח הרי הוא כתבלין דבכלי ראשון בשלה ובכלי שני לא בשלה א״ל אביי תני רבי חייא מלח אינה כתבלין דבכלי שני נמי בשלה ופליגא דרב נחמן דאמר רב נחמן צריכא מילחא בישולא כבשרא דתורא
ואיכא דאמרי סבר רב יוסף למימר מלח הרי הוא כתבלין דבכלי ראשון בשלה בכלי שני לא בשלה א״ל אביי תני ר׳ חייא מלח אינה כתבלין דבכלי ראשון נמי לא בשלה והיינו דאמר רב נחמן צריכא מילחא בישולא כבישרא דתורא:
{\large\emph{מתני׳}} אין נותנין כלי תחת הנר לקבל בו את השמן ואם נתנוה מבעוד יום מותר ואין ניאותין ממנו לפי שאינו מן המוכן:
{\large\emph{גמ׳}} אמר רב חסדא אע״פ שאמרו אין נותנין כלי תחת תרנגולת לקבל ביצתה אבל כופה עליה כלי שלא תשבר אמר רבה מ״ט דרב חסדא קסבר תרנגולת עשויה להטיל ביצתה באשפה ואינה עשויה להטיל ביצתה במקום מדרון והצלה מצויה התירו והצלה שאינה מצויה לא התירו
איתיבי אביי והצלה שאינה מצויה לא התירו והתניא נשברה לו חבית של טבל בראש גגו מביא כלי ומניח תחתיה בגולפי חדתי דשכיחי דפקעי
איתיביה נותנין כלי תחת הנר לקבל ניצוצות ניצוצות נמי שכיחי
\end{multicols}\newpage

\newsection{דף מג}
\begin{multicols}{2}
איתיביה כופין קערה על הנר שלא יאחז בקורה בבתי גחיני דשכיח בהו דליקה
וכן קורה שנשברה סומכין אותה בספסל ובארוכות המטה בכשורי חדתי דעבידי דפקעי
נותנין כלי תחת הדלף בשבת בבתי חדתי דשכיחי דדלפי
רב יוסף אמר היינו טעמא דרב חסדא משום דקא מבטל כלי מהיכנו
איתיביה אביי חבית של טבל שנשברה מביא כלי אחר ומניח תחתיה א״ל טבל מוכן הוא אצל שבת שאם עבר ותקנו מתוקן
נותנין כלי תחת הנר לקבל ניצוצות א״ר הונא בריה דרב יהושע ניצוצות אין בהן ממש
וכן קורה שנשברה סומכין אותה בספסל או בארוכות המטה דרפי דאי בעי שקיל ליה
נותנין כלי תחת הדלף בשבת בדלף הראוי
כופין את הסל לפני האפרוחין שיעלו וירדו קסבר מותר לטלטלו והתניא אסור לטלטלו בעודן עליו והתניא אע״פ שאין עודן עליו אסור א״ר אבהו בעודן עליו כל בין השמשות מיגו דאיתקצאי לבין השמשות איתקצאי לכולי יומא
א״ר יצחק כשם שאין נותנין כלי תחת תרנגולת לקבל ביצתה כך אין כופין עליה כלי בשביל שלא תשבר קסבר אין כלי ניטל אלא לדבר הניטל בשבת מיתיבי כל הני תיובתא ושני בצריך למקומו
תא שמע אחת ביצה שנולדה בשבת ואחת ביצה שנולדה ביום טוב אין מטלטלין לא לכסות בה את הכלי ולסמוך בה כרעי המטה אבל כופה עליה כלי בשביל שלא תשבר הכא נמי בצריך למקומו.
ת״ש פורסין מחצלות על גבי אבנים בשבת באבנים מקורזלות דחזיין לבית הכסא
ת״ש פורסין מחצלות על גבי לבנים בשבת דאישתיור מבנינא דחזיין למיזגא עלייהו
ת״ש פורסין מחצלת על גבי כוורת דבורים בשבת בחמה מפני החמה ובגשמים מפני הגשמים ובלבד שלא יתכוין לצוד הכא במאי עסקינן דאיכא דבש א״ל רב עוקבא ממישן לרב אשי תינח בימות החמה
דאיכא דבש בימות הגשמים דליכא דבש מאי איכא למימר לא נצרכא אלא לאותן שתי חלות והא מוקצות נינהו דחשיב עלייהו הא לא חשיב עלייהו מאי אסור
אי הכי הא דתני ובלבד שלא יתכוין לצוד לפלוג ולתני בדידה בד״א כשחישב עליהן אבל לא חישב עליהן אסור הא קמ״ל אע״פ שחישב עליהן ובלבד שלא יתכוין לצוד
מני אי ר״ש לית לי׳ מוקצה אי ר׳ יהודה כי לא מתכוין מאי הוי הא דבר שאין מתכוין אסור לעולם ר׳ יהודה מאי ובלבד שלא יתכוין לצוד שלא יעשנה כמצודה דלישבוק להו רווחא כי היכי דלא ליתצדו ממילא
רב אשי אמר מי קתני בימות החמה ובימות הגשמים בחמה מפני החמה ובגשמים מפני הגשמים קתני ביומי ניסן וביומי תשרי דאיכא חמה (ואיכא צינה) ואיכא גשמים ואיכא דבש
אמר להו רב ששת פוקו ואמרו ליה לר׳ יצחק כבר תרגמא רב הונא לשמעתיך בבבל דא״ר הונא עושין מחיצה למת בשביל חי ואין עושין מחיצה למת בשביל מת
מאי היא דא״ר שמואל בר יהודה וכן תנא שילא מרי מת המוטל בחמה באים שני בני אדם ויושבין בצדו חם להם מלמטה זה מביא מטה ויושב עליה וזה מביא מטה ויושב עליה חם להם מלמעלה מביאים מחצלת ופורסין עליהן זה זוקף מטתו ונשמט והולך לו וזה זוקף מטתו ונשמט והולך לו ונמצאת מחיצה עשויה מאליה
איתמר מת המוטל בחמה רב יהודה אמר שמואל הופכו ממטה למטה רב חנינא בר שלמיא משמיה דרב אמר מניח עליו ככר או תינוק ומטלטלו היכא דאיכא ככר או תינוק כולי עלמא לא פליגי דשרי כי פליגי דלית ליה מ״ס טלטול מן הצד שמיה טלטול ומ״ס לא שמיה טלטול
לימא כתנאי אין מצילין את המת מפני הדליקה אמר ר׳ יהודה בן לקיש שמעתי שמצילין את המת מפני הדליקה היכי דמי אי דאיכא ככר או תינוק מ״ט דתנא קמא אי דליכא מ״ט דר׳ יהודה בן לקיש אלא לאו בטלטול מן הצד פליגי דמר סבר טלטול מן הצד שמיה טלטול ומ״ס לא שמיה טלטול לא דכ״ע טלטול מן הצד שמיה טלטול והיינו טעמא דר׳ יהודה בן לקיש דמתוך שאדם בהול על מתו
\end{multicols}\newpage

\newsection{דף מד}
\begin{multicols}{2}
אי לא שרית ליה אתי לכבויי א״ר יהודה בן שילא א״ר אסי א״ר יוחנן הלכה כר׳ יהודה בן לקיש במת:
אין ניאותין הימנו לפי שאינו מן המוכן: תנו רבנן מותר השמן שבנר ושבקערה אסור ורבי שמעון מתיר:
{\large\emph{מתני׳}} מטלטלין נר חדש אבל לא ישן רבי שמעון אומר כל הנרות מטלטלין חוץ מן הנר הדולק בשבת:
{\large\emph{גמ׳}} ת״ר מטלטלין נר חדש אבל לא ישן דברי רבי יהודה ר״מ אומר כל הנרות מטלטלין חוץ מן הנר שהדליקו בו בשבת ר׳ שמעון אומר חוץ מן הנר הדולק בשבת כבתה מותר לטלטלה אבל כוס וקערה ועששית לא יזיזם ממקומם ור׳ אליעזר בר׳ שמעון אומר מסתפק מן הנר הכבה ומן השמן המטפטף ואפי׳ בשעה שהנר דולק
אמר אביי רבי אליעזר ברבי שמעון סבר לה כאבוה בחדא ופליג עליה בחדא סבר לה כאבוה בחדא דלית ליה מוקצה ופליג עליה בחדא דאילו אבוה סבר כבה אין לא כבה לא ואיהו סבר אע״ג דלא כבה
אבל כוס וקערה ועששית לא יזיזם ממקומם מאי שנא הני אמר עולא סיפא אתאן לר׳ יהודה
מתקיף לה מר זוטרא אי הכי מאי אבל אלא אמר מר זוטרא לעולם רבי שמעון וכי קשרי רבי שמעון בנר זוטא דדעתיה עלויה אבל הני דנפישי לא
והתניא מותר השמן שבנר ושבקערה אסור ורבי שמעון מתיר התם קערה דומיא דנר הכא קערה דומיא דכוס
א״ר זירא פמוט שהדליקו בו בשבת לדברי המתיר אסור לדברי האוסר מותר למימרא דרבי יהודה מוקצה מחמת מיאוס אית ליה מוקצה מחמת איסור לית ליה והתניא ר׳ יהודה אומר כל הנרות של מתכת מטלטלין חוץ מן הנר שהדליקו בו בשבת אלא אי איתמר הכי איתמר א״ר זירא פמוט שהדליקו עליו בשבת ד״ה אסור לא הדליקו עליו ד״ה מותר:
אמר רב יהודה א״ר מטה שיחדה למעות אסור לטלטלה מיתיבי׳ רב נחמן בר יצחק מטלטלין נר חדש אבל לא ישן
ומה נר דלהכי עבידא כי לא הדליק בה שרי לטלטולה מטה דלאו להכי עבידא לא כ״ש אלא אי איתמר הכי איתמר אמר רב יהודה אמר רב מטה שיחדה למעות הניח עליה מעות אסור לטלטלה לא הניח עליה מעות מותר לטלטלה לא יחדה למעות יש עליה מעות אסור לטלטלה אין עליה מעות מותר לטלטלה והוא שלא היו עליה בין השמשות
אמר עולא מתיב ר׳ אליעזר מוכני שלה בזמן שהיא נשמטת אין חיבור לה ואין נמדדת עמה ואין מצלת עמה באהל המת ואין גוררין אותה בשבת בזמן שיש עליה מעות
הא אין עליה מעות שריא אף על גב דהוו עליה ביה״ש ההיא ר׳ שמעון היא דלית ליה מוקצה ורב כרבי יהודה סבירא ליה
\end{multicols}\newpage

\newsection{דף מה}
\begin{multicols}{2}
הכי נמי מיסתברא דרב כר׳ יהודה סבירא ליה דאמר רב מניחין נר על גבי דקל בשבת ואין מניחין נר ע״ג דקל בי״ט אי אמרת בשלמא דרב כרבי יהודה סבירא ליה היינו דשני בין שבת לי״ט אלא אי אמרת כרבי שמעון סבירא ליה מה לי שבת ומה לי י״ט
ורב כרבי יהודה ס״ל והא בעו מיניה דרב מהו לטלטולי שרגא דחנוכתא מקמי חברי בשבתא ואמר להו שפיר דמי שעת הדחק שאני דהא א״ל רב כהנא ורב אשי לרב הכי הלכתא אמר להו כדי הוא ר׳ שמעון לסמוך עליו בשעת הדחק
בעא מיניה ריש לקיש מר׳ יוחנן חטים שזרען בקרקע וביצים שתחת תרנגולת מהו כי לית ליה לר׳ שמעון מוקצה היכא דלא דחייה בידים היכא דדחייה בידים אית ליה מוקצה או דילמא לא שנא א״ל אין מוקצה לרבי שמעון אלא שמן שבנר בשעה שהוא דולק הואיל והוקצה למצותו הוקצה לאיסורו
ולית ליה הוקצה למצותו והתניא סיככה כהלכתה ועיטרה בקרמים ובסדינין המצויירין ותלה בה אגוזין אפרסקין שקדים ורמונין ואפרכלי של ענבים ועטרות של שבולין יינות שמנים וסלתות אסור להסתפק מהן עד מוצאי י״ט האחרון ואם התנה עליהן הכל לפי תנאו
וממאי דר׳ שמעון היא דתני ר׳ חייא בר יוסף קמיה דר׳ יוחנן אין נוטלין עצים מן הסוכה בי״ט אלא מן הסמוך לה ור׳ שמעון מתיר ושוין בסוכת החג בחג שהיא אסורה ואם התנה עליה הכל לפי תנאו כעין שמן שבנר קאמרינן הואיל והוקצה למצותו הוקצה לאיסורו איתמר נמי א״ר חייא בר אבא א״ר יוחנן אין מוקצה לרבי שמעון אלא כעין שמן שבנר בשעה שהוא דולק הואיל והוקצה למצותו הוקצה לאיסורו
אמר רב יהודה אמר שמואל אין מוקצה לר׳ שמעון אלא גרוגרות וצימוקים בלבד ומידי אחרינא לא והתניא היה אוכל בתאנים והותיר והעלן לגג לעשות מהן גרוגרות בענבים והותיר והעלן לגג לעשות מהן צימוקין לא יאכל עד שיזמין וכן אתה אומר באפרסקין וחבושין ובשאר כל מיני פירות
מני אילימא רבי יהודה ומה היכא דלא דחייה בידים אית ליה מוקצה היכא דדחייה בידים לא כל שכן
אלא לאו ר׳ שמעון היא לעולם רבי יהודה ואוכל אצטריכא ליה סד״א כיון דקאכיל ואזיל לא ליבעי הזמנה קמ״ל כיון דהעלן לגג אסוחי אסחי לדעתיה מינייהו
בעא מיניה רבי שמעון בר רבי מרבי
פצעילי תמרה לרבי שמעון מהו א״ל אין מוקצה לר״ש אלא גרוגרות וצימוקין בלבד
ורבי לית ליה מוקצה והתנן אין משקין ושוחטין את המדבריות אבל משקין ושוחטין את הבייתות ותניא אלו הן מדבריות כל שיוצאות בפסח ונכנסות ברביעה בייתות כל שיוצאות ורועות חוץ לתחום ובאות ולנות בתוך התחום ר׳ אומר אלו ואלו בייתות הן ואלו הן מדבריות כל שרועות באפר ואין נכנסות לישוב לא בימות החמה ולא בימות הגשמים
איבעית אימא הני נמי כגרוגרות וצימוקין דמיין ואי בעית אימא לדבריו דר״ש קאמר ליה וליה לא סבירא ליה
ואיבעית אימא לדבריהם דרבנן קאמר להו לדידי לית לי מוקצה כלל לדידכו אודו לי מיהת דהיכא דיוצאות בפסח ונכנסות ברביעה דבייתות נינהו ורבנן אמרו ליה לא מדבריות נינהו
אמר רבה בר בר חנה אמר ר׳ יוחנן אמרו הלכה כרבי שמעון ומי א״ר יוחנן הכי והא בעא מיניה ההוא סבא קרויא ואמרי לה סרויא מר׳ יוחנן קינה של תרנגולת מהו לטלטולי בשבת אמר ליה כלום עשוי אלא לתרנגולין הכא במאי עסקינן דאית ביה אפרוח מת
הניחא למר בר אמימר משמיה (דרב) דאמר מודה היה ר׳ שמעון בבעלי חיים שמתו שאסורין אלא למר בריה דרב יוסף משמיה דרבא דאמר חלוק היה רבי שמעון [אפי׳] בבעלי חיים שמתו שהן מותרין מאי איכא למימר הכא במאי עסקינן בדאית ביה ביצה
והאמר רב נחמן מאן דאית ליה מוקצה אית ליה נולד דלית ליה מוקצה לית ליה נולד דאית ביה ביצת אפרוח
כי אתא רב יצחק בר׳ יוסף א״ר יוחנן הלכה כר׳ יהודה ור׳ יהושע בן לוי אמר הלכה כרבי שמעון אמר רב יוסף היינו דאמר רבה בר בר חנה א״ר יוחנן אמרו הלכה כרבי שמעון אמרו וליה לא סבירא ליה
א״ל אביי לרב יוסף ואת לא תסברא דר׳ יוחנן כר׳ יהודה הא ר׳ אבא ורבי אסי איקלעו לבי ר׳ אבא דמן חיפא ונפל מנרתא על גלימיה דר׳ אסי ולא טילטלה מאי טעמא לאו משום דרבי אסי תלמידיה דר׳ יוחנן הוה ור׳ יוחנן כרבי יהודה ס״ל דאית ליה מוקצה
א״ל מנרתא קאמרת מנרתא שאני דא״ר אחא בר חנינא א״ר אסי הורה ריש לקיש בצידן מנורה הניטלת בידו אחת מותר לטלטלה בשתי ידיו אסור לטלטלה ור׳ יוחנן אמר אנו אין לנו אלא בנר כרבי שמעון אבל מנורה בין ניטלה בידו אחת בין ניטלה בשתי ידיו אסור לטלטלה
וטעמא מאי רבה ורב יוסף דאמרי תרווייהו הואיל ואדם קובע לה מקום אמר ליה אביי לרב יוסף והרי כילת חתנים דאדם קובע לו מקום ואמר שמואל משום רבי חייא כילת חתנים
\end{multicols}\newpage

\newsection{דף מו}
\begin{multicols}{2}
מותר לנטותה ומותר לפרקה בשבת אלא אמר אביי בשל חוליות אי הכי מ״ט דר״ש בן לקיש דשרי
מאי חוליות כעין חוליות דאית בה חידקי הלכך חוליות בין גדולה בין קטנה אסורה לטלטלה גדולה נמי דאית בה חידקי גזירה אטו גדולה דחוליות כי פליגי בקטנה דאית בה חידקי מר סבר גזרינן ומר סבר לא גזרינן
ומי א״ר יוחנן הכי והאמר ר׳ יוחנן הלכה כסתם משנה ותנן. מוכני שלה בזמן שהיא נשמטת אין חיבור לה ואין נמדדת עמה ואין מצלת עמה באהל המת ואין גוררין אותה בשבת בזמן שיש עליה מעות
הא אין עליה מעות שריא ואע״ג דהוו עליה ביה״ש א״ר זירא תהא משנתינו שלא היו עליה מעות כל ביה״ש שלא לשבור דבריו של ר׳ יוחנן
א״ר יהושע בן לוי פעם אחת הלך רבי לדיוספרא והורה במנורה כר׳ שמעון בנר איבעיא להו הורה במנורה כר׳ שמעון בנר להיתרא או דילמא הורה במנורה לאיסורא וכר׳ שמעון בנר להיתרא תיקו
רב מלכיא איקלע לבי רבי שמלאי וטילטל שרגא ואיקפד ר׳ שמלאי ר׳ יוסי גלילאה איקלע לאתריה דר׳ יוסי ברבי חנינא טילטל שרגא ואיקפד ר׳ יוסי בר׳ חנינא ר׳ אבהו כי איקלע לאתריה דר׳ יהושע בן לוי הוה מטלטל שרגא כי איקלע לאתריה דר׳ יוחנן לא הוה מטלטל שרגא מה נפשך אי כרבי יהודה סבירא ליה ליעבד כרבי יהודה אי כר׳ שמעון סבירא ליה ליעבד כר׳ שמעון לעולם כר׳ שמעון ס״ל ומשום כבודו דר׳ יוחנן הוא דלא הוה עביד
א״ר יהודה שרגא דמשחא שרי לטלטולה דנפטא אסור לטלטולה רבה ורב יוסף דאמרי תרוייהו דנפטא נמי שרי לטלטולה (דהואיל וחזי לכסות ביה מנא)
רב אויא איקלע לבי רבא הוה מאיסן בי כרעיה בטינא אתיבי אפוריא קמיה דרבא איקפד רבא בעא לצעוריה א״ל מ״ט רבה ורב יוסף דאמרי תרוייהו שרגא דנפטא נמי שרי לטלטוליה א״ל הואיל וחזיא לכסויי בה מנא אלא מעתה כל צרורות שבחצר מטלטלין הואיל וחזיא לכסויי בהו מנא א״ל הא איכא תורת כלי עליה הני ליכא תורת כלי עליה מי לא תניא
השירים והנזמים והטבעות הרי הן ככל הכלים הנטלים בחצר ואמר עולא מה טעם הואיל ואיכא תורת כלי עליה הכא נמי הואיל ואיכא תורת כלי עליה א״ר נחמן בר יצחק בריך רחמנא דלא כסיפיה רבא לרב אויא
רמי ליה אביי לרבה תניא מותר השמן שבנר ושבקערה אסור ורבי שמעון מתיר אלמא לר׳ שמעון לית ליה מוקצה ורמינהו רבי שמעון אומר כל שאין מומו ניכר מעי״ט אין זה מן המוכן
הכי השתא התם אדם יושב ומצפה אימתי תכבה נרו הכא אדם יושב ומצפה מתי יפול בו מום מימר אמר מי יימר דנפיל ביה מומא ואת״ל דנפיל ביה מומא מי יימר דנפיל ביה מום קבוע ואם תמצי לומר דנפל ביה מום קבוע מי יימר דמזדקק ליה חכם
מתיב רמי בר חמא מפירין נדרים בשבת [ונשאלין לנדרים שהן] לצורך השבת ואמאי לימא מי יימר דמיזדקק לה בעל
התם כדרב פנחס משמיה דרבא דאמר רב פנחס משמיה דרבא כל הנודרת על דעת בעלה היא נודרת
ת״ש נשאלין לנדרים של צורך השבת בשבת ואמאי לימא מי יימר דמזדקק ליה חכם התם אי לא מיזדקק ליה חכם סגיא ליה בג׳ הדיוטות הכא מי יימר דמיזדקק ליה חכם
רמי ליה אביי לרב יוסף מי אמר ר׳ שמעון כבתה מותר לטלטלה כבתה אין לא כבתה לא מאי טעמא דילמא בהדי דנקיט לה כבתה הא שמעינן ליה לר׳ שמעון דאמר דבר שאין מתכוין מותר דתניא ר׳ שמעון אומר גורר אדם כסא מטה וספסל ובלבד שלא יתכוין לעשות חריץ
כל היכא דכי מיכוין איכא איסורא דאורייתא כי לא מיכוין גזר ר״ש מדרבנן כל היכא דכי מיכוין איכא איסורא דרבנן כי לא מיכוין שרי ר״ש לכתחילה
מתיב רבא מוכרי כסות מוכרין כדרכן ובלבד שלא יתכוין בחמה מפני החמה ובגשמים מפני הגשמים והצנועין מפשילין במקל לאחוריהן והא הכא דכי מיכוין איסורא דאורייתא איכא כי לא מיכוין שרי רבי שמעון לכתחילה
אלא אמר רבא
\end{multicols}\newpage

\newsection{דף מז}
\begin{multicols}{2}
הנח לנר שמן ופתילה הואיל דנעשה בסיס לדבר האסור
א״ר זירא א״ר אסי א״ר יוחנן אמר ר׳ חנינא אמר רבי רומנוס לי התיר רבי לטלטל מחתה באפרה א״ל רבי זירא לרבי אסי מי אמר רבי יוחנן הכי והתנן נוטל אדם בנו והאבן בידו או כלכלה והאבן בתוכה ואמר רבה בר בר חנה א״ר יוחנן בכלכלה מלאה פירות עסקינן טעמא דאית בה פירי הא לית בה פירי לא
(דניאל ד, טז) אישתומם כשעה חדא ואמר הכא נמי דאית בה קרטין אמר אביי קרטין בי רבי מי חשיבי
וכי תימא חזו לעניים והתניא בגדי עניים לעניים בגדי עשירים לעשירים אבל דעניים לעשירים לא אלא אמר אביי מידי דהוה אגרף של ריעי
אמר רבא שתי תשובות בדבר חדא גרף של ריעי מאיס והאי לא מאיס ועוד גרף של ריעי מיגלי והאי מיכסי אלא אמר רבא כי הוינן בי רב נחמן הוה מטלטלינן כנונא אגב קיטמא ואע״ג דאיכא עליה שברי עצים מיתיבי ושוין שאם יש בה שברי פתילה שאסור לטלטל אמר אביי בגלילא שנו
לוי בר שמואל אשכחינהו לרבי אבא ולרב הונא בר חייא דהוו קיימי אפיתחא דבי רב הונא אמר להו מהו להחזיר מטה של טרסיים בשבת אמרו ליה שפיר דמי אתא לקמיה דרב יהודה אמר הא רב ושמואל דאמרי תרוייהו המחזיר מטה של טרסיים בשבת חייב חטאת
מיתיבי המחזיר קנה מנורה בשבת חייב חטאת קנה סיידין לא יחזיר ואם החזיר פטור אבל אסור רבי סימאי אומר קרן עגולה חייב קרן פשוטה פטור אינהו דאמור כי האי תנא דתניא מלבנות המטה וכרעות המטה ולווחים של סקיבס לא יחזיר ואם החזיר פטור
אבל אסור ולא יתקע ואם תקע חייב חטאת רשב״ג אומר אם היה רפוי מותר
בי רב חמא הוה מטה גללניתא הוה מהדרי לה ביומא טבא א״ל ההוא מדרבנן לרבא מאי דעתיך בנין מן הצד הוא נהי דאיסורא דאורייתא ליכא איסורא דרבנן מיהא איכא אמר ליה אנא כרשב״ג סבירא לי דאמר אם היה רפוי מותר:
{\large\emph{מתני׳}} נותנין כלי תחת הנר לקבל ניצוצות ולא יתן לתוכו מים מפני שהוא מכבה:
{\large\emph{גמ׳}} והא קמבטל כלי מהיכנו אמר רב הונא בריה דרב יהושע ניצוצות אין בהן ממש:
ולא יתן לתוכו מים מפני שהוא מכבה: לימא תנן סתמא כרבי יוסי דאמר גורם לכיבוי אסור
ותסברא אימור דאמר ר׳ יוסי בשבת בערב שבת מי אמר וכי תימא הכא נמי בשבת והתניא נותנין כלי תחת הנר לקבל ניצוצות בשבת ואין צריך לומר בע״ש ולא יתן לתוכו מים מפני שהוא מכבה מע״ש ואין צריך לומר בשבת אלא אמר רב אשי אפילו תימא רבנן שאני הכא מפני שמקרב את כיבויו:
\par \par {\large\emph{הדרן עלך כירה}}\par \par 
מתני׳ {\large\emph{במה}} טומנין ובמה אין טומנין אין טומנין לא בגפת ולא בזבל לא במלח ולא בסיד ולא בחול בין לחין בין יבשין
ולא בתבן ולא בזגין ולא במוכין ולא בעשבין בזמן שהן לחין אבל טומנין בהן כשהן יבשין:
{\large\emph{גמ׳}} איבעיא להו גפת של זיתים תנן אבל דשומשמין שפיר דמי או דילמא דשומשמין תנן וכל שכן דזיתים
ת״ש דאמר ר׳ זירא משום חד דבי ר׳ ינאי קופה שטמן בה אסור להניחה על גפת של זיתים ש״מ של זיתים תנן
לעולם אימא לך לענין הטמנה דשומשמין נמי אסור לענין
\end{multicols}\newpage

\addpart{רי"ף שבת}\renewcommand{\partname}[1]{רי"ף שבת}
\fancyhead[CO]{\partname}
\fancyhead[CE]{\chapname}
\renewcommand{\sethebfont}{\fontsize{14pt}{21.0pt} \selectfont}\sethebfont
\newchap{פרק \hebrewnumeral{1} יציאות השבת}
\newsection{דף א}
\begin{multicols}{2}
\textbf{{\largeיציאות}} }השבת שתים שהן ארבע בפנים ושתים שהן ארבע בחוץ.
כיצד. העני עומד בחוץ ובעל הבית בפנים פשט העני את ידו לפנים ונתן לתוך ידו של בעה״ב או שנטל מתוכה והוציא העני חייב ובעל הבית פטור.
פשט בעה״ב את ידו לחוץ ונתן לתוך ידו של עני או שנטל מתוכה והכניס בעל הבית חייב והעני פטור.
פשט העני את ידו לפנים ונטל בעה״ב מתוכה או שנתן לתוכה והוציא שניהן פטורין.
פשט בעל הבית את ידו לחוץ ונטל העני מתוכה או שנתן לתוכה והכניס שניהם פטורין:
\textbf{{\largeגמ׳}} (דף ג.) בעא מניה רב מרבי הטעינו חבירו אוכלין ומשקין והוציאן לחוץ מהו עקירת גופו כעקירת חפץ ממקומו דמי או לא אמר ליה חייב ואינו דומה לידו מאי טעמא ידו לא נח גופו נח:
פשט העני את ידו וכו׳: (דף ד.) והא }בעינן עקירה והנחה }על גבי מקום ארבעה וליכא (דף ה.) אמר רבא ידו של אדם
חשובה לו כארבעה על ארבעה. וכן כי אתא רבין אמר רבי יוחנן ידו של אדם חשובה לו כארבעה על ארבעה:
(דף ג:) תניא היתה ידו מלאה פירות }והוציאה לחוץ }אסור להחזירה ותניא אידך מותר להחזירה הא והא בשוגג ולא קשיא כאן לאותה חצר כאן לחצר אחרת וכדבעי (דף ד.) מיניה רבא מרב נחמן היתה }ידו מלאה פירות והוציאה לחוץ מהו שיחזירנה לאותה חצר אמר ליה מותר לחצר אחרת מאי א״ל אסור ומאי שנא א״ל לכי תיכול עלה כורא דמילחא התם לא אתעבידא מחשבתו הכא אתעבידא מחשבתו:
אמר רב ביבי בר אביי }}}הדביק פת בתנור התירו לו לרדותה קודם שיבא לידי איסור סקילה הא מילתא מקשו בה רבנן מכדי רדיית הפת חכמה היא ואינה מלאכה כדגרסי׳ בפרק כל כתבי הקדש (דף קיז:) תנא דבי *}צ״ל ר׳ ישמעאל}שמואל (ויקרא כ״ג:ז׳) כל מלאכת עבודה לא תעשו יצאו רדיית הפת ותקיעת שופר שהן חכמה ואינם מלאכה ואלא מאי טעמא [לא] שריוה הכא אלא משום שלא יבא לידי איסור סקילה הא לאו הכי אסירא ואמאי והא לאו מלאכה קא עביד ופריקו בה כמה פירוקי דלא דייקי גבן ומשום הכי לא כתבינן להו
אבל אנן הכי מסתברא לן בפירוקא דהאי קושיא אע״ג דרדיית פת בתנור חכמה היא ואינה מלאכה לא שרו לה רבנן אלא היכא דשכח פת בתנור וקדש עליו היום דמציל ממנה מזון שלש סעודות לשבת משום כבוד השבת ואפילו הכי בעי לשנויי כדקתני לקמן (דף קיז:) ולא ירדה במרדה אלא בסכין אבל רדייה שלא לסעודת שבת לא שריוה ליה רבנן }אפי׳ ע״י שינוי ורדייה דהכא לאו לסעודת שבת היא דהא לא חזיא לאכילה ומשו״ה לא שרו ליה רבנן אלא משום שלא יבא לידי
\end{multicols}\newpage

\newsection{דף ב}
\begin{multicols}{2}
איסור סקילה:
(דף ה:) }}}ת״ר המוציא מחנות לפלטיא דרך סטיו חייב ובן עזאי פוטר דקסבר מהלך כעומד דמי (דף ו.) אמר רבי יוחנן מודה בן עזאי בזורק תניא נמי הכי }אחד המוציא ואחד המכניס ואחד המושיט ואחד הזורק חייב }ובן עזאי אומר המכניס והמוציא פטור }והזורק והמושיט חייב
תנו רבנן ארבע רשויות לשבת רשות היחיד ורשות הרבים כרמלית ומקום פטור איזהו רשות היחיד חריץ שהוא עמוק עשרה טפחים ורחב ארבעה וכן גדר שהוא גבוה עשרה ורחב ארבעה זו היא רשות היחיד גמורה ואיזו היא רה״ר }סרטיא ופלטיא גדולה ומבואות המפולשין }היא רשות הרבים גמורה ואין מוציאין מרשות היחיד זו לרשות הרבים זו ואין מכניסין מרשות הרבים זו לרשות היחיד זו ואם הוציא והכניס בשוגג חייב חטאת במזיד ענוש כרת בעדים והתראה נסקל אבל ים ובקעה ואצטונית וכרמלית אינן לא כרשות הרבים ולא כרשות היחיד ואין נושאין ונותנין בתוכן ואם נשא ונתן פטור ואין מוציאין מתוכן לרשות הרבים ולא מרשות הרבים לתוכן ואין מכניסין מתוכן לרשות היחיד ולא מרשות היחיד לתוכן ואם הוציא והכניס פטור חצרות של רבים ומבואות שאינן מפולשין עירבו מותרין לא עירבו אסורין אדם עומד על האסקופה נוטל מבעל הבית ונותן לו ומן העני ונותן לו ובלבד שלא יטול מבעל הבית ויתן לעני ומעני ויתן לבעל הבית ואם נטל ונתן שלשתן }פטורין }:
אבל ים ובקעה ואצטונית וכרמלית אינן לא כרשות היחיד ולא כרה״ר (דף ז.) אטו הני כולהו לאו כרמלית נינהו כי אתא
רב דימי א״ר יוחנן לא נצרכה אלא לקרן זוית הסמוכה לרה״ר }אע״ג דזימנין דדחקי בה רבים ועיילי להתם כיון דלא ניחא תשמישתיה ככרמלית דמיא:
}כי אתא רב דימי א״ר יוחנן בין }העמודים נידון ככרמלית.
אמר *}בגמ׳ וברא״ש איתא רבי זירא אמר רב יהודא}רב יהודה אמר רבי זירא איצטבא שלפני העמודים נידון ככרמלית מ״ד בין העמודים כ״ש איצטבא ומ״ד איצטבא אבל בין העמודים לא דזימנין דדרסי ביה רבים והלכה כר׳ יוחנן:
אמר רבא בר שילא אמר רב חסדא לבינה }זקופה ברה״ר }וזרק ונח בפניה חייב על גבה פטור אביי ורבא דאמרי תרוייהו והוא שגבוה ג׳ ולא דרסי ביה רבים אבל היזמי והיגי אע״ג דלא גבוה ג׳ רב חייא בר אשי אמר אפי׳ היזמי והיגי אבל צואה לא רב אשי אמר אפילו צואה:
כי אתא רב דימי אמר אין כרמלית פחותה מארבעה על ארבעה אמר רב ששת ותופסת עד עשרה דעד עשרה הויא כרמלית למעלה מעשרה הויא מקום פטור (דף ח.) ואקילו בה רבנן מקולי רה״י ומקולי רה״ר מקולי רה״י דאי איכא מקום ד׳ על ד׳ הוא דהויא כרמלית ואי לא לא הויא כרמלית ומקולי רה״ר דעד עשרה הוא דהויא כרמלית למעלה מעשרה לא הויא כרמלית אלא מקום פטור:
אמר רב גידל אמר רב חייא בר יוסף אמר רב בית שאין בו עשרה טפחים וקרויו משלימו לעשרה על גגו מותר לטלטל בכולו בתוכו אין מטלטלין בו אלא בארבע אמות אמר אביי ואם חקק בה ארבעה על ארבעה והשלימו לעשרה מותר לטלטל בכולו מ״ט הוי חורי רשות היחיד וחורי רשות היחיד כרה״י דמו דאיתמר חורי רה״י כרשות היחיד דמו חורי רה״ר אביי אמר כרשות הרבים דמו ורבא אמר לא כרשות הרבים דמו והלכה כרבא וכיון דלאו כרשות הרבים דמו אי אית בהו ארבעה טפחים על ארבעה טפחים וגביהי עשרה הוו להו כרשות היחיד ואי אית בהו ד׳ טפחים על ד׳ טפחים ולא גביהי עשרה הוו להו כרמלית ואי לית בהו ארבעה טפחים על ארבעה טפחים ואף על גב דגביהי עשרה מקום פטור הן
אמר רב חסדא נעץ קנה ברה״י וזרק ונח על גביו אפי׳ גבוה מאה אמה חייב שרשות היחיד עולה עד לרקיע ודוקא ברשות היחיד אבל ברה״ר מקום פטור הוא (דף ח.) אמר אביי זרק כוורת לרשות הרבים גבוהה י׳ ורחבה ששה על ששה פטור אינה רחבה ששה על ששה
\end{multicols}\newpage

\newsection{דף ג}
\begin{multicols}{2}
חייב פי׳ כל מקום שגבוה י׳ ורחב }ארבע על ארבע חולק רשות לעצמו ורה״י הוא:
ולפיכך הזורק כוורת לרה״ר פטור שכיון שגבוהה י׳ טפחים ורחבה ארבע על ארבע חולקת רשות לעצמה והוה ליה כזורק מרשות היחיד לרשות היחיד שהוא פטור ולא אמרו שהיא צריכה רוחב ששה אלא מפני שהיא עגולה שאי אפשר שיהא חללה ארבע על ארבע אלא בשיש בה ברוחב ששה על ששה כדאמרי׳ כל אמתא בריבועא אמתא ותרי חומשי באלכסונא והדבר ידוע שרוחב העיגול הוא אלכסונו של מרובע החקוק בתוכו וזהו צורתו *}גי׳ מהר״ם}כדי שיתברר במראית העין ולפיכך הוצרכנו להוסיף ברוחב העיגול שהוא אלכסונו של מרובע ב׳ חומשין של ד׳ טפחים שהן ח׳ חומשין לכל טפח וטפח ב׳ חומשין כדי שיהא אורך הצלע של מרובע החקוק בתוכו ארבעה טפחים ועוד הוספנו שני חומשין לעובי ב׳ המחיצות של כוורת חומש מכאן וחומש מכאן הרי עשרה חומשין שהן שני טפחים נמצא }}(חללה) ששה טפחים לא פחות ולא יותר
רבא אמר אפילו אינה רחבה ששה על ששה פטור מ״ט אי אפשר לקרומיות של קנה שלא יעלו למעלה מעשרה שהוא מקום פטור.
כפאה על פיה גבוהה שבעה ומשהו חייב שבעה ומחצה פטור רב אשי אמר אפילו גבוהה שבעה ומחצה נמי חייב מ״ט מחיצות לתוכן עשויות }ואין ממנה למעלה מעשרה כלום ולפיכך חייב.
אמר עולא }עמוד גבוה תשעה ברה״ר ורבים מכתפים עליו וזרק ונח על גביו חייב מ״ט כל פחות משלשה מדרס דרסי ליה רבים משלשה ועד תשעה לא מדרס דרסי ליה רבים ולא כתופי קא מכתפי עילויה תשעה ודאי קא מכתפי עילויה א״ל אביי לרב יוסף גומא מאי אמר ליה וכן בגומא רבא אמר גומא לא מ״ט קסבר הילוך ע״י הדחק שמיה הילוך תשמיש על ידי הדחק לא שמיה תשמיש
(דף ח:) אמר רב יהודה האי }}זירזא דקני רמא וזקפה רמא וזקפה לא מחייב }עד דעקר להו פי׳ זירזא דקני חבילה של קנים }(כגון) שהיתה מוטלת על הארץ והגביה הקצה האחד והקצה האחר מונח על הארץ והעמידה והשליכה לפניו וחזר והגביה הקצה האחר שהיה מונח על הארץ והעמידה והשליכה לפניו וחזר והגביה אפי׳ כל היום כולו לא מיחייב שהרי לא עקר אותה מעל הארץ:
אדם עומד על האסקופה נוטל מבעל הבית ונותן לו מן העני ונותן לו:
האי אסקופה מאי היא ואסיקנא אסקופה מקום פטור כגון דלית ביה ארבעה על ארבעה וכי הא דכי אתא רב דימי אמר רבי יוחנן מקום שאין בו ד׳ על ד׳ מותר לבני רשות הרבים ולבני רשות היחיד לכתף עליו ובלבד שלא יחליפו (דף ט.) אחרים אומרים אסקופה משמשת לשתי רשויות בזמן שהפתח פתוח כלפנים
פתח נעול כלחוץ אמר רב יהודה אמר רב הכא באסקופת מבוי עסקינן וחציו מקורה וחציו אינו מקורה וקירויו כלפי פנים פתח פתוח כלפנים פתח נעול }כלחוץ:
(דף ט:) \textbf{{\largeמתני׳}} }}}לא ישב אדם לפני הספר סמוך למנחה עד שיתפלל ולא יכנס לא למרחץ ולא לבורסקי ולא לאכול ולא לדון ואם התחילו אין מפסיקין מפסיקין לקריאת שמע ואין מפסיקין לתפלה:
\end{multicols}\newpage

\newsection{דף ד}
\begin{multicols}{2}
\textbf{{\largeגמ׳}} האי סמוך למנחה }אסיקנא דסמוך למנחה גדולה הוא ואפילו בתספורת דידן ולכתחלה אמאי לא גזירה שמא ישבר הזוג ולא למרחץ אפילו להזיע בעלמא ולכתחילה אמאי לא גזירה שמא יתעלפה ולא לבורסקי אפילו לעיוני בעלמא ולכתחלה אמאי לא גזירה שמא יראה הפסד בזביניה ויטרד ולא לאכול אפילו בסעודה קטנה ולכתחלה אמאי לא דלמא ממשכא ליה סעודתא ולא לדין אפילו בגמר דין ולכתחלה אמאי לא דלמא חזי טענתא וסתר דינא מאימתי התחלת תספורת אמר רבי אבין משיניח מעפורת של ספרין על כרכיו ומאימתי התחלת מרחץ אמר ר׳ אבין משיעביר מעפרתו ממנו ומאימתי התחלת בורסקי אמר רבי אבין משיקשור בין כתיפיו ומאימתי התחלת אכילה רב אמר משיטול ידיו ורבי }יוחנן אמר משיתיר חגורו ולא פליגי הא לן והא להו
אמר *}בגמ׳ הגי׳ אביי}זעירי הני חברין בבלאי למאן דאמר תפלת ערבית רשות כיון דשרא }ליה המייניה לא מטרחינן ליה ולמאן דאמר חובה מטרחינן ליה והא תפלת המנחה דלכ״ע חובה היא ותנן אם התחילו אין מפסיקין ואמר רבי יוחנן משיתיר חגורו (דף י.) התם לא שכיח שכרות הכא שכיח שכרות אי נמי מנחה כיון דקביע לה זימנא מירתת ולא אתי למיפשע ערבית כיון דכולי ליליא זמן תפלה היא לא אתי למירתת ואתי למיפשע והאידנא נהוג עלמא בתפלת ערבית מנהג חובה ואע״ג דשרא המייניה מטרחינן ליה
}רבה בר רב הונא רמי פוזמקי ומצלי אמר (עמוס ד׳:י״ב-י״ג) הכון לקראת אלהיך ישראל רבא }כי איכא ריתחא שדי גלימיה ופכר ידיה ומצלי אמר כעבדא קמי מריה }כי ליכא ריתחא מציין נפשיה אמר רב אשי חזינא ליה לרב כהנא כי איכא ריתחא בעלמא }לביש ומכסי פכר ידיה ומצלי אמר כעבדא קמי מריה כי ליכא ריתחא מציין נפשיה ומצלי אמר הכון לקראת אלהיך ישראל:
רבא אשכחיה לרב המנונא דקא מאריך בצלותא אמר מניחין חיי עולם ועוסקין בחיי שעה והוא סבר זמן תורה לחוד וזמן תפלה לחוד:
רבי ירמיה הוה יתיב קמיה דרבי זירא וקא גריס מטא זמן תפלה
וקא מסרהב קרי עליה רבי זירא (משלי כ״ח:ט׳) מסיר אזנו משמוע תורה גם תפלתו תועבה ודוקא מי שתורתו אומנותו כדבעינן למימר קמן:
מאימתי התחלת הדין ר׳ יונה ור׳ ירמיה חד אמר משיתעטפו הדיינין וחד אמר משיתחילו בעלי דינין ולא פליגי הא דעסיקי ואתו בדינא והא דלא עסיקי ואתו בדינא:
ר׳ אמי ורבי אסי הוו יתבי וגרסי ביני עמודי וכל שעתא ושעתא טפחו אעיבורא דדשא ואמרי מאן דאית ליה דינא ניעול וניתי:
רב חסדא ורבה בר רב הונא הוו יתבי כולי יומא בדינא והוה קא חליש לבייהו תני להו חייא בר רב מדפתי (שמות י״ח:י״ג) ויעמוד העם על משה מן הבקר עד הערב וכי תעלה על דעתך שמשה יושב ודן כל היום כולו ותורה אימת היתה נעשית אלא לומר לך כל דיין ודיין שדן דין אמת לאמתו אפילו שעה אחת מעלה עליו הכתוב כאילו נעשה שותף להקב״ה במעשה בראשית כתיב הכא מן הבקר עד הערב וכתיב התם במעשה בראשית (בראשית א׳:ה׳) ויהי ערב ויהי בקר:
עד מתי יושבין בדין אמר רב ששת עד זמן סעודה והיא שעה ששית שהוא זמן מאכל ת״ח:
א״ר אדא בר אהבה מתפלל אדם תפלתו בבית המרחץ ודוקא במרחץ חדש שעדיין לא רחצו בו בני אדם אבל במרחץ ישן שרחצו בו בני אדם אסור להתפלל בו דתניא הנכנס למרחץ מקום שבני אדם עומדין לבושין יש שם ק״ש ותפלה ואין צריך לומר שאלת שלום ומניח תפילין ואין צריך לומר שאינו חולץ נכנס למקום שבני אדם מקצתן עומדין ערומים ומקצתן עומדין לבושים יש שם שאלת שלום ואין שם ק״ש ותפלה ואינו חולץ תפליו ואינו מניח לכתחלה מקום שבני אדם עומדין ערומים אין שם שאלת שלום ואין צ״ל ק״ש ותפלה וחולץ תפליו וא״צ לומר שאינו מניח (דף י:) מסייע ליה לרב המנונא דאמר רב המנונא משמיה דעולא אסור לאדם שיתן שלום לחבירו בבית המרחץ שנאמר (שופטים ו) ויקרא לו ה׳ שלום:
אמר }רבה בר מחסיא אמר רב חמא בר גוריא אמר רב שרי ליה לאינש למימר }הימנותא }בבית הכסא ואע״ג דכתיב (דברים ז) האל הנאמן ההוא נאמן כמה דאת אמר אלהא מהימנא אבל שם גופיה לא איקרי נאמן.
והנותן מתנה לחבירו אין צריך להודיעו והני מילי במתנה דעבידא לאיגלויי דכתיב (שמות ל״ד:כ״ט) ומשה לא ידע כי קרן עור פניו אבל במתנה דלא עבידא לאיגלויי צריך לאודועיה דכתיב (שמות לא) לדעת כי אני ה׳ מקדשכם אמר ליה הקב״ה למשה משה מתנה טובה יש לי בבית גנזי ושבת שמה ואני רוצה ליתנה לישראל לך והודיעם מכאן א״ר שמעון בן גמליאל כל הנותן פת לתינוק צריך להודיע לאמו במאי מודיע לה אמר רב פפא דשייף ליה מההוא מינא ואף על גב דשבת מילתא דעבידא לאיגלויי היא מתן שכרה לא עבידא לאיגלויי }וה״מ מתנה לעשיר אבל מתנה לעני (משלי כ״א:י״ג-י״ד) מתן בסתר יכפה אף
ואמר רבה בר מחסיא אמר רב חמא בר גוריא אמר רב }}לעולם אל ישנה אדם את בנו בין הבנים שבשביל משקל שני סלעים מילת שהוסיף יעקב לבנו יוסף משאר בניו נתקנאו בו אחיו ונתגלגל הדבר וירדו אבותינו למצרים
(דף יא.) ואמר רבה בר מחסיא אמר רב חמא בר גוריא אמר רב כל עיר }שגגותיה }}גבוהין מבית הכנסת לסוף חריבה שנאמר (עזרא ט׳:ט׳) לרומם את בית אלהינו ולהעמיד את חרבותיו פירוש כשם שהמרומם בית אלהינו מעמיד את חרבותיו כך מי שאינו מרומם בית אלהינו מחריב את העומד ולא אמרן אלא בבתי אבל בקושקשי ואברורי לית לן בה
ואמר רבה בר מחסיא אמר רב חמא בר גוריא אמר רב יפה }}}תענית לחלום כאש לנעורת ואמר רב חסרא ובו ביום ואמר רב יוסף ואפי׳ בשבת רב אושעיא בריה דרב אידי איקלע לבי רב אשי עבדו ליה עיגלא תליתאה אמרו ליה נטעום מר מידי אמר להו בתעניתא יתיבנא אמרו ליה }לוזפיה מר וליפרעיה מי לא סבר לה מר להא דאמר רב יהודה אמר רב לוה אדם תעניתו ופורע אמר להו אנא בתענית חלום יתיבנא דאמר רבה בר מחסיא אמר רב חמא בר גוריא אמר רב יפה תענית לחלום כאש לנעורת ואמר רב חסדא ובו ביום ואמר רב יוסף אפילו בשבת ודוקא תענית חלום הוא דשרי בשבתא אבל תעניתא אחרינא לא וכי הא דגרסינן בפ׳ אין עומדין להתפלל א״ר אלעזר משום ר׳ יוסי בן זמרא כל היושב בתענית בשבת קורעין לו גזר דינו של שבעים שנה אמר רב נחמן וחוזרין ונפרעין
\end{multicols}\newpage

\newsection{דף ה}
\begin{multicols}{2}
ממנו דין עונג שבת מאי תקנתיה ליתיב תעניתא על תעניתא ומפרשי לה רבנן בתענית חלום }בלבד אבל תענית אחר אסור להתענות בשבת:
בתעניות ירושלמי ר׳ אחא ור׳ אבהו משום ר׳ יוסי בר חנינא אמרי אסור }להתענות עד ו׳ שעות בשבת א״ר יוסי בר אבין מתני׳ היא קודם חצות לא ישלימו לאחר חצות ישלימו קודם חצות לא ישלימו עד כדון צפרא הוא לאחר חצות ישלימו כבר עבר רובו של יום בקדושה:
}ואם התחילו אין מפסיקין וכו׳: והתני ליה רישא אין מפסיקין סיפא }}}אתאן לד״ת דתניא חבירים שהיו עוסקים }בתורה מפסיקין לק״ש ואין מפסיקין לתפלה א״ר יוחנן לא שנו אלא כגון רשב״י וחביריו שתורתן אומנותן אבל כגון אנו מפסיקין בין לתפלה בין לק״ש:
\textbf{{\largeמתני׳}} }}}לא יצא החייט }במחטו סמוך לחשיכה שמא ישכח ויצא ולא הלבלר בקולמוסו ולא יפלה את כליו ולא יקרא לאור הנר באמת אמרו החזן רואה מהיכן התינוקות קורין אבל הוא לא יקרא כיוצא בו לא יאכל הזב עם הזבה מפני הרגל עבירה:
\textbf{{\largeגמ׳}} (דף יב.) תנא דבי שמואל יוצא אדם בתפליו ע״ש עם חשיכה מ״ט כיון דאמר רבה בר רב הונא חייב אדם למשמש בתפליו כל שעה ושעה ק״ו מציץ ומה ציץ שאין בו אלא אזכרה אחת אמרה תורה (שמות כ״ח:ל״ח) והיה על מצחו תמיד שלא יסיח דעתו ממנו תפילין שיש בהם כמה אזכרות על אחת כמה וכמה הלכך מדכר }דכירי להו תניא חנניא אומר חייב אדם למשמש בבגדו ע״ש עם חשיכה. אמר רב יוסף הלכתא רבתי לשבת:
ולא }}}יפלה את כליו פירוש לא יפלה אדם את כליו לאור הנר שמא יטה:
אמר רב יהודה אמר שמואל אפילו להבחין בין בגדו לבגדי אשתו לאור הנר לא.
אמר רבא לא אמרן אלא לבני מחוזא }(דלבישי לבנים) אבל דבני חקלייתא מידע ידיעי ודבני מחוזא נמי לא אמרן אלא בזקנות אבל בילדות לא:
ת״ר אין פולין ברה״ר מפני הכבוד כיוצא בו א״ר יהודה ואמרי לה א״ר נחמיה אין עושין אפיקטוזין ברשות הרבים מפני הכבוד:
תנו רבנן המפלה את כלי מולל }וזורק ובלבד שלא יהרוג אבא שאול אומר נוטל וזורק ובלבד שלא ימלול:
אמר רב הונא הלכתא מולל וזורק וזהו כבודו ואפילו בחול
תניא ר״ש בן אלעזר אומר אין הורגין את }המאכולת בשבת דברי ב״ש וב״ה מתירין וכן היה ר״ש בן אלעזר אומר משום רשב״ג אין }}}פוסקין צדקה על הצבור אפילו ליתום ויתומה ואין משדכין על התינוקת ליארס ועל התינוק ללמדו ספר וללמדו אומנות ואין מנחמים אבלים ואין מבקרים חולים בשבת דברי בית שמאי וב״ה מתירין:
ת״ר הנכנס לבקר את החולה בשבת אומר שבת היא מלזעוק ורפואה קרובה לבא ר״מ אומר יכולה היא שבת שבזכותה תרחם (דף יב:) ר׳ יהודה אומר המקום ירחם עליך ועל כל חולי ישראל ר׳ יוסי אומר המקום ירחם עליך בתוך חולי עמו ישראל.
שבנא איש ירושלים אומר }בכניסתו שלום וביציאתו אומר שבת היא מלזעוק ורפואה קרובה לבא ורחמיו מרובים ושבתו לשלום:
כמאן אזלא הא דתניא מי שיש לו חולה בתוך ביתו צריך שיערבנו בתוך חולי ישראל כמאן כרבי יוסי:
א״ר יוחנן בקושי התירו לנחם אבלים ולבקר חולים בשבת פירוש שמא יבא לזעוק בשבת אמר רבה בר בר חנה כי הוה אזלינן }בתר רבי אלעזר לשיולי בתפיחה זימנין הוה א״ל המקום יפקדך לשלום וזמנין הוה א״ל רחמנא ידכרינך לשלמא ואף על גב }}}דאמר רב יהודה אמר רב לעולם אל ישאל אדם צרכיו בלשון ארמית }דאמר רבי יוחנן השואל צרכיו בלשון ארמית אין מלאכי השרת נזקקין לו לפי שאין מלאכי השרת מכירין בלשון ארמית שאני חולה דשכינה }עמו דאמר אביי מנין שהקב״ה סועד את החולה שנאמר (תהלים מא) ה׳ יסעדנו על ערש דוי:
ת״ר הנכנס לבקר את החולה לא ישב ע״ג מטה ולא ע״ג ספסל ולא ע״ג כסא ולא ע״ג שרפרף ולא על גבי מקום גבוה ולא למעלה מראשותיו של חולה אלא מתעטף ויושב לפניו מפני שהשכינה למעלה מראשותיו של חולה שנאמר (בראשית מ״ז:ל״א) וישתחו ישראל על ראש המטה וכתיב ה׳ יסעדנו על ערש דוי:
ולא יקרא לאור }}}הנר: אמר רבה אפי׳ גבוה שני קומות ואפי׳ גבוה ב׳ מרדעות ואפי׳ עשרה בתים זו על גב זו.
חד הוא דלא ליקרי אבל שנים קורין }וה״מ בענין אחד }אבל בשני ענינים אפילו שנים אין קורין:
אמר רב הונא ובמדורה אפילו עשרה בני אדם כאחד אסור לקרות אמר רבא ואדם חשוב שאין דרכו להטות את הנר מותר תני חדא שמש בודק קערות וכוסות לאור הנר ותניא אידך לא יבדוק לא קשיא כאן בשמש קבוע כאן בשמש שאינו קבוע }ואב״א הא והא בשמש שאינו קבוע ולא קשיא באן בדמשחא כאן }בדנפטא:
איבעיא להו שמש קבוע בדמשחא מאי אמר רב הונא }הלכה ואין מורין כן ורב ירמיה בר אבא אמר הלכה ומורין כן }.
שמעינן השתא דבין שמש קבוע ובין שמש שאינו קבוע בודק קערות וכוסות לאור הנר ודוקא בדנפטא דמאיס ונפיש נהוריה ולא בעי לאצלויי אבל בדמשחא שמש שאינו קבוע ודאי אסיר ליה שמש קבוע פליגי והלכה בודק ואין מורין כן:
באמת אמרו החזן רואה וכו׳: אמר רבה בר שמואל אבל מסדר הוא ראשי פרשיות אבל פרשה כולה לא (דף יג.) ותינוקות של בית רבן אפילו פרשה כולה קורין ומסדרין מפני שאימת רבן עליהן }ולא אתו לאצלויי שלא ברשות רבן:
}}}כיוצא בו לא יאכל }הזב עם הזבה וכו׳:
איבעיא להו מהו שתישן נדה אצל בעלה הוא בבגדו והיא בבגדה ת״ש (יחזקאל י״ח:ו׳) ואת אשת רעהו לא טמא ואל אשה נדה לא קרב מקיש אשתו נדה לאשת רעהו מה אשת רעהו הוא בבגדו והיא בבגדה אסור אף אשתו נדה הוא בבגדו והיא
\end{multicols}\newpage

\newsection{דף ו}
\begin{multicols}{2}
בבגדה אסור:
תנא דבי אליהו מעשה בתלמיד אחד שקרא הרבה ושנה הרבה ושמש תלמידי חכמים הרבה ומת בחצי ימיו והיתה אשתו נוטלת תפליו ומחזרתן בבתי כנסיות ובבתי מדרשות ואומרת להן כתוב בתורה (דברים ל׳:כ׳) כי הוא חייך ואורך ימיך בעלי שקרא הרבה ושנה הרבה (דף יג:) ושמש ת״ח הרבה מפני מה מת בחצי ימיו ולא היה אדם משיבה דבר פעם אחת נתארחתי אצלה אמרתי לה בתי בימי נדותיך מהו אצליך אמרה לי חס ושלום אפי׳ באצבע קטנה לא נגע בי בימי לבוניך מהו אצליך אמרה לי אכל עמי ושתה עמי וישן עמי בקירוב בשר אבל חס ושלום לא עלה על דעתו לדבר אחר אמרתי לה ברוך המקום שהרגו שלא נשא פנים לתורה שהרי אמרה תורה (ויקרא יא) ואל אשה בנדת טומאתה לא תקרב כי אתא רב דימי אמר מטה חדא הואי במערבא אמרי א״ר יצחק בר יוסף סינר מפסיק בינו לבינה:
\textbf{{\largeמתני׳}} ואלו מן ההלכות }}}שאמרו בעליית חנניה בן חזקיה בן גרון כשעלו לבקרו נמנו ורבו ב״ש על בית הלל וי״ח דבר גזרו בו ביום (דף יז:) ב״ש אומרים אין שורין דיו סממנין וכרשינין אלא כדי שישורו מבעו״י ובית הלל מתירין ב״ש אומרים אין נותנין אונין של פשתן לתוך התנור אלא כדי שיהבילו מבעוד יום ולא את הצמר ליורה אלא כדי שיקלוט את העין וב״ה מתירין:
\textbf{{\largeגמ׳}} (דף יח.) תנו רבנן פותקין מים לגנה ערב שבת עם חשיכה והיא הולכת ומתמלאה כל היום כולו ומניחין מוגמר תחת הכלים והם הולכין ומתגמרין כל היום כולו ומניחין גפרית תחת הכלים והם הולכין ומתגפרין כל השכת כולה ומניחין קילור על גב העין ואספלנית על גכי מכה והם הולכין ומתרפאין כל השבת כולה ואין נותנין חטין לתוך הריחים של מים
אלא כדי שיטחנו מבעוד יום ואוקימנא אליבא דב״ש דאית להו שביתת כלים אבל לב״ה לית להו שביתת כלים }ושרי ואיכא מ״ד אסור ואפי׳ לב״ה מפני שמשמעת את הקול
(דף יח:) ת״ר לא }}}תמלא אשה קדרה עססיות ותורמוסין ותניח לתוך התנור ע״ש עם חשיכה ואם עשתה כן למוצאי שבת אסורין בכדי שיעשו כיוצא בו לא ימלא נחתום חבית של מים ויניח לתוך התנור ערב שבת עם חשיכה ואם עשה כן למוצאי שבת אסורים בכדי שיעשו מאי טעמא גזירה שמא יחתה בגחלים ואי אמרת מוגמר וגפרית מ״ט לא גזרינן שמא יחתה בגחלים התם לא מחתה להו דאי מחתה להו פסדי דקא סליק בהו קוטרא ואונין של פשתן נמי לא גזרינן דכיון דקשי להו זיקא לא מגלי להו צמר ליורה נמי לא גזרינן ביורה עקורה וטוחה דליכא למיחש לא לאחתויי ולא למיגס }בה והשתא דאמר מר גזירה שמא יחתה בגחלים האי קדרה חייתא }[ובשיל] שפיר דמי }בשיל ולא בשיל אסיר ואי שדא ביה גרמא חייא שפיר דמי מאי טעמא דאסח לדעתיה מינה דאמר השתא ודאי לא בשיל ולא אתי לאחתויי:
והשתא דאמר מר כל מידי דקשי ליה זיקא לא מגלי ליה האי בישרא דגדיא ושריק שפיר דמי דברחא ולא שריק אסיר דגדיא ולא שריק דברחא ושריק רב אשי שרי ורב ירמיה מדפתי אסר:
איכא דאמרי דגדיא בין שריק בין לא שריק שפיר דמי דברחא ושריק נמי שפיר דמי כי פליגי דברחא ולא שריק רב אשי שרי ורב ירמיה מדפתי אסר.
וקא פסקי רבוואתא כי האי לישנא בתרא וכרב אשי ואיכא מאן דפסק כרב ירמיה מדפתי דאסר ואנן נמי כרב ירמיה מדפתי ס״ל דחזינן סוגיא דשמעתא כוותיה דאמרינן אמר רבינא האי קרא חייא כיון דקשי ליה זיקא כבשרא דגדיא דמי ושרי מכלל דאי בברחא אסיר ושמע מינה דברחא אסיר:
\textbf{{\largeמתני׳}} (דף יז:) ב״ש אומרים אין פורשין מצודות חיה ועוף ודגים אלא כדי שיצודו מבעוד יום וב״ה מתירין בש״א אין }}}מוכרין לנכרי ואין טוענין עמו ולא מגביהין
\end{multicols}\newpage

\newsection{דף ז}
\begin{multicols}{2}
עליו אלא כדי שיגיע }*}נ״א לביתו}למקום קרוב וב״ה מתירין ב״ש אומרים אין נותנין עורות לעבדן ולא כלים לכובס נכרי אלא כדי שיעשו מבעוד יום ובכולן ב״ה }מתירין עם השמש:
\textbf{{\largeגמ׳}} (דף יח:) ת״ר ב״ש אומרים לא ימכור אדם חפצו לנכרי ולא }ילונו ולא ישאילנו ולא ימשכננו ולא יתננו לו במתנה אלא כדי שיגיע *}נ״א לביתו}למקום קרוב וב״ה אומרים עד שיגיע לבית הסמוך לחומה ר״ע אומר כדי שיצא מפתח ביתו ר׳ יוסי בר׳ חנינא אומר הן הן דברי ר״ע הן הן דברי ב״ה ולא בא רבי עקיבא אלא לפרש דברי ב״ה:
תניא ב״ש אומרים לא }}ימכור אדם חמצו לנכרי אלא א״כ יודע שיכלה קודם הפסח וב״ה אומרים כל שעה שמותר לאכול מותר למכור (דף יט.) ת״ר }}}נותנין מזונות לכלב בחצר אם נטלן ויצא אין נזקקין לו כיוצא בו נותנין מזונות לפני העובד כוכבים בחצר ואם נטלן ויצא אין נזקקין לו:
}}(ת״ר לא }}}ישכיר אדם כליו לנכרי ע״ש }(עם חשיכה) וברביעי ובחמישי מותר):
ת״ר אין משלחין אגרות ביד נכרי ע״ש עם חשיכה אא״כ קצץ לו דמים ב״ש אומרים כדי שיגיע לביתו ובה״א כדי שיגיע לבית הסמוך לחומה והלא קצץ אמר רב אשי ה״ק אם לא קצץ בש״א כדי שיגיע לביתו ובה״א כדי שיגיע לבית }הסמוך לחומה והא אמרת רישא אין משלחין כלל לא קשיא הא דקביע דואר במתא הא דלא קביע דואר במתא שמעינן השתא היכא }דקצץ לו דמים משלחין אותן בין מגיע בין לא מגיע והיכא דלא קצץ לו דמים אי קביע דואר }במתא אין משלחין אלא עד שיגיע לבית הסמוך לחומה ואי לא קביע דואר במתא אין משלחין כלל אא״כ קצץ.
פי׳ דואר איש ידוע שכל כתב אליו יובל והוא משכיר ומשלח כל אגרת למי שנשתלחה אליו:
ת״ר }}}אין מפליגין בספינה פחות מג׳ ימים קודם השבת
בד״א לדבר הרשות אבל לדבר מצוה מפליגין ופוסק עמו ע״מ }לשבות ואינו שובת דברי רבי רשב״ג אומר אינו צריך ומצור לצידן אפילו בערב שבת מפליגין
יש מי שאומר הא דתניא אין מפליגין בספינה פחות מג׳ ימים קודם השבת בזמן שהספינה גוששת ואין במים עשרה טפחים ומשום גזירת תחומין גזרו בה אבל למעלה מי׳ טפחים לא גזרו ומשום הכי נהגו העם להפליג בים }הגדול והאי טעמא פריכא הוא דאי מהאי טעמא הוא דאין מפליגין הוה ליה למיתנא אין מפליגין בספינה קטנה אמאי תני ספינה סתם דמשמע בין קטנה בין גדולה ועוד מאי איריא ג׳ ימים אפי׳ טפי נמי ועוד לדבר מצוה אמאי שרי והא העמידו דבריהם במקום עשה כדתנן (פסחים צא:) אונן טובל ואוכל את פסחו לערב אבל לא בקדשים ואמרי׳ עלה (פסחים צב.) גבי פסח לא העמידו דבריהם במקום כרת גבי קדשים העמידו דבריהם במקום עשה
אלא היינו טעמא דאין מפליגין בספינה פחות מג׳ ימים קודם השבת משום בטול מצות עונג שבת דכל ג׳ ימים הויא להו שינוי וסת משום נענוע הספינה כדכתיב בהו (תהלים קו) יחוגו וינועו כשכור וגו׳ ולא יכלי למעבד עונג שבת ולאחר שלשת ימים הויא להו נייחא ובעי מיכלא ומקיימי מצות עונג שבת והיינו טעמא דלדבר מצוה שרי משום דפטירי ממצות עונג דאמר מר (סוכה כה.) העוסק במצוה פטור מן המצוה ותנן נמי (שם) שלוחי מצוה פטורין מן הסוכה ולהכי נמי אמרינן (סוכה מד:) אסור להלך בערבי שבתות יתר על שלש פרסאות דאינון שנים עשר מיל משום דמבטיל ליה למצות עונג שבת כדמפרשא בהדיא והוא הדין שאין צרין על עיירות של נכרים פחות משלשה ימים קודם לשבת משום דלא מיתהני להו מיכלא ומשתיא תוך ג׳ ימים משום טרדא ופחדא דליבא ולבתר תלתא יומי פרח פחדייהו ומקיימי ליה לעונג שבת:
גרסינן בפרק לולב וערבה (דף מד:) אמר איבו משום ר׳ אלעזר בר׳ צדוק אל יהלך אדם בערב שבת יותר משלש פרסאות אמר רב כהנא לא אמרן אלא לביתיה אבל לאושפיזיא אמאי דנקיט סמיך איכא דאמרי אמר רב כהנא לא נצרכה אלא אפילו לביתיה ואמר רב כהנא בדידי הוה עובדא ואפילו כסא דהרסנא לא אשכחי:
ת״ר אין }}צרין על עיירות של נכרים פחות משלשה ימים קודם השבת ואם התחילו אין מפסיקין אפילו בשבת וכן היה שמאי דורש (דברים כ) עד רדתה ואפילו בשבת:
גרסינן בעירובין (דף מה.) }אמר רב יהודה אמר רב נכרים שצרו על עיירות של ישראל אין יוצאין עליהן בבלי זיין ואין מחללין עליהן את השבת:
תניא נמי הכי עיירות של ישראל שצרו עליהן נכרים אין יוצאין עליהן בבלי זיין ואין מחללין עליהן את השבת בד״א בשבאו על עסקי ממון אבל באו על עסקי נפשות יוצאין עליהן בכלי זיין ומחללין עליהן את השבת ובעיר הסמוכה לספר אפילו לאבאו אלא על עסקי תבן וקש יוצאין עליהם בכלי זיין ומחללין עליהן את השבת:
(דף יח.) \textbf{{\largeמתני׳}} אמר רבן שמעון בן גמליאל }}}נוהגין היו של בית אבא שהיו נותנין כלי לבן שלהן לכובס נכרי שלשה ימים קודם השבת ושוין אלו ואלו שטוענין בקורות בית הבד ובעיגולי הגת:
\textbf{{\largeגמ׳}} (דף יט.) (מאי שנא כולהו דפליגי }ומאי שנא הנך דלא פליגי ב״ש כולהו אי עביד בשבת חייב חטאת גזרו בהו רבנן ע״ש עם חשיכה הנך דאי עביד להו בשבת לא מחייב חטאת לא גזרו בהו רבנן) מאן תנא דכל מילי דאתא ממילא שפיר דמי }ר׳ יוסי ב״ר
\end{multicols}\newpage

\newsection{דף ח}
\begin{multicols}{2}
חנינא אמר ר׳ ישמעאל היא }דתנן השום }והבוסר והמלילות שריסקן מבעוד יום רבי ישמעאל אומר יגמור משתחשך ר״ע אומר (דף יט:) לא יגמור ואיתמר עלה אמר רבה בר בר חנה א״ר יוחנן בצריך דיכה כולי עלמא לא פליגי דאסיר כי פליגי במחוסרין שחיקה [וה״נ כמחוסרין }דיכה דמו] הורה רבי יוסי ב״ר חנינא כרבי ישמעאל:
שמן של בדדין ומחצלות של בדדין רב אסר ושמואל שרי פי׳ שמן של בדדין שמן היוצא }תחת הקורה של בית הבד בשבת ומחצלות של בדדין הן המחצלאות המוקצות בבתי הבד:
הני כרכי דזוגי }רב אסר ושמואל שרי.
פירוש מחצלות כרוכות ומונחות לסחורה ונקשרות שתים שתים ונקראות זוגי והלכתא כשמואל דשרי דקאי כר׳ שמעון דלית ליה מוקצה:
אמר רב נחמן עז לחלבה ורחל לגיזתה ותרנגולת לביצתה ותמרי דעסקא ותורא דרדיא פלוגתא דר׳ שמעון ור׳ יהודה דרבי יהודה אית ליה מוקצה ורבי שמעון לית ליה מוקצה וקי״ל כר׳ שמעון לענין שבת דהא איפסיקא הלכתא כוותיה בהדיא בסוף פרק מי שהחשיך וה״נ אמרינן ההוא תלמידא דאורי בחרתא דארגיז כרבי שמעון ושמתיה רב }הונא והא כר׳ שמעון סבירא }ליה באתריה דרב הוה ובאתריה דרב לא הוה ליה למעבד הכי ושמעינן מינה דהלכתא כרבי שמעון:
\textbf{{\largeמתני׳}} אין }}}צולין בשר בצל וביצה אלא כדי שיצולו מבעוד יום ואין נותנין }פת לתנור עם חשיכה ולא חררה ע״ג גחלים אלא כדי שיקרמו פניה ר׳ אליעזר אומר כדי שיקרום התחתון שלה:
\textbf{{\largeגמ׳}} (דף כ.) וכמה כדי שיצולו אמר ר׳ זריקא אמר רבי אלעאי אמר רב כמאכל בן }דרוסאי איתמר נמי א״ר אסי א״ר יוחנן כל שהוא כמאכל בן דרוסאי אין בו משום בישולי נכרים:
\textbf{{\largeמתני׳}} משלשלין את הפסח לתנור ערב שבת עם חשיכה ומאחיזין את האור במדורת בית המוקד ובגבולים כדי שתאחז האור ברובן רבי יהודה אומר }בפחמין כל שהוא:
\textbf{{\largeגמ׳}} משלשלין מאי טעמא דבני חבורה זריזין הן ולא אתי לאחתויי בגחלים:
ומאחיזין את האור במדורת בית המוקד: מאי טעמא דכהנים זריזין הם: ובגבולין כדי שיוצת
האור ברובן. מאי ברובן אמר רב ברוב כל אחד ואחד ושמואל אמר כדי שלא יאמרו הבא עצים ונניח תחתיהם:
תני רבי חייא לסיועי לשמואל להעלות נר תמיד כדי שתהא שלהבת עולה מאליה ולא שתהא עולה ע״י דבר אחר:
עץ יחידי רב אמר ברוב עביו ושמואל אמר ברוב היקפו אמר רב פפא הילכך בעינן רוב עביו ורוב הקיפו:
אמר רב יהודה קנים }}צריכין רוב אגדן אין צריכין רוב גרעינים צריכין רוב נתנן בחותלות אין צריכין רוב מתקיף לה רב חסדא אדרבה איפכא מסתברא איתמר נמי (דף כ:) אמר רב כהנא אמר רב קנים שאגדן צריכין רוב תני רב יוסף ד׳ מדורות אין צריכין רוב של }זפת ושל גפרית ושל רבב ושל קירה במתניתא תנא אף של גבבא ושל קש:
\textbf{סליקו להו יציאות השבת} 
\textbf{{\largeבמה}} מדליקין ובמה אין }}}מדליקין אין מדליקין לא בלכש ולא בחוסן ולא בכלך ולא בפתילת האידן ולא בפתילת המדבר ולא בירוקה שעל פני המים לא בזפת ולא בשעוה ולא בשמן קיק ולא בשמן שריפה ולא באליה ולא בחלב נחום המדי אומר מדליקין בחלב מבושל וחכמים אומרים אחד מבושל ואחד שאינו מבושל אין מדליקין בו:
\textbf{{\largeגמ׳}} לכש שוכא דארזא שוכא דארזא עץ בעלמא הוא בעמרניתא דאית ביה:
ולא בחוסן אמר אביי כיתנא דדייק ולא נפיץ:
ולא בכלך אמר שמואל שאלתינהו לכל נחותי ימא ואמרי כולכא שמיה יצחק בר זעירי אמר גושקרא שמיה:
ולא בפתילת האידן }אחואנא בעמרניתא דביני ביני:
ולא בפתילת המדבר שברא: ולא בירוקה שעל פני המים מאי היא אמר דב פפא אוכמתא דארבי תנא הוסיפו עליהן של צמר ושל שער זפת זיפתא שעוה קירותא:
תנא עד כאן פסול פתילות מכאן ואילך פסול שמנים פשיטא שעוה }אצטריכא ליה מהו דתימא נגזור פסול פתילות אטו פסול שמנים קמ״ל דלא (דף כא.) ת״ר כל אלו שאמרו אין מדליקין בהן בשבת אבל עושין בהם מדורה בשבת בין להתחמם כנגדה בין להשתמש לאורה בין על גבי קרקע בין על גבי מנורה ולא אסרו אלא לעשותן פתילה לנר בלבד:
ולא בשמן קיק מאי שמן קיק אמר שמואל עוף אחד יש בכרכי הים וקיק שמו רב אחא בריה דרב יצחק בריה דרב יהודה אמר משחא דקזא ור״ש ב״ל אומר }קיקיון דיונה:
אמר רבה פתילות שאמרו חכמים אין מדליקין בהן מה טעם מפני שהאור מסכסך בהן שמנים שאמרו חכמים אין מדליקין בהן מה טעם מפני שאין נמשכין אחר הפתילה:
בעא מיניה אביי מרבה שמנים שאמרו חכמים אין מדליקין בהן מהו ליתן לתוכן שמן כל שהוא וידליק מי גזרינן דלמא אתי לאדלוקי בעיניהו או לא א״ל אין מדליקין א״ל מה טעם אמר ליה לפי שאין מדליקין בהם והכורך }דבר שמדליקין בו ע״ג דבר
\end{multicols}\newpage

\newchap{פרק \hebrewnumeral{2} במה מדליקין}
\end{multicols}\newpage

\newsection{דף ט}
\begin{multicols}{2}
שאין מדליקין בו אין מדליקין בו במה דברים אמורים להדליק אבל להקפות מותר פירוש להקפות להעבות ראש הפתילה להרבות אורה:
}אמר רב ברונא אמר רב חלב }מהותך וקרבי דגים שנימוחו נותן לתוכן שמן כל שהוא ומדליק:
\textbf{{\largeהלכות}} \textbf{{\largeחנוכה}} (דף כא:) א״ר זירא (אמר רב מתנה) אמר רב פתילות }}}ושמנים שאמרו חכמים אין מדליקין בהן בשבת מדליקין בהן בחנוכה בין בחול בין בשבת מדקאמר מדליקין בהם בחנוכה }שבת ש״מ אסור }להשתמש לאורה דכיון דאסור להשתמש לאורה לא אתי לאטויי ותו מדמדליקין בהן בחנוכה }בשבת מכלל דאי כביא לא מזדקיק לה ש״מ כבתה אין זקוק לה והא דתניא מצותה }משתשקע החמה עד שתכלה רגל מן השוק לאו דאי כביא הדר מדליק לה אלא דאי לא }אדליק מדליק א״נ לשיעורא (כלומר שצריך ליתן }שמן לתוכו כדי שתהא דולק והולך עד השיעור הזה) }היתה דולקת והולכת עד השיעור הזה }ורצה לככותה או להשתמש לאורה הרשות בידו:
עד שתכלה רגל מן השוק עד כמה א״ר יוחנן עד דכליא רגלא דתרמודאי פי׳ עצים ידועים אצלם ונקראים תרמודא ובני אדם המביאין אותם נקראין תרמודאי ומתעכבין עד אחר שקיעת החמה
כמו חצי שעה עד שמגיעין לבתיהם:
ת״ר מצות נר חנוכה נר איש }}}וביתו }והמהדרין נר לכל אחד ואחד והמהדרין מן המהדרין בית שמאי אומרים יום ראשון מדליק שמנה מכאן ואילך פוחת והולך וב״ה אומרים יום ראשון מדליק אחת מכאן ואילך מוסיף והולך.
טעמייהו דב״ש כנגד פרי החג אי נמי כנגד ימים הנכנסין.
וטעמייהו דב״ה כנגד ימים היוצאין אי נמי משום מעלין בקדש ולא מורידין:
אמר רבה בר בר חנה א״ר יוחנן שני זקנים היו בצידן אחד עשה כדברי ב״ש ואחד עשה כב״ה זה נותן טעם לדבריו כנגד פרי החג וזה נותן טעם לדבריו משום מעלין בקדש ולא מורידין:
}(חנוני שהניח נר חנוכה מבחוץ והוזק בה אחר פטור מלשלם מפני שהוא נר של מצוה והאי דקא מניחה מבחוץ הואיל ומצוה לפרסם את הנס):
ת״ר נר חנוכה מצוה }להניחה על פתח ביתו מבחוץ ואם היה דר בעליה מניחה בחלון הסמוכה לרה״ר ובשעת הסכנה מניחה על שולחנו ודיו אמר רבא וצריך נר אחרת להשתמש לאורה ואי איכא מדורה לא בעי ואי אדם חשוב הוא אע״ג דאיכא מדורה צריך:
מאי חנוכה דת״ר בכ״ה בכסליו יומי חנוכה תמניא אינון דלא למספד בהון ודלא להתענאה בהון שכשנכנסו יונים להיכל }וטמאו כל השמנים שבהיכל וכשגברה מלכות בית חשמונאי ונצחום בדקו ולא מצאו אלא פך אחד של שמן שהיה חתום ומונח בחותמו של כה״ג ולא היה }בה להדליק אלא יום אחד ונעשה בה נס והדליקו בה ח׳ ימים לשנה אחרת קבעום ח׳ ימים טובים בהלל ובהודאה והלכך מברכינן אניסא כל יומא ויומא מתמניא יומי דחנוכה הואיל ובכל יום ויום היה הנס מתחדש באותו פך של שמן:
אמר רב כהנא דרש רבי נתן בר מניומי משמיה דרב נחמן (דף כב: ע״ש) נר חנוכה שהניחה }למעלה מכ׳ אמה פסולה כסוכה וכמבוי:
אמר רבא נר חנוכה מצוה להניחה בטפח הסמוך לפתח היכא מנח לה רב אחא בריה דרבא אמר מימין ורב ירמיה מדפתי אמר משמאל והלכתא משמאל כדי שיהא מזוזה מימין ונר חנוכה משמאל (והיכא }}}דקא בעי לאדלוקי נר חנוכה ונר שבת ברישא מדליק דחנוכה והדר דשבתא דאי מדליק דשבתא ברישא איתסר ליה לאדלוקי דחנוכה משום דקביל }שבתא עליה):
אמר רב יהודה אמר רב אסור להרצות מעות כנגד נר חנוכה אמר אביי }כל מילי דמר עביד כרב בר מהני תלת דעביד כשמואל מדליקין }מנר לנר ומתירין מבגד לבגד והלכתא כרבי שמעון בגרירה דתניא ר״ש אומר גורר אדם מטה כסא וספסל ובלבד שלא יתכוין לעשות חריץ (דף כב:) אמר רבא היה תופס נר }}}חנוכה ועומד לא עשה ולא כלום מ״ט מאן דחזי אמר לעסקיה הוא דנקיט לה ואמר רבא הדליקה מבפנים והוציאה לחוץ לא עשה ולא כלום
\end{multicols}\newpage

\newsection{דף י}
\begin{multicols}{2}
מ״ט ה״נ }הרואה אומר לצרכו הוא }דנקיט לה:
האי נר חנוכה (דף כג.) דאדלקה חרש שוטה וקטן לא עשה ולא כלום מ״ט הדלקה עושה מצוה ולא הנחה עושה מצוה }והני לאו בני מיעבד מצוה נינהו אבל אשה ודאי מדלקה דא״ר יהושע ב״ל נשים חייבות בנר חנוכה שאף הן היו באותו הנס:
א״ר יהושע ב״ל עששית שהיתה דולקת כל השבת כולה למוצאי שבת מכבה ומדליקה אמר רב ששת }}אכסנאי חייב בנר חנוכה:
א״ר זירא מריש כי הוינא בי רב הוה משתתפינא בפריטי בהדי אושפיזאי לכתר דנסיבנא אמינא השתא }}ודאי לא צריכנא דקא מדליקי עלאי בגו ביתאי:
וה״מ דאשתתופי בפריטי היכא דלא פתח בבא לנפשיה }אבל אי פתח בבא לנפשיה מיחייב לאדלוקי משום חשדא:
אמר רב חייא בר אשי אמר רב המדליק נר חנוכה צריך לברך ור׳ ירמיה בר אבא אמר הרואה נר חנוכה חייב לברך מאי מברך אמר רב יהודה יום ראשון המדליק מברך שלש והרואה מברך שנים מכאן ואילך מדליק מברך שנים והרואה מברך אחד מאי מברך אקב״ו להדליק נר של חנוכה ושעשה נסים }ושהחיינו אמר רב הונא }}}חצר שיש לה שני פתחים צריכה שני נרות אמר רבה לא אמרן אלא משתי רוחות אבל מרוח }אחת לא צריך:
(דף כג:) א״ר יצחק אמר רב הונא נר שיש לה שתי פיות עולה לשני בני אדם אמר רבא מילא קערה שמן }והקיפה פתילות אם כפה עליה כלי עולה לכמה בני אדם לא כפה עליה כלי נעשית כמדורה ואפילו לאחד אינו עולה:
אמר רבא פשיטא לי נר }}חנוכה ונר ביתו נר ביתו עדיף משום שלום ביתו נר ביתו וקדוש היום נר ביתו עדיף משום שלום ביתו בעי רבא נר חנוכה וקדוש היום איזה מהם עדיף קדוש היום עדיף משום דתדיר או דלמא נר חנוכה עדיף משום פרסומי ניסא בתר דבעיא הדר פשטה נר חנוכה עדיף משום פרסומי ניסא
אמר רב הונא הרגיל בנר חנוכה [ושבת] הויין לו בנים תלמידי חכמים (הרגיל בנר של שבת בניו תלמידי חכמים) הרגיל במזוזה זוכה לדירה נאה הרגיל בציצית זוכה לטלית נאה הזהיר בקידוש היום זוכה וממלא גרבי יין רב הונא הוה רגיל דהוה חליף }אפיתחא דבי אבין נגרא חזא דהוו רגילי בשרגי }אמר תרי גברי רברבי נפקי מהכא נפקי מיניה רב חייא בר אבין ורב אידי בר אבין:
רב חסדא הוה רגיל דחליף }אפיתחא דבי נשא דרב שזבי חזא }דהוו רגילי בשרגא אמר גברא רבה נפיק מהכא נפיק מינייהו רב שזבי דביתהו דרב יוסף הות מאחרה ומדלקה אמר לה רב יוסף תנינא (שמות י״ג:כ״ב) לא ימיש עמוד הענן יומם ועמוד האש לילה מלמד שעמוד הענן משלים לעמוד האש ועמוד האש משלים לעמוד הענן סברה לאקדומי אמר לה ההוא סבא תנינא ובלבד שלא יקדים ובלבד שלא יאחר אמר רבא דרחים רבנן הויין ליה בנין רבנן דמוקיר רבנן הויין ליה חתני רבנן דדחיל מרבנן איהו גופיה הוי צורבא מרבנן ואי לאו בר הכי הוא הויין מליה משתמעין כצורבא מרבנן:
ולא בשמן שריפה: מאי שמן שריפה שמן של תרומה שנטמאה ואמאי קרו ליה שמן שריפה הואיל ולשריפה עומד וביו״ט שחל להיות ע״ש עסקינן לפי שאין שורפים קדשים ביו״ט ותניא נמי הכי (דף כד.) כל אלו שאמרו אין מדליקין בהן בשבת מדליקין בהן ביו״ט חוץ משמן שריפה לפי שאין שורפין }קדשים ביו״ט איבעיא להו מהו }}}להזכיר של חנוכה בבהמ״ז כיון דמדרבנן הוא לא מדכרינן או דלמא משום פרסומי ניסא מדכרינן:
אמר רבא אמר רב סחורה אמר רב הונא אינו מזכיר ואם בא להזכיר מזכיר בהודאה רב הונא בר יהודה איקלע לבי רבא סבר לאדכורי בבנין ירושלים א״ל רבא כתפלה מה תפלה בהודאה אף בהמ״ז בהודאה:
איבעיא להו מהו להזכיר של ר״ח בבהמ״ז אם
\end{multicols}\newpage

\newsection{דף יא}
\begin{multicols}{2}
תמצא לומר חנוכה דרבנן לא צריך ראש חודש דאורייתא צריך או דלמא כיון דלא אסיר בעשיית מלאכה לא מדכרינן רב אמר מזכיר ור׳ חנינא אמר אינו מזכיר א״ר זירא נקוט הא דרב בידך דתני ר׳ אושעיא כותיה דתני ר׳ אושעיא }}ימים שיש בהן קרבן מוסף כגון ר״ח וחולו של מועד ערבית שחרית ומנחה מתפלל י״ח ואומר מעין המאורע בעבודה ואם לא אמר מחזירין אותו ואין בהן קדושה על הכוס ויש בהן הזכרה בברכת המזון וימים שאין בהן קרבן מוסף כגון }חנוכה ופורים }תעניות ומעמדות ושני וחמישי.
שני וחמישי מאי עבידתייהו אלא אימא שני וחמישי של תעניות ושל מעמדות ערבית שחרית ומנחה מתפלל י״ח ואומר מעין המאורע בשומע תפלה ואם לא אמר אין מחזירין אותו ואין בהן קדושה על הכוס ואין בהן הזכרה בברכת המזון:
מהא שמעינן דמי שטעה }ולא אמר עננו בתפלת תענית שאין מחזירין אותו ותו שמעינן דמי שטעה ולא הזכיר על הנסים בתפלה בחנוכה וכפורים שאין מחזירין אותו לפי }שאין מחזירין אותו אלא בימים שיש בהן קרבן מוסף כגון ר״ח וחש״מ אבל יום שאין בו קרבן מוסף כגון תעניות ומעמדות וחנוכה ופורים אין מחזירין אותו ותניא [נמי] בתוספתא בהדיא כל יום שאין בו קרבן מוסף כגון תעניות ומעמדות חנוכה ופורים ערבית שחרית ומנחה מתפלל י״ח ואומר מעין המאורע ואם לא אמר אין מחזירין אותו וכל יום שיש בו קרבן מוסף כגון ר״ח וחולו של מועד ערבית שחרית ומנחה מתפלל שמונה עשרה ואומר קדושת היום בעבודה ואם לא אמר מחזירין אותו:
איבעיא להו מהו להזכיר של }}חנוכה במוספין כיון דלית בה מוסף לא מדכרינן או דילמא יום הוא שנתחייב בארבע תפלות רב הונא ורב יהודה אמרי אינו מזכיר רב נחמן ורבי יוחנן אמרי מזכיר והלכתא כותייהו (דף כד:) דא״ר יהושע בן לוי יוה״כ שחל להיות בשבת המתפלל תפלת נעילה צריך להזכיר של שבת יום הוא שנתחייב בה׳ תפלות הכא נמי יום הוא שנתהייב בד׳ תפלות ולפיכך הוא מזכיר והא דאמר רב גידל אמר רב ר״ח }}שחל להיות בשבת }המפטיר בנביא אינו צריך להזכיר של ר״ח שאלמלא שבת אין נביא בר״ח לית הלכתא כותיה מהא דריב״ל:
אמר רבא }יום טוב שחל להיות בשבת שליח צבור היורד לפני התיבה ערבית א״צ להזכיר של יו״ט שאלמלא שבת אין ש״צ יורד ערבית ביו״ט וכן הלכתא.
ואי קשיא לך הא דאמר ריב״ל יום הוא שנתחייב בחמש תפלות.
הכא בע״ש דין הוא דלא צריך ש״צ ערבית לירד לפני התיבה ורבנן הוא דתקון משום סכנה אבל התם ביום הכפורים יום הוא שנתחייב בחמש תפלות:
והני תמני }}יומי דחנוכה מיחייבינן בהו למיגמר הלילא בכל יומא ויומא (ערכין דף י:) דאמר ר׳ יוחנן משום ר׳ שמעון בן יהוצדק י״ח יום בשנה יחיד גומר בהן את ההלל ואלו הן ח׳ ימי חנוכה ויו״ט הראשון של פסח ויום טוב של עצרת ושמונת ימי החג ובגולה כ״א יום ט׳ ימי החג וח׳ ימי חנוכה וב׳ י״ט של פסח וב׳ י״ט של עצרת אבל הלל דר״ח לאו דאורייתא אלא מנהגא הוא ומשום הכי לא גמרינן ביה הלילא אלא מדלגי דלוגי דאמרינן (תענית כח:) רב איקלע לבבל חזא דקא קרו הלל בר״ח סבר לאפסוקינהו כיון דשמעינהו דמדלגי ואזלי אמר ש״מ מנהג אבותיהם }בידיהם תנא יחיד לא יתחיל ואם התחיל גומר הילכך אי בעי יחיד למיקרי הלל בראש חדש קרי ליה בלא ברכה ומדלג דלוגי:
\textbf{{\largeמתני׳}} אין מדליקין }}}בשמן שריפה ביום טוב ר׳ ישמעאל אומר אין מדליקין בעטרן מפני כבוד השבת וחכמים מתירין בכל השמנים בשמן שומשמין בשמן אגוזים בשמן צנונות בשמן דגים בשמן פקועות בעטרן ובנפט ר׳ טרפון אומר אין מדליקין אלא בשמן זית בלבד:
\textbf{{\largeגמ׳}} אין מדליקין בעטרן (דף כה:) מ״ט אמר רבא מפני שריחו רע גזירה שמא יניחנו ויצא א״ל אביי ויצא }שאני אומר הדלקת נר }בשבת חובה דאמר רב חסדא ואמרי לה אמר רבא בר רב חנין אמר רב הדלקת נר בשבת חובה רחיצת ידים ורגלים בחמין ערבית רשות ואני אומר מצוה מאי מצוה דאמר רב יהודה כך היה מנהגו של רבי יהודה בר׳ אלעאי ערב שבת מביאין לו עריבה מליאה מים חמין ורוחץ בה פניו ידיו ורגליו ומתעטף ויושב בסדינין המצויצין ודומה למלאך ה׳
\end{multicols}\newpage

\newsection{דף יב}
\begin{multicols}{2}
צבאות והיו תלמידיו }מחבין ממנו כנפי כסותן אמר להם בני לא כך שניתי לכם סדין בציצית ב״ש פוטרין ובית הלל מחייבין והלכה כב״ה ואינהו סבור גזירה משום כסות לילה
תניא ר״ש בן אלעזר אומר אין מדליקין בצרי מאי טעמא אמר }רבה מתוך שריחו נודף גזירה שמא יסתפק ממנו א״ל אביי (דף כו.) ולימא מר מפני שהוא עף א״ל חדא ועוד קאמינא חדא מפני שהוא עף ועוד גזירה שמא יסתפק ממנו
ת״ר אין מדליקין בטבל טמא בחול וא״צ לומר בשבת ביוצא בו אין מדליקין בנפט לבן בחול ואין צ״ל בשבת בשלמא נפט לבן מפני שהוא עף אלא טבל טמא מ״ט אמר קרא (במדבר י״ח:ח׳) ואני הנה נתתי לך את משמרת תרומתי בשתי תרומות הכתוב מדבר אחת תרומה טהורה ואחת תרומה טמאה מה תרומה טהורה אין לך בה אלא משעת הרמה ואילך אף תרומה טמאה אין לך בה אלא משעת הרמה ואילך:
(דף כז:) \textbf{{\largeמתני׳}} כל }}}היוצא מן העץ אין מדליקין בו אלא פשתן וכל היוצא מן העץ
אינו מטמא טומאת אהלין אלא פשתן (דף כח:) פתילת הבגד שקפלה ולא }הבהבה ר״א אומר טמאה היא ואין מדליקין בה ר״ע אומר טהורה היא ומדליקין בה:
\textbf{{\largeגמ׳}} פתילת הבגד פלוגתא דר״א ור״ע בגמרא כך היא בשלמא לענין טומאה בהא פליגי }דסבר רבי אליעזר קיפול אינו מועיל ור״ע סבר קיפול מועיל }אלא לענין הדלקה במאי פליגי אמר ר״א א״ר אושעיא וכן אמר רב אדא בר אהבה הכא בשלש על שלש מצומצמות עסקינן וביו״ט שחל להיות בע״ש }הוא ודכ״ע אית להו להא דר״י }אמר רב דאמר מסיקין בכלים ואין מסיקין בשברי כלים }ודכ״ע אית להו }להא דאמר עולא המדליק צריך להדליק ברוב היוצא רבי אליעזר סבר קיפול אינו מועיל ובמילתא קמייתא קיימא וכיון דאדליק ביה פורתא הויא ליה שבר כלי וכי קא מדליק בשבר כלי קא מדליק ור״ע סבר קיפול מועיל }וכי קא מדליק }בהיתרא קא מדליק א״ר }יוסה היינו דתנינא שלש על שלש מצומצמות ולא ידענא למאי הלכתא:
ומדקא מתרץ רב אדא בר אהבה אליבא דר׳ יהודה ש״מ כר׳ יהודה ס״ל ומי אמר רב אדא בר אהבה הכי והאמר רב אדא בר אהבה נכרי שחקק }קב בבקעת ישראל מסיקה ביום טוב ואינו חושש ואי ס״ד דכרבי יהודה ס״ל נולד הוא ואסור ומתרץ בגמרא לדבריהם דר״א ור״ע קאמר ליה וליה לא ס״ל כר׳ יהודה }(ואמאי }כיון דאדליק ביה פורתא שבר כלי קא עבדיה וכי קא מהפך בשבר כלי קא מהפך }א״ל טעמא דמתני׳ קא מפרש וליה לא ס״ל כרבי יהודה) (דף כט.) רבא אמר היינו טעמא דר״א }שאין מדליקין בפתילה שאינה מחורבת ולא בסמרטוטין שאינן מחורכין אלא הא דתני רב יוסף }שלשה שאמרו מצומצמות למאי הלכתא לענין טומאה ובדתנן }שלשה }על שלשה שאמרו חוץ מן המלל דר״ש וחכ״א שלשה }(על שלשה) מכוונות:
רב המנונא אמר הכא בפחות משלשה על שלשה עסקינן }}ואזדא ר״א לטעמיה ור״ע לטעמיה דתנן }פחות משלשה על שלשה שהתקינו לפקק בו את המרחץ או לקנח את הריחים או לנער בו את הקדרה בין מן המוכן בין שאינו מן המוכן [טמא }דר״א רי״א בין מן המוכן בין שאינו מן המוכן] טהור ר׳ עקיבא אומר מן המוכן טמא שלא מן המוכן טהור
ואמר עולא ואיתימא רבה בר בר חנה א״ר יוחנן }זרקו לאשפה דברי הכל טהור (דף כט:) הניחו בקופסא דברי הכל טמא לא נחלקו אלא שתלאו במגוד או שהניחו אחורי הדלת דר״א סבר בין }במגוד בין שהניחו אחורי הדלת מוכן הוא וטמא הוא ור׳ יהושע סבר בין שתלאו במגוד בין שהניחו אחורי הדלת לאו מוכן הוא וטהור הוא
}}ור״א דקאמר [לאו] מוכן הוא }(וטמא הוא) דלגבי קופסא [לאו] מוכן הוא וטמא הוא דקסבר ר״א כיון שלא זרקו לאשפה בין }שהניחו במגוד בין שהניחו אחורי הדלת אחשובי אחשביה וכמי שהניחו בקופסא דמי וטמא הוא }ורבי יהושע דאמר מוכן הוא }וטהור הוא לא קאמר מוכן לגבי קופסא }}דאמרי׳ לעיל הניחו בקופסא דברי הכל טמא אם כן היכי קאמר ר׳ יהושע מוכן הוא וטהור הוא אלא האי מוכן דא״ר יהושע לאו אקופסא קאי אלא לאשפה קאי }(והכי קאמר בין שתלאו כמגוד }בין שהניחו אחורי הדלת כיון שלא הניחו כקופסא מוכן הוא לאשפה וטהור הוא דלאו דעתיה עילויה ואינו חשוב לגביה כלום }והכנה דאשפה לאו שמה הכנה לגבי טומאה) }ור״ע בתלאו במגוד סבר כר׳ אליעזר והוי מוכן לגבי טומאה ובהניחו אחורי הדלת סבר ליה כר׳ יהושע דאמר טהור והוי כמו שזרקה לאשפה והוי טהור דאמרינן דהכנה דאשפה לא הויא הכנה דטומאה.
והא דאמר רב המנונא הכא בפחות משלשה על שלשה עסקינן וכו׳ אשכחנא ביה למקצת רבואתא פירושא דלאו דסמכא ודלא כהלכתא ולהכי איצטריכנא למכתביה ולברורי פירכי׳ ולפרושי שמעתתא פירושא מעליא אליבא דהלכתא והכי }אשכחיה ליה דקא מוקים להא דרב המנונא בפחות מג׳ [אצבעות] על ג׳ וקאמר בתר הכי }ומאן דחזא דרב המנונא דאייתי ראיה }מג׳ }על ג׳ ומוקים לה לפתילה דקתני במתני׳ בפתילה שיש בה פחות משלשה }[טפחים] והא דחויה היא דהיכי קתני ר׳ עקיבא אומר טהורה והלא משנה שלימה שנינו }מפני שאמרו שלש על שלש }שנתמעטה טהורה אבל שלשה על שלשה שנתמעטה אף על פי שטהור מן המדרס טמא בכל הטומאות אלא פחות משלש על שלש הוא וכיון דאוקמה רב אדא בשלש על שלש אוקמה רב המנונא בפחות מג׳ והא דקאמר שלשה דהכנת מטליות בהדיא בפחות משלשה קתני לה }והוא הדין לפחות משלש:
הדין הוא פירושא דאשכחן ואנן מקשינן עליה כי קושיא דיליה מאי אמרת }פחות משלש הוא ולהכי קתני רבי עקיבא טהורה ואימא לאידך גיסא היכי קתני }ר׳ אליעזר אומר טמאה בפחות משלש על שלש והלא משנה שלימה שנינו מפני שאמרו שלש על שלש שנתמעטה טהורה ועוד היכי אמרת בפחות משלשה על שלשה קתני לה והוא הדין לפחות משלש על שלש וכי באיזה דין שוה פחות משלש על שלש לפחות משלשה על שלשה (טפחים) והלא פחות משלשה על שלשה טמאה בכל הטומאות חוץ מטומאת מדרס ופחות משלש על שלש טהורה מכולם ואי אפשר לדון טהור מטמא הרי נתברר לך דהני מילי לית בהו מששא ולאו דסמכא נינהו ולהכי חזינן לפרושי האי שמעתא פירושא מעליא ופירושא דסמכא אליבא דהלכתא
עיקר }האי שמעתא דפחות משלשה על שלשה עד שלש על שלש לא מטמא אלא אי אצנעיה ואי לא אצנעיה לא מטמא דתניא בתוספתא בענין שלש על שלש לעולם אינה טמאה עד שיצניענה לבגד ר״ש אומר לדבר שמקבל טומאה טמאה לדבר שאין מקבל טומאה טהורה וידוע שפחות משלשה על שלשה ושלש על שלש דינן שוה
ובהאי הצנעה קא מיפלגי ר״א ור׳ יהושע ור׳ עקיבא דתנן פחות משלשה על שלשה שהתקינו לפקק בו את המרחץ ולנער בו את הקדרה ולקנח בו את הריחיים ר׳ אליעזר אומר בין מן המוכן בין שאינו מן המוכן טמא רבי יהושע אומר בין מן המוכן בין שאינו מן המוכן טהור רבי עקיבא אומר מן המוכן טמא שאינו מן המוכן טהור
ואמר עולא ואיתימא רבה בר בר חנה אמר רבי יוחנן זרקו לאשפה דברי הכל טהור הניחו בקופסא דברי הכל טמא לא נחלקו אלא שתלאו במגוד או שהניחו אחורי הדלת פי׳ שאם הניחו בקופסא הרי הכינו והצניעו ואף על פי שחזר והתקינו לפקק [בו] את המרחץ ולנער בו את הקדרה שהן מעשה מוך בעלמא ודבר שאינו מקבל טומאה הוא לא בטיל ליה מתורת בגד ולא מפקא ליה ההיא תקנה מתורת בגד וטמא הוא לדברי הכל ואם זרקו לאשפה דברי הכל טהור דהא בטליה ובין התקינו לנער בו את הקדירה ובין לא התקינו טהור דהא כבר בטליה מעיקרא ולא בעי התקנה לאלין מילי כי פליגי בשתלאו במגוד או שהניחו אחורי הדלת ר׳ אליעזר סבר מוכן הוא וטמא הוא דכמי שהניחו בקופסא דמי ואף על פי שחזר והתקינו לנער בו את הקדירה וכיוצא בה לא נפק ליה מתורת בגד בההיא התקנה ורבי יהושע סבר כיון שלא הניחו בקופסא לאו מוכן הוא הילכך טהור הוא ואע״פ שלא התקינו לנער בו את הקדרה שהרי ביטלו מעיקרא והאי דקרי ליה מוכן דלגבי אשפה מוכן קרי ליה ור״ע
\end{multicols}\newpage

\newsection{דף יג}
\begin{multicols}{2}
בתלאו במגוד סבר לה כרבי אליעזר דאמר מוכן הוא וטמא הוא ובהניחו אחורי הדלת סבר לה כרבי יהושע דאמר }(לאו) מוכן הוא וטהור הוא ולפום הכין אוקמא רב המנונא לפלוגתא דרבי אליעזר ור״ע בפתילת הבגד לענין טומאה בפחות משלשה על שלשה ובין שתלאו במגוד ובין הניחו אחורי הדלת לר״א מוכן הוא וטמא הוא ור״ע הדר ביה לגבי דר׳ יהושע ובין תלאו במגוד ובין הניחו אחורי הדלת לאו מוכן הוא ולפיכך אמר טהורה היא מדקרי לה פתילת הבגד }דעדיין בגד הוא וקאמר טהורה בהא פליגי לענין טומאה אבל לענין הדלקה היינו טעמא דרבי אליעזר לפי שאין מדליקין בפתילה שאינה מחורכת ולא בסמרטוטין שאינן מחורכין כדרבא ור״ע סבר מדליקין והלכתא כותיה הדין הוא פירושא דהאי שמעתתא בבירור בלא קושיא ובלא ספק:
\textbf{{\largeמתני׳}} לא }}יקוב אדם שפופרת של ביצה וימלאנה שמן ויתננה על פי הנר בשביל שתהא מנטפת ואפילו היא של חרס ורבי יהודה מתיר ואם חברה היוצר מתחלה מותר מפני שהוא כלי אחד לא ימלא אדם קערה שמן ויתננה בצד הנר ויתן ראש הפתילה בתוכה בשביל שתהא שואבת ורבי יהודה מתיר:
\textbf{{\largeגמ׳}} ורבי יהודה מתיר ולית הלכתא כרבי יהודה דלמא אתי לאיסתפוקי מיניה:
ואם חברה היוצר וכו׳: תנא ואם חברה בסיד ובחרסית מותר והא אנן יוצר תנן אלא אימא כעין יוצר:
\textbf{{\largeמתני׳}} }}}המכבה את הנר מפני שהוא מתירא מפני נכרים מפני לסטים מפני רוח רעה ואם בשביל החולה שיישן פטור:
\textbf{{\largeגמ׳}} סוגיא דשמעתא (דף ל.) אם חולה שיש בו סכנה הוא מותר לכבות לכתחלה ואפילו לרבי יהודה ואם חולה שאין בו סכנה הוא לר׳ שמעון לא יכבה ואם כבה פטור }אבל אסור ולרבי יהודה }חייב חטאת:
\textbf{{\largeמתני׳}} כחס על הנר כחס על השמן כחס על הפתילה חייב רבי יוסי פוטר ככולן חוץ מן הפתילה מפני שהוא עושה פחם:
גמ׳ אסקה (דף לא:) רבי יוחנן דרבי יוסי לעולם כרבי שמעון סבירא ליה והכא בפתילה שצריך להבהבה עסקינן דבהא אפילו רבי שמעון מודה דקא מתקן מנא:
\textbf{{\largeמתני׳}} על }שלש עבירות נשים מתות בשעת לידתן על שאינן זהירות בנדה ובחלה ובהדלקת הנר:
\textbf{{\largeגמ׳}} }בשעת לידתן מאי טעמא א״ר יצחק היא קלקלה בחדרי בטנה לפיכך תלקה בחדרי בטנה תינח נדה חלה והדלקת הנר מאי איכא למימר כדדרש ההוא גלילאה קמיה דרב חסדא אמר הקדוש ברוך הוא רביעית דם נתתי בכם על עסקי דם הזהרתי אתכם (דף לב.) ראשית קראתי אתכם על עסקי ראשית הזהרתי אתכם נשמה נתתי בכם שקרויה נר על עסקי נר הזהרתי אתכם אם אתם משמרים אותם מוטב ואם לאו אני נוטל נשמתכם מכם
ומאי שנא בשעת לידתן }אמר רבא נפל תורא חד חד סכינא [וגברי היכי מבדקי אמר רבא בשעה שעוברין על הגשר על הגשר ותו לא אימא כעין גשר] אמר רבי יוחנן }}}לעולם אל ילך אדם במקום סכנה ויאמר שעושין לו נס שמא אין עושין לו נס ואם עושין לו נס מנכין לו מזכיותיו וא״ר יוחנן מאי קראה (בראשית ל״ב:י״א) קטנתי מכל החסדים:
אמר ר׳ יצחק בריה דרב יהושע לעולם }יבקש אדם רחמים שלא יחלה שאם יחלה אומרים לו הבא זכות והפטר ת״ר מי שחלה ונטה למות אומרים לו התודה שכן דרך כל המומתין מתודין אדם יוצא לשוק דומה למי שנמסר לסרדיוט
חש בראשו דומה למי שנתנוהו לקולר עלה למטה ונפל למשכב דומה למי שהעלוהו לגרדום לידון שכל מי שעלה לגרדום לידון אם יש לו פרקליטין גדולים ניצול ואם לאו אינו ניצול ואלו הם פרקליטין גדולים תשובה ומעשים טובים:
(דף לב:) תניא ר׳ נתן אומר בעון ביטול נדרים מתה אשתו של אדם שנאמר (משלי כ״ב:כ״ז) אם אין לך לשלם למה יקח משכבך מתחתיך ר׳ אומר בעון נדרים בניו ובנותיו של אדם מתים כשהם קטנים שנא׳ אל תתן את פיך לחטיא את בשרך ואל תאמר לפני המלאך כי שגגה הוא למה יקצוף האלהים על קולך וחבל את מעשה ידיך ואיזה הן מעשה ידיו של אדם אלו בניו ובנותיו תנו רבנן בעון נדרים בנים מתים דברי ר״א בנו של ר׳ שמעון בן יוחאי רבי יהודה הנשיא אומר בעון ביטול תורה ויש אומרים בעון ביטול מזוזה ויש אומרים בעון ביטול ציצית:
אמר רבי שמעון בן לקיש הוו }}}זהירין במצות ציצית שכל הזוכה ומקיים מצות ציצית זוכה ומשמשין אותו שני אלפים ושמונה מאות עבדים שנאמר (זכריה ח׳:כ״ג) כי כה אמר ה׳ צבאות בימים ההמה אשר יחזיקו עשרה אנשים מכל לשונות הגוים והחזיקו בכנף איש יהודי:
תניא ר׳ נחמיה אומר בעון שנאת חנם מריבה רבה בתוך ביתו של אדם ואשתו מפלת נפלים ובניו ובנותיו מתים כשהן קטנים רבי אלעזר ברבי יהודה אומר בעון חלה אין ברכה במכונס ומארה בשערים משתלחת וזורעים זרעים ואחרים אוכלין אותם שנא׳ (ויקרא כ״ו:ט״ז) אף אני אעשה זאת לכם }וגו׳ אל תיקרי בהלה אלא בהלה ואם נותנין מתברכין שנאמר (יחזקאל מ״ד:ל׳) וראשית עריסותיכם תתנו לכהן להניח ברכה אל ביתך:
בעון ביטול תרומות ומעשרות שמים נעצרים מלהוריד טל ומטר והיוקר הווה והשכר אבד ובני אדם רצים אחר הפרנסה ואין מגיעין שנאמר (איוב כ״ד:י״ט) ציה גם חום יגזלו מימי שלג שאול חטאו תאנא דבי ר׳ ישמעאל בשביל דברים שצויתי אותם בימות החמה ולא עשאום יגזלו מימי שלג ואם נותנין מתברכין שנא׳ (מלאכי ג׳:י׳) הביאו את כל המעשר אל בית האוצר ויהי טרף בביתי ובחנוני נא בזאת אמר ה׳ צבאות אם לא אפתח לכם את ארובות השמים והריקותי לכם ברכה עד בלי די מאי עד בלי די עד שיבלו שפתותיכם מלומר די:
בעון גזל הגובאי עולה והרעב הווה ובני אדם אוכלין בשר בניהם ובנותיהם שנאמר (עמוס ד׳:א׳) שמעו הדבר הזה פרות הבשן אשר בהר שומרון העושקות דלים הרוצצות אביונים וגו׳
(דף לג.) בעון עינוי הדין ועוות הדין וקלקול הדין ועון ביטול תורה חרב וביזה }ודבר בא לעולם ובצורת באה ובני אדם אוכלין ואינן שבעים ואוכלין לחמם במשקל שנא׳ (ויקרא כ״ו:כ״ה) והבאתי עליכם חרב וגו׳
בעון שבועת שוא ושבועת שקר וחלול השם וחלול שבת חיה רעה באה לעולם ובהמה כלה ובני אדם מתמעטים והדרכים משתוממים שנא׳ (ויקרא כו) ואם באלה לא תוסרו לי אל תקרי באלה אלא באלה וכתיב (ויקרא כו) והשלחתי בכם את חית השדה ושכלה אתכם והכריתה את בהמתכם והמעיטה אתכם ונשמו דרכיכם וכתי׳ בשבועת שקר (ויקרא כו) ולא תשבעו כשמי לשקר וחללת את שם אלהיך אני ה׳ ובחילול השם כתיב (ויקרא כו) ולא תחללו את שם קדשי ובחילול שבת (ויקרא כו) מחלליה מות יומת וילפינן חלול חלול משבועת שקר
בעון שפיכות דמים בית המקדש חרב ושכינה מסתלקת מישראל שנאמר (במדבר ל״ה:ל״ג) ולא תחניפו את הארץ וגו׳ הא אם אתם מטמאין אותה אין אני שוכן בתוכה
בעון גלוי עריות וע״ז והשמטת שמיטין ויובלות גלות בא לעולם ומגלים אותן ממקומן ובאין אחרים ויושבין תחתיהן שנאמר }(ויקרא י״ח:כ״ח) ולא תקיא הארץ אתכם כאשר קאה וגו׳
}בעון נבלות הפה צרות רבות וגזירות קשות מתחדשות ובחורי שונאי ישראל מתים ויתומים ואלמנות צועקין ואינן נענין שנאמר (ישעיהו ט׳:ט״ז) על כן על בחוריו לא ישמח ה׳ ואת יתומיו ואת אלמנותיו לא ירחם כי כלו חנף ומרע וכל פה דובר נבלה בכל זאת לא שב אפו ועוד ידו נטויה מאי ועוד ידו נטויה אמר רב נחמן בר אבא אמר רב הכל יודעין למה כלה נכנסה לחופה אלא כל המנבל את פיו ומוציא דבר נבלה מפיו אפילו חותמין עליו גזר דין של שבעים שנה לטובה הופכין אותו לרעה ואמר רבה בר רב שילא }כל המנבל את פיו מעמיקין לו גיהנם שנא׳ (משלי כ״ב:י״ד) שוחה עמוקה פי זרות רב נחמן בר יצהק אמר אף השומע ושותק שנא׳ (משלי כ״ב:י״ד) זעום ה׳ יפול שם אמר רב אושעיא כל הממרק עצמו לדבר עבירה חבורות ופצעים יוצאות בו שנאמר (משלי כ׳:ל׳) חבורות פצע תמרוק ברע:
ת״ר ארבעה סימנין הן סימן לעבירה הדרוקן סימן לשנאת חנם ירקון סימן לגסות הרוח עניות וסימן ללשון הרע אסכרה גרסינן במס׳ סוטה (דף מב.) }אמר רב חסדא אמר רב ירמיה }ארבע כתות אינן מקבלות פני שכינה כת לצים כת חנפים כת מספרי לשון הרע כת שקרים כת שקרים דכתי׳ (תהילים ק״א:ז׳) דובר שקרים לא יכון לנגד עיני כת לצים דכתיב (הושע ז׳:ה׳) משך ידו את לוצצים כת חנפים דכתיב (איוב י״ג:ט״ז) כי לא לפניו חנף יבא כת מספרי לה״ר דכתיב (תהילים ה׳:ה׳-ו׳) כי לא אל חפץ רשע אתה לא יגורך רע לא יגור במגורך רע וגרסינן }בערכין (דף טו.) תניא א״ר אליעזר בן פרטא בא וראה כמה גדול כח של
\end{multicols}\newpage

\newsection{דף יד}
\begin{multicols}{2}
לשון הרע מנין ממרגלים ומה מרגלים שהוציאו דבה על העצים ועל האבנים כך המוציא דבה על חבירו על אחת כמה וכמה (ערכין טו:) אמר רבי יוחנן משום רבי יוסי בן זמרא מאי דכתיב (תהילים ק״כ:ג׳) מה יתן לך ומה יוסיף לך לשון רמיה אמר לו הקב״ה ללשון לשון מה אעשה לך כל איבריו של אדם זקופין ואתה מוטה כל איבריו של אדם מבחוץ ואתה מבפנים ולא עוד אלא שהקפתי לך ב׳ חומות אחת של עצם ואחת של בשר מה יתן לך ומה יוסיף לך לשון רמיה:
(שם) וא״ר יוחנן משום ר׳ יוסי בן זמרא כל המספר לשון הרע כאילו כופר בהקב״ה שנאמר (תהילים י״ב:ה׳) אשר אמרו ללשוננו נגביר שפתינו אתנו מי אדון לנו וא״ר יוחנן משום ר׳ יוסי בן זמרא כל המספר לשון הרע נגעים באין עליו שנא׳ (תהילים ק״א:ה׳) מלשני בסתר רעהו אותו אצמית וכתיב התם (ויקרא כה) לצמיתות ומתרגמינן לחלוטין ותנן אין בין מצורע מוסגר למצורע מוחלט אלא פריעה ופרימה והיינו דאמר ר״ש בן לקיש מאי דכתיב (ויקרא י״ד:ב׳) זאת תהיה תורת המצורע זו היא תורתו של מוציא שם רע
ה״ד לישנא בישא כגון דאמר איכא נורא בבי פלניא והוא דקא מפיק ליה בלישנא בישא דקאמר הכי היכא משתכחא נורא אלא בביתא דפלניא דנפישי טואי
*}שם טז.}אמר רבא כל מילתא דמתאמרא באנפי מארה לית בה משום לישנא בישא אמר ליה אביי כ״ש חוצפא ולישנא בישא א״ל אנא כי הא דר׳ יוסי סבירא לי דאמר ר׳ יוסי מימי לא אמרתי דבר וחזרתי לאחורי אמר רבה בר רב הונא כל מילתא דמתאמרא באנפי תלתא לית בה }לישנא בישא מ״ט חברך חברא אית ליה וחברא דחברך חברא אית ליה:
כי אתא רב דימי אמר מאי דכתיב (משלי כ״ז:י״ד) מברך רעהו בקול גדול בבקר השכים קללה תחשבלו משום דמברך רעהו בקול גדול קללה תחשב לו אין כגון דאיקלע לאושפיזיה וטרחו קמיה שפיר למחר נפיק ויתיב בשוקא ואמר רחמנא ליברכיה לפלניא דהכי טרח קמאי ושמעי אינשי ואכסני ליה:
תני רב דימי קמיה דרב ספרא לעולם אל יספר אדם בשבחו של חבירו יותר מדאי שמתוך טובתו בא לידי גנותו:
אמר ר׳ שמואל בר נחמני על שבעה גופי עבירות נגעים באין על לשון הרע ועל שפיכות דמים ועל שבועת שוא ועל גילוי עריות ועל גסות הרוח ועל הגזל ועל צרות העין על לה״ר דכתיב (תהילים ק״א:ה׳) מלשני בסתר רעהו אותו אצמית על שפיכות דמים דכתיב (שמואל ב ג׳:כ״ט) ואל יכרת מבית יואב זב ומצורע ועל שכועת שוא דכתיב (מלכים ב ה׳:כ״ג) ויאמר נעמן הואל קח ככרים וכתיב (מלכים ב ה׳:כ״ז) וצרעת נעמן תדבק בך על גילוי עריות דכתיב (בראשית י״ב:י״ז) וינגע ה׳ את פרעה נגעים גדולים על גסות הרוח דכתי׳ (דברי הימים ב כ״ו:ט״ו-ט״ז) ובחזקתו גבה לבו עד להשחית וכתיב (דברי הימים ב כ״ו:י״ט) והצרעת זרחה במצחו על הגזל דכתיב (ויקרא י״ד:ל״ו) וצוה הכהן ופנו את הבית תנא הוא הכניס ממון שאינו שלו יבא כהן ויפזר את ממונו ועל צרות העין דכתיב (ויקרא י״ד:ל״ה) ובא אשר לו הבית תאנא דבי ר׳ ישמעאל מי שמייחד ביתו לו כלומר שאינו מכניס אורחין בתוך ביתו ואינו מהנה לבני אדם מנכסיו
איני והאמר רב ענני בר ששון למה נסמכה פרשת בגדי כהונה לפרשת קרבנות ללמדך מה קרבנות מכפרין אף בגדי כהונה מכפרין כתונת מכפרת על שפיכות דמים דכתיב (בראשית ל״ז:ל״א) ויטבלו את הכתנת בדם ומכנסים מכפרין על גילוי עריות דכתיב (שמות כ״ח:מ״ב) ועשה להם מכנסי בד לכסות בשר ערוה מצנפת מכפרת על גסות הרוח כדרבי חנינא דאמר רבי חנינא יבא דבר שבגובה ויכפר על מעשה גובה אבנט מכפר על הרהור הלב אהיכא דאיתיה דכתיב (שמות כ״ח:ל׳) והיה על לב אהרן חשן מכפר על הדינין דכתיב (שמות כ״ח:ט״ו) ועשית חשן משפט אפור מכפר על עבודה זרה דכתיב (הושע ג׳:ד׳) ואין אפוד ותרפים מעיל מכפר על לשון הרע אמר הקב״ה יבא דבר שבקול ויכפר על מעשה קול ציץ מכפר על עזות פנים דכתיב (שמות כ״ח:ל״ז-ל״ח) והיה על מצח אהרן וכתוב אחד אומר (ירמיהו ג׳:ג׳) ומצח אשה זונה היה לך מאנת הכלם
לא קשיא הא דאהנו מעשיו הא דלא אהנו מעשיו אי דאהנו מעשיו נגעים באין עליו ואי לא אהנו מעשיו מעיל מכפר איני והאמר רבי סימון אמר רבי יהושע בן לוי שני דברים לא מצינו להם כפרה בקרבנות ומצינו להם כפרה בדבר אחר ואלו הן שפיכות דמים ולשון הרע שפיכות דמים בעגלה ערופה ולשון הרע בקטורת דתני רבי חייא למדנו בקטורת שהיא מכפרת שנאמר (במדבר י״ז:י״ב) ויתן את הקטורת ויכפר על העם ותנא דבי ר׳ ישמעאל על מה הקטורת מכפרת על מעשה לשון הרע אמר הקב״ה יבא מעשה חשאי ויכפר על מעשה חשאי קשיא שפיכות דמים אשפיכות דמים קשיא לשון הרע אלשון הרע שפיכות דמים אשפיכות דמים לא קשיא הא דידיע מאן קטליה והא דלא ידיע מאן קטליה אי דידיע מאן קטליה בר קטלא הוא במזיד ולא אתרו ביה לשון הרע אלשון הרע נמי לא קשיא הא בצנעא הא בפרהסיא:
אמר רבי יהושע בן לוי מה נשתנה מצורע שאמרה תורה עליו (ויקרא יג) בדד ישב אמר הקב״ה הוא הבדיל בין איש לאשתו ובין איש לרעהו לפיכך בדד ישב וא״ר יהושע בן לוי מה נשתנה מצורע שאמרה תורה יביא (ויקרא י״ד:ד׳) שתי צפרים לטהרתו אמר הקב״ה הוא עושה מעשה פטיט לפיכך יביא קרבן פטיט:
\textbf{{\largeמתני׳}} (דף לד.) שלשה }}}דברים צריך אדם לומר בתוך ביתו ערב שבת עם חשיכה עשרתם ערבתם הדליקו את הנר:
\textbf{{\largeגמ׳}} מנא הני מילי אמר רבי יהושע בן לוי אמר קרא (איוב ה׳:כ״ד) וידעת כי שלום אהלך ופקדת נוך ולא תחטא אמר רבה בר רב הונא אף על גב דאמור רבנן שלשה דברים צריך אדם לומר בתוך ביתו מיבעי ליה למימרינהו בניחותא כי היכי דליקבלן מיניה אמר רב אשי אנא לא שמיע לי הא דרבה בר רב הונא וקיימתיה מסברא גרסינן בגיטין בפרק ראשון (דף ו:) לעולם אל יטיל אדם אימה יתירה בתוך ביתו שהרי פילגש בגבעה הטיל עליה בעלה אימה יתירה והפילה כמה רבבות
מישראל:
אמר רב יהודה אמר שמואל כל המטיל אימה יתירה בתוך ביתו סוף בא לידי שלש עבירות גילוי עריות ושפיכות דמים וחלול שבת:
\textbf{{\largeמתני׳}} ספק }}חשיכה ספק אינה חשיכה אין מעשרין את הודאי ואין מטבילין את הכלים ואין מדליקין את הנרות אבל מעשרין את הדמאי ומערבין וטומנין את החמין:
\textbf{{\largeגמ׳}} אוקימנא מערבין }עירובי חצרות אבל לא עירובי תחומין:
אמר רבא אמרו לו }שנים צא וערב עלינו אחד עירב עליו מבעוד יום ואחד עירב עליו בין השמשות זה שעירב עליו מבעוד יום נאכל עירובו בין השמשות וזה שעירב עליו בין השמשות נאכל עירובו משחשיכה שניהם קנו עירוב ממה נפשך אי בין השמשות יממא הוא בתרא ליקני קמא לא ליקני ואי בין השמשות לילה הוא בתרא לא ליקני קמא ליקני בין השמשות ספיקא }}דרבנן הוא וספיקא דרבנן לקולא:
אמר רבא מפני מה }}אמרו אין טומנין בדבר המוסיף הבל }(ואפילו) מבעוד יום גזירה שמא ירתיח.
פי׳ הא דתנן במה טומנין ובמה אין טומנין אין טומנין לא בגפת ולא בזבל ולא במלח ולא בסיד ולא בחול והני כולהו דבר המוסיף הבל נינהו ואין טומנין בהן }(אפילו) מבעו״י }גזירה שמא ירתיח א״ל אביי אי הכי בין השמשות נמי ניגזור א״ל סתם קדרות בין השמשות רותחות הן:
אמר רבה מפני מה אמרו אין טומנין בדבר שאין מוסיף הבל משחשכה פירוש כגון פירות וכסות וכנפי יונה ולא התירו לכסות בהן אלא מבעוד יום אי נמי ספק חשיכה ספק אינה חשיכה כדתנן ספק חשיכה ספק אינה חשיכה }אין מעשרין וכו׳ אבל טומנין מכלל דאם חשכה ודאי אין טומנין את החמין גזירה שמא יטמין ברמץ ואתי לאחתויי בגחלים (דף לד:) ת״ר בין }}}השמשות }מן היום ומן הלילה ספק כולו מן היום ספק כולו מן הלילה מטילין אותו לחומרא לשני ימים ואיזהו בין השמשות }משתשקע החמה וכל זמן שפני מזרח מאדימים הכסיף התחתון ולא הכסיף העליון בין השמשות
\end{multicols}\newpage

\newsection{דף טו}
\begin{multicols}{2}
הכסיף העליון והשוה }לתחתון לילה דברי ר׳ יהודה ר׳ נחמיה אומר כדי שילך אדם משתשקע החמה חצי מיל ור׳ יוסי אומר בין השמשות כהרף עין ואי אפשר לעמוד עליו זה נכנס וזה יוצא:
הא גופא קשיא אמרת איזהו בין השמשות משתשקע החמה וכל זמן שפני מזרח מאדימים הא הכסיף התחתון ולא הכסיף העליון לילה }והדר תני הכסיף }התחתון (והשוה העליון לילה הא הכסיף התחתון ולא השוה העליון) בין השמשות אמר רבה כרוך ותני איזהו בין השמשות משתשקע החמה וכל זמן שפני מזרח מאדימים הכסיף התחתון ולא הכסיף העליון נמי בין השמשות הכסיף העליון והשוה לתחתון לילה ורב יוסף אמר רב יהודה אמר שמואל הכי קתני משתשקע החמה וכל זמן שפני מזרח מאדימים יום הכסיף התחתון ולא הכסיף העליון בין השמשות הכסיף העליון והשוה לתחתון לילה (דף לה.) אמר רבה בר בר חנה אמר רבי יוחנן הלכה כרבי יהודה לענין שבת והלכה כרבי יוסי לענין תרומה (דף לה:) דלא אכלי כהנים תרומה עד דשלים בין השמשות דר׳ יוסי:
הא דפסק ר׳ יוחנן הלכה כר׳ יהודה לענין שבת לא ידעינן אליבא דמאן פסק אי אליבא דרבה ואי אליבא דרב יוסף וכיון דלא איבריר לן כמאן מינייהו פסק עבדינן לחומרא דאיסורא הוא וספק איסורא לחומרא ועוד דסוגיא ככוליה תלמודא כל היכא דאיפליגו רבה ורב יוסף הלכה כרבה (ב״ב דף קיד:) בר משדה קנין ומחצה הלכך משתשקע החמה איתקדיש ליה יומא ואסור בעשיית מלאכה:
אמר רב יהודה אמר שמואל כוכב אחד }יום שנים בין השמשות שלשה לילה:
אמר רבי }אסי בר אבין משנראו שלשה כוכבים ולא כוכבים גדולים שנראין ביום ולא כוכבים קטנים שנראין בלילה אלא בינוניים אמר ליה רבא לשמעיה אתון דלא ידעיתו שיעורא דרבנן אדאיכא שמשא בריש דיקלי אדליקו שרגא ביום המעונן במתא חזו תרנגולי בדברא חזו עורבי אי נמי אדאני:
ת״ר שש תקיעות }תוקעין ע״ש תקיעה ראשונה לבטל העם ממלאכה שבשדות שניה לבטל עיר וחנויותיה שלישית להדליק את הנר }ותוקע ומריע ותוקע ושובת דברי רבי נתן רבי יהודה הנשיא אומר שלישית לחלוץ תפילין ושוהה כדי לצלות דג קטן או כדי להדביק פת בתנור ותוקע ומדיע ותוקע }ושובת:
\textbf{סליקו להו במה מדליקין} 
\end{multicols}\newpage

\newchap{פרק \hebrewnumeral{3} כירה}
\begin{multicols}{2}
\textbf{{\largeכירה}} }}שהסיקוה }בקש או בגבבא נותנין עליה תבשיל בגפת או בעצים לא יתן עד שיגרוף או עד שיתן את האפר ב״ש אומרים חמין אבל לא תבשיל וב״ה אומרים חמין ותבשיל ב״ש אומרים נוטלין אבל לא מחזירין וב״ה אומרים אף מחזירין:
\textbf{{\largeגמ׳}} איבעיא להו האי לא יתן לא יחזיר הוא אבל לשהות
\end{multicols}\newpage

\newsection{דף טז}
\begin{multicols}{2}
משהין אע״ג שאינו גרוף ואינו קטום ומתני׳ חנניא היא }או דלמא האי לא יתן לא ישהה הוא ואפילו לשהות אי גרוף וקטום אין ואי לא לא ופליגא דחנניא ושקלינן וטרינן ומסקנא דלא יתן לא ישהה הוא ואפילו לשהות אי גרוף וקטום אין ואי לא לא וכ״ש להחזירה
}ומסתברא לן דהכי הוא מסקנא דהא בתר דשקלינן וטרינן (דף לז.) איבעיא להו מהו לסמוך לה תוכה וגבה הוא דאסור אבל לסמוך שפיר דמי או דלמא לא שנא ומדאמרינן תוכה וגבה הוא דאסיר שמעינן דאסור לשהות על גבי כירה שאינה גרופה וקטומה ועוד דהא עסקינן לבעיין מאי הוי עלה ת״ש כירה שהסיקוה בגפת ובעצים סומכין לה ואין משהין על גבה אלא א״כ גרופה וקטומה וגחלים שעממו או שנתנו עליהן נעורת של פשתן דקה הרי היא כקטומה אלמא דהכין היא הלכתא ור׳ אושעיא נמי הכי סבירא ליה (דף לז:) ורבה בר בר חנה א״ר יוחנן נמי הכי סבירא ליה
ועוד הא אמרינן לקמן לענין תנור שהסיקוהו בקש ובגבבא (דף לח:) סבר רב יוסף למימר תוכו תוכו ממש על גביו על גביו ממש אבל לסמוך שפיר דמי ואותביה אביי כיפה שהסיקוה בקש ובגבבא הרי היא ככירים בגפת וכעצים הרי היא כתנור }הרי היא כתנור דאסור הא ככירה שרי במאי עסקינן אילימא על גבה ובמאי אילימא בשאינה גרופה ואינה קטומה אלא כירה שאינה גרופה וקטומה על גבה מי שרי אלמא דהכין הלכתא דכירה כי אינה גרופה ואינה קטומה על גבה אסור }(דף לז:) והא דאמר רב
ששת א״ר יוחנן דמתניתין להחזיר תנן אבל לשהות משהין ואע״ג שאינו גרוף וקטום לית הלכתא כוותיה ואע״ג דסייעיה רבא ואמר תרוייהו תננהו ההוא סיועא לאו דוקא הוא ולא גמרינן מיניה דעל מסקנא דגמרא סמכינן ומדחזינן למסקנא דגמרא וסוגיא דשמעתא דאסור לשהות על גבי כירה שאינה גרופה וקטומה ממילא שמעינן דמימרא דרב ששת וסייעתא דרבא וכל מאן דדמיא להו דחויות אינון הלכך כירה שאינה גרופה ואינה קטומה אסור לשהות עליה חמין שלא הוחמו כל צרכן ותבשיל שלא בישל כל צרכו אבל תבשיל שבישל כל צרכו וחמין שהוחמו כל צרכן אע״פ שאינה גרופה ואינה קטומה משהין.
ודוקא דבר שהוא מצטמק ורע לו כגון ליפתא ודייסא ותמרי וכיוצא בהן אבל דבר שהוא מצטמק ויפה לו כגון כרוב ופולין ובשר טרוף אסור וכירה שהיא גרופה וקטומה משהין עליה כל מילי בין בישל כל צרכו בין לא בישל כל צרכו וכ״ש מצטמק }[ויפה] לו דעיקרא דגזירה דלמא אתי לאחתויי בגחלים הוא וכיון שהיא גרופה וקטומה לא אתי לאחתויי בגחלים וכן הלכתא.
ואי אינשי על גבי כירה שאינה גרופה וקטומה מידי דלא בשיל כל צרכיה אסיר וכל שכן אם עבר ושהה במזיד (דף לח.) דאמר רב שמואל בר יהודה אמר רב יהודה בתחלה היו אומרים המבשל בשבת בשוגג יאכל במזיד לא יאכל והוא הדין לשוכח משרבו משהין במזיד ואומרים שוגגין אנו חזרו וגזרו אף על השוכח והיכא דשכח מידי דבשיל כל צרכיה ומצטמק ויפה לו הא מילתא איבעיא לן בגמ׳
\end{multicols}\newpage

\newsection{דף יז}
\begin{multicols}{2}
ולא איפשיטא ולקולא עבדינן ולא אסרינן לההוא תבשיל דספיקא דרבנן הוא ולקולא:
גרסינן בפ׳ מרובה (ב״ק דף עא.) }}המבשל בשבת בשוגג יאכל במזיד לא יאכל דברי ר״מ רבי יהודה אומר בשוגג }יאכל למוצאי שבת במזיד לא יאכל עולמית ר׳ יוחנן הסנדלר אומר בשוגג יאכל למוצאי שבת לאחרים ולא לו במזיד לא יאכל עולמית לא לו ולא לאחרים מאי טעמא דר׳ יוחנן הסנדלר כדדרש ר׳ חייא אפיתחא דבי נשיאה (שמות ל״א:י״ד) ושמרתם את השבת כי קדש היא לכם מה קדש אסור באכילה אף מעשה שבת אסור באכילה אי מה קדש אסור בהנאה אף מעשה שבת אסור בהנאה תלמוד לומר לכם שלכם תהא יכול אפילו בשוגג תלמוד לומר (שמות ל״א:י״ד) מחלליה מות יומת במזיד אמרתי לך ולא בשוגג פליגי בה רב אחא ורבינא חד אמר מעשה שבת דאורייתא וחד אמר מעשה שבת דרבנן מאן דאמר מעשה שבת דאורייתא כדאמרן ומאן דאמר דרבנן דאמר קרא כי קדש היא לכם היא קדש ואין מעשיה קדש
וקי״ל דבכל התורה כולה רב אחא ורבינא הלכה כדברי המיקל וש״מ דליתא לדר׳ יוחנן הסנדלר הלכך הלכה }כרבי יהודה דאמר המבשל בשבת בשוגג }יאכל למוצאי שבת בין לו בין לאחרים במזיד יאכל למוצאי שבת לאחרים ולא לו גרסינן בחולין בפ״ק (דף טו:) השוחט לחולה בשבת מותר לבריא באומצא מ״ט כיון דאי אפשר לכזית בשר בלא שחיטה כי קא שחיט אדעתא דחולה קא שחיט המבשל לחולה בשבת אסור
לבריא גזירה שמא ירבה בשבילו:
וב״ה }}}אומרים אף מחזירין: אמר רב ששת לדברי האומר (דף לח:) מחזירין מחזירין }(כמה פעמים) }בשבת ואף רבי אושעיא סבר מחזירין ואפילו בשבת דאמר ר׳ אושעיא פעם אחת היינו עומדין לעילא מרבי חייא רבה והעלינו לו קומקום של חמין מדיוטא התחתונה לדיוטא העליונה ומזגנו לו את הכוס והחזרנוהו למקומו ולא אמר לנו דבר אמר רבי זריקא א״ר אמי אמר רבי תדאי לא שנו אלא שעודן בידו אבל הניחן על גבי קרקע אסור א״ר חזקיה משמיה דאביי הא דאמרן עודן }בידו מותר לא אמרן אלא שדעתו להחזיר אבל אין דעתו להחזיר אסור מכלל דאם הניחן על גבי קרקע אע״פ שדעתו להחזיר אסור:
בעי ר׳ ירמיה תלאן במקל מהו הניחן על גבי מטה מהו תיקו בעי רב אשי פינן ממיחם למיחם מהו תיקו:
\textbf{{\largeמתני׳}} תנור שהסיקוהו בקש או בגבבא לא יתן בין מתוכו בין מעל גביו כיפה שהסיקוה בקש ובגבבא הרי היא ככירים בגפת או בעצים הרי היא כתנור:
\textbf{{\largeגמ׳}} ת״ר תנור }שהסיקוהו בקש או בגבבא אין סומכין לו }ואין צריך לומר בתוכו ואין צריך לומר על גביו ואין צריך לומר שהסיקוהו בגפת או בעצים:
כיפה שהסיקוה: }אמר ליה רב אחא בריה דרבא לרב אשי האי כיפה היכי דמי אי ככירה דמי אפילו בגפת ובעצים נמי לישתרי ואי כתנור דמי אפילו בקש ובגבבא נמי לא לישתרי אמר ליה נפיש הבליה מדכירה וזוטר מתנור:
ושמעינן מינה דכל מידי דאסור לאשהויי על גבי כירה אלא אם כן גרופה וקטומה אסור לאשהויי ע״ג כיפה אפי׳ גרופה וקטומה אבל חמין שהוחמו כל צרכן ותבשיל שבישל כל צרכיה שרי לאשהויי ע״ג כיפה מידי דהוה אכירה דאע״ג דאסיר לאשהויי עלה כי לא גרופה וקטומה מידי דלא בשיל כל צרכיה אבל מידי דבשיל כל צרכיה שרי הכא נמי לא שנא ודוקא מצטמק ורע לו דומיא דכירה אכל מצטמק ויפה לו }אסיר והיכי דמי כירה והיכי דמי כיפה אמר ר׳ יוסי בר׳ חנינא כירה מקום שפיתת ב׳ קדירות
\end{multicols}\newpage

\newsection{דף יח}
\begin{multicols}{2}
כיפה מקום שפיתת קדרה אחת:
\textbf{{\largeמתני׳}} אין נותנין }}}ביצה בצד המיחם בשביל שתתגלגל ולא יפקיענה בסודרין ורבי יוסי מתיר ולא יטמיננה בחול ובאבק דרכים בשביל שתצלה:
\textbf{{\largeגמ׳}} איבעיא להו גלגל מאי אמר רב אסי גלגל חייב חטאת אמר מר זוטרא אף אנן נמי תנינא (דף לט.) כל הבא בחמין מלפני השבת שורין אותו בחמין בשבת וכל שלא בא בחמין מלפני השבת מדיחין אותו בחמין בשבת חוץ מן המליח הישן וקולייס האספנין שהדחתן זו היא גמר מלאכתן:
ולא יפקיענה בסודרין ור׳ יוסי מתיר: והא דתנן נותנין את הצונן בחמה בשביל שיחמו לימא ר׳ יוסי היא ולא רבנן אמר רב נחמן בר יצחק בחמה כ״ע לא פליגי דשרי בתולדות האור כולי עלמא לא פליגי דאסיר כי פליגי בתולדות חמה רבנן סברי גזרינן תולדות חמה אטו תולדות האור ור׳ יוסי סבר לא גזרינן והלכתא כת״ק:
ולא יטמיננה בחול: ובהא אפילו רבי יוסי מודה מאי טעמא רבה אמר גזירה שמא יטמין ברמץ ורב יוסף אמר גזירה שמא יזיז עפר ממקומו מאי בינייהו איכא בינייהו עפר תיחוח:
\textbf{{\largeמתני׳}} (דף לח:) מעשה שעשו אנשי }}טבריא והביאו סילון של צונן בתוך אמה של חמין אמרו להם חכמים אם בשבת כחמין שהוחמו בשבת ואסורין ברחיצה ובשתיה ואם ביום טוב כחמין שהוחמו ביום טוב ואסורין ברחיצה ומותרין בשתיה:
\textbf{{\largeגמ׳}} (דף לט:) אמר עולא הלכה כאנשי טבריא אמר ליה רב נחמן כבר תברינהו אנשי טבריא לסילונייהו והלכה כרב נחמן:
תניא לא ישתטף אדם כל גופו בין בחמין בין בצונן דברי רבי מאיר ור״ש מתיר ר׳ יהודה אומר בחמין אסור בצונן מותר:
אמר רב חסדא מחלוקת בקרקע אבל בכלי דברי הכל אסור.
אמר רבה בר בר חנה א״ר יוחנן הלכה כרבי יהודה והלכתא בחמין אסור בין בכלי בין ע״ג קרקע ודוקא חמי האור אבל חמי טבריא שרו כדבעי׳ למימר קמן:
(דף מ:) איתמר חמין שהוחמו מערב שבת רב אמר למחר רוחץ כהן כל גופו אבר אבר ושמואל אמר לא התירו לרחוץ אלא פניו ידיו ורגליו תניא כותיה דשמואל חמין שהוחמו מערב שבת למחר רוחץ בהן פניו ידיו ורגליו אבל לא כל גופו אבר אבר ואין צריך לומר חמין שהוחמו *}גי׳ הגמ׳ ביו״ט}בשבת מרחץ שפקקו נקביו מערב שבת למוצאי שבת רוחץ בו מיד
אמר רבי שמעון בן פזי אמר רבי יהושע בן לוי משום בר קפרא בתחלה היו רוחצין בחמין שהוחמו מערב שבת התחילו הבלנין להחם חמין בשבת ואומרים מערב שבת הוחמו אסרו להם את החמין והתירו את הזיעה ועדיין היו רוחצין בחמין ואומרים מזיעין אנחנו אסרו להם את הזיעה והתירו }להם חמי טבריא ועדיין היו רוחצין בחמי האור ואומרים בחמי טבריא רחצנו אסרו להם נמי חמי טבריא והתירו להם את הצונן ראו שאין הדבר עומד התירו להם חמי טבריא ועדיין זיעה במקומה עומדת:
אמר רבא האי מאן דעבר אדרבנן }שרי למקרייה עבריינא (דף מ:) ת״ר לא ישוט אדם בבריכה מלאה מים ואפילו עומדת בחצר ואי אית ביה גדודי שרי }מ״ט דא״נ עקר למיא הא
\end{multicols}\newpage

\newsection{דף יט}
\begin{multicols}{2}
איכא גדודי דמהדרי להו:
(דף מ:) ת״ר מתחמם אדם כנגד המדורה ויוצא ומשתטף בצונן ובלבד שלא ישתטף בצונן ויבא ויתחמם כנגד המדורה בשבת מפני שמפשיר מים שעליו:
תנו רבנן מיחם אדם אלונטית ומניחה ע״ג מעים בשבת ובלבד שלא יביא קומקום של חמין ויניח ע״ג מעים בשבת ודבר זה אפילו בחול אסור מפני הסכנה:
ת״ר }}}מביא אדם קיתון }של מים ויניחנו כנגד המדורה לא בשביל שיחמו אלא כדי שתפוג צנתן ר׳ יהודה אומר מביאה אשה פך של שמן ומניחתו כנגד המדורה לא בשביל שיתבשל אלא בשביל שיתפשר רבן שמעון ב״ג אומר סכה אשה ידה }בשמן ומחממת כנגד המדורה וסכה לבנה קטן ואינה חוששת:
אמר רב יהודה אמר שמואל אחד מים ואחד שמן יד סולדת בו אסור אין יד סולדת בו מותר.
היכי דמי יד סולדת בו אמר רחבה כל שכריסו של תינוק נכוית ממנו א״ר יצחק בר אבדימי פעם אחת נכנסתי אחר רבי לבית המרחץ ובקשתי להניח }פך של שמן באמבטי ואמר לי טול בבלי שני ותן ש״מ תלת ש״מ שמן יש בו משום בישול וש״מ כלי שני אינו מבשל וש״מ הפשירו לא זהו בישולו:
}אמר ר׳ אבא אמר ר׳ יוחנן }כל מקום מותר להרהר חוץ מבית הכסא ובית המרחץ אמר אביי דברים של קדש אסור לאמרן בלשון חול ודברים של חול מותר לאמרן בלשון קדש.
ולענין אפרושי מאיסורא אפילו בלשון קדש שרי דהא רבי אמר ליה לרב יצחק בר אבדימי טול בכלי שני ותן ואמרינן נמי אמר רב יהודה אמר שמואל מעשה בתלמידו של ר׳ מאיר שנכנס אחריו לבית המרחץ ובקש להדיח לו קרקע ואמר לו אין מדיחין לסוך לו }(שמן) ואמר לו אין סכין:
\textbf{{\largeמתני׳}} (דף מא.) }}מולייר הגרוף שותין ממנו בשבת אנטיכי אע״פ שהיא גרופה }וקטומה אין שותין הימנה בשבת:
\textbf{{\largeגמ׳}} היכי דמי מולייר הגרוף תאנא מים מבפנים וגחלים מבחוץ:
מאי אנטיכי אמר רב נחמן בי דודי: תניא כותיה דרב נחמן אנטיכי אע״פ שהיא גרופה וקטומה אין שותין הימנה בשבת מפני שנחושתה מחממתה:
\textbf{{\largeמתני׳}} }המיחם שפינהו לא יתן לתוכו צונן בשביל שיחמו אבל נותן הוא לתוכו או לתוך הכוס כדי להפשירן:
\textbf{{\largeגמ׳}} מאי קאמר (דף מא:) אמר אביי הכי קאמר המיחם }שפינה ממנו ויש בו מים חמין לא יתן לתוכו מים מועטין כדי שיחמו אבל נותן הוא לתוכו מים רבים כדי שיפשירו ומיחם שפינה מים ממנו לא יתן לתוכו מים כל עיקר מפני שהוא מצרף ור׳ יהודה היא דאמר דבר שאין מתכוין אסור.
אמר רב לא שנו אלא שיעור }לצרף אבל שיעור להפשיר מותר ושמואל אמר אפילו שיעור לצרף נמי מותר (דף מב.) כר׳ שמעון דאמר דבר שאין מתכוין מותר ואי קשיא לך הא דאמר שמואל }}מכבין }גחלת של מתכת ברשות הרבים בשביל שלא יזוקו בה רבים אבל גחלת של עץ לא דאי כר׳ שמעון סבירא ליה אפילו של עץ נמי מותר בדבר שאין מתכוין סבר לה כר׳ שמעון במלאכה שאינה צריכה לגופה סבר לה כרבי יהודה דאמר חייבין עליה.
אמר רבינא הלכך קוץ ברשות הרבים מוליכו פחות פחות מארבע אמות ובכרמלית אפילו טובא וכן הלכתא:
אבל נותן הוא לתוכו או לתוך הכוס כדי להפשירן:
}}}ת״ר נותן אדם חמין לתוך צונן אבל לא צונן לתוך }חמין דברי בית שמאי
\end{multicols}\newpage

\newsection{דף כ}
\begin{multicols}{2}
ובית הלל אומרים בין חמין לתוך צונן ובין צונן לתוך חמין מותר במה דברים אמורים בכוס אבל באמבטי חמין לתוך צונן אין ולא צונן לתוך חמין ורבי שמעון בן מנסיא אוסר ומסקנא דשמעתא דהלכתא כב״ה דקאמרינן אמר רב הונא בריה דרב יהושע חזינא ליה }לאבא דלא קפיד אמנא מדתני רבי חייא }מערב אדם קיתון של מים לתוך ספל של מים בין חמין לתוך צונן ובין צונן לתוך חמין:
\textbf{{\largeמתני׳}} }האלפס והקדרה שהעבירן מרותחין לא יתן לתוכן תבלין (דף מב:) אבל נותן הוא לתוך הקערה או לתוך התמחוי רבי יהודה אומר לכל הוא נותן חוץ מדבר שיש בו חומץ וציר:
\textbf{{\largeגמ׳}} והלכתא כת״ק סבר רב יוסף למימר מלח הרי הוא כתבלין דבכלי ראשון בשיל ובכלי שני לא בשיל א״ל אכיי תני ר׳ חייא מלח אינו כתבלין דבכלי שני נמי בשיל ופליגא דרב נחמן דאמר רב נחמן צריכא מילחא בישולא כבישרא דתורא:
ואיכא דאמרי סבר רב יוסף למימר מלח הרי הוא כתבלין דבכלי ראשון בשיל ובכלי שני לא בשיל }אמר ליה אביי תני רבי חייא מלח אינו כתבלין דבכלי ראשון נמי לא בשיל והיינו דאמר רב נחמן צריכא מילחא בישולא כבישרא דתורא והלכתא כלישנא בתרא:
\textbf{{\largeמתני׳}} }}}אין נותנין כלי תחת הנר לקבל בו את השמן ואם נתנו מבע״י מות׳ ואין נאותין ממנו לפי שאינו מן המוכן:
\textbf{{\largeגמ׳}} אמר רבא אמר רב חסדא אע״פ שאמרו אין נותנין כלי תחת התרנגולת לקבל ביצתה
לפי שאסור לבטל כלי מהיכנו אבל כופה עליה }כלי שלא תשבר:
(דף מג:) איתמר מת }}}המוטל בחמה אמר רב יהודה אמר שמואל הופכו ממטה למטה ורב חיננא בר שלמיא משמיה דרב אמר מניח עליו ככר או תינוק ומטלטלו היכא דאיכא ככר או תינוק כולי עלמא לא פליגי דשרי כי פליגי היכא }דליכא מר סבר טלטול מן הצד שמיה טלטול ומר סבר לא שמיה טלטול:
לימא כתנאי אין מצילין את המת מפני הדליקה אמר רבי יהודה בן לקיש שמעתי שמצילין את המת מפני הדליקה היכי דמי אי דאיכא ככר או תינוק מ״ט דת״ק דאסר אי דליכא ככר או תינוק מ״ט דרבי יהודה בן לקיש דשרי אלא לאו בטלטול מן הצד פליגי דמר סבר טלטול מן הצד שמיה טלטול ומר סבר לא שמיה טלטול לא דכולי עלמא טלטול מן הצד שמיה טלטול והיינו טעמא דרבי יהודה בן לקיש מתוך שאדם בהול על מתו (דף מד.) אי לא שרית ליה טלטול דרבנן }אתי לידי כבוי דאורייתא:
אמר רב יהודה בר שילא א״ר יוחנן הלכה כרבי יהודה בן לקיש במת:
שמעינן מינה דטלטול מן הצד שמיה טלטול }(והיינו טעמא דרבי יהודה) דלא שרו ליה רבנן אלא גבי מת דלא אתי לידי איסורא דאורייתא אבל בעלמא אסור וקשיא לן הא דגרסינן }בפ׳ כל הכלים (דף קכג.) לענין פגה שטמנה בתבן וחררה שטמנה בגחלים אם מגולה מקצתה מותר לטלטל אלעזר בן תדאי אומר תוחבין לה בכוש או בכרכר והיא ננערת מאיליה א״ר נחמן הלכה כאלעזר בן תדאי דסבר טלטול מן הצד לא שמיה טלטול ופרקינן כי אמרינן טלטול מן הצד שמיה טלטול הני מילי כגון אבנים וכיוצא בהו דאינן לצורך שבת דאסור לטלטל ממקום למקום דומיא דמת כדאמרינן התם בן שמנה חי הרי הוא כאבן ואסור לטלטלו דכמת דמי אבל פגה שטמנה }בתבן ופוגלה וחררה שטמנה ברמץ וכיוצא בהן דאוכלין הן וצריך לאוכלן בשבת טלטול מן הצד בכל כי האי מילתא לא שמיה טלטול:
ואין נאותין ממנו לפי שאינו מן המוכן: תנו רבנן מותר השמן שבנר ושבקערה אסור להסתפק ממנו ורבי שמעון מתיר:
\textbf{{\largeמתני׳}} מטלטלין נר חדש אבל לא ישן דברי דבי יהודה רבי מאיר אומר כל הנרות מטלטלין
\end{multicols}\newpage

\newsection{דף כא}
\begin{multicols}{2}
חוץ מן הנר שהדליקו בו באותה שבת רבי שמעון אומר כל הנרות מטלטלין חוץ מן הנר הדולק בשבת (דף מז:) נותנין כלי תחת הנר לקבל בו ניצוצות אבל לא יתן לתוכו מים מפני }(שהן מכבין):
גמ׳ (דף מד.) [תנו רבנן] מטלטלין וכו׳ חוץ מן }הנר הדולק בשבת כבה מותר לטלטלו.
ואף על גב דקיימא לן הלכתא כרבי שמעון לענין מוקצה בהא }לית הלכתא כוותיה דגרסינן בפרק מי שהחשיך (דף קנז.) פליגי בה רב אחא ורבינא חד אמר בכל השבת כולה הלכה כרבי שמעון לבד ממוקצה מחמת מיאוס ומאי ניהו נר ישן וחד אמר במוקצה מחמת מיאוס נמי הלכתא כוותיה לבר ממוקצה מחמת איסור ומאי ניהו נר שהדליקו בו באותה שבת אבל מוקצה מחמת חסרון כיס אף רבן שמעון מודה דתנן כל הכלים ניטלין בשבת חוץ מן המסר הגדול ויתד של מחרישה.
וקי״ל דכל היכא דפליגי רב אחא ורבינא הלכה כדברי המיקל הלכך נר שהדליקו בה באותה שבת אף על פי שכבה אסור לטלטלו דמוקצה מחמת איסור הוא וכן מותר שמן שבנר ושבקערה שהדליקו בהן באותה שבת אסור לטלטלו ולהסתפק ממנו באותה שבת:
אמר ר׳ זירא }}פמוט שהדליקו עליה באותה שבת דברי הכל אסור לטלטלה בשבת לא }הדליקו עליה באותה שבת דברי הכל מותר לטלטלה בשבת וכן הלכתא (דף מד:) ומטה שיש עליה מעות אסור לטלטלה אין עליה מעות מותר לטלטלה בין יחדה בין לא יחדה והוא שלא היו }עליה בין השמשות אבל אם היו עליה בין השמשות מגו דאיתקצאי בין השמשות איתקצאי לכולי יומא והוה ליה מוקצה מחמת איסור הילכך אסור לטלטלה }(מיתיבי }מוכני שלה בזמן שהיא נשמטת אין חיבור לה ואין נמדדת עמה ואין מצלת עמה באהל המת ואין גוררין אותה בשבת בזמן שיש עליה מעות הא אין עליה מעות שריא אע״ג דהוה עליה בין השמשות ההיא ר׳ שמעון היא)
והלכתא }}}}(דף מה.) מניחין נר על גבי דקל בשבת ואין מניחין נר על גבי דקל ביום טוב מאי טעמא כיון דבשבת בדיל מיניה לא אתי לאשתמושי במחובר וביום טוב דלא בדיל מיניה אתי לאשתמושי במחובר הילכך בשבת שרי וביום טוב אסור:
אמר רב יהודה אמר שמואל אין מוקצה לרבי שמעון אלא גרוגרות וצמוקין בלבד.
ירושלמי מאי שנא הני הואיל ומסריחות בינתים
(דף מז.) אמר רבא כי הוינן בי רב נחמן הוינן מטלטלינן }כנונא }}}אגב קיטמא ואע״ג דאיכא עליה שברי עצים:
תניא מלבנות המטה וכרעי המטה ולווחים של סקיבוס לא יחזיר ואם החזיר פטור (דף מז:) אבל אסור לא יתקע ואם תקע חייב חטאת רשב״ג אומר אם היה רפוי מותר בי רב חמא הוה ליה מטה גללניתא דהוו מהדרי ליה ביומא טבא א״ל ההוא מרבנן לרב חמא מאי דעתיך בנין מן הצד הוא נהי דאיסורא דאורייתא ליכא איסורא דרבנן מיהא איכא אמר ליה אנא כרשב״ג סבירא לי דאמר אם היה רפוי }מותר וכן הלכתא:
}}}נותנין כלי תחת הנר לקבל בו ניצוצות:
והא קא מבטל כלי מהיכנו אמר רב הונא בריה דרב יהושע ניצוצות אין בהן ממש ושרי לטלטולי ולא קא מבטל כלי מהיכנו:
ולא יתן לתוכו מים מפני שהוא מכבה: תניא נותנין כלי תחת הנר לקבל ניצוצות בשבת ואין צריך לומר בערב שבת ולא יתן לתוכו מים }מפני שהוא מכבה מע״ש ואין צריך לומר בשבת מפני שהוא מקרב כיבוי:
\textbf{סליקו להו כירה} 
\textbf{{\largeבמה}} }}}טומנין ובמה אין }טומנין אין טומנין לא בגפת ולא בזבל ולא במלח ולא בסיד ולא בחרסית ולא בחול בין לחין בין יבשים לא בתבן לא בזגין ולא במוכין ולא בעשבין בזמן שהן לחין אבל טומנין בהן כשהן יבשין וטומנין בכסות וכנפי יונה ובפירות ובנסורת של חרשים ובנעורת של פשתן }
\end{multicols}\newpage

\newchap{פרק \hebrewnumeral{4} במה טומנין}
\end{multicols}\newpage

\newsection{דף כב}
\begin{multicols}{2}
רבי יהודה אוסר בדקה ומתיר בגסה:
\textbf{{\largeגמ׳}} איבעיא להו גפת של זיתים תנן אבל של שומשמין שפיר דמי או דלמא דשומשמין תנן וכל שכן דזיתים דלא:
תא שמע דאמר רבי זירא משום חד דבי רבי ינאי קופה שטמן בה אסור להניחה על גבי גפת של זיתים שמע מינה גפת של זיתים תנן לא לעולם אימא לך לענין הטמנה אפילו דשומשמין אסור אבל לענין (דף מח.) אסוקי הבלא דזיתים מסקי הבלא דשומשמין לא מסקי הבלא:
}}רבה ורבי זירא איקלעו לבי ריש גלותא חזיוהו לההוא עבדא דאתנח }כוזא דמיא אפומא דקומקומא נזהיה רבה אמר ליה רבי זירא מאי שנא ממיחם על גבי מיחם אמר ליה התם אוקומי קא מוקים הכא אולודי קא מוליד הדר חזייה דקא פריס סודרא אפומא דכובא ואנח נטלא עילויה נזהיה רבה אמר ליה רבי זירא אמאי אמר ליה השתא קא חזית לסוף חזייה דקא עצר ליה אמר ליה ומאי שנא מפרונקא אמר ליה התם לאקפיד עליה הכא קפיר עליה:
לא בתבן לא בזגין לא במוכין: בעא מיניה רב אדא בר אהבה מאביי }מוכין שטמן בהן מהו לטלטלן בשבת אמר ליה וכי מפני שאין לו לזה קופה של תבן עומד *}נ״א ומפקיר}ומטלטל קופה של מוכין רב חסדא שרא לאהדורי אודרא לבי סדיא בשבת׳ איתיביה רב חנין בר רבא לרב חסדא מתירין בית הצואר בשבת אבל לא פותחין ואין נותנין את המוכין לא לתוך הכר ולא לתוך הכסת ביום טוב וכל שכן בשבת לא קשיא הא בחדתי הא בעתיקי:
תניא נמי הכי אין נותנין את המוכין לא לתוך הכר ולא לתוך הכסת ביום טוב ואין צריך לומר בשבת נשרו מחזירין אותן בשבת ואין צריך לומר ביום טוב:
אמר רב }}}יהודה אמר רב הפותח בית הצואר בשבת חייב חטאת מתקיף ליה רב כהנא וכי (דף מח:) מה בין זה למגופת חבית אמר ליה זה חיבור וזה אינו חיבור:
בזמן
}}}שהן לחין: (דף מט.) איבעיא להו לחין מחמת עצמן תנן או דלמא לחין מחמת דבר אחר וסליקא דמחמת עצמן תנן ומוכין לחין מחמת עצמן היכי משכחת לה כגון מרטא דביני אטמי וכסות לחה דעבידא ממרטא דביני אטמי והוא הצמר שבין ירכותיה של בהמה שיש בו לחלוחית קבועה שאינה עוברת:
\textbf{{\largeמתני׳}} טומנין בשלחין ומטלטלין אותם בגיזי צמר ואין מטלטלין אותן.
כיצד הוא עושה נוטל את הכסוי והן נופלים מאליהן ר׳ אלעזר בן עזריה אומר קופה מטה על צדה ונוטל שמא יטול ואינו יכול להחזיר וחכמים אומרים נוטל ומחזיר.
אם לא כסהו מבעוד יום לא יכסנו משתחשך.
כסהו ונתגלה מותר לכסותו. ממלא אדם את הקיתון מים ונותן תחת הכר או תחת הכסת:
\textbf{{\largeגמ׳}} יתיב ר׳ יוחנן בר ערמאי ור׳ יוחנן בר אלעזר ויתיב ר׳ חנינא בר אבא קמייהו ויתבי וקא מיבעיא להו }שלחין של בעל הבית תנן אבל של אומן כיון דקפיד עלייהו אין מטלטלין או דלמא של אומן תנן וכל שכן של בעל הבית וסליקא כתנאי.
ומסתברא דהלכה כר׳ יוסי דאמר אחד זה ואחד זה מטלטלין אותן דרבי יוסי (דף מט:) שלחא הוה אביו ונתברר לו שאין האומנין מקפידין עליהן וכן פסק רב שרירא גאון משום הראשונים שהלכה כר׳ יוסי:
בגיזי צמר ואין מטלטלין אותן: (דף נ.) אמר רבא לא שנו אלא שלא ייחדן להטמנה אבל אם ייחדן להטמנה מטלטלין אותן:
איתמר נמי הכי כי אתא רבין א״ר יצחק אמר ר׳ יוסי בן שאול אמר רב לא שנו אלא שלא ייחדן להטמנה אבל ייחדן להטמנה מטלטלין אותן:
תני רבה בר בר חנה קמיה דר׳ יוחנן }חריות של }דקל שגדרן לעצים ונמלך עליהן
\end{multicols}\newpage

\newsection{דף כג}
\begin{multicols}{2}
לישיבה צריך לקשרן רשב״ג אומר א״צ לקשרן הוא תני לה והוא אמר לה הלכה כר״ש בן גמליאל:
איתמר רב אמר קושר ושמואל אמר חושב ורב אסי אמר יושב אע״פ שלא קשר ושלא חישב והלכתא כרב אסי }דאמר כי האי תנא דתניא יוצאין בפוקרין }ובציפה אימתי בזמן שצבען בשמן וכרכן במשיחה אבל לא צבען בשמן ולא כרכן במשיחה אין יוצאין בהן ואם יצא בהן שעה אחת מבעוד יום אע״פ שלא צבען בשמן ולא כרכן במשיחה מותר לצאת בהן:
רב אשי אמר הא נמי סייעתא }ממתני׳ דתנן הקש שעל המטה לא ינענעו בידו אבל ינענעו בגופו אם היה מאכל בהמה או שהיה עליו כר או סדין ינענענו בידו:
אמר רב יהודה מכניס אדם מלא קופתו עפר ועושה בו כל צרכו דרש מר זוטרא משמיה דמר זוטרא רבה והוא שיחד לו קרן }זוית:
תניא בכל }}חפין את הכלים חוץ מכלי כסף בגרתקון הא נתר וחול מותר:
(דף נ:) אמר רב יהודה }}עפר לבינתא שרי.
פי׳ למימשא ביה ידיה אמר רב יוסף כוספא דיסמין שרי אמר רבא עפר פלפלין שרי אמר רב ששת ברדא שרי מאי ברדא אמר רב יוסף תילתא אוהלא ותילתא אסא ותילתא סיגלי רב נחמיה בריה דרב יוסף אמר כל היכא דליכא רובא אוהלא שפיר דמי בעו מיניה מרב ששת מהו לפצוע זיתים בשבת פי׳ למימשא }ביה ידיה אמר וכי בחול מי התירו קסבר משום הפסד אוכלין:
רבי אלעזר בן עזריה אומר קופה מטה על צדה ונוטל:
א״ר אבא א״ר חייא }אמר רב הכל מודים שאם }}נתקלקלה הגומא שאסור להחזיר ובחוששין שמא תתקלקל הגומא קא מיפלגי ר׳ אלעזר סבר חוששין ורבנן סברי אין חוששין והלכה כרבנן:
אמר רב הונא האי סליקוסתא דדצה ושלפא והדר דצה שרי ואי לא אסיר:
פירוש סליקוסתא אגודה של בשמים כגון חבצלת השרון וכיוצא בהן שנועצים אותה בטיט או בארץ לחה כדי שלא תתיבש וקסבר רב הונא ואמר שאם נעצה בערב שבת ושלפה וחזר ונעצה כדי שיהא מקומה מרווח וכשיטלנה בשבת לא תתקלקל גומתה מותר לטלטלה בשכת ואם לאו אסור ולית הלכתא כוותיה וכן הא דאמר (רב) שמואל האי סכינא דביני אורבי דצה ושלפה והדר דצה שרי ואי לא אסיר ואמר מר זוטרא ואיתימא רב אשי בגזרתא דקני שפיר דמי }ולית הלכתא כוותייהו דהא אותבינן עלייהו מההיא דתנן הטומן לפת וצנונות תחת הגפן אם היו }מקצת העלין מגולין אינו חושש לא משום כלאים ולא משום שביעית ולא משום מעשר
וניטלין בשבת וסלקא לה בתיובתא אלמא לא בעי למידץ ולמישלף אלא לגלויי מקצתן בלבד }וכן הלכתא:
ממלא אדם את הקיתון: אמר רב יהודה אמר שמואל מותר }להטמין }}את הצונן:
א״ל רב נחמן לדרו עבדיה אטמין לי את הצונן ואייתי לי מיא דאחים קפילא ארמאה שמע ר׳ אמי ואיקפד אמר רב יוסף מ״ט איקפד תרוייהו כרבוואתא עביד חדא כרב וחדא כשמואל כשמואל דאמר שמואל מותר להטמין את הצונן כרב דאמר רב שמואל בר צדוק אמר רב כל שנאכל כמות שהוא חי אין בו משום בשולי נכרים והוא סבר אדם חשוב שאני:
ת״ר אף על פי שאמרו אין טומנין את החמין משחשכה אפילו בדבר שאינו מוסיף הבל אבל אם בא להוסיף }מוסיף כיצד הוא עושה אמר רבן שמעון ב״ג נוטל (אדם) את הסדינים ומניח את הגלופקרין או נוטל את הגלופקרין ומניח את הסדינין וכן היה רשב״ג אומר לא אסרו אלא באותו מיחם אבל פינן ממיחם למיחם }מותר:
טמן וכסה בדבר הניטל בשבת או שטמן בדבר שאין ניטל בשבת וכסה בדבר הניטל בשבת ה״ז נוטל ומחזיר טמן וכסה בדבר שאין ניטל בשבת או שטמן בדבר שניטל וכסה בדבר שאינו ניטל אם מגולה }מקצתו נוטל ומחזיר ואם לאו (דף נא:) אינו נוטל ומחזיר ר׳ יהודה אומר נעורת של פשתן דקה הרי היא כזבל מניחין מיחם על גבי מיחם וקדרה ע״ג קדרה ומיחם ע״ג קדרה וקדרה ע״ג }מיחם וטח את פיהם בבצק ולא בשביל שיחמו אלא בשביל שיהו משומרין וכשם שאין טומנין את החמין כך אין טומנין את הצונן רבי התיר להטמין את הצונן ואין }}}מרסקין את }הברד ואת השלג בשבת כדי שיזובו מימיו אבל נותן הוא בתוך הכוס או בתוך הקערה ואינו חושש:
\textbf{סליקו להו במה טומנין} 
\end{multicols}\newpage

\newchap{פרק \hebrewnumeral{5} במה בהמה}
\end{multicols}\newpage

\newsection{דף כד}
\begin{multicols}{2}
\textbf{{\largeבמה}} }}}בהמה יוצאה ובמה אינה יוצאה יוצא הגמל באפסר והנאקה בחטם והלובדקים בפרומביא והסוס בשיר וכל בעלי השיר יוצאים בשיר ונמשכין בשיר ומזין עליהן וטובלין במקומן:
\textbf{{\largeגמ׳}} מאי נאקה בחטם אמר רבה בר בר חנה }נאקתא חיורתא בזממא דפרזלא:
והלובדקים בפרומביא אמר רבה בר בר חנה חמרא לובא בפגי דפרזלא:
אמר רב יהודה אמר שמואל מחליפין לפני רבי של זו בזו ושל זו בזו מהו נאקה באפסר לא תיבעי לך דכיון דלא מנטרא ביה משוי הוא לה כי תיבעי לך גמל בחטם מאי כיון דסגי ליה באפסר משוי הוא ליה או דלמא כל }נטירותא יתירתא לא אמרינן משוי הוא ליה אמר לפניו ר׳ ישמעאל בר׳ יוסי כך אמר אבא ארבע בהמות יוצאות באפסר בשבת הסוס והפרד הגמל והחמור }מאי לאו למעוטי גמל בחטם לא למעוטי נאקה באפסר }ובמתניתא תאנא הלובדקים והגמל יוצאין באפסר באפסר אין בחטם לא אלמא כל נטירותא יתירתא משוי הוא ליה ואסיר וכן הלכתא.
ואע״ג דאמר שמואל דהלכה כחנניא דאמר נטירותא יתירתא לא אמרינן משוי הוא ליה לית הלכתא כותיה דהא רב פליג עליה ואמר נטירותא יתירתא משוי הוא ליה ואותביה עליה דרב ולא קמה תיובתא אלא פריקו לה תלתא אמוראי בתראי ומדאיכפלו הני תלתא אמוראי בתראי דאינון אביי ורבא ורבינא ופריקו כולהו אליבא דרב שמע מינה דהלכתא כותיה ועוד הא קיי״ל דרב ושמואל הלכתא כרב באיסורי:
(דף נב.) הסוס בשיר וכל בעלי השיר יוצאין בשיר כו׳:
מאי יוצאין ומאי נמשכין רב }הונא אמר או יוצאין כרוכין או יוצאין נמשכין ושמואל
אמר יוצאין נמשכין ואין יוצאין כרוכין ומסקנא דשמעתא דיוצאין כרוכין ויוצאין נמשכין:
\textbf{{\largeמתני׳}} (דף נב:) חמור יוצא במרדעת בזמן שהיא קשורה לו והזכרים יוצאין לבובין והרחלות שחוזות כבולות }וכבונות והעזים יוצאות צרורות ר׳ יוסי אוסר בכולן חוץ מן הרחלות הכבונות ור׳ יהודה אומר עזים יוצאות צרורות ליבש אבל לא לחלב:
\textbf{{\largeגמ׳}} חמור יוצא במרדעת וכו׳ (דף נג.) אמר שמואל והוא שקשורה לו מע״ש:
תניא נמי הכי חמור יוצא במרדעת בזמן שקשורה לו מע״ש ולא באוכף שעליה אע״פ שהוא קשור לה מערב שבת:
בעא מיניה רב אסי בר נתן מרבי חייא בר אשי }מהו ליתן מרדעת ע״ג חמור בשבת (ולא לצאת בו לרה״ר) א״ל מותר וכן הלכתא אבל קרסטל רב שרי ושמואל אסר והלכתא כשמואל דאמרי׳ אזל ר׳ חייא בר יוסף אמרה לשמעתא דרב קמיה דשמואל א״ל אי הכי אמר אבא לא הוה ידע במילי דשבתא כלום }כי סליק ר׳ זירא אשכחיה לר׳ בנימין בר יפת דיתיב וקאמר משמיה דר׳ יוחנן נותנין מרדעת ע״ג חמור בשבת א״ל יישר וכן תרגמה אריוך בבבל ומנו שמואל הא רב נמי אמרה אלא שמעיה דקא מסיים בה אין תולין קרסטל לבהמה בשבת א״ל יישר כבר תרגמה אריוך בבבל דאי רב בקרסטל שרי ועוד הא אותיבנא תרתי תיובתא לרב ושמואל ואיפריקו אליבא דשמואל ולא איפריקו אליבא דרב:
(דף נג:) סכין }}}}ומפרכסין לאדם ואין סכין ומפרכסין לבהמה ודוקא בגמר מכה ומשום תענוג אבל בתחלת מכה ומשום צער מפרכסין נמי לבהמה:
תניא בהמה שאכלה כרשינין הרבה לא יריצנה בחצר בשביל שתתרפה ור׳ יאשיה מיקל:
דרש רבא הלכה כר׳ יאשיה והשתא דדרש רבא הלכה כר׳ יאשיה ומיקל משום צערא דבהמה ולא קא גזר משום שחיקת סממנין ש״מ לית הלכתא כי הא *}צ״ל דתניא}דתנן בהמה }שאחזה דם אין מעמידין אותה במים בשביל שתצטנן אבל אדם שאחזו דם מעמידין }אותו במים בשביל שיצטנן אלא אפילו בהמה נמי מעמידין אותה במים דהאי תנא דאסר משום שחיקת סממנין הוא דאסר ובהא איפסיקא הלכתא כר׳ יאשיה דלא גזר משום שחיקת סממנין דאין אדם בהול כל כך על רפואת בהמתו דליתי }סממנין:
תניא לא }}}יצא הסוס בזנב שועל ולא בזהורית שבין עיניו ולא יצא הזב בכיס שלו ולא העזים בכיסין שבדדיהן ולא פרה בחוסם שבפיה ולא סייחין בקרסטלים שבפיהם ולא בהמה בסנדל שברגלה ולא בקמיע שאינו מומחה לבהמה אע״פ שהוא מומחה לאדם אבל יוצאת היא באגד שעל גבי המכה ובכתיתין שעל גבי השבר ובשליא המדולדלת בה ופוקק לה זוג בצוארה ומטיילת בחצר:
הזכרים יוצאין לבובים מאי לבובים אמר רב תותרי פי׳ תותרי מטלניות מרוקמים שמייפין ומקשטין בה את הבהמה מאי משמע דהאי ליבוב לישנא דקרובי הוא דכתיב (שיר השירים ד׳:ט׳) לבבתני אחותי כלה עולא אמר עור שמקשרין להן כנגד לביהן שלא יפלו עליהם זאבים רב נחמן בר יצחק אמר עור שמקשרין להן תחת זכרותן כדי שלא יעלו על הנקבות ממאי מדקתני סיפא והרחלים יוצאות שחוזות מאי שחוזות שאוחזות אליה שלהן למעלה כדי שיעלו עליהן הזכרים מאי משמע
\end{multicols}\newpage

\newsection{דף כה}
\begin{multicols}{2}
דהאי שחוזות לישנא דגלויי הוא דכתיב (משלי ז) שית זונה:
(דף נד.) כבולות מאי כבולות שמכבלין אליה שלהן למטה שלא יעלו עליהן הזכרים:
כבונות מאי כבונות שמכבנין אותה למילת כדתנן *}נגעים רפ״א}שאת כצמר לכן מאי כצמר לבן אמר רב ביבי אמר רב אשי כצמר נקי רך בן יומו שמכבנין אותו למילת:
והעזים יוצאות צרורות וכו׳: אתמר רב אמר הלכה כרבי יהודה ושמואל אמר הלכה כרבי יוסי ואיכא דמתני להא שמעתא באנפי נפשה רב אמר ליבש מותר לחלב אסור ושמואל אמר אחד זה ואחד זה אסור.
ואיכא דמתני לה אהא עזים יוצאות צרורות ליבש אבל לא לחלב משום רבי יהודה בן בתירה אמרו כך הלכה }*}גי׳ ד״ת אבל}אלא מי מפיס איזו ליבש ואיזו לחלב ומתוך שאין מכירין אחד זה ואחד זה אסור אמר שמואל ואיתימא אמר רב יהודה אמר שמואל הלכה כרבי יהודה בן בתירה ורב אמר הלכה כתנא קמא והלכתא כרב דאמר הלכה כרבי יהודה דמתניתין דאמר עזים יוצאות }צרורות ליבש אבל לא לחלב דקיימא לן רב ושמואל הלכה כרב באיסורי:
\textbf{{\largeמתני׳}} ובמה אינה יוצאה לא יצא גמל }במטולטלת לא עקוד ולא רגול.
וכן שאר כל בהמה. לא יקשור גמלים זה בזה וימשוך אבל מכניס }חבלים לתוך ידו ובלבד שלא יכרוך:
\textbf{{\largeגמ׳}} תאנא לא יצא גמל במטולטלת }הקשורה לו בזנבו אבל יוצא במטולטלת הקשורה לו בזנבו ובחטוטרתו אמר רבה בר רב הונא יוצא גמל במטולטלת הקשורה לו בשלייתה:
לא עקוד ולא רגול: אמר רב יהודה עקוד עקידת יד ורגל כיצחק בן אברהם רגול שלא יכוף ידו ע״ג זרועו ויקשור:
ולא יקשור גמלים זה בזה: מ״ט אמר רב ששת משום דמיחזי כמאן דאזיל לחינגא:
אכל מכניס הוא חבלים לתוך ידו }: אמר רב אשי עד כאן לא שנו אלא לענין כלאים כלאים דמאן אלימא כלאים דאדם והתנן *}ספ״ח דכלאים}אדם מותר עם כולן למשוך ולחרוש ולהנהיג אלא כלאים דחבלים והתנן *}שם פ״ט מ״י}התוכף תכיפה אחת אינה חיבור (אלא) לעולם }כלאים דחבלים והכי קאמר ובלבד שלא יכרוך ויקשור אמר שמואל ובלבד שלא יוציא חבל מתחת ידו טפח והא תאנא דבי שמואל טפחיים אמר אביי }השתא דאמר שמואל טפח ותאנא דבי שמואל טפחיים שמואל הלכה למעשה אתא לאשמועינן טפח והתניא (דף נד:) ובלבד שיגביה אפסר מן הקרקע טפח כי תניא ההיא בחבלא דביני ביני:
\textbf{{\largeמתני׳}} אין חמור יוצא במרדעת בזמן שאינה קשורה לו ולא בזוג אע״פ שהוא פקוק ולא בסולם שבצוארו ולא ברצועות שברגליו ואין התרנגולין יוצאין בחוטין ולא ברצועות שברגליהן ואין הזכרים יוצאין בעגלה שתחת האליה שלהן ואין הרחלים יוצאות חנונות ואין העגל יוצא בגימון ולא פרה בעור הקופר ולא ברצועה שבין קרניה פרתו של ראב״ע היתה יוצאת ברצועה שבין קרניה שלא ברצון חכמים:
\textbf{{\largeגמ׳}} אין חמור יוצא במרדעת מ״ט כדאמרן.
פי׳ כדאמרן לעיל מאי אינה קשורה לו אילימא אינה קשורה לו כלל פשיטא ליחוש דילמא נפלה ואתי לאיתויי אלא לאו שאינה קשורה לו מע״ש:
ולא
בזוג אע״פ שהוא }פקוק משום דמיחזי כמאן דאזיל לחינגא:
ולא בסולם שבצוארו אמר רב הונא בי לועא למאי עבדי לה להיכא דאית לה מכה דלא הדר חייך לה:
ולא ברצועה }דעבדי ליה *}צ״ל לגיזרא}גיזרא:
גיזרא חזינא ביה תרי פירושי איכא מ״ד שפסיעותיהן דחוקות וקצרות ונוגעות עקביהן ורגליהן זו בזו והן נזוקין ועושין להן מטלית או רצועות וקושרים על עקביהן כדי שלא יזוקו ויש מפרשים כשנבקעה פרסה של רגל הבהמה קושרין אותה ברצועה כדי שתחלים ותחזור לכמות שהיתה:
ואין התרנגולין יוצאין בחוטין. דעבדי להו סימנין כי היכי דלא ליחלפן:
ולא ברצועות שברגליהן דעבדי להו כי היכי דלא ליתברו מאני:
ואין הזכרים יוצאין בעגלה שתחת האליה שלהם כי היכי דלא ליחמטן אליותיהן:
ואין הרחלים יוצאות חנונות מאי חנונות אמר רב הונא עץ אחד יש בכרכי הים ויחנון שמו ומביאים קיסם ממנו ומניחים לה בחוטמה והיא מתעטשת ונופלין הדרנין שבראשה א״ה זכרים נמי }כיון דמנגחי אהדדי ממילא נפלי שמעון נזירא אמר קיסמא דריתמא:
ואין העגל יוצא בגימון מאי גימון אמר רב הונא [בר] נירא אמר ר״א מאי משמע דהאי גימון לישנא דמיכף הוא דכתיב הלכוף כאגמון ראשו:
ולא פרה בעור הקופר דעבדי לה דלא לימציוה יאלי:
ולא ברצועה שבין קרניה. איתמר רב חייא בר אשי אמר רב בין לנוי בין לשמור אסור ורב חייא בר אבין אמר לנוי אסור לשמור מותר והלכתא כרב כדכתבינן לעיל דכל נטירותא יתירתא משאוי הוא:
פרתו של ר״א בן עזריה: תאנא לא שלו היתה אלא של שכינתו היתה ומתוך שלא מיחה בה נקראת על שמו איתמר רב ורבי חנינא ורב חביבא ור׳ יוחנן מתנו בכולי סדר מועד כל כי האי זוגא חלופי רבי יוחנן ועייל ר׳ יונתן }}כל שאפשר למחות באנשי ביתו ואינו מוחה הוא נתפש על אנשי ביתו באנשי עירו ואינו מוחה הוא נתפש על אנשי עירו על כל העולם כולו ואינו מוחה הוא נתפש על כל העולם כולו אמר רב פפא הילכך הני דבי ריש גלותא מיתפשי אכולי עלמא:
\textbf{סליקו להו במה בהמה} 
(דף נז.) \textbf{{\largeבמה}} אשה }}יוצאה ובמה אינה יוצאה לא תצא אשה לא בחוטי צמר ולא בחוטי פשתן ולא ברצועות שבראשה ולא תטבול בהן עד שתרפם (וכיון דבחול לא תטבול בהן עד שתרפם בשבת לא תצא בהן דלמא מיתרמיא לה טבילה דמצוה ושרייא להו ואתיא לאיתויינהו ארבע אמות ברשות הרבים):
ולא בטוטפת ולא בסרביטין בזמן שאינן תפורים ולא בכבול }לרה״ר ולא בעיר של זהב ולא בקטלא ולא בנזמים ולא בטבעת שאין עליה חותם ולא במחט שאינה נקובה ואם יצאה אינה חייבת חטאת:
\end{multicols}\newpage

\newchap{פרק \hebrewnumeral{6} במה אשה}
\end{multicols}\newpage

\newsection{דף כו}
\begin{multicols}{2}
\textbf{{\largeגמ׳}} בעי רב כהנא מרב הני תיכי חלילתא מאי א״ל אריג קאמרת כל שהוא אריג לא גזרו }ואיכא דאמרי אמר רב הונא בריה דרב יהושע חזינא }לאחוותאי דלא קפדן עלייהו מאי איכא בין האי לישנא להאי לישנא איכא בינייהו דטניפן להך לישנא דאמרת כל שהוא אריג לא גזרו הנך נמי אריג נינהו להך לישנא דאמרת חזינא להו לאחוותאי דלא קפדי עלייהו }כיון דטניפן מקפד קפדי עלייהו תנן התם }ואלו חוצצים באדם חוטי צמר וחוטי פשתן והרצועות שבראשי הבנות ר׳ יהודה אומר של צמר ושל שער אין חוצצין מפני שהמים באין בהן:
אמר רב הונא כולן בראשי הבנות שנינו אבל חוטין שבצואריהן אינן חוצצין לפי שאין האשה חונקת את עצמה והני מילי בחוטין אבל בחבקין שבצואריהן כגון (דף נז:) קטלא וכיוצא בהן חוצצין דאשה חונקת את עצמה כדי שתראה בעלת בשר:
ר׳ יהודה אומר של צמר ושל }שער וכו׳:
אמר רב יוסף אמר רב יהודה אמר שמואל מודים חכמים לרבי יהודה בחוטי שער שאין חוצצין הלכך יוצאה בהם בשבת }ולא אתיא לאתויינהו ד׳ אמות בר״ה:
ולא בטוטפת: מאי טוטפת אמר רב יוסף חומרתא דקטיפתא.
פירוש חומרתא דקטיפתא חלי ידועה שתולין אותה בצואר מפני עין הרע א״ל אביי ותהוי כקמיע מומחה ותשתרי אלא אמר רב יהודה משמיה דאביי אפזייני.
פי׳ אפזייני ציץ: תנ״ה יוצאת אשה בסבכה המוזהבת ובטוטפת וסרביטין הקבועין בה ואיזו היא טוטפת ואיזו היא סרביטין א״ר אבהו טוטפת המוקפת לה בראשה מאזן לאזן סרביטין המגיעין לה עד לחיים אמר רב הונא }תאנא עניות עושין אותן ממיני צבעונין עשירות עושין אותן של כסף וזהב:
ולא בכבול לרה״ר: א״ר ינאי כבול זה איני יודע מהו אי כבלא דעבדא תנן אבל כיפה של צמר שפיר דמי או דלמא כיפה של צמר תנן }וכ״ש כבלא דעבדא }רב אמר כיפה של צמר }ושמואל אמר כבלא דעבדא }אר״ד אבהו מסתברא כמאן דאמר כיפה של צמר }:
תנ״ה יוצאת אשה בכבול ובאצטמא לחצר ר״ש אומר אפילו בכבול לרה״ר כלל אמר רבי שמעון כל שהוא למטה מן השבכה יוצאין בו וכל שהוא למעלה מן השבכה אין יוצאין בו }}}*}ד״ת מוחק זה}(לחצר אין לרה״ר לא) ושמואל אמר כבלא דעבדא תנן אבל כיפה של צמר שפיר דמי והלכתא
כרבי אבהו דתניא כוותיה.
ופירוש כיפה של צמר חוטי דעמרא דגדילין ועבידן כי }הוצא ורחב שתי אצבעות שיעור ציץ כדאמרינן (חולין קלח.) כיפה של צמר היה מונח בראש כהן גדול ועליו ציץ נתון שנאמר (שמות כ״ח:ל״ז) ושמת אותו על פתיל תכלת:
מאי אצטמא אמר רבי אבהו בי זייני מאי בי זייני כליא פרוחי.
פירוש מטלית שתולין בה מיני צבעונין כגון גלופקרא שתולין אותה לכלה לכלות ממנה זבוב שאם יעמוד זבוב על פניה מתביישת לטרדו ומצטערת בו כגון זה }שתולין לבהמה ולפיכך אין בה משום כלאים דלאו אריג היא ואינה מטמאה בנגעים דלאו שתי וערב היא ואין יוצאין בה לרה״ר שאינה תכשיט:
תנו רבנן שלשה דברים נאמרו באצטמא אין בה }}משום כלאים ואין מטמאין }(בה) בנגעים שאין בה שתי וערב ואין יוצאין בה לרשות הרבים [שאינה] תכשיט משום רבי אליעזר בר ר״ש אמרו אף (דף נח.) אין בה משום עטרות כלות:
אמר שמואל יוצא }}העבד בחותם שבצוארו אבל לא בחותם שבכסותו תנ״ה יוצא העבד בחותם שבצוארו אבל לא בחותם שבכסותו ורמינהי לא יצא העבד לא בחותם שבצוארו ולא בחותם שבכסותו לא קשיא הא דתניא לא יצא בשל מתכת והא דתניא יצא בשל טיט וכדרב נחמן דאמר רב נחמן אמר רבה בר אבוה דבר המקפיד רבו עליו אין יוצאין בו דבר שאין מקפיד רבו עליו יוצאין בו:
תאנא לא תצא בהמה לא בחותם שבצוארה ולא בחותם שבכסותה ולא בזוג שבצוארה ולא בזוג שבכסותה:
ולא }}}בעיר של זהב: (דף נט.) מאי עיר של }זהב אמר רבה בר בר חנה ירושלים דדהבא כדעבד לה רבי עקיבא לדביתהו:
כלילא }(דף נט:) רב אסר ושמואל שרי דאניסכא כולי עלמא לא פליגי דאסיר בי פליגי דארוקתא }מר סבר ניסכא עיקר ומר סבר רוקתא עיקר רב אשי מתני לקולא דארוקתא כ״ע לא פליגי דשרי כי פליגי דאניסכא *}גי׳ ד״ת מר סבר}מאן דאסר דלמא משלפא ומחויא ואתיא לאיתויי ומר סבר מאן דרכה למיפק בכלילא אשה חשובה ואשה חשובה לא שלפא ומחויא:
\end{multicols}\newpage

\newsection{דף כז}
\begin{multicols}{2}
א״ל רב שמואל בר בר חנה לרב יוסף בפירוש אמרת לן משמיה דרב כלילא שרי:
אמרו ליה לרב גברא רבה אריכא ומטלע }ומנו לוי הוה כדאמרינן לוי אחוי קידה ואטלע אתא לנהרדעא ודרש כלילא שרי וכן הלכתא:
פירוש }ניסכא חוט }[של כסף או של זהב] כלומר שהן חתיכות }חתיכות נקובות על המטלית ומכניסין לתוכן חוט להעמידן.
פירוש רוקתא כגון מטלית שאותן חתיכות קבועות על המטלית:
אמר רב יהודה אמר רב ששת קמרא שרי והוא חגור שבמתנים ויש בו חתיכות קבועות כגון כלילא איכא דאמרי דארוקתא }דאמר רב ספרא מידי דהוה אטלית מוזהבת ואיכא דאמרי דאניסכא }}דאמר רב ספרא מידי דהוה אאבנט של מלכים }וכל ישראל בני מלכים הם וכיון דאניסכא שרי ללישנא בתרא כ״ש דארוקתא והלכתא כלישנא בתרא }:
אמר רב אשי האי ריסוקא אי אית ליה מפרחייתא שרי ואי לא אסיר:
ופירוש ריסוקא }חגורה של עור ונקראת בלשון ישמעאל מגנוקא אי אית ליה מפרחייתא שהן כגון שרביטין }הרי הוא תכשיט ומותר לצאת בו ואי לא נעשה כמשוי ואסור:
ולא בקטלא ולא בנזמים: מאי קטלא מנקטא פארי ושמה בלשון ערב מכנקא.
ומאי נזמים נזמי האף: ולא בטבעת שאין עליה חותם הא יש עליה חותם חייבת חטאת וכן הלכה:
(דף ס.) ולא במחט שאינה נקובה: למאי חזיא תרגמא רב אחא נרשאה קמיה דרב יוסף הואיל ואשה חולקת בה שערה ובשבת למאי חזיא תניא כמין טס של זהב יש לה בראשה }בחול חולקת בה שערה ובשבת מנחת על פדחתה:
\textbf{{\largeמתני׳}} לא }}יצא האיש בסנדל מסומר ולא ביחיד בזמן שאין ברגלו מכה ולא בתפלין ולא בקמיע בזמן שאינו *}גי׳ הגמ׳ ובס״י מן המומחה}מומחה ולא בשריון ולא בקסדא ולא במגפיים ואם יצא אינו חייב חטאת:
\textbf{{\largeגמ׳}} סנדל מסומר מ״ט לא א״ר אבא אמר שמואל שלופי השמד היו והיו
נחבאין במערה ואמרו הנכנס יכנס והיוצא אל יצא נהפך סנדלו של אחד מהן כסבורין הם אחר מהן יצא והכירו בהם אויבים ועכשיו באין עליהם דחקו זה את זה והרגו זה את זה יותר ממה שהיו הורגים בהם האויבים אמר רבי אלעזר בן אלישע במערה היו יושבין ושמעו קול מעל גבי המערה דחקו זה את זה והרגו זה את זה יותר ממה שהיו הורגים בהם האויבים:
רמי בר יחזקאל אמר בבהכ״נ היו יושבין ושמעו קול מאחורי בהכ״נ כסבורין הם }אחד מהן יצא והכירו בהם אויבים וכו׳ באותה שעה אמרו לא יצא איש בסנדל המסומר ל״ש בשבת ול״ש ביו״ט (דף ס:) בשבת מ״ט לא דאיכא כנופיא דאיסור מלאכה ביו״ט נמי איכא כנופיא דאיסור מלאכה והיינו דתנן ביו״ט (דף יד:) אין משלחין סנדל מסומר ולא מנעל שאינו תפור:
אמר רב יהודה אמר שמואל לא שנו אלא לחזק אבל לנוי מותר וכמה לנוי ר׳ יוחנן אמר חמש בזה וחמש בזה ור׳ חנינא אמר שבע בזה ושבע בזה א״ל ר׳ יוחנן לרב שמן בר אבא אסברא [לך] לדידי שתים מכאן ושתים מכאן ואחת בתרסיותיו לר׳ חנינא שלש מכאן ושלש מכאן ואחת בתרסיותיו:
א״ל }אילפא לרבה בר בר חנה אתון דתלמידי דר׳ יוחנן אתון עבידו כר׳ יוחנן אנן נעביד כר׳ חנינא בעא מיניה ההוא רצענא מר׳ אמי תפרו מבפנים מהו א״ל מותר ולא ידענא מ״ט אמר רב אסי ולא ידע מר מ״ט כיון דתפרו מבפנים הוה ליה מנעל ובסנדל גזרו ביה רבנן במנעל לא גזרו ביה רבנן:
בעא מיניה ר׳ אבא בר זבדא מר׳ אבא בר אבינא עשאו כמין קלבוס מהו א״ל מותר איתמר נמי א״ר יוסי בר׳ חנינא עשאו כמין קלבוס מותר:
אמר רב ששת חיפהו כולו במסמרים כדי שלא תהא קרקע אוכלתו מותר תניא כותיה דרב ששת לא יצא אדם בסנדל המסומר ולא יטייל בו מבית לבית ואפילו ממטה למטה אבל מטלטלים אותו לכסות בו את הכלי ולסמוך בו כרעי המטה ור׳ אלעזר בר״ש אוסר נשרו רוב מסמרותיו ונשתיירו בו ארבעה או חמשה מותר ור׳ יהודה מתיר עד שבעה חיפהו בעור מלמטה וקבע בו מסמרים מלמעלה מותר עשאו כמין טס או כמין יתד או חיפהו כולו במסמרים כדי שלא תהא קרקע אוכלתו מותר:
ולא ביחיד בזמן שאין ברגלו מכה: טעמא דאין ברגלו מכה (דף סא.) הא יש ברגלו מכה נפיק באותה שאין בה מכה כחייא בר רב וכר׳ יוחנן דאמר }עשית של שמאל מכה:
ולא בתפלין ולא בקמיע: אמר רב פפא לא תימא עד דאיתמחי גברא ואיתמחי }קמיע אלא כיון דאיתמחי גברא אע״ג דלא אתמחי קמיעא דיקא נמי דקתני ולא בקמיע בזמן שאינו מן המומחה ולא קתני בזמן שאינו מומחה שמע מינה ת״ר איזהו קמיע מומחה כל שריפא ושינה
\end{multicols}\newpage

\newsection{דף כח}
\begin{multicols}{2}
ושילש אחד קמיע של כתב ואחד קמיע של עיקרין אחד חולה שיש בו סכנה ואחד חולה שאין בו סכנה ולא שנכפה אלא שלא יכפה ויקשור ויתיר אפילו ברה״ר ובלבד שלא יקשרנו (דף סא:) בשיר ובטבעת ויצא בו לרשות הרבים משום מראית העין והתניא איזהו קמיע מומחה כל שריפא שלשה בני אדם כאחד לא קשיא הא לאתמחויי גברא והא לאתמחויי קמיעא:
אמר רב פפא פשיטא לי תלתא קמיעי לתלתא גברי תלתא תלתא זמני אתמחי גברא ואתמחי קמיעא תלתא קמיעי לתלתא גברי חד חד זימנא גברי אתמחי קמיעא לא אתמחי חד קמיעא לתלתא גברי קמיעא אתמחי גברא לא אתמחי בעי רב פפא תלתא קמיעי לחד גברא מאי קמיעא ודאי לא אתמחי גברא אתמחי או לא אתמחי מי אמרינן האי כיון דאיתסי תלתא זימני ודאי מומחה הוא או דלמא מזליה דההוא גברא דמקבל קמיעא מאי תיקו
}}איבעיא להו קמיעין יש בהן משום קדושה או דילמא אין בהן משום קדושה למאי הלכתא }לאצולינהו מן הדליקה ת״ש הברכות והקמיעין אע״פ שיש בהן אותיות של שם ומעניינות הרבה של תורה אין מצילין אותן מפני הדליקה ונשרפין במקומן ואלא לענין }גניזה מאי ת״ש היה שם כתוב על ידות הכלים ועל כרעי המטה הרי זה יגוד ויגנוז אלא להכנס בהן }}}לבית הכסא מאי יש בהן משום קדושה ואסור או }דילמא אין בהן משום קדושה ושרי ופשטינן (דף סב.) אם היו מחופין }עור שרי למיעל בהו לבית הכסא ואי לא אסיר ותפילין היינו טעמא דאין נכנס בהן לבית הכסא אלא חולצן ברחוק ד׳ אמות ואוחזן בבגדו ובידו ונכנס משום שי״ן שלהן דאמר אביי שי״ן של תפלין הלכה למשה מסיני:
ולא בשריון ולא }}}בקסדא ולא במגפים: שריון זרדא קסדא אמר רב סנורתא ונקראת בלשון ישמעאל ביצה והוא כובע שבמקרא דכתיב (שמואל א י״ז:ה׳) וכובע נחשת על ראשו מגפים אמר רבה בר רב הונא פזמקי פירוש מוקי דעבדי להו מברזלא או מנחשא ולית להו גיותא דכרעא:
\textbf{{\largeמתני׳}} לא תצא }}}אשה }במחט נקובה ולא בטבעת שיש עליה חותם ולא }בכולייר ולא בכובלת ולא בצלוחית של פלייטון ואם יצאתה חייבת חטאת דברי ר׳ מאיר וחכמים פוטרין בכובלת ובצלוחית של פלייטון:
\textbf{{\largeגמ׳}} אמר עולא וחילופיהן }באיש כלומר אם יצא
האיש במחט שאינה נקובה ובטבעת שאין עליה חותם חייב חטאת ואם יצא בטבעת שיש עליה חותם ובמחט נקובה פטור אבל אסור והכין הוא מסקנא דשמעתא ולאו הוצאה כלאחר יד הוא דכיון שפעמים שאדם נותן לאשתו בחול טבעת שיש עליה חותם להוליכה לקופסא ומניחתה בידה }עד שמגעת לקופסא נמצאת דרך הוצאתה בכך ואינה כלאחר יד ופעמים נמי שהאשה נותנת לבעלה בחול טבעת שאין עליה חותם להוליכה אצל האומן ומניחה באצבעו עד שמוליכה לאומן ונמצא דרך הוצאתו בכך בחול וכיון שדרך הוצאתו בחול }באצבעו בכך אם יצא בה בשבת חייב חטאת:
ולא בכולייר: מאי כולייר אמר רב }מכבנתא יש שפירשו אותה טליסאן ויש שפירשו אותה מידי דמקיף לרישיה כדאמרינן ביומא (דף כה.) כהנים מוקפין ועומדין כמין כולייר:
כובלת אמר רב יוסף חמרתא דפליון פירוש חוליא של בושם שמתקשטת בו להעביר ריח רע ממנה ועיקר לשון פליון מלשון (במדבר כח) בלולה בשמן דמתרגמינן דפילא במשח:
(דף סב:) אמר רבי *}אבהו כ״ה בגמ׳}אמי ואמרי לה במתניתא תאנא ג׳ דברים מביאין }את האדם לידי עניות ואלו הן המשתין מים בפני מטתו ערום ומי שאשתו מקללתו בפניו והמזלזל בנטילת ידים המשתין מים בפני מטתו ערום לא אמרן אלא דמהדר אפיה לפוריא אבל לבראי לית לן בה וכי מהדר אפיה לפוריא נמי לא אמרן אלא אארעא אבל במנא לית לן בה והמזלזל בנטילת ידים כדרבי זריקא דאמר רבי זריקא אמר ר׳ אלעזר בל המזלזל בנטילת ידים נעקר מן העולם אמר רבא לא אמרן אלא דלא משא כלל אבל משא ולא משא לית לן בה ולאו מילתא היא מדרב חסדא דאמר רב חסדא אנא משאי מלי חפנאי מיא ויהבו לי מלי חפנאי טיבותא.
ומי שאשתו מקללתו בפניו אמר רב על עסקי תכשיטיה והוא דאית ליה ולא עבד לה:
\textbf{{\largeמתני}} ׳ (דף סג.) לא }}}יצא האיש לא בסייף ולא בקשת ולא בתריס ולא באלה ולא ברומח ואם יצא חייב חטאת ר׳ אליעזר אומר תכשיטין הן לו וחכמים אומרים אינן לו אלא גנאי שנאמר (ישעיהו ב׳:ד׳) וכתתו חרבותם לאתים וגו׳ והלכה כרבנן בירית טהורה ויוצאין בה בשבת כבלים טמאים ואין יוצאין בהן:
\textbf{{\largeגמ׳}} (דף סג:) יתיב רבי אבין ורב הונא קמיה דרבי ירמיה ויתיב רבי ירמיה וקא מנמנם ויתיב רבי אבין וקאמר בירית באחת כבלים בשתים אמר לו רב הונא אלו ואלו בשתים ומטילין שלשלת ביניהם ונעשו כבלים אתער בהו רבי ירמיה ואמר ישר וכן אמר רבי יוחנן (דף סד:) אמר רב ששת למה }}}}מנה הכתוב תכשיטין שבחוץ עם תכשיטים שבפנים טבעת וכומז לומר לך כל המסתכל אפילו באצבע קטנה של אשה כאילו מסתכל במקום התורף:
\textbf{{\largeמתני׳}} יוצאת אשה בחוטי }שער בין משלה בין משל חברתה בין }}}משל בהמה בטוטפת ובשרביטין בזמן שהן תפורין בכבול
\end{multicols}\newpage

\newsection{דף כט}
\begin{multicols}{2}
ובפאה נכרית בחצר במוך שבאזנה ובמוך שבסנדלה ובמוך שהתקינה לנדתה בפלפל ובגרגר מלח ובכל דבר שתתן לתוך פיה ובלבד שלא תתן לכתחלה בשבת ואם נפל לא תחזיר שן תותבת ושן של זהב רבי מתיר וחכמים אוסרים:
\textbf{{\largeגמ׳}} וצריכא דאי אשמעינן דידה משום דלא מאיס לה אבל דחברתה }דמאיס לה אימא לא צריכא ואי אשמעינן דחברתה משום דבת מינה היא אבל דבהמה דמינכר אימא לא צריכא:
תאנא ובלבד שלא תצא ילדה בשל זקנה וזקנה בשל ילדה בשלמא זקנה בשל ילדה שבח הוא לה אבל ילדה בשל זקנה גנאי הוא לה כדי נסבה:
ובכבול ובפאה נכרית בחצר: אמר רב כל שאסרו חכמים לצאת בו לרה״ר אסור לצאת בו לחצר }חוץ מכבול ופאה נכרית ור׳ ענני בר ששון משמיה דר״י סבר הכל ככבול והלכתא כרב }דסתם לן תנא כותיה.
ולרב מאי שנא הני אמר עולא גזירה שמא }}}תתגנה על בעלה ונמצא בעלה מגרשה כדתניא (ויקרא טו) והדוה בנדתה זקנים הראשונים אמרו שלא תכחול ושלא תפרכס ותתקשט בבגדי צבעונים עד שבא רבי עקיבא ולימד א״כ נמצאת מגונה על בעלה ונמצא בעלה מגרשה אלא מה תלמוד לומר והדוה בנדתה תהא בנידותה עד שתבא במים:
(דף סה.) }}}ובמוך שבאזנה: תני רמי בר יחזקאל והוא שקשור לה באזנה:
ובמוך שבסנדלה תני רמי בר יחזקאל והוא שקשור לה בעקיבה }קשר מהודק.
ובמוך שהתקינה לנדתה סבר רבי אמי בר אחא למימר והוא שקשור לה בין יריכותיה אמר ליה רבא אע״פ שאינו קשור לה דכיון דמאיס לה לא אתי לאתויי:
בעא מיניה ר׳ ירמיה בר אבא עשתה לו בית יד מהו א״ל }אע״פ שעשתה לו בית יד מותר איתמר נמי אמר רב נתן בר אושעיא א״ר יוחנן עשתה לו בית יד מותר:
בפלפל ובגרגיר מלח. פלפל }לריח הפה גרגיר מלח דעביד }לדורשני ובכל דבר שתתן לתוך פיה זנגביל אי נמי דארציני:
שן }תותבת ושן של זהב ר׳ מתיר וחכמים אוסרין אמר אביי רבי ור׳ אלעזר ור״ש בן אלעזר כולהו סבירא להו דכל מידי דמיגניא בה לא אתי לאחויי רבי הא ראמרן ר״א דתניא }ר״א פוטר בכובלת ובצלוחית של פלייטון רשב״א דתניא ר״ש בן אלעזר אומר כל שהוא למטה מן הסבכה יוצאה בה למעלה מן הסבכה אין יוצאה בה והללו כולן שטה אחת הן ואין הלכה כאחד מהן:
\textbf{{\largeמתני׳}} יוצאין בסלע שעל הצינית הבנות קטנות יוצאות בחוטין ואפילו בקסמים שבאזניהם ערביות יוצאות רעולות ומדיות פרופות וכל אדם אלא שדברו חכמים בהווה:
\textbf{{\largeגמ׳}} מאי צינית בת ארעא פי׳ בלשון ישמעאל פצה אל״א ר״ץ:
הבנות יוצאות בחוטים. אבוה דשמואל לא }}שביק להו לבנתיה דליהוון נפקן בחוטי ולא שביק להו למיגנא בהדי הדדי ועביד להו }}מקוה ביומי }ניסן ומפצי ביומי תשרי }}ולא שביק להו למיפק בחוטי והתנן הבנות יוצאות בחוטין דאבוה דשמואל צבועין הוו:
ולא שביק להו למיגנא בהדי הדדי (דף סה:) דלא לילפן להו גופא נוכראה:
ערביות רעולות ומדיות }}פרופות וכל אדם אלא שדברו חכמים בהווה.
פירוש ערביות רעולות נשים ערביות יוצאות רעולות כדכתיב (ישעיה ה) הנטיפות השרות והרעלות ושמן בלשון ערבי גרץ ונשים מדיות יוצאות
\end{multicols}\newpage

\newsection{דף ל}
\begin{multicols}{2}
פרופות בבגדיהן:
(דף סה.) \textbf{{\largeמתני׳}} פורפת אשה על האבן ועל האגוז ועל המטבע ובלבד שלא תפרוף לכתחלה בשבת:
\textbf{{\largeגמ׳}} (דף סה:) והא אמרת רישא פורפת אמר אביי סיפא אתאן למטבע כלומר }על המטבע בלבד הוא דלא תפרוף לכתחלה בשבת אבל על האבן ועל האגוז פורפת לכתחלה:
בעי אביי אשה מהו שתערים ותפרוף על האגוז ותוציא לבנה קטן בשבת ועלתה בתיקו וקיימא לן דכל תיקו דאורייתא לחומרא ודרבנן }לקולא:
\textbf{{\largeמתני׳}} }}}הקיטע יוצא בקב שלו דברי רבי מאיר (דף סו.) ור׳ יוסי אוסר }אם יש לו בית קבול כתיתין טמא סמוכות שלו טמאין מדרס ויוצאין בהן בשבת ונכנסין בהן בעזרה כסא וסמוכות שלו טמאין מדרס ואין יוצאין בהן בשבת ואין נכנסין בהן בעזרה לוקטמין טהורים ואין יוצאין בהן בשבת:
\textbf{{\largeגמ׳}} הקיטע יוצא בקב שלו והלכה כר׳ יוסי (דף סו:) מאי לוקטמין א״ר אבהו חמרא דאוכפא רפרם בר פפא אמר קישרי רבה בר רב הונא משמיה דעולא
אמר פראמי.
פירוש פראמי דבר שנותנין על הפה שלא ינתז ממנו הרוק:
גרסינן בפרק הנודר מן הירק (דף נה:) תניא יוצאים בשק }עבה }ובסגוס עבה וביריעה ובחמילה מפני הגשמים הרועים יוצאין בשקים ולא הרועים בלבד אמרו אלא כל אדם אלא שדברו חכמים בהווה.
ותניא בתוספתא אבל לא בתיבה ולא בקופה ולא במחצלת מפני הגשמים.
ועוד תניא בתוספתא יוצאין במוך ובספוג שעל גבי המכה ובלבד שלא יכרוך עליהן חוט או משיחה.
יוצאין בקליפת השום ובקליפת הבצל שעל גבי המכה ואם נפל לא יחזיר ואין צריך לומר שלא יתן לכתחלה בשבת.
יוצאין באגד שע״ג המכה וקושרו ומתירו בשבת אספלנית שפירשה מן האגד מחזירה עם האגד.
יוצאין באספלנית ובמלוגמא וברטיה שעל גבי המכה ואם נפל לא יחזיר וא״צ לומר שלא יתן לכתחלה בשבת והאי דקתני אם נפל לא יחזיר הוינן בה בגמרא דעירובין (דף קב.) הלכה מחזירין רטיה בשבת במקדש ותמן ברירנא לה:
\textbf{{\largeמתני׳}} הבנים יוצאין בקשרים }ובני מלכים בזוגין וכל אדם אלא שדברו חכמים בהווה:
\textbf{{\largeגמ׳}} מאי קשרים אמר אבא מרי אמר רב נחמיה בר ברוך אמר רב אידי בר אבין אמר רב יהודה }קשרי פאה מאי איריא בנים אפילו בנות נמי מאי איריא קטנים אפילו גדולים נמי אלא מאי קשרים כי הא דא״ר אבין בר חיננא אמר רב חמא בר גוריא בן שיש לו געגועין על אביו נוטל רצועה ממנעל של ימין }ונותן וקושר לו בשמאלו:
אמר ר׳ אבין בר רב הונא אמר רב חמא בר גוריא סחופי כסא אטיבורא בשבת שפיר דמי.
ואמר ר׳ אבין בר רב הונא אמר רב חמא בר גוריא מותר לסוך שמן ומלח בשבת כי הא דרב הונא כי הוה אתי מבי רב ורב מבי רבי חייא ור׳ חייא מבי רבי כי הוו מיבסמי הוו מייתו מילחא ומישחא ושייפא להו לגויתא דידייהו }ולכרעייהו ואמרי כי היכי דצייל האי מישחא נצייל חמריה דפלניא בר פלניתא:
ואמר רבי אבין בר רב הונא אמר רב חמא בר גוריא מותר ליחנק בשבת ואמר רבי אבין בר רב הונא אמר רב חמא בר גוריא לפופי ינוקא בשבתא שפיר דמי:
תנו רבנן יוצאין באבן תקומה בשבת משום ר׳ מאיר אמרו אף במשקל אבן תקומה ולא שהפילה אלא שמא תפיל ולא שעיברה אלא שמא תתעברה ותפיל אמר רב יימר בר שלמיה משמיה דרב והוא דאיכוין ותקיל *}מהר״ם מ״ז}לרפואה:
}(ואמר שמואל שמעתי שמציל) אבן מכוונת לא שפחתה ולא הותירה ממשקל אבן תקומה כשנשקלה נגדה:
והלכה כר׳ מאיר מדקא מתרצינן אליביה והוא דאיכוין ותקיל לרפואה ועוד דאמרינן }אמר רב אשי משקל דמשקל מאי מדקא מבעיא ליה משקל דמשקל מכלל דכר׳ מאיר סבירא ליה:
(דף סז.) ובני מלכים בזוגין אוקמה *}בק״ד אבא}רבה באריג שבכסותו לדברי הכל:
\textbf{{\largeמתני׳}} יוצאין }בביצת החרגול ושן של שועל ומסמר הצלוב משום רפואה דברי ר״מ וחכמים אומרים אף בחול אסור משום דרכי האמורי:
\textbf{{\largeגמ׳}} יוצאין בביצת החרגול דעבדי לשיחלא.
פי׳ רפיון הירך כדתנן (בכורות דף מ.) השחול והכסול אי זהו שחול כל שנשמטה ירכו:
ובשן של שועל דעבדי ליה לשינתא דחייא למאן דנאים דמיתא למאן דלא נאים:
ובמסמר הצלוב דעברי לזירפא. פי׳ זירפא כמו }שריפא וקורין בלשון ערבי אחתרקן ומסמר הצלוב מעמידה כדי שלא תוסיף:
אביי ורבא דאמרי תרוייהו כל }דבר }*}גי׳ ד״ת שיש בו}שהוא משום רפואה אין בו משום דרכי האמורי הא אין בו משום רפואה יש בו משום דרכי האמורי והתניא }אילן שמשיר פירותיו סוקרו בסקרא וטוענו באבנים גדולות בשלמא טוענו באבנים גדולות כי היכי דליכחוש חיליה אלא סוקרו בסקרא מאי רפואה קא עביד כי היכי דליחזוה אינשי וליבעו עליה רחמי כדתניא (ויקרא י״ג:מ״ה) וטמא טמא יקרא מכאן שצריך להודיע [צערו] לרבים ורבים יבקשו עליו רחמים אמר רבינא כמאן תלינן האידנא כובסא בדיקלא כמאן כי האי תנא.
מאי כובסא כמו נצר כדכתיב (ישעיהו י״א:א׳) ונצר משרשיו יפרה:
\textbf{סליקו להו במה אשה} 
\end{multicols}\newpage

\newchap{פרק \hebrewnumeral{7} כלל גדול}
\end{multicols}\newpage

\newsection{דף לא}
\begin{multicols}{2}
\textbf{{\largeכלל}} גדול אמרו בשבת כל השוכח עיקר שבת ועשה מלאכות הרבה בשבתות הרבה אינו חייב אלא חטאת אחת.
והיודע עיקר שבת ועשה מלאכות הרבה בשבתות הרבה חייב על כל שבת ושבת.
והיודע שהוא שבת ועשה מלאכות הרבה בשבתות הרבה חייב על כל אב מלאכה ומלאכה והעושה מלאכות הרבה מעין מלאכה אחת אינו חייב אלא חטאת אחת:
\textbf{{\largeגמ׳}} (דף סט:) אמר רב הונא }}היה מהלך במדבר ואינו יודע אימתי שבת מונה ששה ומשמר יום אחד אמר *}בד״י רבה}רבא ובכל יומא ויומא עביד כדי פרנסתו בר מההוא יומא וההוא יומא לימות }ליעביד מאתמול שתי פרנסות ודלמא אתמול שבת הוה אלא כל יום ויום עושה כדי פרנסתו ואפילו ההוא יומא וההוא יומא במאי מינכר ליה בקידוש והבדלה:
אמר רבא ואם היה מכיר סכום יום שיצא בו עושה מלאכה כל אותו היום פשיטא מהו דתימא כיון דבשבת לא נפיק בערב שבת נמי לא נפיק והאי חמשא בשבתא נפיק ותרי יומי לישתרו ליה למיעבד קמ״ל דלא זימנין דמשכח שיירתא
ומיקרי ונפיק:
\textbf{{\largeמתני׳}} }}(דף עג.) אבות מלאכות ארבעים חסר אחת הזורע והחורש והקוצר והמעמר הדש והזורה והבורר והטוחן והמרקד והלש והאופה הגוזז את הצמר והמלבנו והמנפצו והצובע והטווה והמיסך והעושה שתי בתי נירין והאורג שני חוטין והפוצע שני חוטין והקושר והמתיר והתופר שתי תפירות והקורע ע״מ לתפור שתי תפירות הצד צבי השוחטו והמפשיטו והמולחו והמעבדו והממחקו והמחתכו והכותב שתי אותיות והמוחק על מנת לכתוב שתי אותיות הבונה והסותר המבעיר והמכבה והמכה בפטיש והמוציא מרשות לרשות הרי אלו ארבעים מלאכות חסר אחת:
\textbf{{\largeגמ׳}} (דף עג:) תאנא הזורע והזומר והנוטע והמבריך והמרכיב כולן מלאכה אחת הן אמר רב יוסף האי מאן דקטל }}אספסתא בשבתא חייב שתים אחת משום קוצר ואחת משום נוטע:
אמר אביי מאן דקניב סילקא חייב שתים אחת משום קוצר ואחת משום נוטע.
תאנא החורש והחופר והחורץ כולן מלאכה אחת הן:
אמר רב ששת היתה לו גבשושית ונטלה בבית חייב משום בונה בשדה חייב משום חורש גומא וטממה בעפר בבית חייב משום בונה בשדה חייב משום חורש:
א״ר אבא החופר }גומא בשבת ואינו צריך אלא לעפרה פטור עליה ואפילו לר׳ יהודה דאמר כל מלאכה שאינה צריכה לגופה חייב עליה הני מילי מתקן אבל האי מקלקל הוא:
תאנא הקוצר }}והבוצר והגודר והמוסק והאורה כולן מלאכה אחת הן:
אמר רב פפא האי מאן דשדא פיסא לדיקלא ואתר תמרי חייב שתים אחת משום תולש ואחת משום מפרק }אמר רב אסי אין דרך תלישה בכך ואין דרך פריקה בכך:
גרסינן בפרק המצניע (דף צה.) תנו }רבנן }}החולב והמחבץ והמגבן כגרוגרת חייב חטאת ואמרינן חולב משום מאי מיחייב משום מפרק מחבץ חייב
\end{multicols}\newpage

\newsection{דף לב}
\begin{multicols}{2}
משום בורר מגבן חייב משום בונה ומפרק }תולדה דדש הוא:
והמעמר (דף עג:) אמר רבא האי מאן }דכניף מילחא ממלחתא מיחייב משום מעמר }אמר אביי אין דרך עימור בכך אלא בגידולי קרקע:
והדש תאנא }הדש והמנפץ והמנפט כולן מלאכה אחת הן:
והזורה והבורר והטוחן: (דף עד.) תנו רבנן }}}היו לפניו (שני) מיני אוכלין בורר ואוכל בורר ומניח ולא יברור ואם בירר חייב חטאת מאי קאמר אמר אביי הכי קאמר בורר ואוכל לאלתר בורר ומניח לאלתר ולבו ביום לא יברור ואם בירר נעשה כבורר לאוצר וחייב חטאת אמרוה קמיה דרבא ואמר להו שפיר קאמר נחמני היו לפניו }שני מיני אוכלין ובירר ואכל ובירר והניח רב אשי מתני פטור ור׳ ירמיה מדפתי מתני חייב רב אשי מתני פטור והא תניא חייב לא קשיא הא בקנון ותמחוי הא בנפה וכברה:
אמר חזקיה הבורר תורמוסין מתוך פסולת שלהן חייב חטאת לימא קסבר חזקיה אוכל מתוך פסולת אסור לא שאני תורמוסין (דף עד:) דשלקי ליה שבע זמנין בקדירה בקליפתא ואי לא שקלי }ליה מן הפסולת מסרח הילכך אי לא שקלי }ליה מן הפסולת כפסולת מתוך אוכל דמי וחייב:
והטוחן אמר רב פפא האי מאן דפרים }סילקא בשבת חייב משום טוחן ואמר רב מנשיא האי מאן דסלית }סילתי חייב משום טוחן אמר רב אשי ואי קפיד אמשחתא חייב משום מחתך.
פי׳ דסלית סילתי אלו עצי דקלים שעומדין שיבי שיבי פירוש נימין נימין וכשמפרק אותן הנימין יוצא מביניהן כמין קמח דק ומתוך כך נחשב כטוחן:
והלש והאופה אמר *}בס״י רב אשי בר רב עוירא}רב אחא בר אשי בר עוירא האי }}}מאן דשדא }סיכתא לאתונא חייב משום מבשל פשיטא מהו דתימא }לשרורי }מנא קא מכוין קא משמע לן דמירפא רפי והדר קמיט פירוש סיכתא גללי בהמה רקיקין אמר רבא האי מאן דארתח כופרא חייב משום מבשל:
הגוזז את הצמר המנפצו והמלבנו והצובע והטווה: תנו רבנן התולש את הכנף והקוטמו והמורטו חייב שלש חטאות:
הקושר והמתיר
והתופר שתי תפירות והא לא קיימא אמר רבה בר בר חנה א״ר יוחנן והוא שקשרן:
והקורע ע״מ לתפור שתי תפירות רבה ור׳ זירא דאמרי תרוויהו שכן יריעה שנפלה בה דרנא קורעין אותה ותופרין אותה (דף עה.) אמר מר זוטרא בר טוביה אמר רב המותח }חוט של תפירה בשבת חייב חטאת והלמד דבר אחד מן המגוש חייב מיתה והיודע לחשב בתקופות }ומזלות ואינו מחשב אסור לספר הימנו:
הצד }}צבי השוחטו והמפשיטו והמולחו והמעבדו (דף עה:) היינו מולח היינו מעבד רבי יוחנן ור׳ שמעון בן לקיש דאמרי תרוייהו אפיק חדא מינייהו ועייל שרטוט:
}והמוחקו א״ר אחא בר חנינא השף בין }העמודים בשבת חייב משום ממחק א״ר חייא בר אבא ג׳ דברים שח לי ר׳ אסי משום ריב״ל המגרר ראשי כלונסות }חייב משום מחתך והממרח רטיה בשבת חייב משום [ממחק] והמסתת }את האבן בשבת חייב משום מכה בפטיש אר״ש בר ביסנא אר״ש בן לקיש הצר בכלי צורה והמנפח בכלי זכוכית חייב משום מכה בפטיש אמר רב יהודה האי מאן דשקל }אקופי מגלימא חייב משום מכה בפטיש והני מילי דקפיד עלייהו:
והכותב שתי אותיות וכו׳: ת״ר כתב אות אחת גדולה ויש במקומה כדי לכתוב שתים פטור }מחק אות אחת גדולה ויש במקומה כדי לכתוב שתים חייב א״ר מנחם בר׳ יוסי זה חומר במוחק מבכותב:
הבונה והסותר }המכבה והמבעיר והמכה בפטיש: רבה ור׳ זירא דאמרי תרוייהו כל מידי דאית ביה גמר מלאכה חייב משום מכה בפטיש:
\textbf{{\largeמתני׳}} ועוד כלל }אחר אמרו כל הכשר להצניע ומצניעין כמוהו והוציאו בשבת חייב עליו וכל שאינו כשר להצניע ואין מצניעין כמוהו }והוציאו בשבת אינו חייב אלא המצניעו:
\textbf{{\largeגמ׳}} כל הכשר להצניע לאפוקי מאי רב מתנה אמר לאפוקי דם נדה מר עוקבא אמר לאפוקי עצי אשרה מאן דאמר דם נדה כל שכן עצי אשרה ומ״ד עצי אשרה אבל דם נדה לא דמצנע ליה לשונרא ואידך כיון דחלשא לא מצנע ליה:
\textbf{{\largeמתני׳}} (דף עו.) המוציא תבן כמלא פי פרה עצה כמלא פי גמל עמיר כמלא פי טלה }עלי שום
\end{multicols}\newpage

\newsection{דף לג}
\begin{multicols}{2}
ועלי בצלים לחים כגרוגרת יבשים כמלא פי גדי עשבים כמלא פי הגדי ואין מצטרפין זה עם זה מפני שאינן שוין בשיעוריהן המוציא אוכלים כגרוגרת [חייב] ומצטרפין זה עם זה מפני ששוין בשיעוריהן חוץ מקליפיהן וגרעיניהן ועוקציהן וסובן ומורסנן רבי יהודה אומר חוץ מקליפי עדשים המתבשלות עמהן:
\textbf{{\largeגמ׳}} מאי עצה אמר ר׳ יהודה תבן של מיני קטנית:
(דף עו:) ר׳ יהודה אומר חוץ מקליפי עדשים וכו׳:
עדשים אין פולין לא והתניא ר׳ יהודה אומר חוץ מקליפי פולין ועדשים לא קשיא הא בחדתי והא בעתיקי עתיקי מאי טעמא לא מצטרפי א״ר אבהו מפני שנראים כזבובים בקערה:
\textbf{סליקו להו ככל גדול} 
\textbf{{\largeהמוציא}} }יין כדי מזיגת הכוס חלב כדי גמיעה דבש כדי ליתן על הכתית שמן כדי לסוך אבר קטן מים }כדי לשוף את הקילור }ושאר כל המשקין ברביעי׳ וכל השופכין ברביעית רבי שמעון אומר כולן ברביעית ולא נאמרו כל השיעורין הללו אלא למצניעיהן:
\textbf{{\largeגמ׳}} תנא כדי מזיגת כוס יפה מאי כוס יפה כוס של ברכה אמר רב נחמן אמר רבה בר אבוה כוס של ברכה צריך שיהא בו רובע רביעית כדי שימזגנו ויעמוד על רביעית:
חלב כדי }גמיעה: (דף עז:) ת״ר המוציא חלב של בהמה כדי גמיעה וחלב של אשה ולובן של ביצה כדי ליתן }במשיפת [של קילור] קילור כדי לשוף במים:
דבש כדי ליתן על הכתית תנא כדי ליתן על פי כתית שמן כדי לסוך אבר קטן תניא ר׳ שמעון בן אלעזר אומר שמן כדי לסוך אבר קטן של קטן בן יומו:
\textbf{{\largeמתני׳}} (דף עח.) המוציא חבל כדי לעשות אוזן לקופה גמי כדי לעשות תלאי לנפה ולכברה ר׳ יהודה אומר כדי ליקח ממנו מדת מנעל לקטן ונייר כדי לכתוב עליו }קשר של מוכסין והמוציא קשר מוכסין }חייב (דף עח:) נייר מחוק כדי לכרוך על פי צלוחית של פליטון:
\textbf{{\largeגמ׳}} תנו רבנן הוצין כדי לעשות אוזן לכפיפה מצרית.
סיב אחרים אומרים כדי ליתן על פי משפך קטן של יין }רבב כדי לשוף תחת אספוגית קטנה וכמה שיעורה כסלע פירוש אספוגית רקיק כדכתיב (שמות כט) ורקיקי מצות ומתרגמינן אספוגין פטירין:
מוכין כדי לעשות כדור קטנה וכמה שיעורה עד כאגוז:
נייר כדי לכתוב עליו קשר של מוכסין תנא וכמה קשר של מוכסין שתי אותיות:
\textbf{{\largeמתני׳}} עור כדי לעשות קמיע קלף כדי לכתוב עליו פרשה קטנה שבתפלין שהיא שמע ישראל דיו כדי לכתוב בו שתי }אותיות כחול כדי
\end{multicols}\newpage

\newchap{פרק \hebrewnumeral{8} המוציא יין}
\begin{multicols}{2}
לכחול }עין אחת זפת וגפרית כדי לעשות בו נקב קטן שעוה כדי ליתן על פי נקב קטן:
\textbf{{\largeגמ׳}} (דף עט:) אמר רב קלף }}}}הרי הוא כדוכסוסטוס מה דוכסוסטוס כותבין עליו מזוזה אף קלף כותבין עליו מזוזה וכן הלכתא ותפילין אין נכתבים על }הדוכסוסטוס אלא על הקלף בלבד:
\textbf{{\largeמתני׳}} (דף עח:) }דבק כדי ליתן בראש השבשבת חרסית כדי לעשות פי כור של צורפי זהב ר׳ יהודה אומר כדי לעשות פטפוט סובין כדי ליתן על פי כור של צורפי זהב סיד כדי לסוד אצבע קטנה שבבנות רבי יהודה אומר כדי לעשות כלכול רבי נחמיה אומר כדי לעשות אנטיפי (דף פ:) אדמה כחותם המרצופין דברי רבי עקיבא וחכמים אומרים כחותם האגרות זבל וחול דק כדי לזבל קלח של כרוב
\end{multicols}\newpage

\newsection{דף לד}
\begin{multicols}{2}
דברי רבי עקיבא וחכ״א כדי לזבל כרישא חול הגס כדי ליתן על מלא כף סיד קנה כדי לעשות קולמוס אם היה עבה או מרוסס כדי לבשל ביצה קלה שבביצים טרופה ונתונה באלפס (דף פא.) עצם כדי לעשות תרווד רבי יהודה אומר כדי לעשות חף }:
\textbf{{\largeגמ׳}} אמר עולא }חף פותחת:
}זכוכית כדי לגרר }בה ראש הכרכר:
}תאנא כדי לפצע בה שני נימין כאחד }צרור אבן כדי לזרוק בעוף ר׳ אליעזר בן יעקב אומר כדי לזרוק בבהמה }א״ר יעקב כדי שתהא הבהמה מרגשת בה וכמה שיעורה תניא רבי אליעזר }אומר משקל עשרה זוז:
זונין עייל לבי מדרשא אמר להן }}}}אבנים של בית הכסא שיעורן בכמה אמרו לו כזית כאגוז כביצה אמר להן וכי טורטני יכניס נמנו וגמרו כמלא היד:
תנו רבנן שלשה אבנים מקורזלות מותר להכניס לבה״כ וכמה שיעורן ר׳ מאיר אומר כאגוז רבי יהודה אומר כביצה והלכתא נמנו וגמרו כמלא היד:
אמר רב יהודה אבל לא את הפייס מאי פייס א״ר זירא כרשיני בבלייתא פירוש אדמה שהיא קרובה להתפרר:
אמר ר׳ ינאי אם יש מקום }קבוע לבית הכסא מלא היד ואם לאו כהכרע מדוכה קטנה של בשמים פי׳ ראש של בוכנא שדוכין בו את הבושם ופירשוהו בתלמוד ירושלמי מלא רגל מדוכה קטנה של בושם אמר רב ששת }אם יש עליה עד מותר פירשוהו רבנן עד כגון סמרטוטין כדתנן התם *}נדה יא. יב:}משמשת בעדים הן הן תיקוניה הן הן עיוותיה א״ל אביי לרב יוסף ירדו עליהן גשמים ונטשטשו מהו כלומר אם ירדו גשמים על האבנים המקורזלות האלו ונטבעו בקרקע מהו מי חיישינן }שמא יהא כסותר או כטוחן או לא אמר ליה אם היה רשומו ניכר מותר (דף פא:) בעא מיניה רבה בר רב שילא מרב חסדא להעלותן לגג אחריו מהו אמר ליה גדול כבוד הבריות שדוחה את לא תעשה שבתורה:
אמר רב הונא אסור לפנות בשדה ניר בשבת משום דרבה דאמר רבה היתה לו גומא וטממה בבית חייב משום בונה בשדה חייב }משום חורש:
אמר ר׳ שמעון בן לקיש צרור שעלו }בו עשבים מותר לקנח בו בשבת והתולש ממנו בשבת }חייב חטאת:
אמר ר׳ יוחנן אסור לקנח בחרס משום סכנה ואיתימא משום כשפים מאי כשפים כי הא דרב חסדא ורבה בר רב הונא הוו קא אזלי בארבא אמרה להו ההיא מטרוניתא אותבון בהדייכו ולא אותבוה אמרה }(להו) מילתא וקא אסרה לה לספינה אמרו אינהו מילתא ושריוה אמרה להו מאי איעבד לכו (דף פב.) דלא מקנחיתו לכו בחספא ולא קטליתו לכו כינה אמנייכו ולא אכליתו ירקא מכישא דאסר גינאה:
היו לפניו
צרור וחרס מקנח בחרס ואינו מקנח בצרור ודוקא בחרס דאגני כלים:
היו לפניו }צרור ועלין רב חסדא ורב המנונא חד אמר מקנח בצרור ואינו מקנח בעלין וחד אמר מקנח בעלין ואינו מקנח בצרור מיתיבי המקנח בדבר שהאור שולט בו שיניו נושרין לא קשיא הא בלחין הא ביבשין:
איתמר הצריך }להפנות }ואכל רב חסדא ורבינא חד אמר רוח רע שולט עליו וחד אמר רוח זוהמא שורה עליו איתמר הצריך ליפנות ואוכל דומה לתנור שהוסק על אפרו וזו היא התחלת ריח הפה }(אכל ולא שתה אכילתו דם וזו היא התחלת חולי מעים אכל ולא הלך ארבע אמות אכילתו מתרקבת וזו היא תחלת רוח רעה רחץ ולא סך תחלה דומה לזולף מים על גבי חבית רחץ בחמין ולא שתה ממנו דומה לתנור שהסיקוהו מבחוץ ולא הסיקוהו מבפנים:
רחץ בחמין ולא נשתטף בצונן דומה לברזל שהכניסוהו לאש ולא הכניסוהו במים):
הנצרך ליפנות ואינו יכול ליפנות רב חסדא אמר יעמוד וישב יעמוד וישב:
אמר ליה רבי זירא לרב חסדא תא ניתני ותא ניעסוק בחיי דברייתא הנצרך ליפנות ואינו נפנה רב חסדא אמר ילך וישב ויעמוד }הלך ארבע אמות רב המנונא אמר ישתמש *}בגמ׳ ימשמש}בצרורות באותו מקום רב חנין מנהרדעא אמר }יהפך לצדדין ורבנן אמרי יסיח דעתו לדבר אחר אמר ליה רב אחא בריה דרבה לרב אשי וכל שכן שלא יפנה אמר ליה יסיח דעתו מדברים אחרים:
אמר רבי ירמיה מדפתי אנא חזיתיה לההוא טייעא דקם ויתיב דקם ויתיב עד דשפיך כקדרה:
תנו רבנן הנכנס לסעודת קבע צריך שיבדוק את עצמו כיצד מהלך ארבע פעמים של עשר }פסיעות או עשר פעמים של ארבע }פסיעות ונפנה ואחר כך יכנס:
\textbf{{\largeמתני׳}} חרס }כדי ליתן בין פצים }לחבירו דברי רבי יהודה ר׳ מאיר אומר כדי לחתות בו את האור רבי יוסי אומר כדי לקבל בו רביעית אמר ר״מ אע״פ שאין ראיה לדבר זכר לדבר (ישעיהו ל׳:י״ד) ולא ימצא במכתתו חרס לחתות אש מיקוד א״ל רבי יוסי משם ראיה ולחשוף מים מגבא:
\textbf{סליקו להו המוציא} 
(דף פב.) \textbf{{\largeאמר}} רבי עקיבא }מנין לע״ז שהיא מטמאה במשא כנדה שנאמר (ישעיהו ל׳:כ״ב) תזרם כמו דוה צא תאמר לו מה נדה מטמאה במשא אף ע״ז מטמאה במשא (דף פג:) מנין לספינה שהיא טהורה שנאמר (משלי ל׳:י״ט) דרך אניה בלב ים (דף פד:) מנין }}לערוגה שהיא ששה על ששה טפחים שזורעין בתוכה חמשה זרעונין ארבעה בארבע רוחות הערוגה ואחד
\end{multicols}\newpage

\newchap{פרק \hebrewnumeral{9} אמר רבי עקיבא}
\end{multicols}\newpage

\newsection{דף לה}
\begin{multicols}{2}
באמצע שנא׳ (ישעיהו ס״א:י״א) כי כארץ תוציא צמחה וכגנה זרועיה תצמיח זרעה לא נאמר אלא זרועיה תצמיח (דף פו.) מנין }לפולטת שכבת זרע ביום השלישי שהיא טמאה שנאמר (שמות י״ט:ט״ו) היו נכונים לשלשת ימים אל תגשו אל אשה }}מנין שמרחיצין את הקטן ביום השלישי שחל להיות בשבת שנאמר (בראשית ל״ד:כ״ה) ויהי ביום השלישי בהיותם כואבים מנין }שקושרין לשון של זהורית בראש שעיר המשתלח שנאמר (ישעיהו א׳:י״ח) אם יהיו חטאיכם כשנים כשלג ילבינו מנין לסיכה שהיא כשתיה ביום הכיפורים אע״פ שאין ראיה לדבר זכר לדבר שנאמר (תהילים ק״ט:י״ח) ותבא כמים בקרבו וכשמן בעצמותיו:
(דף פט:) המוציא עצים כדי לבשל }ביצה קלה תבלין כדי לתבל ביצה קלה ומצטרפין }זה עם זה קליפי אגוזים וקליפי רמונים אסטיס ופואה כדי לצבוע בהן בגד קטן בסבכה.
מי רגלים נתר ובורית קימיניא ואשלג כדי לכבס כהן בגד קטן בסבכה רכי יהודה אומר כדי להעביר על הכתם (דף צ.) פלפלת כל שהוא ועטרן כל שהוא מיני בשמים ומיני מתכות כל שהן מעפר המזבח ומאבני המזבח ממקק ספרים וממקק מטפחותיהן כל שהן שמצניעין אותן לגנזן רבי יהודה אומר אף המוציא ממשמשי ע״א כל שהן שנאמר (דברים י״ג:י״ח) ולא ידבק בידך מאומה מן החרם:
המוציא קופת הרוכלין אף על פי שיש בה מינין הרבה אינו חייב אלא חטאת אחת זרעוני גנה פחות מכגרוגרת ר׳ יהודה }אומר חמשה (דף צ:) זרע קשואין שנים זרע דלועין שנים זרע פול המצרי שנים חגב חי }כל שהוא מת כגרוגרת צפורת כרמים בין חיה בין מתה כל שהיא שמצניעין אותה לרפואה ר׳ יהודה אומר אף המוציא חגב חי טמא כל שהוא שמצניעין אותו לקטן לשחק בו:
גמ׳ (דף פג:) אמר רב יהודה אמר רב }}לעולם אל ימנע אדם עצמו מבית המדרש אפי׳ שעה אחת ואמר רב יהודה אמר רב לעולם אל ימנע אדם מדברי תורה ואפילו בשעת מיתה שנא׳ (במדבר י״ט:י״ד) זאת התורה אדם כי ימות באהל אפילו בשעת מיתה תורה אמר ר״ש בן לקיש אין דברי תורה מתקיימין אלא במי שממית עצמו עליהן שנא׳ זאת התורה אדם כי ימות באהל:
(דף צ.) אמר רב יהודה אמר רב לעולם }ניזדהר אינש ממקק דספרי תכך דשיראי ואילא דעינכי ופה דתאיני והה דרמוני כולהו סכנתא:
ההוא תלמידא דהוה יתיב קמיה דרבא הוה קא אכיל תאיני אמר ליה רבי קוצים יש בתאיני אמר קטלי׳ פה לדין:
\textbf{סליקו להו אמר רבי עקיבא} 
(דף צ:) U \textbf{{\largeהמצניע}} לזרע ולדוגמא ולרפואה והוציאו בשבת חייב עליו בכל שהוא וכל אדם אין חייבין עליו אלא כשיעורו חזר והכניסו אינו חייב אלא כשיעורו:
\end{multicols}\newpage

\newchap{פרק \hebrewnumeral{10} המצניע}
\begin{multicols}{2}
(דף צא:) המוציא אוכלין ונתנן על האסקופה בין שחזר והוציאן בין שהוציאן אחר פטור שלא עשה מלאכתו בבת אחת קופה שהיא מליאה פירות ונתנה על האסקופה החיצונה אף על פי שרוב הפירות מבחוץ פטור עד שיוציא את כל הקופה (דף צב.) המוציא בין בימינו בין בשמאלו בין בתוך חיקו בין על כתפו חייב שכן משא בני קהת לאחר ידו ברגלו בפיו ובמרפקו באזנו ובשערו ובאפונדתו ופיה למטה בין אפונדתו לחלוקו ובחפת חלוקו במנעלו ובסנדלו פטור שלא הוציא כדרך המוציאים:
\textbf{{\largeגמ׳}} אמר ר׳ אלעזר המוציא משוי למעלה מעשרה טפחים חייב }שכן משא בני קהת:
אמר רב משום ר׳ חייא המוציא משוי על ראשו פטור (דף צב:) ואם תאמר אנשי הוצל עושים כן בטלה דעתן אצל כל אדם:
\textbf{{\largeמתני׳}} }המתכוין להוציא לפניו ובא לו לאחריו פטור לאחריו ובא לו לפניו חייב באמת אמרו האשה החוגרת בסינר בין מלפניה בין מלאחריה חייבת שכן ראוי להיות חוזר רבי יהודה אומר אף במקבלי פתקין:
המוציא ככר לרשות הרבים חייב הוציאוהו שנים פטורים אם לא יכול אחד להוציא והוציאוהו שנים חייבים ור׳ שמעון פוטר (דף צג:) }המוציא אוכלין פחות מכשיעור בכלי פטור אף על הכלי מפני שהכלי טפל לו את החי במטה פטור אף על המטה מפני שהמטה טפלה לו את המת במטה חייב }אף על המטה וכן כזית }מן המת וכזית מן הנבלה וכעדשה מן השרץ חייב ור׳ שמעון פוטר:
\textbf{{\largeגמ׳}} (דף צד:) }}ההוא שיכבא דהוה בדרוקרת שרא להו ר״נ בר יצחק לאפוקי לכרמלית א״ל רב אחא בריה דרבא לר״נ כמאן כר״ש אימר דפטר ר״ש מחיוב חטאת איסורא }דרבנן מיהא איכא א״ל האלהים }עיילת ביה את ואפילו לר׳ יהודה מי קאמינא לרשות הרבים לכרמלית קאמינא }דגדול כבוד הבריות שדוחה את לא תעשה שבתורה:
\end{multicols}\newpage

\newsection{דף לו}
\begin{multicols}{2}
\textbf{{\largeמתני׳}} }}}הנוטל }צפרניו זו בזו או בשיניו }וכן שערו וכן שפמו וכן זקנו וכן }הגודלת וכן }הפוקסת וכן }הכוחלת ר׳ אליעזר מחייב חטאת וחכמים אוסרין משום שבות התולש מעציץ נקוב חייב }ושאינו נקוב פטור ור׳ שמעון פוטר בזה ובזה:
\textbf{{\largeגמ׳}} אמר רבי אלעזר מחלוקת ביד אבל בכלי דברי הכל חייב ואמר רבי אלעזר מחלוקת לעצמו אבל לחבירו דברי הכל פטור:
וכן }}}שערו וכן שפמו: תאנא הנוטל מלא פי הזוג בשבת חייב וכמה הוא מלא פי הזוג אמר רב יהודה שתים:
תניא נמי הכי הנוטל מלא פי הזוג חייב וכמה מלא פי הזוג שתים ר׳ אליעזר אומר אחת ומודים חכמים לרבי אליעזר }}במלקט לבנות מתוך שחורות ואפילו אחת חייב ודבר זה אפי׳ בחול אסור משום (דברים כ״ב:ה׳) לא ילבש גבר שמלת אשה:
}}תניא ר״ש בן אלעזר אומר צפורן }שפירש רובה וציצין שפירשו רובן ביד מותר בכלי פטור אבל אסור לא פירש רובן ביד פטור אבל אסור (ר׳ שמעון בן אלעזר אומר) בכלי חייב חטאת אמר רב יהודה הלכה כר׳ שמעון בן אלעזר אמר רבה אמר ר׳ יוחנן והוא שפירשו כלפי מעלה ומצערות אותו
(דף צה.) אמימר }}}שרא זילחא במחוזא פירוש התיר }רבוץ הבית במחוזא שהיו הקרקעות שלהן }בשיש טעמא מאי אמור רבנן דלמא אתי לאשוויי גומות הכא ליכא גומות רבא תוספאה אשכחיה לרבינא דקא מצטער מהבלא ואמרי לה מר קשישא בריה דרבא אשכחיה לרב אשי דקא מצטער מהבלא אמר ליה לא סבר לה מר להא דתניא הרוצה לרבץ את ביתו בשבת מביא עריבה מליאה מים ורוחץ פניו בזוית זו ורגליו בזוית זו וידיו בזוית זו ונמצא הבית מתרבץ מאליו אמר ליה לאו אדעתאי תנא אשה חכמה מרבצת ביתה בשבת והאידנא דסבירא לן כר״ש אפי׳ לכתחלה שרי:
\textbf{סליקו להו המצניע} 
(דף צו.) \textbf{{\largeהזורק}} }מרשות היחיד לרשות הרבים או מרשות הרבים לרשות היחיד חייב מרשות היחיד לרשות היחיד ורשות הרבים באמצע רבי עקיבא מחייב וחכמים פוטרין כיצד שתי גזוזטראות זו כנגד זו ברשות הרבים המושיט והזורק מזו לזו פטור היו שתיהן בדיוטא אחת הזורק פטור והמושיט חייב שכך היתה עבודת הלוים שתי עגלות זו אחר זו
\end{multicols}\newpage

\newchap{פרק \hebrewnumeral{11} הזורק}
\begin{multicols}{2}
ברשות הרבים מושיטין את הקרשים מזו לזו }ברשות הרבים אבל לא זורקין
(דף צט.) חולית הבור והסלע שהן גבוהין עשרה ורחבין ארבעה הנוטל מהן והנותן על גבן חייב פחות מכאן פטור (דף ק.) הזורק ד׳ אמות בכותל למעלה מי׳ טפחים כזורק באויר למטה מי׳ טפחים כזורק בארץ והזורק בארץ ד׳ אמות חייב זרק לתוך ד׳ אמות ונתגלגל חוץ לד׳ אמות פטור חוץ לארבע אמות ונתגלגל תוך ד׳ אמות חייב
(דף ק:) והזורק בים ארבע אמות פטור אם היה רקק מים שרשות הרבים מהלכת בו הזורק לתוכו ארבע אמות חייב וכמה הוא רקק מים פחות מעשרה טפחים רקק מים ורשות הרבים מהלכת בו הזורק בתוכו ארבע אמות חייב הזורק מן הים ליבשה ומן היבשה לים ומן הים לספינה ומן הספינה לים ומן הספינה לחברתה פטור ספינות קשורות זו לזו מטלטלין מזו לזו אם אינן קשורות אף על פי שהן מוקפות אין מטלטלין מזו לזו:
\textbf{{\largeגמ׳}} איתמר }}}ספינה רב הונא אמר מוציא ממנה זיז כל }שהוא וממלא רב חסדא ורבה בר רב הונא אמרי עושה מקום ארבעה וממלא והלכתא כותייהו דהוה ליה רב הונא חד לגבי תרי ואין דבריו של אחד במקום שנים אבל שופכין דידיה שופך להו אדופני }ספינה ואף על גב דנחתי מכחו לים [והלכתא] כחו בכרמלית לא גזרו רבנן:
(דף קא:) ספינות }}}קשורות זו לזו מטלטלין מזו לזו הני מילי על ידי עירוב וכדרב ספרא }דתניא ספינות קשורות זו לזו מערבין ומטלטלין מזו לזו נפסקו נאסרו חזרו ונקשרו בין שוגגין בין מזידין בין אנוסין בין מוטעין חזרו להיתרן הראשון וכן מחצלאות הפרוסות לרה״ר מערבין ומטלטלין מזו לזו נגללו נאסרו חזרו ונפרשו בין שוגגין בין מזידין בין אנוסין בין מוטעין חזרו להיתרן הראשון שמחיצה הנעשית בשבת בין בשוגג בין במזיד בין באונס בין בטעות שמה מחיצה איני והא אמר רב נחמן לא שנו אלא לזרוק אבל לטלטל אסור כי איתמר דרב נחמן אמזיד איתמר:
\end{multicols}\newpage

\newsection{דף לז}
\begin{multicols}{2}
\textbf{{\largeמתני׳}} (דף קב.) הזורק ונזכר לאחר שיצאה מידו קלטה אחד או קלטה כלב או שנשרפה פטור הזורק לעשות חבורה בין באדם בין בבהמה ונזכר עד שלא נעשה חבורה פטור זה הכלל כל חייבי חטאות אינן חייבין עד שיהא תחלתן וסופן שגגה תחלתן שגגה וסופן זדון או תחלתן זדון וסופן שגגה פטורין עד שתהא תחלתן וסופן שגגה:
\textbf{סליקו להו הזורק} 
(דף קב:) \textbf{{\largeהבונה}} }}כמה יבנה ויהיה חייב הבונה כל שהוא והמסתת והמכה בפטיש ובמעצד והקודח כל שהוא חייב זה הכלל כל העושה מלאכה }ומלאכתו מתקיימת בשבת חייב רבן שמעון ב״ג אומר אף המכה בקורנס על הסדן בשעת גמר מלאכה חייב מפני שהוא כמתקן מלאכה:
\textbf{{\largeגמ׳}} העושה }נקב בלול של תרנגולין רב אמר חייב משום בונה ושמואל אמר חייב משום מכה בפטיש:
עייל שופתא }בקופינא דמרא. פירוש הכניס הקרדום בתוך העץ שלו רב אמר חייב משום בונה ושמואל אמר חייב משום מכה בפטיש:
\textbf{{\largeמתני׳}} (דף קג.) }החורש כל שהוא המנכש והמקרסם והמזרד כל שהוא חייב המלקט עצים אם לתקן כל שהן ואם להסיק כדי לבשל ביצה קלה והמלקט עשבים אם לתקן כל שהן ואם לבהמה כמלא פי הגדי }הכותב שתי אותיות בין בימינו בין בשמאלו בין משם אחד בין משני שמות בין משני }סמניות בכל לשון חייב א״ר יוסי לא חייבו שתי אותיות אלא משום רושם שכן כותבין על קרשי המשכן לידע איזהו בן זוגו א״ר יהודה מצינו שם קטן משם גדול כגון שם משמעון ומשמואל נח מנחור דן מדניאל
\end{multicols}\newpage

\newchap{פרק \hebrewnumeral{12} הבונה}
\begin{multicols}{2}
וגד מגדיאל:
\textbf{{\largeגמ׳}} אוקמה אביי בשולט בשתי ידיו:
\textbf{{\largeמתני׳}} (דף קד:) הכותב שתי אותיות בהעלם אחד חייב כתב בדיו בסם ובסקרא בקומוס ובקנקנתום ובכל דבר שהוא רושם על שני כותלי }הבית על שני לוחי פנקס והן נהגין זה עם זה חייב.
הכותב על בשרו חייב המשרט על בשרו ר׳ אליעזר מחייב חטאת ורבי יהושע פוטר.
כתב במשקין במי פירות באבק דרכים באבק סופרים ובכל דבר שאינו מתקיים פטור.
לאחר ידו ברגלו בפיו ובמרפקו כתב אות אחת סמוך לכתב וכתב ע״ג כתב נתכוין לכתוב }חי״ת וכתב שני זיינין אחת בארץ ואחת בקורה בתב על שני כותלי הבית על שני דפי פנקס ואין נהגין זה עם זה פטור כתב אות אחת נוטריקון רבי יהודה בן בתירא מחייב וחכמים פוטרין:
\textbf{{\largeגמ׳}} תנא כתב אות אחת והשלימה לספר ארג חוט אחד והשלימה לבגד חייב:
\textbf{{\largeמתני׳}} (דף קה.) הכותב שתי אותיות בשתי העלמות אחת בשחרית ואחת בין הערבים רבן גמליאל מחייב וחכמים פוטרין:
\textbf{{\largeגמ׳}} במאי קא מיפלגי רבן גמליאל סבר אין ידיעה לחצי שיעור ורבנן סברי יש ידיעה לחצי שיעור:
\textbf{סליקו להו הבונה} 
\textbf{{\largeרבי}} }}אליעזר אומר האורג שלשה חוטין בתחלה ואחד על האריג חייב וחכמים אומרים בין בתחלה בין בסוף שיעורו שני חוטין העושה שני בתי נירין בנירין בקירוס בנפה ובכברה ובסל חייב התופר שתי תפירות והקורע על מנת לתפור ב׳ תפירות הקורע }בחמתו והקורע על מתו ובל המקלקלין פטורין והמקלקל על מנת לתקן
\end{multicols}\newpage

\newchap{פרק \hebrewnumeral{13} האורג}
\end{multicols}\newpage

\newsection{דף לח}
\begin{multicols}{2}
שיעורו כמתקן:
\textbf{{\largeגמ׳}} תניא ר״ש אומר משום רבי חלפתא בן אגרא משום ר׳ יוחנן }המשבר כליו בחמתו והמקרע כליו והמפזר מעותיו בחמתו יהי בעיניך כעובד ע״ז שכן דרכו של יצר הרע היום אומר לו עשה כך ולמחר אומר לו לך ועבוד ע״ז והולך ועובד א״ר יוחנן מאי קרא (תהילים פ״א:י׳) לא יהיה בך אל זר איזה אל זר שהוא בגופו של אדם הוי אומר זה יצר הרע }}ואי עביד למירמא אימתא לאנשי ביתיה שפיר דמי כי הא דרב יהודה שליף מצובייתא רב אחא בר יעקב תבר מאני תבירי רב ששת רמי לאמתיה מונינא ארישא:
וכל המקלקלין פטורין (דף קו.) ואפי׳ }חובל ומבעיר לר׳ יהודה אבל לר״ש כל המקלקלין פטורין חוץ מחובל ומבעיר כדתני רבי אבהו קמיה דר׳ יוחנן וטעמיה דר׳ שמעון דמדאצטריך קרא למישרי מילה מכלל דחובל בעלמא חייב ומדאסר רחמנא הבערה גבי בת כהן }הא מבעיר בעלמא חייב ורבי יהודה התם מתקן הוא כדאמר רב אשי מה לי לתקן מילה מה לי לתקן כלי מה לי לבשל פתילה מה לי לבשל סממנין:
\textbf{{\largeמתני׳}} (דף קה:) המלבן }}}והמנפץ והצובע והטווה שיעורו כמלא רחב }הסיט כפול והאורג שני חוטין שיעורו כמלא הסיט (דף קו.) ר׳ יהודה אומר הצד צפור למגדל וצבי לבית חייב וחכמים אומרים צפור למגדל (דף קו:) וצבי לבית לגנה ולחצר ולביברין רבן שמעון בן גמליאל אומר לא כל הביברין שוין אלא זה הכלל מחוסר צידה פטור ושאינה מחוסר צידה חייב:
צבי שנכנס לבית ונעל אחד בפניו חייב נעלו שנים פטורין לא יכול אחד לנעול ונעלו שנים חייבין ור״ש פוטר:
\textbf{{\largeגמ׳}} ת״ר הצד חגבים חגזין צרעין ויתושין בשבת חייב דברי ר״מ וחכ״א כל שבמינו ניצוד חייב שאין במינו ניצוד פטור:
\textbf{{\largeמתני׳}} ישב האחד על הפתח ולא מלאהו וישב השני ומלאהו השני חייב ישב הראשון על הפתח ומלאהו ובא השני וישב בצדו אע״פ שעמד הראשון והלך לו הראשון חייב והשני פטור הא למה זה דומה }לנועל ביתו לשמרו ונמצא צבי שמור בתוכו:
\textbf{{\largeגמ׳}} (דף קז.) א״ר חייא
בר אבא א״ר יוחנן נכנסה לו צפור תחת כנפיו יושב ומשמרה עד שתחשך:
}}אמר שמואל כל פטורי דשבת פטור אבל אסור לבר מהני תלת דפטור ומותר חדא הא וממאי דפטור ומותר דתני סיפא הא למה זה דומה לנועל את ביתו לשמרו ונמצא צבי שמור בתוכו ואידך המפיס }מורסא בשבת אם לעשות לה פה חייב ואם להוציא ממנה ליחה פטור וממאי דפטור ומותר דתנן מחט של יד (מותר) ליטול בה את הקוץ ואידך הצד נחש בשבת אם מתעסק בו שלא ישכנו פטור ואם לרפואה חייב וממאי דפטור ומותר דתנן כופין קערה על גבי הנר בשביל שלא תאחוז בקורה ועל צואה של קטן ועל עקרב שלא תשוך }:
\textbf{סליקו להו האורג} 
\textbf{{\largeשמונה}} שרצים }}}האמורין בתורה הצדן והחובל בהן חייב ושאר שקצים ורמשים החובל בהן פטור הצדן }לצורך חייב שלא לצורך פטור חיה ועוף }שברשותו הצדן פטור }והחובל בהן חייב:
\textbf{{\largeגמ׳}} (דף קז:) הצד פרעוש }בשבת פטור וההורגו חייב לדברי הכל:
אמר אביי האי מאן דתליש פיטורי מאודנא דחצבא מיחייב משום עוקר דבר מגידולו:
\textbf{{\largeמתני׳}} }}אין עושין הילמי בשבת (דף קח:) אבל עושין מי מלח וטובל בהן פתו ונותן לתוך התבשיל א״ר יוסי והלא הוא הילמי בין מרובה בין מועט אלא אלו הן מי מלח המותרין נותן שמן בתחלה לתוך המים או לתוך המלח:
\textbf{{\largeגמ׳}} מאי קאמר אמר רב יהודה אמר שמואל הכי קאמר אין עושין מי מלח מרובין אבל עושין מי מלח מועטין אמר ר׳ יוסי והלא הוא הילמי בין מרובין בין מועטין:
איבעיא להו רבי יוסי לאיסור או להיתר וסליקא לאיסור:
תניא נמי הכי אין עושין מי מלח מרובין לתת לתוך הכבשין שבתוך הגסטרא אבל עושין מי }מלח מועטין ואוכל בהן פתו ונותן לתוך התבשיל אמר רבי יוסי וכי מפני שהללו מרובין והללו מועטין הללו אסורין והללו מותרין יאמרו מלאכה מרובה אסורה מלאכה מועטת מותרת אלא אלו ואלו אסורין ואלו הן מי
\end{multicols}\newpage

\newchap{פרק \hebrewnumeral{14} שמנה שרצים}
\end{multicols}\newpage

\newsection{דף לט}
\begin{multicols}{2}
מלח המותרין נותן שמן ומלח או שמן ומים ובלבד שלא יתן מים ומלח לכתחלה ולית הלכתא כר׳ יוסי אלא כתנא קמא דאמרינן בעירובין בפירקא קמא (דף יד:) אמר רב יוסף אמר רב יהודה אמר שמואל אין הלכה כר׳ יוסי לא בהילמי ולא בלחיים:
(מכילתין דף קח:) תני רב יהודה בר חביבא אין עושין מי מלח עזין מאי מי מלח עזין רבה ורב יוסף דאמרי תרוייהו כל שהביצה צפה בהן וכמה אמר אביי תרי תילתי מלחא וחדא מיא למאי עבדי לה אמר רבי אבהו למורייסא ותני רב יהודה בר חביבא אין מולחין צנון וביצה בשבת רב חזקיה משמיה דאביי אמר צנון אסור וביצה מותרת וכן הלכה:
אמר רב נחמן מריש הוה מלחנא פוגלא בשבתא אמינא אפסודי קא מפסידנא ליה דאמר שמואל פוגלא חורפיה מעלי ליה כיון דשמעיתה להא דאמרי׳ כי אתא עולא אמר במערבא מלחי פישרי פישרי מימלח לא מלחנא טבולי מטבילנא:
יין }}}לתוך העין אסור ע״ג העין מותר ורוק תפל אפי׳ ע״ג העין אסור וכן הלכה אמר מר עוקבא אמר שמואל שורה אדם קילורין מערב שבת ונותן ע״ג עיניו בשבת ואינו חושש בר ליואי הוה קאי קמיה דמר עוקבא חזייה דהוה עמיץ ופתח אמר ליה כולי האי לא שרו רבנן ולא שרא מר שמואל
תניא }חנניה אומר משום רבי יהודה טובה טיפת צונן שחרית ורחיצת ידים ורגלים בחמין ערבית מכל קילורין שבעולם הוא היה אומר }}}}יד לעין תקצץ יד לחוטם תקצץ יד לפה תקצץ *}[יד לאזן תקצץ]}יד לחיסודה תקצץ יד לאמה תקצץ יד לפי טבעת תקצץ יד (דף קט.) לגיגית תקצץ יד מחרשת יד מסמאת יד מעלה פוליפוס תנן התם בנדה (נידה דף יג.) כל היד המרבה לבדוק בנשים משובחת באנשים תקצץ מאי שנא נשים ומאי שנא אנשים נשים דלאו בנות הרגשה נינהו משובחת אנשים דבני הרגשה נינהו תקצץ אי הכא מאי אריא מרבה אפילו לא מרבה נמי כי קתני מרבה אנשים:
תנא בד״א לענין שכבת זרע אבל לענין זוב אף באנשים היא משובחת ואפילו לענין שכבת זרע נמי אם בא לבדוק בחרס או בצרור בודק ומטלית עבה הרי היא כחרס:
תניא רבי אליעזר אומר כל האוחז באמה ומשתין כאילו מביא מבול לעולם אמרו לו לר׳ אליעזר והלא ניצוצות מתיזין לו על רגליו ונראה ככרות שפכה ומוציא לעז על בניו שהם ממזרים אמר להם מוטב שיוציא לעז על בניו ואל יעשה עצמו כרשע לפני המקום אפילו שעה אחת תניא אידך אמר להם רבי אליעזר לחכמים אפשר שיעמוד על מקום גבוה ומשתין או ישתין בעפר תיחוח ולא יעשה עצמו שעה אחת רשע לפני המקום
הי מינייהו אמר להם ר׳ אליעזר ברישא אילימא קמייתא בתר דאמר להו איסורא הדר אמר להו תקנתא אלא הא אמר להו ברישא ואמרו ליה לא מצא מקום גבוה או עפר תיחוח מאי אמר להם מוטב שיוציא לעז על בניו ואל יעשה עצמו כרשע לפני המקום אפילו שעה אחת וכל כך למה מפני שלא יוציא שכבת זרע לבטלה דאמר ר׳ יוחנן כל המוציא שכבת זרע לבטלה חייב מיתה שנאמר (בראשית ל״ח:י׳) וירע בעיני ה׳ אשר עשה וימת גם אותו דבי ר׳ אמי אמרי כאילו שופך דמים שנאמר (ישעיהו נ״ז:ה׳) הנחמים באלים תחת כל עץ רענן שוחטי הילדים אל תקרי שוחטי אלא סוחטי רב אשי אמר כאילו עובד ע״ז כתיב הכא תחת כל עץ רענן וכתיב התם (דברים י״ב:ב׳) על ההרים הרמים ועל הגבעות ותחת כל עץ רענן:
שמואל ורב יהודה הוו קיימי אאיגרא דבי כנישתא דשף ויתיב בנהרדעא בליליא א״ל רב יהודה צריכני להשתין מים א״ל שיננא אחוז באמה והשתין מים לחוץ והיכי עביד הכי והתניא ר׳ אליעזר אומר כל
האוחז באמה ומשתין כאילו מביא מבול לעולם אמר אביי עשאוה כבולשת דתנן (ע״ז ע:) בולשת שנכנסה לעיר בשעת שלום חביות פתוחות אסורות סתומות מותרות בשעת מלחמה כולן מותרות מפני שאין להם פנאי לנסך אלמא כיון דבעיתי לא אתי לנסוכי הכא נמי כיון דבעית לא אתי להרהורי והכא מאי בעתותא איכא איבעית אימא בעתותא דליליא ודאיגרא ואיבעית אימא בעתותא דרביה ואיבעית אימא בעתותא דשכינה ואיבעית אימא שאני רב יהודה דאיכא אימתא דמריה עליה דקרי שמואל עליה דרב יהודה אין זה ילוד אשה ואיבעית אימא נשוי הוה דאמר רב נחמן אם היה נשוי מותר ואיבעית אימא כי הא אורי ליה דתני אבא בריה דרב בנימין בר חייא אבל מסייע בביצים מלמטה ואבע״א כי הא אורי ליה דא״ר אבהו א״ר יוחנן גבול יש לה מעטרה ולמעלה אסור מעטרה ולמטה מותר
(נדה יג:) אמר רב כל המקשה עצמו לדעת יהא בנדוי ולימא אסור משום דקא מגרי יצר הרע אנפשיה רבי אמי אמר נקרא כופר שכן אומנותו של יצר הרע היום אומר לו עשה כך ולמחר אומר לו לך ועבוד ע״ז והולך ועובד איכא דאמרי א״ר אמי כל המביא עצמו לידי הרהור אין מכניסין אותו למחיצתו של הקב״ה שנא׳ (בראשית לח) וירע בעיני ה׳ אשר עשה וכתיב (תהילים ה׳:ה׳) כי לא אל חפץ רשע אתה לא יגורך רע א״ר אלעזר מאי לא יגורך רע לא יגור במגורך רע ואמר רבי אלעזר מאי דכתיב (ישעיהו א׳:ט״ו) ידיכם דמים מלאו אלו המנאפים ביד כדתנא דבי רבי ישמעאל (שמות כ) לא תנאף לא תהא נואף בין ביד בין ברגל וקרי ביה לא תנאיף
ת״ר הגרים והמשחקים בתינוקות מעכבין את המשיח בשלמא גרים משום דר׳ חלבו דאמר רבי חלבו קשים גרים לישראל כספחת בבשר החי אלא משחקין בתינוקות מאי היא אילימא משכב זכור בני סקילה נינהו ואלא דרך אברים בני מבול נינהו אלא דנסבי קטנות דלאו בנות אולודי נינהו דא״ר יוסי אין בן דוד בא עד שיכלו כל הנשמות שבגוף שנאמר (ישעיהו נ״ז:ט״ז) כי רוח מלפני יעטוף ונשמות אני עשיתי:
גרסינן בפרק אין מעמידין בהמה (ע״ז כח:) אמר מר זוטרא בר טוביה אמר רב }עין }}שמרדה מותר לכוחלה בשבת סבור מינה הני מילי דשחיקי סמנין מאתמול אבל מישחק ואיתויי לא אמר להו ההוא מרבנן ורבי יעקב שמיה לדידי מיפרשא לי מיניה דמר יהודה דאפילו מישחק ואיתויי מרה״ר נמי:
ההיא אמתא דהות בי מר שמואל דמרדה לה עינא בשבתא וצווחה וליכא דמשגח בה פקע עינא ומתה נפק מר שמואל ודרש עין שמרדה מותר לכוחלה בשבת כגון מאי אמר רב יהודה כגון רירא דיצא דמא ודמעתא וקדחא ותחלת אוכלא לאפוקי סוף אוכלא ופצוחי עינא דלא אמר רבי
\end{multicols}\newpage

\newsection{דף מ}
\begin{multicols}{2}
אבא בר זוטרא אמר רבי חנינא מעלין אזנים בשבת תני רב שמואל בר יהודה ביד אבל לא בסם איכא דאמרי בסם אבל לא ביד מאי טעמא דיד מיזרף זריף (ע״ז כט.) אמר רבי יהושע בן לוי מעלין אונקלי בשבת מאי אונקלי אמר רבי אבא אצטומכא דליבא פירוש אלמע״דה (ע״ז כח.) אמר שמואל האי פידעתא סכנתא היא ומחללין עליה את השבת אמר רב ספרא האי עינבתא פרונקא דמלאך המות היא אמר רבא האי סימטא פרונקא דאשתא הוא (שם) אמר רב שמן בר אבא א״ר יוחנן מכה של חלל אינה צריכה אומד והאי אישתא צמירתא נמי כמכה של חלל דמיא מהיכן היא מכה של חלל פירש דבי אמי מן השפה ולפנים }ומי שאחזו דם מקיזין לו דם בשבת מפני הסכנה וכל מכה של חלל מחללין עליה את השבת:
גרסינן בפ״ק דע״ז (ע״ז דף יב:) א״ר חנינא הבולע נימא של מים בשבת מותר להחם לו חמין בשבת ומעשה באחד שבלע נימא של מים והתיר לו רבי נחמיה להחם }לו חמין בשבת אמר רב הונא בריה דרב יהושע אדהכי והכי ניגמע חלא }ואפשר דפקיד אביתיה
תנן התם ביומא (יומא דף פג.) ועוד אמר רבי מתיא בן חרש החושש בפיו מותר להטיל לו סם בשבת מפני שהוא ספק נפשות וכל ספק נפשות דוחה את השבת לאתויי מאי (יומא פד.) לאתויי ספק שבת אחרת היכי דמי כגון דאמדוה לתמניא יומי ויומא קמא שבת מהו דתימא ניעכביה עד לאורתא כי היכי דלא ניחול עליה תרי שבי קמ״ל דלא תניא נמי הכי מחממין חמין לחולה בשבת בין להשקותו בין להברותו ואין אומרים נמתין לו עד שיבריא אלא מחממין לו מיד וספיקו דוחה את השבת ולא ספק שבת זו אלא אפילו ספק שבת אחרת
ואין עושין דברים הללו לא ע״י }נכרים ולא ע״י קטנים אלא ע״י גדולי ישראל }ואין אומרים לעשות דברים הללו לא ע״י נשים ולא ע״י כותים מפני שמצטרפין לדעת אחרת }ומנא לן דספק נפשות להקל *}שם פה:}דאמר רב יהודה אמר שמואל דאמר קרא (ויקרא י״ח:ה׳) אשר יעשה אותם האדם וחי בהם וחי בהם ולא שימות בהם ואי חיישינן לספיקא זימנין דלא מקיים וחי בהם הלכך נציל מספיקא כי היכי דניקיים וחי בהם ולא שימות בהם וכן הלכה.
 *}שם פד:}תנו רבנן מפקחין פקוח נפש בשבת והזריז הרי זה משובח ואינו צריך ליטול רשות מב״ד כיצד ראה תינוק שנפל לים פורש מצודה ומעלהו והזריז הרי זה משובח ואינו צריך ליטול רשות מב״ד ואע״ג דקא צייד כוורי בהדיה ראה תינוק שנפל לבור עוקר חוליא ומעלהו והזריז הרי זה משובח ואינו צריך ליטול רשות מב״ד ואע״ג דקא מתקן דרגא לחול ננעלה דלת בפני תינוק הרי זה שובר את הדלת ומוציאו והזריז הרי זה משובח ואינו צריך ליטול רשות מבית דין ואע״ג דמתבר ליה ציבי:
}מי שנפל עליו מפולת ספק הוא שם ספק אינו שם ספק חי ספק מת ספק ישראל ספק כותי מפקחין עליו:
*}שם פה.}מאי קאמר הכי קאמר לא מיבעיא ספק הוא שם ספק אינו שם דאי איתיה ישראל הוא דמפקחין ולא מיבעיא ספק חי ספק מת דאי איתיה ישראל הוא דמפקחין אלא אפי׳ ספק כותי ספק ישראל מפקחין מצאוהו חי מתעסקין בו פשיטא לא צריכא דאפילו לחיי שעה מפקחין }:
מכבין ומפסיקין בפני הדליקה והזריז הרי זה משובח ואינו צריך ליטול רשות מב״ד ואע״ג דקא ממכיך מכוכי וצריכא }כדכתבנוה ביומא:
(מכילתין דף קט.) אמר מר עוקבא מי }}}שנגפה ידו או רגלו צומתו ביין ואינו חושש איבעיא להו חלא מאי א״ל ר׳
הלל לרב אשי כי הוינן בי רב כהנא אמרי חלא לא אמר }רב כהנא הני בני מחוזא כיון דמפנקי אפי׳ חמרא מסי להו ואסיר רבינא איקלע לבי רב אשי חזייה דדרכ׳ ליה חמרא אגבא דכרעיה וצמית ליה בחלא א״ל לא סבר לה מר להא דאמר רב הלל חלא לא איכא דאמרי דקא צמית ליה בחמרא א״ל ולא סבר לה מר להא דאמר רבא הני בני מחוזא כיון דמפנקי אפילו חמרא מסי להו ואסיר ומר נמי קא מפנק א״ל גב היד וגב הרגל שאני דאמר רב אדא בר מתנה אמר רב גב היד וגב הרגל הרי היא כמכה של חלל ומחללין עליה את השבת:
ת״ר רוחצים במי }גדר במי חמתן במי עסיא במי טבריא אבל לא בים הגדול ולא במי המשרה ולא בימה של סדום בים הגדול אוקימנא (דף קט:) במים }הרעים שבו אבל במים היפין שבו רוחצין ולא במי המשרה אוקימנא בדאשתהי }אבל לא אשתהי רוחצין:
\textbf{{\largeמתני׳}} אין אוכלין איזובין בשבת מפני שאינו מאכל בריאים אבל אוכל הוא את יועזר ושותה אבובראה כל האוכלין אוכל אדם לרפואה וכל המשקין שותה חוץ ממי דקלים וכוס של עיקרין מפני שהן לירוקין אבל שותה הוא ממי דקלים [לצמאו] וסך שמן של עיקרין שלא לרפואה:
\textbf{{\largeגמ׳}} (דף קי.) כל האוכלין לאתויי מאי לאתויי טחול לשינים וכרשינין לבני מעים כל המשקין לאיתויי מאי לאיתויי מי צלפין בחומץ:
אמר ליה רבינא לרב אשי מהו לשתות מי רגלים בשבת אמר ליה תנינא כל המשקין שותה ומי רגלים לא שתו אינשי:
(דף קי:) תניא מנין }}}}}לסירוס באדם שהוא אסור תלמוד לומר (ויקרא כ״ב:כ״ד) ובארצכם לא תעשו בכם לא תעשו ואפילו מסרס אחר מסרס חייב }אמר רבי חייא בר אבא אמר רבי יוחנן הכל מודים (דף קיא.) במחמץ אחר מחמץ שהוא חייב שנאמר (ויקרא ו׳:י׳) לא תאפה חמץ (ויקרא ב׳:י״א) לא תעשה חמץ במסרס אחר מסרס מנין שהוא חייב שנאמר (ויקרא כ״ב:כ״ד) ומעוך וכתות ונתוק וכרות אם על כורת חייב על נותק לא כל שכן אלא להביא נותק אחר }כורת שהוא חייב:
\textbf{{\largeמתני׳}} }}}החושש בשיניו לא יגמע בהן את החומץ אבל מטבל הוא כדרכו ואם נתרפא נתרפא החושש במתניו לא יסוך }יין וחומץ אבל סך הוא את השמן ולא שמן וורד ובני מלכים סכין על גבי מכותיהן שמן וורד שכן דרכן לסוך בחול רבי שמעון אומר כל ישראל בני מלכים הם:
\textbf{{\largeגמ׳}} החושש בשיניו לא יגמע בהן את החומץ והתניא לא יגמע ופולט אבל מגמע הוא ובולע כי תנן נמי מתניתין מגמע ופולט תנן:
תוספתא *}[דשבת פי״ג סי׳ ז׳]}אין לועסים }מוסכי
\end{multicols}\newpage

\newsection{דף מא}
\begin{multicols}{2}
בשבת אימתי בזמן שמתכוין לרפואה ואם מפני ריח הפה מותר ולא ישוף אדם סם בשיניו בשבת בזמן שמתכוין לרפואה ואם מפני ריח הפה מותר והחושש בשיניו לא יגמע בהן את החומץ ויהא פולט אבל מגמע הוא ובולע ומטבל כדרכו ואינו חושש החושש בגרונו לא יערענו בשמן }אבל נותן שמן הרבה לתוך אניגרון ובולע:
החושש במתניו לא יסוך יין וחומץ וכו׳: אמר רב הלכה }כר׳ שמעון (דף קיא:) ולא מטעמיה דאילו רבי שמעון סבר אע״ג דלא שכיח שרי ורב סבר אי שכיח אין ואי לא לא.
וכן הלכתא:
\textbf{סליקו להו שמונה שרצים} 
\textbf{{\largeואלו}} }}}קשרים שחייבין עליהם קשר הגמלין וקשר הספנין.
כשם שהוא חייב על קשורן כך הוא חייב על התירן ר״מ אומר כל קשר שהוא יכול להתירו בידו אחת אין חייבין עליו:
יש לך קשרים שאין חייבין עליהן כקשר הגמלין וכקשר הספנין.
קושרת אשה מפתחי חלוקה וחוטי סבכה ושל פסיקיא ורצועות מנעל וסנדל ונודות שמן ויין וקדרה של בשר ר״א בן יעקב אומר קושרין לפני בהמה בשביל שלא תצא (דף קיג.) וקושרין דלי בפסיקיא אבל לא בחבל ורבי יהודה מתיר כלל א״ר יהודה כל קשר שאינו של קיימא אין חייבין עליו:
\textbf{{\largeגמ׳}} יש לך }קשרים שאין חייבין עליהן חיובא הוא דליכא הא איסורא איכא ומאי ניהו (דף קיב.) קיטרא דקטרי בזממא וקטרא
\end{multicols}\newpage

\newchap{פרק \hebrewnumeral{15} ואלו קשרים}
\begin{multicols}{2}
דקטרי באצטרידא:
}מפתחי חלוקה וכו׳: מותרין לכתחלה:
ורצועות מנעל וכו׳: איתמר התיר רצועות מנעל וסנדל תני חדא חייב חטאת ותני חדא פטור אבל אסור ותניא אידך מותר לכתחלה קשיא מנעל אמנעל קשיא סנדל אסנדל מנעל אמנעל לא קשיא הא דתני חייב במנעל דאושכפי דמעשה אומן הוא וקשר של קיימא הוא והא דתני פטור אבל אסור בדרבנן דמעשה הדיוט הוא אלא שהוא קשר של קיימא ולפיכך פטור אבל אסור והא דתני מותר לכתחלה בדבני מחוזא כגון רצועות שיוצאות מגופו של מנעל וקושרין אותו על הרגל ועל השוק אחר שנועלין את המנעל דלאו מעשה אומן ולא קשר של קיימא הוא
וסנדל אסנדל לא קשיא הא דתני חייב חטאת בדטייעי דקטרי אושכפי דהוא נמי קשר של קיימא }כגון האי נמי דקרו ליה השתא תאסמה דעביד ליה לפי מדת איסתויר׳ }דכרעיה וקטר ליה קיטרא מעלייתא ואם התירן בשבת חייב חטאת והאי דקתני פטור אבל אסור בדחומרתא דקטרי אינהו והיכי דמי דחומרתא }מיכנפא כולהו רצועות ומעיילן בחומרתא }דכי בעי מעיק הדר קטיר מאבראי דחומרתא וכי בעי מרווח והאי דקתני מותר לכתחלה בסנדל דנפקי ביה }בתרי דזמנין דנפיק ביה האי ומרווח ליה וזמנין דנפיק ביה האי ומעיק ליה והאי לאו קשר של קיימא הוא ולפיכך מותר לכתחלה
}}(ערובין ק:) אמר רב אמי בר אבא אמר רב אשי אסור להלוך על גבי עשבים בשבת שנאמר (משלי י״ט:ב׳) ואץ ברגלים חוטא תני חדא אסור להלוך על גבי עשבים בשבת ותניא אידך מותר לא קשיא }כאן בימות החמה כאן בימות הגשמים איבעית אימא הא והא בימות הגשמים ולא קשיא הא דאיכא שדאכי והא דליכא שדאכי ואיבעית אימא הא והא דאיכא שדאכי ולא קשיא הא דסאים מסאני והא דלא סאים מסאני ואיבעית אימא הא והא דסאים מסאני ולא קשיא הא דאית ליה עקוסא והא דלית ליה עקוסא והשתא }כולהו שריאין
(שבת קיב.) ר׳ ירמיה הוה קא אזיל בתריה דרבי אבהו בכרמלית איפסיקא ליה רצועה דסנדליה א״ל שקול גמי [לח] דחזי למיכלא דבהמה וכרוך עילויה אביי הוה קאים קמיה דרב יוסף בחצר איפסקא ליה רצועה דסנדליה א״ל מה איעבד ליה א״ל שבקיה ומאי שנא מדרבי ירמיה התם לא מינטר הכא מינטר:
ר׳ אליעזר בן יעקב אומר קושרין לפני בהמה בשביל שלא תצא ואע״ג דאית ליה תרי איסורי מהו דתימא (דף קיג.) חד מינייהו בטולי בטליה קמשמע לן אמר רב יוסף אמר רב יהודה אמר שמואל הלכה כר״א בן יעקב:
קושרין דלי וכו׳: א״ר אבא א״ר חייא בר אבא אמר רב מביא אדם }חבל של גרדי מתוך ביתו וקושרו בפרה ובאבוס }}ואמר רב נחמן אמר שמואל }אם כלי }}קיואי הוא מותר לטלטלו בשבת ואוקימנא בחבל של גרדי דהיינו כלי קיואי:
\textbf{{\largeמתני׳}} }}}מקפלין את הכלים אפילו ארבעה וחמשה פעמים ומציעין את המטות מלילי שבת לשבת אבל לא משבת למוצאי שבת רבי ישמעאל אומר
\end{multicols}\newpage

\newsection{דף מב}
\begin{multicols}{2}
מקפלין את הכלים ומציעין את המטות }מיום הכפורים לשבת וחלבי שבת קריבין ביום הכפורים ולא של יום הכפורים בשבת ר״ע אומר לא של שבת קריבין ביום הכפורים ולא של יום הכפורים קריבין בשבת:
\textbf{{\largeגמ׳}} מקפלים את הכלים וכו׳ דבי רבי ינאי אמרו }ל״ש אלא באדם אחד אבל בשני בני אדם לא ובאדם אחד נמי לא אמרן אלא בחדשים אבל בישנים לא ובחדשים נמי לא אמרן אלא בלבנים אבל בצבועין לא ולא אמרן אלא שאין לו להחליף אבל יש לו להחליף לא:
תאנא של בית רבן גמליאל לא היו מקפלין כלי לבן שלהם מפני שהיה להם להחליף:
אמר רב הונא יש לו להחליף יחליף אין לו }}להחליף ישלשל בגדיו משום שנא׳ (ישעיהו נ״ח:י״ג) וכבדתו מעשות דרכיך ממצוא חפצך וכבדתו שלא תהא מלבושך של שבת כמלבושך של חול כי הא דרבי [יוחנן] קרי למאניה מככדותי:
(דף קיד.) א״ר יוחנן מנין לשינוי בגדים מן התורה שנאמר (ויקרא ו׳:ד׳) ופשט את בגדיו ולבש בגדים אחרים ותאנא דבי ר׳ ישמעאל בגדים שבישל בהם קדרה לרבו אל ימזוג בהן כוס לרבו אמר רבי חייא בר אבא אמר רבי יוחנן }גנאי הוא לתלמיד חכם שיצא במנעלים המטולאים והא רבי חייא נפיק אמר מר זוטרא בריה דרב נחמן בטלאי על גבי טלאי והני מילי בעילאי }אבל בגילדאי לית לן בה }ובעילאי נמי לא אמרן אלא בימות החמה אבל בימות הגשמים לא ואמר רבי חייא בר אבא אמר ר׳ יוחנן כל ת״ח שנמצא רבב על בגדו חייב מיתה שנא׳ (משלי ח׳:ל״ו) כל משנאי אהבו מות אל תיקרי משנאי אלא משניאי רבינא אמר רבד איתמר ולא פליגי הא בלבושא הא בגלימא.
פירוש רבב כגון קירא וזיפתא וכיוצא בו ופירוש רבד דם יבש וטעמו של דבר כדי שלא יראה כדם נדה ויבא לידי חשד:
א״ר יוחנן איזהו ת״ח }שממנין אותו פרנס על הצבור כל ששואלין ממנו דבר הלכה בכל מקום ואומרה ואפילו במס׳ כלה א״ר יוחנן איזהו ת״ח שבני עירו מצווין לעשות לו מלאכתו זה שמניח חפציו ועוסק בחפצי שמים למאי נפקא מיניה למיטרח ליה בריפתא וגרסינן ביומא (יומא דף עב:) ר״י רמי כתיב (דברים י׳:א׳-ב׳) ועשית לך ארון עץ וכתיב (שמות כ״ה:י׳) ועשו ארון עצי שטים מכאן לתלמיד חכם שבני עירו מצווין לעשות מלאכתו (שמות כ״ה:י׳) מבית ומחוץ תצפנו אמר רבה כל תלמיד חכם שאין תוכו בברו אינו ת״ח אביי ואיתימא ר׳ אבא אמר נקרא תועבה שנאמר (איוב ט״ו:ט״ז) אף כי נתעב ונאלח
}אמר רבי שמואל בר נחמני א״ר יונתן מאי דכתיב (משלי י״ז:ט״ז) למה זה מחיר ביד כסיל }וגו׳ אוי לת״ח שעוסק בתורה ואין בו יראת שמים א״ר ינאי חבל על דלית ליה ביתא ותרעא לביתא עביד אמר להו רבא לרבנן במטותא מינייכו לא תירתון תרתי גיהנם א״ר יהושע בן לוי מאי דכתיב (דברים ד׳:מ״ד) וזאת התורה אשר שם משה זכה נעשית לו סם חיים לא זכה נעשית לו סם המות
(במכילתין דף קיד.) וא״ר יוחנן איזהו ת״ח כל ששואלין אותו בדבר הלכה בכל מקום ואומרה ואמרינן למאי הלכתא למנוייה פרנס אי במסכתא באתריה ואי בכולי תלמודא }בריש מתיבתא
(דף קיג.) }}}מעשות דרכיך שלא יהא הלוכך של שבת כהלוכך של חול מאי היא דלא ליפסע פסיעה גסה (דף קיג:) כדבעיא מיניה רבי מרבי ישמעאל ברבי יוסי מהו לפסוע פסיעה גסה בשבת אמר ליה וכי בחול מי התירו שאני אומר בחול נמי אסור דאמר מר פסיעה גסה נוטלת אחת מחמש מאות ממאור עיניו של אדם במאי מהדר לה בקדושא דבי שמשי
בעא מיניה רבי מרבי ישמעאל ברבי יוסי מהו לאכול אדמה בשבת אמר לו אף בחול אסור מפני שהוא מלקה }אמר רבי אמי כל האוכל מעפרה של בבל כאילו אוכל }מבשר אבותיו ויש אומרים כאילו אוכל שקצים ורמשים דכתיב (בראשית ז׳:כ״ג) וימח את כל היקום וגו׳ אמר רבי שמעון בן לקיש למה נקרא שמה שנער שכל מתי מבול ננערו לשם א״ר יוחנן למה נקרא שמה מצולה שכל מתי מבול נצללו לשם }ובודאי שקצים ורמשים דאמרי אתמחויי אתמחו }[ואסרי] כיון דמילקא גזרו בה רבנן כי הא דההוא גברא דאכל גרגשתא ואכיל תחלי וקדחה ליה תחלי
לליביה:
ומיהו למירהט לבי כנשתא בשבתא שפיר דמי דגרסי׳ בפרק מאימתי (ברכות דף ו:) א״ר זירא מריש כי הוה חזינא רבנן דהוו רהטי בשבתא לפירקא אמינא קא מחללי רבנן שבתא כיון דשמעיתא להא דאמר רבי תנחום אמר ר׳ יהושע בן לוי לעולם ירוץ אדם לדבר מצוה ואפילו בשבת שנאמר (הושע יא) אחרי ה׳ ילכו כאריה ישאג אנא נמי הוה רהיטנא
(מכילתין קיג:) אמר רב הונא היה מהלך בשבת ופגע באמת המים אם היה יכול להניח רגלו ראשונה קודם שיעקור רגלו שניה מותר ואם לאו אסור מתקיף לה רבא היכי ליעביד ליקיף קא מפיש בהילוכא ליעביר זימנין דמיתווסן מאניה במיא ואתי לידי סחיטה אלא בהא כיון דלא אפשר פסע ליה פסיעה גסה וקופץ לאמת המים ושפיר דמי וכן הלכתא
(דף קיג.) ממצוא חפצך חפציך אסורין חפצי שמים מותרין ודבר דבר שלא יהא דיבורך של שבת כדיבורך של חול דבר דיבור אסור הרהור מותר:
\textbf{סליקו להו ואלו קשרים} 
(דף קטו.) \textbf{{\largeכל}} }}}כתבי הקודש מצילין אותן מפני הדליקה בין שקורין בהן בין שאין קורין בהם ואע״פ שכתובין בכל לשון טעונין גניזה ומפני מה אין קורין בהן מפני ביטול בית המדרש מצילין תיק הספר עם הספר ותיק התפילין עם התפילין אע״פ שיש בתוכו מעות להיכן מצילין אותן למבוי שאינו מפולש בן בתירה אומר אף למפולש:
\textbf{{\largeגמ׳}} כל כתבי הקדש וכו׳: איתמר היו }כתובין תרגום ובכל לשון רב הונא אמר
\end{multicols}\newpage

\newchap{פרק \hebrewnumeral{16} כל כתבי}
\end{multicols}\newpage

\newsection{דף מג}
\begin{multicols}{2}
אין מצילין אותן מפני הדליקה ורב חסדא אמר מצילין אותן מפני הדליקה והלכתא כרב הונא דרב חסדא }תלמיד הוה בפני רב הונא ואין הלכה כתלמיד בפני הרב (אבל) היו כתובים בסם בסקרא בקומוס ובקנקנתום }}אין מצילין אותם מפני הדליקה:
(דף קטו:) ת״ר הברכות והקמיעין אע״פ שיש בהן אותיות של שם ומענינות הרבה של תורה אין מצילין אותן מפני הדליקה אלא נשרפין במקומן מכאן אמרו כותבי ברכות כשורפי }ס״ת ומעשה בא׳ שהיה כותב ברכות בצידן ובאו והודיעו את רבי ישמעאל והלך ר׳ ישמעאל (בצידן) }לבקרו וכשהיה עולה בסולם הרגיש בו נטל טומוס של ברכות ושקעו בספל של מים ובלשון הזה אמר לו גדול עונש האחרון יותר מן הראשון:
}ספר תורה שיש בו ללקט שמנים וחמש אותיות מתוך תיבות שלימות ואפילו בכללן (בראשית לא) יגר שהדותא א״נ פרשה שאין בה שמנים וחמש אותיות אלא שיש בה אזכרות כגון (במדבר י׳:ל״ה) ויהי בנסוע הארון מצילין אותן מפני הדליקה
}(דף קטז.) הגליונים וספרי מינין אין מצילין אותן מפני הדליקה ר׳ יוסי אומר אף בחול קודר האזכרות שבהן וגונזן והשאר שורפן אמר רבי טרפון אקפח את בני שאם יבואו לידי שאני שורפן ואת האזכרות שבהן שאפי׳ רדף אחריו רודף להרגו ונחש }אחריו להכישו נכנס לבית ע״ז ואין נכנס לבתיהן של אלו שאלו מכירין וכופרין והללו אין מכירין ועליהם הכתוב אומר (ישעיהו נ״ז:ח׳) ואחר הדלת והמזוזה שמת זכרונך א״ר ישמעאל קל וחומר ומה לעשות שלום בין איש לאשתו אמרה תורה שמי שנכתב בקדושה ימחה על המים הללו שמטילין איבה ותחרות בין ישראל לאביהם שבשמים על אחת כמה וכמה ועליהם אמר דוד (תהלים קלט) הלא משנאיך ה׳ אשנא ובתקוממיך אתקוטט וכשם שאין מצילין אותן מפני הדליקה כך אין מצילין אותן לא מפני המפולת ולא מן המים ולא מן כל דבר המאבדן:
(דף קטז:) ומפני מה אין קורין בהן מפני ביטול בית המדרש:
אמר רב לא שנו אלא בזמן בית המדרש אבל בזמן שאין בית המדרש קורין ושמואל אמר
אפילו שלא בזמן בית המדרש אין קורין והלכתא כרב:
גרסינן בפרק השולח גט לאשתו (דף לח:) אמר *}בגמ׳ איתא רבה}רבא בהני תלת מילי }נחתי בעלי בתים מנכסיהון דמפקי עבדיהון לחירות וסיירי נכסיהון בשבתא וקבעי סעודתייהו בשבתא בעידן בי מדרשא דא״ר חייא בר אבא א״ר יוחנן שתי משפחות היו בירושלים אחת קבעה סעודתה בשבת ואחת קבעה סעודתה בערב שבת ושניהם נעקרו:
והלכתא אסור לקרות }*}ס״א בשטרי}בספרי הדיוטות בשבת:
להיכן מצילין אותן וכו׳: (דף קיז.) היכי דמי מפולש והיכי דמי שאינו מפולש אמר רב אשי שלש מחיצות ולחי אחד זה מבוי שאינו מפולש שלש מחיצות בלא לחי זהו מפולש ולית הלכתא }כבן בתירא לא במבוי ולא בחצר:
\textbf{{\largeמתני׳}} (דף קיז:) מצילין מזון שלש סעודות הראוי לאדם לאדם והראוי לבהמה לבהמה כיצד נפלה דליקה בלילי שבת מצילין מזון ג׳ סעודות בשחרית מצילין מזון שתי סעודות במנחה מצילין מזון סעודה אחת ר׳ יוסי אומר לעולם מצילין מזון שלש סעודות:
\textbf{{\largeגמ׳}} }}תניא נשברה לו חבית בראש גגו מביא כלי ומניח תחתיה ובלבד שלא יביא כלי אחר ויקלוט כלי אחר ויצרף נזדמנו לו אורחים מביא כלי אחר }ויקלוט כלי אחר ויצרף ולא יקלוט ואח״כ יזמין אלא יזמין ואח״כ יקלוט ואין מערימין בכך משום ר׳ יוסי ברבי יהודה אמרו }מערימין:
תנו רבנן }}הציל פת }נקיה אין מצילין פת הדראה הציל פת הדראה מציל פת נקיה מצילין מיום הכפורים לשבת אבל לא משבת ליום }הכפורים ואין צריך לומר משבת ליום טוב ולא משבת זו לשבת הבאה ת״ר }}שכח פת בתנור וקדש עליו היום מצילין מזון שלש סעודות ואומר לאחרים בואו והצילו לכם וכשהוא רודה לא ירדה במרדה אלא בסכין }ואע״ג דרדיית הפת חכמה היא ואינה מלאכה אפילו הכי במה דאפשר לשנויי משנינן:
אמר רב חסדא }}}לעולם ישכים אדם להוצאת שבת שנאמר (שמות ט״ז:ה׳) והיה ביום הששי והכינו את אשר יביאו לאלתר:
א״ר אבא חייב אדם לבצוע על שתי ככרות בשבת שנאמר (שמות ט״ז:כ״ב) לקטו לחם משנה אמר רב אשי חזינא ליה לרב כהנא דנקיט תרתי ובצע חדא אמר לקטו לחם משנה כתיב ר׳ זירא בצע ליה אכוליה שירותיה אמר ליה רבינא לרב אשי והא מיחזי כרעבתנותא כיון דכל יומא לא עביד והשתא הוא דעביד לא מיחזי כרעבתנותא:
רבי אמי ורבי אסי כד הוה מיקלע להו רפתא דעירובא שרו עליה המוציא אמרי כיון דאיתעבידא ביה מצוה חדא נעביד ביה מצוה אחריתי:
כיצד נפלה דליקה בלילי שבת מצילין מזון ג׳ סעודות:
ת״ר כמה סעודות חייב אדם לאכול בשבת שלש רבי חידקא אומר ארבע א״ר יוחנן ושניהם
\end{multicols}\newpage

\newsection{דף מד}
\begin{multicols}{2}
מקרא אחד דרשו (שמות ט״ז:כ״ה) ויאמר משה אכלוהו היום וגו׳ ר׳ חידקא סבר הני תלתא היום דכתיבי לבר מאורתא ורבנן סברי בהדי דאורתא והלכתא כרבנן:
(דף קיח.) תנו רבנן }}}קערות שאכלו בהן ערבית מדיחן לאכול בהן שחרית שחרית מדיחן לאכול בהן בצהרים בצהרים מדיחן לאכול בהם במנחה מן המנחה ולמעלה שוב אינו מדיח אבל כוסות וקיתוניות וצלוחיות מדיחם והולך כל היום כולו לפי שאין קבע לשתיה:
אמר רבי שמעון בן פזי א״ר יהושע בן לוי משום בר קפרא כל }}המקיים שלש סעודות בשבת ניצול משלש פורעניות מחבלו של משיח ומדינה של גיהנם וממלחמת גוג ומגוג מחבלו של משיח כתיב הכא יום וכתיב התם יום (מלאכי ג׳:כ״ד) הנה אנכי שולח לכם את אליה הנביא לפני בוא יום ה׳ מדינה של גיהנם כתיב הכא יום וכתיב התם יום עברה היום ההוא וממלחמת גוג ומגוג כתיב הכא יום וכתיב התם }(יחזקאל לא) ביום בא גוג
א״ר יוחנן משום רבי יוסי בן זימרא כל מי שמענג עצמו בשבת נותנין לו נחלה בלא מצרים שנאמר (ישעיהו נ״ח:י״ד) אז תתענג על ה׳ והרכבתיך על במתי ארץ והאכלתיך (דף קיח:) נחלת יעקב אביך לא כאברהם דכתיב ביה (בראשית י״ג:י״ז) קום התהלך בארץ לארכה ולרחבה ולא כיצחק דכתיב ביה (בראשית כ״ו:ג׳) כי לך ולזרעך אתן את כל הארצות האל אלא כיעקב דכתיב ביה (בראשית כ״ח:י״ד) ופרצת ימה וקדמה וגו׳ ר״נ בר יצחק אמר ניצול משעבוד מלכיות כתיב הכא }והאכלתיך נחלת יעקב אביך והרכבתיך על במתי ארץ וכתיב התם (דברים ל״ג:כ״ט) ואתה על במותימו תדרוך אמר רב יהודה א״ר כל המענג את השבת נותנין לו משאלות לבו שנאמר }אז תתענג על ה׳ ויתן לך משאלות לבך עונג זה איני יודע מהו כשהוא אומר (ישעיהו נ״ח:י״ג) וקראת לשבת עונג הוי אומר זה תענוג שבת:
במה מענגה אמר רבה בריה דרב יהודה בר שילת משמיה דרב }בדגים קטנים וראשי שומים ותבשיל של תרדים *}בגמ׳ וברא״ש איתא רב חייא בר אשי}רב אשי אמר אפילו דבר מועט ועשאו לכבוד שבת הרי זה עונג מאי היא אמר רב פפא כסא דהרסנא:
א״ר חייא בר אבא א״ר יוחנן כל המשמר את השכת כהלכתו אפילו עובד ע״ז כאנוש מוחלין לו שנאמר (ישעיהו נ״ו:ב׳) אשרי אנוש יעשה זאת ובן אדם יחזיק בה שומר שבת מחללו אל תיקרי מחללו אלא מחול לו:
אמר רב יהודה אמר רב אלמלי שמרו ישראל שבת ראשונה לא שלטה בהם אומה ולשון שנאמר (שמות ט״ז:כ״ז) ויהי ביום השביעי יצאו מן העם ללקוט ולא מצאו וכתיבבתריה ויבא עמלק:
}א״ר שמעון בן יוחאי אלמלי }שמרו ישראל שתי שבתות כהלכתן מיד נגאלין שנאמר (ישעיהו נ״ו:ד׳) כה אמר ה׳ לסריסים אשר ישמרו את שבתותי וכתיב בתריה והביאותים אל הר קדשי:
א״ר יוסי יהא חלקי עם אוכלי שלש סעודות בשבת:
א״ר יוסי יהא חלקי }}עם גומרי הלל בכל יום איני והאמר מר כל הקורא הלל בכל יום הרי זה מחרף ומגדף כי קאמרינן בפסוקי דזמרה ומאי ניהו (תהלים קמד) מתהלה לדוד עד (תהילים ק״נ:ו׳) כל הנשמה תהלל יה:
(דף קיט.) ר׳ חנינא }}}מיעטף וקאי בפניא }דמעלי שבתא אמר בואו ונצא לקראת מלכא רבי ינאי לבש מאניה ומכסי אמר בואי כלה בואי כלה רבה בר רב הונא איקלע לבי רבה בר רב נחמן קריבו ליה תליסר סאוי טחיי אמר להו מי הויתון ידעין דאתינא לגבייכו דטרחיתון כולי האי אמרו ליה מי עדיפת לן מינה ר׳ אבא הוה זבין בתליסר איסתרי פשיטי בשרא מתליסר טבחי ומשלים להו אצינורא דדשא ואמר להו אשור הייא אשור הייא ר׳ אבהו הוה יתיב אתכתכא דשאגא ושייף נורא רב ענן לביש גונדא דתנא (לעיל קיד.) דבי רבי ישמעאל בגדים שבישל בהם קדרה לרבו אל ימזוג בהם כוס לרבו
רב ספרא מחריך רישא רבא מלח שיבוטא רב הונא מדליק שרגא רב פפא גדיל פתילתא רבחסדא פרים סילקא רבה ורב יוסף מצלחי ציבי רבי זירא מצתת צתותי רב נחמן בר יצחק מכתף ועייל מכתף ונפיק אמר אילו מקלעי לי ר׳ אמי ור׳ אסי מי לא מכתיפנא ועיילנא קמייהו ר׳ אמי ור׳ אסי הוו מכתפי ועיילי מכתפי ונפקי אמרי אילו מיקלע גבן ר׳ יוחנן מי לא מכתיפנא קמיה:
בעא מיניה רבי מר׳ ישמעאל ברבי יוסי עשירים שבא״י במה הם זוכין א״ל בשביל שמעשרין דכתי׳ (דברים יד) עשר תעשר עשר בשביל שתתעשר עשירים שבבבל במה הן זוכין בשביל שמכבדין
את התורה ושבשאר ארצות במה הם זוכין בשביל שמכבדין את השבת א״ר חייא בר אבא פעם אחת נתארחתי אצל בעל הבית אחד בלודקיא והביאו לפניו שולחן של זהב משוי ששה עשר בני אדם וששה עשר שלשלאות של כסף קבועות בו וקערות וקיתוניות וכוסות וצלוחיות קבועות בו ועליו כל מיני מאכל וכל מיני מגדים כשהן מניחין אותה אומרים (תהלים כד) לה׳ הארץ ומלואה וכשהן מסלקים אותה אומרים (תהילים קט״ו:ט״ז) השמים שמים לה׳ והארץ נתן לבני אדם אמרתי לו בני במה זכית לכך אמר לי קצב הייתי וכל בהמה נאה שמצאתי אמרתי זו לכבוד שבת אמרתי לו בני אשריך וברוך המקום שזיכך לכך:
גרסינן בפרק יום טוב שחל להיות }ערב שבת (דף טז.) תאני רב תחליפא אבוה דרבנאי חוזאה ואמרי לה אחוה דרבנאי חוזאה }}כל מזונותיו של אדם קצובין לו מר״ה ועד ר״ה חוץ מהוצאות שבתות וי״ט והוצאת בניו לתלמוד תורה שאם מוסיף מוסיפין לו ואם פיחת פוחתין לו:
תניא אמרו על שמאי הזקן שכל ימיו היה אוכל לכבוד שבת מצא בהמה נאה אמר זו לשבת למחר מצא אחרת נאה הימנה מניח את השניה ואוכל את הראשונה אבל הלל הזקן מדה אחרת היתה בו שכל מעשיו היו לשם שמים ואומר (תהלים סח) ברוך ה׳ יום יום תנ״ה שמאי אומר מחד בשביך לשבתיך הלל אומר ברוך ה׳ יום יום:
(מכילתין דף קיט.) אמר ליה ריש גלותא לרב המנונא מאי דכתיב (ישעיה נח) לקדוש ה׳ מכובד א״ל זה יום הכפורים שאין בו לא אכילה ולא שתיה אמרה תורה כבדהו בכסות נקיה וכבדתו רב אמר להקדים ושמואל אמר לאחר אמרו ליה דבי ר׳ ינאי לרב פפא כגון אנן דשכיח לן בכל יומא ויומא בשרא וחמרא במאי נייקריניה אמר להו נישנייה אי רגיליתו לאקדומי אחריתו ואי רגיליתו לאחורי אקדימו רב ששת בקייטא מותיב להו לרבנן היכא דמטי שמשא בסיתוא מותיב להו היכא דמטי טולא כי היכי דלוקמן הייא ר׳ זירא הוה (דף קיט:) מהדר אזוגי }דרבנן ואמר להו במטותא מינייכו דלא תחלוניה:
אמר רב ואיתימא רבי יהושע בן לוי אפילו יחיד }}}המתפלל בערב שבת צריך לומר ויכולו דאמר רב המנונא כל האומר ויכולו כשהוא מתפלל מעלה עליו הכתוב כאילו נעשה שותף להקב״ה במעשה בראשית שנאמר }ויכולו אל תיקרי ויכולו אלא ויכלו:
אמר רב חסדא אמר מר עוקבא המתפלל בערב שבת ואומר ויכולו שני מלאכי השרת המלוין לו לאדם מניחין את ידיהם על ראשו ואומרים לו (ישעיהו ו׳:ז׳) וסר עונך }וחטאתך תכופר:
תניא ר׳ יוסי בר יהודה אומר שני מלאכי השרת מלוין לו לאדם בערב שבת מבית הכנסת לביתו אחד טוב ואחד רע וכשבא לביתו ומצא נר דלוק ושלחן ערוך ומטה מוצעת מלאך טוב אומר יהי רצון שיהיה לשבת אחרת כך ומלאך רע אומר בעל כרחו אמן ואם לאו מלאך רע אומר יהי רצון שיהיה לשבת אחרת כך ומלאך טוב אומר בעל כרחו אמן א״ר אלעזר לעולם יסדר אדם שולחנו בערב שבת אע״פ שאינו צריך אלא לכזית:
אמר רבי חנינא לעולם יסדר אדם שולחנו למוצאי שבת אע״פ שאינו צריך אלא לכזית:
חמין במ״ש מלוגמא פת חמה במ״ש מלוגמא ר׳ אבהו הוו עבדין ליה עיגלא תילתא באפוקי שבתא הוה אכיל מיניה כולייתא כי גדל אבימי בריה א״ל לאבוה למה לך לאפסודי כולי האי }נשבוק עד מעלי שבתא שבקיה ואתא אריא ואכליה:
אמר רב יהודה בריה דרב שמואל בר שילת משמיה דרב אין הדליקה מצויה אלא בבית שיש בה חלול שבת שנאמר (ירמיהו י״ז:כ״ז) ואם לא תשמעו אלי לקדש את יום השבת ולבלתי שאת משא [וגו׳] והצתי אש בשעריה ואכלה ארמנות ירושלים ולא תכבה מאי ולא תכבה אמר רב נחמן בשעה שאין בני אדם מצויין לכבותה:
\textbf{{\largeמתני׳}} (דף קכ.) }}מצילין }סל מלא ככרות בשבת ואע״פ שיש בו מאה סעודות ועיגול של דבילה וחבית של יין ואומר לחבריו בואו }והצילו לכם ואם היו פקחין עושין עמו חשבון אחר השבת להיכן מצילין אותו }לחצר המעורבת בן בתירא אומר אף לשאינה מעורבת:
\textbf{{\largeגמ׳}} מצילין סל מלא ככרות והא תנא רישא ג׳ סעודות ותו לא אמר רב הונא לא קשיא כאן בבא להציל כאן בבא לקפל פירוש }להציל בכלי אחד לקפל בב׳ ובג׳ כלים ר׳ אבא בר׳ זבדא אמר רב אידי ואידי בבא לקפל ול״ק כאן לאותה
\end{multicols}\newpage

\newsection{דף מה}
\begin{multicols}{2}
חצר כאן לחצר אחרת והלכתא כרב הונא:
בעי רב הונא בריה דרב יהושע פירש טליתו וקיפל והניח וקיפל והניח מאי כבא להציל דמי או כבא לקפל דמי מדאמר רבא אטעייה }רב שזבי לרב חסדא ודרש ובלבד שלא יביא כלי מחזיק יותר מג׳ סעודות ש״מ כבא להציל דמי ושפיר דמי:
א״ל ר״נ בר יצחק לרבא מאי טעותא א״ל דתנן ובלבד שלא יביא כלי אחר ויקלוט כלי אחר ויצרף כלי אחר הוא דלא אבל בההוא מנא כמה דבעי מציל:
ואומר לאחרים בואו והצילו לכם וכו׳: חשבון מאי עבידתיה נימרו ליה מהפקירא קא זכינא אמר רבא הכא בירא שמים עסקינן דלא ניחא ליה דליתהני מאחרים ובחנם נמי לא ניחא ליה דליטרח וה״ק אם היו פקחים דידעי דכי האי גונא לאו שכר שבת הוא עושין עמו חשבון לאחר השבת:
\textbf{{\largeמתני׳}} ולשם הוא מוציא כל כלי תשמישו ולובש כל מה שהוא יכול ללבוש ועוטף כל מה שהוא יכול לעטוף ר׳ יוסי אומר י״ח כלים וחוזר ולובש ומוציא ואומר לאחרים בואו והצילו עמי:
והלכתא כת״ק: \textbf{{\large(מתני׳)}} }}ר״ש בן ננס אומר פורסין עור של גדי ע״ג שידה תיבה ומגדל שאחז בהן האור מפני שהוא מחרך ועושין מחיצה בכל הכלים בין מלאים בין ריקנים בשביל שלא תעבור הדליקה ר׳ יוסי אוסר בכלי חרס חדשים מלאים מים שהן מתבקעין ומכבין את הדליקה מפני שאינן יכולין לקבל האור:
\textbf{{\largeגמ׳}} ולית הלכתא כר׳ יוסי דקסבר גרם כיבוי אסור:
}אמר רב יהודה }טלית שאחז בה האור פושטה ומתכסה בה ואם }כבתה כבתה וכן ס״ת שאחז בו האור פושטו וקורא בו ואם כבה כבה:
(דף קכ:) ת״ר }}}נר שע״ג טבלא מנער את הטבלא והיא נופלת מאליה ואם כבתה כבתה אמרי דבי ר׳ ינאי לא שנו אלא בשוכח אבל במניח נעשה בסיס לדבר האסור ונר }שאחורי }}הדלת }אסור לפתוח ולנעול כדרכה דאביי ורבא דאמרי תרוייהו מודה ר״ש בפסיק רישיה ולא ימות ואסור לפתוח הדלת כנגד המדורה בשבת ואפילו ברוח מצויה:
\textbf{{\largeמתני׳}} (דף קכא.) נכרי שבא }}}לכבות אין }אומרים לו כבה ואל תכבה מפני
שאין שביתתו עליהם }קטן שבא לכבות אין שומעין לו מפני ששביתתו עליהם:
\textbf{{\largeגמ׳}} א״ר אמי בדליקה התירו לומר כל המכבה אינו מפסיד ת״ר מעשה שנפלה דליקה בחצרו של יוסף בן סימאי ובאו אנשי גיסתרא של צפורי לכבותה מפני שאפוטרופוס של מלך היה ולא הניחן מפני כבוד השבת ונעשה לו נס וירדו גשמים וכיבוה ולערב שיגר לכל אחד ואחד מהן שתי סלעים ולאפרכוס שבהם חמשים דינרין וכששמעו חכמים בדבר אמרו לא היה צריך לכך שהרי שנינו נכרי שבא לכבות אין אומרים לו כבה ואל תכבה:
\textbf{{\largeמתני׳}} כופין }}}קערה ע״ג הנר בשביל שלא תאחוז בקורה ועל צואה של קטן ועל עקרב שלא תשוך א״ר יהודה מעשה בא לפני רבן יוחנן בן זכאי בערב ואמר חושש אני לו מחטאת:
\textbf{{\largeגמ׳}} ועל צואה של קטן אוקימנא (דף קכא:) בצואה של תרנגולין בשביל הקטן שלא יוזק בהן ובחצר אחרת אבל צואה של תרנגולין ושל בני }אדם ובאותה חצר מותר לכבדה ולהוציאה לבית הכסא דגרף של רעי מטלטלינן ליה ומפקינן ליה:
}}}חמשה }נהרגין בשבת ואלו הן זבוב שבמצרים וצרעה שבנינוה ועקרב }שבהדייב ונחש שבא״י וכלב שוטה בכל מקום ושאר כל המזיקין אם היו רצין אחריו מותר להרגן לדברי הכל:
ת״ר נזדמנו לו נחשים ועקרבים בשבת והרגן בידוע שנזדמנו לו להרגן לא הרגן בידוע שנזדמנו לו להרגו ונעשה לו נס מן השמים אמר עולא ואיתימא רבה בר בר חנה בנישופין בו אמר רב יהודה רוק דורסו לפי תומו אמר רב ששת נחש דורסו לפי תומו אמר רב קטינא עקרב דורסו לפי
\end{multicols}\newpage

\newsection{דף מו}
\begin{multicols}{2}
תומו אבא בר מרתא דהוא אבא בר מניומי הוו מסקי ליה זוזי }בי ריש גלותא אתו קא מצערו ליה הוה שדי רוקא }קמיה אמר להו ריש גלותא אייתו מאנא סחיפו עלייהו אמר להו לא צרכיתו הכי אמר רב יהודה רוק דורסו לפי תומו אמר להו צורבא מרבנן הוא שבקוהו:
א״ר אבא בר כהנא א״ר יוחנן }}}}פמוטות של בית רבי מותר לטלטלן בשבת א״ל ר׳ זירא בניטלין ביד אחת או בשתי ידיו (דף קכב.) א״ל כאותן של בית אביך וא״ר אבא בר כהנא א״ר יוחנן קרונות של בית רבי מותר לטלטלן בשבת א״ל ר׳ זירא בניטלין באדם אחד או בב׳ בני אדם א״ל כאותן של בית אביך:
\textbf{{\largeמתני׳}} נכרי שהדליק את }הנר }}משתמש לאורו ישראל ואם בשביל ישראל אסור:
מילא }מים להשקות לבהמתו
משקה אחריו ישראל ואם בשביל ישראל אסור עשה כבש לירד בו ירד אחריו ישראל ואם בשביל ישראל אסור מעשה ברבן גמליאל וזקנים שהיו באים בספינה ועשה נכרי כבש }לירד בו וירדו אחריו ר״ג וזקנים:
\textbf{{\largeגמ׳}} תנו רבנן נכרי שליקט }עשבים בשבת מאכיל אחריו ישראל ואם בשביל ישראל אסור מילא מים להשקות בהמתו משקה אחריו ישראל בד״א בשאין מכירו אבל במכירו אסור גזירה שמא ירבה בשבילו ודוקא מים ועשבים דחיישינן להכי אבל הדלקת הנר וכבש דליכא למיגזר שמא ירבה בשבילו כדאמרינן נר לאחד נר למאה אע״פ שהוא מכירו אם עשה בשבילו מותר (א״ר חנינא מעמיד אדם בהמתו ע״ג עשבים אע״פ שהם מחוברים ואינו עובר משום (שמות כ״ג:י״ב) למען ינוח שורך וחמורך כדאיתא במכילתא יכול לא יניחנה לעקור ת״ל למען ינוח ואין זה נייח שפעמים שבהמה רעבה אם אתה אוסר לה. ואין מעמידין ע״ג מוקצה)
והאי דקתני נכרי שליקט עשבים בשבת מאכיל אחריו ישראל אוקימנא בדקאי לה באפה ואזלא היא ואכלה אבל אוקומה עלייהו אסיר דהוי ליה }מוקצה ואמר רב הונא }}}מעמיד אדם בהמתו ע״ג עשבים בשבת אבל לא ע״ג מוקצה ואי קאים לה באפה ואזלא היא ואכלה מן המוקצה שפיר דמי ת״ר עיר }}}שדרין בה ישראל ונכרים והיתה בה מרחץ המרחצת בשבת אם רוב נכרים מותר לרחוץ במ״ש מיד ואם רוב ישראל ימתין בכדי שיחמו חמין וכן מחצה על מחצה כרוב ואדעתא דרובא קא מחממי ונר הדלוק במסיבה אם רוב נכרים מותר להשתמש לאורה ואם רוב ישראל אסור מחצה על מחצה אסור מאי טעמא דאדעתא דרובא קא מדלקי:
(דף קכב:) שמואל איקלע לבי אבי תורן אתא ההוא נכרי וקא מדליק שרגא אהדרינהו שמואל לאפיה כיון דחזא דאייתי שטרא וקרי בהו אמר אדעתא דנפשיה קא מדליק הדר אהדרינהו לאפיה }בי שרגא:
גרסינן בע״ז (דף כא:) תניא רבן שמעון בן גמליאל }}אומר לא ישכיר אדם }מרחצאו לנכרי מפני שהיא נקראת על שמו ונכרי זה עושה בו מלאכה בשבתות וי״ט אבל שדהו לנכרי שרי דאמרי אריסותיה הוא דקא עביד מרחץ נמי אריסותיה קא עביד אריסותא למרחץ לא עבדי אינשי:
הנהו מוריקי דנכרי נקיט בשבתא וישראל נקיט בחד בשבתא אתו לקמיה דרבא שרא להו רבא איתיביה רבינא לרבא ישראל ונכרי }שקבלו
\end{multicols}\newpage

\newsection{דף מז}
\begin{multicols}{2}
שדה בשותפות לא יאמר ישראל לנכרי טול אתה חלקך בשבת ואני חלקי בחול ואם התנו מעיקרא מותר ואם באו לחשבון אסור אכסיף רבא לסוף איגלאי מילתא דהתנו מעיקרא הוה:
\textbf{סליקו להו כל כתבי} 
(דף קכב:) \textbf{{\largeכל}} הכלים }}ניטלין בשבת ודלתותיהן עמהן אף על פי שנתפרקו בשבת ואינן דומין לדלתות הבית לפי שאינן מן המוכן:
\textbf{{\largeגמ׳}} אע״פ שנתפרקו בשבת ולא מיבעיא נתפרקו בחול אדרבה נתפרקו בשבת מוכנין אגב אביהן נתפרקו בחול אין מוכנין על גב אביהן אלא אמר אביי הכי קאמר כל הכלים }}הניטלין בשבת ודלתותיהן ניטלים עמהן ואף על פי }שנתפרקו בחול ניטלין בשבת ואינן דומין לדלתות הבית לפי }שאינו מן המוכן שדלתות הבית קבועין בו ולא הוכנו לטלטל:
תנו רבנן דלת של שידה ושל תיבה ושל מגדל ניטלין ולא מחזירין של לול של תרנגולין לא ניטלין ולא }מחזירין גזירה שמא יתקע:
\textbf{{\largeמתני׳}} נוטל אדם קורנס לפצוע בו אגוזים קרדום לחתוך בו את הדבילה מגירה לגרר בה את הגבינה מגריפה }לגרף בה את הגרוגרו׳ את הרחת ואת המזלג
\end{multicols}\newpage

\newchap{פרק \hebrewnumeral{17} כל הכלים}
\begin{multicols}{2}
לתת עליו }פת לקטן את הכוש ואת הכרכר לתחוב בו מחט של יד ליטול }בו את הקוץ ושל סקאים לפתוח בו את הדלת:
\textbf{{\largeגמ׳}} האי קורנס בין קורנס של אגוזים בין קורנס של נפחים בין קורנס של זהבים אבל }קורנס של בשמים אסור (דף קכג.) כדרב חייא בר אשי אמר ר׳ יוחנן דכיון דקפיד עליה מייחד ליה מקום כסיכי זיירי ומזורי דאמר רב חיננא בר שלמיה משמיה דרב הכל מודים בסיכי זיירי ומזורי דכיון דקפיד עלייהו מייחד להו מקום:
ירושלמי זיירי }דעצר בהו מזורי דחביט בהו וסיכי הן יתדות:
את הכוש ואת הכרכר וכו׳: תנו רבנן פגה שטמנה בתבן וחררה שטמנה בגחלים אם מגולה מקצתה מבעוד יום מותר לטלטלה בשבת ואם לאו אסור }אלעזר בן תדאי אומר תוחבין בהן בכוש או בכרכר והן ננערות מאיליהן אמר רב נחמן הלכה }}כאלעזר בן תדאי:
מחט ליטול בו את הקוץ: האי מחט (היכי דמי) מחט שלימה אבל מחט שניטל עוקצה או חורה אסור לטלטלה כדרבא:
}}}אסוכי ינוקא בשבתא רב נחמן אסר ורב ששת שרי וקיי״ל }הלכה כרב נחמן בדיני וכרב ששת באיסורי:
\textbf{{\largeמתני׳}} (דף קכג:) }}}קנה של זיתים אם יש קשר בראשו מקבל טומאה ואם לאו אינו מקבל טומאה בין כך ובין כך ניטל בשבת רבי יוסי אומר כל הכלים ניטלין בשבת חוץ מן המסר הגדול ומן היתד של מחרישה }(דף קכד.) כל הכלים ניטלין לצורך ושלא לצורך רבי נחמיה אומר אין ניטלין אלא לצורך:
\textbf{{\largeגמ׳}} אמר רב נחמן האי אוכלא דקצרי כיתד של מחרישה דמי פירוש כלי חרס מנוקב מעשנין בו את הכלים ומזלפין בו מים }פ״א אוכלא דקצרי אבן גדולה שנותן עליה הכובס את היריעה כשהוא רוצה ללבנה והיא שרויה במים ומכה עליה במזורה ופירוש׳ קמא דוקא דגרסינן בפרק חלק (דף צב:) עד דשוויה לנפשיה כי אוכלא דקצרי דשמעת מינה דכלי מנוקב הוא אמר אביי האי חרבא דאושכפא וסכינא דאשכבתא וחצינא דנגרי
\end{multicols}\newpage

\newsection{דף מח}
\begin{multicols}{2}
כיתד של מחרישה דמי:
כל הכלים ניטלין לצורך מאי לצורך ומאי שלא לצורך אמר רבא דבר שמלאכתו להיתר בין לצורך גופו בין לצורך מקומו }}אפי׳ מחמה לצל שרי דבר שמלאכתו לאיסור לצורך גופו ולצורך מקומו אין מחמה לצל לא ואף רב (דף קכד:) סבר לה להא דרבא דאמר רב מרא שלא יגנב זהו טלטול שלא לצורך }ואסור טעמא דשלא יגנב אבל לצורך גופו ולצורך מקומו מותר רב מארי בר רחל הוו ליה הנהו בי סדיאתא בשמשא אתא לקמיה דרבא א״ל מהו לטלטולינהו א״ל שרי דהא חזו לך אית לי אחריני חזו לאורחין אית לי נמי לאורחין א״ל גלית דעתיך דלאו כוותי סבירא לך אלא כאביי סבירא לך לכולי עלמא שרי ולך אסיר }אלמא חזו למיתב עלייהו לכל דעייל ואתי
אמר רבי אבא א״ר חייא בר אשי אמר רב }}}מכבדות של מילת }מותר לטלטלן בשבת אבל של תמרה לא וכן אמר רבי אלעזר והני מילי מחמה לצל אבל לצורך גופו או לצורך מקומו אפילו של תמרה מותר לטלטלן דקיימא לן דכל דבר שמלאכתו לאיסור מותר לטלטלן בין לצורך גופו בין לצורך מקומו אבל מחמה לצל לא.
פירוש מכבדות של מילת כגון זנב של שועל וכיוצא בו שמכבדין בהן כלי מילת *}[לשון הרי״ף]}והאידנא דקי״ל כר״ש דאמר דבר שאין מתכוין מותר הויין להו מכבדות של תמרה דבר שמלאכתו להיתר ואפילו מחמה לצל שרי:
\textbf{{\largeמתני׳}} }}}כל הכלים הניטלין בשבת שבריהן ניטלין עמהן ובלבד שיהו עושין מעין מלאכה שברי עריבה לכסות בהן את החבית שברי זכוכית לכסות בו את פי הפך רבי יהודה אומר ובלבד שיהו עושין מעין מלאכתן שברי עריבה לצוק לתוכן את המקפה ושל זכוכית לצוק לתוכן שמן:
\textbf{{\largeגמ׳}} אמר רב יהודה אמר שמואל מחלוקת }שנשברו בשבת דמר סבר מוכן הוא ומותר ומר סבר נולד הוא ואסור אבל נשתברו מע״ש דברי הכל מותרין הואיל והוכנו למלאכה אחרת מבעו״י והלכתא כת״ק.
ואמר בעל הלכות משמיה דרב צמח גאון
דהלכתא כרבי יהודה דקי״ל ר״מ ורבי יהודה הלכה כרבי יהודה ומסתברא לן ה״מ היכא }דפליג בהדיא אבל היכא דדברי רבי מאיר סתם ור׳ יהודה הוא דפליג בהדיא לא אמרינן ר״מ ור׳ יהודה הלכה כר׳ יהודה אלא הלכתא כסתם ואע״ג דאפליגו בהדיא בברייתא כיון דסתם לן תנא כר״מ הלכה כר״מ ולא חיישינן למחלוקת דברייתא:
אמר רב נחמן }}הני ליבני דאשתיור מבנינא שרי לטלטולינהו בשבתא הואיל וחזיין למיזגא עלייהו שרגינהו ודאי אקצינהו }ואמר רבא חרס קטנה מותר לטלטלה אפילו ברה״ר ואזדא רבא לטעמיה דרבא הוה קא אזיל ברסתקא דמחוזא איתווס מסאניה טינא אתא שמעיה שקל חספא וקא כפר ליה רמו ביה רבנן קלא אמר להו לא מיסתייא דלא גמירי אלא השתא נמי מיגמר מגמרי אילו הויא בחצר מי לא חזי לכסויי ביה מאנא השתא נמי חזי לדידי:
אמר רב יהודה אמר שמואל מגופת חבית שנתכתתה היא ושבריה מותר לטלטלה בשבת תניא נמי הכי מגופת חבית שנכתתה היא ושבריה מותרין לטלטל בשבת ולא יספות ממנה שבר לכסות את הכלי ולסמוך בה כרעי המטה ואם זרקה לאשפה מבע״י אסור הפירוש לא יספות לא יחתוך כדתנן }}באבטיח:
(דף קכה.) אמר }רב המדוראי אמר שמואל קרומיות של מחצלת מותר לטלטלן בשבת מאי טעמא אמר רבא בר המדוראי אסברה לי מחצלת גופה למאי חזיא לכסויי בה עפרא השתא נמי חזיא לכסויי ביה טינופא פירוש קרומיות של מחצלת שיורי מחצלאות שבלו אמר רבי זירא אמר רב שירי }פרוזמאות אסור לטלטלן בשבת אמר אביי במטלניות שאין בהן שלש על שלש דלא חזיין לא לעניים ולא לעשירים פירוש }פרוזמאות בלאות של טליתות כדאמרינן בסוכה בפ״ק (דף יא.) רב עמרם חסידא רמא תכלתא לפרוזמא דאינשי ביתיה:
(מכילתין דף קכה.) ת״ר שברי תנור ישן הרי הן ככל הכלים ומותר לטלטלן בחצר דברי רבי מאיר רבי יהודה אומר אסור לטלטלן בשבת העיד רבי יוסי משום ר״א בן יעקב על שברי תנור }שמותר לטלטלן בשבת ועל כיסוי שלו שאינו צריך בית יד והלכתא כר״מ כדאמרן לעיל וכר״א בן יעקב דקאמר רבינא כמאן מטלטלינן האידנא כסויי דתנוראי דמתא מחסיא אע״ג דלית בהו בית אחיזה כמאן כרבי אליעזר בן יעקב:
\textbf{{\largeמתני׳}} }האבן שבקירויה אם ממלאין בה ואינה נופלת ממלאין בה בשבת ואם לאו אין ממלאין בה
\end{multicols}\newpage

\newsection{דף מט}
\begin{multicols}{2}
(דף קכה:) זמורה שהיא קשורה בטפיח ממלאין בה בשבת פקק החלון ר״א אומר בזמן שהוא קשור ותלוי פוקקין בו ואם לאו אין פוקקין בו וחכ״א בין כך ובין בך פוקקין בו:
\textbf{{\largeגמ׳}} קשורה אין שאינה קשורה לא גזירה שמא יקטום:
ר״א אומר בזמן שהוא קשור וכו׳: אמר רבה בר בר חנה א״ר יוחנן הכל }}}מודים שאין עושים }אהל עראי לכתחלה ביו״ט ואין צ״ל בשבת לא נחלקו אלא להוסיף שר״א אומר אין מוסיפין ביו״ט ואין צ״ל בשבת וחכ״א מוסיפין בשבת ואין צריך לומר ביו״ט תניא נמי הכי מודים חכמים לרבי אליעזר שאין עושין אהל עראי בתחלה ביו״ט ואין צריך לומר בשבת לא נחלקו אלא להוסיף שר״א אומר אין מוסיפין ביום טוב ואצ״ל בשבת וחכ״א מוסיפין בשבת ואין צריך לומר ביום טוב:
וחכמים אומרים בין כך ובין כך }}}פוקקין בו:
מאי בין כך ובין כך אמר ר׳ אבא בר כהנא
(דף קכו.) בין קשור ובין שאינו קשור והוא שמתוקן והלכה כחכמים ואע״ג דסתם לן תנא כר׳ אליעזר בנגר הנגרר (ערובין קב.) קיי״ל כי האי סתמא אחרינא דתנן *}סוף שבת}מדבריהן למדנו שפוקקין ומודדין וקושרין בשבת (דף קכו:) משום דמעשה רב:
גרסינן בסוף עירובין (דף קב.) שלח ליה רמי בר יחזקאל לרב עמרם לימא לן מר מהני מילי מעלייתא דאמרת לי משמיה דרב אסי בכיפי דארבא ושלח ליה הכי אמר רב אסי הני כיפי דארבא בזמן שיש בהן טפח אי נמי אין בהם טפח ואין בין זה לזה ג׳ מביא מחצלת ופורס עליה מאי טעמא מוסיף על אהל עראי הוא ושפיר דמי:
(שם) הנהו דיכרי דהוו ליה לרב הונא דביממא בעו טולא ובליליא בעו אוירא אתא לקמיה דרב אמר ליה זיל כרוך בודיא ושייר בה טפח ולמחר פרסיה עלייהו דמוסיף על אהל עראי הוא ושפיר דמי:
(כאן דף קכו.) תניא קנה }שהתקינו בעל הבית להיות פותח ונועל בו בזמן שקשור ותלוי פותח ונועל בו אין קשור ותלוי אין פותח ונועל בו רבן שמעון בן גמליאל אומר אם היה מתוקן אע״פ שאין קשור }ותלוי א״ר יהודה בר שילא א״ר אסי א״ר יוחנן הלכה כרבן שמעון בן גמליאל ודוקא כשיש תורת כלי עליו:
וגרסינן בסוף עירובין (דף קב.) ההיא שריתא פירוש קורה אחת דהות בי רבי פדת דהוו שקלי לה בי עשרה ושדו לה אדשא אמר תורת כלי עליה ההיא אסיתא דהוה בי מר שמואל דהוה מחזקא ארבעת קבין והוו שקלי לה בי עשרה ושדו לה אדשא אמר תורת כלי עליה:
(מכילתין קכו:) \textbf{{\largeמתני׳}} }}כל כסויי הכלים שיש להן בית אחיזה ניטלין בשבת א״ר יוסי בד״א בכסויי קרקעות אבל בכסויי הכלים בין כך ובין כך ניטלין בשבת:
\textbf{{\largeגמ׳}} אמר רב יהודה בר שילא א״ר אסי א״ר יוחנן והוא שיש תורת כלי }עליו דכ״ע דכיסויי קרקעות אם יש להן בית אחיזה אין ואי לא לא כסויי הכלים אע״ג דלית בהו בית }אחיזה שרי כי פליגי }בכסויי }(תנור מר מדמי להו לכסויי קרקעות ומר מדמי להו לכסויי) כלים וחברינהו בארעא ת״ק סבר גזרינן ור׳ יוסי סבר לא גזרינן }והלכה כתנא קמא:
\textbf{סליקו להו כל הכלים} 
\end{multicols}\newpage

\newchap{פרק \hebrewnumeral{18} מפנין}
\end{multicols}\newpage

\newsection{דף נ}
\begin{multicols}{2}
\textbf{{\largeמפנין}} אפילו }}ארבע וחמש קופות של תבן ושל תבואה מפני האורחין ומפני ביטול בית המדרש אבל לא את האוצר מפנין תרומה טהורה ודמאי ומעשר ראשון שניטלה תרומתו ומעשר שני והקדש שנפדו ותורמוס היבש מפני שהוא מאכל }לעניים אבל לא את הטבל ולא מעשר ראשון שלא ניטלה תרומתו ולא את מעשר שני והקדש שלא נפדו ולא את הלוף ולא את החרדל רבן שמעון בן גמליאל מתיר בלוף מפני שהוא מאכל לעורבים:
\textbf{{\largeגמ׳}} השתא חמש מפנין ארבע מיבעיא אמר רב חסדא ארבע מחמש וחמש מאוצר גדול ומאי אבל לא את האוצר שלא יתחיל באוצר תחלה ומני ר׳ יהודה היא דאית ליה מוקצה ושמואל אמר ארבע וחמש (דף קכז.) כדאמרי אינשי ואי בעי אפילו טובא ומאי אבל לא את האוצר שלא יגמור את האוצר דלמא אתי לאשוויי גומות אבל אתחולי מתחיל ומני ר״ש היא דלית ליה מוקצה.
}הא מילתא חזינא בה פלוגתא ביני רבוותא איכא מאן דפסק כרב חסדא משום דסוגיא דשמעתא כוותיה וסתם לן תנא }במתניתין כוותיה וגמרא קא שקיל וטרי לפרושי שיעורי דמתני׳ ואיכא מאן דפסיק כשמואל משום דמוקים למתניתין כר״ש דלית ליה }מוקצה דקיימא לן כוותיה:
תנו רבנן אין מתחילין באוצר תחלה אבל עושה בו שביל כדי שיכנס ויצא בו שביל והא אמרת רישא אין מתחילין הכי קאמר עושה בו שביל ברגלו בכניסתו וביציאתו }תנו רבנן תבואה צבורה כל זמן שמתחילין בה בערב שבת מותר לטלטלה בשבת ואם לאו אסור לטלטלה בשבת דברי ר״ש ורבי אחא מתיר כלפי לייא אימא דברי רבי אחא ור״ש מתיר:
תאנא כמה שיעור תבואה צבורה לתך איבעיא להו הני ארבע וחמש קופות דקאמרינן אע״ג דאית ליה אורחין טובא או דילמא הכל לפי האורחין ואם תמצא לומר הכל לפי האורחין חד גברא מפני לכולהו או דלמא כל חד וחד מפני לנפשיה ופשטינן לה הכל לפי האורחין אבל אי כל חד וחד מפני לנפשיה לא איפשיטא וכיון דאיסורא הוא עבדינן
לחומרא:
מפני האורחין }ומפני ביטול בית המדרש דמצוה נינהו דאמר ר׳ יוחנן גדולה הכנסת אורחין כהשכמת בית המדרש דקתני מפני האורחין ומפני ביטול בית המדרש ורב דימי אמר יותר מהשכמת בהמ״ד דקתני מפני האורחין והדר תני ומפני ביטול בהמ״ד אבל שלא במקום מצוה לא אמר רב יהודה אמר רב גדולה הכנסת אורחין מהקבלת פני השכינה שנאמר (בראשית י״ח:ג׳) ויאמר *}[צ״ל אדני]}ה׳ אם נא מצאתי חן בעיניך וגו׳:
אר״א בא וראה שלא כמדת הקב״ה מדת בשר ודם מדת ב״ו אין הקטן יכול לומר לגדול המתן לי עד שאבא אצלך ואילו בהקב״ה כתיב ויאמר *}[צ״ל אדני]}ה׳ [וגו׳] אל נא תעבור מעל עבדך:
אמר רב יהודה בר שילא אמר רב אסי אמר רבי יוחנן ו׳ דברים כשאדם עושה אותם אוכל פירותיהן בעולם הזה והקרן קיימת לו לעולם הבא אלו הן הכנסת אורחין ובקור חולים ועיון תפלה והשכמת בית המדרש והמגדל בניו לתלמוד תורה והדן את חבירו לכף זכות
(דף קכז:) ת״ר הדן את חבירו לכף זכות דנין אותו לכף זכות ומעשה באדם אחד שירד מגליל העליון שהיה נשכר שם ג׳ שנים אצל בעל הבית בדרום כיון שהגיע ערב הרגל אמר לו תן לי שכרי ואלך ואפרנס את אשתי ואת בני א״ל אין לי מעות תן לי בהמה א״ל אין לי תן לי קרקע א״ל אין לי תן לי פירות א״ל אין לי תן לי כרים וכסתות א״ל אין לי הפשיל טליתו לאחוריו והלך לו בפחי נפש לאחר הרגל נטל בעל הבית שכרו בידו ונטל עמו משאוי ג׳ חמורין אחד של מאכל ואחד של משתה ואחד של מיני מגדים והלך לביתו לאחר שאכלו ושתו נתן לו שכרו (בידו) ואמר לו כשאמרת לי תן לי שכרי ואמרתי לך אין לי מעות במה חשדתני אמרתי שמא מצאת פרקמטיא בזול וכשאמרת לי תן לי בהמה ואמרתי לך אין לי בהמה במה חשדתני אמרתי שמא מושכרות הן ביד אחרים וכשאמרת לי תן לי קרקע ואמרתי לך אין לי במה חשדתני אמרתי שמא }מחוכרות הן ביד אחרים וכשאמרת לי תן לי פירות ואמרתי לך אין לי במה חשדתני אמרתי שמא אינן מעושרות וכשאמרת לי תן לי כרים וכסתות ואמרתי לך אין לי במה חשדתני אמרתי שמא הקדישתן לשמים אמר לו העבודה כך היה שהקדשתי כל נכסי לשמים בשביל הורקנוס בני שלא עסק בתורה וכשבאתי אצל חבירי שבדרום התירו לי את נדרי ואתה כשם שדנתני לכף זכות כך }ידין אותך המקום לכף זכות:
ת״ר מעשה בחסיד אחד שפדה ריבה אחת בת ישראל וכשבא למלון השכיבה אצל מרגלותיו למחר השכים וטבל ובא ושנה לתלמידיו ואמר להם כשהשכבתיה אצל מרגלותי במה חשדתוני אמרנו שמא תלמיד אחד יש בינינו שאינו בדוק לרבי וכשירדתי וטבלתי במה חשדתוני אמרנו שמא מטורח הדרך ראה רבינו קרי אמר להם העבודה כך היה וכשם שדנתוני לכף זכות כך ידין המקום אתכם לכף זכות:
ת״ר פעם אחת הוצרך דבר לחכמים אצל מטרוניתא אחת שכל גדולי רומי מצויין אצלה אמרו מי ילך אצלה אמר להם רבי יהושע בן חנניא אני אלך אצלה הלך אצלה הוא ותלמידיו כיון שהגיע לפתח ביתה חלץ תפליו ברחוק ד׳ אמות ונכנס ונעל הדלת אחריו וכשיצא ירד וטבל ושנה לתלמידיו אמר להם בשעה שחלצתי תפילין במה חשדתוני אמרנו שמא קסבר רבי אסור ליבנס בדבר שבקדושה במקום הטנופת וכשנעלתי הדלת בפניכם ונכנסתי במה חשדתוני אמרנו שמא דבר מלכות בינו ובינה וכשיצאתי וירדתי וטבלתי ועליתי ושניתי במה חשדתוני אמרנו שמא נתזה צנורא מפיה על בגדו של רבי אמר להן העבודה כך היה ואתם כשם שדנתוני לכף זכות כך ידין אתכם המקום לכף זכות:
ולא }}את הלוף ולא את החרדל: (דף קכח.) ת״ר }מטלטלין את החצב מפני שהוא מאכל לצביים ואת החרדל מפני שהוא מאכל ליונים ר״ש בן גמליאל אומר אף מטלטלין שברי זכוכית מפני שהוא מאכל לנעמיות א״ל ר׳ נתן אלא מעתה חבילי זמורות יטלטלו מפני שהן מאכל לפילין ור״ש בן גמליאל אמר לך פילין לא שכיחי נעמיות שכיחי:
אמר אביי רשב״ג ור״ש ורבי ישמעאל ור״ע כולהו סבירא להו שכל ישראל בני מלכים הן רשב״ג הא דאמרן רבי שמעון דתנן (דף קיא.) בני מלכים סכין שמן וורד על }גבי מכותיהן רבי שמעון אומר כל ישראל בני מלכים הם ר׳ ישמעאל ור׳ עקיבא דתניא (ב״מ דף קיג:) הרי שהיו נושין בו אלף זוז והיתה לו אצטלא *}בס״י בת״ק מנה}בת מאה מנה מפשיטין אותו ומלבישין אותו אצטלא הראויה לו }ותאנא דבי רבי ישמעאל רבי [ישמעאל] ור׳ עקיבא אמרו כל ישראל ראוין לאותה אצטלא וקיי״ל דכל כי האי גוונא שיטה היא ולית הלכתא כחד מינייהו (ואין מטלטלין את החרדל דקיי״ל בהא כסתם משנה):
\textbf{{\largeמתני׳}} (דף קכו:) חבילי קש וחבילי עצים וחבילי זרדין אם התקינן למאכל בהמה מטלטלין אותם ואם לאו אין מטלטלין אותם (דף קכח:) כופין את הסל לפני האפרוחים שיעלו ושירדו תרנגולת שברחה דוחין אותה עד שתכנס מדדין עגלים וסייחים ואשה מדדה את בנה אמר רבי יהודה אימתי
\end{multicols}\newpage

\newsection{דף נא}
\begin{multicols}{2}
בזמן שנוטל אחת ומניח אחת אבל אם היה גורר אסור:
\textbf{{\largeגמ׳}} (דף קכח.) תנו רבנן חבילי קש וחבילי עצים וחבילי זרדים בזמן שהתקינן למאכל בהמה מטלטלין אותן ואם לאו אין מטלטלין אותן רבן שמעון בן גמליאל אומר חבילה הניטלת בידו אחת מטלטלין אותה בשתי ידיו אסור לטלטלה חבילי }}סיאה ואזוב וקורנית אם הכניסן למאכל בהמה מסתפק מהן בשבת ואם הכניסן לעצים אין מסתפק מהן בשבת וקוטם ואוכל ובלבד שלא יקטום בכלי ומולל ואוכל ובלבד שלא ימלול *}גי׳ ד״ת בכלי וכן הוא בס״י}בידיו הרבה דברי ר׳ יהודה וחכ״א מולל בראשי אצבעותיו ובלבד שלא ימלול בידו הרבה כדרך שהוא מולל בחול ובן באמינתא וכן בפיגם וכן בשאר מיני תבלין
גרסי׳ בפ״ק דביצה (ביצה דף יג:) המולל מלילות מערב שבת למחר מנפח }על יד על יד ואוכל אבל לא בקנון ולא בתמחוי ומסקנא אם בא למלול (כאן קכח.) בשבת היכי מולל אביי משמיה דרב יוסף אמר חדא חדא ורב אויא משמיה דרב יוסף אמר חדא אתרתי רבא אמר כיון דקא משני אפילו טובא נמי וכשהוא מנפח מנפח בידו אחת ובכל כחו והלכתא כרבא דקיימי רבנן כותיה דאמרי בראשי אצבעותיו:
(דף קכח.) מאי אמינתא ניניא: סיאה אמר רב יהודה סנתרי:
אזוב אברתא: קורנית חשי:
ולית הלכתא כרשב״ג דקי״ל כסתם מתני׳:
איתמר }}}בשר }חי מותר לטלטלה בשבת בשר תפוח רב הונא אמר מותר לטלטלה בשבת ורב חסדא אמר אסור לטלטלה והלכתא כרב הונא דרב חסדא תלמיד היה לפני רב הונא ועוד דרב הונא קאי כר״ש דקי״ל כותיה ועוד דתניא דמסייע ליה דת״ר }מטלטלין את העצמות בשבת מפני שהן מאכל לכלבים בשר תפוח מפני שהן מאכל לחיה מים מגולים מפני שהן ראוין לחתול רשב״ג אומר אסור לשהותן מפני הסכנה:
ת״ר דג מליח מותר לטלטלו דג תפל }אסור לטלטלו בשר }חי בין מליח בין שאינו מליח }מותר לטלטלו:
}}}כופין את הסל: (דף קכח:) אמר רב יהודה בהמה שנפלה לאמת המים בשבת מביא כרים וכסתות ומניח תחתיה ואם עלתה עלתה ואע״ג דקא מבטל כלי מהיכנו מ״ט ביטול כלי מהיכנו דרבנן וצער בעלי חיים דאורייתא ואתי דאורייתא ודחי דרבנן וה״מ דלא אפשר למעבד }לה פרנסה התם במקומה אבל אפשר }למעבד לה עושה לה פרנסה במקומה ודיו:
תרנגולת שברחה דוחין אותה: דוחין אין מדדין לא תנינא להא דת״ר מדדין בהמה חיה ועוף בחצר אבל לא את התרנגולת:
תנו רבנן אין עוקרין בהמה חיה ועוף אבל דוחין אותן עד שיכנסו }מדדין
}עגלים וסייחים והאשה מדדה את בנה:
א״ר יהודה אימתי וכו׳: וכן הלכתא דקי״ל *}עירובין דף פא: ופב.}כל היכא דא״ר יהודה אימתי אינו אלא לפרש דברי חכמים:
מתני׳ }}}אין מילדין את הבהמה ביו״ט אבל מסעדין ומילדין את האשה בשבת וקורין *}בס״י הגי׳ חכמה כבגמ׳}המילדת ממקום למקום ומחללין עליה את השבת וקושרין את הטבור רבי יוסי אומר אף חותכין וכל צרכי מילה עושין בשבת:
\textbf{{\largeגמ׳}} כיצד מסעדין רב יהודה אומר אוחז את הולד שלא יפול לארץ תניא כוותיה דרב יהודה כיצד מסעדין אוחז את הולד שלא יפול לארץ ונופח בחוטמו כדי שיכנס בו הרוח ונותן דד לתוך פיו כדי שיינק:
אמר רבי יהודה מרחמין היינו על בהמה טהורה ביו״ט.
כיצד עושה אמר אביי מביא בול של מלח ונותן לה לתוך הרחם כדי שתזכור צערה ותרחם עליו ומזלפין מי שליא על גבי ולד כדי שתריח ריחו ותרחם עליו ודוקא טהורה אבל טמאה לא דלא מרחקה ואי מרחקה לא מקרבה:
ומילדין את }}האשה: ת״ר אם היתה צריכה לנר חברתה מדלקת לה את הנר ואם היתה צריכה לשמן חברתה מביאה לה שמן ביד ואם אין סיפק ביד מביאה לה בשערה ואם אין סיפק בשערה מביאה לה בכלי דרך רשות הרבים אמר מר אם היתה צריכה לנר חברתה מדלקת לה את הנר פשיטא אמר רב אשי לא נצרכה אלא אפילו בסומא מהו דתימא כיון דלא חזיא אסור קמ״ל דיתובא מיתבא דעתה מימר אמרה אי איכא מידי }חזיין חברתאי ועבדין לי אם היתה צריכה לשמן חברתה מביאה לה שמן ביד ואם אין סיפק ביד מביאה לה בשערה ותיפוק ליה משום סחיטה רבה ורב יוסף דאמרי תרוייהו אין סחיטה בשער ורב אשי אמר אפילו תימא יש סחיטה בשער מביאה לה בכלי דרך שערה דכל מה דאפשר לשנויי משנינן
אמר רב יהודה אמר שמואל חיה כל זמן שהקבר פתוח בין שאמרה צריכה אני ובין }שאמרה איני צריכה מחללין עליה את השבת (דף קכט.) נסתם הקבר אמרה צריכה אני מחללין עליה את השבת }אמרה איני צריכה אין מחללין עליה את השבת רב אשי מתני הכי מר זוטרא מתני (הכי) אמר רב יהודה אמר שמואל חיה כל זמן שהקבר פתוח בין אמרה צריכה אני ובין אמרה איני צריכה מחללין עליה את השבת נסתם הקבר אמרה איני צריכה אין מחללין עליה את השבת לא אמרה איני צריכה מחללין עליה את השבת אמר ליה רבינא למרימר מר זוטרא מתני לקולא ורב אשי מתני לחומרא הלכתא כמאן אמר ליה הלכתא כמר זוטרא דספק נפשות להקל
מאימתי פתיחת הקבר אמר אביי משעה שתשב על המשבר }רב הונא בריה דרב יהושע אמר משעה שהדם שותת רורד ואמרי לה משעה שחברותיה נושאות אותה באגפיה ודייקי רבוותא ואמרי מדלא אמר רב הונא עד שיהא הדם שותת ויורד שמע מינה קודם שתשב על המשבר קאמר וכאביי עבדינן ולא מחללין עליה שבתא עד שיהא הדם שותת }ויורד ותשב על המשבר:
עד מתי פתיחת הקבר אסיקנא נהרדעי אמרי חיה }שלשה שבעה ושלשים שלשה בין שאמרה צריכה אני ובין שאמרה איני צריכה מחללין עליה את השבת שבעה אמרה צריכה אני מחללין עליה את השבת אמרה איני צריכה אין מחללין עליה את השבת הא סתמא מחללין שלשים אפי׳ אמרה צריכה אני אין מחללין עליה את השבת אלא עושין לה על ידי נכרי }כדרב עולא בריה דרב עילאי דאמר כל צרכי }חולה עושין לו ע״י נכרי בשבת ובדרב המנונא דאמר כל דבר שאין בו סכנה אומר לנכרי ועושה אמר רב יהודה אמר שמואל החיה שלשים יום למאי הלכתא אמרי נהרדעי לטבילה:
אמר
\end{multicols}\newpage

\newsection{דף נב}
\begin{multicols}{2}
רבא לא אמרן אלא שאין בעלה עמה אבל בעלה עמה בעלה מחממה ברתיה דרב חסדא טבלה בגו תלתין יומין שלא בפני בעלה }ואמטוה לערסה בתריה דרבא לפומבדיתא אמר רב יהודה אמר שמואל עושין }מדורה לחיה בשבת סבור מינה הני מילי לחיה אבל לחולה }לא בימות הגשמים אין בימות החמה לא ואסיקנא ל״ש חיה ול״ש חולה ל״ש בימות החמה ל״ש בימות הגשמים מדאמר רב חייא בר אבין אמר רב ששת הקיז דם ונצטנן עושין לו מדורה אפילו בשבת ואפילו בתקופת תמוז שמואל צלחו ליה תכתקא דשגא רב יהודה צלחו ליה פתורא רבה שרשיפא אמר ליה אביי לרבה קא עביר מר משום (דברים כ) בל תשחית אמר ליה בל תשחית דגופאי עדיף לי:
אמר }רבי חייא בר אשי לעולם ימכור אדם קורות ביתו ויקנה מנעלים לרגליו *}רפואת ההקזה עי׳ במיי׳ פ״ד מהל׳ דעות}הקיז דם ואין לו מה יאכל ימכור מנעלים שברגליו ויספיק מהן צרכי סעודה מאי צרכי סעודה רב אמר בשר ושמואל אמר יין רב אמר בשר נפשא חלף נפשא ושמואל אמר יין סומקא חלף סומקא:
וקושרין את הטבור: (דף קכט:) ת״ר קושרין את הטבור ר׳ יוסי אומר אף }חותכין וטומנין את השליא כדי שיחם את הולד אמר רשב״ג בנות מלכים טומנות בספלים של שמן בנות עשירים בספוגין של צמר בנות עניים במוכין:
אמר רב נחמן אמר רבה בר אבוה הלכה כרבי יוסי:
ואמר רב נחמן אמר רבה בר אבוה מודים חכמים לר׳ יוסי בטבור של שני תינוקות שחותכין מאי טעמא דמנתחי אהדדי אמר רב נחמן אמר רבה בר אבוה כל האמור בפרשת תוכחה עושין לחיה בשבת שנא׳ (יחזקאל ט״ז:ד׳) ומולדותיך ביום הולדת אותך }לא כרת שרך מכאן שחותכין את הטבור בשבת ובמים לא רחצת למשעי מכאן שמרחיצין את הולד והמלח לא המלחת מכאן שמולחים את הולד והחתל לא חותלת מכאן שמלפפין את הולד בשבת:
\textbf{סליקו להו מפנין} 
(דף קל.) \textbf{{\largeרבי}} אליעזר אומר }}}אם לא הביא כלי מע״ש מביאו בשבת מגולה ובסכנה מכסהו על פי עדים ועוד א״ר אליעזר כורתין עצים לעשות פחמים ולעשות ברזל כלל א״ר עקיבא כל מלאכה שאפשר לעשותה מערב שבת אינה דוחה את השבת מילה שאי אפשר לה ליעשות מערב שבת דוחה את השבת:
\textbf{{\largeגמ׳}} א״ר אבין אמר רב אדא א״ר יצחק פעם אחת }שכחו ולא הביאו איזמל מערב שבת והביאוהו בשבת שלא (דף קל:) ברצון ר״א }דשרי דרך רה״ר אלא ברצון רבי שמעון דשרי דרך חצרות דרך גגות דרך קרפיפות דתנן ר״ש אומר אחד גגות ואחד חצרות ואחד קרפיפות רשות אחת לכלים ששבתו לתוכה ולא לכלים ששבתו לתוך הבית
ופסקו רבוואתא דלית הלכתא כר״ש לגבי איזמל ואע״ג דאמר רב בפרק כל גגות העיר (דף צא.) הלכה כר״ש בהא קיימא לן דהלכתא כרבנן }ואפסיקא הלכתא כרבי עקיבא דאמר כל מלאכה שאפשר לעשותה מערב שבת אינה דוחה את השבת ותו דהא אמר }רבה בפרק האשה בפסחים (פסחים דף צב.) ערל והזאה ואיזמל העמידו דבריהם במקום כרת איזמל מאי היא דתניא כשם שאין מביאין אותו דרך רשות הרבים כך אין מביאין אותו דרך חצירות גגות וקרפיפות ותו דרבנן ור״ש יחיד ורבים הלכה כרבים ואיכא מאן דמסייעי למלתא ממעשה דאשכחו בתלמודא דארץ ישראל דגרסינן התם *}בפירקין הל״א}רב שמואל בר אבדימי הוה ליה עובדא למיגזר בי רב שישנא בריה אנשיון מייתי אזמל שאל לרבי מונא א״ל ידחה עד למחר שאל לרבי יצחק
\end{multicols}\newpage

\newchap{פרק \hebrewnumeral{19} רבי אליעזר דמילה}
\begin{multicols}{2}
ברבי אלעזר אמר להם מישחק קונדיטון לא אנשיתון ומייתי איזמל אנשיתון ידחה עד למחר ואנן קשיא לן פסקא הדין דפסקו ראשונים חדא דהא אפסיקא הלכתא בהדיא (עירובין דף צא:) כרבי שמעון דאמר אחד גגות ואחד חצרות ואחד קרפיפות כולן רשות אחת ומותר לטלטל בכולן ואפילו איסורא ליכא בדבר של רשות וכ״ש בדבר מצוה ועוד דהא עבדו עובדא כר״ש כדאמרינן לעיל א״ר אבין אמר רב אדא אמר רבי יצחק פעם אחת שכחו ולא הביאו אזמל וכו׳ וקיימא לן דמעשה רב בכל מקום
והני ראיות כולהו דגמרי מינייהו רבוותא דלא מייתינן איזמל דרך גגות וחצרות וקרפיפות איכא פירכא לכל חדא מינייהו וליכא למיגמר מינייהו דהא דאפסיקא הלכתא כר״ע דאמר כל }דבר שאפשר לה ליעשות מערב שבת אינה דוחה את השבת ליכא למיגמר מינה אלא דלא דחינן שבת ומייתינן דרך רשות הרבים בלבד כר״א אבל לאתויי דרך גגות ודרך חצרות ודרך קרפיפות לא שייכא בהא מילתא כלל אלא פלוגתא דר״ש ורבנן היא דרבנן סברי דרך גגות ודרך חצרות }איכא בהו איסורא משום שבות ולא דחינן לההיא איסורא אפילו במקום מצוה ור״ש סבר דרך גגות וחצרות וקרפיפות רשות אחת ומותר לטלטל במילי דרשות וכ״ש במילי דמצוה וכבר אפסיקא הלכתא כרבי שמעון בהא }נמי
והא דאמר רבא ערל והזאה ואיזמל העמידו דבריהם במקום כרת לאו סברא דידיה קאמר ולאו הלכה נמי קא פסק אלא סברא דרבנן קא מדכר דאית להו דהבאת איזמל דרך גגות יש בה משום שבות ואסרי לה }משום כרת וכיון דהעמידו דבריהם כמקום כרת בא רבא וחשב לה בהדי ערל והזאה שהן מדבריהם והעמידום במקום כרת ורבי שמעון פליג באיזמל על רבנן ואמר דליכא איסורא כלל כמו שביררנו ועוד דר״ש נמי אית ליה הא דרבא דלא פליג ר״ש עלייהו דרבנן אלא בכלים ששבתו בחצר אבל בכלים ששבתו בתוך הבית אית ליה לר״ש דאסור והעמידו חכמים דבריהם במקום כרת הלכך ליכא ראיה נמי מהא דאמר רבא ערל והזאה ואיזמל
ומאי דאמרו נמי משום דר״ש יחיד ופליג בהדי רבים וקי״ל דיחיד ורבים הלכה כרבים הני מילי היכא דלא אפסיקא הלכתא ביחיד בהדיא אבל הכא הא איפסיקא הלכה כר״ש בהדיא ועבדו עובדא נמי כוותיה
והאי עובדא נמי דאשכחו בתלמוד ארץ ישראל ליכא למיגמר מיניה דאסור לאתויי דרך גגות דאיכא למימר דהא דאמר ידחה עד למחר משום דלא אפשר לאתויי דרך גגות וקרפיפות }הוא כגון דקא מפסקא להו דרך רשות הרבים אי נמי משום דלא הוה להו איזמל ששבת בחצר דהא כי אמר ר׳ שמעון לכלים ששבתו לתוכו ולא לכלים ששבתו בתוך הבית ולאו לאפוקי מדר״ש אתא אלא לאפוקי מדר״א אתא ועוד אפילו תימא לאפוקי מדר״ש אתא לא סמכינן אלא אגמרא דילן והדא היא דעתא דילנא בהא שמעתא ומילי דברירן אינון ולית בהון ספיקא כלל:
כלל אמר ר׳ עקיבא: (דף קלג.) אמר רב יהודה הלכה כר״ע (דף קלב.) עד כאן לא פליגי ר׳ אליעזר ור״ע אלא במכשירי }}מילה אבל מילה גופה דחיא שבת מנא לן א״ר יוחנן אמר קרא (ויקרא י״ב:ג׳) וביום השמיני ימול בשר ערלתו ואפילו בשבת:
תנו רבנן מילה דוחה את הצרעת בין בזמנה בין שלא בזמנה יום טוב אינו דוחה אלא בזמנה בלבד:
}}(יבמות דף עב.) ומילה בין בזמנה בין שלא בזמנה לא מהלינן אלא ביממא דתניא (שם) אין לי אלא הנימול לשמונה שיהא נימול ביום לתשעה לעשרה לאחד עשר לשנים עשר ושאר כל הנימולין מנין שלא יהו נימולין אלא ביום תלמוד לומר וביום השמיני ימול בשר ערלתו:
גרסינן בקידושין בפרקא קמא (דף כט.) }}והיכא דלא מהליה אבוה מיחייבי בי דינא למימהליה דכתיב (בראשית י״ז:י׳) המול לכם כל זכר והיכא דלא מהלוה }בי דינא מיחייב איהו למימהל נפשיה דכתיב (בראשית י״ז:י״ד) וערל זכר אשר לא ימול את בשר ערלתו.
\textbf{{\largeמתני}} (כאן קלג.) }}}עושין כל צרכי מילה בשבת מוהלין ופורעין ומוצצין ומשימין עליה אספלנית וכמון ואם לא שחק כמון מערב שבת לועס בשיניו ונותן ואם לא טרף יין ושמן נותן זה בעצמו וזה בעצמו ואין עושין לו חלוק לכתחלה אבל כורך עליה סמרטוט ואם לא התקין מערב שבת כורך על אצבעו ומביא ואפילו מחצר אחרת:
\textbf{{\largeגמ׳}} (דף קלג:) תנו רבנן המל כל זמן שעוסק במילה חוזר כין על }ציצין המעכבין את המילה ובין על ציצין שאין מעכבין את המילה פירש על ציצין המעכבין את המילה חוזר על ציצין שאין מעכבין את המילה אינו
\end{multicols}\newpage

\newsection{דף נג}
\begin{multicols}{2}
חוזר:
ת״ר מהלקטין את המילה ואם לא הלקיט ענוש כרת היכי דמי כגון דאתי בין השמשות ואמרי ליה לא מספקת ואמר להו מספיקנא ועבד ולא איספיק דאשתכח דחבורה הוא דעביד הילכך ענוש כרת:
ומוצצין }}}אמר רב פפא האי אומנא דלא מייץ סכנתא היא ומעברינן ליה:
אם לא שחק כמון (דף קלד.) ת״ר דברים שאין עושין למילה בשבת עושין לה ביו״ט שוחקין לה כמון וטורפין לה יין ושמן:
אין עושין לה חלוק וכו׳: אמר אביי אמרה לי }אמי האי חלוק דינוקא ליפכיה לסיטריה אבראי אמאי דילמא מידבק גירדא מיניה ואתי לידי כרות שפכה:
תניא אמר רבי נתן }}פעם אחת הלכתי לכרכי הים ובאה אשה אחת לפני שמלה בנה ראשון ומת שני ומת שלישי הביאתו לפני *}בגמ׳ ראיתיו שהוא אדום אמרתי}וראיתיו שהוא אדום והצצתי בו ולא ראיתי בו דם }ברית אמרתי לה בתי המתיני לו עד שיבלע בו דמו המתינה לו ומלו אותו וחיה והיו קורין אותו נתן הבבלי על שמי ושוב פעם אחת הלכתי למדינת קפוטקיא ובאה אשה אחת לפני שמלה בנה ראשון ומת שני ומת שלישי הביאתו לפני וראיתיו שהוא ירוק והצצתי בו ולא ראיתי בו דם ברית אמרתי לה בתי המתיני לו עד }שיפול בו דמו המתינה לו ומלה אותו וחיה והיו קורין אותו נתן הבבלי על שמי:
(דף קלד:) \textbf{{\largeמתני׳}} }}}מרחיצין את הקטן בין לפני המילה ובין לאחר המילה ומזלפין עליו ביד אבל לא בכלי ר׳ אלעזר בן עזריה אומר מרחיצין את }הקטן בשלישי שחל להיות בשבת שנאמר (בראשית ל״ד:כ״ה) ויהי ביום השלישי בהיותם כואבים ספק בן שבעה ספק בן שמונה ואנדרוגינוס אין מחללין עליהן את השבת ורבי יהודה מתיר באנדרוגינוס:
\textbf{{\largeגמ׳}} כי אתא רבין אמר רבי אבהו אמר ר׳ אלעזר ואמרי לה א״ר יוחנן הלכה כרבי אלעזר בן עזריה בין בחמין שהוחמו בשבת בין בחמין שהוחמו מבעוד יום בין הרחצת מילה בין הרחצת כל גופו מפני שסכנה היא לו ואמרי רבוואתא שהלכה כראב״ע ביום ג׳ וכ״ש בראשון שמרחיצין אותו בדרכו בין }לפני המילה בין לאחר המילה
}בין בחמין שהוחמו בשבת בין בחמין שהוחמו מע״ש:
אמר רב }}אין מונעין חמין ושמן ליתן על גבי מכה בשבת ושמואל אמר נותנין חוץ למכה ושותת ויורד לתוך המכה והלכתא כשמואל דתניא כוותיה אין נותנין }חמין בשבת ליתן ע״ג מכה אבל נותן חוץ למכה ושותת ויורד למכה אין נותנין חמין ושמן ע״ג מוך ליתן ע״ג }מכה בשבת ולא ע״ג מוך שע״ג מכה בשבת:
ת״ר נותנין מוך יבש וספוג יבש ע״ג מכה בשבת אבל לא גמי יבש ולא }כתיתים יבשין והתניא רישא נותנין כתיתין יבשין קשיא כתיתין אכתיתין לא קשיא הא דקתני נותנין מוך יבש בחדתי והא דקתני אבל לא כתיתין יבישין בעתיקי גרסינן בסוף עירובין (דף קב:) ת״ר רטיה שפירשה מע״ג המכה בשבת מחזירין אותה בשבת רבי יהודה אומר הוחלקה למטה דוחקה למעלה הוחלקה למעלה דוחקה למטה ומגלה קצת רטיה ומקנח פי המכה וחוזר ומגלה מקצת רטיה ומקנח פי המכה אבל לא יקנח רטיה עצמה מפני שהוא ממרח ואם מרח חייב חטאת:
אמר רב חסדא מחלוקת כשפירשה על גבי כלי אבל פירשה ע״ג קרקע דברי הכל אסור להחזירה אמר רב יהודה הלכה כר׳ יהודה ולית הלכתא כוותיה }דרב אשי דהוא בתראה עבד עובדא כוותיה דת״ק
ת״ר (ויקרא יב) }}}ערלתו ערלתו ודאי דוחה את השבת (דף קלה.) ולא ספק דוחה את השבת ערלתו ודאי דוחה את השבת ולא }אנדרוגינוס דוחה את השבת רבי יהודה אומר אנדרוגינוס דוחה את השבת וענוש כרת ערלתו ודאי דוחה את השבת ולא נולד בין השמשות דוחה את השבת ולא נולד כשהוא מהול דוחה את השבת שבש״א צריך להטיף ממנו דם ברית ובה״א אין צריך להטיף ממנו דם ברית א״ר שמעון בן אלעזר לא נחלקו ב״ש וב״ה על }הנולד כשהוא מהול }}}שצריך להטיף ממנו דם ברית על מה נחלקו על גר שנתגייר כשהוא מהול שב״ש אומרים צריך להטיף ממנו דם ברית וב״ה אומרים אינו צריך להטיף ממנו דם ברית
איתמר אמר רב הלכה כת״ק ושמואל אמר הלכה כר״ש בן אלעזר ואע״ג דפליגי רב ושמואל בהא מילתא קיימא לן כבתראי דאינון רבה ורב יוסף דאתמר רבה אמר חיישינן שמא ערלה כבושה היא ורב יוסף אמר }ודאי ערלה כבושה היא ואמר רבה }מנא אמינא לה דתניא ר׳ אלעזר הקפר אומר לא נחלקו בית שמאי ובית הלל על הנולד כשהוא מהול שצריך להטיף ממנו דם
\end{multicols}\newpage

\newsection{דף נד}
\begin{multicols}{2}
ברית על מה נחלקו לחלל עליו את השבת שב״ש אומרים מחללין עליו את השבת וב״ה אומרים אין מחללין מכלל דת״ק סבר ד״ה }אין מחללין עליו את השבת ומדגמר רבה דהוא בתראה מהא מתניתא ש״מ דהלכה היא ואע״ג דפליג עליה רב יוסף הא קי״ל רבה ורב יוסף הלכה כרבה }בר משדה ענין ומחצה הלכך כרבה סבירא לן דאמר חיישינן שמא ערלה כבושה היא וצריך להטיף ממנו דם ברית בחול אבל לא בשבת דמספיקא לא מחללינן שבתא וכן הלכה:
לרבינו האי גאון זצ״ל וששאלת נולד כשהוא מהול צריך להטיף ממנו דם ברית או לא ואם צריך להטיף ממנו דם ברית מברכין עליו או לא כללו של דבר רבנן קמאי במתיבתא הכין אסכימו שצריך להטיף ממנו דם ברית אלא מיהו בנחת וצריכה מילתא למיבדקה יפה יפה בידים ומראית העין ולא בפרזלא דלא ליעייק ליה ואין מברכין על המילה אלא אם כן נראית לו ערלה כבושה ורואין ונזהרים היאך מוהלין אותו ואם [לא] נראית לו ערלה ממתינין לו הרבה שלא יביאוהו לידי סכנה ואין חוששין לשמיני אמר מר ולא ספק דוחה את השבת לאתויי מאי לאתויי הא }דתנו רבנן }}}בן שבעה מחללין עליו את השבת בן שמונה אין מחללין עליו את השבת ספק בן שבעה ספק בן שמונה אין מחללין עליו את השבת בן שמונה }הרי הוא כאבן ואסור לטלטלו בשבת אבל אמו שוחה עליו ומניקתו מפני הסכנה:
אמר }רבא אמר רב אסי כל שאמו טמאה לידה נימול לשמונה וכל שאין אמו טמאה לידה אינו נימול לשמונה שנאמר (ויקרא י״ב:ב׳) אשה כי תזריע וילדה זכר וטמאה שבעת ימים וכתיב (ויקרא י״ב:ג׳) וביום השמיני ימול בשר ערלתו אמר ליה אביי דורות הראשונים יוכיחו שאין אמן טמאות לידה ונימולין לשמונה א״ל נתנה תורה (דף קלה:) ונתחדשה הלכה איני והא איתמר יוצא דופן ומי שיש לו שתי ערלות רב הונא וחייא בר רב חד אמר מחללין עליו את השבת וחד אמר אין מחללין עד כאן לא פליגי אלא לחלל עליו את השבת אבל לשמונה ודאי מהלינן ליה הא בהא תליא כתנאי יש יליד בית שנימול לשמונה
ויש יליד בית שנימול לאחד יש מקנת כסף שנימול לאחד ויש מקנת כסף שנימול לשמונה כיצד לקח שפחה מעוברת ואחר כך ילדה זהו מקנת כסף שנימול לשמונה לקח שפחה וולדה עמה זהו מקנת כסף שנימול לאחד לקח שפחה ונתעברה אצלו וילדה זהו יליד בית שנימול לשמונה רבי חמא אומר ילדה ואח״כ הטבילה זהו יליד בית הנימול לאחד הטבילה ואחר כך ילדה זהו יליד בית שנימול לשמונה ות״ק לא שני ליה בין הטבילה ואח״כ ילדה בין ילדה ואח״כ הטבילה ואע״ג דאין אמו טמאה לידה נימול לשמונה
אמר רבא בשלמא לר׳ חמא משכחת לה יליד }בית נימול לאחד }יליד בית נימול לשמונה כגון שילדה ואחר כך הטבילה הטבילה ואח״כ ילדה מקנת כסף שנמול לשמונה כגון שלקח שפחה מעוברת והטבילה ואח״כ ילדה מקנת כסף שנימול לאחד כגון שלקח זה שפחה וזה עוברה אלא לת״ק בשלמא כולהו משכחת להו אלא יליד בית שנימול לאחד היכי משכחת לה א״ר ירמיה בלוקח שפחה לעוברה הניחא למ״ד קנין פירות לאו כקנין הגוף דמי אלא למ״ד קנין פירות כקנין הגוף דמי מאי איכא למימר אמר רב משרשיא בלוקח שפחה על מנת שלא להטבילה:
תניא רשב״ג אומר כל ששהה }שלשים יום }}באדם אינו נפל שנא׳ (במדבר י״ח:ט״ו-ט״ז) ופדויו מבן חדש תפדה }שמונה ימים בבהמה אינו נפל שנאמר (ויקרא כ״ב:כ״ז) ומיום השמיני והלאה ירצה לקרבן אשה לה׳ הא לא שהה ספיקא הוא }לגבי אשת אח דמחייבי כריתות הן אחמירו בה רבנן ולא נייבם אשת אח מספיקא דכל ספיקא דאורייתא לחומרא אבל לגבי }}}אבילות אקילו בה רבנן (דף קלו.) איבעיא להו
\end{multicols}\newpage

\newsection{דף נה}
\begin{multicols}{2}
מי פליגי רבנן על רשב״ג או לא ואת״ל פליגי עליה הלכה כמותו או לא ת״ש דאמר רב יהודה אמר שמואל הלכה כרשב״ג הלכה מכלל דפליגי ש״מ דפליגי והלכה כמותו והאי דאמרינן שמונה ימים בבהמה אינו נפל הני מילי לקרבן אבל }}}לאכילה אם שהה שבעה ימים מותר לשוחטו בליל שמיני ואוכלו דאמרינן רב פפא ורב הונא בריה דרב יהושע איקלעו לבי רב אידי בר אבין עבדו ליה עיגלא תליתאה ביומא דשבעה ואמרו ליה אי אחריתו עד לאורתא (ח׳) עידן דראוי לשחיטה הוה אכלינן מיניה השתא לא אכלינן מיניה:
בריה דרב דימי בר יוסף איתיליד ליה ההוא ינוקא ובגו תלתין יומין שכיב יתיב וקא מתאבל עליה א״ל אבוה }צודנייתא קא בעית למיכל א״ל קים לי בגוויה שכלו לו חדשיו רב אשי איקלע לבי רב כהנא איתרע ביה מילתא בגו תלתין יומין חזייה דיתיב וקא מתאבל עליה א״ל לא סבר לה מר להא דא״ר יהודה אמר שמואל הלכה כר״ש בן גמליאל אמר ליה קים לי בגויה שכלו לו חדשיו:
איתמר מת בתוך שלשים ועמדה ונתקדשה אמר רבינא משמיה דרבא (דף קלו:) אם אשת ישראל חולצת ואם אשת כהן אינה חולצת וכן הלכה:
רבי }}יהודה מתיר באנדרוגינוס: אמר רב שיזבי אמר רב חסדא לא לכל א״ר יהודה }אנדרוגינוס זכר הוא שאם אתה אומר כן בערכין יערך ומנלן דלא מעריך דתני׳ (ויקרא כז) הזכר ולא טומטום ואנדרוגינוס יכול לא יהא בערך איש אבל יהא בערך אשה תלמוד לומר הזכר ואם נקבה זכר ודאי נקבה ודאית ולא טומטום ואנדרוגינוס איכא מאן דאמר מדקא מפרש רב חסדא לטעמיה דרבי יהודה ש״מ דהלכתא כוותיה וגאון קאמר דלית הלכתא כוותיה:
\textbf{{\largeמתני׳}} (דף קלז.) }מי שהיו לו שני תינוקות אחד למול בשבת ואחד למול בא׳ בשבת ושכח ומל את של אחר השבת בשבת חייב אחד למול בערב שבת ואחד למול בשבת ושכח ומל את של ערב שבת בשבת ר״א מחייב חטאת ורבי יהושע פוטר קטן נימול לשמונה לתשעה ולעשרה ולאחד עשר ולשנים עשר לא פחות ולא יותר כיצד כדרכו נמול לשמונה נולד בין השמשות נימול לתשעה בין השמשות לע״ש לעשרה יום טוב אחר השבת לי״א שני י״ט של ר״ה נימול לשנים עשר קטן החולה אין מוהלין אותו עד שיבריא:
\textbf{{\largeגמ׳}} גרסינן בפרק יוצא דופן (דף מב:) ההוא דאתא לקמיה דרבא אמר לו }}}מהו למימהליה בשבת אמר לו שפיר דמי בתר דנפק אמר רבא ס״ד דלא ידע האי גברא דשרי למימהל בשבתא אהדריה אמר לו אימא }איזי גופא דעובדא היכי הוה אמר לו שמעיתיה דעייק בין השמשות ולא איתיליד עד דאחשיך אמר לו ההוא הוציא ראשו חוץ לפרוזדור הוה דאי לא אפיק לא הוה מעיק דבמה דאיתיה במעי אימיה פיו סתום וטבורו פתוח והוה ליה נולד בין השמשות והויא לה מילה שלא }בזמנה ומילה שלא בזמנה לא דחיא שבת:
קטן החולה אין מוהלין אותו עד שיבריא: אמר שמואל חלצתו }חמה נותנין
לו כל שבעה להברותו איבעיא להו מי בעינן מעת לעת או לא ואיפשיטא דבעינן הני שבעה מעת לעת:
\textbf{{\largeמתני׳}} ואלו הן ציצין }}}המעכבין את המילה בשר החופה את רוב העטרה אם היה כהן אינו אוכל בתרומה ואם היה בעל בשר מתקנו מפני מראית העין (דף קלז:) מל ולא פרע את המילה כאילו לא מל:
\textbf{{\largeגמ׳}} א״ר אבין א״ר ירמיה בר אבא בשר החופה את רוב גובהה של עטרה וכן הלכה:
אמר שמואל קטן המסורבל בבשר רואין אותו כל זמן שמתקשה ונראה מהול אינו צריך למול ואם לאו צריך למול במתניתא תאנא אמר רשב״ג קטן המסורבל בבשר רואין אותו כל זמן שמתקשה ואינו נראה מהול צריך למולו ואם לאו אין צריך למולו מאי בינייהו איכא בינייהו נראה ואינו נראה:
מל ולא פרע וכו׳: ת״ר }}}המל אומר בא״י אמ״ה אקב״ו }על המילה אבי הבן אומר בא״י אמ״ה אקב״ו להכניסו בבריתו של אברהם אבינו והעומדים שם אומרים כשם שהכנסתו לברית כך תכניסהו לתורה ולחופה ולמעשים טובים והמברך אומר בא״י אמ״ה אשר קדש ידיד מבטן וחוק בשארו שם וצאצאיו חתם באות ברית קדש על כן בשכר זה אל חי חלקנו צורנו צוה להציל ידידות שארינו משחת למען בריתו אשר שם בבשרנו בא״י כורת הברית המל את }הגרים אומר בא״י אמ״ה אקב״ו למול את הגרים ולהטיף מהם דם ברית וכו׳ המל את העבדים אומר בא״י אמ״ה אקב״ו למול את העבדים ולהטיף מהם דם ברית שאלמלא דם ברית לא נתקיימו שמים וארץ }שנאמר (ירמיהו ל״ג:כ״ה) אם לא בריתי יומם ולילה חוקות שמים וארץ לא שמתי ברוך אתה ה׳ כורת הברית
ומאן דמהל גר ועבד ניכסי ערוה וניבריך על המילה והדר לימהול ובתר דמהיל ליבריך למול את הגרים וכו׳ וגר אף על גב דאמהיל בארמיותיה בעי להטיף ממנו דם ברית ועד דמתפח }לא מטבלינן ליה וכו׳ מטבלינן ליה צריך למיזל בהדיה תלתא ונימא ליה כי היכי דאמרינן ליה מעיקרא וגייזי ליה ממזייה ושקלי ליה טופריה דידיה ודכרעיה וצריכי למיחזייה כד טביל וכד סליק מברך בא״י אמ״ה אקב״ו על הטבילה וכן עבד משוחרר וכן עבדא ואמתא דזבין להו מעיקרא מן הנכרי אתו תלתא ואמרי ליה מצות דמחייב בהו עבד ומטבלינן ליה כדכתבינן בהלכות }גרים בפ׳ החולץ
והיכא דאשתפיך }}חמימיה דתינוק ישראל ואיבדור סמני׳ בתר דאמהיל עבדינן ליה בשבתא מפני הסכנה אשכחן בהלכות גדולות }}כתיבי והיכא דאייתי איזמל במעלי שבתא לשבתא ואיגניב או איפגים מקמי מילה שרי ליה למימר לנכרי לצבותיה או לאיתויי איזמל
\end{multicols}\newpage

\newsection{דף נו}
\begin{multicols}{2}
אחרינ׳ כההוא (עירובין סז:) ינוקא דאשתפיך חמימיה אמר להו רבה לייתו מגו ביתאי א״ל אביי והא לא עירבו א״ל נסמוך אשיתוף והא לא שתיף לימרו לנכרי ליזיל לייתי לי מגו ביתאי אמר ליה אביי בעאי לאותובי למר ולא שבקן רב יוסף דא״ר כהנא כי הוינן בי רב יהודה אמר לן בדאורייתא מותבינן תיובתא והדר עבדינן עובדא בדרבנן עבדינן עובדא והדר מותבינן תיובתא הדר אמר לו מאי בעית דתותביה }א״ל מהא דתניא הזאה שבות היא ואמירה לנכרי שבות היא (עירובין סח.) מה הזאה שבות ואינה דוחה שבת אף אמירה לנכרי שבות ואינה דוחה שבת א״ל ולא שני לך בין שבות דאית ביה מעשה לשבות דלית ביה מעשה }דהא מר לא א״ל לנכרי זיל אחים לי דהוא מעשה דאסור מדאורייתא אלא זיל אייתי לי דלית ביה מעשה דידים אלא דיבורא בעלמא הוא ושתק ולא מצא תשובה
ועיינינן בהו ואשכחן דלאו מילי דסמכא נינהו ומגופא דהאי עובדא ילפת דאסור למימר לנכרי לצבותיה או לאתויי דרך רה״ר מדקאמר ליה דהא מר לא א״ל }זיל אחים לי מכלל דאסור למימר לנכרי זיל אחים לי
וכמה מרבוואתא טעו בהאי פירושא דהאי מימרא דקסברי דהאי דא״ל ולא שני }ליה למר בין שבות דאית ביה מעשה לשבות דלית ביה מעשה אשבות דהזאה קאמר ליה דשבות דהזאה שבות דאית ביה מעשה ושבות דאמירה לית ביה מעשה בידים דדיבורא בעלמא הוא ואקשו בה הכי ואמרי הואיל ואמירה לנכרי שבות דלית ביה מעשה הוא מה לי א״ל אייתי לי מגו ביתאי ומה לי א״ל זיל אחים לי תרוייהו שבות דלית בהו מעשה נינהו דדיבורא בעלמא הוא ושרי ואטעו לכולהו נוסחאי דגמרא דקסברי דליכא בגמרא דהא מר לא א״ל זיל אחים לי אלא טעותא דנוסחי הוא ולא כדקא סברי הוא
ונוסחי לית בהו טעותא אלא נוסחי מעלייא נינהו }ולא פירושי׳ כדקא סברי להו אלא }האי דא״ל ולא שני ליה למר }}בין שבות דאית ביה מעשה לשבות דלית ביה מעשה לאו אשבות דהזאה קאמר ליה אלא הכי קאמר ליה הא דאמרת אמירה לנכרי שבות וקא }מדמיתו לה לשבות דהזאה דאסירא לא שני לך בשבות דאמירה גופה בין שבות דאית ביה מלאכה לשבות דלית ביה מלאכה דהא מר לא א״ל זיל אחים לי דאית ביה מלאכה אלא אייתי לי מגו ביתאי קאמר ליה דטלטול בעלמא הוא דלית ביה מלאכה דשמעת מינה דכי אמרינן אמירה לנכרי שבות בדבר שהוא מלאכה אבל בדבר שאינו מלאכה כגון האי לא אמרינן ביה אמירה לנכרי שבות
הרי נתברר לך דליכא בנוסחאי טעותא ומילי דברירן אינון והאי דאמרי׳ שבות דלית ביה מעשה ולא אמרינן שבות דלית ביה מלאכה מעשה ומלאכה בהאי ענינא חדא מילתא היא דקרו אינשי למלאכה מעשה כדאמרינן ובין יום השביעי לששת ימי המעשה והיינו מלאכה וכבר נתברר לך דאסור למימר לנכרי לצבותיה או לאתויי דרך רשות הרבים והאי דכתב בעל הלכות טעותא הוא ולא תסמוך עילויה:
ונכרי אסור למימהל דגרסינן במס׳ ע״ז (דף כז:) איתמר מנין למילה }}בנכרי שהיא פסולה דארו בר פפא משמיה דרב אמר כדכתיב (בראשית י״ז:ט׳) ואתה את בריתי תשמור ור׳ יוחנן אמר (בראשית י״ז:י״ג) המול ימול יליד ביתך קרי ביה המל ימול מאי בינייהו איכא בינייהו אשה לרב דאמר ואתה את בריתי תשמור אשה כיון דליתא בברית לא מהלא לר׳ יוחנן דאמר המל ימול כיון }דישראל הוא אפילו ערלים כמהולים דמו איתתא נמי בכלל ישראל היא ומהלא הלכך היכא דליכא גברא יהודאה דידע למימהל ואיכא איתתא יהודיתא דידעה למימהל ומהלא שפיר דמי דהלכה כרבי יוחנן דקיימא לן רב ורבי יוחנן הלכה כרבי יוחנן:
\textbf{סליקו להו ר״א דמילה} 
\textbf{{\largeרבי}} אליעזר אומר תולין את }}}המשמרת ביום טוב ונותנין לתלויה בשבת וחכמים אומרים אין תולין המשמרת ביום טוב ואין נותנין לתלויה בשבת אבל נותנין לתלויה ביום טוב:
\textbf{{\largeגמ׳}} וחכ״א וכו׳: איבעיא להו תלה }מאי אמר רב יוסף תלה חייב חטאת א״ל אביי אלא מעתה תלה כוזא בסיכתא הכי נמי דמחייב (דף קלח.) אלא אמר אביי מדרבנן של יעשה כדרך שהוא עושה בחול וכן הלכה:
מנקיט אביי חומרי מתנייתא ותאני הגוד והמשמרת כילה וכסא גליין
\end{multicols}\newpage

\newchap{פרק \hebrewnumeral{20} תולין}
\begin{multicols}{2}
לא יעשה ואם עשה פטור אבל אסור }אבל אהל קבע לא יעשה ואם עשה חייב חטאת אבל מטה וכסא וטרסקל *}בס״י ובגמ׳ והאסלא}והאיסקלא מותר לנטותן בשבת לכתחלה:
ואין נותנין לתלויה בשבת: איבעיא להו שימר מאי אמר רב כהנא שימר חייב חטאת ומשום מאי מתרין ביה רבה אמר משום בורר ור׳ זירא אמר משום מרקד:
תאני רמי בר יחזקאל }}טלית }כפולה לא יעשה ואם עשה פטור אבל אסור ואם כרך עליה }חוט או משיחה מותר לנטותה לכתחלה פי׳ }כגון טלית כפולה שקושרה בין שני כתלים והיא משולשלת ומגעת לארץ ונכנס בין שני קצותיה וישן תחתיה בצל ואין בגגה טפח ולא בפחות מג׳ סמוך לגגה טפח ולפיכך אינו אהל קבע אלא אהל עראי הוא ומפני שאין בגגה טפח ולא בפחות מג׳ סמוך לגגה טפח }ולפיכך פטור אבל אסור ואם היו עליו חוטין מאתמול ונטה אותם היום מותר:
אמר רב יוסף חזינא להו לכילי דבי רב הונא }באורתא נגידן ולצפרא }חביטן רמיאן:
אמר רב כהנא חזינא להו לכילי דבי רב פפא ורב הונא בריה דרב יהושע דמאורתא נגידן ולצפרא חביטן רמיאן:
אמר רב משום רבי חייא וילון מותר לנטותו ומותר לפרקו פירוש וילון מסך:
ואמר שמואל משום רבי חייא (דף קלח:) כילת חתנים מותר לנטותה ומותר לפרקה:
אמר רב שישא בריה דרב אידי האי }}}סיאנא שרי פי׳ }כובע שמשימין אותו בני אדם על ראשיהם ומאהיל על פניהם מפני השמש והא אמרת סיאנא אסור לא קשיא הא דאית ביה טפח הא דלית ביה טפח אלא מעתה אפיק בגלימיה טפח הכי נמי דמיחייב אלא לא קשיא הא דמיהדק והא דלא מיהדק
שלח ליה רמי בר יחזקאל לרב הונא אימא לן הנך מילי מעלייתא דאמרת לן משמיה דרב תרתי בשבתא וחדא בתורה שלח ליה הא דתניא }הגוד בכיסכיו מותר לנטותו בשבת אמר רב לא שנו אלא בשני בני אדם אבל באדם אחד אסור אמר אביי כילה אפי׳ בעשרה בני אדם אסור אי אפשר דלא }מינתקה פורתא ואידך מאי היא דתניא כירה שנשמטה אחת מירכותיה מותר לטלטלה בשבת שתים אסור לטלטלה בשבת ורב אמר אפילו אחת אסור לטלטלה בשבת גזירה שמא יתקע והלכתא כרב:
תורה מאי
\end{multicols}\newpage

\newsection{דף נז}
\begin{multicols}{2}
היא דאמר רב עתידה תורה שתשתכח מישראל שנאמר (דברים כח) והפלא ה׳ את מכותך הפלאה זו איני יודע מה היא כשהוא אומר (ישעיהו כ״ט:י״ד) לכן הנני יוסיף להפליא את העם הזה הפלא ופלא ואבדה חכמת חכמיו ובינת נבוניו תסתתר הוי אומר הפלאה זו תורה:
(דף קלט.) שלחו ליה בני בשכר ללוי כילה מאי כשותא בכרמא מהו מת ביום טוב מהו עד דאזיל שליחא נח נפשיה דלוי אמר ליה שמואל לרב מנשיא אי חכימת שלח להו שלח להו חזרנו על כל צדדי כילה ולא מצינו לה צד היתר ולישלח להו כדרמי בר יחזקאל }לפי שאינן בני תורה }}}כשותא בכרמא עירבוביא ולישלח להו כרבי טרפון }דתנן כשותא ר׳ טרפון אומר אין כלאים בכרם וחכמים אומרים }יש כלאים ככרם וקיימא לן כל המיקל בארץ הלכה כמותו בחוצה לארץ לפי שאינן בני תורה:
}}}מת ביום טוב לא יתעסקין ביה לא יהודאין ולא ארמאין לא ביו״ט ראשון ולא ביום טוב שני }והאמר רבא (דף קלט:) מת ביום טוב ראשון יתעסקו בו עממים מת ביום טוב שני יתעסקו בו ישראל ואפילו בשני ימים טובים של ראש השנה מה שאין בן בביצה לפי שאינן בני תורה:
אמר רב הונא אמר רב חמא בר גוריא אמר רב מתעטף }}}אדם בכילה }וכיסכסיה ויוצא בה לרשות הרבים בשבת ואינו חושש ומאי שנא מדרב הונא דאמר רב הונא היוצא בטלית }שאינה מצויצת כהלכתה בשבת חייב חטאת.
ציצית לגבי טלית חשיבי ולא בטלי הני לא חשיבי ובטלי אמר רבה בר רב הונא }מערים אדם על }}}המשמרת ביו״ט לתלות בו רמונים ותולה בו שמרים אמר רב הונא }והוא דתלה בו רמונים }ומאי שנא מהא דתנן מטילין שכר במועד לצורך המועד ושלא לצורך המועד אסור אחד שכר תמרים ואחד שכר שעורים ואף על פי שיש לו שכר ישן מערים ושותה מן החדש ומותר התם לא מוכחא מילתא הכא מוכחא מילתא:
אמרו ליה רבנן לרב אשי חזי מר האי }}}צורבא מרבנן ורב הונא בר חיון שמיה ואמרי לה רב הונא בר חליון שמיה דשקיל ברא דתומא ומנח בברזא דדנא ואמר לאצנועי קא מיכוינא ואזיל ונאים במעברא ועבר להך גיסא וסייר פירי ואמר למינם קא מיכוינא אמר להו הערמה }בדרבנן היא וצורבא מרבנן לא אתי למיעבד לכתחלה:
\textbf{{\largeמתני׳}} }}נותנין מים ע״ג שמרים בשביל שיצולו ומסננין את היין בסודרין ובכפיפה מצרית ונותנין ביצה במסננת של חרדל ועושין יינמלין בשבת ר׳ יהודה אומר בשבת בכוס ביום טוב בלגין ובמועד בחבית רבי צדוק אומר הכל לפי האורחין:
\textbf{{\largeגמ׳}} אמר זעירי נותן אדם }יין צלול ומים צלולין לתוך המשמרת בשבת ואינו חושש אבל עכורין לא ואם היה בין הגתות שעדיין היין תוסס ואינו צלול וגם הוא }עכור טורד חבית יינה ושמריה ונותן לתוך המשמרת בשבת ואינו חושש:
ומסננין את היין בסודרין: אמר רב שימי בר חייא ובלבד שלא יעשה גומא:
ובכפיפה מצרית: אמר רב חייא בר אשי אמר רב ובלבד שלא יגביה מקרקעיתו של כלי טפח:
אמר רבה האי }פרוונקא }}}אפומא דכובא כולה אסיר אפלגא דכובא שרי פירוש פרוונקא מטלית אמר רב פפא לא ליהדיק איניש }צבאתא אפומא דכוזניתא משום דמחזי כמשמרת:
דבי רב פפא שפו שיכרא ממנא למנא }בצבאתא א״ל רב אחא מדפתי לרבינא הא איכא ניצוצות ניצוצות בי רב פפא לא חשיבי:
ונותנין ביצה במסננת: }(דף קמ.) תני
\end{multicols}\newpage

\newsection{דף נח}
\begin{multicols}{2}
יעקב קרחה לפי שאין עושין אותה אלא לגוון:
איתמר חרדל שלשו מערב שבת למחר אמר רב }ממחו ביד ואין ממחו בכלי ושמואל אמר ממחו בכלי ואין ממחו ביד אר״א אחד זה ואחד זה אסור ורבי יוחנן אמר אחד זה ואחד זה מותר:
אמר מר זוטרא לית הלכתא ככל הני שמעתא אלא כי הא דאיתמר }}}חרדל שלשו מערב שבת למחר ממחו בין ביד בין בכלי ונותן לתוכו דבש ולא יטרוף אלא מערב שחליים ששחקן מערב שבת למחר נותן לתוכן שמן וחומץ וממשיך לתוכן }אמיתא ולא יטרוף אלא מערב:
שום שריסקו מערב שבת למחר נותן לתוכו פול וגריסין ולא ישחוק אלא מערב וממשיך אמיתא לתוכה מאי אמיתא ניניא אמר אביי ש״מ האי ניניא מעליא לתחלי:
ועושין יינמלין: }ת״ר עושין יינמלין בשבת ואין עושין אלונטית איזו היא יינמלין ואיזו היא אלונטית יינמלין יין ודבש ופלפלין אלונטית יין ישן ומים צלולין ושמן אפרסמון דעבדי ליה לבי מסותא:
\textbf{{\largeמתני׳}} אין שורין את החלתית בפושרין אבל נותנן בתוך החומץ ואין שורין את הכרשינין ואין שפין אותן אבל נותן הוא לתוך הכברה או לתוך הכלכלה אין כוברין את התבן בכברה ולא יניחנה במקום גבוה בשביל שירד המוץ אבל נוטל הוא בכברה ונותן לתוך האיבוס:
\textbf{{\largeגמ׳}} בעא מיניה רבי יוחנן מרבי ינאי מהו לשרות החלתית בצונן אמר ליה אסור והא אנן תנן אין שורין את החלתית בפושרין הא בצונן מותר אמר ליה מתני׳ יחידאה היא דתניא אין שורין את החלתית בין בחמין בין בצונן ר׳ יוסי אומר בצונן מותר בחמין אסור והלכתא כר׳ ינאי דאמר בין בחמין בין בצונן אסור. ואי שרה מאתמול מותר לשתותו בשבת:
ואי שתה יום חמישי ויום ששי מקודם השבת ובשבת לא היה לו שרוי והיה חושש שאם לא ישתה שמא יסתכן מותר לשרות בצונן ומניח בחמה כדי שלא יסתכן כמעשה דרבי חייא בר אבין:
}}מסתמיך ואזיל ר׳ חייא בר יוסף אכתפי׳ דרב נחמן בר יצחק בר אחתיה אמר ליה כי מטינן לבי רב חסדא עיילן כי מטא עייליה בעא מיניה מהו לכסכוסי כתונתא בשבתא לרכוכא קא מכוין ושרי או דילמא לאולודי חיורא קא מכוין ואסיר אמר ליה לרכוכא קא מכוין ושפיר דמי כי נפק אתא א״ל מאי בעי מר מיניה א״ל בעאי מיניה מהו לכסכוסי כתונתא בשבתא ואמר לי שפיר דמי
אמר רב חסדא האי }}כתונתא }(דף קמ:) משלפא לדידה מקניא שרי קניא מיניה אסירא דקניא לאו מנא הוא אמר רבא }אי כלי קיואי הוא מותר (ושרי) פירוש כלי קיואי כלי }אורג אמר רב חסדא האי }}}}הוצא דירקא אי חזיא למאכל בהמה שרי לטלטולה ואי לא אסיר אמר רב חייא בר אשי אמר רב האי תליא דבשרא שרי לטלטוליה אבל }דכוורי אסיר דכיון דסני ריחיה אסוחי אסח דעתיה מיניה והוה ליה מוקצה מחמת מיאוס ומשום הכי אסור לטלטולה }רב לטעמיה דסבירא ליה כרבי יהודה אבל לר״ש }מוקצה מחמת מיאוס לית ליה }והלכתא כותיה אמר רב חסדא בר בי רב לא לשדר מאניה לאושפיזיה לאחוורי ליה משום דלאו אורח ארעא:
\textbf{{\largeמתני׳}} }}}גורפין מלפני הפטם ומסלקין לצדדין מפני הרעי דברי רבי דוסא וחכמים אוסרין ונוטלין מלפני בהמה זו ונותנין לפני בהמה זו בשבת:
\textbf{{\largeגמ׳}} איבעיא להו רבנן ארישא פליגי או אסיפא פליגי }תא שמע דתניא וחכמים אומרים אחד זה ואחד זה לא יסלקנו
לצדדין שמע מינה אתרויהו פליגי:
אמר רב חסדא מחלוקת באבוס של כלי אבל באבוס של קרקע אסור לדברי הכל:
ונוטלין מלפני בהמה וכו׳. תאני חדא נוטלין מלפני בהמה שפיה יפה ונותנין לפני בהמה שפיה רע ותניא אידך נוטלין מלפני בהמה שפיה רע ונותנין לפני בהמה שפיה יפה אמר אביי אידי ואידי מקמי תורא לקמי חמרא לא שקלינן מקמי חמרא לקמי תורא שקלינן והאי דקתני נוטלין מלפני בהמה שפיה יפה בחמרא דלית ליה רירא ונותנין לפני בהמה שפיה רע בפרה (דף קמא:) דאית לה רירי והאי דקתני נוטלין מלפני בהמה שפיה רע בחמור דלא דייק ואכיל ונותנין לפני בהמה שפיה יפה בפרה דדייקא ואכלא:
\textbf{{\largeמתני׳}} }}הקש שעל המטה לא ינענענו בידו אבל מנענעו בגופו ואם היה מאכל בהמה או שהיה עליו כר או סדין מנענעו בידו מכבש של בעלי בתים מתירין אבל לא כובשין ושל כובסין לא יגע בהן ר׳ יהודה אומר אם היה מותר מערב שבת }מתיר את כליו ושומטן:
\textbf{{\largeגמ׳}} אמר רב נחמן האי פוגלא מלמעלה למטה שרי מלמטה למעלה אסור }דקסבר טלטול מן הצד שמיה טלטול ולית הלכתא הכי בין מלמטה למעלה בין מלמעלה למטה שרי דהא הדר ביה רב נחמן }בפרק כל הכלים (דף קכג.) ואמר טלטול מן הצד לא שמיה טלטול.
אמר רב יהודה הני }פלפלי }למידק חדא חדא בקתא דסכינא שרי תרי תרי אסיר רבא אמר כיון דקא משני אפילו טובא נמי איכא דמוקים לה בשבת, ואיכא דמוקים לה ביום טוב ומסתברא כמאן דמוקים לה בשבת דאי ביו״ט ליכא מאן דאמר דבעינן למידק חדא חדא בקתא דסכינא ואמר רב יהודה }האי מאן דסחי במיא }לינגוב נפשיה דילמא אתי לאיתויי ארבע אמות בכרמלית אי הכי כי קא נחית נמי קא דחי ד׳ אמות ואסיר כחו בכרמלית לא גזרו ביה רבנן.
}}אמר *}ס״א רבא}אביי ואיתימא רב יהודה טיט שעל }גבי רגלו מקנחו בקרקע אבל לא בכותל אמר אביי בכותל מאי טעמא לא משום דמיחזי כבונה האי בנין חקלאה הוא אלא אמר רבא מקנחו בכותל אבל לא בקרקע דילמא אתי לאשוויי גומות:
איתמר מר בריה דרבנא אמר אחד זה ואחד זה אסור רב פפא אמר אחד זה ואחד זה מותר }אלא מר בריה דרבנא במאי מקנח ליה בקורה איכא מאן דאמר דהלכה כמר בריה דרבנא משום דקא מפרשינן למיליה, ובעל הלכות פסק }*}ס״א כרבא}כרב פפא ואמר רבא }לא ליצדד איניש כובא בארעא דילמא אתי לאשוויי גומות אמר רבא לא לישתמש איניש אפומא דלחיא דילמא מיגנדר ליה חפץ ואתי לאיתויי:
ואמר רבא לא }}ליהדק איניש אודרא אפומא דאשישה דילמא אתי לידי סחיטה:
אמר רב }}כהנא טיט שעל גבי בגדו מכסכסו מבפנים ואין מכסכסו מבחוץ ומותר לגוררו בצפורן:
תאני ר׳ חייא אין מגרדין לא מנעל חדש }ולא מנעל ישן ולא יסוך את רגלו שמן והוא בתוך
\end{multicols}\newpage

\newsection{דף נט}
\begin{multicols}{2}
המנעל או בתוך הסנדל אבל סך את רגלו שמן ומניחו בתוך המנעל או בתוך הסנדל וסך כל גופו }שמן ומתעגל ע״ג קרטיבלא ואינו חושש אמר רב חסדא לא שנו אלא שיעור לצחצחו אבל שיעור לעבדו אסור:
ת״ר לא }}יצא אדם במנעל גדול אבל יוצא הוא בחלוק גדול ולא תצא אשה במנעל מרופט ולא }}תחלוץ בו ואם חלצה חליצתה כשרה ולא יצא במנעל חדש }ובאיזה מנעל אמרו במנעל של אשה תאני בר קפרא ל״ש אלא שלא יצאת בו שעה אחת מבעוד יום אבל יצאת בו שעה אחת מבעוד יום }מותר }}ומנעל שע״ג האימוס שומטין אותו בשבת }פירוש אימוס דפוס של מנעל גרסי׳ בזבחים בפרק דם חטאת (דף צד:) דרש רבא }מותר לכבס את המנעל אמר ליה רב פפא לרבא והא }זימנין סגיאין הוה קאימנא קמיה דרבא ושכשכו ליה מסאניה במיא שיכשוך אין כיבוס לא }הדר לפניך טעות הן בידי ברם כך אמרו שיכשוך אוקי רבא אמורא עליה ודרש דברים שאמרתי מותר כיבוס אסור:
\textbf{סליקו להו תולין} 
\textbf{{\largeנוטל}} }}אדם את בנו והאבן בידו, והכלכלה והאבן }בתוכה ומטלטלת תרומה טמאה עם הטהורה ועם החולין רבי יהודה אומר אף מעלין את המדומע באחד ומאה:
\textbf{{\largeגמ׳}} אמרי דבי רבי ינאי בתינוק שיש לו געגועין על אביו (דף קמב.) ודוקא אבן דאי נפלה לא אתי לאיתויי אבל דינר לא:
(דף קמב:) \textbf{{\largeמתני׳}} האבן שעל פי החבית מגביה ומטה על צדה והיא נופלת מאליה, היתה בין החביות מגביה ומטה על צדה והיא נופלת מאיליה, }}מעות שעל הכר מנער את הכר והן נופלות מאיליהן היתה עליו לשלשת }מקנחה בסמרטוט היתה של עור נותן עליה מים עד
\end{multicols}\newpage

\newchap{פרק \hebrewnumeral{21} נוטל אדם את בנו}
\begin{multicols}{2}
שתכלה:
\textbf{{\largeגמ׳}} אמר רב הונא אמר רב לא שנו אלא }בשוכח אבל במניח נעשה בסיס לדבר האסור והלכתא כוותיה דרב דקאי רבי אמי בשם רבי יוחנן כוותיה בפרק כל הכלים (דף קכה:):
היתה בין החביות: תניא רבי יוסי אומר היתה חבית מונחת באוצר או שהיו כלי זכוכית מונחין }תחתיה מגביה׳ למקום אחר ומטה על צדה והיא נופלת ונוטל הימנה מה שהוא צריך לו ומחזירה למקומה:
מעות שעל הכר וכו׳: אמר רבי חייא בר אשי ל״ש אלא בשוכח אבל במניח נעשה בסיס לדבר האסור:
אמר רבה בר בר חנה אמר ר׳ יוחנן לא שנו אלא לצורך גופו אבל לצורך מקומו מטלטלן ועודן עליו:
אמר רב אושעיא שכח ארנקי בחצר מניח עליו ככר או תינוק ומטלטלו ולית הלכתא כוותיה דאמר רב אשי לא אמרו ככר או תינוק אלא למת בלבד וכן הלכה:
\textbf{{\largeמתני׳}} (דף קמג.) ב״ש אומרים }}}מעבירין מעל השלחן עצמות וקליפין ובית הלל אומרים מסלק את הטבלא ומנערה מעבירין מעל השלחן פירורין פחותין מכזית ושער של אפונים ושל עדשים מפני שהוא מאכל בהמה ספוג }}}אם יש לו עור בית אחיזה מקנחין בו ואם לאו אין מקנחין בו וחכ״א בין כך ובין כך ניטל בשבת ואינו מקבל טומאה:
\textbf{{\largeגמ׳}} אמר רב נחמן אנו אין לנו אלא בית שמאי כרבי יהודה וב״ה כרבי שמעון וכי קא שרי רבי שמעון לטלטולי כגון עצמות }דחזיין למאכל כלבים וקליפין דחזיין למאכל בהמה אבל מידי דלא חזי לא למאכל אדם ולא למאכל בהמה לא קא שרי רבי שמעון ותנו רבנן (לעיל קכח.) }מטלטלין את העצמות מפני שהן מאכל כלבים בשר תפוח מפני שהוא מאכל לחיה:
מעבירין מעל השלחן וכו׳: מסייע ליה לר׳ יוחנן דאמר }}פירורין שאין בהן כזית מותר לאבדן ביד:
גרעיני דתמרי ארמייתא שרי }}}לטלטלינהו הואיל וחזיין אגב אמייהו פירוש מפני שהן רכין ונאכלין עם התמרה עצמה דפרסייתא אסור לטלטל מפני שהן קשין ואין נאכלין עם התמרים פ״א תמרי ארמייתא קשין הן ובהמה אוכלתן עם הגרעינין ולפיכך מטלטלין אותן ותמרי פרסייתא רכין הן ואין הבהמה אוכלת אותן ולפיכך אין מטלטלין אותן ומסתברא דה״מ במקום שמאכילין את התמרים לבהמות כגון בבל ויריחו אבל במקום שאין מאכילין את התמרים לבהמות אלו ואלו אסור לטלטלן:
שמואל מטלטל להו אגב ריפתא שמואל לטעמיה דאמר עושה אדם כל צרכיו בפת רבה מטלטל להו
\end{multicols}\newpage

\newsection{דף ס}
\begin{multicols}{2}
אגב לקנא דמיא רב ששת זריק להו בלישניה רב פפא זריק להו אחורי המטה אמרו עליו על רבי זכריה בן אבטולס שהיה מחזיר פניו אחורי המטה וזורקן:
\textbf{סליקו להו נוטל} 
\textbf{{\largeחבית}} (דף קמג:) }}}שנשברה מצילין ממנה מזון ג׳ סעודות ואומר לאחרים בואו והצילו לכם ובלבד שלא יספוג אין סוחטין את הפירות להוציא מהן משקין ואם יצאו מעצמן אסורין ר׳ יהודה אומר אם לאוכלין היוצא מהן מותר ואם למשקין היוצא מהן אסור חלות דבש שריסקן מערב שבת ויצאו מעצמן אסורין ור״א מתיר:
\textbf{{\largeגמ׳}} תאנא ובלבד שלא }יספוג ביין ובלבד שלא יטפח בשמן שלא יעשה כדרך שהוא עושה בחול:
ת״ר נתפזרו לו }פירות בחצרו מלקט על יד על יד ואוכל אבל לא יתן לא לתוך הסל ולא לתוך הקופה שלא יעשה כדרך שהוא עושה בחול:
}}}אין סוחטין: }}אמר רב פפא א״ר יהודה אמר שמואל מודים חכמים לר׳ יהודה בשאר מיני פירות ומודה ר׳ יהודה לחכמים בזיתים וענבים א״ל ר׳ ירמיה לר׳ אבא א״כ במאי פליגי א״ל לכי תשכח אמר רב נחמן בר יצחק מסתברא בתותים ורמונים פליגי א״ר יוחנן }הלכה כרכי יהודה בשאר }פירות תניא סוחטין (דף קמד:) בפגעין ובפרישין ובעוזרדין אבל לא ברמונים }מפני שבית מנשיא בן מנחם סוחטין ברמונים בחול ועושין מהן יין ונמצאו הרמונים בני סחיטה הלכך }אסור לסוחטן בשבת א״ר נחמן הלכה כשל בית מנשיא בן מנחם
נמצא עכשיו הכלל הראוי מכל אלו הדברים שזיתים וענבים אין סוחטין אותן בשבת ואם יצאו מעצמן אסורין בין לאכילה בין למשקין לדברי הכל והתותים והרמונים אסור לסוחטן בשבת לדברי הכל ואם יצאו מעצמן אם לאוכלים היוצא מהן מותר ואם למשקין היוצא מהן אסור כרבי יהודה ושאר פירות כגון פגעין ופרישין ועוזרדין }סוחטין אותן לכתחלה בשבת לדברי הכל
ואי איכא מאן דסליק אדעתיה דרמונים בהדי זיתים וענבים הוו דהא אמר רב נחמן דהלכה כשל בית מנשיא בן מנחם ההיא לענין סחיטה לכתחלה איתמר דאין סוחטין אותן בשבת אבל אם יצאו מעצמן והן לאוכלין מותרין }לכתחלה כרבי יהודה והכי נמי מסתברא דקתני סוחטין בפגעין ובפרישין ובעוזרדין אבל לא ברמונים אלמא אסחיטה קאי ועוד מדיהיב טעמא כדרב חסדא דאמר תרדין שסחטן ונתנן בתוך המקוה פוסלין את המקוה בשינוי מראה ואמאי והא לאו בני סחיטה נינהו אלא מאי אית לך למימר כיון דאחשבינהו הוו להו משקה ה״נ כיון דאחשבינהו ה״ל משקה }ש״מ דאסחיטה קאי דלא מצית למימר דאחשבינהו אלא היכא דסחטן:
אמר רב יהודה אמר שמואל סוחט }אדם אשכול של ענבים לתוך הקדרה אבל לא לתוך הקערה פירוש הקדרה יש בה אוכל הלכך הוה ליה משקה הבא לאוכל וכאוכל דמי והקערה אין בה אוכל }}ולפיכך ה״ל משקה ואסור }איכא מאן דאמר האי מימרא לענין יו״ט איתמר ולא בשבת דהא קי״ל }במתניתא אין סוחטין
\end{multicols}\newpage

\newchap{פרק \hebrewnumeral{22} חבית}
\begin{multicols}{2}
את הפירות להוציא מהן את }המשקין ואם יצאו מעצמן אסורין ואוקימנא }בזיתים וענבים }דבין למשקין ובין לאוכלין היוצא מהן אסורין ולפיכך לא מיתוקם האי מימרא דשמואל אלא בי״וט וכן מצינו בעל הלכות }פסוקות שכתבה ביו״ט ולא כתבה בשבת
ואיכא מאן דאמר }האי מימרא דשמואל בשבת היא ומוקים לה למתניתין בסוחט לתוך הקערה דהוו להו משקין אבל לתוך הקדרה סוחט כשמואל ודיוקא דדייק מיניה }רב חסדא בלחוד הוא דהויא ביום טוב וכן הדעת נוטה ורב חננאל ז״ל כך היה דעתו אלא שאמר שדברי שמואל ורב חסדא }דדייקינן מינייהו לאו הלכה אינון והכי אמר והלכתא }כדיוקא דר׳ יוחנן (דף קמה.) }להא מתניתא }שמע מינה }סוחט אדם כבשין ושלקות לגופן אבל למימיהן אסור ואם סחט למימיהן נעשה כמי שסחט זיתים וענבים וחייב חטאת ולא הפריש בסחיטתן למימיהן בין קדרה לקערה אלא הכל אסור דהא }סחיטת זיתים וענבים בין לקדרה בין לקערה למימיהן הוא צריך וא״ר יוחנן שהסוחט כבשין ושלקות למימיהן כסוחט זיתים וענבים וחייב חטאת ומתוך אלו הדברים מתברר שאין הלכה לא כשמואל ולא כרב }דאמר סוחט אדם אשכול של ענבים ונותן לתוך הקדרה אבל לא לתוך הקערה
ויש מי שהעמיד דברי רב ושמואל ביו״ט מדדייק רב חסדא דברי רב ושמואל ואמר מדבריהם נלמוד חולב אדם עז לתוך הקדרה אבל לא לתוך הקערה לקיים דבריהם להיות הלכה ואינן דברים נכונים שאפי׳ ביו״ט אסור לינק בפיו מן הבהמה וכל שכן לחלוב כדגרסינן בפרק חרש שנשא פקחת (יבמות דף קיד.) אבא שאול אומר נוהגין היינו שיונקין מן בהמה טהורה ביו״ט ואקשינן עלה היכי דמי אי דליכא סכנתא אפילו ביו״ט אסור ואי דאיכא סכנתא אפילו בשבת מותר ומפרקינן לא צריכא דאית ביה צערא וסבר מפרק כלאחר יד הוא שבת דאיסור סקילה הוא גזרו ביה רבנן יו״ט דאיסור לאו הוא לא גזרו ביה רבנן הנה שאפילו ליינק מן הבהמה ביו״ט לא התירו אלא במקום צער כל שכן לחלוב במקום שאין צער שודאי אסור ואין לנו לאהדורי }אפירוקי ולאפוקי שמעתא מדוכתה אלא ודאי זה שאמר סוחט לתוך הקדרה אבל לא לתוך הקערה בשבת אמרה וזה שאמר רב חסדא ביום טוב הוא ואין הלכה כמותו לא בשבת ולא ביו״ט וכן קבלנו מרבותינו אלו הן דברי רב חננאל זצ״ל
ואנן עיינינן בהו ואשכחינן פירכא לכולהו חדא דהאי דאמר ולא הפריש בסחיטתן למימיהן בין לקדרה בין לקערה אלא הכל אסור לאו הכי הוא דהאי דאמר ר׳ יוחנן למימיהן חייב חטאת כענין מימרא }דרב ושמואל למימיהן פטור אבל אסור הוא }דכי היכי דהא דאמרי רב ושמואל למימיהן פטור אבל אסור לקערה הוא אבל לא לקדרה ה״נ הא דא״ר יוחנן למימיהן חייב חטאת לקערה הוא }אבל לא לקדרה תדע דהא רב ושמואל אינהו דאמרי סוחט אדם אשכול לתוך הקדרה אבל לא לתוך הקערה ואינהו דאמרי }בענין כבשין ושלקות למימיהן פטור אבל אסור ואי סלקא דעתך למ״ד למימיהן אסור ל״ש לקדרה ולא שנא לקערה הכל אסור קאמר א״כ קשיא דרב אדרב ודשמואל אדשמואל אלא מה אית לך למימר האי למימיהן אסור לתוך הקערה אמרי ולא לתוך הקדרה }הכי נמי לר׳ יוחנן למימיהן דקאמר חייב חטאת לתוך הקערה קאמר ולא לתוך הקדרה ולא פליג רבי יוחנן אדרב ושמואל אלא בפטור אבל אסור ובחיוב חטאת אבל בקערה וקדרה לא דאם כן לפלגו בהדיא בקערה וקדרה ומדלא אשכחן פלוגתייהו אלא במותר לרב ובפטור אבל אסור לשמואל ובחייב חטאת לר׳ יוחנן שמעית מיניה דלא איפליגו בקדרה וקערה
ועוד דלא אשכחן דר׳ יוחנן פליג עלייהו דרב ושמואל אלא בכבשין ושלקות אבל באשכול של ענבים לא אשכחן דפליג ומאן דאמר דפליג בעי ראיה והאי דקאמר דהאי סחיטת זיתים וענבים בין לקדרה בין לקערה למימיהן הוא צריך }והא״ר יוחנן שהסוחט כבשים ושלקות למימיהן בסוחט זיתים וענבים הוא וחייב חטאת גם זו אינה ראיה לפי שהסוחט ענבים לתוך הקדרה שיש בה אוכל אף על פי שהוא צריך למימיהן כיון שהוא סוחט לתוך האוכל אינו כסוחט משקין אלא כמפרר אוכל לתוך אוכל שהוא מותר ודא״ר יוחנן בסוחט כבשין ושלקות למימיהן שהוא כסוחט זיתים וענבים וחייב לא אמר אלא שהסוחט כבשין ושלקות לתוך הקערה כסוחט זיתים וענבים לתוך הקערה אבל לתוך הקדרה אלו ואלו מותרין שאוכל הן ולא משקה
ולעולם הלכה כשמואל לענין סוחט אדם אשכול ענבים לתוך הקדרה ורבי יוחנן לא איירי בסוחט באשכול כלל והא דאמר נמי שאפילו ליינק מן הבהמה בפיו בי״ט אסור וכ״ש לחלוב לאו הכי הוא שהיונק מן הבהמה הוא חמור מן החולב לתוך הקדרה ואע״פ שהוא מפרק כלאחר יד שהיונק בפיו כסוחט משקה הוא והחולב לתוך הקדרה כמפרר אוכל הוא ולפיכך מותר ועוד }הא }מותיב רמי בר חמא ומפרקינן לתיובתיה ומותיב נמי רבינא דהוא בתרא והוא מותיב לה והוא מפרק לה גם ר׳ ירמיה בקש להעמידה כתנאי ולא עמדה אלא }לדברי הכל משקה
\end{multicols}\newpage

\newsection{דף סא}
\begin{multicols}{2}
הבא }לאוכל אוכל הוא ומכל אלו נלמוד שהלכה כרב ושמואל הלכך הלכה נמי כרב חסדא דקא יליף מיניה דחולב }אדם עז בי״ט לתוך הקדרה אבל לא לתוך הקערה והכי חזינן לרב האיי גאון ז״ל כדפסיקנא ולית בה ספיקא:
גרסינן בפרק אף על פי (כתובות דף ס.) }}}תניא רבי מרינוס אומר גונח יונק חלב בשבת מאי טעמא מפרק כלאחר יד הוא ובמקום צער לא גזרו ביה רבנן אמר רב יוסף הלכה כר׳ מרינוס
תניא נחום איש גליא אומר צנור שעלו בו קשקשין ממעכן ברגלו בצנעא בשבת ואינו חושש מ״ט מתקן כלאחר יד הוא ובמקום פסידא לא גזרו ביה רבנן אמר רב יוסף הלכה כנחום איש גליא ואי קשיא לך ההוא דגרסינן בפרק חרש שנשא פקחת (יבמות דף קיד.) אבא שאול אומר נוהגין היינו שיונקין מבהמה טהורה ביו״ט ואמרינן היכי דמי אי דאיכא סכנה אפילו בשבת נמי שרי ואי דליכא סכנה אפילו ביום טוב נמי אסור ומפרקינן לא צריכא דאיכא צערא וקסבר מפרק כלאחר יד הוא שבת דאיסור סקילה הוא גזרו ביה רבנן יום טוב דאינו אלא בלאו לא גזרו ביה רבנן דשמעת מינה דגונח אף על גב דאית ליה צערא אסור ליה ליינק מבהמה בשבת ההיא לא קשיא מידי דההוא מימרא אליבא דאבא שאול הוא ולית הלכתא כותיה דהא פסק רב יוסף הלכתא כרבי מרינוס דאמר אפילו בשבת במקום צערא לא גזרו ביה רבנן
וחזינן מאן דקא מוקים לה להא דר׳ מרינוס בצערא דאית ביה סכנה כי היכי דלא תיקשי ליה האי דאבא שאול והאי פירוקא לאו מעליא הוא דאי הכי }הא דר׳ מרינוס צערא דאית ביה סכנתא הוא מאי איריא מפרק כלאחר יד אפילו מלאכה גמורה גבי סכנה שריא ועוד }האי דקאמר במקום צערא לא גזרו ביה רבנן במקום סכנה איבעי ליה למימר ועוד תינח גבי בני אדם דאית בהו סכנה גבי צנור דקתני מתקן כלאחר יד הוא ובמקום פסידא לא גזרו ביה רבנן התם מאי סכנה איכא הלכך ליתא להאי פירוקא
(דף קמה.) גופא }}}}כבשין שסחטן אמר רב לגופן מותר למימיהן פטור אבל אסור שלקות בין לגופן בין למימיהן מותר ושמואל אמר אחד כבשין ואחד שלקות לגופן מותר למימיהן פטור אבל אסור ור׳ יוחנן אמר אחד כבשין ואחד שלקות לגופן מותר למימיהן חייב חטאת ולית הלכתא כר׳ יוחנן לענין חיוב חטאת דהוה }ליה רב ושמואל בחדא שיטתא דליכא חיוב חטאת אלא בזיתים וענבים והוה ליה ר׳ יוחנן יחיד ואין דבריו של אחד במקום שנים ועוד הא דתנא דבי }מנשה מסייע להו דאמר דבר תורה אינו חייב אלא על דריכת זיתים וענבים בלבד ומסתברא לן דהלכתא כוותיה דשמואל בהא מילתא דקאי ר׳ יוחנן כוותיה לענין איסור וליכא בינייהו פלוגתא אלא לענין חיוב חטאת:
א״ר חייא בר אשי אמר רב דבר תורה אינו חייב אלא על דריכת זיתים וענבים בלבד וכן תאנא דבי מנשה דבר תורה אינו חייב אלא על דריכת זיתים וענבים בלבד ואין עד }}}}מפי עד כשר אלא לעדות אשה בלבד
איבעיא להו עד מפי עד לענין בכור מהו רב אסי אסר ורב אשי שרי א״ל רב אסי לרב אשי והא תאנא דבי מנשה אין עד מפי עד כשר אלא לעדות אשה בלבד א״ל תני לעדות שהאשה כשרה לה כי הא דתניא ר״ש בן קפוסאי אומר בכור }כהן צריך שנים מן השוק להעיד עליו רשב״ג אומר אפילו בנו ובתו ודוקא בנו ובתו אבל אשתו לא מ״ט אשתו כגופו דמיא רב יימר אכשר עד מפי }}עד בבכור קרי עליה מרימר יימר שרי בוכרא והלכתא עד מפי עד כשר בבכור פירוש בכור בהמה כלומר אם יבא עד מפי עד ויאמר כי מום זה שנפל בבכור זה אינו בידי אדם מתירין אותו להאכל מיד במומו על פיו:
חלות דבש }}שריסקן וכו׳: כי אתא רב אושעיא מנהרדעא אייתי מתניתא בידיה זיתים וענבים שריסקן מע״ש ויצאו מעצמן אסורין ור״א ור״ש מתירין ופסק בעל הלכות משמיה דרב צמח בר פלטוי ריש מתיבתא כר״א ורבי שמעון:
\textbf{{\largeמתני׳}} }}כל שבא }בחמין מלפני השבת שורין אותו בחמין בשבת וכל שלא בא בחמין מלפני השבת מדיחין אותו בחמין בשבת חוץ מן המליח הישן וקולייס האספנין שהדחתן בחמין זו היא גמר מלאכתן:
\textbf{{\largeגמ׳}} כל שבא בחמין וכו׳ מאי היא כגון תרנגולתיה דר׳ אבא פירוש שהיתה מלוחה ביותר
וכשמבקשין לאוכלה שורין אותה במים חמין ואם הובא בחמין מע״ש שורין אותה במים חמין בשבת ואם לאו אסור לשרותה בחמין בשבת:
וכל שלא בא בחמין מע״ש }אין שורין אותה וכו׳:
איבעיא להו הדיח מאי אמר רב יוסף הדיח חייב חטאת:
(דף קמו.) \textbf{{\largeמתני׳}} }}}שובר אדם את החבית לאכול ממנה גרוגרות ובלבד שלא יתכוין לעשותה כלי אין נוקבין מגופה של חבית דברי ר׳ יוסי וחכמים מתירין.
ולא יקבנה מצדה ואם היתה נקובה לא יתן עליה שעוה מפני שהוא ממרח אמר ר׳ יהודה מעשה בא לפני רבן יוחנן בן זכאי בערב ואמר חושש אני לו מחטאת:
\textbf{{\largeגמ׳}} תניא רשב״ג אומר מביא אדם חבית של יין ומתיז }את ראשה בסייף ומניחה לפני האורחים בשבת ואינו חושש:
תניא חותלות של תמרים ושל גרוגרות }מתיר ומפקיע וחותך בעו מיני׳ מרב ששת מהו למיברז חביתא בבורטיא בשבת לפיתחא קא מכוין ואסיר או דילמא לעין יפה קא מיכוין ושרי אמר להו לפיתחא קא מיכוין ואסיר:
אין נוקבין מגופה של חבית וכו׳: אמר רב הונא מחלוקת למעלה אבל מן הצד דברי הכל אסור והיינו דקתני ולא יקבנה בצדה וכן הלכתא ת״ר אין נוקבין נקב חדש לכתחלה בשבת ואם בא להוסיף מוסיף ויש אומרים אין מוסיפין ושוין שנוקבין נקב ישן לכתחילה בשבת (דף קמו:) דרש רב נחמן בר רב חסדא משמיה דרב נחמן הלכה כיש אומרים:
ושוין שנוקבין נקב ישן לכתחלה אמר רב יהודה אמר שמואל לא שנו אלא במקום העשוי לשמר אבל במקום העשוי לחזק אסור פירוש לשמר להוציא יין מן השמרים היכי דמי לשמר והיכי דמי לחזק אמר רב חסדא למעלה מן היין:
זהו לחזק }למטה מן היין זהו לשמר }רבה אמר אפי׳ למעלה מן היין נמי זהו לשמר והיכי דמי לחזק כגון שנקבה למטה מן השמרים והלכתא כרבה דתניא דמסייע }ליה:
גובתא }פירוש שפופרת רב אסר ושמואל שרי למיחתך לכתחלה כ״ע לא פליגי דאסיר אהדורי כ״ע לא פליגי דשרי כי פליגי דחתיכא ולא מיתקנא מאן דאסר סבר גזרה דילמא אתי למיחתך לכתחלה ומאן דשרי סבר לא גזרינן דילמא אתי למיחתך לכתחלה כתנאי אין חותכין שפופרת ביו״ט ואין צ״ל בשבת נפלה מחזירין אותה בשבת ואין צ״ל ביו״ט ור׳ יאשיה מיקל אהייא אילימא ארישא הא קא מתקן מנא אלא אסיפא ת״ק נמי מישרא קא שרי אלא דחתיכא ולא מיתקנא איכא בינייהו ת״ק סבר גזרי׳ ר׳ יאשיה סבר לא גזרי׳ דרש רב שישא בריה דרב אידי משמיה דרבי יוחנן הלכה כרבי יאשיה:
ואם היתה נקובה וכו׳: מישחא רב אסר ושמואל שרי מאן דאסר סבר גזרינן משום שעוה ומאן דשרי סבר לא גזרינן משום שעוה והלכתא כרב אמר טבות רישבא אמר שמואל האי טרפא דאסא אסיר פירוש אסור ליתן עלין של הדס }בתוך הנקב של חבית לקלח בו היין כדי שלא יהא שותת בדופני החבית מאי טעמא רב ירמיה מדפתי אמר גזירה
\end{multicols}\newpage

\newsection{דף סב}
\begin{multicols}{2}
משום מרזב רב אשי אמר גזירה שמא יקטום מאי בינייהו איכא בינייהו דקטים ומנח:
\textbf{{\largeמתני׳}} }}נותנין תבשיל לתוך הבור בשביל שיהא שמור ואת המים היפים ברעים בשביל שיצננו ואת הצונן }בחמין בשביל שיחמו:
מי }}}שנשרו כליו במים מהלך בהן ואינו חושש הגיע לחצר החיצונה שוטחן בחמה אבל לא כנגד העם:
\textbf{{\largeגמ׳}} אמר רב יהודה אמר רב כל }מקום שאסרו חכמים מפני מראית העין אפי׳ בחדרי חדרים אסור והא אנן תנן שוטחן בחמה אבל לא כנגד העם תנאי הוא כדתניא שוטחן בחמה אבל לא כנגד העם ר׳ אלעזר ור״ש אוסרין הא מילתא אע״ג דהויא לה סתם }מתני׳ ומחלוקת בברייתא הויא לה הלכה במחלוקת דברייתא דקאי רב כוותה דמקשי׳ מההיא דרב בפ״ק דע״ז (דף יב.) דקאמר מאי אינו נראה אילימא דלא מיתחזי }והאמר רב יהודה אמר רב כל מקום שאסרו חכמים מפני מראית העין אפי׳ בחדרי חדרים אסור אלמא הכי הלכתא (דף קמז.) אמר רב הונא }המנער }}}טליתו בשבת חייב חטאת ולא אמרן אלא בחדתי אבל בעתיקי לית לן בה ובחדתי נמי לא אמרן אלא באוכמי אבל בחיורי וסומקי לית לן בה והוא דקפיד עלייהו.
עולא איקלע לפומבדיתא חזנהו לרבנן דנפצי גלימייהו א״ל הא קא מחללי רבנן שבתא א״ל רב יהודה נפצו ליה באפיה אנן לא קפדינן ולא מידי:
אביי הוה קאי קמיה דרב יוסף א״ל הב לי כומתאי }הוה טלא עלה והוה קא מיחסם למיהבה ניהליה אמר ליה נפוץ ושדי אנן לא קפדינן ולא מידי:
א״ר יצחק בר יוסף א״ר יוחנן היוצא בטלית }}}מקופלת ומונחת לו על כתיפו בשבת חייב חטאת תניא נמי הכי סוחרי כסות היוצאין בטלית מקופלת ומונחת להן על כתפיהן חייבין חטאת ולא סוחרי כסות בלבד אמרו אלא כל אדם אלא שדרכן של סוחרי כסות לצאת בכך וחנוני היוצא במעות הצרורין לו בסדינו חייב חטאת ולא חנוני בלבד אמרו אלא כל אדם אלא שדרכו של חנוני לצאת בכך הרטנין יוצאין בסודר שעל כתפיהן בשבת ולא רטנין בלבד אמרו אלא כל אדם אלא שדרכן של רטנין לצאת בכך.
א״ר יהודה מעשה בהורקנוס בנו של ר׳ אליעזר בן הורקנוס שיצא בסודר שעל כתיפו בשבת אלא שנימא אחת קשורה לו על אצבעו ובא מעשה לפני חכמים ואמרו אע״פ שאין נימא קשורה לו על אצבעו דרש רב נחמן משמיה דרב חסדא משמיה דר׳ יוחנן אע״פ שאין נימא קשורה לו על אצבעו
עולא איקלע לבי איסי בן הינו בעא מיניה מהו לעשות מרזב בשבת א״ל הכי אמר ר׳ אלעזר אסור מאי }מרזב א״ר זירא כישי בבלייתא יש אומרים כי מרזב זה }כשמתעטף אדם בסדינו ומקפל }שני קצותיה כמין דרך קיפול ויניחנה על כתיפו השמאלי ויחזור ויקפל שאר טלית שמשולשל בצדו הימין ויניחנה על כתיפו הימין ונמצא טליתו מקופלת מכאן ומכאן ומשולשלת על כתיפו וחללו כנגד השדרה ונראה כמרזב ובענין הזה כישי בבלייתא שהיו קושרין שתי כריכות וקורין אותה }כישא:
ר׳ ירמיה הוה יתיב קמיה דרבי זירא א״ל הכי מאי א״ל אסור והכי מאי א״ל אסור אמר רב פפא נקוט האי כללא בידך כל אדעתא לכנופי אסור כל אדעתא כדי להתנאות בו שפיר דמי כי הא דרב שישא בריה דרב אידי מתנאה בסדינו [הוה]:
כי אתא רב דימי אמר פעם אחת יצא רבי לשדה והיו שני צידי טליתו מונחין לו על כתיפו אמר לפניו רבי יהושע בן זרוז בן חמיו של ר׳ מאיר בזו לא חייב רבי מאיר חטאת אמר לו דקדק רבי מאיר עד כאן
שלשל רבי את טליתו כי אתא רבין אמר לא היה יהושע בן זרוז אלא יהושע בן קפוסאי היה חתנו של רבי עקיבא ואמר לו בזו לא חייב רבי עקיבא חטאת אמר לו דקדק רבי עקיבא עד כאן שלשל רבי את טליתו כי אתא רב שמואל בר יהודה אמר נשאל איתמר:
\textbf{{\largeמתני}} }}הרוחץ במי מערה או במי טבריא }מסתפג אפילו בעשרה אלונטיות ולא יביא בידו אבל עשרה בני אדם מסתפגין באלונטית אחת פניהם ידיהם ורגליהם ומביאין אותה בידם:
\textbf{{\largeגמ׳}} קתני מי מערה דומיא דמי טבריא מה מי טבריא חמין אף מי מערה חמין:
הרוחץ דיעבד אין לכתחלה לא ואי קשיא לך הא דגרסינן בפרק שמונה שרצים (דף קט.) תנו רבנן רוחצין במי גדר במי חמתן במי טבריא וגרסינן נמי בפרק כירה (דף מ.) ראו שאין הדבר עומד התירו להם חמי טבריא וזיעה במקומה עומדת ושמע מינה דמותר לרחוץ במי טבריא לכתחלה והכא קתני הרוחץ דיעבד אין לכתחלה לא לא קשיא הא דאמרינן דיעבד אין לכתחלה לא לא איתמר אלא במי מערה והא דקתני מי טבריא לגבי מי מערה לגלויי עלייהו דמי מערה }מה מי טבריא חמין אף מי מערה חמין נינהו }דקתני להו גבייהו אבל מי טבריא מותר לרחוץ בהן אפילו לכתחלה ומי מערה מ״ט לא שרו לכתחלה משום דמערה מיטללא כדאמרינן בפרק שור שנגח את הפרה (דף נ:) מערה מרבעא ומטללא הלכך נפיש הבלא דידה ואתי לידי זיעה ומשום הכי לא שרו לכתחלה:
(דף קמז:) }ומסתפג אפי׳ בעשר׳ אלונטיות ולא יביאם בידו וכו׳:
אמר רבי חייא בר אבא אמר רבי יוחנן הלכה מסתפג אדם באלונטית ומביאה בידו לתוך ביתו ולא גזרינן דילמא אתי לידי סחיטה אמר ר׳ חייא בר אבא אמר ר׳ יוחנן האוליירין מביאין בלורי נשים לבי בני ובלבד שיתכסה בהן ראשו ורובו:
סבניתא א״ר אבין בר רב חסדא אמר רבי יוחנן צריך לקשר ב׳ ראשיה למטה מן הכתפיים אמר להו רבא לבני מחוזא כי מעבריתו מאני דבני חילא שרביבו להו למטה מן הכתפיים:
\textbf{{\largeמתני׳}} }}סכין וממשמשין אבל לא מתעמלין ולא מתגרדין אין יורדין לפילומא ואין עושין אפיקטויזין בשבת ואין מעצבים את הקטן ואין מחזירין את השבר:
מי שנפרקה ידו או רגלו לא יטרפם בצונן אבל }מטבל הוא כדרכו ואם נתרפא נתרפא:
\textbf{{\largeגמ׳}} ת״ר סכין }וממשמשין בבני מעים בשבת ובלבד שלא יעשה כדרך שהוא עושה בחול:
והיכי עביד א״ר חמא בר חנינא סך ואח״כ ממשמש ור׳ יוחנן אמר סך וממשמש בבת אחת והלכתא כרבי יוחנן גרסינן בסנהדרין בפרק חלק (דף קא.) ת״ר סכין וממשמשין בבני מעים בשבת ולוחשין לחישת נחשים ועקרבים בשבת ומעבירין כלי על גבי העץ בשבת.
אמר רשב״ג בד״א בכלי הניטל בשבת אבל כלי שאינו ניטל בשבת אסור תנו רבנן אין שואלין דבר מן השד בשבת ר׳ יוסי אומר אף בחול אסור אמר רב הונא *}גי׳ ד״ת}[אין] הלכה כרבי יוסי ואף ר׳ יוסי לא אמר אלא *}גי׳ ד״ת מפני}בשעת הסכנה:
(מכילתין דף קמז:) אבל לא מתעמלין אמר רבי חייא אמר ר׳ יוחנן אסור לעמוד בקרקעיתה של דיומסית מפני שמתעמלת ומרפאת:
ולא מתגרדין. ת״ר אין גורדין במגרדות בשבת ר״ש אומר אם היו *}גי׳ ד״ת רגליו}ידיו מלוכלכות בטיט ובצואה גורד כדרכו ואינו חושש והלכתא כוותיה:
אין יורדין לפילומא: מאי טעמא משום נקא פירוש בקעה ויש שם מים
\end{multicols}\newpage

\newsection{דף סג}
\begin{multicols}{2}
ותחתיו טיט כמו דבק ואם ירד אדם שם חוששין שמא יטבע באותו הטיט וידבק שם ואינו יכול לעלות עד שמתקבצין בני אדם ומעלין אותו משם ויש מי שאומר הרוחץ באותה בקעה מצטנן ואותן המים משלשלין את בני המעים:
ואין עושין אפיקטויזין בשבת: אמר רבה בר בר חנה א״ר יוחנן לא שנו אלא בסם אבל ביד מותר תניא ר׳ נחמיה אומר אף בחול אסור מפני הפסד אוכלין ומסתברא דהני מילי היכא דליכא צער אבל היכא דאית ליה צער וכשמקיא אוכלין שבבטנו מתרפא מותר וכן אמר בעל הלכות משמי׳ דרב צמח בר פלטוי גאון ז״ל:
אין מעצבין את }}הקטן. אמר רבה בר בר חנה לפופי ינוקא בשבתא שפיר דמי והאנן תנן אין מעצבין את הקטן התם בחומרי שדרה דמיחזי כבונה:
ואין מחזירין את השבר וכו׳: אמר רב חנא בגדתאה אמר שמואל הלכה (דף קמח.) מחזירין את השבר בשבת וכן הלכה:
\textbf{סליקו להו חבית} 
\textbf{{\largeשואל}} }}}אדם מחבירו כדי יין וכדי שמן ובלבד שלא יאמר לו הלויני וכן שואלת אשה מחברתה ככרות ואם אינו מאמינו מניח }טליתו אצלו ועושה עמו חשבון לאחר השבת וכן ערבי פסחים בירושלים שחל להיות בשבת מניח טליתו אצלו }ואוכל פסחו ועושה עמו חשבון לאחר יו״ט:
\textbf{{\largeגמ׳}} א״ל רב נתן לאביי מ״ש הלויני ומ״ש השאילני א״ל השאילני לא אתי למיכתב (דלאו זמן מרובה הוא) אבל הלויני (דזמן מרובה הוא) אתי למיכתב איתמר }הלואת }יו״ט רב יוסף אמר לא ניתנה ליתבע }ורבא אמר ניתנה ליתבע הא מילתא אפליגו בה רבוואתא איכא מאן דפסק כרב יוסף ואמר משום דרב יוסף רביה הוה לגביה }דרבא ועוד דהא רב אויא סבר לה כוותיה דשקיל משכנתא ורבה בר }רב הונא נמי קא מערים אערומי שלא היה תובעו בפירוש אלא אמר לו הלויני וכשהלוהו אומר לו הרי יש לי אצלך כך וכך ואיכא מאן דפסק }כרבא דאמר ניתנה ליתבע ואמר משום דהוא בתרא וקי״ל בבתראי:
\textbf{{\largeמתני׳}} }}}מונה אדם את אורחיו ואת פרפרותיו מפיו אבל לא מן הכתב ומפיס אדם עם בניו ועם בני ביתו על השולחן ובלבד שלא יתכוין לעשות מנה גדולה כנגד מנה קטנה משום קוביא:
מטילין חלשים על הקדשים ביום טוב אבל לא על המנות:
\textbf{{\largeגמ׳}} (דף קמט.) מ״ט רב ביבי אמר גזירה שמא ימחוק אביי אמר גזירה שמא יקרא בשטרי }הדיוטות מאי בינייהו איכא בינייהו דכתיבי אכתלי ומדלאי
\end{multicols}\newpage

\newchap{פרק \hebrewnumeral{23} שואל}
\begin{multicols}{2}
למאן דאמר שמא ימחוק לא חיישינן למאן דאמר שמא יקרא חיישינן ולית הלכתא כרב ביבי משום דפליג עליה דרבה דאמר (דף יב:) ולא יקרא לאור הנר ואפילו גבוה שתי קומות ואפילו גבוה שתי מרדעות וקי״ל הלכתא כוותיה והא דתניא מונה אדם את אורחיו ואת פרפרותיו כמה בחוץ וכמה בפנים וכמה עתיד ליתן לפניהם מכתב שעל גבי הכותל אבל לא מכתב }שע״ג טבלא ופנקס אוקימנא בדחייק מיחק דגודא בשטרא לא מיחלף אבל כתב מיכתב אסור בין מדלאי בין מתתאי:
תניא }}אין רואין במראה בשבת ורבי מתיר במראה הקבועה בכותל ואסיקנא במראה של מתכת עסקינן וכדרב נחמן אמר רבה בר אבוה דאמר מפני מה אמרו מראה של מתכת אסור בשבת מפני שאדם עשוי להסיר בה נימין המדולדלין דחריפא כאיזמל ש״מ דמראה שאינה של מתכת בין קבועה בכותל בין שאינה קבועה שריא ושל מתכת }אסור כתנא קמא ואפילו קבועה בכותל:
תנו רבנן }}כתב המהלך תחת הצורה ותחת הדיוקנאות אסור לקרותו בשבת ודיוקנא עצמה אף בחול אסור להסתכל בה משום שנאמר (ויקרא י״ט:ד׳) אל }תפנו אל האלילים }ומאי תלמודא אמר רב אל תפנו אל מדעתכם:
מפיס }אדם עם בניו וכו׳. עם בניו ועם בני ביתו אין עם אחר }}לא מ״ט כדרב יהודה אמר שמואל דאמר רב יהודה אמר שמואל בני חבורה המקפידין זה על זה עוברין }משום מדה משום משקל ומשום מנין ומשום לווין ופורעין ביום טוב (דף קמט:) וכדברי בית הלל אף
\end{multicols}\newpage

\newsection{דף סד}
\begin{multicols}{2}
משום רבית:
מטילין }חלשים וכו׳. מאי אבל לא על המנות א״ר יעקב בריה דבת }יעקב אבל לא על המנות של חול ביו״ט:
(דף קנ.) \textbf{{\largeמתני׳}} }}}לא ישכור אדם פועלים בשבת ולא יאמר אדם לחבירו שכור לי פועלים בשבת אין מחשיכין על התחום בשבת לשכור לו פועלים ולהביא פירות אבל מחשיך הוא לשמור ומביא פירות בידו כלל אמר אבא שאול כל שאני זכאי באמירתו רשאי אני להחשיך עליו:
\textbf{{\largeגמ׳}} תניא לא יאמר אדם לחבירו הנראה שתעמוד עמי לערב ר׳ יהושע בן קרחה אומר אומר אדם לחבירו הנראה שתעמוד עמי לערב אמר רבה בר בר חנה אמר רבי יוחנן הלכה כר׳ יהושע בן קרחה דאמר קרא (ישעיה נט) ממצוא חפצך ודבר דבר }דבור אסור מחשבה מותרת והני מילי דבר הרשות הוא דאסור אבל דבור דמצוה מותר דאמר קרא ממצוא חפצך ודבר דבר חפציך אסורין חפצי שמים מותרין:
רב חסדא ורב המנונא דאמרי תרוייהו חשבונות של מצוה מותר לחשבן בשבת:
ואמר רבי אלעזר פוסקין צדקה לעניים בשבת ואמר ר׳ יעקב }אמר ר׳ יוחנן מפקחין פקוח נפש בשבת ואמר רבי יעקב }א״ר יוחנן הולכין לבתי כנסיות ולבתי מדרשות לפקח על עסקי רבים בשבת וא״ר שמואל בר נחמני א״ר יונתן הולכין לטרטיאות ולקרקסיאות לפקח על עסקי רבים בשבת }}ואני שמואל אומר דהכא לא גרסינן בשבת דאפילו בחול איצטריך למשרי משום פיקוח רבים דאי לאו פיקוח אסור למיזל כדאמרינן במסכת ע״ז משום דכתיב (תהילים א׳:א׳) ובדרך חטאים לא עמד ובמושב לצים לא ישב) תאנא דבי מנשה משדכין על התינוקות ליארס ועל התינוק ללמדו ספר וללמדו }אומנות:
אמר רב יהודה אמר שמואל חשבונות של מה לך ושל מה בכך מותר לחשבן בשבת תניא נמי הכי חשבונות שעברו וחשבונות שעתידין להיות אסור לחשבן בשבת וחשבונות של מה לך (דף קנ:) ושל מה בכך מותר בשבת ורמינהו מחשבין חשבונות שאינן צריכים ואין מחשבים חשבונות הצריכים כיצד אומר אדם לחבירו כך וכך פועלים השכרתי וכך וכך הוצאתי על דירה זו אבל לא יאמר כך וכך הוצאתי כך וכך אני עתיד להוציא ולטעמיך תקשי לך היא גופה אלא לא קשיא הא דאיכא }אגירי דאגירי גביה והא דליכא אגירי דאגירי גביה אי איכא אגירי דאגירי גביה אפילו חשבונות שעברו נמי אסור לחשבן דהא קא בעי למידע כמה בעי למיתב להו והוה להו חשבונות שצריכים ואסור ואי ליכא }אגירי דאגירי גביה הוו להו חשבונות שאינן צריכין ושרי
ת״ר מעשה בחסיד אחד שנפרצה לו פרצה בתוך ביתו ונמלך עליה לגודרה בשבת ונזכר שהוא שבת ולא גדרה ונעשה לו נס ועלה בו צלף וממנו היתה פרנסתו ופרנסת בני ביתו:
אמר רב יהודה אמר שמואל מותר לומר *}בגמ׳ לחברו}לכרך פלוני אני הולך למחר שאם יש שם בורגנין הולך אפילו בשבת:
אבל מחשיך הוא לשמור וכו׳: ומקשינן ואע״ג דלא אבדיל והאמר ר׳ אליעזר בן אנטיגנוס משום רבי אלעזר ברבי ינאי אסור לאדם }}}שיעשה חפציו }קודם שיבדיל וכי תימא דאבדיל בתפלה והאמר רב יהודה אמר שמואל המבדיל בתפלה צריך שיבדיל על הכוס וכי תימא דאבדיל על הכוס כוס בשדה מי איכא תרגמה רב נתן בר אמי קמיה דרבא בין הגתות שנו א״ל רב אדא לרב אשי כי הוינן במערבא }אמרי המבדיל בין קדש לחול ועבדינן }צרכינן כלומר אינו צריך לומר
בין אור לחשך כולהו אלא ברוך אתה י״י אמ״ה המבדיל בין קדש לחול בלבד: אמר רב אשי כי הוינן בי רב כהנא אמרינן המבדיל בין קדש לחול וסלתינן סלתי׳:
כלל אמר אבא שאול וכו׳ }(דף קנא.) אמר רב יהודה אמר שמואל מותר לו לאדם לומר לחבירו שמור לי }פירות שבתחומך ואני אשמור לך פירות שבתחומי דקתני }כל שאני זכאי באמירתו רשאי אני להחשיך עליו מדאחשוכי אחשיך שמור נמי עביד:
ת״ר }אין מחשיכין על התחום להביא בהמה היתה עומדת חוץ לתחום קורא לה והיא באה מחשיכין על התחום לפקח על עסקי כלה להביא לה הדס ועל עסקי המת להביא לו ארון ותכריכין ואומרין לו לך למקום פלוני ואם לא תמצא במקום פלוני לך למקום פלוני ואם לא תמצא במנה הבא במאתים ר׳ יוסי אומר ובלבד שלא יזכור לו סכום מקח:
}השוכר את }}הפועל לשמור לו את הפרה ולשמור לו את התינוק }אין נותנין לו שכרו של שבת לפיכך אין אחריות שבת עליו אם היה שכיר }שנה שכיר שבוע שכיר }שבת נותנין לו שכרו של שבת לפיכך אחריות שבת עליו ולא יאמר תן לי שכרי של שבת אלא אומר לו תן לי שכרי של עשרה ימים:
\textbf{{\largeמתני׳}} מחשיכין על התחום לפקח על עסקי הכלה ועל עסקי המת להביא לו ארון ותכריכין }נכרי שהביא חלילים בשבת לא יספוד בהן ישראל אא״כ באו ממקום קרוב עשו לו ארון וחפרו לו קבר יקבר בו ישראל }ואם בשביל ישראל לא יקבר בו עולמית:
\textbf{{\largeגמ׳}} מאי מקום קרוב רב אמר מקום קרוב ממש ושמואל אמר חיישינן שמא חוץ לחומה לנו כלומר אע״ג דחזינן להו
\end{multicols}\newpage

\newsection{דף סה}
\begin{multicols}{2}
דעיילי בצפרא לא אמרי׳ אי לאו }דעיילי מאתר קריבא לא הוו עיילי בצפרא אלא אמרינן }הכי מאתרא רחיקא אתו והאי דאתו בצפרא בליליא אזלי עד דמטו לחומה וביתו התם וקא עיילי השתא ומאי אא״כ באו ממקום קרוב ה״ק לא יספוד בהן ישראל אלא ימתין }כדי שיבאו ממקום קרוב והלכה כשמואל דדייקא מתניתין כוותיה דתנן במכשירין (מכשירין פרק ב׳) עיר שישראל ונכרים דרין בתוכה והיתה בה מרחץ המרחצת בשבת אם רוב נכרים מותר לרחוץ בה מיד ואם רוב ישראל ימתין בכדי שיחמו חמין מחצה על מחצה ימתין בכדי שיחמו חמין ר׳ יהודה אומר באמבטי קטנה אם יש בה רשות רוחץ בה מיד (דף קנא.) מאי רשות א״ר יצחק ברי׳ דרב יהודה כגון אדם חשוב שיש לו י׳ עבדים שמחממין לו עשרה קומקומוסין בבת אחת מותר לרחוץ בה מיד
(עירובין דף לח:) ת״ר לא }}}יטייל אדם בתוך שדהו לידע מה היא צריכה כיוצא בו לא יטייל אדם על פתח מדינה כדי שתחשך ויכנס למרחץ מיד:
\textbf{{\largeמתני׳}} }}}עושין כל צרכי המת סכין ומדיחין אותו ובלבד שלא יזיז בו אבר ושומטין את הכר מתחתיו ומטילין אותו על החול (דף קנא:) בשביל שיצטנן וקושרין את הלחי לא שתעלה אלא שלא תוסיף וכן קורה שנשברה סומכין אותה בספסל או בארוכות המטה לא שתעלה
אלא שלא תוסיף:
\textbf{{\largeגמ׳}} ת״ר מביאין כלי מיקר וכלי מתכות ומניחין לו על כריסו כדי שלא תפוח ופוקקין את נקביו כדי שלא תכנס בהן הרוח ואף שלמה אמר בחכמתו (קהלת י״ב:ו׳) עד אשר לא ירתק חבל הכסף זה חוט השדרה ותרוץ גולת הזהב זה האמה ותשבר כד על המבוע זה כרס ונרוץ הגלגל אל הבור זה פרש שנאמר (מלאכי ב׳:ג׳) וזריתי פרש על פניכם פרש חגיכם אמר רב הונא אלו בני אדם שמניחין דברי תורה ועושין כל ימיהם כחגים ר׳ לוי אמר משום ר׳ יהושע דסיכני לאחר ג׳ ימים כריסו של אדם נבקעת *}בגמ׳ הגי׳ ונופלת}והופכת לו על פניו ואומרת לו טול מה שנתת בי:
\textbf{{\largeמתני׳}} אין מעמצין את המת בשבת ולא }}בחול עם יציאת נפש והמעמץ עם יציאת נפש הרי זה שופך דמים:
\textbf{{\largeגמ׳}} ת״ר לא יעמץ אדם עיניו של מת עם יציאת נפש וכל המעמץ עם יציאת נפש הרי זה שופך דמים משל לנר שכבתה והולכת אדם מניח *}בגמ׳ הגי׳ אצבעו}ידיו עליה מיד כבתה:
תניא ר״ש בן אלעזר אומר הרוצה שיתעמצו עיניו של מת נופח לו יין בחוטמו ונותן לו שמן בין ריסי עיניו ואוחז בשני }גודליו והן מתעמצות מאיליהן:
תניא ר״ש בן אלעזר אומר תינוק בן יומו חי מחללין עליו את השבת אמרה תורה חלל עליו שבת אחת כדי שישמור שבתות הרבה דוד מלך ישראל מת אין מחללין עליו את השבת שנאמר (תהלים פח) במתים חפשי כיון שמת אדם נעשה חפשי מן המצות:
תניא ר״ש בן אלעזר אומר תינוק בן יומו חי א״צ לשמרו מן העכברים עוג מלך הבשן מת צריך לשמרו שנאמר (בראשית ט׳:ב׳) ומוראכם וחתכם יהיה על כל חית הארץ הכי קאמר כל זמן שאדם חי מוראו מווטל על הבריות כיון שמת ניטל מוראו ממנו:
אמר רב פפא נקטינן אריא אבי תרי לא נפיל והא קא חזינא דנפיל ההוא *}בגמ׳ הגי׳ כדרמי בר אבא}כדר׳ אמי בר אבא דאמר אין חיה רעה }שולטת באדם אא״כ נדמה לה כבהמה שנא׳ (תהילים מ״ט:י״ג) אדם ביקר בל ילין נמשל כבהמות נדמו א״ר יוחנן אסור לאדם לישן בבית בלילה יחידי וכל הישן בלילה יחידי אוחזתו לילית תניא ר׳ שמעון בן אלעזר אומר עשה (צדקה) *}בס״י ובגמ׳ ליתא}עד שאתה מוצא ומצוי בידך ועודך בידך כלומר ויש בידך לעשות ואף שלמה אמר בחכמתו (קהלת י״ב:א׳) וזכור את בוראך בימי בחורותיך עד אשר לא יבאו ימי הרעה אלו ימי הזקנה והגיעו שנים אשר תאמר אין לי בהם חפץ אלו ימי המשיח שאין בהם לא זכות ולא חובה ופליגא אדשמואל דאמר שמואל אין בין העולם הזה לימות המשיח אלא שעבוד מלכויות בלבד שנאמר (דברים ט״ו:י״א) כי לא יחדל אביון מקרב הארץ:
תניא ר׳ אלעזר הקפר אומר לעולם יבקש אדם רחמים על מדה זו כלומר על מדת עניות שאם לא בא הוא בא בנו ואם לא בא בנו בא בן בנו שנאמר (דברים ט״ו:י׳) כי בגלל הדבר הזה ותנא דבי ר׳ ישמעאל גלגל הוא שחוזר בעולם:
אמר רב יוסף נקטינן צורבא מרבנן לא מיעני והא קא חזינן דמיעני }אי איתא דמיעני אהדורי אפיתחא לא מיהדר:
אמר ליה רב אחא לדביתהו כי אתא עניא אקדימו ליה ריפתא כי היכי דליקדמו לבניך אמרה ליה מילט קא לייטת להו אמר לה קרא כתיב כי בגלל הדבר הזה ותאנא דבי ר׳ ישמעאל גלגל הוא שחוזר בעולם:
תניא רשב״ג אומר (דברים י״ג:י״ח) ונתן לך רחמים ורחמך כל המרחם על הבריות מרחמין עליו מן השמים וכל שאין מרחם על הבריות אין מרחמים עליו מן השמים
(דף קנג.) אמר רב יהודה בר שמואל בר שילת משמיה דרב מהספדו של אדם ניכר אם הוא בן עוה״ב ואם לאו איני והא״ל רב לרב שמואל בר שילת אחים לי הספדאי דהתם קאימנא לא קשיא הא דמחממו ליה וחאים הא דמחממו ליה ולא חאים א״ל אביי לרבה כגון מר דסנו ליה כולהו פומבדיתאי מאן מחאים הספדו אמר מיסתיא את ורבא בר רב חנין
בעא מיניה ר״א מרב איזהו בן העולם הבא א״ל (ישעיהו ל׳:כ״א) ואזניך תשמענה דבר מאחריך לאמר }רבי חנינא אומר כל שרוח רבותיו נוחה הימנו איכא דאמרי בעי מיניה רבי אלעזר מרב ורב מרבי חנינא איזהו בן העולם הבא א״ל כל שדעת רבותיו נוחה הימנו
תנן התם רבי אליעזר אומר שוב יום אחד לפני מיתתך שאלו תלמידיו את ר׳ אליעזר וכי אדם יודע מתי ימות ויעשה תשובה אמר להם כל שכן ישוב היום שמא ימות למחר ישוב למחר שמא ימות למחר ונמצא כל ימיו בעל תשובה ואף שלמה אמר בחכמתו (קהלת ט׳:ח׳) בכל עת יהיו בגדיך לבנים:
\textbf{סליקו להו שואל} 
\textbf{{\largeמי}} שהחשיך לו בדרך נותן }את }}}כיסו לנכרי אין עמו נכרי מניחו על החמור הגיע לחצר החיצונה נוטל את הכלים הניטלין בשבת ושאינן ניטלין מתיר את החבלים והשקים נופלין:
\end{multicols}\newpage

\newchap{פרק \hebrewnumeral{24} מי שהחשיך}
\end{multicols}\newpage

\newsection{דף סו}
\begin{multicols}{2}
\textbf{{\largeגמ׳}} מ״ט שרו ליה רבנן למיתן כיסו לנכרי קים להו לרבנן דאין אדם מעמיד עצמו על ממונו ואי לא שרית ליה אתי לאיתויי ד׳ אמות ברשות הרבים אמר רבא דוקא כיסו }אבל מציאה לא:
אם אין עמו נכרי מניחו על החמור: אם אין עמו נכרי מניחו על החמור: טעמא דאין עמו נכרי הא יש עמו נכרי }יהיב ליה מ״ט חמור אתה מצווה על שביתתו נכרי אי אתה מצווה על שביתתו חמור וחרש שוטה וקטן }לחמור יהיב ליה לחרש שוטה וקטן לא יהיב ליה מ״ט הני אדם }והאי לאו אדם חרש ושוטה לשוטה יהיב ליה לחרש לא יהיב ליה שוטה וקטן לשוטה יהיב ליה לקטן לא יהיב ליה
איבעיא להו }חרש וקטן מאי (דף קנג:) איכא דאמרי לחרש יהיב ליה ואיכא דאמרי לקטן יהיב ליה וכיון דלא איפסיקא הלכתא בהדיא דעבד כי האי לישנא עבד ודעבד כי האי לישנא עבד אין עמו }לא נכרי ולא חרש ולא שוטה ולא קטן ולא חמור מאי אמר ר׳ יצחק עוד אחרת התירו ולא רצו לגלותה מאי עוד אחרת מוליכו פחות פחות מד׳ אמות אמר מר אין עמו נכרי מניחו על החמור והלא מחמר הוא ורחמנא אמר (שמות כ) לא
תעשה כל מלאכה אמר רב אדא בר אהבה מניחו עליה כשהיא מהלכת סוף סוף אי אפשר דלא קיימא פורתא להשתין מים או להטיל גללים }והא קא אתי למיעבד }עקירה והנח׳ אלא כשהיא מהלכת מניחו עליה וכשהיא עומדת נוטלו מעליה:
אמר רב אדא בר אהבה }}}היתה חבילתו מונחת לו על כתיפו רץ תחתיה ולא נחית לה עד שמגיע לביתו ודוקא רץ אבל קלי קלי לא מ״ט כיון דליכא היכרא אתי למיעבד עקירה והנחה סוף סוף כי מטי לביתיה אי אפשר דלא קאי וקא מעייל מרה״ר לרה״י דזריק ליה כלאחר יד:
(דף קנד:) הגיע לחצר החיצונה וכו׳: אמר רב הונא }}היתה בהמתו טעונה כלי זכוכית כגון קרני דאומני דאסור לטלטלן משום דלא חזיין מביא כרים וכסתות ומניח תחתיה ומתיר את החבלים והשקין נופלין מאליהן ודוקא בשליפי זוטרי דאי בעי שמיט להו לכרים וכסתות מתותייהו ולא קא מבטל כלי מהיכנו מהא שמעינן דלית הלכתא כר׳ יצחק דאמר בפרק כירה (דף מג.) אין כלי ניטל אלא לדבר הניטל בשבת דהא הכא אוקימנא בקרני דאומני דלא חזיין וקאמר דמביא כר או כסת ומניח תחתיהן תניא רבי שמעון בן יוחי אומר היתה בהמתו טעונה שליף של תבואה מכניס ראשו תחתיו ומסלקו לצד אחר והוא נופל מאליו אביי
\end{multicols}\newpage

\newsection{דף סז}
\begin{multicols}{2}
אשכחיה *}בגמ׳ לרבה}לרבא דהוה קא משפשף ליה לבריה ע״ג חמרא א״ל והא קא משתמש מר בבעלי חיים אמר ליה צדדין נינהו וצדדין לא גזרו בהו רבנן ואסיקנא (דף קנה.) דהלכתא צדדין אסורין צדי צדדין מותרין:
\textbf{{\largeמתני׳}} }מתירין }}פקיעי עמיר בשבת לפני בהמה ומפספסין את הכיפין אבל לא את הזירין אין מרסקין את השחת ולא את החרובין לפני בהמה בין דקה בין גסה רבי יהודה מתיר בחרובין לדקה:
\textbf{{\largeגמ׳}} אמר רב יהודה הן הן פקיעין הן הן זירין קסבר פקיעין תרי זירין תלתא כיפין דארזי וה״ק מתירין פקיעי עמיר לפני בהמה אבל פספוסי לא וכיפין פספוסי נמי מפספסינן אבל לא את הזירין לא לפספס }ולא להתיר מ״ט דרב יהודה אמר *}בגמ׳ איתא רבא}רבה קסבר שוויי אוכלא משוינן מיטרח באוכלא לא טרחינן:
\textbf{{\largeמתני׳}} (דף קנה:) אין אובסין את הגמל ולא דורסין אבל מלעיטין ואין ממרים את העגלים אבל מלעיטין ומהלקטין את התרנגולים ונותנין מים על גבי מורסן אבל לא גובלין ואין נותנין מים לפני דבורים ולפני יוני שובך אבל נותנין לפני אווזין ותרנגולין ולפני יוני הדרסיאות:
\textbf{{\largeגמ׳}} מאי אין אובסין אמר רב יהודה אין עושין לה אבוס בתוך מעיה אין ממרין את העגלים אבל מלעיטין אי זו היא המראה ואי זו היא הלעטה המראה מרביצה ופוקס את פיה ומאכילה כרשינין ומים בבת אחת הלעטה מאכילה מעומד ומשקה מעומד ונותן לה מים בפני עצמן וכרשינין בפני עצמן
תנו רבנן מהלקטין לתרנגולין פירוש דספי ליה בידים ובולע }בפני עצמו אבל אינו נותן בפיו עד מקום שאינו יכול להחזיר ואין צריך לומר שמלקיטין פירוש שנותן לפניו והוא לוקט מאיליו ואין מלקיטין ליוני שובך וליוני עלייה ואין צריך לומר שאין מהלקטין מ״ט אווזין ותרנגולין מזונותן עליך יוני שובך ויוני עלייה }אין מזונותן עליך כיוצא בו נותנין מזונות לפני הכלב ואין נותנין מזונות לפני החזיר ומה הפרש בין זה לזה זה מזונותיו עליך וזה אין מזונותיו עליך
ת״ר }}}אין גובלין את הקלי ויש אומרים גובלין מאן יש אומרים אמר רב חסדא (דף קנו.) ר׳ יוסי בר׳ יהודה היא והני
מילי דמשני והיכי משני }יד על יד פירוש מעט מעט ומדקמתרצי׳ אליבא דרבי יוסי ברבי יהודה ש״מ דהלכתא כוותיה:
ושוין שבוחשין את השתית בשבת ושותין זיתום המצרי ודוקא ברכה אבל בעבה לא ורכה נמי לא אמרן אלא דמשני היכי משני א״ר יוסף בחול נותן את החומץ ואח״כ נותן את השתית ובשבת נותן את השתית ואח״כ נותן את החומץ לוי בריה דרב הונא בר חייא אשכחי׳ לגבלא דבי }נשא דקא גביל וספי להו לתורי פירוש }מגבל המורסן בטש ביה אתא אבוה אשכחיה וא״ל הכי אמר אבוה דאמך משמיה דרב ומנו רבי ירמיה בר אבא גובלין ולא מספין ואי לא לקיט בלישניה מהלקטין יתיה וה״מ דמשני היכי משני אמר רב יימר בר שלמיא משמיה דאביי שתי וערב והא לא מערב שפיר א״ר יהודה מנערו לכלי אחר כלומר מוליך התרווד שתי וערב אבל אינו ממרס בידו ולא מסבב התרווד ואי לא מערב שפיר מנערו בכלי אחר:
כתיב אפנקסיה דזעירי אמרי׳ קמיה דרבי ומנו רבי חנינא מהו לגבל ואמר לי אסור מהו }לפרק ואמר לי מותר אמר רב }מנשיא חד קמי חד תרי קמי תרי ש״ד תלתא קמי תרי אסור רב יוסף אמר אפי׳ קביים עולא אמר אפילו כור ואפילו כוריים והלכתא כעולא:
\textbf{{\largeמתני׳}} (דף קנו:) }}מחתכין את הדלועין לפני הבהמה ואת הנבלה לפני הכלבים רבי יהודה אומר אם לא נתנבלה מע״ש אסורה לפי שאינה מן המוכן:
\textbf{{\largeגמ׳}} }אמר עולא הלכה כרבי יהודה ואף רב סבר הלכה כר״י מדכרכי דזוזי }דרב אסר ושמואל שרי ושמואל אמר הלכה כר״ש ואף זעירי סבר הלכה כר״ש ואף רבי יוחנן סבר הלכה כר״ש והלכתא }בר״ש דקא אמרי׳ (דף קנז.) איפלגו בה רב אחא ורבינא חד אמר בכל השבת כולה הלכה כר״ש לבר ממוקצה מחמת מיאוס ומאי ניהו נר ישן וחד אמר במוקצה מחמת מיאוס נמי הלכה כר״ש בר ממוקצה מחמת איסור מאי ניהו נר שהדליקו בה באותה שבת אבל מוקצה מחמת חסרון כיס אפי׳ רבי שמעון מודה דתנן כל הכלים ניטלין בשבת חוץ מן המסר הגדול ויתד המחרישה וקיי״ל דכל היכא }דפליג רב אחא ורבינא הלכה כדברי המיקל:
\textbf{{\largeמתני׳}} מפירין }}}נדרים בשבת ונשאלין *}בגמ׳ ובס״י לנדרים}את הנדרים שהן לצורך השבת ופוקקין את המאור ומודדין את המטלית ואת המקוה מעשה בימי אביו של ר׳ צדוק ובימי אבא שאול בן בטנית שפקקו את המאור
\end{multicols}\newpage

\newsection{דף סח}
\begin{multicols}{2}
בטפיח וקשרו את המקידה בגמי לידע אם יש בגיגית פותח טפח אם לאו ומדבריהם למדנו שפוקקין ומודדין וקושרין בשבת:
\textbf{{\largeגמ׳}} ומיבעי לן הא דקתני מפירין נדרים בשבת לצורך השבת הוא אבל שלא לצורך השבת לא אלמא הפרת נדרים מעת לעת או דלמא כי קתני לצורך אשאלה בלבד הוא דקתני אבל הפרת נדרים בין לצורך בין שלא לצורך אלמא הפרת נדרים כל היום ותו לא }ואסיק תנאי היא דתניא הפרת נדרים כל היום ר׳ יוסי בר׳ יהודה ור׳ אלעזר בר׳ שמעון אומרים מעת לעת והלכתא הפרת נדרים כל היום בלבד וליתא לדרבי יוסי ורבי אלעזר דגרסינן בפרק נערה מאורסה אביה ובעלה מפירין נדריה (נדרים עו:) אמר רבי שמעון בן פזי אמר רבי יהושע בן לוי אין הלכה כאותו זוג ושמעינן מינה דמפירין נדרים בשבת בין לצורך בין שלא לצורך
והיכא דנדרה שלא בפני בעלה והפר לה בעלה והיא לא ידעה שהפר לה הפרתו הפרה דתניא (במדבר ל) (נזיר כג.) אישה הפרם וה׳ יסלח לה במה הכתוב מדבר באשה שנדרה בנזיר ושמע בעלה והפר לה והיא לא ידעה שהפר לה והיא היתה שותה ביין ומטמאה למתים וכשמגיע ר׳ עקיבא אצל פסוק זה היה בוכה ומה מי שנתכוין לאכול בשר חזיר ועלה בידו בשר טלה אמרה תורה צריך כפרה וסליחה מי שנתכוין לאכול בשר חזיר ועלה בידו בשר חזיר על אחת כמה וכמה ואיתא להאי גירסא בסוף גמרא דקדושין:
ונשאלין לנדרים לצורך: איבעיא להו בשלא היה לו פנאי או בשהיה לו פנאי ת״ש דאזדקיקו רבנן לרב זוטרא בריה דרבי זירא ושרו ליה נדריה אע״ג דהוה ליה פנאי גרסינן בפרק נערה המאורסה (נדרים עז.) סבר רב יוסף למימר נשאלין לנדרים בשבת ביחיד מומחה אין בשלשה לא דמיחזי כבי דינא אמר אביי כיון דסבירא לן אפילו מעומד ואפילו בקרובים ואפילו בלילה לא מיחזי כבי דינא אמר רבא אמר רב נחמן הלכתא נשאלין לנדרים מעומד וביחידי ובלילה ובקרובים ובשבת }אפילו אפשר לו מבעוד יום:
}}}ומודדין את המקוה (מכילתין דף קנז:) עולא איקלע לבי ריש גלותא חזייה לרבה בר בר חנה דיתיב באגנא דמיא ומשח ליה א״ל אימר דאמור רבנן מדידה דמצוה מדידה דלאו מצוה מי אמור אמר ליה מתעסק בעלמא אנא:
\textbf{סליקו להו מי שהחשיך וסליקא לה מסכת שבת} 
\end{multicols}
\newpage
\addpart{חידושי רמב"ן על שבת}\renewcommand{\partname}[1]{חידושי רמב"ן על שבת}
\fancyhead[CO]{\chapname}
\fancyhead[CE]{\partname}
\renewcommand{\sethebfont}{\fontsize{12pt}{18.0pt} \selectfont}\sethebfont
\newchap{פרק \hebrewnumeral{1} יציאות השבת}
\newsection{דף ב}
\textblock{}
\textblock{מתני: \textbf{יציאת השבת שתים שהן ד׳ בפנים.} פירש״י ז״ל, בפנים לאותו העומד בפנים ושתים הוצאה והכנסה דבעל הבית לחיוב שהן ארבע הוצאתו והכנסתו לפטור וכן בחוץ שתים שהן ארבע הוצאה והכנסה דעני לחיוב ולפטור דפתח לסדורי דעני לחיוב בסדורי דרישא. וי״מ בפנים היינו לחפץ שהוא נכנס בפנים והיינו הכנסה ולפי שאין חיוב המלאכה אלא בהכנסה והיא היא דאתיא לידי חטאת קרי להכנסה מלאכה דבפנים ושתים הכנסה דבעל הבית והכנסה דעני לחיוב שהן ארבע בשתי הכנסות של פטור ובחוץ שתי הוצאות לחיוב שהן ארבע שתי הוצאות של פטור ופי׳ הכנסה בתחלה מהיכא דפתח בי׳ וליכא קפידא במלתא כדמפורש בריש מס׳ נדרים דזמנין תנא מפרש מהיכא דסליק מיני׳ וזמנין מהיכא דפתח ביה ולא קפיד. והא דתני הכנסה והוצאה דעני בחדא ולא נקיט אורחא דרישא כיון דפתח במילי דעני מסיים להו. ואחרים פירשו, שתים בפנים היינו הוצאה דבע״ה לחיוב והוצאה לפטור שהן ארבע הכנסה לחיוב והכנסה לפטור. ושתים שהן ד׳ בחוץ נמי היינו הכנסה דעני לחיוב והכנסה לפטור שהן ד׳ הוצאתו לחיוב והוצאתו לפטור ואין הפי׳ הזה נכון לפי שאין הפטור כדאי להוליד תולדות ואין ראוי לקרותו אב ודאמרי׳ בגמ׳ וכ״ת מהן לחיוב ומהן לפטור לאו אאבות קיימי אלא ה״ק וכ״ת התם לא קתני אלא אבות, והיינו הוצאות שתים לחיוב ותולדותיהן הוצאות שתים לפטור. ואיכא דקשי׳ לי׳ היכי מנו רבנן הוצאות בשתים הא שם הוצאה חד הוא כדמקשינן בגמ׳ בפ׳ ידיעות הטומאה והא שם טומאה אחת היא ואע״ג דאסיקנא התם דתרתי נינהו משום טומאה דקדש וטומאה דמקדש אבל שם הוצאה ודאי אחת הוא. ובתוס׳ רבותינו הצרפתים ז״ל אמרו לכך מנו אותם חכמים בשתים משום דפלגינהו רחמנא בשתים דמתרי קראי נפקא לן חד מפ׳ הקודם מדכתיב ויכלא העם מהביא נמנעו להוציא מביתם למחנה לויה שהוא רה״ר והיינו הוצאה דבעה״ב והוצאה דעני נפקא לן במס׳ ערובין בפ״ק מדכתיב אל יצא איש ממקומו והוינן בה וכי לוקין על לאו שניתן לאזהרת מיתת ב״ד לומר שזה ניתן לאזהרת מיתת ב״ד והוצאה בכלל הוא וקרי בה אל יוציא ואע״ג דהתם מסקינן אמר רב אשי מי כתיב אל יוציא אל יצא קרינן התם ה״ק ועיקר קרא לתחומין הוא אבל הוצאה נפקא מכללי׳ וכ״ש לרבנן דאמרי תחומין דרבנן ואל יוציא בלחוד הוא. וא״ת קראי גופייהו ל״ל צריכי סד״א הוצאה חידוש הוא שברה״י מותר לישא משא גדול ואם הוציא לרה״ר כגרוגרות חייב משא״כ בשאר אבות מלאכות שאינן חלוקות ברשויות אלא אסורן מחמת עצמן והואיל וחידוש הוא ואין לך בה אלא חדושה בלבד והלכות עקרות הן ואין למדות זו מזו לפיכך פירש הכתוב לשתיהן ומיהו הואיל ומלאכה אחת היא לא נמנין בכלל אבות מלאכות כשתים והביאו ראי׳ לדבר מדלא גמרינן לה ממשכן והא ודאי הוצאה היתה במשכן ולא למדוה משם כמו שלמדנו לל״ח מלאכות הנשארות ואמרי׳ נמי בפ׳ הזורק ממאי דבשבת קאי דלמא בחול ומשום דשלימה לי׳ מלאכה ומסקנא גמר העברה העברה מיה״כ אלמא ליכא למיגמר ממשכן אלא צריכה היתה התורה לפורטה לעצמה. ואי קשי׳ לך, א״כ לימא ללאו יצאתה ואזהרה שמענו עונש מנין יש להשיב לך כיון שהיתה במשכן ואשכחן דקפיד עלה רחמנא בלאו ואזהר עלה רחמנא הדרא לכלל שאר כל המלאכות שהיו במשכן. ואין דברים הללו מחוורין כל צרכן. וי״א דלא צריכי תרי קראי להוצאה דעני ודבעה״ב שאלו כן היו מנינן להו באבות מלאכות בתרתי ונמצאו ארבעים אלא משום דאי כ׳ רחמנא ה״א ללאו יצאתה ואין בה מיתת ב״ד לכך שנה עליו הכתוב להביאה לכלל שאר מלאכות ודקא קשיא לך למה מנו אותן חכמים שתים לאו קושי׳ הוא דלא דמי למאי דאקשינן בפ׳ ידיעת הטומאה שם טומאה אחת הוא והתם בין נגע בקודש בין שנכנס במקדש בהעלמת טומאה ובידיעתה אי אתה צריך להודיעו אלא שהוא טמא הלכך לעולם העלמה אחת היא לגמרי אבל הכא הואיל ושתי הוצאות הם בשני ענינים תרתי חשיבי וכ״ש שיש להתחייב על הוצאות בעה״ב והכנסת עני שדרך האדם למשוך לעצמו והיא המלאכה התשובה אצלם וזה הטעם נכון בתי׳ הקושיא. אבל נראה דל״צ תרי קראי להוצאה ופלוגתא דתנאי היא ומאן דסבר תחומין דאורייתא מפיק לה מויכלא ואל יצא איש ממקומו לתחומין בלבד הוא דאתא אל יצא כתיב ואל יצא קרינן ומאן דסבר תחומין לאו דאורייתא מפיק לה מאל יצא קרינן בי׳ אל יוציא ויכלא לא משמע לי׳ דלא גמר העברה העברה מיה״כ אלא בחול קאי ומשום דשלימה מלאכה א״נ לכ״ע מויכלא ואל יצא אל יוציא ליכא דדריש הכי אלא ר׳ יונתן בלחוד וקשי׳ לי׳ התם ודחיי׳ רב אשי לגמרי לכ״ע ולמ״ד תחומין לאו דאורייתא לאזהרת (יוציא) המן אתאי והא ע״כ למ״ד תחומין דאורייתא ולא דריש אל יצא אל יוציא צריך אתה לומר ששתי הוצאות מויכלא ופי׳ שהזכרנו בזה למעלה פי׳ משובש ורחוק הוא דכיון דמקרא ומסורת אל יצא לתחומין אתא אל יוציא מנ״ל ועוד דא״כ לא נפיק מלאו שניתן לאזהרת ב״ד לעולם. ירושלמי (א,א) א״ר יוסי עני ועשיר א׳ הם ומנו אותם חכמים שתים וכו׳ פי׳ מפני שמשונה הוצאת זה מזה מפני שהעני מביא לרה״ר שהוא עומד ודרך הוצאה בכך ובע״ה מוציא מרה״י שהוא בו לרשות שאינו עומד שם. ראיתי לבעלי הקונדריסין שפי׳ למה פתח תנא דמכילתין בהוצאה משום דבעי למינקט סדרא מע״ש כדקתני לא יצא החייט במחטו סמוך לחשיכה והדר מייתי סדורא דיומא במה מדליקין שהוא סמוך משתחשך והדר במה טומנין והדר תנא כללא דיומא גופי׳ לפיכך שנה תחלה איסור הוצאות כדי שיהא ראוי לגזור עליהן סמוך לחשיכה שמא ישכח ויוציא אע״ג דקתני אין שורין את הדיו ואין נותנין דיו וכו׳ ומתני׳ לא מקדים אבות דידהו מ״מ התחיל סדרו בהוצאה לומר הוצאה מלאכה הוא וגזרו עלי׳ סמוך לחשיכה וכן גזרו בדיו ואונין וכולה מתני׳ ולהכי תנא הכא לא ישב אדם לפני הספר סמוך למנחה מפני שסמכו ענין לו לאל יצא החייט במחטו סמוך לחשיכה:
}
\textblock{גמ׳: \textbf{הא דתנן שבועות שתים וידיעת הטומאה ב׳ וכו׳.} אפרשי׳ במקומה (שבועות ב:) בס״ד:
}
\textblock{הא דמקשינן \textbf{מ״ש הכא דתני ב׳ שהן ד׳ בפנים וב׳ שהן ד׳ בחוץ ומ״ש התם דקתני וכו׳.} מקשי׳ בתוס׳, הא בשבועות לא מצי למיתני כי הכא ודומי׳ דמראות נגעים בעי למיתני והם החמירו בקושי׳ זו ולאו מילתא היא דאי יציאת שבת ד׳ שהן שמנה כדקתני להו הכא לאו דומי׳ דשבועות ומראות נגעים הן ולא הוה למיתנינהו התם בשתים שהן ד׳ ותו לא:
}
\textblock{והא דאמרי׳ \textbf{התם דלאו עיקר שבת אבות תני תולדות לא תני.} פי׳ ס״ל דאיכא אבות ותולדות הוצאות אב הכנסה תולדות והכא דעיקר׳ שבת קתני הוצאה והכנסה התם לא קתני אלא הוצאה ולהאי טעמא הא דקתני התם ב׳ שהן ד׳ באבות ה״ק שתי שמות שהן ד׳ מלאכות וכן שבועות שתי שמות הרעה והטבה שהן ד׳ להבא ולמפרע וכולן אבות הן:
}
\textblock{\textbf{אלא אר״פ הכא דעיקר שבת תני חיוב ופטור.} נ״ל דרב פפא הדר בי׳ ממאי דאמר דאבות היינו יציאות וסבר׳ דהכנסות נמי אבות הן וכדתנן התם ב׳ שהן ד׳ ב׳ היינו הוצאה והכנסה דעני שהן ד׳ הוצאה והכנסה דבעה״ב ולא מפני שהן תולדות אלא שתי שמות הוצא׳ והכנס׳ שהן ד׳ מלאכות בעני ובעה״ב כדפרש״י וכולן אבות הן ויש עיקר גדול בתורה לארבע כמו לשתים והא לא מפרשי בתור׳ הוצאות והכנסות של עני יותר מבעה״ב ולא מבעה״ב יותר משל עני:
}
\textblock{ואקשינן עלה \textbf{חיובי׳ מאי נינהו יציאות.} כלומר חיובי דקתני במתני׳ דהתם אינן אלא שתים יציאה דעני ויציאה דבעה״ב. והא יציאות קתני. ואי קשי׳ מתני׳ דהכא תיקשי לי׳ נמי שהן שמנה בהוצאות היכי משכחת לה י״ל אה״נ אלא אשיטה דהתם קיימי. עי״ל בשלמא מתני׳ לא תיקשי דלמא הכי קתני הוצאות שתים דעני ודבעה״ב. לחיוב שהן ד׳ לפטור לכלי שהי׳ עומד בפנים ושתים מלאכות שהן ד׳ לכלי שהי׳ עומד בחוץ והיינו הכנסות וסיפא לאו איציאות דרישא קאי אלא בפנים הוצאה בחוץ היינו הכנסה מ״ה אלימא לי׳ למפרך אמתני, דשבועות דלא קתני תרתי היינו פנים וחוץ:
}
\textblock{ופרקי׳ \textbf{תנא להכנסה נמי הוצאה קרי לי׳ תדע מדתנן המוציא מרשות לרשות חייב מי לא עסקינן דקא מעייל עיולי.} פי׳ ״מי לא עסקינן״ לאו דוקא אלא מלישנא דמתני׳ מדלא קתני המוציא סתם וקתני מרשות לרשות ודאי משמע דהכנסה נמי אתא למכלל בה דתרוייהו מלאכה אחת ושם אחד ואין בה חלוק להפרישן בשני אבות ולא באב ותולדה. וי״מ דרב אשי סברי קאמר, דע״כ במעייל עיולי [נמי] עסקינן (והא) [דהא] הכנסה נמי מכלל האבות הוא כמו ההוצאה וצריך הוא לשנותה בכללי האבות א״ו הוצאה קרי לה וכיון שהן מלאכה אחת ושם אחד לא מנה אותן בשתי אבות. ולאו מילתא היא, דהתם בפ״ק דשבועות אמרי׳ בהדיא ודילמא דקא מפיק מרשות היחיד לרה״ר א״כ ליתני מרה״י מאי מרשות לרשות וכו׳. וא״ת הכנסה דמקריא אב מנ״ל סברא הוא דכל עקירת חפץ ממקומו הוצאה הוא לר״א [וא״כ אמאי] לרבא הכנסה תולדה וי״ל הכנסה דמחייב עלה נפקא לן מההוא דאמרינן בפ׳ במה טומנין דאמרי׳ התם הם העלו את הקדשים מקרקע לעגלה אתם אל תכניסו מרה״ר לרה״י ואי קשיא הא דמפקי׳ להוצאה בפ׳ הזורק מויכלא העם מהביא תיפוק לי׳ ממשכן כדקתני התם הם הורידו אתם אל תוציאו לא תיקשי משום כתנאי הוא כדאמרן א״נ אלו לא כ׳ רחמנא ויכלא ה״א הוצאה והכנסה לאו מלאכות נינהו כדפרישי׳ לעיל אבל השתא דכ׳ רחמנא בהדיא הוצאה גלי עלה דהכנסה [והוצאה] כשאר מלאכות הוא וגמרי׳ לה ממשכן והיינו נמי דמחייב סקילה עלה כדפרי׳ לעיל. ואי קשי׳ לך הא דאמרי׳ בהזורק הך דהואי במשכן חשיבא קרי לי׳ אבות אלמא הכנסה לא הוי׳ במשכן ולא כתיבא ולא חשיבא איכא למימר ההוא טעמא לר״א קאמרינן ולשאר אבות מלאכות ותולדותיהן אבל הוצאה והכנסה כתיבא וחשיבא ותרוייהו חד אב נינהו לההוא טעמא ודאמרי׳ התם מיהו הוצאה אב הכנסה תולדה ההוא לישנא דלא כר״פ מיהו לאו למימרא דלא הוית במשכן דודאי מיהוי הוית התם כדאמרי׳ בפ׳ במה טומנין אלא כיון דמלאכ׳ אחת הן לגמרי לא חשיבי להו בתרתי כדאמרי׳ כפ׳ כלל גדול שובט הרי הוא בכלל מיסך ומדקדק הרי הוא בכלל אורג והוו תולדות ואע״ג דהוו במשכן, וכן נמי כותש הוי במשבן וחשבי׳ לי׳ תולדה. וא״ת מאי אולמה דהוצא׳ מהכנסה א״ל משום דמפרשי בלאו דויכלא ומיהו ר״פ סבר כיון דתרוייהו הוי במשכן ולא חשיבא חדא טפי מחברתה שתיהן שנאן בכלל לשון המוציא מרשות לרשות וקסבר תנא דעיקר חיובא משום עקירת חפץ ממקומו הלכך הוי חדא מלאכה לגמרי ואין תולד׳ זו כשאר התולדות שלא נשנו שהרי כתיבא וחשיבא וכל היכא דכתיבי וחשיבי אע״ג דאיכא דדמיא לה חשיב לה כדאמרי׳ בפ׳ כלל גדול (שבת עג:) גבי זורה ובורר הלכך ע״כ הויא בכלל המוציא מרשות לרשות ומשום דהוא חד שם וחד מלאכ׳ תנא הכי. ולפ״ז יצאתי מקושית כל התלמידים שמקשים היאך אמרי׳ הכא ובכלל המוציא מרשות לרשות שנה להכנסה והא הכנסה תולדה היא והתם אבות תני תולדות לא תני ולא ראיתי לא׳ מהן נבון בדבר. ורבא אמר רשויות קתני, סבר דהוצאות אבות והכנסות לא נשנו בכלל גדול כדין שאר התולדות שהיו במשכן שלא נשנו מפני שאינן חשובות כ״כ והך נמי לא חשיבא כהוצא׳ משום דהוצאה היתה עיקר הבאה למחנה לוי׳ ואין הכנסה צורך משכן כ״כ וכן אפשר לפרש ההוא דאמרי׳ לר״א הך דהוי במשכן חשיבא קרי לי׳ אב כלומ׳ דלא הויא במשכן א״נ הויא במשכן ולא חשיבי כגון הכנסה וכגון כותש קרי לי׳ תולד׳ וע״כ אתה אומר שהרי היו במשכן מלאכות ולא חשיבי לי׳ לתנא וקרי להו תולדות כגון כותש וכדאמרן:
}
\textblock{\textbf{מתני׳ נמי דייקי דקא מפרש הכנס׳ לאלתר.} פרש״י ז״ל דקתני כיצד פשט העני את ידו לפנים ונתן וכו׳ וכן עיקר ואי קשיא ע״כ מתני׳ הכנסות קתני וצ״ל דיוקא א״ל מתני׳ דהכא איכא לפרושי דסיפא לאו אריש׳ קיימי כדאמר וא״ל היאך דייקא לאפוקי מדרבא דאמר רשויות קתני דאי תני רשות ל״ל לאקדומי להכנסה אדרכה הוצאה דמיפרשה טפי הו״ל לאקדומי ועוד שהוא אב אליבא דרבא אלא ש״מ והכנס׳ והוצא׳ שם אחד להן שכולן נקראו הוצא׳ ולפיכך הקדימה ללמדך שהוא בכלל יציאות והיינו דקאמרי׳ לאלתר דאלמא אי מאחר לה אחורי לא שמעי׳ מינה כלום. וי״מ מדקתני בפנים קא דייק דהיינו הכנסה כדפרש״י במתני׳, ולא מחוור:
}
\textblock{\textbf{רבא אמר רשויות קתני.} פרש״י ז״ל דלרבא ה״ק, בין הכא בין התם רשויות שבת שתים רה״ר ורה״י ועל ידיהן ד׳ איסורין בפנים וכנגדן בחוץ וק״ל דלא תני התם דומיא דמראות נגעים אלא שתים שהן עם שנים אחרים ד׳ ולא דמי כלל ועוד ושתים מבחוץ ל״ל הו״ל למיתני ושהן ד׳ מבחוץ אלא משמע דה״פ לרבא רשויות שבת שתי מלאכות שהן ד׳ וה״ג בנוסחא דוקנא רבא אמר רשויות קתני רשויות שבת ב׳ שהן ד׳ ונוסחי דכ׳ בהו רשויות שבת ב׳ ותו לא קייטי נינהו ולא דייקי:
}
\newsection{דף ג}
\textblock{הא דאמר שמואל \textbf{כל פטורי שבת פטור אבל אסור בר מהני תלת.} לאו כל היכי דתני במס׳ שבת פטור קאמר והא בפ׳ ב״מ תנן בשביל החולה שיישן פטור ואוקמי׳ בחולה שיש בו סכנה ופטור ומותר הוא וכל הזריז ה״ז משובח וא״צ ליטול רשות מב״ד אלא כ״מ שאמרו מי שעושה מלאכה פלונית פטור עליה מפני שאינה מלאכה גמורה אותו הפטור היא פטור אבל אסור דומיא דמתני׳ דהכא. ואי קשיא הא דתניא בפ׳ במה אשה ר״א פוטר בכובלת ובצלוחית של פלייטון וההוא פטור ומותר הוא כדתני׳ אידך יוצאה אשה בכובלת לכתחל׳ ל״ק דההוא פטור לאו פטור ומותר קאמר אלא לר״מ קאמר לי׳ דקאמר חייב וא״ל איהו אינו חייב ולא בא עדיין לומר היתר בה לכתחלה עד שבא במקום א׳ וא״ל לרבנן מותר לכתחלה וכיון דמפרש בהדיא מותר לכתחלה לאו פטורי דשבת הוא אלא לר״מ נסיב ליה והדר מפרש ליה באידך ואפשר דר״א לא התיר לכתחל׳ אלא בכובלת דמדשבקי׳ בההוא ברייתא לבר זוגי׳ ולא תנא יוצאה אשה לכתחלה בכובלת ובצלוחית של פלייטון הלכך ההוא פטור לאו פטור ומותר הוא דאיכא צלוחית שפטור אבל אסור. ופר״ת, הא דקא מני שמואל צידת נחש ומפיס מורסא לאו אליבא דנפשי׳ קאמר דלדידי׳ חייב עלה כדאסיק בפ׳ כירה במלאכה שא״צ לנופה ס״ל לר״י וצידת נחש ומפיס מורסא אינו פטור אלא לר״ש כמ״ש בפ׳ ח׳ שרצים אלא אליבא דמאן דתני פטור קאמר, ומותר הוא: }
\textblock{\textbf{פטורי דאתא בהו לידי חיוב חטאת קא חשיב.} פרש״י ז״ל דהיינו עקירה, דמצי למייתי לידי חיוב חטאת אלו עביד הנחה. ולהאי פי׳ קא חשיב פשט העני את ידו [לפנים][ונטל בעה״ב מתוכה] ופשט בעה״ב את ידו ונתן העני לתוכה והיינו שתי עקירות דעני בתרתי וב׳ דבעה״ב בתרתי, וכן פר״ת ז״ל. ואיכא דקשי׳ להו כיון דתרי פטורי דעני עקירות והכנסו׳ נינהו ותרי פטורי דבעה״ב עקירות והוצאות אמאי חשיב להו בתרתי ואחרים פי׳ שעני שפשט ידו לפנים בין מלאה בין ריקנית היא תחלת מלאכה מפני שיכול ליטול ולהוציא ולבא לידי חיוב חטאת וכן של בעה״ב ותרתי נינהו חדא דהוצאה וחדא דהכנסה ואין זה נכון שהוא מונה מי שלא עשה תחלת מלאכה מפני שיכול הוא להתחיל בה ועוד שהוא פטור ומותר בהכנסת היד אם לא עשה דבר אחר. ואחרים פי׳ שהמניח הוא העושה מלאכה ומני פטורי דאתי בהו איניש לידי חיוב חטאת דהיינו גמר המלאכה שהוא המחייב אותו חטאת ונטל בעה״ב מתוכה חדא ונתן בתוכה והכניס חדא והיינו תרתי הכנסות דבעה״ב ושתי הוצאות דעני וזה הפי׳ נכון משום דמתני׳ הני פטורי דהכנסות אתא לאשמעינן דפטרי לי׳ בסמוך מבעשותה ודקא קשי׳ לך היכי הוו תרתי כיון שהמלאכות ע״י שניהם משתנות זו מזו אע״פ שהוא לא מנה אלא פטור הבא לידי חיוב חטאת תרתי חשיב (ליה) [להו]: }
\textblock{ה״ג: \textbf{והא אתעבידא מלאכה,} ול״ג כדכתב במקצת נוסחי מבינייהו, וה״פ והא אתעבידא מלאכה עי״ז שעושה הנחה ויתחייב אבל הראשון ודאי פטור שאם התחיל במלאכה ובא שני וגמרה אינו בדין שיתחייב ראשון: }
\textblock{\textbf{איתמר נמי א״ר חייא בר גמדא נזרק׳ מפי חבורה.} פי׳ מייתי ליה ראי׳ [בא ליישב קושית התוס׳] דהכי הוא עיקר והכי אסכימו אמוראי בבי מדרשא ולא תימא מתני ר׳ הוא ויחידאה הוא ורבנן פליגו עלי׳ ומפקי לי׳ מנפש אחת למעוטי זה עוקר וזה מניח ולא גרסי׳ נזרקה מפי חבורה ואמרו בעשותה: }
\textblock{\textbf{תפשוט דלא התירו ואב״ע לעולם לא תפשוט.} &lt;חסר&gt;... וכיון דיש כאן ב׳ תירוצים חדא קולא וחדא חומרא לחומרא אזלינן ולא נפשוט בעי׳ דרב ביבי וצריך לומר דמזיד ג״כ לא לפשוט דודאי יש חילוק ביניהם דכאן וודאי לא יעשה בידים אבל בהך דרב ביבי ממילא נאפ׳ ואסקינן לאותה תצר מותר לחצר אחרת אסור דמיתעביד מחשבתו אבל גבי רודה פת מן התנור רצונו לאפות הפת ורודה אותו קודם שנאפית ומיהו במזיד אפי׳ לאותו חצר אסור כדקאמרינן:
}
\textblock{\textbf{ואב״ע דכ״ע לא קנסו שוגג אטו מזיד.} כלומר אידי ואידי בשוגג בדקאמרת ולא משום קנס אטו מזיד אלא בשוגג גופי׳ אסור לחצר אחרת אלמא במזיד קנסו וכן נראה ממ״ש רבינו הגדול. והא בשוגג ומ״מ לפי לשון שפי׳ דוקא מבעוד יום אבל משחשיכה מותר לאותה חצר דהא איפשט התירא בין בשוגג בין במזיד לחצר אחרת אסור לעולם דהא אפשר לי׳ להחזיר לאותה חצר, ואין זה מתוקן בהלכות:
}
\newsection{דף ד}
\textblock{\textbf{אילימא בשוגג ולא אידכר למאן התירו.} איכא דק״ל, דילמא דחזי אינשי ואמרי לי׳ רדה ואיכ׳ דקשי׳ להו עוד דלמא איירי בהזיד בלאו ושגג בכרת דחשוב שגגה כמבואר לקמן פרק כלל גדול ונראה דלאו קושי׳ הוא כלל דמשמע לי׳ דמשום חיוב חטאת דוקא התירו לו ולא משום דאתי׳ לידי חלול שבת וכשמודיעי׳ לו מן האיסור דאז ליכא תו חיוב חטאת באמת אסור לו לרדות וא״כ כיון דאינו יודע שיש חיוב חטאת ואינו יודע רק מחיוב לאו איך ירדה אותו וגם כשמודיעין לו איך יאמרו לו רדה כיון דליכא תו חיוב חטאת הרי אסור לו לרדות וכשמתרץ הש״ס במזיד ותני שלא יבוא לידי איסור סקילה ה״ה באידכר או בשאר דברים:
}
\textblock{\textbf{וכי אומרים לאדם חטא בשביל שיזכה חבירך.} קשי׳ לי׳ לר״ת ז״ל, והא אמרינן בעירובין פ׳ בכל מערבין ניחא לי׳ לחבר דליעבד איהו איסורא קלילא ולא ליעבד ע״ה איסורא רבא ומפרקי שאני התם שע״י החבר הוא עושה שא״ל מלא לך כלכלה זו תאנים מתאנתי וקשי׳ להו הא אמרי׳ בפסחים שהכהן עובר על עשה דהשלמה ומקריב קרבן מחוסר כפורים בערבי פסחים כדי שלא יבא חברו לידי איסור כרת וא״ל שאני כהנים דשלוחי ישראל שוינהו רחמנא והוא אינו יכול לזכות מעצמו וא״נ שאני הדביק פת בתנור בשוגג דלאו איסור הוא שאפשר לו להתכפר בחטאת ואין רדייתה אלא זכות לו שיפטר מקרבן, כנ״ל:
}
\textblock{\textbf{והא בעי עקיר׳ והנחה מע״ג מקום ד׳.} פי׳ גמרא גמרי לה הכי ומקראי לית לן ואפי׳ ממתני׳ ובריית׳ לית לן בהדי׳ וכן הא דמקשי ודילמא הנחה הוא דלא בעיא הא עקיר׳ בעינן סברא הוא ועדיפ׳ ליה עקיר׳ ולא כמו שפי׳:
}
\textblock{\textbf{למימרא דפשיטא לי׳ לרבה דבקלוטה כמי שהונחה פליגי וכו׳.} איכא למידק ודלמא רבה אכתי בספוקי׳ קאי ומיהו בדר״ע לא מסתפקי לעולם דבודאי סבר קלוטה כמו שהונחה וע״כ ל״ק הא מני רבנן דבדרבנן מספקא ליה ואיכ׳ דמיפרקי׳ אי מספקא לי׳ לרבה הול״ל דודאי למעלה מעשרה פליגי אבל למטה מעשרה ד״ה קלוטה כמי שהונחה כי היכי דתיקום מתני׳ כד״ה מדקאמר ר״ע הוא אלמא ליכא לאוקמי לעולם כרבנן. ולדידי ל״ק לי כלל, דאנן מר״ע גמרי׳ ולא מדברי רבה ומתני׳ למעלה בלחוד קתני דר״ע מחייב וחכמים פוטרין אבל למטה מעשרה לא תנן ואפי׳ תימא בין למטה בין למעלה קתני בדר״ע משום דיליף זורק ממושיט הוא מתחייב בשתיהן אבל משום קלוטה כמי שהונחה לא שמעי׳ לי׳ לר״ע דמחייב בדלא שמעינן להו לרבנן ודבר פשוט הוא זה. ויש שמוסיפין לומר, דאי במיליף זורק ממושיט כי פליגי בדיוטא אחת פליגי אבל בשתי דיוטאות אפי׳ במושיט כדאי׳ בפ׳ הזורק הלכך בשתי דיוטאו׳ למטה מעשרה משום טעמא דקלוטה כמי שהונח׳ לא שמענו שחייב ר״ע כלל ואע״ג דאמר רבה אבל למט׳ מעשרה ד״ה חייב הא סברא דנפשי׳ הוא כלומר דאפשר לומר כן אבל מכיון דלא שמעי׳ לי׳ לר״ע דאמר הכי ליכא למימר הא מני ר״ע דהא אפשר דלא סבר הכי כי היכי דלא אמרינן הא מני רבנן היא אע״פ שאפשר. ולשון אחר מתרצין עוד בתוספות, דאי אמרי׳ למעלה מי׳ פליגי ור״ע יליף זורק ממושיט לית ליה קלוטה כמו שהונחה דאי ס״ל הכי קלוטה למעלה כמי שהונחה שם היא ופטור דמונחה במקום פטור הוא אלא הא דאמרי׳ אבל למטה מי׳ ד״ה חייב לר״ע משום דיליף זורק ממושיט הוא ולרבנן משום דנסב לה טעמא משום קלוטה כמי שהונחה הלכך אי מבעיא לי׳ לרבה לא מיתוקמא מתני׳ כר״ע כלל. ולא נראה, דלמעלה מי׳ כיון דאיתפליג לי׳ רשותא מקרקע והאי רשות ליכא למימר כמי שהונח׳ לפוטרו דאלת״ה למטה מעשר׳ בשתי דיוטות הו״ל ר״ע לפטור ורבנן לחיוב ואנן איפכא קא משכחינן וד״ה חייב קאמרינן:
}
\textblock{וד\textbf{אמר רבה אבל למטה מי׳ ד״ה חייב משום דאמרינן קלוטה כמי שהונחה.} ק״ל, ל״ל למימר הכי א״ה קשי׳ לרבנן אדרבנן כדאמרי׳ בכולי׳ שמעתין דאינהו בעי הנחה ע״ג מקום חשוב וא״ל ניחא ליה לאהדורי למילתייהו דרבנן במתני׳ א״נ לאו סברא דדוקא קאמר [אלא ה״ק] אבל למטה מעשרה א״ל דד״ה חייב וא״ל נמי למטה מי׳ נמי פליגי (ובסיפוקא) [וכספיקא] שלישית היא:
}
\textblock{\textbf{ואר״י א״ש מחייב היה רבי שתים אחת משום הוצאה וא׳ משום הכנסה.} איכא דקשי׳ לי׳, והא רב יוסף דהוא מארי׳ דההוא פירוקא אמר בפ׳ הזורק דר׳ לא מחייב אלא אחת לעולם ול״ק דר״י לא אמר אלא ר׳ הוא סתם וכיון (דחזיא) [דדחינן] לי׳ מר׳ (דדחי) [דזיז בזה] מהדר גמ׳ לאוקמי כאידך דר׳ ואליבא דרב ושמואל, כי היכי דלא תיקשי לי׳ ודלמא הנחה הוא דלא בעי׳ הא עקיר׳ בעי׳:
}
\newsection{דף ה}
\textblock{\textbf{הכא למטה מי׳.} ק״ל, למט׳ מי׳ נמי תעש׳ כרמלי׳, ואע״פ שהאדם תופס׳ בידו כשם שלמעל׳ נעשי׳ ר״ה ואע״פ שהוא תופס אותה כדאמרי׳ אלא טרסקל בר״ה רה״י הוא ולאו מילתא הוא דכיון שהוא כלי לא עשו אותה כרמלית כדאמרי׳ לקמן בקופה עשרה רה״י פחות מכן מטלטלין ממנה לר״ה ומר״ה לתוכה, וכמו שפי׳ [רש״י], ולא גזור רבנן לבטלה מתורת כלי הואיל וכלי הוא ואפשר דהאי למטה מי׳ לאו דוקרן אלא למטה מג׳ נמי הוא ואיידי דאמר הא למעלה מי׳ אמר נמי למטה מי׳ והל״ל למטה מג׳ ואע״ג דלא מתוקמא במשלשל ידו למטה מג׳ משום דקתני עומד הכא כיון דיש טרסקל בידו אפשר שעומד ויש לטרסקל בית יד גדול עד שהוא עצמו עומד למטה מג׳:
}
\textblock{\textbf{אכפל תנא לאשמעינן כל הני.} פירש״י ליתני פשט כנף בגדו לפנים וקשיא לן הא טובא קמ״ל דלמטה מג׳ חייב בידו ופלוגתא דאמוראי הוא לקמן בפ׳ המצניע ומלתא אגב אורחא קמ״ל דהא הנחה הוא ולאו מלתא הוא דהתם כשהוציא ידו לרשות אחרת דאפי׳ למטה מג׳ א״ל בתר גופא גרירה אבל למי ששלשל ידו למטה מג׳ ברשות גופו פשיטא דהנחה היא ול״צ למימר וסיפא דקתני פטורין ודאי אי בשלשל המכניס והמוציא ידו למטה מג׳ מצריך צריכי וקמ״ל דאגד ידו שמי׳ אגד אלא דלא ניחא לי׳ למימר דנקט רישא בהאי גונא לאשמעינן פטורא דסיפא ומיהו רישא נמי קתני פשט העני ידו לפנים ונטל ואי קסבר אגד ידו לא שמי׳ אגד אפי׳ נתן בעה״ב ידו לתוכה העני חייב אלא י״ל ס״ל הכא כמאן דמחייב התם ולא ניחא לן לאוקמי לסיפא בשלשל המכניס והמוציא אלא בשלשל המקבל הלכך קשי׳ למה איכפל תנא לאשמעינן כל הני. וי״מ איכפל תנא לאשמעינן הני לומר דהו״ל לאשמעי׳ מאי דאשמעי׳ במילי דאורחא הוא אבל למיתני עשיר ועני סתם בשוחה וגומא וננס דלאו אורחא הוא:
}
\textblock{\textbf{ידו של אדם חשובה כד׳ על ד׳.} איכא דקשי׳ לי׳ בלא״ה נמי לחייב דמחשבתו משוי׳ לי׳ מקום כדאמרי׳ בשלהי פ׳ הזורק גבי זרק בפי כלב ובפי הכבשן חייב ובמס׳ עירובין פ׳ המוצא תפלין השתין ורק חייב חטאת ומפ׳ לה טעמא דמתשבתו משוי לי׳ מקום ואיכא דמפרקי שאני התם דנעשית מחשבתו לגמרי אבל הכא לא נעשית לגמרי שאין עיקר מחשבת נתינת אוכלין אלא משום אכילתן או תשמישן וכן נטילתן, כך מפורש בתוס׳. ואחרי׳ אומרים דבמתני׳ ליכא למימר הכי דבשלמא נתן בתוכן איכא מחשבה אלא נטל ממנה מאי מחשבה איכא וכי הנוטל מחשב מאיזה מקום הוא נוטל. ואי קשיא, אי מתני׳ ליכא מחשבה היכי אמרי׳ בסמוך מהו דתימא ה״מ היכי דאחשבי׳ לידי׳ א״ל ה״ק מ״ד דר״י לאו אמתני׳ איתמר אלא בעלמא איתמר ובהיכי דאחשבי׳ קמ״ל. ואני אומר שאין עיקר קושייתם קושי׳ כלל שאין אומר במלאכת המלאכה עצמה שתהא מחשבה עושה אותה מלאכה שהרי הוא מחשב להניח ואינו מניח. אבל עיקר הדבר כך הוא: כשאדם מחשב למלאכה ועושה אותה כגון זרק בפי הכלב שהוא מתכוין להאכילו והוא מאכילו אותה האכילה משווי׳ פיו של כלב מקום וכן שריפת פי הכבשן לפי שהנחה זו חשוב׳ לו לפי כוונתו יותר ממקום אחר ודרך הנחה זו בכך וכן השתנת השתן ויריקת הרוק שהוא מכוין לנקות עצמו מהן ועושה כך ומגו דהוה מלאכה לכך הוה מלאכה לענין שבת שזו מלאכת מחשבת אבל אם דעתו להניח על מקום שאין בו ד׳ במה יעשה אותה מלאכה גמורה הרי אינו מתכוין אלא להניח שם ואין אותה ההנחה כלום נמצא שלא מתכוין להניחה כשם שאין הנחתו מלאכה כך מחשבתו אינו למלאכה. תדע דהא אמרי׳ התם בשלהי הזורק גבי הא דתנן בכריתות (יג:) אם היתה שבת והוציאו חייב ואמאי הא אין דרך הוצאה בכך כלומר בפיו אלא מחשבתו משוי׳ ליה מקום ואי ס״ד כל מילי דחשיב עליה מחשבתו משוי׳ לי׳ למקום היכי תנן בפיו ובמרפיקו פטור הא מכוין להוציא בפיו וכן נמי נגר היוצא בקיסם שבאזניו ושולחני בדינר שבצוארו לר״מ אמאי פטור לימא מחשבתו משוי לי׳ הוצאה אע״פ שאין דרך הוצאה בכך אלא ש״מ דשאני מתכוין לאכילה ואוכל דמגו דמהני מחשבתו לאכילה מהני נמי לשבת לשווי׳ מלאכה דדרך מלאכה זו בכך מה שאין לומר כן במחשב למלאכה עצמה ולא לדבר אחר שאם אינה מלאכה הרי לא חשב לכלום, כנ״ל:
}
\textblock{כתוב בכל הנוסחאות: \textbf{שתי כחות כאדם א׳ דמו ופטור} וכן הוא גרסת ר״ח ז״ל, ופי׳ כגון שנעקר ממקומו וקיבל וכשני ב״א דמי וחייב שהרי עקיר׳ והנח׳ הוא או דלמא כאדם א׳ הוא ופטור כאלו נותן מימינו לשמאלו דאע״פ שהעבירה ד׳ אמות פטור כך מפורש בתוס׳ ולא דאיק לי׳ דהני לאו שני כחות נינהו דעקיר׳ והנחה בכח א׳ היא. לשון אחר: ״כגון שקיבל בעקירה ממקומו״, וכך פירושא: כשני בנ״א דמי בעומד ומקבל, דהא עקירה והנח׳ תרוייהו מכחו אתו א״ד כאדם א׳ דמי שעשה עקירה ולא הניחה במקום א׳ דפטור וה״נ ההנחה לא נעשית מכחו הראשון. ויש לרש״י ז״ל פי׳ אחר טוב מזה, אבל אין הגירסא מודה לו:
}
\textblock{\textbf{בכותל משופע.} מפורש בתוס׳ בשרבים מכתפין עליו דאי אין רבים מכתפין עליו הו״ל כרמלית. ואני אומר אע״פ שאין רבים מכתפין עליו דכל אויר ר״ה עד עשרה ר״ה ופני כותל ומשופע שוין הן, ותנן (ק.) למטה מעשרה בזורק בארץ ואוקימנ׳ בפני הכותל ממש ובדבלה שמנה כדלקמן (שבת ז:):
}
\textblock{\textbf{א״ד בתר כלי אזלינן והא לא נח.} ק״ל, והא אמרי׳ במס׳ ב״מ פ״ק (ט:) אלא מעתה היה מהלך בספינה וקפצו דגים ונפלו לתוך הספינה ה״נ דלא קני ומהדרי׳ מי דמי התם ספינה מינח נייחי דמיא הוא דקא ממטי לה ולאו קושי׳ הוא כלל דהתם כל חצר דאינה מהלכת קניא וספינה אינה מהלכת אלא מיא ממטי לה, אבל גבי שבת הנחה בעי והא ניידא ולא נייחא:
}
\textblock{הא דסבר בן עזאי \textbf{מהלך כעומד דמי,} לית לי׳ הא דאר״י המפנה חפצים מזויות לזויות ונמלך עליהן והוציאן דלדידי׳ כיון שנמלך ויצא כעומד לפוש ועוקר דמי וה״נ מפורש בר״פ א״נ (כתובות לא.). וקשי׳ להו לרבנן, איפכא מעביר ד׳ אמות בר״ה גמרא גמירי לה וא״כ לילף מיני׳ דמהלך לאו כעומד דמי דלמא שאני התם דמקום חיוב הוא לכ״ע אבל הכא מקום פטור הוא לכ״ע ואין דנין קל מחמור להחמיר עליו. תו קשיא להו, לר״ע דאמר קלוטה כמי שהונחה זורק, ד׳ אמות בר״ה היכי משכחת לה ומפרקי לה דלא אר״ע חלוט׳ כמי שהונח׳ דוקא בהכנס׳ לרשות אחרת דקלוט׳ מחיצות, אבל בחד רשות לא אמרי׳. ובירושל׳ (א,ג) מצאתי רב חסדא שאל לרב הונא לדעת ב״ע אין אדם מתחייב בתוך ד׳ אמות לעולם כיון שהוציאה נעשה כמי שהונחה על כל אמה ואמה ויהא פטור , ואין שם פירוק מפורש על זה:
}
\newsection{דף ו}
\textblock{\textbf{אלא מידי דהוה אמוציא כו׳.} אי קשיא, צדי ר״ה גופי׳ מנ״ל, עיין תוס׳ ואפשר שכן היה במשכן מיתרי האהלים זה נכנס וזה יוצא במחנה לוי׳ דהוי ר״ה וקים להו לרבנן דאורחי׳ בהכי:
}
\textblock{\textbf{אימור דשמעת לי׳ לר״א היכא דליכא חפופי אבל היכא דאיכא חפופי מי שמעת ליה.} וק״ל, דגרסי׳ במס׳ עירובין פ׳ כל הגגות (צד.), ונפלגו בצידי ר״ה דעלמא אי פליגי בצידי רה״ר דעלמא ה״א ע״כ ל״פ רבנן עלי׳ דר״א אלא היכא דאיכא חפופי אבל היכא דליכא תפופי לא קמ״ל אלמא בכולהו פליגו א״ל דה״ק ע״כ ל״פ רבנן אלא היכי וכו׳ קמ״ל דאדרבה היכי דליכא חפופי פליגי והיכא דאיכא חפופי מודה ר״א לרבנן דכרמלית הוא. א״נ סוגיא דהתם כר״ע דהכא דסבר אפי׳ איכא חפופי נמי פליגי, ור׳ אחא הוא דאמר דלא פליגי דלא שמעי׳ לי׳ לר״א הכי:
}
\textblock{\textbf{ארבע רשויות לשבת.} הקשו בתוס׳ ליתני חמשה, דהא קרפף יותר מב׳ סאתים שלא הוקף לדירה הזורק מתוכו לר״ה חייב ואין מטלטלין בו אלא בד׳ אמות והעלוה בגמגום ולדידי ל״ק דההוא רה״י דאורייתא והני בין מדאוריית׳ בין מדרבנן שמותם עליו אלא שדיניהן שלשה מן התורה ומדבריהם ארבעה:
}
\textblock{\textbf{למעוטי הא דר״י דתניא יתר ע״כ אר״י מי שיש לו ב׳ בתים וכו׳.} אי קשי׳ אמאי לא מייתי ההוא דתנן בפ׳ כל הגגין ועוד אר״י מערבין למבוי המפולש א״ל אי מהתם ה״א כי אמר ר׳ יהודה מערבין בצורת פתח או בדלתות ואע״ג דהתם בגמ׳ מוכח דאפי׳ בלחי או קורה קאמר, ממתני׳ בהדיא לית לן:
}
\textblock{\textbf{זהו ר״ה גמורה וכו׳ למעוטי אידך דר״י.} ק״ל מיהא מימעטו תרוייהו דקתני מבואות המפולשין ר״ה גמורה אלמא שתי מחיצות לאו דאורייתא וכיון דר״ה דאוריית׳ הוא לא מהני בה לחי או קורה א״ל כי קתני ר״ה הם ברחבים י״ו אמה קתני דפחות מכן לאו ר״ה הוא כדגמרי׳ בפ׳ הזורק וכי אר״י מי שיש לו ב׳ בתים וכו׳ דאלמא שתי מחיצות הוי רה״י דאורייתא דלמא ברחב עשר או מעשר ועד י״ג אמה ושליש בלחוד קאמר דומי׳ דפסי ביראות וכדסבר רב אחא קמי׳ דרב אשי למימר בפ״ק דעירובין ומ״ה דיינינן לי׳ מרישא וזהו רה״י שהוא גמורה במחיצתה אבל שתי מחיצות לעולם מדאוריית׳ לא הוה רה״י:
}
\textblock{\textbf{אלא אימא אינו חייב על אחת מהן.} כ׳ רש״י ז״ל, יש אחת בהן שאינו חייב עליו מיתה וכן הוא בודאי דחייבין על כולן במזיד ובשגגתן חטאת דאי ס״ד דפטור מ״מ קשיא מנינא דאיסי דקתני ארבעים חסר אחת לומר לך שאם עשאן כולן בהעלם אחת שהוא חייב ארבעים חטאת חסר אחת א״ו לא פטר איסי אלא מסקילה והיינו דאמרי׳ ענוש כרת ונסקל אצטריכ׳ ליה וענוש כרת לאו דוקא אלא נסקל אצטריכא לי׳ דכיון דבשוגג חייב חטאת פשיטא דענוש כרת ואיסי נמי מסקילה קאמר דאינו חייב על אחת מהן:
}
\newsection{דף ז}
\textblock{\textbf{אילימא דאיכא מחיצה עשרה הוא דהויא כרמלית ואי לא וכו׳, והא״ר גידל וכו׳.} איכא למידק ולקשי לי׳ מדתניא לעיל וכן גדר שהיא גבוה י ורחב ד׳ זו היא רה״י גמור׳ א״נ מדתנן חולית הבור והסלע בזמן שהן גבוהין י׳ הזורק לתוכן וכו׳ וא״ל דלהכי מקשי לי׳ מרב גידל משום דאית בה תרתי מדקאמר גבוה י׳ רה״י פחות מי׳ כרמלית ולא דאיק אלא הכא בבקעה שאינה מוקפת לדיר׳ והוא יותר מב׳ סאתים וקאמר והיא כרמלית, אבל פחות מי׳ בין הוקף לדירה בין לא הוקף לא הוי כרמלית, (ואי) [ומסיפא] קשיא דקאמר אין מטלטלין בו אלא בד׳ אמות כדפרש״י ז״ל. ומיהו הוי יכול לאקשויי משמעת׳ דעולי׳ בקרפף מדר׳ יוחנן לדר׳ יוחנן, דבמחיצ׳ עשרה רה״י הוא מדאוריית׳ אלא הא מפרשי לי׳ טפי דקאמר פחות מכן הוא כרמלית א״נ בשאין בה אלא שתי דפנות שאינ׳ רה״י וקאמר דלא הוי כרמלית אלא במחיצה עשרה כדין מחיצת רה״י ומסיפא קשי׳ כדאמרן. ובתוס׳ מתרצים דה״ק: אילימ׳ דאי איכא אויר מחיצה עשרה [נ״א – אויר הראוי למחיצה עשרה] הוא דהוי כרמלית, אבל ודאי מחיצה עשרה פשיטא ליה דרה״י הוא. והאי דנקט מחיצה ולא נקט אויר (מחיצה) משום דבעי לאיתויי מלתא דרב גידל והוא אויר מחיצה [שאין בו] עשרה כדאמר בית שאין תוכו עשרה וקרויו משלימו, ואין צורך: }
\textblock{ה״ג וכן גרסו הגאונים ז״ל: \textbf{ואם חקק בו ד׳ על ד׳ והשלימו לעשרה מותר לטלטל (בתוכו) בכולו.} ואיכא למידק אמאי לא בעינן שיהיו מחיצות תוך ג׳ לשפת החקק כדאמרי׳ לענין סוכה אם יש משפת חקקו ולכותל ג׳ טפחים פסולה א״ל לענין סוכה דפנות בעי דכתיב בסכת תשבו בסכת בסכות הלכך בעי׳ תוך ג׳ דליהוי לבוד אבל לענין שבת כל היכי דלא בקעי רבים כלל רה״י הוא והא איתנהו למחיצות מבחוץ דהוו עשרה הלכך תוך התקק נמי תוך מחיצות דעשרה הוי ורה״י הוי. ואיכא דקשי׳ ליה, הא אמרי׳ בפ׳ המביא גפי תניין (גיטין טו:) גדור חמשה ומחיצה עשרה אין מצטרפין וליכ׳ למימר דשאני הכא דרובא מחיצ׳ דהתם משמע דאפי׳ רובא מחיצה לעולם אין מצטרפין כדקאמרינן עד שיהא או כולו בגדור או כולו במחיצה ואיכא דמפרקי שאני חקק בית שהיא ראוי לדירה שהרי יש לו מחיצות גמורו׳ ונעשה בית גמור ועוד שאותן מחיצות ראויות לקירוי לפיכך מצטרף הגדור עמהם. ומיהו קשיא, הא אמרינן בפ׳ הזורק בור וחולייתה מצטרפין לעשרה וא״ל שאני חוליות הבור שכן דרך כל הבורות לעשות להם חולי׳ והו״ל כגדור עשרה משא״כ בחריץ בעלמא. ורב נתן בעל הערוך פי׳ גדור חמשה כגון גבשושית שבקרקע גדור מלשון גדודו דפ״ק דעירובין ומלשון גודא וגוד אסיק מחיצתה, ולפ״ז י״ל דחריץ מצטרף עם מחיצה שעל שפתו שהכל כמחיצה למי שעומד בתוכו. כך תרצו מקצת חכמים הצרפתים ז״ל, שלא כדברי רש״י ז״ל שפי׳ שם חריץ עמוק חמשה וכותל תמשה על שפתו. ואחרים פי׳, ההוא דהתם לענין ערובי חצרות כההוא דמייתי לה בפ׳ כל הגגות בעירובין שתי חצרות זו למעלה מזו ועליונה גבוה י׳ טפחים אם יש גדור חמשה ומחיצה תמשה מערבין שנים ואין מערבין א׳ וכו׳ שמעתא התם וי״ל דהא דאמר אביי הכא מצטרפין מימרא הוא וההוא דפ׳ הזורק נמי מימרא הוא דר׳ יוחנן ומימרא לר״ח לא ס״ל ואע״ג דהתם בפ׳ הזורק מסייע לה להא דר׳ יוחנן ממתני׳ דקתני חולית הבור והסלע ולא תני הבור והסלע ר״ח לית ליה ההוא דיוקא ומיהו לית הלכתא כדר׳ חסדא כדפסקינן התם בגמ׳ בהדיא ובפ׳ הזורק תניא כוותי׳ דר׳ יוחנן, ומיני׳ הוא דדחי לר״ח מהלכתא: }
\textblock{\textbf{נעץ קנה ברה״י וזרק ונח ע״ג אפי׳ גבוה ק׳ אמות חייב.} נראה שרש״י ז״ל מפרש דלא בעי ר״ח הנחה ע״ג מקום ד׳ ברה״י ואע״פ שמשמען של דברים כן תימא הוא א״כ היכי פשיטי להו לרבנן דגמ׳ דכעיגן עקירה והנחה ע״ג מקום ד׳ דאקשו לה בהדיא לעיל אמתני׳ ומשמע דאביי כר׳ חסדא ס״ל מדקא מהדר לאוקמי מימרי׳ אליבא דד״ה ותו דכתב לה רבינו הגדול ז״ל בפסק הלכה ואע״ג דק״ל הנחה ע״ג מקום ד׳ בעינן. ומצאתי לר״ח ז״ל שכ׳ וק״ל דלמעלה מי׳ א״צ ד׳ על ד׳ אלא קנה אפי׳ עולה למעלה מי״ט וזרק ונח ע״ג חייב. ומדברי כולם נלמוד לפרש דהיינו נמי דאתא ר״ח לאשמעינן דרה״י עולה עד לרקיע והויא כמונחת בקרק׳ ואע״ג דלא ס״ל קלוטה כמי שהונחה דהנחה כל דהו [ברה״י] מיהו הוי׳ הנחה (נמי) [כמו] שהוא מקום ד׳ הלכך לא בעינן הנחה ע״ג מקום ד׳ אלא כגון בר״ה א״נ לאדם שהוא עומד ברה״י שאינו דומה לקנה נעוץ דהוא (בר״ה) [כרה״י] עצמה. וא״ת הא לעיל (שבת ד:) אמרי׳ דר׳ לא בעי הנחה ע״ג מקום ד׳ מקמי דתייתי פירוקא דאביי א״ל התם לא ס״ד נמי דכרב חסדא והוי מצי למדחי התם טעמא דר׳ כר״ח אלא הואיל ואתי לדחויי ניחא לי׳ לדחויי כדאביי והוא מסקנא דשמעתא: }
\newsection{דף ח}
\textblock{\textbf{רחבה ששה פטור.} כ׳ רש״י ז״ל, דל״ד ולחומרא לא דק דהא סגי בחמשה וג׳ חומשין. ולא מחוור, חדא - דכל היכי דמצריך להכי מקשי ליה בגמ׳ בהדיא, ועוד - שאין זו חומרא דאנן לא לכתחלה קאמרינן דאפי׳ רחבה כמה פטור אבל אסור הוא אבל לפטור ממית׳ וקרבן קאמרינן ואם אינה רחבה ששה חייב מיתה וקרבן קאמר. ורבי׳ אלפסי ז״ל נתן אותן שני חומשין למחיצות נמצא עובי המחיצה אצבע צרדה שהוא חומשו של טפח כדמפורש בפ׳ התכלת (מנחות דף מ״א ב׳, וע״ש ברש״י ד״ה בתילתא) וגבוה עשרה דקאמרינן אפי׳ בהדי שולים נמי והא ק״ל גדור חמשה ומחיצה חמשה מצטרפין וכדכתיבנ׳ לעיל (שבת ז:):
}
\textblock{\textbf{כפאה על פיה וכו׳ שבעה ומחצה פטור.} פרש״י משום דהוי׳ למעלה מעשרה ובשאינה רחבה ו׳ עסקי׳ שאלו הי׳ רחבה ששה כיון דמטא לעשרה בשבעה ומשהו נמי פטור דאמרי׳ גוד אחית מחיצתה והוה רה״י. ומקצת חכמים הצרפתים ז״ל אמרו דליכא למימר לבוד בשאינ׳ רחבה ששה דכשאר חפצים דעלמא הוא לפיכך פרשוה ברחבה ששה ואפ״ה בשבעה ומשהו חייב משום דשולים ממעטי בה כיון דאינה גבוה עשרה ומדין לבוד עושין אותה רשות אין אומרי׳ פחות משלשה מצטרף עם המחיצו׳ דל״א לבוד אלא במחיצות והני לאו מחיצות נינהו ול״ד לבית דקירויו משלים לעשרה דע״ג רה״י דהתם הא איכא עשרה וה״נ אם היתה גבוה עשרה עם השולים בלא אויר שולים מצטרפין עמה וכשיש בה שבעה ומחצה פטור שנמצא אויר שתחת שולים עשרה ונעשית רה״י. ואני איני אומר כן, דלעולם בין ברחבה ששה בין בשאינה רחבה ל״א בבלים לבוד ועכשיו כלי הוא אלא עיקר פטורה משום שכיון שהוא בתוך ג׳ נעשית כמונחת וכשהוא גבוה ז׳ ומחצה הרי הונחה ואותבה למעלה מי׳ לפיכך פטור ולא משום מחיצות כלל &lt;ע׳ ברשב״א&gt; וכן דעת רב אלפס ז״ל דלא מסיים הכא רחבה כמה, ויותר מזה כתבתי בס׳ המלחמות:
}
\textblock{\textbf{וכן בגומא.} י״מ וכן בגומא תשעה ר״ה וכ״ש פחות מתשעה דמשתמשי בה רבים ולא מדמי גומא דתשע׳ לעמוד אלא כל גומא עמוקה מג׳ מדמי לעמוד תשעה והיינו דאמרי׳ מאי לאו אסיפא וקתני פחות מכן מטלטלין וכל פחות מכן במשמע אפי׳ שמנה שבעה. ולא מסתבר, אלא פחות מט׳ לא משתמשי בה להניח שם כליהם כי היכי דליתברו ברגלי ב״א ובהמה ובדברים הנגררין בר״ה אבל עמוק תשעה משתמרי כלים בגוייהו שאין סתם כלים נוגעין בשפתה ופחות מכן דגומא, אתשעה:
}
\textblock{\textbf{לא ארישא.} פי׳ אאין מטלטלין מתוכה לר״ה ולא מר״ה לתוכה אבל לא קאי אשיעורא [דפחות מכאן] כלל. ותימא הוא, דהא וכן קתני כדאקשינן בעלמא (פסחים טז.). א״נ [א״ל] רישא אם טלטל חייב חטאת ועלה קתני וכן בפחות מכן אע״פ שאין מטלטלין אם טלטל פטור:
}
\textblock{\textbf{האי זירזא דקני רמה וזקפה לא מחייב עד דעקר ליה.} איכא למידק והתני׳ [דף צא:] הגונב כיס בשבת כו׳ היה מגררו ויוצא מגררו ויוצא (חייב) [פטור] שהרי איסור שבת ואיסור גניבה אין באין כא׳ אלמא אע״ג דלא עקר לי׳ מחייב. וראיתי מתרצין בשם הראב״ד ז״ל דשאני מוציא מרשות לרשות דעקירת הרשויות עקירה הוא לחייב עלה, ולפ״ד אינו כן שא״כ אף המפנה חפצים מזויות לזויות ונמלך עליהן כשהן ברה״י והוציאן למה הוא פטור נימא עקיר׳ גופו מרה״י לר״ה עקירה הוא ואע״פ שהגבהה הראשונה לא היתה לכך ואע״פ שלא עמד דעקירת הרשות עקירה הוא ועקירת גופו עקירה הוא דיהא המגרר עצמו כמגרר חפץ. אלא הך מימרא דזירזי דקני, א״ל דשאני רמה וזקפה ממגרר שהמגרר משעה שגירר הוציאה ממקומה הראשון ואפי׳ לא משכה במילואה אבל רמא וזקפה עדיין ראש אחד במקומה הראשון ומונחת היא שם וכי חזר והפך פעם אחרת לא עשה מלאכתו כאחת ועקירות הרבה הן והנחות הרבה הן לא עקר בבת א׳ כל הכלי אבל מגרר עקירה אחת הוא עקיר׳ בתחלת ד׳ והנחה בסוף ד׳ ולא הוי מונח בנתיים והיינו דלא אשמעינן רב יהודא מגרר ויוצא ולא בדברים המתגלגלי׳ אלא דוקא בדברים המרובעים שעוקר ומניח כגון זרזא דקני ורמא וזקפי׳. ועי״ל דההוא כגון שהיתה רה״י גבוה מר״ה וכשהוא מוציאה משפת אסקופת הבית נעקרה ונפלה לר״ה או לצדי ר״ה והרי יש כאן עקיר׳ וכ״ש למאן דאוקמא בפ׳ אלו נערות כגון ששלשל יד למטה מג׳ וקבלה והא ודאי עקירה, וזה מחוור לכל. אבל מצאתי בחבור הרמב״ם ז״ל כתי׳ שכתבתי, והוא העיקר. וה״נ מוכח בפ׳ כיצד צולין (פסחים פה:) בנגררין, דעקירה והנחה הוא:
}
\newsection{דף ט}
\textblock{הא דאקשינן \textbf{והא״ר חמא בר גוריא אמר רב תוך הפתח צריך לחי אחר להתירו.} איכא למידק עליה לימא לי׳ הכא בפתוח לר״ה הוא והתם בפתוח לכרמלי׳ והכי אוקמא רבא בפ״ק דערובין (ט.). וא״ל לא בעי לתרוצי הכי משום דהתינח לרבא לאביי דאמר בין הלחיים אסו׳ לעולם מא״ל ואל תתמה למה לא בעי לשנויי אליבא דרבא דמקשי׳ סתם אליבא דאביי דאשכחן כוותי׳ טובא בי ההוא דאקשינן בר״פ האשה שנתארמלה ליזול בתר רובא דרוב נשים בתולות נשאת וההוא קושי׳ מעיקרא אליבא דרב דאמר הולכין בממון אתר הרוב אבל אנן קיי״ל כשמואל דאמר אין הולכין בממון אחר הרוב ולדידי׳ ל״ק כלל כדמפורש בפ׳ המובר פירות ואפ״ה מקשי׳ לי׳ סתם כ״ש בזו וכיוצא בה אקשי׳ במס׳ תמורה ונברור חד לגבי כלב אליבא דמ״ד יש ברירה וקי״ל אין ברירה בדאורייתא ותירוציה דרב יהודא אמר רב הלכה הוא ולפתוח לכרמלית אבל פתות לרה״ר ל״צ למיהוי מקורה מדק״ל כרבא. ואעפ״כ אין זה מחוור לי, משום דרב אמאי איצטרך לפרוקי באסקופת מבוי אי כרבא ס״ל ואי כאביי קשי׳ לרבא מדרב דרב קשיש מינייהו טובא ולא לפרוקא אליבא דאביי אתי אא״כ אתה אומר דחד מחד פריקא נקט. ואחרים אומרים דלא מתוקמא בפתוח לרה״ר כי התם דא״כ אפי׳ פתח נעול כלפנים ומותר להשתמש בו דרבא ל״פ התם אלא סתם אמר בין הלחיים בין פתוח בין נעול בין מקורה בין ניתר בלחיים הכל מותר וכלפנים הוא כפשטא דשמעתא דהתם ובפתוח לכרמלית בין הלחיים אסור דמודה לי׳ רבא לאביי ותחת הקורה לעולם מותר דחודו החיצון יורד וסותם. והא דפירש״י ז״ל בהא דקאמרינן וקריו כלפי פנים פתת פתוח כלפנים מחודה החיצון של קורה לא מחוור כלל דאכולה אסקופא משמע, ובירושלמי (א,א) כל זמן שהפתח פתוח כולה כלפנים והטעם והואיל והותרה מקצתה של אסקופ׳ הותר׳ כולה דהא איכא היכרא טוב׳ דמיפרשן מרה״ר הוא כולה אסקופה ולא אתי לאפוקי לחוץ והיינו נמי דלא מוקי לה באסקופת בית משוס דלעולם בבית אין מה שנמשך לקרויו נמשך אחר הקרויו משום דבית אין היכרו והיתרו בקירוי אלא במחיצות:
}
\textblock{והא דאמרי׳ \textbf{מסייע לי׳ לר״י בר אבדימי.} נ״ל דרב אשי הוא שאמר כן לפי פירוקו אבל לפי פירוקא של ר״י אמר רב א״צ דה״ק ואם היתה גבוה ורחבה ד׳ אינה בכלל (ארבעה) המבוי ואין דינה כדין אסקופ׳ המבוי שתהא ניתרת בקירוי ושיהא הפתח פתוח אלא רה״י גמורה הוא ומותר לטלטל בכולה בלא קורת מבוי ואפי׳ במבוי מפולש שהוא אסור, היא מותרת. אבל לדברי (רש״י) [רב אשי]דאמר באסקופת בית מאי ה״ז רשות לעצמה דא״נ הויא כולה אסקופה ארבעה בזמן שהפתח נעול אסור דהא לא הוי בחוץ אלא פחות מד׳ ולא מצטרף בהדי מה שבפנים ובזמן שהפתח פתוח בלחוד הוא דמשתרייא וא״כ היינו רישא ועוד דאפי׳ הויא אסקופ׳ חוץ לפתח ד׳ דמותרת הול״ל הר״ז רה״י או לומר אע״פ שהפתח נעול כלפנים דהא בית לעולם רה״י הוה וכ״ש דאם היתה אסקופ׳ רחבה ד׳ דלא משמע אלא כולה בין דתוך הפתח בין דחוץ הפתח ומ״ה דייקי לרב אשי מדקתני ה״ז רשות לעצמה ש״מ דאסור לטלטל מזו לזו ומסייע לי׳ לר׳ (יוחנן) [יצחק] ב״א. ובודאי לריב״א לית הלכתא כאחרים דר״ש פליג עליו דאמר בפ׳ בל גגות (ערובין צא.) א׳ גגות וא׳ חצרות רשות אחת והתס פסקינן כר״ש לפי׳ סמך לו רבינו אלפסי ז״ל אפירוקא דר״י אמר רב דלדידהו לא אשכחן דפליגי רבנן אאחרים ורב אשי נמי לא פליג אההוא דינא אלא מר פרק לה בכה״ג ומר בכה״ג ול״פ. ולפיכך לא כתב רבינו ז״ל הא דאמרי׳ מסייע לי׳ לריב״א אע״פ שכתבה לההוא דאחרים וכן לא כתב סיפא דאם היתה אסקופה דכולה פשיטא כדפרי׳ דמלתא גופא דאר״מ אסור לכתף עליו מפורש התם בעירובין דלא אר״מ אלא בעמוד ואמת המיס אבל מכתש׳ וכ״ש קופה וכלי׳ גדולים לא:
}
\newsection{דף יא}
\textblock{\textbf{לוה אדם תעניתו ופורע.} אפ׳ במקומו במס׳ תעני׳ בס״ד:
}
\textblock{\textbf{אמר אביי היא היא.} פי׳ לאו למימרא דסבר אביי בכ״מ גזרי׳ נזרה לגזרה אלא בכיוצא בוו לפי שעשו כרמלית כרה״ר לכל דבר וכן אסרו לצאת בה״ש בכל מה שאסור לצאת משחשיכה וה״נ אומר אביי בפ״ק דביצה דמקשי היא גופא גזירה וכו׳ וכולה שמעתין מהוצאות בלחוד מייתי ומצאתי במקצת הנוסחאות ואביי אמר לך כולה חדא גזירה הוא ולרבא נמי איכא דוכתי דגזור בכה״ג כדאמרי׳ בפ׳ במה אשה כל שאסור לצא׳ ברה״ר אסור לצאת לחצר ואפי׳ בדברים שאם יצא בהץ לרה״ר פטור שאינו אלא מדרבנן אלא התם בתכשיטין שאין אשה עשוי לפושטן כשיוצאה לחוץ:
}
\textblock{והא דאמרי׳ \textbf{וכן בגת לענין מעשר.} י״מ שלא התירו אלא על הגת משום דמיחזי עראי אבל חוץ לגת אינו מוכיח ונראה קבע ולפיכך אסור ומאי וכן אראשו ורובו. ור״ח ז״ל כתב דה״ק: כשם שהעומד ברה״ר ושותה ברה״י אין מותר לו לשתות אא״כ יכניס ראשו ורובו למקום שהוא שותה שאם ישארו לו המים מותר להחזיר למקומן כן בנח לענין מעשר אין מותר לו לשתות מן היין שבגת קודם שיעשר אלא אם הם צונן שיתכן לו להחזיר המותר אבל בחמין שמפםידין היין וא״א לו להחזיר המותר אסור שכיון שמזגו בחמין נקבע למעשר ואסור. זה לשון ר״ח ז״ל:
}
\newsection{דף יב}
\textblock{הא ד\textbf{תני רבה בר שמואל יוצא אדם בתפיליו ע״ש עם חשיכה.} קשה לן, והא היוצא בשבת בתפילין פטור הוא כדאי׳ בפ׳ במה אשה יוצאה, ואם כן ל״ל משמוש בתפילין, הא אמרן לרבא דכל שאם יצא בשבת פטור אין גוזרין בו. ואפשר לומר דשאני תפילין דדרך מלבוש הוא ואין אדם פושט עצמו בחוץ ואי לאו משמושין הוה אסור (כד״א) [כדאמרן] בתכשיטין בחצר:
}
\textblock{הא דאמרינן \textbf{הלכה מולל וזורק וזהו כבודו אפי׳ בחול.} משום כבודו קאמר ולא מפני שהוא אסור, מדלא קאמר הלבה מולל וזורק ובלבד שלא יהרג אלמא א״צ להזהר כ״כ שלא יהרג ותנא דאמר ובלבד שלא יהרוג ובלבד שלא ימלול ר״א היא ור״א כב״ש ס״ל דאמרי אין הורגין את המאכולת בשבת דאיהו מדב״ש הוה כדאמרי׳ בכמה דוכתי (שבת קל:; נדה ז:) ר״א שמותי הוא, שפירושו מתלמידי שמאי, כדמוכח בהדיא בירושלמי במס׳ ביצה (א,ד) ובמס׳ סוכ׳ פ׳ הישן (ב,ח). א״נ קסבר ר״א לא נחלקו ב״ש וב״ה בדבר הזה דלא ס״ד לאוקמי מתני׳ כב״ש. מ״מ מסקנא דב״ה סברי מותר להרוג מאכולת בשבת אפי׳ לכתחלה ורבה דשדי להו לקנא דמיא ור״נ אמר לברתי׳ קטלן בשבת הוה דדבר הלמד מענינו הוא ומוכח בירושלמי דגרסי׳ התם חזקי׳ אמר כל ההורג כינה בשבת באלו הורג גמל שמואל מקטע ידא ורגלא ויהב לי׳ קומי עוקא ר׳ יוסי בר בון יהיב לי׳ גו צליחותא אר״ש בן חלפתא ולא מחלזון למות וחלזון יש בו גידין ועצמות ולא קתנו כל דבר שאין בו גידין אינו חי יותר מששה חדשים וכו׳. ומיהו בגמרא דילן בפ׳ שמונה שרצים (קז.) לא אמרינן הכי, אלא מה אלים שפרין ורבין אף בל דפרין ורביץ דלפום הבי נמי אמרי׳ התם שהורג פרעוש בשבת חייב ותמה אני על ר״ת ז״ל שכ׳ כאן הכינה היא הפרעוש ואי אפשר וגמר׳ בהדיא הוא בפ׳ ח׳ שרצים מ״מ נפרש שהפרעוש מין רמש פרה ורבה הוא. וי״א שהוא מין הידוע ההווה מן העפר בימות החמה שקוריץ בערבי אלבויגו״ת והוא אינו פרה ורבה אלא שחייבין עליו כמי שהוא פרה ורבה מזכר ונקבה שכן ההורג עכבר שהשריץ מן העפר חייב ולא פטרו מאלים מאדמים אלא כגון כינה שהווה מן הזיעה וכן תילעים שבאשפות ובדברים המוסרחין, וז״ד הרמב״ם ז״ל [הל׳ שבת יא:]. ול״נ דעכבר שהשריץ מן העפר חייבין עליו ודאי מפני שמינו פרה ורבה הוא אינו אלא שהוא כמין הסריסים אבל מין שאין בהם מפרי׳ ורבים כלל כגון אלברנו״ת פטור ולכתחלה נמי מותר בין לצוד בין להרוג דבכלל מאכולת הן וכדפרישית. וא״ת על מה שאמרנו ר״א מתלמידי בית שמאי הוא והלא תלמידו של ריב״ז הי׳ ור׳ יוחנן תלמידו של הלל הי׳ כמו דאמרי׳ בב״ב קטן שבכולן ריב״ז א״ל חזר ושמש שמאי או אחד מתלמידיו או שהסכים עמהם בדבריהם ונמנה עמהן כדאמרינן בו ביום גדשו סאה ור׳ יהושע שהי׳ מדב״ה אמר בו ביום מחקו סאה לפי שלא נעשה כרצון ב״ה שאותו היום כפופין היו בית הלל לבית שמאי. אי נמי ריב״ז קיבל מהלל ומשמאי כדתנן באבות, ושנה לו מדותיו של הלל ומדותיו של שמאי, והוא ראה דבריו של שמאי ונתחבר לתלמידיו:
}
\textblock{הא דאמרינן \textbf{ל״ק כאן בשמש קבוע כאן בשמש שאינו קבוע.} ק״ל, דמי עדיף שמש קבוע מאדם עצמו שאסור לו להבחין בין בגדו לבגד אשתו שהוא עיון הגס יותר מבדיקת כוסות וקערות ושם לא חלקו נמי בין משחא לנפטא אלא הבל אסור ונ״ל שהטעם כמו שמפורש בירושלמי אית דבעי מימר מפני נקיות ומפני הסכנה מותר ואמרינן נמי התם לקנב חוזרין מפגי נקיות ומפני הסכנה מותר ואף על פי כן בשמש שאינו קבוע ובמשח׳ אסרו מפני שהוא עיון דק יותר מאלו מפני שאינו רגיל לשמש לאחד וקרוב להטות:
}
\textblock{\textbf{וכולה פרשה לא, מיתיבי.} ק״ל, וכולה פרשה היכי סליק אדעתי׳ דשרי והא מתניתא קתני בהדיא אבל הוא לא יקרא וא״ל קרייה לחוד וסדור לחוד וכי קתני היכן התינוקת קורין לא לאור הנר אלא היכן קורין למחר והכי אקשי ובולה פרשה לא מסדר והארשב״ג מסדרין פרשיותיהן לאור הנר ומיהו למאי דאמרינן שאני תינוקת של ב״ר דאימת רבן עליהן מותר להן אפי׳ לקרות ודיקא נמי מדקתני מתני׳ אבל הוא לא יקרא משמע אבל הם קורין והא דקתני ברייתא מסדרין פרשיותיהן משום דאורחא דמילתא דאיגן אלא מסדרין בעלמא, ואינו מחוור. וא״ל, סדור כולה פרשה וקרייה כולה חד הוא, וכי אקשי׳ אדרבה בר שמואל וכולה פרשה לא, משום דס״ד דמתני׳ הכי קתני היכן התיניקות קורין בלילה וקורא עמהם מפני שהם שנים בענין א׳ ומותר לקרות אבל הוא לא יקרא בפ״ע ומ״ה אקשי׳ וכולה פרשה לא שריא בהדי תינוקת דודאי רבה בר שמואל כהדי תינוקות אמר דסידור משמע שהוא מסדר לתינוקות ואלו בפני עצמו אין זה מסדר ואיזה הוא מסדר זה המלמד סדר הפרשיות והם גומרין אותן והא דתניא מסדרין פרשיותיהן והיינו לפני רבן שמסדרין לפניו ללמוד סדר היום והפרשיות ופריק שאני תינוקות דאימת רבן עליהן והן מותרין והוא אסור שהם אין מונעין אותו מלתקן אם ירצה מפני אימתן, והן ודאי לא יתקנו, כן נ״ל:
}
\newsection{דף יג}
\textblock{הא דאמרי׳ \textbf{ופליגא דר׳ פדת.} פירש״י ז״ל הא דרב יוסף דפשט בעיין מדרשא דהך קרא פליגא דר״פ דאמר לא אסרה תורה אלא קריבה של גילוי עריות והיינו תשמיש ממש ושאר קורבה אפי׳ קרוב בשר מדרבנן הוא והיא בבגדה אפי׳ מדרבנן ליכא למיגזר. ולא מחוור, דמסקנא דעוף וגבינה לא עולה ולא נאכל אפי׳ למאן דאמר בשר עוף מדרבנן ואע״ג דרב יוסף לא הי׳ סבר הכי מעיקרא בפרק כל הבשר ועוד ק׳ לי אטו היא בבגדה לידי שאר קורבה דרבנן אתי לידי גילוי ערוה לא אתי ולא יאכל הזב עם הזבה מפני הרגל עבירה דתשמיש קאמר וכל עיקר בעיין משום שנוי ודיעות הוא דאתיא עלה ולא משום קולא דקריבה ותו קשי׳ בהקישא גופה מה ענין אשת איש לאשתו נדה בהיא בבגדה והוא בבגדו גבי אשתו כיון דלא רגילי בהכי איכא שינוי, גבי א״א ליכא אלא חוצפה ופריצותא. אלא נראה דה״פ דסוגיא: דבעיא דמעיקרא היא בבגדה והוא בבגדו אי שרי ואע״ג דאיכא קריבה דסתם ישנים במטה אחת נגעו אהדדי כדאמרי׳ בברכות והא איכא עגבות ומיתהנו מהדדו כדכתיב אם ישכבו שנים וחם להם ואיכא נמי חבוק וחשש דביאה ואתא רב יוסף למיפשט בה איסורא משום חששא דביאה ואפי׳ כשת״ל דקריבה שריא ואידחי משום שינוי ודיעות והדר פשטינן דאיסורא דקריבה גופא אסורא מהקישא דא״א דהתם ודאי כל קריבה אסורה ואפי׳ [בלא] יחוד דלא אתי השתא לידי ביאה, דהא אפי׳ להסתכל בה אסור בערוה, אף אשתו נדה בכל ענין אסור. וקאמרינן דהך הקישא פליגי אדר״פ דאמר לא אסרה תורה בשאר בשר אלא גלוי ערוה בלבד ואלו בא״א דבר ברור הוא שנאסרה כל קריבה שבעולם והכתוב אומר הרחק מעלי׳ אל תקרב אל פתח ביתה ולאו דמיפשטה בעיין לר״פ להתירא והתם איכא חששא דביאה ור״פ לא איירי בשינוי ודיעות כלל אלא דר״פ לית לי׳ הך הקישא ושרי קריבה בעריות דשאר ופירושו לא אסרה תורה לומר שהוא מותר כדאמרי׳ בעלמא לא אסרה תורה אלא פשוט ידים ורגלים ולא אסרה תורה אלא דמות ד׳ פנים בהדי הדדי דהיינו לומר דשרי והכי איתמר התם כל כי האי ריבותא ליכול לא אסרה תורה אלא רבית הבאה מלוה למלוה. ואי קשיא, והא תנן (קדושין פ:) לענין בנים קטנים, וישן עמהם בקרוב בשר, הגדילו זה ישן בכסותו וזה ישן בכסותו. התם ביחוד וקירוב בשר חששא דביאה איכא ור״פ היכי דליכא יחוד וחששא דביאה הוא דשרי קריבה ואי שינוי ודיעות סגי להיכר באשתו נדה לא מתסר משום קריבה וכן עולא דמנשק לאחוותי׳ אבי ידייהו בשינוי הוא שלא כדרך הנושקין ובלא יחוד ור״פ בשינוי ודיעות לא איירי כלל: }
\textblock{\textbf{בימי לבונך מהו אצלך.} פי׳ הי׳ יודע אבא אליהו שטעה אותו תלמיד בכך לפיכך שאל (לו) [לה] כן אבל לא מפני שיהא שום קולא בימי לבון מבימי נדות ונראה שזה הליכון הוא לאחר שתיקן רבי שאם ראתה יום א׳ או ב׳ וג׳ שיהא מונה ששה [נ״א - שבעה] והם, ומאחר שהשלימה שבעה ימים מתחלת ראיי׳ קורין אותן [המותר] ימי ליבון לומר שאינו ימי נדות שכבר יצאה מהן והן ימי לבון לפי שהם נקיים, והי׳ אותו התלמיד מיקל בדבר מפני שהיא חומרא מדבריהם. ואף על פי שלא טבלה הי׳ מיקל, לפי שלא הי׳ יודע לטבילת נדה עיקר מן התורה, ולא הי׳ יודע הא דתניא (סד:) תהא בנדתה עד שתבוא במים כדכ׳ כאן רש״י ז״ל, ולפיכך נענש. ואין זה מחוור שיטעה מי ששימש ת״ח הרבה בכך ולא היה יודע שטבילת נדה מן התורה, וכן קשיא לי שנלמוד טבילת נדה מכאן, תהא בנדתה עד שתבוא במים, דא״כ זקנים הראשונים דמוקמי לי׳ לא תכחול ולא תתקשט במיני צבעונין, טבילה לנדה מנ״ל. אבל נ״ל שטבילה לנדה ולזבה מפורשת הוא מן התורה, בזב כתוב (ויקרא טו:ב) איש כי יהי׳ זב כו׳ פרשה דטומאה, וכתיב וכי יטהר הזב ונו׳ וספר לו ורחץ בשרו במים חיים וטהר, וכשבא הכתוב לפרש בין נדה וזבה כתיב ואשה כי תהי׳ זבה דם יהי׳ זובה כלומר טומאת הזב בלובן והזבה באודם וכתיב וספרה ואחר תטהר בטהרה האמורה לענין הזב למעלה וזהו נמי פשוטו של מקרא ומ״ש שאין הזבה צריכה מים חיים אפשר שראו ריבוי בכתוב מן ואחר תטהר או ממקום אתר אבל טבילה מפורשת היא ובהרבה מקומות בתורה כן שכתוב וטמא עד הערב וטהר ואין כתוב ורחץ במים ואנן לומדין מן המפורשין שכתוב שם ורחץ בשרו במים וטמא עד הערב ואין טומאה פורחת דמ״מ א״א למי ששימש ת״ח שלא תהא לו נדה בהכרת עד שתטבול. ומ״ש ר״ע תהא בנדתה עד שתבוא במים, משמע לי׳ דהיינו אזהרה לאשה שתהא נזכרת לנדתה ותהי׳ מתרחקת עצמה ומתנדה מבעלה ומטהרת כל ימי נדותה והאי דקאמר עד שתבוא במים לישנא בעלמא כל ימי נדותה עד שתשיהר ובלשון הראשון של זקנים הראשונים תניא נמי לא תכחול ולא תפקוס עד שתבוא במים, וזו ראי׳ לדברינו. ויש לדחוק ולומר שטבילה שבפרשה שאמרנו, לטהרות היא, אבל לבעלה או מדר״ע או מדרשה התם אך במי נדה יתחטא מים שהנדה טובלת בהן וגם זה אינו מתחוור ומ״מ א״א למשמש ת״ח הרבה שיקל מחמת שלא תהא צריכה טבילה. אלא נפרש דטובלת היתה אשתו לסוף ימי נדותה של תורה ולפי שהי׳ מיקל ופורץ גדר של חכמים נשכו נחש שכל דבריהם כגחלי אש, וכן מפורש בס׳ הישר (סי׳ קנד). וי״א דשומרת יום כנגד יום היא שנקרא ימי ליבון והי׳ מיקל לפי שהיא חששא בעלמא ושמא לא תראה ולא תבוא לידי זיבה, וסוף דבר שלא כהוגן נעשה וראה מה עלתה בו. אבל לשון הראשון יותר נכון, שאף בזה לא הי׳ ראוי לטעות שאינה חששא אלא ספירה היא שכשם שסופרת שבעה לזיבה כך סופרת אחד לאחד שהיא זיבה קטנה ואם טבלה אפי׳ בשמיני ושמשה אינו אלא תרבו׳ רעה. ואפשר דמשום חומרא דאתי׳ לידי קולא הוא דבטלוה לטבילתה בדורות האחרונים ומשום טהרות נהגו בה תחילה: }
\textblock{\textbf{האוכל אוכל ראשון ואוכל שני.} פיר״ת ז״ל שאין זה פסולי גויה דההוא בדורות ראשונים נתקנו. ובמס׳ יומא (פ:) מספקא להו נמי בפסול גוי׳ אי דאוריייתא וההוא אינו פוסל במגע תרומה אלא נפסל גופו מלאכול בתרומה וטעמא דההוא משום דמחזי כמאן דנגעי בהדדי ומטמו במעיו ולהכי בעו חצי פרס שנתמעט בעיכול מכביצה וחזרו ב״ש וב״ה ונזרו אוכל כביצה ושיפסול במגע תרומה ומפורש בס׳ הישר (סי׳ ר״ה,שי״ג): }
\textblock{\textbf{והבא ראשו ורובו במים שאובין וטהור שנפלו על ראשו ועל רובו ג׳ לוגין מים שאובין.} פי׳ נ״ל דאף טהור קאמר וכ״ש טבל ועלה ונפלו על ראשו ועל רובו ג׳ לוגין מים שאובין ומכלל טהור לא נפיק. ועוד דעיקר גזירה בנפילה הוה שהיו נותנין עליהן ג׳ לוגין מים שאובין וכל עיקר לא גזרו ביאה אלא משום נתינה שהוקבעה להם ובדין הוא דהו״ל למיגזר תרווייהו בטמא שטבל ולא כלל בטהור, אלא דאי לא הא לא קיימא הא כדמפורש בגמ׳. וא״ת ביאה נמי הא לא קיימא, כיון דטהור לא מיטמי (לב״ה) [בה], א״ל דאי ביאה לא קיימא כולה האי לא איכפת לן שעיקרה של גזירה נפילה הוא וא״ת לטהור נמי לגזור ביאה ומתקיימא כולה מילתא שפיר נוכל לומר דבהכי סגי להו א״נ שא״א ליגזר בטהור ביאה שא״כ אין לך אדם רוחץ במרחץ ולא בשום מקום במים שאובין וא״א לרוב ציבור לעמוד בגזירה זו לפיכך לא גזרו כן וזה הטעם מספיק לכל דמ״ה גזור ביאה מפני שא״א לעמוד בה ואי אתו לבטלה תיבטל שא״א לגזור עוד בה גזירה אחרת, כן נ״ל. וראי׳ לדבר מההוא דגרסי׳ במס׳ גיטין (טז.), חציו בנפילה וחציו בביאה מהו, ומשכחת לה בטמא שטבל, וש״מ שמטמא בכל אחת משתיהן שאלמלא לא היו מטמאין אלא זה בנפילה וזה בביאה פשיטא דחציו בנפילה וחציו בביאה טהור דהא כולו בנפילה [לזה] או כולו בביאה [לזה] נמי טהור וליכא דמטמא בשתיהן, והתם לא קמבעיא לי׳ אלא אי מצטרפות הואיל וכל אחת מטמא בפ״ע וכדמוכח שמעתא: }
\newsection{דף יד}
\textblock{\textbf{ושדי לפומיה ופסיל ליה.} פי׳ משום שגזרו במשקין [של] (כגון ש)טבול יום, נקט פסיל להו, אבל אוכלין טמאין טמויי נמי מטמא להו דמשקין בתרומה וי״א דמשקין שבפיו מאוסין הן ואי שדי להו לא מטמו אחרינא, אלא מפסיל פסול להו מלאכול:
}
\textblock{והא דאמרינן \textbf{מ״ד הא שכיחא והא לא שכיחא.} ק״ל, הא נמי שכיחא, משום דילמא אכיל אוכלין דתרומה ושקיל משקין טמאין ושדו לפומי׳ ופסול ואמאי לא אמר טעמא מהא ותרוייהו שכיחי. ומפרקי אוכל תרומה לדעת מזהיר זהיר מאוכלין טמאין אבל אוכל אוכלין חולין טמאין שמא יתן תרומה לתוך פיו שלא לדעת, ואינו מחוור. וי״ל שאין חוששין שמא יאכל תרומה ויטמא אותו בפיו במשקין טמאין הבאין עלי׳ לאחר מכאן לפי שכבר נמאסה משלעסה ועמד בפיו ואין חכמים חוששין כ״כ לטומאתה שיגזרו עלי׳ שאינה ראוי׳ אבל עיקר החששא לשמא יאכל אוכלי חולין טמאין או משקין ויביא תרומה ויתננה לתוך פיו שנמצא שהוא מטמאה עם כניסתה לפיו וראוי׳ היא שעדיין לא נפסלה מלאכול ויש בידו שתים א׳ שהוא אוכל תרומה טמאה שעדיין במקום שיכול להחזירה הוא וא׳ שהוא מטמא תרומה טהורה הראוי׳. ולדידי לאו קושי׳ הוא כלל, לפי שגזרו על מי שכבר אכל אוכלין טמאין שיפסול תרומה משום מי שהי׳ הדברים טמאין בפיו שמטמאה ודאי, אבל לגזור משום מי שבפיו אוכלין טהורין אינו [השם], ודבר ברור הוא:
}
\textblock{\textbf{אילימא הך גזור ברישא הא תו ל״ל.} פירש״י ז״ל הלא גזרו סמוך לנטילתן מגע תרומה ואכתי ק׳ לי הא נ״מ שאפי׳ נטל ידיו ע״מ לאכול ושמר ידיו ולא הסיח דעתו מהן אלא במגע הספר שיטמאו מחמת הספר ואין בכלל גזרה ראשונה אלא מסיח דעתו דהיינו סתם ידים וא״ל דהא ודאי לא גזור דהא טעמא משו דר׳ פרנך וטעמי׳ דר׳ פרנך משום דהידים עסקניות הן אבל מי שנטל ידיו תיכף ושמרן מותר לו לאחוז ס״ת ערום הלכך אין הספר מטמא אותן אלא כשהסיח דעתו מהן. ואין תי׳ זה נכון בעיני, לפי שרבותינו שהיו בתר ב״ש וב״ה כגון ר״מ ור״י וחבריו נחלקו אתה ספר מטמא הידים ומחלוקות אחרות בענין זה במס׳ ידים ומנ״מ להו מאחר דגזור בכל הידים. לפי׳ נראה דה״ק הא תו ל״ל למגזר אי בשהסיח דעתו ודאי בלא מגע ספר טמאות ואי בשלא הסיח לא היו צריכין לגזור עליהן שאין טעמא של גזירה אלא משום דרבי פרנך ודר׳ פרנך ליכא אלא משום היסח הדעת ומפני שהידים עסקניות הן וכבר הי׳ זה בכלל הגזירה הראשונה ולא היו צריכין לגזור עלי׳ גזירה אחרת בפ״ע ומפרקי׳ אלא הא גזור ברישא וגזרו סתם על הספר שיטמא את הידים ולא חילקו בין מי שהסיח דעתו למי שלא הסיח ופשטה תקנה זו בישראל ואח״כ גזרו על כל הידים שהסיח דעתו מהן וגזירה ראשונה לא זזה ממקומה שהי׳ צריך מנין אחר להתיר נמצא שהספר פוסל כל הידים עכשיו דגזירה ראשונה כך היתה ובמקומה עומדת לפיכך נחלקו חכמי ישראל והאחרונים איזה ספר מטמא והיאך מטמא:
}
\textblock{הא דאמרינן \textbf{אילימא במשקין הבאין מחמת שרץ מן התורה הוא.} ק״ל והא טומאת אוכלין ומשקין לטמא אחרים לאו דאוריית׳ כדאי׳ בפסחים וי״ל ההוא [לאו] מי״ח דבר הוא אלא מימות נביאים אחרונים הוא דכתיב הן ישא איש בשר קודש ונגע בכנפו אל הנזיד ואל היין ואל השמן היקדש ומעלות דרביעי בקודש ושלישי בתרומה דהוו מימות הראשונים לא משכחת להו אלא בהדי משקין טמאין והא דקאמר׳ דאוריי׳ הוא אליבא דמ״ד אפי׳ לטמא אחרים דאורייתא א״נ ה״ק עיקר טומאה דאורייתא שהרי שניים הן וכשמטמאין אחרים כבר נגזר שאם אין אתה אומר כן אף האוכלין עצמן מי״ח דבר הן:
}
\textblock{\textbf{אלימא במשקן הזב דאורייתא הוא.} בדין הוא דהו״ל לאקשויי א״ה טמויי׳ נמי ליטמי, דמשקה דזב אב הטומאה הוא והכלים ראשון ומטמא את התרומה אלא איידי דאקשה לעיל דאוריי׳ הוא אקשי נמי הכי:
}
\textblock{\textbf{אלא במשקין הבאין מחמת שרץ.} פי׳ אפי׳ במשקין הבאין מחמת שרץ מי״ח דבר אבל משקין הבאים מחמת ידים מטמאין כלים כדאמרי׳ פ׳ שלשה שאכלו שמא יטמאו משקין שבידו ויחזרו ויטמאו את הכוס. תדע דהא שנטמאו במשקין בחד גונא קתני , לאוכלין וכלים, כן נ״ל:
}
\newsection{דף טו}
\textblock{הא דאמרי׳ \textbf{י״ח דבר גזרו וי״ח דבר נחלקו.} נראה שרש״י מפרש שאינן אלא י״ח שגזרו ובהן נחלקו אלא שעמדו מנין ורבו ב״ש וגזרו בהן בכולן ולפי פי׳ למחר חזרו בהן ב״ה והודו למניינם. ולפי פשט הדבר בי״ח אחרים נחנקו ולמחר הושוו ובספר הישן מא״י מצאתי השיטה מוחלפת די״ח דבר גזרו ובי״ח דבר נחלקו ולמחר הושוו ובפי׳ ירושל׳ מצאתי השיטה בי״ח דבר נחלקו ובי״ח דבר הושוו וי״ח גזרו פי׳ י״ח דבר נחלקו וגזרו וי״ח דבר שהשוו שמנה מיציאות שבת והכנסות תשיעי ספר עשירי מרחץ י״א בורסקי י״ב לאכול י״ג לדין י״ד חייט ט״ו לבלר ט״ז מפלה כליו ״׳ז לא יקרא י״ח לא יאכל הזב עם זבה ושמנה עשר שנחלקו שבסוף המשנה דיו סממנין כרשינין אונין צמר חיות עופות דנים מוכרין טוענין מגביהין עורות כלים וחמשה מהן שהוזכרו בבריית׳ (יח:) לא ישאילנו ולא ילוונו ולא ימשכנו ולא יתן לו ואיגרות, הרי די״ח במחלוקת:
}
\textblock{\textbf{שמאי אומר מקב לחלה.} שמעתי בשם ר״ת דכיון דכתיב (במדבר טו,כ) בחלה תרימו תרומה, פי׳ ותנתן לכהן, וקיי״ל (יומא פ.)שיעור אוכלין בכביצה פחות מכן אינה חלה ולא מקיימא בי׳ מצות נתינה ושמאי אזיל בתר דעתו של בע״ה שהוא מפריש א׳ מכ״ד נמצא שהוא מפריש מקב כביצה והלל אזיל בתר דעתו של נחתום שהוא מפריש א׳ ממ״ח נמצא מפריש מקביים חלה של כביצה. ואחרים מפרשים דבעריסותיכם דכתב רחמנא פליגי, מ״ס עריסותיכם תרתי והעומר הוא עשירית האיפה (מלגאו) [מלבר] שהן שני קבין, ומ״ס עריסותיכם דכל חד וחד משמע, וכ״פ הראב״ד ז״ל. ואינו נכון, דקב וקביים דשמאי והלל וקב ומחצה דרבנן תרוייהו ירושלמית ולא מדבריות, ועוד שאין זה עישור אלא תשיעי:
}
\textblock{\textbf{מלא הין מים שאובין פוסלין את המקוה שחייב אדם לומר בלשון רבו.} פי׳ הראב״ד ז״ל דטעמי׳ דהלל מפני שהוא המדה הגדושה שנאמרה בתורה כדכתיב ושמן זית הין ואע״פשנאמרה בתורה מדות קטנות כיון דמים שאובין לפסול את המקוה דרבנן אזלי׳ לקול׳ ולא מיפסל אלא בשיעור הין ולפיכך שנה זו רבו הין לגלות לי שמפני שנאמרה בתורה הוא פוסל את את המקוה ושמאי סבר ט׳ קבין לפי שהן ראוין לשטיפת כל הגוף ועזרא תיקן לבעלי קריין לפיכך חשיבי כמקוה פסול ופוסלין. וחכ״א ג׳ לוגין מפני שהן חשובין שנתנם תורה לקרבנות צבור והוא רביעית ההין ואזלי׳ בתר שיעורא זוטא ואע״ג דאשכחן לוג שמן בצבור מיהא לא אשכחן פחות מג׳ לוגין ולא אשכחן לוג אלא לשמן אבל יין אין פחות מרביעית ההין והוא ג׳ לוגין:
}
\textblock{והא דתנן \textbf{שמאי אומר כל הנשים דיין שעתן} אפרש במקומה (נדה ב׳ א׳ ד״ה ב״ש) בס״ד:
}
\textblock{לא גרסי׳ בדר׳ ישמעאל בר׳ יוסי בשמונים שנה של \textbf{כלי זכוכית,} דא״ה קשי׳ דהא יוסף בן יועזר ויוסף בן יוחנן גזור ואע״ג דאיכא למימר דגזור מעיקרא ולא קבילו מינייהו וגזור בשמונים שנה וקבילו מינייהו מדלא מקשי׳ ומפרקי׳ לה בגמ׳ כדמקשינן ומפרקינן לה מארץ העמים, ש״מ לא גרסינן לה:
}
\newsection{דף טז}
\textblock{ה״ג וכן במשניות מדוקדקות: \textbf{כלי חרס וכלי נתר טומאתו שוה מיטמאין ומטמאין.} דהיינו גבן ואין מקבלין טומאה מגבן אלא מתוכן, וגבן ואחוריהן חדא מילתא הוא אלא דגב משמע כל חוץ שלהם ולפרושי מילתא נקט הכי לומר דאין להם טומאה כלל לא מאונגן ולא מאוזנן דכלהו גב נינהו. ומה שפירש״י ז״ל אחוריהן כגון חקק בית מושב שהוא תוך אינו ענין במשנה זו דאי דרך תשמישן הוא מיטמאין ואם לאו אינן מיטמאין וה״נ מוכח בכמה מקומות במס׳ כלים דאחוריים היינו גב דכלי, דוק ותשכח התם בפ׳ כ״ה [מ״ו]:
}
\textblock{\textbf{הניחא למ״ד לא לכל הטמאות אלא לטומאת מת.} פלוגתא דרשב״ג ורבנן הוא בסיפא דהך מתני׳ גופא, והוא במס׳ כלים בר״פ [י״א], דתנן כלי מתכות פשוטיהן ומקבליהן טמאין נשברו טהרו חזר ועשה מהן כלים חזרו לטומאתן הישנה רשבג״א לא לכל טמאות אלא לטומאת נפש. ודאמרי׳ נמי (יז.) הניחא למ״ד כלי טמא חושב משקין פלוגתא דר״מ ור׳ יוסי הוא בתוספתא דמסכת מכשירין דתניא עריבה שירד דלף לתוכה המים הניתנין והצפין בכי יותן נטלה לשפכה בש״א בכי יותן ובה״א אינן בכי יותן בד״א בטהורה אבל בטמאה הכל מודים שהן בכי יותן דר״מ רי״א א׳ טמאה וא׳ טהורה בש״א בכי יותן ובה״א אינן בכי יותן:
}
\textblock{\textbf{המניח כלים תחת הצינור א׳ כלים גדולים כו׳ פוסלין את המקוה.} בצינור שקבעו ולבסוף חקקו הוא ואליבא דמ״ד בתס׳ ב״ב אינו פוסל את המקוה, וכן פירש״י ז״ל. ויש לפרשה בצינור מפולש שאינו עשוי לקבלה שאינו פוסל את המקוה ותנן במס׳ מקואות פ״ג סילון שהוא צר מיכן ומיכן ורחב באמצע אינו פוסל מפני שלא נעשה לקבלה. ושנינו עוד המניח טבלה תחת הצינור אם יש לה לבזבז פוסלת את המקוה ואם לאו אינה פוסלת ותנן נמי החוטט בצינור לקבל צרורות פוסל את המקוה מכל הני שמעת מינה שכל שהצינור אינו עשוי לקבלה כגון אלו שלנו אינו פוסל את המקוה ופי׳ כלים גדולים העשוין לנחת שאינן מקבלין טומאה שלא תאמר אין שם כלי עלייהו כדקתני נמי כלי גללים ואצ״ל קטני׳, ולא כדפירש״י:
}
\newsection{דף יז}
\textblock{\textbf{טמאוהו משום כלים המאהילין על המת.} פירש״י ז״ל טמאוהו טומאת ערב ומאן דאזא סבר שטמאוהו טומאת שבעה מדין אוהל. ואית דמקשו הכא ה״ד, אי בפשוטי כלי עץ אין מקבלין טומא׳ אי איכא חרב חרב הרי הוא כחלל וניחא להו הב״ע שיש במרדע בית קבול לחרב המרדע וטמא משום מקבלי כלי עץ ואע״פ שהיא עשוי׳ למלאות שמי׳ קבול כדאמרי׳ בסוכה בנקבות פסול וכו׳ ואינו טמא משום כלי מתכת בדליתי׳ לחרב המרדע יש בו חרב המרדע בטול הוא אגב גופא של מרדע כדתנן (כלים יג,ו) מתכת המשמש את העץ טהור ולפיכך אין אומרים בה חרב הרי היא כחלל. ולדידי לא ניחא לי בהאי תירוצא, דכי אמרי׳ חרב ה״ה כחלל לאו דוקא חרב אלא ה״ה לכלי שטף הנוגעין באב הטומאה של מת דתנן בריש מס׳ אהלות ד׳ טמאין במת ג׳ טמאין טומאת ז׳ וא׳ טמא טומאת ערב כו׳ עד כיצד כלים הנוגעין במת ואדם בכלים וכלים באדם טמאין טומאת ז׳ הרביעי בין אדם בין כלים טמאין טומאת ערב אר״ע יש לי חמישי השפוד התחוב באוהל האהל והשפוד ואדם הנוגע בשפוד וכלים באדם טמאין טומאת ז׳ אמרו לו אין האהל מתחשב מדקתני כלים סתם ש״מ שכל כלי הנוגע באב הטומאה הבא מן המת או אהל על המת הוא נעשה אב הטומאה ויש ראיות אחרות ובמקום אחר אכתוב מה ששמעתי מהן בעז״ה ולפ״ז קשה איך אר״ע משום כלים המאהילין טומאת ערב משום חרב הוא טמא טומאת ז׳. ואני אומר דה״נ קאמר ר״ט שהראשונים שאמרו האיכרעובר ומרדע על כתפו אהל צדו האחת על הקבר טמאוהו טומאת ז׳ משום כלים הנוגעים במת אמרו ודוקא במרדע שמקבל טומאה וה״ה לכלי כל שמקבל טומאה ור״ע תיקן שמשום אהל טמאוהו ואפי׳ אין המרדע מקבל טומאה נמי ודוקא בעובי המרדע ואפשר דנ״מ שהנזיר מגלח על אותו טומאה ומזה שלישי ושביעי ואלו משום מגע כלים הנוגעין במת אמרי׳ בהדיא במסכת נזיר אטו מאן דנגע בכלים בר הזאה הוא, בתמי׳. והוי יודע דר״ט קיבל מר״ע שהוא שמע וטעה והוא דרש והסכים לשמועה וכמו שאמרו לו בהרבה מקומות כיוצא בזה כל הפורש ממך כפורש מחיים והיינו דלא אמרי׳ בגמ׳ א״ה לר״ט בצרי לה וכדאקשי׳ לר״י ואע״ג דאיכא למימר דר״ט מני אוכלין וכלים בתרתי כיון דלא אמרי׳ הכי בהדי׳ בגמ׳ לא ניחא להו למימר הכי:
}
\textblock{\textbf{גזירה משום הנושכות.} פירש״י ז״ל אשכולות נושכות זו את זו וכשבא להפרידן נסחט המשקה עליהם וכיון דעבד בידים ולא אפשר בלא סחיטה מכשר ואינו מחוור. ור״ח ז״ל פי׳ משום הנושכות יש מי שאומר שנושך אדם מאשכול ונוטפין ממנו משקין. ובשם גאון ז״ל מצאתי כשאדם בוצר כרמו יש מהן שהגרגרים שלהם מדובקין זה עם זה ונושכין זה את זה מפני דבוקן ואע״פ שהמשקה יוצא מהן אינו הולך לאיבוד והמשקה עומד ונשמר בדבוק אותן גרגרים ואינו נופל בקרקע ומ״ה הוכשר, זה פי׳ גאון ז״ל, ויותר מחוור מן הראשונים:
}
\textblock{\textbf{והלא מגיס.} והוא צובע שכן דרך הצובעין להגיס בבגדים. ומיהו גבי בישול נמי המגיס חייב משום מבשל כדתני׳ בתוספ׳ ומייתי לה בגמ׳ בפ׳ המביא כדי יין א׳ נותן את האור וא׳ נותן את העצים וא׳ נותן את הקדירה וא׳ מגיס כולן חייבין ומיהו לא גזרינן דילמא מגיס כדגזרי׳ בצובע שאין דרך כ״ב להגיס בה אלא משעה ראשונה כדי שיתערב הכל ויתבשל. ונראה דלא מחייב אלא בהגסה ראשונה שאינו מתבשל מהרה אלא בהגסה זו שנמצא אף הוא מבשל &lt;הג״ה - פי׳ שכבר הגיס פ״א מע״ש&gt; ובההוא ליכא למיחש כדאמרי׳ קדרה חיתה שרי אבל משהגיס דבלא מגיס מתבשל אף המגיס פטור דמאי עביד הלכך ליכא למיגזר בשבת משום מגיס דלאו קרובי בישולא איכא דבלא״ה מתבשל א״נ א״ל דמש״ה לא אדכרו מגיס משום שחתוי הגחלים כוללת יותר שיש לחוש אפר במבושל כמאכל ב״ד ובצלי ופת ואלו משום מגיס ליכא לפיכך הזכיר חיתוי גחלים בכ״מ ובקדרה חיתה ליכא משום מגיס כי היכי דלית בה משום חיתוי גתלים דהא מסח לי׳ לדעתי׳ מיני׳, ולא נגע בו כלל. וי״ל עוד דגבי צבע יש לחוש לפי שהוא קרוב להיורה והוא׳ מפסיד הרבה, וכ״נ בירושלמי (א,ו).
}
\textblock{\textbf{בשיל ולא בשיל אסור.} פי׳ משהגיע למקצת בישול שהוא קודם למאכל ב״ד עד שיתבשל לגמרי כמאכל כל אדם ואפי׳ מצטמק ויפה לו נמי בכלל זה שהכל אסור ואפשר דלא איירי הכא במצטמק, ולאו משום דשרי אלא דלא פי׳ רבינא הכא, וכך אמר בעל הלכות גדולות ז״ל מאכל ב״ד בשיל ולא בשיל, ודינין הללו יתבררו לקמן (שבת לז:) בס״ד:
}
\newsection{דף כ}
\textblock{\textbf{קרא לאברים ופדרים הוא דאתא.} פי׳ לאברים ופדרים של חול שהעלן בחול ע״ג המזבח והן נשרפין והולכין שהוא מותר לחתות ולהבעיר בהן ולהוסיף בהן אש באותה המערכה עצמה שלהן, אבל להעלות או לעשות מערכה אחרת בפ״ע לשל חול אסור ושל שבת ודאי מקטירן והולך, ובתוס׳ מאריכין בזה. (יי״נ ובנותיהן ופתן כולן אפרש אותן במקומם במס׳ ע״ז בס״ד.)
}
\newchap{פרק \hebrewnumeral{2} במה מדליקין}
\newsection{דף כא}
\textblock{}
\textblock{\textbf{חלב מהותך וקרבי דגים שנמוחו.} פי׳ רש״י ז״ל חלב מהותך היינו חלב מבושל ולא מסתבר דבמתני׳ לגמרי אסרי רבנן דומיא דשארא ועוד נימא חלב מבושל כלישנא דמתני׳. אלא חלב שהותך ועדיין הוא מהותך ודומה לשמן וחלב מבושל הוא שנתבשל ונקרש וכן מוכח בירושל׳ דגרסי׳ התם רב ברונא אמר חלב נותן לתוכו שמן ותני שמואל כל שמתיכין אותו ואינו קרוי קרוש מחוי הוא ומדליקין בו ואי בשאינו מחוי אפי׳ נתן שמן לתוכה אסור רב חנינא ורב ברונא בשם רב חלב מהותך וקרבי דגים מדליקין בו משמע דחלב שמתיכין (בו) אותו ואינו קרוש הוא ראוי להדלק׳ ועלי׳ איתמר בגמ׳ דמימשך בעיני׳ ובדין הוא דבעיני׳ שרי אלא דרבנן דגמ׳ גזור בה ובירושל׳ לא גזרו אבל הקרוש שאינו מחוי אפי׳ נתן לתוכו שמן אסור כדמפורש התם. והא דאמרי׳ א״ב דרב ברונא אמר רב ולא מסיימי ה״פ דת״ק אמר ולא בחלב אפי׳ מחוי אפי׳ ע״י תערובו׳ משמע וחכמים בתראי לא אסרי אלא חלב מבושל שקרש אבל מחוי לא אסרי אלא מותר ע״י תערובות. א״נ מבושל דאסרי רבנן בתראי אפי׳ מחוי ולית להו דרב ברונא וחלב דרבנן קמאי כשהוא כעין חלב שהוא קרוש אבל מחוי ונעשה כעין שמן שרי בתערובות מ״ה לא מסיימי. ואי קשיא ונימא כל תנא בתרא לטפויי קאתי כדאמרי׳ בפ׳ המוכר פירות א״ל כיון דאנחום פליגי ולדידי׳ מהדרי לי׳ ליכא למימר הכי דהא לטפויי עלי׳ אתי ואיכא דמפרקי דת״ק תנא חלב דומיא דשאר פיסולי מה התם לית להו תקנתא לעולם ואפי׳ ע״י תערובות אף חלב לית לי׳ תקנתא לעולם:
}
\textblock{הא דאמרי׳ \textbf{כבתה זקוק לה ומותר להשתמש לאורה.} ק״ל, מנ״ל שמותר להשתמש לאורה דלמא אף בשבת אין מדליקין בהן לפי שכבתה זקוק לה ובשבת א״א להזקק לה (וא״ל דס״ל לתלמודא כיון דבשבת א״א להזקק לה) לא חיישי׳ ומדליקין בכל השמנים מכיון שאפשר שלא יכבו אע״פ שפעמים שכבין לא אכפת לן ושבת למ״ד כבתה זקוק לה דינו שוה לחול למ״ד כבתה אין זקוק לה כשם שבחול מותר להדליק בשמנים הללו ואם כבו אין זקוק לה אף בשבת כן לד״ה ואין מחזרין להדליק בפתילות ושמנים אתרים שבשבת לד״ה כבתה אין זקוק לה ואין לך בה אלא מצות הדלקה בלבד. אבל קשיא לן מה שפי׳ רבינו הגדול ז״ל, ש״מ כבתה אין זקוק לה מדקאמר מדליקין בהן בשבת דאי כביא לא מצי מדליק בה ולפי פירושנו אין ראי׳ משבת דלכ״ע אין זקוק לה אלא מחול מדלא חיישינן דלמא פשע כדפרש״י ז״ל ולפי דברי רבינו אלפס ז״ל יש לדחוק ולומר הא דאמרי׳ קסבר מותר להשתמש לאורה ומשום דלא ליפלוג אדר׳ חסדא אלא בחדא אמרי׳ הכא כלומר בדכבתה ודאי פליגי בלהשתמש לאורה לא פליגי דמותר הוא:
}
\textblock{ודאמרי׳ \textbf{אסור להשתמש לאורה.} כל תשמישין במשמע בין דמצוה כגון סעודת שבת בין דרשות ורב אסר לקמן הוסיף אפי׳ הרצאות מעות שהוא ענין קל ואינו נראה כנהנה שמרחוק הוא מרצה אותן אפ״ה אסור ומש״ה מתמה שמואל עלה דההוא וכי נר קדושה יש בו. וא״ת בסעודת שבת מאי ביזוי מצוה איכא והא הוא מצוה לאו מלתא הוא דכל שאדם משתמש ממצוה זו לאחרת ביזוי מצוה היא זו דמיחזי כמאן דלא חביבא לי׳ הך מצוה ועביד מצוה אחריתא וע״כ לא שרי אלא להדליק מנר לנר משום דתרוייהו חדא מצוה נינהו כלומר מצות נר חנוכה ומין במינו אינו בטל אבל שאר מצות נראות כמבטלות זו את זו. ומסתייע הך סברא מדאמרי׳ בגמרא דבני מערבאי (ב.) ועוד מן הדא דר׳ תחליפא שאיל לרב חסדא לא כן אלפן רבי שבת שחל להיות בחנוכה שאסור לראות מטבע לאור נר חנוכה הרי אינו שכח ומוציא את הפתיל׳ לית טעמא אלא שלא הוצת האור ברוב הדלק. פי׳ רב תחליפא מקשי לרב חסדא לא כך למדתני רבינו שאפי׳ שבת שחלה להיות בחנוכה אין מדליקין בפתילות ושמני׳ שמנו חכמים ואמאי אסור אם טעם איסור הפתילות ושמנים הוא מפני שהאור אינו נמשך אחריו והוא שכח ומוציא את הפתילה כדקאמר התם לעיל הרי שאחר שאסור לראות מטבע לנר חנוכה וכיון שכן לא אתי לאטויי אלמא מדאסר לראות מטבע כל תשמישין נמי אסור דאי שרי בשאר תשמישין הא אתי לאטויי. וכ״ד רבי׳ הגדול ז״ל שפסק הלכה כדברי כולם להחמיר וכ״נ מדברי ר״ח ז״ל שאין כאן בית מיחוש דודאי אסור בכל תשמישין דע״כ הלכה כרב מתנא אמר רב דר׳ נחמי׳ ור׳ יוחנן הכי ס״ל ואביי קבלה בסוף וקרא נפשי׳ כדלא זכי משום דלא גמרה מעיקרא כי היכי דלא תשתכח מיני׳. ומצאתי למקצת הגאונים ז״ל שאמרו, שאם כבתה ונשאר שמן בנר ביום ראשון מוסיף עליו ומדליקו ביום שני וכן שאר הימים ואם נשאר בה ביום אחרון עושה לה מדורה ושורפה במקומה שהרי הוקצה למצותיו. ואם קבלה היא נקבל, ונאמר דלא דמי לעצי סוכה ולנוי סוכה שמותרין אחר התג משום דהתם לא נתנם אלא למצות החג ודעתו עליהן לאחר שבעה אבל הכא דעתו הי׳ שיכלה כל שמן שבנר דלית הלכתא כר״ש דאמר אדם מצפה אימתי תכבה נרו לכך מכיון שהוא נתנו והקצהו שיכלה במצוה נאסר לו לעולם כמי שהקדישו לשמים ולא דמי לנר של שבת דלכ״ע מותר לאחר השבת דההוא אפי׳ בשעת מצותו משתמש בו שלכך בא מתחלתו הלכך אין איסור הנאה חלה עליו. ואמת הדבר שלא הי׳ נראה כן, לפי שלא נאסר בעוד שהוא דלוק אלא משום ביזוי מצוה וכיון שכבה בדין הוא שיהא מותר שכבר נשלמה מצותו ואפשר שזה דומה לסוכת החג ונויין שלה שאע״פ שנפלה בחג אסורה דהכי מוכח במס׳ ביצה בפ׳ המביא (ל:). אבל רבינו הגדול ז״ל כ׳ ״א״נ לשיעורא״ - כלומר שאם הית׳ דולקת והולכת עד השיעור הזה ורצה לכבותה או להשתמש לאורה הרשות בידו ונראה מזה שאם הותיר בה שמן שמותר אפי׳ ביום ראשון ולכבותה נמי ולהסתפק בשמן מותר שכיון שמותר להשתמש לאורה בעוד שדולקת במצותה כ״ש שאם כבתה מותר ואע״פ שי״ל ה״מ בשמן שהותיר על השיעור אבל כבתה בתוך זמנה אסור לעולם שהרי הוקצה למצוה, אינו נראה כן:
}
\textblock{הא ד\textbf{דריש רב נתן משמי׳ דרבא כסוכה וכמבוי.} משמע לי דרבה גרסי׳ בה״א דקסבר טעמא דסוכה משום דלא שלטא בי׳ עינא ומ״ה שייך בנר חנוכה דבעי׳ פרסומי ניסא ומשלט עינא אבל לרבא דאמר משום דירת קבע מאי שייטא דסוכה גבי נר חנוכה. וא״ל סימנא בעלמא נקט שלשה שיעורן עשרי׳ אמה ובנר חנוכה ודאי הטעם משום דלא שלוט בי׳ עינא דעיקר הטעם דלא אמרינן בסוכה האי טעמא לשאר אמוראי לאו משום דסברי דשלטי בי׳ עינא אלא דסברי דקרא דלמען ידעו ידיעה לדורות הוא ולא בעינן שלטא ביה עינא:
}
\newsection{דף כב}
\textblock{\textbf{וכי נר קדושה יש בו.} ואתקיף לה רב יוסף וכי דם קדושה יש בו נ״ל דה״פ דמדאסר רב אשי אפילו הרצאות מעות דליכא משום שמא יאמרו לצורכו הוא דאדלקי׳ לפום הכי אקשי וכי מצות דברים שבקדושה הן כגון תפלין וס״ת ותשמישיהן שהן אסורין בהנאת תשמיש חול וא״ל רב יוסף וכי דם קדושה יש בו שלא נהג בו במה שירצה והלא תשמישי מצוה נזרקין הן אלא אע״פ שנזרקין לאחר מצותן בשעת מצותן נוהגין בהן קדושה שלא יהי׳ מצות בזויות עליו ה״נ נוהגין בהן כדברים שבקדושה ואסורין בהן שמוש חול בשעת מצותן ולא משום דדמי בזיון לבזיון מדמי להו אלא כדפרישית. וכן הא דמדמי ריב״ל עצי סוכה משמע דקסבר שחל שם שמים על הנר הוא ומשום קדושה דא״ל ר״י היכי תלי תניא בדלא תניא אפי׳ לדבריו אלא א״ר יוסף משום בזיון הוא שבשעת מצותן אסורין הן וכיון שנאסרו אף זה שבא להסתפק מהן ולסלקן מן המצוה אסור דהא איתקצאי למצוה א״נ שאם בא לאוכלן ולסלקן מהמצוה בזוי מצוה הוא (דעדיף בין) רשות ממצוה:
}
\textblock{\textbf{זו נר מערבי.} פרש״י ז״ל במס׳ מנחות (צח:) דלמ״ד מזרח ומערב היו עומדין קרי נר מערבי האחרון שהוא סמוך למערב יותר מן הכל וכאן חזר בו ופי׳ שהוא השני של מזרח כדתניא בספרא (אמור י״ג), וכן הוא האמת. אבל ק״ל בהאי ברייתא דתניא התם, נכנס ומצא שתי נרות מזרחיים דולקות (משמן) מדשן את המזרחי ומניח את המערבי דולק שממנו מדליק את המנורה בין הערביים דאלמא אינו מדליק למזרחי ובמס׳ תמיד תנן בפ״ג (לג.) מי שזכה בדישון המנורה נכנס ומצא שני נרות מזרחיים דולקים מדשן את השאר ומניח את אלו במקמן מצאן שכבו מדשנן (ומדליקין את) [ומדליקן מן] הדולקין ואח״כ מדשן את השאר ומשמע דברייתא דספרא משבשתא הוא וסמי ברייתא מקמי מתני׳ וטעמא דמלתא לפי שאין השני שבמזרח נקרא מערבי לעולם אא״כ דולק לו נר א׳ מזרחי שהוא לזה שבצדו מערבי לפיכך הוא מדליק את המזרחי כדי שיהא השני מערבי לפני ה׳. וה״ד מדשן את השאר היינו אותן שכבו אבל אותן שלא כבו [אינו] מדשנן ומוחט את הפתילות ומניחן כמה שהן ובערב הוא נותן בהן שמן אחר כמדה ראשונה חצי לוג ופתילה אחרת ומדליקה. והיינו דאמרי׳ במסכת מנחות פרק שתי מדות נראית שכבתה נתדשן שמן ונתדשנה הפתילה כיצד עושה מטיבה ונותן בה שמן אחר ומדליקה פי׳ מטיבה בשחר ונותן בה שמן אחר ולערב מדליקה אלמא אם לא כבתה לא נחדשנה ואינו נוגע בה אלא בערב. וה״נ תניא בספרי (בהעלותך) יאירו שבעת הנרות שומע אני שיהיו דולקין לעולם ת״ל מערב ועד בוקר אי מערב ועד בוקר יכבה ת״ל לפני ה׳ תמיד הא כיצד יאירו שבעת הנרות מערב עד בוקר לפני ה׳ תמיד שיהא נר מערבי תדיר שממנו מדליקין את המנור׳ בין הערבים אלמא בבוקר אינו מכבה ומדשן אלא אותן שכבו אבל בערב מפנה ומדשן ומדליק שאם לא תאמר כן נר מערבי לא הודלק אלא פעם אחת מעולם. ונמצאת לפי מדה זו שפעמים שהוא מטיב בערב כגון אותן שלא כבו שחרית וכגון נר מערבי והיינו דתנן במס׳ יומא גבי תמיד של בין הערבים ומטיב את הנרות ולא הדלק׳ דוקא כמו שפירש״י ז״ל שם אלא לכך נקרית הטבה שאף הטבה שבה לפעמים שהוא מדשן אותן שלא כבו ונותן בהן שמן אחר ופתילה אחרת ומדליק ואע״פ שאף בשחר מדליק שתי נרות מזרחים אינה נקראת הטבה לפי שעיקר מצות הדלקה בערב שהוא מדליק את כלן וזהו שאמרה תורה ובהעלות אהרן את הנרות בין הערבים יקטירנה ומצינו בקבלה לבער בערב, אבל ראיתי לר׳ משה הספרדי ז״ל שאמר שאפי׳ בבוקר מדליקן כל אותן שכבו וכן בערב, ולא ידעתי מהו:
}
\textblock{למ״ד \textbf{הנחה עושה מצוה.} צריך להדליק ולהניח הנר דלוק במקומו ולברך באותו הנחה להניח נר של חנוכה ולפום הכי גרסי׳ לקמן ומגביהה ומדליקה וחוזר ומניחה מיבעי לי׳, ול״ג ומגביהה ומניחה וחוזר ומדליקה. ומיהו קשי׳ לי, אמאי מכבה כלל, הא אמרי׳ השתא דאמרת הדלקה עושה מצוה הדליקה חש״ו לא עשה כלום אלמא למ״ד הנחה עושה מצוה אף על פי שהדלקתה בפסול הנחה כשרה מכשרת, ה״נ אמאי מכבה במגביהה וחוזר ומניחה דיו וי״ל כיון דמאתמול הוא הרואה אומר לצרכה הוא דאדלקה אתמול דלא ידעי השתא למאי הודלקה אתמול (ובין) [ועוד דבין] שהדליקה לשם נר דשבת או לנר חנוכה כל הדלקה דמאתמול לא מתכשרא היום. ועי״ל דלמ״ד הנחה עושה מצוה צריך להדליקה כשהוא בידו ואח״כ יניחה הא אלו הונחו בפסול נפסלה אותה הדלקה לגמרי הלכך צריך לכבותה ולהגביהה ולהדליק ולעשות הנחה בהכשר, כן נ״ל:
}
\textblock{ומדקאמרי׳ \textbf{אי הנחה עושה מצוה אין מדליקין מגר לנר.} אע״ג דלצורך נר מצות חנוכה הוא עושה משמע לי דלעולם ע״י קינסא אסור ולא שרי אלא מנר לגר דהיינו מצוה ממצוה דאי ס״ד אפי׳ ע״י קינסא שרי כי הנחה עושה מצוה אמאי אין מדלירין לא יהא נר חנוכה עד שלא הגיחו פתות מקינסא אלא ש״מ ע״י קינסא אסור ונרות דגונור׳ בפתילות ארוכות מתוקמא וכדאוקימתא דרבא, אבל ראיתי למקצת המחברים האחרונים ז״ל שהתירו אפילו ע״י קינסא, ואינה:
}
\newsection{דף כג}
\textblock{\textbf{ספק דדבריהם לא בעי ברכה.} דהיינו דמאי, ה״ק: אפי׳ לא נתעשר אין הברכה שלו אלא מדבריהם וכן לגבי ברכות שבתפלה ספק בשל דבריהם הוא. ורבא אמר רוב ע״ה מעשרין הן. ונ״ל דל״פ אהאי טעמא דהא בהדיא אמרי׳ בברכות (כא.) ספק קרא ק״ש ספק לא קרא אינו חוזר וקורא ק״ש ספק אמר אמת ויציב ס׳ לא אמר אמת חוזר ואומר אמת מ״ט ק״ש דרבנן אמת ויציב דאורייתא וש״מ דספק דדבריהם לא בעי מיהדר וברוכי. אלא טעמא דרבא דכיון דחקינו רבנן וגזרו על הדמאי כמצוה של דבריהם דמיא ממש דומיא די״ט שני דהוא תקנתא דרבנן לא חיישינן לספיקא ובעי ברכה כנר׳ תנוכה אלא משום דכיון דרוב ע״ה מעשרין הן עבוד רבנן הכירא דלא לימרו כדאוריי׳ דמי להו לרבנן ונ״מ לכמה מילין. ואפשר לומרדלרבא, כל שברכה לעצמה מצוה כגון ק״ש ותפלה וברכת המזון ומגלה והלל וספק אמר ספק לא אמר בשל תורה כברכת המזון וק״ש חוזר בשל סופרים כגון תפלה למ״ו דסבר הכי א״נ הטוב והמטיב שבברכת המזון אינו חוזר ואומר אבל שאר ברכת המצות דאוריית׳ כגון הפרשת מעשרות כיון שהמציה דאוריית׳ והוא צריך לחזור ולעשות המצוה מספק בשל תורה בזה סבני רבא שאף הברכה של דבריהם צריך לעשות עמה ולזה הטעם נטילת לולב יום ראשון ותקיעת שופר וכיוצא בהן בכל ספיקן חוזר ומברך כשם שצריך לעשות המצוה מספק, וזה הלשון צ״ע ובדיקה בתלמוד, אבל רבינו כ׳ בפ׳ המיל׳ בהסכמת הנאונים בספק מילה שאין מברכין עליה, ומשם נלמוד מכאן:
}
\textblock{\textbf{נר חנוכה ונר ביתו נר ביתו עדיף.} משמע דאף להדליק אמרי׳ מקדימין נר ביתו לנר חנוכה שכל התדיר והמשובח מחברו קודם לחברו. אבל ראיתי לבעל הלכות ז״ל שאמר והיכי דקא בעי לאדלוקי נר חנוכה ונר שבת ברישא מדליק דחנוכה והדר מדליק של שבת דאי אדליק דשבת ברישא (איתסר לי׳ לאדלוקי דחנוכה משום דקבלת שבת עליה. וכמה רחוק זה הטעם, אם אמרו ביה״כ כן (אם אמר) [בשאמר] זמן, משום שמוסיפין מחול על הקודש ואוכלין ופוסקין מבע״י ויכול הוא להפסיק ולקבל עליו שלא יאכל מבע״י וכיון שאמר זמן הרי קבלו מעתה אבל הדלקת נר שבת אם הדליקו מבע״י מה קיבול שבת יש בכך הרי כדי שלא יהא טרוד בערב הוא עושה א״ב הדליק נר של שבת לא ידליק נר אחרת ואדרבה לא מפני שהוא שבת הוא מדליק אלא מפני שעדיין אינה שבת מדליק וכי גמר הדלקה קונה שביתה ועוד הרי אמרו בשילהי פרק׳ן שלישית להדליק את הנר הדליק המדליק ושוהה כדי לצלות דג קטן או כדי להדביק פת בתנור אלמא בתר הדלקת הנר מותר לצלות דגים ולהדביק פת בתנור דכולהו ודאי מדלקו וזמן דכלהו אינשי להדלקה ובתר הכי מאן דאדליק גופי׳ מדביק פת בתנור וצולה דג קטן מדלא הויא תקיעה בתריית׳ דהדלקה ש״מ דכל כה״ג לאו קבלה הוא ושרי: }
\textblock{\textbf{א״ל אביי אלא מעתה בי״ט לישתרי.} ואיכא דקשי׳ לי׳ ולימא בי״ט ביומי׳ ממש הוא דקתני אבל רישא דקתני אין מדליקין גבי שבת כיון דמע״ש הוא מדליק הוי שרי אי לאו משום גזירה דהטייה. ומפרקי רבנן ז״ל דרבה ואביי תרוייהו ס״ל שאין בתרומה משום שריפת קדשים בר׳ט שלא אמרו אלא בקדשים שאסור ליהנות מהן בשעת שריפתן דהויא לה הדלקה שלא לצורך אבל תרומה כיון שמותר ליהנות ממנה בשעת שריפה ובדרך הנאתה הוא מבערה אף בר׳ט מותר להדליק בה והא דאמרי׳ לקמן מ״ט לפי שאין שורפין קדשים בי״ט אליבא דר״ח איתמר דהא תניא כוותי׳ ואע״ג דאיירי עלה אביי ואמר איהו בה טעמא אאין שורפין אמר אבל צאו משום דאיהו סבר (לההוא) [לא דהוא] טעמא דמתני׳. וליכא למימר דאביי לא סבר לה כרבה [ולרבה פריך] דאי הוה סבר אביי שיש בשריפת שמן שריפה ביו״ט משום שריפת קדשים מאי הוה קשי׳ לי׳ עלי׳ דרבה אלא מעתה בי״ט לישתרי הא לא שמיעי לי׳ מיגר דרבה דלית בר משום שריפת קדשים דאיהו לא אמר טעמרן אלא גבי שבת מפני שהוא מדליק מבערב ואין זה שורף בשבת כלל כלום ובודאי אפשר למימר דאביי סבר שאלו הר בו משום שריפת קדשים אף (מעי״ט) [מבע״י] אסור כיון שעיקר השריפה בשבת ובירושל׳ נמי איכא הכי ולא דמיא להדלק׳ נר דחולין דהכא הבערה גופי׳ חשיבא לר מלאכה ולפיכך תאמר דאביי אליבא דרבה אקשי ולאו משום דסבר לי׳ כותי׳ אלא מיהו כיון דחזינא לד לר״ת דאית לי׳ משום שריפת קדשים ואפ״ה מבע״י שרי והוא צריך לדחוקי נפשר ולאוקמי למתני׳ בי״ט שחל להיות בע״ש א״א לומר דפליג אביי בהא דלפום סוגיין משמע דמלתא ברירא הוא גבייהו לכ״ע דמבע״י לית בה משום שריפת קדשים אע״פ שהוא הולך ודולק בי״ט ובשבת ובירושלמי פליגי אגמ׳ דילן, והתם נמי פליגי בהך סברא. ומסתברא לי, דאפילו לרב חסדא דאמר משום שריפת קדשים נזירה דרבנן בעלמא הוא דומיא דשארא אבל מדאורייתא שרי לכ״ע ודאמרי׳ לקמן מה״מ אאין שורפין קדשים אבל שמן שריפה דמתני׳ ודאי מדרבנן הוא ולא לחלוק בשריפת קדשים דחזינן לכולהו קראי דמייתי לקמן דליכא לאוכוחי מינייהו אלא קדשים שאין אדם נהנה מהן וכ״ש לרב אשי דאמר שבתון עשה הוא ואין עשה דוחה ל״ת ועשה דהא לאו דיחוי הוא ועיקרא דהך שמעתא איתמר בפ׳ כיצד צולין אשריפת קדשים דעלמא דתנן העצמות והגידין והנותר ישרפו בששה עשר והוינן בה וניתי עשה ונדחי ל״ת אמר חזקי׳ וכו׳ שמעתין התם עיקר והכא משום גררא דהתם, (וגזרי׳) [וגזירה] בעלמא מדרבנן. ורש״י ז״ל כ׳ עלה דההיא, ואע״ג דהדלקת נר בי״ט מלאכה המותרת הוא הואיל דהבערת שאר קדשים שאין נהנין נאסרה גם זו לא יצאת מן הכלל. ואיני יודע לטעם זה טעם שאלו הי׳ הכתוב מפרש שאין שורפין קדשים הי׳ אפשר בדרך דילמא למימ׳ שהכל בכלל אבל אם הכתוב מזכיר שלא לחלל י״ט במלאכה שא״צ האיך יהא בכלנה מלאכה הצריכה אלא מדרבנן וכדאמרן. אבל בפ׳ אין צדין (ביצה כז:) מצאתי שכ׳ רש״י ז״ל אע״ג דהדלקת הנר בי״ט צורך אכילה הוא ומותר ואפ״ה בשמן שריפה לא מגזירת הכתוב שאין קדשים טמאין מתבערין בי״כי דרחמנא אתשבי׳ להבערתן דכתיב באש ישרף הלכך מלאכה הוא פי׳ לפירושי משום דעיקר כוונתו של אדם בשריפה זי למצוה של גבוה וצורך הדיוט ממילא הוא דאתי ובתוס׳ מייתי לה מנדרים ונדבות דאיכא מ״ד אין קרבין בי״ט משום דכתיב לכם ולא לגבוה והתם צורך הדיוט נמי הוא דאכלי כהנים ובעלים חלקן וכיון דעיקר המלאכה לשם המצוה ומינה הוא דמתהני הדיוט ולסור ומאן דשרי טעמא מפ׳ התם שלא יהא שולחנך מלא ושלחן רבך ריקם ואע״פ שיש צדון אחר רא״ זו אפשר שהוא כן כדברי רש״י ז״ל. והא דאמר רבה }
\textblock{\textbf{גזירה י״ט אטו שבת.} אי קשיא מ״ש נוכל פתילות ושמנים דלא גזרינן י״ט אטו שבת א״ל דהתם לא שייכא גזירתן בי״ט כלל אבל שריפת קדשים בי״ט גמי שייכא שמא יבא להבעיר שלא לצורך וכיון שבעיקר האיסור י״ט ושבת שוין אסרו זה עם זה: }
\textblock{\textbf{מנה״מ.} ואי קשיא, מהיכי תיתי דשרי דבעית קרא למיסר א״ל משום דסד״א י״ט ל״ת הוא ואתי עשה ודחי ל״ת מדקא מתרץ רב אשי בהדיא משוס די״ט עשה ול״ת ולא מידחי אלמא מעיקרא לא ס״ד הכי אלא ל״ת בלתוו הוא וסברי׳ דאתי עשה ודחי ל״ת. ואי קשי׳, אי הכי רבא לימא הכי אמאי אצטריך לי׳ לבדו ולא מילה דהא איהו ס״ל די״ט עשה ול״ת הוא כדמקשינן פ״ק דביצה ובפ׳ כיסוי הדם דאמרי׳ אתי עשה ודחי ל״ת מכדי י״ט עשה ול״ת הוו ואין עשה דוחה ל״ת ועשה אלא (אר״פ) [אמר רבא] אפר כירה דעתו (עליו) [לודאי] וא״ל דרבא התם לאו משום האי קושי׳ הדר בי׳ ואפשר דלדידי׳ ה״נ דאתי עשה דכיסוי ודתי ל״ת די״ט ולא ממעכי מלבדו אלא מילה שלא בזמנו דמקיים עשה דידה למחר אבל כיסוי כיון ששחט דחי י״ט (לרבה) [לרבא] וי״ל דלא קפיד רבא בהא וכיסוי נמי לא דתי מלבדו ומיהו לא מהך קושי׳ מתרץ לה רבא אלא או משום דמסתבר לי׳ טעמא בתרא או משום דגמ׳ הוא דמקשי׳ עלי׳ הך קושי׳ דמכדי י״ט עשה ול״ת הוא ולא רבא. ומפרשי׳ בתוס׳ דשריפת תרומה לאו עשה דאורייתא הוא דלא אשכחן אלא ואסמכוה רבנן לשריפת קדשים (שיטמא) דז וכן משמע מדברי ר״ש מצוה לשרוף תרומה שנטמאת משום דדמיא לקדש ועוד משום תקלה. והכא ה״פ דבעי׳ דמתני׳ ומפרשי׳ לפי שאין שורפין קדשים בי״ט וכיון שעשו חכמים שריפת תרומה כשריפת קדשים למצוה אף לאיסור עשאוהו כמותו. ולפי שפי׳ למעלה דגזירת חכמים הוא בשמן שריפה משום שמא יבא לשרוף שלא לצורך הכא הומ״ל משום הך נזירה ואפי׳ היה שורפין קדשים בי״ט התם שרי משום דעשה הוא אבל בתרומה דליכא עשה שלא לצורך אסור מן התורה ולצורך איכא נזירה אלא כיון דהכי הוא דאין שורפין קדשים לעולם נסוב לה טעמא הבי וכי בער מה״מ אשריפת קדשים ממש בעי׳ ואפשר שאף שריפת תרומה מצות עשה, של תורה ובכלל בקודש באש תשרף הוא וכ״נ.בירושל׳ דהוא עשה דאורייתא. ואיכא דקשיא לי׳ ותיהני נמי י״ט ל״ת ושריפת קדשים עשה היכי דחי האי עשה להאי ל״ת הרי אפשר לעשה זה להתקיים למחר ואר״ש ב״ל ב״מ שאתה מוצא עשה ול״ת אם אתה יכול לקיים שניהן מוטב ואם לאו יבא עשה וידחי לל״ת וכיון שכן מ״ט צריכי להנך קראי ורב אשי נמי ל״ל למימר שבתון עשה הוא ואם באנו לפרש דהכא בקדשים שא״א לשהות כגון איברים ופדרים שניתותרו מעי״ט וכעין פירוקא דאביי שאע״פ שאפשר שמעלן ומלינן בראשו של מזבח כיון דאם ירדו פסולין מ״ט הוא לשורפן היום ולא יפסלו בלינתן ולא יספיק לנו פרקונו זה לכיעמא דחזקיה ורבא דמהתם להכא לא משמע אלא שאין שריפת קדשים שנפםלו שאפשר לשהות דותה יו״ט הלכך ק״ל ל״ל קרא. ולדידי הא ל״ק לי, משום דלא אמר רשב״ל אפשר לקיים את שניהן אלא כנון ההוא דאקשי׳ במנחות בפ׳ התכלת גבי סדין בציצית שאפשר לו לעשות לבן ממינו ולא עבדי׳ לה צמר בכיוצא בזה נאמרה אבל מ״ט שאין לקיימה היום כיון דחביבה מצוה בשעתה וכמו דקאמר הש״ס במנחות אר״ש בא וראה במה חביב׳ מצוה בשעתה שהרי אברים ופדרין כשרין בל הלילה יע״כ אי אפשר לקיים את שתיהן מיקרי אע״פ שאפשר למחר שכל שמחוסר זמן כמחוסר הכל דמי והרי עכשו א״א לקיים שניהן. ולהאי פי׳, הא דתנן אין חותכין לא בדבר שהוא משום שבות ולא בדבר שהוא משום ל״ת ומפ׳ טעמא בגמ׳ משום די״ט עשה ול״ת ולא אתי עשה דשופר ודחי לי׳ ההוא סוגי׳ דבתראי הוא ואליבא דרב אתמר ולאו דצריכין לההוא טעמא דהא מסקי׳ לקמן בפ׳ מילה דכי אמרי׳ אתי עשה ודחי ל״ת כגון מילה בצרעת וכו׳ הא מכשירי מילה ושופר בעידנ׳ דעבר ללאו לא מקיים עשה ולא כלום וטעמ׳ דשיטפ׳ הוא לומר דהכי הוא ודאי כדרב אשי:
}
\textblock{הא דאמרי׳ \textbf{בא הכתוב ליתן לו בקר שני לשריפתו.} ודאי לאו למימרא דבליל מי״ט אין שורף עד בקר שני דהא אפר ביום ראשון נשרף אלא משום שאין שורפין קדשים בי״ט ועוד הו״ל למכתב בבקר באש תשרופו אלא ה״ק לא תותירו ממנו עד בקר ראשון והנותר ממנו תשרפו אותו בלילה עד בקר שגי קודם שתלכו מירושלים כדכתיב ופנית בבקר והלכת לאהליך דהיינו בוקרו של שגי ומ״מ קרא יתירה הוא לומר שאין שורפין קדשים בי״ט, כן נ״ל. ורש״י כתב, והנותר ממנו לראשון עד בקר שני המתינהו ותשרפוהו ולא נהיר׳ דא״כ משמע דלא ישרף בלילה ומנ״ל דמשום דאין שורפין קדשים בי״ט הוא הרי אפי׳ במי״ט אינן נשרפין עד למתר דאלמא גזירת הכתוב הוא לשורפן בבקר שני ואין שריפתן סמוך לאיסורץ אלא משמע כדפרישי׳. ושוב מצאתי בירושל׳ (ב.) כדברי רש״י דגרסי׳ התם התורה אמרה אין שורפין קדשים בי״ט ואצ״ל בשבת מה תמית מימר כן לא תותירו ממנו עד בקר והנותר ממנו עד בקר באש תשרופו אחר שני בקרים אחר בקרו של ט״ו ואחר בקרו של נז׳ין וכתיב והנותר מבשר הזבת ביום השלישי באנדישרף אחר שני בקרים אחר בקרו של ט״ו ואחר בקרו של ט״ז, וזה כלשון רש״י. ומה שהוקשה (לו) [לי] מן שריפת הלילה, כך הוא ודאי, ששם אמרו בפר מעתה אין מדליקין בשמן שריפה בלילה שאין שורפין קדשים בלילה ומתרץ אר״י (ילדו) לה בשיטת ר׳ ישמעאל דר׳ ישמעאל אמר תינוק שעבר זמנו נימול בין ביום ובין בלילה כלומר אף שריפת קדשים בזמנן אין נשרפין בלילה לאחר זמנן אפר בלילה ומקשינץ מה אית לך למימר שמן שריפה שעבר זמנה (נשרפת בין ביום בין בלילה) אר״ש בן פזי מכיון שנטמא כמי שעבר זמנו למדנו בפי׳ שאין הקדשים נשרפין עד בקרו של ט״ז בט״ו מפני שהוא י״ט ובליל ט״ז שאין שורפין קדשים בתחלה בלילה:
}
\textblock{\textbf{אביי אמר אמר קרא עולת שבת בשבתו ולא עולת חול בשבת ולא עולת חול בי״ט.} איכא למידק עולת חול בי״ט מנ״ל הא מקרא לית לי׳ אלא עולת חול בשבת והיכי גמר י״ט משבת וא״ל תרי מיעוטא כתיבא וה״ל למיכתב על עולת התמיד כדכתיב בר״ח ודרשי׳ עולת שבת ולא עולת חול בשבת ודרשי׳ בשבתו לי״ט, זה הנ״ל. ומצאתי לרבותי׳ הצרפתים ז״ל בתוס׳ שמעמיקין בה לומר אביי סבר לה כר״ע דאמר בשילהי ואלו קשרים עולת שבת לימד על חלבי שבת שקרבין בי״ט יכול אף ביה״כ ת״ל בשבתו דאלמא עולת שבת קריבה בי״ט ולא עולת חול בי״ט ואצ״ל לעולת חול שאינה קריבה בשבת ואדרשה דר״ע סמך אביי ועיקר טעמא דר״ע מפורש התם לדברי ר״ע נדרים ונדבות אין קרבין בי״ט וכי אצטריך קרא למישרי י״ט. ואי קשי׳, כיון דעיקר טעמא משום דנדרים ונדבות אין קרבין בי״ט וההוא נפקא לן מלכם ולא לגבוה כדמשמע בפסחים ובמס׳ ביצה לימא אביי הכא לכם ולא לגבוה אלמא אין שורפין קדשים בי״ט א״ל קמ״ל רבותא דאפי׳ עולת חול שא״ר לשהויי אינה קריבה בי״ט דאי מלכם ולא לגבוה ה״א ה״מ בנדרים ונדבות שלא נשחט עדיין לפי שלא הותחלו מצותן כלל. ואי קשי׳, היכי קאמר ר״ע כיון דנדרים ונדבות אינן קרבין בי״ט כי אצטריך קרא למישרי י״ט דילמא אע״ג דקיי״ל נדרים ונדבות אינן קרבים בי״ט דנפקא לן מלכם ולא לגבוה דה״מ בנדרים ונדבות שאפשר להשהותן אבל אברים שא״א להשהותן קרבין הלכך כי אצטריך קרא ליה״כ א״ל ס״ל לר״ע דכיון דכ׳ רחמנא לכם ולא לגבוה לומר שנדרים ונדבות אין קרבין בי״ט אע״פ שאפשר לנו לומר שלא אסר הכתוב אלא באותן שאפשר נשהותן כיון שי״ל נמי שהכל בכלל מצרך הוה צריך קרא אחרינא למשרנהו לנדרים ונדבות שאי אפשר לשהותן שיקרבו בי״ט ולא הוה קרא רהיט ושרי יה״כ הלכך לי״ט אצטריך קראי וממילא למדנו שאפי׳ נדרים ונדבות שא״א לשהותן אינן קרבין בי״ט והיינו דאביי דאייתי האי קרא כדכתיבנ׳. ואי קשי׳ אדר״ע דאמרן דאית לי׳ לכם ולא לגבוה, והא אמרי׳ במס׳ ביצה בפ׳ יו״ט (ביצה כא.) רע״א אפי׳ נפש בהמה במשמע א״כ מת״ל לכם לכם ולא לגויים לימא נמי הכי ולא לגבוה א״ל חדא מתרתי נקיט ללמדך שאין נפש גוי במשמע אע״ג שנפש בהמה משמע. ואפשר לפרש דהא דאמר ר״ע נדרים ונדבות אין קרבין בי״ט ה״ק אינו בדין שיקרבו הואיל שלא מצינו שהתירן הכתוב שי״ט כשבת חוץ מאוכל נפש וקרא לי״ט אצטריך וממילא למדנו איסורן וכן נמי ר׳ ישמעאל דאמר נדרים ונדבות קרבין ה״נ קאמר דין הוא שיקרבו שלא יהא שולחנך מלא ושולחן רבך ריקם ול״צ קרא אלא למישרי יה״כ:
}
\textblock{\textbf{ולא מילה שלא בזמנה דאתי׳ בק״ו.} איכא דל״ג הכא דאתיא בק״ו, מפני הטעם שאמרתי דלרבא י״ט ל״ת הוא בלחוד ואתי עשה דמילה ודחי לי׳ בלא ק״ו ועוד דאי מילה שלא בזמנה בק״ו אתי׳ לא מימעטא מלבדו דהא אצטרך לבדו לאין שורפין קדשים בי״ט ועוד שבפ׳ ר״א דמילה אמרי׳ מה״מ דמילה אינה דוחה י״ט אלא בזמנה ואמר רב אשי נמי התם שבתון עשה הוא וכו׳ ואי בק״ו אתי׳ אע״ג דבעלמ׳ לא אתי עשה ודחי ל״ת ועשה הכא דחי דהא איכא ק״ו, אבל מה אעשה שהוא כ׳ בעיקר הנוסחאות. ורש״י ז״ל פי׳ ק״ו זה שהוא האמור בפ׳ ר״א דמילה ומה צרעת שדוחה את העבודה ועבודה דוחה שבת מילה שלא בזמנה דוחה אותה שבת שנדחית מפני עבודה א״ד שתהא מילה דוחה אותה ומקשו עלה דהא רבא לית לי׳ האי ק״ו בפ׳ ר״א כדא״ל לרב ספרא לאו משום חומרא דצרעת הוא אלא משום דגברא לא חזי וכו׳. ושמעתי שהר״נ [גאון] ז״ל פי׳ במגלת סתרים ומה (עשית) [עשיית] באוכל נפש שהוא רשות כגון השחיט׳ והבישול דוחה י״ט, מילה שלא בזמנה שהוא מצוה לא כ״ש שתדחה י״ט. ואחרים פי׳ כך, ומה צרעת שאינה נדחית מפני צורך הדיוט נדחית מפני מילה שלא בזמנה י״ט שנדחה מפני צורך הדיוט כגון שחיטה ובישול א״ד שיהא נדחה מפני מילה שלא בזמנה. ואלו ב׳ הפירושים תשובה א׳ להן, שאם שחיטה ובישול דוחין י״ט מפני שהוא צרכו של יום ושמחתו בכך. וא״א לשהות תדחנו מילה שלא בזמנה שאין ליום עסק בה ואפשר לשהות עוד למחר. אלא עיקר ק״ו זה, דרבא סבר נדרים ונדבות שלא נכרתו עליהן י״ג בריתות דוחה י״ט מילה שנכרתו עלי׳ י״ג ברייתות א״ד שתדחה יו״ט כך פי׳ רבותי׳ הצרפתים ז״ל ולפ״ז אפשר לקיים גי׳ הספרים ולומר דרבא סבר די״ט עשה ול״ת הוא ואפ״ה אצטריך קרא למילה שלא בזמנה משום דאתי׳ בק״ו ורב אשי סבר נדרים ונדבות אין קריבין בי״ט הלכך ק״ו ליכא וא״ת אי ס״ל לרבא די״ט עשה ול״ת הוא הכא למאי אצטריך לטעמ׳ אתרינ׳ א״ל דהך מימר׳ התם אתמר בפ׳ ר״א דמילה ולמילה שלא בזמנ׳ אצטריכ׳ כדאי׳ התם וגמ׳ הוח דמסדר לה הכא לומר דשריפת קדשים ודאי לא דחי אי מלבדו אי משום עשה ול״ת וסוגי׳ דמס׳ י״ט כפשטא אתי׳ הכי:
}
\newsection{דף כה}
\textblock{\textbf{שכן פנ״ק עכ״ס.} האי דלא חשיב כפרה כלומר שמחוסרי כפרה אוכלין בתרומ׳ משהעריבו שמשן ואין אוכלין בקדשים, משום דמה״ט דפנ״ק עכ״ס הוא דקמה לי׳ לתנא בדוכתי׳ כדאיתא בפ׳ הערל (יבמות עד:) והתם נמי לא חשיב איסור הנאה בשעת ביעור:
}
\textblock{\textbf{מ״ח פ״ז.} פירש״י ז״ל כהן שנטמא, ולא נהיר דהא איכא בקדש כרת והוא חמיר אלא לזר האוכלה קאמרי׳ ודקאמ׳ ודאסור׳ לזרים לומר דגבי קדש אפי׳ איסור אין בה:
}
\textblock{\textbf{מתוך שריחו רע, גזירה שמא יניחנו ויצא.} איכא דק״ל, בנפט נמי לימא ר״י הכי שהרי ריחו רע כמו שפרש״י ז״ל מדאמרי׳ ביומא (לט.) בא למדוד לו נפט אומר לו מדוד לעצמך והך קושי׳ מפרקא לי׳ ממאי דאשכחית בירושל׳ גבי קרבי דגים דמקשי מה בין עטרן מה בין קרבי דגים קרבי דגים כ״ז שהוא דולק אין ריחו רע כבה ריחו רע עטרן בין כבה בין לא כבה ריחו רע [שלא תאמר] הואיל וריחו רע יהא טעון הרחק ד׳ אמות, לפום כן צ״ל אין מדליקין אף אנו נאמר כן בנפט שאינו מסריח אלא כשאדם מנענע דהא תלמוד החנוני יש לו בחנותו נפט ואינו מקפיד וכשבא (למחר) [למדוד] א״ל מדוד לעצמך:
}
\textblock{\textbf{שהיו מחבאין ממנו כנפי כסותן.} לפי שלא הי׳ להם בם אלא לבן ולא תכלת, מתייראין שלא יחשדום במיחוי מצות שמתוך שהתכלת דמיו יקרים אין לוקחין אותן ואמר להם יודע אני שאתם חוששין לדברי ב״ש ולפיכך אין אתם מטילין בם תכלת שלא יהא אסורה שלא במקום מצוה לא כך שניתי לכם שהלכה כדברי ב״ה שחייב בציצית מן התורה ודרשי׳ סמוכין למישרי בהו כלאים ולמה אתם חוששין לדברי ב״ש כלל והרי מן התורה הוא חייב ומותר ואינהו סבור אע״ג דאין הלכה כדברי ב״ש דפטרי לי׳ מדאורייתא ולא דרשי סמוכין בהא מיהו איכא למיחש ולמגזר משום כסות לילה וכמו שעשו אנשי ירושלים שכל המטיל לו תכלת בסדינו אינו אלא מן המתמיהים. ולישנ׳ דגמ׳ דייקי כדאמרן, ולא כדברי מי שאומר שהלכה כדברי ב״ש וטעמא דב״ש משום כסות לילה (ידע) שאין זה טעמן של ב״ש אלא גזירה חדשה היא שחששו הם וטעמא דב״ש משום דלא דרש הכא סמוכין (ופטרי) [ולא פטרי] לה נמי מלבן כדפרישית. ואי קשיא, הא במשנה תורה כ״ע דרשי סמוכין א״ל טעמ׳ דב״ש נמי משום דדרשי סמוכין להזהיר על הכלאים בציצית שלא תאמר יבא עשה וידחה ל״ת. וי״מ דב״ש ממעטי להו מכסותך, דתניא בספרי (דברים רלד) ״כסותך״ פרט לסדין והם מעמידין אותה כדברי ב״ש ואין זה מחוור לי דהכי מתניא התם כסותך פרט לטיגא לבורסין ולברדסין עד כסותך פרט לסגוס. ונ״ל שכל אלו כלי מטה הם ומשום כסות לילה ממעט להו אצטריך שלא תאמר פעמים שישן בהם ביום קמ״ל כיון שהם מיוחדין ללילה פטורין, וכדברי ב״ה היא סתמא. וכבר פירשתי השמועה יפה במס׳ מנחות לסייע לדברי רבינו הגדול ז״ל שחייב בלבן ולא גזרו אלא על התכלת, וכ״ד רש״י ז״ל וכל הפורש ממנו כפורש מחיים, ויש לרבים אחרים דברים מרבים הבל:
}
\newsection{דף כו}
\textblock{\textbf{סומכוס היינו ת״ק, א״ב דרב ברונא, ולא מסיימי.} פרש״י ז״ל, א״ב דרב ברונא דאמר צריך ליתן לתוכו שמן כל שהוא א״נ דאפשר חלב מהותך בתערובת כל שהוא חד מהנך סבר לרב ברונא דחלב ניתר בתערובות אבל שמן דגים אפי׳ בעיני׳ וחד סבר שמן דגים ע״י תערובות אבל חלב כלל לא. וזה הפי׳ איננו נכון, דא״כ הני תרי תנאי תרוייהו לית להו דרב ברונא לגמרי אלא מ״ס לה כוותי׳ בחדא ופליג עליה בחדא ומ״ס לה כוותי׳ בחדא ופליג עלי׳ בחדא. אלא עיקר פי׳, דשמן דגים לאו היינו קרבי דגים וחד מהנך תנאי סבר מדליקין בשמן דגים בעיני׳ אבל בקרבי דגים לא אלא ע״י תערובות ואית לי׳ דרב ברונא וחד סבר בשמן דגים ע״י תערובות הא בקרבי אפי׳ ע״י תערובו׳ לא ולית לי׳ דרב ברונא א״נ תרוייהו בשמן דגים בעיני׳ שרי אלא ח״א שמן דגים בעיני׳ ולא קרבי דגים לעולם וח״א שמן דגים בעיני׳ אבל קרבי דגים צריך תערובות. ומצאתי בפי׳ ר״ח ז״ל שכ׳ כן ואסיקנ׳ לקמן דשאני הני משמן דגים לפי ששמן דגים מזוקק הוא ומדליק הוא ונמשך אחר הפתילה וא״צ תערובות אבל מהותך וקרבי דגים אינן כשמן מזוקק וצריך תערובות:
}
\textblock{והא (דתנן) [דתניא] \textbf{כל היוצא מן העץ אין בו משום ג׳ על ג׳.} משום דבעי למימר חוץ מפשתן נקט יוצא מן העץ דהא למעוטי כל שאני בגדים אתא ולרבא דוקא נקט ג׳ על ג׳ אבל לאביי ה״ק כל היוצא מן העץ אינו מטמא חוץ מפשתן ונקט לי׳ ג׳ על ג׳ שהוא שיעור של פשתן ואגב אורחי׳ קמ״ל בג׳ על ג׳ וכדבעי׳ בסמוך דכ״ע מיהת. והא דקתני ומסככין בו, ה״ק: שכל היוצא מן העץ טהור ומסככין בו חוץ מפשתן דטמא ואין מסככין בו ופרש״י דכיון דלא קתני חוץ מבגד של פשתן משמע כל דטמא בפשתן אין מסככין דהיינו שתי וערב של פשתן ולקמן (שבת כז:) אפרש:
}
\textblock{\textbf{(ג׳ על ג׳ דתנן) [דחזו] בין לעניים בין לעשירים אתי בק״ו.} פי׳ לאו דוקא בק״ו אלא אתי בתורת טעמא שדינו כדין בגד שלם וכ״מ שאמר הכתוב בגד אף (ג׳ על ג׳) במשמע כיון דחזו לעניים ולעשירים תדע דקאמרינן חד למעטינהו משלש על שלש וחד נמעטינהו משלש׳ על שלשה ואכתי כיון דכולהו צריכי מיעוטי מנ״ל דבגד שלם אינו מטמ׳ דקאמר אף כל בגד ופשתן והוה צריך תלתא חד למעוטי בגד שלם וחד למעוטי שלשה על שלשה וחד למעוטי שלש על שלש אלא ש״מ שלשה על שלש׳ היינו בגד שלם. ועוד דקאמ׳ אביי דהאי או בגד מבעי לי׳ לרבות שלש על שלש בצו״פ דמטמא בשרצים ואמאי נימא כי אצטריך לשלשה על שלשה דהאי לא אתי בק״ו דשתי וערב בשרצים לא מטמא וקאמר רבא נמי אי ס״ד נגעים חמירי לכתוב רחמנא בשרצים וליתו נגעים ולגמרו מיני׳ ואמאי אי כ׳ רחמנא בשרצים ה״א ריבויא לשלשה על שלשלה דלא אתו בשרצים בק״ו כדאמרן ועיקר בגד לבגד שלם וכן נמי כי הדר בי׳ רבא לקמן כי היכי דלא (לימא) דתנא דבי ר׳ ישמעאל מאידך תנא דבי ר׳ ישמעאל הול״ל דשלשה על שלשה בשאר בגדים לרשב״א אית לי׳ לתנא דבי ר״י לית לי׳ ואו בגד לשאר בגדים שלמים אתא ומנ״ל לרבוי מיניה ג׳ על ג׳ אלא ש״מ מכיון דאמרי׳ דחזו בין לעניים ובין לעשירים לא בעי ק״ו דבכלל בגד הוא ואין חילוק ביניהם לבגד שלם שהרי לא נתנה שיעור לבגד אלא כל שראוי לעניים ולעשירים בגד האמור בתורה הוא:
}
\textblock{\textbf{ואימא לרבות (ג׳ על ג׳) בשאר בגדים.} פירש״י ז״ל ושלש על שלש בצמר ופשתים מיטמו בק״ו והאי ריבוי (לג׳ על ג׳) אתא ולשאר בגדים, והזקיקו לרש״י ז״ל לפרש כן משום דאמרי׳ בסמוך ואימא כי אמעטו משלש על שלש אבל שלשה על שלשה מיטמו והוא ז״ל סובר דשלשה על שלשה מריבוי דוהבגד קאמר דאיטמו וכיון שכן שלש על שלש מנ״ל דקא מצרכת לי׳ מיעוטא בשאר בגדים אלא ש״מ דאתי׳ בק״ו. וזה פי׳ של תימא שהרי עדיין בין שיניו הוא מה שאמר שלש על שלש לא אתי בק״ו והיאך חזר בו בשתיקה ועוד דא״כ נמי מקשה ידע שיש חילוק בין צמר ופשתים לשאר בגדים דבצמר ופשתים שלש על שלש טמאין דאתי בקל וחומר ובשאר בגדים שלשה על שלשה ומה חילוק בין קושי׳ ראשונה לשני׳ ומה תירץ לו. בשאמר א״ק צמר ופשתים הרי לא בא המקשה לשוותן ולעקור המיעוט שמיעטן התורה ויצטרך רש״י ז״ל לפי פי׳ לומר דהאי מתרץ טעמא וא״ל איהו הכי קאמינא אימא כי אימעטי משלש על שלש דאתי בק״ו אבל משלשה על שלשה לא וזה דוחק ומ״מ אי אפשר כדאמרן. לפיכך נ״ל דמעיקרא קס״ד דליכא מיעוטא בקרא למעט שאר בגדים אלא שאין מפורשין בתורה אלא צמר ופשתים ולפ״ז הקשו ואימא ריבויא לרבות שלשה על שלשה בשאר בגדים ולא לרבות (שלשה על שלשה) בצמר ופשתים ויהיו שאר בגדים וצמר ופשתים שוין ואעפ״י שאמרה תורה צו״פ הרי עשתה תורה שאר בגדים כצו״פ מריבוי הכתוב נמצא צו״פ מן המקרא ושאר בגדים מריבוי כדאשכחן בכמה דוכתי. וא״ת מסתברא קאי אצו״פ מרבה צו״פ לא היא דאדרבה מסתברא (קאי) דמרבה שלשה על שלשה דחזי בין לעניים בין לעשירים ומפרקי א״א לרבות שאר בגדים להשוותן לצו״פ שהרי מיעטן הכתוב בפי׳ וע״כ לא בא הכתוב אלא למיעוט דאל״ה לא לכתוב רחמנא לא צו״פ ולא ריבוי והבגד והיו כולן באין:
}
\textblock{והדר אקשי׳ \textbf{ואימא כי אימעוט משלש על שלש.} שנתרבו בצו״פ מוהבגד, אבל שלשה על שלשה מיטמו ול״צ ריבוי (בין שהיו) ראוין בין לעניים בין לעשירים שכיון שנתרבה שלש על שלש בצו״פ אע״ג שכ׳ מיעוט לשאר בגדים אין אנו מעמידין אותו המיעוט אלא למעט המיעוט שבצו״פ שהוא שלש על שלש אבל שלשה על שלשה דין הוא שלא נמעט אותו שבא מן הדין שהרי ראוי בין לעניים בין לעשירי׳ וא״צ פסוק לרבות מאחר שאנו מקיימין המקרא שאמר צו״פ וחלקנו בין צו״פ לשאר בגדים וכבר פירשנו דכי אמרינן שלשה על שלשה היינו בגד שלם של שאר בגדים שאין חילוק בין בגד שלם לשלשה על שלשה, ואף לדברי רש״י ז״ל צריך אתה לומר כן:
}
\newsection{דף כז}
\textblock{\textbf{נפקא לי׳ מאו בגד.} פי׳ דכתיב בשרצים או ריבוי הוא ואע״ג דאו בגד דנגעים הוה מיעוטא משום דבמיעוטי כתיבי נמצא לרבא אליבא דרשב״א לגבי (בגדים) נתמעטו שאר בגדים לגמרי דתרי מיעוט כתיבי בהו ולגבי שרצים צו״פ טמאין אפי׳ ג׳ על ג׳ ושאר בגדים לא מיטמו דגלי רחמנא בנגעים אבל שלשה על שלשה בשאר בגדים נתרבה בשרצים מאו בגד ומסקנא דהיינו נמי סבריה דרבי ישמעאל בשתי הברייתות. ומיהו לאביי דסבר שרצים מנגעים לא גמרי׳ לשלש על שלש אידך תנא דבי ר״י דמפיק מאו בגד לשאר בגדים שלש על שלש בטומאת שרצים לית לי׳ כלל ואפשר לומר דאפי׳ אביי לא קאמר הכי אלא לדעתי׳ דר״ש אבל מודה הוא דלאידך תנא דר״י גמר שרצים מנגעים ולא פריך מדקא מוקי או בגד לשאר בגדים ולא איצטרך לג׳ על ג׳ דצו״פ ורבא נמי דקא גמר צו״פ שרצים מנגעים לר״ש קאמר אבל לר׳ ישמעאל דלית לי׳ שלשה על שלשה בשאר בגדים או בגד בשרצים מיבעי לי׳ כאביי לרבות שלש על שלש בצו״פ ולא פליגי אביי ורבא אלא אליבא דרשב״א לחוד הלכך א״ל כדאמרן דאביי לאידך תנא דבי ר״י אית לי׳ ג׳ על ג׳ בשרצים תדע דלא משתמיט תנא בשום דוכתא לטהורי שלש על שלש דצו״פ בשרצים ומת ומיהו מסתברא ודאי דדוקא בצו״פ אבל שלש על שלש בשאר בגדים לית לי׳ ולא פליג אדרבא בהך סברא דמוקי מלתא דרשב״א בשלש דשרצים:
}
\textblock{כתוב בכל הנוסחאות \textbf{דתניא בגד אין לי אלא בגד שלשה על שלשה וכו׳.} וכתב רש״י ז״ל דלא גרסי׳ דתניא, דא״ה ק׳ ברייתא לאביי ולדידן ל״ק דאביי מוקי לה לברייתא כאידך תנא דבי ר״י דמרבי שאר בגדים לקמן בשמעתין מאו בגד והך ברייתא היינו ממש ההוא דלקמן דקתני מנין לרבות צמר גמלים והאי דקתני שלשה על שלשה משום דשלש על שלש לית לי׳ בשאר בגדים גבי שרצים כלל וכן עיקר א״נ קמ״ל דאפי׳ לשלשה על שלשה דהיינו בגד שלם צריך ריבויי בשאר בגדים. וי״א דבעיקר ברייתא לא תניא אלא הכי אין לי אלא בגד (שלם) צו״פ שאר בגדים מנין וכו׳ ואתיא לאביי כאידך תנא דר״י דמרבי שאר בגדים מאו בגד לגמרי ומטמא אפי׳ ג׳ על ג׳ דסבר [לא] ילפינן מנגעים ומפיק מאידך תנא דר״י קמא לגמרי וסבר דאתא או בגד לגלוי על בגדים האמורין בתורה סתם שאינן צו״פ ואינן למדים מנגעים בזה אבל למדין מהן דשלש על שלש כבגד שלם הוא לכל טומאה אבל רבא מגרי בה ומתרץ לה הכי שלשה על שלשה בשאר בגדים מנין ואיכא בתלמודא טובא כה״ג דאמוראי מוספי בה לפרושי מלתא. ובמילתא דאמוראי נמי בשמעתין כ׳ רש״י ז״ל אחרת, בהא דאמרינן ר״פ אמר אף כל לאתויי כלאים ופירש״י ז״ל דלא איתמר פירוקא כלישנא דאמרן אליבא דרבא ומיפלג בין שלש על שלש לשלשה על שלשה אלא הכי איתמר ר״פ אמר לא מפיק דאף כל לאתויי כלאים. אלא שאיני מודה בזה (שא״א) [שאפשר] לומר דמדחזי לי׳ לר״פ דמיהדר למימר דלא מפיק ואמר למעוטי כלאים כי איתותב מהאי הדר אמר הא ול״נ דה״פ דהא ודאי ר״פ אמרה מדקאמר אף כל למעוטי כלאים דאי ס״ד להך תנא דר״י לית לי׳ אפי׳ שלשה על שלשה בשאר בגדים למעוטי טומאת שרצים איצטריך ול״ל לאהדורי בתר כלאים אלא ש״מ מטומאת שרצים לא מפקינן אלא שלש על שלש דאלו שלשה על שלשה נתרבו מאו בגד כאידך תנא וכיון שכן אף כל דאמר ר׳ ישמעאל כלו׳ שנלמוד סתום מן המפורש ל״ל פשיטא דשלש על שלש בשאר טומאה מנ״ל אי מדגלי רחמנא בנגעים אדרבא מה התם צמר ופשתים אף כל צמר ופשתים א״ו ע״כ אף כל דאמר רבי ישמעאל לשאר דברים משמע כגון כלאים והך דר״פ סיומא הוא לטעמא קמא כדאמרן כי לית לי׳ להך תנא שלשה על שלשה כו׳, ואסיקנא דכלאים בריתא היא ואף כל לציצית:
}
\textblock{הא דפריך אביי \textbf{נגעים משרצים לא גמרי. דמה לשרצים שכן מטמאין בכעדשה.} ורבא לא פריך משום דכעדשה באורך ובעובי לרבא לא פחות מגריס כנגע דליתי׳ אלא שטח ויש בכעדשה לרקע ולחתך ולשטח כגריס ואביי משעור הנר׳ פרכי׳ דכול׳ מקום נגע בעובי כל הגוף חשיב נגע ולא מצטריך גמרא למיחת להכי ופשיטא לי׳, כך נ״ל:
}
\textblock{\textbf{הדר בי׳ רבא מההיא.} פי׳ ולא ס״ל כאביי נמי אלא כדאמרן ותימא אביי ל״ל למימר דר״ש ור׳ ישמעאל אמרו דבר א׳ ודר״י מפקא מאידך תנא דבי ר״י אדרבה הול״ל דרשב״א דוקא שלש על שלש לית לי׳ ופשיטא דמילתי׳ הכי משמע ור״י לית לי׳ שלש על שלש אבל שלשה על שלשה בשרצים אית לי׳ דמרבה להו מאו בגד כדאידך תנא דר״י. ואפשר משום דס״ל לאביי דההוא או בגד לתנא דר״י בתרא לגמרי מרבה להו לשאר בגדים ואפי׳ שלש על שלש הילכך ע״כ האי תנא דר״י מפיק מאידך תנא דר״י ומ״ה נמי ר״ש ור״י אמרו דבר א׳ שאם תאמר שר״ש ור״י לא אמרו דבר א׳ נמצאו שלש מחלוקות בדבר ולא דייק וא״ל דאביי גופא ס״ל מסברא דנפשי׳ דשאר בגדים כיון דאפילו משלשה על שלשה אימעט בנגעים דתרי מיעוטא כתיבי לגבי שרצים נמי לגמרי אימעוט להו דאין היקשא זו למחצה כדאמר ת״ק דר״י אף כל צו״פ משמע ולא בגדי׳ אחרי׳ כלל הלכך ע״כ מפיק הוא מאידך תנא דר״י דמרבי׳ להו כלל ונ״ל דכ״ע בין לרבא בין לר״פ ור״נ כולהו ס״ל אליבא דתנא דר״י דכ״מ שהוזכרו בגדים סתם אינן אלא צו״פ ור״פ דאמר למעוטי כלאים לא ס״ל דלא מימעט נמי ציצית אלא מודה דשאר בגדים לתנא דר״י אינן חייבין בציצית וכן רבא דאמר למעוטי טומא׳ שרצים משלש על שלש ל״ד מיעוטא לטומאה אלא אדרבה לטומאה ל״צ מיעוטא כדפרישית ואף כל לציצית או לכלאים איצטרך ור״פ ור״נ סיומא היא לתרוצי לרבא כי לית לי׳ וכו׳ כדפרישית. ומ״מ לדברי כל הפירושים ״אף כל״ דר״י כללא הוא שנלמוד כל בגדים סתומים שבתורה מן המפורש מספק מכללא אלא מקום שרבתה אותן תורה כגון שלשה על שלשה לטומאה ובדידה נמי מהני מעוטא דשלש על שלש טמא בפשתים ולא בשאר בגדים אלא שלשה ובציצית ליכא ריבוי ואפילו לרבא והוא דהכנף מין כנף לא מרבי׳ מינה שאר בגדים דכיון שבגדים שנאמרה בתורה סתם צמר ופשתן הן מין כנף נמי צו״פ ולא אתייא אלא לומר דבעי׳ לבן ממין כנף כדאי׳ בפ׳ התכלת ורבא לא יליף מינה לשאר מינין ראייה וס״ל בלא״ה דשאר מינין חייבין בציצית אלא לתרוצי קראי אתא. והיינו דאקשי׳ בריש מס׳ יבמות (ד:) והא תנא דר״י ל״ל לרבא ופריק סד״א וכו׳ ואמאי והא אמרי׳ בשמעתין דתנא דר״י למעוטי שרצים אתא א״נ למעוטי כלאים הלכך טעמא דכ׳ רחמנא צו״פ הם לא״ה ה״א כלאים בציצית [אסור] ואע״ג דכתיב כנפי בגדיהם בגדים שנאמרו בכאן סתם אינן צו״פ דוקא אלא כדרבא ומאי קושי׳. אלא ש״מ כדפי׳, דתנא דר״י בכ״מ ממעיט להו דל״מ בגדים ואע״ג דאיפשר לאוקמי לההוא אליבא דרנב״י דאמר הכא דתנא דר״י למעוטי ציצית אתא כיון דרבא גופי׳ לא אמרה ומקשי׳ מינה התם בסתמא ש״מ דלד״ה תנא דבי ר״י מתפקא מדרבא וכ״ש למאי דפרי׳ דהא רנב״י אתיר לתרוצי דרבא דהכא ועוד דחזי׳ סוגי׳ דשמעתא דבהתכלת דפשיטא להו דתנא דר״י מפיק מדרבא ומימעט מיני׳ ציצית הילכך כיון דאשכחן לי׳ לתנא דר״י דאמר דוקא צו״פ ולא אשכחן תנא דפליג עלי׳ בהדיא ואינך תנאי א״ל דלא מפקי מיני׳ הילכתא כוותי׳ ור״נ הכי ס״ל בפ׳ התכלת ואע״ג דאמרי׳ בריש מס׳ יבמות התינח לתנא דר״י לרבנן מא״ל אליבא דרבא ורחב״א ור״י קאמרי׳ דסברי דפליגי רבנן עלי׳ אבל לדידן ל״פ ולא מוקמי פלוגתא ביני תנאי שלא לצורך. ויש מי שאמר דלית הלכתא כתנא דר״י דאיתוקם בשיטה דאמר אביי רשב״א ותנא דר״י אמרו ד״א ולאו מלתא הוא דאביי לפרושי מילתייהו אתי ולאו לאוקמינהו בשיטה וכיוצא בה אמרו בפסחים בפ׳ כ״ש אמר אביי ר״ע וריב״נ ור״א כולהו ס״ל חמץ בפסח אסור בהנאה והא ודאי אפי׳ לאביי הלכה פסוקה הוא ולא מקריי׳ שיטה. ומיהו לאביי דאמר הך תנא דר״י מפיק מהנך תנאי את״ל דכי מרבה שאר בגדים לטומאת שרצים לגמרי משוי להו כצנור ופשתן ודאי לגבי ציצית כולן חייבין דמאי חזית דמקשי לנגעים נילף מבגד האמור בשרצים שכולן בכלל. אלא הך סברא דאביי הוא ורבא לית לי׳ וכ״ש דאתא למיסמך אאידך תנא דר״י ואדרשב״א ועוד שכבר פירשתי למעלה והוא הנכון לפרש דלאביי גופא אליבא דהך תנא נמי בגדים שנאמרו בשרצים סתם היינו צו״פ והן הן שמטמאין בשלש על שלש דאמרי׳ בנגעי׳ וכשריבה הכתוב שאר בגדים לשלשה על שלשה ריבה אותן הילכך ציצית צו״פ הוא דמן בגד האמור בתורה נלמוד ולא מריבוי דאו דכתבינהו רחמנא במקום א׳ למקצת (ספרים) [דברים] דבכ״מעיקר בגד הכתוב בתורה צו״פ הוא הא כ״מ שלא ריבה אותן הכתוב כלל אינן אלא צו״פ בלבד וגבי ציצית ליכא ריבוי לפיכך פסק רבינו הגדול ז״ל כר״נ דאמר שיראין פטורין מן הציצית דתניא כוותי׳ בתנא דר״י, והיא האמת:
}
\textblock{הא דאמרי׳ \textbf{רשב״א וסומכוס אמרו דבר א׳.} נראה שרש״י ז״ל מפרש מפני שהן (פסולין) [פוסלין] לסוכה כל שמטמא בנגעים אע״פ שאינו מטמא בשרצים וק״ל א״כ אפי׳ נמי פליגי רבנן עלייהו, לקולא פליגי עלייהו והא אמרי׳ במס׳ סוכה בהוצני פשתן פסולה באניצי פשתן פשרה בהושני איני יודע ומפרשי׳ בגמרא דלא פשיטא להו להכשיר אלא (להתרי) [לא תרי] ולא דייק ולא נפיץ אבל תרי וכ״ש דייק ולא נפיץ ואע״פ שאינו טווי ולא ראוי ליטמא בנגעים כלל. אלא נ״ל דלרשב״א מסככין בטווי דשאר מינין ולאביי אפי׳ בשלשה על שלשה קאמרי׳ הכא ר״ש דבעי ממש ראוי להטמא לגבי סוכה וסומכוס אמרו דבר א׳ דאלו רבנן כיון שהתחילו בו לתקנו לטווי כגון דתרי א״נ תרי ודייק ואפי׳ בשאר מינין אין מסככין דלאו פסולת גורן ויקב הוא אי נמי מדרבנן, וההיא תפתר כרבנן:
}
\textblock{הא דתנן \textbf{כל היוצא מן העץ אין מדליקין בו אלא פשתן.} פירש״י ז״ל כגון קנבוס. ולמעלה (כו.) כתב קנבוס וצמר גפן ובצמר גפן (קי״ל) [קשה לי] דיוצאמן הפרי הוא ולאו בכלל יוצא מן העץ הוא דתרי גוונא נינהו יוצא מן העץ לחוד ומן הפרי לחוד כד מוכח לעיל (שבת כו.) בשמנים ועוד אין לך פתילה מושכת בשמנים כצמר גפן. ובין בקנבוס ובין בצ״ג. קשה לי הא דאמרי׳ לעיל ולא בלכש דהוא עמרניתא דארזא ומשמע דוקא בלכש אבל בעמרניתא אחריתא מדליקין אלא משמע שאין יוצא מן העץ אלא דבר שכותשין עצו ונעשה פתילה כגון קנבוס וצמר גפן עצו לחוד וצמר (שלא) נתלש ממנו, כן נראה לי. ורבי׳ תם ז״ל פי׳ דקנבוס וצמר גפן לאו יוצאין מן העץ אלא מיני זרעים הם כדאמרי׳ במס׳ מנחות לא אסרה תורה אלא קנבוס ולוף אבל שאר זרעים מדרבנן הוא דאסירי וה״נ מוכח בפרק כיצד מברכין דאמרינן כל היכי דאי שקלת לי׳ לפרתא הדר אילנא ומפיק גוזא בפה״ע ואין כן בקנבוס ובצ״ג ובמיני זרעים יש שמדליקין אלא פשתן הוצרך להתיר מפני שקראו הכתוב עץ, ותטמנם בפשתי העץ (יהושע ב,ו):
}
\textblock{הא דקתני \textbf{אינו מטמא טומאת אוהלין.} כשעשה ממנו אהל וחברו בקרקע דהו״ל כבית הא במטלטלין כל המאהילין מטמאין, והכי מוכח בכמה מקומות:
}
\newsection{דף כח}
\textblock{\textbf{אתיא אהל אהל ממשכן. כתיב הכא אדם כי ימות באהל וכתיב התם ויפרוש את האהל על המשכן.} פי׳ כל מה שבמשכן נקרא אהל, קרוי אהל ניטמא טומאת אהל ואשכחן פשתן דאקרי אהל כדכתיב משכן אהל מועד מכאן שהפשתן מטמא טומאת אוהלין אבל מהאי קרא גופי׳ דויפרוש את האהל לא גמר לפשתן דהאי אהל של יריעות עזים הוא ואיהו נמי מיטמא כדאמרי׳ לקמן ומה נוצה של עזים והיינו נמי דאקשי׳ אי מה להלן קרשים כלומר מה גילינו במשכן שהקרשים קרוים משכן וכל שנקרא משכן נקרא אהל כדכתיב משכן אהל נימא אף לענין טומאת אוהלין יהיו אהל ויטמא טומאת אוהלים, ואין לשון זה מחוור ולא נכון כלל. אלא י״ל מדכתיב ויקם משה את המשכן ויתן את אדניו וגו׳ קא מסדר לי׳ אדני׳ קרשי׳ ובריחי׳ ועמודי׳ ולמעל׳ מהם יריעות דשש הי׳ פורש עליהן שהן קרוין משכן והי׳ לו לכתוב ויפרוש את המשכן וכתיב ויפרוש את האהל על המשכן מ״ה אר״א דהאי קרא בפרשת יריעות דשש מיירי ומקרא מסורס הוא וה״ק ויפרוש את האהל ויפרוש על המשכן כלומר פרש האהל דהוא משכן כלו׳ יריעות דשש ופי׳ עליהן אהל אחר יריעות ועזים והא דמייתי לה ממשכן אהל מועד משום דמשמע משכן של אהל דהוא יריעות ועזים כדכתיב לאהל על המשכן, כן נ״ל:
}
\textblock{\textbf{אלא הא דתני רב יוסף לא הוכשרו למלאכת שמים למאי הלכתא.} פירש״י ז״ל דמאי דהוה הוה ולא מחוור לי דהא אמרי׳ מאי הוה עלה דתחש ורב יוסף נפק מיני׳ למילף דתחש בהמה בהמה טהורה הי׳ ואפשר דאי לאו להלכותיואיתמר לא הול״ל לא הוכשרו אלא הול״ל בהדי׳ תחש שהי׳ בימי משה ברי׳ טהורה הוא:
}
\textblock{והא דאקשינן \textbf{תפילין בהדי׳ כתיב בהו.} ק״ל, דלמא היא גופא אתי רב יוסף לאשמועינן [טעמא] דבתפילין וספר תורה מתנא תנא בהדיא אין כותבים על עור בהמה טמאה וכו׳ וטעמא משום תורת ה׳ בפיך ונ״ל דרב יוסף מן תחש גמיר ומיני׳ קאמר לא מצינו שהוכשרו למלאכת שמים דהוא מלאכת המשכן אלא עור בהמה טהורה בלבד ומ״ה אקשינן הני בהדי׳ כתיבי ולא מתחש אתו והא דקאמרינן בתר הכי מאי הוה עלה דתחש בעיא בתר פשטא הוא וכיוצא בה לקמן בפ׳ במה בהמה יוצאה (שבת נג.) ואחרות בתלמוד. ומסקי׳ לרצועות דתשמישין נינהו ואינו גוף קדושה ולא אתי מתורת ה׳ בפיך ואע״פ שאין הלכה דהא גמירי שחורות [ולא טהורות]. אלא מתחש דהוא משמש לקדש גמרי׳. ומיהו בהא דאמרן אלא לרצועותיהן ק״ל והא יו״ד ודל״ת של תפילין הלמ״מ כדאמרי׳ לקמן בפ׳ במה אשה יוצאה ואי׳ התם במנחות קשר של תפילין הלמ״מ וכיון שהשם כתוב ברצועות מותר בפיך בעי כדבעי׳ משום שי״ן של תפילין. ואפשר דמרצועות דיד קאמר, ועיקר קשר של תפילין קשר של ראש שיש בו רובו של שם כדאמרי׳ וראית את אחורי מלמד שהראהו הקב״ה למשה קשר של תפילין ואמרי׳ נמי וראו כל עמי הארץ כי שם ה׳ נקרא עליך אלו תפילין שבראש לפי שרובו של שם כתוב ברצועותיהן מבחוץ. וצ״ע וא״ל שי״ן דהוא כתובה קיימת הוי כתורת ה׳ ד׳ ויוד דקשרים הן, לא:
}
\textblock{\textbf{אמר עולא מחלוקת בקטנים אבל בגדולים ד״ה מותר.} כך נמצאת הגי׳ בספרים שלנו, וק״ל דהא ר״י דבר שאינו מתכוין אסור לי׳ בכ״מ גבי בכור שאחזו דם אין מקיזין לו דם ואע״פ שהוא כגדולים דלא אפשר ותנן נמי כל הכלים אינן נגררין חוץ מן העגלה מפני שכובשת קתני כל וקתני כלים דומיא דעגלה שהיא גדולה וטעמא שכובשת הא בעושין חריץ אסור. ובמקצת נוסחאות מצאו: מחלוקות בקטנים אבל בגדולים ד״ה אסור ומפרשי לי׳ משום דה״ל פסיק רישי׳ ולא ימות ומיהו התם בי״ט אוקי מתני׳ דקתני גבי עגלה ואינה נגררת אלא ע״ג כלים דלא כר״ש ולא קאמר דפסיק רישי׳ ולא ימות הוא וי״ל עגלה של קטן כלי קטן מקריא א״נ התם משום דזמנין שכובשת אבל ממה שהשמיטה רבינו והגאונים לא כתבוה נראה שאין הגירסא כן:
}
\newsection{דף ל}
\textblock{הא דאמרי׳ \textbf{במאי בחולה שיש בו סכנה מותר מבעי לי׳.} פירש״י ז״ל דה״ה לשאר פטורי דמתני׳. ויש לפרש שיודע היה שבגוים ולסטים ורוח רעה יש בהם סכנה שכן דרכן לעולם וספיקן נמי מתירין ולא הוה קשי׳ לי׳ בהו נמיתני מותר משום דקס״ד דמשום חולה תני בשארא פטור ומ״ה קשי׳ לי׳ אי בחולה שיש בו סכנה בכולן הי׳ בו לשנות מותר אי בחולה שאין בו סכנה לא הו״ל למיתנייא בהדי שארא והו״ל למיתניי׳ בהא חייב ובשארא מותר:
}
\newsection{דף לב}
\textblock{הא דאמרינן בהגדה \textbf{הלכות הקדש תרומה ומעשרות הן הן גופי תורה.} פירש״י ז״ל הקדש מסור לכל אדם להקדש ואין חוששין במטלטלין שלהן שמא הקדישום וכן בתרומה נאמנין לומר הפרישו תרומה ומעשרות ואע״ג דתיקון רבנן דמאי מ״מ תורה האמינתם. ורבותי פירשו הלכות הקדש כגון יין לנסכים כדמפרש בחומר בקודש (חגיגה כז:) שע״ה נאמן עליו וק״ל אי מדרבנן קאמר לא הול״ל תרומות ומעשרות דרבנן לא האמינוהו, אלו דברי רש״י ז״ל. אבל בירושלמי (ב,ז) מצאתי כך רשב״ג אומר הלכות הקדש חטאת וההכשרות הן הן גופי הלכות ושלשתן נמסרו לע״ה הלכות הקדש דתנינן אם אמר הפרשתי לתוכה רביעית קדש נאמן חטאות דתנינן הכל נאמנין על החטאת ההכשרות דתנינן ועל כולם ע״ה נאמן לומר טהורין הן:
}
\newsection{דף לד}
\textblock{הא דתנינן \textbf{עשרתם ערבתם הדליקו את הנר.} פירש״י ז״ל דהני תרתי שייכי למימר בלשון שאלה שכבר עשו אבל בנר לא שייך למימר בזה הדלקתם דדבר הנראה לעין וקא חזי אי אדליק אי לא אדליק. ובירושל׳ גרסי׳ הכי: לא צירכא דלא הדליקו את הנר עשרתם ערבתם אמר רבי חני בר אדי שאני מתמיר עלה בקלה אף הוא מחמיר על עצמו בחמורה פי׳ ומ״ה מקדים עשרתם ערבתם:
}
\textblock{\textbf{אמר רבא א״ל שנים צא וערב עלינו.} פירש״י ז״ל עירובי תחומין וקשה לי והא אמרן דספק חשיכה וס׳ אינה חשיכה אין מערבין עירובי תחומין ואפשר שאין מערבין אבל אם עירב עירובו עירוב. ור״ח ז״ל פירש בעירובי חצירות וכן עיקר והא קמ״ל שאע״פ ששנים אמרו לא׳ לערב לא אמרי׳ ממ״ג ואע״ג דדמיא למאי דאמרי׳ בעלמא בבאין לישאל בבת אחת שניהן טמאין הכא בין השמשות ספיקא הוא ועירובין דרבנן וספיקא דרבנן לקולא ולא אמרי׳ בכה״ג ממ״נ:
}
\textblock{ ה״ג ר״ח ז״ל ור״י אלפסי ז״ל, וכ״כ במקצת נוסחי: \textbf{מפני מה אמרו אין טומנין בדבר המוסיף הבל אפילו מבע״י גזירה שמא ירתיח.} פי׳ לפי שהוא מניח קדרתו שאינה רותחת כ״כ והוא מוציאה מרותחת הרבה, אף היא בא ומרתיחה לכתחלה ועוד דכיון שהוא מטמין בדבר המוסיף אלמא ברותחת ניחא לי׳ וזמנין דפסק רתיחתה ומרתח לה איהו ואקשי׳ א״ה בה״ש כלומר בה״ש דמתני׳ בדבר שאינו מוסיף ליתסר דכיון דשהה עד הערב ומטמין ברותח ניחא לי׳ לאורתא ודילמא נמי מצטריך ומרתח לה ופריק סתם קדרות בה״ש רותחות הן ולא תפסוק רתיחתן בשעה קלה ואי למחר בעי לה הא לא איכפת לי׳ ברותחת כיון דמטמין בדבר שאינו מוסיף לצורך מחר. ודבר שאינו מוסיף (נגזר) משום רמץ שאף הוא אינו מוסיף הבל משום דהו״ל כקטומה ופחות ממנה לפי שהם גחלים עוממות מוטמנות בתוך האפר וא״ה אסור משום גזירה דלמא אתי לחתויי בגחלים, כ״פ הראב״ד ז״ל. וכתב רב אלפסי ז״ל בהלכותיו שזה שתמצא בשאר ספרים אין טומנים בדבר שאינו מוסיף משחשיכה גזירה שמא ירתיח אל תסמוך עליו שהוא שבוש גדול לפי שדבר שאינו מוסיף אינו מרתיח וא״כ איך נגזור עליו. אבל ראיתי מי שכתב פי׳ השמועה לפי גרסתו גזירה שמא ירתיח מאליו ויצטרך לגלות עד שתנוח רתיחתה ויחזיר ויטמין בשבת ואקשי׳ א״ה בה״ש ליגזור כלומר שלא יטמין בה״ש בדבר המוסיף הבל לצורך מחר ופריק כבר הם מרותתות והוא לא יטמין עד שיגלה אותן ויוציא זוהמתן מהן ואין לחוש ומותר וזה חידוש שאין הדעת מקבלו שיהא מבע״י אסורה ובה״ש מותר. ודברי רבינו הגדול ז״ל סתומין אבל מטין לזה הדעת. שוב מצאתי בירושלמי בריש במה טומנין. בלשון הזה לפי שהדברים הללו רותחין ומרותחין והוא נוטלן והן חשין לתוך ידו והוא מחזירן והן מוסיפין רותתת לפיכך אסור לטמון בהן ע״כ תה מפורש כדבריו. וא״ת ואם מגלה נמי וחוזר יטומן בשבת מה איסור יש אם צריך וזה משמע לקמן בטומן בשבת מגלה ונוטל וחוזר ומטמין בשבת התם בדבר שאינו מוסיף וכיון שטמן מבע״י כבר הסיח דעתו מאותו בישול אלא שיהא משומר הילכך נוטל ומחזיר אבל בדבר המוסיף אם מגלה ומרתיח וחוזר ומטמין כתחלת הטמנה הוא ואיכא גזירה דחתוי גחלים ומגיס אבל זה שאמרנו בגמ׳ דילן א״ה בה״ש נמי נגזר דהיינו בה״ש דשרי׳ במתני׳ בדבר שאינו מוסיף לפי שבזה נמי יש נחוש מתוך שהוא מסלק קדירה מעל גבי כירה עם חשיכה והיא רותחת ומטמין שמא תרתיח ויצטרך לגלותה וחוזר ומטמין לכתחילה בשבת דהא לא הוא טמונה בשבת כלל ופריק כבר הם מרותחות ויוצאות זוהמתן מבע״י ואין חוששין שמא בזמן מועט ירתיח ויצטרך ליטלן שהרי דבר שאינו מוסיף אין חוזר ומרתיח. ומ״מ גזירה שמא יטמין ברמץ לדבר שאינו מוסיף קשה דרמץ ודאי מוסיף דכל שמחמם את הצונן נקרא מוסיף:
}
\textblock{מתני׳: \textbf{כירה שהסיקוה בקש ובגבבא נותנין עלי׳ תבשיל.} רש״י ז״ל מפרש ענין הכירה אף בהטמנה והוא ז״ל סובר שהכל גרוף וקטום וכן קש וגבבא אינן מוסיפין הבל ולשהות המוזכר כאן במשנה ובגמ׳ (לדעתי) [לדעתו] הוא מותר בין בהטמנה בין בלא הטמנה אבל להחזיר המוזכר בכאן כלא הטמנה מותר בהטמנה אסור שאין טומנין בשבת אלא י״ל אם טמן מבע״י ונטל בשבת מותר הוא להחזיר שהרי הטמנה מבע״י היא זו כדתנן לקמן בענין קופה נוטל ומחזיר וכ״כ רש״י עצמו נקמן גבי מחזירין אפי׳ בשבת. ור״ח ז״ל אמר שאינה הטמנה אלא כיסוי של ברזל והקדירה שובת עליו והיא תלוי׳ באבנים או בכיוצא בהם אבל הטמינם ע״ג גחלים ד״ה אסור דקי״ל הטמנה בדבר המוסיף מבע״י אסור ואמת אמר שלא התיר ר׳ חנניא כמאכל ב״ד להטמין אלא לשהות דאי מטמין כיון דאינו גרוף וקטום הו״ל הטמנה בגחלים ממש דהיא אסורה מבע״י וכן רש״י ז״ל עצמו נזהר מזה בגמ׳ (בעינן) [בענין] תוכה וגבה דתוכה אסור בשאינה גרופה משום דהו״ל הטמנה ברמץ דאסרן מבע״י וכיון שהוא מודה שאין דברי רבי חנניא אמורין בהטמנה ושיטת כל השמועה (בדברי) [כדברי] ר״ח היא שנויה, כבר למדנו שאין הטמנה בכלל השמועה זו:
}
\textblock{\textbf{עד שיגרוף או עד שיתן האפר.} מפורש בירושלמי (ג,א) הגורף עד שיגרוף כל צרכו מן מה דתני׳ הגורף צריך לטאטו בידו הדא אמרה עד שיגרוף כ״צ פי׳ שיגרוף לגמרי עד שלא ישארו בו גחלים כלל ולא שלהבת. הקוטם עד שיקטם כ״צ מן מה דתני מלבה עלי׳ נעורת של פשתן הדא אמרה אפי׳ לא קטם כל צרכו פי׳ א״צ לקטום עד שלא יהא ניכר כלל שם אש. ובגמ׳ דילן נמי בקוטם הכי משמע שהרי אפי׳ גחלים שעממו הרי הן כקטומים וכן קטמה ונתלבתה ומתני׳ נמי דייקא דקתני יתן את האפר משמע נתינה בעלמא מדלא קתני שיטמין באפר והטעם לפי שהמערב אפר ואש לגמרי סליק דעתי׳ מחיתוי האש וכבר קלקלו והסיח דעתו ממנו אבל הגורף אם נשאר שם אפי׳ גחלת קטנה כיון שהיא לוחשת מבערת ומדלקת כירה גדולה וא״ת אם צריך נגרוף לנמרי בתנור למה אסור א״ל לפי שהבלה גדול ומרתיח אינו נראה כגרוף ואתי להשהותע״ג כירה של גחלים ואתי לחתויי. וכתב ר״ח ז״ל וקיימא לן מתני׳ בתנור של נחתומין ותנור של נחתום מפורש בפ׳ לא יחפור (ב״ב כ:), אבל תנור דידן ככירה של נחתומין היא:
}
\textblock{}
\textblock{גמרא: \textbf{או דלמא לשהות תנן, ואי גרוף וקטום אין אי לא לא.} ומנו רבנן דפליגי אדר״ח וליכא למימר דכי תנן נמי לשהות כחנניא אתיא והא דקתני לא ישהה עד שיגרוף בפחות ממאכל ב״ד דמודה בזה חנניא הא ליכא למימר חדא דתבשיל בפחות ממאכל ב״ד לא הוי ועוד בין חמין לתבשיל בפחות ממאכל ב״ד ליכא לאפלוגי דהא מתבשלין בתחלה בשבת ועוד נוטלין ומחזירין בפחות ממאכל ב״ד ליכא למימר כלל דהא מבשל בתחלה בשבת הוא וענוש כרות ונסקל הוא. אלא כולה מתני׳ כמאכל ב״ד היא, ואי לשהות תנן דלא כחנניא הוא. והאי דלא אמרי׳ בגמ׳ ומני רבנן הוא משום דאכתי לא קמה לן פלוגתא דרבנן עלי׳ דחנני׳ אלא מיני׳ דמתני׳ אי לשהות תנן ומיהו לקמן נפקא לן ממתני׳ דר״מ ור״י. ואע״ג דכולה מתני׳ כמאכל ב״ד הוא לשהות (אפילו) בגרופה וקטומה אפי׳ בפחות מכן מותר דהא אשכחן כל היכי דליכא למיחש לחתוי דשרי כדתנן במשלשלין את הפסח לתוך התנור עם חשיכה ובקדירה חייתא דשרי ואע״פ שהוא מתבשל לגמרי בשבת (ממילא דהא כל דאתי) בהכי שרי ב״ה כדאי׳ בפ״ק (יז:):
}
\newsection{דף לז}
\textblock{הא דאמרי׳ \textbf{אא״ב להחזיר תנן היינו דשני בין תוכה לגבה,} פירש״י ז״ל אא״ב להחזיר תנן אבל לשהות משהין אפי׳ בשאינה קטומה ור׳ חלבו אשריותא דלשהות בשאינה קטומה קאי היינו דשני דכי משהה בתוכה בשאינה גרופה מטמין ממש ברמץ אלא א״א לשהות תנן דאפי׳ לשהות בעי׳ גרופה כיון דגרופה ומבע״י הוא נותנין לתוכה מה לי תוכה מה לי על גבה ותימה הוא מי דוחקו בפי׳ הזה והרי בסמוך אמרינן דבלהחזיר עצמו שני בין תוכה לגבה ול״ל (כר׳) [דר׳] חלבו אשריותא דלשהות קאי בשאינה קטומה (כלומר) [ומ״מ] לפי דבריו שהוא מפרש חומר זה דתוכה משום הטמנה אמסקנ׳ דלהחזיר נמי כך יש לנו להעלות דאסור להחזיר בתוכה משום הטמנה דהו״ל מטמין משחשיכה דחזרה בשבת הוא והוא ז״ל אינו אומר כן בסוגיא דלקמן כמ״ש למעלה. ומיהו פי׳ דהך אתקפתא כפשטא הוא, דמסתבר להו בגמר׳ בחזרה דחמיר תוכה משום דנפיש הבלא והוי כמבשל בשבת ואע״פ שגרוף וקטום ואין כאן הטמנה דלאו בתוך הגחלים הוא מטמין וגופה של כירה נמי אין הטמנה כלל ובן תוכו דתנור במתני׳ לאו הטמנה מיקרי. ומצאתי בירושלמי (ג,א)שמפרש טעמא, וה״ג התם וב״ה או׳ יחזיר ר׳ חלבו בשם רב ל״ש אלא עליה אבל לתוכו לא עד אמר עולא עד שלשה ר׳ מונא אומר עד מקום שהוא עושה חרץ א״ר יוסי בר בון מפני שהוא שליט במקום שהיד שולטת בו ופי׳ דעולא סבר עד ג׳ טפחים בעומקה של כירה נקרא עליה ומותר ר׳ מונא מתיר עד מקום חרץ שעושין בכירה ומפרש ר׳ יוסי טעמא דרב מפני שאין רשאי להכניס קדרתו בשבת במקום שאין היד שולטת בו מפני רוב החמימות ולא משום הטמנה נגעו בה:
}
\textblock{\textbf{תא שמע דאמר ר׳ חלבו אר״ח בר גוריא אמר רב ל״ש אלא על גבה וכו׳.} ק״ל, כיון דרב מהימן לי׳ ולפשוט מהא דאמרי׳ לקמן רב ושמואל דאמרי תרווייהו מצטמק ויפה לו דהיינו נתבשל כ״צ מצטמק ויפה לו (אין) צ״ל שהוא אוסר כל תבשיל כמאכל בן דרוסאי שכל במאכל ב״ד מתבשל מצטמק ויפה לו הוא וח״ל דמהתם לא מיפשט לי׳ היכי תנן אלא דרב סבר אסור לשהות ודילמא אין להחזיר תנן ורב לא ס״ל כמתניתין דחנניא הוא ורבנן פליגי עלי׳ ומש״ה לא פשטוה מדרב ששת ח״ר יוחנן דלקמן ואע״ג דבגמרא מפרש לטעמייהו דלהחזיר תנן דלמא לא היא אלא לעולם לשהות תנן ואיהו דאמר כחנניא ודלא כמתני׳ ומש״ה אתי למיפשט מהא דאמר רב ל״ש דאלמא עלה דמתני׳ קאיואפשר דר׳ יוחנן לא מהימן לי׳ ורב מהימן לי׳ משום דהוה במניני׳ דרבי דסתמינהו למתני׳ א״נ משום דאמוראי נינהו אליבי׳ דאמר רב (ושמואל) אר״י כירה שהסיקוה בגפת ובעצים משהין עלי׳ חמין אלמא כל צרכן אין שלא כל צרכן לא ואין הטעם הראשון נכון בדר׳ יוחנן משום דע״כ ר״י כסתם מתני׳ ס״ל דא״ר יוחנן הלכה כסתם משנה:
}
\textblock{\textbf{מהו לסמוך.} פירוש ר״י אלפסי ז״ל לשהות בסמיכה, וכן הלכה אבל אין ראיותיו ברורות לנו כל הצורך אלא כיון דמתניתין למאי דאוקימנא להחזיר חסורי מחסרא ועוד דצריכת למימר דסבר לה כר״י בחדא מ״ה ניחא לן למימר לשהות תנן דאשנויי לא סמכינן בכה״ג ואף על פי שמקצת הגאונים ז״ל חלוקין בדבר ואומרים דהלכתא כלישנא דלהחזיר אנו אין לנו אלא כלשון רבינו הגדול ז״ל משוה דחזי׳ דבתר כל שקלא וטרי׳ דשמעתין כולהו אמרו לקמן הלכתא מצטמק ויפה לו אסור וכבר פירשתי שכל שאסור מצטמק ויפה לו א״א שלא יאסור לשהות כמאכל ב״ד שכל תבשיל שבעולם ואפי׳ חמין כמאכל ב״ד מתבשל ומצטמק ויפה לו הילכך אסור ואפי׳ לרב (ושמואל) משמי׳ דר׳ יוחנן דאמר מצטמק ויפה לו מותר כמאכל ב״ד שהוא מתבשל ויפה לו אסור דאל״ה ליתני כמאכל ב״ד וכ״ש האי דנתבשל כ״צ דשרי (ועיין במאור) ועוד דקאמר בהדי׳ חמין שהוחמו כל צורכן אלמא לא הוחמו כל צרכן נא ואע״פ שהוא מצטמק ורע להן והא דר״י דסמכא הוא דחזינא לי׳ לרב עיקבא דא״ל רב אשי דסמוך אהא דר׳ יוחנן ואם יאמר לך אדם דלמא כמאכל ב״ד שרי לשהות לפי שהוא מסיח דעתו מלאכול עד למחר ולמחר הרי כבר נתבשל ונצטמק אבל כשהוא כל צרכו וא״צ אלא להצטמק ויפה לו אסור שהוא אוכלו בלילה ואתי לחתיי אף אתה אמור לו הרי משנתינו שנינו תבשיל סתם ואין במשמע מאכל ב״ד בלבד אלא כל שהוא ממאכל ב״ד ולמעלה ואפ״ה שרי לשהות ללישנא דאמר להחזיר תנן ועוד דתניא בברייתא דר״י ובה״א חמין ותבשיל וההוא אפי׳ בנתבשל כל צרכו ובמצטמק ויפה לו הוא מדאקשי׳ לקמן קשי׳ דר״י אדר״י דקא אסר מצטמק ויפה לו (וא״ת) אמרי׳ דתנא דמתני׳ פליג עליה בחדא וסבר דשרי אלמא למאן דמתני לישנא דלהחזיר מצטמק ויפה לו מותר לשהות ועוד דא״כ מצטמק ורע לו נמי כשהוא קרוב לבישולו יאסור וכן חמין דהא חזי ללילה ואתי לחתוי בי׳ ודברי הבאי הן דבמאכל בן דרוסאי ללילה קבעי לי׳ ולא מסח דעתי׳ מיני׳ כלל וחנניא כלל גדול קאמר כל שהוא כמאכל ב״ד וממנו ולמעלה בין מצטמק בין מתבשל ואמת הדבר דכל סתם תבשיל דתנן במתני׳ ובבריי׳ מצטמק ויפה לו הוא וכיון שכן ממסקנא דאסרי׳ כל מצטמק ויפה לו איפשטא לי׳ בעיין דהאי לשהות נמי אסור ועוד ר״ח שפסק דלהחזיר תנן אמר שמצטמק ויפה לו מותר וסמוך אדר׳ יוחנן והרי השמועה לפניך פשוטה דמצטמק ויפה לו אסור ועוד דר׳ יוחנן כמאכל ב״ד אסור וכדכתיבנא לעיל ואין אני יודע דרך וטענה לדברי הגאונים אא״כ נפרש דהא דאמר ר״נ מצטמק ויפה לו אסור לא לשהות אלא להחזיר למר כדאית לי׳ ולמר כדא״ל לומר דמבושל כל צרכו ומצטמק ויפה לו הרי הוא כאינו מבושל כ״צ וגם זה אינו עולה משום דאסוגי׳ דשהי׳ איתמר (ולא מפסק) [ולמיפסק] בה אתא. ועוד דלא אשכחן התירא להחזיר אפי׳ במצטמק ורע לו אלא בגרופה וקטומה ולא שרי׳ חמין להחזיר אלא בקטומה אפילו הוחמו כ״צ והכי מוכחא סוגי׳ דלקמן דמוכחא חזרה דמתני׳ מחזרה דקומקום של חמין דר׳ הוא דהוחם כ״צ ואי אמרת דההוא אפי׳ בלא גרוף וקטום שרי מה ענין זה לחזרה דמתני׳ דהיא צריכה לבשול או להצטמק ואסורה אלא בקטומה דההוא כיון דצריכה בשבת אסורה הוא ש״מ כל חזרה בין דמצטמק ויפה לו בין דרע לו בין דהוחמו כ״צ הכל דין א׳ הוא. ומסתברא דאמרי׳ בפ״ק בשיל ולא בשיל אסור מכיון שהתחיל לבשל עד שיהי׳ מבושל כ״צ שהוא למעלה מתבשילו של ב״ד הוא נקרא בשיל ולא בשיל וכ״כ בה״ג של ר׳ שמעון קיארא ז״ל דבשיל ולא בשיל היינו כמאכל ב״ד דאמרי׳ בפ״ק דאסור לשהותה ע״ג כירה דלא כחנניא. ואשכחינן תמן בהלכות גדולות ובפסוקות קדירה דמטא בשולא מבע״י אם מנטמקת ויפה לו אסור לשהותו ע״ג כירה בשבת דכיון דבעי למיכל מיני׳ בה״ש מסרהב ומפיש לה נורא ובתר דקדיש יומא מישתלא ואתי לחתוי בגחלים מצטמק ורע לו מותר לשהותה ע״ג כירה גרופה וקטומה דמסקי׳ אשמעתי׳ אמר ר״נ וכו׳ וזה תימא ואתי׳ אליבא דר״מ, ובודאי לית הלכתא כוותיה:
}
\textblock{\textbf{ש״מ מצטמק ויפה לו מותר.} פירוש דסתם תבשיל כמצטמק ויפה לו ומ״ה נקטו חמין ותבשיל ועוד דאי במצטמק ורע לו פשיטא מאי שריותא אתי לאשמועינן דקאמר משהין וקס״ד דקטמה לאו דוקא. אלא הא קמ״ל דאע״ג דקטמה וגלי אדעתי׳ דלא ניחא לי׳ בחתוי גחלים אפ״ה דוקא מבושל כל צרכו אבל לא נתבשל כ״צ לא ואגב אורח׳ קמ״ל דלית לן דחנניא ומפרקי׳ ש״ה דקטמה כלומר משום דקטמה הוא ומצטמק ויפה לו אסור. ואקשי׳ אי חשוב לי׳ כקטומ׳ מאי למימרא כלומר היא גופי׳ מאי למימר׳ דמיתר בנתבשל כ״צ אפי׳ בלא נתבשל כ״צ נמי מותר ואפי׳ בפחות ממאכל ב״ד שלא מצינו שום איסור בכירה קטומה בשהיי׳ ומדקאמר משהין אלמא התירה נמי קמשמע לן. ופריק הובערה אצטריכא לי׳ מ״ד כיון דהובער׳ הדרא למילתא קמייתא ולתסר קמ״ל וה״ה למאכל ב״ד או פחות ממנו אלא דהא קמ״ל דאע״ג דבשיל כ״צ אי קטומה אין אי לא אסור דמצטמק ויפה לו אסור והיינו דלא כחנניא כדפרישית. וזהו שכ׳ רבינו הגדול ז״ל ורב אושעיא ורבה בב״ח אמר ר״י הכי ס״ל. ומיהו קשיא, אמאי קאמר חמין שהוחמו כ״צ דהא אפילו בשאינה קטומה נמי שרי וי״ל משום סרכא דתבשיל, ועיקר. ואפשר לפרש כך: א״ה מאי למימרא לשהות אפי׳ להחזיר נמי הובערה איצטרכא לי׳ מ״ד הדרה לה להא מילתא קמייתא ותתסר אפי׳ לשהות קמ״ל דלא הדרא לגמרי ומיהו דוקא לשהות תבשיל שנתבשל כ״צ וחמין שהוחמו כ״צ דלא אתי לחתוי כיון שכבר נתבשל אבל לא הוחמו כ״צ אפי׳ לשהות אסור דכיון דהובערה אתי לחתוי. ולהאי פי׳ הא דאמר ר״ס קטמה ונתלבתה סומכין לה דוקא כשנתבשל כל צרכו אבל לא נתבשל כ״צ לא. ופירוש מצטמק ויפה לו בכולה שמעתין היינו תבשיל שנתבשל כ״צ כמאכל כל אדם אלא שצימוקו יפה לו ומאן דאסר כר״י ומאן דשרי סבר ע״כ לא אסר ר׳ יהודה דמתניתין אלא כמאכל ב״ד ולמעלה ממנו כ״ז שהוא מתבשל וצריך לו אבל נתבשל כ״צ משום צימוק לא אתי לחתוי בגחלים ואפילו אינה קטומה מותר ואע״ג דר״י דבריית׳ דלקמן אסר אפילו דיעבד בההיא כר״מ ס״ל דלית לי׳ צימוק אוסר וסבירא לן (כר״מ) [דר״מ] אפילו לכתחלה שרי וההיא דקתני לעיל חמין אבל לא תבשיל כמאכל ב״ד. והא דאסיק בגמרא הא לכתחלה הא דיעבד אליבא דמ״ד מצטמק ויפה לו קאמרי׳ דאסר א״נ י״ל דבלכתחלה לא ס״ל כוותי׳ דר״מ דאסר משום דאיהו אפי׳ בקטומה אסר דגזר בתבשיל לעולם ולא גזר בחמין לעולם כיון שהגיע למאכל ב״ד ומיהו תבשיל קודם שנתבשל כ״צ מצטמק ויפה לו ומצטמק ורע לו שוין הן אלא שב״ש מחמירין בו במתני׳ מפני שהוא צריך יותר בישול ואיפשר שב״ש בכל תבשיל הן מחמירין ומה שפירשתי דבר ברור וראיתי בתוספת בזה שבוש שאני תמי׳ עליהן:
}
\textblock{הא ד\textbf{אמר רבא תרווייהו תנינא.} לדברי רב ששת קאמר כלומר רב ששת דיוקא דמתני׳ קמ״ל א״נ שיעורא דר״ש תנינא דאפי׳ היכי דלא בשיל כ״צ כמבושל חשבינן לי׳ דהיינו משקרמו פני׳ אבל איהו דיוקא דמתני׳ קמ״ל דאפי׳ בכירה ותבשיל נמי הולכין אחר השיעור הזה דלהחזיר תנן מיהו מאן דאית לי׳ משום מחתה בגחלים אפילו כמאכל ב״ד ל״ק מתני׳ דפת דש״ה דליכא גזירה דחתוי דכיון שהוא על האש אי מחתה לי׳ סליק בי׳ קיטרא וכן בבשר בצל וביצה משום דמחרכי ומיהו קודם שהגיע למאכל ב״ד חיישי׳ דלמא מחתי להו דכל שלא בשיל כ״צ לא מיחרך וזה דבר ידוע א״נ מפיק להו בר דלא קשי להו זיקא. תדע, דהא רב דאית לי׳ הכא מצטמק ויפה לו אסור אמר התם וכמה כדי שיצלו כמאכל ב״ד ולפי הפי׳ הזה נדחו דברי המביאים ראי׳ מסייעתי׳ דרבא לפסוק הלכה דלהחזיר תנן ונסתייעו דברי רבינו הגדול ז״ל, ודברי רש״י ז״ל מטין שהוא מן המתירין:
}
\newsection{דף לח}
\textblock{\textbf{אי קודם גזירה קשי׳ מזיד ואי לאחר גזירה קשי׳ שוגג.} ק״ל, ואי ס״ל דאיכא גזיר׳ היכי אמר להתירא וא״ל ה״ק אי קודם גזירה איירי קשי׳ מזיד ואי לאחר גזירה אפי׳ שוגג אסור ופשיטא דלא איירי איהו לגזירה אלא סבר דלא גזרו או דמקמי גזירה הוה דבימי רב גזרו כך בה ושינוי דלישנא הוא אי לאחר גזירה:
}
\textblock{\textbf{איבעיא להו עבר ושהה מאי.} פי׳ מצטמק ויפה לו קמבעי׳ לי׳ אי כר״מ אי כר״י ואע״ג דתני׳ לעיל אשכחן כה״ג פ״ק דכתובו׳ איבעי׳ להו מאי לבעול לכתחל׳ בשבת ואע״ג דתני׳ לעיל מיני׳ לא יבעול מפני שהוא עושה חבורה והאי פי׳ ליתא חדא דהתם בריי׳ סתם תניא וקמבעי׳ לי׳ אי אתי׳ כדברי הכל או לא כדמוכח בשמעת׳ דפליגי בה תנאי ועוד דהכא דאמוראי פליגי בה לעיל מאי מיבעי׳ לי׳ תו. ואחרים פי׳ דקא מיבעי׳ האי מותרים ואסור דר״י לכתחל׳ או אפי׳ דיעבד גם זה אינו נכון חדא דמותרין משמע דיעבד מדלא קאמר מותר ועוד דאי לכתחל׳ אפי׳ ר״מ אסר ומאי בינייהו דבלישנא דבריית׳ משמע בהדי׳ דפלוגתייהו בתבשיל הוא. והפי׳ הנכון דבשוגג מיבעי׳ לי׳ מי אמרי׳ כי פליג ר׳ יהודה אמזיד דר״מ או אפי׳ אשוגג וכי אמרי׳ נמי קנסו אף על השוכח דילמא בשאינו מבושל כ״צ אבל מצטמק ויפה לו בשוגג לא אסרו א״ד ל״ש ואתי למפשט לי׳ מדר׳ יוסי שמצא בצלי שנשתהה ע״ג כירה והתם ודאי שוגגין או שוכחין היו שבמזיד ודאי לא שהו ואעפ״כ אמר כן ופריק לשבת אחרת שיזהרו שלא יעשה כן עוד לשבת אחרת אע״פ שהיו שוכחין או שמא שוגגין היו, וכ״נ מדברי רבינו הגדול ז״ל, ואיהו גריס עבר ושכח, וכולם לא נתכוונו אלא לדבר א׳:
}
\textblock{\textbf{מחזירין ואפי׳ בשבת.} פי׳ רש״י ז״ל יום שבת שלא יאמר שלא התירו אלא סילק בלילה והחזיר [אבל] מכיון שסילק ביום שבת דעתו שלא להחזיר ואסור (דבעי) [דדמי] למניח תחל׳, ואחרים מפרשים שבת היינו משתחשך שלא תאמר לא התירו להחזיר אלא בה״ש. ובספר התרומ׳ ראיתי דמתני׳ דמחזיר מבע״י מדקתני עד שיגרוף ולא קתני אא״כ גרף אלמא בשעה שמותר לגרוף היא אלא דכל שהוא בכדי שתהא הקדרה כולה מתחמם מבע״י אם הית׳ צוננות קרי משהה וכל שהוא סמוך לחשכה שחמימות׳ בשבת קרי מחזיר והיינו מתני׳ נחתי בגמ׳ למישרי מחזיר ממש משחשיכה וזה עולה יפה יותר מן הלשונות שזכרתי. ובירושל׳ (ג.) מצאתי נטלו מבע״י מחזיר מבע״י משחשיכה, נטלו מבע״י וקידש עליו היום ר׳ סימן בר תריי בשם ר׳ אושעי׳ אמר אם בארץ אסור לטלטלו ר׳ אושעי׳ אומר משרת הייתי לר׳ חייא הגדול והייתי מעלה לו חמין מדיוטא התחתונ׳ לדיוט׳ העליונ׳ ומחזירין לכירה וכו׳ ואיבעי׳ להו נמי התם תלאו ביתד הניחו ע״ג ספסל מהו. וזה מוכיח שפיר ״אפי׳ בשבת״ כן, שאפי׳ סילק מע״ש וקידש היום מותר להחזיר שנמצאת החזר׳ לבד בשבת דמתני׳ נוטלין ומחזירין בשבת קתני ולפיכך אסרו שהניחו ע״ג קרקע משום דדמי׳ למתחיל להניח ולבשל מאחר שסילקו מע״ש ונרא׳ לפי הירושל׳ הזה שאם סילק בשבת מותר להחזיר ואפי׳ הניח ע״ג קרקע:
}
\textblock{\textbf{אלא לאו לסמוך וכו׳.} יש בכאן מחלוקת, שיש מי שאומר כי היכי דבקש וגבבא תנור אסור בין לתוכו בין על גבו ובין לסמוך דהא תני׳ כוותי׳ ה״נ אם הוסק בגפת ועצים אסור בכל ענין אע״פ שגרוף וקטום ויש מי שאומר דדוקא בקש ובגבבא בלא גרוף וקטום דאמרי׳ נפיש הבלי׳ כתנור מדלקי ואתי לחתויי אבל גרוף וקטום אפי׳ בתנור מותר לשהות ולהחזיר ועיקר הדבר כדברי האוסר. וא״ת א״כ אביי אמאי לא מוקי לה למתני׳ בגרוף וקטום משום דקסבר דכירה גרופ׳ אע״פ שהדליק׳ בגפת ובעצים מותרת ואין דינה כתנור אלא בשאינה גרופי וקטומ׳ והוסק׳ בגפת ובעצים והיינו מתני׳ דקתני בקש ובגבבא הרי הוא ככירים ואין בין בקש ובגבבא לגפת ועצים בגרופה כלום. זהו דעתי בדבר זה, עד שראיתי לרבינו הגדול שאין זה דעתו שהוא אסר בכיר׳ אע״פ שגרף וקטם. כתב ר״ח ז״ל ודאביי ל״פ אדרב אדא בר אהבה כלל והאי דלא מוקי מתני׳ בהכי א״ל סתמא בכל ענין משמע להו דגפת ועצים בלא גרוף וקטום משמע ולהכי מהדר לאשכוחי התירא דכיר׳ בלא גרוף וקטום ואסור לתנור והיינו לסמוך. וכיון שסמכנו על פי׳ ר״ח ז״ל שאמר שאין זו הטמנה ועל דברי רבינו הגדול נמצינו למדים בשהייה דחיי לגמרי מותר דלצורך מ״ש הוא. בשול מעט עד שיתבשל ויצטמק כ״צ לגמרי דמכאן ואילך רע לו אסו׳. בגרוף וקטום בין היה לצורך שבת בין מבושלת קצת לעולם מותר. נהחזיר בלא גרוף וקטום אסור לעולם אפי׳ מצטמק ורע לו. בגרוף וקטום מותר. כמאכל ב״ד על גבה בתוכה אסור. פחות ממאכל ב״ד לעולם אסור להחזיר שלא הותר לבשל בתחלה בשבת בגרוף וקטום. ולענין הטמנה אין ספק שאסור להטמין אפי׳ כמאכל ב״ד ואפי׳ הגרופ׳ וקטומ׳ אפשר שהוא אסור ונראין הדברים שהוא מוסיף הבל באמת שאפי׳ בקטום לגמרי שהוא הולך ופוחת כל שמוציא חמימות מגופו ומחמם את הצונן דבר המוסיף מקרי וזה אפי׳ בחי גמור מבשל כ״ש שיחמם וכ״ש שהקטימ׳ האמור׳ בכ״מ אינה קטומ׳ לגמרי אלא מעט להיכר. אבל לפי פרש״י ז״ל מותר בגרופ׳ וקטומ׳ ואפשר שעליו סמכו במנהגי הטמנה שבמקומותינו והסמוכים לנו. אבל עיקר הדבר שאין הטמנותינו אלא בשהויי זה שבפ׳ זה לפי שאין משהין אותו ע״ג כירה ואינו יושב ואינו נוגע בגחלים אלא בכיסוי טיט או ברזל כמו שפי׳ ר״ת ז״ל ואע״פ שמכסין אותו בבגדים אין זו הטמנ׳ לגחלים אלא השהוי מותר לעצמו וההטמנה לעצמה שהוא דבר (שהוא) מוסיף ומבע״י ולא עוד אלא שההטמנ׳ כמתיר לשהויי דדמי קצת לתנורא דשריק ולא ניחא לחתויי בהו. והענין שפירשתי בתוכן דתנור וכירה דלא הוי הטמנ׳ בסיוע לי׳ הלכך בגרופ׳ וקטומ׳ שרי לפי דברינו ודברי רבינו הגדול ז״ל. ולפי דברי אחרים אפי׳ בשאינ׳ גרופה (ככל) [כ]מאכל ב״ד:
}
\newsection{דף לט}
\textblock{\textbf{רבה אמר גזרה שמא יטמין ברמץ.} לפי טעמ׳ דרבה מהראוי לאסור אף מבע״י כמו בכל דבר המוסיף הבל תדע דבמתני׳ קאמר מעשה דאנשי טבריא משום דדמי להטמנה ואסור אף מבע״י כדאמר רב חסדא בסמוך ואמנם לטעמי׳ דרב יוסף משום שמזיז עפר ממקומו אינו אסור אלא בשבת והא דלא קאמר הש״ס א״ב מבע״י וצ״ל דלא מיתסר אף לרבה מבע״י משום דחול ואבק דרכים שהוא חם בע״ש מן החמה בלילי שבת הוא מצטנן לגמרי וס״ס הו״ל בשבת דבר שאינו מוסיף. וא״ת, והא חול בעצמו דבר המוסיף הבל הוא כדתנן לקמן ולא בחול הא אתמר עלה בפ׳ לא יחפור חול דחמימי מחמם דקרורי מקרר הלכך מטמינין בו ביצה וכיוצא בו דמתקרר והולך הוא ואינו מתחמם אבל בשאר דברים המוסיפין לגמרי אין טומנין בהן אפי׳ צונן גמור מדר״ח בסמוך:
}
\textblock{\textbf{מתיבי רשבג״א מגלגלין ביצה ע״ג גג רותח.} משום דליכ׳ שמא יטמין ברמץ ולא משום שמא יזיז עפר ממקומו לית הלכת׳ כוותי׳ דאיהו כר׳ יוסי ס״ל אבל רבנן דאסרי בסודרין הה״נ דאסרי בגג רותח דתרוייהו תולדת חמה נינהו וטעמ׳ דשמא יטמין ברמץ ושמא יזיז עפר ממקומו לר״י צריכה לרבנן ל״צ להו ולפיכך השמיט׳ רבינו הגדול ז״ל אע״פ שכ׳ טעמייהו דרבה ורב יוסף לר׳ יוסי כדי לפרש משנתינו דמודה בה ר״י:
}
\textblock{\textbf{ממעשה שעשו אנשי טברי׳ בטלה הטמנ׳ בדבר המוסיף ואפי׳ מבע״י.} איכא דקשי׳ להו, פשיט׳ דאסור׳ הוא, דהא מתני׳ הוא בפ׳ במה טומנין, וא״ת משנתינו בשבת, א״כ [היאך] טומנין בכנפי יונה ומטלטלין אותן ותנן נמי בגיזי צמר ואין מטלטלין אותן ותניא בבריית׳ בשילהי במה טומנין ת״ר אע״פ שאמרו אין טומנין את החמין בדבר שאינו מוסיף אבל אם בא להוסיף מוסיף כיצד נוטל את הסדינין ומניח את הגלופקורין ומפרקי׳ דלמא הוה מוקמי׳ לי׳ בספק חשיכה ואליבא דר׳ דאמר כל שהוא משום שבות לא גזרו עליו בה״ש. ואפשר דה״ק, ממעשה שעשו אנשי טברי׳ בטלה שעל אותו מעשה גזרו לבטלה דומה למה שאמרו מעדותו של ר״י בן גודגדא נשנית משנה זו. ושמעינן מדר׳ חסדא דמטמין בדבר המוסיף אסור התבשיל אפי׳ דיעבד שלא תאמר אין טומנין אמרו לכתחל׳ אבל עבר והטמין מותר ומיהו דוקא בצונן שנתחמם או נצטמק ויפה לו אבל בעומד בחמימותו בשעה ראשונ׳ אין לנו ועולא דאמר הלכה כאנשי טבריא מודה לומר דלא דמי׳ להטמנה דהכא ליכא למיגזר משום חיתוי גחלים א״נ הטמנה מבע״י בתולדת חמה ל״ל שלא אמרו אלא בשבת עצמה א״נ ס״ל כרב יוסף דאפי׳ בשבת לא אסר אלא משום שמא יזיז עפר ממקומו (בתולדת חמה) הא לא״ה הטמנה עצמה מותר׳ לר״י בשבת בע״ש מיהת מותרות ולא גזרינן בהו משום רמץ וחיתוי גחלים ור״נ א״ל כבר תברינהו לסולנייהו וחזרו בהם וכיון דלא אהדר ליה עולא משמע שקבלה ממנו וכעין אתקפתא ותיובת׳ וכל כיוצא בה אין הלכה כמאן דאיתותב, וכ״כ ר״ח ורב אלפס ז״ל:
}
\textblock{האי דאמרי׳ \textbf{מאי רחיצה. אילימא רחיצת כל גופו.} לאו למימרא דמקצת גופו מותר דחמין שהוחמו בשבת הוו ואסרינהו אפי׳ בשתי׳ ואיסור הנא׳ הן לזמן אלא ה״ק אילימא רחיצת כל גופו נמי הוצרכו לאסור להן משום דהוי כחמין שהוחמו בשבת אטו הוחמו מע״ש מי שרי והתנן וכו׳ וכן כולה סוגי׳ דלהשתטף לומר שאסרום בשטיפה בין בשבת בין בי״ט מפני שהוחמו ביומן הא מבערב מותרות כר״ש ומיהו בשבת בשתיי׳ ובכל הנאה הן אסורים אפי׳ להושיט אצבעו קטנה בהן כנ״ל ודבר ברור הוא. וראיתי בתוס׳ בזה שבוש:
}
\textblock{הא דתנן \textbf{בש״א לא יחם אדם חמין לרגליו אא״כ ראויים לשתיי׳.} כגון שדעתו עליהם לשתות מהן קאמר שאם דעתו לרחיצה בלבד אע״פ שראויין לשתיי׳ אסורין דהא ב״ש לית להו מתוך שהותרה מלאכה לצורך הותרה נמי שלא לצורך וכ״ש דל״ל הואיל תדע דהא מתני׳ דחמי טבריא ראויין לשתיי׳ הן כדקתני ומותרין בשתיי׳ ואסורין ברחיצ׳ לב״ש וב״ה מתירין דעבדו הנאת הגוף דרחיצ׳ כאוכל נפש כדשרי מדורה להתחמם כנגדה במס׳ י״ט והוא שתהא הנאה דומי׳ דאכילה שוה לכל נפש כדאי׳ התם בכתובות. ומשמע נמי ודאי לב״ה רחיצה דכל הגוף נמי מותרת בי״ט וכ״ש הוא דהנאת כל הגוף דמיא לאכילה ושתיי׳ טפי מהנאת אבר אחד כדקס״ד לב״ש בפ״ב דביצה ודומיא דזיעה נמי דמותרת נמי בי״ט קודם גזירה ואפי׳ בהותם בי״ט כדאמרי׳ התם בפ׳ א״צ ואין עושין פחמין פשיטא למאי חזי לא נצרכא אלא למוסרן לאוליירין בו ביום ובו ביום מי שרי כדאמר רבא להזיע וקודם גזירה ה״נ להזיע וקודם גזירה אלמא קודם גזירה מותר לגמרי ואפי׳ להתם להן בי״ט. ויש כאן שאלה, כיון שרחיצת כל הגוף וזיעה כולן מותרין ואפי׳ להחם בי״ט למה גזרו כלל עליהן בי״ט ומאי איסורא אתי מינה וא״ת משום שבת אין ראוי לגזור י״ט אטו שבת בדבר של אוכל נפש וי״ל כיון דבי״ט נמי איכא בהו איסור כגון מדיח קרקע וסך א״נ סחיטה דאלונתית וכיוצא בהן עשו י״ט כשבת לכך ואינו מחוור. ובתוס׳ מפרשים רחיצת כל הגוף אינה צורך כל נפש ומן התורה הוא אסורה ודומיא דמוגמר הוא אבל פניו ידיו ורגליו צורך כל נפש הוא וכן זיעה וכיון דרחיצת כל הגוף אסורה מן התורה משום לתא דידה גזרו בהן וסומכין דבריהן מן הירושל׳ דבעי התם מותר לשתות ואסור לרחוץ ומתרץ אך אשר יאכל לכל נפש הוא לבדו יעשה לכם. ואלו דברי נביאות הן, שרחיצת כל הגוף תאסר וזיעה מותרת ורחיצת מקצת הגוף נמי מותר ואדרבה הנאת כל הגוף טפי צריכא וטפי שריא דומיא דמדורה כדפרישית. ועוד מדקתני הכא בברייתא ואצ״ל חמין שהוחמו בי״ט דאסורין ברחיצת כל גופו ואפי׳ אבר אבר משמע דאיסור דרבנן הוא דרישא (דהתם) [והתם] בירושל׳ שרי כל גופו אברים אברים בהוחמו בי״ט אליבא דרב אלמא רחיצת כל הגוף צורך כל נפש הוא דמה לי בבת אחת מה לי אבר אבר לצורך כל נפש ומ״ש שם משום דכתיב אשר יאכל לכל נפש וכו׳ טעמא דגזירה הוא לומר דלא משוינן אוכל נפש לשאר דברים א״נ בירושל׳ לא שרי שאר הנאות וכן הסוגיא שלהם מוכחת שם ואין לי להאריך בזה דלגמ׳ דילן הדבר פשוט הוא דכל רחיצה וזיעה מותרת ואפי׳ להחם (ול״כ) [ולא נראה] טעמא דמילתא:
}
\textblock{ה״ג וכ״כ בכל הנוסחאות: \textbf{א״ר חסדא מחלוקת בכלי אבל בקרקע ד״ה מותר, והא מעשה דאנשי טבריא בקרקע הוה ואסרינהו רבנן אלא אי איתמר וכו׳.} וה״פ: מחלוקת בכלי שר״מ ור״י אומרים אסור משום גזירת מרחץ לפי שהרואים אומרים היום הוחמו ואתי למיחם חמין בשבת כגון ליתן צונן בחמין כדי שיוחמו אבל בקרקע ד״ה מותר דלא מוכחא מילתא שהן חמי האור וליכא למיגזר. ואקשי׳ והא מעשה דאנשי טבריא בקרקע הוה ואסרי להו רבנן פי׳ רבנן דאמרן דרבנן דפליגי עני׳ דר״ש אסרו אפי׳ הוחמו מע״ש ומתני׳ ר״ש הוא אבל רבנן אסרו אפי׳ להשתטף ואפי׳ בקרקע וא״ת נהדר ונימא מתני׳ רבנן נמי הוא ובקרקע ד״ה כר״ח אה״נ אלא אנן מסיני׳ דשמעת׳ אקשי׳ דהא אסכימו רבנן דמתני׳ ר״ש ורב [בגמ׳ הגי׳ רב איקא] דהוא רבי׳ דר״ח אמרה ולא פליג ר״ח ארביה. והדרי׳ ואמרי׳ אלא אי איתמר הכי איתמר מחלוקת בקרקע דר״ש סבר בשטיפ׳ בקרקע ליכא משום גזירת מרחצאות (דבין) [דכיון] דבקרקע היא ולא עביד נמי אלא שטיפ׳ אינו נראה כמי שהוחמו בי״ט שאפי׳ בצונן אדם משתטף כמה פעמים ועוד שהוא בקרקע ונראה כצונן אבל בכלי ד״ה אסור אפי׳ שטיפה שהוא נראה כמיחם שפינהו עכשיו מן האש שכן דרך הרוחצים לעשות ואיכא למיגזר בי׳. ונוסחי בהך שמעתא משובשת טובא דאיכא דגריס לה כולה איפכא ולא סמכי׳ עני׳ משום דחזי׳ לגאונים ור׳ יצחק ז״ל דפסקו מחלוקת בקרקע אבל בכלי ד״ה אסור אלמא לישנא בתרא הכי. ואפשר דגרסי׳ מעיקרא מחלוקת בכלי אבל בקרקע (וכו׳) [דברי הכל אסור] כדכתיב במקצת נוסחי והיינו דאקשי׳ והא מעשה דאנשי טבריא בשבת הוה ואסרי להו רבנן להביא סילון של צונן באמבטי של חמין לשתיי׳ ורחיצ׳ אלמא אם אינן מביאין אותו דלא הוי חמין שהוחמו בשבת שרי אלא אי אתמר הכי אתמר מחלוקת בקרקע אבל בכלי ד״ה אסור כלומר איפכא אתמר, וצ״ע בספרי הגאונים ז״ל, ושוב פירשתי השמועה בטעמה ובגרסתה בספר מלחמות יפה יפה:
}
\newsection{דף מ}
\textblock{\textbf{דאי מכללא מאי דלמא ה״מ במתני׳ אבל ברייתא לא.} איכא דקשי׳ לי׳ מי אמר במשנתינו כל מקום קאמר דהכי אמרי׳ במס׳ מנחות בפ׳ הקומץ רבה אמר רב אשי אמר לי מר זוטרא קשי בה רב חנינא מסורא פשיטא מי קאמר במשנתינו כל מקום קאמר אלמא כל מקום משמע אפי׳ בברייתא וה״נ אמרי׳ בפ׳ שני דייני גזירות גבי כ״מ שאמר ר״ג רואה אני את דברי אדמון הלכה כמותו. והי׳ נ״ל לפרש דהכא סתם מתני׳ כר״ש ואר״י הלכה כסתם משנה הלכך אע״ג דאר״י בכ״מ הלכה כדברי המכריע וסמכי׳ עלה ודחי סתמא דמתני׳ מחמת מכריע כדאמר כ״מ ה״מ כשההכרעה במשנתינו דאתיא הכרעה דמתני׳ ודחי׳ סתמא דמתני׳ אבל כשההכרעה הוא בבריית׳ לא דחי׳ סתמא דמתני׳ מקמי הכרעה דברייתא דאם רבי שנאה סתם ולא הכריע ר׳ חייא מניין לו. אבל ראיתי שהגאונים אומרים אין הלכה כדברי המכריע בברייתא וכ״כ רבינו הגדול ז״ל בפ״ק דקדושין ואפשר שמה שאמרו מי קאמר במשנתינו אינו כלל ללמוד הימנו שהרי אפי׳ התם בכתובות איכא דבעי לי׳ מיבעיא ובמנחות נמי שלח לי׳ רב יימר לר״פ הא דאמר רבנן הלכה כר״ש בן שזורי ולא עוד אלא כ״מ ששנה ר״ש שזורי הלכה כמותו אף בנתערב לו טבל בחולין ואהדר לי׳ אין ולא א״ל מי קאמר במשנתינו ואע״ג דקשי בה ר׳ חנינא מסורא פשיטא אנן לאו אקושי׳ סמכי׳ אלא אעיקר מימרי׳ הלכך כיון דחזינן גבי הכרעה דאמרי׳ דוקא במתני׳ עלה סמכי׳ ומיהו כ״מ שאמרו כ״מ סתם אף בבריית׳ במשמע [עד שנראה] שאין סוגיין הגמ׳ כן. א״נ ה״ק דילמא ר׳ יוחנן במתני׳ דוקא אמרה בפי׳ דלא שמיעה להו מימרא בלישנא דוקא. וא״ת קשי׳ הך, היכי פסק ר׳ יוחנן כר׳ יהודה הא איהו אמר כסתם משנה ומתני׳ ר״ש הוא א״ל (דאי לא) מדתנן לקמן בפ׳ חביות הרוחץ במי מערה במי טברי׳ מסתפג אפי׳ בעשר אלונטאות ולא יביאם בידו ור׳ יוחנן מתני לה להא כבן (בינאי ודייקינן עלה בגמ׳ קתני מי מערה דומיא דמי טבריא מה מי טברי׳ בחמין אף מי מערה בחמין הרוחץ דיעבד אין לכתחלה לא מכלל דלהשתטף אפי׳ לכתחל׳ מני ר״ש הוא וכיון דההיא אתי׳ כר״ש ותני לה בן (בינאי) הו״ל מתני׳ דהכא סתם ואח״כ מחלוקת ואין הלכה כסתם כנ״ל:
}
\textblock{\textbf{ואצ״ל חמין שהוחמו בי״ט.} (א״כ) פי׳ אבל הוחמו מעי״ט לא הי׳ צריך (שום) הוחמו בי״ט. ורבינו הגדול ז״ל התיר בשם גאון ז״ל לרחוץ בהן כל גופו, וכן האמת:
}
\textblock{הא דאמרי׳ \textbf{ועדיין היו רוחצין בחמין ואומרים מזיעין אנו אסרו להם.} נראה שלא אסרוה בבת אחת אלא מתחלה אסרוה בשבת והתירוה בי״ט לפי (שאין) [שאף] החמין מותרין לרחוץ פניו ידיו ורגליו ואח״כ אסרוה בי״ט ממאי דקתני בברייתא פקקו נקביו מעי״ט למחר נכנס ומזיע אלמא בשבת אסור וי״ל דה״ה לשבת והאי דקתני בי״ט לאשמעי׳ שאפי׳ בי״ט זיעה דוקא א״נ משום דאין דרך להזיע בלא שטוף חמין בסוף ובשבת אסור להכי נקיט י״ט, ולשון ראשון אינו הגון:
}
\textblock{\textbf{ובלבד שלא ישתטף ויתחמם כנגד המדורה מפני שמפשיר מים שעליו.} פי׳ משום גזיר׳ מרחצאות נגעו בה שנמצא כרוחץ במים פושרין אבל משום הפשרה עצמה מותרת שהרי שנינו אבל נותן לתוכו מים מרובין כדי להפשירן ותנ״ה לא בשביל שיחמו אלא שתפיג צינתן דהיינו הפשר כדמוכח שמעתא בבעי׳ בשמן (מי לא אמר ת״ק) [מה לי אמר ת״ק]. ודקאמרי׳ שמן אע״פ שהיד סולדת בו מותר כלומר אינו דומה למים פושרין דשרי׳ אין היד סולדת בו דהיינו הפשר ואסרי׳ יד סולדת בו וכן כולה שמועה מתפרשת ור״י אמר שמואל שרי בתרווייהו אין יד סולדת בהן דהיינו הפשר וש״מ דהפשר אפי׳ כנגד המדורה ממש מותר ועוד מדר״י נשמע לרבנן דכיון דסבר ר״י בשמן הפשרו לא זהו בשלו שרי לי׳ כנגד המדורה ממש. ובס׳ התרומה מתבלבלין בדבר זה מן הירושל׳ שיש בסוגי׳ זו דברים סותרין גמ׳ שלנו ואין לי להאריך. ומסתברא דלאו דוקא כנגד המדור׳ אלא על גבה ממש כדתנן במיחם אבל נותן לתוכן מים מרובין וכו׳ ודא ודא אחת היא. 
}
\textblock{\textbf{לאפרושי מאיסורא שאני.} שמעי׳ (מדקא) [מהא ד]שרי לי׳ לאיניש למימר לי׳ בבית המרחץ עביד לי הכי והכי ואע״ג דממילא הויא הוראה ובלבד שלא יאמר דרך הוראה (דהכא) והכא דהוה מצי למימר לי׳ לא תתהפך באמבטי והופרש מן האיסור וא״ל טול בכלי שני ותן וכי הוה ק״ל בגמ׳ אמאי עביד הכי משום שמנעו מלהביא באמבטי הוא וכדמייתי לה תדע ממעשה דר״מ. ואפשר שלא הותר אלא להפריש מאיסורא דרך אותה הוראה דכיון דאית לי׳ לאפרושי מאיסורא מצי למימר מילתא אגב אורחי׳ ומיהו הכירא הוא דשרי וא״ת היאך עלה על דעת שיניחנו לעשות איסור ולא יאמר לא תעשה א״ל משום דהוי מצי לי׳ למימר כמי שאינו רוצה בדבר זה ולא שימנענו בשביל איסור. ובירושל׳ התירו אפי׳ לשאול. וגרס התם שואלין הלכות המרחץ בבית המרחץ והלכות בית הכסא בבה״כ כחדא רשב״י על מסחי עם ר״מ א״ל מה שנדיח א״ל אסור מהו שנקנח א״ל אסור ולא כן שאל שמואל לרב מהו לענות אמן במקום המטונף א״ל אסור ואסור דאמרית לך אסור (אשבת) [אשכח] תני שואלין הלכות בית המרחץ בבית המרחץ והלכות בית הכסא בבה״כ ומסתמא בגמ׳ דילן לא שרי אלא לאפרושי מאיסורא ומעשה דר״מ נמי לאו הכי הוה כדמייתי לה בירושלמי אלא כדאמרינן הכא ביקש להדיח וכו׳:
}
\newsection{דף מא}
\textblock{הא דתנן \textbf{מוליאר הגרוף שותין ממנו בשבת.} מקשי עלה בירושלמי הא אם אינו גרוף אין שותין ממנו פי׳ והא חמין מצטמק ורע להן ושרי לשהויינהו מכיון שהוחמו כ״צ ועוד קטום אמאי לא מהני בי׳ ופריק א״ר חמא ברי׳ דר׳ הלל מפני שהרוח נכנסה לתוכה והגחלים בוערות א״ר יוסי בר׳ בון מפני שהוא עשוי פרקין פרקין והוא ירא שלא יתאכל דבקי והוא מוסיף מים אנטוכי אע״פ שהן גרופים אין שותין הימנה ר׳ חנינא ר׳ יוסי אבא בר בר חנא בשם ר״י מפני שהיא מתחממת מכותלי רבנן דקסרין בשם ר׳ הונא אמר אם היתה גרופה ופתוחה מותר:
}
\textblock{הא דאמר אביי \textbf{ומיחם שפינה ממנו מים לא יתן לתוכו מים כל עיקר מפני שהוא מצרף ור״י הוא.} ק״ל, ל״ל למימר הכי, והלא דבר זה אינו בכלל לשון משנתינו כל עיקר דהא לא קתני אלא המיחם שפינהו לא יתן לתוכו צונן דהיינו מועטין כדי שיחמו אבל נותן לתוכו מים מרובין כדי להפשירן והבו דלא לוסיף עלה. ואם באנו לפרש דה״ק ואפי׳ כר״י מצית לאוקמי קשה לשון הגמ׳ עלינו דמשמע דכר״י מוקי לה דוקא ועוד ל״ל התוס׳ כי היכי דתיתי דוקא כר״י הו״ל למישבקי׳ כדאי׳ ואתי׳ כד״ה. ואפשר מפני ששנינו כדי להפשירן משמע דוקא שיעור להפשירן אבל מרובין כ״כ שיצננו למים שבתוכו ויצרפו אותו כדאמר רב אבל שיעור לצרף אסור מש״ה קאמר הא פינה ממנו (מי) לא יתן וכו׳ א״נ מדקתני לה במיחם שפינהו ולא קתני שפינה ממנו מים:
}
\newsection{דף מב}
\textblock{\textbf{מכבין גחלת של מתכת.} פרש״י ז״ל, דלא שייך כיבוי בהכי מדאורייתא ומדרבנן הוא דאסור והיכי דאיכא נזקא דרבים לא גזרו על השבות ואפשר שכל צירוף דרבנן ואפשר שלא אמרו כן אלא בגחלת דלא שייך צירוף אלא בכלים או ברוצה לעשות כלי משום דהוה מתקן אבל בגחלת לאו מתקן הוא ומדאוריית׳ שרי. והא דאמרי׳ במס׳ יומא פ׳ א״ל הממונה אבל הכא צירוף דרבנן הוא ה״נ קאמר דצירוף עששיות דרבנן דהא לאו לתקוני כלי הוא עושה ואפי׳ לר״י דאמר דבר שאינו מתכוין אסור הכא לאו תיקון הוא כלל אלא בכלי וכיוצא בו ודמי׳ למאי דאמרי׳ לא יקטמנו לחצוץ בו שיניו ואם קטם חייב חטאת ואם לא קטם לחצוץ בו שיניו פטור אפי׳ לר״י דכל משום תקוני מנא הוא מאן דלא מתקן פטור. אבל בה״ג מצאתי גבי גחלת של עץ לית בה היזק רבים מ״ט כמה דלא כביא אית בה סומקא וקא חזו לה ולא איתזק בה אבל גחלת של מתכת אע״ג דאזיל סומקא קליא ולא חזו לה ואתו לאיתזוקי בה ור״י לית לי׳ היזק דרבים ושמואל במידי דאית בי׳ היזק רבים פליג עלי׳ ואע״ג דבעלמ׳ מלאכה שאצ״ל כותי׳ ס״ל בגחלת של מתכת דאית בה היזק רבים פליג עלי׳ וכן בצידת נחש דאיכא היזק שרי שמואל וכ״כ ר״ת כדבריו. ותימה הוא, איך אנו מתירים מלאכה גמור׳ משום היזק שלא במקום סכנת נפשות ושמא כל היזק של רבים כסכנת נפשות חשיב לי׳ שמואל אבל אינו נכון שא״כ מנ״ל דר״י גופי׳ לא מודה בהא וי״ל עקרב שלא תשוך היזק דרבים הוא מ״מ מדבריהם למדנו שצירוף דאוריית׳ וקשיא ההוא דיומא ואפשר שאין בעששיות צירוף כמו בגחלת לפי שהן חמות הרבה ומפשירות המים ואין מצרפין לגמרי אלא במים צוננין ואי קשי׳ הא דאמרי׳ בפר״א דמילה ממתקן את החרדל בגחלת ואוקימנ׳ בשל מתכת אלמא ליכא צירוף גמור במתכת א״ל שאני חרדל שאינו מצרף ואין צירוף אלא במים:
}
\textblock{\textbf{אבל נותן לתוך הכוס כדי להפשירן.} למאי דס״ד מעיקר׳ דספל כאמבטי דאע״פ שהוא כלי שני מבשל הוא במים דוקא כדי להפשירן הוא דמותר אפי׳ בכוס ולמאי דקאמרי׳ השתא דספלי מותר דכל כלי שני אינו מבשל כדי להפשירן דקתני משום דכל כלי שני אינו כהפשר א״נ משום מיחם קתני הכי. ומשמע בגמ׳ דבמסקנ׳ ספל אינו כאמבטי ובין צונן לתוך חמין ובין חמין לתוך צונן מותר וכן בכל כלי שני אבל אמבטי בין צונן לתוך חמין או חמין לתוך צונן אסור כדא״ר נחמן הלכה כר״ש בן מנסיא סיפא כפשטא קאי. אבל רבינו ור״ח ז״ל אמרו דר״ש בן מנסיא ארישא פליג, ולית הלכתא כוותי׳ מדרבא דהוא בתרא ומעשה רב ואינו מחוור לי דהא למאי דס״ד מעיקרא איתמר אלא שהם ז״ל סבורין דכיון דאמרי׳ בגמ׳ מי סברת אסיפא פליג כמתמיה ש״מ בסברא דקושטא איתמר דארישא קאי. דכ״ע הא דשרי במיחס כדי הפשר ובאמבטי אסור לעולם דסתמא קתני צונן לתוך חמין אסור משום דסתם אמבטי לחמם הוא והוא הטעם לספל למאי דס״ד ספל הרי הוא כאמבטי ולהכי נמי קתני אמבטי (אלא) [ולא] קתני כלי ראשון דהתם להפשיר מיהת מותר:
}
\textblock{\textbf{נשברה לו חביות בראש גגו.} פרש״י ז״ל חביות של טבל שאינה ראוי׳ לטלטל ואין זה נכון חדא דא״כ ליתני של טבל כדקתני במתני׳ דלקמן ועוד דקתני סיפ׳ נזדמנה לו אורחים מביא כלי א׳ וקולט כלי אחר ויצטרף ולא יקלוט ואח״כ יזמן ואי בטבל אורחים מאי בעי לי׳ א״ו במעושר וכי אמר הצלה שאינה מצויה לא התירו אפי׳ בדבר הניטל בשבת קאמר תדע מדאקשי׳ דלף ולא מפרקינן בדלף הראוי כדלקמן ש״מ אפי׳ בדבר הראוי לא התירו הצלה שאינה מצוי׳ כך מפורש בתוס׳. ותימה היא (לדברי) [לדבר] הראוי אמאי לא, וא״ל משום דבכל הצלה איכא למיחש לההוא דאמרי׳ בפ׳ כל כתבי מתוך שאדם בהול על ממונו אי שרית לי׳ אתי לאתויי כלים דרך רה״ר ואע״פ שמשום איבוד ממון התירו בכלי א׳ ואי אמרת לא חיישי רבנן להצלה שאינה מצויה הי׳ להם לאסור הכל משום גזירה דשמא יביא כלי כנ״ל ומשמע דרבה ל״ל משום ביטול כלי מהיכנו כלל דאי אית לי׳ אפי׳ עשוי להטיל ביצתה במקום מדרון נמי אסור ורבינו הגדול ז״ל פסק כרב יוסף משום דסוגיין בכולהי מכילת׳ דבטל מהיכנו אסורהוא והיינו כטעמי׳ דרב יוסף:
}
\newsection{דף מג}
\textblock{\textbf{איתבי׳ כל הני תיובתא ומשני לי׳ בצריך למקומן.} פי׳ בר מדלף, דמתוקם לי׳ בדלף הראוי. א״נ ניחא לי׳ לתרוצינהו בכלל בצריך למקומן:
}
\textblock{\textbf{מני אי ר״ש ל״ל מוקצה.} פי׳ ואמאי אמרת בשחשב עליהן והוה מצי למימר לי׳ הדרי ומתוקמא כר״ש,ואתי׳ בריית׳ כפשטי׳, אלא יגדיל תורה:
}
\textblock{\textbf{אי ר׳ יהודה כי לא מכוין מאי הוי.} ואי קשיא לר״ש נמי הא פסיק רישי׳ ולא ימות הוא א״ל בשלמא לר״ש הכי קתני יפרוש בענין שהן יכולין לברוח ואם ברחו ברחו ואם נצודו נצודו אלא לר׳ יהודה לעולם אסור והא דלא אוקמא בהא נמי בצריך למקומן כדאוקימנ׳ אחרנייתא משום דהתם כיון דצריך לטלטל ואינו עושה מעשה בהנחתם תחת הקורה וכיוצא בה מותר אבל (בפירש) [בפורש] הטלטול לחוד והפרישה לחוד, ואין זאת מתרת זאת:
}
\textblock{\textbf{אתמר מת המוטל בחמה, אמר רב יהודה אמר שמואל הופכו ממטה למטה, ורב חנניא בר שנמיא משמי׳ דרב אמר מניח עליו ככר או תינוק ומטלטלו היכא דאיכא ככר או תינוק דכ״ע ל״פ דמותר כי פליגי דליכא, מר סבר טלטול מן הצד לא שמי׳ טלטול ומ״ס טלטול מן הצד שמיה טלטול. לימא כתנאי אין מצילין את המת מפני הדליקה. אמר ר׳ יהודה בן לקיש שמעתי שמצילין את המת מפני הדליק׳ היכי דמי אי דאיכא ככר או תינוק מ״ט דת״ק דאסר. ואי דליכא ככר או תינוק מ״ט דר״י בן לקיש, אלא לאו בטלטול מן הצד פליגי. נא דכ״ע שמי׳ טלטול, והיינו טעמא דר״י בן לקיש מתוך שאדם בהול על מתו. אי לא שרית לי׳ טלטול דרבנן. אתי לידי כבוי דאורייתא, אמר רב יהודה בר שילא אמר ר׳ יוחנן הלכה כר״י בן לקיש במת.} והלכתא כרב דאמר מניח עליו ככר או תינוק ומטלטלו. ואי ליכא ככר או תינוק. אין מטלטלין אותו אפילו ממטה למטה. דטלטול מן הצד במת אסור. חדא דרב ושמואל הלכה כרב באיסורי. ועוד דהא מסקינן בגמרא כותי׳ דכ״ע כילטול מה״ל שמי׳ טלטול. ות״ק אין מצילין אפי׳ על ידי טלטול מן הצד קאמר. ור״י בן לקיש לא פליג אלא בדליקה ור׳ יוחנן נמי דפסק כריב״ל במת הכי ס״ל כרב משום דאדם בהול על מתו. ולאו התירא דטלטול מה״צ אתא לאשמעינן הכא הילכך הלכתא כרב דמוקים לי׳ בגמרא לדברי הכל. וכן פסק רבינו הגדול בהלכות כרב. ולענין דליק׳ ריב״ל אפילו טלטול גמור שרי. דלא ליתי לידי כיבוי והלכתא כוותי׳. וכן פירש ר״ש, אבל דברי רבינו יצחק אלפסי ז״ל מטין. שלא התיר ריב״ל אלא טלטול מן הצד, ואין הענין מחוור כן:
}
\textblock{\textbf{מר סבר טלטול מן הצד שמי׳ טלטול.} ק׳ להו לרבוותא ז״ל ורב ל״ל הא דתנן בפ׳ טומנין בגיזי צמר ואין מטלטלין אותן כיצד הוא עושה נוטל את הכיסוי והן נופלות ותנן נמי בפ׳ נוטל האבן שע״פ החביות מטה על צדה והיא נופלת מעות שעל הכר נוטל את הכר והן נופלות ואמרי׳ נמי בפ׳ כל כתבי הנר שעל הטבלא מנער את הטבלא ולימא תיובתא דרב מהני. וכ״ת אמר לך רב אנא דאמרי כת״ק דר״י בן לקיש הא לאו מילתא הוא דמיני׳ ליכא לאוכוחי דסבר דטלטול מן הצד שמי׳ טלטול ודילמא מאי אין מצילין דאסר להדיא ולאפוקי מדר״י בן לקיש דשרי להדיא משום שאדם בהול על מתו ואי לא שרית לי׳ אתי לכבויי והא דאמר דטלטול מן הצד שמי׳ טלטול לאקומי לדרב כד״ה אמרי׳ הכי ולומר דאפי׳ מן הצד אסר ת״ק ולאו משום דמוכחא מלתא הכי. וכ״ת רב דאמר כת״ק דר״א בן תדאי וכדגרסי׳ בר״פ כל הכלי׳ אר״נ הלכה כראב״ת למימרא ר״נ דטלטול מן הצד לא שמי׳ טלטול ומשמע לך מינה דת״ק דר״א סבר שמה טלטול זו אינה תורה שאלו היה ת״ק סובר שמה טלטול לא היה מתיר במקצתו מגולה דהא איכא נמי טלטול מן הצד ואכתי רב דאמר כמאן הא הנך מתני׳ הוי תיובת׳ ועוד אמאי לא מקשי מינייהו לרב בגמ׳ ולימא אנא כי הא ס״ל ועוד דקשי׳ דרב אדרב דאמרי׳ בפ׳ אין תולין אמרי בי׳ רב תנינא הקש שע״ג המטה לא ינענענו בידו אבל בגופו מנענעו ואם היה מאכל בהמה או שהי׳ עליו כר או סדין מנענעו בידו ש״מ טלטול מן הצד לא שמי׳ טלטול ש״מ ובי רב תלמידו דרב הוו ומיניה דרב שמעו וכעובדי דרב עבדי כדאמרי׳ בעלמא אלו מקילין לעצמן ואלו מקילין לעצמ׳ וכדאמרי׳ לקמן לאו משום דרב אסי תלמיד דר״י הוי ור׳ יוחנן כר״י ס״ל ואמרי׳ נמי והא רב הונא תלמידו דרב, ורב כר״י ס״ל. ותירץ רבינו הגדול ז״ל להאי קושיא ואמר תרי גווני טלטול מן הצד הוה חד באוכלין ודבר הראוי שהוא עם דבר שאינו ראוי כראב״ת וחד כגון טלטול דמת שלא כדרכו ורב סבר דאסור, והלכת׳ כוותי׳:
}
\newsection{דף מד}
\textblock{הא דאמרי׳ \textbf{הלכה כר״י בן לקיש במת.} למעוטי ממון קאמרי׳, כלומר דלענין ממון ל״א מתוך שאדם בהול על ממונו אי לא שרית לי׳ אתי לכבויי אלא אדרבה אמרי׳ אי׳ שרית לי׳ אתי לכבויי כדאמרי׳ במכלתין נקמן דהתם כיון דמדינא מציל כדבעי אי שרית ליה לגמרי שוכח עיקר שבת הוא הלכך מציל בהני גווני דשרי לי׳ כדי שיהא נזכר הכא דאסור אי לא שרית לי׳ כלל מתו נשרף בפניו ואתי לכבויי וכה״ג נמי אמרי׳ בממון לקמן בפ׳ במה אשה גבי מערימין בדליק׳ אי לא שרית לי׳ אתי לכבויי אלמא תקנת חכמים שלא יתירו לו לגמרי שמא ישכח עיקר שבת ושיתירו לו של הצלות שלא יעבור ויכבה א״נ י״ל כיון שאדם בהול על מתו יותר מכל דבר אינו יודע לעכב עצמו כלל מלהצילו בכל ענין שיציל:
}
\textblock{\textbf{התם קערה דומי׳ דנר.} ואע״ג דלשמן שבה (כגחלים) [כגדולים] דמי׳ דאפשר שלא תכבה עד שיכלה שמן שבו כיון דיהיב דעתי׳ אימתי תכבה משום נר עצמו לשמן נמי יהיב דעתי׳ אבל ברברבי לגמרי מקצה להו מדעתי׳:
}
\textblock{\textbf{פמוט שהדליקו בשבת לדברי המתיר אסור לדברי האוסר מתיר.} פרש״י ז״ל המתיר ר״מ דלית לי׳ מוקצה מחמת מיאוס ובהא מודה לפי שהוא מוקצה מחמת איסור ולפי פי׳ זה אין ר״ש מודה בפמוט לאיסור דאיהו לית ליה מוקצה אפי׳ מחמת איסור לפיכך אין הפי׳ נכון דקאמרי׳ כלישנא בתרא דברי הכל לא יצא ר״ש מן הכלל. ור״ח ז״ל כתב לדברי המתיר היינו ר״ש אוסר ובפמוט של מתכת שהוא גדול וכן מצאתי בתוס׳ הצרפתים בשם ר״ת ז״ל דפמוט דמי לכוס וקערה ועששית ולפיכך אמרו לדברי המתיר שהוא ר״ש שהוא מתיר בכל איסור דמוקצה מחמת איסור כגון זה לדברי האוסר שהוא ר״מ ור״י מתיר שר״י לית ליה אלא מוקצה מחמת מיאוס ר״מ לית ליה אלא מוקצה מחמת איסור ומחמת מיאוס כגון נר שהדליקו בו בשבת ואקשי׳ למימרא כר״י וכו׳ אלא אי איתמר הכי איתמר פמוט שהדליקו עליו באותה שבת דברי הכל אסור דר״י ור״מ נמי אית להו מוקצה מחמת איסור לחודי׳ ואפי׳ אינו מחמת מיאוס כלל ור״ש נמי בכגון זה מודה דדמי לכוס וקערה ועששי׳ כדפרישי׳ ואין צורך לפרש כן אלא ר״מ ור״ש מתירין הן ותרוייהו בכלל המתיר, ור״י הוא האוסר:
}
\textblock{\textbf{ומה נר דלהכי עבידא כי לא אדליק בהו שרי.} פי׳ רש״י ז״ל דלא מיתסר בהזמנה מדלא קתני אבל לא את המיוחד לכך וק״ל דילמא היא גופה קא משמע לן דישן אף על פי שלא יחדו אסור דמוקצה מחמת מיאוס הוא ואסור (דבנר) [דהא נר] של מתכת שרי ר״י אפי׳ ישן ופרש״י בשלא ייחדו שאלו ייחדו לכך אף על פי שאין בו מיאוס אסור דהו״ל מטה שייחדוהו והניח עלי׳ דאסורה אם כן של חרס ישן דאסרי׳ ליה מחמת מיאוס, ע״כ בשלא ייחדו הוא. ובתוס׳ ראיתי שמפרשים, דנר של חרס עשוי להדליק בה בכל יום אבל נר של מתכת אינו עשוי להדליק בה אלא לפעמים לפיכך אף על פי שהדליקו עליו מטלטלין אותן אלא אם כן הדליקו אותה באותה שבת ולא דמי למטה שייחדה למעות שאם הניח בה אסור לטלטל מפני שהיא עשוי להניח בה מעות כל שעה ודמי כמי שהניחו מעות שם בשבת משא״כ בנר של מתכת שאין מדליקין בה אלא לפעמים אף על פי שהוא מיוחד לכך. ואף זה ק״ל דמ״מ של חרס מיוחד הוא לכך וישן בלא מיאוס (מיתסר) [ליתסר]. ול״נ, שמטה שייחדה למעות לאו משום מוקצה מחמת איסור נאסרה דהא כלים שמלאכתן לאיסור מותרין בין לצורך גופן בין לצורך מקומן אלא מתוך שאדם מקצה דעתו מהן מישובה שלה ואין אדם משנה אותה לשום תשמיש אחר בטל תורת כלי ממנה וזו היא הקצאה שלה ולדעת ר״י אתמר דומיא דקנה של תרנגולים לקמן בשמעתין וקאמר רב מעיקרא ר״י אוסרה אלמא לענין מוקצה הזמנה כמעשה וכל שאלו עשה בה מעשה נאסרה במוקצה מחשבתו אוסרתו. ולהכי אקשי׳, אם כן נר של חרס חדש כיון דישן מוקצה מחמת מיאוס הוא הזמנה שלו תחשב כמעשה דכל נר מיוחד ועשוי להדלק׳ הוא ומעתה הוא מסלק דעתו ממנו ואין דעתו עליו לשום תשמיש אחר כיון דאינו ראוי לתשמיש אחר עם תשמישו המיוחד לו כמטה ודחי׳ לה דבעי׳ במטה עד שייחדנה ויעשה מעשה ותבטל ממש מתורת כלי מעתה נר של מתכת ישן מותר אפי׳ מיוחד דהא חזי לכמה מילי ושל חרס מיאוסו אוסרו כך נראה לי, ודבר נכון הוא:
}
\textblock{\textbf{מוכני.} פרש״י ז״ל דגבי שידה תנן במס׳ כלים והיא עגלה של עץ מוקפת מחיצות למרכב ופי׳ מוכני אופן אין חיבור לה דחשוב כני לעצמה ואם נטמאת השידה כגון שאינה מחזקת מ׳ סאה דבת קבוני טומאה הוא לא נטמאת המוכני ואם נטמאת המוכני לא נטמאת השידה. ואין דבריו הללו נכונים, דא״כ מוכני פשוטי כלי עץ הוא ואינה מקבלת טומאה לעצמה ועוד שאין אופן עשוי להניח עליו מעות ולא ראוי לכך. ועוד הקשו עליו בתוס׳ שאם השידה עשוי למרכב אנשים כמו שאמר למה נמדדת שהרי כל הראוי למדרס טמא אע״פ שהוא מחזיק מ׳ סאה בלח שהם כוריים ביבש ולא גמרי׳ משק דהנך חזו למדרסות כדאמרי׳ במס׳ בכורות בפ׳ אלו מומין וכן פרש״י ז״ל עצמו בפ׳ במה אשה שלא הוזכרה טהרת פשוטי כלי עץ אלא לטומאת מגע ואוהל משום דאיתקש לשק והא נמי משום הקישא דשק הוא דבעי׳ מטלטל מלא וריקן וכיון שטמאה מדרס מטמא טמא מת ועוד דכיון שמקבלת שום טומאה אינה מצלת באהל כדאי׳ בפרק אר״ע לקמן. וכן מה שפי׳ באינה מצלת עמה באהל המת כגון שגובה המוכני מגין על הכלים שלמעלה מדפני השידה ואין מצלת משום שהמוכני כלי לעצמו ומקבל טומאה אע״פ שהשידה גדולה כ״ז אינו מחוור שהרי המוכני ראוי׳ להציל בעצמה לפי שהוא פשוטי כלי עץ ואינה מקבלת טומאה. ור״ח ז״ל פי׳ מלשון את הכיור ואת כנו וכתיב ואת המכונות ות״י בסיסייא וכך פירושה שכני קיבול הוא אינה חיבור לה לפיכך אינה מקבלת טומאה עמה ואין מצלת עמה באהל המת פי׳ בשהיו שניהם בבית שהמת בתוכו ואע״פ שגג האוהל מאהיל על הכל מה שבתוך השידה טהור לפי שהשידה מכוסה מלמעל׳. והכי משמע במס׳ נזיר דאמרי׳ הנכנס לארץ העמים בשידה תיבה ומגדל טהור דלא גזרי׳ שמא יוציא את ראשו לפי שהוא מכוסה מלמעל׳ בקרון או בספינ׳ טמא משום שמא יוציא את ראשו שהן מגולין הלכך שידה גדולה שאינ׳ מקבלת טומאה היא עצמה ודאי מצלת על כלים שבתוכה אפי׳ בתוך אהל של מת אבל מוכני בזמן שהיא נשמטת אינה מצלת שאם היתה השידה נקובה למטה בפותח טפח ומוכני סותם הנקב או ממעטו מפותח טפח טמאין לפי שהמוכני מקבל טומאה ואינו חוצץ ואם אינה נשמטת שאינה עשוי׳ לינטל משם לעולם מצלת לפי שהיא טהורה עמה. וי״מ בין טמאה בין טהורה נשמטת ועשוי׳ לינטל משם אינה מצלת דהו״ל כפתוח ואינו ממעט בשיעורו טומאה ואי ק״ל הא דאמרי׳ בפ׳ לא יחפור וחביות גופה תיחוץ ואע״פ שהן עתידין נינטל משם ש״ה שהוא סותם כל הנקב אבל הכא בסותם מקצת הנקב אינו ממעט ומיירי שאין הטומאה כנגד מה שנשאר מן הנקב (אנ) כנגד הסתום או בשאר השידה שאם היה הטומאה כנגד הנקב ממש אע״פ שאין שם פותח טפח טמא. ואיכא דמפרשי לה אפי׳ בסותם כל הנקב שאין דבר העתיד לינטל חוצץ בפני הטומאה לעולם והתם בשבטלה לחביות ואע״ג דאמרי׳ במס׳ מגילה לישקלוה לתיבותא ולנחה התם והו״ל כלי העשוי לנחת וחוצץ לפני הטומאה ש״ה שמתוך שהוא עשוי לנחת כמי שאינו נשמט דמי שהרי אינו מטלטל מלא וריקן והרי שמו מוכיח עליו שלפיכך קראוהו כלי העשוי לנחת ואלו סברותיהן ושם במקומה אפרש מה שחדשו בו הצרפתים ז״ל בתוס׳ בס״ד. וזה הפי׳ שכתבנו במוכני הוא הנכון ודרך ארץ להניח מעות עליו. וק״ל אמתני׳ בזמן שאינ׳ נשמטת גוררין אותה ואע״פ שאינה נשמטת גוררין אותה ואע״פ שיש עליו מעות והרי נעשה בסיס לדבר האסור. ובשם ה״ר אפרים ז״ל אמרו משכחת לה בשיש פירות בשידה או בגדים שסתמא כך הוא ומעות במוכני ובזמן שאינה נשמטת ה״ל כלי א׳ ומותר לגוררה כדתנן נוטל אדם כלכלה של פירות והאבן בתוכה ויפה פי׳. ורבותינו הצרפתים ז״ל מפרשים בתוס׳ דבזמן שהיא נשמטת נעשית מוכני שהוא כלי בפני עצמו בסיס לדבר האסור אבל בזמן שאינה נשמטת כוון שהשידה עיקר הכל מותר:
}
\textblock{\textbf{ואע״ג דהוה עלה כל בהש״מ,} וק״ל ולימא ליה מתני׳ בשוכח ובזמן שיש עליו מעות אסור נפלו מותר ואף על גב דהוי עלי׳ בה״ש והכי איתא לקמן בפ׳ נוטל ול״ק דאי בשוכח הא אמרי׳ התם ל״ש אלא לצורך גופן אבל לצורך מקומו מטלטלין ועודן עליו ואפי׳ לצורך גופו נמי בשאי אפשר לו מטלטלן ועודן עליו והכי מוכחא שמעתא התם והכא קתני אין גוררין אותה בשבת דמשמע כלל כלל לא:
}
\newsection{דף מה}
\textblock{\textbf{ה״נ מסתברא דרב כר״י סבירא ליה וכו׳. ק׳ ל״ל סברא אחריתא מהא ש״מ דקאמר יחדה והניח עליה מעות אסור לטלטלה (ואלא) לר״ש ל״ל וניחא אי מהתם ה״א אפי׳ ר״ש מודה דה״ל מוקצה מחמת חסרון כיס זה תירצו בתוס׳.} ול״נ כיון דהך מימרא תירוצא הוא דקאמרי אלא אי איתמר הכי איתמר בעי לאתויי ראי׳ ממקום אחר: }
\textblock{\textbf{אין מוקצה לר״ש אלא שמן שבנר.} פי׳ שמן המטפטף בשעה שהוא דולק שאסור דר״ש ל״ל לדר״י כדאמרי׳ לעיל ואלו שמן שבנר ממש המסתפק ממנו חייב הוא משום מכבה ולטלטלו נמי אסור משום דנעשה בסיס לדבר האסור כדלקמן אלא המטפטף משמן קאמרי׳ דהשתא לאו בסיס הוא אלא שהקצהו לאיסורו ולמצותו ולא הי׳ דעתו עליו כלל עד שיכבה. וא״ת והא אית לי׳ נמי גרוגרת וצימוקין לא דמו לחטים שזרען דהתם אינן דחוין כלל. ואי קשי׳ מאי קמבעי׳ ליה והא נר שכבה לר״ש מותר אע״פ שדחאו בידים שהדליקו לא תקשה דהתם כיון שכבה חזר ונראה אבל חטים שזרען ובקרקע עדיין כדחויין עומדין ולפיכך דמו אותן לשמן המטפטף דמכל מקום כל זמן שהנר עצמו דולק דיחוי ראשון קיים אע״פ שזה מטפטף ממנו אלא דהתם הוקצה לאיסורו ולמצותו וקס״ד לאיסורו ולמצותו בעינן כלומר דאית ביה תרתי שאסור להסתפק ממנו משום מכבה ושהדלקתו מצוה ומ״ה אקשי׳ מעצים דסוכה. ופרקי׳ כעין שמן שבנר קאמינא הואיל והוקצה לאיסורו הוקצה (נמי) למצותו פי׳ כל שהוקצה משום מצותו דהיינו כעין שמן שבנר דאלו שמן שבנר לאיסורו ולמצותו תרוייהו הוקצה. ויש מחליפין וגורסין הואיל והוקצה למצותו הוקצה לאיסורו דאי לא היינו קמייתא ולאו מלתא הוא אלא כדפי׳: }
\textblock{\textbf{וממאי דר״ש הוא.} פיר״ת ז״ל ממאי דר״ש מודה בה אבל לאו ר״ש הוא דקתני עד מוצאי יום טוב האחרון של חג והיינו משום דכיון דאתקצאי לבין השמשות אתקצי לכולא יומא והיינו סברא דר״י ור״מ בנר ולעיל במטה ואלו לר״ש ל״ל הכי דהא אם הי׳ עליו מעות כל בין השמשות שרי ר״ש. ולא מסתבר לן, דהכא כיון דבמניח איירי ומעות עודן עליו אסורין בטלטול אפי׳ לר״ש משום דלא חזי למידי א״כ ה״ל בסיס לדבר האסור כעין נר ופתילה דאסרי׳ לקמן משום דהוה בסיס לדבר האסור וע״כ הו״ל לר״ש למיסר אלא משום דאדם מצפה אימתי ינטלו מעות מעל המוכני כטעמא דנר ולא הוה כקערה ועששית וכיון שכן מוכני ומעות כדין נר ושמן עם השלהבת כבר אמרו דדעתי׳ עלי׳ אימתי תכבה ואינו מקצה דעתו ממנו אלא בעוד שהוא בהקצאתו אבל הכא הקצהו לכל החג והוקצה נמי למצותו לבין השמשות של שמיני שע״כ צריך הוא לישב בסוכה ואפי׳ היתה נמי סוכה רעועה אסורה דכי אמרי, התם בסוכה רעועה מאתמול דעתי׳ עילויה ה״מ בסוכה דעלמא אבל בסוכה דמצוה לאו דעתי׳ עלוי׳ דתפיל אלא אדרבה דעתי׳ עלוי׳ דלא תפול ואם נפלה הדר בני לה בתש״מ וע״כ הוקצה נמי לבין השמשות דשמיני ואיתקצי לכולי יומא דל״ל אדם יושב ומצפה שתפול ביום שמיני עד השתא לא נפלה והשתא נפלה בשמיני בכונה. וא״ת עטורין לישתרי בשמיני דהא מצי שקיל להו כיון שהוא מקצה אותם לכל החג אין דעתו עליהן עד שיתיר סוכתו ויטול את הכל במוצאי י״ט האחרון. וראיה לדברי, דהא בפ׳ לולב וערבה א״ר יוחנן סוכה אף בשמיני אסורה ומפרש לה משום מגו דאתקצאי לבה״ש ולא מוכח׳ לקמן בשמעתין דכר״י ס״ל והתם נמי פליגי כמה אמוראי דס״ל כר״ש אלמא סוכה בשמיני אסורה ד״ה היא. ואיכא למידק אשמעתין, דהכא משמע דמשום מוקצה הוא אסור ובפ׳ במה מדליקין אמרו משום שלא יהא מצות בזויות עליו וא״ל משום שלא יהיה מצות בזויות עליו הוא מקצה אותו למצותו שאלמלא היה מותר לו לבזות המצות היה מותר לו להשתמש בהן אף בשעת מצותן נמצא שלא הוקצה אלא משום כבוד המצות הוקצה בשעת מצותן וכיון שהוקצו אם באו לסלקן מן המצוה ולהסתפק מהן אסור משום מוקצה ואפי׳ בסוכ׳ נופלת. וק״ל, דהכא משמע דמשום מוקצה ומשום בזוי מצוה אסורה עצי סוכה ובמס׳ ביצה פר׳ המביא מפ׳ לה מהא דאר״ש מנין לעצי סוכה שאסורין כ״ש שנאמר חג הסוכות תעשה לך שבעת ימים מה חג לה׳ אף סוכה לה׳ ולא תימא התם אסמכתא דרבנן דהא מקשי׳ מיניה פ״ק דסוכה וב״ש מיבעי׳ להו להכי אלמא דוקא הוא ולאו אסמכתא היא. וא״ל כי אמרינן אף סוכה לה׳ ה״מ כשהיא סוכה ויוצא בה י״ח אבל נפלה עציה מותרין דאורייתא שכבר בטלה ואין שם סוכה עליה אלא שנאסרו עצים מדבריהם מפני שהוקצה למצותן והא דמייתי הכא ושוין בסוכת החג בחג שהיא אסורה משום דרישא ור״ש מתיר אוקימנא התם בסוכה נופלת וסיפא נמי אפי׳ בנופלת קאסרי׳. וההוא דאקשי׳ התם (ביצה ל:)ומי מהני בה תנאה והא״ר ששת וכו׳ בתוס׳ ראיתי שמפרשים ומי מהני בה תנאה אפילו בנפילה והא״ר ששת וכו׳ וכיון דקיימא איסורא מן התורה דין הוא שלא יהני בה תנאה אפילו נפלה משום קיימת יהדר אקשי׳ והא מהני בה תנאה אנוי סוכה משום שהן מדרבנן ולא גזרינן משום סוכה עצמה, ואין זה נכון. ואפשר לי לפרש ולומר, דמעיקרא נמי בנופלות קאמרינן דרישא דברייתא בנופלות אוקים בגמרא והכי מקש ומי מהני בה תנאה להתירה כשתפול והאר״ש עצי סוכה אסורין כל ז׳ שלא יטול ממנה אפי׳ עץ (אחר) [אחד] כ״ז שיצא בה בעוד שהיא קיימת וכיון שכן היאך הוא מתנה לכשתפול תהא מותרת ואפילו היא סוכה רעועה או בחש״מ שיהא דעתו לסתרה (ואע״כ) סוכה דמצוה הוא וחלה עלי׳ קדושה דאורייתא מעתה ואין תנאו כלום מעכשיו והאיך יהא כשר לכשתפול הו״ל כמתנה על ההקדש לכשיפדה דלא הוה תנאי׳ תנאה ועצי סוכה נמי השתא בעודה קיימת הקדש גמור לשמים הוה ומוקצה למצוה הם בע״כ ופריק אה״נ אלא (איסור׳) [אסוכה] דעלמא. והדר אקשי ולא מהני בה תנאה ואפילו לסוכה עצמה שיהא מותר בה לכשתפול או שיסתור אותה בידים ויהנה בעצים של זו ויעשה סוכה אחרת למצותו והא תניא ואם התנה עליה הכל לפי תנאו שכשם שחל שם שמים על הסוכה כך חל על עטוריה ואסור מפני בזוי מצוה ואפ״ה כי מתני עלייהו מהני תנאו אף אני אביא סוכה עצמה דמהני בה תנאה להתירה לסתור אותה ולהשתמש בהן וכ״ש נפלה ומפרקי באומר איני בודל מהן כל בה״ש דלא חלה קדושה עלייהו כלל וליכא אפילו משום ביזוי מצוה כיון שהתנה אבל עצי סוכה עצמה אינו יכול להתנות עליהן תנאי זה וכיון שחלה עליהן קדושת סוכה מן התורה איתקצאי להו כל ז׳ שאפי׳ נפלה נמי אסורה שלא חל תנאו כלל שאלו לא נפלו היו אסורין מה״ת כל ז׳ והלכך כי נפלה אסורי שהרי הוקצה למצותן, וזה הלשון יותר נכון. ול״נ דעצי סוכה בין קיימת בין נופלת אסורין בהנאה כ״ש מן התורה בלא איסור מוקצה ועטורין נמי אסורין משום ביזוי מצוה ואפשר דאפילו בסוכה נופלת איכא ביזוי מצוה דכ״ז שעצי סוכה אסורין אם בא להשתמש בעטורין איכא ביזוי מצוה דהא צריך לחזור ולבנותה ולעטרה בהן אלא הא דמייתי הכא משום מוקצה למצותו היינו מסיפא דקתני עד מוצאי י״ט האחרון דע״כ היינו משום מוקצה מגו דאתקצאי למצותו בה״ש אתקצאי לכולי יומא כדפריש׳ לעיל וכן הא דמייתי מדתני רחב״י ושוין בסוכת החג בחג שהיא אסורה משום דסוכת החג בחג קתני וכל החג במשמע ואפי׳ י״ט האחרון וקתני נמי ושוין אלמא כי היכי דלת״ק אסורין כל החג ה״נ לר״ש וי״ט האחרון ע״כ משום מוקצה הוא ומודה בה ר״ש וכדפירש, וזה הלשון מחוור כל השמועות הללו. ומ״ש שם בפרק המביא אבל עצי סוכה דחלה קדושה עלייהו איתקצי לאו לאיסורייהו משום מוקצה אלא ה״ק כיון דסוכה דמצוה הוא וחלה עלייהו מה״ת שעה א׳ קדושה כעצי מצוה חלה עלייהו כ״ש אע״פ שהותנה שע״כ סוכה דמצוה הוא ומצותה מייחדה ומקצה אותה כ״ש לכך שהרי נתפסה בקדושה וכל שעה קדושתה עלי׳ ושמה עלי׳ סוכה דמצוה הוא אלא (מדר״י) שמיני והוא משום מוקצה ומהני בה תנאה (והרי מקשה) [ולהכי מקשה] עלי׳ מאתרוג דמדרבנן נשמע לדאורייתא. וראיתי בתוס׳ שאלה כי היכי דמהני תנאה בסוכה רעועה לר״י אמאי לא מהני נמי גבי נרות לר״י ור״מ בנר שהדליקו בה באותה שבת ולר״ש בנר הדלוק לשמן המטפטף והם השיבו על שאלה זו תשובות שאין בהם ממש. ויכולני לומר דעצי סוכה כיון דדעתי׳ עלייהו ודאי לאחר יום טוב אי מתנה בי״ט מהני בהו תנאה אבל שמן כיון דסבר שהוא כלה בנר בשבת היאך יתנה על הספק אם יכבה והוא אין דעתו שיכבה וכ״ש כוס וקערה לר״ש וכיון שאין תנאי מועיל למותר השמן אין מועיל לנר עצמו שהרי הוא בסיס לשמן ופתילה וזו סברא מתקבלת לדעת לפי גמ׳ שלנו שלא הזכיר במוקצה דנרות תנאי בכל השמועות הללו. עד שמצאתי בירושלמי (ג,ז) כך, תני אם התנה עליו מותר מה אנן קיימין אי כר״מ אפי׳ התנה יהא אסור ואי כר״ש אפי׳ לא התנה יהא מותר אלא כר״י נר מאוס הוה מן דתני׳ אם התנה עליו יהא מותר ר״ש דתני אבל כוס וקערה ועששית אע״פ שכבו אסור ליגע בהן ר׳ טבי בשם ר׳ חסדא אפי׳ ר״ש דאמר תמן מותר מודה הוא הכא שהוא אסור וכו׳ ופי׳ שהם הקשו על ברייתא זו דתני בנר דשבת אם התנה עליו מותר מני אי ר״מ אין תנאי מועיל בו לפי שהם סוברים שם בירושלמי לר״מ דכל המיוחד לאסור אסור ואי ר״י כ״ש שאין אדם יכול להתנות על הדבר המאום שלא יהא מאוסו מקצהו ובסוף העמידוהו בר״ש ככוס וקערה ועששית דמודה בהו ר״ש דאסירי ואם התנה עליהן מותר ואף אנו נאמר בנר של מתכות לר״מ ור״י דמהני בהו תנאה ודומיא דסוכה רעיעה. ואפשר דאפי׳ בסוכה בריאה מהני, ואין אנו ולא רבותינו הצרפתים ז״ל צריכין מעתה לדחוק בסברות: }
\textblock{\textbf{א״ל אין מוקצה לר״ש אלא גרוגרות וצמוקים בלבד ור׳ ל״ל מוקצה.} תימה הוא הא רבי בהדיא אמר לר״ש ור״ש בנו נמי אליבא דר״ש קא״ל וכדאמר ר׳ יוחנן אנו אין לנו אלא בנר כר״ש ואע״ג דלא ס״ל כוותי׳ ק״ל היכי אמרינן אב״ע אליבא דר״ש קאמר ולי׳ לא סביר׳ לי׳ והא אמרי׳ לקמן הורה רבי במנורה כר״ש בנר וע״כ בנר להתירא הורה כר״ש וי״ל לא שמיע לי׳ הא דריב״ל וכי מטינן התם לא בעי לאקשויי מהא דסמך אשאר פירוקי׳ דאתמר הכא כך אמרו בתוס׳. ולי נראה דבמוקצ׳ מחמת איסור ס״ל כר״ש מפני שדעתו עליו אימת תכבה אבל במוקצה שאינו ראוי ואינו מצוי כגון פצעילי תמרים וגרוגרות וצמוקים ובייתיות ומדבריית אפשר דלא ס״ל כוותי׳ וה״נ משמע בפ׳ אין צדין דרבי לית לי׳ מוקצה דשרי בכור תם שנפל בו מום ביום טוב כר״ש ועדיפין מדר״ש משום דקסבר רואין מומין בי״ט ולא אסיר לי׳ משום מי יימר דנפיל בי׳ וכו׳:
}
\textblock{\textbf{והאר״נ למאן דאית ליה מוקצה אית לי׳ נולד.} איכא למידק למאן קאמרי׳ לר׳ יוחנן ור״י לית ליה דר״נ דאיהו סבר טעמא דביצה משום גזירה דמשקין שזבו ועוד דר״נ גופי׳ הוקשה מסברא זו בדוכתא בריש מס׳ ביצה ולית לה פתרי אלא שאלו מן הדברים שנאמרו בגמ׳ לרווחא דמילתא לתרוצי סוגיא אליבא דכ״ע כלומר מהא לא תפשוט פלוגתא, ויש כיוצא בה בתלמוד:
}
\newsection{דף מו}
\textblock{הא דאמר רבא \textbf{שרגא דמשחא שרי לטלטולי, דנפטא אסור.} ה״פ לפי שהנר של נפט מסריח הוא ואין משתמשין בו אלא להדליקה נמצא שהיא בשבת כשברי כלי שאין עושין מעין מלאכה ופריק אע״פ שאינו עושה מעשה כלי חזי למידי לכסויי מנא. ואקשי ליה הו״ל כצרורות שבחצר שאין עושין מעין מלאכה כלל ופריק כיון דכלי גמור הוא ותורת כלי עליו להשתמש בו לאחר השבת מטלטל הוא ככלי הראוי דבשבת גופי׳ חזי למידי הוא: }
\textblock{\textbf{כל היכי דכי מכוין איסורא דאורייתא כי לא מכוין גזר ר״ש.} איכא דק׳ ליה, והכא כי מכבה נמי ליכא איסורא דאוריית׳ שהרי מלאכה שאינה צריכה לגופה הוא ולר״ש פטור עליה וניחא ליה הכא בפתילה שצריך להבהבה עסקי׳ דבההוא אפי׳ ר״ש מודה ואפשר שאפי׳ לא הבהבה נמי כיון שאם כבה הוא אב מלאכה ויש מקום שהוא חייב עלי׳ מן התורה גזר ר״ש בכל טלטול שכשאדם רואה את חבירו מטלטל נר הדלוק מי מפיס לו איזה מהן הבהבה ואיזו לא הבהבה מה שאין כן בגרירת מטה שאין גרירת מטה וספסל בעולם שיכול לבא עלי׳ איסורא דאורייתא:
}
\newsection{דף מז}
\textblock{\textbf{מחתה באפרה.} פי׳ בשהוסק מערב י״ט ואם תאמר והא אפר כירה מוכן הוא א״ל דוקא אפר כירה אבל אפר מחתה לית בה מששא ואין דעתו עליו לפי שהוא מלבונה וכיוצא בה ונעשה דק מאוד ואינו ראוי לכסות בו שם דבר והוא מוקצה עד שיהא דעתו עליו מאתמול א״נ בשאין דעתו עליו וא״צ לו כדברי ר״ש ומסקנא מתרץ לה מדאמר רבא מטלטלין כנונא אגב קיטמא אע״ג דאיכא עלי׳ שברי עצים וכנונא הוא כלי נחושת הנקרא אלכנון בלשון קדר וקיטמא מוכן הוא דומיא דאפר כירה אלא ששברי העצים אינן ראויין ודרבי נמי באפר מוכן כלומר שצריך לו וקא משמע לן דאיכא עלי׳ קרטי לבונה ועוקצין מותר, זה תורף פרש״י ז״ל. ואי ק״ל א״כ היינו מתני׳ דכלכלה והיכי אקשי׳ עלי׳ מהא ושווין שאם יש עלי׳ שברי פתילה שאסור לסייעה ממתני׳ דכלכלה וא״ל דה״ק שאין שברי עצים בטלין אגב הקיטמא כמו שאין שברי הפתילה בטלה אגב שמן ונראה לפי שהוא חשוב לגבה מה שאין בכלכלה שהפירות חשובין והאבן אינה חשובה ועוד שאין דרך האבן ליטלה בכלכלה ועיקר הטלטול לפירות אבל שברי עצים בכנונא ושברי פתילה בנר מקומן הוא ועיקר הטלטול הוא עושה בשבילן. ואחרים פי׳ דאפר מחתה שהוסק בשבת קאמרי׳ ואליבא דר״ש דשרי במותר שמן שבנר ומעיקרא קס״ד דאסור לפי שאין אפר שבמחתה ראוי ובתר הכי מתרץ רבא דאפר מחתה מוכן הוא ומטלטלין אפי׳ שברי עצים אגבו ופי׳ כנונא מחתה וכן פי׳ ר׳ נתן בעל הערוך וא״ת תפשוט דרבי כר״ש ס״ל אין הכי נמי וכדאמר לעיל, וכך פירשו רבותינו הצרפתים ז״ל בתוס׳. ושתי לשונות הללו להן דרך א׳, ואינם נכונים בעיני משום דלא מפלגי׳ בגמ׳ באפרה של מחתה ועוד דכנונא נמי אגב מנא מטלטל הוא אפר כדאקשי׳ בגמ׳ דיום טוב גבי שיורי כוסות ולטלטלינהו אגב מנא מי לא אמר רבא כי הוינן בי ר״נ מטלטלין כנונא אגב קיטמא. והפי׳ הנכון כך הוא: שהאפר שבמחתה הוסק בשבת וטלטול המחתה עצמה מותר כר״ש בנר וקא משמע לן שאין האפר שאינו ראוי מעכב ומטלטל הוא אגב כלי שלא נעשה הכלי בסיס להן מערב יום טוב לדעת והן בטלין אגבו וכן קיטמא דלא חשיבי ואפילו שברי עצים דחשיבי קצת מטלטלי אגב כנונא דכולהו בטלי לגבי כלי שאין נטילת הכלי בשבילן כלל אלא בשביל עצמו וה״ט דשיורי כוסות ולא מבעי׳ לצורך מקומו. א״נ לצורך גופו בשאי אפשר לנערן דמותר דהיינו מעות שעל הכר שמטלטלן ועודן עליו כדאיתא בפ׳ נוטל אלא כל שמטלטל מחתה בלא אפר וכוסות בלא שיורין מטלטלין עמהן בין לצורך בין שלא לצורך דבטלי אגב כלי ולא חשיבי כלל תדע מדלא גזרי׳ בשיורי כוסות כלל משום טלטול שלא לצורך, כן נ״ל ועלתה השמועה בחוור:
}
\textblock{\textbf{קרטין בי רבי מי חשיבי.} מהא שמעי׳ דמוקצה לעשירים בבית עשירים מוקצה הוא לעשירים ואפי׳ עניים אין מטלטלין אותן שכבר הוקצו מדעתו של בעל הבית מערב יום טוב, וכן נמי משמע לעיל גבי מפירין ונשאלין לנדרים לצורך השבת ואמאי לימא מי יימר דמזדקק בעל וחכם אלמא מוקצה הוא מפני שאסרה הככר על עצמה שלא כדברי מי שפי׳ שאסרה ככר שלה אכולה עלמא:
}
\textblock{הא ד\textbf{תניא בגדי עניים לעניים.} לאו למימרא דשלש על שלש לעשירים לא מטמא דהא תנן הבגד מטמא משום שלשה על שלשה למדרס ומשום שלש על שלש לטמא מת בפרק כ״ז דמסכת כלים והיינו לעשירים דאלו לעניים תנן התם בפ׳ כ״ח בגדי עניים אף על פי שאין בהם שלשה על שלשה טמאים טומאת מדרס אלמא אפילו לעשירים טמאין טומאת מת בשלש על שלש (וכן דייק הר״ש בפ׳ כ״ח מ״ח ע״ש) אלא לומר לך שאין טמא טומאת מדרס בפחות משלשה לעשירים אלא לעניים משום דחזו להו ומדרס בייחוד תלי׳ מילת׳ ואין עשירים מייחדים פחות משלש על שלש לישיבה ואפי׳ ייחדו בטלה דעתו אצל עשירים דעלמא אבל לשאר טומאות כל שהוא שלש על שלש טמא דהא חזו תדע דהא בפרק במה מדליקין לא מפלגינן בין עניים לעשירים לא בקרא ולא בשמעתי׳ כולי׳ ואי ק״ל אי הכי לייתי מתני׳ א״ל כיון דלא מפרש (בקרא ומשנה) [בחדא משנה] ניחא לי׳ לאתויי בריית׳ דמתניא בהדיא. וי״מ כגון בגדים הגסים לעניים אבל לעשירים אינן טמאין לפי שאין ראויין להם בגדי עשירים הדקים לעשירים טמאים וכ״ש לעניים ולאו מילתא הוא:
}
\textblock{\textbf{בגלילא שנו.} שמעתי בשם ה״ר אפרים ז״ל לפי שהיה להם שמן הרבה והיתה הפתילה חשיבה להן יותר מן השמן כדאמרי׳ בסנהדרין לבני גלילא עילאה ולבני גלילא תתאה שלומכן יסגא מהודעין אנחנא לכון דזמן ביעור מטא לאפרושי מעשריא ממעטני זיתיא ואמרי׳ במס׳ נדרים בגלילא שאני דחמרי יקיר להון ממשחא אלמא שמן הרבה הי׳ להם, וכן פר״ת ז״ל. אבל רש״י ז״ל אמר שהיו עניים ולא הי׳ להם פשתן ואינו נכון דאמרי׳ בפרק הגוזל מקבלין מהן דבר מועט כגון פשתן בגליל. ור״ח ז״ל כתב שדרכן להדליק בשברי פתילות ואפשר מפני שפשתנם יפה ואין פתילה שכבתה עושה פחם וראוי׳ להדליק בי׳:
}
\textblock{הא דאמרי רב ושמואל \textbf{המחזיר מטה של טרסיים בשבת חייב חטאת.} האי דוקא בתקע אבל לא תקע פטור דהא בין לת״ק בין לרשב״ג בלא תקע פטור ולית דחש ליה לב״ש דאמרי במס׳ ביצה גבי מנורה של חוליות יש בנין בכלים וכיון דבשתקע מיירי רב ושמואל ור״א ור״ה בשלא תקע דהא מוקמי׳ להא כרבן שמעון בן גמליאל דאמר אם הי׳ רפוי מותר מאי קושי׳ דמקשי להו רב יהודה מדרב ושמואל א״ל דה״ק להו כיון דרב ושמואל תרוייהו אמרי בשתקע חייב חטאת ע״כ בשלא תקע אסור לכתחלה ואינהו סבר אע״ג דתקע חייב חטאת לא תקע מותר לכתחלה ולא גזרי׳ והיינו דמקשי׳ עלייהו המחזיר קני מנורה וכו׳ וכן מפורש בתוס׳. ויש לפרש דמטה של טרסיים לאו היינו מלבנות המטה דבריית׳ ובמטה של טרסיים א״צ לתקוע דבלא תקיעה נגמרה מלאכתה ואקשי׳ עלייהו מהא דתניא קנה של סיידין לא יחזיר ואם החזיר פטור אבל אסור לומר נהי דסביר׳ לכו דחזרתה לאו היינו גמר מלאכתה מ״מ הי׳ להם לאסור לכתחלה והם אמרו כרבן שמעון בן גמליאל וברפוי שכיון שאין חזרתה בלא תקיעה גמר מלאכתה אפי׳ בשאינו רפוי ברפוי מותר לכתחלה כדאמר רבן שמעון בן גמליאל אם הי׳ רפוי מותר והא דאמרי׳ במס׳ ביצה גבי מנורה של חוליות שב״ה מתיר להחזירה בשלא תקע וברפוי כרבן שמעון בן גמליאל ואפשר אפילו כרבנן ומשום צורך יום טוב וכבודו ודאמרי׳ לעיל בשל חוליות גזירה שמא יתקע ויבא לידי חיוב חטאת קאמרי׳, ועוד תתברר לפנינו (ק״ב ע״ב ד״ה רב) בס״ד:
}
\textblock{הא דאמרי׳ \textbf{קופה שטמן בה אסור.} מים צונן הוה כדאמרי׳ הכא אולודי קא מוליד, וכן פירש״י ז״ל. וק״ל רבי זירא היכי אמר מאי שנא ממיחם ע״ג מיחם כלומר דשרי בבריי׳ לקמן הא קתני עלה דההיא לא בשביל שיחמו אלא בשביל שיהיו משומרין. ואפשר לומר בריית׳ לא שמיע לי׳ והכי קאמר מאי שנא ממיחם על גבי מיחם דשרי שכך עושים מעשים בכל יום. ותו קשיא, רבי זירא לית ליה הא דתניא אין נותנין ביצה בצד המיחם בשביל שתתגלגל ולא יפקיענה בסודרין ואף על פי שהן תולדת האור ולא אור עצמו וא״ת לא הי׳ אלא מפשיר אם כן למה אסר רבא לית לי׳ הא דתניא מניח אדם קיתון של מים כנגד המדורה כדי שתפיג צנתן ומתני׳ נמי הוא אבל נותן לתוכו מים מרובין כדי להפשירן וא״ת לא בתוך המיחם הי׳ מניח אלא בתוך המים והמיחם הי׳ מונח על האש והי׳ רבי זירא סבור לחלוק בהן ול״ל הא דאמר ר״ש בכלי שני נותן. וא״ל שהי׳ מניח על גב פי המיחם שפינהו מעל האש ולא הי׳ שפתו חם לחמם אלא שהי׳ עולה הבל מן המים החמין ומתחממין אותו והי׳ סבור שכיון שאין הכוזא נוגע במים אף הבל המים שהן עצמן תולדת האור אינן אוסרין אותו כמו שמותר ליתן מיחם ע״ג מיחם אף על פי שהתחתון מלא מים חמין וההבל עולה ופריק שאני הכא דצונן הוא ואולידי בשבת אסור:
}
\newchap{פרק \hebrewnumeral{4} במה טומנין}
\textblock{}
\textblock{\textbf{מה בין זו למגופת חבית.} פירש״י ז״ל דתניא בפ׳ חביות מביא אדם חביות של יין ומתיז את ראשה בסייף ומניחה לפני האורחין בשבת ואינו חושש. וזה אינו מחוור ונכון, דהתם מחביות גופה קאמר ממאי מדבעי מיני׳ דרב ששת התם מה למברז חביתא בבורסיא וא״ל אסור דאקשי׳ עלה מהא ואי במגופה מתניתין היא דתנן אין נוקבין מגופה של חבית דר״י וחכמים מתירין ועוד קשי׳ אשמעתא דהא בהדיא אמרי׳ התם לפיתחא קמכוין ואסור א״ד לעין יפה קמכוין וש״ד והכא לפתחא קא מכוין ופתחא עבוד דמאי עין יפה איכא הכא. ועוד מי דמי בשלמא התם פתח שאינו עשוי להכניס ולהוציא הוא ולהכי הותר בלא מכוין דאיסורא דרבנן הוא אבל הכא פתח שהוא עשוי להכניס ולהוציא הוא ומן התורה הוא אסור. ומצאתי בתוס׳ דהכא בבית הצואר של נשים שאינו מן החלוק עצמה והכי פריך וכי מה בין זה למגופת החבית דכי היכי דאיכא חילוק בין גופה של חבית למגופתה כך יש חילוק בין בית הצואר לחלוק עצמה. ופריק שאני הכא שהוא חבור. ועדיין לא נתברר לי כהוגן:
}
\textblock{\textbf{שלל של כובסין והבגד שהוא תפור בכלאים חבור לטומאה ואין חבור להזייה.} וקתני בבריי׳ דהאי חבור לטומאה היינו עד שיתחיל להתיר, ומ״ה סתם חבור ומשום דעיקר בריי׳ זו שנוי׳ במשנתינו קאמר בגמרא תנן, וכיוצא בה בתלמוד הרבה. והא דלא מייתי׳ הכא מתני׳, משום דקאמר עד שיתחיל להתיר דהיינו שלא בשעת מלאכה ואוקי׳ כרבי מאיר וה״ה למקל של קרדום דלרבי מאיר הוא אבל לרבנן אינו חבור כלל ואף על גב דגבי מספורת של פרקים הוה חבור לטומאה אפי׳ לרבנן שאני התם שהן עושין מעין מלאכה אחת ואי אפשר לזה בלא זה משא״כ בשלל של כובסין ובית הפך ובית התבלין. ואם תאמר גבי מקל של קרדום אמאי לא גזרו שלא בשעת מלאכה אטו שעת מלאכה אפשר לומר מפני שאין אדם מצניעו להחזירו לקורדם ואדם עשוי לזורקו בין העצי׳:
}
\textblock{והא דאמרי׳ \textbf{עבדו להו רבנן הכירא דלא לייתו למישרף עלייהו תרומה וקדשים.} איכא למידק ולעבד נמי הכירא בשלל של כובסין ואפשר לומר כיון שאינו חבור להזאה היינו הכירא אבל כלי חרס דלאו בני הזאה נינהו צריכי הכירא:
}
\newsection{דף מט}
\textblock{\textbf{מאי טעמא, אביי אמר שלא יפיח בהם.} פי׳ למאי הלכתא בעי׳ כאלישע וכי כל מי שאינו צדיק כמותו לא יניח תפילין. אמר אביי, לא לענין חסידתו אלא לענין זריזתו, דאיהו אדם גדול וזרוז בהן הי׳ שהרי נעשה לו נס גדול ורבא אמר שלא יישן בהם אחד שינת קבע ואחד שינת ארעי ואף על גב דבמס׳ סוכה איפלגי בהא מלתא תנאי אדרבא דהוה בתרא סמכי, זהו כדעת רבינו הגדול ז״ל שהשמיטן לאותן ברייתות שבפ׳ הישן ולא כתבן בהל׳ תפילין. ואי קשי׳, אלישע שהי׳ גוף נקי למה נטלן מראשו, הא בשעת שמד אפי׳ אערקתא דמסאנא יהרג ואל יעבור ה״מ לעבור על מצות לא תעשה אבל אם גזרו לבטל מצות עשה ודאי תבטל ואל יהרג דהא שב ואל תעשה שאני. ועוד שהן יכולין לבטל ע״כ שיניחוהו בבית האסורין וממילא תבטל דאף על פי שהי׳ מניחן שמא הי׳ סביר שלא יראהו אדם או שהי׳ מוסר עצמו למיתה שלא לבטל מעליו מלכות שמים ורשאי הוא בכך ואינו כמתחייב בנפשו אף על פי שאמרו יבטל ואל יהרג שכל מצוה שהחזיקו בה ישראל בשעת השמד נוטלין עלי׳ שכר הרבה ועדיין מוחזקת ואמרו בהגדה בבראשית רבה מה לך יוצא ליסקל על שמלתי את בני מה לך יוצא ליצלב על שנטלתי את הלולב וכו׳, וכן מצינו בדניאל שמסר עצמו למיתה על התפלה שהוא מצות עשה דרבנן:
}
\textblock{\textbf{הביאו שלחין ונשב עליהן.} פירש״י ז״ל בחול הי׳ המעשה אלמא לא קפיד ואינו נכון שאף על פי שפעם א׳ נהג כן אינה ראי׳ שלא יקפידו עליהן שאפשר שנהג עין יפה או מפני טעם אחר ועוד דבריי׳ דלקמן ר״י אומר א׳ זה וא׳ זה מטנטלין אותן מפורש׳ להביא ראי׳ מזה. והגאונים אמרו, בשבת הי׳ מעשה ור״י לטעמי׳ דאמר לקמן אחד זה ואחד זה מטלטלין אותן ופסקו הלכה כן משום דכיון דשלחא הוי קים להו בהו ועוד אי לאו דקים לי׳, לא הוא עביד מעשה:
}
\textblock{הא דאמרי׳ \textbf{כנגד מלאכה מלאכתו ומלאכת שבתורה ארבעים חסר א׳.} פר״ח ז״ל שכל מלאכות האמורות בתורה ס״א, טול מהם ג׳ דויכולו שאינם צווי וד׳ מלאכה ומלאכת שכתב בהן (עשיות ויעש ותעשה) [ועשית יעשה תעשה], ולרגל המלאכה אשר לפני, וי״ד שכ׳ בהן כל מלאכת עבודה הרי כ״א, נשתיירו מ׳ ובכללן ששת ימים תאכל מצות כו׳ לא תעש׳ מלאכה ולמה זאת שבפי׳ אמר בתלמוד ארץ ישראל שבא להשלים ל״ט מלאכות שבתורה. ויש מי שמוציא מכל אלו המ׳ ויבא הביתה לעשות מלאכתו נשארו ל״ט ויש מי שמוציא והמלאכה היתה דים ולדברי הכל נשארו ל״ט, ע״כ דברי ר״ח ז״ל. והוי יודע שלד״ה אותן מלאכות שהיו במשכן חשובין וקרי להו אב ואותן שלא הי׳ במשכן קרי להו תולדה דאל״ה למה קרא לזה אב ולזה תולדה אלא למר גמר כולה מלתא ממשכן ולמר היו חכמים עושין כולן אבות אלא שמצאו להן סמך במנין מלאכות שבתורה וכיון שידעו מנין למדו חשיבותן ממשכן. א״נ באבות ותולדותיהן הכל מודים שלמדו ממשכן, אלא לחשוב עוד ענינים אחרים למלאכות שלא היו במשכן לא אב ולא תולדה שלו פליגי, למר כיון דלא הוי במשכן כלל [ממילא] פטור עלי׳ ולמר מן המנין למדו זה. ואף על גב דמסייעי׳ למ״ד כנגד עבודות שבמשכן מהא דתניא אין חייבין אלא על מלאכה שכיוצא בה היתה במשכן לאו משום דלא מודו כ״ע בהא אלא משום דקתני הם זרעו ואתם לא תזרעו דמשמע דמהתם גמר וכי אזהרינהו רחמנא ממלאכת המשכן אזהרינהו ומ״ה סמך פ׳ שבת לפ׳ משכן ולמ״ד כנגד עבודות שבתורה לא גמרי׳ ממשכן אלא למחשבינהו כאבות:
}
\textblock{ודאמרי׳ \textbf{ואתם אל תוציאו מרשות היחיד לרשות הרבים.} כבר מפורש בפרק ראשון (ב׳ ע״ב):
}
\newsection{דף נ}
\textblock{\textbf{רבינא אמר בשל הפתק.} פירוש ומתני׳ הכי קתני בגיזי צמר ואין מטלטלין אותן להטמין בהן [פי׳ להוסיף] או לדבר אחר ואם הי׳ של הפתק אין מטלטלין אותן לעולם אף על פי שטמן בהן א״נ מתני׳ בשל הפתק ודרבא הכי אתמר ל״ש אלא של הפתק אבל גיזי צמר אחרים אי טמן בהן מטלטלין אותן אי נמי דרבא לאו אמתני׳ אתמר, כך פירוש רש״י ז״ל. ויש הפרש בין מוכין לגיזי צמר שהמוכין אף על פי שאינן של הפתק בגיזי צמר של הפתק דמי ואסורין לטלטל כדאמרי׳ לעיל וכי מפני שאין לזה קופה של תבן עומד ומפקיר קופה של מוכן. ואפשר דההוא טלטול דאסקי׳ לעיל ליומא אחרי כלומר שאם טמן בהן היום אינו מותר לטלטלן לשבת אחרת שאינן מיוחדין לכך אבל ליומי׳ מותר שעדיין ייחודן עליהן תדע דלא קתני במתני׳ במוכין כיצד הוא עושה וכו׳ כדקתני בגיזי צמר של הפתק וכי שרינן גיזי צמר שאינן של הפת׳ לבו ביום אבל לא לשבת אחר׳ שכבר בטל ייחודן הואיל ולא טמן בהן לשבת זו שמפני שלא הי׳ לזה קופה של תבן עומד ומפקיר קופה של מוכין ומוכין וגיזי צמר שאינן של הפתק שוין:
}
\textblock{\textbf{ולא ידענא אי בית האבל אי בית המשתה הוי.} פי׳ ר׳ חנינ׳ ב״ע אמר כן לתלמידיו ״כדי שנשב עליהן למחר בבית האבל או בבית המשתה״, שאלו לא אמר כן, אע״פ שהי׳ מעשה בבית האבל או בבית המשתה שמא אף במקום אחר הי׳ מתיר כך אלא תא חזי מאן גברא רבא דקא מסהיד עלי׳ ואפשר שאלו הי׳ במקום אחר מותר בכך כאן בבית המשתה נא הי׳ צריך לכך שבודאי חילוק יש בין בית האבל ובית המשתה לשאר מקומות, ואין זה נכון:
}
\textblock{\textbf{ורב אסי אמר יושב אף על פי שלא קשר ולא חשב.} פי׳ ר״א פליג עליה דרבי שמעון בן גמליאל, וסבר שצריך מעשה כ״ש שנשתמש בהן לאותו ענין הצריך לו עכשו כעין התחלת מלאכה אבל מחשבה גרידא לא מהניא ומעשה בלא מחשבה נמי מהניא והיינו דקאמרי׳ אלא ר״א דאמר כמאן שאלמלא הי׳ סובר דחשב מהני לי׳ מאי קמבעי׳ לי׳ רבן שמעון בן גמליאל נמי במעשה כל דהו בלא מחשבה ודאי מודה אלא ש״מ כדפרי׳ וקיי״ל כר״א ולא קי״ל כרבן שמעון בן גמליאל אלא אם כן יש עליו תורת כלי ובההוא מודה ר״א כדמוכחא שמעתא בפ׳ כל הכלים (שבת קכ״ו ע״א):
}
\textblock{הא דתנן \textbf{נזיר חופף ומפספס.} בנתר ובחול קאמר, מדתנן בפרק שלשה מינין עלה ר״י ברבי יהודה אומר (יחיד לא יחוף) [נזיר לא יחוף] ראשו באדמה מכלל דרבי שמעון שרי והיינו דמקשי׳ מינה הכא להא דתניא אבל לא יחוף בהן שערו:
}
\textblock{\textbf{מהו לפצוע זתים בשבת.} פירש״י ז״ל על סלע כדי למתק מרירתן ותימא הא בחול מי אסרן וכי לא יתקן אדם מאכלו כדי שיהא ראוי לו ועוד בשבת מה איסור יש בה אי משום סחיטה דבר הלמד מעניינו הוא שלא נאסר בשביל סחיטה שא״כ לא הי׳ לשאול כאן אלא לקמן במקום הלכות סחיטה. ויפה פי׳ הגאונים ז״ל דלמשמשא בהו קא מיבעי׳ לי׳ מפני שהן משירין שער ולפי׳ אמר לו בחול מי התירן לאוכלין לרחוץ בהן פניו ידיו ורגליו קסבר משום הפסד אוכלין. 
}
\textblock{\textbf{אם היו מקצת עליו מגולין אינו חושש.} אמרו מקצת הגאונים ז״ל שזה הגי׳ שיבוש אלא ״מקצתם מגולין״ לפי שאם לא הי׳ מגולה אלא עלין זריעה מעולה היא שכן דרך כל זריעתן. ורש״י ז״ל גורס מקצת עלין, וכן במשניות ומתרצין לה בטומן אגודה של לפתות שהוא שלא כדרך זריעה ולהכי שרי וכן אמרו בירושלמי שם. והוי יודע דדוקא בדלא אשריש ודוקא טומן אבל נתכוין לזריעה לא וכן אמרו בירושלמי במס׳ כלאים לפי שאינו רוצה בהשרשתן. ומדקתני ולא משום מעשר, משמע כשנתוספו, שאם לא נתוספו אפי׳ אשרוש ונזרעה זריעה גמורה לשם זריעה נמי אין בו משום מעשר מיהו בשבת דוקא בדלא אשרוש דכיון דמעשיו מוכיחין עליו שאינו רוצה בהשרשתן ועדיין אינן מחוברין מותר, אבל נתכוין לשם זריעה לעולם אסור:
}
\newsection{דף נא}
\textblock{\textbf{אמר רב יודא אמר שמואל מותר להטמין את הצונן.} פירש״י ז״ל להטמינו שלא יחמו ולא גזרי׳ אטו הטמנה כדי שיחמו סבור הרב ז״ל דלחמם אסור אף על פי שהוא דבר שאינו מוסיף. וכמדומה שהגאונים ז״ל סוברין שלחמם נאמרו הדברים כלו׳ שלא יצטנן יותר או שתפוג צינתו קצת ומתניתין הכי דייקא דכולה בהטמנה דחמום נשנית וכן לשון הגמ׳ שאמרו ממתני׳ ה״א ה״מ דבר שאין דרכו להטמין דהיינו מים הכי משמע דלצנן ודאי דרך המים להטמין יותר מכל דבר ולכך הוצרך רש״י לידחק בזה מעט ולומר דאגזירה דלחמם קאי וכן משמע מבריי׳ דלקמן דכיון דשרינן פינן ממיחם למיחם כל שכן בצונן דלא אסרו אלא בחמין ובמיחם ראשון שהוא מצטמק בחמום שלו ודרך הטמנה בכך ואף על גב דסיפא דבריי׳ אסר ת״ק להטמין צונן משום דת״ק לית לי׳ קולא דרבן שמעון בן גמליאל בפינן ממיחם למיחם (ואיהו) [והיינו] דברי׳ סבירא ליה דצונן מותר. ורש״י מפרש גם זו כפי שיטה שלו דפינן ממיחם למיחם כדי לקרר שרי רבן שמעון בן גמליאל להטמין שלא יצטננו ביותר משום דליכא למיגזר בי׳ דהשתא אקורי קא מיקר לה במתכוין אחומי קא מחמם לה וכל זה יותר מחוור ללשון אחר שכתבנו ומיהו דוקא בדבר שאינו מוסיף ומשחשיכה אבל להטמין בדבר המוסיף אפילו מבע״י אף הצונן אסור מדרב חסדא במעשה דאנשי טבריא:
}
\textblock{\textbf{כיצד הוא עושה רבן שמעון בן גמליאל אומר נוטל את הסדינין ומניח הגלופקרין.} פי׳ רבן שמעון בן גמליאל לא בא לחלוק אלא לפרש דברי חכמים כדקתני אמר ר׳ שמעון בן גמליאל כיצד ומשמע דוקא נוטל ומניח אבל לא להוסיף בלא נוטל ואף על גב דתנן כסהו ונתגלה מותר לכסותו ה״מ להחזיר כיסוי הראשונים אבל לא להוסיף אלא ע״י הערמה דהיינו נוטל, וטעמא דמילתא מפני שנראה כטומן בשבת, ואינו נכון. ואומרים אחרים, דוקא נתגלה מאליו אבל גלהו אדם אסור לכסותו וכאן אף על פי שגלהו מותר להוסיף עליו הואיל ולא גלהו לגמרי אלא שנטל (אש) [אחד] מן הבגדים שעליו, ואין זה נכון, שהרי בקופה שטמן בה שנינו נוטל ומחזיר אף על פי שגלהו אדם. אבל נראה לי דלאו דוקא נוטל ומניח אלא משום דבעי [למיתני] תרתי, נוטל סדין ומניח גלופקרין להוסיף, ונוטל גלופקרין ומניח סדין למעט מש״ה קתני הכי. א״נ קמ״ל דאפי׳ אין עליו אלא סדינין מותר להוסיף וטמון קרינא בי׳. ובירושלמי (ד,ג) כ׳, אין טומנין חמין משחשכה אבל מוסיפין עליהן כסות וכלים וכמה יהא עליו ויהא מותר לכסותו ר׳ זריקא בשם ר׳ חנינא אמר אפי׳ מפה אמר ר׳ זעירא ובלבד שיהא דבר שהוא מועיל אמר רבי חנינא כל הדברים מועילין אמר רבי מתניא ויאות אלא מאן דנסיב סמרטוט ויהיב לי׳ על ראשה בשעת צנתה דילמא דלא כבש צנתה:
}
\textblock{גרסת הגאונים ז״ל כך היא: \textbf{מניחין מיחם על גבי מיחם וקדירה על גבי קדירה ומיחם ע״ג קדירה וקדירה על גבי מיחם.} וכ״ג רב אלפס ז״ל, וכן מצינו אנו בתוספתא (ד,יד). ובספרים כ׳ אבל לא מיחם על גבי קדירה ולא קדירה על גבי מיחם ואחרים גורסין ומיחם ע״ג קדירה אבל לא קדירה ע״ג מיחם, וגי׳ הגאונים אמת:
}
\textblock{והא דקתני \textbf{אין מרסקין לא את השלג וכו׳.} פי׳ רש״י דדמי למלאכה שבורא המים האלה אבל נותן הוא לתוך הכוס ואף על פי שנמוח מאליו ובתוספתא תני אבל מרסק הוא לתוך הקערה ונראה דמשום סרך מלאכה נגעו בה הא אלו הניחן בחמה ונפשרו ואפי׳ כנגד המדורה מותרין הן דלאו נולד הוא ולא דמי למשקין שזבו משום דהיינו בעודן קרושין נמי תורת משקין עליהן לכל דבר. ובעל ספר התרומות (ס׳ רל״ה) כ׳ שאסורין משום נולד, ואסור ליתן קדירה שקרש שמנינותא כנגד המדורה משום דמעיקרא עבה וקפוי ועכשיו נמחה ונעשה צלול והו״ל נולד ולפי טעמו אף בחמה אסור השומן דנולד הוא וק״ל בשמן דכיון דלעולם אוכל הוא אפילו לרסק ולסחוט מותר דהא בפירות דלאו בני סחיטה נינהו סוחטין אותן לכתחילה ואפי׳ בתותים ורמונים היוצא מעצמן אם לאוכלין מותר הוא, אלא נראה דהכל מותר בין בחמה בין כנגד המדורה דהא ממילא אתו דומיא דלתוך הכוס ומשום חמום לא מתסר דאולידי הפשר הוא:
}
\newchap{פרק \hebrewnumeral{6} במה אשה}
\newsection{דף נז}
\textblock{}
\textblock{מתני׳. \textbf{לא בחוטי צמר ולא בחוטי פשתן.} פי׳ ושפחות נמי יוצאות בהן דהא בלא תצא אשה קתני להו א״נ נשים ועבדים כלל א׳ הן (ואם כן) מ״ש דקתני נרה״ר בשנמא בכיפה שג צמר קתני לקמן כיעמא שלא תתגנה על בעלה א״ג בתכשיטין גזרו שאין דרך לשלפן מחצר לרה״ר בסימני עבדים נא נזרו וקשיא גי אי כיפה של צמר תנן אמאי אסורה הא תחת שבכה היא כדקתני בדר״ש ותנן יוצאה אשה בטוטפת ובסרביטין בזמן שהן תפורין בשבכה ובדקתני בברייתא יוצאה אשה בשבכה המוזהב׳ ובטוטפת ובסרביטין הקבועים בה משום דלא מישלפה שבכה ואינה פורעת ראסה ושמא י״ל שהכיפה זו למטה מן השבכה היא עונידת אבל השבכה היא מונחת על שערה בגובה של הראש והכיפה טו פדחתה ושער שלפני׳ בראש ומקצת שערה מכסה ע. צד הביכה שכנגד הראש למעלה ויכולה היא לשלוף כיפתה בנא גנוי ראשה מן השבכה ורשב״א שרי כיון שעל השער מגיש היא למטה יון השבכה קצת:
}
\textblock{\textbf{אין בהן משוס כנאים.} כל גדולי המפורשים ז״ל אמרו מפני שאינו ארוג ותינוה הוא שהרי שנינו הלבדין אסורין מפני שהן שוע אעפ״י שאינן ארוג וטווי ולא ידעתי בה תירוץ ואם נאיי. דמדאוריי׳ קאמרי׳ שודאי אין אסור משום כלאים אלא שוע טווי ונוז כדאיירינן במסבת נדה מדאפקינהו רחמנא בחד לישנא הא ניתא דלא תני תנא סתם אין בה משום כלאים אם היה בה כנאים מדבריהם ואפשר שלא גזרו חכמים באסטמא מפני שהוא דבר קטן ואינו עשויה למלבוש אלא לתכשיט בעלמא ואע״פ שגזרו בלבדין כנ״ל לדעת הראשונים אבל בעלי התוס׳ ז״ל אמרו שהוא קשה ולפיכך אין בה משום כלאים כדאמרי׳ במס׳ ביצה האי גמדא דנרש שרי ואפשר לפרש שאין בה חימום כלל ואין אדם מתכוין בה אלא לתכשיט, וזה הוא התירם:
}
\textblock{\textbf{ואין מטמאה בנגעים.} לפי ששם נמי בגד כתיב בהן ובעי׳ טווי וארוג ואצ״ל בשרצים וטמא מת. ובעלי התוס׳ ז״ל אמרו דוקא בנגעים אבל שרץ ומת טמאין מידי דהוה אשכין קטנים שטהורין בזב וטמאין במת וה״נ תזי׳ בכופת שאורשייחדה לישיבה טמאה מדרס ואין בזה טעם כלל שאם הוא בגד אף כנגעים תטמא ואם אינה בגד אף בשרצים לא תטמא דתרוייהו בגד כתיב בהו ועוד דהא גמירי מהדדי לקמן בג״ש מבגד ועור שכל שנתרבה בזה מתורת בגד ועור נתרבה בזה ולענין מדרס ודאי טמאה אם ייחדה למדרס שאפי׳ פשוטי כני עץ נתרבה למדרס. ורבי׳ הגדול ז״ל כ׳ אינה מטמאה בנגעים שאינה שתי וערב וכ״כ ר״ח ז״ל לפי שאלו הי׳ טווי הי׳ טמא משום שתי וערב אע״פ שאינו אריג ואפש׳ שמפני זה אמר בנגעים ואצ״ל בשרצים שאפי׳ יש בה שתי וערב טהורה מאחר שאינו אריג ושמעתי שי״מ לפי שהוא ממיני צבעונין ואין בגד מטמא בנגעים אלא לבן כדאי׳ בתו״כ, ואגב שארא נקט לה:
}
\newsection{דף נח}
\textblock{\textbf{ומי אמר שמואל הכי והאמר שמואל יוצא העבד בחותם שבצוארו.} פי׳ בדין הוא דהו״ל לתרוצי כי אמר שמואל יוצא העבד בשל טיט וכי תנן לא יצא בשל מתכות וכדאמר לקמן ואפשר דמסתברא לי׳ דמתני׳ בין בשל טיט בין בשל מתכות וכן אמר רש״י ז״ל שסתם חותמות של טיט ומ״ה מהדר לאוקמי מתני׳ בשל טיט דעבד איהו לנפשי׳ וכן נמי הי׳ בדין לאוקומי בכל מילי ואפשר שאינו נקרא כבול אלא אותו שבצואני אבל שבכסות לא וכ״נ מדברי רש״י ז״ל שאמר חותם שבצוארו הוא ככלא דעבדא והייתי סובר לומר דהשתא לא קס״ד לאיפלוגי בין טיט למתכות ולקמן במסקנא דמפלגי׳ בינייהו אידחי ליה האי פירוקא הלכך בין עביד ליה רבי׳ בין עביד איהו לנפשיה בשל טיט מותר בשל מתכות אסור. וכ״ז להעמיד דברי רבינו הגדול ז״ל שלא חלק ולא הזכיר כלל עבד ליה רבי׳ ועבד איהו לנפשיה אבל אינו מספיק ליה מפני שאמר לקמן אידי ואידי דעבד ליה רבי׳ ולא קשיא כאן בשל מתכות כאן בשל טיט משמע דעבד ליה רביה אבל עביד איהו לנפשיה אפילו של טיט אסור מדלא קאמר לא שנא עבד ליה רביה ולא שנא עבד איהו לנפשיה:
}
\textblock{\textbf{כולהו רבנן לא ליפקוה בסרבלא חתומי.} פירש״י זכרונו לברכה שהוא סימן לאימת ריש גלותא:
}
\textblock{\textbf{כל דבר אשר יבוא באש, אפי׳ דבור אסור.} אי קשיא, התם תשמישי אדם הכא תשמישי בהמה א״ל מדכתיב כל דבר משמע אפי׳ דבר בהמה ויש אומרים הכא תשמיש אדם הוא דניחא ליה שישמע קול ענבול כדי שלא תגנוב. ואי קשיא, קרא בענין גיעולי כלים כתיב ולא לענין טומאה א״ל אם אינו ענין קאמרינן דגבי אסורין של כלים אין לחלוק בין יש לו ענבל ומשמיע קול לשאין לו ענבל ואסיפא דקרא קאי דכתיב אך במי נדה יתחטא:
}
\textblock{\textbf{הואיל והדיוט יכול להחזירו.} פירש״י ז״ל, לפיכך לא פרחה ממנו טומאתו הישנה אבל מכאן ולהבא אינו מקבל טומאה דהיינו אין להם ענבל טהורין וכן פי׳ גבי טעמיה דרבא דאמר הואיל וראוי להקישו ע״ג כלי חרס דהוה כניקב כפחות מכמוציא רמון שטמא לשעבר וטהור מכאן ולהבא. ורבותינו הצרפתים ז״ל פי׳, טומאתן עליהן לגמרי לקבל טומאה ואע״ג דקתני אין להם ענבל טהורין מפני שנשתנה דין הזוג שלא היה ענבל מעולם ולא נגמרה מלאכתו מהיות כלי מזוג שהיה לו ענבל שזה כיון שנתייחד להשמעת קול עדיין ראוי להשתמש בו מעין מלאכה לפיכך הוא כלי גמור ומקבל טומאה מכאן ולהבא והיינו דאמרינן ואין אומרין בטמא מת עמוד ונעשה מלאכתו כגון בלי שנשבר וראוי למלאכה אחרת שמקבל טומאה אפילו מכאן ולהבא ושלא נטהר מטומאה ישנה:
}
\textblock{\textbf{מתיב רבא הזוג והענבל חיבור.} פירש״י ז״ל חבור לטומאה ולהזאה וכיון שהן חבור ודאי נשברו כיהרו וכ״ת דה״ק דאפילו מפורדין קרי להו חבור ואתו לאשמועינן שאם נפרדו משנטמאו לא עלו מטומאתן והא תניא גבה דההיא מספורת של פרקים. ואין פירוש זה מחוור ונכון, דלא משמע דמשום לשון תבור מייתי לה ועוד דלאו גבה דההוא תגיא דהא הזוג והענבל מתניתין הוא במסכת פרה וזו ברייתא אמרו שהוא בתוספתא דכלים ובר מן דין מנ״ל ממתניתין דכל שהוא כלי אחד אם נשבר אע״פ שהדיוט יכול להחזירו טהור הא מתניתין לא מיירי בההוא כלל והא אביי אמר דכיון דהדיוט יכול להחזיר לא הוה כנשבר כלל ומאי קושיא דמייתי לה. ועוד הקשו לה רבותי׳ הצרפתים ז״ל, מדגרסי׳ במסכת ברכות בפ׳ שלשה שאכלו (ברכות נ.) מטה שאבדה חציה או שנגנבה חציה או שחלקוה האחין או השותפין טהורה החזירוה מקבלת טומאה מכאן ולהבא ומפרש רבא דלמפרע לא משום דפרחה עיומאה ממנה וש״מ דאי לא אבדה כיון דעתיד להחזירה לא פרחה טומאה ממנה והכא קאמר רבא גופיה דפרחה ממנה טומאה. לפי׳ פירשו בתוספות, שלא נאמרו דברים הללו אלא לענין קבלת טומאה אבל הכו מודים שלא פרחה טומאתה ממגה והכי מקשה הזוג והענבל חבור והיינו לענין הזאה ומשמע דוקא מחוברין אבל נפרדו וחזר וחברן אין הזאה שהזה [לזו] מועלת לזה כלום ואי מקבלין טומאה מפני שהדיוט יכול להחזירן ודאי נטמא זה בלא זה וחזרן וחברן שניהן טומאה אחת להן וכן הדין בהזאה אלא ש״מ דכשני כלים חשיבי, וכ״ת אפילו נפרדו נמי והתניא מספורת של פרקים תבור ואמר רבא דוקא בשעת מלאכה ש״מ שאם נתפרדה אינן חבור וכיון שאינם חבור אין מקבלין טומאה ואע״פ שהדיוט מחזירן ומיהו לא דמי לגמרי דהא הזוג והענבל חבור אפילו להזאה, ואין לשון זה כהוגן. ואחרים פירשו דרבא הכי קשיא ליה, כיון דאמרינן שהן חבור לטומאה ולהזיה דגבי הזי׳ נשנית משנה זו ולשון חבור סתם נמי בין לטומאה בין להזיה אלמא כלי אחת הם ואם נטל אחד מהם העיקר טמא והב׳ אינו אלא כידות הכלים שאין מקבלין טומאה אלא ע״נ אביהן והן חבור להזיה כדתנן בסמוך כל ידות הכלים הקדוחות חבור ורבי יוחנן בן נורי אומר אף החדוקות והענבל כיד לזוג הלכך אע״פ שאבד הענבל הזוג טמא ואע״פ שאינו יכול להחזירו לו וקשיא לאביי וכ״ת ה״ק אע״ג דלא מחבר כמחבר דמי מפני שהדיוט יכול להחזירו וצריך הוא לחבר הא זה בלא זה אינן ראוין כלל (א״נ) [א״כ] הוה להו כשני כלים שמשמשין מעין מלאכה א׳ שאין א׳ מהם כלום בלא חבירו וחבור לטומאה ולא להזאה אלא אמר רבא הואיל וראוי להקישו ע״ג כלי חרס לפיכך טמא ואפילו אבד הענבל שאינו אלא כידות הכלים כ״כ הר״ב ברבי יוסף ז״ל. ואין לפי׳ זה שום עיקר למי שמעיין בו, דהא זיג וענבל לאביי כיון דכל חד וחד בפ״ע לא חזי למידי לא יד וכלי הוה ולא שני כלים עושים מלאכה אחת אלא בשניהם עושה כלי וחבור לטומאה ולהזיה ולכל דבר אלא שאם נפרדו כיון שהדיוט יכול להחזירו כמחוברין ועדיין טומאתן עליהן ולא מתני׳ קשיא לאביי ולא ברייתא קשיא ליה. ול״נ דבר ברור, שכלפי שאמר אביי הואיל והדיוט יכול להחזירו דאלמא כלי א׳ הוא ממש וכשנחלקו ואינו ראוי לקבל טומאה דלא תזי למידי אלא מפני שהדיוט יכול להחזירו הוא שמקבל טומאה לפום הכי אקשינן מתניתין דקתני חור דהיינו לומר שמביאין טומאה זה לזה ואי כלי א׳ הוו מאי חבור אין זה מביא טומאה על זה אלא במקצתו של בלי נגעה טומאה ושניהן הוא שמקבלין טומאה לא הא׳ מהן ולא מיתני האי לישנא אלא בשני כלים מחוברין כענין בשלל של כובסין וכן שנינו בשלשלת המפתחות וכץ בידות הכלים. וכ״ת ה״ק, אע״ג דלא מחבר כמאן דמחבר דמי לומר שמקבל טומאה בפ״ע מפני שהענבל כמחובר לו והתניא במספורת של פרקים ואיזמל של רהיטני שהדיוט ודאי מחזירן ואעפ״כ אינו חיבור מן התורה אלא בשעת מלאכה וטעמא דמלתא כיון שדרך אומנין להפרידן שלא בשעת מלאכה כמפורדין דמי וקשיא לאביי דלדידי׳ אפי׳ בשעת פירודן דהוו כמחוברין וברייתא דקתני אפילו בשעת חבורן הוו כמפורדין והלכך כל היכי דלא חזי למידי בפ״ע לא מקבל א׳ מהן טומאה בשעת פירודן בשביל חזרת הדיוט וכענין ששנינו מספורת שנחלקה לשנים ר״י מטמא וחכמים מטהרין ובדין הוא דליקשי מהך ברייתא בפ״ע מעיקרא אלא אורחא דתלמודא הוא בכמה דוכתי, וזה פי׳ נכון ומחוור:
}
\textblock{הא דאמרינן \textbf{יש עליה חותם חייבת.} בדין הוא דהוה ליה לאתויי מסיפא דקתני ואם יצאת חייבת חטאת אלא אורחא דתלמודא כההוא דאמרינן לעיל בפרק כירה (שבת מו:) כבתה אין לא כבתה לא ואף ע״ג דהיה יכול לאתויי מרישא דקתני בהדיא חוץ מן הנר הדולק בשבת וכיוצא בה בפ׳ אלו מציאות וכיוצא בה בפ״ק דב״ב (כז.) ורבות אחרות:
}
\textblock{הא דקתני ברייתא \textbf{ור׳ נחמיה מטמא.} וה״ה דפליג בהיא של מתכות וחותמה של אלמוג דרישא שטהורה והיינו דמפרש עלה בטבעת הלך אחר חותמה והיינו בין לטומאה בין לטהרה וה״נ תני לה בהדיא בתוספתא בדוכתא[כלים ב״מ ג,ח]. ובירושלמי דפרקין נמי נותני רבי נחמי׳ מחליף, וכן פירש״י ז״ל, והיינו נמי דמחייב חטאת במתני׳דאי ס״ד היא של מתכות וחותמה של אלמוג דרישא מודה בה רבי נחמיה אם כן כבר הודה דתכשיט הוא משום (המעמד) [הטבעת] כברייתא דמטמא. ומיהו צריך טעם למה שנו כאן מטמא [ולא מחליף]:
}
\newsection{דף ס}
\textblock{\textbf{אמר ליה אביי ותיהוי כבורית טהורה ותשתרי.} לפי פירש״י ז״ל ללו דוקא טהורה, אלא שם בורית נקט כלישנא דמתניתין דקתני בורית טהורה ויוצאין בה. ואחרים פי׳ דתרתי קפריך א״כ תיהוי כבורית לגמרי ותהא טהורה שאינה לאו כלי ולא תכשיט אלא שמוש צניעות בפולמא, ואנן אמרינן לעיל (שבת נב:) אבל לענין טומאה זה וזה שוין ותהא נמי מותרת לצאת בה דהואיל ולצניעותא עבודי לא משלפא:
}
\textblock{והא דאמרינן \textbf{תינח בחול בשבת למאי חזיא.} פירש״י ז״ל, הרי בשבת אינה חולקת שערה ואיני יודע מי אסרה והרי אמרו נזיר חופף ומפספס וה״ה לשבת דקי״ל כר״ש ואין זה פסיק רישי׳ ולא ימות ועוד אם מותרין לחלוק בה שער תצא בה לרה״ר, והרי אין כאן תכשיט אלא כלי. אלא [ה״פ]: הא תינח בחול כלו׳ דחזי ליה ולפיכך טמאה בשבת לענין הוצאה מאי חזיא לה ובי חולקת היוץ שערה ברשות הרבים ומפני שחולקת שערה יהא מותר להוציאה בתכשיט ומפרק כמין טס של זהב יש לה והיא נוי לה ותכשיט כשמניחתה כנגד פדחתה:
}
\textblock{הא דתניא \textbf{רבי נהוראי אומר חמש מותר שבע אסור.} הול״ל שש אסור, אלא שאין הסנדלין עשוין לעולם בזוגין אלא או ה׳ או ז׳ כדאמרינן שנים מכאן ושנים מכאן וא׳ מתרסיותיווהא דתניא לקמן ארבע או חמש ההוא כשנאסרו והא דא״ל רבי אסי לר״ה שמנה אסור לאו דוקא איידי דאמר ליה שבע מותר אמר נמי שמנה אסור, או שמא יש שעשוין בשמנה ואין עשוין בשש:
}
\newsection{דף סב}
\textblock{גמרא: \textbf{אמר עולא וחלופיהן באיש.} כתב רבינו הגדול ז״ל דאטבעת ואמחט קאי, ולא ידעתי למה פי׳ כן ואם מפני שאמרו וחלופיהן באיש אין בזה נ״ט מפני שהטבעות שתים כמ״ש למעלה ורבי׳ שרירא ורבינו האי גאונים ז״ל פרשוה אטבעת בלחוד, וכן פיר״ח ורש״י ז״ל, וכל סוגי׳ דשמעתא אטבעת איתמר. ויכולין אנו לפרש פי׳ רבינו הגדול ז״ל, דמחט נקובה לאשה משוי וחייבת חטאת משום דרך הוצאה לאשה במחט שבראשה ואיש פטור אבל אסור עליה שאין דרך הוצאה לאיש במחט תחובה בבגדו ובראשו וכך אמרו בירושלמי ושאינה נקובה אשה פטורה אבל אסורה דתכשיט הוא ואיש חייב עליו חטאת שכל המוייא מחט שאינה נקובה לשום דבר כגון ליטול בה את הקוץ תחובה בבגדו הוא מוציאה ודרכו בכך שתהא מצוי׳ לו תמיד משא״כ בנקובה שדרך הוצאתה בבד של מחטין וצינוריות היא ואין דרך להוציא אחת דלמאי צריך לה ואם צריך לה ביוו הוא מוציא כשאר כל הדברים ומיהו הא דהוו בה בגמ׳ והא הוצא׳ כלאחר יד הוא אטבעת קאי בלחוד, וכ״ז אינו נכון:
}
\textblock{והא דאמרי׳ \textbf{תירצת אשה, איש מאי א״ל.} ק״ל, לפי פי׳ הנגיד ז״ל שהזכרנו, לימא דעולא ה״ק, חילופיהן באיש לתכשיט ומשוי דשאין עלי׳ חותם תכשיט לאשה ומשוי לאיש ויש עליו חותם משוי לאשה ותכשיט לאיש ולא שיהא זה חייב עליו חטאת [ו]זה פטור אבל אסור וזה מותר לכתחלה, אלא משמע דחילופיהן באיש אדינין דמתניתין דחייב עלי׳ ופטור קאי או משום שחוששין בתכשיטי האיש למשליף ולאחויי או משום דגזרי׳ איש אטו אשה כדפרשי׳ [בתכשיטין] השוין בשניהם כגון טבעות קרוב לטעמו של ר״ח ז״ל ובפ׳ נוטל (שבת קמב.)תני׳ המוציא כליו מקופלין ומונחין לו על כתפו וסנדליו וטבעותיובידו חייב ואם הי׳ מלובש בהן פטור וקי״ל כל פטורי דשבת פטור אבל אסור דתרגומא אטבעות ומדלא קתני מותר ובאיש, ש״מ אפי׳ איש אסור:
}
\textblock{\textbf{אריג כ״ש טמא.} פי׳ כגון שארג דבר קטן בפ״ע לשמש ואינו בא מבגד גדול ולא ראוי להוסיף עליו א״נ אפי׳ בא מבגד גדול כיון שכ״ש שלו ראוי ואינו ראוי להיות גדול יותר כנון איבי חלילא וכיוצא בהן. והאי דמפקי׳ לה מאו בגד, ק״ל הא אפיקתי׳ בפ׳ במה מדליקין לרבא לנ׳ על ג׳ בשאר בגדים לאביי לרכות שלש על שלש בצמר ופשתים דמטמא בשרצים ואפשר דהאי או בגד כולה יתירה הוא דמכדי לגזירה שוה בעור סגי ובנד גופא משכבת זרע ילפי׳ הילכך בגד דכתב רחמנא לאריג כל שהוא או לשלשה על שלשה בשאר בגדים, א״נ איפכא:
}
\textblock{\textbf{האי במדין כתיב.} פירש״י ז״ל גבי טמא מת, בשרץ מנ״ל. והא קשי׳ הא גמרי׳ (סד:) גזירה שוה בגד ועור בגד ועור שרץ ממת, ועוד דהא רבינן נמי מאו בגד דשרצים ולא מקשינן האי בשרצים כתיב מת מנ״ל דהא איכא נמי פירכא לקמן בהא ועוד לישנא דגמרא לא משמע הכי דהול״ל תינח מת בשרץ מנ״ל. אלא ה״פ: האי במדין כתיב גבי קרבן שהקריבו אנשי הצבא ולא גבי טומאת מת כלל ואתם ושביכם לא משמע כל שביכם דהאיכא בהמות ואדם ומפרקי׳ גמר כלי כלי מהתם כתיב הכא וכל כלי מעש׳ וכתיב התם וכל בנד ובג כלי עור וכל מעשה עזים וכל כלי עץ תתחטאו וכיון דאית למת גמרי׳ לשרץ בגזירה שוה:
}
\newsection{דף סד}
\textblock{\textbf{דמת נמי אפנויי מופנה מכדי מת איתקש לש״ז.} איכא למידק, א״כ גזירה שוה ל״ל כיון דכתיב או שק בשרץ לרבות דבר הבא מזנב הסוס ומזנב הפרה ממילא נתרבה אף בש״ז וכיון שנתרבה בש״ז ממילא שמעי׳ למת ומיעוכי דחבלים נמי דאתי משק מיהא גמור וכן נמי איכא למיגמרשרץ ממת לענין קנקי וחבק דהא איתקש תרווייהו לש״ז ואי אתה יכול לומר משוס דהו״ל דבר הלמד מהיקש ואינו חוזר ומלמד דלא איתמר הכי אלא לענין קדשים אבל בכל התורה למידין למד מלמד אלא א״ל דמלתא דאתי׳ בהקישא כה״ג טרח וכ׳ לה קרא בג״ש, וה״נ משמע במס׳ ב״ק (כה:) גבי פכין קטנים:
}
\textblock{מתני׳: \textbf{בכבול ובפיאה נכרית לחצר.} איכא דקשי׳ לי׳, הא מרישא שמעת מינה דקתני (נז.) לרשה״ר, הא לחצר שרי. וכן נמי טוטפת וסרביטין בזמן שתפורין מרישא שמעת מינה דקתני ולא בטוטפ׳ ולא בסרביטין בזמן שאינן תפורין הא תפורין ש״ד וא״ל דהנך נמי מיצרך צריכו דאי מרישא ה״א ה״ה גמי לטוטפת וסרביטין אלא דאתא לאשמועי׳ דאפי׳ אינן תפורין נמי אינה חייכת חטאת וכן אתה מפרש במחט וטבעת וכן נמי אי לא הדר ותני בכבול וכפיאה נכרית לחצר ה״א דכולהו נמי לא אסרי אלא נרה״ר אבו לחצר לא מיירי מידי דהוה אכבול וז״ה הדר ותני גי׳ למימרא דוקא כבוו התירו לחצר ולא שאר התכשיטין. וק״ל אמאי לא תנא ברישא פיאה נכרית בהדי בבול ואפשר שבכבול עצמי די נו ללמוד לשאר התכשיטין שלא יצא בהן לחצר. ול״נ דאורחא דתנא הוא ל׳ייתני איסורא ומהדר ומתנינהו להתירא משוס חדא רבותא בעלי.א כדתנן באלו טריפות (חולין מב.) ואלו כשרות (שם נד.) וכן בפסולי אתרוג (סוכה כט:), וכן כיוצא בהן, אף כאן משום חוטי שער והנך דהתירא תני נמי הנך להתירא. והוי יודע שלדברי הכל כבול דשנוי ברישא די׳תני׳ הוא השנוי בסיפ׳, ועולא דמפרש כדי שלא תתגנה על בעל׳ ס״ל כמ״ד כיפה של צמר תנן, ומ״ד כבלא דעבדא תנן אמר האי טעמא בפיאה נכרית וגבי כבול שלא יכעוס רבה עליה לומר שהיא מראה עצמה לדעת כבת חורין, א״נ לא גזרו בסימני עבדית בהצר דנאו תכשיטין נינהו וכי נפקא שלפא, דאדרבה בהכי ניחא לי׳. ואין דברי רש״י נ״ל נוחין בזה, שאמר דהכא לד״ה כיפהשל צמר תנן, ואם כן קשי׳ סיפא למאן דאמר כבלא דעבדא תנן:
}
\textblock{\textbf{אמר רב כל שאסרו חכמים לצאת בו לרה״ר אסור לצאת בו לחצר.} לאו דוקא חצר, אלא אסור ללבשן בבית כלל וכל שאסרו חכמים היינו בין אותן שאמרו אינו חייב חטאת בין אותן שאמרו חייב חטאת ובלבד דרך מלבוש כיון דלבוש בהן בבית חוששין שמא יצא בהן לשוק שאין דרך לפשוט ושוכח ויוצא הוא וליכא הכא למימר היא גופה גזירה בשאינו חייב חטאת ואנן נגזור בבית וחצר אטו רה״ר דהא אמרן בתכשיטין שהוא לבוש אין דרך לפשטן והיינו דאמרי׳ בפ׳ כירה השירין והנזמין והטבעות הרי הן ככל הכלים הניטלין בחצר ואמר עולא מ״פ׳ הואיל ואיכא תורת כלי עליהן אלמא אסור ללבשן בין בבית בין בחצר והיינו דאמרי׳ ישרינן כבול שלא תתגנה על בעלה והיינו בבית ומיהו אלה ורומח ביד בחצר מותר דהא לאו דרך מלבוש הוא וטלטול כעלנוא הוא וכלים הללו נמי לצורך מטלטלין הן והכי איתא לקמן בדוכתא. ושמא נאמר דבבית מותר ולא אסרו אלא בחצר דדיירי בי׳ רבים ודמי לרה״ר ושלפא ומחוייא התם ואיכא לא תתגנה על בעלה דחזי לה בחצר שאלו ברשות הרבים אינו רואה אותה ומה שאמרו הרי הן ככל הכלים הניטלין בחצר לומר דאף על גב דלא חזי התם ללבישה, ואף להכניסן לבית בידו לא חזי כשאינה מעורבת נטלין הן בחצר משום דחזי לכסויי מאני. וראיתי לר׳ משה הספרדי ז״ל שפי׳ אסור לצאת בהן לחצר בשאינה מעורבת אבל למעורבת מותר ואצ״ל בבית וק״ל אם כן אפילו בחצר שאינה מעורבת ראוין הן ללבישה שם כשם שראוין לכסויי מנא שהרי מותר לטלטל בכולה (ולא) [דהא לפ״ז לא] אסרו אלא לצאת בהן מבית לחצר מפגי שאסור להוציא כלום דרך טלטול מבית לחצר או עבדר כעין גזירה דילמא שלפא ומחוייא ומפקא להו ביד מבית לחצר, ועוד דלא נהיר, דחצר סתם לאו שאינה מעורבת משמע:
}
\textblock{הא דתנן \textbf{שן תותבת ושן של זהב.} לא חדא קתני, שן תותבת שהיא של זהב כמו שפרש״י ז״ל, דא״כ שן שן ל״ל ולאו אהתירא דרישא קאי כל״א שפרש״י ז״ל דהוה ליה למימר ובשן תותבת. אלא תרתי קתני, ובתרוייהו פליגי. ומפורש בירושלמי (ו,ה) א״ר יוסי מסתברא של זהב שעמדה לו ביוקר לא תצא דאי נפלה ודאי מחזרי׳ ליה ואקשי׳ שן תותבת מאי אית לך ומפרקי׳ עוד הוא דמבהתא למימר לנגרה דעבד לה חורי וכי נפלה ידאי מחזרי׳ לי׳ נראה שהוא של עץ ופי׳ תותבת גיורת ר״ל נכרית מלשון תותב והכוונה שנתישבה תחת שינה שנפלה עיין בפי׳ המשנה להרמב״ם ז״ל ומלת תושב פי׳ עיין בהראב״ע ז״ל פרשת לך לך ולפי שהזכירו בירושלמי נגר ואסרוה מפני שהיא בושה לומר לנגר לעשות לה לפי שהנגר יכיר בה שאין לה כל (שיעורו) [שיניה] שאם היה לה למה היא עושה שן של עץ אבל של כסף ושל זהב אין בה משום חששא זו לפי שיכולה היא לומר שלנוי היא עושה ליטלה בידה כמו שמנהג הנשים לעשות צורת כל דבר מזהב ומתוך שדמי׳ יקרים אסור מ״מ לגמ׳ דילן לא משמע דמשום דלמא נפלה היא כלל אלא משום דילמא שלפא ומחוייא ושל כסף דמותר משום דדומה לשן ושל זהב כיון שמכירין בה מחוייא היא להו תכשיט דידה ולר׳ כיון שנשארה חסרה שן לא מחויא ושן תותבת נמי לחכמים מחוייא דתכשיט הוא לה אע״ג דמגנייא בה. וי״מ שן תותבת שהיא עשוי׳ להושיבה ע״ג שן שחורה והיא כעין דפוס לשן ומפני כך היא בושה להראות לנגר או לאומן לעשות לה אתרת שהרי צריך הוא לעשותה בדפיסה אבל שן של זהב שהיה מונחת במקום שן שהיא חסרה אינה צריכה כל כך ואינ׳ צריכ׳ להראות שני׳ לאומן:
}
\newsection{דף סה}
\textblock{מתני׳: \textbf{הבנות קטנות יוצאות בחוטין.} פירש״י ז״ל, הבנות הקטנות שמנקבות אזניהן (ועושין) [ואין עושין] להן נזמין עד שיגדלו יונותנין חוטין או קסמים באזניהם שלא יסתמו נקבי אזניהן. וי״מ חוטין שקולעת בהן שערה דאסרי׳ נכי גדולה בר״פ משום דשכיחא בהו טבילה אבל קטנה דלא שכיחא בה טבילה ש״ד. ולשון ראשון הגון מזה, חדא דקטנות גמי צריכות טבילה דטומאת מגע משום טהרות ואף נדות נמי שכיחא בהו כדתניא מעשה שהיה והטבילוה קודם לאמה ועוד מדקתני אפי׳ בקסמין איכא למשמע דחוטין נמי אאזנים קיימי דאי חוטי הראש לא שייך למיתני אפי׳ בקסמין אלו דברי רש״י ז״ל. אבל בירושל׳ [ו,ה]נראה שהן חוטין שבראשן דגרסי׳ התם אמתני׳ דיוצאה אשה בחוטי שער בין משלה בין משל חברתה ר׳ חנן בר אמי קומי רבנן יהודה בר מנשה בר ירמי׳ בלבד שלא תצא לא ילדה כשל זקנה ולא זקנה בשל ילדה והא תנינן הבנות יוצאות בחוטין ר׳ בא בשם ר׳ יהודה אפי׳ כרוך על צוארה פי׳ ואע״פ שאין דומין לשערה וה״ל כילדה בשל זקינה וזקינה בשל ילדה ופריק תמן שאינה יכולה להביא חוט (שאינו) [שהוא] דומה לשערה היא יוצאה בהם, הכא בלבד שלא תצא ילדה כשל זקנה וזקנה בשל ילדה פי׳ משוס דמיגנייא בהו ואתי לשלפויי, נראה מכאן שאינן חוטין שבאזנים. ומה שהקשה רש״י ליחוש משום דילמא מתרמיא לה טבילה של מצוה י״ל שלא תששו לכך אלא בגדולה דשכיחא לה דמים והיא רדופה לביתה אבל קטנה א״נ טבלה לטהרות אינה רדופה כל כך ועוד דלא שכיחא בה טבילה ומילתא דלא שכיחא לא גזרו בה רבנן ועוד שהחוטין לבנות שיוצאת וראשן פרוע צריכי להו טפי ולא מנשיא להו כי היכידלייתינהו ארבע אמות ברשות הרבים. ומיהו מאי דקשי׳ ליה לרש״י מאי אפי׳ משמע ולבני מערבא נמי קשר דגרסי׳ תו התם אבא בר ממל מפקיד לשמואל ברר לא תפק תקבל עלך מיתני אלא אבל לא קסמין שבאזני׳ מ״מ לפי גמרא שלנו לא משמע דהתניא בריש פרקין הבנות יוצאות בחוטין אבל לא בחבקים ומשום טבילה ועוד בנתר דאבוה דשמואל נשואות היו ומקשי׳ עלה ממתני׳ אלא משמע בחוטין שבאזניהן הן ואינו מהודקין לחוץ בטביל׳ וכדתניא לעיל וקסמין נמי אינן חוצצין בטכילה ומשום נפילה נמי קשורין באזנים הוו, אי נמי אין מקפידות עליהן להביאן:
}
\textblock{\textbf{ומפצי ביומי תשרי.} פרש״י ז״ל מפני טיט הנהרות, ואפש׳ שהי׳ עבה ביותר, אבל אין טיט חוצץ אלא טיט היוצרים והבורות וכיוצא בהן, ואע״פ ששנינו הטביל את המטה אע״פ שרגליה שוקעת בטיט העבה טהורה מפני שהמים מקדמין ומשמע שאין טיט העבה חוצץ בתוך המים התם דוקא במטה אבל ברגלי אדם חוצץ שהוא נכנס לו בין האצבעות ונדחק שם וכן פר במס׳ נדה דאמרי׳ אשה לא תטבול בנמל מפני הטיט ואין זה מחוור שאין בנמל טיט יותר משאר בורות. והגאונים ז״ל פירשו לשתיהן מפני הצניעות לפי שהן מתייראות שמא יבוא אדם ויראה אותן ואינן טובלות כתקנן לפיכך הי׳ עושה להן מפצי לצניעות:
}
\textblock{\textbf{נשים המסוללות זו בזו פסולות לכהונה.} פירש״י ז״ל לכהן גדול לפי שאינן בתולות. ואין הלשון נוח לפרש כן, ועוד דגרסי׳ בפ׳ הערל (יבמות עו.) אמר רבא לית הלכת׳ לא כברא ולא כאבא, מאי הוא, דאר״ה נשים המסוללות זו בזו פסולות לכהונה, ואפי׳ לר״א דאמר פנוי הבא על הפנויה שלא לשם אישות עשאה זונה ה״מ איש אבל אשה פריצותא בעלמא הוא, ומשמע דר״ה זונה קרי לה ואפי׳ לכהן הדיוט אסרן, וכ״כ שם רש״י ז״ל עצמו פסולות לכהונה משום זנות. ומיהו איכא למידק מההוא דגרסי׳ בפ׳ הבא על יבמתו אנוסת חברו ומפותת חברו לא ישא ואם נשא ראב״י אומר הולד חלל ואמרי׳ מ״ט דר״א ואר״ה א״ר וכן אמר רב גידל א״ר סביר׳ ליה כר״א דאמר פנוי הבא על הפנויה שלא לשם אישות עשאה זונה ואקשי׳ ומי ס״ל והא קיימא לן משנת ראב״י קב ונקי ואלו בהא אמר רב עמרם אין הלכה כר״א קשיא ומאי פרכא דהא ר״ה כר״א סביר׳ ליה א״ל מדרב קא מקשי׳ לרב עמרם דרב גברא רבא הוא ורביהו דכולהו ולא פליגי עלי׳ א״נ לגמרא קשיא ולאו לרב הונא ויש לפרש לר״ה נשים עושות זו עם זו זונות אפי׳ לרבנן מפגי שאין בהן אישות, ורבא קאמר דאפי׳ לר״א אינן עושות:
}
\textblock{\textbf{שמא ירבו נוטפין על הזוחלין.} [יפה] פירש״י ז״ל, ומ״ש במקצת הספרים והוו להו שאובין אינו נכון ואין כ״כ בנוסחי עתיקי. אבל ר״ח ז״ל כ׳ כלשון הזה, שמא ירבו מי גשמים שהן נוטפין מן הגנות ונעשו שאובין על מי פרת שהן זוחלין כדתנן במקוואות הזוחלין כמעיין והנוטפין כמקוה כו׳ פי׳ לפי שהשאובין פוסלין את המעיין ברוב ולא לגמרי אלא דבעי אשבורן כדתנן התם במס׳ מקואות למעלה מהן מעיין שמימיו מועטין שרבו עליהן מים שאובין שוין למקוה לטהר באשבורן ולמעיין לטהר בכל שהוא ומשום הכי עביד להו , מקוואות ומושך לתוכן מאותן מימות דאשבורן מטהרין. אבל אין לשון נוטפין אלא כדברי רש״י ז״ל כדתנן והוטפין כמקוה ואי שאובין הן האיך הן כמקוה אלא מי גשמים נקרין נוטפין וכ״נ שאף היורדין מן המעיין דרך אויר נקרין נוטפין וכן דרך המשניות במקומן (מכשירין ב,ד). וק״ל, היכי אמר [שמא] ירבו הנוטפין על הזוחלין אפי׳ מחצה נמי פסולין כדתנן בעדיות העיד ר׳ צדוק על הזוחלין שרבו על הנוטפין שכשרין אלמא מחצה על מחצה פסולין ואפשר דלאו דוקא וה״ה למחצה על מחצה שפסולין אלא לפי שא״א לצמצם לא קפיד בלישנא. ול״נ שהזוחלין שהלכו למקום הנוטפין ונתערבו הרי הן כנוטפין עד שירבו עליהן אכל הנוטפין שנתערבו בזוחלים כפרת ביומי ניסן אינן פוסלין לעולם עד שירבו הנוטפין על הזוחלין, שבכל כיוצא בזה במקומן חשיבי. והוי יודע שהנוטפין בעו אשבורן ואינן מטהרין בזוחלין אבל מעיין מטהר כין בזוחלין בין באשבורץ ממאי מדתנן (מקואות א,ז)שוה למקוה לטהר באשבורן, וקתני עלה למעלה מהן מים מובין שהן מטהרין בזוחלין והיינו אפי׳ בזוחלין דאי בזוחלין דוקא מאי למעל׳ מהן הכא בחד גוונא מטהרי והכא בחד גונא מטהרי ותנץ נמי מעין שהוא צר וחקקו ועשאו רחב מטהר באשבורן ואינו מטהר בזוחלין אלא במקום שהמים יכולין לילך שם מתחלה אלמא מעין נמי מטהר באשבורן ותנן במס׳ פרה פ״ו המפנה מעיין לתוך הגת או לתוך הגג פסולין לזבין ולמצורעין ולמי חטאת אלגנא לשאר כל אדם כשר אע״פ שהוא אשבורן, ולקמן בפ׳ ח׳ שרצים (שבת קט.) אמרי׳ כל הימים כמקוה, ר׳ יוסי אומר כל הימים מטהרין בזוחלין והיינו אף בזוחלין שלא פסלן באשבורן תדע שאם לא יאמר כן פרת לא יטהר נמי באשבורן שמא רבו זוחלין על הנוטפין ונא אמרו אלא אין המים מטהרין בזוחלין הא באשבורן כולן מטהרין הן: }
\textblock{הא ד\textbf{אמר שמואל נהרא מכיפי׳ מבריך.} תימה הוא, שהרי אנו רואין שכשיורדין נשמים הנהר רבה וי״ל מפני שאמרו שאין לך כל טפח וטפח יורד מלמעלה שאין תהום עולה לקראתו טפחיים כדאיתא בשלהי פ׳ סדר תעניות וא״ת א״כ היכי שמטית מינה דפליגי דשמואל אדרב דאמר סהדא רבא פרת א״ל דלשמואל כיון דנהרא מכיפי׳ מבריך פעמים שאין הנשמים יורדין ותהום עולה ומשקה הארץ ונהרי מברכי אבל לרב אין עלייתו של תהום מבריך הנהרות אלא הגשמים היורדין לתוכן ועוד שאין זה סהדא רבא שפעמים גשמים מועטין ותהום עולה הרבה. ונראה מדברי ר״ח ז״ל שהלכה כרב ואף רבינו הגדול כתבה לדאבוה דשמואל אע״פ שלא כתבה לדרב וכן סלקא שמעתא דהא שמואל נמי ס״ל כאידך ואיהו לא סמך נפשי׳ לענין איסורא אנן היכי סמכינן ואף על גב דאמרי׳ בגמ׳ ופליגא דידי׳ אדידי׳ טעמא פליגי אבל מימר א״ל דלא פליגי אלא תייש לשמא רבו מן הנוטפין וכיוצא בזו כפסחים ופליגי דידי׳ אדידי׳ גבי ספיחי שביעית דטעמי פליגי אבל עיקר דינא א״ל דרבי שמעון בן יוחאי ראוי לסמוך עליו בפניו דרבה בב״ח קטן ממנו הי׳ ולא חשש לכבוד עצמו להסמיך אחרים עליו אפי׳ בפניו, ועוד דאמרי׳ כי ההוא דפרקי׳ לעיל (שבת נד.) השתא דאמר שמואל מכיפיה ואמר אין מטהרין ש״מ הלכה למעשה קמ״ל (וע״כ) הלכה כרב, דרב ושמואל הלכה כרב באיסורי, ועוד דרבים נינהו דהא איכא אבוה דשמואל. אלא שקשה עלינו שנהנו לטבול בנהרות בין ביומי ניסן בין ביומי תשרי. ויש מי שאומר [ר״ת] הלכתא כשמואל דאמר מכיפיה מבריך מדאמרי׳ במס׳ בכורות תניא רמ״א יובל שמו ולמא נקרא שמו פרת שמימיו פרין ורבין מסייע ליה לשמואל דאמר שמואל נהרא מכיפיה מבריך וכן פסק בס׳ היראים בשם (ר״ח) [ר״ת] ז״ל, ואין הפסק נכון ולא ראוי לסמוך עליו שיותר יש לנו לסמוך עג הלכת׳ פסיקת׳ דאתמר גבי׳.עיקר ושמטתין כרב סלקא והתם לא מסייעא לשמואל אלא מדר״מ ורבנן הא אמרו פרת שמו ולא אמרו שמימיו פרין ורבין והיינו טעמא נמי דרב ואבוה דשמואל וא״ת א״כ מאי סייעתא דר״מ י״ל משום דלא חזי׳ לרבנן דפליגי עלי׳ בהא שלא יהיו מימיו פרין ורבין ומיהו רב ואבוה דשמואל אמרי לך פליגי ואנן דאמרי׳ כרבנן ומיהו יש נוסחאות שכ׳ בהן וחכמים אומרים פרת שמו שמימיו פרין ורבין ולמה נקרא שמו יובל וכו׳ ומסייע ליה לשמואל מדרבנן ומדר״מ, והכי כתוב בס׳ היראים. וק״ל מאי סייעתא פרת הוא שמימיו פרין ורבין ולא נהר אחר ובשביל כך נקרא כן ואי ס״ד שכל הנהרות מתברכין מכיפיהו מה נשתנה זה שנשתנה שמו וי״ל שפרת רבה הרבה יותר משאר נהרות לפיכך נקרא כן שהוא גדול לפעמים יותר על מה שהי׳ מתחלה ואין זה מצוי בשאר הנהרות ובמס׳ ברכות פ׳ הרואה נמי אמרי׳ למה נקרא שמו פרת שמימיו פרין ורבין. עי״ל שכל הנהרות כפרת כדאמרי׳ התם הנודר ממי פרת אסור בכל מימות שבעולם דאפי׳ עינתא דמדליין סולמי דפרת נינהו הלכך אם רבוי פרת מכיפיה שאר נהרות ממנו ושמא יש להקל ולומר שנדין בזה לפי מה שנראה בנהרות שהרי אנו רואין שאין הנהרות רבות בימות הגשמים יותר עג שהיו כל השנה אלא מעט או יום א׳ וב׳ לאחר הנשמים. ואי , אפשר שיהיו בו נוטפין רבים על הזוחלין אלא שהי׳ פרת [רבה] הרבה כימות הגשמים קרוב למה שהי׳ בתחלהולפעמים יותר, ולפיכך הי׳ חושש אבוה דשמואל וכן יש לחוש במקצת נהרות קטנים שהן יבשין בימות החמה וכל ימות הגשמים ומקצת ימות החמה הן מושכין מים הרבה אפילו שלא בשעת הגשמים עד שיכלה הפשרת שלגים בימות הקיץ שהן מתייבשין מ״ה אמר שמואל אין המים מטהרין בזוחלין אלא פרת ביומי תשרי לבד על כיוצא באלו שאמרתי אבל לא כלל כל הנהרות בכלל זה. וי״א, בבל עמוקה וכל מימי הגשמים יורדין שם כדאמרי׳ למה נקרא שמה שנער ששם ננערו כל מתי מבול ולפיכך חוששין בנהרותי׳ אבל לא אמרו במקומות אחרים ובס׳ היראים נמי הכי כתוב אע״פ שהלכה כשמואל מיהו לנהרות קטנים יש לחוש לרביית נוטפין מסברא ממתני׳ דתנן במס׳ פרה פ״ח המים המכזבין פסולין אלו הן המים המכזבין א׳ בשבוע פי׳ א׳ לז׳ שנים המכזבין בפולמוסיו׳ בשני בצורת כשרין ור״י פוסל ולדעתנו אנו יש לחוש בכל נהר עד שיתברר לך מעין שלו כמה הוא ומאיזה מקום המשכתו ורבויו באין בין בימות החמה בין בימות הגשמים והלכתא ודאי כרב ואין תולין ברביית כיפה ועליית תהום כשמואל. וכבר הקשו בשמועה זו, ואפי׳ רבו נוטפין על הזוחלין מאי הוו והרי פרת נהר גדול הוא ומושך הוא לעולם ותנן במקוואת מעין שהוא מושך כנדל ורבה עליו והמשיכו הרי הוא כמות שהיה היה עומד ורבה עליו והמשיכו שוה למקוה לטהר באשבורן ולמעיין נהטביל בכל שהוא וי״א מאי ורבה עליו דקתני שהוסיף עליו והמשיכי הא אלו רבה במעיין מים מרובין על שלו אינו מטהר אלא באשבורן, והקשו בזה א״כ ליתני הוסיף. ול״נ לפי ענין המשניות שהמעיין נקרא גומת העיין עצמה ואמה המושכת ממנו נקרא זוחלין כענין ששנייות הזוחלין כמעין (והזוחלין) [והנוטפין]כמקוה, ושנו בתוספתא (מקואות א.) לענין החופר בצד הנהר והחופר בצד המעין שאין הנהר המוסיף נקרא מעין אלא עין המים בלבד ולפיכך אמנ׳ו שהמעין שהוא מושך אפי׳ כנדל ורבה שאובין בתוך המעיין עצמו עדיין הוא כמות שהיה ומטהר במים הנזחלין ממנו ובלבד במקום זחילתן הראשונה וכן מפורש בתוספתא. היה עומד אע״פ שרבה ממש לתוכה של מעין כיון שכל זחילתו מחמת הנוטפין צריך אשבורן וזה ששנינו בתחלת מקוואות למעלה מהן מעין שמימיו מועטין שרבו עליו מים שאובין שוה למקוה לטהר באשבורן היינו מימיו ממועטין ועומדין הא רבים ומושכים אע״פ שרבה עליו כשרין נטהר בזוחלין וכ״ז במרבה לתוך המעין עצמו אבל במרבה לתוך המים הנזחלין לעונם נפסלין ברוב שאובין וברוב נוטפין והיינו שמעתין, וכ״נ דברי ה״ר משה ז״ל ומדבריו למדתי׳. וכן יש לתרץ שמועה זו בטובלות על שפת הנהר (מקור) [מקום] שלא היה בו זוחל תחלה, וכענין ששנו בתוספתא כמו שהזכרתי: }
\newsection{דף סו}
\textblock{\textbf{מאן לא הודה לו ר׳ יוסי.} פי׳ לא הודה לו שיצאו בו בשבת אבל לגבי חליצה הודה גו כדאמרי׳ בפ׳ בתרא דיומא לעולם מנעל הוא והכא בהא קמפלגי מר סבר גזרי׳ דלמא משתמיט ואתי לאתויי ומר סבר לא גזרי׳ ושל סידין נמי לאו להילוכה עבי׳ ולא מיהדק ומישתליף (דאפשר) [ואפשר] דר״ה לא סבר לההוא אוקימתא דהתם. והוא סביר׳ לי׳ דר׳ יוסי אוסר משום דלאו סנדל הוא כלל והיינו דלא בעי (רבא) [נ״א - רבה, וי״ג - רב יוסף] למימר הכא לא הודו לו דר״י (דר״ה) [כרב הונא], ופריק פירוקא אחרינא ואמר מאן לא הודה לו ר״י ב״נ ושמואל נמי דאמר ר״מ הוא סבר דלר׳ יוסי לאו סנדל הוא אי נמי לאו לאפוקי מדר׳ יוסי אמר ר״מ הוא אלא לאפוקי מדרבנן דפליגי עליה דר״מ ור״ע כדתניא לו הודו לו וקסבר שמואל רבנן פליגי עלי׳ ולא הודו לו כלל דלהוי סנדל. ובמס׳ יבמות נמי בפ׳ מצות חליצה תניא בסנדל של שעם בקב הקיטע חליצה כשרה ואמר רבא עלה מדרישא ר״מ סיפא ר״מ ומשמע דסבירא ליה לרבא דלרבנן לאו מנעל הוא ולאו ר״מ לאפוקי מדר׳ יוסי דהא רבא גופיה הוא דאמר במס׳ יומא דלר׳ יוסי נמי מנעל הוא אלא לאפוקי מדרבנן דפליגו עלייהו ומשום דהא מתני׳ דהכא שמיע להו טפי ובכמה דוכתי אמרי׳ הא מני ר״מ הוא דאמר אדם מקנה דבר שלא בא לעולם וכמה רבנן קשישי סברי הכי במס׳ יבמות בפ׳ האשה רבה ולא הזכירו א׳ מהן אלא ר״מ והא דאמרי׳ התם במס׳ יומא לעולם מנעל הוא לר״מ ור״י קאמרינן דהא תלמוד לענין חליצה לאו מנעל כדאמרי׳ מנעל התפור בפשתן איץ חולצין בו [-והיינו כרבנן] והתם במס׳ יומא אמרי׳ נמי דרבא גופי׳ לפיק בדכילי שהוא סנדל של קש דומיא דשעם והא איהו דאמר דכולי עלמא מנעל הוא אלא שמע מינה דדוקא לר״י ור״מ ולא לרבנן. וה״פ דשמעתין דהתם: מהו לצאת בסנדל של שעם דפשיטא לן שאינו מנעל כרבנן דלא הודו לו ומיהו שמא אפ״ה אסור ביה״כ ואמרי׳ דמותר והדר אקשי׳ מהא דר׳ יוסי אוסר אלמא לאו מנעל הוא ומש״ה אמר שאסור ביום הכפורים אבל לדידן דלאו מנעל הוא מותר שלא אסרו אלא מנעל. ובס׳ התרומה (ס׳ ר״מ) מפורש דכל הני אמוראי כרבא ס״ל דאמר דכ״ע מנעל הוא ובשבת בהא פליגי דחיישינן דלמא משתלף ואתי לאתויי ארבע אמות ברשות הרבים כלומר דכיון דהוא עץ לא מהדק ומשתליף ולענין חליצה נמי כיון דמשתליף הו״ל כמנעל גדול שאינו יכיל להלוך בו דפסולה. וזה פי׳ משובש דסנדל דחליצה ודאי יכול להלוך בו הוא ולא פליגי בשבת אלא בקב הקיטע מפני שאין לו רגל ומשתליף מיניה ועוד דסנדל דחליצה דאית לי׳ רצועות ושנצן בין של עץ בין של עור יכול להלוך בו ועוד אי משום דעץ קשה הוא ולא מיהדק כי מחופה עור היכי מיהדק והלא קושייתו של עץ במקומו עומד ובר מן דין אי סנדל דעץ אינו יכול להלוך בו הוא עד דהוי׳ חליצתו פסולה בכך כבר הוצאת אותו מכלל מנעל ומדבריהם הן (יושבין) [מושבין(-פי׳ מופרכין)]. אבל רש״י ז״ל פי׳ שם דכ״ע מנעל הוא של עץ וכי נפק רבא בשל קש ושל שעם שהוא מן קש נפיק דלא מגין כעץ ולאו מנעל הוא ולפי פירושו היינו נמי דאמר רבא מאי לא הודה לו ריב״נ כלו׳ בשל עץ הכל מודים דסנדל הוא וכי פליגי בקש הוא דפליגי וריב״נ הוא. ואי קשיא לך הא דאמרי׳ בפ׳ מצות חליצה של שעם ושל סיב ושל עץ חליצה כשרה ואוקי׳ כר״מ אלמא של שעם ושל עץ שוין א״ל ר״מ כר״ע סבירא להו דאמר הכא חולצת ואמר התם מטמא שעליו שנוי בבריית׳ הודו לו שסנדל של סיידין שהוא של קש סנדל הוא אבל (לריב״נ) לא הודה לו ותני בברייתא בלשון רבים ורבנן דיומא ס״ל כוותי׳. והא דאמר מנעל התפור ופשתן אין חולצין בו א״ל לכתחלה קאמר אבל דיעבד חליצה כשרה דומיא דעץ וא״ל אפי׳ דיעבד חליצתה פסולה משום דבעי׳ של עור או של עץ דומיא דעור שהוא מגין משום דנעל ריבה אבל תפור בפשתן כשם שאם היה כולו של פשתן בגד הוא אף כשהוא בפשתן אינו מנעל. ולפי פירוש זה, הא דתנן בסנדל של עץ חליצתה כשרה לרבא אפי׳ כרבנן אתיא דל״פ רבנן עלייהו (ור״מ) [דר״מ] ור״ע בשל עץ דמנעל הוא וחליצתה כשרה ובשל קש לחוד הוא דפליג ריב״נ וק״ל אם כן למה פסק רבינו הגדול בסנדל של עץ חליצתה פסולה אלא אם כן מחופה עור ושמא חששו להחמיר להנך אמוראי. ועדיין אני תמה האיך שתיק רבא [ביבמות] מאותה סוגיא ולא העמיד משנתינו כד״ה שם ואפשר דסבירא ליה דרב דאמר אין חולצין במנעל התפור בפשתן אפילו דיעבד קאמר כדגמר לה מואנעלך תחש לגבי חליצה הא לשאר מילי מנעל [הןא, ולא פליגי במדרס], ושבת אלא ריב״נ דפליג בשל קש מיהו לגבי חליצה פליגי והיינו דאבוה דשמואל ושמואל דאמרי התם מאן תנא של עץ חליצתה כשרה ר״מ הא לרבנן פסולה דגמר גמירי לה דרבנן לא קרו מנעל לחליצה אלא של עור ופליגי אר״י ור״מ דהודו לו לר״ע ואע״ג דלא אשכחן היכא כדאמרי׳ בעלמא אין הלכה כדברי ר״מ ור״י אלא פורס מפה ומקדש כלו׳ שחכמים אומרים כן ואמרי׳ בכמה דוכתי זו דברי (ר״מ) [רבי] פלוני סתימתאה אבל חכמים אומרים וכו׳, ואף על גב דלא אשכחן להו בהדיא. ויש לדקדק באותה שמועה של מס׳ יומא האיך יוצאין בסנדל של שעם ביום הכפורים כיון שאינו סנדל נמצא שאסור לצאת בו משום משוי דהא קיי״ל ערוב והוצאה ליום הכפורים כשבת וא״ת בחצר דוקא והלא כל שהוא דרך מלבוש הרי הוא כתכשיט כדפי׳ לעיל לאסור לצאת בו לחצר כלל ובמתני׳ נמי תנן בכולן לא יצא חוץ מכבול לחצר. ומתוך הדוחק י״ל דאפילו למאן דאמר לאו מנעל הוא לא משוי הוה דלהנאתו עביד ליה ודומיא דאנפלייא של בגד דלאו מנעל הוא כלל ויוצאין בה בשבת וביום הכפורים אלא טעמא משום דכיון דלא סיימו אינשי אלא עורדמגין ורכיך גזרו בו דילמא תליץ ליה ואזיל לסיומי מסאני׳ואתילאתויי ד׳ אמות ברשות הרבים וגבי קב הקיטע נמי חליץ ליה דכאיב ליה הלכך ביום הכפורים דלא סייס מסאני׳ לא חליץ להו ומותר דלאו במקום מנעל סאים להו וזהו התירן. עוד יש לדחוק ולומר, הני אמוראי כולהו דלא סליק להו טעמא במנעל ולאו מנעל פליגי וכל דלאו מנעל אסור בשבת ורבא מוקי טעמא דפלוגתא בגזירה ולדידי׳ אפי׳ כי לא הוה מנעל כל היכי דליכא למיגזר מותר וכולהו רבנן דיומא דמסיימו ביום הכפורים הנהו מיני כרבא סביר להו ועד״ז מתקיימות כל השמועות כהלכה ובקושי ובדוחק ועוד דאביי נפיק ומתרץ דשרי ול״ל דרבא ואם נאמר דסבירא ליה לחצר מותר, אינו נכון כלל. ומצאתי בירושלמי במס׳ יבמות (יב:) סנדל של עץ חבריא בשם רב שיהיו חבטיו של עור תמו אמרין בשם רב והוא שיהיו תרסיותיו של עור פי׳ חבטיו מה שהרגל חובט עליו והיינו מחופה מבפנים עור ותרסיותיו עור ר׳ אילא בשם ר׳ יוחנן ואפי׳ כולו של עץ כשר במתני׳ מסייע לר׳ יוחנן סנדל של קש טמא מדרס והאשה חולצת בו דר״ע אבל חכמים לא הודו לא למדרס פי׳ למדרס ר״י היא דקא״ל ופריש ולא הודו לו דקתני למדרס למדנו מכאן דסנדל של סיידין דקש הוא ילמדנו לדעתו של ר׳ יוחנן דהכל מודים בסנדל של עץ שהוא כשר לחליצה וכן של קש דתריוייהו מנעל נינהו ואין זה מסכים לסוגי׳ שבגמ׳ שלנו. עוד ראיתי בתוספתא דיבמות (יב,ח) גבי ברייתא דקתני חלצה בסנדל של שעם בקב הקיטע חליצתה כשרה אר״י אלו ראה ר׳ אלעזר בסנדל של עץ עתה היה אומר עליו הריהוא כסנדל לכל דבריו משמע דר״י גמר דפלוגתא דרבי אלעזר ורבנן הוא בסנדל של עץ לגבי טומאה, ועכשיו הי׳ ר״א מסתייע ראה שחולצין בו לעשותו נמי כסנדל לשאר כל הדברים. ובתוספתא דשבת (ו.) קתני ר״א אומר קב הקיטע אם יש בו בית קיבול כתיתין יוצאים בו ואם לאו אין יוצאין בו פי׳ דס״ל כיון דאית בה בית קבול כתיתין הדוקי מהדק לי׳ ברגלו ולא משתמיט מינה וזה קצת סיוע לדברי שאני אומר סנדל של עץ רבנן פליגי עלי׳ לומר לאו מנעל הוא ואפשר שנאמר בזו שאמרו מותר לצאת בסנדל של שעם ביום הכפורים דלאו דוקא אלא לנעול בבית התירו וכפי לשון אחרון שכתבתי למעלה אלא לפי שתפסו להם לצאת בתענית צבור אמרו כן ביום הכפורים א״נ לומר שהוא מותר לצאת בהם בפני רבים דלא באו להתיר איסור אלא נעילה אבל איסור הוצאה לא מיירי בה הכא אי עירוב והוצאה ליום הכפורים ואי לחצר מותר אלא הנך מתפרשי בדוכתייהו:
}
\textblock{ודאמרי׳ \textbf{מאן הודו לו ר״מ.} איכא למידק והא לא אמר ר״מ אלא דיעבד אבל לא לכתחלה ור״ע חולצת לכתחלה קאמר, א״ל לר״מ נמי לכתחלה חולצת והא דקתני חליצתה כשרה משום סנדל שאינו שלו ומשום של שמאל בימין:
}
\textblock{מתני׳: \textbf{היודע עיקר שבת ועשה מלאכות הרבה בשבתות הרבה חייב על כל שבת ושבת.} מפורש במס׳ כריתות בפ׳ אמרו לו משום דימים שבנתיים הויא ידיעה ופרש״י ז״ל שאי אפשר שלא שמע בנתיים שאותו היום שבת היה ואע״פ שלא נודע לו שחטא שלא היה זכור שעשה בו מלאכה כיון דשגגה ראשונה משום שגגת שבת הוה והרי נודע לו הוה ידיעה לחלק ויש להשיב והא ידיעה דחטא בעינן כדאמר בגמ׳ קצר וטחן כגרוגרות בשגגת שבת וזדון מלאכות וקצר וטחן כגרוגרות בזדון שבת ושגגת מלאכות וכו׳ אלמא העלם אחת הוא ואין חלוקות לחטאת מפני שלא נודע לו שחטא ואע״פ שנודעה לו שגגתו הראשונה שלא הי׳ יודע שהוא שבת דבעינן או הודע אליו חטאתו אשר חטא בה וי״ל ל״ד דבשלמא התם לא נודע לו חטאתו שהרי יש לו שגגת מלאכות אבל הכא הרי נודע לו חטאתו שהרי לא היה לו אלא שגגת שבת והרי נודע לו שהוא שבת ונודע לו שהמלאכות אסורות בו ושהללו מלאכות הן ואע״פ (שנודע) [שלא נודע] לו שחטא הוא שהרי אינו זכור שעשה בו מלאכה. ואיכא דקשי׳ ליה הא דאמרי׳ התם במס׳ כריתות (יז.) אלא נדה מאי ימים שבנתיים איכא כגון שבא עליה וטבלה וחזר ובא עליה והכא מאי איכא למימר אי אפשר שלא ידע שהיתה נדה והרי אפשר שלא ראה ושלא שמע ואפי׳ ראה אותה טובלת אפשר שהיה סבור שטבלה בשביל טומאה אחרת כגון ששמשה או מגע שרץ וכיוצא בה ואפשר לומר שכיון שראה אותה טובלת מעלה הוא על דעתו שטובלת לנדתה שרוב טבילת הנשים לנדתן. עוד הקשו א״כ היכי מפרשינן לה ואמרי בגמ׳ הכא משום דכתיב שמירה אחת לכל שבת ושבת למה לי קרא׳ פשיטא הא איכא ידיעה בנתיים והא נמי לאו קושי׳ הוא דקמ״ל אע״פ שלא נודע לו שחטא כיון שנודע לו שגגתו כולה ידיעה היא. ול״נ ימים שבנתיים דהוו ידיעה לא ששמע שאותו היום שבת הי׳ ולא שנזכר אלא שלא נתן לבו לדבר כלל אבל מאחר שהוא יודע עיקר שבת ויודע שיום שביעי לעולם הוא שבת ולאחר שבוע אחר שיודע בכל ימות השבוע שהוא חול ואלו נתן דעתו לחשב ודאי ידע שהיום שבת וכשעשה מלאכה בשבוע שעבר שבת היתה אע״פ שלא נתן דעתו לכך הוה ידיע׳ וכן בנדה כשראה אותה טובלת לנדתה ולא נתן אל לבו שבא עליה בשעת נדתה ואפשר שאף רש״י ז״ל לזה נתכוין ולפי פירוש זה אין משנתינו אלא בידע עיקר שבת שהוא יום שביעי. אבל הר״ר שמואל ז״ל פי׳ שכיון שנודע לו יום החול הוי הפסק לההעלמות שהרי יצא מן הספק הא׳ ויודע שעכשיו היא חול וכן בנדה מאחר שראה אותה טובלת ויודע שטהורה מנדתה ועכשיו אין לו ספק שיודע בודאי שהיא טהורה ידיעה היא ולזה הפי׳ ימים שבנדה וימים שבשבת שוין הן לגמרי והוא הפי׳ הנכון. ואיכא למידק, היכי אמרי׳ במס׳ כריתות (טז.) ק״ו מנדה והא ימים שבנתיים דהוו ידיעה בשבת כתיבא כדאמרי׳ בגמ׳ והכא משום דכתיב שמירה א׳ לכל שבת ושבת והיכי תלי ר״א כתיבה בדלא כתיבה ואפשר דסביר׳ ליה תרתי בעי מיניה ימים שבנתיים אי הוה ידיעה לחלק ואי שבתות כגופין מחולקין דמיין ופשיט ליה תרוייהו מנדה ואגב גופין מחולקין דגמרי׳ מנדה אמר נמי ימים שבנתיים מנדה אבל אדרבה כי כתיבי בשבת כתיבא. א״נ אסמכתא נינהו:
}
\textblock{\textbf{חייב על כל אב מלאכה ומלאכה.} פי׳ למ״ד שבתות כגופין מחולקין דמי חייב על כל אב מלאכה ומלאכה של כל השבתות חייב אחת דהכ׳ ליכא למימ׳ ימים שבנתיים הוה ידיעה לחלק שאין כל השגגות בימים אלא במלאכות:
}
\newchap{פרק \hebrewnumeral{7} כלל גדול}
\newsection{דף סח}
\textblock{}
\textblock{גמ׳: \textbf{שביעית נמי משום דקבעי למיתני עוד כלל אחר אמרו.} ק״ל גבי שביעית מאי גודלי׳ דראשון טפי מאחר דכמה מילי כייל ואפשר לומר משום דקמאי יש להן ביעור ובתרא אין להם ביעור. ואי קשי׳ הא באידך פרקין דשביעית (ח.)קתני כלל גדול אמרו בשביעית כל המיוחד לאוכל אדם וכו׳ ולא קתני בתרי׳ כלל אחר ההיא נמי משו׳ דהוא כלל גדול מכלל אחר אמרו דקתני מקודם בפ״ק קתני נמי בהאי כלל גדול, כך מפורש בתוס׳:
}
\textblock{\textbf{ולבר קפרא דתני (עוד כלל אחר אמרו) [כלל גדול במעשר] מאי אבות ומאי תולדות איכא.} בדין הוא דהול״ל ב״ק משום דתני כלל אחר אמרו קתני כלל גדול. וא״ל [ד]ב״ק לא תני במתניתיה כלל אחר, הלכך לא ה״ל למיתני כלל גדול משום כלל אחר שבמשנה. א״נ משום דבעי חד טעמא דשוי לתרוייהו לתנא דמתני׳ ומתנית׳ דב״ק והיינו גדול עונשה דתרוייהו ס״ל הכי אלא דמר לא תני ענשה של תאנה וירק מפני שהוא מדבריהם ומר תני לי׳:
}
\textblock{\textbf{ואלו פיאה ליתי׳ בתאנה וירק.} פירשו רבותינו הצרפתי׳ ז״ל שלפיכך לא תקנו פיאה בתאנה וירק לפי שאין לקיטתו כא׳ ויש בו משום ביטול עניים שיאמרו עכשיו מניח בעל הבית פיאה ויפסידו יותר ממה שירויחו וכן בירק מפני שאין לו גורן ואין לקיטתו קבוע וידועה ונ״ל שזהו הטעם שלהם מדברי תורה הוא דדומיא דקציר חייבה תורה כל שלקיטתן כאחד כדתניא בתורת כהנים ול״ל דאסמכתא נינהו דמנ״ל למעטינהו הא לא כתיב בה דגן ותירוש כדכתיב במעשר אלא כולן מן התורה חייבין כולן שלקיטתן כא׳ והשאר פטורין ולא רצו להתקין על דברי תורה שאין בזה תקנה לעניים וכשמיעטו תאנה רבותא קמשמע לן אף על פי שחייבת בבכורים וכ״ש בשאר אילנות שאין לקיטתן כא׳ וכך שנינו בפ״א של פיאה ובאילן האוג והחרובין האגוזים והשקדים והגפנים והרמונים הזתים והתמרים חייבין בפיאה ש״מ הא כל שאר האילנות פטורין:
}
\textblock{\textbf{גר שנתגייר לבין העכו״ם.} כגון שנתגייר בפני ג׳ והודיעוהו מקצת מצוות אחרות ולא הודיעוהו מצות שבת, או שטעו ולא הודיעוהו, שאלו נתגייר בינו לבין עצמו אינו גר:
}
\textblock{\textbf{אדתני היודע עיקר ליתני הכיר ולבסוף שכח.} וא״ת דילמא הא קמ״ל שאפי׳ יודע עיקר שבת אינו חייב אלא על כל שבת ושבת כדקתני רישא:
}
\textblock{והא דאקשי׳ נמי \textbf{אבל לא שכחה מאי חייב על כל מלאכה ומלאכה.} קשיא, דילמא לא שכחה נמי אינו חייב אלא על כל שבת ושבת. ורש״י ז״ל תירץ בזה דאי סלקא דעתין אינו חייב אלא על כל שבת ושבת הוה ליה למיכרך ולמיתני הכי: מי שיודע [עיקר שבת (ושכחה) וכן היודע] עיקר שבת ועשה מלאכות הרבה בשבתות הרבה חייב על כל שבת ושבת, ונכון הוא. אלא שקשיא לי הא דאמרי׳ מתני׳ בהכיר ולבסוף שכח ורב ושמואל דאמרי תרוייהו אפי׳ תינוק שנשבה לבין הגוים כהכיר ולבסוף שכח דמי, ופרש״י ז״ל דמתני׳ אצטריך למנקט הכיר ולבסוף שכח דלא תימא חייב על כל שבת ושבת ואמאי ליתני תרוייהו וליכרוך הכי מי שלא ידע עיקר שבת והשוכח עיקר שבת. ואפשר דהכי נמי אמר שמואל מאי השוכח עיקר שבת שאינו יודע עיקר שבת בין שהית׳ שכוחה ממנו בין ששמע ושכח וכי אמרי׳ מתני׳ בהכיר ולבסוף שכח לומר שהוא בכלל משנתינו לדברי הכל ומסייעא לזה הפירוש לשון הברייתא ששנה נמי השוכח עיקר שבת וחזר ופי׳ כיצד תינוק שנשבה לבין הגוים שהכל בכלל לשון זה, כן נ״ל. אי נמי סבירא להו לרב ושמואל דשקולין נינהו שבזה ובזה אין בהן ידיעה ואין לחייב בשניהם אלא אחת ולא חייש תנא למיתני לאפוקי מטעותיה דר׳ יוחנן אבל יודע עיקר שבת ושכחה ולא שכחה תרי גוני נינהו ולזה יש ידיעה ולזה אין ידיעה והלכך צריך היה לפרש כיון דתני כלל א״נ רישא דקתני אינו חייב אלא אחת מוכחא מילתא דהא אתא לאשמעי׳ דהא הכיר ולבסוף שכח אינו חייב אלא אחת &lt;הג״ה - אבל לא הכיר נמי חייב&gt;:
}
\textblock{הא דאמרי׳ \textbf{מאי טעמא דמונבז.} פרש״י ז״ל דאבנין אב ודאי לא סמיך דשוגג חלוף דמזיד הוא וליכא לדמויינהו וק״ל אי כולה מהקישא אתיא לי׳ מ״ש דפטור בשלא היתה ידיעה כלל ומ״ש בידיעה דשעת מעשה דלא פטר ליה, אלא י״ל דהבנין אב ודאי דנין וכל שלא היתה לו ידיעה כלל אינו בכלל החטא אלא מאי טעמא דמונבז בידיעה בשעת מעשה דהא ליכא לאתויי בבנין אב שאם אינו נקרא חוטא עד שתהא לו ידיעה בשעת מעשה א״כ לעולם פטור עד שיזיד בלאו וכרת והא לית ליה למונבז כדתני׳ לקמן שגג בזה זהו שוגג ומפרשי׳ דידיע׳ דשעת מעש׳ מהקישא מרבי לה ואהני בנין אב ואהני היקש:
}
\newsection{דף סט}
\textblock{\textbf{מה להלן דבר שחייבין על זדונו כרת ושגגתו חטאת.} ואי קשי׳, ולימא מה התם זדונו סקילה אף כל שזדונו סקילה ולאו מלתא דהיא גופה קמשמע לן מדכתיב תורה אחת לעושה בשגגה הוקשה בפי׳ כל התורה כולה לע״ז וסמיך ליה והנפש אשר תעשה ביד רמה ונכרתה וגו׳ לכרת הקישה הכתוב. ואי קשיא מונבז הא מנ״ל, לאו קושיא הוא דאדרבה למונבז היקש גמור הוא ואינו למחצה ולרבנן היקש למחצה הוא אלא דסברי רבנן שאין מקישין שוגג למזיד דלא דמי. וי״מ דמונבז נמר לה מדגרסי׳ במס׳ יבמות עליה עליה מג״ש נאמר כאן אשר חטא עליה ונאמר להלן לצרור לגלות ערותה עליה מה להלן דבר שחייבין על זדונו כרת ועל שגגתו חטאת אף כאן כל שזדונו כרת ושגגתו חטאת ואלו היה פי׳ זה נכון הוה בעי בגמ׳ ומונבז הך סברא מנ״ל:
}
\textblock{\textbf{עד שישגוג בלאו שבה.} פי׳ אע״פ שהוא יודע איסור עשה שבה כיון שהוא שוגג בלאו שגגה הוא. והא דאקשי׳ דידע ליה במאי ואוקי׳ בתחומין ואליבא דר״ע ולא מוקי דידע ליה בעשה דשבתון ותשבות משום דאי לא ידע בלאו העלם זה וזה בידו ולקמן (שבת ע:) אמרי׳ שאינו חייב אלא אחת, וכדבעי׳ לפרוש תמן בס״ד. ויש לפרש דלאו שבה היינו איסור שבה שהוא סבר מותר והאי דסמך ליה אקרא דלא תעוזה היינו איסור שבהן, וכן עיקר:
}
\textblock{והא דא״ר יוחנן \textbf{מנינא ל״ל שאם עשאן כולן בהעלם אחד שחייב על כל א׳ וא׳.} פי׳ ללמד ששגגת כרת שמה שגגה אע״פ שהזיד בלאו או שידיעת תחומין ידיעה אבל לחלוק מלאכות מרישא ש״מ דקתני חייב על כל אב מלאכה ומלאכה וכדקאמרי׳ עלה חלוק מלאכות מנא ליה אלא דהוי אמינא התם ביודע מקצת ושוגג במקצת לגמרי:
}
\textblock{והא דאמרי׳ \textbf{בשלמא לר׳ יוחנן דאמר כגון ששגג בכרת אע״פ שהזיד בלאו משכחת לי׳ דידע ליה לשבת בלאו.} ק״ל מה ידיעה היא זו אדרבה לר׳ יוחנן כיון דשגגת כרת שגגה וידיעת לאו לא שמה ידיעה זה שיודע בשבת לאו ושגג בכרת שגגת שבת הוא, ולאו זדון שבת מקרי:
}
\textblock{והא דאוקימנא \textbf{דידע לה בתחומין.} ה״ה דהו״ל לאוקמי כמונבז והוא הדין דמצי למימר דידע ליה לשבת בלאו דמחמר ואליבא דדברי הכל אלא משום דאי אפשר שביתת עצמו לא ידע שביתת בהמתו ידע וכן נמי לא אמר דידע ליה בהבערה ואליבא דר׳ יוסי (ע.) משום דמתני׳ לא אתיא כוותי׳ דקתני מבעיר באבות מלאכות:
}
\textblock{\textbf{הא מני מונבז הוא.} פרש״י ז״ל, אבל רבנן דפליגי עליה לית ליה קרבן שבועת ביטוי לשעבר דסביר׳ ליה כר׳ ישמעאל דאמר אינו חייב אלא על העתיד לבוא. והקשו בתוס׳, מכדי מאן דפליג עלי׳ דמונבז ר״ע ור״ע הוא דמחייב לשעבר במס׳ שבועות (כו:), ויש לי לדחוק שלא חלק ר״ע עלי׳ דמונבז אלא על שהיה פוטר כל שלא ידועה בבנין אב דידי׳ אבל איהו סבירא ליה דבין בשלא היתה לו ידיעה כלל בין בשיש לו נמי ידיעה בשעת מעשה כולן חייבין דשגגת קרבן שמה שגגה וכשא״ל הריני מוסיף על דבריך לומר שאם תדרוש ההיקש כמי שתדרוש בנין אב ידיעת בשעת מעשה תיבעי וא״ל מונבז הן אדרוש וא״ל ר״ע הן זה נקרא שוגג סתם אם תפטור כל שאין לו ידיעה בשעת מעשה אלא מזיד נמי הוא במקצת ולחנם תפסה לו תורה לשון שוגג הא מ״מ מודה ר״ע דשגגת קרבן שמה שגגה מבנין אב הוא דתחטא ודקאמרי׳ ורבנן שגגה במאי שאלה הוא לשאר רבנן מה לי אמרו בה א״נ לאביי נימא הכי וסוגיין דר״ע פליג אמונבז בשגגת קרבן לגמרי דלא כאביי והיינו דקאמרי׳ ורבנן שגגה במאי. ויש מקצת נוסחאות שכ׳ בהן: מני הא, אילימא מונבז [השתא בכל התורה כולה דלאו חידוש הוא אמר מונבז]שגגת קרבן שמה שגגה הכא לא כל שכן אלא לאו רבנן וכו׳ תיובתא דאביי תיובתא. ורש״י ז״ל השיב עלי׳ דמאי פשיטותא, הא אצטריך לאשמעינן דחלוק׳ שגגה זו משאר שגגות דאלו בכל התורה כולה למונבז בין שגג בקרבן בין שגג בלאו וכרת שמה שגגה וכאן בלשעבר שגג בלאו שבה לאו שגגה הוא דהאדם בשבועה (ויקרא ד,ד) פרט לאנוס. אלא אכל וסבור שלא אכל ונשבע לא אכלתי כדמייתי׳ עלה בפ׳ שבועות שתים (שבועות כו.) עובדא דתלמידי דרב דמר אמר שבועתא הכי אמר רב ומר אמר הכי אמר רב ואמר רב כחד מינייהו וא״ל לאידך את לבך אונסך אבל טועה באיסור השבועה ואמר מותר אינו בכלל אנוס זה אלא חייב כאן כמו שחייב בכל התורה תינוק שנשבה לבין הגוים ושוכח עיקר שבת ואיסורן וכ״ש היתה לו ידיעה באיסור זה ושכחה דהיינו אומר מותר דכל איסורין שבתורה דחייב הלכך הא דקתני הך ברייתא איזהו שבועת ביטוי לאו למעוטי אומר מותר ואביי נמי דאמר עד שישגוג בלאו שבה היינו שנשבע לא אוכל וזכור בשבועתו אלא שהוא שגג באיסור שבה וסבר מותר לעבור על שבועות אלו אבל שוכח שבועתו לא שוגג בלאו הוא אלא בשבועה שלו והלכך לאביי נמי בכה״ג אית לי׳ שבועת ביטוי לשעבר באומר מותר הלכך ברייתא דקתני איזהו שבועת ביטוי לשעבר לא בא אלא ללמד שאף זו שבועת ביטוי הוא ולא בעי׳ שוגג בלאו שבה. וזה הפי׳ הנכון, אלא שיש לדקדק מההוא דבמס׳ שבועות דבעי מיני׳ רבאמר״נ שבועות ביטוי לשעבר ה״ד ופשיט ליה באומר יודע אני ששבועה זו אסורה אבל איני יודע אם חייבין עליה קרבן אם לא משמע לכאורה כה״ג אבל אומר איני יודע ששבועה זו אסורה פטור משום האדם בשבועה ולאו קושי׳ הוא דרבא בעי מיני׳ איזו שבועת ביטוי לשעבר משום דליתי׳ דומי׳ דלהבא בשוכח ופטיש לי׳ בטועה בקרבן לומר דשגגת קרבן הוה שגגה לדברי הכל דלא כאביי וה״ה בשגגת לאו וכ״ש. וי״מ דלאביי נמי [דאמר] עד שישגוג בלאו שבה [סבר] לרבנן, ע״כ אית לן למימר דשבועת ביטוי.ל שעבר בשאומר איני יודע ששבועה זו אסורה ומוקמי׳ האדם בשבועה פרט שלא נשבע לשקר (דל״פ) [דאל״ה] קרבן שבועת ביטוי לשעבר דרבי רחמנא היכי משכחת לה ומ״ה מקשי׳ עלה מברייתא דקתני איזו היא שבועת ביטוי לשעבר דמשמע איני יודע ששבועה זו אסורה לאו שבועת ביטוי הוא ומימעטא מהאדם בשבועה וכ״ת הא מני מונבז ולאו למעוטי אומר מותר אלא לרבות שאף על זו חייב דשגגת קרבן שמיה שגגה השתא בכל התורה כולה וכו׳ אלא לאו רבנן הוא כלו׳ אלא לאו, דוקא קאמר זו הוא שבועת ביטוי לשעבר (אלא) [ולא] אחרת, דממעטי׳ לה מהאדם בשבועה פרט לאנוס ואפי׳ לרבנן הוא וה״ה למונבז שאין מחלוקת בין מונבז לחכמים באומר איני יודע ששבועה זו אסורה ובדרשת האדם בשבועה ומונבז היינו רבנן הלכך קשי׳ לאביי. וזה פי׳ מקולקל בזה שהרי לזה הפי׳ עיקר ההויה לאביי ממה שלא הזכיר [בסוגיא], והוא מדין האדם בשבועה אבל לזה הדרך יותר נכונה גירסת הספרים שגורסים הא מני מונבז הוא ועיקר הקושי׳ משגגת קרבן הוא כפשטה אבל נאמר ודאי דלאביי למונבז דאמר שגגת קרבן שמה שגגה ומשכחת שבועת ביטוי לשעבר בשגג בקרבנה ממעטי׳ מהאדם בשבועה אומר איני יודע ששבועה זו אסורה אע״פ שהיתה ידיעה לו מתחלה אבל לרבנן דאמרי עד שישגוג בלאו שבה לא ממעטינן מינה אלא סבור שלא נשבע לשקר שכך דרך בכ״מ שמקרא א׳ מרבה ומקרא א׳ ממעט תפוס מיעוטו של זה וריבוי של זה בכדי שלא יהיו סותרין זה את זה והיינו דפשט ליה ר״נ לרבא שבועת ביטוי לשעבר באומר יודע אני ששבועה זו אסורה אבל איני יודע אם חייבין עליה קרבן משום דלא ס״ל כאביי אלא שאני הכא דחידוש הוא או שנתמעט מהאדם בשבועה בודאי כסברא דמאן דמותיב עליה דאביי:
}
\textblock{\textbf{במאי מנכר ליה בקדושא ואבדלתא.} פירשו רבותינו הצרפתים ז״ל, אבל מותר לילך חוץ לתחום אפי׳ באותו יום שהוא מקדש לפי שתחומין מדבריהם וכל שהוא מדבריהם לא גזרו, ואפי׳ למ״ד (סוטה ל:) תחומין דאוריית׳, כיון דאיסור עשה הוא ואינו איסור סקילה לא החמירו, תדע מדלא אמרי׳ במאי מנכר ליה בתחומין, וליכא למימ׳ דשאר יומי נמי לא אזיל חוץ לתחום א״כ ישאר לעולם במדבר ויחלל שבתות וימות בארץ גזרה:
}
\newsection{דף ע}
\textblock{\textbf{חלוק מלאכות מנ״ל אמר שמואל וכו׳.} לאו טעמא דמתני׳ מפרש שמואל דהא אסיקנ׳ דכר׳ יוסי ס״ל ומתני׳ לא אתי כר׳ יוסי דקתני מבעיר באבות מלאכות אלא טעמא דנפשיה קא מפרש:
}
\textblock{והא דאקשי׳ \textbf{האי במזיד כתיב.} וליכא למימר מיתות הרבה רבה כאן דבתרי קטלי ליכא למקטלי׳ אלא ע״כ דברה תורה כלשון בני אדם או שאם אי אתה יכול להמית במיתה הכתובה בו שתמיתו בכל מיתה שאתה יכול להמיתו. ופריק אם אינו ענין למזיד תנהו ענין לשוגג ודרוש בי׳ הכי מאי יומת, בממון:
}
\textblock{כתוב בכל הספרים: \textbf{יכול עשאן כולם בהעלם אינו חייב אלא א׳ תלמוד לומר בחריש ובקציר תשבות.} ועדיין אני אומר על חרישה ועל קצירה חייב שתים ועל כולן לא יהא חייב אלא אחת. ושאל רש״י ז״ל והא מבחריש ובקציר נפקא דהו״ל דבר שהיה בכלל ויצא מן הכלל ללמד על הכלל כולו והשיב משום דהו״ל כלל בלא תעשה ופרט בעשה ואין דנין אותו בכלל ופרט ואין זה מחוור משום דבכלל נמי איכא עשה דוביום השביעי תשבות. וטעם אחר אמר, משום דה״ל חריש וקציר ב׳ כתובין הבאין כא׳ ואין מלמדין וא״ת א״כ בהבערה ה״ל שלשה וכ״ש שאין מלמדין ובזה י״ל שכיון שאין שלשתן בלאו או בעשה אין דנין אותו בג׳ כתובין הבאין כא׳ א״נ כיון דשנים אין מלמדין ויצא שלישי ללמד ע״כ מוקמי׳ להו לשנים לדרשא אחריתי ול״ל דלמא הבערה ללמד על עצמה יצאת לעשות דינה כיוצא באלו שיתחייב עליו בפ״ע כחרישה וקצירה מדלא כתבה רחמנא התם בחריש ובקציר ומהבערה תשבות וכן פרש״י ז״ל וקסבר שני כתובין נמי אין מלמדין והשתא דיצאת הבערה לחלק חריש וקציר ל״ל לדרשא אחריתא אתא אי לר״ע לענין תוספ׳ שביעית אי לר׳ ישמעאל לענין שבת למידרש מה חריש רשות אף קציר רשות יצא קצירת העומר שהוא מצוה ואינו מחוור שהיה לו לפרש כן בברייתא בהדיא. ובירושלמי (ז,א) מצאתי כן על ענין אחר, שני דברים שיצאו מן הכלל אינן חולקין וכר׳ ישמעאלחולקין דאמר ר׳ בון בן חנניא דברי ר׳ ישמעא שני דברים שיצאו מן הכלל חולקין פי׳ קסבר שני כתובין הבאין כא׳ מלמדין. ובשאר ספרים גרסי׳ להנך בריית׳ הכי, יכול לא יהא חייב אלא אם כן עשאן כולן בהעלם א׳, תלמוד לומר בחריש ובקציר תשבות ונכון הוא שבא חריש וקציר ללמד על הכלל שחייב על כל אחת ואחת בשעשאה בפני עצמה ועדיין אני אומר שאם עשאן כולןבהעלם א׳ נמי לא יהיו חלוקות ולא יהא חייב אלא אחת ובא הבערה ללמד על הכלל כולו שחייב על כל א׳ וא׳ וק״ל דהא שוגג לקרבן דומיא דמזיד לסקילה ואכל חדא ודאי מחייב דהכי אזהר רחמנא כל מלאכה ומקושש נמי חדא עבר ואיחייב מיתה, ושמא שוגג שאני:
}
\textblock{\textbf{אלא אמר רב אשי חזינן אי משום שבת קא פריש.} ה״פ, פעמים שאומרים לו שבת הוא ופורש ואע״פ שהיתה לו העלמות מלאכות הוא נותן אל לבו ונזכר לפיכך אינו חייב אלא אחת שעיקר העלמתו היתה בשביל שבת והוכיח סופו על דעתו, שאילו היה יודע מתחלה שהוא שבת היה מחשב באיסור המלאכות ונזכר ופורש ואי משום מלאכות קא פריש חייב על כל א׳ וא׳. ורבינ׳ השיב, כלום פריש משבת אלא משום מלאכות כלו׳ שהעלם זה וזה בידו ודאי אינו חייב אלא אחת שמ״מ עכשיו העלמת שתיהן בידו אבל כשאדם אומר לו כן (או) שאומר בדעתו כמו שטעיתי בשבת טעיתי במלאכות ונזכר הכל ועכשיו הוא שבאה לו ידיעה זו אבל מתחלה העלם זה וזה בידו היה אלא לא שנא ואינו חייב אלא אחת. ורב אשי גופי׳, לאו למיפשט בעי׳ דרבא אתא, דודאי עדיין איבעי׳ דרבא קיימת שפעמים אינו מניח מלאכתו אלא כשיודיעוהו שתיהם, אלא תפשוט מיהא מקצת קאמר:
}
\textblock{\textbf{אא״ב העלם זה וזה בידו חייב על כל אחת ואחת שפיר.} איכא למידק הכא אי העלם זה וזה בידו מכל אבות מלאכות היינו שוכח עיקר שבת שאינו חייב אלא אחת בשלמא בעי׳ דרבה לא הוה ק״ל דהתם בשהעלם שבת הוא ומקצת מלאכות בידו מפרשי׳ לה אלא הא קשי׳ ונ״ל דהכא במאי עסקינן כגון בהעלם ל״ח מלאכות בידו ויודע עיקר שבת שמלאכה אחת נאסרה בו אלא שהוא סבור שאין היום שבת נמצא חייב על ל״ח מלאכות בהעלם זה וזה ל״ח חטאת מפני העלמות המלאכות ועל אותה שהוא יודע יתחייב עליה משום העלם שבת. ושוב מצאתי בחדושי הר״ר משה בר׳ יוסף ז״ל דמשני לד דבעידנא דעבד להו למקצתייהו לא הוה ידע להו דאסירי אבל יודע הוא שנאסרו המלאכות אחרות בשבת ולאורת׳ עביד להו ולא ידע שאסורות ונזכר על האחרות שהן אסורות בשבת אבל (לעולם) העלם שבת היה בידו בכולן ולפיכך הוא העלמה אחת (אפי׳ לר״ז) [לר״נ]. ואי ק״ל אי דוחקי׳ בהאי גוונא אמאי מוקמי׳ לה בהעלם זה וזה בידו נוקמי בזדון שבת ושגגת מלאכות ובכה״ג דנולדה לו מחשבה אחרת ושכח מלאכות הראשונות ולא ידע לו באחרונות ומשני א״כ תירוצי׳ דר׳ יוחנן מאי אתא לאשמעי׳ מתני׳ היא דתנן היודע שהוא שבת וכו׳ והא דמוקמי לקמן בהכי מ״מ אשמעי׳ כל דהו לר׳ יוחנן דס״לכוותי׳ לר״ל תחומין ואליבא דר״ע, אלו דברי הרב ז״ל. ויש להשיב על (דבריו) [דברי ועל דבריו] מהא דאמרי׳ לקמן (שבת עב:) אלא באומר מותר ע״כ לא בעי מיניה רבא מר״נ העלם זה וזה בידו אלא לאיחיובי חדא או לאיחיובי תרתי אבל מיפטרא לגמרי לא ואם כדברינו אין בעי׳ דרבא באומר מותר אלא בהעלם מקצת מלאכות בידו ויודע שהשאר אסורות והעלמה גרידא הוא ואדרבה אומר מותר היינו שוכח עיקר דמתני׳ והו״ל לאקשויי ממתני׳. וא״ל, כל שתי שגגות אומר מותר קרי לי׳ ומשכחת לה בע״ז כגון דסבר הקטרה מותרת לכל ע״ז וסבור נמי כי אסורא ע״ז של כסף וזהב אב דמיני אחריני שריא לגמרי וכן בשבת דסבור קצירה וטחינה מותרת בכל שבת ואין היום שבת ואומר מותר מיקרי דהא העלם שתי הצדדין בידו, וצ״ע בכריתות בפ״ק (ג:). ויש לי לפרש מלתא כפשטא, דאת״ל העלם זה וזה בידו במקצת מלאכות חייב על כל אחת דשגגת שבת אע״פ שהיא כוללת הכל אינה פוטרתה באחת דכיון שהוא צריך לידיעות הרבה שגגות הרבה נינהו. דמשכחת לה כפשטה שנעלמו ממנו מלאכות כולן ואע״פ שהעלמות הללו נותנת לו העלמה של שבת ע״כ כיון דמכת מלאכות הוא עושה ומחוסר הוא ידיעות הרבה חייב על כל אחת ואחת. ושוכח עיקר שבת דרישא, היינו שאינו יודע מה היום מיומיים כגון תינוק שנשבה לבין הגויים שאפי׳ שאתה אומר לו שבת הוא ומלאכות הן אינו פורש אבל כאן יודע שנאסרה שבת ונאסרו מלאכות. וא״ת כיון שהוא טועה בכולן וסבור שאינן מלאכות היינו העלם זה וזה דאפי׳ ביודע שהוא שבת העלם שבת מיקרי וכיון שהוא בא לו מכח מלאכות דהויין טובא שגגות וחייב על כל אחת ואחת שהרי יודע עיקר שבת (או שהוא) [ושהיא] שבת ואינו נותן דעתו מה הן מלאכות שבו א״נ דאמר הנך דהויין במשכן שריין דלא הוו במשכן קרויין מלאכות ומ״מ אינו יודע עיקר שבת מחמת שגגת מלאכות הוא והיינו העלם זה וזה בידו אלא אי אמרת התם אינו חייב אלא אחת הכא צריך הוא לידיעה גמורה של שבת ושיהא בשגגות מלאכות בלחוד. וזה מסכים ללשון הגמ׳ דקאמר סתם א״א בשלמא העלם זה וזה חייב על כל אחת ואחת שפיר ואלו ללשון הר״ר משה ז״ל היה לו לפרש משכחת לה דידע מקצת ונעלם וחזר וידע ונעלם. כמו שפי׳. ולהאי פירושא, איכא לתרוצי הא דאמרי׳ משכחת לה דידע לה לשבת בלאו ואע״ג דידיעה דלאו לא הויא ידיעה ה״מ בבאין לחייבו בשגגה זו אבל הכא בשגג מלאכות דכרת מחייבין לי׳ אלא דלא הוי שוכח עיקר שבת מחמת שגגת מלאכות אלא בשוכח אותן לגמרי אבל האי דידע דהיום הוא שבת ומלאכות הללו אסורות בו בלאו הא אית לי׳ בשבת ידיעה הלכך משום שגגת מלאכות מצי לחיובו, כן נ״ל. ואי קשיא, מ״מ לד״ה הוי יכול לאקשויי עלה דההיא ממתני׳ בדרך ק״ו, ומה השוכח עיקר שבת חייב כ״ש האי ל״ק דדילמא איכא טעמא למיפטר יודע מקצת ואומר מותר טפי משוכח עיקר שבת משום דזיל קרי בי רב הוא דהואיל והוא יודע מקצת ישאל על מקצת וילך בי רב ולמד שהרי לאו מפורש בה בתורה וכידיעה דמי אבל השוכח עיקרו אינו בא לבית רבו לשאול שאינו יודע כלום ומידון מיני׳ ומינה עדיפא (מדמוי מינה) [מדמויי משאינה מינה]:
}
\newsection{דף עא}
\textblock{הא דתנן \textbf{אכל חלב ודם ונותר ופיגול בהעלם א׳ חייב עכוא״א וכו׳.} ומשמע דב׳ מינין היינו נותר פיגול (וטמא), ק״ל עלה הא דגרסי׳ במס׳ מעילה בפ׳ קדשי מזבח (מעילה יז:) אמתני׳ דתנ׳ הפגול והנותר אין מצטרפין ואמרי׳ עלה ל״ש אלא לענין טומאה אבל לענין אכילה מצטרפין כדתניא כל בקדש פסול ליתן ל״ת על אכילתו. וא״ל הכי קתני, אכל חלב פיגול ונותר חייב עכאו״א וזה החומר שיש בשני שמות שהוא חייב עכאו״א ישנו כן במינין הרבה ואינו במין א׳ אבל לא שהפיגול והנותר הן שני שמות לענין איסור אכילה שלא יצטרפו והיינו דאמרי׳ משני מינין פטור צ״ל ואוקי במין א׳ בתמחויין הרבה (דילמא) [דאלמא] לאו ארישא קאי ולא כמו שמפרשים דבכלל מינין הרבה בין נותר ופיגול טמא ברישא ובין תמחויין (ארישא קאי הרבה כרבי יהושע). ויש לפרש, משני מינין פטור, היינו חלב ודם ונותר או חלב ודם ופיגול אבל פיגול ונותר חלוקין לחטאות מפני שהן שני שמות ואף על פי כן מצטרפין שכלל הכתוב בלאו א׳ שכל שפיסולו בקודש הרי הוא בבל תאכל:
}
\textblock{\textbf{למ״ד אשם ודאי לא בעי ידיעה בתחלה.} פירש״י ז״ל דהיינו רבי טרפון דאמר מה לזה מביא שתי אשמות ור״ע דאמר (כריתות כב:) מביא שתי אשמות סבר בעי ידיעה ויש עליו קושי׳ ממה ששנינו שם ומודה ר״ע לרבי טרפון במעילה מועטת ואמרי׳ בהדיא בגמ׳ מדברי שניהן נלמוד אשם ודאי לא בעי ידיעה בתחלה אלא ר״ע עצה טובה קמ״ל אלא מ״ד אשם ודאי בעי ידיעה בתחלה היינו ר׳ יוסי דאמר התם ר׳ יוסי אומר אין שנים מביאין אשם א׳ מה״ט משום דבעי׳ ידיעה בתחלה כדבעי׳ בחטאת. וק״ל הכא מי איכא מאן דלא בעי ידיעה בתחלה והרי א״א לו להתנות ולומר אם ודאי מעלתי זה אשמי ואם ספק זה אשם תלוי כמו שמתנה גבי ספקמעילות התם שאין מביאין אשם תלוי על שפחה חרופה וי״ אם רצה להביאו יכול להתנות ולומר אם ודאי מעלתי זה אשמי ואם ספק תהא נדבה וא״ת הרי מביא קדשים לבית הפסול א״ל דלמא סבירא ליה כמ״ד מותר להביא קדשים לבית הפסול א״נ (משום) [שאם] הקדישו קדוש ולא בעי ידיעה, כך תירצו בתוספת. ול״נ דהכא נמי אפשר בלא ידיעה כגון שבאו הוא וחבירו על שתי נשים וא׳ מהן שפחה חרופה ויכולין שניהן להתנות ולומר כ״א אם ודאי אני (מעלתי) [בעלתי] זה אשמי ולמאן דבעי ידיעה בתחלה אי אפשר, וזה יותר נכון:
}
\newsection{דף עב}
\textblock{\textbf{אמר לי׳ במעשה דלאחר הפרשה קאמרת וכדרב המנונא א״ל אין.} פירוש ופליגא דעולא דאיהו אמר נמ״ד אשם ודאי בעי ידיעה בתחלה ואילו עולא אפילו למ״ד לא בעי ידיעה בתחלה במעשה דלאחר הפרשה מודה וטעמא דר״ד משום דמדמה להו לעולות שאם עבר על עשה והפריש קרבן וחזר ועבר על עשה כולן מתכפרין בקרבן אחד והכי נמי למאן דלא בעי ידיעה לא שנא ועולא סבר מי דמי התם לא קבע לי׳ על איזה מהם קרבנו בא אבל הכא קרבנו קבוע ונקבע על בעילה ראשונה שהופרש עלי׳ ואתא רבין ופליג ואוקמה כעולא ופריש לתרתי מילי, כך מפורש בתוס׳. וק״ל, דהא משמע מתיוהא דרב המנונא דבמעשה דלאחר הפרשה ליכא מאן דאמר. וי״ל דה״ק, דלמא לאחר הפרשה קאמרת וכרב המנונא וה״ק אפילו למאן דבעי ידיעה בתחלה אינו חייב עכאו״א אלא לאחר הפרשה הא קודם הפרשה מחלוקת ר״י ור״להוא ועולא נמי דאמר קודם הפרשה למ״ד לא בעי ידיעה אינו חייב אלא אחת לאו למימרא למאן דבעי ידיעה חייב עכאו״א אלא מחלוקת נמצאו עולא ור״ד בורחין מן הספקות ומן המחלוקת, ואתי רבין וכללוהו לתרווייהו ופרשינהו, כן נ״ל:
}
\textblock{\textbf{} פירש רש״י ז״ל כגון סכין המוטל בערוגת ירק ונתכוין להגביהו וחתך את הירק וכן יש עליו לפרש לחתוך תלוש זה וחתך מחובר זה. ואין זה מחוור ונכון, וק׳ עלי׳ הא דגרסי׳ (בפיר׳) [בפ׳ הזורק דף צז:] נתכוין לזרוק ד׳ וזרק שמנה אם אמר כ״מ שתרצה תנוח חייב ואם לאו פטור ואצ״ל נתכוין לזרוק במזרח וזרק במערב ולקמן בשמעתי׳ נמי שוגג בלא מתכוין כגון דסבר שומן הוא ואכלו דהיינו נתכוין לחתיכה זו שהיא שומן ואכלה דכוותה גבי שבת נתכוין לחתוך התלוש זה ונמצא שהוא מחובר דחתכו ועוד דנתכוין לזה ועשה מלאכה בזה אפילו מחיוב לחיוב מלתא אחריתא במס׳ כריתות דרב ושמואל פטרי ליה משום מתעסק ואפילו בחד מילתא כגון נתכוין ללקוט תאנים ולקט תאנים וכל שכן תאנים וענבים כולה כדאי׳ התם:
}
\textblock{\textbf{אלא מאהבה ומיראה.} פירש רש״י ז״ל מאהבת אדם ומיראתו ואחרים פירשו מאהבת עבודה זרה עצמה לפי שהוא נאה בעיניו או מיראתו שלא תריע לו והכל עולה לטעם א׳ שכל שאינו חושב שהוא אלוק׳ ולא נתכוין לעבודת אלקות בע״ז אף על פי שהוא עובדה לרבא פטור, ואף על פי שאמרו (סנהדרין עד.) יהרג ואל יעבור אם עבר אינו נהרג בב״ד, ואפילו מאהבה ומיראה דליכא אונס ממש אלא שהוא מתיירא ממנו, פטור. וא״ת היכי קתני משא״כ בשבת. אלא אמיראה וכגון דאנסו לי׳ ממש דבשבת פטור שהרי אמרו יעבור ואל יהרג. וי״א בדין הוא דהו״ל לאקשויי הכי אלא ארישא נמי אקשי לה, וראי׳ לדבריהם הא דאמרי׳, ואביי אמר לא, שגג בלא מתכוין כגון סבור רוק הוא ובולען ל״ל למימר הכי הא אפשר לי׳ לפרשה במאהבה ומיראה כולה בע״ז דל״ק אלא לרבא אלא ש״מ דלאביי נמי לא מתוקמא לה בהכי אלא דעדיפא לי׳ אקשי ארישא גיפא ולדברינו הראשוני׳ אביי לדבריו דרבה קא מתרץ, דמהכא לא ש״מ אלא משום דתיקשי לי׳ לדידיה:
}
\newsection{דף עג}
\textblock{\textbf{מתניתין. אבות [מלאכות] ארבעים חסר א׳.} ירושלמי (ז:): ולמה לא תנינן הושטה עמהון רבי סימון בשם ר׳ יהושע ב״ל מפני המחלוקות ר״ע וחכמים ר׳ יהושע ב״ל בשם ר׳ יתר עליהון הושטה ולמה לא תניתוה עמהון שכל המלאכות באחד וזו בשנים שכל המלאכות יש להן תולדות וזו אין לה תולדה ע״כ, וטעםנכון הוא זה. ומעביר ד״א בר״ה ה״ט דלא תני לי׳, משום דתולדה דמוציא הוא, דכל ד״א של אדם רשותו הוא וכשמוציא חוץ לד״א שלו מוציא מיקרי ובודאי הוא גמרא גמירי לה לחייבו משום מוציא. תדע דאלו לא סמכו לי׳ אמוציא אינו חייב מיתה שאין עונשין מהלמ״מ אלא א״כ נסמכה לתורה שבכתב:
}
\textblock{\textbf{מכה בפטיש.} פירש רש״י, הוי גמר מלאכה, שכן האומן מכה על הסדין בשעת גמר מלאכה. ואיני יודע טעם לזה שאין זו גמר מלאכה אלא אימון ידיו של אומן הוא וכה״ג לאו מלאכה כדאי׳ בהדי׳ בר״פ הבונה (שבת קב:). אלא מכה בפטיש היינו שהאומן מכה בפטיש על הכלי להשוות עקמימותו, וכן על האבן שבבנין להשוות׳ לחברותי׳, והיינו גמר מלאכה, כך פיר״ח ז״ל:
}
\textblock{גמרא: \textbf{אמר רב פפא האי מאן דשדא פיסא לדיקלא ואתר תמרי חייב... משום מפרק.} פירש״י ז״ל שאף זה מפרק תמרים מן הדקל. ותמה אני, בוצר וגודר ומוסק ליחייב שתים. אלא עיקר הפי׳ כדברי האומר שהן נתלשין עם המכבדות ואחר כך דשין אותן שמחמת דשדא פיסא לדיקלא נתלשו המכבדות מהדקל וכשהן נופלין לארץ מכוחו הן נחבטין ונדושין ונפרקין התמרים המכבדות, והיא תולדה דדש:
}
\newsection{דף עד}
\textblock{\textbf{שכן עני אוכל פתו בלא כתישה.} יפה פירש רש״י ז״ל דלעולם במקדש הוי ומחייב עלי׳ אבל מאחר שהוא בכלל בורר ויפה כת בורר מכותש כי הכל בוררין ואין הכל כותשין לפיכך מנו בורר ולא כותש וקרוב לזה פר״ח ז״ל שאמר אף על גב דכתישה מלאכה הוא האי דלא חשיב בהדי אבות משום דלא חשיבא שלא כל אדם אוכל בכתישה לפיכך אינה אלא כתולדה ע״כ ואף על פי שאמרו בר״פ הזורק (שבת צו:) דהך דהוה במשכן חשוב וקרי לי׳ אב, שאני כתישה שישנה בכלל חברותי׳. ורבינו תם ז״ל נדחק בה, ופי׳ וליחשוב נמי כותש שהיו כותשין שעורים לנסיוני דתכלת שטומנין אותו בתוך פת של שעורים ואופין אותו אם נשתנה פסול ואם למעליותא נשתנה כשר וקסלקא דעתך שאין אופין פת אלא בכתישה ופריק הכי כמה עניים אוכלין פיתן שלא בכתישה ולא היו כותשין ג״כ במשכן. וקשיא לי׳ הא דאר״פ שביק תנא דידן בשול סממנין דהוה במשכן ונקט אופה אלמא אפייה ליתי׳ במשכן ופריק ה״ק שביק תנא דידן בישול שהוא מן המלאכות הצריכות מחמת עצמן ונקט אופה שאינו אלא בשביל נסיון תכלת. ואין הפי׳ הזה נכון אצלי, לפי שלא נאמר נסיון זה במנחות אלא בלוקח תכלת צבוע ומשום חששא דקלא אילן ובמשכן הן עצמן צובעין אותו כדתנן (עג.) והצובע ועוד שאלו היו אופין במשכן כלל היו מונין אבות מלאכות ארבעים שאף על פי שאופה ומבשל שוין אינן לגמרי דבר א׳ עד שלא ימנו שתיהן ויותר הי׳ ראוי למנותן שתים מבורר וממרקד וזורה: }
\textblock{\textbf{וכי מותר לאפות לבו ביום וכי מותר לבשל לבו ביום, אלא אמר רבה וכו׳.} פי׳ מסברא קסבר רבה שאין בבורר פחות מכשיעור (חייב) [איסור] כאופ׳ פחות מכשיעור, שאין דרך ברירה בכך אלא (באוכל) [כאוכל] ומצא פסולת בחתיכ׳ שהוא אוכל ומפריש פסולת מתוך אוכל הוא, שמותר, ורב יוסף אמר דכיון דכשיעור חייב פחות מכשיעור נמי אסור דדרך בירור הוא שכל בורר קימעא קימעא הוא בורר עד שמצטרף לחשבון גדול אי נמי רבה לאו לפרוקי קושיא דר״ח אתא אלא לפרוקי ברייתא אתא וקושי׳ לא שמיע לי׳ ועולא ורבה תרווייהו אברייתא קיימי וגמרא דקא מסדר להו בתר אתקפתא וכן במקומות הרבה בתלמוד כאותה דבפ״ק (ד.) בהנחה על מקום ד׳: }
\textblock{והא ד\textbf{אמר רב יוסף בקנון ובתמחוי לא יברור ואם בירר חייב חטאת.} אוסופי הוא דקא מוסיף אלישנא דבריי׳ משום דאורחא דתנא הכי שאינו (מדליק) [מדלג] מיד דמותר לנפה וכברה שהוא חייב ומניח בנתיים קנון ותמחוי שפטור אבל אסור ומ״ה קאמר (והא) [דהאי] דלא יברור דקתני בבא באפי נפשי׳ הוא דפטור אבל אסור. והא דאקשי׳ הכא מידי קנון ותמחוי קתני לאו עיקר קושיא משום קנון ותמחוי בלחוד שא״כ הי׳ לו אפשר לאומרה בלא תוספת ולא הו״ל לר״ה להחליף לגמרי אלא משום דקשיא נמי מידי נפה וכברה קתני כמ״ש רש״י ז״ל. ויש מקשים, היכי תיקשי מידי נפה וכברה קתני, שהרי סתם בורר בנפה וכברה הוא מדחייב עלה אלמא עיקר דרך בכך ותניא נמי לקמן חייב ואוקים בנפה וכברה ולא מקשי׳ מידי נפה וכברה קתני. ולאו קושיא הוא כלל, דאנן הכי מקשי׳, כיון דרישא מוקמת ביד סיפא נמי ביד הוא ולא בנפה וכברה ולא בקנון ולא בתמחוי שאם הי׳ כן הי׳ לו לתנא לפרש כן אבל בתי׳ דאביי לק״מ משום דבורר ואוכל בורר ומניח לאלתר משמע ובורר סתם לבו ביום משמע: }
\textblock{\textbf{והתניא חייב.} פירש רש״י ז״ל, היינו הא דתניא לעיל בתירוצו דאביי ולבו ביום לא יברור ואם בירר נעשה כבורר לאוצר וחייב חטאת ואי הא דרב אשי בבורר ואוכל לאלתר מותר לכתחלה הוא וליכא פטור אבל אסור. ואין הפי׳ הזה נכון, משום דמוקמי׳ לה בנפה וכברה ואם כן היכי קתני רישא לאלתר מותר והא אמרי׳ במס׳ ביצה (יד:) ובנפה ובכברה לא יברור כ״ש בשבת שאסור. ויש לפרש דבריי׳ אחרת הוא ופלוגתא דרב אשי ורב ירמי׳ מדיפתי בשני מיני אוכלין שאין בהן פסולת כלל לפיכך נחלקו בה בקנון ובתמחוי ולבו ביום דמר פטור ומר מחייב אבל לאלתר מיתר וברייתא דלעיל באוכל ופסולת מעורבין ובקנון ובתמחוי ומשום הכי לבו ביום חייב חטאת ומיהו לאלתר מותר. אבל מצינו לר״ח זכרונו לברכה שפירש לבריי׳ דלעיל לאוקימתן דנחמני בבורר ביד ואפילו הכי לבו ביום (מותר) [חייב], ופלוגתא דרב אשי ורב ירמי׳ בקנון ובתמחוי ולאלתר ובנפה וכברה לעולם חייב חטאת שנעשה כבורר לאוצר וכן עיקר דליכא ברירה [בכלי] מותרת בשבת כלל בין בקנון בין בנפה דתניא התם בביצה המולל מלילות בע״ש למחר מנפה על יד אבל לא בקנון ולא בתמחוי והיינו לאכול לאלתר דאלו להניח אפילו בו ביום (אסור) כדתנן המולל מלילות של חטים מנפח על יד ואוכל ואם נפח ונתן לתוך חיקו חייב ואר״א וכן לשבת א״ו בבורר ואוכל לאלתר היא ואפילו הכי בקנון אסור וכן משמע דמהך בריי׳ הוינן ומתרץ לה כפי אוקימתא קמייתא (לרווחא) [ולרווחא] דמילתא [נוכל] לומר דאינון פליגי בקנון לאלתר אבל בנפה אפילו לאלתר מודה והיינו בריי׳ ומיהו אפי׳ לאוקימתא דאביי דינא הכי הוא. וכתב, ואף על גב דגרסינן לענין מי שאבד לו גט אם מצאו לאלתר כשר ואמרי׳ ה״ד לאלתר ואמר הלכה כל זמן שלא עבר אדם שם ואחרים אמרו כל שלא שהה אדם שם הכא כיון דאשכחן מפורש בתלמוד א״י כוותי׳ עבדינן ומסקי לשמעתי׳ שיעור מה שמיסב על השולחן באותה סעודה בלבד ופירש הרב ז״ל דה״מ אובל מתוך פסולת אבל פסולת מתוך אוכל לעולם אסור וחייב חטאת הוא מדחזקי׳ וכן פסקה רבינו הגדול ז״ל בהלכות. וא״ת, הרי התירו לנפח על יד, י״ל התם משום דמשני ואין דרך ניפוח בכך ואף על פי שהוא פסולת מתוך אוכל מותר. ואיכא דקשי׳ לי׳ מהא דגרסי׳ בפ׳ תולין (שבת קלח.) משמר משום מאי מחייב, ואמרי׳ משום בורר מה דרכו של בורר נוטל אובל ומניח פסולת אף כאן נוטל אוכל ומניח פסולת והכא אמרי׳ דדרך בורר ליטול פסולת מתוך אוכל ולאו קושיא הוא דהא אמרן שאף בבורר אוכל מתוך פסולת משכחת חיובא בנפה וכברה והתם ה״ק אף אי אתרו בי׳ משום בורר התראתו התראה דאיכא בורר דדמי לה כגון אוכל מתוך פסולת. אבל בתוס׳ מתרצים בשאוכל מרובה על הפסולת דרך בורר לברור (אוכל) ולהניח (פסולת) וכשהפסולת מרובה על האוכל דרך לברור (הפסולת) ולהניח (האוכל) ולאמסתברא דגבי שבת לעולם אסור לברור פסול׳ ולהניח אוכל, ואע״ג דתרווייהו דרך בורר נינהו בשבת בהתירא טרחינן באיסורא לא טרחינן ומאן דטרח באיסורא ובירר הפסולת הו״ל כבורר לאוצר שהרי אין דעתו לאכול מה שבירר ולפיכך חייב: }
\textblock{\textbf{מהו דתימא לשרורי מנא קא מכוין קמ״ל דמרפא רפי והדר קמיט.} קשי׳ להו לתלמידי מהא דמסקי׳ במס׳ ע״ז גבי בישולי גוים דלשרורי מנא קא מכוין וי״ל אף על גב דאיהי מסיק לשרורי מנא כיון דאי אפשר אא״כ רפו מעיקרא גבי שבת חייב דמודה ר״ש בפסיק רישי׳ ולא ימות, כך מתרצי׳ בתוס׳:
}
\textblock{\textbf{והוא שקשרן.} פי׳ לא קשירה ממש הוא דא״כ תיפוק לי׳ משום קושר ואם לחייבו שתים אין זו מלאכה בעצמה אלא שמרכיבין זע״ז כעין קשר:
}
\textblock{הא דאמרי׳ \textbf{תולש היינו גוזז.} שמעי׳ מינה דאינו מחויב בתולש מחיים יותר מלאחר שחיטה כיון דלאר שחיטה שייך נמי גזיזה דלאו משום עוקר דבר מגידולו הוא דנימא דחייב עלה מחיים ולא אחר שחיטה דא״כ היינו קוצר אלא דוקא משום גוזז חייב ולא משום קוצר דקצירה איננה רק בגידולי קרקע. ובתוספתאנמי תניא הגוזז מן הבהמה ומן החיה ומן העופות אפילו מן השלח מלא הסיט כפול חייב אלמא חיוב משום גוזז ישנו לאחר שחיטה כמו בחיים. ומכאן אני אומר שמותר לתלוש ביום טוב מן הנוצה של עוף קודם שחיטה לצורך שחיטה כיון דלאחר שחיטה ע״כ יתלוש הנוצות לצורך האכילה הרי ניתן מלאכה זו לדחות לצורך אכילה אף קודם שחיטה מותר ועכשיו נמי צורך קצת הוא דבשלמא אי הוה אמרי׳ דאחר שחיטה איננה מלאכה כלל וקודם שחיטה הוי מלאכה א״כ כיון דלא ניתן מלאכה זו לדחות לצורך אוכל נפש כיון דלצורך היינו לאחר שחיטה איננה מלאכה א״כ לא ניתן לדחות מלאכה זו כלל ממילא קודם שחיטה אסור אבל כיון שבררנו דאף אחר שחיטה חשיב גוזז ומותר משום צורך אוכל נפש ה״ה קודם שחיטה. והא דקאמר הש״ס במסכת בכורות פ״ג (כד:)השוחט את הבכור עושה מקום לקופיץ ובעי בש״ס כנגדו בי״ט מהו ומסיק דתולש לאו היינו גוזז ואקשי׳ עלה מהא דתניא התולש את הכנף הקוטמו והמורטו חייב ג׳ חטאות ומשני שאני כנף דהיינו אורחי׳ לאו למימרא דאסור בי״ט לתלוש מן העוף [במקום שחיטה] אלא גבי צמר מן הכנף שאין סופו לתלוש ואין תלישתו צורך י״ט ושפיר הקשה מ״ש גבי עור (דאמר׳) [דאמרת] דתולש בשבת חייב אלמא דתולש היינו גוזז. ובירוש׳ פ׳ כלל גדול (ה״ב) איתא, וה״ג התם: תמן תנינן השוחט את הבכור עושה מקום לקופיץ מכאן ומכאן ותולש את השיער ובלבד שלא יזיזנו ממקומו וכן תולש את השיער לראות את המום רבי הילא בשם רבי שמעון בן לקיש מזו התולש בקדשים פטור א״ר יעקב בר אחא רשב״ל כדעתי׳ דאתפלנין התולש בקדשים ר׳ יוחנן אמר חייב רבי שמעון בן לקיש אמר פטור רבי ירמיה בעי מחלפי שיטתי׳ התולש כנף מן העוף המורטה הקוטמה חייב ולא דמי׳ עוף שאין לו גיזה תלישתה גיזתה ברם הכא אינו חייב עד שיגזוז ע״כ. למדנו עכשיו שהתולש מן העוף המתה חייב משום גוזז כדרך שחייב גיזז עצמו בשלחין לאחר מיתה וכיון שכן ממילא למדנו שביום טוב מותר שאין חילוק בין מחיים לאחר מיתה ונתנה יום טוב לדחות אצל מלאכה זו כמו אצל אוכל נפש וליכא למימר דבעוף מחיים אסור משום תולש דבר מגדולו והאי דנקט משום גוזז לחייבו לאחר שחיטה נמי דחדא תולדה לשני אבות ליכא לעולם ועוד אי הכי בבהמה (לתסור) [ליחייב] דהא אורחיה הוא לתלוש ביד דבר מגידולו ולאו אורחי׳ דגזיזה אלא בכלי אלא אנן הכי קאמרי׳ כיון דמשום גוזז הוא כי אורחי׳ חייב כגון בעוף ובבהמה דלאו אורחי׳ הוי מלאכה כלאחר יד ופטור ומשום תולש ליכא דלא הוי תולש אלא בגידולי קרקע (צז.) הא מבעלי חיים משום גוזז איכא משום תולש ליכא ומעיקרא קס״ד דכל ביד תולדה דתולש הוא וכל בכלי תולדה דגוזז והשתא אמרי׳ דכולן משום גוזז הלכך ליכא למיסר משום תולש כלל. אבל ראיתי למקצת מחברים האחרונים ובכללם המחבר הגדול רמב״ם ז״ל [שאוסרים], ואני איני כדאי לחלוק, אבל האמת יורה דרכו:
}
\newsection{דף עה}
\textblock{\textbf{תנו רבנן הצד חלזון והפוצעו אינו חייב אלא אחת.} תימה הוא שהי׳ לו לשנות הפוצע חלזון פטור ר׳ יהודה מחייב:
}
\textblock{והא דאמרי׳ \textbf{ש״ה דכמה דאית בי׳ נשמה ניחא טפי כי היכי דליציל צבעי׳.} ופירש רש״י ז״ל שהיה טורח לשמרו שלא ימות. פי׳ לפירושו: והוא (פוצעו) בדרך שאפשר שלא ימות ואם מת לא נעשה מחשבתו והיינו דקרינן לי׳ מתעסק. ואחרי׳ פירשו דהו״ל מלאכה שא״צ לגופה דלא ניחא לי׳ במיתה, ולא דייק. וק״ל, לרבי יהודה מיהא ליחייב תלת כלומר חדא משום נטילת נשמה ואפשר דלדבריהם דרבנן קא״ל דאף על גב דפטריתו לי׳ משום נטילת נשמה אודי לי מיהת דחייב נמי משום דישה. ונ״ל דלרבי יהודה נמי מתעסק פטור דהא קשה לי׳ ולא איתעבידא מחשבתו:
}
\textblock{ולרב \textbf{משום צובע אין משום נטילת נשמה לא.} פי׳ א״כ קשי׳ מתני׳ דקתני צובע ושוחטו בתרתי, ופריק אימא אף משום צובע:
}
\textblock{\textbf{השף בין העמודים.} פירש רש״י ז״ל, השף בקרקעות הבנין שהעמודים נשענין עליו כדי שיהא חלק חייב משום ממחק ולא מסתברא אלא משום בונה ובשדה משום חורש כדאמרי׳ בגומא. אלא ה״פ: השף העור בין העמודים כדי לרככו חייב משום ממחק שכן דרך הרצענין להחליקו ולרככו בכך ומפורש בירושלמי המוחקו מה מחיקה הי׳ במשכן זעירא בר חנינא בשם ר׳ חנינא אמר שהיו שפין את העור ע״ג העמוד. חייב משום מאי חייב ר׳ יוסי בשם רבי יודא בן לוי ור׳ אחא ר׳ יודא בן לוי בשם ר׳ יהושע בן לוי, משום ממחק:
}
\textblock{\textbf{לאפוקי מדר״ח דמחייב אתולדה במקום אב.} איכא דקשי׳, לאפוקי מדר״א מרישא שמעת מיני׳ דקתני חייב על כל אב מלאכה ומלאכה ומשמע ולאו אתולדה במקום אב וקתני נמי העושה מלאכות הרבה מעין מלאכה אחת אינו חייב אלא חטאת אחת. ומפרקי׳ אי מרישא ה״א על כל אב מלאכה לאפוקי קצר וקצר בהעלם אחד ועושה מלאכות הרבה מעין מלאכה אחת היינו רישא והכי קתני חייב על כל אב מלאכה ולא על אב א׳ שתים שהעוש׳ מלאכו׳ הרבה מעין מלאכ׳ א׳ אינו חייב אלא אחת. ובנוסחאות מדוקדקות ל״ג במתני׳ חייב על כל אב מלאכה אלא חייב על כל מלאכה ומלאכה. ומיהו בכריתות (טז:) מוכח מרישא דאתי׳ דלא כר״א ואפשר דסמיך אסיפא כיון דסיפא אתיא דלא כר״א לא דחקי׳ לאוקמי לרישא כר״א ומוקמי׳ לה כפשטא ואתיא כרבנן וי״ל דאף על גב דמרישא שמעת מינה לאפוקי מדר״א סתים נמי בהך משנה אחר׳יתי דלא כוותי׳ דכיון דאיצטרכא לי׳ לתנא למתניי׳ לאפוקי מדר׳ יהודה תנא אלו אגב אורחי׳ לאפוקי לה נמי מדר״א וכיוצא בה הרבה בתלמוד:
}
\textblock{ודאמרי׳ נמי \textbf{חסר אחת לאפוקי מדרבי יהודה.} קשי׳, הא תני לה רישא אבות מלאכות ארבעים חסר א׳ וא״ל אי מרישא ה״א תנא ושייר אבל סיפא כיון דתנא אלו ליכא למימר תנא ושייר כדאקשי׳ בעלמא תני תנא אלו ואת אמרת תני ושייר. ועוד ממשנה יתירה דוקא אלו ולא יותר:
}
\textblock{ודא״ל רבנן \textbf{שובט הרי הוא בכלל מיסך מדקדק הרי הוא בכלל אורג.} קשיא נמי, דהא כל מלתא דהוי במשכן אע״ג דדמיא לה חשיב לה וי״ל שובט ומיסך מדקדק ואורג א׳ הן לגמרי ולא שנים הדומין:
}
\textblock{\textbf{אר״י בר חנינא הא דלא כר״ש.} פירוש כולה מתניתין, דמתני׳ קתני כל הכשר להצניעה ומצניעין כמוהו אזלי׳ בתר רובא דעלמא ואלו ר״ש לא חייב אלא מצניעיהן ממש וסיפא נמי דקתני אינו חייב אלא המצניעו דאלמא מצניעו לא בעי שיעורא כלל דלא כרבי שמעון דאלו רבי שמעון קאמר כל השיעורין הללו למצניעיהן, כך פירש רש״י ז״ל. ואף על גב דבתר הכי מייתי׳ בגמרא פיסקא דסיפא דמתני׳ משום דארישא איתמר מתני׳ דלא כרבי שמעון וכולה בכללא והדר מייתי לסיפא דהוא בנחוד אתיא דלא כרשב״א. ומצינו פסקא אחרונה שבמשנתינו שהביאו אותה בגמ׳ ראשונה ואחר מביאים הראשונה ומדברים עליה ואחת מהן בפ׳ חזקת הבתים (ב״ב לו.):
}
\newsection{דף עו}
\textblock{\textbf{והתניא כגרוגרות אידי ואידי חד שיעורא הוא.} ואי קשיא לימא כגרוגרת, יש לומר אי תנא הכי הוי אמינא משום דחזי לאדם קמ״ל דטעמא משום טלה הוא דחזי לי׳ וכן אתה מפרש בכ״מ שאמרו (ו)כיוצא בה, כי ההיא דלקמן באידך פירקין (שבת ע״ח ע״ב):
}
\textblock{\textbf{המוציא כדי גמיעה.} אין שיעורו ידוע לנו, והוא פחות מכדי רביעית כדמוכחא מתני׳ ואין צריך לומר שהוא פחות מכגרוגרות ואע״פ ששיעור חולב כגרוגרות כדאמרינן בפ׳ המצניע התם הוא משום דאין אדם טורח לחלוב בפחות מכן אבל הוצאה בכדי גמיעה חייב ובמס׳ יומא גבי יוה״כ פליגי תנאי מ״ס כדי רביעית ומר אמר מלא לוגמיו ומר אמר כדי גמיעה למדנו מ״מ שאינו רביעית ולא מלא לוגמיו ומסתברא גבי שבת כדי גמיעת אדם בינוני קאמרינן ולא לעוג מלך הבשן ולא לכל אדם בשלו אלא בשל עולם שמין בבינונית, ואפי׳ למ״ד גבי יוה״כ לכל חד וחד בדידי׳ משערינן:
}
\newchap{פרק \hebrewnumeral{8} המוציא יין}
\newsection{דף עז}
\textblock{}
\textblock{\textbf{ועוד מים בכד ומצטרפין.} ויש לפרש דאביי פליג אשיעורא דרביעית ופשיט׳ הוא. ולי נרא׳ דכ״ע כוס של ברכה שיעורו ברביעי׳ הוא ואביי ה״ק מים בכד ומצטרפין אלא מתני׳ לאו יין שיכול אדם למזוג בו במים כוס יפה קתני אלא יין כמו שהאדם עשוי ליתן בכוס של ברכה קאמר דהוי קרוב לרביעית וחשיב דאית בי׳ שיעור שתי׳ כמלא לוגמיו או יותר מעט באפי נפשי׳ בלא צירף דמים אבל רובע רביעית לא חשיב לשתי׳ כלום. ואיכא למידק הכא, ואביי מי לית לי׳ ששיעור יין ברביעית ולא חייבו אלא בכדי שימזגנו ויעמוד על רביעית והא א״ל אביי לרב יוסף דלמא ע״כ לא אמר ר׳ נתן הכא דכזית בעי רביעות וכו׳ אלמא בקרוש לאו בר מזיגה הוא בעי רביעית ממש ובשאינו קרוש מזיגתו מצטרפת עמו ואפשר לומר לדבריו דר״י אקשי לי׳ אבל איהו סבירא ליה שיעור בכדי מזיגת הכוס חשיב ואפי׳ לא מצטרף מים בהדי׳ ויין קרוש בכזית לדידי׳ הוא שיעור מזינת הכוס בשאינו קרוש. ויש לפרש דאביי נמי סבר דשיעור יין בכדי שימזגנו ויעמוד על רביעית ולא פליג אלא אמזיגה דרבא וה״ק אא״ב מזיגה שני חלקי מים ואחד יין כיון שא״א לשתות יין אלא במזיגה כיוצא בה השתא נמי חשיב לשתיה שכל שעומד למזוג כמזוג דמי אלא אי אמרת רובע רביעית וכדי שימזגנו בחד תלתא מיא הרי אינו עומד למזוג כולי האי שמעוט אנשים הן מוזגין כן ולמה מצטרפין דודאי אפי׳ לרבא דאמר כל חמרא דלא דרי על חד תלתא מיא לאו חמרא הוא מזיגה יתירתא הוא ואין רוב ב״א מוזגין כן כדאמר ר״י במס׳ נדרים (נה.) דמי האי מזיגה למזיגה דרבא ברי׳ דר״י בר חמא אלמא מזיגה יתירה הוא. ובתוס׳ מקשי׳ מהא (דתניא) דגרסי׳ בפ׳ ע״פ (קח:) ד׳ כוסות הללו צריך שיהא בהן כדי רביעית אלמא שיעור כוס רובע רביעית דהוי לד׳ כוסות רביעית וזו אינה קושיא כלל דהכי קאמר צריך שיהא בכל א׳ רביעית שתאו חי רביעית דחי שתאו מזוג רביעית מזוג ולא פרש שם מזיגתו בכמה ובודאי מודה הי׳ אביי דשיעור כוס של ברכה רביעית שכך הוא שיעור כל המשקין רביעית יין לנזיר רביעית ליוה״כ. ויש מקצת נוסחאות משובשות שכ׳ בהן בההוא דערבי פסחים מאי כוס יפה דקאמר שמואל בהאי שעור׳ לכל חד וחד דהו״ל רביעית לכולהו ואין בנוסחאות מדוקדקות כן אלא אידי ואידי חד שיעורא הוא וכן גריס הר״ר שמואל ז״ל ואפי׳ להאי גירס׳ נמי (ק׳) דסוגיא דהתם אליבא דרבא, אבל אביי מפרק לה כדאמרן:
}
\newsection{דף עח}
\textblock{\textbf{אמר אביי בגלילא שנו.} פר״ח ז״ל מי גליל ידועין הן שמועלין לעין כקילור אבל מיא דעלמא ברביעית כשתיה פי׳ לפירושו שמאחר שהן ידועין לרפואה ובכ״מ מתרפאין מהן רפואתן מצוי׳ בכ״מ כשתייתן במקומן ויותר ודאמרי׳ נמי דשמואל דאמר כל שקיינו מסו ומטללי לבר ממיא ר״ח ז״ל גריס בה מסו ומצערי בר ממיא דמסו ולא מצערי ופי׳ שהמים כולן אפי׳ שתייתן לרפואה היא וכיון דעיקר שתייתן לרפואה שעורן ברפואה פי׳ לפי׳ אע״פ שהשתיי׳ אינה אלא ברביעית כיון שהוא עושה רפואה ופעמים שעושין מהן רפואה לקילור רפואתן שכיחא והולכין אחר רפוא׳ קטנה שלהן לחומרא שהוא רפואות הקילור. ולפי הפי׳ הזה ראוי לפרש כל שקיינו כגון העשויין לרפוא׳ כגון אספרגוס וחבריו. ויש לדקדק אחר פי׳ של רש״י ז״ל דהואיל ושתייתן שכיחא ורפואתן לא שכיחא אע״פ שהוא רפואה טובה לעין יותר מכל הקילורין בתר שתי׳ הו״ל למיזל ויש לפרש דמעיקרא כי אמרי׳ רפואתן לא שכיחא משום דאיכא שקיינו אחרינא דמסו בהו ועדיפי ממיא קאמרי׳ והשתא מתרץ רפואתו נמי שכיחא דהא מיא עדיפא מכולהו שקיינו ואע״ג דשתייתן שכיח׳ טפי דשכיח׳ ושכיתא אזיל רבנן לחומרא אע״ג דלא שכיחא כולי האי:
}
\textblock{\textbf{אטו מצניע לאו מוציא הוא.} נראה שכך פירושו הול״ל בד״א כשאר כל אדם אבל המצניעו בכל שהוא מדקאמר במוציא משמע דמצניע לאו מוציא הוא ואלומי אלי׳ לקושי׳ דהו״ל למידק אטר (אחר שלא הצניעו) [מצניע שלא הוציאו] מי מחייב במשהו:
}
\textblock{\textbf{אמר אביי הכא במאי עסקינן בתלמיד.} וה״ק במה דברי׳ אמורים במוציא שאינו מחשב כלל אלא להוצאה שנצטוה עלי׳ אבל במצניעו רבו חייב זה במחשבתו של רבו לפי שדעתו תלוי בדעת רבו ואלו תנא במה דברי׳ אמורים בשאר כל אדם אבל במצניע כל שהוא הייתי אומר שאין שום אדם מתחייב בכ״ש אא״כ (המצניעו) [הצניעו] הוא עצמו ולפיכך שנה בד״א במוציא כלומר אדם שאינו מתכוין אלא להוצאה בלבד ואינו יודע חשיבותו של דבר ואם הוא מוצנע חייב. ומני, ר״ש הא פליג עלי׳ ואלו רשב״א בלא רבו נמי מחייב זה במחשבתו של זה, [אלא כת״ק]. ורש״י ז״ל מוקי לה כרשב״א, ואינו מחוור דאיהו לכל אדם מחייב בזה במחשבתו של זה ולא היה צריך לאוקמי בתלמיד שא״ל רבו וכולה אוקימתא דאביי ועוד למה לי לדוחקה ולאוקמי כיחידאה ה״נ הוי יכול למימרא לפי פירושו הכי בד״א במוציא שלא הצניעו אבל הצניעו כ״ש ואי משום דברישא איירי רשב״א אדרבה משמע דרבנן הוא דקתני רישא סתם דם וכל מיני משקין ברביעי׳ והך סיפא לאו בתר רשב״א גרירא דנימא כולה מילתא דידיה הוא אלא סיימו רשב״א ורשב״ג מילתייהו ואתאן לת״ק ועוד דאי הוי אפשר למימר דה״ק בד״א במוציא דבר שלא הצניעו וכו׳ וכרבנן לא הוי מקשי בגמ׳ אטו מצניע לאו מוציא הוא דפשיטא לן דאיכא חילוק בין מוציא שאינו מצניע למצניע דהא איפלגי בה במתני׳. ואיתא נמי בפ׳ כלל גדול וכלישנא דברייתא לא משני מידי אלא ש״מ דלישנא דברייתא לא דייקי׳ לי׳ כדפריש׳:
}
\textblock{מתני׳: \textbf{נייר כדי לכתוב קשר של מוכסין והמוציא קשר מוכסין חייב.} יש לפרש דסד״א דהמוציא קשר מוכסין פטור והמוציא נייר כדי לכתוב עליו קשר מוכסין לא נתחייב בשביל שהוא מיוחד לכך שהרי אפי׳ בקשר עצמו פטור אלא מפני שהוא שיעור לכתוב עליו שתי אותיות דידן שהוא שם משמעון ובית אחיזה שהוא קשר מוכסין לפיכך חזר ושנה שהמוציא קשר מוכסין חייב שהוא דבר תשוב בעצמו. ויש לפרש והמוציא קשר מוכסין חייב אפי׳ לאחר שהראהו למוכס, וסתמא כר׳ יהודה:
}
\newsection{דף עט}
\textblock{\textbf{עד שיאמר לוה פרעתי ולא פרעתי.} פי׳ הכא במאי עסקינן שבקש ולא מצא עדים שמתו אותן שחתמוהו ואין שום אדם מכיר התם ידן שאלו היה הדבר ספק השטר חשוב וראוי שמא יביאו עדים ויקיימוהו ואין לומר שצריך הוא להחזיר השטר ללוה ולפיכך הוא פטור, שהואיל ולא מצא עדים שהוא מזוייף בודאי, כך ה״ה שיהא מונח בידו של מלוה זה שמא ימצא עדים לקיימו או עד שימצא הלה עדים שהוא מזוייף או עד שיבוא אליהו. ואפשר לפרש שאפי׳ הדבר ספק הואיל ולא מצא עדים עכשיו אינו מתחייב משום שמא ימצא עדים שמ״מ אינו ידוע אם הוא דבר הראוי אם לאו. וא״ת עד שלא אמר פרעתי אמאי חייב למאי מהני לי׳ שטרא אי מודה לי׳ אפי׳ בלא שטר נמי חייב אי לא מודה לי׳ הא אמרת פטור והשטר פסול וי״ל לא חציף אינש למימר שטרא זייפא הוא אא״כ הוא אמת שהוא מתיירא שמא יבואו עדים ויעידו ונמצא שהוחזק כפרץ ובפרעון נמי מתבייש דא״ל אידך אי פרעתני שטרך בידי מאי בעי ואי לא נפיק שטרא מתותי ידי׳ א״ל לא היו דברי׳ מעולם וי״א מהני לזה דאי מודה בה גובה מהיתומים ואינן יכולין לומר פרענו ואי לאו שטרא יכולין לומר כן, ולמ״ד (ב״ב קעה:) מלוה ע״פ לא גבי מיתמי לא גבי כלל א״נ מהני ללקוחות שלקחו ממנו אחר שיודה לו בב״ד:
}
\textblock{\textbf{ור״י סבר אין כותבין שובר.} פרש״י ז״ל ומ״ה צריך ליה שאם לא יחזיר לו שטרו הדין נותן שיחזיר לו ממון שפרעו ומכאן למדנו דבר זה ואין לנו ראיה ממקום אחר שהיה במשמע שאחר שפרעו אפי׳ למאן דאמ׳ אין כותבין שובר אינו חייב להחזיר לו שאינה אלא חששא שמא יאבד שטרו והואיל ופרעו פרעו. ור״ח ז״ל פי׳ כגון שפרעו ולא עמד עמו בדין ולרבנן דאמרי כותבין שובר א״צ לי׳ זה לשטר שהוא מוציא שוברו ולר״י דאמר אין כותבין שובר הוא צריך לו כדי שלא יוציאנו מלוה ויגבה בו פעם אחרת. ונראה שהוא ר״ל דלרבנן כיון דכותבין שובר וחששא דצריך להיות שומר שוברו מן העכברים לאו חששא הוא אין שטר זה כלום לר״י כיון דחששא היא שטר זה ראוי הוא ומתחייב עליו המוציא:
}
\textblock{\textbf{כדי לכתוב עליו את הגט.} פרש״י ז״ל תורפו של גט, דהיינו שמו ושמה ומקום העדים והזמן והרי את מותרת לכל אדם והזקיקו לומר כן כדי שיהא שיעור דפתרא דמליח וקמיח מועט משיעור עור דמליח ולא קמיח ויפה פירש וא״ת בגט הא בעינן ודן ותורפי׳ דגיטא נ״מ להתגרש בדאורייתא ויכולין אנו לומר שאע״פ שהוא יותר מעובד שיעורו מרובה שהרי קלף מעובד הוא ושיעורו גדול משיעור עור כדתנן במתני׳ ופרש״י ז״ל עצמו דאיידי דדמיו יקרים לא עביד מיני׳ קשר מוכסין אלא לתפילין ומזוזות, ולא מתחייב בשיעורא זוטא. מ״מ למדנו דסתם עיבוד היינו מלוח דהא אמרן לעיל דבין מעובד ובין שאינו מעובד שיעורן לעשות ממנו קמיע וכאן במליח וקמיח שיעור אחר לכתוב גט אלמא סתמא לא לקמוחי קאי אלא למלוחי דהיינו עבוד כדאמרן בפ׳ כלל גדול (שבת עה:) היינו מולח היינו מעבד וכו׳:
}
\textblock{\textbf{ורמינהו קלף ודוכסוסטוס כדי לכתוב עליו מזוזה וקלף כדי לכתוב עליו פרשה קטנה שבתפלין שמע ישראל.} תימה הוא ותיקשי ליה רישא אסיפא כדקשי׳ לי׳ במתני׳. ויש לי לפרש דקס״ד דהכי קתני קלף ודוכסוסטוס סתמן למזוזה והמוציאן בסתם כדי לכתוב עליהן מזוזה והמוציא קלף מפורש לכתוב עליו תפילין או שהצניעו לכך שיעורו בכדי פרשה קטנה שבתפילין ומש״ה מקשי׳ סתמא אסתמא דמתני׳ דודאי מתני׳ לאו להצניעו לתפלין קאמר דסתמא קתני ודומיא דהנך כולהו. ומקצת נוסחאות יש שגורסין תחלה קלף ודוכסוסטוס כדי לכתוב עליו מזוזה ותו לא והדר מייתי הא מדקתני סיפא קלף כדי לכתוב עליו תפילין וכו׳ ומקשי׳ ולטעמיך תיקשי לך היא גופה וזו מיד מגיה סופרים הוא:
}
\textblock{\textbf{אלא כי תניא ההוא בס״ת.} מהא שמעי׳ שס״ת נכתב על הגויל ועל הדוכסוסטוס ועל הקלף ומיהו מצותה על הגויל ומ״ה קתני על הקלף ועל הדוכסוסטוס כשרה משמע אבל לא תעשה וארחה דבני תלמודא למכתב אגויל כדתנן בגיטין גוילין שבו לא עבדתין לשמן. והא דאמרי׳ בב״ב שיעור ס״ת בכמה בקלף אינו יודע משום דיעבד כשרה ובדין היא דהו״ל למישאל בדוכסוסטוס אבל שמא יודע היה שלא נתנו חכמים שיעור לדוכסוסטוס א״נ אם הי׳ אומר לו שיעור לקלף הי׳ חוזר ושואל בדוכסוסטוס אבל מבקלף לא ידע לו שיעור כ״ש בדוכסוסטוס. וכ׳ ר״ח ז״ל שם בשלהי פ״ק דב״ב אע״ג דגרסי׳ בירושל׳ (מגילה א,ט) ובקלפים לא נתנו חכמים שיעור וכו׳ ורב אמר כששאלו אותו שיעור ס״ת בכמה איני יודע דמשמע שכותבין ס״ת על הקלף ביון דלא אשכחן בתלמוד מי שיעשה כך אין מתירין אותו לכתחלה לעשות בקלף ומ״מ מצוה מן המובחר אינה אלא בגויל, ע״כ. והרמב״ם ז״ל אמר כתב על הדוכסוסטוס ס״ת פסול, ואיני יודע מה הוא דהא תניא בס״ת כולן כשרין ואע״ג דבמסקנא מזוזה כשרה בקלף ובדוכסוסטוס ברייתא בס״ת קיימא כדמעיקרא חדא במזוזה בגויל לא מפרשא להתירא ועוד דלא פלגי׳ אנן אאוקימתא דגמ׳ דמוקי׳ ס״ת כשרה בכולן בלא ראיה ואפי׳ אפשר לאוקים במסקנא במזוזה. ופי׳ לשון גויל, שלא נקלף ממנו כלום והדוכסוסטוס שהוא עור א׳ חלוק לשנים כמו שפירשו הגאונים ז״ל ואמרי׳ בב״ב גויל אבנא דלא משפרין ולא כאשר עלה על דעת רבים שנקרא כן מפני שהוא מעופץ בעפצים ונקראים בלשון לע״ז גל״א ולפי דברי קצת חכמי ספרד שלש עורות הללו שהם הגויל והקלף והדוכסוסטוס שלשתן בעפצים שנו. ובתוס׳ חכמי הצרפתים ז״ל ראיתי שאמרו שלא אמרו חכמים עפיץ דוקא אלא יש עבודין אחרים שהן מכשירין כעפצים הללו. וראי׳ לדבר מהא דגרסי׳ בפ׳ הקומץ (מנחות לא:) קרע הבא בתוך שנים יתפור בתוך ג׳ לא יתפור והא תניא לא יתפור ל״ק הא דעפיצין הא דלא עפיצין אלמא כותבין ס״ת אדלא עפיצן ואינו נקרא דפתרא שיש עבוד א׳ מכשיר, כגון שלנו. וזו אינה ראי׳, לפי שאין פי׳ דלא עפיצן אלא שאין עפיצן ניכר מחמת יושנן או מחמת דבר אחר כדאמרי׳ ה״מ בעתיקתא אבל בחדתא יתפור ולא עתיקת׳ עתיקתא ממש ולא חדתא חדתא ממש אלא הא דעפיצי והא דלא עפיצי, וכן פי׳ הרמב״ם ז״ל הספרדי. ומיהו אפשר שהוא אמת, לפי שלא הי׳ להן עיבוד של קיימ׳ אלא העפצים כדאמרי׳ בגטין והא בעינן כתב שאינו יכול להזדייף בדעפיצין משמע שכל שאר העבודין שלהן יכולין היו להזדייף ואפשר שמפני כך פסלום לס״ת דבעינן כתיבה תמה וליכא א״נ דלאו ספר מיקרי, אבל לדידן דאית לן עבודין אחרים שאינן יכולין להזדייף וכותבין בהן גטין ושטרות אפשר שיהיו כשרים אף לס״ת:
}
\textblock{\textbf{השתא דאתית להכי לרב נמי לא תימא דוכסוסטוס הרי הוא כקלף אלא אימא קלף הרי הוא כדוכסוסטוס.} לפי פשטה של שמועה רב לית ליה הא דתניא הלכה למשה מסיני מזוזה אדוכסוסטוס למצוה דהא איהו כר״מ ס״ל וכדרשב״א ואיהו הא אמר היה כותבה לכתחלה משמע וא״נ אף ע״ג הדוכסוסטוס קתני. אבל רבינו הגדול ז״ל כ׳ בהל׳ מזוזה (ה.) תניא הלכה למשה מסיני תפילין על הקלף מזוזה על הדוכסוסטוס קלף במקום בשר דוכסוסטוס במקום שער. והא דאמרי׳ מזוזה על הדוכסוסטוס למצוה אבל אם כתבה על הקלף כשרה. (וא״א) [ואפשר] דאע״ג דתניא רבי׳ מאיר היה כותבה לא לכתחלה אלא פעמים שלא נזדמן לו דוכסוסטוס וכותבה על הקלף וכדתניא בפ׳ הקומץ (מנחות לא:), רבי מאיר היה כותבה על הדוכסוסטוס כמין דף ועושה לה ריוח מלמטה וריוח מלמעלה ופרשיותי׳ פתוחות ואוקי׳ מצוה למעבדינהו סתומות ואי עבדינהו פתוחות ש״ד ואע״ג דלא דמי דהתם י״ל דמשום שאר דיני קתני כותבה ועוד דהכא תני טעמא מפני שמשתמרת ואפשר דה״ק אע״פ שמצותה על הדוכסוסטוס פעמים כותבה על הקלף כשירה כשצריכה שימור כגון שעומדת במקום התורפה או של רבים לפי שאינה נבדקת אלא פעמים ביובל וכ״ז כדי לקיים דברי רבינו ז״ל שהוא סובר דרב נמי אית לי׳ הלכה לכתחלה כדאמר קלף הרי הוא כדוכסוסטוס ומשמע דכולהו גמירי דוכסוסטוס למזוזה ואע״ג דרב נמי אמר כותבין לאו דוקא דהא מעיקרא כי אמרינן דרב אתפלין קאמר כותבין לאו דוקא לכתחלה אלא דיעבד השתא נמי ל״ש. ור״ח ז״ל כ׳ קלף ודוכסוסטוס לענין מזוזה שהיא נכתבת על זה ועל זה אבל תפלין אין נכתבין על דוכסוסטוס וכן הלכה ע״כ ומקצת מחברי ספרים שפסקו שאם שנה בזה ובזה פסולין [אלא לעולם תפלין] אקלף מזוז׳ אדוכסוס׳ אין לנו לסמוך עליהן דהלכה כרב דהוא אמורא ומעשה דר׳ [מאיר] (נמי) כמו שפסקו הראשונים ז״ל. ומזוזה בגויל אינה מתחוורת בשמועה, ומדמקשי׳ לעיל מזוזה אקלף מי כחבי׳ ולא אמרי׳ אקלף וגויל ועוד מדקתני ברייתא דמזוזה קלף ונייר ומטלית ולא קתני גויל ש״מ בגויל כשרה דההיא ברייתא כולהו פסולי [אתא] למיתני. ועוד טעמא דמשתמרת בגויל איתא לר״מ דאזיל בה בתר שמור הלכה לכאורה כשרה בגויל נמצאו ס״ת ומזוזה כשרין בכולן ותפילין בקלף לבד. ומצאתי בירושלמי (ח,ג) קלף כדי לכתוב עליו פרשה קטנה שבתפלין שמע ישראל, הדא דתימא בקולף פני העור אכסוסטין כדי לכתוב שתי פרשיות שבמזוזה ולשון פני העור משמע שקלף הוא הצד הדבוק לבשר וכותבין עליו במקום בשר ממש ודוכסוסטוס הוא הצד שע״ג בהמה שהשיער בו וכותבין עליו במקום שיער ממש וכן הוא עיקר אע״פ שנחלקו בו הראשונים:
}
\newsection{דף פ}
\textblock{\textbf{למימרא דשיעורא דר״י נפיש.} פי׳ פלוגתייהו דר״י ורבנן בשכיח טובא ושיעורו גדול ושכיח אבל לא כל כך ושיעור קטן דרבנן סברי בתר שכיח טפי אזלינן אע״ג דנפיש שיעורי׳. ור״י סבר כיון דאידך נמי שכיח אע״ג דלא שכיח כולי האי בתר שיעור׳ זוטא אזלינן לחומרא ובעלי הגמרא יודעין שכל אלו הולכין אחר זה הדרך ולפיכך הקשו מזה לזה ואם לא מפני כן אין אומרים בשיעורי שבת זה דומה לזה שכל אחד לפי חשיבותו שיעורו:
}
\newsection{דף פא}
\textblock{\textbf{תנו רבנן חפי פותחת טהורין קבען בפותחת טמאין.} פרש״י ז״ל דכלי עצם ככלי מתכות ופשוטיהן טמאין, בהכל שוחטין (חולין כה:). ואיכא למידק עליה, דהא תניא בפ״ק דמכילתי׳ (טז.) כלי עץ כלי עצם כלי עור וכלי זכוכית פשוטיהן טהורין ומקבליהן טמאין ומתני׳ היא בכלים וא״ת קשיא ההוא דחולין ההוא בגולמי כלי עצם מפורשת שם אבל פשוטיהן ודאי טהורין והך מתני׳ בפשוטי כלי מתכת היא ועליהן היא שנויה בתוספתא דמס׳ כלים ומ״ה טהורין קודם שקבען מפני שהן כגולמי כלי מתכת ושקבען נגמרה מלאכתן וטמאין משום כלי מתכת ואע״פ שהפותחת של עץ ואין דינה לקבל טומאה בפותחת הולכין אחר חפין כדתנן בפרק י״ג (דמס׳ כלים מ״ו), המשמש את המתכת טמא ומתכת המשמשת את העץ טהור כיצד פותחת של עץ וחפין שלה של מתכת טמאה פותחת של מתכת וחפין שלה של עץ טהורה שלא כדברי ר״ש שאמר דפותחת כמו שיש לה בית קבול דמי ומקבלת טומאה אלא משום חפין טמאה וכשהן בפ״ע לפי שלא נגמרה מלאכתן טהורין כדפרישית וקתני סיפא ושל גל אע״פ שחברן בדות וקבען במסמרים (כולו) [כלומר] שכבר נגמרה מלאכתן טהורין:
}
\textblock{\textbf{אסור למשמש בצרור בשבת.} פירש״י ז״ל משום השרת נימין, וקשיא עלי׳, והא דבר שאינו מתכוין הוא, ורבא כר״ש ס״ל ואפשר שכל המשמש כדרך שמשמשין בחול כדי לפתוח א״א בלא השרת נימין, ואינו נכון. וטוב מזה פר״ח ז״ל שאמר שאסור למשמש באבן כדי להחליקה דמחזי כמתקן מנא ומ״ה אקשינן לסתכן משום דגידודי דאבנא משתמטא להו לשיני כרכשתא:
}
\textblock{\textbf{ואם לאו כהכרע מדוכה קטנה של בשמים.} רש״י ז״ל הפסיק בה ופי׳ ואם לאו שאין מקום קבוע לבית הכסא מותר לו לטלטל כשעור המכריע, שהוא אגוז, דנפיש מכזית וזוטר מכביצה. וכמה הוא רחוק, חדא שאין לשון הכרע נופל ע״ז שלא נאמר כאן בדרך הכרעה ועוד שהי׳ לו לומר כדברי המכריע. [ועוד] וכי דרכן של בני אדם לקנח עצמן במדוכה קטנה של בשמים דקאמרת אם יש עליה עד מותר, ועוד בלא עד נמי לישתרי שהרי כל הכלים ניטלין לצורך גופן ומקומן. והגאונים ז״ל פירשו, ואם לאו מותר לטלטל אבן לקנוח שהוא כהכרע מדוכה קטנה של בשמים שהיא פחותה ממלא היד והדר אתמר מימרא באפי נפשה אמר רב ששת אם יש על האבן עד מותר לטלטל ולקנח בה ואפי׳ היא כמלא היד מפני שהיא כמזומנת לקנוח, ומן הירושלמי (ח,ו) נסתייעו דגרסינן בה מעשה בביתוס בן זונין שהי׳ יושב לפני רבו אמר כך אני אומר צריך שיהא יושב ושוקל ביד א״ל הכל מותר חוץ מן העשוי בחפיזה ר׳ הלל בשם ר׳ ינאי ויש בו כמלא רגל מדוכה קטנה של בשם א״ר יוסי לא אמר אלא מדוכה קטנה של בשם הא של בשמים לא:
}
\newsection{דף פב}
\textblock{הא דתנן \textbf{מי שהיה ביתו סמוך לע״ז והיה שלו ושל ע״ז.} פירש״י ז״ל לבית שע״ז בתוכו ונדון מחצה על מחצה שהמחצה שלו נחשבת בד׳ אמות של כניסה. ור״ח ז״ל כ׳ בפירושיו וקשיא מכדי בין לרבה בין לר״א משמשי ע״ז איתקש לשרץ כדכתיב שקץ תשקצנו, והתנן (ע״ז מז:) לענין בית ע״ז אבניו ועציו ועפרו מטמאין כשרץ ר״ע אומר כנדה ומפרקי׳ דהא בית גופי׳ כע״ז הוא כדתנן ג׳ בתים הן בית שבנאו מתחלה לשם ע״ז אסור ואמר רבי המשתחוה לבית אסרו וע״ז מטמא כנדה אבל משמשיה כשרץ אלו דברי ר״ח ז״ל ועיקר הן לפי גמרתינו. ובירושלמי (ט,א) גרסי׳, אית מתני׳ אמרה ע״ז כנדה ומשמשיה כנדה ואית מתני׳ אמרה ע״ז כנדה ומשמשיה כשרץ כל עצמה אינה קרוי נדה אלא במשמשין וטמאתם את צפוי פסילי כספך ואת אפדת מסכת זהבך תפתר בחקיקין על גופה ר׳ יעקב דכפר חנין אמר תפתר במשתחוה לאפוד עצמו כענין שנאמר ויעש אותו גדעון לאפוד מתני׳ כמ״ד ע״ז כנדה ומשמשיה כנדה דתנינן אבניו ועציו ועפריו מטמאין כשרץ ר״ע אומר כנדה תפתר במשתחוה לבית עצמו וזה מפורש כדברי ר״ח ז״ל לפי מאי דאמרי׳ בגגו׳ דילן דמשמשיה כשרץ:
}
\newchap{פרק \hebrewnumeral{9} אמר רבי עקיבא}
\textblock{}
\textblock{ופי׳ \textbf{ידין כמחצה על מחצה.} שמחצית אבניו ועציו ועפריו מותר אע״פ שעבדוהו לכותל לא אסרו דבר שאינו שלהם. [ועוד] שלא השתחוו אלא לשכנגדם ולא לצד האחר שהוא של ישראל, וכן פי׳ בירושלמי במס׳ ע״ז (ג,ו). ומוקמי לה כגון שבנה תחלה ואח״כ נסמכה לו ע״ז אבל אם בנו הבית לשם ע״ז א״נ השתחוו לו ואח״כ נסמך להם הכל אסור ומשמע דאסור נמי להסמך לו דהא מתהני מע״ז ומשמע נמי דדוקא במכיר עצים ואבנים שכנגדו. ואי קשיא, למ״ד (ע״ז כט:) ע״ז שנשתברה מאליה מותרת הכי נמי לשתרי כבר הקשו כן בירושלמי ופירשוה כגון שהשתחוו לכל אבן ואבן ואח״כ בניי׳ א״נ כשעתיד להחזירן (לכליין) [לבניין]. והא דבעי׳ בגמ׳ דילן לקמן (שבת פג.) ע״ז ישנה לאברים בששברה גוי ופרקה מחליותיה וקא מיבעי׳ לי׳ אי בטל הגוי הזה אם לאו אבל נשתברה מאליה ויכול להחזירה ודאי אסורה:
}
\textblock{\textbf{ומטמאה באבן מסמא.} פרש״י ז״ל, אבן ששמוה על היתדות וכלים תחתיה אע״פ שלא הכביד על הכלים טמאין כדכתיב כל אשר יהיה תחתיו (וההיא) לעליונו של זב אתא וכל אשר יהיה תחתיו הזב קאמר בפרק בנות כותיים במס׳ נדה ובת״כ מפיק לה מדכתיב והיושב על הכלי אשר ישב עליו [ובדבר הראוי למשכב ומושב] והיינו נמי דתנן כל שהזב נשא עליו טהור חוץ מן הראוי למשכב ומושב והאדם ואלו בקרא דכל אשר יהיה תחתיו לא כתיב משכב ומושב, [ועוד] ל״ל אבן אפי׳ כל דבר שהוא תלוי באויר נמי מטמא למה שתחתיו מדין זה. ופי׳ שם בתוס׳ שהיא אבן גדולה שומא על משכבות ומושבות והזב עליה ואינו מכביד כלל על המשכבות וזו היא דרך פירושו של רש״י ז״ל וכ״כ במס׳ נדה (סט:). וק׳ להו, דהכא בשמעתין משמע דאבן מוסמא מטמא משום משא כדתנן מנין לע״ז שמטמאה במשא ומוקמ׳ לה באבן מוסמא, ותנן נמי התם (ע״ז מז:) ר״ע אומר כנדה מה נדה מטמאה במשא דהיינו אבן מוסמא מדקרי לאבן מוסמא משא משמע דמשום משא היא מטמאה וכן לקמן באוקימתי׳ דרב אשי אליבא דרבה מוכח בהדיא דהיסט היינו אבן מוסמא והיינו פלוגתייהו דר״ע ורבנן לרבה ואלו לפי הפי׳ שכתבנו אין טומאה אלא מטעם דומה לטומאת משכב ומושב ואמרי׳ במס׳ נדה בפ׳ בתרא הזב והזבה והנדה והיולדת והמצורע שמתו מטמאה במשא עד שימות הבשר ומפרש בגמ׳ מאי משא אבן מוסמא. ועוד נמי אמרי׳ בפ׳ דם הנדה מה״מ דדם הנדה מטמא אמר חזקי׳ אמר קרא והדוה בנדתה מדוה כמותה אי מה הוא עושה משכב ומושב לטמא אדם ולטמא בגדים אף מדוה עושה משכב ומושב לטמא אדם ולטמא בגדים אמר קרא אשר היא יושבת עליו ולא דמה אי מה היא מיטמאה באבן מוסמא אף מדוה נמי מיטמא באבן מוסמא א״ר אשי אמר קרא והנושא אותם יכבס אותם מיעוטא הוא אלמא לאו משום משכב ומושב נתרבה אע״פ שנתמעט משכב ומושב מדם הנדה היה סבור לרבות בו אבן מוסמא. עוד הביא ר״נ ז״ל במגלת סתרים שלו שמצא ברייתא בתוספתא כלים (ב״ק ו.) אין טומאה בכלי חרם אלא מאוירו ובהיסט הזב ע״ג אבן מוסמא ואלו מדין משכב ומושב אין טומא׳ לכלי חרס כל עיקר כדאמרי׳ לקמן (שבת פג:). ובפר״ח ז״ל מצאתי שפי׳ שאם יש אבן על משכב הזב והזבה וישב אדם טהור על האבן אע״פ שלא נגע במשכב כי האבן הזאת חוצצת בינו ובין המשכב טמא אע״פ שהאבן אינה מקבלת טומאה וזה נכון לפ״ד ונפרש שטעם טומאת אבן מוסמא מפני שהמשכב והמושב שתחתיה שאף הן מטמאין באבן מוסמא וה״ה לזב עצמו נושאין את הטהור והוא כעין הסיטו של זב שהוא טמא שלא מצינו לו חבר בכל התורה וזה ששנינו אצבעו של זב תחת הנדבך הטהור מלמעלן מטמא שנים ופוסל אחד ואבן מוסמא דע״ז נמי כגון שהיא נושאה את האדם עליה או בכף מאזנים וכרעה היא וכן דם הנדה נמצא׳ טומאת אבן מוסמא משום משא ואינה תלויה בטומאת משכב ומושב. וצ״ע שאם מתורת היסט דוקא שלא מצינו לו חבר בכל התורה הרי מצורע מטמא באבן מוסמא בפ׳ בתרא דנדה, זה צ״ע. ונראה דאבן מוסמא כולל בין נושא בין נשא דהא תניא בספרא. ישב ע״ג עשרה מצעות אפי׳ ע״ג אבן מוסמא וכו׳ וזו שהיא כן ישנה בכל מטמאי משכב ומושב אבל זו שבשמועתינו שהיא מטעם היסט בטומאה למטה וטהור למעלה וכך שנינו בספרא והיושב על הכלי אשר ישב עליו הזב מנין לעשרה מושבות אפי׳ ע״ג אבן מוסמא תלמוד לומר והיושב מקום שהוא יושב ומטמא יושב הטהור ומיטמא:
}
\textblock{\textbf{ולוקשה רחמנא לנבילה.} ק״ל, אי אקשה לנבלה אין משא שלה מטמאה אלא אדם המסיט שכך שנינו גבי נבלה אבל עכשיו שהוקשה לנדה כל הראוי למשכב ומושב והאדם מיטמאין במשא שלה ושמא אין דין משכב ומושב בע״ז:
}
\textblock{\textbf{ולר״ע למאי הלכתא איתקיש לשרץ למשמשי׳.} פי׳ שטפא דגמרא הוא דהא איבעי לי׳ למימר דלא מיטמא באבן מוסמא אלא דאמר הכי לעיל לרבה אמר נמי הכא לר״א ואפשר דלרווחא דמילתא נקט לה דאפילו למסקנא דאמרי׳ דאיתקש למת ומת לא מטמא באבן מוסמא ומשום מיעוטא דאבן מוסמא ל״צ הקישא דשרץ ואצטריך משום משמשי׳ ואע״ג דעד השתא לא מדכרינן להקישא דמת ול״נ לר״ע לחומרא מקיש:
}
\newsection{דף פג}
\textblock{\textbf{מי אלימא ממתני׳.} פי׳ ה״ר משה בר׳ יוסף בחדושיו דמקשה דקרו ליה סבר רישא דומיא דסיפא מה משמשי׳ כשרץ לגמרי ולא מיטמי במשא אף ע״ז עצמה כשרץ ולא מיטמאה במשא ויפה פי׳ אף על פי שמצינו בכמה מקומות בתלמוד פרכות כגון זו ואין לנו בהם טעמי הרב ז״ל:
}
\textblock{\textbf{מתקיף לה רב אשי א״ה מאי הן.} פי׳ הר״ר יוסף הלוי בן מגאש בתשובה דהכא קא קשיא ליה כיון דנכרי ונכרית דינן בפני עצמן לטמא בכל ענין משמשי ע״ז אינן מטמאין במשא ולא באבן מוסמא נמצא דינה חלוק מן הכל ועל מה אמרו הן ולר״א הן דת״ק ניחא דקאי אע״ז ואמשמשי׳ ששניהן מטמאין במגע ולא במשא ולא באבן מוסמא אבל הן דר״ע קשי׳ דלא קפליג ודאי אמשמשין דאיהו נמי במגע ולא במשא ס״ל כשרץ אלא אע״ז בלחוד קאי ומאי הן וליכא למימר דאנכרי ואנכרית קאי נמי׳ דהא ל״ל לאדכורי נכרי ונכרית ועוד דהא לא שוו לאבן מוסמא. ואתא רב אשי ופריק אליבא דרבה דה״ק נכרי ונכרית וכו׳ ע״ז הן בני אדם שהסיטו אותן טמאין ולא אם נסטו ממנו ואין צריך לומר אבן מוסמא ר״ע אומר הן האחרים שהסיטו אותה ושנסטו ממנה טמאין וממילא אליבא דר״ע מתרצא ברייתא הכי ע״ז ומשמשיה הן ב״א הנוגעין בה טמאין שבמגע מטמאין כשרץ אבל לא הסיטן בין הם שהסיטו אחרים בין אחרים שהסיטו אותן ור״ע סבר ע״ז הן ב״א הנוגעין בה והמסיטין אותה טמאין אבל משמשיה היא ולא הסיטה זהו תורף פירושו של זה הרב ז״ל, ונכון הוא. אבל הוה זהיר במה שפי׳ הסיט ומשא הנזכרים בתחלה בשני דברים שאינו כן אלא הסיט שאדם טהור נושא הטומאה הוא נקרא משא בלשון תורה והסיט שהטמא נושא הטהור הוא הסיט סתם שבזב טמא ולא מצינו לו חבר ואבן מוסמא נתרבה מתורת הסיט זה וזה מתוקן בפרש״י ז״ל, ולא עלו בתשובת ה״ר יהוסף הלוי כהוגן. ויש ספרים שכ׳ בהן בריש הברייתא מגע נכרי ונכרית ואתיא פירכא דרב אשי בפשיטות דלא מתני לי׳ מגע והן והסיטן דמאי (מן) [הן] חוץ ממגע והסיט והו״ל למיתני מגע נכרי ונכרית והסיטן ותירוציה אתיא כרבה ולא מיתוקמא לי׳ ברייתא לר״א כלל אמגע דהא קתני לי׳ למגע בהדיא והן והסיטן בר ממגע קתני ליה והיינו דאוקי ר״א כרבה ולא כר׳ אלעזר כלל ופשוטה היא:
}
\textblock{ה״ג בכל הנוסחאות: \textbf{כדתנן הזב בכף מאזנים ואוכלין ומשקין בכף שני׳ כרע הזב טמאין כרעו הן טהורין.} ופרש״י ז״ל שלא מצינו נושא את הזב שיהא טמא אלא במשכב ומושב והאדם אבל כרע הזב טמאין שזהו משא הזב האמור בתורה מגעו שהוא ככולו וזהו הסיט והא דמייתי כדתנן לענין כף מאזנים אבל לא לענין הדין שכאן טהורין וכאן טמאין. והוי יודע שאין לשון משנה כן, אבל כך היא שנוי׳ הזב בכף מאזנים ומשכבות ומושבות בכף שני׳ כרע הזב טהורין פי׳ טהורין מדין מדרס שהיא הסיטו של זב ואינו מטמא משכב לעשותו אב הטומאה כרעו הן טמאין משום משכב שהוא תחתונו של זב וקתני סיפא הזב בכף מאזנים ואוכלין ומשקין בכף שניי׳ טמאין ובמת הכל טהור חוץ מן האדם ומשכחת לה בשכרע הזב ולומר דהיכא דברישא טהורין הכא באוכלין ומשקין טמאין. והאי דקתני ובמת הכל טהור חוץ מן האדם ה״ק הכל בין משכבות ומושבות ברישא בין אוכלין ומשקין בין שכרע הוא בין שכרעו הן הכל טהור חוץ מן האדם בזמן שכרע הוא שזהו משא שבתורה. (ואפי׳) [ואפשר] שבברייתא שנויי׳ בלשון הזה בגמ׳ ולפי שהענין הזה שנויי׳ במשנה קאמר דתנן וכן במקומות הרבה כאותה שהביאו בסוף ב״מ ותנן נמי גבי אילן כה״ג ואין המשנה כמו שהביאוהו שם אבל מקצתה בפ׳ המוכר את הספינה. ובפ׳ האומר במס׳ קדושין דתנן האומר לאשה הרי את מקודשת לי לאחר שאתגייר לאחר שתתגיירי ר״מ אומר מקודשת ר״י הסנדלר אומר אינה מקודשת ר״י הנשיא אומר וכו׳ ואלו במתני׳ תנן אינה מקודשת סתם ולא שנינו בה לא ר״מ ולא ר״י הסנדלר ולא ר״י הנשיא אלא מפני שהיא רמוזה במשנתינו ומאי דלא פריש במתני׳ פריש בבריית׳ מזכירה בשם משנה ועוד יש כמה משניות מסדר טהרות שהובאו בתלמוד בחסר ויתיר כההוא דתנן מראות נגעי׳ ב׳ שהן ד׳ בהרת עזה כשלג ולא כך היא שנוי׳ לגמרי ואחרת בפ׳ העור והרוטב:
}
\textblock{\textbf{ישנה לאברים או אינה לאברים.} נ״ל דלענין טומא׳ קא מבעי׳ לי׳ וכשנשתברה מאיליה ודאי כה״ג אסורה דקיי׳ל (ע״ז כט:) ע״ז שנשתברה מאיליה אסורה אלא הכי קמבעי׳ לי׳ כיון דטומאת ע״ז דרבנן לא אחמור רבנן בשבורה א״ד כיון דיכול להחזירה כמאן דמחברא דמיא ומתני׳ דתנן מטמאין כשרץ בשעבדן לאבן ואבן בפ״ע כדאמרן וזה הפי׳ למד מעניינו שלענין טומאה נשאלה הלכה זו ולא לענין ע״ז ר״ל לענין איסור, והפי׳ הראשון דברי בעלי התוס׳ ז״ל:
}
\textblock{\textbf{זה זבוב בעל עקרון.} תימה הוא למה הביאו כאן לזאת הברייתא אי למימר דזבוב ע״ז היא פשיטא דהא קרא כתיב ואת זבוב אלקי עקרון וא״ל אי מקרא ה״א גוף גדול אלא שנקרא כן אבל השתא דכתיב וישימו משמע שמשימין אותה בתוך חיקם אלמא גוף קטן הוא ואין זה נכון אפשר שהיא גדולה מכזית. ורש״י ז״ל פי׳ שזבוב הוא כמשמעו אבל לכך צריך לברייתא לומר שהוא ע״ז גמורה ואדוקין בה מלשון ברית שהוא לשון אהבה וחבה כלומר שלא תאמר בעקרון הוא שעשו כך לזכר ע״ז בעלמא אבל אינה ע״ז גמורה ואינן אדוקין בה. ובירושלמי (ט,א) מצאתי, אבל בע״ז שלימה אפי׳ כל שהוא דא״ר יוסי בר בון רב חמא בר גוריא בשם רב הבעל ראש הגויה הי׳ וכאפין היה וישימו להם בעל ברית לאלקים פי׳ ברית ראש הגויה שבו הברית והוא נמי לשון בעל ונראה שהגירסא וכאפון היה כלומר שהיה קטן כגרעין של אפונים ולפ״ז לא מייתי ראיה משום לשון זבוב אלא מלשון בעל ברית ולומר שהיו אדוקין בצורת ראש הגויה שנקרא בעל והא דקתני זה זבוב בעל עקרין מפני שנקרא בעל אמר כן או שהוא דבר קטן כזבוב:
}
\textblock{הא דאמרי׳ \textbf{טומאת ע״ז דרבנן ולקולא ולחומרא לקול׳ מקשינן.} לרבנן אליבא דר״א, אבל לרבה אליבא דר״ע איפכא ס״ל דכיון דבעי׳ לארחוקי ע״ז לחומרא מקשי׳ ועוד דקראי לחומרא כתיבי וטמאתם וגו׳ תזרם כמו דוה ושקץ תשקצנו הילכך לכולהו מחמרי׳ דתטמא אפי׳ באבן מוסמא ומשמשין דלא כתיבי בהו טומאה כשרץ ומיהו לשיעור דשרץ בכעדשה לא מקשי׳ למת דלא מחמרי׳ בטומא׳ דרבנן מכלל מאי דאחמיר רחמנ׳ בדאורייתא ולא אשכחן אבן מוסמא ואוהל בדאוריי׳ בחד דוכת׳ ולרבנן כיון דאבן מוסמא דנדה לא מצינו לו חבר בכל התור׳ לא מחמרי׳ בה אבל במשא מקשי׳ לחומרא וה״ט דר״ע לר״א:
}
\textblock{מתני׳: \textbf{ספינה שהיא טהורה.} להכי נקט בספינ׳ לאשמעי׳ דטלטול שאדם מטלטל בים לאו טלטול הוא דע״י מים לאו טלטול אדם הוא ופירשו בתוס׳ שאינה מיוחדת לישיבה דאלו מיוחדת לכך אפי׳ גדולה טמאה דלא מקשי׳ לשק בהנך דחזי למדרסאות כדאמרי׳ בפ׳ על אלו מומין וכן הא דתנן הבאה במדה טהורה מכלום אפתחה למעלה קאי אבל פתחה מצדה כיון דחזיא למדרס טמאה ואע״פ שהספינה עשוי׳ לישיבה עיקרה לפרקמטיא היא ואומר ליושב עמוד ונעשה מלאכתנו. ואינו מחוור לי כלל, דספינה עשויה ומיוחדת לישיבה עם מלאכתה הוא שפתחה מצדה דעיקרה לכלים ומשמשת ישיבה עמה והכי תני לה בת״כ אין לי אלא כסא ספסל וקתדרא המיוחדין לישיבה מנין תיבת הבלנין ותיבה שפתחה מצדה וכו׳ עד מרבה אני את אלו שהם משמשין ישיבה עם מלאכתן ומוציא אני את אלו שאינן משמשין ישיבה עם מלאכתן יי״ל ובספינה על כסא וספסל יושבין בה ועל מטה שוכבין בה וכל עצמה מעשה קרקע בעלמא ואינ׳ משמשת לישיבה כנ״ל לפי שיטה זו שלהם. ושמא אין שיטתן עיקר אלא כל הבאין במדה טהורין אע״פ שהן ראויין למדרס דגרעי מפשוטי כלי עץ ושמא איכא רבויי בקרא לפשוטי כלי עץ דחזו למדרסאות והבאין במדה נתמעטו (מכלים) [מכלום]:
}
\newsection{דף פד}
\textblock{\textbf{וא״ר יוחנן אם יש בה בית קבול רמונים טמאה.} פי׳ משום דטלטול ע״י שורים שמיה טלטול, ואי לא מדר״י, ה״א במטה טמאה מפני שעשוי׳ לטלטל ע״י אדם ע״י הדחק ושל אבנים טהורה מפני שאינה עשוי׳ לטלטל אלא ע״י שורים מ״ה מייתי הא דר״י ומיהו לאו מינה מייתי ראיה שעדיין י״ל דר״י נמי בקטנה שעשויה לטלטל ע״י אדם מוקי לה למתני׳ ומ״ה מייתי סיפא דקתני בשלש תיבות הבאה במדה טהורה. מכלום פי׳ לפי שאין תיבה גדולה המחזקת כורים ביבש עשויה לטלטל ע״י שורים כמו העגלות מכלל דרישא אפי׳ באה במדה טמאה מפני שעשוי׳ לטלטל ע״י שורים דאי לא ליתני ד׳ עגלות הן ואייתי הא דר״י כי היכי דלא (תיקשי ליה) כדאמרן ומיהו הו״ל לאתויי מציעתא דקתני ג׳ עריבות הן וכו׳ וקתני הבאה במדה טהורה מכלום אלא שאין הגמרא מקפדת בהבאת המשניות על סדרן לגמרי ועוד משום ששתים אלו שוות לגמרי אבל בעגלות קתני משני לוג ועד תשע׳ קבין שנסדקה וכו׳ (בפ״ב) [בפ״כ] דכלים (מ״ב):
}
\textblock{\textbf{וספינה של חרס טמאה כחנניא.} פי׳ טמאה מגע רבי יוסי אומר אף הספינה טהורה ממגע מתקיף לה רב פפא א״ה מאי אף דמשמע כי היכי דשאר כלי חרס טהור אף ספינה טהורה והלא אין כלי חרס טהור ממגע אלא אר״פ ה״ק ושל עץ בין מדרסו בין מגעו טמא וספינת הירדן טהור׳ מדין מגע כתנא דדין ר׳ יוסי אומר אף הספינה טמאה מגע, זהו תורף פרש״י ז״ל. ויש מקשין, ל״ל לר״פ לאוסופי ביה ושל עץ בין מדרסו בין מגעו לימא הכי מדרס כלי חרס טהור ומגעו טמא וספינה של חרס טהורה פי׳ אף ממגע ר׳ יוסי אומר אף הספינה טמאה מגע כשאר כני חרם ואין זו קושי׳ שמאחר שהברייתא משובשת וע״כ אתה מוסיף בה ומגעו טמא ודין [דטהורה] ספינה לת״ק יש להוסיף בה כמו שאתה רוצה כדי שיתפרש בה הדין לגמרי וכבר כתבתי לך גדולה מזו בפ׳ דלעיל ויש בו בזו טעם ורש״י ז״ל משום דשכיחא ניה ספינה של עץ משל חרס והואיל ובשל עץ בעי לאוקמי צריך היה להוסיף ושל עץ דאי לא היכי ליתני דין ספינה של עץ והא לא אידכר שם כלי עץ. ובתוס׳ מפורש בשם ה״ר שמואל ז״ל, דהא דקאמר ושל עץ בין מדרסו בין מגעו טמא לא להוסיף על הברייתא אנא דיוקא דברייתא הוא ודייק הא של עץ בין מדרסו בין מגעו טמא מדלא פליג נמי בכלי עץ. וספינת הירדן טהורה בין ממגע בין ממדרס כתנא דידן ר׳ יוסי אומר אף הספינה כמגע כלי חרס וטמאה כחנניא. ובפר״ח ז״ל גרסי׳, א״ר זביד מדרס כלי חרם טהור ומגעו טמא וספינה של חרס טהורה כתנא דידן רי״א אף הספינה טמאה כחנניא מתקיף לה ר״פ א״ה מה אף אלא אר״פ ה״ק מדרס כלי חרס טהורה ומגעו טמא ושל עץ בין מדרסו בין מגעו טמא וספינה של חרס טמאה כחנניא רי״א אף הספינה טהורה ופי׳ ר״ח ז״ל לזו הגירסא דמאי דאמרי׳ מגעו טמא היינו בדיוק כלומר הא מגעו טמא מ״ה אקשי ר״פ מאי אף כלומר אמאי קאי אף הא מגע ליתא בברייתא אלא דיוקא הוא דקאמרינן ופריק דאף הספינה טהורה מדין מגע כמו מדרס שטהר ת״ק. ואין פי׳ זה נכון, שהרי יכול היה להוסיף בעיקר הבריית׳ ומגעו טמא שהוא כמשמעה כמו שהוסיף וספינה של חרם טהורה או טמאה ונימא דאף קאי עלי׳ ממה שנימא דקאי אדיוקא דלישנא דת״ק:
}
\textblock{\textbf{מה הוא אית ליה טהרה במקוה.} פי׳ לאו דוקא במקוה דהא זב מים חיים בעי אלא סירכא נקיט ומשכבו לאו לביאת מים חיים הוקש דלא בעי אלא טבילת מקוה כדתניא בתו״כ, אלא לטבילה הוקש ומש״ה נקט מקוה בהוא עצמו. וי״ל יש לו טהרה במקוה בשאר טומאות קאמר ומיהו משום הקישא דמשכב קאמר הכי כדפרישית:
}
\textblock{\textbf{אף משכבו דאית ליה טהרה במקוה.} אי קשיא הרי פת שאור שייחדה לישיבה מטמא מדרס א״ל הואיל ואיכ׳ במינו דהיינו גידולי קרקע כדמתרץ ואזיל ומיהו מעיקרא הו״ל לאותובי מהא אלא חדא מתרתי פירכי נקט. וי״א דאפי׳ למ״ד גבי מפץ דרחמנא ליצלן מהאי דעתא שאני [מפץ] דנמי גדולי מים הן ואפ״ה בסוף מרבי׳ כל דדמי ליה כלל [ואפי׳ מפץ] מדכתיב וכל המשכב, ואין טעם זה נכון:
}
\textblock{\textbf{ומה פכין קטנים שטהורים בזב.} פירש״י ז״ל פכין קטנים של חרס שטהורין מכל טומאת הזב דלמדרס אין כלי חרס נעשה מדרס ובמגע אינן יכולין לטמא שאינן מטמאין אלא מאוירן ואינו יכול להכניס אפי׳ אצבעו קטנה בהן ואי משום היסט כל שבא לכלל מגע בא לכלל משא מדאפקי׳ רחמנא בלשון נגיעה כענין שאמרו בפ׳ העור והרוטב (חולין קכד:) גבי משא נבלה ומגעה, ואי משום שערה שיכול ליכנס באוירן שהוא מטמא כל שאינו ראוי לנגיעת בשר אינו מטמא בנגיעת שער דמדכתיב והנוגע בבשר הזב מפקי׳ ליה בספרא. ויש לדקדק אחר טעמו, ליטמו במשקין דזב, דלא אפקינהו רחמנא בלשון נגיעה דכתיב וכי ירוק הזב בטהור ועוד מקשה ר״ת ז״ל הרי צמיד פתיל מטמא במשא הזב אע״פ שאינו במגע כדאמרי׳ במס׳ גטין שמא תסיטם אשתו נדה ובמס׳ נדה נמי אמרי׳ מוקף צמיד פתיל שטמא במעת לעת שבנדה וה״נ מוכח בחולין ומתרץ ש״ה שעומד לפתוח וכפתוח דמי וק״ל הא אמרי׳ בחולין גבי קולית המת מחוסר נקיבה כמחוסר מעשה דמי. ופריק ש״ה שאינו נקוב אלא שחשב עכשיו לנקבו ואין מחשבתו בלבד עושה שאינו נקוב כנקוב אבל צמיד פתיל נקוב הוא וכפתוח דמי. וכן היה ר״ת ז״ל מפרש מ״ש במס׳ ביצה (לב.) אלפסין עירניות טמאות במשא הזב וטהורות באוהל המת שהן חקוקית וסתומות בכיסוין עד שיצרפם בכבשן ויפתחם וגבי משא טמאות משום דכפתוחות דמיין דומיא דמוקף צמיד פתיל וכן משמע בתוספתא דביצה דקתני אין פותחין אלפסין סתומין ורשב״ג מתיר אלמא אלפסין עירניות הללו סתומות בכיסוין הן יש להם בית קיבול אלא שנסתם בכיסוין. אבל רש״י ז״ל פי׳ שאין להם תוך ואף על פי כן במשא הזב טמאות דלא בעי תוך ואינו מחוור שהרי פכין קטנים טהורים במשא מפני שאין באין לכלל מגע תוך כ״ש אלפסין עירניות שאין להם תוך כלל. ועוד ששנינו הטהורין שבכלי חרם טבלה שאין לה לבזבז ומחתה פרוצה כו׳ והמטה והכסא וכו׳ מתני׳ בפ״ג דכלים משמע שאין לאלו טומאה כלל. ועוד דהשיב ר״ת ז״ל דתניא בת״כ יכול יטמאנה מאחוריו נאמר כאן בו ונאמר להלן בו מה להלן מאוירו אף כאן מאוירו כמ״ש רש״י ז״ל למעלה וכל שאינו בא לכלל מגע אינו בא לכלל היסט:
}
\textblock{\textbf{מפץ שטמא בזב א״ד שיהא טמא במת.} הקשה ר״ת ז״ל, צמיד פתיל יוכיח שטמא בזב ואינ׳ מטמא במת ופריק נייתי לה בתורת טעמא צמיד פתיל שטמא בזב מפני שהוא מסיטו ומגעו ככולו ואפילו באוירו ואלו במת מאוירו טמא ולא נטהר אלא מפני שהוא טומאת אחורים שטהורה אף בזב. והוא השיב על דבריו, והא אמרי׳ במס׳ נדה פ״ק (ה:), ולאו ק״ו הוא, ומה מוקף צמיד פתיל שטהור במת אינו ניצול מעת לעת שבנדה משכבות ומושבות שטמאין במת א״ד שלא יהיו ניצולין במעל״ע שבנדה. ומאי ק״ו שאני מעל״ע שבנדה שהוא טומאת תוך כדאמרן ומשני התם ה״ק ומה צמיד פתיל שמצינו לו צד טהרה באהל המת עשו לו מעל״ע שבנדה מדבריהם כנדה עצמה משכבות ומושבות שלא מצינו בהן צד טהרה א״ד שנעשה בהן מעל״ע שבנדה כנדה עצמה וה״ט נמי דלא אמר התם פכין קטנים יוכיחו שטמאין במת וטהורין במעל״ע שבצדה לפי שהן טהורין אף בנדה עצמה ואין אותו ק״ו אלא ליתן טעם וראי׳ שעשו חכמים מע״ל שבנדה כנדה עצמה. ומיהו לא נהירא לי, דהא מ״מ קולא היא שטמאין בזב בהיסט ואין טמאין במת בהיסט, שהסיטו של זב לא מצינו לו חבר בכל התורה כולו. וי״א דהכא ה״פ ומה פכין קטנים שטהורין בזב לכל טומאותיו בין במשא בין בהיסט בין במגע הן טמאים במת לכל טומאותיו מפץ שטמ׳ בזב לכל טומאותיו א״ד שיה׳ טמא במת בכל טומאותיו וא״ל צמיד פתיל יוכיח שהרי אינו טמא אלא במשא הזב ולא במגע. גם זה אינו נכון שאם טהור במת אף הוא טהור במגע דזב חוץ מן המדרס אבל אמת הוא דלהכי נקיט פכין קטנים מפני שטהורין בכל טומאת הזב דאי לא לימא פכין גדולים שטהורין מן המדרס לפי שאין כלי חרס נעשה מדרס כדאמרן אלא דרויחא לי׳ נקיט:
}
\textblock{\textbf{ואמאי והא לית לי׳ טהרה במקוה.} פירש רש״י ז״ל דכי כתיבא טבילה בפ׳ שרצים ופ׳ מדין אכלים דכתיבא התם אבל פשוטי כלי עץ שלא נכתבו שם לית להו טבילה אלא הרי הן כאוכלין ומשקין שאין להם טהרה במקוה לפי שלא נכתבו בהן בתורה. ויש מקשים מהא דגרסי׳ במס׳ סוכה (טז.) מטה מטמאה חבלה ומטהרת חבלה מטמאה איברים ומטהרת איברים ומפרש לה בגמ׳ בארוכה ושתי כרעים בקצרה ושתי כרעים והיינו פשוטי כלי עץ אלמא יש להן טהרה במקוה ומפרשי לה במפץ של שיפה ושל גמי אבל של עץ יש לו טהר׳ במקוה שהרי נתפרש׳ טבילה במקבליהן וכיון שכן אף פשוטיהן במקום שמקבלין טומאה יש להן טהרה. וי״א הרי כתיב וכל כלי עץ תתחטאו וסמיך לי׳ בפ׳ שניי׳ את הבדיל ואת העופרת כל דבר אשר יבא באש תעבירו באש וטהר וכל אשר לא יבא באש דהיינו של עץ תעבירו במים מקיש עץ למתכות מה הן בפשוטין אף כל כלי עץ בפשוטין. ועוד דהתם לא כתיב שק דמשמע שיש בו בית קבול אלא מעשה עזים אף על פי שאין לו קבול כגון הנך דלעיל אריג כל שהוא תתחטאו. ואיכא למידק, דאמרי׳ עלה דהא בריי׳ במס׳ ב״ק בפ׳ כיצד וקמייתי לה בין לטומאת ז׳ בין לטומאת ערב ואם אין לו טהרה במקוה מה טומאת ז׳ טומאת ערב יש כאן הרי טמא הוא לעולם וא״ת מאי טומאת ז׳ שהוא נעשה אב הטומאה כדין הטמאין טומאת ז׳ והרי כל שאין לו טהרה במקוה אין נעשה אב הטומאה ומפרקי דברייתא דהכא בין במפץ של עץ בין במפץ של שיפה ושל גמי היא שנוי׳ והתם פריך משל עץ שיש לו טהרה במקו׳ כדאמרן והכא פריך משל שיפה ושל גמי. והגאונים ז״ל פי׳, פכין קטנים של עץ וטהורין במדרס הזב ואפי׳ ייחדן לישיבה נמי טהורין שאינם ראוים לכך ואפ״ה טמאין במת בכל טומאותיו מפץ שטמא למדרס הזבה א״ד שיטמ׳ באוהל המת וה״ה לשרץ וכן לפי׳ הראשון מת לאו דוקא:
}
\textblock{\textbf{מי לא עסקי׳ דייחדינהו לאשתו נדה וקאמר רחמנא טהור.} פירש רש״י ז״ל, דאי טמא הוא מדרס היכי מציל והא כל דבר טמא אינו חוצץ בפני הטומאה הילכך כ״ח המוקף צמיד פתיל טהור מכל טומאה דבהיסט נמי לא מיטמי כדפרי׳ דכל שאינו בא לכלל מגע לא בא לכלל משא ולפיכך הוא מציל על מה שבתוכו. ויש תשובה על דברי רש״י ז״ל הללו ממ״ש למעלה דמוקף צמיד פתיל טמא אפי׳ במע״ל שבנדה ואמרי׳ [גיטין סא:] שמא תיסטם אשתו נדה. וי״א לעולם הכלי עצמו טהור, שאלמלא כן אינו חוצץ אבל מה שבתוכו טמא בהיסט שהרי ניסט ואינו נכון בטעם לפי שאם בא לכלל מגע אי קרית לי׳ מפני שעתיד ליפתח שניהן טמאין ואם לאו שניהן אינן באין לכלל היסט. ועוד דההוא דאלפסין עירניות קשי׳ לי׳ כפי מה שפירש רבינו תם ז״ל דמדקתני טמאות וטהורות משמע טומאות עצמן וקתני סיפא ר״א בר צדוק אומר אף טהורות במשא הזב לפי שלא נגמרה מלאכתן משמע שעל עצמן של אלפסין הדברים אמורים. אלא ה״פ: מי לא עסקי׳ דייחדינהו לאשתו נדה שאם אתה אומר הן טמאין במדרס הזב א״כ מצינו להן טומאה מאחוריהן ואיך חוצצין בפני הטומאה אבל טומאת היסט שאני שהיא טומאת תוך דמגעו שהוא ככולו הוא לפיכך אחורים חוצצין שלא מצינו להם טומאה בכל התורה כולה. (שמועה זו מפורשת בתוספות לרבינו הצרפתים ז״ל, וכתבתי׳ והוספתי בה דברים להגדיל תורה):
}
\textblock{הא דתנן \textbf{שזורעין בתוכה ה׳ זרעונין.} ה׳ [מיני] גרעינין הוא, וכ״פ רבינו תם ז״ל. [וקס״ד] (ו)במחריב את הקרנות וזורעין באמצע כל רוח, ושיעור יניקות הזרעים ג׳ טפחים הוא כדתנן (ב״ב יז.) מרחיקין את הזרעים מן הכותל ג״ט ואף על פי שהיניקות מתערבות כיון שאין מין זה יונק מגופא של מין זה מותר כדאמרי׳ לא ינקו מהדדי ולא אמרי׳ לא ינקי בהדי הדדי. והראשונים אמרו שיניקת הירקות הנזרעין בערוגה אינה אלא טפח ומחצה ואם תאמר מכל מקום למה אינן תשעה ד׳ בד׳ קרנות הערוגה ובאמצע כל שורה ד׳ ואחת בתוך זו אינה קושיא למאן אי לרב דאמר ערוגה בחורבה שנינו הרי ממלא הוא שתי הקרנות גרעינין כגון זה. ולזרע תשעה ולא ימלא הקרנות לא אמרינן משו׳ האי טעמא שמא ימלא הקרנות ממין אחר ויהי׳ סמוכין יותר מדינן אי לשמואל דאמר ערוגה בין הערוגות שנינו לדידי׳ נמי אפילו בממלא את הקרנות היא כדקא מקשינן עליה והא קא מערבן אהדדי כלומר דבין בממלא את הקרנות בין שאינו ממלא מתערבין הן כיון דהוא מקפת ערוגות ומהדר בנוטה שורה לכאן ומותר אפילו בממלא והיינו דמקשי׳ מעיקרא והאיכא מקום קרנות כלומר דערוגה בין הערוגות שנינו ומש״ה מחריב קרנות של ערוגה זו כדי שיזרע סמוך להן בערוגה אחת ויהי׳ בין גרעינים של זו ולאחרת ג״ט ומפרקי׳ לה בממלא את הקרנות וגזירה משום ממלא ולעולם הוא מרבה בזריעה ה׳ זרעונים הללו כמו שיכול ואם בא ליטע בה ט׳ זרעונין אסור שמא יאריך בזריעת הקרנות ולא יצמצם להיות ביניהם הפרש ויש לפרש דקים להו לרבנן דחמשה בשיתא לא ינקי אבל תשעה ינקי שאין שלשה ביניהם די להן להניקן בערוגה אחת לד״ה:
}
\newsection{דף פה}
\textblock{\textbf{גבוליה כמה.} פי׳ כמה הוא חשוב גבול להפרישה משאר השדה ולקרות זאת ערוגה בפ״ע כדתנן לענין כלאים כמלא רוחב פרסה וכיון דלענין כלאים הוי הפרש למראית העין ולהוציא מדין ערבוב לענין גבול נמי הפרש הוא וטעמי׳ דר״י נמי מדכתיב והשקית ברגליך כגן הירק דכיון דחשוב והוי גבול אלמא הפרש הוי, ומגיהי ספרים מיעברי קולמס א״מאי טעמא דר״י״ ללא צורך:
}
\textblock{הא דגזרי׳ \textbf{שמא ימלא את הקרנות.} ולא גזרינן בזריעה דאידך גיסא משום דדרך זריעה כך הוא לזרוע בערוגה שורה אחת כולה ושורה אחת כנגדה אפי׳ בגרעין א׳ די להם וליכא למיגזר א״נ לפי שאדם נזהר בערוגה א׳ ואינו נזהר מערוגה לערוגה שאין אדם זורע ערוגה א׳ לחצאין ונזהר הוא בגרעינין הנזרעין בה להרחיק זה מזה כדינן אבל בשתי ערוגות פעמים שהוא זורע ערוגה אחת היום וממלא שתי קרנות שבה ולמחר כשהוא בא לזרוע ערוגה אחרת אינו משתמר להרחיק גרעונה של ערוגה זו מאותה ערוגה אחרת, וכן עיקר. ובירושלמי (כלאים ט:) ניתני תשעה א״ר תנחום בן צדיא אלא בערוגה שבערוגות היא מתניתין, וזהו מה שפירשנו ולא אתיא דגמרא דילן אלא כדקא ס״ד מעיקרא. ויש בפי׳ שמועה דברים לר׳ יהוסף הלוי בן מגא״ש ז״ל ולמקצת הגאונים הראשונים ולפמ״ש השמועה פשוטה לפניך. והוי יודע שאין יניקת הזרעונין ג״ט או טפח ומחצה לכל א׳ אלא עם מקומו שהוא זרועה בו שהרו אין תוכה של ערוגה אלא ששה ואין בין זרע אמצעי לשבצדדיו אלא פחות משלשה טפחים א״ו ש״מ מקום הזרעין עצמן מן המנין. וי״א טפחים הללו שוחקות הן וכי מדלית נמי מקום הזרעין איכא שלשה טפחים מצומצמים. ובירושל׳ (כלאים ג.) מצאתי ר׳ יוחנן בשם ר׳ ינאי כולהון בתוך ששה פירוש ששה אף במקומן של זרעים כדפרישית. כהנה בשם ר׳ שמעון בן לקיש כולהון חוץ לששה אי כולהון חוץ לששה נהוי תשעה א״ר תנחום בן צדיא כיני ערוגה שבערוגות היא מתני׳ פירוש אא״ב מקום הזרעים נמי בתוך אותן ששה טפחים של ערוגה א״א לזרוע בה אלא חמש לפי שיניקתן ג״ט וצריך לזרוע ד׳ גרעינן בד׳ קרנות וא׳ באמצע כדי שיהא ביניהן ג״ט חוץ ממקום הזרעים שהרי זרועין באלכסון דאיכא חומש דאלכסונא שבהן הזרעים זרועין אלא אי אמרת חוץ לששה הן זרועים א״כ ארבע באמצע שורות הערוגה הוא זורען אם רצה ולמה לא יזרע אחרים בקרנותיה ויהיו תשעה ופריק בערוגה שבערוגות הוא כדאמרן. ולפ״ד הירושלמי נפרש לגמרא דילן חוץ ממקום זרעים ומ״מ נלמוד שאין הערוגה מצלת מדין ערבוב שא״כ ל״ל קים להו דחמשא בשיתא לא ינקי אפי׳ ביונקים מותר כדאמרי׳ לגבי ראש תור ירק הנכנס ובממלא כל גנתו דר׳ יוחנן ופריק תני סופה דהא מתני בכלאים אלא ערוגה ערבוב יש בה. מיהו כיון שאין בה יניקה זה מזה מותר שאיסור כלאים ביניקה ומראית העין היא תלוי׳:
}
\textblock{\textbf{הרוצה למלאות כל גינתו ירק עושה ערוגה ששה ועוגל בה חמשה.} פירש רש״י ז״ל עוגל בתוכה ה׳ עגולין לה׳ זרעונים ואין משמע הלשון כן. ועוד אפי׳ יותר יעגל בה. ובתוס׳ בשם רבינו תם ז״ל עוגל בתוכה עגול א׳ של חמשה וזורע אותו מין א׳ והוא השנוי במשנתינו וא׳ באמצע וזורע ד׳ על ד׳ רוחות הערוגה בשאר הערוגות ממה שירצה והם המורשות שהמרובע יתר על העגול רואין כאלו ערוגה עגולה ששה על ששה ומורשות הקרנות שחוץ לעיגול נזרע ממה שירצה ונמצא חצי טפח בערוגה סביב העיגול הפנימי. ואקשי׳ והיאך נתמלאת כל גינתו ירק והרי יש חצי מפח רחב לכל צד ואמרי דבי ר׳ ינאי ודאי צריך להחריב אותו בין הבינים ור׳ יוחנן בההוא פורתא לא איירי ורב אשי אמר דאפי׳ ההוא חצי טפח ממלא ואם היו זרעוני הערוגה שתי זורע אלו ערב ומותר:
}
\newsection{דף פו}
\textblock{\textbf{מאן דלא מוקים כתנאי תני רישא טהורה.} פירוש, איכא דתני לה הכי ומאן דלא מוקי כתנאי אמר דהכי דוקא ומאן דמוקי כתנאי לא מתני לישנא דמרגלא אפומייהו ומוקים לה כתנאי והוצרכנו לפי׳ הזה שלא מצינו כתנאי שיחליף לשון המשנה וישנה אותה כרצונו:
}
\textblock{\textbf{אלא ר״ע כמאן.} פירוש הול״ל שש עונות אי ס״ל דלא אמרי׳ מקצת היום ככולו פעמים שהן ד׳ פעמים שש כר׳ ישמעאל ובקרא ימים כתיבי אי מקצת אי כולן [ומשני בהשכמה עלה, והילכך] בעי׳, חמש עונות דכיון דבהשכמה הוה ש״מ לאו אימים קפיד רחמנא אלא אעונות:
}
\textblock{ה״ג בכל הספרים: \textbf{אבל חכ״א ג׳ עונות (קרא) שלימות בעי׳.} ופי׳ עונה שלימה, יום ולילה, ולא דבעי׳ לילו עמו אלא מע״ל ג׳ ימים. וגמרא גמר לה שמואל, וכן מצאתי בירושלמי (ט,ג) א״ר יוחנן זו דברי ראב״ע ורבי ישמעאל ור״ע אבל דברי חכמים עד ג׳ ימים מכאן ואילך היא נסרחת אתי׳ כי דמר ר׳ זעירא בשם ר׳ יוחנן זאת תורת הזב ואשר תצא ממנו שכבת זרע מה תורת הזב עד ג׳ ימים אף ש״ז עד ג׳ ימים ולא כדברי מי שפירש מאן חכמים רבי אלעזר בן עזריה ורש״י ז״ל גורס שש עונות שלימות בעינן ונכון הי׳ אלו הודו לו הספרים:
}
\textblock{\textbf{אבל פירשה מן האיש טמאה כ״ז שהיא לחה.} הקשו בתוס׳, והתנן לה במס׳ נדה בפ׳ דם הנדה וש״ז מטמאין לחין ואינן מטמאין יבשין, ומתרצי בשם ר״ת ז״ל דהכ׳ בזרע שנרבע ופלט דומיא דאשה וזה אינו כלום שהי׳ לו לדקדק אי חביל דלית בי׳ פרזדור. והקושיא אין בה ממש דהתם קמ״ל דאין מטמאין יבשין אבג לחים ה״א בתוך עונות שלה מטמא ולא חוץ לזמנה:
}
\newsection{דף פז}
\textblock{\textbf{הוסיף יום א׳ מדעתו מאי דרש כו׳.} זה ק״ל, אי מדרש דרש לאו מדעתו הוה ולא הסכים על ידו הוה והרבה כיוצא בו עשה משה ושאר נביאים וי״ל ודאי אם רצה הקב״ה הי׳ אומר לו היו נכונים ליום הרביעי כדלקמן אלא הוא ודאי לשלישי אמר אלא גלוי הי׳ לפניו דעתו של משה ולפיכך משאמר לו ליום השלישי חזר ואמר היום ומחר כדי שיהא ברצונו של משה רבינו להתלות במדרשו ולא יהא כמעביר על דבריו במה שאמר ליום השלישי אבל לא שיהא משה רבינו מוכרח לדרוש כן שאפי׳ בחצי היום שייך למימר היום ומחר:
}
\textblock{\textbf{פירש מן האשה והסכים הקב״ה על ידו.} וי״ל אלולי שמדעתו עשה, שכינה למה לא אמרה כן [עד] לאחר מתן תורה והלא עמו היה בדיבור מקודם לכן כלאחר מיכן אלא ש״מ עד שפי׳ הוא מדעתו מק״ו ומיהו ק״ו גופי׳ לאו דוקא הוא דהא מצי משמש וטובל ומדבר בכל יום אלא הוא מדעתו שנשא ק״ו בעצמו להתקדש שיהי׳ ראוי לדבר בכל עת וכן משיבור הלוחות אינו מחוור האיך עשה מדעתו והלא ק״ו דרש אלא חומרא בעלמא הוא שעשה לומר שלא יהא נתפס בשבירתו אבל אינו ק״ו דאדרבה צריכין הם לתורה כדי שיחזרו ויעשו תשובה אף על פי שאין אוכלין בקדשים עכשיו. וכן צ״ע מנלן שהסכים הקב״ה עמו אי משום אשר הרי במה כתובין בתורה שאינן לשון אישור. ושמעתי משום דכתיב אשר שברת ושמתם בארון ושברי לוחות מונחים בארון, ואלמלא הי׳ בשבירתן חטא אין קטיגור במקום סניגור, אלא מלמד שהיתה שבירתן חביבה לפניו. ומדרש אגדה יהושע ושבעים זקנים תופסין בידו שלא ישברם ולא יכלו לו, אמר הקב״ה תהא שלו באותו היד שנא׳ ולכל היד החזקה וגו׳ אשר עשה משה, ואפשר מ״ה דרש האי אשר לשון אישור:
}
\textblock{\textbf{אתחומין לא אפקוד.} וא״ת והרי התם כתיב אל יצא איש ממקומו דהיינו תחומין להני דאית להו תחומין דאוריי׳ א״ל בסיני קאמר אלא שנכתב במקומו ומפורש בתוס׳ דה״ה להוצאה דלא אפקוד דהיאך היו מוליכין כלים ואהלים ואפשר דאלאו דמחמר לא אפקוד דאיסורי לאוין הן א״נ ס״ל כמ״ד אל יוציא ותחומין מערי מקלט גמרי׳ והנהו אמוראי כר״ע אמרי דאמר תחומין דאוריי׳ משום דלרבי יוסי איירי׳ ור״י שמעת לי׳ בפ׳ בכל מערבין דאמר תחומין דאוריי׳. וי״א דתחומין דהכא היינו הוצאה דד״א ברה״ר ורשויות דיחיד ורבים תחומין מיקרי ועוד פירשו תחומין אלו בג׳ פרסאות. וכדברי רבי׳ אלפסי ז״ל שאמר דלכ״ע של ג׳ פרסאות דאוריי׳ כדברי הירושלמי (ערובין ה,ד), ואינו כלום:
}
\newsection{דף פח}
\textblock{\textbf{לרבנן שמנה חסרים עבוד.} שלא לצורך אמרו כן, דהא אמרן אייר דהאי שתא עבורי׳ עברוה אלא משום דר״י א״נ דמוקי להנך ברייתא דלעיל כולהון כר״י:
}
\textblock{והא דאמרי׳ בענין אגדה \textbf{הא מודעא רבא לאורייתא} ומתרץ כבר קבלוה בימי אחשורוש. ק״ל, וכי מה קבלה זו עושה מסופו של עולם לתחלתו, אם קודם אחשורוש לא היו מצווים למה נענשו, ואם נאמר מפני שעברו על גזירת מלכם א״כ בטלה מודעא זו. ועוד למה הצריכה לקבלה וברית. ונ״ל לומר, דמתחלה אף על פי שהיה להם מודעא מ״מ לא נתן להם הארץ אלא כדי שיקיימו התורה כמו שמפורש בתורה בכמה פרשיות, וכתיב ויתן להם ארצות גוים ועמל לאומים ירשו בעבור ישמרו חוקיו ותורותיו ינצורו, והם עצמן מתחלה לא עכבו בדבר כלל ולא אמרו במודעא כלום, אלא ברצון נפשם מעצמם אמרו כל אשר דבר ה׳ נעשה ונשמע, (לפי) [לפיכך] כשעברו על התורה עמד והגלם מן הארץ משגלו מסרו מודעא על הדבר מדכתיב והעולה על רוחכם הי׳ לא תהי׳ אשר אתם אומרים נהיה כגוים וכמשפחות האדמה לשרת עץ ואבן, וכדאמרי׳ באגדה (סנהדרין קה.) רבי׳ יחזקאל עבד שמכרו רבו כלום יש לו עליו וכו׳ לפיכך כשבאו לארץ בביא׳ שניה בימי עזרא עמדו מעצמם וקבלוה ברצון שלא יטענו עוד שום תרעומות, והיינו בימי אחשורוש שהוציאם ממות לחיים, והי׳ זה חביב עליהם מגאולה של מצרים:
}
\textblock{הא דתנן \textbf{תבלין ב׳ וג׳ שמות.} פירש״י ז״ל כגון פלפלין שיש פלפל ארוך ופלפל שחור ופלפל לבן והן מין א׳ ערלה או כלאי הכרם ודקתני סיפא רש״א בין שנים ושלשה שמות ומין אחד בין שני מינין ושם א׳ משכחת לה כגון כרפס של גינה ושל נהרות ושל אפר ששמן א׳ והן ב׳ מינין ואף על פי שזה נקרא של אפר וזה של גינה אין אותו השם שם עצמן אלא שם לוויי הוא אבל עיקר שמם א׳. ור״ח ז״ל מפרש שנים ושלשה שמות כגון ערלה וכלאי הכרם ועצי אשרה כאותה ששנינו (מכות ד.) שלא השם המביאן לידי מכות מביאן לידי תשלומין. ופי׳ מצטרפין, לאסור את עירובן, וכן מפורש בסוף משנה זו תבלין של תרומה ושל כלאי הכרם אין בזה כדי לתבל ואין בזה כדי לתבל ונצטרפו ותבלו אסור לזרים ומותר לכהנים ר״ש מתיר לזרים ולכהנים וכו׳ משנה ולא ללקות עליהן בצירופן לכזית נשנית דא״ל דלא מצטרפי דכיון דשמות חלוקין הן אלא לאסור תערובתן קאמר והא דקתני אסור ומצטרפין ה״ק אסור אם תבלו בהן את הקדרה ומצטרפין לתבלה כמ״ש רש״י א״נ אסור לתבל בהן ומצטרפין לאיסורא בדיעבד, ולישנא דאיסורין קתני דאלו לתבל בהן לכתחלה פשיטא דאין מבטלין איסורין לכתחלה ומ״מ צירוף למלקות לא תנן ור״ש דאמר בסיפא אין מצטרפין לא מצריך צירוף כדתנן רש״א כל שהוא למכות ולא אמרו כזית אלא לענין קרבן בלבד אלא לענין איסור ערובין נשנית שצריך נ״ט או מאתים כשיעור המפורש להן במקומן והואיל וראויין למתק מצטרפין לאסור ע״י עירובן ומיהו במס׳ ע״ז מוקי לה רבא כר״מ דאמר מצטרפין למלקות ושם אפרש בס״ד והלשון הזה של ר״ח ז״ל מסתייע מן הירושלמי במקומה (ערלה ב.):
}
\textblock{והא דאקשינן \textbf{ורמינהו.} ק״ל, מי דמי איסור אכילה להוצאת שבת התם כל שני מינין מצטרפין הכא כל ששיעורן שוה מצטרפין כדתנן המוציא אוכלין כגרוגרת ומצטרפת זה עם זה ול״ל למימר ה״נ דחזי למתק (וליכא לפרושי מצטרפין כמתני׳ אלא לענין שבת) וא״ל בשלמא אוכלין ששיעורן כגרוגרת מפני שכך שיעור אכילה לשבת ושיעורן מעצמן דין הוא שיצטרפו לענין הוצאה דלא אזלי׳ בתר שמא וטעמא בהוצאה אבל אלו שאין חייבין עליהן אלא מפני (שאין) [שהן] ראוין לתבל ביצה שאם תצרפם ותתבל בהן ביצה אינה מתובלת בכך שהרי אין טעמן שוה, ומוקי לה בדחזי למתק שהרי יש בין כולן שיעור וראוי לתבל בהן ביצה כנ״ל. ובירושלמי וקשיא כמון ומלח מצטרפין ר׳ הילא בשם ר״א במיני מתיקה שנו:
}
\newsection{דף צ}
\textblock{\textbf{ורמינהו המוציא סמנין שרוין.} פי׳ למאי דקס״ד מעיקרא שרויין לאו דוקא אלא ה״ה לשאינו שרויין א״נ סבר מתניתין אפי׳ בשרוין: }
\textblock{\textbf{מי רגלים בן מ׳ יום.} רש״י ז״ל מפרש שנשתהו מ׳ יום ובמס׳ נדה משמע שדי להם בג׳ ימים ומפרשי׳ מג׳ ועד מ׳ אבל מכאן ואילך תשש כחן. ואחרים אמרו מי רגלים של תינוק בן מ׳ יום וכן בענין בדיקת תכלת במס׳ מנחות (מג.): }
\textblock{הא דאותבי׳ הכא \textbf{הוסיפו עליהן החלוסת והלעינין והבורות והאהל.} בתוספתא שביעית (ה,ה) נישנית, וכך היא שנויה שם יש להן שביעית ולדמיהן שביעית יש להם ביעור ולדמיהן ביעור ולפי שנשנו במשנתי׳ מאלו שיש להן ולדמיהן שביעית וביעור וברייתא קתני דהני נמי הכי לכך אמרו בה בגמ׳ הוסיפו עליהן: }
\textblock{והא דאמרי׳ בגמ׳ \textbf{והתנן זה הכלל כל שיש לו עיקר י״ל שביעית כל שאין לו עיקר אין לו שביעית.} אינה משנה ולא ברייתא בשום מקום וכן פירושו אינו מחוור שהרי מיני תבואה ומיני ירקות גמי אין להם עיקר ויש להן שביעית. אלא שיש לדחוק לפי ששנינו במשנתינו (שביעית ז.) בעלה הלוף השוטה ועלה דנדנה יש להם שביעית ולדמיהן שביעית יש להן ביעור ולדמיהן ביעור וקתני סיפא בעיקר הלוף השוטה ובעיקר הדנדנה וש להן שביעית ולדמיהן שביעית ואין להן ביעור ולדמיהן ביעור להכי אמר בגמ׳ שזה הכלל כל שיש לו עיקר א׳ המתקיים בארץ חוץ מן הנלקט כגון עלי הלוף יש להן שביעית לכל הלכותיה דהיינו ביעור וכל שאין לו עיקר א׳ אלא (אחר) הנלקט הזה הוא עיקרו המתקיים לו כגון עיקר הלוף השוטה אין לו שביעית לכל הלכותיה דאין לו ביעור וכיון שכך האיך מנו בכלל הוסיפו על אותן שיש להן שביעית וביעור (כבריתא) שהוא עיקר עשב מתקיים לעולם ומפני שעיקר שביעית היינו ביעור קאמרי שביעית ומפני שהענין שנוי במשנתינו אמרו והא תנן כמו שכתבתי אחרות הרבה, וז״א מחוור, אלא שהצורך מזקיקנו לפרש כן: }
\textblock{מתני׳: \textbf{רי״א אף המוציא ממשמשי ע״ז כ״ש.} פי׳ רש״י דאחשבי׳ ע״ז קרא לאיסורא מדכתיב מאומה. והא דתנן כל הכשר להצניע וכו׳ ואוקים למעוטי עצי אשרה דלא כר״י, ולא נהירא אלא מתני׳ דהכא במוציאן לשרפן או להוליכן לים המלח דומיא דרישא וכשר להצניע הוא להכי אלא דר״י לטעמי׳ דאמר מלאכה שא״צ לגופה חייב עליה ולר״ש אפי׳ במקק ספרים נמי פטור וכולה סתמא דלא כר״ש דודאי לר״ש פטור דהו״ל כמוציא מת לקברו דפטר ר״ש: }
\textblock{מתני׳: \textbf{המוציא קופת הרוכלין אע״פ שיש בה מינין הרבה אינו חייב אלא חטאת אחת.} מקשו עלה בירושלמי וקשיא אלו הוציא והוציא בהעלם אחת כלום הוא חייב אלא אחת ומפרקי למי נצרכה לר״א שלא תאמר מינין הרבה נעשו כדי העלמות הרבה ויהא חייב על כל אחת ואחת א״כ צריך למימר אילו חייב אלא אחת פי׳ לר״א דאמר במס׳ כריתות פ׳ אמרו לו (כריתות טו.) שאם בא על ה׳ נשים נדות ואפי׳ קטנות חייב על כל אחת ואחת מפני שגופין מוחלקין ומיחייב נמי התם אתולדה במקום אב סלקא דעתך אמינא מינין הרבה ליחייב קא משמע לן דכיון דבבת אחת עשאה ובהעלם אחת אינו חייב אלא אחת: }
\newchap{פרק \hebrewnumeral{10} המצניע}
\newsection{דף צא}
\textblock{}
\textblock{\textbf{מתקיף לה ר׳ יצחק ברי׳ דר׳ יהודה אלא מעתה חשב להוציא כל ביתו ה״נ דלא מחייב עד דמפיק ליה כולה.} פי׳ אדאבוה מקשי היכי מחייב ר״מ אחטה אחת והלא אין דרך ב״א להוציא פחות מכזית לזריעה ולשום דבר נמי לא חשוב בכל התורה אלא אמרי׳ בתר דעתי׳ דידיה אזלי׳ לגמרי א״כ חשב להוציא כל ביתו לא ליחייב עד דמפיק לי׳ כוליה אם הלכנו אחר מחשבתו (לקולא) [לחומרא] לא נלך אחר מחשבתו (לחומרא) [לקולא]. ומפרקי׳ בטלה דעתו אצל כל אדם שאין מחשבתו מבטלת מה שהוא ראוי לכל בני אדם אבל היא ראויה לעשות מה שאינו חשוב לכל כסתם. חשוב לו שהרי חשב עליו לזריעה וזריעה כל דהו ראויה היא וכל אדם מחשבין כן וזה הפירוש יותר ראוי ממה שפי׳ רש״י ז״ל במקום הזה:
}
\textblock{הא דבעי רבא \textbf{הוציא חצי גרוגרת לזריעה ותפחה ונמלך עליה לאכילה מהו.} ק״ל עלה מאי קמבעיא ליה דהא איהו דאמר במס׳ מנחות בפרק כל המנחות כל היכי דמעיקרא הוה בי׳ והשתא לית ביה הא לית בי׳ וכל היכי דמעיקרא לא הוי בי׳ והשתא אית ביה מדרבנן כי פליגי היכי דמעיקרא הוה בי׳ וצמק וחזר ותפח מ״ס יש דיחוי באיסורין ומ״ס אין דיחוי באיסורין וכיון שכן גבי שבת נמי בדלית ביה ותפח היכי מיחייב דהא בהנחה נמי לית ביה שיעור אוכלין לענין שום דבר כדקתני התם בברייתא הרי אלו טהורין ואין חייבין עליהן משום פגול ונותר וחלב וכן נמי לענין תרומה ואין בידי טעם נכון לחלק בין שבת לשאר איסורין בדבר זה. אבל יש לי לתרץ, דהכא בגרוגרות ממש עסקי׳ ואמרי׳ התם שאני גרוגרות הואיל ויכול נשלקן ולהחזירן לכמות שהיו וכיון שכן כשתפחה יש בה שיעור ומיהו בעוד שלא תפחה לא חשיבא שיעור ואפי׳ למ״ד התם גבי תרומה דשיעור הוא א״נ באוכלין שיש בהן שיעור וצמקו והוציאן ותפחן דודאי השתא יש בהן שיעור דהכי אסקי׳ התם תיובתא למ״ד דיחוי באיסורין לענין שיעורא:
}
\textblock{\textbf{זרק כזית תרומה לבית טמא.} פרש״י ז״ל בכמה מקומות בתלמוד שאוכלין מקבלין טומאה בכזית ולטמא אחרים בכביצה ושמעתין ל״ק עלי׳ דכיון דחשיב צירוף לענין לטמא אחרים מחייב נמי משום שבת אע״פ שכבר נטמאת. ואינו מחוור, דמשמע דהכי מבעי ליה מדמחייב משום מטמא תרומה מחייב נמי משום שבת ואי כזית מקבל טומאה אין עליו תוספת חיוב בצירוף אוכלין בפחות מכביצה עם כזית וכ״ש דלטומאת אחרים לא השתא מתעבדא בהנחתה שאפי׳ נטמאו זו בפ״ע וזו בפ״ע אף בשעת טומאת אחרים אינן מצטרפין ומטמאין אותן שהרי מכיון שנכנסה נטמאת באויר וצירוף זה שבשעת הנחה אינו מעלה ואינו מוריד ושנינו במסכת אהלות בפי״ג אלו ממעטין בחלון פחות מכביצה אוכלין וקתני כזיפא זה הכלל הטהור ממעט והטמא אינו ממעט משמע מיהא שאין פחות מכביצה אוכלין מקבלין טומאה לא מד״ת ולא מד״ס. וה״ר משה הספרדי ז״ל מפרק להא בשלא הוכשרו וכביצה אין ממעטין לפי שהן חשובין [ואינו מבטלן] (ובמס׳ חולין פ׳ השוחט (חולין דף לד ע״א) אשלים זה הענין בסייעתא דשמיא). בתוס׳ מקשים בשלמא הנחה איכא שיעורא מדמצטרף לענין טומאה אלא אעקירה הא ליכא שיעורא ולדידי ל״ק דאנן הכי אמרי׳ מדמחייב לענין מטמא תרומה חשיבא זריקה ומחייב נמי לענין שבת דחשיבא זריקה דאי אפשר בלא עקירה והנחה וכדאקשינן מדלענין יוצא בכזית לענין שבת נמי בכזית וכענין ששנינו המוציא כזית מן המת וכעדשה מן השרץ דמגו דחשוב לענין טומאה חשיב נמי לענין שבת אע״פ שאינו דומה לזו לגמרי. והם אמרו בשם ה״ר שמואל, דאשעת עקירה ודאי חייב כיון שהוא אוכל לענין זר שהזר האוכלה לוקה עליו בכזית ואלו זרקה לבית טהור חייב אבל מכיון שנטמאת פרח ממנה איסור זרות שאין זר לוקה עלי׳ עוד ומ״ה בעי צירוף להנחה, ע״כ. ואיני יודע מה הוא, שהזר בין טמאה בין טהורה לוקה עליו בכזית וחייב עליה מיתה ומפני שהיתה ראויה לאכילת כהנים אף החולין ראוין לאכילת כל אדם וכן נמי בהא דמתרצין בלחם הפנים מדאפקי׳ אפסיל ליה אכתי ק׳ דהא חשיב דזר וכהן לוקין עליו בכזית. ועוד שא״כ על כל האוכלין האסורין לחייב בכזית ולא שנינו אלא בגרוגרות לכולן ולא דמי לכזית מן המת דהתם מהני ליה הוצאתו להציל שלא יטמא לעומדין שם ומש״ה הויא הוצאה גופא חשובה ודוקא נקט נמי תרומה, דבחולין כיון שאין טומאתן אוסרת אינה חשובה כלום:
}
\textblock{\textbf{ואי סלקא דעתך אגד כלי שמי׳ אגד קדים ליה איסור גניבה לאיסור שבת.} פי׳ דאלו לענין גניבה כל פרוטה ופרוטה מן הכיס כיון שנמשכה מרשות בעלים למקום הקונה לו קנאה גנב ואע״ג דכלי ודאי לא קני׳ עד דאפקי׳ כולי׳ משיכה קונה הוא בכלי׳ דמוכר וכן דאמרי׳ הא איכא מקום חלמא ושנצין ה״נ קאמר למיקנא מאי דנפק וש״מ דהיכא דלא משתקל ליה לא קני. וא״ל דס״ל כמאן דמוקי לה במס׳ כתובות (לא:) בדאפקי לרה״ר ולענין מיקנא ממש לא קני ומיהו לענין גניבה מחייב מכיון דאפקיה מרשותיה ומ״ה מקשי הכא קדים ליה איסור גניבה שכל פרוטה ופרוטה שהוציא מרשותו נתחייב עליה הואיל ויכול ליטלה ואם אינו יכול ליטלה לא נתחייב בה. ואל תתמה, א״כ איך הקשו כאן סתם אליבא דההוא פירוקא דהא בפ׳ המוכר את הספינה (ב״ב פו.) מקשי׳ סתם אליבא דההוא פירוקא אחרינא דאוקים התם בשהוציאה לצדי רה״ר ודלא כמ״ד בששלשל ידו למטה משלשה וקיבלה ושמעתין נמי לכאורא לאו אתיא כמאן דאמר כגון ששלשל ידו למטה משלשה וקיבלה דהא בבת א׳ הוא מקבלה בידו כולה ואין איסור גניבה באה לו לחצאין. אבל רש״י ז״ל כתב כאן ובכתובות פריך אי דאפקי׳ לרה״ר איסור גניבה ליכא דרה״ר לאו מקום קניה הוא ומוקי לה בצדי רה״ר א״נ שצירף ידו למטה מג׳ וקבלה:
}
\newsection{דף צב}
\textblock{\textbf{איפוך.} י״מ איפוך דאביי לרבא ודרבא לאביי לגמרי ואי קשיא הא אמר רבא תוך ג׳ לרבנן צריך הנחה ע״ג משהו והאיך אמר ביד חייב ל״ק דהכא מעביר והא אוקימנא לההיא דרבא בזורק בפ׳ המוציא (פ.). ואי ק׳, הא אמרי׳ בפ״ק (ה.) כגון ששלשל ידו למטה מג׳ וקבלה ואפ״ה קתני סיפא פטור וכ״ת ר׳ אבוה הוא דס״ל כאביי ורבא פליג עלי׳ והא התם לא אקשי׳ עלה אלא איכפל תנא לאשמעי׳ כל הני ובא ותי׳ תי׳ א׳ אלא אי לא איכפל תנא ל״ק לי׳ לרבא. ובמקצת נוסחי התם אמר רבא איכפל תנא לאשמעי׳ הני אלמא כר׳ אבוה ס״ל הא לאו מילתא הוא דרבא גופי׳ אע״ג דלא ס״ל לר״א ליכא עלי׳ קושי׳ אלא משום איכפל תנא ומ״ה מקשי עלה הא ותי׳ בה תי׳ א׳ ולא ס״ל כר״א ולרוב הספרים נמי י״ל דגמ׳ הוא דמקשי לי׳ לר׳ אבוה הכי ולא רבא ואע״ג דאיהו מפרק פירוקא אחרינא ואיכא נמי למימר דאדרבה הך סוגיא כרבא אתיא ורישא בשלשל ידו למטה מג׳ וסיפא בשלא שלשל והיינו דאקשי׳ עלה פשיטא ולא מתרצי׳ דמשום סיפא נקט רישא וקא משמע לן שאין זו הנחה ולאפוקי מדרבא כדכתבי׳ בפ״ק. וי״מ, איפוך ואימא הכי אביי אמר בין ביד בין בכלי חייב רבא אמר בין ביד בין בכלי פטור וס״ל כר׳ אבוהו וסברא דאגוד כלי בלחוד מפיך ממר למר דמקמי דהדור בהו אתאמרא הך:
}
\textblock{מתני׳: \textbf{ובמרפקו.} פרש״י ז״ל אצילי ידיו שקורין הלעוזות אשילי״ש ולפי דעתי מסוף הפרק הראשון שהוא קנה היד עד הכתף נקרא מרפק ונקרא עציל דאמרי׳ במס׳ ערוכין בפ׳ האומר שהמרפק והעציל א׳ הוא דתנן באומר משקל ידי עלי ממלא חבית ומכניסה עד מרפקו ותניא עלה עד העציל, ואי אפשר לומר אשילי״ש שא״כ אתה מחמיר בנדרים יותר מבתפלין שאלו בתפלין אין נקרא יד אלא עד קבורת שלו והתם אמרי׳ בנדרים הלך אחר לשון ב״א ומקילין בהו טפי א״ו היינו סוף אותו הזרוע הסמוך לכתף שקורין קובדי״י ועוד דאמרי׳ במס׳ סופרים לא יקרא אדם בספר ושני אצילי ידיו עליו והיינו קובדי״ש שכן דרך ב״א מוטין ועושין כן, ואין פרש״י ז״ל נדחת ממקומו כמו שפירשתי שכל אותו הפרק נקרא מרפק ועציל. וראי׳ לדבר ממ״ש הכתוב בירמי׳ (לח:) תחת אצילי ידיו והיינו אשילי״ש שכן דרך הנמשכים להעלותם מן הבור וכן משנתינו נמי קוראה מרפק אשילי״ש שכן דרך מקצת אנשים להוציא תחת האשילי״ש ולא בסוף המרפק שנקרא קובד״י ומה ששנינו במס׳ אהלות בפ״א נבי רמ״ח איברים שנים בקנה שנים במרפק ואחד בזרוע הוא המקום שהזכרנו לפי שבאותו מקום ב׳ עצמות ארוכים שקירין הרופאים קובד״י עליון וקובד״י שפל ונקרא בלשון ערבי זכ״ר והן יודעין בנתוח:
}
\textblock{והא דתנן נמי \textbf{אף מקבלי פתקין.} במתכוין להוציא לפניו ובא לו לאחריו קאמר שכן דרך לבלרי מלכות לחגור קופסא א׳ קטנה שיש בה לולאות בחגור שלהן והלולאות רחבין ומתהפכת הקופסא מלפניו ובאה לו לאחריו ודרכה להיות חוזרת כן, ואין פירש״י ז״ל מחוור כלל:
}
\textblock{\textbf{מדקאמר אי אתם מודים לו לאו מכלל דפטרי רבנן.} פי׳ ולא משום שמשמע לשון זה כן דאדרבה משמע שמחייביןשאל״כ מאי ראי׳ הביא ר״י לדבריו וכל השנויים בתלמוד נמי דרכם כן אלא מהא דאמר והן לא מצאו תשובה לדברי משמע דה״ק והם לא מצאו תשובה וטעם לפירכא שאני מקשה להם שזה דומה לנתכוין להוציא לאחריו ובא לו לאחריו דודאי הודו לי שהן דומין אלא שהן פוטרין בכולן דבלא טעם ודאי לא פטרי בהא ומיחייבי בהא, אלא לא מצאו תשובה ופטרו קאמר ואסיפא דבריית׳ קא סמיך בפטורא דרבנן ולא מלישנא דאי אתם מודים דייק לה כלל ומפרקי׳ לה אלא מר מדמי לי׳ לחיובה ומר מדמי לי׳ לפטורא ולא מצאו לה תשובה לזה ופירכא אלא עדיין היא מחלוקת:
}
\textblock{\textbf{זה יכול וזה יכול.} מפורש בתוס׳, לא שיש לו כח לתקן ולאחזה כרצונו ולהוציאה אלא שיש בו יכולת להוציא׳ באחיזה זו שהוא אוחז בה עכשיו, וזה פי׳ נכון:
}
\newsection{דף צג}
\textblock{\textbf{ור״ש לטעמי׳ דאמר יחיד שעשה בהוראת ב״ד חייב.} איכא דקשי׳ לי׳, והא תנן במס׳ הוריות (ג:) הורו ב״ד וידעו שטעו וחזרו בהן בין שהביאו כפרתן בין שלא הביאו כפרתן והלך היחיד ועשה על פיהן ר״ש פוטר ר״א אומר ספק וה״ר משה ב״ר יוסף ז״ל תירץ דהתם איכא כפרה הכא ליכא כפרה ואחרים תירצו דשאני התם כיון דחזרו בהם ונזכרו (דמי הנזכר) [כמי שנזכר הוא] דמי שהרי עליהן הוא סומך וידיעתן כידיעתו. וכ״ז אינו מספיק, חדא מאי לטעמי׳, היכן אמרה ר״ש לזו וכ״ת משום דנקט לה התם בדאיכא כפרה א״נ בשנזכר דלמא משום דר״א ועוד דגרסי׳ בפ׳ האשה רבה (יבמות צא.) הורו ב״ד כזדון איש ואשה לא מתיא קרבן ולר״ש קאמר, אלמא הורו ב״ד ליחיד ועשה על פיהן פטור. לכך נראה כגי׳ ר״ח ז״ל שהוא גורס, ור״ש, יחיד שעשאה ב״ד ל״צ קרא פי׳ דאנוס הוה ומאי הו״ל למיעבד א״נ דממילא ממעט לגמרי מקרא קמא דכתיב ועשו כל העדה הא יחיד פטור אפי׳ מקרבן דשגגת מעשה:
}
\textblock{\textbf{הי מינייהו מחייב.} יש שואלין כאן פשיטא דמאן דיכול מחייב והאיך אפשר לומ׳ שזה שיכול פטור וזה שאינו יכול חייב ורב המנונא היכי ס״ד לחייב המסייע ולפטור העושה ומפרש לה דה״ק מדקאמרת ד״ה חייב ש״מ חד מחייב הי מינייהו מחייב (טפי) וא״ר חסדא זה שיכול דזה שאינו יכול מאי קא עביד בודאי בטל הוא כחו שלו אצל היכול ואע״פ שחייב ר״י זה אינו יכול וזה אינו יכול הכא לא ורב המנונא אמר לעולם אימא לך ר״י האי נמי מחייב משום דמסייע כדמחייב בזה אינו יכול וזה אינו יכול וא״ל מסייע אין בו ממש כיון שהא׳ יכול אבל זה אינו יכול
}
\textblock{\textbf{ההוא שכבא דהוה בדרוקרת. שרא להו רב נחמן בר יצחק לאפוקי לכרמלית. א״ל ר׳ יוחנאי אחוה דמר בר״י דרבנא לר׳ נחמן בר יצחק. כמאן כר׳ שמעון.} דאמר המוציא את המת במטה פטור דמלאכה שא״צ לגופה היא. אימור דפטר ר״ש מחיוב חטאת איסור דרבנן מיהא איכא א״ל האלהים עיילת ביה את אפילו לר׳ יהודה מי קאמינא לרה״ר לכרמלית קאמינא. גדול כבוד הבריות שדוחה את ל״ת שבתורה. פי׳ הא דשרא ר׳ נחמן איסורא כרמלית שרי להו. אבל איסור טלטול לא שרא להו. אלא ע״י ככר או תינוק. כדאמרי׳ פ׳ במה מדליקין [ד״ל ע״ב] שאפילו דוד מלך ישראל מת ומוטל בחמה מניח עליו ככר או תינוק ומטלטלו. וטעמא דמילתא, שלא התירו אצל כבוד הבריות אלא איסור שאין לו תקנה כגון הוצאת הכרמלית אבל טלטול שאפשר לתקן ע״י ככר או תינוק. לא התירו בו. אלא על ידיהן. והוא שמוטל בחמה דלא אמרו ככר או תינוק אלא למת בלבד משום כבודו. וזהו דעת ר״ש ורובי המפרשים ז״ל. ולי נראה שלא אמרו ככר או תינוק. אלא במטלטל המת ברשות היחיד מחמה לצל אבל במקום שצריך להוציאו לכרמלית כשם שהתירו הוצאתו אעפ״י שיש בה משום שבות. כך התירו לטלטלו. שאינו בדין להוסיף בהוצאה אם המת מותר בהוצאה משום כבודו. ככר ותינוק האיך הותירו להוציאן. והן אינן טפלים למת כמטה. ואין מסייעין בהוצאה כלל. ועוד הוצאה דככר קרובה לבא לידי איסור תורה. יותר מהוצאת המת, דאפילו מוציאו לקברו מלאכה שאצ״ל היא. הילכך מותר להוציאו בפני עצמו. ולא שיוציאוהו ע״י שום דבר אחר. וכן נראה דעת הרמב״ם ז״ל. ואי קשיא לך הא דאמרינן בפרק נוטל (שבת קמ״ב ע״ב). פעם אחת שכחו דסקיא מלאה מעות בסטרטיא. ואמרינן הניחו עליה ככר או תינוק וטלטלוה. אומר אני דהתם במחיצה של בני אדם טלטלוה. שאלו בפחות פחות מארבע אמות אסור הוא. ואם הי׳ מותר כן לא היו מוסיפין בו הולכת הככר פחות פחות מד״א כדפרישית. ואיכא למידק אשמעתין, לר״ש דאמר הוצאת מת אפילו לרה״ר איסורא דרבנן הוא במוטל לחמה יהא מותר להוציאו אפילו לרה״ר דומיא דכרמלית לר׳ יהודה. ועוד קשיא לן לר״ש כיון דלר״ה איסורא דרבנן הוא לכרמלית יהא מותר לכתחלה אפילו בלא כבודו דהא אמרינן בפרק יציאות השבת (די״א ע״ב) גבי לא יעמוד אדם בר״ה וישתה ברה״י. איבעיא להו לכרמלית מהו אמר אביי היא היא. רבא אמר היא גופה גזירה. ואגן ניקום ונגזור גזירה לגזירה. אלמא כל מלתא דמדאורייתא שרי ברשות הרבים. לכרמלית מותר לכתחלה לכל אדם. ויש לומר שהכרמלית עשאוהו מדבריהם כרשות הרבים ולא התירו בו. אלא משום כבוד הבריות. אבל כשהתיר רבא לעומד ברה״י לשתות בכרמלית מפני שאינו עושה מלאכה כלל. ואינה אסורה בשום מקום. וכן התיר ברה״ר יציאת החייט במחט התחובה לו בבגדו וכיוצא בו סמוך לחשיכה שאין גוזרי׳ שמא ישכח ויוציא כיון שהוצאתו בשבת עצמה אינה אלא משום שבות. כללו של דבר הגזירות שאין בהם מעשה כלל התיר הא במוציא ממש לכרמלית. דברים שהוצאתן לרה״ר משום שבות אסור. שלא מצינו היתר להוציא פחות מכשיעור לכרמלית. וכן כזית מן המת וכזית מן הנבלה. אלא עשאו הכרמלית כרה״ר למלאכות. אפילו לדבריהם. אלא שלא עשאוהו כרשות הרבים. לגזירת שמא יעשה. ולענין הוצאת מת המוטל בחמה לרשות הרבים שהזכרנו י״ל שהוא מותר על ידי תינוק. ורנב״י דשרא לכרמלית. משום מעשה שהיה לכרמלית היה. ואמר דאפילו כרבי יהודה אורי הא אלו הן צריכין להוציאו לרה״ר. היו מתירין כדברי ר״ש שגדול כבוד הבריות. שדוחה כל לא תעשה של דבריהם. ודוקא ע״י תינוק כדברי רש״י ז״ל. אבל ע״י ככר אסור דהא ככר אינו טפל למת. ולא דמי למת במטה לפי שאינו מסייע בהוצאה. ואיכא איסורא דאורייתא בככר. ומדברי רבינו חננאל נראה שהוא ז״ל אוסר. והטעם, שלא התיר כבוד הבריות הזה. אלא דבר שעיקרו מד״ס. כגון כרמלית הא רה״ר לא. ואע״פ שהתירו כלאים דרבנן וטומאה דרבנן בכהן משום כבוד הבריות כדאיתא פרק מי שמתו (די״ט ע״ב) בכאן אסרוה. דמי מפיס במלאכה שלו אם הוא צריך לגופה אם לאו. שאפילו בהוצאת מת משכחת לה לגופה כמו שאמרו בירושלמי (י,ה) בארמי שהוציאו לכלבו וכ״ש שכיון שהטלטול צריך לככר או לתינוק. חוששין לו שמא בככר הוציאוהו. דאית ביה חיובא דאורייתא. והרואה אותו מוציא לר״ה. אומר הותרה שבות דאית ביה חיובא דאורייתא. וכש״כ בשאר מלאכות. כיון שעיקרן תורה דאסור. תדע שהרי אמרו גבי טלטול. אי לא שרית ליה אתי לכבויי. ואם תאמר שכל מלאכה של דבריהם נתיר לכבוד המת אפילו כיבוי עצמו נתיר. כדברי ר״ש דמלאכה שאצ״ל היא ועוד מצינו דאפילו באמירה לארמי. דהוא שבות שאין בו מעשה אסורה כדאמרינן פרק ארבעה ראשי שנים (ר״ה כ.). שבת ויוה״כ דלא אפשר בעממין דחינן. ואם תשאל אם מפני איסור המלאכות. למה התירו בי״ט אמירה לארמי והלא מלאכה בי״ט תורה היא. וקי״ל כל בדאורייתא. לא שנא איסור לאו. ול״ש איסור כרת. כדאיתא בדוכתא ביבמות (דצ״ד ע״א). י״ל י״ט אין [איסור] מלאכה מתפרסמת עליו כשבת. שהרבה מלאכות מותרות לנו. ועוד דקרוב למכשירין הוא. והתירו משום שמחת י״ט. כדאמרינן ביצה (כ״ב ע״א) במכבין את הבקעת בשביל שלא יתעשן הבית ושלא יתעשן הקדרה. ובקבורת המת נמי דומיא דשלא יתעשן הבית איכא. הילכך שבות דאמירה מותר בו. אי נמי משום כבוד המת הקילו בשבות דלאו. יותר משבות דאיסור סקילה. כדאמרינן בפ״ב דביצה (דכ״א ע״א) שאני שבות דשב׳ משבו׳ די״ט. לענין קדשי׳ שלא יבואו לבית הפסול. ואמרינן ביבמות בפ׳ חרש (דקי״ד ע״א) יונק מפרק כלאחר יד הוא. שבת דאיסור סקילה גזרו בי׳ רבנן יו״ט דאיסור לאו לא גזרו בי׳ רבנן משום צערא. ה״נ משום כבוד המת. הקילו בשבות דיום טוב. ואע״ג דלשאר מילי י״ט כשבת. לענין שבות בשבת הכל אסור. ולא חששו לכבוד הבריות שכל מלאכה שיש לה עיקר בתורה בין ע״י ישראל ובין ע״י ארמי אסורה בשבת. ואי קשיא הא טלטול מלאכה שאין לה עיקר בתורה היא דומיא דכרמלית. ולא התירוה משום כבוד הבריות. אפילו בדליקה בלא ככר או תינוק. ורבי יהודה בן לקיש דשרי משום דלמא אתי לכבויי. אבל לא משום כבוד הבריות. י״ל תנא קמא אסר. משום דאפשר למיתן עלה ככר או תינוק. ואין לך מקום דחוק. שלא יהא המציל יכול ליתן עליו ככר או תינוק. או אחד מן הכלים שהוא לבוש וכיוצא בהן. ומשום הכי אסר. וריב״ל סבר אי מצרכת ליה לאהדורי בתר ככר או תינוק. אתי לכבויי משום שאדם בהול על מתו. הלכך שרינן לי׳ להצילו להדיא כדבעי. והילכתא כוותיה. אבל מלאכה שאין לה עיקר בתורה התירוה ע״י ישראל כגון כרמלית לכבוד הבריות. ולא התירו בה מלאכה שיש לה עיקר בתורה. אפילו באמירה כגון בעממין. וכן הא דאמרינן בכירה (דמ״ג ע״ב) עושין מחיצ׳ למת בשביל חי ואין עושין מחיצה למת בשביל מת. משום דאית לי׳ תקנת׳ הוא דאפשר למיעבד כדתני שילא מרי מת המוטל בחמה באין שני ב״א ויושבין בצדו. חם להם מלמטה זה מביא מטה ויושב עליה וזה מביא וכו׳. ואלו לא הי׳ אפשר מטלטלין. או עושין מחיצה של דבריהם לכבוד המת. ויש מי שאומר, כיון שמצינו בכל מקום. אמירה לארמי קלה ממלאכה דשבות ע״י ישראל. כדאמרן לענין חולה שאב״ס. ש״מ טעמא דשבת. דלא שרינן בעממין משום כבוד המת הוא. שלא יאמרו פלוני נתחלל עליו שבת במיתתו. וכדאמרי׳ בפרק שואל (דקנ״א ע״א) לגבי עשו לו ארון. חפרו את הקבר. לא יקבר בהן עולמית. בעומד באסרטיא. ובמונח על קברו. שלא יאמרו. זהו שנתחלל עליו שבת לצורך פלוני. וזה הטעם בעצמו. הוא שנאמר לענין הוצאת המת לרה״ר על ידי ישראל. שאלו בכרמלית אין בו משום מראית העין שהכל יודעין שאינו בכלל מלאכות. הילכך לא מבעיא חפירת הקבר והכוך דאית בהו מלאכות גמורות דאסור בעממין. אלא אפילו הוצאתו בבה״ק להניחו בכוך העשוי מאתמול דהיא ע״י ישראל שבות. וע״י עממין שבות דשבות אפילו הכי אסור. ואין נוהגים בו כבוד כלל. במקום איסורים המתפרסמים אפילו של דבריהם:
}
\newsection{דף צה}
\textblock{\textbf{חולב חייב משום מפרק.} אי קשיא, והא אמרינן בפ׳ כלל גדול (שבת ע״ה ע״א) אין דישה אלא בגידולי קרקע, ומפרק תולדה דדש הוא. וי״ל דישה בעצמו של פרי ליתא אלא בגידולי קרקע, אבל להוציא ממנו פירות מכונסים וטמונים בתוך תיק שלהם, כגון חולב, דומיא דדישת ג״ק הוא. ואפשר דבכלל ג״ק - בהמה, שמצינו שנקראת ג״ק בקצת מקומות כדאמרי׳ מה הפרט מפורש פרי מפרי וגדולי קרקע אף כל פרי מפרי וגדולי קרקע והיינו בקר וצאן. א״נ ר״א הוא דסבר הכי אבל רבנן דפטרי קסברי אין דישה אלא בגדולי קרקע ורבנן אכולהו פליגי דאי לא ל״ל למיתניי הכא ומחלוקת הוא בין הראשונים ור״ח סייע דברי האומר דחולב ומחבץ דברי הבל הוא מדאמרי׳ בכתובות וביבמו׳ יונק מפרק כלאחר יד הוא הא חולב ממש מפרק גמור הוא, ועוד אכתוב לפנינו בזה (קמד:) בס״ד:
}
\textblock{טעמא דר״א ב\textbf{מכבד ומרבץ.} לאו משום אשוויי גומות דכיון דמיחייב בהו חטאת ומיתה לא הול״ל לאיחיובי אא״כ ידוע שהשוה אותן ממש אלא דהיא גופה מלאכה שהקרקע משתווה ומתיפה בכך והוי בונה או גמר מלאכה דבנין ומכה בפטיש הוא ולרבנן ליתה מלאכה בעצמה אלא שמא ישוה גומות:
}
\textblock{והא דקתני בריית׳ \textbf{הזיד בי״ט לוקה את הארבעים.} אמכבד ואמרבץ קאי, דאלו חולב ומחבץ אוכל נפש הוא ובי״ט מותר מן התורה ובפרק ר״א דמילה (שבת קלד.) דאסרי לגבן משום דאפשר לעשות מעי״ט איסורא דרבנן קאמר דלא שרו ליה עובדין דחול באפשר דאלו מן התורה ודאי שרי דהא אוכל נפש עצמו הוא ולא אמרי׳ אפשר ולא אפשר אלא במכשירין דכתיב הוא וכתיב לכם בדאי׳ בביצה ובמגילה ועוד דהא ר״א שרי במכשורין אע״ג דאפשר כדאית׳ בפ׳ תולין וא״צ לומר באוכל נפש וכ״ש חולב דלא אפשר מעי״ט והוא אוכל נפש ומיהו למאן דלית ליה אליבא דר״א הואיל בפסחים משכחת לה בחולב מי״ט לחול א״נ לכ״ע בחולב בהמה טמאה לגוים ומחבץ ומגבן באיסורין כגון שהעמיד בשרף הערלה וחומץ של יי״נ דאיסורי הנאה נינהו. אלא דמחוורתא ברייתא כדפרישית אמכבד ואמרבץ דלא כייל תנא שבת וי״ט אלא בשאר מלאכות אבל באוכל נפש לא שוו ולא תני לה בחד כללא ומסתברא דאכולהו בבא תנינא קאי ארודה חלות דבש נמי דכיון דסבר ר״א יער הוא והתולש ממנו כתולש מן המחובר בי״ט נמי אסור דתלישה בי״ט אסורה מן התורה ואפי׳ בדברים שא״א לעשותן מבערב כגון תותים ותאינים ובירושל׳ (ביצה ה,י) נפיק לה מדכתיב אך הוא לבדו הרי אלו מיעוטין כנ״ל. וראיתי בס׳ התרומה ובתוס׳ שרבים משתבשין בזה:
}
\textblock{\textbf{הכא ליכא גומות.} איכא דקשי׳ להו, והא אמרי׳ בפ׳ ב״מ גזירה עליתא דשישא משום עליתא דעלמא ומפרקין שאני התם מפני שהגרירה משמעת קולה ומאן דשמע לא ידע אי דשישא הוא. ואחרים אומרים שאני הכא שכל העיר עשויה כן ברצפה של אבנים וזה אינו כלום. ואני אומר אין אומרין באיסורי שבת זו דומה לזו שלא מסרן הכתוב אלא לחכמים והם גזרו בזו מפני שהיא מלאכה מצויה ובשאינה מצויה לא גזרו או שהגורר קרוב לבוא לידי חריץ יותר ממה שהמרבץ בא לידי השוואת גומות [לפיכך גזרו בזו ולא בזו. וכן נראה שהרי התירו כאן להערים] (למה יהיו מתירין הערמה זו) וכ״ש [דנוכל לומר] דההיא מימרא הוא ולא ס״ל הכי:
}
\textblock{והא דאמרי׳ \textbf{והאידנא דס״ל כר״ש שרי אפי׳ לכתחלה.} כ׳ בעל הלכות ז״ל דאפילו מכבד קאמר, ואפילו בדהוצני מותר ועוד יתברר לקמן בפ׳ כל הכלים (שבת קכד:) בס״ד:
}
\textblock{\textbf{אלמא נקוב לר״ש כשאינו נקוב משוי לי׳.} פי׳ מדקתני ר״ש פוטר בזה ובזה משמע דה״ק ר״ש פוטר בזה ובזה לומר דשוין הן לכל דבר מדלא קתני התולש מעציץ נקוב חייב ור״ש פוטר והא איכא לענין הכשר דחשיב ליה ר״ש כמחובר ופריק מתני׳ לא בא אלא לרבות שאר כל הדברים וכמתני׳ דהתם אבל הכשר אינו בכלל זה שהתורה רבתה בו טהרה וראי׳ לפי׳ זה דלא אמרי׳ אלמא נקוב לר״ש כתלוש הוא אלא אמר נקוב כשאינו נקוב משוי לי׳ כלומר לכל דבר, וכזה מפרש ר״ח ז״ל. ובנמוקי ה״ר משה ב״ר יוסף מצאתי שפי׳ כלשון הזה היא גופה ל״ק לן, דילמא לענין מעשר בלחוד איתני ולחומרא דגזרי׳ נקוב אטו שאינו נקוב דלא ליתי לאפרושי מן הפטור על החיוב ולענין הכשר זרעים לא גזרי׳ דאפושי טומאה לא מפשינן אבל הא דמתני׳ דשבת קשיא ליה מדמקילינן לענין שבת ופטור על נקוב כשאינו נקוב דבר תורה הוא ויש לנו להחמיר לענין זרעים. וזה הפי׳ אינו נכון, שהרי שנינו במשנה כל[א]ים פ״ז (מ״ח) עציץ נקוב מקדש בכרם ושאינו נקוב אינו מקדש רש״א זה וזה אסורין ולא מקדשין אלמא ר״ש לקולא בנקוב וזו הברייתא נמי ידועה היא שעל משנה זו נשנית בתוספתא דכלאים לקולא. ורבותינו הצרפתים ז״ל פירשו, דמעיקרא קס״ד דלפוטרן מן המעשר ומדין שביעית קתני דנקוב כשאינו נקוב משום דכתיב בהו שדה וכן לענין כלאים כתיב כרם ושדה אבל לענין זרעים דלא כתיב בהו שדה ולא אזלי׳ בהו אלא בתר תלוש ומחובר לא שוו גבי שבת נמי בתר תלוש ומחובר אזלי׳ וכ״ז מפני שלא היה אביי יודע לחלק בין הכשר זרעים לשאר דברים כגון איסורי שבת דכל דהו חשוב תלוש בזה ומש״ה קס״ד לדמויי איסור שבת לדין הכשר זרעים:
}
\textblock{\textbf{עד שיפחת רובו.} פרש״י ז״ל אם היה מוקף צמיד פתיל והוא באהל המת, אין טומאה נכנסת לו דרך הנקב ומציל על כל מה שבתוכו עד שיפחת רובו דכל כלי פתוח כתיב:
}
\textblock{ובהא דאמר \textbf{ל״ק הא ברברבי הא בזוטרי.} פירש״י ז״ל רוב ברברבי, ומוציא רמון בזוטרי שהוא יותר מרובו, ולא מחוור, מדתנן (כלים י,ו) חבית שנקבה וסתמוה שמרים הצילה והאיך השמרים סותמין רובה של חביות ואע״פ שרש״י ז״ל פירשה בפ׳ הגוזל (ב״ק קה.) בנקובה בכונס משקה ומונחת ע״פ ארובה אין פירושו נכון חדא דלגבי צמיד פתיל תנן לה במס׳ כלים פ׳ אלו כלים מצילין בצמיד ופתיל וקתני עלה בתוספתא רי״א אין צמיד פתיל מבפנים כיצד חביות שנקבה וסתמוה שמרים אינה מצלת וחכ״א מצלת וכו׳. אלא ה״פ: בזוטרי רוב שהוא פחות ממוציא רמון ברברבי מוציא רמון ולחומרא אע״ג דקתני סיפא פקקה בזמורה עד שימרח מן הצדדין ובין זמורה לחברת׳ יש זמורות גדולות שסותמין נקב כרמון. ואי קשיא, והא לענין צמיד ופתיל סתימה גמור׳ בעי׳ דתנן התם במס׳ כלים (י:) אין מקיפין לא בבעץ ולא בעופרת מפני שהוא צמיד ואינו פתיל אלמא סתום לגמרי בעי׳ וכדמוכח התם במתני׳ והכא אמאי בעי׳ במוציא רמון ולאו מילתא הוא דודאי דרך פיו שיעורו בכל שהוא אבל דרך נקב שנפחת מן הצדדין עד שנפחת או רובו או במוציא רמון וכשניקב בשיעור בין דרך פיו בין מן הצדדין בעי׳ סתימה גמורה לגמרי שיהא צמיד ופתיל ואם לאו אין אותה סתימה כלום. וא״ת, והא בעי רבא במס׳ ב״ק (קה.) אגף חציה והניח חציה מהי אלמא לא בעי׳ סתימה גמורה לגמרי התם נמי היינו דבעי רבא מי אמרי׳ כיון שאגף אותה בטיט נעשית כדופן החבית וכאלו לא נקב אלא חציה וה״ט דלא מיבעי׳ ליה התם סתמו שמרים חציה מהו דפשיטא ליה כיון דסתימת שמרים משום צמיד פתיל הוא דמציל בעי סתימה לגמרי. ואי קשיא א״כ מאי ענין בעי׳ דרבא לההוא מתני׳ ה״ק גבי סתימת שמרים ודאי בעי׳ סתימה לגמרי אבל באגף מהו ואי לאו מתני׳ ה״א כיון שנקבה שיעור פתח שוב אין לה הצלה לעולם אלא בהסקה בכבשן אבל מכיון דקיי״ל דאית לה הצלה בצמיד פתיל במידי דקאי יש לה מיעוט לשיעור הפתח, זה כתבתי לפי מה שמצאתי בדברי ר״ש ובתוס׳. ותמהני האיך נחלקו אבות העולם וטעו בזה שהרי משניות שלמות הן בפ״ט דכלים (מ״ח) דתנן נקבו העשוי לאוכלין שיעורן כזתים העשוי למשקין שיעורו כמשקין העשוי לכך ולכך מטילין אותו לחומרא בצמיד פתיל ובכונס משקה מ״מ דברי בעלי תוס׳ עומדין שיפרשו שיעור זתים לעומד לאוכלין סתם והוסיפו בגמרא במיוחד לרמונים מוציא רמון ובזוטרי רובו והוא שמיוחד לרמונים וזה מקום הדוחק שלהן שאין כלי קטן מיוחד לרמון א׳ שאפי׳ שנים אינו מחזיק שהרי רובו פחות ממוציא רמון ומ״מ חבית שנקבה בכונס משקה קאמר דהא עשויה היא למשקין. אבל ר״ת ז״ל היה מפרש לענין צמיד ופתיל לעולם מציל עד שיפחת רובו אבל מכיון שנפחת רובו אינו מציל אע״פ שסתם אותו הנקב כולו בצמיד פתיל דה״ל כאוכלין שגבלן בטיט וכן הוא עיקר הפי׳ והא דאמרי׳ התם אגף חציה והניח חציה ה״ק אגף חציה והניח חציה סתום מן השמרים מהו שיצטרפו הסתימות הללו, וכ״כ ה״ר משה ז״ל הספרדי בחבורו, ויפה נתבארו שם הדברים הללו:
}
\newsection{דף צו}
\textblock{והא דאמרי׳ \textbf{פליגי בה תרי אמוראי במערבא חד אמר במוציא רמון.} פרש״י ז״ל בהכשר זרעים פליגי, ובתוס׳ הקשו והתנן בפ״ב דעוקצין (מ״י) גבי הכשר כמה הוא שיעורו של נקב כדי שיצא בו שורש קטן והם פירשו לענין צמיד ופתיל ובסתם כלי ואני תמה דסתם כלי חרס לאוכלין הוא ושיעורו (סתום או יותר) [כזיתים או יותר] ושורש קטן פחות מיכן הוא, אלא שי״ל לענין הכשר זרעים לר״ש קאמרינן דאיהו מיקל בנקב ואפי׳ בהכשר בעי טפי: }
\newchap{פרק \hebrewnumeral{11} הזורק}
\textblock{}
\textblock{גמ׳: \textbf{ולר״א דמחייב אתולדה במקום אב אמאי קרי להו אב ואמאי קרי לי׳ תולדה.} פי׳ גמרא גמיר לה דכולהו רבנן קרי להו אבות וקרי להו תולדות אבל אנן לא שמעי׳ לי׳ לר״א דקרי להו אבות ותולדות דאי ממתני׳ דפ׳ כלל גדול הא אוקמי׳ שם דלא כר״א ופי׳ הדברים אמאי קרי ליה אב ואמאי קרי ליה תולדה למאי הלכת׳ ומאי נ״מ לדינא דאלו טעמא דמלתא ודאי פשיטא לן אפי׳ לרבנן משום דהוי במשכן ואע״ג דאיכא דהוו ומיקרי תולדות לפי שהן בכלל האחרות לגמרי ודין הוא שלא יהיו חלוקות מהן לענין חטאת כדאמרי׳ פ׳ כלל גדול (שבת עג:) גבי כותש ובורר וזורה. ואי קשי׳, דלמא להכי קרי ליה אב וקרי ליה תולדה משום דאי עביד אב (ואב) דידי׳ מיחייב תרתי (והכי) [דהכי] אמרי׳ במס׳ כריתות פ׳ אמרו לו לאו מילתא הוא דאנן הכי קאמרי׳ למאי הלכתא קרי להו תולדות ליקרינהו לכולהו אבות מלאכות סתם ויהיו חלוקות באמת. ומיהו הא ק״ל, דלמא להכי קרי להו תולדה דאי מתרי ביה משום אב דידיה מיחייב כדאמרי׳ לקמן (שבת דף קל״ח ע״א) משמר משום מאי מתרי׳ בי׳ רבה אמר משום בורר ר׳ זירא אמר משום מרקד ואלו הוו כולהו אבות והתרה באב זה משום אב אחר אינה התראה כלל, וי״ל לר״א כיון דהן חייבין בעצמן צריכות התראה בשמן ממש. ומפרקי׳, לר״א באמת לא נ״מ מידי לענין דינא אלא הך דהוי במשכן חשיבא קרי ליה אב והך דלא הוי במשכן חשיבא קרי ליה תולדה ואע״ג דאיכא דהוי במשכן וקרי ליה תולדות כגון שובט ומדקדק וכותש ומכניס דלמא ר״א מני להו אבות הואיל והוו במשכן דהא לית ליה מתני׳ דאבות מלאכות א״נ לר״א חד אב מיקרי ומני כותש ובורר בחד אב וכן מיסך ושובט שניהם דבר א׳ הם והם אב א׳ דומיא דכותש וכותש. ומ״מ אני תמה, למה באו אלו השאלות על הכנסה והוצאה ולא שאלו סתם על מתני׳ דאבות מלאכות וי״ל מפני שזו תולדה ראשונה ששנינו בסדר המשנה וכיון שהשלים דין הוצאה והתחיל בהכנסה מיד שאל. ויש לפרש בענין אחר, דקס״ד שהאבות הן מלאכות עיקריות ותולדותיהן מלאכות שטפלות להן כגון מעמר וכניף מילחא ובונה וחופר גומא או מכבד שאלו באת לעשותן אבות אי אפשר שאין להן שם אלא לטפלן לאבותיהן אבל הוצאה והכנסה מעלתה וחיובה של מלאכה זו כזו מ״ט הוי חדא אב וחדא תולדה ופריק נ״מ להכי ומסתברא הוצאה דמפרש בה קרא הוא אב ולר״א מאי נ״מ לן מינה ומתרץ משום חשיבות דמשכן: }
\textblock{וגירסת הספרי׳ \textbf{הך דהואי במשכן חשיבא הך דלא הוה במשכן לא חשיבא הך דכתיבא קרי ליה אב הך דלא כתיבא קרי ליה תולדה.} ופי׳ הואי במשכן משום שאר אבות וקאמר כתיבא משום הוצאה דהכנסה נמי הוי התם אלא הך דמפרש בה קרא תפס ליה אב. ואיכא דגרסי, הך דהוי במשכן וחשיבא קרי ליה אב הך דלא הוי במשכן ולא חשיבא קרי ליה תולדה וכדאמרן לרבנן ולא מחוורי הך גירסא משום דכל דהוו במשכן ודאי חשיבא אלא שיש מהן שהם בכלל האחרות ולר״א טפי מחשב להו כי קרי להו תולדה מכי קרי להו אב דכי קרי להו אב בהדי אב דדמי ליה מיחשבי כגון כותש בכלל בורר, ותולדה חשיב לעצמה: }
\newsection{דף צז}
\textblock{\textbf{אבל למעלה מעשרה ד״ה פטור ולא ילפי׳ זורק ממושיט.} כ׳ רש״י ז״ל וסיפא דקתני הזורק פטור והמושיט חייב ד״ה היא פי׳ לפירושו דכי קתני זורק פטור למעלה מעשרה וק״ל א״כ מאי כיצד ומתרץ בתוס׳ דהכי קתני שתי גזוזטראות כיצד דינם וכיוצא בזו בפ׳ ר״ג במס׳ יבמות (נ.) וחכ״א יש גט אחר גט ויש חליצה אחר חליצה כיצד וכו׳ ולא קאי כיצד ארישאאלא כיצד דין יבם ויבמה קתני כדאיתא התם (נב.). ונ״ל דלמ״ד קלוטה כמי שהונחה ליכא למיפטר זורק אפי׳ בשתי גזוזטראות זו כנגד זו בר״ה דהא אע״ג דלא הוי גזוזטראות כי הני במשכן הרי מוציא הוא מרה״י לרה״ר שהרי קלוטה כמי שהונחה ומתני׳ רבנן קאמרי לה וה״ק המושיט והזורק פטור לעולם אפי׳ למטה מעשרה מושיט משום דלא הוי הכי במשכן זורק משום דל״א קלוטה כמי שהונחה. היו שתיהן בדיוטא אחת הזורק פטור לעולם אפי׳ למטה מעשרה והמושיט חייב לעולם שכך היתה עבודת הלוים ולמ״ד למעלה מעשרה פליגי אבל למטה מעשרה חייב לד״ה רישא דקתני הזורק והמושיט פטור למעלה מעשרה וד״ה דהא ליכא למילף זורק זה ממושיט דמושיט כי האי לא הוה במשכן ואי למטה מעשרה זורק חייב משום דהיינו מוציא כיון דס״ל דקלוטה כמי שהונחה וסיפא נמי למעל׳ מעשרה ורבנן מש״ה קאמר הזורק פטורוכי קתני כיצד אסיפא קתני. ובירושל׳ (יא.) מצאתי, שמואל אמר לא שנו אלא למטה מעשרה טפחים הא למעלה מעשרה אסור מילתיה דר״א אמרה אפי׳ למעלה מעשרה דאמר ר׳ הילא בשם ר״א מעגלות למד ר״ע ועגלות לאו למעלה מעשרה אינון כיצד שתי גזוזטראות וכו׳ אית תנא דתני כיצד ואית תנא דלא תני כיצד פי׳ אסור דקאמר שמואל אסור ואין חייבין עליו סבר מחלוקת למטה מעשרה אבל למעלה ד״ה פטור ולא תני כיצד דד״ה הוא וזה סיוע לרש״י ז״ל ול״ג בדר״א מותר אלא ה״ג אפי׳ למעלה מעשרה, והיינו ר״א דגמ׳ דילן:
}
\textblock{\textbf{אלימא במעביר למטה מעשרה הוא דמחייב למעלה מעשרה לא מחייב.} אי קשיא, דהו״ל למירמא מ״ט דרבנן דפטרי בלמטה מעשרה ולאו מילתא הוא דרבנן סברי מהלך לאו כעומד דמי ופטור כיון דלא הניח ברה״ר ור״ע סבר כעומד דמי לחייבו והשתא מקשי מ״מ מה לי למעלה מה לי למטה במעביר והאר״א וכו׳ וקשי׳ לר״ע דמקראי נפקא לי׳:
}
\textblock{הא דתניא \textbf{תוך ג׳ ד״ה חייב.} ק״ל, והאמ׳ רבא לקמן (שבת ק.) תוך ג׳ לרבנן צריך הנחה ע״ג משהו ואיכ׳ דמפרקי דרבא ס״ל כר״י דאמ׳ בפ׳ המוצ׳ תפלין (צ״ח:) אפי׳ אינו מסולק מן הארץ אלא כמלא החוט גוללו אצלו, ומפרשי׳ התם טעמא משום דבעי׳ הנחה ע״ג משהו. ואי קשיא, והא אתמהי׳ עלה התם לימא רבא כתנאי אמרה לשמעתי׳ ומחסרי׳ לי׳ למתני׳ משום הכי והא ע״כ כתנאי אמרה לשמעתי׳ א״ל תנאי דברייתא לאו תנאי נינהו למיקשו מינייהו לרבא הואיל ומתני׳ כולי׳ אתא כוותי׳. ובתוס׳ מקשו מדר״י אדר״י דהתם אפי׳ בתוך ג׳ בעי הנחה, ולקמן בשמעתין (שבת צז:) סבר קלוטה כמי שהונחה. ולא סליק להו פירוקא ולדידי ל״ק שאני התם שאיןסופו לנוח שהרי אגדו בידו ואע״ג דגזרי׳ אטו אין אגדו בידו בשאין סופו לנוח קאמרי׳ כגון שנתעכב בכותל שאינו משופע ותלוי ועומד לעולם דל״ל קלוטה לא במעביר ולא בתולה אלא בזורק בלבד שנקלטה באויר [וסופו לנוח בארץ], ומ״ה בעי׳ הנחה ע״ג משהו אבל בזורק דהכא כיון שנקלטה בעצמה באויר כהנחה דמשהו דמיא שהרי (קלוטה בו אויר) [קלטה האויר], וכ״ש בשנח באותו רשות בעצמו בהא אית ליה לר״י קלוטה כמי שהונחה. ואי קשי׳, השתא רבא דאמר כמאןדהא רבא אמרה בזורק ודילמא עד כאן לא אמר ר״י התם אלא בשאין סופו לנוח בקרקע עצמו אלא תלוי ועומד בפחות מג׳ ואין זה קלוט בעצמו אבל בשסופו לנוח לא אמר דהא אפי׳ למטה מעשרה אית ליה קלוטה כמי שהונחה לאו קושי׳ הוא דהא רבאלרבנן קאמר דלית להו לעולם קלוטה כמי שהונחה הלכך קאמרמדר״י בתלוי באויר נשמע לרבנן בקלוטה דאפי׳ בפחות מג׳ לאו הנחה היא. ור״ח ז״ל כ׳ דהך ברייתא פליגא אדרבא והלכת׳ כרב חלקיה דתני׳ כוותיה והוא העיקר דההיא ר״י אוקימתא דרבא גופיה הוא והא דמתמה התם לימא דרבא תנאי הוא משום דבשלמא בבריית׳ איכא למימר דלא שמיע ליה כדלא שמיע ליה לר״א אלא מתני׳ היכי שביק רבא רבנן ואמר כר׳ יהודה, היינו קושיין:
}
\textblock{\textbf{ולאו מי אוקימנא להאי כגון דמדלי חד וכו׳.} לא אתפריש עיקר אוקימתא היכן, דבמס׳ עירובין (פה:) נמי אמרי׳ ולאו מי אוקימנאאולי יש שם חילוף בנוסחת רש״י ז״ל שכ׳ כאן לאו מי אוקימנא בעירובין:
}
\textblock{\textbf{כגון דאמר עד דנפקא לי׳ לרה״ר תנוח.} פי׳ לא שאמר תנוח סתם דא״כ תנוח ותעמוד משמע והרי לא נחה אע״פ שלענין שבת הויא כמונחת אלא ה״ק רוצה אני שתהי׳ כמונחת לענין שבת עם יציאתה שלא אתחייב בהוצאתה אלא בהנחת יציאתה ברה״ר ומ״ה מחייב לר״י כיון דקלוטה כמי שהונחה:
}
\newsection{דף צט}
\textblock{רש״י ז״ל מפרש שעל \textbf{עגלה} אחת מניחין קרש א׳ ועודף לכאן ולכאן והויין חמיסר לפי ששתי העגלות הולכין זו אצל זו כל אחת ה׳ אמות הרי עשר וה׳ אמות שבין עגלה לעגלה הרי חמיסר. והקשה על דברי עצמו ז״ל, שהרי יותר מעשרים הם שהרי הקרשים ארכן י׳ אמות והן מוטלין על דופני עגלה לרחבן ונמצא ארכן של קרשים לרחב הדרך הרי עשרים אמה לבד ממה שהי׳ צריך למשוך קרשי כל עגלה ועגלה צד החיצון של עגלה כדי שלא יגעו קרשים זה בזה ומעכבין את הילוך העגלות והוא השיב אין זה תשובה שהקרשים למעלה מי׳ הן ולגבי רה״ר לית לי׳ למיחשב אלא מקום העגלות ורוחב שביניהם. וזה הפי׳ כמה רחוק, חדא דאקשי׳ ל״נ באמתא ופלגא סגי׳ ואם הקרשים לרחבן שלעגלות המה האיך שיער זה המקשה שבכך די והלא אף בשתי אמות ומחצה אין די שהרי הקרשים ארוכים מאד וכבדים ושמא יפלו ועוד היכי הוו חמיסרי ושיתסרי האיך היו הולכות בשוה ואין מקדים רגע א׳ ואין מאחר רגע דלמא זו אחר זו לגמרי היו מהלכות כיון שהקרשים בעגלה אחת מונחין ומנ״ל דהוו שתיםמהלכות בשוה ולא ארבעתן ועוד דא״כ טובא הויין. ואין תשובתומחוורת , שאע״פ שהקרשים למעלה מי׳ מ״מ היו צריכין להרחיקהאהלים כמלא אותן של קרשים נמצא הכל רחב דרך הרבים והרי אתה מחשב שתי העגלות וריוח שביניהן לרשות א׳ וכיון שכן או הן כ׳ אמה ברוחב שתי העגלות וצידיהן והרחק שביניהן עם בליטתןמכאן ומכאן או לא תחשוב אלא כל העגלה רשות בפני עצמה שאין הקרשים שעליהן עושין רשות מפני שהן למעלה מכ׳ ונמצא בין קרש לקרש ה׳ אמות והוא רחב רה״ר ואין דרך זו עולה. והגאונים ז״לאמרו שעל רוחב שתי העגלות הקרש מוטל. והשמועה פשוטה לפי דרך זה, ומפורשת בצורתה בחיבור ה״ר יהודה אלברצלוני. ובס׳ המאור לר׳ זרחי׳ הלוי ז״ל:
}
\textblock{\textbf{א״ד כיון דממקום פטור קא אתי׳ לא.} איכא למידק הכא ולפשטי׳ מהא דתני׳ (ה:) מחנות לפלטיא דרך סטיו וכו׳ וא״ר יוחנן עלה מודה בן עזאי בזורק ותניא כוותיה אלמא אע״ג דאתי ממקום פטור חייב. ובתוס׳ מתרצים ש״ה דכה״ג הוי במשכן אבל הכא לא הוה במשכן ואין טעם זה מספיק שאעפ״כ היה להם לפטור כי היכי דפטר ב״ע מהלך אע״ג דהוה במשכן ועוד דאכתי קשי׳ דהאמר רבא המעביר חפץ מתחלת ד׳ לסוף ד׳ אע״פ שהעבירו דרך עלי׳ חייב ועוד אמאי לא פשטוה מהא דתניא ר״י בר׳ יהודה אומר נעץ קנה ברה״ר ובראשו טרסקל וזרק ונח על גביו חייב ואוקימנא למעלה מי׳. ועוד מהא דתני׳ עמוד ברה״ר גבוה י׳ ורחב ד׳ ואין בעיקרו ד׳ ויש בקצר שלו ג׳ וזרק ונח ע״ג חייב ובכל הני ממקום פטור קאתי וי״ל דמאמוראי לא תפשוט לי׳ לרב מרדכי דמילתא דפשיטא להו מבעי לי׳ וטרסקל ועמוד לא עדיפי ממתני׳ דמתוקמי במורשי וחרצא וכוון וזרק בהן וברייתא דמודה ב״ע בזורק לא שמיע להו. ונ״ל לתרץ, דרב מרדכי פשיטא לי׳ דכל דאתי ממקום פטור למקום חיוב ונח במקום חיוב ובאויר של חיוב בזורק חייב וכי קא מבעי לי׳ כגון דהשתא נמי ברשות של אויר פטור קא מנח כגון כלי (של) [שעל] עמוד זה שכל גופו במקום פטור קאי כלומר למעלה מי׳ אלא שהוא מונח ע״ג י׳ וכיון דגופו למעלה אפי׳ אחר שנח דילמא פטור דחיובא ממקום פטור קאתיא שהרשות הזה אויר פטור הוא כמו שפרש״י דאינו אויר רה״ר ואויר רה״י נמי לא הוי דכי אמרינן (ז.)רה״י עולה עד לרקיע דילמא באויר חצר אבל בעמוד לא הלכך ההוא דמחנות לפלטיא דרך סטיו ל״ק דבמקום חיוב מנחא וממקום חיוב נמי אתיא שהרי יצאת (לרה״ר) ואח״כ נחה ומעביר חפץ דרך עליו חייב דממקום חיוב קאתיא להנחתה ודר״י בר׳ יהודה בטרסקל לרב מרדכי בשיש לו דפנות לטרסקל שנמצא מונח ברה״י וכן הא דתניא עמוד גבוה י׳ ורחב ד׳ נמי בהכי מתוקמ׳ כגון שיש לו דפנות. ואי קשיא, אי הכי מתני׳ נמי לוקמה בהכי ל״ק דסלע לית ליה דפנות וכן הא דא״ר חייא (ק.) זרק למעלה מי׳ והלכה ונחה ע״ג חור כל שהוא באנו למחלוקת ר״מ ורבנן ש״ה שהנחתה במקום חיוב כולה וממקום חיוב אתי׳ לי׳ אע״ג דעברה במקום פטור כך י״ל מתוך הדחק ונכון הוא דר׳ מרדכי ודאי דיחוי מידחי ומ״מ עיקר כדאביי ורבא:
}
\textblock{\textbf{התם לא מבטל לה הכא מבטל לה.} הקשה בתוס׳, וכי לא מבטל לה מאי הוה הא ליכא ד׳ אמות אטו אם היה בידו חץ של ד״א והניחו חייב הרי כל שאינו כולו חוץ לד״א לא מחייב ולדידן לאו מלתא דהתם כולו מונח, ואינו מונח בתוך ד׳ אבל הכא כיון שהוא מונח על פניו המדובקות בכותל והיא (ר״ל הכותל) ההנחה שלו חוץ לד״א הוא מונח הלכך כ״ז דלא מבטל ליה ולא מיחשב כמחיצה עצמה בדין הוא דליחייב. והם הקשו, ומנ״ל למקשה דבד״א מצומצמות הוא והרי כ״מ ששנו זורק ד״א בשזרק חפץ כולו חוץ לד״א היא והכא בשיש ד״א על פני הדבלה כחץ דלא ממעטא מד״א. ול״ק, דכיון דהניחה בפנים המדובקים בכותל, ודאי ד״א מהתם משערינן וקושטא מתרצים דאפי׳ במצומצמות חייב ואי הוה צריך הוה מתרץ דלאו במצומצמות הוא ובתוס׳ (הקשו מפרש״י) [מפרשים] דאי לאו משום דלא מבטל לה הוה לי׳ דבילה ככותל ולאו הנחה היא אלא עשיית כותל הוא ולאו מילתא הוא דא״כ גבי בור בלא מיעוט נמי סילוק מחיצה הוי:
}
\newsection{דף ק}
\textblock{\textbf{פירות מיבטלא מחיצה.} פי׳ לא שאם זרק פירות לבור שיהא פטור שיהיו הן עצמן מבטלין המחיצות דהא אמרן לעיל (שבת צט:) דפירות דלא מבטל להו לא מבטלי מחיצתה ול״א בהו הנחת חפץ וסילוק מחיצה בהדי הדדי אתו דהא איכא מחיצות ניכרות בשעת זריקה אלא דכיון דמעיקרא ממלא פירות הזורק שם פטור שהרי נתבטלו קודם זריקה זו מתורת מחיצות שאין עומקו ורחבו ניכר ואינו משתמש לו. ואיכא לפרושי הא דאמר התם לא מיבטל לה משום שסופו ליפול משם ואינו בנין והנחה גמורה שתתקיים לעולם כדי שיבטל המחיצות אבל הנך פירות בבור ודאי מבטלי מחיצות ובתוס׳ פירשו דהכא מבטל להו שאינו יכול להעלותןמן הבור עד לאחר השבת ואינו כלום. ור״ח ז״ל כתב אבל פירות שכבר הושמו בבור אם נתן כלום למעלה פטור דפירי מבטלי מחיצה וה״מ דבטלינהו לגבי בור אבל אי אצנעינהו לפירי בבור למיהדר ולמיכל מינייהו לבתר הכי לא ממעטי בבור וגם זה לא נתחוור לי דאין ביטול פירות בבור כלום דבטלה דעתו אצל כל אדם אלא מעצמן הם מבטלין מחיצות מאחר שכבר הושמו שם כדפי׳. יש מקשים מההוא דגרסי׳ בפ׳ חלון חריץ שבין ב׳ חצרות מערבין שנים ואין מערבין א׳ ואפי׳ מלא תבן או קש דתבן וקש לא מבטלו מחיצה וכ״ש פירות ומתרצי לה בכה״ג הכא בקופה של טבל עסקינן דאינן ראויין לטלטל ולאו מילתא הוא דודאי פירות לא מבטלי למהוי אינהו מחיצה ולשוויי ב׳ רשויות רשות א׳ ולצרפן אבל מ״מ לענין רשות שבת מבטלי מחיצה דעמוק י׳ ורחב ד׳ בעי׳ ומחיצות ניכרות בעי׳ והא ליכא וכדאמרן וכן נמי גבי סוכה גבוהה פירות לא מבטלי גובהה שהן אינן עושין מחיצות אלא דשבת שאני כטעמא קמא דאמרן, וטעם נכון הוא: }
\textblock{הא דאמר ר״ה ד\textbf{מוציא זיז כל שהוא.} ודאי למטה מי׳ משפת מיא קאמרי׳ דלמעלה במשהו סגי מיהו אף על פי כן ק״ל דהאמרן בפ״ק דלמטה מי׳ נמי אי איכא ד׳ הוי כרמלית ואי לא מקום פטור בעלמא הוי הלכך בזיז כ״ש סגי וי״ל דאויר כרמלית הוא ואין (הוצאת) [הנחת] זיז כלום, עי״ל דס״ל אפי׳ ברשויות דרבנן ובלבד שלא יחליפו והלכך אסור להוציא מכרמלית דים לספינה שהוא רה״י דרך זה אע״פ שהוא מקום פטור וה״נ משמע בפרק כיצד משתתפין לפום מסקנא דרבינא דר׳ יוחנן אפי׳ בדרבנן נמי קאמר ובלבד שלא יחליפו:
}
\textblock{\textbf{עושה מקום ד׳ וממלא.} ק״ל טובא, מקום ד׳ נמי או כרמלית הוי או רה״י אחר הוי כגון שגבוה עשרה וקמפיק מכרמלית לרה״י ומאי תקנתי׳. עד שמצאנו לרבינו האי גאון ז״ל שכ׳ והי׳ אדונינו גאון יהודה זקננו ז״ל אומר ששמע מן הזקנים שאותו מקום כגון תיבה פחותה או סל פחותה. ונמצא סיוע בתלמוד א״י פי׳ לפירושו באותה שאמרו במס׳ עירובין בפ׳ כיצד משתתפין ר׳ חנינא בן עקיבא אומר גזוזטרא של ד׳ על ד׳ אמות חוקק בה ד׳ על ד׳ וממלא משום דאמרי׳ כוף וגוד ומיהא הכא לא בעי׳ ד׳ אמות אלא חקק ד׳ ואפי׳ לרבנן דפליגי עליה התם הכא שרי משום דלא אפשר ולפ״ז י״ל בין בזיז בין בארבעה (שהוא) [שאינו] עומד עליהם וממלא אלא שהוא מעביר דליו תוך הזיז חלול כל שהוא ולפיכך אסרוהו עד שיעבור דרך מקום דא״ל כוף וגוד דמקום חשיב הוא ובין למעלה מי׳ לשפת מים בין למטה דינן שוה. וזה מה שמצינו בירושלמי (יא,ה) רב המנונא אמר נסר שהוא נתון לספינה ואין בו רחב ד׳ מותר לישב בו ולעשות צרכיו בשבת א״ר מנא אלו אמר תיבה פחותה יאות א״ר בון מאן דבעי למעבד תקנה לאלפא מוציא נסר חוץ לג׳ שאין בו רחב ד׳ ואני רואה את המחיצות כאלו עולות דא״ר יעקב בר אחא בשם רב המנונא כל ג׳ וג׳ שהן סמוכין למחיצה כמחיצה הן ר׳ יצחק בר׳ אליעזר מפקד ר׳ יהושע בר שמיי דהוה פרוש מעבדא לי׳ סל פחות ואינו מסכים לגמרי עם גמרתינו מיהו משמע שעל הזיז הוא יושב וממלא. ושמעתי שר״ת ז״ל מפרש, מקום ד׳ לחקק, אבל צריך ד״א שהן עשרה טפחים לצדדין כדאמרינן בגזוזטרא דבפחות מכן ליכא למימר כוף וגוד ואינו במשמע:
}
\textblock{\textbf{כחו בכרמלית לא גזרו רבנן ומנא תימרא וכו׳.} איכא למידק, אדמסייע ליה מדר״י תיקשי ליה מדרבנן דאמרי לא מתוכה לים ולא מן הים לתוכה. ובתוספ׳ מפרקי׳ לה מדתניא בתוספתא ספינה שבים גבוה י׳ טפחים אין מטלטלין לא מתוכה לים ולא מן הים לתוכה רי״א עמוקה י׳ ואין גבוה י׳ וכו׳ ובהא פליגי תנא קמא סבר אפי׳ גבוה י׳ אסור לטלטל מן הים לתוכה ומתוכה לים להדי׳ ור׳ יהודה סבר גבוה י׳ מותר דמכרמלית למקום [פטור הוא ודרך מקום] פטור אתי לרשות היחיד ואפי׳ אינו גבוה י׳ מתוכה לים מותר דרך חורה ובהא ל״פ רבנן עליו כלל ולהאי פירושא האי דבעי זיז ומקום ד׳ לרבנן משום דאסור להחליף, ול״נ דר״י אפי׳ אינה גבוה י׳ קאמר, דלא תימא כולה כרמלית הוא:
}
\textblock{\textbf{קשרה בדבר המעמידה מביא לה טומאה בדבר שאין מעמיד׳ אין מביא לה טומאה ואמר שמואל הוא שקשרה בשלשלות של ברזל לענין טומאה היא דכתיב בחלל חרב חרב ה״ה כחלל.} כך הוא בכל הנוסחאות וכך גורס רש״י ז״ל ומפרש דה״ק קשרה בדבר שרגיל להעמיד בה דהיינו שלשלת של ברזל מביא לה טומאה אם ראשו א׳ קשור באהל המת מביא טומאה לספינה ולכלים שבתוכה דחרב ה״ה כחלל והרי הוא כאבי אבות הטומאה ואיירי בספינה המקבלת טומאה. בדבר שאינה מעמידה כלומר שאינו רגיל להעמידה כגון מיתרים דלאו מתכות אפילו במקבלי טומאה אין מביא לה טומאה דאלו ספינה לא מיטמאה מחמת אותו כלי להיותה אב הטומאה אלא ראשון לטומאה ואין היא מטמאה כלים שבתוכה. ואין זה הפירוש נכון, חדא ל״ל קשרה אפילו נוגעת נמי ועוד דבדבר המעמידה לא משמע הכי ועוד דמביא לה טומאה דרך אהל משמע ועל אהל הוא שנוי בכ״מ ועוד דאמרי׳ שמואל לאפוקי מדנפשי׳ אתי ומה ענין זו לזו שיעלה על הדעת לדמותן ולמיבעי בהא נמי שלשלת של ברזל כי התם כי היכי דליצטרך שמואל לאפוקי מהך סברא. וה״ר שמואל גורס קשרה בדבר המעמידה מביאה את הטומאה ובדבר שאין מעמידה אינה מביאה את הטומאה ול״ג בחלל חרב כו׳ אלא ה״ג גבי טומאה הוא דכתיב כל הבא אל האהל אבל לענין שבת כיון דיכול להעמידה היכרא בעלמא הוא וה״פ הכא כגון שיש טומאה למטה מן הספינה והספינה מאהלת על הטומאה ועל האדם ועל הכלים וכל זמן שהספינה שטה ומתנדנדת אינה מביאה את הטומאה שאין אהל מביא טומאה אלא אהל קבוע דתנן ברישא אלו לא מביאין ולא חוצצין הזרעים והעוף הפורח והטלית המנפנפת וספינה שהיא שטה ע״פ המים ועלה קתני קשר את הספינה בדבר המעמידה מביאה את הטומאה וזו היא המשנה שהביאו כאן בגמ׳ ול״ג דתניא אלא דתנן, וכן בפר״ח ז״ל. והא דמסיימו בדבר שאין מעמידה אינה מביאה את הטומאה דיוקא דגמ׳ הוא הא בדבר שאין מעמידה כו׳ ומ״ה אוקמה שמואל בשקשרה בשלשלת של ברזל ליתד התקועה בארץ או לדבר העומד והעמידה שלא תשוט. וזה הפי׳ הי׳ נכון ומחוור אלו היתה הגי׳ כן כתובה בספרים. ולזה אפשר שיהא אמת מה שפי׳ מקצת המפורשים דכי אמרי׳ חרב ה״ה כחלל לאו דוקא חרב אלא [ה״ה] לכלי שטף ולא אמרה תורה חרב אלא להוציא כלי חרס ולפי שאין נעשין אב הטומאה. אבל אחיו רבינו תם ז״ל מקיים הגי׳ הכתובה בספרים ומפרש לה בשלשלת טמאה שנגעה במת וחזרו הספינה ואהלה על השלשלת ועל הכלים וטמאם לפי שחרב היא כחלל ומטמאה באהל ודוקא קשרה לפי שאיננה נפרדת ממנה ולעולם הספינה מאהלת עלי׳ ועל הכלים נמצאת כאהל קבוע ומביאה את הטומאה בכלים שהספינה עצמה אינה מקבלת טומאה. ואכתי ק״ל, למה איצטריך שמואל לאפוקי מההיא מה ענין שבת אצל טומאה ואפשר דכי היכי דלא תיסק אדעתין [דכיון] דאפילו במאהלת על המת בעי שמואל קשירה בשלשלת משום דליהוי אהל קבוע וה״נ ליבעי קביעותא ואי לא קביעי כמפורדות דמיין קמ״ל (משום דליהוי שלשלת גופה מטמאה קאמר וה״ה לכל דבר המעמיד דהוי אהל) והנה בפי׳ זה גילה רבינו תם ז״ל דעתו דחרב ה״ה כחלל אפי׳ לטמוי באהל וחבריו חולקין עליו, ובפ׳ לא יחפור (ב״ב כ׳ ע״א) שמעתי בזה דברים ארוכים ושם אכתוב הכל בס״ד:
}
\newchap{פרק \hebrewnumeral{12} הבונה}
\textblock{}
\textblock{ נ״ל דהכי קאמר ולטעמיך דחשבית דכולהי בניני שווין היאך אפשר דר״י סבר אפי׳ בהנחה בעלמא והא לאו כלום קעביד ועד שאתה תמה עלי לך ותמה על ר׳ יוסי ביותר מזה ויש כזה ולטעמיך במס׳ ב״ב (קו.). \textbf{ולטעמיך אימא סיפא ר׳ יוסי אומר כו׳.} ק׳ להו לרבוותא ז״ל והא אמרי׳ במס׳ ביצה אין בנין בכלים ואין סתירה בכלים ויש לנו לתרץ דכי אמרי׳ אין בנין בכלים ה״מ בכלי שנתפרק כגון מנורה של חליות אבל הכא עושה כלי או מתקן כלי מתחלתו אין לך בנין גדול מזה ואין זה נקרא בנין בכלים שהרי אינו כלי אבל עושה כלי מיקרי. ואפשר לפ״ד זו, שכל כלי שצריך אומן בחזרתו מיחייב עליה משום בונה דהו״ל כעושה כלי מתחלתו שהרי משעה שנתפרק ואין ההדיוט יכול להחזירו בטל מתורת כלי. והיינו דגרסינן בפ׳ כירה במנורה של חוליות אסרינן להו אפי לטלטלה גזירה משום שמא תפול ותתפרק לגמרי ויחזירנה ונמצא עושה כלי בתחלה בשבת. וההיא דאמרי׳ במס׳ ביצה (יא:) גבי תריסי חניות מ״ד טעמא דב״ה משום דאין בנין בכלים ואין סתירה בכלים ואפי׳ דבתים נמי קמ״ל הותרו סופן משום תחלתן דחניות אין דבתים לא לאו למימרא דסברי ב״ה יש בנין בכלים ויש סתירה בכלים אלא ה״ק מ״ד טעמייהו לאו משום שמחת י״ט שאלמלא כן היו גוזרין עליהן בשיש להן ציר באמצע אטו מן הצד ומן הצד אסור דהו״ל מלאכת אומן וכעושה כלי לכתחלה דמי ומשום שמחת י״ט הוא דלא גזרי הלכך בדבתים דליכא משום שמחת י״ט גזרי׳. וי״א דכשיש להן ציר מן הצד אסור משום גזירה שמא יתקע שהרי הדלת תסוב על צירה אבל בשיש להן ציר באמצע אינו תוקע שהרי יש לו לפתוח כל היום ואינה סובבת לפיכך לא אסרוהו אלא משום גזירה דמן הצד ואע״ג דגזירה לגזירה ל״ג בכל כה״ג דשכיחא גזרינן גזירה כדמוכח בפ׳ י״ט (יח.). ויש לפרש דהני נמי בנין בכלים הוא וסברי ב״ה אליבא דעולא יש בנין בכלי׳. והא דאמרי׳ גבי מנור׳ זוקפין מנור׳ משום דלא מיחזי להו זקיפתה בלחוד כבנין שלא נתפרקה כלל והיינו נמי דאמר ר׳ יוחנן גבי מתני׳ דתריסין מוחלפת השיטה דסבר לה כעולא אבל אנן דקי״ל אין בנין בכלים לא מפכינן לה. ומשמע לי נמי דכל היכי דתקע ביתידות חייב משום בונה והיינו דאמרי׳ בפ׳ כל הכלים גזירה שמא יתקע כלומר ומחייב משום בונה ולא כמו שפי׳ רש״י ז״ל משום מכה בפטיש ואפשר דהא דאמר רשב״ג בפ׳ כירה אם הי׳ רפוי ה״ז מותר משום דס״ל אין בנין בכלים בלא תקיע׳ הילכך הואיל וכשאינו רפוי לאו אב מלאכה הוא כשהוא רפוי מותר לכתחילה ות״ק אף על פי דסבר אין בנין בכלים בלא תקיעה כדקאמר ואם תקע חייב מ״מ גזר כל חזרה אפי׳ כשהוא רפוי לכתחלה שמא יתקע ואלו הי׳ סובר רשב״ג שהמחזיר חייב חטאת ה״ל לאפלוגי בהא דאמר ת״ק אם תקע חייב ועוד דלא הוי שרי ברפוי לכתחלה כדמקשינן בפ׳ המצניע מי איכא מידי דבכלי חייב חטאת וביד מותר לכתחלה אלא ש״מ קסבר אין בנין בכלים ולפמ״ש הא דאמרי׳ דמאן דעביד תלתא חייב י״ג חטאות חד מינייהו משום בונה וכך אמרו בירו׳. עוד יש לי בירור שוב יותר מזה בס׳ המלחמות:
}
\newsection{דף קג}
\textblock{\textbf{ל״צ דקעביד בארעא דחברי׳.} פי׳ ול״צ ליפוי׳ והו״ל מלאכה ש״צ לגופה ופטור עלי׳ ור״ש הוא כנ״ל ועוד אכתוב דעת בעל הערוך בס״ד:
}
\textblock{\textbf{} @ק״ג ע״ב
}
\textblock{הא דאמרינן \textbf{שלא יכתוב אלפין עייני״ן עייני״ן אלפי״ן.} פי׳ רש״י ז״ל מפני שדומין בקריאתן ואחרים פי׳ שהן דומין בכתיבתן לפי שגוף האל״ף שהוא הקו האמצעי צריך לכתו׳ כפוף כגון זה __ ואם מרחיק הירך שלמטה הימנה ולא יחברנה לה ויעשה הקו האמצעי יותר כפוף מדינו כזה __ ידמה לעי״ן. וצדי״ן גמלין. נראה שפי׳ לא יפריש צדדי של צד״י מגופה שא״כ ידמה ליו״ד וגימ״ל כזה __ ואף על פי שהיא דומה יותר לנו״ן משום גמלי״ן צדי״ן אמר כן שאם יעשה ראשה של גימל כפופה כלפי מעלה עושה צד״י כזו __ ושוב ראיתי במקצת נוסחאות צדי״ן נו״ן נו״ן צדי״ן והוא הנכון כי הנו״ן יותר עלול להתדמות לצד״י והצד״י לנו״ן צד״י לנו״ן __ ונו״ן לצד״י 6) כשיהי׳ ראשה כפוף למעלה ובלבד שלא יהא מן המגיהין. טיתי״ן פיפי״ן. פי׳ שלא יפריש קו הטי״ת האחרון מגופה ותדמה לפ״א כזה __ ומכאן יש ללמוד שהיא כפופה כגון זה __ ולא ככתיבת מקצת סופרים הצרפתים שכותבין אותה פשוטה כזה __ ומדאמרינן זייני״ן נוני״ן שפי׳ נו״ן פשוטה ש״מ שהנו״ן ראשה גס ואינו כפוף כוי״ו כזה __ אלא כגון הזיין לגמרי כזה __ מדלא אמרי׳ ווי״ן נוני״ן. ודאמרי׳ לקמן מ״ט פשוטה כרעי׳ דדל״ת לגבי גימ״ל י״מ שכן דרך כתיבה אשורית המאושר׳ להיות רגלה של דל״ת נוטה לגבי ג׳ כגון זו __ ולא לגבי (עצמה) כזו 31) ולא מחוור אלא לא לגבי (גימ״ל) [דל״ת] ולא לגבי ה״א מכרעה אלא זקופה ועומדת ופי׳ פשוטה כרעה מפני שרגלה כלפי הגימ״ל. ודאמרר מ״ט מהדר אפי׳ דקו״ף מרי״ש פי׳ מ״ט אינה מראה לו פנים כגון זו __. ומדאמרינן כרעי׳ דקוף תליא ש״מ שהיא עשויה כנון ירכה של ה״א שאינה נוגעת כלל לקו העליון כזו (ק) וש״מ דתי״ו נוגעת כחי״ת דאי לא ליבעי טעמא בתי״ו כדבעי טעמ׳ בקו״ף ומדאמרי׳ שי״ן שקר אין לו רגלים ולא קאי ש״מ שהיא עשוי במדרון כגון זו __ ולא כמו שטעו מקצת סופרים לכופפה כזו __ ומפני שצריך ליזהר בכל זה בכתיבת ס״ת ותפילין ומזוזות לפיכך כתבתי דברי׳ הללו שכל הטוע׳ בציור האותיו׳ פסול והפסיל הכל (זהו דבר חדש, והפוסקי׳ כתבו דבדיעבד כשר בשאינו דומה לצורת אות אחר אף ששינה צורתה):
}
\textblock{\textbf{ומדפתוח ועשאו סתום כשר סתום ועשאו פתוח נמי כשר.} ק׳ ל״ל למימר הכי בלא״ה נמי משכחת לה כגון שהי׳ דעתו לכתוב שמעון וכתבו במ״ם סתומה שכשר הוא לשמעון ולא קושי׳ הוא דכיון דעיקר כתיבת שמעון במ״ם פתוחה והוא כ׳ במ״ם סתומ׳ ולא כ׳ אלא שם פשיטא דחייב שמתחלת כוונתו נראית יותר לכתיבת שם מכתיב׳ שמעון אבל שם במ״ם פתוחה ודאי היינו רבותא:
}
\textblock{\textbf{} @דף ק״ד ע״א
}
\textblock{הא דאמרי׳ \textbf{מנצפ״ך צופים אמרום.} פי׳ צופים ר״א ור׳ יהושע שהם חזרו ויסדום ולפיכך אמרו מנצפ״ך ולא כמנפ״ץ כסדר כתיבתן לסי׳ הצופים כלומר מן צפך כלו׳ אלו אותיות שקבלן ישראל מן צופיך:
}
\textblock{\textbf{} }
\textblock{\textbf{} ובירו׳ דמגלה מהו מנצפך ר׳ ירמי׳ בשם ר״ש בר׳ יצחק מה שתקנו לך הצופים מה ענין צופים מעשה הי׳ ביום סגריר שלא נכנסו חכמים לבית הוועד נכנסו תינוקות אמרין נעביד בית ועדה שלא יתבטל אמרין מהו דכתיב מ״ם נו״ן פ״ף צ״ץ כ״ך ממאמר למאמר מנאמן לנאמן מפה אל פה מצדיק לצדיק מכף לכף מכף ידו של הקב״ה לכף ידו של משה סימנו אותן חכמים ועמדו כולן בני אדם גדולים בתורה אמרו ר״א ור״י מנהון הוין. ומוחלפת שיטה זו מההוא דמס׳ מגילה דהכא משמע סתמא דעל הפתוחות אמרו ושם משמע דעל הסתומות אמרו אלא קאי הכא ומקשי קאי התם ומקשי כדפירש״י ז״ל:
}
\textblock{\textbf{} @ק״ד ע״ב
}
\textblock{הא דאמרינן \textbf{נתכוין לכתוב אות א׳ ועלו בידו ב׳ חייב.} ק״ל דהא אמרי׳ נתכוין לזרוק שתים וזרק ד׳ פטור וא״ל דש״ה דבכ״מ השתים נתכוין לכתוב אותה האחת אבל התם לא נתכוין לזרוק חוץ למקום השתים. ואיכא דפריק דהכא בשדעתו לכתוב שתים זה אח״ז אלא שנתכוין לכתוב אחת בתחלה ועלו בידו השתים כאחת והוא הי׳ רוצה שיכתוב אותן אחת אחת ולפיכך חייב דהו״ל כנתכוין לזרוק ד׳ וזרק ח׳ ואמר כ״מ שתרצה תנוח, זה תי׳ בעל התוספת ז״ל, ואינו נכון:
}
\textblock{\textbf{} @ק״ה ע״3
}
\textblock{ודאקשי׳ \textbf{והתניא פטור.} ק״ל, וליקשי לי׳ ממתני׳ דקתני נתכוין לכתוב חי״ת ועלו בידו שתי זייני״ן פטור וא״ל בשלמא מתני׳ ל״ק דעלו בידו שתי זיינין בלא זיוני משמע אבל בריי׳ דקתני ועלו בידו שתים משמע שתי אותיות כתיקונן ורש״י ז״ל גורס והתנן פטור אבל בנסחאות הישנות ובפי׳ ר״ח ז״ל אשכחן והא תניא:
}
\newchap{פרק \hebrewnumeral{13} האורג}
\textblock{}
\textblock{\textbf{תניא הקורע בחמתו ובאבלו על מת ואעפ״י שחלל את השבת יצא ידי קריעה.} ומקשינן עלה (בירושלמי פי״ג דשבת) בעון קומי ר׳ יוסי. לא כן א״ר יוחנן בשם ר׳ שמעון בן יוצדק. מצה נזולה אין אדם יצא בה י״ח בפסח פי׳ משום דהוה מצוה הבאה בעביר׳ אף קריעה זו שקרע בשבת מצוה הבאה בעבירה היא. ופריק אמר לון תמן גופה עבירה. ברם הכא הוא עבר עבירה. כך אנו אומרים הוציא מצה מרה״י לרה״ר יוצא בה ידי חובתו בפסח. ושמעינן מינה שאם קרע בחלוק הגזול. אינו יוצא ידי קריעה. ואין קורעין בי״ט שני של גליות אפילו קרוביו של מת:
}
\textblock{\textbf{חמתו אחמתו נמי ל״ק הא ר״י הא ר״ש.} איכא למידק אפי׳ מתו נמי מלאכה שא״צ לגופה היא וי״ל אה״נ. וי״מ לר״ש נמי כיון דצריך הוא ללבוש בגד קרוע וכל שבעה קרעו לפניו צריכה לגופה מיקרי. ותו קשי׳ אי ר״ש היכי קתני כל המקלקלין פטורין והא לר׳ חוץ מחובל ומבעיר הו״ל למיתני וא״ל בדין הוא דהו״ל לאקשויי הכי אלא דעדיפא מיני׳ אקשי לי׳ ומיהו לפום תירוצא דמסקנא דאוקי׳ דעביד למירמא אימתא לאינשי ביתי׳ תרוייהו ר״י נינהו ומתני׳ בדלא רמי אימתא ובריית׳ בדרמי אימתא ועל מת דעלמא אפי׳ לר״י פטור דקלקול גמור הוא (א״נ) בדקא עביד דרך כבוד חיים דדמי לרמי אימת׳ אלא דקעביד דרך חמה וארחי דחמתו של אדם קתני ור״ח ורש״י ז״ל שניהן כתבו רישא ר״ש וסיפא ר״י ואינו מחוור לי דהא חד בבא הוא:
}
\newsection{דף קו}
\textblock{\textbf{כל המקלקלין פטורין חוץ מחובל ומבעיר.} אוקי׳ לר״ש דאמר מקלקל בחבורה חייב ולזה המחלוקת לא מצינו עיקר בתלמוד ורש״י ז״ל הביאו בקשר ענין ואמר דנפקא לן מפלוגתייהו במלאכה שא״צ לגופה דשמעי׳ לי׳ לר״ש דאמר מלאכה שא״צ לגופה פטור עלי׳ וחובל ומבעיר חידוש הוא שחדשה התורה בהן שאין לך מבעיר ואפי׳ מדליק עצים לבשל קדרתו שיתחייב מן הדין שכולן מקלקלין ומה שהוא מתקן אצל אחרים כגון הבישול מלאכה שא״צ לגופה היא ומן הדין פטור עלי׳ ומדחייבה אותן תורה ש״מ מקלקל בהבערה חייב וכן בחבלה מדאיצטרך רחמנא למישרי מילה ולר״י התם מתקן הוא אצל אחרים ומלאכה שא״צ לגופה חייב עלי׳ והיינו דמפרש בגמ׳ ואזיל מ״ט דר״ש מהאי טעמא גופי׳ גמרי׳ פלוגתייהו זהו פירוש רש״י ז״ל. ומזה הפירוש נתחייב הר״ר שמואל תלמידו ז״ל לומר דהא דאמר ר״י ואת״ל חובל בצריך לכלבו ה״ק לי׳ אין הלכה כר״ש אלא כר״י דאלו ר״ש לא בעי תיקון כלל דמנא תייתי לה וא״ת נפקא לן ממילה תיקון דמילה מאי הוא אי חשבית לי׳ תיקון דגופה א״כ כל טעמא של ר״ש בטל מעתה והרי חזר לדעתו של ר״י ואי לאו תיקון דגופה אלא במלאכה שא״צ לגופה היא א״כ לאו תיקון הוא כלל לדברי ר״ש וא״ת דמודה ר״ש בתיקון דחובל ומבעיר שאף על פי שאינה תיקון דגופה חיובי מיחייב א״כ היינו ר״י דס״ל דמקלקל בחבורה חייב ולמה אמרי׳ בכולה תלמודא במקלקל בחבורה פלוגתא דר״י ור״ש אלא ה״פ חובל ומבעיר אינה משנה שיתחייב יותר משאר המקלקלין דכר״י קי״ל ואת״ל משנה חובל בצריך לכלבו מבעיר בצריך לאפרו והוא מתקן גמור ורש״י עצמו ז״ל כ׳ דחובל וצריך לכלבו לר״י כיון דמתקן הוא אצל הכלב כה״ג מלאכה הוא וזה הפירוש אינו נכון דאי ר׳ אבוהו כר״ש היכי משתיק לי׳ ר׳ יוחנן כדמקשי׳ בפ״ק חולין ורב משום דס״ל כר״י מאן דתני כר״מ שתוקי משתיק לי׳ ועוד דלר״ש אפילו בחובל ומבעיר בעי׳ שתהא המלאכה צריכה לגופה כדאמרי׳ בפרק כל כתבי גבי חמשה נהרגין בשבת אלא לאו ר״ש דאמר מלאכה שא״צ לגופה פטור עלי׳ אלמא אפי׳ במקלקל בחבורה בעי׳ מלאכה הצריכה לגופה ובפ׳ ואלו הן הנחנקין אמרי׳ והא דתנן מחט של יד ליטול בה את הקוץ ליחוש דילמא חביל בי׳ ומתרץ התם אי לר״י מקלקל בחבורה אי לר״ש מלאכה שא״צ לגופה היא ופטור עלי׳ אלמא לר״ש מלאכה שא״צ לגופה בחבורה פטור וכן זו שאמר רש״י ז״ל עצמו דאין לך מבעיר שאינו מקלקל ואפי׳ מבעיר עצים לבשל קדירה אינו נכון שהרי גופה של אש צריך לו וכן במבעיר להתחמם שהנאתו וביעורו שוין והחורש ותולדותי׳ אין הנאתן בעצמן ממש אלא שהן תשמישו של עולם להנאת דברים אחרים וכה״ג מלאכה הצריכה לגופה מיקרי והרי בגמרא מפורש שאין טעמו של ר״ש מהבערה סתם אלא מהבערה דבת כהן ולא מחבלה דשוחט אלא ממילה לפיכך אמרו דהא דאמרי׳ במחלוקת ר״י ור״ש בחובל ומבעיר גמרא גמירי לה כי הא דאמרי׳ לקמן בפ׳ חביות רבי יוחנן ההוא כבן בינאי מתני לה ולא שמעי׳ לי׳ לבן בינאי אלא גמרא ואמרי׳ בב״ב חרוב המורכב וסדן השקמה מחלוקת ר׳ מנחם בר׳ יוסי ורבנן ולא מצינו בשום מקום מחלוקת זו אלא דגמירי רבנן דגמרא הכי ואחרים כיוצא בזה יש הרבה או שמא בריי׳ היא באחת מן החיצונות כגון משנת בר קפרא ולוי ושאר התנאים. זה כתוב בתוס׳ חכמי הצרפתים ז״ל. ואין אני אומר כן, דמילה ודאי מלאכה הצריכה לגופה הוא שהוא צריך שיהא אדם זה מהול ומהדומה לזה הנוטל צפרני׳ בכלי וכן שפמו שהוא חייב משום תולדה דגוזז ואף על פי שאינו צריך לגוף הצפרנים והשיער מלאכה הצריכה לגופ׳ נקראת וכן הזורה והבורר ומאן דשקיל איקופי מגלימא כולן דומות למילה והבערה דבת כהן נמי כיון שהוא מדליק האש לבשל בו פתילה הרי היא מלאכה הצריכה לגופה דומיא דמדליק את האור להתחמם כנגדו ומשתמש לאורה ולא שמענו מלאכה שא״צ לגופה במדליק אלא במכבה אלא ודאי תרוייהו מלאכה הצריכה לגופה נינהו אלא שהצורך עצמו נקרא קלקול בגופו שהנימול נשאר חבול וחולה והבערה של בת כהן כיון שאין הנאה לאדם ממנה אינו אלא כמקלקל בעצים אלא שאדם צריך לזו המלאכה גופה מ״מ ויש בה תיקון שאינה מגופה של מלאכה שהיא קיום המצוה ולר״ש אין אותו התיקון כלום הלכך לר״ש כל מלאכה שא״צ לא מחמת גופה ולא מחמת תיקון פטור עלי׳ אפילו בחובל ומבעיר והיינו נוטל קוצו וחבל בעצמו דמקלקל גמור וא״צ לכלום היינו מלאכה שא״צ לגופה ומ״ה בעי ר״ש חובל וצריך לכלבו מבעיר וצריך לאפרו כדי שתהא המלאכה צריכה לגופה לא משום דבעי שום תיקון וה״ק לי׳ רבי יוחנן חובל ומבעיר אינה משנה שיתחייב בדרך מקלקל מפני שהוא מלאכה שאין צריך לגופה ופטור עלי׳ ואם תמצי לומר משנה חובל בצריך לכלבו ומבעיר בצריך לאפרו. וא״ת והרי אף על פי כן מלאכה שא״צ לגופה היא דומי׳ דחופר גומא וא״צ אלא לעפרה לא תטעה בזה שהחבלה בכאן היא הדם עצמו שהוא צריך וכבר רמזתי לגוזז ונוטל שערו וצפרנו בין לתקן גופו ולייפות עצמו ובין לצורך השער והגיזה כולן מלאכה הצריכה לגופה הוא מפני שהכל דבר א׳ והנטילה היא המלאכה משא״כ בחופר גומא שהבנין בקרקע הוא והוא הנקרא מלאכה וכן הבערה לאפר צריך לגופו. וכאן הבן שואל, בין לדברינו בין לדברי רבותינו הצרפתים ז״ל צריך לכלבו וצריך לאפרו מתקן גמור הוא ואמאי לא מיחייב לכ״ע והשיבו בזה שכל מה שאין דרכן של בני אדם לקלקל בשביל אותו התיקון אין זה מתקן אצל שבת והטעם משום דמלאכת מחשבת בעינן ואין זו מלאכת מחשבת שיהא חובל באדם לצורך הכלב וידליק גדיש לצורך האפר אבל שוחט ומדליק עצים לבשל בהן מתקן הוא אצל שבת לד״ה דהא תנן באבות מלאכות מבעיר ומדמוסיף ר״י שובט ומדקדק ש״מ מודה בכולהו ואף על גב דלענין ע״ז אמרי׳ בפ״ק דחולין סכין של ע״ז מותר לשחוט בה משום דמקלקל הוא גבי שבת מיהו מחייב. ורבינו תם ז״ל מפרש דהא דחשיב מקלקל אפילו צריך לכלכו וצריך לאפרו לפי שאין התיקון בא בשעת הקלקול דלאחר שנעשית החבורה בא הדם ולאחר שנעשית הבערה בא האפר. ואין טעם זה מחוור ולא מסכים לדעתינו. ואשוב לדברי לפרש טעמייהו דרבי אבוהו ורבי יוחנן הוי יודע דטעמא דר״ש דפטר מלאכה שא״צ לגופה משום מלאכת מחשב׳ דטעמא דכל המקלקלין פטורין נמי משום דבעי׳ מלאכת מחשבת כדמוכח בפרק קמא דחגיגה משום הכי סבר ר׳ אבוהו למימר הרי למדנו לר״ש דמילה קלקול הוא לגופו של נימול ותיקון הוא ודאי אצל אחרים שעושה נחת רוח ליוצרו ודרכן של ב״א לעשות כן אלא כיון שאין התיקון ממש בגופו כתיקון (שהוא) מגופה של מלאכה מתחשב ונמצינו מחייבי׳ קלקול בחבורה לר״ש הלכך אפי׳ בשא״צ לגופה נמי מחייב דהיינו מקלקל גמור בלא שום תיקון וכוונה א״נ נוטל את הקוץ וחבל דהא גלי רחמנא דלא בעי׳ בהא מלאכת מחשבת וה״נ אמרי׳ כה״ג במס׳ כריתות הנח לחבור׳ הואיל ומקלקל חייב מתעסק נמי חייב דלא קפדי׳ בחבורה במלאכת מחשבת דמתעסק משום מלאכת מחשבת היא פטור. ר׳ יוחנן סבר מילה והבערה דבת כהן תרווייהו מלאכה הצריכה לגופה נינהו שהרי דעתו שיהא גופו נימול ודעתו להשתמש באש הזה שהדליק ומלאכת מחשבת קרינן בהו ואין דעתו להזיק ואין לך אלא חידושו בלבד תיקון לא בעי׳ להו כלל אבל מלאכה הצריכה לגופה בעי׳ הילכך שאר חובלין ומבעירין להזיק כגון שהדליק גדיש בשבת שאין לו בגופו של מלאכה שום תשמיש פטור. והא דאמרי׳ בפ׳ כל כתבי ובפ׳ אלו הן הנחנקין דמלאכה הצריכה לגופה בעי׳ כדכתב׳ לעיל כולהו אליבא דר׳ יוחנן אתיא דלר׳ אבוהו לא בעי׳ מלאכה הצריכה לגופה. עי״ל דלר׳ אבוהו מלאכה הצריכה לגופה בעי דומיא דמילה ומ״ה פטרי׳ נוטל את הקוץ מפני שהוא מתכוין לקוץ ולא לחבורה אבל סבר ר״א דחובל בחבירו מאחר שהוא מתכוין לאותו קלקול ולא לדבר אחר ונמצינו שקלקול בחובל חייב דמלאכה הצריכה לגופה מיקרי שהרי הוא צריך לקלקול והתורה עשתה מקלקל כמתקן בחבורה ובמילה עצמה לא חייבה אותו התורה אלא מפני הקלקול והרי החופר גומא כדי שיכשלו בה ב״א ודאי צריך לגופה מיקרי אלא שהוא קלקול ופטור הלכך בחובל שמקלקל בו חייב ואלו היה צריך לדם כאותה שאמרו בריש כתוב׳ אי דם חבורי מחבר חייב דלדם הוא צריך. א״נ ר״א לא פליג מידי ברייתא קא תני ורבי יוחנן פירשה ניהליה דצריך לגופה מיהא בעי׳ ומתכוין לקלקל לא כלום הוא הלכך מלאכה שא״צ לגופה הוא עד שיהא לו תשמיש בקלקולו שמחמתו הוא עושה זה הקלקול שנמצא קלקול חבלה זו מלאכה הצריכה לגופה מצד א׳ ואף על פי שאין צרכו כנגד קלקולו בדברים שבני אדם עושין כן כמו שפירשתי זהו מה שהעליתי בשמועה זו מתוך פלפולו של רש״י ז״ל ואין בדבריו ממש ודבר נכון הוא ועולה כהוגן. וא״ת ומה ראו לומר כן לר״ש בחובל ומבעיר מחייב במקלקל משום דלא חשיב תיקון דמצוה אדרבה ליגמר מינייהו מלאכה שא״צ לגופה חייב עליה וממילא הו״ל מתקן כדרבי יהודה י״ל מלאכה שא״צ לגופה מקרא מלא הוא מלאכת מחשבת וחידוש הוא שחדשה תורה באלו שתיהן. והא דאמרי׳ בטעמא דר״י מה לי לתקן מיל׳ מ״ל לתקן כלי. לפי עניינו ודברי רש״י ז״ל צ״ל דה״פ מ״ל לתקן כלי מ״ל לתקן מילה כיון שהתור׳ פסלה בנו ערלה ה״ז תיקון במצו׳ מה לי לבשל סממנין מה לי לבשל פתילה הרי אנו צריכים לבישול הזה לשם מצוה וצרכנו זה כצורך הנאה של בישול אוכלין שאם תפרש דלרש״י תיקון נמי דגופו משוי לי׳ הרי כל הפי׳ שלנו בטל דמנל״ל דר״ש הוא לא חשיב תיקון דגופו דאי לא ברייתא אמאן תרמייה אכתי לר״י ל״ל לשויי תיקון דגופו לימא משום דמצוה הוא ולפ״ד הללו הא דאמרי׳ במס׳ פסחים בפ׳ א״ד כי אזלת לקמי׳ דר׳ זריקא בעי מיני׳ לדברי האומר מקלקל בחבורה פטור שוחט שלא לאוכליו מאי תיקן להוציא אבר מן החי דוקא לר׳ יהודה פריך דאמר מקלקל בחבורה פטור דלר״ש דאמר חייב אע״פ שאין בו שום תיקון נמי מחייב בין לר׳ אבוהו בין לר׳ יוחנן דהתם מלאכה הצריכה לגופה הוא שהוא שוגג וסבור לשחוט יפה ומלאכת מחשבת קרינא בי׳ וזה סיוע לדברי. ולדברי רבותינו הצרפתים ז״ל דלר׳ יוחנן בעי ואליבא דר״ש הוא דקבעי והוא ניהו מ״ד מקלקל בחבורה פטור והביאו ראי׳ לדברי׳ משום דההוא מתני׳ דהתם דמחייבא שוחט שלא לאוכליו לאו ר׳ יהודה היא דקתני רישא ושאר כל הזבחים ששחטן לשם פסח אם אינן ראוין חייב ואוקימנא בגמרא בטועה והיינו מתעסק אלא למאן דמחייב מקלקל בחבורה. ואין זה נכון דמ״ד מקלקל פטור היינו ר׳ יהודה ועליו אמרו כן בכ״מ ובכמה מקומות אמרו לר״ש דאמר מקלקל בחבורה חייב ודקא קשי׳ להו רישא דלא כר״י ניהו דרישא דמתני׳ לאו ר״י הוא מ״מ יכול הוא לישאל מה תיקן שלא מצינו מי שחולק בזה לפטור. ועוד שאינו אלא שאלה ואלו לא מחייב מ״ד מקלקל בחבורה פטור בשוחט שלא לאוכליו הי׳ יכול להשיב מאן שמעת לי׳ דאמר הכי ר״י לר״י ה״נ דפטור ועוד יש חילוק גדול בין מתעסק דשני תינוקות לשאר הזבחים. ובירושלמי בר קפרא אמר אפי׳ א״צ לדם ולאפר ר״י אמר והוא שיהי׳ צריך לדם ואפר מתני׳ פליגא על ר״י שורו שהדליק גדיש בשבת חייב בתשלומין הוא שהדליק גדיש בשבת פטור מן תשלומין לפי שמתחייב בנפשו שורו שלא לצורך אף הוא שלא לצורך ומ״מ מתחייב בנפשו אמר רב חנינא ברי׳ דרב הלל ויאות מאחר שאלו לצורך היה מתחייב בנפשו פטור מתשלומין מן הדא מכה בהמה ישלמנה מכה אדם יומת מה מכה בהמה לא חלקת בין שוגג למזיד אף מכה אדם לא חלקת לחייבו אלא לפוטרו ממון וקשי׳ על בר קפרא (וכו׳) (אי) סבר כרבי יוסי דהבערה ללאו יצאת על עצמה, שאינה במיתה. ולית לי׳ חילוק מלאכות בתמי׳ אשכח תנא דדריש מאחת מהנה לחייב על כל אחת ומזה ילפינן חילוק מלאכות ולא מהבערה. ודניחא הבערה, חבורה, אב לצורך ותולדה שלא לצורך, מחליף פי׳ חבורה אב ושוחט תולדה ולמה תנינן עמהון פי׳ דתני שוחט בין האבות משום סדר סעודה תני עמהן פי׳ סדר סעוד׳ תני (אופ׳ חורש) [מחורש עד אופה] תני ג״כ שוחט וה״ל למיתני חובל. והוצרכנו לכתוב כ״ז כדי שלא יתקשה אדם בו על דברינו:
}
\textblock{\textbf{ואין נותנין לפניהם מזונות.} פירש רש״י ז״ל כיון דמוקצין הן לא מצי למיטרח עלייהו. ובתוספ׳ מקשו עלי׳ דהא תנן (קנו:) מחתכין את הדלועין לפני הבהמה ואת הנבלה לפני הכלבים ולקמן נמי בפרק מפנין אמרינן דמטלטלין אף הלוף מפני שהוא מאכל לעורבים ומאן דאסר נמי התם לא אסר אלא משום שאין דרכן של ישראל לגדל עורבים. ומפרשי׳ טעמא שמא יצוד אותן אבל חי׳ ועוף שאין מחוסרין צידה אף על פי שהן מוקצין שרי שהרי מזונותן עליך ולא דמי ליוני שובך וליוני עליי׳ שאסור משום שאין מזונותן עליך כדאי׳ בפ׳ מי שהחשיך (שבת קנה:). ומייתי סייעתא להאי פירושא מסוגי׳ דפ׳ אין צדין דאמרי׳ ה״ד מחוסר צידה אר״י אמר שמואל כל שאומר הבא מצודה ונצודנו א״ל ר״י והא אווזין ותרנגולין שאומר הבא מצודה ונצודם ותניא הצד אווזין ותרנגולי ויוני הדרסיאות פטור ומאי קושיא הוא פטור אבל אסור הוא דכל פטורי דשבת פטור אבל אסור הוא וגבי שבת תניא אלא ה״פ דמתני׳ קתני כל שמחוסר צידה אסור ליתן לפניהם מזונות משום גזירה שמא יצוד אלמא צידה זו דאורייתא הוא והתניא פטור וכיון דפטור היכי גזרי׳ משום שמא יצוד. אבל בירושל׳ (יג,ז) מפורש לפי שאין עושין תקנה לדבר שאינו מן המוכן ודקא קשי׳ להו כלבים ועורבים ש״ה לפי שהן מוכנין לצרכן ומזומנין ונותן דעתו עליהן מאתמול אבל דגים שבביברים וכן חיה ועוף לפי שאינן מזומנים לאדם ואינן ניצודין לו אסור ליתן להם מזונות אבל הניצודים כגון בביבר קטן מתוך שדעתו של אדם עליהן כמצויין לך דמו ומותר שאלו תרנגולת העומדת לגדל בצים אע״פ שמוקצת נותנין מזונות לה. ומיהו קשה לרב יוסף דשרי למשדי קמייהו דיוני שובך ויוני עליי׳ אע״פ שהצדן חייב ואיןמזונותן עליך למה אין נותנין מזונות לפני דגים וי״ל לרב יוסף משום דשכיחי להו מזונות בביברין כדאמרי׳ בפ׳ מי שהחשיך שאני מיא דשכיחי וכן חיה ועוף בביברין מזונותן מצוי להו ולהכי קתני בבריית׳ ואין נותנין לפניהן מזונות ומיהו במתני׳ כיון דצדין תנן אע״פ שמזונותן מצויין להן נותנין לפניהן דכל שברשותך מותר אפי׳ לפטם שא״א לומר בתרנגולת העומדת באשפה שלא נתן לה מזונות אע״פ שמנקרת ואוכלת והיינו שיטתיה דרב יוסף בפ׳ מי שהחשיך אבל לר״י התם כל שאינו ברשותינו אין מזונותם עלינו ואסור לגמרי. וההוא דמקשי׳ בפ׳ אין צדין ממתני׳ דקתני פטור אמתני׳ דקתני אסור משום דמתני׳ [דקתני אסור] חייב הוא מדקא פטר ליה גבי שבת לצד לתוכו דהני תרי משגיות בנות חד בקיתא נינהו ועוד דכיון דבשבת פטור דאלמא ניצודין ועומדין הן גבי י״ט מותר לכתחלה דהא צריך לאוכל נפש אבל גבי שבת למאי צייד להו:
}
\textblock{הא דתנן \textbf{למה זה דומה לנועל את ביתו לשמרו ונמצא צבי שמור בתוכו.} נ״ל דה״פ שהיה צבי שמור בתוך הבית והדלת מגופה (ר״ל סגורה כמו הגפת דלתות דיומא) והוא שמור בכך ובא זה ונעל במנעל כדי שלא יבוא שום אדם ויפתח הבית שפטור ומותר אע״פ שהוא מוסיף לו שמירה שעכשיו אין אדם יכול לפתחה אף בישב הראשון על הפתח ובא שני וישב בצדו אע״פ שהוא מוסיף שמירה על שמירתו שאם עמד הראשון והלך לו נמצא שמור מכחו השני פטור. עוד יש לי לאומרה כגון שהיה צבי קשור בבית ושמור ונעל ביתו לשמרו אע״פ שנתק הצבי לאחר כן הקשר ונמצא משתמר בנעילתו הואיל ושמור היה ונצוד פטור הנועל. אבל בתוס׳ אמרו שלא אמרה משנתינו למה זה דומה שיהיו שוין בענינן אלא שוין הן בדינן כלומר שהוא מותר לגמרי ואינו נכון ולכל הנך פירושי הא דקתני נועל את ביתו לשמרו ל״ד אלא אפי׳ לשמור הצבי מותר שהרי שמור הוא. אבל בתוס׳ ובירושל׳ משמע ענין אחר דגרסי׳ התם (ירו׳ יג,ו) ר׳ יוסי בר בון בשם רב הונא אמר היה צבי רץ כדרכו ונתכוון לנעול בעדו ונעל בעדו ובעד הצבי מותר. ובתוספתא (יג,ו)תניא ישב א׳ על הפתח ונמצא צבי שמור בתוכו אע״פ שמתכוין לישב עד שתחשך פטור מפני שקדמה צידה למחשבה למה זה דומה לנועל את המגדל ונמצא צבי שמור בתוכה ולמתכסה בטלית ונמצאת צפורת בתוכה אע״פ שמתכוין לישב עד שתחשך פטור מפני שקדמה צידה למחשבה ומתני׳ נמי הכי קתני שנתכוין לנעול את ביתו לשמרו ולא נודע לו כלל שיהי׳ שם צבי דבשעת צידה ודאי דומה לנתכוין להגביה תלוש וחתך מחובר הוא ואנוס נמי הוא וקתני פטור ומותר לעשות כן שהרי הוא שומר מאליו ואעפ״י שנתכוין שלא לפתתה עד הערב או משום שלא יפתחנה אחר מותר אבל ביושב ומשמר בפתח קס״ד מעיקרא לאסור משום דדמי לצד ומסקנא שהכל מותר הואיל וקדמה צידה למחשבה:
}
\newsection{דף קז}
\textblock{מתני׳. \textbf{ח׳ שרצים כו׳ הצדן והחובל בהן חייב.} כבר מפורש בתוס׳ מאי חובל שנצרר הדם ולא יצא לפיכך ח׳ שרצים שיש להן עורות חייב שאין החבורה חוזרת אלא שהעור מעכב את הדם שלא יצא אבל שאר שקצים ורמשים פטור בנצרר הדם לפי שלא נעקר לגמרי ועתיד הוא לחזור שאלמלא נעקר לגמרי היה יוצא לחוץ שאין כאן עור המעכבו ואיסור חבלה הוא משום נטילת נשמה כי הדם הוא הנפש. אבל רש״י ז״ל שפי׳ משום צובע לא מחוור טעמי׳ דבפ׳ אלו טרפות מוכח דיצא הדם לחוץ אפי׳ בשאר שקצים ורמשי׳ חייב ושם פי׳ שאם נצרר הדם אע״פ שלא יצא משום (דהדרא) [דלא הדרא] ברי׳. אבל בירושל׳ פ׳ כלל גדול מצאתי מה צביעה היתה במשכן שהיו משרבטין בבהמה ועורות אלים מאדמים הדא אמרה העושה חבורה ונצרר הדם חייב משום צובע המאדים אודם (בטפה) [בשפה] חייב והמוציא דם חייב משום נטילת נשמה שבאותו מקום ע״כ ומיהו לא ידענא צובע במאי ניחא ליה בשקצים ורמשים וכן במילה למאי מבעי ליה צבע וכן מציצה דאמרי׳ בפ׳ ר״א דחילול שבת ואינו צובע כלום אלא נטילת נשמה יש בה. אלא כל חבורה משום נטילת נשמה הוא ובמקום שהוא רוצה בצביעתה איכא נמי משום צובע כשוחט גופיה דאית בה תרתי כדאי׳ בפ׳ כלל גדול. וק״ל פוצע חלזון ומוציא דמו ליחייב משום נטילת נשמה שבאותו מקום אע״ג דניחא לי׳ דלא לימות ממש וליכא לפרושי דפציעת חלזון אינו מוציא דם אלא נצרר הוא ופטור דאין לו עור שאם אינו מוציא דם האיך הוא בכלל דישה לר״י:
}
\textblock{\textbf{והצדן לצורך חייב ושלא לצורך פטור.} פרש״י ז״ל דאין במינו נצוד והיא מלאכה שא״צ לגופה ור״ש הוא כדמוקי לה בגמ׳ ואין אני יודע מה הוא שח שאם אין במינו ניצוד אפי׳ לצורך פטור אלא משמע דח׳ שרצים האמורים בתורה אין דרכן להזיק וכל צידתך לצורך הוא אבל שאר שקצים ורמשים שדרכן להזיק ופעמים אדם צדן שלא לצורך כלומר כדי שלא יזיקו ומ״ה תני פלוגת׳ בסיפא וה״ה לרישא:
}
\newchap{פרק \hebrewnumeral{14} שמנה שרצים}
\textblock{}
\textblock{גמ׳: \textbf{הא הורגן חייב מאן תנא א״ר ירמיה ר״א הוא.} פי׳ מדקתני רמשים משמע דחייב משום נטילת נשמה בכל דבר שהוא חי ואלו לרבנן דפטרי בכינה בכל דבר שאין בו גידין ועצמות נמי פטרי משום שאינו מתקיים י״ב חדש והו״ל כמת וגמרי׳ נמי מאלים מאדמים מה התם יש בהן גידין ועצמות אף כל שיש בהן גידין ועצמות והכי איתא בירושל׳ פ״ק ואתא רב יוסף ופליג ואמר דלא גמירי רבנן מאלים מאדמים אלא דבעי פרה ורבה אבל גידין ועצמות לא בעי׳ הלכך ברמשים שהן פרין ורבין חייבין לד״ה:
}
\textblock{הא דאמרי׳ \textbf{במושיט ידו למעי בהמה ודלדל עובר שבמעיה דחייב משום עוקר דבר מגדולו.} ק״ל דהא גוזז ותולש כנף מן העוף כשהן חיין לא מחייבינן להו תרתי חדא משום עוקר דבר מגידולו וכן הנוטל שערו וצפרניו ושפמו אין בהן משום דין עוקר דבר מגדולו דליחייב תרתי אלמא אין משום עוקר דבר מגדולו דהוא תולדה דקוצר אלא בגדולי קרקע וה״נ משמע בבכורות גבי תולש צמר מבכור דתולש לאו היינו גוזז וכנגדו בי״ט מותר דלית ביה משום עוקר דבר מגדולו ועוד דודאי קוצר ודש תרוייהו בחד גוונא גמרי׳ להו ממשכן כי היכי דאמרי רבנן אין דישה אלא בגדולי קרקע ה״נ אמרי׳ ודאי אין קצירה אלא בגדולי קרקע ואפי׳ ר״י אפשר דמודה בקצירה דהוא ממש מן הקרקע ותולש מבעלי חיים בכלל גוזז הוא. וי״ל דמדלדל עובר, משום נטילת נשמה הוא חייב וה״ק אע״ג דהאי עובר לית בי׳ בדידי׳ נשמה כיון דגידולו תלוי בנשמת אמו העוקרו משם נוטל נשמתו ממנו דלאו מי א״ר ששת בגידולי קרקע כן דמאן דתלש כשותא מהיזמי שיניקתו תלוי בהיזמי כדאמרינן בעירובין פ׳ בכל מערבין דקטלי לי׳ להיזמתא ויבשה כשותא ומחייבו משום עוקר דבר מגידולו דהוא תולדה דקוצר ה״נ בבעלי חיים מחייב משום עוקר דבר מגידולו דהוא משום נטילת נשמה דומיא דחובל דהוא נטילת נשמה מאבר א׳ וסירכא דלישני׳ נקט וכ״נ הפי׳ הזה מדברי ה״ר משה הספרדי ז״ל בפ״א מהלכותיו. ומ״ש בירושל׳ בפ׳ כלל גדול רבנן דקסרין אמרין ההוא מאן דצד נונא וכל דבר שמבדילו מחיותו חייב משום קוצר לא אתי׳ כשיטתא דגמ׳ דילן:
}
\newsection{דף קח}
\textblock{\textbf{דקא דלי מיא ועכירי.} פר״ח ז״ל דשמואל שמע דגברא רבא אתי ולפי שראה המים עכורים חשב בלבו ששתה מהם והיה יודע בדרך חכמת הרפואות שהמים ההם העכורים משלשלין הבטן ולפיכך אמר וחש במעי׳ ול״ג ר״ח וקא דלו מיא לאקבולי אפי׳ ואפשר דה״ג גברא רבא אתיא וחש במעי׳ דקא דלו מיא זיל לאקבולי אפי׳, ותהי לי׳ בקנקני׳ והוא דרך משל לבודקו לפי שזה היה אומנתו של קרנא לבדוק היין שבקנקנים כמו שמפורש בפ׳ ב׳ דייני גזרות (כתובות קה.):
}
\newsection{דף קט}
\textblock{\textbf{עלין אין בהן משום רפואה.} פר״ח ז״ל אם נתן עלין של ירק וכיוצא בהן בעין לצנן מותר והטעם שנראה כמיקר ויותר הוא נכון מדברי רש״י ז״ל שפי׳ לאכילה ומאי קמ״ל מתני׳ הוא כל האוכלין אדם אוכל ועוד דמשמע מדאמר רב ששת גרגירא אפי׳ לדידי מעלי ליה שאסור לאכלו בשבת ואמאי והתנן כל האוכלין אדם אוכל לרפואה ואל יעלה על הדעת לומר שהגרגיר אינו ראוי לאכילה ואינו בכלל אוכלין דהא אמרי׳ במס׳ סוכה הפיגם והגרגר פטורין מן המעשר לפי שאין משתמרין ואם אינו אוכל תיפוק ליה משום אוכל:
}
\textblock{\textbf{שריקא טויא.} פרש״י ז״ל גדי צלי שמחפין אותו בבצים ומותר ובלבד שלא יהא רותח כדי לבשל ומשום תקון אוכלין נגעו בה ואינו נכון לפי שאין זה תיקון שיאסר ופשיטא דמותר והאיך ר״ח בר אבא אוסר. ועוד שאין הדמיון דיין צלול ומים צלולין לתוך המשמרת דומה יפה ועוד דדבר הלמד מעניינו הוא שדרך הרפואה. לפיכך נראה כפר״ח ז״ל שהוא מי אבטיחין שמסננין אותן במשמרת לנקותן ועבדינן לי׳ לשלשולי וכיון דמתאכל האבטיח עם בני מעיו בלא סנון מותר ולשון שריקא טוויא נראה שהוא הענין הידוע שעושין בדלעת היונית הטפולה בבצק ונצלת באור:
}
\textblock{\textbf{ופעפועי ביעי.} שם עשב הוא כדאמרי׳ בעירובין מערבין בפעפועים (בפ׳ בכל מערבין) ובמשניות הפעפועין ופי׳ בירושל׳ בסוף מס׳ פיאה הקולי. ורש״י ז״ל כ׳ שם במס׳ עירובין שהם ירקות ששמן יוטל״ש בלעז:
}
\textblock{\textbf{[אבל לא בים הגדול ולא במי משרה ולא בימה של סדום.]} מדקתני לים הגדול בהדי מי משרה וימה של סדום והנהו אוקים בדאישתהי ש״מ דים הגדול נמי מותר בדלא אשתהי פי׳ ברעים שבו וביפים שבו מותר לשהות ומאן דפליג בהו לא דק ומי טבריא לעולם מותר וכי קתני הך ברייתא אבל נשתהי אסור אמי משרה וימה של סדום:
}
\textblock{\textbf{הני מילי היכא דקא מכוין הכא ממילא הוא.} ואסיקנא דאפ״ה אסור. א״ד דאתיא כר״י דאמר דבר שאין מתכוין אסור אבל לר״ש ה״נ דשרי ולא מחוור דא״ה מאי קושי׳ דילמא כי אמרי׳ לר״ש ורב אחא משבחא גאון ז״ל כ׳ בשאלתא דאמור אל הכהנים דאע״ג דדבר שאין מתכוין הוא אסור דכי קיי״ל כר״ש ה״מ באיסורי שבת אבל בשאר איסורין הלכה כר״י. ורבותינו הצרפתים ז״ל השיבו עליו מדאמרי׳ לעיל ומי א״ר יוחנן הכי והאר״י הל׳ כסתם משנה ותנן נזיר חופף ומפספס אבל לא שורק אלמא ר׳ יוחנן כר״ש ס״ל ועוד מדפרכי׳ בפ׳ ר״א דמילה ל״ל קרא דבר שאין מתכוין הוא ועוד מדגרסי׳ במס׳ בכורות בפ׳ יש בכור לנחלה מנין לבכור שאחזו דם שמותר להקיז אותו וכו׳ אמר שמואל הלכה כר״ש ופרכי׳ הי ר״ש אילימא ר״ש דמתני׳ אטו עד השתא לא אשמעי׳ שמואל דדבר שאין מתכוין מותר אלמא הלכה כר״ש בכל איסורי אלא י״ל הכא פסיק רישי׳ ולא ימות הוא ואפ״ה קס״ד מעיקרא למישרא הואיל וממילא הוא שלא אסרה תורה [אלא] כורת [ו]נותק גיד או בצים:
}
\textblock{הא דאמרי׳ \textbf{אלא בסריס והאר״י א״ר חייא הכל מודים במסרס וכו׳.} תמיה לי בשלמא התם בדקא עביד מעשה כגון נותק אחר כורת ומחמץ אחר מחמץ נמי רחמנא קפיד אמחמץ ואלש ומקטף ואופה אלא סריס ששותה כוס של עיקרין מאי קעביד וא״ל הכא בסריס חמה שאינו מוליד אבל מתאוה ובועל וכשהוא שותה כוס של עיקרין הללו מיעקר לגמרי ואינו מתקשה כלל ואינו יכול להיות נזקק לנשים וזה אסור לפי שהקפיד׳ תורה אתוספת סירוס. וא״ת מנ״ל דשרינן בכה״ג לא נוקי לי׳ בהכי ולא תיקשי א״ל מדקאמרי׳ ומיעקר אלמא בשיש בה תוספת עיקור וסירוס קאמר והוה יכלי׳ למימר דהא דקאמר ומיעקר לאסורא לומר דלא לישתי אלא סריס גמור אלא ניחא ליה לאשכוחי בשריותא. ומסקנא אוקי׳ באשה זקנה א״נ עקרה ואע״פ שיש בה תוספת סירוס שהיא מצטננת ואינה נזקקת שוב לבעל אפי״ה באשה שרי שאין בה דין סירוס ואפשר דכיון דחייבה תורה על נותק אחר כורת אע״פ שהוא סריס גמור ואינו מוסיף כלום כיון שעשה מעשה סירוס אסור אף זה כן:
}
\newsection{דף קיא}
\textblock{\textbf{האי מסוכרייתא דנזייתא אסור להדוקה ביומא טבא.} פרש״י ז״ל בגד שכורכין בברזא ואסור משום סחיטה כהאי דאמרי׳ בפ׳ תולין לא ליהדוק איניש אודרא אפומא דשישא דלמא אתי לידי סחיטה. ור״נ בעל הערוך כ׳ שכל מלאכה שאין אדם עושה אותה להנאתו ואינו נהנה בה אע״פ שהיא פסיק רישי׳ ולא ימות מותר לר״ש דדבר שאינו מתכוין מקרי וכשהודה ר״ש לא הודה אלא באומר אפסיק רישי׳ לצרכו ולא ימות והוא נהנה בפסיקת הראש ומיתתו אלא שלא נתכוין אלא לפסיקת הראש. וראיותיו מדאמרי׳ במסכת סוכה ואסור למעטן בי״ט ואי אית ליה הושענא אחריתי שרי אלמא דבר שאינו מתכוין מיקרי אע״פ שהמלאכה נעשית בודאי כיון שאינו נהנה באותה המלאכה ולא להנאתו הוא עושה. ועוד מדאמרי׳ בפ׳ הבונה גבי התולש עולשין מתוך שדהו אם ליפות את הקרקע בכל שהוא ופרכי׳ אטו כולהו לאו ליפות את הקרקע נינהו ופרקי׳ ל״צ דקעביד בארעא דלאו דילי׳ ומאי תירוצא הא מ״מ מיפה הוא ופסיק רישי׳ ולא ימות הוא ובפ׳ ר״א דמילה אמרי׳ ואי דאיכא אחר ליעבד אחר ומתרץ דליכא אחר ולרבא נמי פריך ולא כמו שפרש״י ז״ל דלאביי פריך מקמי דסברה. ועוד אחרת מדאמרי׳ בפ׳ ב״מ מוכרי כסות מוכרין כדרכן ובלבד שלא יתכוין בחמה מפני החמה ובגשמים מפני הגשמים אלמא שאינו מתכוין מקרי אע״פ שהוא לובש בודאי. ועוד אחרת מדאמרי׳ בסדר יומא בפ׳ א״ל הממונה א״ר יהודה עששיות של ברזל מחמין לו מערב יה״כ ומטילין בתוך הצונן בשביל שתפוג צינתן ואקשי׳ והלא מצרף ופריק אביי אפי׳ תימא שהגיע לצירוף דבר שאינו מתכוין מותר והיכי הוי דבר שאינו מתכוין ששם בודאי הוא נעשה אותו הצירוף. ולפ״ז הדעת פי׳ בעל הערוך ז״ל במסובריתא דנזייתא שהיא סתימה בפי הכד או סתימת נקב של מעלה כההוא דאמרי׳ בפ׳ תולין לא ליהדוק אינש אודרא אפומא דשישא וכו׳ שהשמן חוזר לפך הא סתימת ברזי הדפנות מותר שהרי אינו מתכוין לסחיטה ואינו נהנה בה אנא הוי פסיק רישי׳ ולא ימות דלאבוד אזיל. וזאת הסברא לא הסכימה לדעתי ואלו הראיות יש לדחות. שזו שאמרו במס׳ סוכה דא״ל הושענא אחריתי כך טעמא שאין מיעוכי ענבי ההדס תיקון אלא למי שרוצה לצאת בו י״ח אבל למאן דבעי ליה למידי אחרינא לאו תיקון הוא ואין ההדס עומד ומיוחד לצאת בו אלא להריח בו או לדבר אחר לפיכך כשאין לו הדס אחר והוא ממעט כדי לתקן אותו לחובתו הוא כעושה כלי ואסור אבל כשהוא ממעט לדבר אחר ויש לו הושענא אחריתי נמצא שאינו מתקן שהרי אין בזה תיקון להדס שהרי הוא עומד להריח ולא עשה שום מלאכה לא במתכוין ולא בשאינו מתכוין ולא שמענו בממעט ענבי הדס בפסח שיהא אסור וזו כיוצא בה. ומ״ש בפ׳ הבונה בארעא דלאו דילי׳ אפשר לפרש שפטור מפני שהוא מלאכה שא״צ לגופה וכבר כתבתי זה במקומה. וענ״ל שיפויי השדה אינה מלאכה אבל תלישת עשבים היא המלאכה המחייבתו ומשום תולש מתרינן בי׳ אלא שהתלישה חשובה להתחייב עליה בכל שהוא כשהוא מתכוין ליפויי השדה וכשהשדה אינו שלו אינו חשוב בעיניו יפוי השדה כלום ואין התלישה חשובה להתחייב עליה במשהו וזה דומה למצניע דחייב בכל שהוא ושאר בני אדם אינן חייבין עליהן אלא בשיעורן. אבל בתוספתא מצאתי דמייפה את הקרקע מלאכה בפני עצמה הוא ומשום חורש מתרינן בי׳ דקתני אם נתכוין בכולן חייב שתים אלא שיש לתרץ כדאמרןדחורש וא״צ אלא לעפר מלאכה שא״צ לגופה הוא וה״נ מוכח בירושל׳ בפ׳ כלל גדול. ומה שהתיר ר״ש למוכרי כסות ללבוש כלאים משום דלא אסרה תורה אלא מלבוש שסתמו להנאה אבל כשאין לו ממנו הנאה לאו מלבוש הוא אלא משאוי בעלמא והכי אית׳ בפ״ק דיבמות ואתא ר״ש ושרי בשאינו מתכוין בחמה מפני החמה ובגשמים מפני הגשמי׳ ומן הטעם הזה אמרו בגדי כהונה קשין הן והאי נמטא גמדא דנרש שרי (בסדר יומא ס״ט) לפי שאין בהן הנאה אע״פ שהוא לבוש בכלאים. ומ״ש בס׳ יומא בעששיות של ברזל שהוא דברי שאינו מתכוין אינה ראי׳ של כלום דההוא למאי דקס״ד מעיקרא אביי אמרה מקמי דגמר מרבא דמודה ר״ש בפסיק רישי׳ ולא ימות ומש״ה איקשי עלה מההוא דמילה בצרעת דאיתמר מקמי הכי כדמפורש בפ׳ ר״א דמילה ובתר הכי מסקי׳ דהואיל וצירוף דרבנן הוא והוא אינו עושה לשם כך מותר שאין כאן ליגזר משום מתקן שהרי אינו רואה כמתקן. אבל אין פרש״י ז״ל במסוכרייתא דנזייתא מחוור עדיין לפי שהסחיטה אינה אב מלאכה בפני עצמו שיתחייב עלי׳ בכל ענין אבל מפרק הוא ותולדה דדש הוא כסחיטת זתים וענבים וכשא״צ למשקין הנסחטין אינו דומה לדש כלל כדמשמע גבי הסוחט כבשין דלגופן מותר ולמימיהן חייב ואפשר דלרב אסור כיון דקא סחיט מ״מ אע״פ שא״צ לסחיטה זו שהרי דרך סחיטה בהן ואסור לכתחלה וכן עיקר. עוד לשון אחר פי׳ בעל הערוך שהוא סתימת הנקב של חבית בחתיכת עץ או בברזא מהו דקתני ואסור משום מחבר כלו׳ שהוא משקעהו לחבית בהדוק ונמצא עושה דופן שמבטלין אותו בגיגית ימים הרבה אלא שאין מתכוין לדופן אלא לסתימת השכר יפה ואי קשי׳ הא אפי׳ ברא דתומא אסרי׳ בפרק תולין התם דלא מתקן להכי הכא כיון דמתקן שרי אלא להדוקי טפי אסור דהוי דופן. ואי קשיא להא דאמרי׳ סחיטה תולדה דדש הוא מהא דאמרי׳ לקמן בפ׳ ואלו קשרים ליעבר זימנין דמתווסן מאני במיא ואתי לידי סחיטה ובפ׳ חבית תנן גבי אלונתית ולא יביאם בידו ומפרש בגמ׳ משום סחיט׳ וכן במס׳ ביצה גבי טבילת כלים אור״ת ז״ל דתרי גווני סחיטה נינהו חד תולדה דמלבן וחד תולדה דדש וההוא דהוה תולדה דמלבן ליתא אלא בבגד שמתלבן ובמים שמלבנין וחייבין עליו אף על פי שא״צ למשקין אבל ביין ושמן ושאר דברים ליתי׳ וכשהוא צריך למשקין הזבין מהן הוא תולדה דדש וחייב כדאמרי׳ לא חייבה תורה אלא בסחיטת זתים וענבים בלבד ואיתא להא בין במים בין בשאר הנסחטין וכו׳ וכך הם דבריו בס׳ הישר ואין הלשון הזה מתוקן. אבל כך ראוי לומר כל סוחט פירות תולדת מפרק בצריך למשקין ושיעורן כגרוגרות ואין דישה אלא בגדולי קרקע (כלל) כפירות וכיוצא בהן והסוחט בגד תולדת צובע כדרך מלבן והוא נמי בכל שמכבס בין במים בין ביין ודאמרי׳ התם לאכלה ולא למשרה ולא לכביסה והוא שמתכבס הבגד בכך מועט ושיעורו כמלא רחב הסיט כפול בחוטין ובאריגה ג׳ על ג׳ כשיעורן להוצאה וכענין הזה כ׳ ה״ר משה הספרדי ז״ל ושנינו בתוספתא המכבס והסוחט מלאכה אחת ר׳ ישמעאל בר׳ יוחנן ב״ב אומר צבעים שבירושלים קבעו סחיטה מלאכה בפ״ע וההוא דאיתמר בפ׳ תולין מסננין את היין בסודרין ולא חיישינן דילמא סחיט לא מפני שאין ביין משום סחיטה דליבון אלא לפי שאין הסודרין מתלבנין בכך ואין דרך לסחטן מן היין עד שמתלכלכין הרב׳ ומכבסין אותן במי׳ וכל הגזירות כדרכן חששו להן כדחש ליה רבה בדסתודר אפומי דכובא בריש ב״ט ואע״פ שאין בכובא אלא יין: }
\textblock{הא דאקשי׳ דרב אדרב דאמר הלכה כר״ש \textbf{למימרא דרב כר״ש ס״ל.} קשיא טובא מי דמי אטו משום דס״ל כר״ש בהא בכולהו מילי ס״ל כותיה. ומתוך הדחק יש לי לפרש דגמרא גמירי ליה דרב משום דאזיל במילי דשבת לקולא פסק כר״ש ולאו למימרא דבעלמא כל ישראל בני מלכים הן ומש״ה אקשי׳ למימרא דרב בכולהו קולי דשבת כר״ש ס״ל וכ״נ מדברי רש״י ז״ל ואנן השתא אע״ג דס״ל כר״ש בדבר שאין מתכוין בכל ישראל בני מלכים הן לא ס״ל כותיה דלא מיתליא חדא בחברתה ואע״פ שבגמ׳ תלאום זו בזו לא מפני שכל מי שפוסק כאן כל ישראל בני מלכים הן משום קילא דשבת אמרה וכר״ש ס״ל בכל קולי דשבת כדפרי׳ ואדרבה מינה משמע שאין הלכה שכל ישראל בני מלכים הן ולא ס״ל הכי אלא מאן דמיקל לגמרי בכל אסורי שבת לפום מאי דפרישית ושמעי׳ לאביי דאוקי׳ בשטה לקמן בפ׳ מפנין הלכך ליתי׳ ומיהו באתרא דשכיח שרי כרב והכי פסק רבינו הגדול ז״ל ובתוס׳ חכמינו הצרפתים ז״ל ראיתי דהא בהא תלי׳ דלהכי שרי ר״ש ואמר דכל ישראל בני מלכים הם משו׳ דהוי דבר שאין מתכוין אלא לתענוג ולא מפני ייחוסן אמר אלא כל ישראל בני מלכים הן לדבר זה דכולן מותרין בו וכ״נ מדברי ר״ח ז״ל והוא פסק כר״ש אפי׳ בהא ועם כל זה אין אני מבין טעמם בתירוצם זה כלל ובה״ג מצאתי להלכה כר״ש דשרי: }
\newchap{פרק \hebrewnumeral{15} ואלו קשרים}
\newsection{דף קיב}
\textblock{}
\textblock{\textbf{הכא למילתיה לאו מנא הוא.} פרש״י ז״ל דהא לר״י לענין שבת וטומאה לאו מנא הוא ובתוס׳ הקשו דהא לרבנן נמי למילתי׳ לאו מנא הוא אלא משום דאמרי׳ מגו דמצי לאפוכי משמאל לימין הוי מנא וא״כ לרבנן נמי נימא דלאו מנא הוא לחליצה ולאן קושיא הוא דה״פ לרבנן דאמרי לענין טומא׳ ושבת מנא הוא משום דאי בעי מהפך לי׳ משמאל לימין לענין חליצה נמי מנא הוא דהשתא בסנדל דעלמא של שמאל שהפכו לימין שעומד לשמאל אמרת כשרה כ״ש בזה שכל עצמו אינו קרוי סנדל אלא מפני שעומד להתהפך ולמיהוי של ימין אבל לר״י דאמר לא מיקרי סנדל אלמא ס״ל שאינו עשוי להתהפך אלא להיות של שמאל כשהי׳ וכבר בטל מתורת סנדל בשמאל הלכך אם הפכו נמי וחלץ בו חליצתו פסולה:
}
\textblock{\textbf{שניה נמי מתקנה ראשונה.} פי׳ פשטי׳ דר׳ יוחנן אזלא כאוקמתי׳ דאוקמי בסנדל שיש לו ד׳ אזנים ונפסקה אחת מהן ותקנה ואח״כ נפסקה חברתה שבאותו צד עצמו ואע״ג דמתני׳ ר״י הוא שאין לו הפוך ובנפסקה אחת מתרסיותיו של סנדל דעלמא טהור מפני שאין ראוי עוד לאותו הרגל ה״מ בשלא תקנו שאינו עומד לתקן שאין דרכן של בני אדם לתקן כן ולצאת במנעלים המטולאים כגון אלו ועוד דהו״ל כלי שנשבר דאפי׳ עומד לתיקון כגון כלי מתכות וטלית שנקרעה טהור אבל כשתקנה ודאי מקבלת טומאה מכאן ולהבא הלכך גבי סנדל שיש לו שתי אזנים [בצד אחד] וכשנפסקה אחת לא טהור מטומאתו הראשונה כשתקנה וגלה בדעתו שהוא רוצה באותו טלאי ואח״כ נפסקה שניי׳ אמאי טהור מאותו טומאה שקבל תחלה הא מתקנה וקיימא חדא אלא ש״מ פנים חדשות באו לכאן דהואיל ואתה בא לטמאו מחמת תקון זה שלאחר הטומאה פנים חדשות הן שנולדו בו וכבר נתקלקל משקבל טומאה זו לפיכך טהור מטומאתו הישנה שהיתה עליו וטמא מגע מדרס כלומר שנגע בעצמו וה״נ אע״פ שנקב ונסתם ונקב ונסתם כיון שהגיעו נקבים למוציא רמון הרי הן כמנוקבים עכשיו דמשירדה להן טומאה זו נסתמו כולן ופנים חדשות הן:
}
\newsection{דף קיג}
\textblock{מתני׳: \textbf{כלל אמר ר״י כל קשר שאינו של קיימא אין חייבין עליו.} ק״ל דמשמע הא איסורא איכא וא״כ האיך ר״י מתיר לכתחלה ועוד הא רבנן מודו שאין חייבין עליו ואיכא דבעי מימר ה״ק להו ר״י לרבנן אפי׳ בחבל דעלמא אין בו אלא איסורא בעלמא הלכך בחבל דלי דין הוא שיהא מותר לכתחל׳ וזה אינו נכון משום דאמרי׳ בגמ׳ אלימא חבל דעלמא בהא ר״י מתיר קשר של קיימא הוא ויש לדחוק קשר של מקצת קיימא הוא להיות פטור אבל אסור. ובדקתי ומצאתי בירושל׳ (טו:) בלשון הזה כלל אר״י וכו׳ הא רבנן לא אלא בגין דתנינן קדמייתא בשם ר״י תנינן אף הזה בשם ר״י כלל אר״י וכו׳ אלמא מלתא באפי נפשיה הוא ולא קאי אפלוגתא דר״י ורבנן אלא אדר״מ דלעיל קאי דבר פלוגתי׳ היינו ר׳ יהודה:
}
\textblock{\textbf{שלא יהא דבורך של שבת כדבורך של חול.} פרש״י ז״ל כגון מקח וממכר וחשבונות וק״ל היינו ממצוא חפציך. ועיקר הפי׳ מפורש בירושל׳ א״ר חנינא בדוחק התירו שאילת שלום בשבת אר״ש בר אבא ר״ש בן יוחאי כד הוה חמי לאימי׳ משתעי סגי הוה א״ל אימא שבתא הוא. תני אסור לאדם לתבוע צרכיו בשבת, ר׳ זעירא שאל לר״ח בר אבא מהו מימר (רוענו) [רענו] פרנסנו א״ל טופס ברכה כך היא מכאן נראה שמנהגם לא היה לאומרו אלא לומר ברכת נחמה כמו שאנו אומרים שאם היה מנהג לאומרו לא היה שואל מהו למימר הכי אלא היה לו להקשות והרי אומר (רוענו) [רענו]פרנסנו ומ״מ אם רצה אומר כן שטופס ברכה כך הוא בחול וא״צ לשנות אבל לשאול צרכיו מעצמו אסור. ואני תמה בין בחול בין בשבת (רוענו) [רענו] פרנסנו בבונה ירושלים מאי בעי ושמא בברכת הזן היו אומרים כן כמו שאנו אומרים בה לא חסר לנו ולא יחסר לעולם ועד:
}
\textblock{\textbf{אי רבי ישמעאל ליתקע כי היכי דלידעו דחלבי שבת קריבין ביה״כ.} פרש״י ז״ל כיון דקיל לר׳ ישמעאל יה״כ משבת ליתקע להבדיל מלהקריב חלבי יה״כ בשבת ובתקיעה זו יבינו שהוא יום קל ולכשיחול יה״כ במוצאי שבת לא יתקעו וידעו שמותר להקריב בו חלבי שבת ולר״ע כיון ששניהם שוים אין תוקעין לידע שאין זה קרב בזה ולא זה בזה. ואפשר שהטעם לפי שאין במשמע התקיעה אלא שיבדלו מלעשות מלאכה שהיום קל הוא מהיום הבא הלכך לר״ע אין תוקעין כלל אבל לר׳ ישמעאל היה להם לתקוע. וא״ת והלא אין דוחין שבות להתיר והאיך נתקע לידע דחלבי שבת קריבין ביה״כ ועוד שזו שבות רחוקה היא אין זו שאלה לפי שהתקיעה לאיסור הוא לומר שאין חלבי יה״כ זה קריבין בשבת וקרובה היא וה״ק לר״ע ודאי אין כאן תקיעה והבדלה אלא לר׳ ישמעאל ליתקע להבדיל מלהקריב חלבי יה״כ בשבת שמקל לחמור הוא צריך תקיעה לידע חומר השבת וממילא יבינו קולת יה״כ שאין חלבים שלו קרבים בשבת אבל חלבי שבת קרבין בו שבזה תוקעין ובזה אין תוקעין שאלמלא לא היו קרבין לא של זה בזה ולא של זה בזה לא היו תוקעין שאין תוקעין אלא מקל לחמור כדפרישית ובתוס׳ פי׳ דכי אמרי׳ שבות קרובה התירו וכו׳ לתרוצי נמי הך קושיין דחלבי שבת קריבין ביה״כ אתא. וי״ג וליבדול כי היכי דלידעו דחלבי שבת קרבין ביה״כ כלו׳ כשחל יה״כ להיות במוצאי שבת וזו הגירסא אין לה עיקר בשום נוסחא אבל למדה המגי׳ מן הירושלמי דגרסי׳ התם מה כר״ע ברם כר׳ ישמעאל יבדיל שכן חלבי שבת קריבין ביה״כ א״ר זעירא קומי ר׳ מוני אפי׳ כר׳ ישמעאל לא יבדיל כלום הוא מבדיל אלא להתיר דבר האסור לו אלו הקטר חלבי שבת שמא אינו מותר א״ר שמואל אחוי׳ דר׳ ברכי׳ ויבדיל שכן הוא מותר להדיח כבשין ושלקות א״ר יוסי כלום הוא מותר להדיח כבשין ושלקות אלא מן המנחה ולמעלה ויבדיל מן המנחה ולמעלה ממ״נ כוס אין כאן נר אין כאן במה הוא מבדיל א״ר אבין בתפלה עכ״ל הירושלמי. ואין הגירסא נכונה לפי הגמ׳ דילן דאמרה כי היכי דלידעו וכו׳ דבהבדלה לא ידעו דהא לא אוושי מילתא כדאמרי׳ לקמן בסמוך וליתקע כי היכי דלידעו דשרי בשחיטה לאלתר ומאי קושיא ש״ה דמהבדלה ידעי. והירושל׳ קושי׳ אחרת קא מקשי ודרך אחרת יש לו לפי שכל יום שהוא קל מחבירו מבדילין ביניהם אפי׳ שלא להודיע שום היתר אלא שמצוה להבדיל בין קודש חמור לקודש קל וכיון שכן לר׳ ישמעאל מבדיל שהרי יש ביניהן קולא זו ופריקו ליה שפיר וקושי׳ דמקשי ר׳ אבין יבדיל במנחה בתפלה לאו קושי׳ היא שאין מבדילין בחצי היום ולגמ׳ דילן פשיטא לה ההוא טעמא ובירושלמי לא חשו מש״ה לאקשויי ולאפרוקי. ומזה הירושלמי נתברר לך קניבת ירק שאמור כאן שהוא הדחה בעלמא וכן שנינו בתוספתא שאלו מקנב ממש בורר הוא וחייב אבל בתלוש מניח הוא ופי׳ עגמת נפש שלא תהא נפשו עגומה במוצאי יה״כ ויהא טורח לקנב ואין פרש״י ז״ל [קט״ו ע״א ד״ה מותר] נח כלל:
}
\textblock{והא דתנן \textbf{במ״ש מבדילין ולא תוקעין.} לאו למימרא דבשאר מ״ש תוקעין להתיר העם למלאכתן דהא קתני רישא כ״מ שיש הבדלה אין תקיעה וכללא קתני לכל הימים ועוד דלא תני בברייתא בפ׳ ב״מ אלא תקיעות של ע״ש ולא משתמיט תנא ותני בשום דוכתא תקיעות של מ״ש כיצד הן ובאיזה שעה הן אלא מדקתני רישא כ״מ שיש תקיעה אין הבדלה דהיינו כל ע״ש וכ״מ שיש הבדלה אין תקיעה דהיינו מ״ש ואפי׳ י״ט סמוך לו מלפניה או מלאחרי׳ ולהכי קתני י״ט שחל להיות בע״ש תוקעין כשאר ע״ש ובמ״ש מבדילין ולא תוקעין כשאר מ״ש. והא דאקשי׳ ליתקע דלידעו דשרי׳ בשחיט׳ לאלתר לאו למימרא דבעלמא תקעי׳ אלא אפילו בעלמא לא תקעי׳ משום דמלאכת רשות היא מאי איכפת לן עלה אלא בי״ט לאחר שבת ראוי לתקוע להתיר מלאכת אוכל נפש שהיא מצוה אלא שאין דוחין בה שבות דאיסור תקיעה. והר״ר משה ז״ל כתב דתוקעין במ״ש אין דברים שלו נכונים וכבר עלתה השמועה כהוגן:
}
\textblock{\textbf{לעבור עליו בעשה ולא תעשה.} פירשו המפרשים דפליג ר׳ יוחנן אברייתא וס״ל כאידך דתניא כוותיה וא״כ הול״ל ר״י ס״ל כי הא דתניא וא״ל ר״י תנא הוא ופליג ומיהו כיוצא בזה יש במקומות אחרים ואע״פ שאינו מביא ברייתא אחרת בפ׳ ד׳ מיתות אמר אביי כי כתיב ולא יהיה קדש וכן בפ׳ בן סורר ומורה:
}
\newsection{דף קטו}
\textblock{\textbf{מפצעין באגוזים ומפרכסין ברמונים מן המנחה ולמעלה.} טעמא דמלתא לפי שהוא שעה שבני אדם מתקנין מאכלן לערב אבל מקמי הכי נראה כמתקן לצורך היום ודילמא אתי למיכל ועוד טעם אחרת שמעתי דמן המנחה ולמעלה כיון שעבר רובו של יום אין נפשו של אדם מתאוה לו לאכול כ״כ משום דהו״ל כמי שיש לו פת בסלו וכך פי׳ בעל המאור אבל קודם לכן לפי שאין לו פת בסלו גריר ליביה ואתי למיכל: }
\textblock{מתני׳: \textbf{כל כתבי הקודש מצילין אותן מפני הדליקה.} פי׳ אבל לא דבר אחר משום דאי שרית ליה אתי לכבויי כדמפ׳ בגמרא ואמרו בתוס׳ דוקא בשהדליקה באותה חצר אבל בחצר אחרת מציל כל דבר כי היכי דבעי וליכא למיחש דלמא אתי לכבויי הואיל ואין הדליקה באותה חצר וזו סברא בלא ראי׳. ובעל ס׳ התרומות כתב דוקא לחצר ולמבוי דאינן מקורין וסמוכין לרה״ר ואתי לאחלופי ברה״ר אבל היה בית חבירו סמוך לביתו וערבו מציל על כל מה שירצה ואין סברא זו נכונה בעיני כלל שהרי היתה לו חבית בראש גגו אינו מביא כלי מביתו ולא מן הגג עצמו ויצרף ואע״ג דגג ברה״ר לא מיחלף ועוד אמר הטענה המשובשת שהוא רגיל בה דהשתא לית לן רה״ר הלכך מצילין בכל חצר המעורבת ומבוי משותף כל מה שירצה וכבר כתבתי שאלו דברי הבאי הם וכ״ש בדליקה דמשום דאתי לכבויי הוא וקרוב הוא לכיבוי בעיירות שאין רה״ר גמורה כמו ברה״ר גמורה: }
\textblock{}
\textblock{גמ׳: \textbf{ר״ה אמר אין מצילין דהא לא ניתנו ליכתב.} פסק רבינו הגדול ז״ל הלכתא כר״ה ואומר אני שהוצרך לפסק הזה לומר שאם היו כתובין בכל לשון אין מצילין אבל תרגום מצילין שהרי ניתנה תורה שבע״פ לכתוב בזמן הזה משום עת לעשות לה׳ הפרו תורתך כדאמרי׳ במס׳ סוטה גבי ר׳ יוחנן דמעיין בספרא דאגדת׳ בשבתאוכן מצאתי לרב בעל ה״ג ז״ל שאמר מצילין ספר אפטרת׳ מפני הדליקה ושרי לטלטולי ולמיקרי ביה מדר׳ יוחנן וכן במ״ס מצאתי דמצילין ספר אגדתא ואפשר דמאי תרגום תרגום דעלמא, אבל תרגום דידן דהיינו אונקלוס ויונתן ב״ע כשאר תורה שבע״פ הוא ומצילין: }
\textblock{ה״ג וכ״כ בכל הנוסחאות: \textbf{והאמר רב המנונא תנא מצילין אמר לי׳ אי תניא תניא.} ול״ג מאן תניא והיכי תניא וליתא בשום נוסחא וברייתא הוא ומצאתי׳ שנוי׳ בפ׳ זה בתוספתא אבל רש״י ז״ל גורס מאי תניא דתניא עד שתהא כתובה אשורית על הספר ובדיו וכו׳ (אפי׳) [ופי׳] שכל אלו הן הדברים שבין ספרים למגלה ועלייהו קתני להו בהך מתניתא וש״מ שאר ספרים נכתבין בסם ובסיקרא ואין זה נכון ולא מחוור דבעי׳ דריש גלותא מפשט פשיטא לי׳ דלא ניתנו לקרות בהן דהיכא דניתנו לקרות בהן כ״ע לא פליגי דמצילין וכיון שכן השתא היכי מייתי לה ברייתא למימר דקורין בהן ועוד דלאו הכי תניא בשום דוכתא. ותו ק״ל, דאמרי׳ עלה דמתני׳ בדוכתא מה״מ אתי׳ כתיבה כתיבה כתיב הכא ותכתוב אסתר וכתיב התם מפיו יקרא אלי את הדברים האלה ואני כותב על הספר ובדיו וכל כתיבה דספרים תפלין ומזוזות מהתם גמרי׳ אלמא בכולהו בעי׳ על הספר ובדיו. ועוד מצאתי דתניא בהדיא במס׳ סופרים גבי ספרים אין כותבין ע״ג דפתרא ולא על נייר מחוק ולא בשחור ולא בשיחור ולא בקומוס ובקנקנתום ולא במי טריא וכו׳ ועוד דמגמ׳ דילן מוכחא דאמרן בפ׳ המוציא גבי ס״ת כתובה על הנייר ועל המטלית פסולה על הקלף ועל הגויל ועל דוכסוסטוס כשרה אלמא בס״ת נמי בעי׳ [ספר] ובפ׳ הבונה (שבת קג:) תניא או שלא כתב בדיו או שכתב האזכרות בזהב הרי אלו יגנזו וההוא לאו במגלה תניא דמייתי לה וכתבתם אלא גבי ס״ת תניא ממאי מדקתני או שכתב את השירה כיוצא בה ש״מ דיו בעינן ובפ׳ הקומץ רבה נמי הכי מוקי לה בס״ת אלמא דיו בעי׳. ושוב מצאתי בס׳ התרומות (סי׳ רמ״ה) כדברינו והוא מפרש זו הברייתא דקתני עד שתהא כתובה על הספר ובדיו דלגבי הצלה תני וה״ק שאר ספרים כיון שיש בהן אזכרות אפי׳ כתובין בסם ובסיקרא מצילין אותן אבל מגלה דלית בה הזכרה עד שתהא כתובה אשורית על הספר ובדיו ואין אנו צריכין לדחוק מפני שהגירסא אינה עיקר בנוסחאות:
}
\textblock{\textbf{רב אשי אמר לעולם כדאמרי׳ מעיקרא ושמואל דאמר כר׳ נחמי׳.} פרש״י ז״ל כיון דר׳ נחמיה יחידאה הוה שמואל הנהיג בפני עצמו במקומו כדברי חכמים [עיין תוס׳ ד״ה ושמואל] ונכון הוא וא״נ הוא לא הנהיגן דאיהו כר׳ נחמי׳ ס״ל אפשר דאינהו לא נייתי ליה למעבד כיחידאה כדאמרי׳ בעלמא משום דס״ל כשמואל הלכה כר׳ מחברו ולא מחבריו ובהא אפילו מחבריו. [וי״מ] דמפסיק סידרא שאני שהיו קורין הסדר של שבוע בצבור וחוזרין ודורשין ואין בקריאה כגון זו משום גזרת שטרי הדיוטות ומש״ה שרי נמצא ר׳ נחמיה מחמיר שלא לקרות אפי׳ שלא בזמן בית המדרש ומיקל לפסוק הסדר אפי׳ בזמן בית המדרש ומעשה דאנשי נהרדעא כך היה. א״נ ר׳ נחמי׳ תרתי אית ליה משום גזרת שטרי הדיוטות ומשום בטול בית המדרש לפיכך לא התירו אלא למפסק סדרא במנחתא דליכא לא משום בטול דלאו זמן בית המדרש הוא ולא משום גזרת שטרי הדיוטות, כך פי׳ ה״ר משה בר׳ יוסף ז״ל: }
\textblock{מתני׳: \textbf{ואעפ״י שיש בתוכן מעות.} יש מקשים וכי יש בתוכן מעות מאי הוי והתנן כלכלה והאבן בתוכה והא שמלאה פירות ומתרצי דהכא (מעורב) חשיבי ואלמלא משום כבוד ספרי הקדש היה הכלי טפל להן. ול״נ דלאו משום איסור טלטול שנינו אלא משום הצלה לומר דמצילין דבר שאינו ראוי להצלה עם כתבי הקדש ואפי׳ למבוי שאסור להוציא שם: }
\textblock{ה״ג בנוסחי: \textbf{רבנן בתרתי פליגי פליגי בטלטול ופליגי במלאכה.} וה״פ׃ רבנן דפליגי עליה דרבי ישמעאל אפי׳ שהופשט כולו מתירין לטלטל העור אגב הבשר משום כבוד בשר שלא יתלכלך ור׳ ישמעאל משהופשט עד החזה אסר ליה בטלטול כלל וה״ק לי׳ וכו׳ ואיכא נוסחי דמפקי רבנן משום דמשמע להו דרב אשי חומר׳ בדר׳ ישמעאל אתי לאשמעי׳ דפליג אפי׳ בטלטול ובהנך נוסחי פלוגתייהו קודם הפשטה כדר׳ ישמעאל דקאמר מפשיטו עד החזה ומוציא אמורין ומניחן במקומן דאסור הוא אפי׳ בטלטול ורבנן שרו אפי׳ הפשטה ודקאמרי׳ לי׳ ניהו דאסרת הפשטה [טלטול מיהא] לישתרי דומיא דתיק עם הספר: }
\textblock{\textbf{לא לטלטל [נטלטל] עור אגב בשר.} פירוש מחמה לצל, לפי שהבשר ודאי מותר לטלטלו מחמה לצל כדי שלא יסריח שאין קדשי שמים כנבלות שיהיו אסורין בטלטול דצריכין לדברי הכל. ואקשי׳ מי דמי התם עושה בסיס לדבר המותר בטלטול מן הדין והתירו בו הצלה דחדא מלתא הוא הכא נעשה בסיס לדבר שהוא אסור מן הדין ודייך שהקלת עליו לטלטלו אם הי׳ בעצמו משום כבוד קדשי שמים אבל העור אגבו אסור דכה״ג לא שרי׳ משום כבוד דאנן משום הבסיס אסרנו הכל בטלטול זה ולא משום הבשר שהוא קדשים. וי״ל התם גבי תיק נעשה בסיס לס״ת שהוא תשמישו ואע״פ שמונח בתוכו הכא אדרבה הבשר נעשה בסיס לעור שהוא חצי מופשט והוא דבר האסור. אבל רש״י ז״ל פי׳ בו דמעיקרא קס״ד דרבנן שהבשר מותר לטלטל משום כבוד קדשי שמי׳ ובעי למישרי נמי עור אף ע״פ שהוא אסור מן הדין ואקשי׳ מאי קושי׳ לרבי ישמעאל הא איהו לית לי׳ משום כבוד שמים בנאכל להדיוט וכיון שכן אפי׳ הבשר אסור ואפילו יהא העור מותר לטלטל לעצמו אגב בשר אסור דנעשה בסיס לדבר האסור לדברי ר״י והדר אתי למימר דאע״פ שאסור, משום כבוד שמים שרי לטלטולי תרווייהו שלא העמידו דברי טלטול במקום כבוד קדשי הקדש ומאי אגב בשר (לא יהא הבשר) מותר ודאי, אלא ה״ק אגב הבשר שהוא קדשי שמים ראוי לטלטל אף העור, ור׳ ישמעאל פליג בעור ופליג בבשר. ול״נ דבשר לאחר הפשט הוא שאסור בטלטול ועור מותר דהיינו שלחין דחזי למיזגא עלייהו והא דקאמרי׳ נטלטל עור אגב בשר כדקאמרי׳ כנונא אגב קיטמא שפירושו כנונא בקיטמי׳ דקיטמא אסור וכנונא מותר מ״ה קאמרי׳ מי דמי התם דתיק משמש את הספר ונעשה היתר כמותו הכא העור משמש את הבשר ונעשה כמותו לאיסור שדין הבסיס לעולם כדין הדבר שהוא נעשה לו בסיס ובהכי אתי׳ כולה שמעתא בפשיטות: }
\newsection{דף קיז}
\textblock{והא דמסקי׳ \textbf{לעולם כדאמרי׳ מעיקרא ודקא קשיא לך הכא טלטול והכא מלאכה וכו׳.} תימה, הא בטלטול לא מצי רבנן לאשכוחי התירא במלאכה דאית בה תרתי שבות דמלאכה בברזא ושבות דטלטול היכי משכחי התירא השתא נמי ליפרוך מי דמי התם נעשה בסיס לדבר האסור ולדבר המותר [הכא נעשה בסיס לדבר האסור] וכה״ג [מי] שרי וי״ל כי פליגי בטלטול דלאחר הפשטת כולו כדפרי׳ לא דמי לתיק הספר עם הספר אבל השתא דפליגי בהפשטה גופה דשבות דידי׳ בבשר קדש עצמו הוה והוא מתעסק בבשר ועור להציל בשר כדי שלא יסריח דמי להצלה דספר עם התיק. וי״מ דלרבנן מפשיט את כולו קודם הרצאת אמורין דאנון דבר המותר עם דבר האסור וכדאמרי׳ מעיקרא דבהפשטה פליגי האי דמייתי ראי׳ להפשטה מתיק הספר לומר כשם שטלטול תיק הספר והצלתו מותר עם הספר כך הפשטת הבשר מותר עם האמורין דהיתר ואיסור כא׳ הוא מציל, כך מפורש בס׳ המאור:
}
\textblock{\textbf{ועוד לרבנן נציל לתוכו אוכלין ומשקין.} פירוש לא ממש דומי׳ דכתבי הקודש אלא בשתוף ואדרהיט ותנא סיפא חצר לפלוג וליתני בהא מבוי גופה כתבי הקדש בלא בשתוף ואוכלין ומשקין בשתוף וכן כולה סוגי׳ עד דאתי רב אשי ואוקים האי מבוי בלחי א׳ וכרבי אליעזר וכיון דלאו בר שיתוף הוא לא מצלינן אוכלין ומשקין לתוכו לעולם ומ״ה פריש תנא ממבוי ותני חצר וה״ה למבוי בלחיים ושיתוף. ול״נ דה״פ: מדקא מהדר רב חסדא לאוקמי בשני לחיים כר״א ש״מ דת״ק דמתני׳ לא שרי בס״ת אלא דברים המותרים בשאר הטלטולין כלו׳ מבוי המותר שיש לו שתי לחיים ומשותף אבל שאינו מותר לא ומ״ה אקשי׳ א״כ היינו חצר המעורבת שדין מבוי זה המשותף וחצר המעורבת א׳ הוא וליצול לתוכו אוכלין ומשקין ופריק רבה פירוקא אחרינא ואוקימנא בשתי מחיצות ולחי ואליבא דר״י דשרי בהכי ולדידי׳ נמי אקשי׳ כי הך קושי׳ ליצול לתוכו אוכלין ומשקין. וק״ל, כיון דרבה אקשי לי׳ לר״ח הך קושי׳ היכי לא אזדהר מיני׳ בפירוקי׳ ואפשר דסבר רבה דמבוי שאין לו אלא ב׳ מחיצות ולחי אסור מפני שהוא דומה לרה״ר וגבי הצלת דליקה חיישינן שמתוך שאדם בהול על ממונו אי שרית לי׳ אתי לאפוקי לרה״ר כדגזרי׳ בסמוך שמא יביא כלי דרך רה״ר ומ״ה אסרו הצלת אוכלין ומשקין לתוכו ומשום כבוד ספרי הקדש התירו ואביי לא ניחא לי׳ בהאי טעמא כיון דלא שרו רבנן אלא במבוי שלם במחיצות שלו משותף נמי בעי וי״ל דהך ועוד לרבנן קושי׳ דגמרא הוא ולאו דרבה הוא. ואתא רב אשי ופריק לאו כדקס״ד מעיקרא דלא שרי רבנן אלא מבוי המותר דודאי שרי מבוי בלחי א׳ אף על פי שאינו משותף ולא מיבעי׳ לב״ה דלא (אסר) [חסר] אלא שיתוף אלא אפי׳ לר״א שהוא (אסר) [חסר] לחי ושיתוף גבי ס״ת שרי ומתני׳ אתי׳ כד״ה ולא כר״א בלחוד דלא אמר רב אשי ותרווייהו אליבא דר״א כדקאמרי ר״ח ורבה אלא אפי׳ לר״א אתי׳ וה״ה לב״ה. ונתקיימו בזה דברי רבינו הגדול ז״ל שכתב במשנתינו לפסק הלכהודחי דברי בן בתירא ואלו הי׳ פירוש דברי רב אשי כדברי רש״י ז״ל דמוקים לה תרווייהו אליבא דר״א אין למשנתינו עמידה דהא קי״ל כב״ה ומיהו יפה כתבה רבינו ז״ל דלרבנן נמי אין מצילין אלא בלחי א׳ שאין בו שיתוף הא למפולש לא וכן נלמוד מדברי ר״ח ז״ל שכ״כ להיכן מצילין למבוי שאינו מפולש ואף על פי שאין בו לא שיתוף ולא עירוב וזה להציל ספרי הקודש אבל אוכלין ומשקין אין מצילין אלא לחצר המעורבת דהלכתא כת״ק ע״כ אלא [שאפשר] שאין משנתינו אלא כדברי ר׳ אליעזר לפי׳ (ראשון) [רש״י] אף על פי שהדין אמת, ומה שפירשנו נכון ועולה כהוגן:
}
\textblock{\textbf{והא תני דבי ר״י כל מלאכת עבודה לא תעשו יצא תקיעת שופר ורדיית הפת וכו׳.} ק׳ טובא, דהאי קרא בי״ט כתיב וליכא למיגמר שבת מינה דהא ממלאכת עבודה ממעטת לי׳ ובשבת כל מלאכה כתיב ומצאתי שה״ר שמואל ז״ל הי׳ גורס כל מלאכה לא תעשו ואין הנוסחאות כן ועוד דהא משמע דמקרא קממעט לי׳ ואלו מכל מלאכה לא ממעט מידי דלא משתמע מיני׳ מיעוטא כלל. ול״נ דה״פ, דמתוך שהשבת אסורה בכל מלאכה ואפי׳ באוכל נפש כתיב בה לעולם כל מלאכה וכן ביום הכפורים ומתוך שי״ט מותר באוכל נפש כתיב בו כל מלאכת עבודה לא תעשו דאוכל נפש אינה מלאכת עבודה אלא מלאכת הנאה ובפר׳ חג המצות כתיב כל מלאכה לא יעשה בהם אך אשר יאכל לכל נפש הוא לבדו יעשה לכם מפני שפי׳ אוכל נפש כתב כל מלאכה ובפרש׳ כל הבכור כתיב בחג המצות לא תעשה מלאכה לא כ׳ כל ולא הוצרך לומר מלאכת עבודה שכבר פי׳ ובסדר אמור אל הכהנים כתיב בחג המצות עצמו כל מלאכת עבודה והאי הוא דדאיק ר׳ ישמעאל שאם למעט אוכל נפש כבר פירש בו ולכתוב רחמנא מלאכה סתם כמ״ש במקום אחר אלא לא בא למעט אלא מלאכה שאינה אלא חכמה כגון תקיעת שופר ולענין זה יו״ט למד משבת שהרי כתיב אף בי״ט זה ״כל״ חוץ מאוכל נפש כנ״ל והוא פי׳ נכון. ומן הענין הזה נתרץ מה שרגילין לשאול מהיכן למדו היתר אוכל נפש בשאר י״ט חוץ מחג המצו׳ וכבר נתפרש שהוא נלמד מלשון מלאכת עבודה והדבר ברור הוא ממ״ש בכולן כן חוץ מן המקומות כמ״ש למעלה:
}
\textblock{\textbf{[ת״ר כמה סעודות חייב אדם לאכול בשבת שלש.]} מצאתי לרב שמעון בעל הלכות ז״ל שכתב בדין ג׳ סעודות ולא למיכל יא) מחייב אלא לקיומי ג׳ סעודות ואף על גב דלא מסלק תכא יב) (מ״מ) פורס מפה ומברך יג) שארי המוציא ואכול כביצה ומברך ולפ״ז הא דאמרי׳ בגמרא במנחה אורחא דמלתא קתני ולאו דוקא. ואיכא נמי מאן דאמר דמשלים להו בפירא ובמיני תרגימא כדאמרי׳ במסכת סוכה (כז.) לדברי ר״א שאמר י״ד סעודות חייב אדם לאכול בסוכה במאי משלים להו ומהדרי׳ במיני תרגימאומיני תרגימא היינו פירות ואף על גב דרבא פליג עלה במס׳ יומא ואמר פירי לא בעי סוכה מסקנא דההיא שמעתא לאו כרבא אתיא ומשום אורחא דמלתא בפת יד) כדקתני מתני׳ גבי הצלה לקולא ולא כן דעת רבינו תם ז״ל. ואומר ר״ת דנשים חייבות בג׳ סעודות שאף הן היו בנס המן וחייבות לבצוע על שתי ככרות מטעם זה ואין צורך שבכל מעשה שבת איש ואשה שוין:
}
\newsection{דף קיח}
\textblock{\textbf{אין פוחתין לעני העובר ממקום למקום מככר בפונדיון.} פירוש, דהכי מפקדין למעבד לי׳ צרכי׳ בההוא יומא שמא לא ילין במקום ישראל ואם לן בכאן אוכל אותה סעודה שני׳ בלילה ולמחר כשיסע נותנין לו ככר בפונדיון והיינו דאקשי׳ מאי פרנסת לינה כלומר הא קביל לה ומיהו כי אזיל לאו ריקן אזיל וכן אם הי׳ שבת נותנין לו מזון ג׳ סעודות לשבת וכשיסע בא׳ בשבת נותנין לו ככר בפונדיון והיינו דלא ק״ל לרבנן וכי אזיל בריקן אזיל ומיהו לר׳ חדקא כיון דמוקמי׳ לה בדאייתי איהו סעודה בהדיה ואמרי׳ לי׳ (אכלי׳) [אכלה] קשיא להו בגמרא וכי דעתנו לפטרו בולא כלום שאנו אומרים לו אכול את שלו הי׳ לנו לפרנסו בשבת ומה שהוא חסר לאחר השבת ניתן לו עד מזון שתי סעודות ופרקי׳ אין אנו נוטלין סעודה זו אלא בתורת הלואה וכשיסע בא׳ בשבת נחזירנ׳ לו וניתן לו משלנו אחרת משום שכיון שעכשיו א״צ לשבת אלא שתי סעודות מוטב שיאכלנו ונפרנס אותו בשעתו ולא ליתן לו עכשיו יותר מדאי לצורך א׳ בשבת. ואפשר שהטעם מפני גזל עניים שמא לא יהי׳ מצוי לעניים אחרים אבל ודאי לעולם לא פטרינן עני בלא מזון אותו יום ולילה כדאמרי׳ בפירושן:
}
\newsection{דף קכ}
\textblock{\textbf{בא לקפל.} נ״ל שפירוש כגון שפירש טליתו ונטל הכלי והניחו עם האוכלין שבתוכו לתוך הטלית וכן עשה משנים או ג׳ כלים ובא להוציא טליתו מקופלת עם הכלים שבתוכה ובא להציל הוא שהי׳ מתחלה בכלי אחד ועכשיו בא זה והצילו ובעי׳ דרב הונא ברי׳ דר׳ יהושע היא מעין שתיהן שלא הציל להכלים בתוך טליתו אלא שפכן והניח האוכלין שבב׳ או ג׳ כלים בתוך טליתו והצילן ואסיק׳ דבא להציל הוא דמדמי׳ לה לנשברה לו חבית שכל טפה וטפה בפ״ע היא באה לתוך כלי זה של הצלה. וזה הפירוש עולה כהוגן ומדברי רש״י ז״ל למדתיו שכ׳ בפ״א גבי הנהו תרי תלמידי דחד מציל בתרי ותלתא מאני שמקפלן בתוך כלי גדול ונושאן בבת אחת כר׳ אבא בר זבדא והוה קשי׳ (לרבנן) [לרבינו] ז״ל דהא לאו כר׳ אבא אתיא אלא מציל הוא כדפשיטי׳ לי׳ לר״ה ברי׳ דר״י ולפמ״ש הכל מתוקן:
}
\textblock{הא ד\textbf{אר״י טלית שאחז בה את האור מצד א׳ נותן עליהן מים מצד אחר.} לא (כ׳ בה) [כתבה] רבינו ז״ל בהלכות ואף על גב דאתי׳ כת״ק דשרי כל גרם כיבוי דהלכתא כוותי׳ ולא ידעתי מאיזה טעם ושמא הוא סובר לדמותה למאי דאמרי׳ בשלהי כירה ולא יתן לתוכה מים מפני שהוא מכבה ואמר רב אשי עלה אפילו תימא רבנן שאני הכא מפני שמקרב את כיבויו ופי׳ דלא דמי כלי מלא מים אף על פי שיש לחוש לשמא יבקע לנותן מים ממש שהוא מקרב גוף הכיבוי להדיא ואסור לד״ה ור״י דשרי הכא סבר ההיא מתני׳ ר׳ יוסי הוא ובשבת ובריי׳ לא שמיע לי׳ א״נ אית לי׳ טעמא אחרינא ומיהו אנן אתירוצא דרב אשי סמכינן ואתיא דלא כר״י. ומצאתי [ל]סברא זו רגלים בירושלמי (ג,ח) דגרסי׳ עלה דההיא ר׳ ישמעאל בשם ר׳ זעירא ר׳ יוסי הוא א״ר יוסא הוינן סברין מימר מה פליגין ר׳ יוסי ורבנן בשעשה לה מחיצה של כלים אבל אם עשה לה מחיצה של מים לא מן דאמר שמואל בשם ר׳ זעירא דר׳ יוסי הוא הדא אמרה אפילו עשה לה מחיצה של מים הוא המחלוקת עכ״ל זו הגמ׳ וכאן בפ׳ זה שנו כן על מימרא זו דטלית שאחז בו את האור. ולמדנו ממנו דר״י דשרי מחיצה של מים ס״ל מתני׳ דהתם ר׳ יוסי הוא ורב אשי דמוקי לה כד״ה מחלק לה בין מחיצה של מים למחיצת כלים אע״פ שדרכן להשתבר וטעם נכון יש בדבר ואעפ״כ יותר נכון לפרש דהתם ודאי מכבה והכא כמחיצת כלים דמי דאפשר דלא מכבה כדקאמר ואם כבתה כבתה אבל סוגי׳ שבירושלמי כולה מעידה על דברי רבינו הגדול ז״ל:
}
\textblock{ה״ג רש״י \textbf{צואה של קטן הא חזיא לכלבים וכ״ת דלא חזי׳ מאתמול והתניא נהרות המושכין וכו׳.} ופי׳ דאלמא כיון דאורחא בהכי דעתי׳ עלי׳ ה״נ כיון דאורחא דקטן בה היא דעתי׳ עלי׳ דלכי תיתי יאכילוה לכלבים. ואיכא דקשי׳ לי׳ מההוא דגרסינן במס׳ ביצה אלא בתרנגולת העומדת לגדל ביצים מוקצה הוא ואמאי והא אף על פי שהתרנגולת עומדת לגדל ביצים כיון דעבידו דאתו דעתי׳ עלייהו וביצים גופייהו לאכילה קיימי. ואיכא דמפרשי לה בתרנגולת העומדת לגדל ביצים לאפרוחין שהן בעצמן מוקצין וזה הפירוש מוקצה חדא דאנן אתרנגולת קאמרי׳כדקא מקשי׳ נמי מעיקרא אלימא בתרנגולת העומדת לאכילה והשתא נמי אמרינן לגדל ביצים ואלו הי׳ הקצאתה מחמת הביצים העומדיםלאפרוחים מאי קאמר כי עומדת לאכילה (ולגדל) [או לגדל] ביצים כי עומדת היא לאכילה נמי אם דעתו שהבצים שתלד בינתיים יהיו עומדין להוציא מהן אפרוחים אסורה וכי עומדת לגדל ביצים אין אסורין עד שיהיו הן מוקצים מחמת עצמן. ועוד גרסי׳ התם ביצה תאכל ע״ג אמה ה״ד כנון שלקחה סתם נשחטה הובררה דלאכילה עומדת לא נשחטה הובררה דלגדל ביצים עומדת ואמאי הא אף על פי שלא נשחטה לא הוברר שהביצים (אינם) עומדים לגדל אפרוחים שהרי הוא רוצה לאכול הביצים עצמן ואמאי לא תאכל בלא אמה. וקושיין לאו קושי׳ הוא, דכיון דגופה של תרנגולת אסורה הפורש ממנה אסור דכיון דאיברי בגופה כגופה דמי׳ ואלו שחטה ומצא ביצים במעי׳ מי לא אסורה השתא נמי אסורה שאין לידתה מתירתה בי״ט שאם כן בטלת תורת מוקצה ועוד נמי לא עבידי דאתי דדילמא שלימא שחלא דידה ולא תטיל ביצה בי״ט. ותו ק״ל מדתנן (קמג.) וחכ״א מסלק את הטבלא כולה ומנערה והתם ודאי בשראוין לבהמה וא״ה מאן דאית לי׳ מוקצה אסר אמאי הא כיון דעבידי עצמות וקליפין דאתי לימא מוכנין נינהו וה״נ קשי׳ הא דאמרינן בפ׳ קמא דביצה גבי אפר כירה שהוסק בי״ט אסור ואמאי והלא הוא יודע שיש לו להסיק עציו המוכנין לתבשילו והאפר יוצא מהן. ולדידן ל״ק שכל שראוי ועומד למאכל אדם אין דעתו אלא על עיקרו ואינו נותן אל לבו לזמן את הפסולת לאחר שיעשה בו צרכיו שאין אדם מכין הבשר משום עצמותיו לאתר שיאכלנו ודעתו על הראוי לו אבל כשאינו עומד למאכל אדם הוא נותן דעתו אף משום בהמה ומכין לבהמה לחמה. אבל הרי״ף לא גריס בשמעתא צואה של קטן הא חזיא וכו׳ ולא התיר אלא משום גרף של רעי:
}
\textblock{\textbf{ברצין אחריו ודברי הכל.} נראה מדברי בה״ג ז״ל שהוא מפרש דברייתא קא מתרץ אליבא דר״י וברצין אחריו ומשום פקוח נפש שמותר להורגן לד״ה ומיהו ברייתא ר״י הוא דקתני ה׳ נהרגין ואלו לרבי שמעון כל המזיקין נמי נהרגין ואף על פי שאינןרצין כריב״ל. וזה הפירוש מוקצה מן הדעת, דהול״ל לעולם ר״י הוא וברצין אחריו דברייתא ודאי ר״י הוא ולא לד״ה ואם בא לומר שאפילו לרבי שמעון מותר להורגן פשיטא אפילו שאר מזיקין ואפילו אין רצין אחריו נמי אמרת דשרי ואמת הדבר שבירושלמי העמידוהו לברייתא זו בבאין להזיק אבל גמ׳ דילן לא אתיא הכי ודרך אחרת יש להן בירושלמי במזיקין הללו. ור׳ נתן בעל הערוך פי׳ מלתא דריב״ל ברצין אחריו וד״ה ומשום פ״נ וברייתא כשאינן רצין וד״ה שה׳ אלו שאינן רצין כשאר המזיקין שרצין. ואין זה נכון, משום דמדין פ״נ אין להורגן אלא בשאין יכול להציל עצמו בענין אחר ובשאינן רצין אחריו למה יהרג אותן אין זה פ״נ למי שנפשו פקוחה ועוד שאמר ז״ל ברייתא בשאינן רצין וד״ה אינו בגמרא ומעצמו אמר כן שאלו הי׳ המתרץ משנה הברייתא מכמות שהיינו מפרשים אותה בגמר׳ מתחלה לדברי ר״ש הי׳ לו לפרש בגמרא. והפירוש נכון שעל דברי ריב״ל אמרו שהן ברצין אחריו ומותר להרגן לד״ה אבל ברייתא ר״ש הוא וכן פי׳ רש״י ור״י אלפסי ז״ל וה״ג ואנא מתריצנא כלו׳ לריב״ל וכן מצינו בנוסחא עתיקי ולא גרסי׳ לה כדכתיב במקצתן דאי הכי משמע דברייתא משנה:
}
\textblock{ומסקנא דשמעתא \textbf{נחש דורסו לפי תומו.} ופירוש אפילו במתכוין משום דס״ל כר״ש במלאכה שאין צריך לגופה. ואא״ל בלא מתכוין ומשום דקי״ל כוותי׳ בדבר שאין מתכוין חדא דא״כ מאי דורסו דמשמע לכתחלה עושה כן וכ״ת דה״ק לכתחלה דורס עליו כדרך הלוכו ואם נהרג נהרג דבדריסה לא הוי פסיק רישי׳ ולא ימות אכתי ק׳ דקאמר ר׳ ינאי צרעה אני הורג דמשמע במתכוין לכתחלה הורגן ועוד מאי רבותא דאבא בר מרתא דא״ל לריש גלותא רוק דורסו לפי תומו הא מ״מ צריכין הן למסחף מנא עלוי׳ עד שידרס בלא מתכוין ועוד כל הני רבנן מאי אתו לאשמועינן אטו עד השתא לא שמעי׳ בדבר שאין מתכוין הלכה כר״ש וליכא למימר דאתי לאשמועינן דלא שרי להורגן להדי׳ דהא משמע דהתירא אתא לאשמועינן ואי הכי פשיטא ואמאי נקוט מזיקין לישמעי׳ בשאר מילי אלא ש״מ אפי׳ במתכוין וכרבי שמעון. נמצא עכשיו שכל המזיקין נהרגין כדרכן בשרצין אחריו ומשום פ״נ והזרת ה״ז משובח וא״צ לשנות דהיינו דריב״ל ה׳ האמורין בברייתא נהרגין כדרכן ואף על פי שאינן רצין אחריו והיינו בריר ואתיא כרבי שמעון שאר נחשים ועקרבים שאינן רצין אחריו צריך לשנות ומותר לדרסן אפילו במתכוין כאלו אין מתכוין והיינו לפ״ת. ומכאן אתה למד שהלכה כר״ש במלאכה שא״צ לגופה, וכ״ד רבינו הגדול ז״ל. ומוכח נמי מדאייתי מתני׳ דחוץ מן הפתילה מפני שהוא עושה פחם ואוקמה כר״ש (והיינו) [והני] מתני׳ דמפיס מורסא וצידת נחש דאייתי בשילהי פ׳ האורג אלמא הכי ס״ל ומ״ש בפ׳ כירה גבי שמואל במלאכה שא״צ ס״ל כר״י בא ללמדנו שלמאי דס״ל כר״ש אפי׳ של עץ נמי מותר. וכ״פ ר״ח ז״ל כר״ש במלאכה שא״צ מדאשכחן לי׳ לרבא דהוא בתרא דס״ל כוותי׳ בפ׳ נוטל ובפרק כלל גדול נמי מוכח דאמר רבא החופר גומא בשבת פטור ואפי׳ לר״י אלמא כר״ש ס״ל ורבינו הגדול ז״ל כתבה לההוא ושמעתין נמי ראי׳ גדולה למי שפוסק כר״ש אף במלאכה שא״צ לגופה. עוד אני אומר שיש ראי׳ ממ״ש בפ׳ מי שהחשיך (שבת קנז.) בכל השבת כולה הלכה כר״ש והוי כלל גדול לכל מילי דר״ש במלאכה שא״צ לגופה ובדבר שאין מתכוין ובמוקצה ועוד אאריך במקומו בס״ד:
}
\newsection{דף קכב}
\textblock{\textbf{א״ל כאותן של בית אביך.} פירש רש״י ז״ל שהיו קטנות ודאמרינן לקמן בפרק כל הכלים ניטלין בשבת אפי׳ בשתי ידים ובשני בני אדם דוקא בשאר כל הכלים אבל פמוטות אין דינן כשאר כל הכלי׳ (דלא קפדי) [דילמא קפדי] עלייהו ומייחדי להו מקום. ומיהו ק״ל דבפרק כירה (שבת מו.) לא משמע הכי דאסרי׳ התם גדולה דאית בי׳ חידקי משום גדולה דחוליות והיינו נטולה בשתי ידים כדמוכחא שמעתתא דהתם ואף על פי כן לא אסרוה אלא משום גדולה דחוליות ובאית בה חידקי ואין בין פטומות למנורה כלום אלא שהמנורה יש בה קנים אילך ואילך ופמוט אין בו כדאמרינן בפרק הקומץ רבה אינה באה זהב אינה באה קנים ההוא פמוט מיקרי. א״ו משמע שפמוטות של בית אביו גדולו׳ היו ואביו של ר״ז גבאי של מלך היה והיו בקיאין בביתו ורואין שפמוטות שלו גדולו׳ ואותן של בית ר׳ לא הי׳ יודעין שמא פמוטות קטנים היו ואע״פ שאין דרך נשיאים בכך מנורו׳ גדולות הי׳ לו ופמוטות קטנים של כסף וכיוצא בו להשתמש לפניו ובמה שהיו רואין שיערו דשל אביו גדולות בודאי ונוטלין בשתי ידים ואף על פי כן התירן לו וה״ה לנטלות בב׳ בני אדם אלא שלא שאלו רק על ניטלות בב׳ ידים ולא בב׳ בני אדם לפי שעלה בדעתו שהי׳ אסור לטלטל בב׳ ב״א וכששמע ממנו עוד שהי׳ מתיר קרונות והדבר ידוע שדרכן לטלטל בב׳ ב״א שאל ממנו אם הוא מתיר אפילו בב׳ ב״א כדרכן ואמר לו הן וממילא ידוע שאף בפמוטות ששאל בנטלות בב׳ ידים והתיר אף בב׳ ב״א הי׳ מתיר אלו שאל ממנו כנ״ל אבל בתוס׳ הפרישו בין פמוטות לשאר כלים ולא התירו פמוטות אלא בב׳ ידים. ובירושלמי (יז.) נחלקו בשלשה דאמר התם כלי שניטל בשנים מטלטלין אותה בג׳ בד׳ בה׳ אין מטלטלין אותה א״ר זעירא מכיון דאת אמר בב׳ מותר מעתה אפי׳ ד׳ וה׳ וכן הלכתא מעובדא דאסיתא דבי מר שמואל דבשלהי מס׳ עירובין דף ק״ב דהוין שקלי עשרה ב״א וכו׳:
}
\textblock{\textbf{מתני׳ עשה נכרי כבש לירד בו יורד אחריו ישראל.} דוקא בכבש וכיוצא בו דליכא משום מלאכה דלצורך גוי נעשית וליכא משום מוקצה דמשתמש בעלמא הוא וכיושב על האבן דמי אבל לטלטל הכבש אסור וי״ל נמי דמותר אפי׳ בטלטול דמכיון שהעצים ברשות הגוי אין בו משום מוקצה לפי שאין הגוי מקצה כלום ודעתו על הכל וא״צ הכנה כמ״ש בירושלמי אבל בשתלשן לעצי׳ ועשה כבש אסור לטלטל ומ״מ יורד בה ישראל. ומכאן אתה דן לפירות שתלשן לצורך עצמו דאסורים ודברים התלושי׳ שאפתן ובשלן שמותרים והוא שראוין מתחלתן לכוס הא אינן ראוין לכוס מוקצין הם ואסורין למי שיש לו איסור מוקצה בכך. וראיתי בס׳ התרומה (סי׳ רמ״ז) בלבול גדול בגוי שבישל בשבת לצורך עצמו משום דקשי׳ לי׳ תרתי בשביל ישראל למה אסור הא קי״ל המבשל בשבת בשוגג יאכל ולצורך גוי למה מותר מ״ש ממשקין שזבו ופירות הנושרין דגזרי׳ שמא יסחוט ושמא יעלה ויתלוש. ולפ״ד שהקושיות הללו אין בהן ממש דמה ענין מעשה ע״י ישראל למעשה שבת דגוי בישראל גזרו לר״מ במזיד ולר״י גזר נמי שוגג אטו מזיד אבל בגוי אם לצורך עצמו ד״ה מותר ואם לצורך ישראל ד״ה אסור שאפילו לר״מ גזרו בו שמא יאמר לו עשה דכיון דלדעת ישראל חעביד ואתו לאיערומי החמירו בו יותר משוגג ובעי׳ בכדי שיעשו וזה דבר פשוט הוא וכן זו שאמרו גזירה שמא יתלוש ושמא יסחוט בדברים הנעשין מאליהן כגון משקין שזבו ופירות שנשרו הכל מודים שאסורין דכיון דאיכא גזירה עשאום כמעשה שבת של ישראל אבל גוי לעצמו ליכא למיגזר דישראל מגוי לא גמר מלאכות דשבת וכל היכי דליכא סרך לא גזרי׳ בישראל שמא יעשה מלאכה בשבת כדאמרי׳ הכא בעשבים ובפ׳ כירה ובכמה דוכתי ומ״מ משקין שסחטן הגוי ופירות שתלשן לצורך עצמן אסורין משום גזירה הוא בלא דין מוקצה דכיון דמילי דאתו ממילא נינהו כי עשאן גוי נמי אסורין דכולה חדא גזירה הוא ואפי׳ עשבין שתלשן גוי לצרכו בכלל משום דבכל הנתלשין מן המחובר גזרו גזירה זו ויש לומר בעשבים דמשום מוקצה הן אסורין ואפילו לרבי שמעון והוה ליה כבעלי חיים. וראיתי בתוספתא (יד,יא) דקאמר אבל גוי שמכירו ה״ז אסור משום שמרגילו ועושה עמו לשבת אחרת וכולן שעשאן ישראל בין אנוסין בין שוגגין בין מזידין בין מועטין הרי זה אסור רישא ד״ה והך סיפא לא ידענא מאן קתני לה דמשווה שוגג ומזיד אלא שי״ל אסור לו קתני ורבי יהודה היא:
}
\textblock{(הא דאמרי׳ \textbf{אין בנין בכלים.} כבר מפורש לעיל ריש פ׳ הבונה (שבת ק״ב ע״ב) יע״ש):
}
\textblock{גמרא: \textbf{קורנוס של אגוזים לפצוע בו את האגוזים קסבר דבר שמלאכתו לאיסור אפילו לצורך גופו אסור.} יש מפרשים דמתני׳ דלקמן דקתני נטלין לצורך ר״י מוקים לה בדבר שמלאכתו להיתר ולצורך צורך גופו ולא לצורך צורך מקומו ולא לצורך גופו אבל מלאכתו לאסור כלל כלל לא. ורבי יוסי פליג ומוסיף דאפילו מלאכתו לאיסור מטלטלין כולן חוץ מן המיסר הגדול ויתד שמוקצין מחמת חסרון כיס. וקשיא לי כיון דלצורך מקומו שלא לצורך קרית ליה רישא דמתניתין היכי קתני לפצוע בו את האגוזים שהוא לצורך אפילו שלא לצורך נמי שרי ויש לומר דלהכי קתני לצורך גופו לומר דקורנס של אגוזים [הוא דמותר] לצורך גופו (הוא [הא] של נפחים אפילו לגופה אסור וכדי שלא תאמר של אגוזים נמי לגופו דוקא הדר מפרש כל הכלים הניטלין שאמרנו בין לצורך בין שלא לצורך ניטלין, כן נ״ל. ולה״ר משה בר׳ יוסף מצאתי שפי׳ לרב יהודה לצורך צורך תשמישו שלא לצורך צורך גופו אף על פי שאינו לצורך תשמישו וכולם בדבר שמלאכתו להיתר אבל מלאכתו לאיסור כלל כלל לא. וזה הפירוש מתרחק מפני זה הטעם דקורנס של אגוזים לפצוע בו האגוזים צורך תשמישו הוא ואפילו שלא לצורך נמי ואפשר מאי קורנס של אגוזים שהוא עשוי לשחוק בו האגוזים לשכר או לדבר אחר ואין עשוין לפצע ואיני יודע מי הכניסו בכך. ובירושלמי הא שלא לפצע בו את אגוזים לא מתני׳ דרבי נחמיה היא דאמר אין ניטלין אלא לצורך ומיהו רחת ומזלג לתת עליו לקטן קשי׳ שאין תשמישו בכך ואפשר שאף על פי שאינן מיוחדין לקטן כיון שדרך תשמישן כך הוא לתת עליהן או לזרות ולהבר דרך תשמישן מיקרי ומיהו לרבא ברי׳ דריב״ח כיון דצורך גופו וצורך מקומו שוין רישא דקתני לפצע בו לאו דוקא אלא ה״ה לצורך מקומו דתרוייהו שוין ולצורך מיקרי. ואיכא דק׳ להו היכי אמר ר״י כל דבר שמלאכתו לאיסור אפילו לצורך גופו ולצורך מקומו אסור והא איהו אמר בפ׳ כירה שרגא דמישחא שרי לטלטלה אלמא דבר שמלאכתו לאיסור לצורך גופו מותר. וי״מ דנרות לאו מלאכתו לאיסור הוא שאינו אלא בסיס לפתילה ושמן ואין עושין מלאכה ולא אמרו אלא כגון קורנס ומגרה שבהן עושין מלאכת איסור ולהכי נמי ל״ק בריי׳ דנרות ופמוטות דר״מ ור״ש מתירין בהן ור״י אוסר במוקצה מחמת מיאוס ואיסור היום הא משום מלאכתן ד״ה מטלטלין, כ״כ הראב״ד ז״ל. ול״נ שנרות משמשי שבת הן ואין מלאכתן לאיסור אלא כגון קורנס ורחת שאין להן תשמיש קבוע בשבת א״נ ר״י ממתני׳ גופה איתותב והדר בי׳ תדע דהא אמר לעיל בפ׳ ואלו קשרים (שבת קיג.) כלי קיואי מותר לטלטלן בשבת ומלאכתן לאיסור הן:
}
\newchap{פרק \hebrewnumeral{17} כל הכלים}
\newsection{דף קכג}
\textblock{}
\textblock{\textbf{אסובי ינוקי.} פירש רש״י ז״ל להחליק סדר אבריו ביד כשהוא נולד איבריו מתפרקין והוא צריך ליישבן ולא דמי ללפופי דשרי לד״ה בשילהי פ׳ חביות דהתם הוא מחבש אותו ומאליו הוא מתוקן אבל אסובי בידים דמי לתקוני מנא ומאי דמדמה לה בגמרא לאפקטוזין לא מחוור ואי דמיא טפי דמי׳ למעצבין דאמרי׳ לקמן דאסור. ור״ח ז״ל פירש שהוא אסוקי גרמא דפומא דנפלה לינוקא ומגביהין אותו ביד ופעמים שהוא מקיא באותה רפואה ומשום הכי דמיא לאפקטוזין וזה אינו נכון כלל שאין הרפואה מחמת הקאתו והקאה זו נמי שריא דדבר שאינו מתכוין הוא ויש ליישב דהכי מדמי כשם שסתימת פי האצטומכא אסור לעשו׳ לה פה ברפואת אפיקטוזין כך סתימת הפה אסור׳ לעשות לה רפואה באסובי ינוקא בדי לפותחה, כך מפורש בתוס׳:
}
\textblock{הא דאותבי׳ \textbf{אביי לרבא חברי׳ מדוכה.} ואקשי לי׳ איהו נמי לרבי׳ [רבה, לעיל ע״א] מדוכה, ולא אסיק אדעתי׳ לאוקומה בלצורך מקומה דילמא לאפוקי תירוצא מיני׳ דרבה איכוין וכי אותבי׳ לרבי׳ לאו למיפלג עליה אמר אלא גמרא גמיר א״נ בתר דשמעה סברה. ואיכא דמיעבר קולמס עלה דההוא ול״ג אביי אלא איתבי׳ סתם ומה יעשה זה במאי דאקשי׳ לקמן למר לר׳ נחמי׳ הני קערות היכי מטלטלי להו דרבה ודאי אפשר דס״ל אסור לטלטלה לעולם קאמר דסתמא הכי משמע ולרבי נחמיה ודאי אינה נטלת לעולם דלצורך מקום כל הכלים אסורין ולצורך גופה של תשמיש המיוחד לה אין אותו תשמיש מותר בשבת כדי שתהא נטלת בכך וה״נ סבר ר״א אין מטלטלין אותה לעולם ומשום הכי אוקמה קודם התרת כלים ורבא ענה בתרי׳ לפום סבריה ואוקמה כר׳ נחמי׳ אבל לדידי׳ מתוקמא מחמה לצל א״נ רבה בר נחמני הוה ומדאוקי׳ דכא ואם לאו אין מטלטלין אותה מחמה לצל ש״מ דיש בה שום מטלטלת מחמה לצל ולאו אגב שום שהרי לא אמרו ככר או תינוק אלא למת בלבד ולא לשאר המוקצין נמי כגון נר וכיוצא בו אלא מלאכתו לאיסור כיון דלית בי׳ משום (הוצאה) [הקצאה] אלא איסור מלאכתו גורמת לו כל זמן שהוא משמש היתר מותר ודוקא במשמשין לו אבל להניח עליו ככר ולטלטלו מחמה לצל אסור לעולם כנ״ל:
}
\newsection{דף קכד}
\textblock{הא דאמרינן לרבה \textbf{ואתא ר׳ נחמי׳ למימר דאפי׳ דבר שמלאכתו להיתר לצורך גופו אין לצורך מקומו לא.} אפשר דדבר שמלאכתו לאיסור אפילו לצורך גופו אסור וכן לרבא דהא לצורך דר׳ נחמי׳ לאו לכל צורך גופו הוא דומי׳ דת״ק אלא פליג במתניתין אשלא לצורך לגמרי והדר פריש בברייתא דלצורך גופו נמי דוקא תשמישו ובמלאכתו לאיסור לא משכחת לה. וכ״כ רש״י ז״ל ודוקא לדבר שמלאכתו להיתר אבל לדבר שמלאכתו לאיסור אפילו לצורך גופו לא שרי רבי נחמי׳ דהאי לצורך גופו לאו תשמיש המיוחד לו דהא מיוחד למלאכת איסור. ואפשר דלרבי נחמיה דבר שמלאכתו לאיסור ודבר שמלאכתו להיתר שוים הם ודקא קשיא לך כיון דבעי ר׳ נחמיה צורך תשמישו המיוחד לו דבר שמלאכתו לאיסור היכי משכחת לה דשרי לצורך גופו זה אינו קושי׳ דכל שמיוחד לאותו מלאכה אף על פי שהוא לאיסור מותר בכיוצא בה להיתר כגון קורנס של נפחים או של בשמים שהוא מיוחד להכות ולשחוק והם מלאכות של איסור אם בא לפצע בה אגוזים תשמישו המיוחד לו מיקרי שאף זו שחיקה והכאה כיוצא בה אלא שהיא מותרת וכגון מכבדות של תמרה (לכבד בהן השלחן) מלאכתן לאיסור הן ויש להן היתר בשבת לתשמישן. ולפיכך אמרו בפ׳ חביות (קמו.) ששובר אדם את החביות לאכול ממלה גרוגרות ומטלטל את הסכין לכך בדרוסות וחותכין ממש בקורדם ובסכין וההיא ר׳ נחמיה הוא ושריא אף שאין הסכין וקורדם מיוחדין לחתיכת גרוגרת מ״מ מיוחדין הן לחתיכה ולשבירה וכל שם חתיכה א׳ הוא ולא בעי ר׳ נחמיה אלא למעט טלטול השופר לגמע בו מים לתינוק אף על פי שהוא ראוי לכך הואיל ואינו עשוי למלאכה זו ולא לכיוצא בה ומיהו אם בא לשבר אגוזים וכיוצא בזה בתוך המדוכה אסור לטלטלה לכך ולא תשמישה הוא אלא לשתיקה גמורה שאסור בשבת ואין זה מחוור. וכן נמי סכינא למפסק בה מיתנא לפי זה הדרך דלאו תשמישו מיקרי שאי הסכין עשוי למפסק וחתיכת הסכין ופסיקת המיתנא תרי מיל נינהו כדמוכח בפרק בכל מערבין. והא דאמרי׳ בפ׳ חביות (דף קמ״ו) חותלות של תמרים ושל גרוגרות מתיר אבל לא מפקיע ולא חותך לר׳ נחמי׳ משום שאין כלי ניטל אלא לצורך תשמישו ובסכין אסור. ומ״מ קשה יפקיע ויחתוך במספרים או באותו כלי העשוי לחתיכה ואף על פי שמלאכתו לאיסור וא״א שלא יהא שום כלי עשויה לכך או לכיוצא בזה ומיהו משמע שכל כלי שמלאכתו לאיסוראין מטלטלין אותו כלל ואין משנין אותו מאיסור להיתר וכדאמרי׳ במדוכה. ומיהו אפשר דלרבא צורך מקומו כתשמיש המיוחד לו ושרי אפי׳ במלאכתו לאיסור וא״ת כיון דלצורך גופו אסור אף לצורך מקומו אסור שלא מצינו היתר יותר בצורך מקומו מצורך גופו י״ל אה״נ דלר״נ מצינו לפי שאין משנין כלי ממלאכתו ולכן אסור לגופו והיינו דאקשי׳ לי׳ אביי לרבא חברי׳ מדוכה ולא מוקי לה כר׳ נחמיה דלדידי׳ צורך מקומו מותר אפי׳ לרבי נחמי׳ דאס״ד מלאכתו לאיסור בין לגופו בין למקומו אסור היכי מקשי לי׳ אביי מיני׳ והא ניחא לי׳ לרבא כדניחא לי׳ לאביי דאי כרבי נחמיה מתוקמא לי׳ ואי כרבנן נדידי׳ לצורך מקומה ולדידי׳ מחמה לצל ומאי קשר לי׳ אלא כדפרישית, כן נ״ל:
}
\textblock{\textbf{וב״ה מתירין להוציא את הקטן ואת הלולב וס״ת.} כ׳ ר״ח ז״ל נקטי׳ קטן למולו ולולב למצותו וס״ת לקרות בו ומשמע משום דאמרי׳ בהו מתוך שהותרה הוצאה לצורך הותרה נמי שלא לצורך אבל שלא לשם מצוה ל״א מתוך ואין צורך לכך אלא אפי׳ גדול התירו דהא ילתא שרי לה ר״נ למינפק באלונקמה [ביצה כה:] שכל שהוא לצורך אדם ולהנאתו מותר. ובירושל׳ אמרי׳ התם בביצה הא גדול אסור ר׳ שמואל ברי׳ דר׳ יוסי בר בון אמר אפי׳ גדול מותר מי לא תנינן קטן בא להודיעך כחן של ב״ש עד היכן היו מחמירין ואמרי׳ נמי התם מטלטלין האסתניסים ומיהו שלא לצורך כלל כגון מי שאין דרכן בכך ואינו אסתניס יא) אסור אבל אין חייבין עליי׳ דמ״מ כיון שהותרה הוצאה לצורך הותרה נמי שלא לצורך יב) והא דאמרי׳ בפ״ק דביצה אלא מעתה הוציא את האבנים לב״ה ה״נ דלא מחייב היינו אבנים לבנין שלא לצורך היום כלל ובהא ל״א מתוך ומילקא נמי לקי בכל כה״ג. וראי׳ לדבר מהא דאמרי׳ בפסחיםהאופה מי״ט לחול רבה אמר אינו לוקה רב חסדא אמר לוקה ואפי׳ רבה לא אמר אינו לוקה אלא משום דאמרי׳ הואיל ואי מקלעי ליה אורחים חזי ליה השתא נמי חזי ליה אבל אי לא״ה לוקה ול״א מתוך שהותרה הבערה לצורך הותרה נמי שלא לצורך ובפ׳ ב״מ אמרו אין שורפין קדשים בי״ט דלא אתי עשה ודחי ל״ת ועשה די״ט ואמאי לימא מתוך אלא ש״מ שלא לכל דבר אמרו ב״ה מתוך אלא לדבר מצוה וצורך היום. ובפ״ק דכתובות (ז.) אמרו מאי דעתך מתוך שהותרה חבורה לצורך הותרה נמי שלא לצורך אלא מעתה מותר להניח את המוגמר בי״ט מתוך וכו׳ ופריק עליך אמר קרא לכל נפש וכו׳ ומשמע מינה שאין אומרים מתוך אלא בדבר השוה לכל נפש כלומר מתוך מלאכת אכילה שהיא צורך כל נפש הותרה נמי כל מלאכת הנאה שהוא צורך כל נפש וה״ה לכל מה שמשתמש בו אדם ביומו דהא ס״ת לקרות בו לאו משום הנאה הוא אלא שתשמיש הרגיל ליומו מותר לו ול״ד למילה שלא בזמנה דמצוה גרידא הוה ולא מתהני בעשייתה כלל מחמת גופה. ומיהו מילה בזמנה צורך היום מיקרי ומש״ה מוציאין קטן כשם שמשלחים כלי להשתמש בו בי״ט וכן לולב לצאת בו דאלת״ה הויא לה הוצאה מכשירי מצוה ומכשירי מצוה אין דוחין י״ט. ומיהו קשה מההוא דאמרי׳ פ״ק דביצה השוחט עולת נדבה בי״ט לוקה ואמרי׳ עלה דאמר לך מני ב״ש הוא דסברי ל״א מתוך שהותרה וכו׳ אלמא לב״ה אמרי׳ מתוך אע״פ שאין צורך היום וי״ל דההוא נמי צורך היום הוא שלא יהא שולחנך מלא ושולחן רבך ריקם. ול״נ דהתם מתוך שהותרה לצורך גבוה בעולת ראי׳ ובשלמי חגיגה דאינון (צורך) היום לגבוה הותרה נמי (שלא) לצורך גבוה דהיא עולת נדבה (דהיא הנאה דהדיוט). ומיהו אפי׳ לב״ה אסור אלא שאין לוקין עליו. ותו ק״ל הא דאמרי׳ פ״ק דפסחים גבי ביעור חמץ של שריפה וש״מ ל״א מתוך שהותרה הבערה לצורך הותרה נמי שלא לצורך. וי״ל כל חובת היום בדבר שהוא [מקרי] צורך היום קצת אלא שאינו לאכילה דומיא דאוכל נפש שהוא צורך וחובה ליום ולא מיחייב בי״ט אלא בדבר שהוא צורך מחר כגון אופה מי״ט לחול ומילה שלא בזמנה. וק׳ מההיא דתנן המבשל גיד הנשה בי״ט ואוכלו לוקה חמש ולא לקי משום מבשל בי״ט דמשום דאמרי׳ מתוך וכו׳ ואע״פ שאינו צורך היום כלל ואיסורא נמי איכא וא״ל ש״ה שראוי הוא לאכילה והרי נהנה ממנו בי״ט ואכלו ואע״פ שהוא אסור אמרי׳ ביה מתוך אבל לא בשאר דברים שאינן צריכים. נמצאת אומר כמה מחלוקות בדבר הוצאה שהיא לצורך היום ולהנאת אדם כגון טלטול האסטניסוס והקטנים מותר אבל בכסא אסור מפני שנראה כרוצה להוליכו למקום רחוק ואם היו רבים צריכים לו מותר וכן הוצאת כלים הצריכין לו לצורך היום מותרת והוצאת כלים שאינן צריכין לו כגון מפתח אסור כדאמרי׳ בירושל׳ תני ולא את המפתח וב״ה מתירין הדא דתימא במפתח של אוכלין אבל במפתח של כלים לא והא רבי אבוהו הוה יתיב ומתני ומפתח הפלטירין הו״ל בגוויה וזה הירושל׳ כתבו ר״ח ז״ל ואע״פ שמשלחין תפלין בי״ט מפני שאפשר שיהיו צריכין היום לאותו שנשתלחו לו להתלמד בהלכות עשייתן והו״ל כס״ת וכל המשתלחין בי״ט כלים תפורין ושאינן תפורין משום דחזו לבו ביום ושמא יצטרך להם הוא. והוצאת לולב וס״ת עצמו כיון שהוא מצוה וצורך היום דבעי׳ לה׳ ולכם מותרת אבל שחיטת עולת נדבה אע״פ שהוא מצוה אסור׳ ומיהו אין לוקין עליו דאמרי׳ מתוך אבל הוצא׳ אבנים לבנין לוקה וכן אופה מי״ט לחול אלא דקיי״ל כרבה דאמר הואיל וכן בזורע וכיוצא בהן מותק וכותב ומלאכות שאינן נעשות לאוכל נפש אצ״ל דלוקה עליהן וכן בדבר שאינו נהנה בגופה של מלאכה אלא שהוא נעשה מכשירין לדבר הצריך ל״א ביה מתוך כדאמרי׳ במכבה את הנר מפני דבר אחר בפ״ב דביצה וכן מכשירי מילה בזמנה ולולב ושופר כולן אסורין ול״א בהו מתוך. וק״ל דהא גבי שבת קרי הוצאת קטן למולו מכשירי מילה וא״כ בי״ט ליתסרו ועוד מוליכין בהמה אצל טבח ליתסר ואנן אפי׳ לב״ש קאמרי׳ דמותר מן התורה בפ״ק דביצה ול״ק דגבי אוכל נפש הולכת הדבר הנאכל אוכל נפש עצמו הוא דכתיב אשר יאכל לכל נפש (וכן) הדבר הצריך מותר לב״ה מדין מתוך אבל בשבת דמילה עצמה דוחה היא הוצאת קטן וכל שאפשר לעשותה מע״ש אסור. ומכאן אתה למד דל״א באוכל נפש עצמו אפשר לעשות אסור אלא במכשירין לר״י וה״נ מוכח במסכת מגילה דף ז׳ ע״ב:
}
\textblock{הא דאמרי׳ \textbf{אבל של תמרה אסור וכן אמר ר״א.} אומר רבינו הגדול ז״ל, האידנא דקיי״ל כר״ש שרי כבוד הבית דדבר שאין מתכוין הוא והו״ל כלהו מכבדות דבר שמלאכתו להיתר שהוא סבור שעל הרבוץ והכבוד נאמר כן בפ׳ המצניע וכ״כ ר״ח ז״ל וכ״ד בעל הלכות ראשונות ז״ל. ונראה׳ מדבריו של בעל הלכות ז״ל שגירסתו מוחלפ׳ שכ״כ ר״א אמר אף של תמרה סבר לה כר״ש דאמר דבר שאין מתכוין מותר אלמא כי לא מכוין כניש תמרא רככא דאוצר ע״כ וזו הגירסא יותר מחוורת דלמאן דגריס אימא וכן אמר ר״א קשי׳ אמאי דחיק נפשי׳ לחלופא לימא ר״א כר״ש ס״ל ומתוקם אליבא דהלכתא. ואי ק״ל הא דאמרי׳ ר״פ מפנין (שבת קכז.) אבל לא את האוצר שלא יגמור את האוצר כולו וכ׳ במקצת נוסחי פי׳ ובהלכות רבינו הגדול ז״ל דילמא אתי לאשוויי גומות ומתני׳ דהתם ר״ש הוא אלמא לד״ה אסור הא בורכא דהתם לאו במכבד עסקי׳ אלא בגומר אותו בלא כבוד ואעפ״כ אסור משום שמא יראה שם גומות וישוה אותם לדעת לפי שלא ראה אותם עד עכשיו ודרך מכבדי אוצר בכך לפי שנעשה כולו גומות גומות ועשויות למלאות מפני האורחים. ונשאלה שאלה לרבינו הגדול ז״ל, זה שאמרו בי״ט אף הוא אומר ג׳ דברים להקל מכבדין בין המטות אם כבוד זה הוא שמכבדין את הבית במכבדות וחכמים אוסרין וכן הלכה א״כ קשה מה שיפסק בהל׳ שבת בענין מכבדות של תמרה שמותר לכבד בהן דקיי״ל הלכה כר״ש ואם בי״ט אסור בשבת לא כ״ש. תשובה זה המשנה שנאמרה בה וחכמים אוסרין א״ל מאן חכמים ר״י הוא כדאתמר והשתא דס״ל כר״ש כולהו שריין ודבר שאין מתכוין מותר אלו דברי רבינו ז״ל וכן שיטת המשנה דסיפא גבי קדור וקרצוף קתני וחכ״א אליבא דר״י דוק ותשכח. ורש״י ז״ל חולק ואומר דבכל כבוד מודה ר״ש דאסור דפסיק רישי׳ ולא ימות הוא וקבלת הגאונים ז״ל תכריע. ועוד שהדעת נוטה כן לפי שאנו רואין שאין בכל כבוד השואת גומות ולא ברובם, והאמת יורה דרכו:
}
\textblock{הא דאמרי׳ ב\textbf{חספא} דשרי ברה״ר משום דחזי לדידי׳ אע״ג דליכא עלי׳ תורת כלי וגרע ודאי מכיסוי כלים שיש להם בית אחיזה דאסירי עד שיהא עליהן תורת כלי ה״ט דמילתא משום דכיון דהוו כלים והוכנו למלאכתן עדיין הכנתן ותורתן עליהן כ״ז שראוים למלאכ׳ אבל אבנים וכל שלא היה כלי כגון צרורות שבחצר אסור לטלטלן שלא הוכנו מעולם. וקרומיות של מחצלת נמי שרי והוא הטעם דכיון דמכלים אתו וחזו לתשמיש הלכך מותר לטלטלןומיהו אם היו קרומיות. מבע״י וזרקן לאשפה אסורין דהו״ל כצרורות שבחצר ודמי׳ לשברי חביות שזרקן לאשפה מבע״י דאסורין:
}
\newsection{דף קכה}
\textblock{\textbf{השתא תנור גופיה לר״י לא הוה מנא שברים מבעיא.} ק״ל, תנור גופי׳ לענין שבת אמאי לא הוי מנא אע״פ שאינו יורד לידי טומא׳ גבי שבת מיהא כיון דראוי הוא לכל מידי אמאי לא הוי מנא מ״ש מגולמי דאמרי׳ לקמן דלענין שבת הוי מנא ואפשר כיון דתנור זה מיוחד למלאכתו ואינו עשוי למלאכה אחרת והתורה אמרה שדרך תנור להיות מחובר ולפיכך אינו יורד לידי טומאתו ועדיין עתיד הוא לחברו לארץ לפיכך אינו נקרא תנור ולא כלי עד שיחברנו שם הלכך לענין שבת נמי לאו מנא הוא ומיהו יש לפרש מעיקרא הוא דקס״ד לדמויינהו ורב אשי לטעמי׳ קפריך אי איכא לדמויי טומאה לשבת השתא תנור גופא אמרת לאו מנא הוא הא לדידן לעולם מנא הוי לשבת דטומאה להא מילעא נזרת הכתוב הוא שהרי מתחזק הוא ונגמר בצואר גמל כבקרקע:
}
\textblock{הא דאמר ראב״י על כסוי תנור \textbf{שא״צ בית יד.} משמע דדוקא כיסוי תנור אבל שאר כיסוי הכלים מחוברין צריכין הן בית יד מדאמר רבינא כמאן מטלטלינן האידנ׳ כסוי דתנור אלמא לא מטלטלי שאר כיסוי דכלים וטעמ׳ דמילתא מפני שכסוי התנור בתנור עצמו שאף הוא מסייע באפייתו של תנור וכל עצמו אין אופין בו אלא מחמת סיועו לפיכך דינו כתנור עצמו שמטלטל לעולם או שהוא כשבר ממנו שאף הוא עושה טפקא. והרמב״ם ז״ל הספרדי פי שלא נחלקו ר׳ יוסי וחכמים במתני׳ אלא כגון חביות הקבורות בקרקע לגמרי דרבנן אסרי מפני שאינן נראות והרי הן ככסוי הקרקעות אבל בכסוי הכלים המגולין אף על פי שמחוברין לקרקע ל״פ רבנן במתני׳. ומצאתי בנוסחאות ישנות: העיד ר׳ יוסי הכהן ולמאן דגריס נמי ר׳ יוסי סתם לק״מ שאע״פ שר׳ יוסי התיר עדות זו אפשר שבזה נחלקו שהוא סובר ה״ה לשאר כלים וחכ״א לא התיר ר״א אלא בזה בלבד מן הטעם שפירשנו אנו ומפני כן פ׳ רבינו הגדול ז״ל כדברי ר״א בכיסוי תנור וכדברי חכמים בשאר כלים וכ״פ בעל הלכות ז״ל בדברי חכמים אבל רש״י ז״ל כ׳ עלה דפלוגתא הוא במתני׳ דאיכא למ״ד בעינן בית אחיזה ואולי לדבריו שלש מחלוקות בדבר דת״ק דמתני׳ אסר בכל הכלים ור״י שרי בכולהו ור״א שרי בתנור אבל לא בשאר כלים:
}
\newsection{דף קכו}
\textblock{הא דאמרי׳ בכולהו שמעתי׳ בתלייה וקשירה \textbf{בין בפקק בין בנגר בין בקנה.} בכולן תלייה משום גזירת בנין נגעו בה כר״א דכל דתלוי מטלטל הוא ולא מבטל ליה וקשירה דכולהו משום היתר טלטול ותו לא מידי:
}
\textblock{מתני׳: \textbf{מפנין תרומה טהורה.} אבל לא טמאה לפי שאינ׳ ראויה אלא לשריפה ושריפתה מצוה כדאמרי׳ בפ׳ ב״מ כשם שמצוה לשרוף את הקדשים שנטמאו כך מצוה לשרוף התרומה שנטמאת ואע״ג דאמר רחמנא בשעת שריפה תהנה ממנה אפ״ה אסור לבערה בשום הנאה אחרת ומשנה שלימה שנינו בשילהי פ׳ בתרא דתרומה אלו הן הנשרפין חמץ בפסח ותרומה טמאה וכו׳ ותני עלה התם אוכלין בשריפה ומשקין בקבורה ומ״ש בפסחים בפ׳ כ״ש שאם רצה כהן מריצה לפני כלבו על תרומת חמץ טהורה אמרו ואליבא דר׳ יוסי דאמר מותר בהנאה, ודברי רש״י ז״ל אינן אלא מן המתמיהין:
}
\newchap{פרק \hebrewnumeral{18} מפנין}
\newsection{דף קכז}
\textblock{}
\textblock{ה״ג: \textbf{תבואה צבורה בזמן שהתחיל בה מע״ש מותר לטלטלה.} ול״ג מותר להסתפק ממנה. ונדחה מדברי הגאונים ז״ל שהם מפרשים אותה לטלטלה ממקום למקום לפנות מפני האורחין ומפני בטול בית המדרש. ור׳ אחא אוסר משום מוקצה ור״ש מתיר דלית ליה איסור מוקצה והלכתא כוותיה. וכיון שמותר לטלטל תבואה צבורה ממקום למקום מפני האורחין בעי׳ במתני׳ כמה שיעור תבואה צבורה כלו׳ שמפנין אותה לפי שלא שמענו במתניתן אלא כשהיא בקופות אבל כשאינה בקופה כמה מפנין ממנה. ומפרש לתך ואם נסמוך על הירושל׳ הוא השיעור של ה׳ קופות דגרסי׳ התם ר׳ זעירא שאל לר׳ יאשיה כמה שיעורן של קופות א״ר נלמוד סתום מן המפורש דתנינן בג׳ קופות של ג׳ ג׳ סאין תורמין את הלשכה ור״ח ז״ל כתב זה הירושל׳ אבל לפום גמ׳ דילן משמע שכל ה׳ קופות מצילין ואפי׳ גדולות יותר שלא נתנו חכמים שיעור בקופות וע״ז הפי׳ סמך רבינו הגדול ז״ל שכתבה להא מתני׳ דכמה שיעור תבואה משום דאתי׳ כר״ש דהלכתא כוותי׳ ובריית׳ ראשונה דאין מתחילין באוצר תחלה כר׳ יהודה אתיא ולא כתבה לפסוק הלכה כמותה שהרי כ׳ עלי׳ מחלוקתה בצדה, תניא אידך וכו׳. ונראה מדברי הרמב״ם ז״ל הספרדי דהא דתניא אין מתחילין באוצר תחלה היינו שלא לצורך אורחים ולא מפני בטול בית המדרש אלא לצורך עצמו דמתני׳ קתני ד׳ וה׳ מפני האורחין אבל שלא במקום מצוה לא שרי כולי האי אבל קופה א׳ או שתים ודאי שרי שאין אדם נמנע מלטלטל קופות מפנה לפנה לצורך מקומן אבל אתחולי באוצר נראה כמתקן מקום ואסור אפי׳ בקופה א׳ ולא מסתברא. ועוד דגרסי׳ בירושל׳ אמתני׳ מהו לפנות מן האוצר כסדר הזה נשמעינ׳ מן הדא ושוין שלא יגע באוצר אבל הוא עושה שביל ונכנס ויוצא ע״כ משמע שלפנות ד׳ וה׳ קופות מפני האורחים נשנית ברייתא זו. ומסתברא דהלכתא כרב חסדא בפירושא דמתני׳ ואע״פ שלא הכריע בה רבינו הגדול ז״ל הרי כתבה לבעיא דלקמן דאתיא כוותיה דרב חסדא ומיהו בלא יגמור את האוצר נקטי׳ כפירושא דשמואל כי היכי דתיקום מתני׳ כד״ה. ומיהו משמע למ״ד ארבע מחמש וה׳ מאוצר גדול ליכא לפרושי לא יגמור את האוצר דהא מרישא ד׳ מה׳ שמעי׳ לה אבל כולהו ה׳ לא. מיהו אפי׳ הוה הלכתא כר״ח בפי׳ דמתני׳ לית הלכתא כר״י בהתחלה משום דפליג עליה ר״ש והאי הוא טעמא דרבינו ז״ל וח״ו שלא פסק אדם מעולם כמתני׳ בלא יתחיל ותמה הוא למה לא כתבה רבינו ז״ל לבעי׳ קמא למעוטי בהלוכה עדיף או למעוטי במשוי עדיף. ורב אחא משבחא גאון ז״ל גריס בדשמואל ד׳ וה׳ כדאמרי אינשי ול״ג ואי בעי ואפי׳ טובא וכמה הגיר׳ זו מיושרת בעיני דהיכי שייך למימר ד׳ וה׳ במילת׳ דאפי׳ מאה אומאתים מפנין כדאמרי׳ ועימר ר׳ כל השדה כולה ואלו היה דבר שאינו מצוי לפנות יותר היינו אומרים אורחא דמילתא נקט אלא מפני האורחים ומפני בית המדרש יותר מצוי להיות צריכין לפנות כעשר וכעשרים מלפנות כד׳ וה׳ וכה״ג לא משתעי אינשי ולהך גירסא הלכתא כשמואל משום דתירוצי׳ במתני׳ עדיף דמוקי כולה כר״ש ולא נקיטי׳ ליה לחצאין ולא פלגינן דבורא נמינקט כמר חדא וכמר חדא מאחר דסוגיא דשמעתא ל״ק לי׳. ועוד דשמואל רביה דרב חסדא הוה ועדיף מיניה:
}
\textblock{הא דאמרי׳ \textbf{דמאי הא לא חזי לי׳ כיון דאי בעי מיפקר נכסי׳ והוי עני.} ק״ל ל״ל האי טעמא הא חזי לעניים א״נ אלו מקלע לי׳ עני׳ בביתי׳ א״נ אכסניא שמאכיל אותם דמאי ואיכא דמתרץ גבי תרומה דע״כ יהיב ליה לכהן א״ל כיון דחזי׳ ליה שרי וכן גבי לוף מתוך שאינו ראוי אלא למאכל עורבים אבל דמאי אין מטלטלין אותו אא״כ ראוי לו לפי שהוא ממתין עד למחר ומפריש ואוכל. ובעלי התוס׳ אמרו דל״ג ״ליה״, וכ״נ כדאמרי׳ לקמן אלא ראוי ה״נ ראוי ואי גרסי׳ לי׳ שיטפא דגמ׳ הוא משום דאתמר הכי בפסחים ובברכות ובמנחו׳ אמרי ה״נ הכא (דילמא) בעי למימר חזי לעניים ולאכסני׳ דתנן מאכילין את העניים דמאי וכו׳ ומ״ה תפיס ליה כולי׳ שיטה דהתם דלדידי׳ נמי חזי אבל לא היה צריך לכך:
}
\newsection{דף קכח}
\textblock{\textbf{במוקצה לאכילה ס״ל כר״י במוקצה דטלטול ס״ל כר״ש.} פרש״י ז״ל באיסור אכילת מוקצה ס״ל כר״י שאסור לאכלו אבל מותר לטלטלו כר״ש ומקשו והא אמרי׳ בשילהי פ״ק במחצלת של בדדין וכרכי דזוגי דרב אסר כר״י אלמא אף במוקצה לטלטול סבר רב כר״י והדרא קושי׳ לדוכתי׳ ומפרקו בדבר העומד לאכילה לגבי טלטול ס״ל כר״ש דדיו שאסרת עליו באכיל׳ אבל בדבר שאינו ראוי לאכילה אסור טלטול כר״י. ואי קשיא, הא ר״ה דאמר בפ׳ כירה מניחין נר על גבי דקל בשבת ואוקי׳ כר״י ההוא נמי דבר שאינו עומד לאכילה הוא אפי׳ השמן א״נ במוקצה מחמת איסור ס״ל כר״י אפי׳ במוקצה לטלטול:
}
\newsection{דף קכט}
\textblock{\textbf{אמר רב יודא אמר שמואל חיה כ״ז שהקבר פתוח בין אמרה צריכה אני בין שלא אמרה צריכה אני מחללין עליה את השבת.} נ״ל בשאין שם חכמה ורופא עסקי׳, אבל הדברים שעושין לחיה בחול ידועין לכל לפיכך אע״פ שאמרה אין אני צריכה שיחללו עלי שבת שיכולה אני להמתין לערב כיון שידוע הוא שדבר זה עושין אותו לחיה וכל החיות צריכות לכך, אע״פ שאין אנו יודעין אם יכולה זו להמתין עד ערב ולא תבוא לידי סכנה מחללין עליה מספק ואין שומעין לה דילמא תונבא הוא דנקט לה, אבל משנסתם הקבר שומעין לה שאינה בחזקת מסוכנת לאותן הדברים שרגילות חברותי׳ לעשות לה. ועוד שיכולה להמתין. שלשים אפי׳ אמרה צריכה כיון שאנו יודעין שאין לה חולי אחר ואף היא אינה אומר׳ כן הכל בקיאין דמשום לידה אין לה סכנה לחמין ואי לאו דמסתפינא מרבוותא הייתי מפרש צריכה אני לרחוץ אבל אמרה לחלל עלי אפי׳ שלשים נמי מחללין. והא דאמרי נהרדעי שבעה אמרה צריכה אני מחללין איני צריכה אין מחללין קשי׳ דיוקא דרישא אסיפא אלא דספק נפשות להקל וסתמא (נמי) שאמרה צריכה אני כדברי רבינו ז״ל. אבל איני יודע עיקר דקדוקם של רבינו ושל ר״ח ז״ל בהא דהא ר״ה (דמקמי) אביי הוא שאמר משעה שהדם שותת דאיהו אמאימתי פתיחת הקבר קאי ומש״ה אמר משעה ולא אמר עד שעה ומשום דלא אמר משעה שיתחיל הדם להיות שותת אין בזה דיוק ותמהני לדבריהם למה פסקו כאביי והא ספק נפשות הוא ולהקל ובודאי הוה משמע דהלכתא כלישנא בתרא דאמרי לה משעה שחברותיה נושאות אותה אלא משום דלאו סימנא דסמכא הוא שיש זריזות ויש עצלות לא סמכי רבנן עלה והרמב״ם ז״ל הספרדי פסק משעה שהדם שותת ויורד:
}
\textblock{\textbf{אמר רב יהודה אמר שמואל חיה כל זמן שהקבר פתוח בין אמרה צריכה אני בין אמרה אין צריכה אני מחללין עלי׳ את השבת. נסתם הקבר אמרה אין צריכה אני אין מחללין עליה את השבת. לא אמרה אין צריכה אני מחללין עלי׳ את השבת. מאימתי פתיחת הקבר. אמר אביי משעה שתשב על המשבר רב הונא ברי׳ דרב יהושע אמר משעה שהדם שותת ויורד ואמרי לה משעה שחברותיה נושאות אותה באגפיה.} כתב רבינו הגדול בהלכות דייקי רבואתה ואמרו מדלא אמר רב הונא עד שיתחיל הדם להיות שותת ש״מ קודם שתשב על המשבר קאמר וכאביי עבדינן ולא מחללינן שבתא עד שיתחיל הדם להיות שותת ויורד ותשב על המשבר. והדין מימרא רבואתה קשי׳ עלי׳ חדא דרב הונא בריה דר״י היינו טעמ׳ דלא אמר עד שיתחיל הדם להיות שותת משום דאיהו נמי אמאימתי פתיחת הקבר קאי וסוגי׳ דלישנא הכי וכדתניא בעלמא (ברכות ב׳ ב׳) מאימתי מתחילין לקרות שמע בערבין משעה שקדש היום בערבי שבתות דברי ר׳אליעזר ר׳ יהושע אומר משעה שהכהנים נכנסים לאכול בתרומתן סימן לדבר צאת הכוכבי׳ שיעורא דר׳ אליעזר קדים וקאמר משעה ובפסחים [ב׳ ב׳] מאימתי ארבעה עשר אסור בעשיית מלאכה ר׳ אליעזר בן יעקב אומר משעת האור. ר׳ יהודה אומר משעת נץ החמה. ותנו רבנן [מועד קטן כ״ג א׳] לענין אבילות מאימתי כופין את המטות. משיצא מפתת הבית דברי ר״א ר׳ יהושע אומר משיסתם הגולל. וכן כיוצא בהן הרבה. (ובדידן) [ובר מן דין] אמאי פסקי כאביי והא קי״ל ספק נפשות להקל וכההיא שמעתא גופי׳ אתמר בלישני אמר לי׳ רבינא למרימר מר זוטרא מתני לקולא ורב אשי מתני לחומרא הלכה כמאן א״ל הלכה כמר זוטרא דספק נפשות להקל. הילכך מסתברא דשיעורי לא תלו בהדדי. ואית בנשי דלא יתבה על משבר כלל שהולד ממהר לצאת. אלא משעה שישבה על המשבר או ששותת הדם ויורד או שחברותי׳ נושאות אותה באגפי׳ הריהיא בחזקת מסוכנת ועושין לה. והרב ר׳ משה ז״ל כתב משעה שהדם שותת ויורד. סבור הרב ז״ל לומר שהוא השיעור הקודם. ופוסק בספק נפשות להקל:
}
\textblock{\textbf{עד מתי פתיחת הקבר.} ואסיקנא, נהרדעי אמרי חיה. שלשה שבעה ושלשים. שלשה בין אמרה צריכה אני בין אמרה אין צריכה אני מחללין עלי׳ את השבת. שבעה אמרה צריכה אני מחללין. אמרה אין צריכה אני אין מחללין. וסתמ׳ נמי כמי שאמרה צריכה אני ומחללין. שלשים אפי׳ אמרה צריכה אני אין מחללין עלי׳ את השבת. אלא עושין לה ע״י ארמית כדרב עולא ברי׳ דרב עילאי. דאמר כל צרכי חולה נעשים ע״י ארמי בשבת. וכל היכא דאמרינן מחללין. בין אמרה צריכה בין אמרה אין צריכה. בכגון דליכא חכמה ורופא הוא אלא שהדברים שעושיןלחיה בחול ידועים הן. לפיכך אעפ״י שאמרה אינה צריכה אני שיחללו עלי את השבת ויכולה אני להמתין עד הערב כיון שחברותי׳ עושין אותו בחול לכל חיה וכלן צריכות לכך הרי היא בחזקת מסוכנ׳ לשעתה לגבי מלאכה זו. ומחללין עליה את השבת. ואין שומעין לה דילמא תונבא הוא דנקיט לה. אבל לאחר שלשה אינה בחזקת מסוכנת לשעתה אלא אם אמרה אינה צריכה שומעין לה שלשים אפי׳ אמרה צריכה אני. כיון שאנו יודעים שאין זה חולי אחר ואף היא אינה אומרת כן הכל בקיאין דמשום לידה אין לה סכנה לחמין ולשאר צרכי היולדת עד הערב שיעשו לה בחול. והכי פריש רב אחא בשאלתות. ואלו אשה שילדה בשבת. ואפילו חל שלישי שלה בשבת. אע״ג דהיא אמרה בריאה אנא. ולא צריכנא מדורה חמין. ורופאין נמי אמרי לא צריכה אמור רבנן יולדת כל שלשה ימים הראשונים. ההוא שריות לאו בריותא הוא ועבדינן לה מדורה. ומחמין לה חמין. ומבשלין לה בשילא. וכל צרכיו לה לחיה עבדינן לה. ומיהו הא דקאמר גאון אפי׳ היא ורופאים נמי אמרי לא צריכה לא מסתברא כוותי׳ הרבה שבחול אינן עושין מדורה וחמין בכל יום. ובשאלתות נמי, חיה כל שלשים אסורה להתענות ביום הכפורים. ולא דייקי לן דהא שלשים באמרה צריכה אני תליא מילתא. ואנן נעילת הסנדל תנן. אכילה ושתיה דאיסור כרת לא תנן. ועוד מדאמר שמואל לחיה שלשים. והוינן בה למאי הילכתא ואמרינן לטבילה. ולא מסיימינן בה לתענית שמע מיניה לאו חיה הוא לענין תענית אלא מיהו שלשה ודאי אינה מתענה. שבעה אם אמרה אינה צריכה מתענ׳ והולכ׳. ואם לאו מאכילין אותה. מכאן ואילך הרי היא ככל הנשים להתענות ביוה״כ. אמררב יהודה אמר שמואל. עושין מדורה לחי׳ בשבת. סבור מינה לחיה אין לחולה לא. בימות הגשמים אין בימות החמה לא. איתמראמר רב יהודה אמר שמואל הקיז דם ונצטנן עושין לו מדורה אפי׳ בתקופת תמוז. כל שכן חולה הצריך שעושים לו מדורה. אבל הרב ר׳ משה ז״ל כתב עושין מדורה לחיה ואין עושין מדורה לחולה. ולא נתכוונו דבריו:
}
\newchap{פרק \hebrewnumeral{19} רבי אליעזר דמילה}
\newsection{דף קל}
\textblock{}
\textblock{מתני׳: \textbf{אם לא הביא מע״ש מביאו בשבת.} מקשו בתוס׳ וליתי תינוק לגבי מקום איזמל דהו״ל אסורא דרבנן דק״ל באדם חי נושא את עצמו ואיכא דמפרקא לה בדלא אפשר שיש לו לולד להחזירו אצל אמו וכשנימול הו״ל כפות דחולה הוא והא ודאי לאו מילתא דתינוק קטן כ״כ ככפות דמי אפי׳ קודם מילה. וי״א שאפי׳ אפשר להחזיר שלא לחלל את השבת בהבאה זו לא מהדרי׳ מדאמרי׳ בגמ׳ דרך חצרות וגגות וקרפיפות שלא ברצון ר״א אלמא לא מהדרי׳ כלל שלא לחלל את השבת כיון דאמר רחמנא מכשירי מילה אפי׳ בשבת ואע״ג דאמרי׳ בגמ׳ אי דאיכא אחר ליעבד אחר משום דכ״מ שאתה מוצא עשה ול״ת אם אתה יכול לקיים את שניהם מוטב ה״מ במקום שאפש׳ שיתקיימו שניהם לגמרי אבל שבת ומילה הואיל וניתנה שבת לדחות אצל מילה גופה ניתנה לדחות אף במכשירי מילה ואין מחזירין עליהן אם א״א בענין אחר כיון שמלאכה זו לצורך מילה או מכשוריה היא נעשית דכולי מעשה במיל׳ אריכתא הוא. ומסתברא הכי מדלא מצריך ר״א לאתויי איזמל בשעריה כדאמרי׳ במפנין לגבי חיה אלמא כי אורחא עביד ואפ״ אפשר לשנויי לא משנינן וליכא דוכת׳ דלא מצי לאתויי איזמל דרך פטור בשערו בפיו או באחר ידו ובמנענו בחד מהנך גווני דתנן בפ׳ המצניע פטור א״ו מכשירי מילה כמילה עצמה וניתנה שבת לדחות אצלן ומסקנא נמי בפ׳ המוצא תפלין הכי משמע דמכשירי מצוה לר״א אע״ג דאפשר לשנויי לא משנינן וכ״ש במכשירי מילה. ומיהו אם אפשר להביא איזמל דרך רה״ר אסור לעשות כלי בשבת מפני שמלאכה זו מעיקרא שלא לצורך הוא ונמצאת עשיי׳ זו לצורך הכלי לא לצורך המצוה ולמה זה דומה למי שיש לו כלי בביתו והלך והביא אחר דרך רה״ר. ואפשר לומר עוד דשאני דרך גגות דלאו דרך הבאה היא ומצוה מתאחרת בכך א״נ שלא כרצון ר״א דקאמר (להו ע״כ) אלא שאין זו הוראתו של ר״א [אלא כר״ש], אבל הלשון האחר נראה כמו שפירשתי: }
\textblock{הא דתניא \textbf{כשם שאין מביאין אותו דרך רה״ר כך אין מביאין אותו דרך גגות חצרות וקרפיפות.} נראה לומר דכי דחינא למילה עד מחר ה״מ בדליכא גוי אבל איכא גוי מביאו דרך גגות וקרפיפות לדעת רבינו אלפסי ז״ל. ולפ״ד הגאונים ז״ל אפי׳ דרך רה״ר ואפי׳ לעשות איזמל לכתחלה כיון דאמירה לגוי שבות דלית ביה מעשה הוא שרי. וק״ל דגבי מת אשכחן דההוא שכבא דהוה בדרוקרת ושרא להו ר״נ לאפוקי לכרמלית משום שגדול כבוד הבריות שדוחה את ל״ת שבתורה ואפ״ה לא שריא אמירה לגוי כדאמרי׳ במס׳ ר״ה דשבת ויה״כ לא אפשר בעממין אלמא אמירה הוא אמירה לגוי במלאכה גמור׳ משבות עצמה ע״י ישראל עצמו וי״א ה״ט שאין קוברין אותו בעממין לפי שהוא בזיון למת שחללו עליו את השבת כדתנן לקמן ה״ז לא יקבר בו עולמית ואלו בשאר מלאכות בכדי שיעשו הותרו והוא טעם נכון לגאונים ז״ל. ומפני זה הטעם למדנו שטעה המתיר צרכי מילה בי״ט שני ע״י ישראל מק״ו ממת דשויוה רבנן כחול לפי שהמת חמור ממילה שהרי התירו בו כרמלית בשבת מה שאין כן במילה ולא אסרו בעממין אלא מפני כבודו וכ״ש לדברי רבינו ז״ל שאסר שאין ספק שמת חמור ממילה. וראיתי לרבותינו הצרפתים ז״ל שהכריעו כדבריו מההיא דאתמר התם בההוא ינוקא אחרינא דאשתפוך חמימי׳ א״ל רבא לישיילוה לאימי׳ אי צריכה ליחם גוי אגב אימי׳ אלמא אי לא צריכה ל״א לגוי דליחום ליה לינוקא ומדחיי׳ מילה וזו ראיה גמורה שאפי׳ אם באו הראשונים לשבש הנוסחאות גם בזו ולגרוס ליחום ליה אגב אימיה א״א להם להתיר אלא ע״י גוי (ועוד) דמלאכה דישראל לא מרבי׳ בשביל הקטן כדאמרי׳ במנחות בחולה (לשתי גמורו׳) יא) וכו׳ כדאית׳ התם. ויש משיבין על דבריו מהאי דאמרי׳ הלוקח בא״י כותבין עליו אונו אפי׳ בשבת ואע״ג דאמיר׳ לגוי שבות הכא משום יישוב א״י לא גזרו ביה רבנן אלמא אמירה אפי׳ לעשות מלאכה גמורה במקום מצוה ל״ג רבנן. ולדידן ל״ק ולא מידי דהתם מצוה ותועלת לכל ישראל הוא שלא תחרב ארן קדושה ועוד שאין אנו שוקלין מצות זו בזו ואין אנו יודעין מתן שכרה של מצוה כדי שנא׳ כמו שהתירו זו נתיר אף זו כמו שמצינו שהתירו שבות עצמה ע״י ישראל גבי מת ולא מצינו לו היתר למילה אע״פ שמצות עשה שבה יותר מפורשת בתורה ונכרתו עליה י״ג בריתות וכן מצינו במקצת מקומות שאמרו במקום פסידא ל״ג רבנן בשבות עצמה כההוא דצינור דמס׳ כתובות ובשאר מקומות ואין אנו מתירין כל שבות משום פסידא ואין אומרים בדברים הללו זו דומה לזו. וכן אני אומר שאין מתירין אמירה לגוי אפי׳ בדבר שהוא משום שבות אלא לגבי מילה לפי שניתנה תורה לדחות אצלה אבל בשאר מצות אין דוחין שבות כלל אפי׳ ע״י גוי שאין לומר היתר בשבות דשבות יותר משבות דמלאכה חוץ מזו וכן לענין חולה דאמרי׳ כל צרכי חולי נעשין ע״י ארמאי בשבת בין במלאכה גמור׳ בין בשבות אמרו כדמוכח לעיל ובכמה דוכתי כי ההוא דבפ׳ הדר׳ דמחמין לאימי׳ ע״י גוי בשבת ואמרי׳ בפ״ב דביצה אמימר כחל מנוי וא״ל רב אשי מאי דעתיך משום דצרכי חולה נעשין ע״י גוי מר הא מסייע בהדיה דקא עמיץ ופתח א״ל מסייע אין בו ממש והא כוחל אפי׳ בישראל שבות הוא ואין נעשה ע״י גוי אלא גבי חולה שמלאכה עצמה מותרת בו וכן משום פסידא אין מתירין דתנן נכרי הבא לכבות אין אומרים לו כבה ואע״פ שהוא שבות דמלאכה שא״צ לגופה ומתני׳ ר״ש קתני לה אלמא אמירה דשבות אפי׳ במקום פסידא אין מתירין אותה. והר״מ הספרדי ז״ל התיר שבות דשבות לגבי שופר ולולב: }
\newsection{דף קלא}
\textblock{הא דאמרינן \textbf{מ״ט ואר״י לפי שאין קבוע להם זמן.} ה״ק מ״ט לא התירה התורה מכשירין של אלו כמו שהתירה את של אלו אבל לא איבעי׳ לן מ״ט דר״א באלו דכיון דלא רבינהו רחמנא ממילא ממעטי דכל הנך מכשירין דדחי שבת מקראי מרבי׳ להו בשמעתין. וי״מ דהכי קא מיבעי ליה מ״ט ממעט להני הא אפשר לה לאייתיונהו מסוכה דמה פרכת מה לסוכה שכן נוהגת בלילות כבימים מזוזה נמי נוהגת בלילות כבימים ולילה נמי לרבנן זמן ציצית הוא וראיתם אותו מוקמי׳ ליה לכסות סומא:
}
\textblock{\textbf{הואיל ובידו להפקירן.} מלתא דשוי׳ לתרוייהו קאמר אבל גבי ציצית אפשר לו שלא ללבוש טלית. ורש״י ז״ל פי׳ שבכל יום שמשהה הבגד בלי ציצית עובר בעשה ואפי׳ מונחת בקופסא וזהו שלא כהלכתא דאנן קיי״ל ציצית חובת גברא ולא חובת מנא כדכתב רב אלפס ז״ל. ועי״ל דמש״ה אמר בידו להפקירן שמכיון שהפקירן אף על פי שלבשן פטורין דהו״ל טלית שאולה שכל שלשים יום פטורה מן הציצית ומכאן ואילך נמי מד״ס בלבד היא חייבת אבל מן התורה פטורה דכסותך אמר רחמנא וכן בבית אע״פ שהוא דר בה בשל הפקר פטורה היא מן המזוזה דלא ביתך הוא ולא בית של רבים הוא:
}
\textblock{הא דאמרי׳ \textbf{אלא גמר שבעת ימים מלולב מה להלן מכשירין וכו׳.} אי קשי׳ הא לאו מופנה מב׳ צדדין הוא דע״כ שבעת ימים דגבי לולב צריכה, וכן ט״ו ט״ו דלקמן ודאי לאו מופנה וא״ל דכיון דגלי רחמנא בכל הנך מצות דמכשירין דוחה שבת גלוי מילתא בעלמא הוא ולמדין בג״ש ואין משיבין ומיהו במה מצינו ליכא למיגמר דכל דהו פרכינן במה מצינו:
}
\newsection{דף קלב}
\textblock{\textbf{תפילין דכתיב בהו אות לידחו.} פי׳ מכשיריהן דאלו נטילתן אינה דחוי׳ אצל שבת כדאמרי׳ בפ׳ במה אשה דאפי׳ למ״ד שבת לאו זמן תפלין הוא אם יצא פטור וכן נמי הא דאמרי׳ ציצית דכתיב ביה דורות לידחי ה״נ קאמר מכשיריו לידחי דאלו ציצית עצמו דרך מלבוש הוא ומותר לכתחלה ואינו חייב בו. וא״ת והא לית להו לרבנן מכשירי מילה דוחין ל״ק דכיון דגמרי רבנן ג״ש גבי מילה דאיכא לאוקמי במילה עצמה מכשירין לא דחי אבל גבי תפילין וציצית דלא אצטריך רחמנא למישרי גופה דמצוה גז״ש כי אתא למכשירין אתיא:
}
\textblock{\textbf{יהא זר כשר בהן ויהא אונן כשר בהן.} איכא דק״ל זר הא כהן כתיב בפרשה ולדידי ל״ק דה״ק רחמנא אפשר בכהן כהן ליכא כהן יהא זר כשר כדי שלא יתעכב מלאכול בקדשים ומפרקי׳ הא אהדרי קרא בלילה. אבל בפי׳ ר״ח ז״ל מצאתי והתקיף רבינא הכי אם חס עלייהו יהא זר או אונן כשר בהן ולא יטריחם להמתין עד שימצא כהן טהור כשר לטהרם ופריק הא אהדרי׳ קרא אע״פ שהם עליו והקל בקרבנו החזירו לטהרת הכהן שנאמר והביא אותם ביום השמיני לטהרתו אל הכהן:
}
\textblock{\textbf{תניא כוותיה דר״י דלא כראב״י.} פי׳ כלפי ששנינו בבריי׳ שמיני ימול ואפי׳ בשבת ומשמע כדראב״י משום הכי אצטריך גמ׳ למימר דלא כראב״י לומר דכד מעיינת בי׳ שפיר רישא וסיפא דלא כראב״י וכדמתרץ לה ואזיל ומיהו דלא כרנב״י נמי הויא דאיהו אמר ג״ש וכ״ש למ״ד הלכה דהא איתותב. וא״ת לר״י מכשירי מילה לר״א מנ״ל א״ל לר״א מילה מקרא נפקא ליה מכשירין דנין אות ברית ודורות [הלכות גדולות] ולהאי לישנא היינו נמי דל״א דלא כרנב״י דלר״א מיהא שפיר קאמר והני תנאי לא דרשי וי״ו דוביום ולמאן דדריש צ״ע וי״ו דמחוסרי כפרה מאי דרשי בהו. ובנוסחא דבה״ג לא כתיב אלא תניא כוותיה דר״י סתם ור״ש פי׳ בב״ב (מה:) דלאו כלום משמש אלא כי אתמר בבי מדרשא בהאי לישנא תניא דלא כפלוני אמרי הכי וה״נ הכי אתמר, דר׳ אחא שמיע להו למאן דאותבה ואמרה הכי:
}
\textblock{ה״ג: \textbf{ת״ל בשר מ״ט דאתי עשה ודחי ל״ת.} וכ״כ בכל הנוסחאות וכ׳ עלה ה״ר יהוסף הלוי בן מג״ש ז״ל בתשובה מתחזייא מילתא לכולהי תלמידי דלישנא אחריתי הוא דכיון דפריש לה גמרא להא מתניתא ואסקה כולה למאי מהדרי׳ בתר הכי למימר מ״ט. ולאו הכין הוא, אלא מעיקרא פריש מילה בזמנה דאתי׳ בק״ו משבת והדר אצטריך לפרושי מילה שלא בזמנה היכא ניחא לי׳ מעיקר׳ דאתי עשה ודחי ל״ת ולבסוף לא ניחא ליה הא דמשום עשה ול״ת הוא:
}
\textblock{\textbf{רבא אמר מילה בזמנה ל״צ קרא שבת חמירא דוחה צרעת לא כ״ש.} פי׳ לא כדקא מתרצת ליה מעיקר׳ מילה דוחה דאתיא בק״ו שבת חמירא דוחה צרעת לא כ״ש ואו אינו דקאמר ה״ק ממאי דשבת חמירא וכו׳ אלא תריץ הכי או אינו תינח מילה בזמנה שלא בזמנה מאי איכא למימר ת״ל בשר ואם אינו ענין למילה בזמנה תנוהו ענין למילה שלא בזמנה א״נ כולה כלישנא דמ״ט דאתי עשה ודחי ל״ת:
}
\textblock{והא דאמרי׳ \textbf{נגעים טהורים מא״ל.} קים להו לרבנן דלא יקוץ אפי׳ נגעים טהורים משום עבודה אבל אנו במקום דמדחיי׳ עבודה לא אשכחן דבשלמא בטמאים קיי״ל בפסחים דטומאת מת נדחית בקרבן צבור ולא זבין ומצורעין ונדחין לפסח שני אבל טהורים לא ידעי׳ מנ״ל אלא דהכי קים להו לרבנן:
}
\newsection{דף קלג}
\textblock{ודאמרינן \textbf{כגון מילה בצרעת סדין בציצית.} נ״ל דאליבי׳ דלישנ׳ קמא איפריך הכי אבל למאן דמתני להא דלעשות אי אתה עושה אבל אתה עושה בסיב וכו׳ ואוקמה אביי לר״י וכיון דבשאינו מתכוין ליטהר אפי׳ לדבר הרשות מותר קרא גבי מילה ל״ל ועוד דאפי׳ לרבא נמי כיון דאוקמה בדפסיק רישי׳ ולא ימות אלמא לא אסרה תורה קוצץ בהרתו אלא במתכוין להטהר הא למלאכתו אפי׳ בדפסיק רישי׳ ולא ימות מותר הלכך גבי מילה מותר הוא ולמאי אצטריך בשר אלא לא בא הכתוב אלא להתיר אפילו מתכוין לקוץ במקום מילה שאינו לוקה ומותר וכדמפרש ואזיל ולא מתוקמי כר״י. ומיהו לאביי דמעיקרא ניחא ליה [כדמשני באומר וכו׳], אלא לרבא ולאביי דבסוף דמוקמי לעשות בדפסיק רישיה אכתי קשי׳ דכיון דשרי רחמנא בפסיק רישי׳ מתכוין נמי שרי אגב מלאכה דהאי נמי כמתכוין הוא. ושמא נאמר דאע״ג דגלי רחמנא בפסיק רישי׳ באומר לקוץ בהרתו הוא מתכוין אסור וגבי מילה שרי ורבא נמי מתרץ לה כדאתמר הכא לאביי אליבא דר״ש באומר לקוץ בהרתו הוא מתכוין ולא מוקי [מעיקרא] ברייתא דר׳ יאשי׳ נמי בהכי משום דקתני סתם ימול משמע ליה דאפי׳ לשאינו מתכוין מצריך ליה בשר לר״י:
}
\textblock{\textbf{ותנן נמי גבי פסח כה״ג ואר״י אמר רב הלכה כר״ע.} איכא דק׳ להו גבי פסח ל״ל הלכה אטו פסקי׳ הלכתא למשיחא כדמקשי׳ במס׳ סנהדרין בפ׳ ד׳ מיתות ובמס׳ זבחים פ׳ ב״ש (זבחים מה.), ומש״ה מעברי עלי׳ קולמס מקצת מגיהי ספרים. וזה שבוש גדול הוא, שכל עצמנו לא אמרנו בגמ׳ וצריכא אלא על מימרא דר״י וכן פרש״י ז״ל אבל מתני׳ דר״ע לא מצרכי׳ להו דאיהו פליג אדר״א בפסח ופליג נמי במילה. ובמס׳ מנחות פ׳ שתי הלחם גבי חביתי כה״ג אר״ע וכו׳ ולא מצרכי׳ הכא כלל. ובר מן דין מה הועילו בתקנתן דהא בשלהי פ׳ התכלת פסקי׳ הלכתא כאבא בן דוסתי דאמר חביתי כהן שני קמצים ובפ׳ י׳ יוחסין אמרי׳ הלכה כר׳ יוסי דאמר ממזרין ונתינין טהורין לעתיד לבא. ועוד נמי במס׳ יומא גבי כה״ג שאירע בו קרי ביה״כ ומינו אחר תחתיו דאיכא מ״ד השני אינו ראוי לכה״ג ולא לכהן הדיוט ופסיק הלכתא הכי ומתרצי׳ בכל הני כדמתרץ לה אביי לרב יוסף התם הלכה איתמר וכיון דתריצנא הכי בחד דוכתא ל״צ למיהדר ולמפרקי בשאר דוכתי. א״נ דליכא בכולהו אמוראי מאן דק׳ לי׳ הלכתא למשיחא אלא רב יוסף, אבל שאר רבנן פסקי הלכתא בכולו סדר קדשים, כך מפורש בתוספות:
}
\textblock{\textbf{מאן תנא ר׳ ישמעאל בנו של ריב״ב היא.} פרש״י ז״ל ומפשיט את הפסח עד החזה היינו כדתנן בסדר התמיד הגיע לחזה חותך את הראש ונותנו למי שזוכה בו וגבי פסח נמי משהפשיט עד החזה דיכול להוציא את האימורין בלא נימין קורעו ומוציא מיד והו״ל פירש מן ההפשט. וק״ל אי הוציא את אמוריו אפי׳ בטלטול נמי אסור ופלוגתא בשלא הוציא את אמוריו היא דלא נעשה העור בסיס לדבר האסור כדמסיק בפ׳ כל כתבי ואפשר שכיון שאמרו בסדר התמיד כשמגיע לחזה עוסק בנתוח וגומר הפשטו ואח״כ גומר נתוחו ה״נ כשהגיע לחזה קורעו ומוציא את החלבים ולדברי חכמים גומר הפשטו ואח״כ מנתח ונוטל את האליה שלא היה נוטלה עם החלבים כדי שלא יהא גמר נתוח קודם לגמר הפשט בסדר התמיד ולר׳ ישמעאל כיון שפי׳ אינו גומר ומיהו בטלטול שרי אגב אליה ובסדר התמיד נמי האלי׳ אינה מנתחה אלא לאחר הפשט מאוחרת לרוב הנתחים כדמפורש במשנת סדר התמיד (לא:). ואי קשי׳, הא אפי׳ רבנן לא שרו התם הפשט אלא בברזי דליכא אלא משום שבות כדאמר בפ׳ כל כתבי (שבת קיז.) והכא גבי ציצין שאינן מעכבין איכא איסורא דאוריית׳ א״ל ההוא רבנן קא״ל לר׳ ישמעאל לדבריך יפשיט בברזא כמו שמצילין תיק הספר עם הספר אבל אינהו אה״נ דאפי׳ כדרכו נמי מפשיט. ומדמסקי דרבנן דפליגי עלי׳ דר׳ יוסי בלחם הפנים הוא אידחי לי׳ מאי דאמרן מעיקרא דדחי׳ ליה מדר׳ יוסי ומדר׳ ישמעאל וכ״ש מדרבנן דפליגי עלייהו אלא אפי׳ לרבנן דפליגי בפסח נמי שאינו פורש עד שגומר ורבנן דקדוש החודש אפי׳ להתחיל נמי מתירין שלא תהא מכשילן לעתיד לבא לפיכך פסקו רבינו והגאונים ז״ל כברייתא זו דליכא דפליג עלה אלא ר׳ יוסי דלחם הפנים הוא יחיד במקום רבים:
}
\textblock{\textbf{הלכה כראב״ע בין בחמין שהוחמו בשבת בין בחמין שהוחמו בע״ש.} היה נראה דדוקא לאחר המילה אבל קודם המילה מזלפין משום דאוקימתא דסבי עיקר ורבנן לא שרו בין לפני המילה בין לאחר המילה אלא בזילוף וראב״ע לא שמעי׳ לי׳ דפליג אלא בלאחר מילה. אבל קבלה היא ביד רבינו שקיבל מן הגאונים ז״ל דאפי׳ לפני המילה נמי וראב״ע ה״ק להו לא מבעי׳ ביום ראשון שמרחיצין אותו כדרכו בין לפני מילה בין לאחר מילה אלא אפי׳ ביום הג׳ נמי מרחיצין אותו כדרכו בין בחמין שהוחמו בשבת בין בחמין שהוחמו מע״ש או שהם סומכין על הברייתא יותר מדיוק׳ של רבא דהא מתני׳ נמי מרחיצין קתני וברייתא לית לה זילוף אלא לאחר מילה בשלישי וכיון דהלכה כראב״ע אין לנו זילוף אלא רחיצה. וא״ת לפני מילה היאך שבת נדחית תדחי מילה כדמוכח בפ׳ הדר ובפסחים פ׳ א״ד קרי רחיצה מכשירי מילה ולר״א דפרקי׳ מותר הא לרבנן אסור ופשוט הוא אלא אנן בחמין שהוחמו קאמרי׳ ולא שנתיר להחם אלא שאם הוחמו ע״י גוי או לאמו כשאמרה צריכה אני מותר להרחיצו בהם ובא לומר שאע״פ שיש כאן משום גזירת מרחצאות מותר שאם מלין אותו בלא רחיצה סכנה הוא לו. ומיהו לאחר מילה ביום א׳ או ביום ג׳ אי אשתפיך חמימי׳ מחמין לו אפי׳ ע״י ישראל כדרך כל פקוח נפש שדוחה שבת. א״נ בין בחמין שהוחמו בשבת אלאחר מילה בלחוד קאי לראב״ע וכן הלשון עצמו שכ׳ רבינו ז״ל בין לפני המילה בין לאחר מילה בין בחמין שהוחמו בשבת תפתר לאחר מילה סמך לו על מ״ש לקמי׳ והיכא דאשתפוך בתר דאימהל וכולה שמעתא דעירובין. וכן בעל הלכו׳ אמר וה״מ דלא דחי שבת מכשירין דקמי מילה אבל מכשירין דבתר דאתמהיל כגון דאיתשיד חמימי או דאיבדור סממנא ולא אפשר אלא באחולי שבתא עלייהו סכנה הוא ומחללין שבתא עלייהו התם הוא דאשתפוכי חמימי׳ או דאיבדר סמנא או דאיפגם איזמל מקמי דמימהלי דאמרי׳ תדחה מילה למחר הכא כיון דאיכא סכנה מחללין שבתא עלי׳ כ״ז כ׳ בעל הלכות ז״ל אלא ששיבש במקום אחר ואמר מחמין חמין לחולה ולקטן ולחיה בשבת בין להשקותו בין להברותו א׳ קטן בריא ואחד קטן חולה מחמין לו חמין להברותו ולמולו בשבת וזה לא אמרוהו אלא לדברי ר״א וא״צ לפנים. אבל יש לי בכאן ספק אם היה לו חמין כדי רחיצה שלפני מילה ואין לו כדי רחיצה שניי׳ שלאחר מילה או שנשתפכו חמין שהכין לה מקמי מילה שאני אומר רוחצין אותו ומלין אותו שאין כאן מכשירין דוחין כלום ואחר שמל הרי כאן סכנת נפשות שדוחה שבת ואין אומרים תדחה מילה כדי שלא להביא אותו לסכנה ונדתה שבת אלא מילה עצמה דוחה שבת וסכנת נפשות נמי דוחה ואין למצוה אלא שעתא שאין לדחות מילה מפני דחיית שבת שיבא לאחר מכאן מפני הסכנה. וראיתי מי שסובר שאין מלין אא״כ היה לו חמין וסמנין לאחר המילה ואם נשפכו קודם מילה תדחה (לאחר מילה) מחללין ולשון בעל הלכות מסייעו לפי פשוטו אלא שהדברים עצמן מכריעין כמ״ש. ומתני׳ נמי דייקא דתנן לא שחק מע״ש לועס בשניו ונותן אם לא התקין מע״ש כורך על אצבעו ומביא ואפי׳ מחצר אחרת ולא קתני בשאין לו כמון בביתו או שא״א ללעוס בשניו תדחה ומדקתני תקנתא ולא תני דחיי׳ ש״מ כדאמרן. ועוד דכורך על אצבעו ומביא דרך רה״ר איסורא דרבנן הוא ומשאוי הוא לו וא״נ מחצר אחרת שלא נשתתפו נמי איסורא הוא ונדחה מילה מידי דהוה אאיזמל שאין מביאין אותו דרך שער בשנוי ואפי׳ במבוי שאינו משותף אלא שאין צרכי סכנה שלאחר מילה דוחין אותו מתחלה אלא מל ואח״כ מחלל לצורך הסכנה כדפי׳. וי״א דראב״ע וכ״ש בשני קאמר. ואי ק״ל הא כתיב ביום הג׳ בהיותם כאבים אלמא טפי מסתכן בג׳ מבשני לאו קושי׳ הוא שבג׳ היו חלושים ביותר ולא היו יכולין לברוח ולהלחם אבל ביום הב׳ עדיין לא תשש כחם אע״פ שסכנת יום ב׳ גדולה משל ג׳ ויום ראשון תוכיח שלא נגעו בהן מפני שעדיין כחם עליהם אע״פ שהן מסוכנין יותר. ובירושלמי ויהי ביום הג׳ בהיותם כואבים כתוב כאן ולא בהיותו כואב בשעה שכל איבריהם כאבים עליהן. ודעת רש״י ז״ל שאפילו לדברי חכמים יום ב׳ כיום א׳, וזה אינו נכון:
}
\newsection{דף קלה}
\textblock{ה״ג וכ״ג ר״ח ז״ל ור״י אלפסי ז״ל: \textbf{אמר רבה מנא אמינא לה דתניא ר״א הקפר וכו׳ לאו מכלל דת״ק סבר ד״ה אין מחללין.} ופירושה, דקים לי׳ לרבה דת״ק דר״א הקפר הוא רשב״א וה״נ אשכחן לה בתוס׳ ומש״ה דייק רבה אי ס״ד לר״ש צריך להטיף ממנו דם ברית אפי׳ בשבת דודאי ערלה כבושה היא היכי א״ל ר״א לא נחלקו ב״ש וב״ה בחול שצריך להטיף אמר ת״ק אפי׳ בשבת וא״ל איהו אפי׳ בחול אלא ש״מ דת״ק דהיינו רשב״א בחול אמר ולא בשבת ומ״ה א״ל ר״א מה שאתה אומר שלדברי שניהן בחול צריך להטיף מודינא לך אבל מחלוקת בשבת הוא נמצא רבה דאמר כד״ה ורב יוסף דלא כחד וא״נ ר״א הקפר את״ק קאי ולדידי׳ קא״ל כיון דר״א הקפר ורשב״א תרוייהו חד לישנא קא״ל לא נחלקו שצריך להטיף לתרוייהו בחד גונא משמע צריך להטיף בחול ולא בשבת ואקשי׳ א״ה ר״א הקפר לא בא ללמד אלא דברי ב״ש. [ופרקינן ה״ק, לא נחלקו ב״ש] וב״ה בדבר זה שאתה אומר כלום בגר שנתגייר כשהוא מהול דלד״ה צריך להטיף ממנו דם ברית, כן נ״ל. והא דפסק שמואל הלכה כרשב״א לא פסק להוציא ממחלוקת של ר״א אלא להוציא מת״ק ומדרב דפוסק הלכתא כוותי׳ אבל בגר שנתגייר כשהוא מהול דהיינו ערבי מהול לא אפסיקא הלכתא בהדיא כההוא דאמרי׳ בפסחים מאי לאו לאכול לא לבער וכיון שכן דר״א הקפר עדיפא משום דסוגיין כוותי׳ כדתני׳ בפ׳ הערל אין לי ביום אלא שנמול בזמנו לט׳ לי׳ לי״א עד שנתגייר כשהוא מהול מנין אלמא צריך להטיף ממנו דם ברית ועוד דקיי״ל בפ׳ החולץ כר׳ יוסי דאמר לעולם אינו גר עד שימול ויטבול. וזה הכלל משמע בין לערל בין לנולד כשהוא מהול בין לערבי מהול בין לגבעוני מהול וכן הא דתני׳ התם הרי שבא ואמר מלתי ולא טבלתי מטבילין אותו ומה בכך דר״י ר׳ יוסי אומר אין מטבילין משום דחייש דלמא ערבי מהול הוא וצריך להטיף ממנו דם ברית וכן פרש״י ז״ל ואפסקא הלכתא התם כר׳ יוסי לפיכך אמרו בעל הלכות ורבינו הגדול ז״ל דגר דאתמהיל בארמיותי׳ צריך להטיף ממנו דם ברית. ויש דוחה דלמא רבי יוסי משום שמא נולד כשהוא מהול הוא דאמר אין מטבילין אותו. ויש לסייע דברי ראשונים, דכיון דא״ר יוסי שאינו גר ליכנס תחת כנפי השכינה אלא במילה וטבילה ולא גמרי׳ מאמהות גר שנתגייר כשהוא מהול ודאי צריך להטיף ממנו שאין נכנסין תחת כנפי השכינה בטבילה בלבד ומילה ראשונ׳ לא מהניא שלא נעשית כהלכתה ואע״פ שמל ערל היה כדמוכח בהדיא בר״פ הערל אבל יש לדחוק בגר שנתגייר כשהוא מהול הואיל ומהיל הוא לגמרי הו״ל כאשה ונכנס תחת כנפי השכינה בטבילה בלבד דבשלמא אדם אחר לא גמר ר׳ יוסי מאשה שאין דנין אפשר משאי אפשר אבל האי אי אפשר הוא ומיהו מסתברא שאפשר מקרי שאפשר להטיף ממנו דם ברית. ור״ח ז״ל כ׳ גר שנתגייר כשהוא מהול אין לו תקנה אבל בניו נמולין לשמונה ונכנסין בקהל דהא איגייר בטבילה וכבר חשוב להכשיר זרעו אבל בעצמו לא. ולא ידעתי זו מנין לו לרב ז״ל, דאי כרשב״א א״צ להטיף קאמר לומר שנעשה ישראל גמור להכשיר עצמו בטבילה ושמא הוא סבור דכיון דקיי״ל אינו גר עד שימול ויטבול וזה א״א למולו אין לו תקנה ואינו נכון כלל דהכא משמע שהטפת דם ברית כמילה וזהו שאמר ר׳ יוסי התם אין מטבילין אותו כלומר עד שיטיף שאלמלא אין לו תקנה לעולם מטבילין אותו להכשיר זרעו אבל לא הוא עצמו. ונ״ל שהמל גר ועבד שנתגייר כשהוא מהול מברך אקב״ו להטיף דם ברית מן הגרים או מן העבדים שאין הטפת דם זו מפני ספק אלא חייבין אנו להטיף מהם דם זו של ברית ונכנסין בה תחת כנפי השכינה. אבל בישראל שנולד כשהוא מהול, כיון שא״צ להטיף אלא מפני ספק ערלה כבושה הורו הגאונים ז״ל שאין מברכין שאין זו מילה וה״ה שאין מברכין עלי׳ להטיף שאף אותו הטפה אינה אלא משום ספק ועוד שלא תקנו ברכה זו בישראל כלל וה״ה להכניסו בבריתו של אברהם אבינו שאין מברכין דשמא אין כאן ערלה כלל ולא נצטוינו להטיף דם ממנו ונמצא שאין זו דם ברית. ומי שהורה לברך עלי׳ להטיף או לאחריה כורת הברית - טועה גמור:
}
\textblock{\textbf{אמר רבא א״ר אסי כל שאין אמו טמאה לידה אינו נימול לשמונה.} משמע לן דלית הלכתא כוותי׳ משום דקם לי׳ כרב חמא והוא יחידאה ואלמלא לא הי׳ אמורא חולק עליו היינו אומרים שהלכה כמותו אבל מכיון דפלוגתא דאמוראי הוא ור״ה וחייא בר רב פליגי בה וקם ליה מ״ד נימול לשמונה כת״ק דרבים נינהו הלכתא כוותיה ועוד דרבה ורב משרשיא שקלו וטרו אליבא דת״ק ונראה שדעתו של רבינו הגדול ז״ל כך שלא פסק הלכה כדברי מי וכל היכי דלא איפסק הלכתא כת״ק קי״ל. ועוד שכתבה לכולה סוגיא דהיכי משכחת לה אליבא דת״ק. וי״א כולן נמולין לשמונה ואין דוחין שבת דספיקא הוא ולא נתכוין רבינו הגדול ז״ל לפי דעתם. ולענין עובדא ודאי מוטב שתדחה ולא מחללי עלי׳ שבת שאף לבעל הלכות ז״ל ראיתי שהדבר ספק אצלו:
}
\textblock{הא ד\textbf{ארשב״ג כל ששהה ל׳ יום באדם אינו נפל שנא׳ ופדויו מבן חדש תפדה הא לא שהה ספיקא הוי.} פרש״י ז״ל בשאינו ידוע אם בן ח׳ אם בן ט׳ שאלו בן ט׳ היינו כלו לו חדשיו דאפילו רשב״ג מודה ואי בן ח׳ ודאי לא חיי ובין לרבנן ובין לרשב״ג נפל הוא אלא בדלא ידעי׳ פליגי דרבנן סברי כיון שלא מת מיד בחזקת כלו לו חדשיו הוא ורשב״ג סבר עד ל׳ ספיקא הוא וכ״פ לקמן גבי נפל מן הגג ואכלו ארי. והאי פי׳ ק״ל טובא מהא דגרסי׳ בפ׳ הערל (יבמות פ.)א״ר אבוהו סימני סריס ואיילוני׳ ובן ח׳ אין עושין בהן מעשה עד שיהיו בן כ׳ ואקשי׳ ובן ח׳ מי קא חיי והתניא בן ח׳ הרי הוא כאבן ואסור לטלטלו בשבת אבל אמו שוחה עליו ומניקתו מפני הסכנה ומפרקי׳ הב״ע בשלא גמרו סימניו דתניא איזהו בן ח׳ כל שלא כלו לו חדשיו ור׳ אומר סימניו מוכחין עליו שערו וצפרניושלא גמרו טעמא דלא גמרו הא גמרו אע״פ שלא כלו לו חדשיו אמרי׳ האי בן ז׳ הוא ואשתהוי הוא דאשתהי אלא הא דעבד רבה תוספאה עובדא באשה שהלך בעלה למדינת הים ואשתהי עד תריסר ירחי שתא ואכשרי׳ כמאן כר׳ דאמר משתהי כיון דאיכא רשב״ג דאמר משתהי כרבים עביד מאי רשב״ג דתניא רשבג״א כל ששהה ל׳ יום באדם אינו נפל הא למדת דפלוגתא דרשב״ג ורבנן בשנולד לשמונה ושערו וצפרניו גמרו דת״ק סבר כיון שלא כלו לו חדשיו אע״פ שגמרו סימניו נפל גמור הוא ואסור לטלטלו ור׳ סבר כיון שגמרו ולד מעליא הוא ורשב״ג סבר תוך ל׳ יום ספק מכאן ואילך ודאי, ובפרקי׳ בתוספתא (טז.) תני להו לשלשתן. והשתא ק״ל, היכי מסתפקא לן בגמ׳ אי פליגי רבנן עלי׳ דרשב״ג ולא פשטו׳ אלא מכלל דרב יהודה אמר שמואל והרי כאן ג׳ מחלוקות מפורשים ותו דבכולה שמעתין משמע דרבנן לקולא פליגי וכדאמרי׳ נמי לקמן אם אשת כהן אינה חולצת היינו משום דסמכי׳ אדרבנן דאמר ולד מעליא הוא וכדמפרש עלה בהדיא בפ׳ החולץ והא רבנן לחומרא פליגי עלי׳. ואפשר לומר דלא שמיע להו ההוא ברייתא ואי שמיע להו סברי דילמא הכי קתני איזה בן ח׳ שאסור לטלטלו כל שלא כלו לו חדשיו כלו׳ שאנו יודעין בודאי שלא כלו לו חדשיו וה״ה שצריך נמי שלא גמרו סימניו ורבי אומר סימניו מוכחין עליו שכיון שלא גמרו בידוע שלא כלו לו חדשיו ואין אתה צריך לחזר אחר חשבון עבור חדשים אבל בשגמרו סימניו ולא כלו לו חדשיו ל דבר ת״ק ובא רשב״ג ופי׳ דכל ל׳ יום ספק מכאן ואילך ודאי ובא רב יהודה ופי׳ דפליגו רבנן עלי׳ והדרא מתני׳ לפשטא וכדפרי׳ לעיל. והא דאמר בפ׳ החולץ ורמיזא הכא דרבנן סברי תוך ל׳ ולד מעלי׳ הוא ואיתא נמי בפ׳ יוצא דופן מאן חכמים רבי הוא דקים להו לרבנן דגמ׳ דרבים ס״ל כוותיה א״נ מכיון דרבי ורשב״ג סברי משתהי הו״ל לת״ק כיחיד לגבייהו ולא חיישי׳ ליה כלל ומש״ה באשת כהן אינה חולצת דאפי׳ לרשב״ג אינו אלא ספק וסמכי׳ אדרבי דאמר ודאי ולד מעלי׳ הוא ולא אסרינן איתתא אגברא. ואי ק״ל, כיון דרשב״ג בבן ח׳ קאמר היכי אמרינן בגמרא בפשיטא ממהל היכי ממהלינן ליה מאן אמרה דבן ח׳ ודאי נימול לשמנה אע״פ שגמרו סימניו דהא לת״ק דרשב״ג ודאי לא מהלינן ליה א״ל מעשים היה להם בכל יום שמלין אותן מכיון שגמרו סימניהון אע״פ שאין יודעין אם כלו חדשיהם או שיודעין שלא כלו וכן כולה שמעתא דעגל שנולד כולן בנולדין לספק חדשים בסימנין גמורין מתוקמי דבלא סימנין לכ״ע ספק נפל אלא (קא ס״ד דבסימנים) גמורים הן ומ״ה לרבי דהיינו רבנן ולד מעלי׳ נינהו [הוא]. ואי ק׳, היכי מתוקמי אליבי׳ קרא דופדויו יש לומר ה״ק התורה לא נתנה דברים לשעורין ולפי שיש שצריכין לשהות עד ל׳ יום כגון ודאי בן ח׳ וספק בן ח׳ נתנה שיעור א׳ לכל ל׳ יום כנ״ל פירושא. עד שראיתי להר״ר אברהם ז״ל שאמרה בלשון אחר כדברי רש״י ז״ל דרשב״ג ורבנן בתרתי פליגי רבנן סברי כל שנולד חי ולא פיהק ומת מיד חזקה לט׳ נולד ואם בידוע שנולד לח׳ ה״ז נפל גמור ורשב״ג סבר הכל תלוי בל׳ יום אם שהה ל׳ יום אפילו נולד לח׳ אינו נפל דקסבר אפשר הוא דאשתהי ומקיים ל׳ ואם לא שהה ספיקא הוי עד שיודע לך שנולד לט׳ וכולה שמעתא סלקא ליה שפיר דקים לן דפליגי בבן ח׳ ודאי בגמרו סימנים ולא ידעי׳ אי פליגי בנולד לספק חדשים אי בעו רבנן דליקום בגווי׳ שכלו לו חדשיו. ולענין פסקא משמע לכאורה דמהלי׳ ספקות ממ״נ דאי חי הוא שפיר קמהיל ואי מת הוא כלומר נפל מחתך בבשר בעלמא הוא דהכי סלקא שמעתא. אבל מצינו לרבינו הגדול שכ׳ להך ברייתא דקתני ספק בן ט׳ ספק ח׳ אין מחללין עליו את השבת בפסק הלכה ואנן בגמ׳ אוקימנא אליבא דר״א דלית הלכתא כוותי׳. אלא משמע דרבינו הגדול ז״ל לא ס״ל הא דאמרי׳ מהלינן ליה ממ״נ וכו׳ ובקשנו לו חבר ומצאנו לר״ח ז״ל שכ׳ בפירושיו ולית הלכתא הכי דשינויא הוא ואיברא קושין דבגמ׳ דאקשי׳ מימהל היכי מהלינן ליה לאו קושי׳ אלימתא הוא דאפשר דלא מהלינן לי׳ וכדפרי׳ לעיל אלא מיהו כיון דגמ׳ סליק דמהלינן לי׳ ממ״נ היכי יכלי׳ לדחוי׳ סוגיא דגמ׳ בלא טעמא. ונראה מדברי ר״ח ז״ל שהוא מפרש הא דאמר אביי כתנאי דאעיקר מלתא דרשב״ג קאי דקאמר בבן ח׳ בבהמה שגמרו סימניו ספיקא הוי דמ״ד שחיטתו מטהרתו סבר כיון שגמרו סימניו אין בו ספק אלא חי גמור הוא ומ״ד אין שחיטתו מטהרתו סבר ספק מת והתורה טמאתו להודיעך שאינו ולד עד שישהה ח׳ ימים ולפיכך אין שחיטתו מטהרתו דהו״ל כמחתך בשר בעלמא וא״ל רבא אי סבר מר חי הוא ולא חיישינן לי׳ לישתרי באכילה אלא דכ״ע מת הוא כרשב״ג וא״ה לאו מחתך בשר בעלמא הוא שכ״ז שחיותו עליו כחי הוא ושחיטתו מטהרתו טעמא אחרינא הוא משום דלא דמי׳ לטרפה ולרשב״ג לאחר ח׳ ימים אפי׳ באכילה שרי ובתוך ח׳ הו״ל ספק נבלה (ולפיכך) אחר שכתבתי נמי י״ל כלפי שאמר רשב״ג כל שלא שהה בכל הנולדין בין לח׳ בין לספיקן ספקות הן אע״פ שגמרו להן סימנין אמר אביי כתנאי דאפי׳ לח׳ ודאי אמרי ר׳ יוסי בר״י ור״א בן ח׳ חי והיינו בשגמרו סימניו דאי לאו חי גמור הוא הו״ל מחתך בשר בעלמא ורבא דחאה דמת הוא כרשב״ג ואפ״ה שחיטתו מטהרתו ואפי׳ אין בהן סימנין גמורין. וזה הפי׳ יותר נכון מפרש״י ז״ל, שפירש בזה כתנאי אי הוי מחתך בשר בעלמא אי לא דכל היכי דאתמר בשמעתא חי הוא היינו דחי לעולם ואינו נפל כלל ועוד דלדידי׳ מאי פירכי׳ דרבא כיון דלכ״ע נפל הוא באכילה א״א דלאו מחתך בשר בעלמא שהרי חייבין על שחיטתה בשבת כדאי׳ פא״ד בפסחים וכן על חבורה של זה חייבין ודאי אשתכח לפירושא דילן דאפי׳ למ״ד נפל הוא שחיטתו מטהרתו ואינו כמחתך בשר בעלמא ואפי׳ למ״ד אין שחיטתו מטהרתו טעמא אחרינא הוא משום דלאו דומי׳ דטריפה הוא הלכך לענין שבת חבורה הוא ואין מחללין עליה דהיכי אפשר שתהא שחיטתו שחיטה ולא תהא חבורה לענין שבת וכיון דרבא לא ס״ל ההוא טעמא דא״ר אדא מחתך בשר בעלמא הוא לחומרא נקטינן ולא מחללין שבתא ואם יש טעם אחר טוב מזה גם את הטוב נקבל. ובפ׳ נושאין על האנוסה גרסי׳ הכי ר׳ יוסי בעי אילין תינוקות ספיקות מהו לחלל עליו את השבת א״ר יוסי ב״ר בון ויאות הוא בן ח׳ נעשה כמחתך בשר שלא לצורך רבנן דקיסרין א״ר יעקב בר דוסי מי אמרר שמותר לחתוך בשר שלא לצורך נמצא לדברינו שאפי׳ נולד לספק ח׳ וגמרו סימניו אין מלין אותו בשבת אין לך נימול בשבת אלא דקים לן בי׳ שכלו חדשיו ולא כן דעת הגאונים ז״ל והמחברים:
}
\textblock{ה״ג בהלכות רבינו ז״ל: \textbf{המל את הגרים אומר אקב״ו למול את הגרים ולהטיף ממנו דם ברית וכו׳.} וק״ל למה תקנו חכמים בגר לומר למול משא״כ בשאר הנימולין שאומר על המילה היה להם לומר על מילת הגרים ואפשר מפני שעיקר מצות הגרים לגר עצמו שהוא גדול והוא ממציא עצמו לכך ונמצא כמל את עצמו משא״כ בקטנים הנימולין שהן אין להן דעת והמצוה אינה מוטלת על המוהל הזה בלבד כדאמרי׳ במס׳ פסחים פ״ק לא סגי׳ דלאו איהו מהיל ואע״ג דאבי הבן מברך נמי על המילה דומיא דעל ביעור כדמוכח התם טעמא דמילתא משום שהיא מצוה שאפשר לעשותה ע״י שליח וכל מצוה שהיא כן תקנו בה על וכן אנו מפורשים בעל ביעור אבל כאן שכיון שכל הגרים כאלו הן מלין א״ע אע״ג דאינהו אכתי לא חזי לברכה מ״מ כיון שהחובה מוטלת עליהם ולא על המל לפיכך תקנו בה למול ואפשר דמילת גר חובה גמורה על המל ובעי לברוכי למול דהא לא מתעבדא ע״י שליח אלא כל דמהיל מצוה דנפשי׳ עביד ועיקר. ותמהני עוד למה כללו שתי ברכות של מילה כאחת ולא ברכו בה תחלה וסוף כשאר הנמולין וי״ל לפי שאינו גר עד שימול ויטבול לא ברכו עליו על המילה כמו שאין אנו מברכין על הטבילה אלא הוא עצמו בעלייתו מברך עלי׳ אבל כללו הכל בברכה זו לפי שדם מילה בריתן של ישראל ועל הברית אנו מברכין ובענין הזה הוא ברכת אירוסין כמו שאפרש במקומה בס״ד. ומה שתקנו בה להטיף מהן לפי שיש כמה גרים כשמתגיירים כשהן נמולין ועיקר מילתן הטפת דם ברית לפיכך תקנו לכולן כן. ואחרים אומרים לפי שהם גדולים ודואגין על דמן הנוטף תקנו להם זאת לנחמם. ובעל ה״ג גורס: המל אומר על המילה והמברך אומר למול וכן במקצת הספרים ותרתי תקנו בתחלה ובסוף כשאר המילות ומה שאמרו בה למול דמשמע להבא אין מדקדקים בכך מפני ברכה שבסוף והדבר ניכר דלא להבא משמע ולא לשעבר שכבר ברכנו לפני׳ ברכה אחרת אלא נוסח ברכה הוא שקדשנו למול את הגרים כשהן באין לנו להכניסן בברית. ועוד דמתוך שהוא מילת חובה עליו ועלינו תקנו בה למול וכורת הברית כאחת ולפי שהיא מופלגת מן הטבילה שהוא מתגייר בה לפיכך לא הניחו לו שיברך הוא עצמו שאי אפשר וגירסת רבינו שפרשתי׳ יפה בעיני מזו:
}
\newchap{פרק \hebrewnumeral{20} תולין}
\newsection{דף קלח}
\textblock{}
\textblock{\textbf{אלא אמר אביי מדרבנן שלא יעשה כדרך שהוא עושה בחול.} פרש״י ז״ל דלאו משום אוהל ארעי הוא אלא שלא יתקן לשמור בקביעות כחול וראייתו דאמרי׳ לקמן מערים אדם על המשמרת בי״ט לתלות בה רמונים ואי משום אהל מה לי רמונים מה לי שמרים אלא אהל ליכא כלל ומשום עובדין דחול הוא. וק״ל הא דאקשינן לעיל השתא לר״א אוסופי על אהל עראי לא מוספי׳ למיעבד לכתחלה שרי מאי קושי׳ נימא דהכא לאו אהל הוא כלל ורבנן נמי משום עובדין דחול אסרי ור״א לא חייש להכי ויש לדחוק דסוגיין דלעיל בשיטה דר׳ יוסף אתמר דסבר דרך (חול) [אהל] הוא, ולרווחא דמלתא מקשי , דמאהל עראי מיהת לא נפיק. ול״נ דאביי משום אהל עראי אסר לה ומשום דא״ל רב יוסף חייב חטאת אמר איהו דליכא אלא שבות בעלמא דהיינו עובדין דחול והיינו מתני׳ דאביי דתני משמרת בהדי כילה דמשום אהל עראי והאי דשרינן ברמונים משום דהתולה לשמרים מותחה יפה וקובעה והתולה לרמונים א״צ לקובעה ולא למותחה יפה ולאו אהל הוא כלל:
}
\newsection{דף קלט}
\textblock{הא ד\textbf{יהיב רב משרשיא פרוטה לתינוק גוי וזרע לי׳ כשותא לכרמא.} בחו״ל דוקא, משום דס״ל כר״ט ומיהו בישראל הי׳ נוהג מנהג איסור דילמא אתי למיזרע מין אחר וגוי גדול נמי חש דלמא מחליף לי׳ בישראל ומסתברא דחטה וכשותא וחרצן במפולת יד הי׳ זורע א״נ חטה וכשותא בכרם שאלמלא כן הוא עצמו הי׳ מותר לזרוע חטה או כל מין אחר שירצה בין הגפנים כדאמרי׳ במס׳ קדושין לא קי״ל כר׳ יאשי׳ דאמר אינו חייב עד שיזרע חטה ושעורה וחרצן במפולת יד ולדברי מי שמפרש שלענין איסור הפירות נאסרין אע״פ שאין שם אלא מין א׳ בכרם אפשר שהיה עושה כן שלא יקדשו ומאן דגמר מיהא למין האסור לזורעו בחו״ל ע״י גוי טועה גמור הוא שאפי׳ מקיים כלאים בכרם אסור ומקדש כ״ש זורע ע״י גוי:
}
\textblock{\textbf{ואמר רבא מת בי״ט ראשון יתעסקו בו עממין.} פירש״י ז״ל מת שנשתהא כגון שמת בשבת ולמחר היה י״ט וטעמי׳ משום דאמרי׳ במעשה דמעון בי״ט הסמוך לשבת ולפי פירושו הא דאמרי׳ פ״ק דביצה לא אמרן אלא אשתהי הא לא אשתהי משהינן לי׳ רב אשי אמר אף על גב דלא אישתהי לא משהינן ליה דוקא אי״ט שני ולאיעסוקי בו אפי׳ ישראל אבל י״ט ראשון ודאי אשתהי דוקא ואפשר נמי דמר זוטרא דאמר לא אמרן אלא דאשתהי קאי אתרווייהו אי״ט ראשון ואי״ט שני ורב אשי לא פליג עליו אלא אי״ט שני אבל אי״ט ראשון דכ״ע אשתהי דוקא. ואחרים אומרים דמעשה דמעון הוא שהי׳ בי״ט הסמוך לשבת ומפני כך הצריכו לשאול אבל ר׳ יוחנן התיר לעולם ואפי׳ בי״ט שאינו סמוך לשבת. והא דאמרי׳ התם לא אמרן אלא דאשתהי אי״ט שני ולאיעסקי בי׳ ישראל משום דמחלל י״ט שני בידים אבל י״ט ראשון לאיעסוקי בי׳ עממין דכ״ע אע״ג דלא אשתהי [לא] משהינן לי׳. וי״א דמר זוטרא אתרווייהו קאי, ורב אשי נמי פליג אתרווייהו והא דקא יהיב טעמא י״ט שני לגבי מת כחול שווי׳ רבנן משום דמחללין עליו בידים הוצרך לפרש טעמו של דבר והוצרכנו לפרש משום דאי ס״ד אי״ט שני בלחוד קאי הול״ל הא דאמרת בי״ט שני יתעסקו בו ישראל לא אמרן אלא דאשתהי מדקאמר סתם ש״מ אתרוייהו קאי. וזה הלשון אינו, שאם מר זוטרא אמר בשניהן דוקא אשתהי הי׳ לו לרב אשי לפרש ולומר אע״ג דלא אשתהי לא משהי׳ לי׳ ואפי׳ בי״ט שני מ״ט וכו׳. אבל נראה שדעת הגאונים ז״ל דבין אשתהי בין לא אשתהי בי״ט ראשון יתעסקו בו עממין בי״ט שני יתעסקו בו ישראל. ומצינו לרב אחא משבחא גאון ז״ל שכ׳ בשאלתות דפ׳ אחרי די״ט שני בדליכא עממין הוא שיתעסקו בו ישראל אבל איכא עממין יתעסקו בו עממין אבל בי״ט ראשון אע״ג דליכא עממין (מדחיא) [טרחינן] ומייתי׳ להו דאי אפשר לעולם ע״י ישראל. ולא ידענא מי הזקיקו לכך דכיון דאמרי כחול שווי׳ רבנן לגמרי משמע בין אפשר בין לא אפשר. ונהוג עלמא האידנא דאע״ג דאיכא עממין שיעסקו בי׳ ישראל לגמרי ואפי׳ לחפור בו קבר. והא דאמרי׳ התם אפר למיגד ליה גלימא לאו למימר דקבר דאית בי׳ טירחא יתירא אסור אלא לומר דאפי׳ אסא וגלימא שאפשר בלא כן שהרי אפשר לכרוך אותו בתכריכיו בלא תפירה אפ״ה שרי משום כבודו של מת כ״ש בגופה של קבורה שדוחין עליו י״ט שני לגמרי והכי אתמר בה״ג והלכתא מת בי״ט שני של ר״ה מותר להתעסק בו בכל צרכיו בין צרכי דידי׳ בין צרכי קבר דאמר רבא וכו׳. ואיכא דרמו , והתנן (מו״ק ח:) אין חופרין כוכין וקברות במועד מפני זה כ׳ רש״י ז״ל במס׳ מ״ק דכ״ש בי״ט שני שאסור ולא התירו אלא למיגד לי׳ גלימא ולמיגז לי׳ אסא וכיוצא בהן אבל טירחא יתירה כגון חפירה לא טרחי׳. והראב״ד ז״ל פי׳ ואמר שמנהגן של ראשונים לחפור בקרקע חפירה כ״ש וקוברין בה לפי שעה ואח״כ קוברין בכוכין וקברותלפיכך אסרו כוכין וקברות במועד שהרי אפשר בקבר עראי שאין טורחו מרובה והביא סעד לדבר מדאמרינן בירושלמי בראשונה היו קוברין במהמורות נתאכל הבשר מלקטין עצמות וקוברים אותו בארונים אותו היום מתאכל למחר הי׳ נימח ולפ״ז הפי׳ אפשר דהאידנא שרי אפי׳ קבר גמור לפי שאין מנהג להוציאו מקברו ואם הוציאוהו שלא כדרכן נמצא המת מתבזה בכך. וזה הפירוש אינו מקובל על הלב , שמעולם לא הי׳ מנהג לכל לשנות אותו ולתתן בארון אלא למי שרוצה להעלות עצמות מתו מחו״ל לארץ ובכמה מקומות אמרו בתלמוד שאסור לשנותן ואם דרכן הי׳ להוציא משם העצמות למה קנו מקומן ועוד שכיון שמהמורות עצמן בחפירה הן אף הכוכין מותרין שלא חלקו בכיוצא בזו בין תפירה לחפירה והמהמורות עצמן חפירה גדולה ועמוקה הן כענין שנאמר יפילם במהמורות בל יקומו. ועוד אימתי היו קוברין בקברות וכוכין א״ת לאחר שנתאכל הבשר והלא בארונות היו נותנים אותו, ועוד ששיעור הכוכין מפורש בפ׳ המוכר את הבית כפי מדתו של אדם א״ו מחתלתן היו קוברין להן ולא כשנתעכל הבשר ואם כשמת במועד טומנין אותן לפי שעה ולאחר המועד קוברין אותו בקבר וכוך ה״ז גנאי גדול למת ואסור שאף הן לא הי׳ דרכן לשנותן עד שיתעכל הבשר שנותנים אותו בארונות ואעפ״כ לא הותרו הכוכין והרי אף בזמן הזה אסורים נמי לפי פי׳ זה. ואחרים אמרו שחמור חולו של מועד מי״ט שני הואיל וידעינן בקביעא דירחא לפיכך בי״ט שני של ר״ה ועצרת וי״ט אחרון של חג ופסח חופרין קבר וכוך למת אע״פ שבחה״מ אסור גם זה הבל. והמנהג והפי׳ הנכון מ״ש רבינו הגדול ז״ל שמה ששנינו אין חופרין כוכין וקברות במועד לצורך מתים שימותו והראב״ד תמה א״כ למה מאריך בהן ומרחיב בו ואני אומר שכך הוא הדין שמשנה שלימה שנינו ומתקנין את קלקול המים שבר״ה וחוטטין אותן ואוקי׳ בשאין הדברים צריכין להן ואעפ״כ כיון שצרכי רבים מותר אבל לא להתחיל לחפור כמפורש בדוכתא אף כאן נמי להתחיל ולחפור אסור הואיל ולא מת המת עדיין נראה כמכוין מלאכתו במועד אבל לתקן מותר דהאי נמי צורך רבים הוא ואפי׳ בקבר של בני משפחה נמי צורך רבים הן ואפי׳ בקבר יחיד נמי אפשר דשרו רבנן דכיון דמצוה הוא התירו בה מקצת מלאכה אפי׳ שלא לצורך כמו שהתירו בצרכי רבים וכ״ש שמא יצטרך במועד. והאי דאמרי׳ יתעסקו בו עממין, י״א דוקא בקבורתו אבל טלטולו מותר ע״י ככר או תינוק והוצאתו נמי מותרת כב״ה דאמרי מתוך שהותרה הוצאה לצורך הותרה נמי שלא לצורך. ואחרים אמרו שאסור וכן במשמע הלשון שאמרו יתעסקו בו עממין בכל עסקו קאמר וטעמא דמלתא דכיון דא״א לקבורה בישראל לא התירו הוצאה ע״י ישראל שאם היו מתירין להם מקצת מלאכה אף הם יגמורו. ועוד שהן כעוסקין בקבורה עצמה ומסייעין בה שהוא חלול י״ט לנמרי והא דמיא למאי דאמרי בפ׳ כל הכלים אטו טלטול לאו לצורך הוצאה הוא ועוד שאין כאן משום כבוד הבריות הואיל וסופו להעשות בידי עממין ולא דמיא לההוא דפ׳ המצניע דשרו בו טלטול והוצאה לכרמלית אפי׳ בשבת. ולפי דעתי, הוצאת מת לקבורה כהוצאת אבנים לבנין שאין לומר בו מתוך משום דלא שייך בו צורך היום. וא״ת שהוא מצוה הרי שריפת קדשים מ״ע ואינו דוחה וכ״ש זה. ועוד דהא אתמר בפ״ק דכתובות שאין מתוך אלא בהנאת כל נפש והרי אין כאן נפש. וממ״ש בפ׳ כל הכלים יתברר זה. ואם מפני שאין הוצאה זו צריכה לגופה אם כן בשבת נמי יתירו וכללו של דבר ששבת וי״ט שוין בדבר זה מה שמותר בזה מותר בזה: }
\textblock{הא ד\textbf{אמר רב אשי והוא דתלה בו רמונים.} ואקשי׳ עלה מערים ושותה. ק״ל התם נמי שותה קאמרי׳ אבל אומר לשתות ואינו שותה לא התירו ונראה דה״ק והוא דתלה בו רמונים ברישא מקמי׳ דיתלה בה שמרים הא תולה שמרים וחוזר ותולה בה רמונים אסור ואקשי׳ מערים ושותה מן החדש אע״פ שמתחלה הוא טורח הוכיח סופו על תחלתו ומתרץ התם לא מוכחא מילת׳ דהא א״א דשתי עד דעבר הלכך לא ידעי [דלחול] עביד וכי חזו בסוף דשותה מן החדש דיו בכך להוציא ממראית העין הכא מוכחא מילתא דלשמר עביד וכי תולה בסוף רמונים אמרי׳ דלשמר תלאה ועכשיו ששמר וא״צ לו משתמש בה לשאר צרכיו וכן דברי ר״ש מטין: }
\textblock{\textbf{אבל עכורין לא.} פי׳ אע״ג דאפשר למישתינהו הכי כיון דאיכא פסולת הו״ל בורר ואסור ומש״ה אקשי׳ עלה מדרשב״ג דאמר טורד אדם חבית ויינה ושמריה כו׳ אלמא כיון דאפשר דמשתתי בהכי שרי ולא דמי׳ למתני׳ דמתני׳ שמרים הן שא״א לשתותן כלל והו״ל בורר ומפרקי׳ תרגומא זעירא בין הגתות שלו ולפיכך התיר רשב״ג משום דאורח ארעא נמי הוא דמשתתי הכי אבל בשאר ימות השנה אף על גב דאפשר לאו אורח ארעא הוא למישתינהו עכורין: }
\textblock{\textbf{לא ליהדק איניש צבתא (צינייתא) אפומא דכוזא (דכוזני).} פי׳ לשמור בה משום דמיחזי כמשמרת דאע״ג דלאו משמרת ממש הוא שהרי עוברים השמרים בה כיון דאיכ׳ קיסמין וטנופת דמערב בהו ולא עברי, למשמר דמי: }
\textblock{\textbf{דבי רב פפא שפו שיכרא ממנא למנא בצבתא וא״ל ר׳ אחא מדיפתי לרבינא והא איכא ניצוצית.} היינו טנופי׳ דאמרי׳ וא״ל ניצוצית לר״פ לא חשיב ואינן מקפידין למצותו כ״כ וכשמגיע לניצוצית משליך הן ופסולתן ולאו בורר הוא כדפירש״י ז״ל: }
\newsection{דף קמ}
\textblock{\textbf{הכא מיחזי כי אולידי חיורי הכא לא מיחזי כי אולידי חיורי.} רש״י ז״ל פירש שהסודר אסור דאולידי חיורא הוא ואם כן תימה הוא למה לא כתבה רבינו הגדול ז״ל ונראה שהוא מפרש דר״ה פשט לי׳ בסודרא להתירא ואפילו הכי קס״ד דכיתנא אסור ושרא לי׳ ר״ח אפי׳ בכיתנא ולישנא דגמ׳ דייק כדאמרי׳ ותפשוט לי׳ למר מסודרא ואלו הוה ר״ה אסר בסודרא כיון דר״ח שרא בכיתנא לא הוה אמר ותפשוט לי׳ אלא מקשי קשו פשטי אהדדי:
}
\newsection{דף קמא}
\textblock{הא דאמרינן \textbf{הני פלפלי מידק חדא חדא בקתא דסכינא שרי.} כתב רבינו הגדול ז״ל דאיכא דמוקי לה ביום טוב ואיכא דמוקי לה בשבת והכריע הרב ז״ל כמאן דמוקי לה בשבת ודאי הכי הוא דמ״ש מתבלין שנדוכין כדרכן ואפי׳ מלח בהצלאה נדוך בכל דבר ותנן נמי אין שוחקין את הפלפלין בריחיים שלהן בריחיים הוא דפליגי משום דהוה טוחן הא כדרכן מותר ועוד אשכחן דתניא בתוספתא דתני אין כותשין את המלח במדוך של עץ אבל מרסק הוא ביד של סכין בעץ הפרור ואינו חושש אין מרסקין דבילה וגרוגרת ואת החרובין לפני זקנים בשבת אבל מרסק הוא בעץ של סכין ביד הפרור ואינו חושש שמעי׳ השתא דכל דמשני אפילו בשבת שרי ומיהו בפלפלי דמטחני טפי וצריכי מידק טובא פליגי והלכתא כרבא:
}
\textblock{\textbf{לא ליצדד אינש כובא בארעא דילמא אתי לאשוויי גומות.} פירוש לפי שהכובא צריכא מקום שוה לישב בודאי איכא גבשושית מצדדה וכל עצמו אינו מצדדה אלא שתניח במקום השוה ולא תתנדנד לפיכך חוששין שמא ישוה הגומות בצדוד הכובא או שמא אפילו ביד וכן בקנוח הטיט חוששין כן מפני שהוא מקנחו בגומות שבמקום שוה אינו מתקנח יפה לפיכך חוששין שמא יתכוין אבל בלא מתכוין מותר דקי״ל כר״ש:
}
\textblock{גמרא: \textbf{ולחייב משום תינוק, רבא כרבי נתן סבירא ליה.} איכא דקש׳ לי׳ והא מודו לי׳ רבנן לר׳ נתן באדם ול״ק דתינוק משרבט נפשיה כבהמה. ואי ק׳, הא דאמרינן בפרק מפנין האשה מדדה את בנה ברה״ר ולא גזרינן דילמא אתי לטלטולי משום דאדם נושא את עצמו ול״פ רבנן התם, וכמו שפירש רש״י ז״ל. א״ל התם בנוטל א׳ ומניח א׳ כיון שיש לו דעת בהילוך והולך ברגליו לא משרביט אבל הכא בתינוק גורר שאינו יכול לילך בלא סמיכה דמשרביט נפשי׳ וה״ל כבהמה פלוגתא דרבי נתן ורבנן. ומיהו אפילו לר״נ בגורר אסור לדדות תחלה וכן בבהמה דלא פליג ר״נ התם כלל ולפי סברתינו בפרק המילה שקטן שבקטנים ככפות דמי ואפילו ר״נ מודה נמצאו ג׳ דינים בקטנים. ומיהו ק״ל ל״ל לאוקמי לדרבא כר״נ לוקמה בתינוק גדול שהולך בעצמו ויש לו דעת בהילוך כאדם גדול ואפי׳ רבנן מודו אפשר לומר דילמא אין נותנין כיס אלא בצוארי תינוק קטן אבל לא משיגדיל:
}
\textblock{\textbf{א״ה מאי איריא אבן אפי׳ דינר נמי.} פי׳ אא״ב בתינוק שאין לו געגועין ואין בו משום חולי היינו דשרי לי׳ אבן משום דכמאן דמבטל לי׳ גבי תינוק דמי וכיס נמי כמאן דמבטל לי׳ לגבי תינוק הוא הואיל ואורחי׳ בהכי אבל דינר לאו אורח ארעא ליתן דינר בלא כיס ביד תינוק אלא משום אביו שהוא נוטלו מסרוהו לו וכמאן דאב נקיט דינר בידי׳ דמי ומ״ה אסור אלא אי אמרת כל דלאו צורך תשמישו הוא לא מבטיל לי׳ והכא משום געגועין התירו אבן אפילו דינר נמי. ומפרקי׳ אבן אי נפלה לא אתי לאתויי דינר אי נפיל אתי לאתויי וכיון שטלטלה אף ביד תינוק כטלטול דמי ואתי לאחלופי בטלטול בידים אי נפיל לי׳, לא רצו להתיר אותו טלטול קטן שהוא אסור בעצמו משום געגועין כדי שלא יבא אף לידי טלטול גדול. ומיהו כל היכא דלא מטלטלי׳ לתינוק שרי דלא גזרי׳ משום דלמא נפיל ואתי לאתויי דכל היכי דלא עביד איהו מעשה ולא שייך בטלטול זה מאי גזרי׳ שלא יראה אדם תינוק מטלטל דינר ולא יעמוד אצלו בקרוב ד׳ אמות והאוחזו בידו נמי לאו מידי עביד לי׳, כן נראה לי. אבל רש״י ז״ל כ׳ דלה״ט דחיישי׳ דילמא אתי לאתויי טעמא לאו משום טלטול הוא דאפילו לאחזו בידו והתינוק מהלך ברגליו אסר לי׳ רבא דלמא נפיל ואתי לאתויי, ולא מחוור:
}
\newchap{פרק \hebrewnumeral{21} נוטל אדם את בנו}
\textblock{}
\textblock{\textbf{מאן תנא דבל היכי דאיכא התירא ואיסורא בהתירא טרחי׳ באיסורא לא טרחי׳.} פי׳ לפי שכיון שהאבן מונחת ע״פ חביות שיש בה יין שהוא צריך לו ולא נעשית חבית בסיס לאבן אלא דין הוא שתנטל משם הי׳ ראוי שתעשה אבן זו כפסולת שבאוכל שהוא עפרויות וטנופת ואף על פי כן ניטלין בפ״ע ה״נ תנטל בעצמה כדי שיהיו המשקין שבחבית מתוקנין לאכילה דמ״ש מאבן שעל גבה מאבן שבתוך הפירות עצמן דניטלת כפסולת שבאוכלין אלא מפני שפסולת מרובה אין טורחין אלא בהיתר דהא צריך למשקלה לחבית כדמפרש ואזיל:
}
\newsection{דף קמג}
\textblock{הא דאמרן בגרעיני פרסיתא \textbf{שמואל מטלטל להו אגב ריפתא.} משום דמוקצין הן בשבת עסקינן דביו״ט שרי לטלטלינהו דחזו להסקה ואף על גב דשברי כלים נינהו שמואל כר״ש ס״ל בפ״ק (יט:) ובפרק בתרא דמכילתין (קנו:), אבל בשבת אינן ראוין לכלום לפי שאין מיוחדין אלא להסקה. והא דשדינהו רב לחיותא בפרק ב״מ בי״ט מתוך שראוין להסקה וה״נ משמע (בפ״ק) דביצה. והא דמטלטל להו אגב ריפתא י״מ דלא ס״ל כרב אשי דאמר לא אמרו ככר או תינוק למת בלבד ואינו נכון בעיני לפי שאינו דומה מטלטל גוף האיסור ע״י דבר היתר המונח עליו דומי׳ דמת בככר או תינוק למטלטלי גוף המותר אע״פ שיש עליו איסור ודשמואל נמי הכי הוה דמנח להו אגב ריפתא ומטלטל ריפתא ומטלטלי אגבי׳ דומיא דליקנא דמיא דודאי גרעיני הוא דיהיב בגווה ולא ליקנא עלייהו דגרעיני והיינו דאמרי׳ שמואל לטעמי׳ לפי שהפת נעשה תשמיש לגרעינין אבל במניח ככר על המת ועל הארנקי לא מסיימי בגמרא ס״ל כשמואל דאמר עושה אדם כל צרכיו בפת וכיון שעיקר הטלטול בפת דמיא למעות שעל הכר שמטלטלין ועודן עליו לפי שעיקר הטלטול בדבר המותר ודאי אינו נעשה בסיס שכל עצמו אינו עושה כן אלא לזורקן וכן זו ששנו לב״ש מסלק את הטבלא כולה ומנערה ה״נ הוא שמטלטל הטבלא עם הקליפין שעלי׳ עד מקום החנות ומנערה שם. ואי קשי׳ הא אמרי׳ בפ׳ כירה במטה שלא יחדה למעות אסור לטלטלה י״ל התם אסורה בטלטול קאמר לו שאינה מטלטל לצורך גופה להדי׳ בלא ניעור הא אלו רצה מנער והן נופלין, וכן אם הוצרך למקומה מטלטלה ועודן עליו עד המקום שהוא צריך להניחה שם דהיינו מתניתא מעות שעל הכר. והראב״ד ז״ל אמר התם אף על פי שלא ייחדה בטלה לגבי מעות הכא גרעינין בטלי לגבי הפת וקליפין לגבי טבלא ומפני זה כ׳ רבי׳ הגדול ז״ל לדשמואל ורבא אף על גב דקי״ל לא אמרו ככר או תינוק אלא למת בלבד. ובתוספ׳ בשם רבינו תם ז״ל מפרשים דגרעיני תמרים כולהו מותרין לר״ש דחזו למאכל בהמה ושמואל אף על גב דס״ל כוותי׳ משום דהוא אדם חשוב מצריך להו רפתא. ועוד אמרו דלרבא דסבר כר״י מותר הוא דהני כיון דלא אפשר להניח עליהן או לעשות בהן הכנה מבע״י לכתחלה מותרין אגב רפתא ומנא אבל פירא ובר יונה דאפשר מבע״י לא שרי׳ לכתחלה אלא בדבר המותר לאדם חשוב ודברים משובשין הן ויש לי להשיב עליהן הרבה, ומה שפירשתי נכון וברור:
}
\newchap{פרק \hebrewnumeral{22} חבית}
\textblock{}
\textblock{הא דתני׳ \textbf{נתפזרו לו פירות בחצירו מלקט על יד על יד ואוכל אבל לא לתוך הסל.} לא נתפרש לי מ״ט אלא שראיתי עניינה בתוספתא שהיא שנוי׳ כך פירות שנתפזרו מלקט א׳ א׳ ואוכל נתערבו לו פירות בפירות בורר ואוכל בורר ומניח על השולחן בורר ומשליך לפי בהמתו בדק אלו בפ״ע ואלו בפ״ע או שליקט מתוכן עפר וצרורות ה״ז חייב. לפי זה נראה שכשנתפזרו במקום עפר וצרורות עסקי׳ ואסור ללקטן ולתתן בתוך הסל משום דמיחזי כבורר ואפי׳ בפירות גסין ולהכי קתני שלא יעשה כדרך שהוא עושה בחול אבל לאכול מותר והיינו דקתני בגמרא בחצירו ולא קתני בבית מפני שסת׳ חצר יש בו עפרוריו׳ וצרורות ונמצא בורר משא״כ בבית שהוא עשוי להתכבד ככל יום:
}
\textblock{הא דאמר ר״י \textbf{אם לאוכלין היוצא מהן מותר אם למשקין היוצא מהן אסור.} משום שכשהכניסן לאוכלין היוצא מהן אוכל הוא דהא לא ניחא לי׳ כלל במשקה וכאוכלא דאפרת הוא ואין כאן משום נולד והכי מפרש לה בפ״ק דחולין ואם למשקין היוצא מהן משקה הוא ובכה״ג ודאי נולד הוא ואסור והיינו דאמרי׳ מודה הי׳ ר״י לחכמים בזיתים וענבים דכיון דבני סחיטה נינהו יהיב דעתי׳ פי׳ והו״ל משקין דאסורין דנולד הוא. ומיהו בפרק קמא דביצה משמע דאפי׳ מאן דל״ל נמי נולד משקין שזבו אסורין משום גזירה שמא יסחוט וה״ט דרבנן ואיכא התם מאן דמפרש טעמא דכל איסור נולד גזירה משום משקין שזבו שהוא משום שמא יסחוט וכדפירש״י ז״ל. וק״ל עלה דההיא דאסרי׳ ביצה משום משקין שזבו גזירה דשמא יסחוט והא פירות גופי׳ שרו שהרי מודים חכמים לר״י בשאר פירות דלאו בני סחיטה נינהו כ״ש בביצה דהו״ל למישרי דלא שייכי בי׳ סחיטה כלל. וי״ל שאר פירות דעבידי לאכילה ולא למשקין וכשאדם סוחטן הוא מפסידן מלאכילה לא גזרי׳ בהו דליכא משום שמא יסחוט כנ״ל אבל ביצה כיון דעבידא להטיל ביצים ולא מפסיד אוכלא בהכי דמי׳ לזיתים וענבים דמה זיתים וענבים משקין שבהן בלועין ועשוין להוציא מהן אף ביצה נמי בלועה ועשוי׳ לצאת וי״ל ר״י סבירא לי , דבכל פירות פליגי ומשמע דרב נמי הכי ס״ל ופשטא דמילתא דר׳ יוחנן הכי אתי׳ ורבי יצחק ור״י אמרו דבר א׳ כדאיתא התם בגמרא, וזה יותר מרווח:
}
\textblock{\textbf{וסבר ר׳ יהודה סתם אסור.} כת׳ רש״י ז״ל דה״ה דהוה ק״ל אי רבנן קתני לה דהא מידי דלרבנן הוה משקה לרבי יהודה לא הוי משקה וכן פירש בהא דאמרי׳ השתא ומה זיתים וענבים דבני סחיטה נינהו שלא לרצון ולא כלום תותים ורמונים דלאו בני סחיטה נינהו לכ״ש דהו״ל לאקשויי זיתים וענבים אזיתים וענבים דקתני לעיל כר׳ יהודה לאוכלין היוצא מהן אסור אלא אלומי אלים לאקשויי. והא דמפרקי׳ שלא לרצון דגלי אדעתי׳ ואמר לא ניחא לי והוי טפי גלוי דעתא מהכניסן לאוכלין. ויש לפרש דמדרבנן ל״ק לי׳ כלל משום דכיון דאמרי רבנן אפילו בתותים ורמונים שהכניסן לאוכלין היוצא מהן אסור שמע מינה דלאו משום דמשקין נינהו אלא משום דגזרי׳ בהו משום שמא יסחוט או אטו זיתים וענבים דהא דמו למשקין וכן בזיתים וענבים לר׳ יהודה כיון דאפילו הכניסן לאוכלין אסור א״א לומר משום דחשיב להו משקין הוא אלא משום גזירה שכיון שדרכן לסחוט אותן וממליך עלייהו אסורין שמא ימלוך עליהן לסחטן א״נ לר״י גזירה משום נולד דאיהו אית לי׳ נולד ומוקצה בכ״מ אלא בתותים ורמוני׳ דשרי ר״י לאוכלין ואסר למשקין ולסתם ש״מ דמשקין נינהובסתמא דסתם כמפרש דמי למשקין ומ״ה מקשי׳ מק״ו דתותים ורמונים ומדר״י ולא מזתים וענבים מדרבנן וא״ת ולימא לי׳ מתני׳ רבנן הוא ולא תיקשי ל״ק דהא עדיף תירוצי׳ טפי טובא ועוד דלגבי משקין ולאו משקין ל״פ ומדר״י נשמע לרבנן דסתם משקין הוו. וה״ר משה ב״ר יוסף ז״ל כ׳ טעמא משום דיהיב דעתי׳ לבסוף וגזירה שמא יסחוט אבל לגבי (ס״ל) זתים וענבים טעמא אחרינא הוא ואף על גב דמשקה הוא כיון דשלא לרצון הוא דלא ניחא לי׳ טהור וכי קא מקשה לר״י בדוקא קא מקשי לי׳ מאי לאו לרצון דניחא לי׳ שלא לרצון סתמא. ורבנן אפילו בתותים ורמונים ור״י כזתים וענבים כולהו סבירא ליה אף על גב דלאוכלין קיימי שהיוצא מהן משקה הוא ואפילו שלא לרצון דגלי דעתי׳ דלא ניחא לי׳ במשקה ואפ״ה הוי משקה והו״ל נולד ואסור בשבת אבל לענין הכשר אף על פי שכולן מודין דמשקין הן בזיתים וענבים אפילו הכי לא מכשירי דומי׳ דמי גשמים שתחלתן וסופן משקין ואינן מכשירין אלא לרצון ומ״ה לא מקשי אלא לר״י ומסתם בלבד מתותים ורמונים אלו דברי זה הרב ז״ל פי׳ לפי׳ מ״ש דגבי שבת סתם תותים כמפרש למשקין ומאי שנא לענין טומאה דסתמן אפי׳ בענבים לאוכלין זו היא הקושי׳. ואעפ״כ אינן דברים נכונים, דכל שהן משקין בסתמא ניחא לי׳ וטמאין דמתני׳ לאו לענין הכשר בלחוד תנן אלא טהורין וטמאין קתני לומר דהוה משקה ולא הוי משקה כדפירש״י ז״ל והלכך כל דהוי משקה טמא. ומ״ש משום מי גשמים טעם אחר הוא במי גשמים לפי שהן עשוין לקרקעות ומפורש הוא במשנתי׳ בשלהי מס׳ מכשירין אר״ש מכאן ואילך היינו משיבין לפניו מי גשמים יוכיחו שתחלתן וסופן משקה ואין מטמאין אלא לרצון אמר לנו לא אם אמרתם במי גשמים שאין רובן לאדם אלא לארצות ולאילנות ורוב החלב לאדם וכן הוא ודאי דמי גשמים שירדו על הטמ׳ אינן מקבלין טומאה וטהורין, וכ״פ ה״ר משה הספרדי ז״ל:
}
\newsection{דף קמד}
\textblock{\textbf{ואב״ע שאני סלי זתים וענבים דכיון דלאיבוד קיימי וכו׳.} ואי ק׳ הא הבוצר לגת הוכשר ואפי׳ בסלים שאינן מזופתין, י״ל התם בבוצר לגת בפי׳ גזירה משום קופות זפופו׳ שהרי עשוי הוא לבצור בהן דהא למשקין נינהו הא בבוצר סתם בסלין לא גזרו:
}
\textblock{\textbf{תרדין שסחטן ונתנן במקוה פוסלין את המקוה בשינוי מראה והא לאו בני סחיטה אלא מאי אית לך למימר וכו׳.} פירש רש״י ז״ל והכי מפרש ר״ל לבריי׳ דלעיל סוחטין בפגעין ובפרישין כדי למתק הפרי ולא לצורך משקה אבל לא ברמונים ואפילו למתקן דשל בית מנשיא היו סוחטין אותן בחול לצורך משקה הלכך בשבת אסור ואפילו למתקן הואיל ואיכא חד דעביד לשם משקה דלמא אתי למיעבד הכי ולא מחוור לי אלא תרדין כתותים ורמונים ובכולהו אמרי׳ דלאו בני סחיטה נינהו כלומר שאין סתמן לסחיטה כענבים. והיינו דאמרי׳ אלא כדר״ח כלומר ה״ט דחיישי׳ לבית מנשיא בן מנחם מפני שזה נהג כמנהגו ואחשבינהו וכדר״ח דאמר תרדין שסחטן פוסלין את המקוה בשנוי מראה והא לא בני סחיטה נינהו דרובא דעלמא לא סחטי להו אלא כיון דאיכא מיעוטא דסחטי להו וזה נהג כמנהג אותו מיעוט וסחטן למשקין וחשבן הו״ל משקין ואלו בשאר פירות שאין שום אדם סוחטן למשקין אעפ״י שזה סחטן בטלה דעתו אצל כל אדם ונכון הוא וכ״נ מדברי רבינו הגדול ז״ל:
}
\textblock{הא ד\textbf{אמר רב חסדא מדברי רבינו נלמוד חולב אדם עז לתוך הקדרה אבל לא לתוך הקערה.} פירש רבינו הגדול ז״ל בי״ט ולא ידעתי מי הכניסנו בצרה הזאת אם מפני שאמרו נוהגין שיונקין מן הבהמה בי״ט אבל בשבת אסור הא התם אפי׳ בי״ט לא שרו אלא משום צערא אלא מאי אית לך למימר דההיא שאני כדפרשה רבינו ז״ל דכחולב לתוך הקערה דמי וכיון שכן אף בשבת נמי לישרי ובאמת דבר הלמד מענינו דבשבת קאמר. מ״מ צריך לפנים, היכי גמר להכריח מדברי רבינו בשלמא גבי אשכול של ענבים כיון דיכול לאכול האשכול כי סחיט לי׳ נמי לתוך הקדרה שיש בה אוכל שרי דלאו מפרק הוא אלא כמפרד אוכל בעלמא הוא אלא חולב נהי נמי דלאו משקה הוא אמאי שרי הא כמפרק אוכל הוא דבהמה גופה ודאי דפסולת הוא ולא חזי׳ לאכילה כדאי׳ עם חלבה והו״ל תולדה דדש ואסור. ולדברי רבי׳ ז״ל דמוקי לה בי״ט ל״ק דכל תיקוני מיכלא בי״ט שרי ומיהו לתוך הקערה אסור דהו״ל נולד ומעיקרא לאו משקה והשתא משקה א״נ כל למשקין גזירה משום משקין שזבו דהוא משום שמא יסחוט ובהן עצמן י״ל שמא יחלוב לגבן דהוי דישה אסורה הא ליומי׳ לאוכל מותר ואף על פי שאין דשין בי״ט כדמפיק לה בירושלמי מאשר יאכל לכל נפש ושמרתם את המצות במס׳ ביצה ה״מ דישה בעלמא שדרך בני אדם לדוש הרבה ולתקן ממנו לימים הרבה הא בכגון זה שדרך לחלוב בכל יום והוא נפסד מיום לחבירו מותר לצורך היום שהרי התירו נמי לשמר בתלוי׳ שהוא תולדה דבורר ואין בוררין בי״ט ואין (מדקדקין) [מרקדין], שכל אלו מעטן הכתוב כדמפורש התם. ולרבינו הגדול מצינו בתשובותיו שכתב כלשון הזה ודקס״ד בהא דאמר ר״ח מדברי רבינו נלמד חולב אדם וכו׳ דבשבת הוא לאו הכין הוא מלתא דהא ליכא מאן דשרי בשבת ודקאמרת שלא מצינו חולב שהוא מאבות מלאכו׳ ולא מן התולדות אשתמיטתי׳ הא דגרסי׳ בפרק המצניע ת״ר החולב והמגבן כו׳ ומההוא דאבא שאול שמעת דחולב הוא חייב משום מפרק דאמר לענין יונק מפרק כלאחר יד הוא הא חולב בידו מפרק גמור הוא ומתוך שמלאכת אוכל נפש בי״ט בדברים שא״א לעשותן מעי״ט שרי חולב נמי בתוך האוכל שרי ואף על גב דמלאכה הוא וחולב לתוך הקערה דמלאכת משקין הוא לא הותרה בי״ט ע״כ. ומה שאמר דחולב חייב משום מפרק ומשמע דל״פ רבנן עלי׳ דר״א בהא א״נ לאו דברי ר״א אלא ד״ה אינו מחוור שא״כ למה שנאוה כאן וכי אבות מלאכות או תולדותיהן באו לפרש בברייתא זו ועוד שמצינו בירושל׳ הדא דאת אמר בחדש אבל בישן מחלוקת ר״א וחכמים דאיתפלגון המכבד והמרבץ המגבן והמחבץ החולב והרודה חלת דבש חייב חטאת דר״א וחכמים אוסרין משום שבות וכו׳ אלמא בכולהו פליגי. ודאמרי׳ התם מפרק כלאחר יד איונק וה״ה לחולב שאין זה מפרק הדומה לדש ומדינא מותר הי׳ לחלוב אלא משום דאתי לאחלופי בסחיטה כיון דמלאכה גמורה הוא הוו ינקי ועוד שעיקר רפואות הגונח ביניקת החלב כשהוא רותח. מיהו רש״י נמי פי׳ שם בדבריו שהחולב מפרק גמור הוא וכן י״ל שאע״פ שהחולב מדבריהם כיון דדמי למפרק וכי אורחי׳ קעביד אסור ולא התירו אלא שבות בשינוי דהיינו יונק והכי משמע לישנא דגמרא. ועוד יש לי לפרש ולומר דגבי חולב לקדירה דאפי׳ בשבת שרי דלא דמי לדישה כל זמן שהוא בא לאוכל לפי שהוה מפרד אוכל מאוכל שהעז עצמו אוכל הוא ואף על פי שהיא אסורה לשחוט בשבת אין איסור זה מוציאתה מתורת אוכל לעשותה פסולת אוכלין דהא חזיא לגוים ולחולה וחזיא למחר אטו בשר חי בשבת אין תורת אוכל עליו מפני שהוא חסר בישול הלכך אין כאן משום מלאכ׳ דישה. וא״ת וכיון שהעז מוקצית מחמת איסור יאסר נמי חלבה, י״ל אה״נ ומדברי שמואל למד ר״ח דל״ל מוקצה ונולד א״נ אין מוקצה שקפץ מעצמו אוסר אלא למלאכתו כגון שוחט בשבת שאסורה לעלמא נמי משום מוקצה הא כל שלא עבר על השבת אפילו בשוגג ממילא, הותרה כדאמרינן בביצה שנולדה בשבת דעלמא תשתרי ולא אסרי׳ לי׳ משום מוקצה וכבר פירשתי זה בס׳ מלחמות השם במס׳ ביצה (יא. ברי״ף). ובתוס׳ מצאתי בשם ר״ת ז״ל הואיל ובהמה בשבת לא חזי׳ לאכילה הו״ל כבורר אוכל מתוך פסולת ופסולת נפיש דהיינו הבהמה אלא מפרש לה ר״ת בי״ט דחזי׳ לאכילה והו״ל אוכלא דאפרת ואין זה נכון שיהא איסור שבין שבת לי״ט עושה את הבהמה אוכל או פסולת, אלא כמו שפירשתי עיקר:
}
\newsection{דף קמה}
\textblock{ה״ג: וכ״כ בכל נוסחי דוקאני וזה הוא גרסתו של רבינו הגדול ז״ל: \textbf{אר״פ דכ״ע משקה הבא לאוכל אוכל הוא.} וה״פ: דמעיקרא קס״ד דהאי משקה של ענבים נ״ט בפת ומפני כך הוא נעשה אוכל וה״ט דמ״ד לא הוכשר דאי אינו בא לאוכל מ״ט ובא עכשיו ר״פ ופי׳ אם משקה זה בא לאוכל הוא דכ״ע הוי אוכל אלא הכא אין משקה זה נ״ט בפת כלל אלא בעוד שהוא לח האור מהבהבו ושורפו ונמצא שאינו בא לאוכל ואין נעשין כן אלא שלא ישרוף האור פנים של לחם וטעמא דמ״ד לא הוכשר מפני שהוא משקה העומד לאיבוד דקסבר לאו משקה הוא, כן נ״ל. אבל רש״י ז״ל הגי׳, דכ״ע משקה הבא׳ לאוכל לאו אוכל הוא ולא אתי׳ אליבא דהלכתא דקי״ל אוכל הוא כרב ושמואל:
}
\newsection{דף קמז}
\textblock{הא דתנן \textbf{מי מערה.} דומי׳ דמי טברי׳ דאינון חמין. ומ״ה קתני הרוחץ דיעבד אין לכתחלה לא בחמין שהוחמו מע״ש הוא דאסורין ברחיצה ומותרין בשטוף לר״ש. והא דקתני מערה אורחא דמילתא קתני שהוא מעמיד חומו מע״ש לשבת אבל מי טברי׳ אפי׳ במקוה מותר ואין זיעה אסורה אלא בתולדת האור במרחץ שלא יהיו רוחצין ואומרי׳ מזיעין אנו הא בחמי טבריה שרחיצתן מותרת אף זיעתן מותרת. אבל רבינו הגדול ז״ל אמר מערה מיטללא ונפיש הבלה ואתי לידי זיעה נראה מדבריו שמערה של חמי טברי׳ שהיא מותרת ברחיצה אבל זיעה אסורה בכולן שלא התירו רחיצת טברי׳ אלא מפני שאין הדבר עומד וזיעה במקומה עומדת לגמרי לאסור בכולן וכ״כ הר׳ משה תלמידו כדבריו ז״ל. ומיהו לא מחוור לנו דאי כולה מתני׳ בחמי טברי׳ להשטף ודאי מותר הוא אפי׳ לר״י דרחיצה נמי מותרת שלא במקום זיעה דשיטוף ליכא זיעה:
}
\textblock{הא ד\textbf{תני׳ ר׳ נחמי׳ אומר אף בחול אסור משום הפסד אוכלין. פי׳ רבינו הגדול ז״ל דוקא בדלא מצטער ועביד כדי לאכול הרכה לפיכך אסור אבל היכי דמצטער שרי דהפסד דגופי׳ עדיף ואפילו ליכא חולי והיינו דר׳ יוחנן דאמר ביד מותר אלמא ליכא משום הפסד אוכלין ומשמע לי׳ דל״פ רבנן עלי׳ דר׳ נחמי וברוך שבחר בדבריה׳ שזו מדה מגונה ביותר ואינו נאה לזרע קדש:
} תני׳ ר׳ נחמי׳ אומר אף בחול אסור משום הפסד אוכלין. פי׳ רבינו הגדול ז״ל דוקא בדלא מצטער ועביד כדי לאכול הרכה לפיכך אסור אבל היכי דמצטער שרי דהפסד דגופי׳ עדיף ואפילו ליכא חולי והיינו דר׳ יוחנן דאמר ביד מותר אלמא ליכא משום הפסד אוכלין ומשמע לי׳ דל״פ רבנן עלי׳ דר׳ נחמי וברוך שבחר בדבריה׳ שזו מדה מגונה ביותר ואינו נאה לזרע קדש:
}
\newsection{דף קמח}
\textblock{\textbf{א״ל השאילני לא אתי למיכתב הלויני אתו למיכתב.} פירשו המפרשים מפני שסתם הלואה ל׳ יום ושמעי׳ מינה שאין סתם שאלה ל׳ יום כדברי מקצת המורים אלא כל אימת דבעי [המשאיל] מיהדר לי׳. והא דאמרי׳ (מנחות מד.) טלית שאולה ל׳ יום פטורה מן הציצית לאו משום דסתם שאלה ל׳ יום אלא משום שאין אדם עשוי להשאיל יותר אבל עשוי הוא להשאיל שלשים יום ומדעתו אבל שלא מדעתו לאלתר מהדר ליה. ואקשינ׳ וכיון דבחול זמנין דבעי מימר הלויני בכגון זה (שחוזר) [שאינו חוזר] בעינו ואמר ליה השאילני אתי למיכתב שאין לשון זה עיקר ואין משגיחין עליו בשבת נמי אתי למכתב ופריק בחול ליכא קפידא אבל בשבת הואיל (ואסר׳) [ואמרי] ליה רבנן למימר השאילני אפילו בדבר שאינו חוזר בעינו מידע ידע שלא אמר זה השאילני אלא לומר שזמן הלואה זו כזמן שאלה ולא אתי למיכתב:
}
\newchap{פרק \hebrewnumeral{23} שואל}
\textblock{}
\textblock{הא דאקשי׳ \textbf{בשבת הוא דאסר אבל בחול ש״ד וכו׳.} ולא אקשי׳ השאילני ליתסר מה״ט גופיה הוא שסתם שאלה כל היום אין חוששין שמא יוקרו החטים עד שיהא רבית בככרות אבל בלשון (שאלה) [הלואה] שהוא לזמן מרובה חוששין שמא יוקרו החטים הרבה ונמצא רבית אפי׳ בככרות. מ״מ ק״ל, וכי להלל משעת הלואה הוא אסור והלא אפי׳ סאה בסאה אינה אסורה דל״ל בה (אלא) משעת הלואה עביד לי׳ שומא אלא לוה אדם סאה (בסאה) לכתחלה אם הוזנו נותן לו חטיו ואם הוקרו נותן לו דמיהם. וי״ל לפי שאין הנשים מקפידות ביוקר הככרות אוסר הלל הלואתן מתחלה עד שיעשנה דמים גזירה שמא יוקרו ולא תרצה לשום שלא תראה כצרת עין א״נ שלא תדע ביוקר החטים לפיכך הצריכו לשומן מתחלה וכך משמע הלשון דקאמר והלל אוסר שמא יוקרו החטים:
}
\textblock{ה״ג וכ״כ בכולהו נוסחי: \textbf{אא״ב ניתנה ליתבע מש״ה משמט דקרינא ביה לא יגוש אלא אי אמרת לא ניתנה ליתבע מאי משמט.} פי׳ אי אמרת מאי משמט שלא יגוש הא שמוט ועומד הוא ואי אמרת משמט שצ״ל לו משמט אני אמאי הא לא קרינא ביה לא יגוש. ורש״י ז״ל [מחק] אא״ב וכו׳ שלא לצורך משום דפירש מאי משמט הא שמוט ועומד הוא ואינו מחוור שא״כ יכול היה לתרץ כי סיפ׳ שצ״ל משמט אני אלא ש״מ מתרוייהו נמי מקשי ומשום דכתיב לא יגוש וכדפרי׳:
}
\textblock{הא ד\textbf{רבא בר עולא דמערים ערומי.} איכא לפרושי שאפי׳ היו תובעין ממנו בדין כיון דתפס תפס ואע״פ שלא ניתנה ליתבע לומר שאין ב״ד נזקקין לה אבל מ״מ אי תפס לא מפקינן מיניה. ואי קשי׳ א״ה כי אקשי׳ מסיפא דמתני׳ דקתני ואם לאו אינו משמט לימא נ״מ דאי תפס בתר הכי תפס אה״נ אלא אפי׳ בלא תפיסה נמי בעי לאוקמי ומסתברא נמי דאפי׳ הדר ביה ותפס מיני׳ משכון או מעות כשיעור מאי דתפס מלוה מדידי׳ מפקינן מיניה דכיון דתפס נפרע מלוה והשתא לאו הלואת י״ט מגבי׳ מיניה אלא גזלתו של זה מוציאין מתחת ידו וה״מ דתפס מלוה מיני׳ מעות אבל תפס מיני׳ משכון וחזר הלה ולקחו לאחר זמן מידו אין נזקקין לו דהא לא קנה משכון והלה את שלו ראה ונטל. וי״ל שלא היתה הערמתו מועילה לו לרבא בר עולא אלא שהיה בעל דינו בוש מליתבע ממנו ולא מסתבר שלא אמרו חכמים אלא שלא ניתנה ליתבע בב״ד כדי שלא יבא לכתוב ומכיון שאין ב״ד נזקקין לו בודאי יודע הוא שעל אמונתו הלוהו ולא יכתוב אלא מיד יבקש ממנו ומיהו אם תפס תפס כדפרי׳:
}
\newsection{דף קמט}
\textblock{\textbf{לא לעולם דכתיב אכותל ומידלי.} פי׳ לעולם כדאמרינן מעיקרא דכתיב אכותל ומידלי למ״ד שמא ימחוק לא חיישי׳ דלא ס״ל כרבה ומשום שמא יקרא נמי ליכא דגודא בשטרא לא מיחלף למ״ד שמא יקרא חיישי׳ דגודא בשטרא נמי מיחלף ואיהו נמי לית ליה דרבה ולתרוייהו לא חיישי׳ להא דרבה והא דתניא אבל לא מן הכתב ור״א מתיר מכתב שע״ג הכותל בדכתב אכותל ומידלי ולא משום שמא ימחוק אלא ת״ק אסר לה משום שמא יקרא דגודא בשטרא נמי מיחלף ור׳ אחא מתיר דלא מיחלף ולרבה ת״ק משום שמא ימחוק נסיב, וס״ל כוותי׳. והא דאמרי׳, ודקא קשי׳ לך דרבה תנאי הוא לומר דלאו ד״ה הוא ואשכחן תנאי דפליגי עלי׳ י״ל דר׳ אחא ודאי ליא סבר כוותי׳ מיהו רבה מוקי נפשי׳ כת״ק ורב ביבי ואביי מפיק ליה נמי מיני׳ ומיהו תנא דאין רואין במראה כותי׳ ס״ל. והא דתניא מונה אדם את אורחיו מכתב שע״ג הכותל אבל לא מכתב שע״ג טבלה ופנקס כר״א אתי׳ דאלו לת״ק דאית ליה גודא בשטרא מיחלף אפי׳ חייק ומידלי נמי אסור וה״ה לטבלה ופנקס דגבוה ומידלי דשרי דומיא דנר אלא אורחא נקט תנא. ואי קשיא לך, ל״ל לאוקמי לפלוגתייהו כתנאי ואמוראי נמי בהכי נימא א״ב דחייק מיחק ובגודא בשטרא מיחלף פליגי וכולהו אית להו דרבה א״ל אה״נ אלא כיון דהא דרבה תנאי הוא מדתניא אין רואין במראה ניחא לן למיהדר לאוקימתן קמייתא דאפרוך לה מדרבה. ולענין הלכה סמך רבינו ז״ל על דברי רב ביבי במ״ש גודא בשטרא לא מיחלף לקולא אבל במה שסובר דהיכי דמידלי ליכא למיחש למחיקה לית הלכתא כותי׳ אלא כרבה דאמר אפי׳ גבוה שתי קומות הלכך כתיב מיכתב לעולם אסור חק מיחק בכותל לעולם מותר. ובס׳ מלחמות כתבנו (ענין אחר) [לשון אחר], דרב ביבי ואביי תרוייהו לשוויי׳ חלוק בגזירה דמתני׳ אתו רב ביבי שמא ימחוק וכשאין מחיקה מצוי׳ לו מותר והיינו כגון כותל דמידלי ואתי אביי למימר במחיקה ליכא לאפלוגי אלא (נימא) שמא יקרא וכשאין בו משום קריאה מותר והיינו חק מיחק בכותל הא דכתיב׳ לעולם אסור ודכ״ע גודא בשטרא לא מיחלף כסתם ברייתא קמייתא והיינו דלא מוקמי׳ פלוגתא דתנאי ואמוראי בדחייק בכותל ובגודא בשטרא מיחלף פליגי דסברא הוא גודא בשטרא לא מיחלף ועוד דתיקום מתני׳ קמיית׳ כד״ה ובדחייק הוא:
}
\textblock{\textbf{הכא במראה מתכת עסקי׳.} פי׳ ובה אסר ת״ק אפי׳ בקבועה משום דגזרי׳ קבוע אטו אינו קבוע כדגזר רבה בנר גבוה אטו אינו גבוה ור״מ התיר בקבוע׳ דל״ל דרבה אבל שאר המראות לד״ה מותרות בין קבועו׳ בין שאינן קבועות ולא גזרי׳ מראה של זכוכית אטו של מתכות דלא אשכחן ליה לרבה דגזר מין זה אטו מין אחר שא״כ היתה גזירה לגזירה אלא באותו המין שנגזרו בה גזירה ראשונה הוא שאמר רבה שהשוו חכמים מדותיהן ולא חילקו בו בין גבוה לנמוך ובין קבוע לשאינו קבוע תדע דאמרי׳ אי אדם חשוב הוא מותר ולא גזרי׳ אטו שאינו חשוב ובשמן נמי חילקו בין משחא לנפטא ולא גזרו זה מפני זה, וזה דעת רבינו הגדול ז״ל והוא האמת. ור׳ משה בן יוסף ז״ל פי׳ הכא במראה של מתכת עסקינ׳ וכדר״נ וכל המראות נמי גזרי׳ אטו של מתכות כההיא דמתרץ רבה הכא בי״ט שחל להיות אחר השבת ומשום הכנה וה״ה לשאר ימים דגזרי׳ הא אטו הא ותרצו נמי לענין שריפה הכא בי״ט שחל להיות בע״ש ולפי שאין שורפין קדשים בי״ט וחדא מחתא נינהו וטובא אשכחן כה״ג אלו דברי הרב ז״ל. ואינן נכוניןדאי הכי ת״ק קאמר אין רואין במראה בכל מראה שבעולם ור״מ דל״ל דרבה היה לו לומר ור״מ מתיר כל המראות חוץ ממראה של מתכת שאינו קבוע. וא״ת דר״מ אסר בשאר מראות שאינן קבועות א״כ הא אית ליה דרבה דאע״ג דאיכא למימר מידכר גזרי׳ והאי דלא גזר בקבועה ע״כ משום דבין קבוע לשאינו קבוע לא טעו אינשי:
}
\textblock{הא ד\textbf{אמר ר״י אמר שמואל מותר אדם לומר לחבירו לכרך פלוני אני הולך למחר.} לאו דוקא שיאמר כן בלחוד דהגדה זו לא מהניא ולא מעלה אלא אפי׳ אומר לו לשם אני הולך לך עמי מותר והוינן עלה מאין מחשיכין להביא פירות משום דלשמואל כיון דרשאי הוא באמירתו רשאי הוא בחשיכתו כאבא שאול וש״מ דמותר לומר לחבירו הבא לי פירות ממקום פלוני למחר שכן הוא עצמו ע״י בורגנין ומחיצה מביא, ורשאי הוא בחשיכתו ובאמירתו:
}
\textblock{\textbf{מת נמי משכחת לה למיגזא ליה גלימא.} וכן נמי לעשות לו ארון והיינו דקתני מתני׳ להביא לו ארון ותכריכין כלומר לתקן ולהביא וכתבו בתוס׳ דשמעינן מהא שמותר לאדם לילך בשבת לגינתו ולחורבתו שבתוך התחום להחשיך ולתלוש עשבים ושאר דברים הצריכים לו שלא אסרו למיגז אסא לאחר דלא כלה אלא להחשיך על התחום דאינון עובדין דחול אבל להחשיך תוך התחום מותר והא דאמרי׳ במס׳ עירובין ומייתי לה רבינו הגדול ז״ל לא יטייל אדם בתוך שדהו לידע מה היא צריכה כיוצא בו לא יטייל אדם על פתחה של מדינה כדי שתחשך ויכנס למרחץ מיד ההוא משום חשדא שכל הרואה אותו מטייל ועומד שם מכיר ויודע שלכך הוא מתכוין שיכנס למרחץ, ומפני כך אסרוה:
}
\newsection{דף קנא}
\textblock{הא דק\textbf{אמר אבא שאול כל שאני רשאי באמירתו רשאי אני להחשיך עליו.} דמשמע דמילתא פשיטא הוא שרשאי באמירה דעסק המת והכלה ואומר לו לך למקום פלוני לא מצאת במקום פלוני הבא ממקום פלוני לא מצאת במנהוכו׳ דוקא אמירה דישראל וה״ה לגוי כלומר לך למקום פלוני במוצאי שבת והביא במנה ובמאתים הא לומר לגוי לך בשבת והביא ודאי אסור שאפי׳ הביא לצורך ישראל תני ואזיל במתני׳ דאסור ואע״פ שאמורה לגוי שבות בין שיאמר עשה מלאכה היום בשבת בין למחר דהיינו עמוד עמי לערב התירו שבות דאמירה דלמחר אפי׳ מישראל לישראל משום צרכי כלה ומת ולא התירו אמירה ולא מעשה בשבילו בדבר הנעשה בשבת שבזו חלול שבת ובזו אין חלול שבת וכן אינו רשאי באמירה דערב שבת וכדאמרי׳ בפ״ק (יט.) גבי שליחות איגרות ואפי׳ במת וכלה נמי אסור דהא מתחלל השבת למחר וכל שהשבת מתחלל בשבילו אסור ובמלאכה הנעשית בשבת לא אמרי׳ שכן דא״ה באגרות נמי נימא שאם יש שם מחיצות מביא דמשום שטרי הדיוטות ליכא באגרות שלום ושל עסקי רבים, וכולן אסורין:
}
\textblock{מתני׳: \textbf{מחשיכין על התחום לפקח על עסקי כלה ועל עסקי המת להביא לו ארון ותכריכין. ארמי שהביא חלילין בשבת. לא יספוד בהן ישראל. אא״כ באו ממקום קרוב. עשו לו ארון חפרו לו את הקבר. יקבור בהן ישראל. ואם בשביל ישראל לא יקבר בהן עולמית.} פרש״י ז״ל, ארמי שהביא חלולין בשבת כשהביא בשביל ישראל לא יספוד בהם ישראל לעולם קאמר. וקנסא הוא משום דמוכחא מילתא דבשביל ישראל הוא. דאין דרך להביא חלילין אלא בשביל מת. אלא אם כן באו ממקום קרוב. מתוך התחום. ואנו תמהין ע״ז הפירוש. כיון שהוא קונס על איסור המלאכה לעולם. כשבאו ממקום קרוב תוך התחום נמי אסורין לעולם. דהא מ״מ הביאו ארבע אמות ברה״ר. ונעשה בהן איסור מלאכה גמורה בשביל ישראל. ומה ענין תחומין לכאן הנחת איסור אב מלאכה הנעשית בשביל ישראל. והלכת אחר איסור תחומין. וכן כל ענין הגמרא חיישינן שמא חוץ לחומה לנו בפלוגתא דרב ושמואל. אין לה ענין בפירוש הזה. מה לי חוץ לחומה ותוך לחומה. מה לי חוץ לכמה תחומין. מ״מ נעשית בהבאתן מלאכה בשביל ישראל ואסורין לעולם. ועל דרך רחוקה נתרץ לדברי הרב ז״ל, דהכא כשהביא ארמי חלילין למכור עסקינן שאין ישראל לוקחין ממנו בשבת. לפיכך אם הביאן מתוך התחום מותרים מיד שלא נעשה בהן בשביל שום מלאכה של ישראל בהנאה שהרי החלילין בביתו של ארמי רחוקים הם בשבת מישראל זה. כמו שהיו רחוקים ממנו באותו מקום שבסוף אלפים אמה שהובאו משם. דמה לי רחוקים ממנו ד׳ אמות או אלפים אמה כאן וכאן לרשותו של ישראל אינו יכול להביאן בשבת וכאן וכאן הי׳ יכול לילך ולבקרן וליקח ממנו בשבת. ולהשתמש נמי בהן אם היה צריך להן באותו מקום לשום ענין המותר. אבל מחוץ לתחום כיון שאין ישראל יכול לבא לכאן ולבקרו וללוקחן והארמי הביאן בשבילו לעיר והזמינן לו הרי נעשית בהם מלאכה המקרבת אותן לידי ישראל ומלאכה הנאה היא שנעשית בשביל ישראל והלכך קנסינן. ומ״מ אין הפירוש הזה נכון. ועוד סיפא דמתני׳ קתני לא יקבר בהן עולמית. ורישא לא קתני לעולם. משמע דבכדי שיעשו מותר. אלא ה״פ: דבין מתוך התחום בין מחוץ לתחום בעינן כדי שיבאו מאותו תחום. ומשום הכי קתני לא יספוד בהן ישראל משחשיכה. אא״כ באו ממקום קרוב שיהא בהן בשעת הספד כדי שיעשו למוצאי שבת מיד. ומשום דסתם הספדן אינו ממש בשעה שחשיכה מוצאי שבת. ואם באו ממקום קרוב. מסתמה בשעת הספדו יש בו שיעור כדי שיעשו משום הכי קתני סתם אא״כ באו ממקום קרוב. ולא קא יהיב שיעורא כלל אבל מכל מקום בין רחוק בין קרוב. צריך כדי שיבאו משם. דהא בשבת תוך החתום נמי איכא הונאה והעברת ארבע אמות ברשות הרבים ולאו בשהביאן בשביל ישראל דוקא קתני דאכתי הוה לי׳ למיפלג ומיתני בהדיא כדקתני סיפא אלא סתם חלילין בשביל ישראל הן. שאין הארמים סופדין בהן. ורובן בשביל ישראל הן באין. ואתמר עלה בגמרא, מאי ממקום קרוב. רב אמר ממקום קרוב ממש כל מקום ומקום לפי קרבתו איסורו. והכי קתני. ארמי שהביא חלילין בשבת. לא יספוד בהן ישראל. אא״כ באו לפנינו ממקום קרוב. שיהא בהן בשעת הספד כדי שיבאו מאותו מקום. ושמואל אמר חיישינן שמא חוץ לחומה לנו. כלומר לא כדי שיבאו ממקום קרוב בלבד. אלא כדי שאי אפשר שיבאו יותר ממקום רחוק שאפילו באו לפנינו בשבת שחרית חוששין שמא חוץ לחומה לנו. ונותנין להם זמן כדי שתחשך להם (ברחוק) [ברחוב] העיר (ולא) [ולנו] חוץ לחומה סמוך לחומה. וכל שכן במקום שיש לחוש ברחוק מקום חוץ לתחום. שתולה בו שמואל להחמיר. אלא שלא חשש שמא הלכו בהן כל הלילה שאין דרך בני אדם לילך בלילה. אלא שמא שקעה עליהן חמה בסוף התחום ולנו בלילה חוץ לחומ׳ וזה כענין מה שאמרו בפסחים (יג.) חיישי׳ שמא חוץ לחומה לנו אורחי׳. ואמרי׳ דייק׳ מתני׳ כוותי׳ דשמואל, דתנן במכשירין (ב,ה) עיר שישראל וארמים דרים בתוכ׳ יהיה בה מרחץ המרחצת בשבת אם רוב ארמים רוחץ מיד. אם רוב ישראל ימתין בכדי שיחמו. ר׳ יהודה אומר באמבטי קטנה אם יש בה רשות רוחץ מיד. שמע מינה לרבנן אפי׳ יש בה רשות כגון אדם חשוב שיש לו עשרה עבדי׳ שמחממין לו עשרה קומקמין של מים. שיש לומר עבדיו חממו לו. ולא נעשה ע״י הבלנין אעפ״כ אסור לרבנן. חוששין שמא על ידי הבלנין נעשה. ובשביל ישראל. ואעפ״י שהעיר מחצה על מחצה ומיפלג פליגי רבנן עלי׳ דרבי יהודה. כדפליגי נמי התם בתלת בבי דההיא מתניתין שמעינן מינה מכל מקום דתולין במעשה שבת להחמיר כוותי׳ דשמואל וזה הפירוש הנכון. והוא דרכו של רבינו ז״ל בהלכות. ויש מפרשים דרב לחומרא ממקום קרוב ממש. שידוע לנו בבירור דממקום קרוב באו. כגון שראינום בביתו של ארמי. ושמואל לקולא. שאפילו הביאם מחוץ לעיר. תולין להיתר שמא בתוך התחום לנו בע״ש. ומצאו חיישי׳ להקל כגון חיישי׳ שמא באמבטי עיברה לדעתם. וכך פרש״י ז״ל. וסיעתא דשמואל מדרבי יהודה דתלי ברשות. וקשה לי לפירוש הזה. אם כן דשמואל לקולא תלי בכל מלאכת שבת אפילו בשל תורה כגון הבאת חלילין הללו. דאיכא איסור הוצאה והעברה ברשות הרבים. וכן בחמין שהוחמו בשבת תלינן לקולא כדאמר ר׳ יהודה. אם כן קשיא להו הא דתנן (ביצה דכ״ד ע״א) מצודות חיה ועופות ודגים שעשאן מערב י״ט לא יטול מהן בי״ט אלא אם כן ידוע שניצודו מערב י״ט. ומעשה בארמי אחד שהביא דגים לפני ר׳ גמליאל. ואמר מותרין הן אלא שאין רצוני לקבל הימנו. ואמר שמואל אין הלכה כר״ג. ותניא נמי ר׳ שמעון בן אבא אומר בא ומצאן מקולקלין מערב י״ט. בידוע שמעי״ט ניצודו וספק נעשה כמי שניצודו בי״ט ואסורין.אמר שמואל הלכה כר׳ שמעון בר אבא. ומספק מוכן נמי פסקו שם דאסור. וא״ת לא הקלו במלאכות שבת ליומן. אלא בשיעור כדישיעשו הוא דתלי שמואל לקול׳. אכתי קשיא הילכתא אהילכתא דקי״ל הלכתא כשמואל בדיני. דדיקא מתניתין כוותי׳ והתם אמר רב פפא הלכתא ארמי שהביא דורן לישראל. אם יש מאותו המיןבמחובר אסורין. ולערב אסורין נמי בכדי שיעשו. אלמא לערב נמי תולין בדבר להחמור. במעשה שבת על ידי ארמי. ואעפ״י שאינו יודע שתלישתן בשביל ישראל היתה. וסוגיא דהתם נמי מדשמואל דאמר שמואל לחומרא פסק רב פפא לחומרא. דתרווייהו בחדא שיטה אזלי במעשה שבת ויום טוב ועוד להאי פירושא לשמואל לא הוי ליה לתנא למיתני אא״כ באו ממקום קרוב. דאפשר שבאו. אלא אדרבא הוה ליה למיתני ארמי שהביא חלילין יספוד בהן סתם. ואם באו ממקום רחוק לא יספוד בהן ישראל שהרי הסתם מתיר. ובידוע הוא שאסורין. וכן משמע, שאפילו ספק תחומין אסורים בי״ט. מדאמרינן בפרק בכל מערבין (ד״מ ע״א) גבי ההיא ליפתא דאתאי למחוזא. חזייא רבא דכמישא אמר הא ודאי מאתמול עקירא. מאי אמרת מחוץ לתחום (חייש). ולא תלי שמא ממקום קרוב אי נמי מחוץ לחומה הביאוה ושמא לנו משתחשך. אלמא ש״מ כל מעשה שבת וי״ט לחומרא. וכן הדין לכל דבר שיש לו מתירין אפי׳ בדרבנן ספיקו אסור. הלכתא ארמי שהביא חלילין מן הסתם אסורין דבשביל ישראל הביאו ומחוץ לתחום באו עד שיודע שממקו׳ קרוב באו ודאי ושוהין להם כדי שיעשו שיביאו מאותו מקו׳ וסופד בהן. עשו לו ארון חפרו לו את הקבר דתנן ואם בשביל ישראל לא יקבר בהן עולמי׳ אקשינן עלה בגמרא. ואמאי הכי נמי לימא כדי שיעשו. אמר עולא קבר בעומד בסרטיא. ארון אמר ר׳ אבהו במוטל על קברו דכיון דמלתא דפרהסיא הוא גנאי גדול הוא למת שיאמרו עליו זה קברו של פלוני שחללו את השבת בשבילו, כך פירשו הנאונים ז״ל. ובתוספתא תניא התם (יח,ח) ארמי שהביא חלילין בשביל ישראל בשבת. לא יספוד בהן אותו ישראל אבל ישראל אחר מותר. עשו לו ארון חפרו לו את הקבר לא יקבר בהן אותו ישראל. אבל ישראל אחר מותר. ענין התוספתא כשהביא החלילין בשביל ישראל שאמר לו כן ועשו הקבר בשבילו. וקנסו באותו ישראל לעולם. בקבר הידוע לו. וחלילין נמי ענין של פומבי הוא שאמר אתמול באו. והיינו דלא קתני אא״כ באו ממקום קרוב. והתירו בישראל אחר לאחר כדי שיעשו. אבל פחות מכדי שיעשו לכל ישראל אסורין. וכן בכל מעשה שבת. וכן מוכיח במסכת ביצה:
}
\textblock{מתני׳: \textbf{גוי שהביא חלילין בשבת לא יספוד בהן ישראל.} פרש״י ז״ל לעולם. ולא מחוור, מדקתני סיפא גבי ארון וקבר עולמית והכא לא קתני ש״מ לאו לעולם קאמר אלא בכדי שיעשו ועוד אי לעולם קאמר דאסורין דכיון שנעשית בהן מלאכה בשביל ישראל קנסי׳ עולמי׳ כי באו ממקום קרוב מאי הוה הרי הביאן ד׳ אמות ברה״ר ונעשית בהן מלאכה של תורה בשביל ישראל ומה לנו חוץ לחומה ומה לנו חוץ לאלף תחומין. ונ״ל דמתני׳ לא בשהביאן בשבת בשביל ישראל זה קאמר דא״כ הוה ליה למיפלג ולמיתני כדקתני בסיפא בקבר וארון אלא מתני׳ בשהביא חלילין סתם ומסתמא לשם ישראל הביאן שאין דרכן של גוים לקונן ולספוד בחלילין ורובן בשביל ישראל שמקוננין בהם מביאין אותן לפיכך אסורין לכל ישראל עד כדי שיבואו ממקום קרוב ואם עשו ארון וחפרו קבר מסתמא לדעת עצמן עשאוהו ולפיכך מותר מיד והיינו דקתני עשו לו ואם בשביל ישראל לא יקבר בו עולמית שום ישראל ועלה הוא דמקשי׳ בגמ׳ ואמאי ה״נ לימא בכדי שיעשו ואמאי אסור לעולם ומפרקי׳ בעומד באסרטי׳ וגנאי גדול הוא למת שיקבר בקבר שחללו בשבילו את השבת כך פירשו הגאונים ז״ל דבר זה ולא כדברי רש״י ז״ל:
}
\textblock{גמ׳: \textbf{מאי ממקום קרוב רב אמר ממקום קרוב ממש.} פי׳ רבינו הגדול ז״ל שאם באו עכשיו ממקום קרוב מותרין ואין חוששין לדבר אחר ושמואל אמר אע״פ שאנו רואין עכשיו שהוא מביאן ממקום קרוב חוששין שמא ממקום רחוק באו הלילה ולנו חוץ לחומה במקום קרוב ועכשיו מביאן מחוץ לחומה ודכ״ע מתני׳ הכי קתני אלא אם כן באו ממקום שיש בו בשעת הספד כדי שיבואו משם אלא משום דסתם הספידן אינו ממש בשעה שחשיכה ואם באו ממקום קרוב סתמא יש בו שיעור משום הכי קתני לה סתם כדתנן נמי (רוחץ בה מיד ולוקח מיד) אבל מ״מ בין קרוב בין רחוק צריך כדי שיבואו משם דהא בשבת תוך התחום נמי איכא הוצאה והעברה ברה״ר שנעשית בשביל המת:
}
\textblock{ה״ג בכל נוסחי ספרי ספרד וזו היא גרסתן של גאונים ז״ל: \textbf{דיקא מתני׳ כוותי׳ דשמואל דתנן עיר שישראל וגוים דרין בה וכו׳.} ומתני׳ היא במסכת מכשירין פ״ב ומייתי ראי׳ מדת״ק דאמר מחצה על מחצה אסור ואף על פי שיש שם רשות ולא תלי׳ ברשות אף על פי שדרכו לרחוץ יותר מכל אדם ורגלים לדבר אפ״ה אסור א״נ מחצה על מחצה גופי׳ ראיה שאין תולין במעשה שבת להקל אלא להחמיר. ול״נ דדיוקא מדקתני מרחץ המרחצת בשבת דמשמע דסתמא אסור ואע״פ שהיינו יכולין לתלות ולומר מע״ש הוחמו חמיו או שפקקו נקביו וכן סיפא מצא בה ירק נלקט אסור ואע״פ שי״ל מע״ש נלקט. ורש״י ז״ל מפרש חיישי׳ לקולא, ואמר שמצא כיוצא בו חיישינן שמא באמבטי עיברה בפ׳ אין דורשין וגירסא מוחלפת היא זו אבל לפי הגירסא האמיתית אין ראי׳ ממשנתינו שבמס׳ מכשירין לשמואל אלא מדר״י דאמר אם יש בה רשות רוחץ בה מיד שתולין להקל ולומר שמא משחשיכה נעשה ע״י גוים ועבדים הרבה ופי׳ לשמואל דהכי תנן לא יספוד בהן ישראל אא״כ יש לתלות שבאו ממקום קרוב. ואינו מחוור לי, חדא דא״כ קשה מתני׳ לשמואל דקתני גוי שהביא חלילין וכו׳ אא״כ באו ממקום קרוב ואי ס״ד תלינן לקולא אע״ג דלא ידעינן מנא אתו ולא אסרו אלא בבאין ממקום רחוק בידוע הכי הו״ל למיתני גוי שהביא חלילין ממקום רחוק או מחוץ לתחום לא יספוד בהן ישראל דמשמע במביא בידוע מחוץ לתחום. ועוד דאא״כ באו ממקום קרוב לא משמע שאפשר לתלות שיבואו אלא באו ממש משמע ועוד מאי ראיה מדר״י התם פשיטא דשרי ואפי׳ נעשו בשבת הואיל ואיכא שיעור שאפשר להחם אותן משחשיכ׳ דהא איכא כדי שיעשו אבל יכולני לפרש לקולא ולומר דלשמואל הכי קתני לא יספוד בהן ישראל אא״כ באו ממש ממקום קרוב שיש בו עכשיו כדי שיעשו לאפוקי באו ממקום רחוק הא ספיקא לקולא הוא וספיקא לא קתני מתני׳ כאותה ששנינו במצודות חיה לא יטול מהן ביו״ט אא״כ ניצודו מבע״י ספק ר״ג מותר וכו׳. מיהו הא ק״ל, דמשמע מדסייעוה לשמואל ממתני׳ דגבי מעשה שבת נמי תליא לקולא ואפי׳ היכי דאיכא איסור׳ דאוריי׳ כגון בחמין שהוחמו בשבת וכן הבאה דשבת עצמה דאורייתא היא ואמאי הרי אמרו גבי י״ט אם יש מאותו המין במחובר אסור ועוד שנינו מצודות חיה ועופות ודגים לא יטול מהן ביו״ט אא״כ ידוע שניצודו מבע״י ואע״ג דממילא נצודו ואין בהן אלא איסור מוקצה דרבנן דספק הכן אסור ולא תלינן לקולא ומה בין ספק דבר שנתלש היום לספק תבשיל שנתבשל היום שזה מותר וזה אסור וכ״מ לי מהא דאמרי׳ במס׳ עירובין בפ׳ בכל מערבין ההוא ליפתא דאתא למחוזא חזיי׳ הבא דמכמשה אמר הא ודאי מאתמול עקורה מאי אמרת מחוץ לתחום אתא וכו׳ ורבא ודאי לא היה יודע מהיכן באו אלא שחשש שמא מחוץ לתחום באו וא״ת שמא יודע היה שבא מחוץ לתחום א״א לומר כן מדאמר מאי אמרת משמע דה״ק מה י״ל וליחוש דאי ודאי חוץ לתחום אתא הכי הול״ל מאי איכא דמחוץ לתחום אתאי וכן הלשון הגון ונהוג בכ״מ אלא ש״מ שספק באו מחוץ לתחום אסור, וכ״ש ספק מעשה שבת. וי״ל דלא הקיל שמואל בספק מלאכת שבת ליומו אלא בכדי שיעשו למ״ש הקיל מיהו בההיא דאר״פ (ביצה כד:) בי״ט אם יש מאותו המין במחובר אסור ולערב נמי אסור בכדי שיעשו משמע דספיקא לחומרא ל״ש ליומי׳ ול״ש לערב:
}
\textblock{מהא דתנן \textbf{ובלבד שלא יזיזו בו אבר.} ש״מ דטלטול מקצת שמיה טלטול שלא תאמר עד שיגבי׳ או שיגרור אלא כיון שמזיז ראשו של דבר שאינו ניטול אסור והא דתנן לקמן בפ׳ בתרא בשליף של תבואה מכניס ראשו תחתיה ומסלקו לצד א׳ משום צער בעלי חיים התירו כן:
}
\newsection{דף קנג}
\textblock{הא דאמרי׳ \textbf{קים להו לרבנן דאין אדם מעמיד עצמו על ממונו.} ה״ק שאין אדם יכול לזרוק ממונו מידו להפסידו ומש״ה שרי לי׳ שבות דלא ליתי לידי איסור דאורייתא משא״כ בדליקה שלא התירו לו אמירה לגוי ואדרבה אמרו אי שרית לי׳ אתי לכבוי. ומסתברא דוקא החשיך לו בדרך שבו ביום גדשו סאה וחששו שמא יקל לומר עדיין יש שהות ביום אבל שכח בשבת והוציא כיסו לרה״ר או לדרך לא התירו לו כלום. והא דאמרי׳ בפ׳ נוטל פעם א׳ שכחו ארנקי מלאה מעות בסרטיא ואמרו הניחו עלי׳ ככר או תינוק וטלטלוה במחיצה של בני אדם קאמר אבל לטלטלה פחות מד״א לא אמרו אלא במי שהחשיך בדרך דהא מסקנא התם לא אמרו ככר או תינוק אלא למת בלבד והכא שרי טלטול והולכה פחות מד״א אלא ש״מ שאני כיס שבידו שאין אדם מוצא לבו לזרקו. ומכאן נלמוד למי שבאו לסטים בביתו שאינו נוטל מעות מפניהם להטמינן אלא הרי לסטים כדליקה ושנינו תיק הספר עם הספר אע״פ שיש בתוכו מעות הא בשביל הפסד המעות ושאר הנכסים לא נתיר הצלתן ולא אמירת כבה ואל תכבה ואע״ג דהתם נמי איכא למיחש מתוך שאדם בהול על ממונו אי לא שרית ליה אתי לכבויי כדאמרי׳ פ׳ במה אשה במערימין בדליקה ואפ״ה לא שרי׳ אלא הצלה דדבר הראוי למקום הראוי ואם הצלה זו דומה לכיס שבידו בדרך יציל לכל מקום שלא יהא בו איסור תורה ויאמר לגוי להדיא כבה אלא ש״מ דין אחר הוא במחשיך וכיס בידו ממש. ומסתברא נמי דוקא במחשיך אבל בשוכח ומוציא כיסו לדרך אסור. ובעל התרומות (סי׳ רכ״ו) מתיר להציל כיסו מפני הליסטין הבאים בביתו ואין בדבריו טעם ורש״י ז״ל כתב בפ׳ נוטל גבי ארנקי הניחו עליו ככר או תינוק וטלטלוה פחות פחות מד״א או במחיצה של ב״א ולא מסתברא כלל ולמה יאסרו טלטול אפי׳ ע״י ככר או תינוק מאחר שהותר להם פחות פחות מד״א דהוא איסור חמור ממנו. ועוד היאך יוסיף בהעברה שלו הככר הזה אלא במחיצה של ב״א קאמר ולא שרו לי׳ כלו׳ אפר בשל דבריהם. ושמא יוכל בעה״ת לטעון דהתם בסרטיא שבעיר יכול לישב ולשמור ממונו משא״כ במחשיך בדרך ומ״מ אין הדברים נכונים אלא לסטים היינו דליקה ממש. ולא שייך לישנא דאין אדם מעמיד עצמו על ממונו אלא במפסיד ממונו בידים:
}
\textblock{\textbf{דוקא כיסו אבל מציאה לא.} ודבר פשוט הוא דכי היכי דאסרי׳ במציאה למיתבה לנכרי ה״נ לא יהיב ליה לחש״ו ולא ע״ג חמור וכ״ש שאין מתירין לו להוליכה פחות פחות מד״א שהוא ההיתר הגדול שבכולן. אלא שהרמב״ם ז״ל כ׳ שמותר במציאה להוליכה פחות פחות מד״א ואע״פ שאסור לתתה לנכרי עלה בדעתו שאין אסור בהולכה פחות פחות מד״א אלא שחששו בו דלמא אתי לאתויי ד״א ברה״ר ואין זה כלום, שהרי מ״מ איסור טלטול יש בדבר וכיון שלא התירו אמירה לנכרי גבי מציאה מנ״ל דשרי איסור טלטול וכ״ש להוליכה פחות פחות מד״א הוא קשה להתיר יותר מפני שהוא קרוב לבוא לידי איסור תורה מפני שאדם עשוי שלא לדקדק בשיעור הד׳ אמות ואינו מטעם הפסד ממון כמ״ש הרב ז״ל ובפרק המוצא תפלין איכא מאן דלא שרי להוליך פחות פחות מד״א בתפלין של מציאה ואע״פ שהן של מצוה ורחוק שתהיה מציאה שאינה של מצוה יותר מותרת ממציאה של מצוה:
}
\textblock{\textbf{או דלמא כיון דלא טרח לא אתי לאתויי.} אע״פ שזה הלשון מוכיח שבכל מציאה אמרו ואע״פ שבאה לידו מזמן רחוק קשה לומר כן שא״כ נמצאת צריך לדקדק על כל ממון שבידו אם טרח בה אם לאו א״ו כשאמרו מציא׳ הבא לידו לא אמרו אלא שבאה לידו משעה קרובה:
}
\textblock{\textbf{חמור, חרש, שוטה, וקטן.} אע״ג דלא תנן לה במתני׳ חרש שוטה וקטן בהדיא מ״מ בכלל חמור הם שאינן מצווין על שמירת שבת אלא שאתה מצווה על שביתתן לעשות מלאכתו על ידן ועדיפי מיני׳ דהני אדם וחמור לאו אדם כדמפרש ואזיל ואין לך לתמוה האיך עובר אדם על איסור של תורה שכבר פירשו לפנינו שמניחין עליה כשהיא מהלכת ועל אותו הכוונה הם אומרים כן וכן כשהזכירו כאן חש״ו על אותה הכוונה הוא שיניחו עליהן כשהן מהלכין:
}
\newchap{פרק \hebrewnumeral{24} מי שהחשיך}
\textblock{}
\textblock{\textbf{והלא מחמר.} ה״נ הו״ל לאקשויי והלא שביתתן עליך מן התורה אלא דהא עדיפא ליה משום דהוה יכול למימר בחמור של גוי ואע״ג דקתני אין עמו נכרי מניחו על החמור דמשמע הא יש עמו נכרי לנכרי יהיב ליה ואמרי׳ לעיל משום דחמור אתה מצווה על שביתתו אפי׳ בחמור שאינו שלו אית לן למימר דלנכרי יהיב לי׳ משום דחמור אפשר לבוא לידי איסור תורה אם יהא מחמר אחרי׳ דזימנין דקיימו ולאו אדעתי׳. וה״נ אית לן למימר בחמור שלו אף בשאינו מחמר אחריו שמפני ששביתתו עליו מן התורה אין לו להניחו עליו אלא כענין זה שאמרו כאן שמניח עלי׳ כשהיא מהלכת וכשהיא עומדת נוטלה הימנה ולא להניחו עלי׳ בענין שתעשה עקירה והנחה שאין לו לעבור על ד״ת וכשאמרו למעל׳ חמור אתה מצווה עליו מן התורה עניינו מפני שאפשר לו לבוא לאיסור תורה. או אפשר שנאמר שמפני שמחמר אסור לאו כדבעי׳ למימר לקמן (מדלא כתיב) [מדכתיב] לא תעשה כל מלאכה אתה ובהמתך ובהמה לבדה אין שביתתה עליך אלא בעשה מדכתיב למען ינוח שורך וחמורך ניחא ליה לאקשויי ממחמר דהוא איסור לאו ואע״ג דאסיקנ׳ דפטור מכלום משום דכתיב אתה דאיהו ניהו דמחייב בבהמתו לא מחייב היינו דלא לקי כדאמרי׳ מעיקרא לסברא דלאו שניתן לאזהרת מיתת ב״ד שאין לוקין עליו שדעתם לומר שאין המלאכה שהוא עושה עם בהמתו חשובה מלאכה להתחייב עליה מלקות אלא כמלאכת עבדו ובנו שהוא מוזהר עליהן ואינו לוקה שאין מלקות אלא במעשה עצמו לא במעשה אחרים אבל ודאי איסור לאו יש בדבר כהרבה לאוין שיש בתורה שאין לוקין עליהן כגון לאו שבכללות ולאו שאין בו מעשה והכי איתא בהדיא ביבמות פ״ק דמחמר לאו גרידא הוא. וה״ר משה ז״ל כ׳ בפ״ב מהל׳ שבת שאם הוציא משא על הבהמה אע״פ שהוא מצווה על שביתתה אינו לוקה לפי שאיסורו בא מכלל עשה ואחז״ל שהלאו המפורש בתורה לא תעשה כל מלאכה אתה ובהמתך הוא שלא יחרוש בה ולדבריו המחמר אחר בהמתו אין בו איסור אלא משום שביתת הבהמה ויבא כפשוטו מה שהקשו והלא מחמר ולא הקשו משביתת בהמתו שהכוונה במחמר הוא משום שביתת הבהמה בלא שיהי׳ הוא מחמר אחרי׳ שאף כשעושה בו הגוי מלאכה כדאי׳ בפ״ק דע״ז וז״ש שעל החורש נאמר בתורה לא תעשה כל מלאכה אתה ובהמתך אינו כלום שהרי מפורש כאן שהלאו של לא תעשה כל מלאכה אתה ובהמתך אין לוקין עליו והחורש בבהמה בשבת הוא חייב משום מלאכת עצמו ובשבת חייב סקיל׳ ובי״ט לוקה כמו ששנינו בפ׳ ואלו הן הלוקין יש חורש תלם א׳ וחייב עלי׳ משום ח׳ לאוין החורש בשור וחמור והן מוקדשין וכלאים בכרם ושביעית וי״ט וכו׳ והוא ודאי לחיוב מלקות כמו שמוכיח שם באותו משנה ובגמ׳ שעליה. אלא שמכל מקום צריך טעם בדבר שהרי החורש בבהמות אתה ובהמתך הוא כמחמר שהחרישה בכח הבהמה היא נעשה ונאמר בזה שמפני שהחורש בבהמ׳ הוא נותן עליה עול והוא כובש אותה תחת ידו וברשותו היא עומדת כל המלאכה על שם האדם היא ובו היא תלוי׳ ואין הבהמה אלא ככלי ביד אומן ואינו דומה למחמר שהבהמה היא הולכת לנפשה אלא שיש לה התעוררות מעט מן המחמר:
}
\textblock{הא דאמר ר״א בר אהבה \textbf{רץ תחתיה עד שמגיע לביתו.} הו״ל לאקשויי עלה אמאי לא תני לה במתני׳ אלא שלא רצו חכמים לגלותה כההיא דר׳ יצחק ואע״ג דלא אמרי׳ עליה בהדיא הכי חדא מכלל חברתה איתמר. אלא שקשה הדבר למה לא שנו את שתיהן כאחת מטלטלו פחות פחות מד״א או רץ בה עד שמגיע לביתו והנר׳ לרב מורי נר״ו דבמתני׳ דקתני כיסו לא שרי׳ הכי דדוקא בחבילה הוא דשרי׳ דאית לה הכירא דלאו אורחי׳ למירהט בחבילה אבל בכיס אורחי׳ ולית ליה הכירא ואפשר שנאמר דרב אדא בר אהבה אתא לאשמעינן תקנה בחביל׳ משום דלא אפשר בפחות פחות מד״א ואפשר לו שירוץ בו עד שיגיע לביתו אבל בכיס דמתני׳ א״צ לזה כיון דאפשר בטלטול פחות פחות מד״א אבל אם בא לרוץ בה רשאי שהרי ודאי שיש לו היתר אלא שאין דרכן של ב״א שיהיו רצים בדרכים ל״ש:
}
\textblock{\textbf{אמר רמי בר חמא המחמר אחר בהמתו בשבת בשוגג חייב חטאת וכו׳.} איכא דקשי׳ להו לרמב״ח אמאי לא תני לה גבי אבות מלאכות ואינה קושי׳ דמחמר לאו מלאכה בפ״ע היא שאינו חייב אלא על המלאכות שנישנו בפ׳ כלל גדול אלא שהוא חייב כשהוא עושה ע״י בהמתו כמו שהוא עושה ע״י עצמו אבל אב מלאכה אחר אין לנו אלא אלו שנישנו שם:
}
\textblock{\textbf{והא אנן תנן נוטל את הכלים הניטלין בשבת.} אין קושי׳ זו כלום דהא ודאי ר״ה בכלים שאינן ניטלין איירי דקתני מתני׳ מתיר החבלים והשקין נופלין ואתי איהו למימר דבכלי זכוכית המשתברין מביא כרים וכסתות ומניח תחתיהן אלא משום דבכלים שאינן ניטלין נמי לא ניחא ליה משום דקא בעי לאותובי עלי׳ והא קמבטל כלי מהיכנו מקשי עלייהו מעיקרא והא אנן תנן נוטל את הכלים הניטלין לומר דלא ניחא ליה לא בכלים הניטלין ולא בכלים שאינן ניטלין:
}
\textblock{\textbf{כי קאמר ר״ה בקרני דאומני דלא חזי׳ ליה.} נראה דקרני דאומני ישנים קאמרי׳ שהן מוקצין מחמת מיאוס ואתי׳ כר״י דאית לי׳ מוקצה א״נ כר״ש דהוו כשרגא דנפטא אמרי׳ דחזי לכסויי מנא הכא לכסויי מנא נמי לא חזי לפי שהן מאוסין הרבה כדאמרי׳ האי מאן דשתי בקרנא דאומני קא עבר על בל תשקצו את נפשותיכם אבל אם היו כלים חדשים אע״פ שמלאכתן לאיסור לא הו״ל לאקשויי והא קא מבטל כלי מהיכנו שיכול הוא ליטלן מע״ג כרים וכסתות אם הוא צריך למקומן דהיינו הכרים והכסתות שתחתי׳ אבל השתא דאמרן דלא חזייאן כלל ולית בהו תורת כלי שאינן נטלין לא לצורך גופו ולא לצורך מקומן מקשי׳ שפיר והא קא מבטל כלי מהיכנו:
}
\textblock{\textbf{בשליפי זוטרי.} אם נפרש שאין בשליפי זוטרי דין ביטול כלי מהיכנו נמצא שלא התיר ר״ה כלום בשביל שבירת כלי זכוכית וא״כ היכי מקשי׳ עלה מדתני׳ מתיר חבלים והשקין נופלין אע״פ שמשתברין והא ר״ה לא שרי אלא מאי דמשתרי בעלמא ועוד שאמרו בסוף השמוע׳ מ״ד להפסיד מועט נמי חששו קמ״ל נראה שטעמו של ר״ה אינו אלא מפני (שאין חוששין) [שחוששין] להפסד. לפיכך יש לנו לומר כדברי בעל המאור ז״ל דר״ה אית לי׳ דר׳ יצחק דאמר אין כלי ניטל אלא לדבר הניטל בשבת והכי אמרי׳ בפ׳ כירה פוקו וא״ל לר׳ יצחק כבר תרגומה רב הונא לשמעתיך בבבל והכא אתא לאשמעי׳ דשרי׳ איסור׳ מפני הפסד כלי זכוכית. אלא שקשה לדבריו שא״כ מתחלה (לא) היו יודעין שר״ה מתיר איסור משום הפסד ולא הו״ל לאקשויי והא קא מבטל כלי מהיכנו אלא שהרב ז״ל דוחה דאיסורא דמבטל כלי מהיכנו חמיר טפי ואינו מחוור. ועוד דלא אפשר למימר דאית ליה לר״ה דר׳ יצחק והא חזינן לי׳ לר׳ יצחק דלא שרי אפי׳ במקום הפסד כדאי׳ בר״פ משילין פירות (ביצה לו.).[ועוד] כיון דאנן לא קי״ל כר׳ יצחק אפי׳ בבולסא נמי הוה לן [למישרי] להלכה ואית לן למידחי בריית׳ דקתני ואע״פ שמשתברין אלא שהרב ז״ל אומר בזה שמפני שאין הבולסא ראוי׳ לכלום ואין בה תורת כלי כלל הוא אסור (לדבריהם) [לטרוח טרחא יתירה בשבילה], ואינו מחוור. אבל נ״ל שהיתרו של ר״ה הוא משום דבשליפי זוטרי נמי יש משום בטול כלי מהיכנו ואע״פ שאפשר לו לנער הרי דעתו להניחן שם כל שעה והו״ל ככלי תחת התרנגולת לקבל ביצתה וכלי תחת הנר לקבל ניצוצית דאמרי׳ בפ׳ כירה שאסורין משום בטול כלי מהיכנו ואע״פ שאפשר לנער ואתי׳ השתא למימר שבמקום שבירת כלי הזכוכית הוא שהתיר ר״ה כן אלא דבשליפי רברבי לא שרי משום דביטול כלי מהיכנו במה שא״א לי׳ לנער כל היום חמיר טפי ומקשי׳ מדתניא היתה בהמתו טעונה טבל ועששית מתיר חבלין והשקין נופלין דקתני אסורה בסתמא ומשמע דלא שרי׳ ביטול כלי מהיכנו לאו בשליפי רברבי ולא בשליפי זוטא. וא״ת ואמאי לא מוקמינן בשליפי רברבי שלא לשבור דבריו של ר״ה א״ל משום דקתני עששית דומיא דטבל משמע לן דל״ש בשליפי רברבי ול״ש בשליפי זוטרי אסור דל״ל דבשליפי רברבי דוקא ואסור משום ביטול כלי מהיכנו משום דטבל מוכן הוא אצל שבת שאם עבר ותקנו מתוקן אלא ה״ט דאסרינן בטבל משום שאין הפירות נפסדין כשהן נופלין לארץ וא״כ ל״ש שליפי רברבי ול״ש זוטרי אסורי כיון דדומיא דטבל קתני ומ״ה מקשי׳ מינה לר״ה ומתרצינן התם בבולסא והוא כלי זכוכית דקין שאין ראוין אלא להתיך ואין בשבירתן הפסד של כלום:
}
\textblock{\textbf{דייקי נמי דקתני דומיא דטבל. מה טבל דלא חזי ליה.} כלומר שאין מלאכתו נגמרה בלא תיקון לא בחול ולא בשבת אף דאי נמי דלא חזי ליה שאין מלאכתה נגמרה בלא התכה וא״א לפרש מה טבל דלא חזי ליה לשבת דאי מוקמי׳ לה נמי בקרנא דאומנא הא לא חזי ליה בשבת. ומאי אע״פ שמשתברין. כיון דאמרת שאין הפסד בשבירתן. מהו דתימא להפסד מועט נמי חששו, (קמ״ל) שאע״פ שאינן כלים אלא שהן עומדין להתיך יש הפסד בשבירתן שהשברים הדקין שבהן (א״א) לו (כלומר) והן הולכין לאיבוד וקמ״ל שלא חששו להפסד מועט להתיר בשבילו דבר שיש בו איסור כלל כך היא זו השטה והיא מתחוורת יפה. אלא שאין לנו לסמוך במה שאמרנו שיש ביטול כלי מהיכנו לשעה שלשון ביטול כלי מהיכנו אינו אלא כשהוא מבטלו לכל השבת וכ״ש זה שאינו מניח שם הכרים והכסתות אלא עד שיפלו השקין ודעתו ליטלן משם מיד ואינו דומה לכלי לקבל ביצה וכלי לקבל ניצוצו׳ שדעתו שיהי׳ שם כל היום. גם נראה שדעת הרב אלפסי ז״ל אינו כן שלמד מדברי ר״ה שאין הלכה כר׳ יצחק, ואם נאמר שר״ה מתיר איסור של ביטול כלי מהיכנו בשביל ההפסד, מנין לנו שלא יהא סבור שאין כלי ניטל לדבר הניטל בשבת ויהא מתיר בשביל ההפסד כמו שהוא מתיר איסור של ביטול כלי מהיכנו. אלא דעתו של הרב ז״ל שאין ר״ה מתיר איסור קבוע בשביל ההפסד, אבל א״א שיהא מותר כמו כן בלא הפסד כלל שהרי הסוגיא מוכחת שההיתר הוא בשביל הפסד כמ״ש. אבל המחוור בזה הוא שנאמר שהתירו של ר״ה הוא שמתיר טורח טלטול הכלים במקום הצלה בשביל הפסד זה שאתה רואה שבחבית שנשברה אסרו להביא כלים להניח תחתי׳ אעפ״י שהיין הוא ראוי כמ״ש בפ׳ כירה והטעם הוא מפני שאדם בהול על ממונו ואי שרית ליה אתי לאתויי כלי דרך רה״ר או משום דהוי כעובדין דחול כמו שאתה רואה שאף בי״ט שאין בו עירוב והוצאה לא התירו לכסות פירות הראויין לי״ט אלא משום הפסד ואף כאן מפני שהבהמה טעונה והיא מצטערת יש לחוש להבאת הכרים והכסתות ואע״פ שאין בהן איסור כלל שמתוך שאדם בהול על ממונו בצער הבהמה אתי לאתויי דרך רה״ר אבל כשהיא טעונה כלי זכוכית מפני הפסד הזכוכית התירו כן מפני שהוא צריך לאבדו בידים ואין אדם מעמיד על ממונו לאבדו בידים כדאמרי׳ לעיל גבי נותן כיסו לנכרי אבל בבולסא לא התירו מפני שהוא הפסד מועט ואע״פ שחששו בפ׳ משילין בהפסד מועט לענין כדי יין וכדי שמן שמא בי״ט הקילו שאין לחוש לשמא יביא כלי דרך רה״ר. ועכשיו נאמר שלא התיר ר״ה מה שיש בו איסור ממש בטלטולו אלא חשש על איסור זה שיש בכלים המותרים ונאמר שר״ה לא ס״ל כר׳ יצחק כדברי הרב אלפסי ז״ל אבל קשה הדבר במ״ש בפ׳ כירה כבר תרגומה ר״ה לשמעתיך בבבל ויש שאין גורסין שם ר״ה או נאמר שתרגומו של ר״ה אינו לאמרה כדברי ר׳ יצחק לגמרי שלא יהא כלי ניטל לדבר שאינו ניטל אלא ר״ה תרגומו דלא אמרי׳ הכי אלא לגבי מת שהוא דבר שאין בו צורך כלל אבל לדבר שיש בו צורך לא אמרה ר״ה, וכ״ד הרב ז״ל. או נאמר דרך אחרת, דר״ה ודאי אית ליה דר״י לגמרי שאין כלי ניטל לדבר שאינו ניטל ואע״פ שיש הפסד אבל דבריו של ר״ה בטעונה כלי זכוכית אינה בדבר שאינו ניטל אבל נאמר שהוא דבר הניטל דאע״ג דאוקי׳ בקרנא דאומני אפשר לנו לפרש שהן כלים חדשים שאינן מוקצין מחמת מיאוס ותורת כלי יש בהן אלא שמלאכתן לאיסור ולפיכך הן בכלל כלים שאינן ניטלין בשבת שאמר עליהן במשנתינו מתיר החבלים והשקין נופלין שהרי א״א לו ליטלן מע״ג החמור ואף על פי שניטלין לצורך מקומן אין זה צריך למקומן שא״צ לגבו של הבהמה שהן עליה אלא שהוא נוטלן להצניען וזהו מחמה לצל שאסור בכלים שמלאכתן לאיסור וכיון דר״ה מיירי בהכי אין לומר שאסור מפני שאין כלי ניטל אלא לדבר הניטל שאחר שיש בהן תורת כלי ניטלין הן לצורך גופן ולצורך מקומן ומותר בכולן ואע״פ שאמרנו שאין בכלים שמלאכתן לאיסור משום אין כלי ניטל אלא לדבר הניטל אם הניח כלים שמלאכתן לאיסור ע״ג כלים שמלאכתן להיתר יש בו משום ביטול כלי מהיכנו שהרי מ״מ בטל אותן שהכלי שמלאכתו להיתר היה ראוי לטלטלו אלא בשיהא צריך להן לצורך גופן או לצורך מקומן והואיל ואין דעתו לסלקן (משום דהו״ל) [הו״ל] ביטול כלי מהיכנו גמור והיינו דמקשו והא קא מבטל כלי מהיכנו ואע״פ שהיו אותן קרני דאומני ניטלין לצורך גופן ולצורך מקומן מפני שהכרים והכסתות הן ניטלין אף מחמה לצל ואוקמוה בשליפי זוטרי שאינן מבטלן שם אלא לשעה מפי הרב מורי נר״ו:
}
\textblock{\textbf{היתה בהמתו טעונה שליף תבואה מכניס ראשו תחתי׳ ומסלקו לצד אחר.} פרש״י ז״ל תבואה של טבל שאם אינה של טבל ראויה היא למאכל בהמה ותנן נוטל את הכלים הניטלין בשבת. ותימה הוא כיון שהיא תבואה של טבל שאינה ניטלית למה התירו להכניס ראשו תחתי׳ ולסלקו לצד אחר שאע״פ שאינו מטלטלה לגמרי טלטול במקצת שמיה טלטול כדמוכח בסוף פ׳ שואל מדתנן ולא יזיז בו אבר והיה אפשר לומר דמשום צער ב״ח הוא דשרי ליה אלא שאין הסוגיא סובלת שהתירו כלום משום צער ב״ח כמו שנפרש ואפשר לומר משום דלא הוי טלטול כי אורחיה הוא דשרי ליה משום צער ב״ח ויותר נראה לומר דמכניס ראשו תחתיה אפי׳ בעלמא בליכא צער ב״ח שרי משום שטלטול שבגופו לא גזרו בו רבנן כדתנן בפ׳ תולין הקש שע״ג המטה לא ינענענו בידו אבל מנענע בגופו ואמרי׳ נמי בפ׳ מפנין עושה בו שביל ברגלו בכניסתו וביציאתו:
}
\textblock{\textbf{ויביא כרים וכסתות ויניח תחתיה.} לפי השיטה האחרונה שכתבנו הוי מצי למימר דס״ל דר׳ יצחק דדבש והדביש דבר שאינו ניטל כלל הוא דר״ה לא התיר להביא כרים וכסתות אלא בקרני דאומני שהן ניטלין לצורך גופן ולצורך מקומן כמו שאמרנו אלא משום דלא ס״ל כר׳ יצחק לא אוקמה לדרשב״ג כותיה:
}
\textblock{\textbf{והאיכא צער ב״ח.} פירש״י וליתי צער ב״ח דאורייתא ולדחי׳ דרבנן כדרך מ״ש בפרק מפנין מבטל כלי מהיכנו דרבנן וצער ב״ח דאורייתא ואתי דאורייתא ודחי דרבנן. ולא דמי, דהתם לא אפשר בגווני אחרינא אבל הכא אמאי דחית כלי מהיכנו דהא אפשר לו להתיר חבלים ויפלו שקים ואע״ג דמיצטרו זיקי הא עדיפא לי׳ טפי ולא לידחי איסורא דרבנן. ועוד אי מאן דס״ל צער ב״ח דאורייתא דחי ביטול כלי מהיכנו כי מקשי׳ לעיל לר״ה והא קמבטל כלי מהיכנו אמאי לא מתרץ דקסבר צער ב״ח דאורייתא דהא קיי״ל הכי בפ׳ מפנין. א״ו כשהקשו כאן והאיכא צער ב״ח לא ליבטל כלי מהיכנו מקשו אלא למה שהקשה ויתיר חבלים ויפלו שקין ומשני מצטרי זיקי ואתי׳ למימר דכיון דצער ב״ח דאורייתא כי מצטרו זיקי מאי הוה דהו״ל למיפסדינהו משום צב״ח כיון דלא הו״ל לפרקה בענין אחר ואמרי׳ צער ב״ח דרבנן ובמקום פסידא לא גזרו רבנן בי׳ שיהא צריך לאבד ממונו בידים כדין נותן כיסו לנכרי וכמ״ש למעלה ונצטרך לומר שכיון שצער ב״ח אינו אלא מדרבנן אינו חמור [כמו] לבטל כלי מהיכנו ובשאר האסורין הקבועין בשבת שאינן נדחין מפני הפסד כמו שזכרנו למעלה. זהו פי׳ השמועה על נכון, ואין כאן שיתירו ביטול כלי מהיכנו משום צער ב״ח דאורייתא וכן בדין שהרי העמידו דבריהן במקום מצוה כדאיתא בפ׳ המילה ומ״ש בפ׳ מפנין דאתי דאורייתא ודחי דרבנן היינו משום דאיכא צער ב״ח ואיכא פסידא ואי ליכא פסידא לא הוו שרי דהא אמרי׳ אפשר בפרנסה ואי לא מביא כרים וכסתות ומניח תחתיהן דכיון דאיכא תרתי שרי ליה ביטול כלי מהיכנו שאינו שבות גמור אבל להעלותה כדרכה לא התירו לו מדלא אמרי׳ ואי לא אפשר להביא כרים וכסתות מעלן כדרכו. וא״ת והכא נמי איכא תרתי, צער ב״ח והפסד בהמה שהרי מתה למ״ש שאע״פ שר״ג לא העלה בדעתו שתמות שאלו כן יותר היה חושש להפסד הבהמה מהפסד הזיקין מ״מ בביטול כלי מהיכנו הי׳ יוצא ידי שתיהן אם הי׳ מביא כרים וכסתות ומניח תחתיהן י״ל בהפסד הזיקין כך הוא יוצא ידי שתיהן שבכך הוא נשמר מלהפסיד הבהמה ואינו עושה בה שום איסור ומעתה מוטב שיתיר חבלים ויפלו שקים משיביא כרים וכסתות ויניח תחתי׳ שאין ביטול כלי מהיכנו נדחה מפני הפסד כמו שמפורש בהלכה זו אבל ודאי אי לא הוה אפשר לי׳ להתיר חבלי׳ ויפלו שקים הוה קא שרי׳ ליה ביטול כלי מהיכנו משום הפסד הבהמה ומשום צער ב״ח כההוא דפ׳ מפנין אי הוה ס״ל דצב״ח דאורייתא אבל כאן אמרו דקסבר צב״ח דרבנן לומר שאף להפסד הזיקין לא הצריכוה מפני כך וכ״ש לבטל כלי מהיכנו שהוא אסור:
}
\textblock{\textbf{שתים בידי אדם ואחת באילן וכו׳ מתני׳.} כתבנו פירושה במס׳ סוכה בס״ד:
}
\newsection{דף קנה}
\textblock{\textbf{והלכתא צדדין אסורין צידי צדדין מותרין.} פי׳ לאו כאביי אלא אע״ג דאפשר דמאן דאסר בהני תנאי בצדדין אסר נמי בצדי צדדין ומאן דשרי שרי בתרוייהו כרבא אפשר שפסקו בגמ׳ צדדין אסורין צידי צדדין מותרין דקיי״ל כוותיה דמר בחדא וכותיה דמר בחדא דאי ס״ל כאביי דבצידי צדדין פליגי לא הוה לן למיפסק בצידי צדדין להתירא דודאי כת״ק קיי״ל דאסור ואפשר שנאמר (כמאן) דפסק האי פיסקא דצדדין אסורין וצדי צדדין מותרין (ופליגי) בצדדין וקיי״ל כת״ק דאסור:
}
\textblock{\textbf{אבל לא את הזירין לא לפספס אלא להתיר.} כך הוא הגירסא בספרים שלנו ותימה היא כיון דס״ל לר״י דמטרח באוכלא לא טרחינן אמאי שרי להתיר ורש״י ז״ל מפרש דהתיר זירין שוויי׳ אוכלא הוא ואין כן שא״כ מ״ט דר״ה דשרי בפקיעין ובכפין ויותר נראה לומר דהתיר לכ״ע שרי שאין בו משום עובדין דחול שדבר קל הוא ובפספוס הוא שחלקו לאסרו משום מטרח באוכלא או משום שוויי אוכלא. ויש ספרים שכ׳ בהן אבל לא את הזירין לא לפספס ולא להתיר וזהו גירסתו של הרב אלפסי ז״ל ודייקא דכיון דר״י מפרש דכיפין לחוד ופקיעין לחוד משמע דתלתא בבי קתני מתירין פקיעי עמיר אבל לא מפספסין ומפספסין את הכיפין וה״ה שמתירין דתרוייהו שוי׳ אוכלא הוא אבל לא את הזירין לא לפספס ולא להתיר דתרווייהו הוו מיטרח באוכלא והיתר פקיעין ה״ט דשרי דכיון דפקיעין תרי אין בו טורח של כלום אבל כזירין דתלתא איכא טירחא ואסור:
}
\textblock{\textbf{ר״י מתיר בחרובין לדקה כ״ש לגסה.} דלא ס״ל (כר״י) [לומר דר״י] הוא דאית לי׳ סברא אחריתי למימר דשוי׳ אוכלא משוי׳ ומטרח לא טרחי׳ ומ״ה קאמר לדקה שרי אבל לא לגסה דודאי לא משמע דפליגי בהכי מדלא פליגי נמי ארישא דמתניתין:
}
\textblock{\textbf{מאי דקה גסה ואמאי קרי לי׳ דקה דדיקה ואכלה.} כלומר שהיא כוססת יפה בשיני׳ והיא אוכלת קשין ורכין ומ״ה שרי דהו״ל מיטרח באוכלא ואין פי׳ דדיקה ואכלה כאותו שנאמר בפר׳ תולין (שבת קמא.) [גבי] נוטלין מן הבהמה (שפירוש) [שפיה] יפה כפי הפי׳ שפירש רש״י ז״ל בה שם. ויש כיוצא בזה הרבה בתלמוד לשון אחד בשני פירושין ואחד יש בהלכה זו בסמוך בתבנא סריא ופי׳ תבן קשה שהוא ראוי למאכל בהמה ע״י פירוך שאל״כ לא יהי׳ ראוי לטלטלו וכן הוא האמור בפרק כדי יין ורישא בתבנא סריא שהוא קשה וראוי להסקה ותבנא סריא האמור בפרק שואל גבי אין מחשיכין על התחום הוא תבן מוסרח שאינו ראוי לכלו׳ ולפיכך אין מחשיכין על התחו׳ בשבילו. וכפי הפירוש שפירש שם דיקה ואכלה האמור שם אף הוא כזה שנאמר כאן וכן שנאמר שתבנא סריא האמור בפ׳ השואל הוא שוה עם האמור כאן שבשניהן אין משמעותן על התבן המוסרח שאינו ראוי לכלום שהרי כאן אמרו שהוא ראוי לבהמה ושם כמו כן הוא צריך לו כיון שהוא מחשיך עליו אלא כאן וכאן הכוונה הוא על התבן המחובר לקרקע שפעמים שאין התבואה מצלחת ואין שם אלא התבן בלבד ואין קוצרין אותה אלא נשארת מחוברת בקרקע ולפיכך אמרו כאן שהיא צריכה פירוך מפני שהיא קשה שלא נדושה עם התבואה כמנהג התבן וכן אותה של פרק המביא כדי יין מפני זה היא קשה ועומדת להסקה ולענין מחשיכין על התחו׳ אינה ראויה לו בשבת מפני שהיא מחוברת וכן אמרו שם בשלמא קש משכחת לה במחובר אלא תבן המ״ל בתיבנא סריא כלומר שאף היא מחובר כמו הקש, וזה הפירוש נכון:
}
\textblock{\textbf{הני מזונותן עליך והני אין מזונותן עליך.} אפילו למישדא קמייהו נמי אסור, ואתיא כההיא דתנינן (ביצה כג:) אין צדין דגים מן הביברין בי״ט ואין נותנין לפניהן מזונות דאוקי׳ בפרק האורג בביבר גדול שאין מזונותן עליו מפני שאין ניצודין לו כמ״ש שם ורב יוסף ס״ל דמלקיטין הוא דספי לי׳ בידים למקום שיכולה להחזיר אבל למישדא קמייהו אפי׳ למי שאין מזונותן עליך שרי. והא דאמרי׳ אין נותנין מזונות לפני חזיר, דוקא חזיר שאינו בן גידול כלל כמ״ש (סוטה מט:) ארור מגדל חזירים או מפני שמצוין לו מזונות בכל שעה כדאמרי׳ לית עתיר מחזירא וכיון שאין מזונותן עליו ושכיחי לי׳ מזונות מודה ר״י דאפילו למישדא קמי׳ אסור כדבעי׳ לפרושי לקמן. וההיא דתנן שאין נותנין לפניהן מזונות דאוקים בביבר גדול מיבעי׳ לי׳ לרב יוסף למימר דשכיחי להו נמי מזונות התם כיון שהוא ביבר גדול דכיון דשמעי׳ לי׳ דאמר באין נותנין מים לפני יונים שהוא משום דשכיחי לי׳ מיא ה״נ אית לן למימר בשאר המזונות דכל דשכיחי להו מזונות טרחא בכדי הוא מיהו דיקאבאין מזונותן עליו אבל אם מזונותן עליו אף על גב דשכיחי להו מזונות שרי למישדא קמייהו ודתנן אבל נותנין לפני אווזין ותרנגולין ולפני יוני הדרסיאות לא תימא שמים בלבד נותנין להו משום דלא שכיחי להו מיא אלא אפי׳ אי שכיחי להו מיא נמי שרי (לומר) למיתב קמייהו. ודתניא בברייתא מהלקטיןלתרגנולין ואצ״ל שמלקיטין בענין זה היא שנויה אף על גב דשכיחי להו מזונות ואף על פי שאוכלין מעצמן מהלקטין להם ומלקיטין שכל שמזונותן עליך ל״ש להאכיל ול״ש לפטם שרי דלאו טירחא בכדי הוא כיון שמזונותן עליך וזה הדין כך הוא לר״י ויותר פשוט דהא איהו לא קפיד בדשכיחי להו כלל אלא בכל שיכולה לחזור הוא מתיר אף ע״פ שהוא טורח שאינו מועיל כלום במי שאוכל מעצמו. ומ״ש לפנינו גובלין אבל לא מוספין ודלא נקט בלישני׳ מלקיטין יתי׳ במקום שאינה יכולה לחזור הוא שאמרו שאין מוספין דאתיא כרב יהודה דתניא כוותיה [ולא כרב יוסף, דלדידיה] כיון שהוא מתיר אף למקום שאינה יכולה לחזור אף בשכיחי להו מזונות יש לו להתיר דהא לא קפדינן אטירחא דל״צ בכל שמזונותיו עליו כמו שכתבתי. ומ״ש ודלא נקט בלישני׳ מלקיטין יתי׳ אורחא דמילתא קאמר שאין דרך לעשות כן למי שהוא נקיט בלישנא דלא קפדי אטירחא בכדי בכל שמזונותיו עליו כמ״ש:
}
\textblock{\textbf{דהא תניא בהדיא אין נותנין מים למורסן.} היה נראה לפרש דאע״ג דמורסן לאו בר גיבול הוא שרי ר׳ יוסי בר׳ יהודה אלא שא״א לומר כן דהא בפ״ק מספקא לן אי שרי ר׳ [יוסי בר׳] יהודה בדיו וכאפר דלאו בני גיבול נינהו ולא פשטי׳ לה מהא מש״ה אית לו למימר דמאי דשרי ריב״י במורסן לא פשיטא לן אי משום דעביד [מורסן] (בקמח) [כקמח] דבר גיבול הוא או משום דשרי אפילו היכי דלאו בר גיבול הוא:
}
\newsection{דף קנו}
\textblock{\textbf{מאן י״א אר״ח ר׳ יוסי בר׳ יהודה היא וה״מ הוא דמשני.} אבל לרבי דס״ל דנתינת מים זה גיבול לא שריא כלל דהא א״א לשנוי שכל נתינת מים אחת ואין גובלין דקתני אף נתינת מים בכלל אבל לר״י ב״י דלא מחייב אלא בגיבול אפשר לה לשנוי על יד על יד שאע״פ שהוא מגבל כל צרכו אין דרך גיבול בכך שאין דרכו אלא במדוכה. וא״ת וכיון די״א דאמרי גובלין היינו ריב״י דקא שרי במשנה היכי מוקמי׳ ליה למתני׳ דקתני אבל לא גובלין כר״י ב״י וא״ל במורסן ותרנגולין ליכא למישרי בשנוי חדא דכל גיבול לתרנגולין הוי כי אורחי׳ שא״צ גיבול שלם שהתרנגולין עצמן מגבילין אותו ע״י ניקור ועוד דכיון דל״צ גיבול לא טרחינ׳ בכדי: }
\textblock{\textbf{לא קשי׳ הא בעבה הא ברכה.} לר״י ב״י נמי מיבעי׳ לן לאוקמה ברכה כדי להתיר לו לגבול כל צרכו ואפילו במרובה והוא שישנה בנתינת שתית ואח״כ חומץ: }
\textblock{\textbf{גובלין ולא מוספין. וה״מ הוא דמשני היכי משני אמר רב יימר בר שלמיא שתי וערב.} בכאן התיר במרובה בשינוי שלא יגבל כל צרכו אבל לגבל כל צרכו אסור אפילו על יד על יד שאע״פ שהתירו כן בשתית גבי אדם לא התירו בשוורים אי משום דעדיף לן למיטרח במאכל אדם יותר ממאכל בהמה אי משום דשתית לא מתאכלי בלא גיבול כל צרכו אבל מורסן לשוורים מתאכל בלא גיבול כל צרכו. ומ״ש מהו לגבל וא״ל אסור מהו לפרק וא״ל מותר היינו גיבול לשוורים שגיבול כל צרכו אסור ולפרק מותר שהוא גיבול כלאחר יד וזהו דעת הרב אלפסי ז״ל שכתבה בהלכות להא דזעירי אבל לפי הנראה [מפי׳ ר״ח ז״ל] לא שאלו בגיבול כל צרכו בלבד מכיון שחזרו ושאלו מהו לפרק שנראה ודאי שלא התירו שום גיבול בכלי עצמו אלא לפרק בלבד שהוא ניעור מכלי אל כלי שלא נראה כמכוין מלאכתו לגיבול ולדבריו שמא בכל גיבול הוא אסר אף בשל אדם אלא שאם נפרש כן נאמר שאין הלכה כדבריו [של זעירי] שלא נבטל הסוגיא מלפניו דאיהו כר׳ ס״ל וכבר פסקו הלכה כר׳ יוסי בר״י. והעולה מן הסוגיא כך הוא: שג׳ דינין יש בגיבול לתרנגולין ולשוורין ולאדם לתרנגולין כל גיבול אסור מפני שא״צ גיבול שלם לשוורים מותר והוא שלא יגבל כל צרכו ולאדם מותר ואע״פ שמגבל כל צרכו והוא שיגבל על יד על יד וה״מ בעבה אבל ברכה אפילו במרובה אלא שצריך שינוי בנתינת השתית תחלה ואח״כ החומץ: }
\textblock{מתני׳: \textbf{מחתכין את הדלועין לפני הבהמה.} בדלועין שנתלשו מע״ש מכיון דלא פליג עלה ר׳ יהודה ולא אתי לאשמועינן אלא דמיטרח באוכלא [שרי] כדאמרי׳ לעיל (שבת קנה.). ואע״ג דהאי תנא ל״ל מוקצה, כיון דקתני ואת הנבילה לפני הכלבים דהיינו אף כשנתנבלה בשבת מכיון דפליג עלה ר״י ואמר אם לא היתה נבילה מע״ש אסורה וה״ט דמודה בדלועין משום דמה שנתלש מן המחובר אסור לד״ה שאין אדם מצפה מתי יתלוש מן המחובר והו״ל כבכור שנפל בו מום דקאמר ר״ש אם אין מומו ניכר מעי״ט אסור לפי שאינו מן המוכן משום שאין אדם מצפה מתי יפול בו מום כדאיתא בפ׳ כירה ואף אם תאמר שאף בדלועין אדם מצפה להן מ״מ יש להן איסור פירות הנושרין משום גזירה שמא יעלה ויתלוש. ובנבלה שנתנבלה בשבת דשרי ואע״ג דאית לן למימר שאין אדם מצפה מתי תמות פליגי בה אמוראי איכא מ״ד דמודה ר״ש בבעלי חיים שמתו שאסורין ומוקי לה בחולה או במסוכן כדאיתא במסכת ביצה ואיכא מ״ד שאין ר״ש מודה בב״ח שמתו דכיון דבהמה למיתה קיימה אפשר שהוא מצפה למיתתה ואע״פ שהיא בריאה, ואינו דומה למום בבהמה שאינו מצוי ולא עביד דאתי:
}
\textblock{\textbf{לפי שאינה מן המוכן.} ואע״פ שהיתה מוכנת לאדם כגון שהיתה עומדת לאכילה הא אמרי׳ בפ׳ מקום שנהגו ובפ״ק דביצה דס״ל לר׳ יהודה דמוכן לאדם לא הוי מוכן לכלבים כלומר שאין הביצה מפורשת לכלבים בכלל ההכנה שהיא לאדם ולפיכך כשנתנבלה בשבת שאין הכנת האדם מתירתה מפני שהקצית לו מחמת איסור א״א להתירה בשביל הכנת הכלבים שלא הקצית להם מפני שלא היתה מוכנת לכלבים אלא ע״י הכנת האדם ועוד יתבאר זה ביאור שלם פ״ק דביצה ות״ק דשרי א״ל דטעמיה משום דמוכן לאדם הוי מוכן לכלבים אלא דל״צ להכי שאף במקום שאין הכנה כלל הוא מתיר דהא לית ליה מוקצה כלל:
}
\textblock{גמרא: \textbf{ואף ר׳ יוחנן סבר הלכה כר״ש.} לא הביאו בגמ׳ האיך אמרה ר׳ יוחנן מפני שאין הדבר מוכרע מדבריו כמו שתראה בפרק כירה שלא העלו בזה כלום וכשאמרו כאן שר״י סבר הלכה כר״ש לומר שכך היא קבלה בידם שכך היא סברתו של ר״י:
}
\textblock{\textbf{ומי אר״י הכי והאר״י הלכה כסתם משנה ותנן אין מבקעין עצים מן הקורות.} תימא היא, מאי קשיא ליה דהא מתני׳ נמי סתמא היא כר״ש דקתני ואת הנבילה לפני הכלבים ולא מצית למימר משום דפליג עלה רבי יהודה לאו סתמא היא דהא אמרינן בפרק כל הכלים נגר הנגרר נמי סתמא היא ואע״ג דפליג עלה ר״י ובהא מתניתא גופא אמרינן בפ״ק דביצה דהיא סתמא דאמרינן גבי שבת דסתם לן תנא כר״ש דתנן מחתכין את הדלועין וכו׳. וא״ל ודאי אע״ג דפליג ר״י עליו סתמא היא ואית לן למיפסק הלכתא הכי שהרי לכל הפחות שנויה היא בלשון חכמים ואמרינן בפ׳ אותו ואת בנו ראה ר׳ דבריו של ר״ש באותו ואת בנו ושנאוה בלשון חכמים אבל מכיון שר״י חולק עליו אינה מכלל סתם משנה שאמר בה רבי יוחנן הלכה כסתם משנה כדאמרי׳ בפרק החולץ דרמי׳ הא לא חש לקמחי׳ סתם ואח״כ מחלוקת היא ופי׳ רש״י ז״ל שמחלוקתה בצדה הילכך כיון דסתמא אין מבקעין עצים דהיא סתמא אלימתא פליגא אהא ההיא עדיפא ליה לר״י. מיהו לענין פסק הלכה לגבי יחיד סתמא גמורה הוא לפסוק כמותה כההיא דנגר הנגרר דאמרי׳ בס״פ כל הכלים מאי דעתך משום דקתני סתמא נגר הנגרר הוי נמי סתמא כלומר אין אתה צריך להביא ראיה מאותה סתמא שיש בזו הסתמא די לפסוק הלכה כמותה ולפיכך אמרינן בפ״ק דביצה גבי שבת דסתם לן תנא כר״ש שיש סמך וסעד בסתמא זו לפסוק כר״ש אכל בכאן שהן סבורין שהסתמות חולקות זו בזו יותר יש לסמוך על הסתמא שאין בה מחלוקת כמ״ש. ודאמר לקמן ר״י סתמא אחרינא אשכח בש״א מגביהין מעל השולחן וכו׳ משום דב״ש במקום ב״ה אינה משנה וכיון שדברי ב״ה הם כר״ש דינם כסתמא גמורה בלא מחלוקת ובדין היא דהו״ל לתרוצי דההוא סתמא בי״ט הוא ובי״ט הוא דסתם לן תנא כר״י כדאיתא בפ״ק דביצה אלא שאין הסוגיא זו מסכמת כן הרי השוו מדותיהן בשבת ויום טוב ורנב״י הוא דאית ליה האי סברא בפרק קמא דביצה ומתרץ לה למתנית׳ בהכי אבל רבי יוחנן לית ליה האי סברא דהא מוקים לה למתניתין התם באוקימתא אחריתא גזירה משום משקין שזבו מיהו אנן כרב נחמן ב״י קיימא לן דמסתבר טעמיה לאוקמיה לכולהו סתמא אליבא דהלכתא ובמקומו נפרש בע״ה:
}
\newsection{דף קנז}
\textblock{\textbf{ת״ש אין משקין ושוחטין את המדבריות.} הו״ל לתרוצי כי היכי דמתרצי׳ בפרק כירה דמדבריות כגרוגרות וצמוקין דמי דמודי בהו ר״ש אלא דהכא משמע דס״ל כאידך תירוצא דתרצינן התם דלדבריהם דרבנן קאמר וליה לא סבירא ליה אי נמי לא תירצו התם הכי אלא אליבאו דרבי דקאמר ומדבריות הן שאינן נכנסות ליישוב לעולם והנהו דאי כגרוגרות וצמוקין דמי אבל מדבריות דרבנן כיון שהם נכנסות ליישוב ליכא לדמויינהו לגרוגרות וצמוקין:
}
\textblock{כך הוא הגירסא במקצת הספרים: \textbf{בכל השבת כולה הלכה כר״ש.} ולפי גירסא זו נראה (שאין) [שאנו] למדין לפי לשון זה אם שונין במשנתינו לצורך הפרת נדרים מדין הפרת נדרים מעת לעת או הפרת נדרים כל היום שאם הפרת נדרים מעת לעת אית לן למימר דלצורך אתרווייהו קתני דהא אפשר ליה להפר למחר ונמצא שאין ההפרה בשבת שעה עוברת ואין מפירין אלא לצורך השבת ואם הפרת נדרים כל היום אין לך כל נדר שאין הפרתה שעה עוברת בשבת ואם כן אפילו שלא לצורך מפירין. ובלשון הראשון היתה השאלה אם מפירין שלא לצורך אף כשנאמר שהפרת נדרים מעת לעת מפני שהפרת נדרים אין צריך בית דין אלא שאומר לה טלי ואכלי טלי ושתי. וללשון האחרון קשה, שאף עם נאמר שהפרת נדרים מעת לעת אפשר שמפירין שלא לצורך השבת כגון שהיא שעה עוברת כגון שמערב שבת שמע שא״א לו שיפר אלא בשבת דלא מצית למימר אפשר לו להפר מערב שבת מאחר ששמע שאחר שעכשיו הוא השעה עוברת אין נמנעין בכך מידי דהוה אשאלת נדרים לצורך דאמרינן לקמן שנשאלין אף בשהיה לו פנאי וי״ל כי אמרינן נמי שלא לצורך לא הני מילי בדאפשר אחר השבת כגון דשמע בשבת אבל שמע מערב שבת הרי זה לצורך ומותר. ויש שאמרו בטלו שאינו מצוי מפני מצוי וכיון שהפרת נדרים מעת לעת הכל אסור כיון שרוב נדרים (אי) אפשר להפר למו״ש:
}
\textblock{\textbf{סליק חדושי הרמב״ן על מסכת שבת
} סליק חדושי הרמב״ן על מסכת שבת
}
\addpart{רשב״א על שבת}\renewcommand{\partname}[1]{רשב״א על שבת}
\fancyhead[CO]{ \partname\space\textendash\space \chapname}
\fancyhead[CE]{\partname}
\renewcommand{\sethebfont}{\fontsize{10.5pt}{13.1pt} \selectfont}\sethebfont
\newchap{פרק \hebrewnumeral{1} יציאות השבת}
\clearpage
\newsection{דף ב}
\textblock{}
\textblock{מתני׳:\textbf{ יציאות השבת שתים שהן ארבע בפנים וכו׳.} פירש רש״י ז״ל: שתים לחיוב לאותן העומדים בפנים דהיינו הוצאה והכנסה דבעל הבית, דתנא להכנסה נמי הוצאה קרי ליה, שהן ארבע עם הכנסה והוצאה דבעל הבית לפטור, וכן לעני. והדר קא מפרש שתים דחוץ דסליק מינייהו. והוא הנכון. ואחרים פירשו שתים שהן ארבע בפנים דהיינו הכנסות שנעשות בפנים והן הכנסה דעני והכנסה דבעל הבית. ואינו מחוור דכשבא לפרש לא היה ליה לערב הכנסות והוצאות יחד.\par \textbf{} ואיכא למידק אמתניתין מאי טעמא פלגינהו רבנן בתרתי דהוצאה והכנסה דעני היינו דבעל הבית, ומאי שנא הא מהא, וכדאמרינן בשבועות (יד, ב) גבי ידיעות הטומאה שתים שהן ארבע, והא שם טומאה אחת היא. ותירצו בתוס׳, משום דקרא נמי פלגינהו בתרתי קראי ואיצטריך תרי קראי בהוצאה, דהא בפרק הזורק (לקמן שבת צו, ב) מייתינן לה מויכלא העם מהביא (שמות לו, ו) ובפרק קמא דעירובין (יז, ב) משמע דנפקא לן נמי הוצאה מאל יצא איש ממקומו (שם טז, כט) ודרשינן ביה אל יוציא, דאמרינן התם לוקין על ערובי תחומין דאורייתא, ופריך והא לאו שניתן לאזהרת מיתת בית דין הוא, פירוש משום דדרשינן מיניה אל יוציא. ואע״ג דמשני התם מי כתיב אל יוציא אל יצא כתיב ואל יצא קרינן, לאו למימרא דלא נדרש מיניה כלל אל יוציא, אלא לומר דעיקר קרא כי אתא לתחומין הוא דאתא מדכתיב אל יצא וקרינן ליה אל יצא, אבל לעולם לדרוש מיניה נמי אל יוציא. ותדע לך דהא לרבנן דלית להו ערובי תחומין דאורייתא אל יצא למאי אתא.\par \textbf{} וכי תימא קרא נמי למאי פלגינהו. אפשר לומר משום דהיא מלאכה גרועה שאילו פנה מזוית לזוית משא גדול פטור והוציא קצת מרשות לרשות חייב, מה שאין כן בשאר כל המלאכות שאינן אסורות אלא מצד עצמן באיזה רשות שתעשה. והיינו נמי דלא גמרינן לה ממשכן כדגמרינן כולהו שאר מלאכות, ותדע לך נמי דהא בפרק הזורק (לקמן שבת צו, ב) אמרינן הוצאה מנין שנאמר (שם לו, ו) ויכלא העם מהביא, ואקשינן ודילמא בחול וקאמר להן משה לא תיתו דשלימא לה מלאכה, ואיצטריך לשנויי דגמר העברה העברה מיום הכפורים, אלמא אף על גב דהוי במשכן אפילו הכי איצטריך קרא לפורטה מה שלא הוצרך בשאר מלאכות, והלכך הוצאה חדוש הוא ואין לך בו אלא חדושו ואי לאו דגלי רחמנא בתרווייהו לא ילפינן חדא מחברתה. ואם תאמר אם כן לחשוב להו בתרתי והוי להו ארבעים מלאכות. יש לומר דמכל מקום מלאכה אחת היא ושם אחד יש לה ונחשבת היא כאחד, אלא שהוצרך הכתוב לגלות בהן שזו וזו אחת הן. ואם תאמר אזהרה שמענו עונש מנין. יש לומר כיון דגלי רחמנא דמלאכות נינהו הרי הן בכלל כל העושה בו מלאכה יומת.\par \textbf{} (והר״מ) [ור״ת] ז״ל היה אומר שאין שני מקראות להוצאה דעני ועשיר, וכן יש שפירשו כדבריו, שאילו כן בתרתי חשבינן להו והוו להו מ׳ מלאכות שלמות. ומכל מקום לא תקשי לן מתניתין דפלגינהו בתרתי דלא דמי לההיא דשבועות (יד, ב) דאמרינן שם טומאה אחת היא, דהתם ודאי אחת היא לפי שאין אתה צריך להודיעו אלא שהוא טמא וכל שהוא טמא אסור ליגע במקדש וקדשיו, אבל הכא דהוצאות משתנות זו מזו, דהוצאת העני חשובה שמושך לעצמו אינה דומה להוצאת בעל הבית שמוציא ממנו, לפיכך מנאום חכמים בשתים. ור״ת ז״ל הביא ראיה לדבריו מדגרסינן בירושלמי בפירקין (ה״א): אמר רבי יוסי עני ועשיר אחד הם ומנו אותם חכמים שנים, ומשמע דהכי פירושו: אחד הם דהוצאת שניהם ממקרא אחד נפקא ואפילו הכי מנו אותן במתניתין בשנים לפי שהן הוצאות משתנות. ואם תאמר לדבריהם אל יצא איש ממקומו למאי אתא לרבנן דאמרי שאין תחומין דאורייתא, תירץ הרמב״ן ז״ל דדילמא לאזהרת יוצא המן הוא דאתא.}
\textblock{ [גמרא:] \textbf{וכי תימא מהן לחיוב ומהן לפטור. } כלומר: והכי קתני, אבות מלאכות שתים לחיוב, שהן ארבע עם היציאות של פטור. דאף על גב דלא קתני אפילו תולדות של חיוב אפילו הכי איירי בהוצאות של פטור, משום דכיון דאיירי בהוצאות של חיוב איירי נמי בהוצאות של פטור הואיל ויש בהן איסור של דבריהם.}
\textblock{      \textbf{והא יציאות קתני. } איכא למידק למאי דסבירא ליה דהכנסות לא קרינן להו יציאות, היכי ניחא מתניתין דהכא ושמונה יציאות היכי משכח להו. וי״ל דאין הכי נמי דקשיא ליה מתניתין אלא איידי דאיירי במשנה דשבועות פריך עלה. ורבנו יצחק בר׳ אשר ז״ל תירץ (בתוד״ה והא) דאמתניתין דשבועות דתני כולהו לחיובא בהא הוא דקא קשיא ליה, אבל מתניתין דהכא משכחת לה שנים בלחוד לחיובא דהיינו הוצאה דעני והוצאה דבעל הבית שיש בהן עקירה והנחה, ואינך כולהו לפטורא בין פטור ומותר ופטור אבל אסור. ולדבריו היינו דאמר רבינא מתניתין נמי דייקא דקתני יציאות וקא מפרש הכנסות לאלתר, כלומר: פשט העני את ידו לפנים ונתן לתוך ידו של בעל הבית, דאלמא מפשטא דמתניתין דקתני שתים שהן ארבע לא מצו למידק מידי כדאמרן.}
\textblock{\textbf{אמר רב אשי תנא להכנסה נמי הוצאה קרי לה, ממאי מדתנן המוציא מרשות לרשות חייב. } פירוש: מדקתני מרשות לרשות ולא קתני מרשות היחיד לרשות הרבים, וכדאיתא בהדיא בפרק קמא דשבועות (ה, ב) דפרכינן עלה ואימא דקא אפיק מרשות היחיד לרשות הרבים, ומשני אם כן ליתני המוציא מרשות היחיד לרשות הרבים מאי מרשות לרשות דאפילו מרשות הרבים לרשות היחיד. והאי דקאמר}
\textblock{\textbf{מי לא עסקינן,} מדיוקא דלישנא קא דייק לה.\par \textbf{} ואיכא למידק דאם איתא דאף מוציא מרשות הרבים לרשות היחיד קאמר, אמאי תני לה במתניתין, והא לא תני התם אלא אבות אבל תולדות לא תני התם כלל, והכנסה תולדה דהוצאה הוא כדאמרינן הכא אבות מאי ניהו יציאות, ובפרק הזורק (לקמן שבת צו, ב) נמי אמרינן הוצאה אב הכנסה תולדה. ותירץ ר״ת ז״ל לפי שהוצאה מלאכה גרועה היא והוה אמינא שאין לה תולדה לפיכך הוצרך לכוללה עמה.\par \textbf{} והרמב״ן ז״ל כתב דרב פפא לית ליה דהכנסה תולדה, אלא הכנסה כיציאה גמורה ושניהם אב אחד ומלאכה אחת, ולפיכך מני לה התם ולא מני לה אלא בחדא. ואם תאמר הכנסה דמיקריא אב מנא ליה. יש לומר דנפקא ליה מההיא דאמרינן בפרק במה טומנין (לקמן שבת מט, ב) דתניא התם הם העלו הקרשים מקרקע לעגלה ואתם אל תכניסו מרשות הרבים לרשות היחיד.\par \textbf{} ואי קשיא לך אם כן למאי איצטרכינן למילף הוצאה מויכלא העם בפרק הזורק (שם), תיפוק ליה ממשכן כדקתני התם הם הורידו אתם אל תוציאו. יש לומר דלא איצטריך ויכלא אלא לגלויי דהכנסה והוצאה שבמשכן מכלל המלאכות הן וגמרינן לה ממשכן. ואי קשיא לך הא דאמרינן התם בפרק הזורק (שם) ולרבי אליעזר דמחייב אתולדה במקום אב אמאי קרי ליה אב ואמאי קרי לה תולדה הך דהואי במשכן חשיבא הך דלא הואי במשכן לא חשיבא וכו׳, דאלמא הכנסה לא הואי במשכן. איכא למימר דאשאר אבות מלאכות ותולדותיהן קאמר ולר״א קאמר אבל הכנסה איתא במשכן, ודאמרינן התם מיהו הוצאה אב הכנסה תולדה היא, ההוא לישנא דלא כרב פפא. ומיהו מיהא הות במשכן כדאמרינן בפרק במה טומנין (שם) ואפילו הכי קרי לה תולדה כדאמרינן, כדקרי שובט ומדקדק תולדות כדאיתא בפרק כלל גדול, ואף על גב דהוי במשכן לא חשיבנא להו באפי נפשייהו כדאמרינן התם שובט בכלל מיסך מדקדק הוי בכלל אורג.\par \textbf{} ואם תאמר ומאי אולמא דהוצאה מהכנסה דקרי לה אב ולהכנסה קרי תולדה כיון דתרווייהו הוי במשכן. יש לומר משום דויכלא העם מהביא טפי משמע הוצאה, ומיהו לרב פפא כיון דתרווייהו הוי במשכן ולא חשיבא חדא מחברתה שנאן שתיהן בכלל המוציא מרשות לרשות דמלאכה אחת לגמרי הן. ורבא דקא מתרץ רשויות קתני סבר דהכנסות תולדות ולא נשנו בכלל אבות מלאכות, זו היא שיטתו של רבנו ז״ל.\par \textbf{} ולפי מה שכתבו בתוס׳ אפשר דההיא דפרק הזורק כוותיה דרב פפא נמי אתיא, שהם ז״ל כתבו (בעמוד א ד״ה פשט) דהכנסה דהתם היינו הכנסה דבעל הבית דליתא במשכן, דאילו הכנסה דעני למה לן למילפא התם מסברא דמה לי אפוקי מה לי עיולי בלאו הכי נמי תיתי לן מדהוות במשכן וכדאמרינן בפרק במה טומנין (לקמן שבת מט, ב), אלא ודאי לא נפקא לן מסברא דמה לי אפוקי מה לי עיולי אלא הכנסה דבעל הבית, דהם העלו קרשים מקרקע לעגלה היינו הכנסה דעני העומד בחוץ.}
\textblock{ הא דאמר רבא: \textbf{רשויות קתני. } פירש רש״י ז״ל דלרבא הכי קתני בין במתניתין דהכא בין במתניתין דהתם, רשויות שבת שתים רשות הרבים ורשות היחיד, ועל ידיהן יש לנו ד׳ איסורין בפנים וכנגדן בחוץ. והקשו בתוס׳ דאם כן הוי ליה למיתני שתים שהן ארבע בפנים וארבע בחוץ, שתים שתים למה לי. ועוד הקשה הרמב״ן ז״ל דאם כן לא הוי דומיא דמראות נגעים דהתם השנים מכלל הארבעה. ופירש רבנו יצחר בר׳ אשר (בתוס׳ שם) רשויות שבת יש בהם שתים איסורים שהם ארבע בפנים ושתים שהן ארבע בחוץ. ולמאי דקאמר רבא הוה יציאות כמו תוצאות כדכתיב (במדבר לד, ה) והיו תוצאותיו הימה.\par \textbf{} וכתב בספר המאור ודוקיא דרבינא דדייק ממתניתין דקא מפרש הכנסות לאלתר לא איתחזי ליה לרבא, משום דהכין סידורא דמתניתין וכו׳ כמו שכתוב שם. ולי נראה דמעיקרא כי קא סלקא דעתין דיציאות ממש קתני הוה דייק מינה רבינא מדקא מפרש הכנסה לאלתר, אבל השתא דמפרש רבא דלאו הוצאות קאמר אלא רשויות ליכא למידק מינה מידי.}
\clearpage
\newsection{דף ג}
\textblock{ הא דאמרינן:\textbf{ בבא דרישא דפטור ומותר לא קשיא לי.} קשיא לרבותינו הצרפתים ז״ל והא איכא משום ולפני עור לא תתן מכשול. וכי תימא כיון דבלא הושטה יכול הוא להניח בקרקע לית ביה משום ולפני עור כדאיתא בפרק קמא דע״ז (ו, ב) דאוקימנא ההיא דלא יושיט כוס יין לנזיר דקאי בתרי עברי נהרא, מכל מקום איסורא דרבנן איכא שהוא חייב להפרישו מאיסור. ואוקמוה בעכו״ם. וזה דוחק גדול. ועוד דהעני חייב קאמר ובגוי מי איכא למימר הכי. ומסתברא דפטור ומותר דקאמרינן הכא היינו שאין בו משום נדנוד עבירה מחמת הוצאת עצמו והכנסת עצמו, דאילו בכולהו פטורי דסיפא איכא נדנוד עבירה משום דנגמרה המלאכה על ידו וכאילו הוא עושה המלאכה. תדע דאיצטריך קרא לאשמועינן שהוא פטור מבעשותה או מנפש אחת (ויקרא ד, כז), וא״נ בעקירות דאיכא נדנוד עבירה כיון שעקר מרה״ר והכניס לפנים דאיכא למיחש שמא יניח או שעקר ונתן לתוך ידו של חברו דאיכא למיחש שמא יגמור כיון שהתחיל, אבל פטורי דבבא דרישא דלית להו שום נדנוד עבירה מצד עצמן לא חשיב במתניתין, משום דלא איירי התם אלא באיסורין הבאין להן משום הוצאות והכנסות של עצמן. כך נראה לי.}
\textblock{\textbf{אמר שמואל כל פטורי דשבת פטור אבל אסור בר מהני תלת וכו׳.} לא אמר אלא היכא דאמרו פטור מחמת שאינה מלאכה גמורה דומיא דמתני׳ דהכא, אבל פטורי אחריני איכא במסכת שבת שהוא פטור ומותר, דתנן בפרק במה מדליקין (לקמן שבת כט, ב) בשביל החולה שישן פטור ואוקימנא (שם ל, א) בחולה שיש בו סכנה, ופטור ומותר הוא והזריז הרי זה משובח (יומא פד, ב). והקשו בתוס׳ דהא איכא דתניא לקמן בפרק במה אשה יוצאה (שבת סב, א) רבי אליעזר פוטר בכובלת ובצלוחית של פלייטון, וזה הוא ודאי פטור ומותר הוא כדתניא אידך (שם) יוצאת אשה בכובלת לכתחילה. והם תירצו דהא דנקט התם פטור משום דאדרבי מאיר קאי דאמר (שם) חייבת חטאת ואמר ליה איהו פטור והתם פטור מקרבן קאמר, אלא דבאידך ברייתא גלי לן דמותר לכתחילה. ויש מפרשים דשמא ר׳ אליעזר לא התיר בו לכתחילה אלא בצלוחית של פלייטון, מדשבקה לבר זוגה בברייתא ולא תני אלא בצלוחית, והלכך תני התם פטור משום כובלת דפטור אבל אסור. ואיכא מאן דמפרש (בתוד״ה בר) דלא אמר שמואל אלא בפטור דתניא במתניתין בלבד.}
\textblock{ הא ד\textbf{אמר שמואל דצד נחש ומפיס מורסא פטור ומותר.} כתב ר״ת ז״ל (בתוס׳ ד״ה הצד): לאו אליבא דנפשיה קאמר, דהא אוקימנא לה בפרק שמונה שרצים (לקמן שבת קז, ב) כרבי שמעון דאמר מלאכה שאין צריכה לגופה פטור עליה, ואיהו במלאכה שאין צריכה לגופה כר׳ יהודה סבירא ליה דאמר חייב עליה כדאיתא בפרק כירה (לקמן שבת מב, א) ובזבחים בשלהי פרק כל התדיר (זבחים צא, א). אלא הכא אליבא דמאן דאמר פטור קאמר, ואנן קיימא לן כרבי שמעון דפטור.}
\textblock{\textbf{פטורי דאתי בהו לידי חיוב חטאת קא חשיב.} פירש רש״י: דהיינו עקירות דאפשר דאתי בהו לידי חיוב חטאת אבל הנחות דהיינו פשט העני את ידו לפנים ונטל בעל הבית מתוכה ופשט העני ידו ריקם ונתן בעל הבית בתוכה והוציא לחוץ והניח, וכנגדן בבעל הבית, אי אפשר דאתי בה[ו] לידי חיוב חטאת, הלכך לא חשיב להו. וא״ת עקירות דעני ששתיהן נעשות בחוץ אמאי חשיב להו בתרתי, וכן נמי של בעל הבית. ויש לומר כיון שהן משתנות זו מזו שהאחד עקר והכניס ידו מליאה והשני עקר ולא הכניס כלל אלא שהטעין ידו של חבירו ברשות שהוא עומד בה חשיב להו בתרתי.\par \textbf{} ויש מפרשים איפכא, שהנחות קא חשיב משום שעל ידה נגמרה מלאכה והיא עיקר המלאכה שע״י הנחה אתי בהו בעלמא לידי חיוב חטאת, אבל עקירות לאו מידי קא עביד זה. וזה נראה עיקר. והיינו חידושיה דאשמעינן שהוא פטור אף על גב דאיתעביד מלאכה, ועלה הוא דאתמהינן והא אתעבידא מלאכה ואיצטרכינן למיתלי בטעמא, משום דכתיב (ויקרא ד, כז) בעשותה או נפש אחת דשנים שעשאוה פטורין. ורבנו חננאל פירש כפירוש של רש״י ז״ל.\par \textbf{} והנכון מה שפירש הרב בעל המאור ז״ל דלא קא חשיב אלא הכנסות והוצאות שמרשות לרשות, כגון פשט בעל הבית ידו לחוץ ונטל העני מתוכה או שפשט ידו ריקה ונתן העני לתוכה והכניסה בפנים, וכן פשט העני ידו מלאה בפנים ונטל בעל הבית מתוכה או שנתן לתוכה והוציאה העני בחוץ.}
\textblock{      הכי גרסינן:\textbf{ איתמר נמי א״ר חייא בר גמדא נזרקה מפי חבורה ואמרו יחיד שעשאה חייב שנים שעשאוה פטורין.} ולא גרסינן ואמרו בעשותה יחיד שעשאה כו׳, דאם איתא למאי איצטריך לאתויי מימרא סייעתא לברייתא. אלא הכי קאמר לא תימא מתניתין רבי דוקא היא דדריש בעשותה למיעוטא אבל רבנן פליגי עליה משום דאיכא דלא דריש ליה למיעוטא, אבל נזרקה מפי חבורה כולהו רבנן מודו בה ולא משום בעשותה אלא מנפש או מאחת, כדנפקא לן בפרק המצניע (לקמן שבת צג, א) זה עוקר וזה מניח מנפש או מאחת.}
\textblock{ הכי גריס ר״ח ז״ל:\textbf{ ידו לא נייח גופו נייח ידו בתר גופו גרירא.} ולא בעי׳ עקירה גופו הוי עקירה. ורש״י ז״ל כתב דלא גרסינן ידו בתר גופו גרירא דיתירתא היא, אלא הכי גרס: ידו לא נייח גופו נייח. פירוש: ידו לא נייח על גבי קרקע גופו נייח על גבי קרקע והלכך הוי עקירה. ואם תאמר אם כן נתן לתוך ידו של בעל הבית או שנטל מתוכה אמאי חייב, והא בעינן עקירה והנחה על גבי מקום נח. יש לומר דלא אמרו ידו לא נייחא אלא כשהיא עומדת ברשות אחרת על גבי קרקע והטעינו חברו בידו והוציא שיהא חייב, דלעולם אין לה הנחה בפני עצמה אלא ברשות שהגוף עומד שם שהיא מונחת אגב גופו, אלא לפי שהיא אין דרכה להיות מונחת על גבי קרקע קאמר [ד]הגוף הוא שעומד על גבי קרקע ולעולם היד נגררת אחריו למאן דאמר בפרק המצניע (לקמן שבת צב, א) דאגד יד שמיה אגד, וכיון דאגדו בידו הוא אף על פי שהניח ידו בקרקע אין זו הנחה, וכדאמרינן לקמן (שבת ה, ב) גבי היה קורא על האסקופה שאף על פי שנח על גבי קרקע לא חשבינן ליה כמונח משום דאגדו בידו הוא, ולענין יד נמי הוא הדין והוא הטעם למ״ד התם אגד יד שמיה אגד, אלא מיהו לרבא דאמר התם בפרק המצניע דאגד יד לא שמיה אגד חייב כל שהיא סמוכה לקרקע פחות משלשה כדאיתא התם, אבל ברשות שהגוף שם ידו מונחת היא לכולי עלמא והנותן לתוכה או שנטל מתוכה חייב, וכן אם הטעינו חבירו אפילו בידו ויצא הוא והניח ברשות הרבים חייב כאילו הטעינו על כתפו, והיינו דכתב הוא ז״ל דידו בתר גופו גרירא יתירתא הוא, דהיינו ידו לא נייח כלומר, משום דבתר גופו גרירא.\par \textbf{} ויש ספרים דגרסי: אי נמי ידו בתר גופו גרירא היא. ומסתברא דאפילו לאותן ספרים לאו תרי טעמי נינהו אלא דא ודא לחד טעמא סלקן, אלא דהוי חד טעמא בתרתי לישני, כלומר: ואיכא מאן דאמר לה בהאי לישנא, וכענין שאמרו לקמן (שבת ה, ב) אמוראי נינהו ואליבא דרבי יוחנן מר אמר לה בהאי לישנא ומר אמר לה בהאי לישנא. ומדברי רבותינו בעלי התוס׳ ז״ל נראה דשני ענינין הן, וכן פירש רבנו הרב ז״ל, ואיכא בינייהו כגון שהטעינו חברו בידו והוציא ידו לחוץ וזרק, למאן דאמר ידו לא נייח לא מחייב דהא לא עקר, ולמאן דאמר ידו בתר גופו גרירא הויא עקירה. ואינו מחוור בעיני, דאם כן למאן דאמר ידו לא נייח תקשי ליה מתניתין דקתני פשט העני ידו לפנים ונתן לתוך ידו של בעל הבית או שנטל מתוכה והוציא העני חייב, ואמאי הא בעינא עקירה והנחה על גבי מקום מונח והא ליכא, דכיון דידו לא חשיבא כמונחת אפילו ברשות שהגוף עומד שם מה לי לענין לתת לתוכה וליטול ממנה מה לי אם הוציאה לחוץ וזרק, אי כמונחת היא כאן וכאן ליחייב אי לאו כמונחת היא כאן וכאן ליפטר, אלא ודאי נראה כדפרישית דלישני בעלמא נינהו.}
\textblock{\textbf{היה טעון אוכלין ומשקין מבעוד יום והוציאן לחוץ משחשיכה חייב.} הוא הדין להטעין עצמו משחשיכה לפנות מזוית לזוית ועמד לפוש ונמלך להוציאן, אלא דאי נקט הכי הוה אמינא דלא עקירת גופו הויא עקירה ולא הנחת גופו הויא הנחה ואינו חייב אלא אעקירה ראשונה, קא משמע לן בהיה טעון מבעוד יום ועמד לפוש משחשיכה דעקירת גופו הויא עקירה והנחת גופו הויא הנחה.}
\textblock{ הכי גריס רש״י ז״ל:\textbf{ אמר אביי ידו של אדם אינה לא כרשות היחיד וכו׳ כרשות היחיד לא דמיא מידו דבעל הבית כרשות הרבים לא דמיא מידו של עני.} ופירש דאינה כרשות שהגוף עומד שם לבטל רשות שהיא פשוטה שם, בין שהיא פשוטה לרשות הרבים והוא ברשות היחיד בין שהיא פשוטה לרשות היחיד והוא עומד ברשות הרבים, כרשות הרבים לא דמיא מידו דעני שפשוטה לרשות היחיד ואפילו הכי נטל בעל הבית מתוכה או שנתן לתוכה פטור, ואי חשבת ליה כרשות הרבים שהגוף עומד שם היה חייב, וכן לפשט בעל הבית את ידו לחוץ. ואינו מחוור. חדא, דפשיטא שאין היד הפשוטה לרשות היחיד חולקת רשות לעצמה שתעשה רשות הרבים. ועוד, שהבעיא דאיבעיא ליה אם תעשה כרמלית אינה מענין הפשיטות, דידו אינה נגררת אחר הגוף לחייב הנותן בתוכה והנוטל ממנה אבל לענין חזרתה אצלו אינה כמוחלקת ממנו, ואין הבעיא אלא אם היא כמוחלקת ממנו להחזירה אצלו.}
\textblock{ ורבנו חננאל ז״ל גרס איפכא:\textbf{ כרשות הרבים לא דמיא מידו דבעל הבית כרשות היחיד לא דמיא מידו דעני.} כלומר: פשיטא לי דידו בתר גופו גרירא כדאמרינן, בין הפשוטה לרשות הרבים בין הפשוטה לרשות היחיד ולעולם אינה בטלה לגבי הרשות שפשוטה שם, אלא הא קא מיבעיא לי אם הוציאוה מלאה פירות אם היא כרשות לעצמה ותהא ככרמלית או נגררת לעולם בתר גופה ומותר להחזירה אצלו.\par \textbf{} וקשיא לי לפשוט ממתניתין, דקתני פשט בעל הבית את ידו לחוץ ונטל העני מתוכה או שנתן לתוכה בעל הבית פטור, ופטור אבל אסור קאמר דפטורי דאית בהו מעשה נינהו כדאמרינן לעיל, אלמא ידו של בעל הבית שפשוטה לרשות הרבים ככרמלית היא ואסור להכניסה אצלו. וליתא דהתם משום דאתעבידא מלאכה היא דזה עוקר וזה מניח ולא משום שתעשה כרמלית. ותדע לך דהא פשט העני את ידו לפנים בודאי אינה נעשית כרמלית ברשות היחיד, דרשות היחיד עולה עד לרקיע ואפילו הכי אסור, אבל הכא שהוציאה מלאה פירות ואפילו מבעוד יום קא מיבעיא ליה אם נעשית כרמלית ברשות הרבים. ודוקא להחזירה אצלו הא לטלטל ברשות הרבים מותר, כדתנן בפרק בתרא דעירובין (צח, ב) עומד אדם ברשות היחיד ומטלטל ברשות הרבים ברשות הרבים ומטלטל ברשות היחיד.}
\textblock{\textbf{מי קנסוה רבנן לאהדורי לגביה או לא.} כתוב בתוס׳ דלא גרסינן קנסוה, דהא בסמוך דמוקמינן דלאו ככרמלית דמיא, אפילו הכי אמרינן דאם הוציאה משחשיכה קנסוה להחזירה. אלא הכי גריס מי אסרו להחזירה, דאף על גב דשרי לטלטולי ברשות הרבים כדתנן התם, מכל מקום עבדוה ככרמלית לענין רשות שאינה היא שם שדומין יותר שתי רשויות, ונפקא [מינה] דאם הוציאה בשוגג מבעוד יום אסור.}
\textblock{\textbf{כאן למטה מעשרה כאן למעלה מעשרה.} איכא למידק למעלה מעשרה אפילו לכתחילה נמי יהא מותר להוציאה ולהחזירה, דמקום פטור הוא. ויש מי שתירץ, דלאו בשהוציאה למעלה מעשרה קאמר, אלא שמותר להחזירה למעלה מעשרה, כלומר: הוציאה למטה מעשרה מותר להגביה למעלה מעשרה ולהחזיר משם לפנים, משום דלמעלה מעשרה מקום פטור הוא ומותר להוציא מכרמלית למקום פטור וממקום פטור לרשות היחיד. ואף על גב דאמרינן לקמן (שבת ו, א) גבי אסקופה ובלבד שלא יטול מבעל הבית ויתן לעני, ואמרינן נמי לקמן (שבת ח, ב) ובלבד שלא יחליפו וכן נמי בעירובין (ט, א), הני מילי ברשויות דאורייתא אבל ברשויות דרבנן מותר להחליף. ויש מי שפירש דבכרמלית של יד הקילו לפי שאינה כרמלית גמורה.\par \textbf{} ואינו מחוור בעיני מדאקשינן בסמוך משחשיכה דאי שדי להו אתי בהו לידי תקלת חיוב חטאת לא ליקנסיה, ואם איתא דאפשר לאהדורה בכי האי גוונא מהדר ולא אתי לידי חיוב חטאת, ודוחק הוא להעמידה באויר שאי אפשר להגביה. ועוד דאם איתא הוו מילי דרבנן כחוכא לא ליהדר מלמטה מעשרה אלא מגביה ליה לידיה לעיל ומהדרה.\par \textbf{} ולי נראה דאפילו להוציא למעלה מעשרה אסור לכתחילה, דחיישינן דילמא שדי להו, וכדאמרינן בסמוך אידי ואידי למטה מעשרה ולאו ככרמלית דמיא ולא קשיא כאן מבעוד יום כאן משחשיכה, דאלמא דאף על גב דלאו ככרמלית דמיא אפילו הכי קנסוה כשהוציאה משחשיכה או משום דשדי להו או משום דילמא אתי לאפוקי לכתחילה מרשות היחיד לרשות הרבים, והכא נמי למעלה מעשרה אף על גב דלא קנסוה ליה מ״מ אסור לעשות כן משום דילמא משתלי ושדי, וכל דאי שרית למעלה מעשרה אתי לאפוקי למטה מעשרה ולאו גזירה לגזירה היא דאי לא הא לא קיימא הא. ורבנו הרב כתב דלכתחילה אסור משום דדמיא למוציא למעלה מעשרה דאסור, כדאמרינן בפרק הזורק (לקמן שבת צז, א) א״ר (אלעאי) [אלעזר] המוציא משוי למעלה מעשרה טפחים חייב.}
\textblock{ הא דאמרינן:\textbf{ אדרבא איפכא מסתברא מבעוד יום דאי שדי ליה לא אתי לידי חיוב חטאת ליקנסיה, משחשיכה דאי שדי ליה אתי לידי חיוב חטאת לא ליקנסיה וכו׳.} איכא למידק היכי דמי אי בשוגג ולא אידכר למאן אסרו ולמאן התירו להחזירה, ואי בשוגג ואידכר חטאת מי מיחייב וכדאקשינן בסמוך (ד, א) אדרב ביבי. ותו מאי קא מדמי ליה לדרב ביבי, דהא אסיקנא בדרב ביבי דלאו שוגג קאמר ומשום חיוב חטאת אלא במזיד. ואי אפשר לומר דהכא לישנא דמעיקרא נקט ומיניה אתה שומע למזיד דמסקנא, דשאני מזיד דהכא משום דמזהר זהיר ביה כי היכי דלא ליתי לידי חיוב סקילה מה שאין כן בדרב ביבי דממילא קא אתי, וכדמוכח בסמוך דאמרינן איבעית אימא לא תפשוט ולא קשיא כאן בשוגג כאן במזיד, כלומר: וממזיד דהכא לא תפשוט דרב ביבי דהכא שאני דמזהר זהיר ביה וכדפירש רש״י ז״ל. ותדע לך דהא במסקנא אמרינן דבמזיד דהכא      קנסוה רבנן, ובעיא דרב ביבי איפשיטא דהתירו. ויש לומר דהכא ודאי בשוגג ואידכר, וחיוב חטאת דקאמר לאו חיוב ממש שיביא עליו קרבן חטאת קאמר אלא חלול שבת שיש בו חיוב קאמר.\par \textbf{} ורב ביבי בהכי נמי מתוקמא למסקנא, כלומר: בין במזיד גמור בין בשוגג ואידכר כדהכא, דלאו חיוב סקילה ממש קאמר, דהא אי אפשר דאפילו במזיד גמור מעיקרו כיון דעכשיו הוא בא לרדות ואינו נמנע אלא מחמת שסבור שהוא אסור לעשות כן אם כן סופו אינו אלא שוגג ואינו חייב סקילה, אלא קודם שיבא לידי איסור סקילה קאמר, כלומר: קודם שיתחלל שבת שהוא באיסור סקילה על ידו. והא דאקשינן בדרב ביבי אילימא בשוגג ואידכר מי מחייב, ודאי הוה מצי לתרוצי הכין, [אלא דתירץ איסור סקילה משום] דההוא לישנא כולל הכל ולא חיוב חטאת (שאני) [שאינו כולל], אבל הכא לא דק בלישנא (מכל מקום) [למימר] דאי שדי [ליה אתי לידי] איסור סקילה משום דלישנא דמעיקרא דרב ביבי נקט.\par \textbf{} ואם תאמר מכל מקום אכתי היכי אתי למיפשט דרב ביבי מיהא דשאני שוגג ואידכר, דהכא משום דמזהר זהיר ביה כדאמרינן בסמוך בלישנא דלא תפשוט כאן בשוגג כאן במזיד. איכא למימר ההיא דבסמוך במזיד גמור דאי שדי להו אתי לידי חיוב סקילה מזהר זהיר ביה, אבל בשוגג מעיקרו אף על גב דאידכר כיון דאינו מתחייב סקילה ואפילו כרת נמי לא מחייב לא זהיר ביה כיון דאית ליה טרחא יתירתא למינקט ידו הכין כוליה יומא. ואם תאמר מכל מקום אכתי לא דמיא לדרב ביבי, דהכא אפשר דאפילו בשוגג גמור ולא אידכר דלא שדי ליה, אבל בדרב ביבי דודאי אתי לידי כך ממילא לא ליקנסוה. איכא למימר דהכי קאמר מדקנסוה הכא שמע מינה דליכא למיחש כלל שמא יבא לידי איסור שבת ואפילו באיסורא דאתי ודאי ממילא, דאם איתא דחיישינן הכא לא ליקנסוה כיון דאפילו החזירה ליכא איסור כלל ואפילו דרבנן, אלא מדלא חיישינן הכא וקנסוה ביה ודאי בבעיא דרב ביבי דאיכא איסור מלאכה מדבריהם שנאסר במנין לא התירו ולא חיישינן לאיסור הבא לו ממילא.\par \textbf{} ובתוס׳ נראה שהם סבורים לומר דאפילו ממזיד גמור דהכא איכא למיפשט מזיד דרב ביבי, דהכא נמי אי אפשר לעכב שם ידו כל היום מלאה. והא דאמרינן בלישנא דאיבעית אימא לא תפשוט כאן בשוגג כאן במזיד. יש לומר דהכי קאמר לעולם מתניתא תרווייהו איכא לאוקמינהו בין בשוגג בין במזיד, ואפילו הכי לא תפשוט [ד]דילמא תרווייהו מבעוד יום, ואי בשוגג מוקמת להו מדתניא מותר להחזירה ליכא למיפשט בדרב ביבי דהתירו דהתם מבעוד יום ולא קנסוה אבל בדרב ביבי דמשחשיכה קנסוה, ומאידך דאסר ליכא למיפשט לחומרא, דשאני התם דלא אתי לידי איסור סקילה, אבל בדרב ביבי דאתי לידי איסור חששו, ואי במזיד מוקמת להו נמי ליכא למיפשט מידי מהאי טעמא דאמרן. והיינו דלא פשטיה לה נמי משנויא דקאמר דאף על גב דשנינן כאן למעלה מעשרה כאן למטה מעשרה ליכא למיפשט מינה דלא התירו דדילמא התם מבעוד יום.}
\textblock{\textbf{כאן לאותה חצר כאן לחצר אחרת.} ואף על גב דאסור להחזירה קאמר, להחזירה לרשות היחיד קאמר. ולחצר אחרת דקאמר בששתיהן מאדם אחד, אי נמי בשעירבו.}
\clearpage
\newsection{דף ד}
\textblock{ הא דאמרינן:\textbf{ כדבעא מיניה רבא מרב נחמן.} לפום מאי דפרישנא דמתניתא מבעוד יום לא דמי לגמרי לבעיא דרבא, דהכא בשהוציאה משחשיכה ואיכא למיקנס בה טפי דלא להדרה לחצר אחרת שלא תעשה מחשבתו, והכי קאמר, [דכשם] דמשחשיכה מפליג רב נחמן בין אותה חצר לחצר אחרת, דילמא מבעוד יום איכא לאיפלוגי נמי.\par \textbf{} מדשנינן בקנסו שוגג אטו מזיד קמיפלגי, והדר אמרינן איבעית אימא דכולי עלמא לא קנסינן שוגג אטו מזיד, אלמא להאי שנויא נמי בשוגג מוקמינן לה, ובשוגג הוא דשרינן ליה לאהדורה לאותה חצר אבל במזיד אפילו לאותה חצר אסור דקנסינן ליה, כדאמרינן לא קנסינן שוגג אטו מזיד אלמא במזיד קנסינן. ומדקנסינן ליה במזיד ולא שרינן ליה כדשרינן להדביק פת בתנור למסקנא, אלמא הא דידו כשהוציאה מבעוד יום דלא אתי לידי איסור שבת דאורייתא והלכך לא חיישינן לדילמא שדי ליה.\par \textbf{} והלכך פסקא דהאי שמעתא, הוציא ידו מלאה מבעוד יום בשוגג אסור להחזירה לחצר אחרת אבל לאותה חצר שרי, ואם הוציאה במזיד אפילו לאותה חצר אסור. ואם הוציא משחשיכה בין בשוגג בין במזיד לאותה חצר שרי דחיישינן דילמא שדי להו כדחיישינן בדרב ביבי, אבל לחצר אחרת לעולם אסור. אבל הרמב״ם ז״ל (פי״ג מהלכות שבת ה״כ) פסק דבמזיד אסור אפילו לאותה חצר.}
\textblock{\textbf{אילימא בשוגג ולא אידכר למאן התירו.} איכא למידק לוקמה דחזו אינשי ואמרי ליה רדה. ואיכא למימר דכיון דהשתא הוה סבירא לן דמשום חיוב חטאת ממש קא מיבעיא ליה, בכי הא נמי ליכא חיוב חטאת, דכי מודעי ליה ליכא חיוב חטאת דהיינו שוגג ואידכר. ואינו מחוור בעיני דמכל מקום אי לא מודעי ליה אתי לידי חיוב חטאת, ואנן כדי שלא      יבא לידי חיוב חטאת קאמרינן. ונראה בעיני דהתירו לו לא משמע אלא בבא לישאל.}
\textblock{ הא דאמרינן:\textbf{ וכי אומרים לו לאדם עמוד וחטוא כדי שיזכה חבירך.} קשיא ליה לר״ת ז״ל והא אמרינן בפרק בכל מערבין (עירובין לב, ב) ניחא ליה לחבר דלעביד איסורא קלילא ולא ליעביד עם הארץ איסורא רבה. ופריק שאני התם שעל ידו הוא נעשה, שאומר לו מלא לך כלכלה זו תאנים מתאנתי. ואם תאמר תפשוט מהתם הא דרב ביבי דהתירו, דהא התירו לו לחבר דליעביד איסורא זוטא כי היכי דלא ליעביד עם הארץ על ידיה איסורא רבה, וכל שכן דהתירו לו כי היכי דלא ליעביד איהו ממש איסורא רבה. יש לומר דשאני התם דלא נעשה האיסור עדיין ויכול הוא לתקן על ידי איסור קל, אבל הכא שכבר נעשה האיסור אלא שעתיד ליגמר ממילא דילמא לא התירו.\par \textbf{} וא״ת אכתי אשכחן שאומרים לו לאדם עמוד וחטוא כדי שיזכה חבירך, מדתנן בגיטין פרק השולח (גיטין מא, ב) מי שחציו עבד וחציו בן חורין שכופין את רבו ועושה אותו בן חורין כדי שיקיים מצות פריה ורביה, ואף על גב דאמר רב יהודה (ברכות מז, ב) המשחרר את עבדו עובר בעשה שנאמר (ויקרא כה, מו) בהם תעבודו. יש לומר דמצות פריה ורביה שהיא מצוה רבה שעל ידה מתקיים העולם שאני, וכדמשנינן התם גבי רבי אליעזר שנכנס לבית הכנסת ולא מצא שם עשרה ושחרר עבדו והשלימו לעשרה משום דמצוה דרבים שאני שהיא גדולה. ולי נראה דשאני התם כיון דחציו בן חורין לית ביה משום בהם תעבודו משום צד חירות שבו. וההיא אמתא (גיטין לח, א) דהוו עבדי בה אינשי איסורא, משום שהרבים היו נכשלין בה התירו לו לשחררה וכפוהו.}
\textblock{ הא דאמרינן:\textbf{ והא בעיא עקירה והנחה על גבי מקום ארבעה וליכא.} איכא למימר דמסברא בעלמא קאמר, לפי שאין דרכן של בני אדם להניח כליהם אלא במקום רחב שיכול להשתמר בו שלא יפול, אבל מקראי לית לן. ושמא נאמר דמפלוגתא דר״מ ורבנן דפליגי בזרק ונח בחור כל שהוא בפרק הזורק (לקמן שבת ק, א) יליף לה, דאפילו ר״מ לא מחייב התם אלא משום דכל שיכול לחוק כחקוק דמי, אלמא הנחה על גבי מקום ארבעה בעיא לכולי עלמא, והיינו דקא מקשה הכא להדיא. ואפשר דתנאי גמרא אגמרי לה. ובתוס׳ אמרו דאפשר דדייקי לה מדכתיב (שמות טז, כט) אל יצא איש ממקומו, ודרשינן מינה אל יוציא בעירובין (יז, ב), ומשמע נמי ממקומו ממקום החפץ, ואינו ראוי ליקרות מקום פחות מארבעה.}
\textblock{\textbf{למימרא דפשיטא ליה לרבה דבקלוטה כמי שהונחה דמי ובתוך עשרה פליגי והא מיבעיא בעי ליה לרבה וכו׳.} איכא למידק ממאי, דילמא אכתי לא איפשיטא ליה, אלא דבין ללישנא קמא בין ללישנא בתרא מתניתין לא נפקא מדר״ע דלכולהו לישני אית ליה קלוטה כמי שהונחה דמיא. איכא מ״ד דאי מספקא ליה לרבה, הוה ליה למימר דודאי למעלה מעשרה פליגי אבל למטה מעשרה דברי הכל קלוטה כמי שהונחה דמיא, כי היכי דתיקום מתניתין לכולי עלמא, מדקאמר רבי עקיבא היא אלמא ליכא לאוקומה לעולם כרבנן. ובתוס׳ תירצו דלרבה הוא הדין דאיסתפקא ליה מעיקרא אי אמרינן לכולי עלמא קלוטה לאו כמי שהונחה דמיא, ובשתי דיוטות זו כנגד זו כולי עלמא לא פליגי בין למטה מעשרה בין למעלה מעשרה דאינו חייב, דמשום קלוטה לא מחייב, ומשום דילפינן זורק ממושיט ליכא, דמושיט גופיה בכי הא פטור, אלא הכא בדיוטא אחת קמיפלגי בין למעלה מעשרה בין למטה מעשרה, דר״ע יליף זורק ממושיט ורבנן סברי לא ילפינן, דמאי שנא הא דלא מספקא, אלא ודאי ספוקי מספקא ליה, אלא דלא חש להאריך כולי האי, דמתוך הני בעיי איכא למשמע דאיכא לספוקי בפלוגתייהו כל מאי דאפשר לספוקי בה, וכיון שכן מנא ליה לרבה דר״ע היא, אלא משום דפשיטא ליה בעייא.}
\textblock{ הא דאמרינן:\textbf{ לכולי עלמא לא ילפינן זורק ממושיט.} הקשו בתוספות: נהי דלא ילפינן זורק ממושיט, ממעביר מיהא ליליף, דחייב מעביר מרשות היחיד לרשות היחיד דרך רשות הרבים אפילו למעלה מעשרה, כדאמרינן בפרק הזורק (לקמן שבת צז, א) בעי רב חסדא למטה מעשרה פליגי וכו׳, ופשטה ניהליה רב המנונא מרשות היחיד לרשות היחיד ועובר ברשות הרבים עצמה רבי עקיבא מחייב וחכמים פוטרים, מדקאמר ברשות הרבים עצמה מכלל דבלמטה מעשרה פליגי, ובמאי אילימא במעביר למטה מעשרה מחייב למעלה מעשרה לא מחייב והאמר ר׳ אלעזר המוציא משוי למעלה מעשרה חייב שכן משא בני קהת כו׳ אלמא במעביר כהאי גוונא חייב, וזורק ודאי ממוציא ילפינן ליה כדאמרינן בריש פרק הזורק (לקמן שבת צו, ב) מכדי זריקה תולדה דהוצאה היא הוצאה גופא היכא כתיבא, דהא דאמרינן התם כל ד׳ אמות ברשות הרבים גמרא גמירי להו, היינו בין בזורק בין      במעביר ברשות הרבים עצמה, אבל זורק מרשות היחיד לרשות הרבים ילפינן ממוציא, והכי נמי הוה לן למילף זורק מרשות היחיד לרשות היחיד דרך רשות הרבים ממעביר דמאי שנא. ותירצו דלא ילפינן ממוציא אלא זורק מרשות היחיד לרשות הרבים, דזריקה זו היתה במשכן, אבל מרשות היחיד לרשות היחיד דרך רשות הרבים לא היתה במשכן ולא ילפינן לה ממוציא, ואין לנו לחייב בתולדה דהוצאה אלא אותן שהיו במשכן. ואף על גב דבסמוך בעינן למימר דלר׳ עקיבא ילפינן זורק ממושיט אף על גב דלא הוה במשכן. איכא למימר דהוה סלקא דעתך דלרבי עקיבא טפי דמי זורק למושיט ממאי דדמי שאר תולדות להוצאה.}
\textblock{\textbf{אבל למטה מעשרה דברי הכל חייב מאי טעמא דאמרינן קלוטה כמי שהונחה דמיא.} קשיא אם כן זורק ד׳ אמות ברשות הרבים היכי משכחת לה. ויש מי שתירץ (תוס׳ ישנים כאן) דלא אמר רבי עקיבא אלא לחייב אבל לפטור לא קאמר. ואינו מחוור בעיני, דהא לענין גיטין אמרו כן נמי בפרק הזורק (גיטין עט, א), והתם מאי חיוב איכא ומאי פטור איכא. ויש מי שפירש דלא אמרו אלא בקלוטה ברשות אחרת אבל ברשות אחת לא אמרינן. ואיכא למימר דזורק ומעביר ארבע אמות ברשות הרבים הלכתא גמירי לה, כדאיתא בפרק הזורק (לקמן שבת צו, ב).}
\textblock{\textbf{זרק ונח על גבי זיז כל שהוא חייב.} קשיא לן מאי קא מתרץ, דהא אכתי תקשי לן דילמא הנחה הוא דלא בעינן הא עקירה בעינן. ויש מי שתירץ (בתוד״ה כך) דדוקא על דברי ר״ע דלא בעי הנחה כלל איכא למיפרך הכין דאין נראה לומר גבי עקירה כן, אבל לרבי דבעי הנחה על גבי משהו הוה סבירא ליה לרב יוסף דאף בעקירה כן. ומיהו אי לאו דאיכא לאקשויי עליה עדיפא מיניה הוה מקשה ליה הכין כדפרכינן לקמן (שבת ה, א) על ר׳ זירא דאמר הא מני אחרים היא.}
\textblock{ הא דפרכינן גבי הא דר׳ עקיבא:\textbf{ ודילמא הנחה הוא דלא בעיא הא עקירה בעינן.} פירש רש״י ז״ל: מדלא קאמר ר״ע מחייב שתים. ואינו מחוור בעיני, חדא, דמחייב כללא הוא ואיכא למימר מחייב שתים קאמר, וכדאמרינן בסמוך גבי הזורק מרשות הרבים לרשות הרבים ורשות היחיד באמצע רבי מחייב אמר רב יהודה אמר שמואל מחייב היה רבי שתים. ועוד, דאף לכשתמצא לומר דלא מחייב אלא אחת, דילמא לא משום עקירה אלא משום דלא מחייב אתולדה במקום אב, וכדאמרינן בפרק הזורק (לקמן שבת צז, ב) דלא מחייב אתולדה במקום אב. ועוד קשיא לי, דאי מדיוקא דלישניה קא דייק לא הוה ליה למימר בלשון דילמא. אלא סברא בעלמא קאמר, דדילמא בגמר מעשה כל דהו סגי ליה אבל בתחילת מעשה עקירה חשובה בעיא.}
\textblock{ הא דאמרינן:\textbf{ התם כדבעינן למימר לקמן כדאביי דאמר אביי הכא באילן העומד ברשות היחיד ונופו נוטה לרשות הרבים וכו׳.} פירש רש״י ז״ל בשזרק מתחילת ד׳ לסוף ד׳, דרבי סבר שדי נופו בתר עיקרו והוי כמונח ברה״ר על מקום ארבעה, דעיקר מחשב הנוף כשם שיש בעיקרו ארבעה כך אנו חושבין שיש בנופו ארבעה, דהכל הולך אחר העיקר. ואינו מחוור, דאם כן מאי דוחקיה דאביי דאוקומה בשתי רשויות העיקר ברשות היחיד ונופו ברשות הרבים, לימא באילן העומד ברשות הרבים רבי סבר שדי נופו בתר עיקרו ורבנן סברי לא אמרינן, אלא דעל כרחין לא מצי לאוקומה בכי הא, דאי נופו למטה משלשה בכי הא לא פטרי רבנן דכארעא סמיכתא היא בין איכא ארבעה בין ליכא ארבעה, ואי למעלה משלשה מקום פטור הוא ולא הוה מחייב רבי, והלכך בנופו נוטה לרשות הרבים נמי אי אפשר דמחייב רבי משום טעמא דחשבינן כאילו יש בנופו ארבעה ומשום זורק מתחילת ארבע לסוף ארבע כדפירש רש״י ז״ל מהאי טעמא דאמרן.\par       \textbf{} ובתוס׳ פירשוה משום זורק מרשות הרבים לרשות היחיד, דרבי סבר שדי נופו בתר עיקרו והוי ליה כזורק מרשות הרבים לרשות היחיד, וכיון דשדיא נופו בתר עיקרו להחשיבו כרשות שהוא עומד שדיא ליה נמי בתר עיקרו להחשיבו כארבעה כעיקרו. ובודאי טעמיה דאביי לאו משום דשדיא ליה בתר עיקרו למהוי כארבעה על ארבעה קאמר אלא למהוי כרשות היחיד שהעיקר עומד בו, דהא אביי לתרוצה אליבא דרב חסדא דאמר דברשות היחיד בהנחה כל דהו סגי כדאיתא לקמן (שבת ז, ב) קא אתי, והלכך אפילו אין בנופו ולא בעיקרו ארבעה חייב לרבי דאמר שדי נופו בתר עיקר, כלומר: למהוי כרשות שהעיקר עומד בו, אלא שהוצרך לפרש כן בכאן משום דמשמע להו דסוגיין דהכא אזלא דבעינן הנחה על גבי מקום ארבעה אפילו ברשות היחיד.\par \textbf{} והביאו ראיה מדפריך לקמן (שבת ה, א) והא ידו קתני תני טרסקל שבידו ובתר הכי פריך התינח טרסקל ברשות היחיד, דאלמא אף ברשות היחיד מהדרינן לה לאשכוחי מקום ארבעה, אלא דלא מייתי הא דאביי אלא לומר כשם שאמר אביי דשדיא נופו בתר מקום עיקרו, דאלמא מקום עיקרו הוא העיקר וכאילו נח בעיקרו, אף אנו נאמר שהוא כמונח בעיקרו שיש בו ארבעה.\par \textbf{} ובודאי לכאורה כולה שמעתין הכין מוכחא, ויש לי ללמד מדאמרינן אלא הא רבי דתניא זרק מרשות הרבים לרשות הרבים ורשות היחיד באמצע רבי מחייב ואמר רב יהודה אמר שמואל מחייב היה רבי שתים, ואם איתא מאי קא מייתי מינה ראיה דשאני התם דעברה דרך רשות היחיד דסגי לן בזה בעקירה והנחה כל שהוא. ועוד, למה לן לאתויי הא דרב יהודה אמר שמואל ההיא ברשות היחיד וברשות היחיד לא מיפלגי בין עקירה והנחה. ועוד, מאי קא מייתי ראיה מדתניא זרק ונח על גבי זיז כל שהוא דההוא על כרחין בזיז העומד ברשות היחיד היא והתם איכא למימר דבכל שהוא סגי. ועוד, דאמרינן תינח רשות היחיד מקורה רשות הרבים מקורה מי מחייב. ועוד, דאמרינן לקמן (שבת ה, א) גבי הכניס ידו לחצר חברו וקיבל מי גשמים והא בעיא עקירה מעל גבי מקום ארבעה וליכא, אלמא אף ברשות היחיד בעינן עקירה והנחה על גבי מקום ארבעה.\par \textbf{} אלא דקשיא לי דאם איתא דלא מייתי הכא הא דאביי אלא לומר דשדיא נופו בתר עיקרו כדאביי ולא כדאביי לגמרי, אם כן לא הוה ליה למימר סתמא כדאביי כדבעינן למימר קמן דאדרבה היפך דאביי היא, דאילו לאביי לא שדיא נופו בתר עיקרו לאחשוביה לנוף כארבעה דאם כן קשיא ליה לרב חסדא דלא בעי הנחה על גבי ארבעה ברשות היחיד והכא בעינן, אלא הוה ליה לפרושי בהדיא כאביי ולא כאביי, או לימא התם משום דשדיא נופו בתר עיקרו ולא לידכריה לאביי כלל. ועוד, דאם כן תקשי לן דהכא משמע דבעינן עקירה והנחה על גבי מקום ארבעה אפילו ברשות היחיד, ולקמן משמע דלכולי עלמא ברשות היחיד בכל דהו סגי לן כרב חסדא וקיימא לן כוותיה.\par \textbf{} ומשום הכי נראה לי דהכא נמי לא קשיא ליה אלא מפשט בעל הבית את ידו לחוץ ונתן לתוך ידו של עני או שנטל מתוכה, אבל מידו של בעל הבית לא קשיא ליה דברשות היחיד [בהנחה] כל דהו סגי.\par \textbf{} והא דאמרינן: אילימא הא רבי דתניא זרק ונח על גבי זיז כל שהוא חייב דילמא התם כדאביי, הכי קאמר אילימא הא רבי זרק ונח, ההיא על כרחך ברשות היחיד הוא וטעמא כדאביי [ו]ברשות היחיד לא מיבעיא לי. ואם תאמר אם כן למה ליה לאורוכי כולי האי ולמימר התם כדאביי, לימא ליה התם ברשות היחיד וברשות היחיד לא קא אמינא ותו לא. לא היא, דאילו אמר הכי אכתי תקשי לן מאי טעמייהו דרבנן דפטרי, משום הכי איצטריך להא כדאביי לתרוצי דרבי ורבנן ולומר דרבי מחייב היינו משום דשדי נופו בתר עיקרו וכאילו נח ברשות היחיד ולהכי סגי ליה בכל שהוא, ורבנן דפטרי משום דלא שדי נופו בתר עיקרו והוה ליה כזורק מתחילת ארבע לסוף ארבע (וזרק) ונח למעלה משלשה דפטור דמקום פטור הוא.\par \textbf{} והא דאמרינן: אלא הא רבי דתניא זרק מרשות הרבים לרשות הרבים ורשות היחיד באמצע, היינו משום דקלוטה גריעא טובא מהנחה על גבי משהו. ותדע לך דבקלוטה קיימא לן דלאו כמי שהונחה דמיא ואפילו הכי קיימא לן דברשות היחיד בהנחה כל דהו סגי, ומשום הכי קא מייתי ראיה מקלוטה דרשות היחיד להנחה כל דהו דרשות הרבים, לומר דאי אמרת בשלמא דברשות הרבים בהנחה כל דהו סגי היינו דברשות היחיד סגי לן בקלוטה משום דאלים כחו של רשות היחיד למהוי קלוטה דידיה כהנחת כל דהו דרשות הרבים, אלא אי אמרת דברשות הרבים הנחת ארבעה בעינן מי עדיף רשות היחיד כולי האי דנחשוב ביה קלוטה כמונחת.\par \textbf{} והא דאמרינן: תינח רשות היחיד מקורה תינח טרסקל ברשות היחיד, לאו למימרא דניבעי הנחה על גבי מקום ארבעה ברשות היחיד אלא רבותא בעלמא היא, כלומר: תינח רשות היחיד דאפילו תמצא לומר דהנחה על גבי ארבעה בעינן משכחת לה, ומשום דקתני ברישא דמתניתין פשט העני את ידו קא מדכר הכא רשות היחיד, ומיהו לא איצטרכינן לתרוצי ולא לאדכורי כלל אלא רשות הרבים בלבד דהיינו פשט בעל הבית את ידו לחוץ.\par \textbf{} והא דאמרינן: גבי הכניס ידו לחצר חברו (ולקט) [וקלט] מי גשמים והא בעינן עקירה מעל גבי מקום ארבעה לאו דוקא מקום ארבעה, אלא הכי קאמר והא בעינן עקירה מעל גבי מקום וליכא. ותדע לך דהתם לאו משום שיעור מקום אתי אלא משום דבעינן מקום לאפוקי קולט.\par \textbf{} ומיהו אכתי קשיא לי מהא דתנן בעירובין פרק המוצא תפילין (עירובין צח, ב) לא יעמוד אדם ברשות היחיד וישתין ברשות הרבים וכו׳ וכן לא ירוק, ואמרינן עלה בגמרא אמר רב יוסף השתין ורק חייב חטאת, ואקשינן והא בעינן עקירה מעל גבי מקום ארבעה, והא הכא דעומד ברשות היחיד ואפילו הכי בעינן עקירה מעל גבי מקום ארבעה. ויש לומר דאסיפא דמתניתין קאי דקתני ברשות הרבים וישתין ברשות היחיד      ועלה קא מקשה והא בעינן עקירה מעל גבי מקום ארבעה. אי נמי יש לומר דהנחה הוא דלא בעיא הא עקירה בעיא, ולעולם לכולי עלמא ברשות היחיד בהנחה כל דהו סגי כרב חסדא וכאביי דמתרץ ברייתא דרבי ורבנן אליבא דידיה. וכן פסקו רב אלפסי ורבנו חננאל ז״ל.\par \textbf{} והא דתניא בפרק הזורק (לקמן שבת ק, א): זרק למעלה מעשרה והלכה ונחה בחור כל שהוא ר׳ מאיר מחייב, כלומר: משום דחוקקין להשלים וחכמים פוטרין, דאלמא אי לאו דחוקקין להשלים אפילו ר׳ מאיר מודה דרשות היחיד בעי מקום ארבעה, נראה לי דטעמא דהתם לאו משום דבעינן הנחה ברשות היחיד על גבי ארבעה, אלא דאינו רשות היחיד אלא אם כן יש בחור ארבעה על ארבעה דחורי רשות הרבים הוא, אבל לר׳ מאיר דאמר חוקקין הרי זה כמונח על גבי מקום ארבעה. כך נראה לי.}
\textblock{\textbf{ואמר רב יהודה אמר שמואל מחייב היה רבי שתים.} איכא דקשיא ליה והא אנן אהא דאמר רב יוסף רבי היא אתינן לפרושי, ורב יוסף הא אמר בפרק הזורק (לקמן שבת צז, ב) דלא מחייב רבי אלא אחת דלא מחייב אתולדה במקום אב. ויש לומר דרב יוסף לא אמר אלא רבי היא סתם, וכיון דדחינן לה מהא דרבי דזיז, קא מהדר תלמודא לאוקמוה כאידך דרבי אליבא דרב ושמואל כי היכי דלא תקשי ליה ודילמא הנחה הוא דלא בעינן הא עקירה בעינן.}
\textblock{\textbf{ואמר רב יהודה אמר שמואל מחייב היה רבי שתים.} פירוש: וסבירא ליה להאי מתרץ השתא דהאי דנקטה רבי בזורק מרשות הרבים לרשות הרבים ורשות היחיד באמצע לאו דוקא, אלא הוא הדין זורק מרשות היחיד לרשות היחיד ורשות הרבים באמצע.}
\clearpage
\newsection{דף ה}
\textblock{ הא דאמרינן:\textbf{ אלא טרסקל ברשות הרבים רשות היחיד היא לימא דלא כרבי יוסי ברבי יהודה.} הקשו בתוס׳ מאי קשיא ליה מדרבי יוסי ברבי יהודה, והא מתניתין על כרחין בלמטה מעשרה מיירי, דאי למעלה מעשרה מקום פטור הוא, ודרבי יוסי ברבי יהודה פשיטא דלמעלה מעשרה היא, דאי למטה מעשרה לא הוי רשות היחיד, ומתניתין היא בפרק הזורק (לקמן שבת צט, א) דתנן חולית הבור והסלע בזמן שהן גבוהין עשרה ורחבן ארבעה הנוטל מהם והנותן על גבן חייב פחות מכן פטור, ותניא לקמן (שבת ו, א) גדר שהוא גבוה עשרה ורחב ארבעה זו היא רשות היחיד גמורה. והעלוה בגמגום.\par \textbf{} ואיכא מרבוותא דכתבו דמתניתין אפילו למעלה מעשרה היא, ואף על גב דלמעלה מעשרה ברשות הרבים מקום פטור הוא, כיון דקיימא לן (לקמן שבת צב, א) המוציא משוי למעלה מעשרה חייב וכן (לקמן שבת ח, ב) המעביר ברשות הרבים למעלה מעשרה חייב, הכי נמי אף על פי שידו של עני למעלה מעשרה הנותן לתוכו כאילו נותן ברשות הרבים. וסייעו להא משמעתין דהכא דאקשינן טרסקל ברשות הרבים רשות היחיד היא.\par \textbf{} ואינו נראה בעיני מכמה טעמים. חדא, דהמוציא למעלה או המעביר למעלה מעשרה דחייב על כרחין דוקא באדם אחד שעמד לפוש דעקירת גופו והנחת גופו כעקירת חפץ והנחת חפץ דמי ומשום דגמרינן לה ממשא בני קהת (לקמן שבת צב, א), אבל אם הניחה במקום (אחד) [אחר] דוקא כשהניחה למטה משלשה או שטחה בכותל למטה מי׳ בדבילה שמינה וכיוצא בה כדאמרינן לקמן (שבת ז, ב), אבל אם העבירה למעלה מעשרה והניחה למעלה מעשרה פטור. ועוד, דאם איתא דכי אקשינן הכא טרסקל ברשות הרבים רשות היחיד היא משום דמשמע ליה דידו דעני למעלה מעשרה היא, קשיא טובא והא עומד קתני, כלומר: כדרכו כדאקשינן בסמוך, ולא שיהא שוחה ולא ידיו פרושות לשמים, ואי משום דמתניתין סתמא קתני דמשמע דבכל ענין היא ואפילו בלמעלה מעשרה, מאי קא משני ליה אפילו תימא רבי יוסי ברבי יהודה התם למעלה מעשרה הכא למטה מעשרה, מכל מקום מתניתין סתמא תנן ואפילו בלמעלה מעשרה ולמעלה מעשרה רשות היחיד היא. ועוד, דגרסינן בירושלמי (פ״א, ה״א) העני חייב, אמר רב יהודה בשם שמואל והוא שתהיה ידו של עני בתוך עשרה לקרקע.\par \textbf{} ומסתברא לי דקושיא מעיקרא ליתא, והא דקא מקשינן טרסקל ברשות הרבים רשות היחיד היא לאו למימרא דמשמע לן מתניתין בלמעלה מעשרה, אלא הכי קאמר: מדמוקמת מתניתין בשקבלה העני בטרסקל שבידו ואפילו הכי בעל הבית חייב, אלמא סבירא ליה לתנא דטרסקל לא פליג רשות לנפשיה כעמוד או תל גבוה למעלה מג׳ דברשות הרבים חולקין רשות לעצמן להיות כרמלית או מקום פטור, דקא סבר תנא דכלים אין חולקין רשות לעצמן, אם כן פליג      אדרבי יוסי ברבי יהודה דאילו לרבי יוסי ברבי יהודה אפילו רשות היחיד הוי שהוא רשות גמור דאורייתא בזמן שהוא ברשות הרבים למעלה מעשרה וכל שכן שיחלקו רשות לעצמן ברשות דרבנן להיותן כרמלית למטה מעשרה ונתן בעל הבית לתוכו פטור. ומשני כי אמר רבי יוסי ברבי יהודה הני מילי למעלה מעשרה דאין שם רשות שיבטל הכלי לגמרי דרשות הרבים אינו עולה למעלה מעשרה, אבל הכא למטה מעשרה, כלומר: ולמטה מעשרה דהוי רשות הרבים אין כלי חולק רשות לעצמו להיותו ככרמלית אלא בטל הוא לגבי הרשות, ומעתה לא תקשי לך למטה מעשרה נמי ליהוי כרמלית דהיינו קושיין והיינו דפרקינן דלמטה מעשרה בטיל הוא לגבי הרשות, ולעולם נתן למעלה מעשרה לתוך ידו של עני או שנטל מתוכה פטור דמקום פטור הוא, ואף על גב דכשהיא מג׳ ועד י׳ חייב מה שאין כן בעמוד או בתל, הא לא קשיא דאדם אינו נעשה רשות אלא עומד ברשות וכל שידו תוך רשות הרבים דהיינו למטה מעשרה הרי זה כמונח בארץ, אבל למעלה מעשרה אף על פי שאינו נעשה רשות מכל מקום לא נח ברשות הרבים ולא גרע מזרק וטח בכותל למעלה מעשרה דהוי כזורק באויר. כך נראה לי.}
\textblock{\textbf{איבעית אימא בגומא.} כלומר: בגומא שהיא רשות הרבים דרחבה ארבעה ואינה עמוקה עשרה, וכדרב יוסף דאמר לקמן (שבת ח, א) תשמיש על ידי הדחק שמיה תשמיש, דאילו לרבא דאמר לא שמיה תשמיש וככרמלית הויא הכא אינו חייב (ד)אף על גב דידו מונחת ברשות הרבים, דהא ידו בתר גופו גרירא. ולי נראה דאפילו לרבא נמי אתיא דהא סבירא ליה לרבא אגד יד לא שמיה אגד, והלכך כל שידו ברשות הרבים למטה משלשה אף על פי שגופו עומד בכרמלית והניח בידו כמונח ברשות הרבים וכדאיתא בפרק המצניע (לקמן שבת צב, א).}
\textblock{\textbf{איכפל תנא לאשמועינן כל הני.} הכא משמע דלרבא לא קשיא ליה אלא דלא איכפל תנא לאשמועינן הני, הא דינא גופיה מודי ביה. וקשיא לי דהא רבא אגד יד לא שמיה אגד סבירא ליה בריש פרק המצניע (שם) ואוקי מתניתין דפשט העני את ידו לפנים ונטל בעל הבית מתוכה ופשט בעל הבית את ידו לחוץ ונטל העני מתוכה דוקא כשהיה למעלה משלשה הא למטה משלשה המוציא והמכניס חייב, ומסתמא כולה מתניתין בחד גוונא מיירי וכולהו בבי בלמעלה משלשה נינהו. ויש לי לומר דהכא לטעמיהו קאמר להו הא לדידיה בלמעלה משלשה היא דוקא כולה מתניתין. אי נמי הכא מקמיה דקם רבא בשיטתיה הוה, כדאיתא התם דרבא מעיקרא סבירא ליה אגד יד שמיה אגד אביי סבר אגד יד לא שמיה אגד, ולבסוף קם רבא בשיטתיה דאביי וקם אביי בשיטתיה דרבא, והכא מקמי דליקום בשיטתיה דאביי.}
\textblock{\textbf{ידו של אדם חשובה לו כד׳ על ד׳.} איכא למידק אשמעתין מאי קשיא ליה ומאי איצטריך רבא למימר דידו חשובה לו כד׳ על ד׳ בלאו הכי נמי מחשבתו משויא ליה מקום כדאמרינן בשלהי הזורק (לקמן שבת קב, א) גבי זרק לפי הכלב או לפי הכבשן חייב, ובעירובין פרק המוצא תפילין מכניסן זוג זוג (צט, א) נמי אמרינן השתין ורק חייב ופריך והא בעינן מקום ד׳ וליכא ומשני מחשבתו משויא ליה מקום. ותירץ ר״ת ז״ל (לעיל שבת ד, ב בתוד״ה אלא) דהתם הוא דלא ניחא ליה בענין אחר, דזרק לפי הכלב או לפי הכבשן כוונתו להאכיל הכלב או לשרוף העץ, וכן השתין ורק אי אפשר לנקות את עצמו אלא בכי האי גוונא ומשום כך מחשבתו משויא ליה מקום, אבל הכא לא איכפת ליה בין ידו למניח בכתפו. והא דאמרינן בסמוך הני מילי היכא דאחשבה לידיה, דמשמע דאפילו היכא דניחא ליה בענין אחר מחשבתו משויא ליה מקום. התם הוא לבתר דאמרינן דידו חשובה לו כד׳ על ד׳, והכי קאמר: הני מילי דידו חשובה לו כד׳ על ד׳ היכא דאחשבה לידיה אבל בשאר דברים לבד מידו לא הוה אמרינן הכי במאי דניחא ליה בענין אחר.\par \textbf{} ומצאתי לראב״ד ז״ל בעירובין פרק המוצא תפילין שהקשה בשם הראשונים ז״ל ידו היכי דמי, אי למטה מג׳ מאי איריא משום דהויא כד׳ על ד׳ הא כל פחות מג׳ לא בעיא הנחה אלא על גבי משהו, ואי למעלה מג׳ הויא ליה כרמלית, ואי למעלה מעשרה ליהוי רשות היחיד כעמוד. ותירץ הוא ז״ל שאין אדם חולק רשות לעצמו. ותדע לך שהרי למדנו      (לקמן שבת צב, א) ממשא בני קהת שהמוציא משא למעלה מעשרה חייב הרי שאין דנין באדם דין עמוד בלמעלה מעשרה, וכן אין דנין בו דין כרמלית ולא מקום פטור, לפי שהוא נע ונד. וזה ראיה למה שכתבתי אני למעלה (בסוד״ה הא דאמרינן אלא) וכבר זכיתי לומר כדבריו.\par \textbf{} עוד שאל, הא דבעינן הנחה על גבי מקום ד׳ ברשות הרבים היכי דמי, אי למטה משלשה לא צריך, אי למעלה משלשה כרמלית הוא ופטור. הא ודאי לא אשכחן לה אלא בזורק לרשות הרבים ונח על גבי בעלי חיים כמו שאמרנו, או בזורק לעמוד תשעה ורבים מכתפין עליו דהוא רשות הרבים, ואי הוי ד׳ מחייב עלה בהנחתו ואי לא לא. והתירוץ הזה השני שלא כדעת רש״י ז״ל (לקמן שבת ח, א) והרמב״ם ז״ל (פי״ד מהל׳ שבת, ה״ח), שהם ז״ל כתבו דעמוד תשעה ורבים מכתפין עליו היינו אפילו אינו רחב ד׳ וכמו שכתבתי למטה במקומה (להלן שבת ח, א ד״ה עמוד).}
\textblock{ הכי גריס ר״ח ז״ל:\textbf{ שתי כוחות באדם אחד כשני בני אדם דמי וחייב או כאדם אחד דמי ופטור.} ופירשו כגון שנעקר ממקומו וקדם ונח במקומו וקיבלה, וכשני בני אדם דמי שזה זורק וזה מקבל וחייב, או דילמא כאדם אחד דמי כאילו נתן מימינו לשמאלו דפטור ואע״פ שהעבירה ד׳ אמות. וכן פירשו בתוס׳. ואינו מחוור לי דכיון דא״ר יוחנן עמד במקומו וקיבל חייב נעקר ממקומו וקיבל פטור נעקר הוא ממקומו וקיבל מאי, ודאי משמע דהא [ד]נעקר הוא ממקומו הוי כעין נעקר חברו ממקומו וקיבל, כלומר: שקיבלו דרך עקירה שלא נח החפץ מכוחו של ראשון אלא מכוחו של שני, והכא נמי כשנעקר וקיבלו שלא נח מכח זריקה ראשונה. ועוד הקשה הרמב״ן ז״ל דהני לאו שני כוחות נינהו, דעקירה והנחה בכח אחד הן. ויש מי שפירש שקבלה בעקירה ממקומו, והכי קאמר כשני בני אדם זה זורק וזה מקבל דמי וחייב דאף זה עקירה והנחה תרווייהו מכוחו אתי, או דילמא כאדם אחד שעשה עקירה ולא הניחה במקום (אחד) [אחר] דמי, דהא באותו מקום שנעקר ממנו חזר ונח והלכך פטור. וגם זה אינו מחוור בעיני דכשני בני אדם דקאמרינן משמע כשני בני אדם דכוותיה, כלומר: זה זורק וזה מקבל דרך עקירה דפטור.\par \textbf{} ורש״י ז״ל גריס איפכא: כשני בני אדם דמי ופטור, כלומר: בזה זורק וזה מקבל דרך עקירה לפי שההנחה לא הויא מכח העקירה והויא ליה כשנים שעשאוה, או דילמא חייב דהא אין כאן שנים אלא אחד והעקירה וההנחה תרווייהו מכוחו אתו והוה ליה אחד שעשאה. וזה נכון.\par \textbf{}  ירושלמי (פ״א, ה״א):אמר ר׳ יודן פשיטא לר׳ יוחנן בשזרק בימין וקלט בשמאל שהוא חייב, מה צריכה ליה בשזרק בימין וקלט בימין, רבנן דקסרין ר׳ שמי בשם ר׳ אחא אפילו זרק בימין וקלט בשמאל צריכה ליה חייב, אין תימר פיו ופיו כיון שאכלה כאחר הוא ברם הכא ידו כאחר היא. ר׳ מונא בעי מעתה הוציא כגרוגרת בשתי ידיו יהא פטור משום שנים שעשו מלאכה אחת, א״ל ר׳ חייא בר אדא והדא היא בעשותה לא כן היחיד שעשאה חייב שנים שלשה שעשו פטורין.}
\textblock{\textbf{בכותל משופע.} פירשו בתוס׳ דדוקא בשרבים מכתפין עליו, הא לאו הכי כרמלית הויא. והרמב״ן ז״ל כתב אף על פי שאין רבים מכתפין עליו, דכל אויר רשות הרבים עד עשרה רשות הרבים הוא, ופני הכותל בין משופע בין אינו משופע שוין הן, ותנן (לקמן שבת ק, א) למטה מעשרה כזורק בארץ ואוקימנא בפני הכותל ממש ובדבילה שמנה. ומסתברא כלישנא קמא, דהתם הוא בדלא נח במקום מסויים אלא שנחה על פני הכותל מצד הדבילה שהיא לחה ושמנה, אבל כאן שהוא מקום מסויים לנוח בו החפץ הרי הוא כתל שאם מכתפין בו הוי כרשות הרבים ואי לא הוי כרמלית.}
\textblock{\textbf{היה קורא בספר על האיסקופה ונתגלגל הספר מידו גוללו אצלו וכו׳.} אוקמה אביי בפרק בתרא דעירובין (צח, א) באסקופא כרמלית ורשות הרבים עוברת לפניה, וכיון דאגדו בידו אפילו שבות נמי ליכא. ולאו דוקא ספר אלא הוא הדין לשאר כל הדברים, וספר לרבותא נקטיה משום סיפא דאפילו היה קורא בראש הגג ונתגלגל למטה מעשרה אינו גוללו אלא הופכו על הכתב.}
\textblock{ הא ד\textbf{בעי רבא אגוז בכלי וכלי צף על גבי מים אי אזלינן בתר כלי או בתר אגוז.} איכא למידק ומאי שנא מהא דאמרינן בפרק קמא דמציעא (ט, ב) גבי דגים שקפצו בספינה ספינה מינח נייחא ומיא הוא דקא ממטו לה. ויש לומר דהתם גבי קנין כל חצר שאין היא עצמה מהלכת קונה אף על פי שאחרים מוליכין אותה כספינה וכקלתה הקשורה בה, אבל גבי שבת בעינן שיטול החפץ ממקום נח, להכי מספקא לן דילמא אף כשהכלי צף במים ואגוז בתוכו בעינן שיהא האגוז מונח במקום שאינו מתנענע אפילו מצד תנועת אחרים.}
\textblock{הא דאמרינן:\textbf{ בשלמא בן עזאי קסבר מהלך כעומד דמי.} קשיא לן דלא הוה ליה למימר אלא בשלמא בן עזאי לא אשכחן כי האי גונא דחייב אלא רבנן היכי אשכחן כי האי גונא דחייב, ומי הצריכו לומר דטעמא דבן עזאי משום דקסבר מהלך כעומד דמי. ויש לומר דגמרא גמירי לה דטעמא דבן עזאי משום דמהלך כעומד דמי. אי נמי יש לומר דאמסקנא סמיך דאשכחן כי הא דחייב, וטעמיה דבן עזאי על כרחין משום מהלך כעומד דמי, אלא אורחא דתלמודא לברר דבריו דרך קושיא ותירוץ.}
\clearpage
\newsection{דף ו}
\textblock{ ואסיקנא:\textbf{ מידי דהוה אצידי רשות הרבים.} ואם תאמר וצידי רשות הרבים גופייהו מנא לן דחייב. יש לומר משום דמסתמא כך הוה במשכן, שכן דרך השוכנים באהלים שזה נכנס וזה יוצא מפני מיתרי האהלים שלא יעכבו זה את זה. ואם תאמר אמאי לא אמר ליה מידי דהוה אמעביר ד׳ אמות ברשות הרבים דרך עליו דחייב לפי שכן היה משא בני קהת. מסתברא דהתם משום שהמעביר בעצמו במקום חיוב עומד ולא עובר במקום פטור, אבל כאן שהוא עובר במקום פטור ממש בכי הא לא אשכחן אלא צידי רשות הרבים. ואם תאמר לימא מידי דהוה (אמוציא) [אמושיט] למעלה מעשרה דחייב שכן העגלות למעלה מעשרה ואף על פי שמעביר במקום פטור. מסתברא דהתם לגבי מושיט לאו מקום פטור הוא, אלא גזירת הכתוב. ותדע לך דהא רשות הרבים אינו עולה למעלה מעשרה ואף על פי כן מושיט מרשות היחיד לרשות היחיד דרך רשות הרבים אפילו למעלה מעשרה ולא נח ולא עבר לא בתחילה ולא בסוף ברשות הרבים חייב, והכא במעביר מרשות היחיד לרשות הרבים דרך סטיו וכן במעביר ד׳ אמות למעלה מעשרה או מוציא דרך עליו שהוא למעלה מעשרה לכולי עלמא אינו חייב אלא בשנח לבסוף ברשות הרבים עצמה, הא לא נח לא. ועוד תדע לך דאילו במושיט ממש אינו חייב אלא במושיט בדיוטא אחת הא במושיט בשתי דיוטות זו כנגד זו ורשות הרבים באמצע פטור, אלא דבדיוטא אחת חייב מגזירת הכתוב אף על פי שלא נח ולא עבר ברשות הרבים ולא באוירו. כך נראה לי.}
\textblock{ הא דאמרינן:\textbf{ היכא דאיכא חיפופי מי שמעת ליה.} תימא דהא ודאי בעירובין משמע דאפילו בדאיכא חיפופי אמרה רבי אליעזר, דאמרינן התם בפרק כל גגות העיר (צד, א) גבי פלוגתייהו דרבי אליעזר ורבנן, וליפלגו בצידי רשות הרבים דעלמא, כלומר: אמאי מיפלגי בחצר שנפרצה לרשות הרבים, ומשני אי איפלגו בעלמא הוה אמינא כי פליגי רבנן עליה הני מילי היכא דאיכא חיפופי אבל היכא דליכא חיפופי מודו קא משמע לן, אלמא רבי אליעזר אפילו היכא דאיכא חיפופי פליג. ויש לומר דהכי קאמר קא משמע לן דאדרבא היכא דליכא חיפופי פליג אבל היכא דאיכא חיפופי מודה להו רבי אליעזר לרבנן דכרמלית היא, ובמשכן חיפופי הוה ליה דהיינו יתידות ומיתרים שבאהלים הלכך להא דמיא.\par \textbf{} וטעמיה דבן עזאי משום דמהלך כעומד דמי. (לעיל שבת ה, ב). ואם תאמר אם כן מעביר ד׳ אמות ברשות הרבים היכי משכחת לה דחייב לבן עזאי. יש לומר דד׳ אמות ברשות הרבים הלכתא גמירי להו. ואם תאמר אם כן אפילו עמד לפוש תוך ד׳ יהא חייב. יש לומר דהא נמי הלכתא גמירי לה מהלך חייב עומד פטור. ואם תאמר ליליף מינה בעלמא דמהלך לאו כעומד דמי. יש לומר דילמא שאני התם דמקום חיוב לכולי עלמא כדאמרינן בגמרא, אבל סטיו דמקום פטור הוא לכולי עלמא אין דנין קל מחמור להחמיר עליו. ובירושלמי (פ״א, ה״א) מקשו לה ולא אסיקו לה פירוקא, דגרסינן התם: רב חסדא שאל לרב הונא לדעת בן עזאי אין אדם מתחייב בתוך ד׳ אמות לעולם, כיון שהוציאה יעשה כמי שהניחה בכל אמה ואמה ויהא פטור.\par \textbf{} וקיימא לן כרבנן דמהלך לאו כעומד דמי. ואף על גב דא״ר יוחנן מודה היה בן עזאי בזורק, לאו למימרא דסבירא ליה כוותיה, אלא פירושי הוא דקא מפרש לה וליה לא סבירא ליה. ותדע לך דהא רבי יוחנן גופיה הוא דאמר המפנה חפצים מזוית לזוית ונמלך עליהן והוציאן פטור, ואילו לבן עזאי חייב כדאמרינן בהדיא בכתובות בפרק אלו נערות (כתובות לא, א-ב). ועוד דבהדיא גרסינן בירושלמי (פ״א, ה״א) א״ר יוחנן המפנה מרשות היחיד לרשות הרבים דרך כרמלית חייב. והתימה מן הרב אלפסי ז״ל שכתבה להא דבן עזאי ולהא דא״ר יוחנן מודה היה בן עזאי בזורק.}
\textblock{      \textbf{ארבע רשויות לשבת.} הקשו בתוס׳ ז״ל: אמאי לא תני חמש, וליתני קרפף שהוא יותר מבית סאתים שלא הוקף לדירה שדינו מחולק מאלו, כדאמרינן בסמוך שאסור לטלטל בתוכו אלא בד׳ אמות ואם הוציא והכניס לרשות הרבים חייב. ויש לומר דההוא רשות היחיד הוא דבר תורה והא תנא ליה אלא דרבנן הוא דגזרו והכי נמי תני הכא בברייתא חצרות של רבים ומבואות שאינן מפולשים ולא חשיב להו במנין הרשויות מהאי טעמא דאמרן דרשות היחיד דאורייתא הן וכבר תני רשות היחיד וכמו שפירש רש״י זכרונו לברכה.}
\textblock{ הא ד\textbf{תניא יתר על כן אמר רבי יהודה מי שיש לו שני בתים בשני צידי רשות הרבים וכו׳.} תוספתא היא דמתניא בפרק בתרא דעירובין (לפנינו שם הוא בפ״ז, ה״ט) והכי מתניא התם: אחד חלון שבין שני בתים ואחד חלון שבין שתי חצרות ואחד חלון שבין שני דיורין וגשרים ונפשות ורשות הרבים מקורה מטלטלין תחתיהן בשבת דברי ר׳ יהודה וחכמים אוסרין, יתר על כן אמר רבי יהודה.}
\textblock{\textbf{ואמאי קרו לה גמורה מהו דתימא וכו׳.} איכא למידק אדרבא מדקתני גמורה משמע דזו היא גמורה ויש לך אחרת שאינה גמורה אלא הוי קצת כרשות היחיד. ונראה שלפיכך פירש רש״י ז״ל זו היא שנגמר מנין מחיצות שלה מכל צד, פירוש לפירושו דגמורה לא קאי ארשות אלא אמחיצות, כלומר: זו היא רשות היחיד שנגמרו מחיצות שלה, הא רשות אחרת שאינה גמורה במחיצות שאין לה אלא שתי מחיצות כגון ההיא דרבי יהודה אינה רשות היחיד כלל ואפילו להתחייב הזורק מרשות הרבים לתוכה.}
\textblock{\textbf{אימא אינו חייב על אחת מהן.} פירש רש״י ז״ל: אינו חייב מיתה על אחת מהן. וכן נראה ודאי דדוקא מיתה הוא דאינו חייב אבל חטאת חייב על שגגתה מדמנינן נמי ארבעים חסר אחת, דאי לא מחייב עליה לא כרת ולא חטאת אם כן אמאי מני להו בארבעים חסר אחת. ועוד מדאקשינן חייב חטאת פשיטא ושנינן כרת ונסקל איצטריכא ליה ופרישנא כדרב, אלמא חיוב חטאת לא מספקא להו דאכולהו חייב חטאת, דאי לא כי אקשינן חייב חטאת פשיטא מיד הוה ליה לשנויי קא משמע לן כדרב וכיון דבשוגג חייב חטאת פשיטא דבמזיד חייב כרת. ולפי פירוש זה הא דאמרינן בריש פרק מי שהחשיך (לקמן שבת קנד, א) אמר רמי בר חמא הכי קאמר כל ששגגתו בשבת חייב חטאת זדונו סקילה אתיא דלא כאיסי בן יהודה, דהא איכא דשגגתו חטאת ואין זדונו בסקילה.}
\clearpage
\newsection{דף ז}
\textblock{\textbf{א״ר יוחנן קרפף יותר מבית סאתים שלא הוקף לדירה ואפילו כור ואפילו כוריים וכו׳ מחיצה היא אלא שמחוסרת דיורין.} כלומר: ואסור לטלטל בכולה אלא בד׳ אמות מדרבנן. וגרסינן בפרק עושין פסין (עירובין כד, א) אמר רב נחמן אמר שמואל קרפף יותר מבית סאתים שלא הוקף לדירה פורץ בה פרצה יותר מעשר אמות וגודרו ומעמידו על עשר ודיו, כלומר: מעמידו על עשר לשם דירה ודיו דדמי כמי שהוקף כולו לדירה, ופרצה זו של עשר הוי כפתח ומותר לטלטל בכולו כחצר.}
\textblock{\textbf{לא נצרכה אלא לקרן זוית.} הוא הדין דהוה מצי למימר לא נצרכה אלא לצידי רשות הרבים, אלא משום דפליגי ביה ר׳ אליעזר ורבנן (לעיל שבת ו, א) דלרבי אליעזר הוי רשות הרבים לא בעי לעיולי נפשיה בפלוגתא, ואפילו בצדדים דאית בהו חפופי לא בעי לאוקמה משום דמספקא ליה אי מודה בהו רבי אליעזר.}
\textblock{\textbf{אבל לפני העמודים כרשות הרבים דמו.} ואף על פי שאין בין עמוד לעמוד רוחב שש עשרה שהוא רוחב רשות הרבים דאורייתא כדאיתא בפרק הזורק (לקמן שבת צט, א), אלא דהכא לא מסתברא שיהא העמוד ממעט בשיעור הרשות, שאם כן אפילו נעץ קנה ברשות הרבים ימעטנו. ומיהו קיימא לן כרבי יוחנן דאמר בין העמודים נידון ככרמלית, וכל שכן אצטבא, משום דרבי יוחנן לגבי רב קיימא לן כרבי יוחנן וכל שכן לגבי רב יהודה דהוא תלמידיה דרב. אבל הרמב״ם ז״ל (פי״ד מהל׳ שבת ה״ו) פסק כרב יהודה, ולא ירדתי לסוף דעתו. ואולי משום דקתני בברייתא דלעיל (שבת ו, א) האיצטוונית ולא מני בין העמודים בהדייהו דהויא רבותא טפי. ואינו מחוור דאם כן ליקשי מינה לרבי יוחנן, ואם בעלי הגמ׳ לא סמכו על זה להקשות ממנה לרבי יוחנן איך נסמוך עליה לדחות דברי רבי יוחנן דהלכה כמותו בכל מקום לגבי רב ושמואל וכל שכן לגבי רב יהודה תלמידם. ורב אלפסי ז״ל לא כתב מכל זה בהלכות כלום.}
\textblock{ הא ד\textbf{אמר רב יהודה איצטבא שלפני העמודים נידון ככרמלית.} תמיהא לי מאי קא משמע לן הא בהדיא קתני בברייתא והאצטוונית. ועוד מאי שלפני העמודים דקאמר לימא האיצטבא סתם. ואולי אפילו באיצטבא שאינה גבוהה שלשה קאמר, ומפני שהיא לפני העמודים אינה נוחה לידרס      והלכך נפקא ליה מתורת רשות הרבים. אלא שמצאתי בירושלמי (פ״א סוה״א) בהיפך מזה, דגרסינן התם ר׳ זעירא בשם רב יהודה זעיר בר חיננא בשם ר׳ חנינא סמטיות שבין העמודים נידונין ככרמלית. רבי שמואל בר חייא בשם ר׳ חנינא פירחי העמודים נידונין ככרמלית, לכך צריכא בגבוהין שלשה, עד כאן בירושלמי. ועדיין אני אומר כי יש שבוש בגירסת הירושלמי, ובשאינן גבוהין היא דהא צריכה, דאילו בגבוהין שלשה פשיטא. וצ״ע.}
\textblock{\textbf{לבינה זקופה ברשות הרבים וזרק וטח בפניה חייב.} פירוש: ואף על פי שאין רוחב הלבינה אלא שלשה דסתם לבינות רחבן שלשה וארכן שלשה, ואע״ג דברשות הרבים בעינן הנחה על גבי מקום ארבעה, הני מילי למעלה משלשה אבל למטה משלשה כארעא סמיכתא דמו. אבל בלבינה שהיא גבוהה יותר משלשה אם זרק וטח בפניה למעלה משלשה אומר ר״ת ז״ל (בתוד״ה וטח) שאינו חייב דהא בעינן הנחה על גבי מקום ארבעה על ארבעה, ומקצת מרבני הצרפתים ז״ל אמרו (ריבא בתוס׳ שם) דכל שרואה פני קרקע רשות הרבים כרשות הרבים דמי, וכן נראה מדברי רש״י ז״ל.\par \textbf{} והקשו בתוספות (שם) אם כן מאי דחיק לעיל (שבת ד, א) גבי מתניתין דפשט בעל הבית ידו והא בעינן עקירה והנחה מע״ג מקום ארבעה, דילמא מתניתין שמחזיק החפץ בידו בענין זה שהוא רואה את הקרקע. ואינה קושיא בעיני דפירות לא נייחי בכענין זה, דנתן לתוך ידו של עני או שנטל מתוכה מידו פשוטה משמע, דאילו תפסם ממנו העני או שהיה תופס אותם בידו כדי שלא יפלו, אם כן כשתפסם וסגר ידו או שהיתה ידו סגורה ופשטה כדי ליתן לו את החפץ הא קא עביד מעשה, וכעין נתינה או לקיחה היא, ורישא הא אמרינן בריש מכלתין (לעיל שבת ג, א) דלא עביד מעשה ופטור ומותר קאמר.}
\textblock{\textbf{אביי ורבא דאמרי תרווייהו והוא שגבוה שלשה וכו׳ אבל היזמי והיגי אף על גב דלא גביהי שלשה.} פירוש: משום דלא דרסי בה רבים. קשיא לי והא רבא הוא דאמר בפרק המצניע (לקמן שבת צב, א) המוציא פירות ברשות הרבים למטה משלשה ביד חייב משום דכל סמוך לקרקע בתוך שלשה כקרקע דמי, ויד הא לא דרסי בה רבים. וי״ל דהא דאמרינן אבל היזמי והיגי לאו סיומא דמימרא דאביי ורבא הוא, אלא אנן הא דקאמרינן ומסקינן בה הכי. ואינו מחוור. אלא מסתברא שלא אמרו אלא בדברים שדרך העולם להניחם ולהשליכם ברשות הרבים כגון היזמי והיגי וכיוצא באלו, שהן מושלכין ברשות הרבים ואינם עומדים להסתלק, והלכך כיון שמונעין את הרבים מלדרוס שם אין שם רשות הרבים באותו מקום, שאין דין רשות הרבים אלא במקום המוכן לדריסת רבים, אבל יד או כלי שאינן עשוין להתעכב שם אלא לשעה אין חולקין רשות לעצמן אלא עומדין הן ברשות ובטלין לגבי הרשות. ותדע לך, דהא אפילו כלי גבוה משלשה ועד עשרה אינו חולק רשות לעצמו ולהיותו ככרמלית, דכל שהוא עומד בתוך אויר הרשות אינו חולק רשות לעצמו, וטעמא כדאמרינן, והוא הדין והוא הטעם כאן. כך נראה לי.}
\textblock{ הא דאמרינן:\textbf{ אילימא דאי איכא מחיצה עשרה הוא דהויא כרמלית ואי לא לא הויא כרמלית והא אמר רב גידל וכו׳.} איכא למידק למה ליה לאקשויי ממימרא דרב גידל, לותביה מדתניא לעיל (שבת ו, א) וכן גדר שהוא גבוה עשרה ורוחב ארבעה זו היא רשות היחיד גמורה, פירוש והיכי קאמר איהו דאי איכא מחיצה עשרה הויא כרמלית. ותנן (לקמן שבת צט, א) חולית הבור והסלע בזמן שהן גבוהין עשרה הנותן לתוכן והנוטל מתוכן חייב. ורש״י ז״ל פירש דהאי לאו אגופא דמחיצות ממש קאי, אלא הכא בבקעה שאינה מוקפת לדירה והוא יותר מבית סאתים, וקאמר דאי איכא מחיצה עשרה הויא כרמלית אבל פחות מעשרה לא הויא כרמלית, ומשום הכי אקשי ליה מדרב גידל ומסיפא אקשי ליה. והקשו בתוס׳ דהא לא אפשר, דאי בדאית ליה מחיצות גבוהות עשרה לכולי עלמא רשות היחיד הוא, ובין עולא ובין רב אשי קרי ליה לעיל רשות היחיד ולא הוה צריך לאתויי הא דרב גידל, ור׳ יוחנן נמי אמר לעיל קרפף יתר מבית סאתים שלא הוקף לדירה וכו׳ מחיצה היא אלא שמחוסרת דיורין. ועוד דהיכי מצי למימר דאי ליכא מחיצה עשרה לא הויא כרמלית, והא קתני לעיל (שבת ו, א) דים ובקעה ואיצטוונית הוו כרמלית והנהו לית להו מחיצות. ופירשו הם ז״ל דלאו מחיצה ממש קאמר אלא אויר הראוי למחיצה עשרה, וכגון בקעה יתירה מבית סאתים ומסוככת על גבה והסכך על גבי קונדסין ואינו גבוה עשרה, והשתא בעינן למימר דאי אין בו אויר הראוי למחיצה עשרה דלא הויא כרמלית אלא מקום פטור ומותר לטלטל בכולו, והשתא פריך שפיר מדרב גידל.}
\textblock{\textbf{ואם חקק בו ארבעה על ארבעה והשלימו לעשרה מותר לטלטל בכולו. } קשיא לן דהכא משמע דבין שהמחיצות רחוקות מן החקק בין מקורבות לעולם מותר, וסתמא קאמר, ומאי שנא מהאי דאמרינן בפרק קמא דסוכה (ד, א) סוכה שאינה גבוהה עשרה וחקק בה והשלימה      לעשרה אם יש משפת חקק ולכותל שלשה טפחים פסולה. ותירצו בתוס׳ דהתם בעינן מחיצות ממש, והכשר סוכה במחיצות היא ושיהיו סמוכות לסכך ולמטה מן הסיכוך כדאמרינן (סוכה יז, א) הרחיק את הסיכוך מן הדפנות פסולה, אבל לענין שבת אין ההקפדה במחיצות ממש שיהיו מכשירות את החקק אלא שיהא החקק נשמר על ידיהן שלא יהיו הרבים בוקעין בו, וכיון שיש בחקק רוחב ארבעה על ארבעה ויש בגבהו כשיעור רשות היחיד עם גובה המחיצות והמחיצות שומרות אותו מבחוץ, לא חיישינן בין שיהיו על שפת החקק בין שיהיו מרוחקות ממנו כמה. ויש מי שתירץ דהכא בשיש עשרה בעומק החקק קאמר. ואינו מחוור, דבכולהו נוסחאי חקק בו והשלימו לעשרה גרסינן.}
\textblock{\textbf{ורבא אמר חורי רשות הרבים לאו כרשות הרבים דמו.} ודוקא בחור שהוא למעלה משלשה, אבל למטה משלשה כארעא סמיכתא היא. ואביי דמשוי ליה כרשות הרבים ואף על גב דהוי למעלה משלשה, ואף על גב דהוי מקום מסויים כעמוד גבוה שלשה ברוחב ארבעה דמשוינן ליה ככרמלית. חור שאני לפי שבני רשות הרבים מצניעין שם כליהם והוה ליה כתל גבוה תשעה דמשוינן ליה רשות הרבים מפני שרבים מכתפין עליו.}
\clearpage
\newsection{דף ח}
\textblock{\textbf{אמר אביי ברשות היחיד כולי עלמא לא פליגי כדרב חסדא.} כלומר: דלא בעינן הנחה על גבי מקום ארבעה. ואי קשיא לך דהא משמע לעיל (שבת ז, ב) דאפילו אביי גופיה הוה סבירא ליה דבעינן הנחה על גבי מקום ארבעה, מדאמרינן וכי תימא מתניתין דלית ביה ארבעה והא אמר רב יהודה אמר רב אמר ר׳ חייא אמר ר׳ זרק ונחה בחור כל שהוא באנו למחלוקת ר״מ ורבנן, כלומר: ר״מ סבר חוקקין להשלים. ועוד דמהתם נמי משמע דבין לר״מ בין לרבנן בעינן הנחה על גבי מקום ארבעה אלא דר״מ סבר רואין אותו כאילו חקק ורבנן לית להו רואין. ותניא בתוספתא (פי״א, ה״ז) הזורק בכותל למעלה מעשרה טפחים והלכה וישבה לה בחור שהוא ד׳ טפחים על ד׳ טפחים חייב. מסתברא [מ]כל הני לשוייה רשות היחיד משום דאינו נעשה רשות היחיד עד שיהא בו ארבעה על ארבעה, אבל ברשות היחיד ממש בהנחה כל דהו סגי. וכן נמי לרבי מאיר דסבירא ליה חוקקין הרי זה רשות היחיד גמורה כאילו חקק בו כבר וכיון שהוא רשות היחיד בהנחה כל דהו סגי. כך נראה לי.}
\textblock{\textbf{רחבה ששה פטור.} פירוש: לפי שיש לרבע בו ארבעה על ארבעה. ואי קשיא לך ששה למה לי בחמשה ושלשה חומשין סגיא, דכל טפח ברבוע טפח ושני חומשין באלכסון, נמצאו שמונה חומשין דהוו להו טפח ושלשה חומשין וד׳ טפחים הרי חמשה ושלשה חומשין. פירש רש״י ז״ל דלא דק ולחומרא לא דק לאפרושי מאיסור שבת. ולא מיחוור דהא אמר אינה רחבה ששה חייב, וחייב משמע בשוגג חטאת ובמזיד והתראה סקילה, ואדרבא חומרא דאתי לידי קולא הוא. ועוד, דאנן לאו לכתחילה שריא ליה דאפילו רחבה ששה פטור אבל אסור הוא, ואם כן כי קאמרינן פטור לפוטרו ממיתה וקרבן קאמרינן, ובשאינה רחבה ששה חייב מיתה וקרבן. ועוד, דכל היכא דלא דק מקשה לה בגמרא ומתרץ לה.\par \textbf{} ור״ח ורב אלפסי ז״ל פירשו דרחבה ששה עם הדפנות קאמר, ור״ל חומש לדופן זה וחומש לדופן זה, פש ליה חמשה טפחים ושלשה חומשין בחללה כדי לרבע בחלל ארבעה על ארבעה. ואם תאמר אם כן אף גבוהה עשרה שאמרו היינו עם עובי השוליים, דכי היכי דעובי הדפנות מכלל הששה גם עובי השוליים מכלל עשרה, ואם כן היכי הוא רשות היחיד. איכא למימר דהוי ליה כבית שאין בתוכו עשרה וקרויו משלימו לעשרה דעל גגו מותר להשתמש בכולו, וכגדוד ה׳ ומחיצה ה׳ שמצטרפין (עירובין צג, ב).\par \textbf{} אלא דקשיא להו לרבותינו הצרפתים ז״ל דאכתי הוה לן לצרופי עובי הדפנות לרוחב ארבעה על ארבעה, דהא אי בעי מנח עליה מידי ומשתמש. והביאו ראיה מדאמרינן במסכת עירובין בריש פרק חלון (עירובין עח, א) אמר רב הונא עמוד ברשות הרבים גבוה עשרה ורחב ארבעה ונעץ בו יתד כל שהוא       מיעטו, אמר רב אדא בר אהבה ובגבוה שלשה, רב אשי אמר אפילו אינו גבוה שלשה מאי טעמא אפשר דתלי ביה מידי, אמר ליה רב אחא בריה דרבא לרב אשי מלאו כולו ביתדות מהו, א״ל לא שמיע לך דאמר ר׳ יוחנן בור וחוליתה מצטרפין לעשרה, אמאי הא לא משתמש ליה, אלא מאי אית לך למימר דמנח מידי ומשתמש הכא נמי דמנח מידי ומשתמש. והא דאמר ר׳ יוחנן בור וחולייתה מצטרפין ממתניתין דפרק הזורק יליף לה, דתנן (לקמן שבת צט, א) חולית הבור והסלע בזמן שהן גבוהין עשרה ורחבין ארבעה הנוטל מהן והנותן על גבן חייב, ואקשינן עלה בגמ׳ למה ליה למיתני חולית הבור ליתני הבור והסלע, אלא הא קמ״ל דבור וחולייתה מצטרפין לעשרה, ופרש״י ז״ל שם בעירובין וכי היכי דשמעינן מינה לגובהה שמעינן מינה לפותיה דהא קתני ורחבין ארבעה, ואם כן הכא נמי גבי כוורת הא אפשר דמנח מידי עליה ומשתמש. והעלו זה בקושיא לדברי ר״ח ז״ל.\par \textbf{} ואני תמה לדבריהם דהא משמע בהדיא בגמ׳ דלא אמרינן הכי, מדאמרינן בפרק הזורק (לקמן שבת ק, א) אמר אביי בור עמוק עשרה ורחב שמונה וזרק מחצלת לתוכו חייב, חלקו במחצלת פטור, דהנחת מחצלת וסלוק המחיצה בהדי הדדי קא אתו. ואם איתא מאי סילוק מחיצה איכא, והלא אף המחצלת אף על פי שנעשה כותל לבור עולה היא מן המנין, אלא על כרחנו לא אמרו כן בכל כתלים אלא בכתלים עבים העשוים לכסות עליהם ולהשתמש מה שאין כן בכוורת ומחצלת. ואפשר דמשום הכי נקט אביי מחצלת ולא נקט דף, והכא נמי נקט כוורת ולא נקט תיבה או מגדל.\par \textbf{} ובירושלמי אמרו שם בפרק הזורק (פי״א, ה״ב) גבי ‏מתניתין דחולית הבור: אמר רבי יוחנן העומד והחלל מצטרפין בארבעה, והוא שיהא העומד מרובה על החלל, רבי זעירא בעי עד שיהא עומד שכאן ועומד שכאן רבה, א״ר יוסא פשיטא לר׳ זעירא שאין עומד מצד אחד מצטרף, פשיטא לא שיהא עומד מצד אחד רבה, לא צורכה אלא אפילו עומד השני.}
\textblock{\textbf{רבא אמר אפילו אינה רחבה ששה פטור.} פירוש: והוי אוגדה במקום פטור. ודוקא בזורק מרשות היחיד לרשות הרבים דאין אויר רשות הרבים עולה למעלה מעשרה, אבל זרק מרשות הרבים לרשות היחיד כל שאינה רחבה ששה חייב דרשות היחיד עולה עד לרקיע. ולי נראה דאפילו ברשות היחיד פטור, דכל שגבוהה עשרה ורחבה ארבעה יצאה מתורת כלי והויא לה רשות לעצמה, וכזורק מרשות היחיד לרשות היחיד הוא, דכל מקום שהיא רשות היחיד היא.}
\textblock{\textbf{כפאה על פיה שבעה ומשהו חייב.} יש מי שפירש (רש״י ד״ה כפאה) דדוקא בשאינה רחבה ששה והלכך אף לכשתכנס למטה משלשה משהו והויא כלבוד הויא כולה תוך רשות הרבים וחייב, אבל ברחבה ששה פטור דפחות משלשה לקרקע כלבוד דמי והויא ליה רשות היחיד דהא גבוהה עשרה ורחבה ששה. ובתוס׳ אמרו דאדרבא ברחבה ששה אמרו, דאי בשאינה רחבה ששה אפילו גבוהה שבעה ומחצה חייב, דליכא למימר לבוד אלא במחיצות אבל בדפני כלי לא אמרינן לבוד, ולא חשיבי מחיצות אלא ברחבה ששה אבל כשאינה רחבה ששה אינה אלא כשאר חפצים דעלמא, אלא הכא ודאי ברחבה ששה קאמר.\par \textbf{} ואם תאמר אם כן גבוהה שבעה ומשהו ליפטר, דהא אמרינן לבוד והויא לה כגבוהה עשרה ורחבה ששה. יש לומר דכיון שאינה גבוהה עשרה אלא מחמת לבוד ולא אמרינן לבוד אלא במחיצות, אם כן השוליים אינן מצטרפין לגובה עשרה והרי היא כשאר חפצים דעלמא, אבל בגבוהה שבעה ומחצה כל שהוא סמוך לקרקע שלשה פחות משהו נמצא שיש בגובהן של מחיצות ממש עשרה בלא צירוף השוליים והוו להו מחיצות, והלכך אמרינן בהו לבוד.\par \textbf{} והרמב״ן ז״ל כתב דלעולם בין ברחבה ששה בין שאינה רחבה ששה לא אמרינן לבוד בכלים, ועכשיו כלי הוא, אלא עיקר פטורא משום דכיון שהיא בתוך שלשה נעשית כמונחת, וכשהיא גבוהה שבעה ומחצה הרי הונחה ואוגדה למעלה [מעשרה], לפיכך פטור ולא משום מחיצות כלל, עד כאן. ולא ירדתי לסוף דעתו, דאם כדבריו למה ליה למימר כפאה על פיה, הוא הדין נמי לשוליה כדאיירינן עד השתא. ואפשר דמשום הכי נקט כפאה על פיה, דאי לא כפאה אפילו בשבעה ומשהו פטור משום דאי אפשר לקרומיות של קנים שבפיה שלא יעלו למעלה מעשרה, אבל השולים חלקים הן ואין קרומיות עולין מהן.\par      \textbf{} ומיהו אכתי קשה לי דעל כרחין רבא משום לבוד אתי בה, מדאמר רב אשי אפילו שבעה ומחצה חייב מאי טעמא מחיצות לתוכן עשויות, כלומר: ואינן מחיצות לומר בהן לבוד, אלמא רבא דפטר משום לבוד נגע בה. ועוד דאי מתורת מונח אתית לה רב אשי אמאי מחייב, וכי פליגי רבא ורב אשי בשלשה סמוך לקרקע אי כמונח דמי או לא.\par \textbf{} ואם תאמר מכל מקום אפילו כשזרקה דרך שוליה ויש בה שבעה ומחצה ליפטר מהאי טעמא דאמרן, דכל שלשה סמוך לקרקע כקרקע דמי. ראיתי לרבנו ז״ל (הרמב״ן) שכתב דלא קשיא, דקיימא לן כרבא דאמר בפרק הזורק לקמן (שבת ק, א) דתוך שלשה בעינן הנחה על גבי משהו.\par \textbf{} ומכל מקום תקשי לן לפי מה שפסק שם ר״ח ז״ל כרב חלקיה בר טובי, דאמר בריש פרק הזורק (לקמן שבת צז, א) דלמטה משלשה דברי הכל חייב, דתניא כוותיה התם, ובגיטין נמי בפרק הזורק (גיטין עט, א) [תנן] היתה עומדת בראש הגג וזרקו לה כיון שהגיע לאויר הגג מגורשת, ואקשינן והא לא מינטר, ואמר עולא משמיה דאבימי הכא בפחות מג׳ סמוך לגג עסקינן, דכל פחות מג׳ סמוך לגג כגג דמי. אלא שיש לתרץ בזו דגט שאני דאפילו קלוט חשבי ליה רבנן כמונח, מה שאין כן לענין שבת כדאיתא התם בפרק הזורק.\par \textbf{} ומסתברא לי דלא אמרו תוך ג׳ סמוך לקרקע כקרקע דמי אלא להתחייב בו כאילו זרקו לקרקע ונח בו ממש, אי נמי לקנותו בו כאילו נח ממש בקרקע, אי נמי כאילו נתמלא כל האויר עפר והויא כארעא סמיכתא ואילו היה אותו אויר קרקע מתלקט ועולה עד החפץ מאותו מקום אתה מודד עשרה טפחים לאויר רשות הרבים, אבל שנראה כאילו נשתרבבו שולי הכלי או הגט והגיע לארץ ונחשוב הכלי כאילו אגדו למעלה מעשרה בכי הא לא אמרינן לעולם. כנ״ל.}
\textblock{\textbf{עמוד תשעה ברשות הרבים ורבים מכתפין כו׳.} פירש רש״י ז״ל: ואף על גב דאינה רחבה ארבעה, וכן כתב הרמב״ם ז״ל (פי״ד מהל׳ שבת ה״ח). ותמיהא לי דהא בעינן הנחה על גבי מקום ארבעה וליכא. ויש לומר דכיון דמשום דמכתפי ביה משויא לה רשות הרבים, ודרך המכתפין לכתף אף על פי שאין בה ארבעה לפי שאין מניחים שם משאם אלא נסמכין בו הרי הוא מקום חשוב וראוי לתשמישו, והרי זה כידו של אדם החשובה לו כארבעה על ארבעה לפי שדרכם של בני אדם להעמיד כליהם בידיהם. ועוד דכיון דדרכן לכתף עליו מחשבת הרבים משויא ליה מקום, ולא גרע מזרק ונח בפי הכלב או בפי הכבשן (לקמן שבת קב, א). כך נראה לי.\par \textbf{} והראב״ד ז״ל פירש (הביאו רבנו לעיל (שבת ה, א) ד״ה ידו): דוקא ברחב ד׳. ודוקא שמכתפין עליו אבל כשאין מכתפין עליו ואע״ג דראוי לכתף עליו לא, וכדמוכח בהדיא בעירובין (לב, ב) גבי אילן העומד חוץ לד׳ אמות ונתכוין לשבות בעיקרו דמייתי עלה הא דעולא (לג, א). והא דאמרינן הכא תשעה ודאי מכתפי עליה, הכי קאמר: תשעה ודאי ראוי לכתף עליו ודרכן של בני אדם לכתף עליו וכיון שהוא ראוי לכתף ומכתפין עליו עכשיו כרה״ר הוא, שאילו לא היה ראוי לכתף עליו אע״פ שהרבים מכתפין עליו לא היינו נותנין לו דין רשות הרבים כיון שאינו ראוי לכך.}
\textblock{\textbf{גומא מאי אמר ליה וכן בגומא.} פירשו בתוס׳: וכן בגומא בת תשעה וכל שכן בת שמונה ובת שבעה, דכל שעמוקה מעט ראוי יותר להשתמש בה, אלא שאפילו בת תשעה משתמשין בה, וכדמוכח לכאורה בהא דקופה דמייתי עלה דקתני פחות מכן מטלטלין וכן בגומא, ופחות מכאן כל שהוא פחות מעשרה משמע ואפילו שמונה ושבעה. וכן נמי ההיא דהניח עירובו למטה מי׳ (בעמ׳ ב) דאמרינן בבור דלית ביה עשרה, ומשמע כל שאינו עשרה ואפילו שמונה ושבעה. ואחרים (רש״י ורמב״ן) פירשו דוקא גומא תשעה כעמוד תשעה, דכל שהיא תשעה מצנעי בה הרבים כליהם שאין נדרסין ברגלי בני אדם, הא פחות מכן נדרסין ברגלי בני אדם      שהן יוצאין קצת לחוץ ולא משתמשי בה. ופחות מכאן דקתני בברייתא דקופה וכן בגומא פחות מעשרה והוא דהויא תשעה קאמר, וכן ההיא דבור. והראשון נראה עיקר כפשטה של ברייתא. ועוד דעל כרחין בקופה כל שהיא יותר פחותה ואפילו חמשה וארבעה מותר לטלטל מתוכה לרשות הרבים דכלי הוא ואם כן גומא נמי דכוותה.}
\textblock{ הא דאמר רבא:\textbf{ הלוך על ידי הדחק שמיה הלוך.} תמיהא לי הלוך דנקט, דהוה ליה למימר כיתוף על ידי הדחק שמיה כיתוף, דהלוך מאן דכר שמיה. ועוד דלא כיתוף על ידי הדחק הוא אלא בהדיא מכתפי עליה, ומאי הלוך על ידי הדחק דקאמר. וראיתי בהלכות מורי הרב רבנו יונה זכרונו לברכה דרב יוסף דאמר וכן בגומא לאו לעמוד תשעה מדמי ליה, דהתם לאו על ידי הדחק הוא דבהדיא מכתפי עליה, אלא רב יוסף לרקק מים שרשות הרבים מהלכת בו מדמי ליה וכדמייתינן לה עלה במסקנא (בעמוד ב), ובהא מיתרצא לי לישנא דהתם הלוך על ידי הדחק הוא, והיינו דאהדר ליה רבא, הלוך על ידי הדחק שמיה הלוך.}
\textblock{\textbf{אלא לאו בבור דלית ביה עשרה וקתני עירובו עירוב.} מכאן דקדק מורי הרב ז״ל וכן בתוס׳ דמותר לטלטל לכתחילה פחות מד׳ אמות ברשות הרבים, דהא למאן דאמר וכן בגומא דהויא רשות הרבים מטלטלין מתוכה לרשות הרבים לכתחילה דהא עירובו עירוב. והא דאמרינן לקמן בפרק מי שהחשיך (שבת קנג, ב) גבי נותן כיסו לנכרי אם אין שם נכרי וכו׳, אמר ר׳ יצחק עוד אחרת היתה ולא רצו חכמים לגלותה מטלטלו פחות פחות מד׳ אמות ואמאי לא רצו לגלותה דילמא אתי לאתויי ד׳ אמות ברשות הרבים, דאלמא מדוחק התירו למי שהחשיך לו בדרך דוקא, הא בעלמא לא. התם במוציאו פחות פחות מד׳ אמות ואפילו מהלך רב, דגזרינן דילמא אתי לאתויי בזמנא חדא ארבע אמות, אבל בתוך ארבע אמות בלבד שרי. וכן דעת הרמב״ם ז״ל (פי״ב מהל׳ שבת, הט״ו).\par \textbf{} ולכאורה הכין משמע כדבריהם, מהא דתנן במסכת עירובין בפרק המוצא תפילין (עירובין צח, ב) עומד אדם ברשות היחיד ומטלטל ברשות הרבים ובלבד שלא יוציא חוץ לד׳ אמות, דאלמא פחות מד׳ אמות אפילו לכתחילה שרי.\par \textbf{} ופחות מד׳ אמות דקאמרינן היינו ארבע ואלכסונן שהן ה׳ אמות ושלשה חומשין, דכל אמתא בריבועא אמתא ותרי חומשי באלכסונא, דכל שיעורי דשבת הן ואלכסונן הן, וכדילפינן (עירובין נא, א) ממגרשי הערים (במדבר לה, ה) שהן אלפים אמה מרובעות, וכדאמרינן בעירובין בפרק מי שהוציאוהו (שם) גמרא ויש לו אלפים אמה לכל רוח אמר רב אחא בר יעקב המעביר ד׳ אמות ברשות הרבים אינו חייב עד שיעבור ד׳ אמות הן ואלכסונן. ויש בזה דעת אחרת למקצת מרבותינו הצרפתים ז״ל, ושם כתבתיה בארוכה בס״ד.\par \textbf{} והראב״ד ז״ל כתב (וכ״ה בהשגותיו פי״ב משבת הט״ו) דלכתחילה אסור לטלטל ברשות הרבים פחות מד׳ אמות, דבין ד׳ אמות לפחות מד׳ אמות טעו אינשי וגזרינן דילמא אתי לאתויי ד׳ אמות, כפשטיה דההיא דפרק מי שהחשיך, ולא התירו לעולם אלא במקום הדחק, כגון בא בדרך וכיסו עמו, אי נמי במוצא תפילין, אי נמי מי שיצא חוץ לתחום בין לאונסו בין לרצונו, ואי נמי אם יש לו שם מקום קביעות כגון ששבת שם והיינו הא דעירוב שנתנו בבור.\par \textbf{} וההיא דעומד אדם ברשות היחיד ומטלטל ברשות הרבים שבמסכת עירובין פירשה שם הראב״ד ז״ל במטלטל למעלה מעשרה, דאיכא תרתי לפטורא דהוא פחות מד׳ אמות ומטלטלו למעלה מעשרה שהוא מקום פטור, ואפילו הכי חוץ לד׳ אמות חייב כדרבא דאמר המעביר מתחילת ארבע לסוף ארבע דרך עליו חייב.\par \textbf{} ומכל מקום יש לי לומר, דלכולי עלמא לא גזרו בבין השמשות לטלטל ד׳ אמות ברשות אחרת.}
\textblock{\textbf{האי זירזא דקני רמא וזקפה פטור.} ולא דמיא למגרר דאמרינן במסכת כתובות (לא, א) הגונב כיס בשבת וכו׳ היה מגרר ויוצא פטור דאיסור שבת ואיסור גניבה באין לו כאחד, דמגרר שאני דכל שהוא מגרר הרי עקר את כולו ממקומו בבת      אחת, מה שאין כן בזירזא דקני דבעידנא דזקיף רישא אכתי סופיה מונח בדוכתיה וכי הדר זקיף רישא אחרינא רישא קדמאה מינח נייח על גבי קרקע, ויש כאן עקירות הרבה בלא הנחות והנחות הרבה בלא עקירות. ועוד דבלאו הכי ההיא דמגרר לא קשיא, דהא אוקמוה התם בפרק אלו נערות (בעמוד ב) בשצירף ידו למטה מג׳ וקבלה דאיכא עקירה גמורה והנחה.}
\clearpage
\newsection{דף ט}
\textblock{\textbf{אסקופה משמשת שתי רשויות.} האי אסקופה דאיירי בה אחרים, אסקופת בית וחצר או אסקופת מבוי, אם של בית וחצר נתונה היא בין שתי מזוזות הבית והמשקוף מקורה עליו, ואם של מבוי הוא מקום גבוה מעט והוה כפתח למבוי ומניחים שם בעובי האסקופה לחי מכאן ולחי מכאן. ופירש רש״י ז״ל דהשתא קא סלקא דעתך דבאסקופת מבוי עסקינן דהכשירו בלחי, והא דאמר רב יהודה הכא באסקופת מבוי עסקינן, דמשמע לכאורה דעד השתא לאו באסקופת מבוי איירי, הכי קאמר: הכא באסקופת מבוי כדקאמרת אבל לא בניתר בלחי אלא שהכשירו בקורה, וכאילו אמר הכא באסקופת מבוי שהכשירו בקורה עסקינן. וקא מתמה ואף על גב דלית ליה לחי, כלומר: שאין לחי אחר חוץ ללחי העומד כנגד רוחב האסקופה, שכן דרך הלחיים להעמידן זקופין כנגד עובי האסקופה דהיינו פתח המבוי, ואם כן אפילו פתח פתוח אמאי כלפנים, והא אמר רב חמא בר גוריא אמר רב תוך הפתח צריך לחי אחר להתירו, וסתמא קאמר אפילו פתח פתוח, דקא סבר אסור להשתמש בין הלחיים שאין הלחי מתיר אלא מחודו הפנימי ולפנים. ואוקמה רב יהודה במקורה, וכמאן דאמר מותר להשתמש תחת הקורה דחודה החיצון יורד וסותם, וקרויו כלפי פנים, ולפיכך פתח פתוח כלפנים, כלומר: כל מה שתחת הקורה וממנה ולפנים, אבל פתח נעול אפילו מה שתחת הקורה כלחוץ דכל שהפתח נעול בטלה לה תורת קורה, ובשאין ברחבה ארבעה לא שייך בה תורת גוד אחית.\par \textbf{} והא דלא אוקמה באסקופת בית, משום דסתם בית פתח שלו יש משקוף מלמעלה וסתמו רחב ארבעה ונמצא כולו מקורה, הלכך אפילו פתח נעול כלפנים דחודו החיצון יורד וסותם. ואם תאמר לוקמה בקירויו כלפי חוץ ובשאין ברוחב האסקופה ארבעה, והלכך פתח נעול כלחוץ דבטלה לה תורת קורה, שאין קורה מתרת אלא מקום חשוב שיש ברחבו ארבעה. יש לומר משום דסתם אסקופת מבוי נמי רחבה ארבעה. וכרב אשי נמי לא אוקמא דלא ניחא ליה לאוקמה בשתי קירויות. זהו פירוש שמועה זו לפי שיטתו של רש״י ז״ל.\par \textbf{} ויש מרבוותא ז״ל שהקשו לפירושו דהא אוקימנא (לעיל שבת ח, ב) אסקופה דרישא באסקופת מקום פטור שאין ברחבה ארבעה, והלכך טפי הוה ניחא לאוקומה נמי האי אסקופה דאחרים בשאין בה ארבעה. ועוד דכי אמרינן וכי תימא בדלית ליה ארבעה על ארבעה והאמר רב חנן כו׳, לימא ליה לעולם בדלית בה ארבעה ובמקורה וקרויה כלפי חוץ. ומסתברא לי דהא לא קשה ולא מידי דהא ודאי סתם אסקופה בדאית בה ארבעה, ובפרק בתרא דעירובין (קא, ב) בשמעתא דשערי גינה קרי לה לאסקופה כרמלית, אלמא סתמא רחבה ארבעה היא, ולעיל נמי הכי משמע מדאקשינן האי אסקופה היכי דמיא אי אסקופת רשות היחיד כו׳ אי אסקופת רשות הרבים אי אסקופת כרמלית דאלמא סתמן ארבעה הן, ודאקשינן נמי וכי תימא בדלית ביה ארבעה הכי מוכח. והלכך רב יהודה נמי ניחא ליה לאוקומה הכי וכדמשמע להו מעיקרא, דאי הוה מוקי לה בדלית בה ארבעה הוה משמע דבדאית ליה לא הוה משכח להו פתרי.\par \textbf{} ומיהו מה שפירש הוא ז״ל דפתח פתוח כלפנים דקתני דוקא מחודה החיצון של קורה ולפנים קאמר לא מחוור, דבהדיא גרסינן בירושלמי (פ״א, ה״א) כל זמן שהפתח פתוח כולה כלפנים. ולישנא ודאי הכין משמע, וטעמא דהואיל והותרה מקצתה הותרה כולה, דהא איכא היכרא טובא דמופרשת היא כולה מרשות הרבים או מן הכרמלית העוברת לפניה ולא אתי לאפוקי לחוץ.\par       \textbf{} ומורי הרב ז״ל פירש: דמעיקרא הוה משמע לן דאיירי בין באסקופת בית בין באסקופת מבוי ובשאין ברחבה ד׳, ואקשינן אי אסקופת מבוי הניתר בלחי אפילו פתח פתוח אמאי כלפנים, ולפיכך קאמר רב יהודה הכא באסקופת מבוי עסקינן ובשהכשירו בקורה וקירויו כלפי פנים. והא דלא אוקמה במקורה כולה ואפילו הכי פתח נעול כלחוץ הואיל ואין ברחבה ארבעה, משום דפתח נעול כלחוץ משמע שמותר לטלטל מן האסקופה שתחת הקירוי לחוצה לה שאינו מקורה לפי ששניהם כרמלית מדרבנן ורשות היחיד מן התורה, ואי במקורה כולה אי נמי כשהקירוי כלפי חוץ לא מתוקמא ליה לרב, דמאי אמרת בזמן שהפתח נעול כלחוץ ומותר להשתמש עם הרשות העוברת לפניה, הא לא אפשר, דאי רשות הרבים עוברת לפניה, הא קא מפיק מן האסקופה שהיא רשות היחיד דאורייתא דהא אית לה שתי מחיצות והדלת שהיא נעולה שהיא לה כמחיצה שלישית וקא מפיק לרשות הרבים, ואי כרמלית עוברת לפניה קסבר רב דאף הוא אסור להשתמש מן האסקופה שהיא כרמלית דרבנן לכרמלית גמורה, דהכי סבירא להו לאמוראי בפרק הדר עם הנכרי (עירובין סז, ב) גבי סלע שבים גבוה עשרה ורחב ארבעה שהוא יותר מבית סאתים, דאמרינן התם דאסור לטלטל ממנו לים ומן הים לתוכה. ורב אשי דאוקמה באסקופת בית שהיא מקורה כולה בשתי קירויות ופתח פתוח כלחוץ, בזמן שכרמלית גמורה עוברת לפניה דקסבר דמותר לטלטל מכרמלית דרבנן לכרמלית גמורה, כדאיתא התם בפרק הדר (שם) בההיא דסלע שבים שכתבנו.\par \textbf{} ואם תאמר לענין שמעתין אמאי לא אוקמוה במבוי הניתר בלחי ובפתוח לרשות הרבים דהשתא מותר להשתמש בין הלחיים, ודרב חמא בר גוריא בפתוח לכרמלית דמצא מין את מינו וניעור, דהכי אוקמה רבא בפרק קמא דעירובין (ט, א). יש לומר משום דסבירא ליה דאי בפתוח לרשות הרבים אפילו פתח נעול כלפנים דסתמא קאמר. וכבר כתבתיה שם במקומה בפרק קמא דעירובין בסייעתא דשמיא.}
\textblock{\textbf{כגון עמוד ברשות היחיד גבוה עשרה ורחב ארבעה אסור לכתף עליו גזירה משום תל ברשות הרבים.} מסתברא לי, דדוקא בחצר השותפין והוא שלא עירבו, ומשום דדמיא קצת לרשות הרבים שאסור להכניס ולהוציא ממנה לבתים ומן הבתים לתוכה, אבל ברשות היחיד של אדם אחד אי נמי בחצר השותפין ועירבו לא. ואינו נראה כן מדברי רש״י זכרונו לברכה במסכת עירובין (פט, א). וכבר כתבתיה בארוכה במקומה בעירובין בפרק כל גגות (שם) בסייעתא דשמיא.}
\textblock{\textbf{רב אחא בר יעקב אמר.} לעולם סמוך למנחה גדולה ובתספורת דידן גזירה שמא ישבר הזוג.\par \textbf{} ולענין פסק הלכה: הרב אלפסי ז״ל פסק הלכה כהאי לישנא דרב אחא בר יעקב, וכן פסק מורי הרב ז״ל. אבל הר״ז הלוי כתב דעיקרא דמילתא כלישנא קמא דגמרא דהוו מוקי לה למתניתין במנחה קטנה ובסעודה קטנה, דעד כאן לא דייקינן עלה ולא נדינן מינה אלא מדר׳ יהושע בן לוי דאמר אסור לטעום כלום קודם שיתפלל, וההיא דרבי יהושע בן לוי הא אידחיא לה מהלכתא במסכת ברכות (כח, ב) בהדיא, דאמרינן התם בדוכתא ולית הלכתא כריב״ל, והלכך מתניתין כפשטה שמעינן לה, דסמוך למנחה גדולה הא איכא שהות טובא ואפילו סעודה גדולה שריא. ונראין דבריו. ואלא מיהו אפשר דסעודה גדולה דהיינו סעודת אירוסין ונשואין וכיוצא בה אפילו סמוך למנחה גדולה אסורה משום שכרות, וכטעמא דאמרינן לקמן גבי תפלת ערבית למאן דאמר חובה דאפילו שרי המייניה מטרחינן ליה ומשום דבלילה שכרות שכיחא ואע״ג דאיכא שהות טובא דהא מתפלל והולך כל הלילה, ואפילו התיר חגורו מפסיק.\par       \textbf{} ומיהו אם התחיל לאכול מסתברא שאינו מפסיק בין בסעודה גדולה בין בערבית אפילו למאן דאמר תפלת ערבית חובה, דלמיסר המייניה דליכא טירחא כולי האי הוא דאמרו דמטרחינן ליה אבל כשהתחיל באכילה לא אמרו. ותדע לך מדאקשינן עלה דההיא למ״ד חובה מטרחינן ליה, והא תנן אם התחילו אין מפסיקין ואמר רב משיתיר חגורו, ולמה לן לאורוכיה כולי האי ולאתויי הא דאמר רב משיתיר חגורו, ממתניתין גופא איכא לאקשויי דקתני ואם התחילו אין מפסיקין, אלא דאי לאו הא דאמר רב משיתיר חגורו הוה אמינא דהתם דוקא בשהתחיל לאכול קאמר ובדין הוא שלא יפסיק אבל בהתרת חגורו מפסיק, ואנן מאן דשרי המייניה הוא דאמרינן אבל במתחיל לאכול לא. כך נראה לי.\par \textbf{} עוד נראה לי דאפילו סמוך למנחה גדולה בסעודה גדולה ובתספורת בן אלעשה ובכולה מילתא דמרחץ ובבורסקי גדולה אסור, דהא אוקימנא לה במנחה גדולה ובהני דאמרן לא הוה קשה לן מידי, דאלמא בכל כי הני ליכא למימר הא איכא שהות טובא, דאינהו נמי ממשכי טובא ודילמא פשע, אבל אם התחילו אין מפסיקין, אלא דבסעודה גדולה מסתברא דאף על גב דהתחיל והתיר חגורו מפסיק, אבל כשהתחיל לאכול אינו מפסיק, והוא הדין לתפלת ערבית שאם התחיל לאכול אינו מפסיק שאינו מפסיק למילי דרבנן כל שיש שהות ביום או בלילה.\par \textbf{} אבל לכל מה שהוא מחויב דאורייתא מפסיק כגון לולב ושופר וכיוצא בהן, וכמו ששנינו במסכת סוכה (לח, א) לא נטלו נוטלו על שולחנו ואוקימנא התם אפילו יש שהות ביום, וכיון שכן אפילו לקריאת שמע מפסיקין, ואפילו בשהתחיל לאכול ואפילו איכא שהות. והלכך מסתברא דקורא פסוק ראשון דאורייתא והדר אכיל, אי נמי קורא כל קריאת שמע, ולאחר שיגמור כל סעודתו חוזר וקורא קריאת שמע בברכותיה וסומך גאולה לתפלה, וכן כתב מורי הרב זכרונו לברכה.}
\clearpage
\newsection{דף יא}
\textblock{\textbf{[לוה אדם תעניתו ופורע].} מהא דאמר רב אשי לוזיף מר וליפרע איכא למשמע דאפילו קבל עליו תענית ואמר יום זה לוה ופורע, דאי לא היכי אמר ליה להדיא לוזיף דמנא ליה דלא אמר יום זה. וכן כתב הראב״ד ז״ל במסכת תענית. אבל שמעתי שאין כן דעת רבותינו הצרפתים בעלי התוספות ז״ל, דכל שאמר יום זה אינו לוה. וסבור אני ששמעתי מפי מורי הרב זכרונו לברכה דהא דרב אשי מדת נדיבות ומוסר היה, שהיה מחזר שיאכל אצלו ואמר לו שמא לא יום זה אמרת ולוה ותפרע. והראשון נראה עיקר, דהא אפילו באומר סלע זו לצדקה כל שלא באת ליד גבאי לוה ופורע.}
\textblock{\textbf{אמר אביי היא היא.} דקסבר אביי דכולא חדא גזירה היא, דכל כי הני אי לא הא לא קיימי הא. ואיכא נוסחאי דכתיב בהו הכין בהדיא. ולרבא נמי איכא דוכתא דגזר נמי בכי האי גוונא, דאמרינן בפרק במה אשה יוצאה (לקמן שבת סד, ב) כל שאסור לצאת לרשות הרבים אסור לצאת לחצר, ואפילו בדברים שאם יצא בהם לרשות הרבים פטור       דאינו אלא מדרבנן, וטעמא התם מפני שהדבר קרוב הוא לבא לידי פשיעה, לפי שאין האשה עשויה לפשוט תכשיטיה כשיוצאה לחוץ וכן האיש להניח כליו שרגיל בהן. והיינו נמי טעמא דתפילין דאמרינן לקמן (שבת יב, א) דמותר לצאת בהן בערב שבת, וקא יהיב טעמא משום דאדם עשוי למשמש בהן בכל שעה, ואילו לרבא למאי איצטריכא להאי טעמא, דהא לדידיה כל שאילו יוצא בשבת ליכא אלא איסורא דרבנן לא גזרינן ביה בערב שבת, ומאי שנא ממחט התחובה בבגדו לר׳ מאיר, אלא דטעמא כדאמרן דקרוב הוא לבא לידי פשיעה לפי שהתפילין דרך מלבוש הוא ואין דרך אדם לפושטן כשיוצא לחוץ, והלכך אי לאו דמצוה למשמש בהן בכל שעה הוה אסרינן לצאת בהן בערב שבת.}
\textblock{ הא דאמרינן:\textbf{ וכן בגת לענין מעשר.} פירש רבנו חננאל ז״ל וז״ל: כשם שהעומד ברשות הרבים ושותה ברשות היחיד אינו מותר לו לשתות אלא אם כן יכניס ראשו ורובו למקום שהוא שותה שאם ישארו לו המים מותר להחזירן במקומן, וכן בגת לענין מעשר אינו מותר לו לשתות מן היין שבגת קודם שיעשר אלא אם הן צוננין שיתכן לו להחזיר המותר. אבל בחמין שמפסידין את היין דאי אפשר לו להחזיר המותר אסור, שמזגו לחמין וקבע למעשר אסור.}
\clearpage
\newsection{דף יב}
\textblock{ מדאמרינן:\textbf{ מולל וזורק וזהו כבודו ואפילו בחול.} ולא קאמר ובלבד שלא יהרוג, שמעינן מינה דמשום כבודו בלחוד קאמר, אבל אם בא להרוג ואפילו בשבת מותר. ומ״ד ובלבד שלא ימלול ובלבד שלא יהרוג רבי אליעזר היא דאמר הורג כנה כהורג גמל, ולא קיימא לן כוותיה דמדבית שמאי הוא, וקיימא לן כבית הלל דמתירין להרוג מאכולת, ורבא נמי דשדי להו לקנא דמיא ורב נחמן דאמר לבנתיה קטלן, בשבת הוי, דדבר הלמד מענינו הוא. וכן מוכח בירושלמי (פ״א, ה״ג). וטעמא דכנה שמותר להורגה לפי שאינה פרה ורבה דאינה גדלה אלא מן הזיעה, ואנן מאילים ילפינן כדאיתא בריש פרק שמונה שרצים (לקמן שבת קז, ב) מה אילים שפרים ורבים אף כל שפרה ורבה, ומכאן אמרו שם דאסור להרוג פרעוש בשבת לפי שהוא פרה ורבה.\par \textbf{} ופרעוש יש שפירשו אותו המין השחור שנקרא בערבי בורגות. ויש שאמרו שאותו המין הנקרא בורגות אינו פרה ורבה ואינו נעשה אלא מן העפר, ועל כן פירש כאן ר״ח ז״ל שהכנה היא הפרעוש, לפי שכתיב (שמות ח, יב) והך את עפר הארץ והיה לכנים, אלמא הכנה היא דהוה מן העפר.\par \textbf{} ואלא מיהו אי אפשר לומר שהיא הפרעוש הנזכר בגמרא, דהא משמע התם בפרק שמונה שרצים שהפרעוש הוא מין שפרה ורבה ואסור להורגו, ולא נחלקו בו רבי אליעזר ורבי יהושע אלא לצודו לפי שאין במינו ניצוד אבל להורגו כולי עלמא מודו דאסור. ואפשר שאותו פרעוש מין רחש אחר הוא שפרה ורבה ולא הבורגות.\par \textbf{} והרמב״ן ז״ל כתב שהבורגות מותר לצודו ולהורגו שבכלל מאכולת הוא. ומסתברא ודאי הכין, מדאיבעיא לן היכי קתני אין פולין כדי שלא יהרוג ואין קורין לאור הנר ור׳ אליעזר היא דאמר ההורג כינה בשבת כהורג גמל, או דילמא אין פולין לאור הנר קאמר אבל כדי שלא יהרוג לא, כלומר: דמותר להרוג ודלא כרבי אליעזר, ואסיקנא דאין פולין לאור הנר קאמר הא ביום שרי, דאלמא משום שמא יהרוג לא דמותר להרוג, ואם איתא דאסור להרוג בורגות הרי אותו המין מצוי בבני אדם ככנה ואם כן יהא אסור לפלות כליו אפילו ביום כדי שלא יהרוג, אלא ודאי משמע מהכא דאותו המין מותר להרגו.}
\textblock{\textbf{וכולה פרשה לא והתניא התינוקות מסדרין פרשיותיהן,} ופרקינן:}
\textblock{\textbf{ שאני תינוקות דאימת רבן עליהן.} קשיא לי דקארי לה מאי קארי לה דפשיטא, דאי לא אדמהדר לאקשויי מברייתא דתינוקות ליקשי ליה ממתניתין דקתני אבל הוא לא יקרא עמהם, דאלמא הוא לא יקרא עמהם אבל הם קורין לעצמן ומשום דאימת רבן עליהם. ונראה לי דברייתא שמיעא ליה דהתינוקות מסדרין לעצמן אפילו כשאין רבן עומד עליהן, מדלא קתני התינוקות מסדרין פרשיותיהן בפני רבן, אלמא סדור כולה פרשה שרי, ומה לי הן בלא רבן ומה לי הן ורבן, ופריק שאני תינוקות דאע״ג דאין רבן מסדר עמהם ליכא למיחש משום דכל שעתא ושעתא אימת רבן עליהן. כך נראה לי.}
\clearpage
\newsection{דף יג}
\textblock{\textbf{תניא ר׳ שמעון בן אלעזר אומר בא וראה עד היכן פרצה טהרה בישראל.} תוספתא (פ״א, ה״ז): אמר רבי שמעון בן אלעזר בא וראה עד היכן פרצה טהרה שלא גזרו הראשונים לומר לא יאכל הטהור עם הנדה, שהראשונים לא היו אוכלים עם הנדות, אלא אמרו לא יאכל הזב עם הזבה.}
\textblock{\textbf{בימי לבוניך מהו אצלך אמרה לו אכל עמי ושתה עמי וישן עמי בקירוב בשר אבל לא עלה על לבו לדבר אחר.} יש מקשים ואותו תלמיד ששנה הרבה ושימש תלמידי חכמים הרבה במה היה טועה, וכי לא היה יודע שאינה טהורה עד שתטבול ותספור, והא כתיב (ויקרא טו, כח) וספרה לה שבעת ימים ואחר תטהר, ומה הפרש היה לו בין ימי ראייתה לימי ספורים שלה. ויש מפרשים שלא בזבה דאורייתא קאמר, אלא בשראתה בימי נדה דדבר תורה אין צריך ספירת ימים נקיים שאפילו ראתה כל שבעה ולערב פסקה קודם שתשקע החמה טובלת וטהורה לביתה, אלא לאחר שנהגו בנות ישראל שאפילו רואות טפת דם כחרדל יושבת עליו שבעה ימים נקיים (נדה סו, א), ובאותן שבעה שהיו מתקנת בנות ישראל הוא שהיה מיקל בכך.\par \textbf{} ועדיין קשה דכל שלא טבלה הרי היא בנדתה וכדאמרינן (לקמן שבת סד, ב) שבעת ימים תהיה בנדה בנדתה תהיה עד שתבא במים. ויש אומרים שזה מדרש חכמים הוא ולא היה יודעו. ואינו נכון, מי ששנה הרבה ושמש תלמידי חכמים הרבה שלא היה יודע מדרש חכמים. ועוד דהא משמע דטבילת נדה וזבה מפורשת היא מן הכתוב, דכתיב (במדבר לא, כג) אך במי נדה יתחטא, מים שהנדה טובלת בהם (חגיגה כג, ב). ועוד דכתיב (ויקרא טו, יג) וכי יטהר הזב מזובו וספר לו שבעת ימים ורחץ בשרו במים חיים וטהר, וכתיב בתריה בזבה (שם כח) ואם טהרה מזובה וספרה לה שבעת ימים ואחר תטהר, כלומר: אחר כך תטהר בטהרה שהזכיר למעלה בזב, ואף על פי שטבילת הזב במים חיים ושל זבה במי מקוה, שמא ממקום אחר מיעטו.\par \textbf{} אלא המחוור, שבאותן הימים לאחר תקנת בנות ישראל כשהיו משלימות שבעת ימי נדה דאורייתא היו טובלות וטהורות דבר תורה, ואחר כך סופרות שבעה נקיים, והיו נוהגות כן מפני תקנת הטהרות שמא תגע בטהרות, ובאותן הימים היה אותו תלמיד מיקל הואיל וכבר טבלה וטהורה היא דבר תורה.}
\textblock{\textbf{האוכל אוכל ראשון ואוכל שני.} פירש רבנו תם ז״ל: שאין זה פסול הגויה, דההיא בדורות הראשונים נתקנה, ובמסכת יומא (פ, ב) נסתפקא להו בפסול הגויה אי דאורייתא, וההוא נמי אינו פוסל במגעו אלא באכילתו, וכדאמרינן נפסל גופו מלאכול בתרומה. וטעמא דההיא משום דנגעי בהדדי במעיו וגזרו משום מגע תרומה באוכלין טמאים, ולהכי בעי אכילת [חצי] פרס מפני שנתמעט בעיכול עד כביצה ואילו אכל פחות (מאכילת) [מכחצי] פרס היה מתמעט יותר מכביצה ואינו מטמא. ואתו תלמידי שמאי והלל וגזרו אף על כביצה ושיפסול תרומה אף במגעו.}
\textblock{\textbf{והבא ראשו ורובו במים שאובין וטהור שנפלו על ראשו ועל רובו שלשה לוגין מים שאובין.} פירש הרמב״ן ז״ל: דאף טהור קאמר וכל שכן בטבל ועלה דהיא היא עיקר הגזירה והכל בכלל טהור, אלא שבביאת מים לא גזרו אלא בטבל ועלה, לפי שאילו גזרו אף על טהור ממש אין רוב הצבור יכולין לעמוד, שא״כ אין לך רוחץ במים שאובים. וא״ת א״כ אי לא הא לא קיימא הא כדאמרינן בסמוך (לקמן שבת יד, א). לא היא, דעיקר גזירה משום מים שאובים היא    ובההיא הוא דחיישינן שמא תבטל, אבל בביאת מים לא חיישינן בה כולי האי דאי קיימא קיימא ואי לא תבטל דלאו עיקר גזירה הוא.}
\clearpage
\newsection{דף יד}
\textblock{ הא דאמרינן:\textbf{ ושדי לפומיה ופסיל להו.} איכא למידק אם כן טמויי נמי ליטמו. איכא למימר דמשקין שבפיו מאוסין הן ואי שדו לא מטמו אחרים, אלא מיפסל פסיל להו מלאכול.}
\textblock{ הא דאמרינן:\textbf{ מהו דתימא האי שכיח והא לא שכיח.} איכא למידק ולימא משום דזימנין דאכיל אוכלין דתרומה ושקיל משקין טמאין ושדי לפומיה, דהא ודאי שכיחא. יש לומר דבכי הא ליכא למיחש, דלמאן דאכיל אוכלין טמאין איכא למיחש דילמא לא רמי אדעתיה ושדי לפומיה תרומה טהורה דלאו מיזהר זהיר, אבל לאוכל אוכלין טהורים דתרומה ליכא למיחש דההוא מזהר זהיר ולאוכל טהרות לא גזרו שמא ישכח ויטמאם. ועוד יש לומר דבהנהו לא גזרו משום דכבר נמאסו אוכלין שבפיו משלעסן ואין חוששין לטומאתן כל כך שיגזרו עליהם, אבל לאוכל אוכלין טמאין חששו משום שמא יביא תרומה טהורה ויתננה לתוך פיו, וההיא שעתא ראויה היא דעדיין לא נפסלה מאכילת אדם והרי הוא מטמא תרומה טהורה הראויה.}
\textblock{\textbf{אילימא הך גזור ברישא הא תו למה לי.} פרש״י ז״ל (בד״ה אלא) והלא גזרו סמוך לנטילתן מגע תרומה. וא״ת אכתי איצטריך משום נוטל ידיו לאכילה ומשמרן דנפקי להו מסתם ידים, ואפילו הכי כשבאות מחמת ספר מטמו. יש לומר דבכי הא אפילו נגעו בספר לא מטמו, דלא אמר רבי פרנך אלא מחמת נוגע בספר בסתם ידיו שמא מסואבות הן.\par \textbf{} והרמב״ן ז״ל (בד״ה אילימא) הקשה ע״ז דא״כ למה נחלקו התנאים האחרונים כר״מ ור׳ יהודה (למה נחלקו) במסכת ידים (פ״ג מ״ה) איזה ספר מטמא את הידים ומאי נפקא להו מינה, דמכל מקום כל סתם ידים מטמו את התרומה מתקנת הראשונים מתלמידי שמאי והלל דקדימי להו טובא. ופירש הוא ז״ל דהכי קאמר: אילימא הא דסתם ידים גזרו תחילה הא תו למה לי למיגזר במגע ספר, אי בסתם ידיו הא תקנו ואפילו בלא מגע ספר, ואי אפילו במשמר ידיו למה להו למיגזר בכי הא גזירה בפני עצמה, דההיא דרבי פרנך ליתא אלא משום היסח הדעת ושהידים עסקניות הם. ופריק אלא הך דספר גזרו תחילה וגזרו סתם ולא חלקו בין סתם ידים למי ששמר ידיו, ואחר כך הוצרכו לגזור משום תרומה אפילו בסתם ידים שלא באו מחמת ספר, ותקנה ראשונה כבר פשטה בישראל ולא זזה ממקומה, ולפיכך הוצרכו האחרונים לגלות איזה ספר מטמא את הידים, שאפילו ידים משומרות הבאות מחמת ספר פוסלות את התרומה.}
\textblock{\textbf{אילימא במשקין הבאין מחמת שרץ דאורייתא היא.} אף על גב דפלוגתא היא בפסחים (טו, ב) טומאת משקין לטמא אחרים דאורייתא או דרבנן. איכא למימר דהכא נקט לה אליבא דמאן דאמר התם דאורייתא, ונקיט לה בדרך הניחא למאן דאמר דרבנן אלא למאן דאמר דאורייתא מאי איכא למימר, כנהוג בהרבה מקומות בתלמוד. עוד יש לפרש דההיא אפילו למ״ד לאו דאורייתא תקנת הנביאים היתה דכתיב (חגי ב, יב) הן ישא איש בשר קדש וגו׳, ולא משכחת לה מעלות דרביעי בקדש ושלישי בתרומה אלא על ידי משקין טמאין כדאיתא התם (יז, א), ולפיכך קרי ליה הכא דאורייתא כאילו היא דאורייתא ולא מגזירת שמונה עשר דבר.}
\textblock{      \textbf{אילימא במשקה הזב דאורייתא הוא.} איידי דאקשינן לעיל דאורייתא היא נקט לה נמי הכא, ואפשר היה לאקשויי ההיא טמויי נמי מיטמו.}
\textblock{ הא דאמרינן:\textbf{ שמונה עשר דבר גזרו ובשמונה עשר נחלקו.} מדברי רש״י ז״ל נראה שבאותן שגזרו בהן הוא שנחלקו תחילה, אלא שרבו בית שמאי על בית הלל וגזרו בהם ולמחר חזרו בהם בית הלל והושוו להן והודו למנינן. ויש מפרשים שבשמונה עשר גזרו ובשמונה עשר אחרות נחלקו בו ביום ולמחר הושוו. ולפי דבריהם לא גרסינן הכא והתניא הושוו בו ביום נחלקו ולמחר הושוו, דלפי גירסא זו לא היו אלא שמונה עשר דבר בלבד ובהם נחלקו ובהם הושוו, אלא הכי גרסינן שמונה עשר דבר גזרו ובשמונה עשר נחלקו ובשמונה עשר הושוו. ושמונה עשר שגזרו הם שמפורשים והולכים בשמועתנו, ושמונה עשר שהושוו הרי שמונה יציאות והכנסות דריש פירקא דעני ודבעל הבית, תשעה ספר, עשרה מרחץ, י״א בורסקי, י״ב לאכול, י״ג לדין, י״ד חייט, ט״ו לבלר, ט״ז מפלה כליו, י״ז קורא, י״ח לא יאכל הזב, ושמונה עשר נחלקו אלו שבסוף המשנה והם, דיו, סממנין, כרשינין, אונין, צמר, חיה, עופות, דגים, מוכרין, טוענין, מגביהין, עורות, כלים, הרי י״ג, והחמשה הנשארים הם המוזכרים בברייתא לא ימכרנו, לא ישאילנו, לא ילונו, לא ימשכננו, ולא יתן לו אגרות, הרי חמשה.}
\clearpage
\newsection{דף טו}
\textblock{ הא דתנן:\textbf{ שמאי אומר מקב חלה.} כתבו משמו של ר״ת ז״ל, דכיון דכתיב (במדבר טו, כ) חלה תרימו תרומה ותתנו לכהן, בעינן שיהא בה כדי נתינה, ואין נתינה פחותה מכביצה, וכדקיימא לן דשעור אוכלין בכביצה, ושמאי אזיל בתר דעתא דבעל הבית שהוא מפריש אחד מכ״ד, נמצא מפריש מקב כביצה, והלל אזיל בתר דעתו של נחתום שהוא מפריש אחד משמונה וארבעים, ונמצא מפריש מקביים כביצה.}
\textblock{\textbf{מלא הין מים שאובין פוסלין את המקוה שחייב אדם לומר בלשון רבו.} פירש הראב״ד ז״ל: דטעמיה דהלל מפני שהיא המדה הגדולה הנאמרה בתורה, כדכתיב (שמות ל, כד) ושמן זית הין. ואף על פי שנאמרו בתורה מדות קטנות, כיון דמים שאובין לפסול את המקוה דרבנן אזלינן לקולא ולא מיפסל אלא בשיעורא רבה, ולפיכך שנה לו רבו הין, לגלות לו שמפני שנאמרה בתורה הוא פוסל את המקוה. ושמאי סבר תשעה קבין לפי שהן ראויין לשטיפת כל הגוף ועזרא תקנם לבעלי קריים, לפיכך חשיבו כמקוה פסול ופוסלין. וחכמים אומרים שלשת לוגין מפני שהן חשובין, שנתן הכתוב לקרבנות צבור מדה זו שהיא רביעית ההין ואזלינן בתר שיעורא זוטא. ואף על גב דאשכחן בה לוג שמן, בצבור מיהא לא אשכחן פחות משלשת לוגין, ולא אשכחן לוג אלא שמן, אבל יין אין פחות מרביעית [ההין] שהוא שלשת לוגין.}
\textblock{ הא דאמרינן: \textbf{הא ניחא למ״ד לא לכל הטומאות אמרו אלא לטומאת מת בלבד.} פלוגתא היא דרשב״ג ורבנן (וסיפא) [בסיפא] דהך מתני׳ גופא היא במסכת כלים (פי״א מ״א), דתנן התם כלי מתכות פשוטיהן ומקבליהן טמאין, נשברו טהרו, חזר ועשה מהן כלים חזרו לטומאתן הישנה, רשב״ג אומר לא לכל הטומאות אמרו אלא לטומאת נפש.}
\textblock{ והא דאמרינן נמי (לקמן שבת יז, א):\textbf{ הניחא למ״ד כלי טמא חושב משקה.} פלוגתא היא דר״מ ור׳ יוסי בתוספתא דמסכת מכשירין (פ״ב, ה״ב), דתניא התם עריבה שירד דלף לתוכה המים הניתזין והצפין בכי יותן, נטלה לשפכה בית שמאי אומרים בכי יותן ובית הלל אומרים אינן בכי יותן, במה דברים אמורים בטהורה אבל בטמאה הכל מודים שהם בכי יותן דברי ר״מ, ר׳ יוסי אומר אחת טהורה ואחת טמאה בית שמאי אומרים בכי יותן ובית הלל אומרים אינן בכי יותן.}
\textblock{ הא דתנן: \textbf{המניח כלים תחת הצנור אחד כלים גדולים ואחד כלים קטנים.} פירש רש״י ז״ל: דרבותא משום כלים קטנים, דלא תימא מחמת קטנן אינן נחשבים כלים. ואחרים פירשו, כלים גדולים אפילו העשויין לנחת שאין מקבלין טומאה. והיינו רבותא, דלא תימא אין שם כלי עליהם, כדקתני נמי כלי גללים, ואין צריך לומר קטנים.\par \textbf{} וצנור זה שאמרו כאן שקבעו ולבסוף חקקו, אבל בשחקקו ולבסוף קבעו תיפוק ליה משום צנור דהוא גופיה      כלי הוא ופוסל את המקוה, כדאיתא בפרק המוכר את הבית (בבא בתרא סו, א). ועוד יש לפרש בצנור שאין לו בית קבול, שהוא מפולש ופתוח משני צדדיו כרעפים הללו שלנו או מן הצד האחד, שכל כיוצא בזה אינו פוסל את המקוה לפי שאינו עשוי לקבלה, וכדתנן בפרק ד׳ דמקואות (משנה ג׳) סילון שהוא צר מכאן ומכאן ורחב באמצע אינו פוסל את המקוה לפי שאינו עשוי לקבלה. והוא שלא יהיה לו חקק באמצעיתו לקבל אפילו צרורות, שאילו כן כלי העשוי לקבלה הוא ופוסל את המקוה, וכדתנן (שם) החוטט לקבל צרורות פוסל את המקוה.}
\clearpage
\newsection{דף יז}
\textblock{\textbf{טמאוהו משום כלים המאהילים על המת.} פירש רש״י ז״ל: טמאוהו טומאת ערב, ומאן דחזא טעה וסבר דמשום טומאת אהל טמאוהו טומאת שבעה מדין אהל, ור׳ טרפון אמר דלא טמאוהו אלא טומאת ערב משום דנגע במרדע בעצמו ולא משום אהל. ויש מקשים והלא המרדע פשוטי כלי עץ הוא ואינו מקבל טומאה, ואי בשל מתכת מן הדין טמא טומאת שבעה דחרב הרי הוא כחלל. וניחא להו דמשום נקבות שבו נכנס חרב המרדע נעשה בית קבול לחרב, ובית קבול העשוי למלאת בית קבול הוא, כדאמר בסוכה (יב, ב) גבי מסככין בזכרים ואין מסככין בנקבות. ואי נמי יש בו חרב המרדע, בטל הוא לגבי המרדע, שהמרדע עיקר וחרב שבו משמש המרדע, וכדתנן (כלים פי״ג, מ״ו) מתכת המשמשת את העץ טהור, ולפיכך אין אומרין בה חרב הרי הוא כחלל.\par \textbf{} והרמב״ן ז״ל הקשה דאפילו בכלי עץ כל שהוא מטמא באהל המת או בנגיעתו באב הטומאה שבמת אומר בו חרב הרי הוא כחלל ומטמא טומאת שבעה, כדמוכח במסכת אהלות (פ״ג מ״ג). אלא טמאוהו טומאת שבעה קאמר, אלא שטעה השומע וסבור שמחמת אהל טמאוהו, ולא טמאוהו מחמת שהאהיל עליו המרדע אלא מחמת שהמרדע נטמא מחמת שהאהיל על המת וטימא את האיכר בנגיעתו, ורבי עקיבא פירש דאפשר דמשום אהל טמאוהו כיון שהוא נושא אותו, אבל לאדם אחר וכלים לא. ואם תאמר אם כן מאי נפקא לן מינה כיון שהוא מטמא טומאת שבעה בין כך ובין כך. תירץ הוא ז״ל דדילמא נפקא מינה לנזיר, שאילו משום אהל הנזיר מגלח עליו ומזה שלישי ושביעי, ואילו משום מגע אמרינן במסכת נזיר (נד, ב) בהדיא אטו מאן דנגע בכלים בר הזאה הוא, בתמיהא. ורבי טרפון הודה לו לר׳ עקיבא, וכמו שאמר לו בכמה מקומות (קדושין סו, ב) עקיבא כל הפורש ממך כפורש מחייו אני שמעתי ושכחתי ואתה יושב ודורש ומסכים להלכה. והיינו דלא בצרי ליה לר׳ טרפון.\par \textbf{} ואיכא נסחאי דמקשו לר׳ טרפון בצרי להו, ופריק אמר רב נחמן בר יצחק בנות הכותיים נדות מעריסתן בו ביום גזרו. ולפירושו זה סבירא ליה לר׳ טרפון בכלים שתחת הצנור כר״מ דאמר נמנו ורבו בית שמאי על בית הלל.}
\textblock{\textbf{גזירה משום הנושכות.} פירש רש״י ז״ל: אשכולות הנושכות זו את זו וכשבא להפרידן נסחט המשקה עליהן, וכיון דעביד בידים ולא אפשר בלא סחיטה מכשר. ור״ח ז״ל גורס: הנשוכות, ופירש: כשנושך האדם מן האשכול נוטפין ממנו משקה. והרמב״ן ז״ל פירש משמו של הגאון ז״ל: כשאדם בוצר כרמו יש מהם שהגרגרים שלהם מדובקים זה לזה ונושכין זו את זו מפני דיבוקן, ואף על פי שמשקה יוצא מהן אינו הולך לאבוד שהמשקה עומד נשמר בדיבוק אותן הגרגרים ואינו נופל בקרקע, ומשום הכי הוכשר.}
\textblock{ מתני׳:\textbf{ אין פורשין מצודות חיה ועוף.} פירוש: אף על פי שאינו צד בידים אפילו הכי גזרו דדילמא אתי ביה לידי חיוב חטאת, דפעמים שבשעת פרישתו ילכד חיה או עוף וחייב, כדתניא בתוספתא (פי״ג, ה״ה) הפורש מצודה לבהמה או עוף אם נכנסו לתוכה חייב, כך כתבו בתוס׳. ואין אנו      צריכין לכך אלא לרבה דקא מפרש טעמא דבית שמאי משום דכל מלאכה שהוא חייב עליה חטאת גזרו עליה מערב שבת, אבל לרב יוסף דמוקי פלוגתייהו בשביתת כלים אי דאורייתא אי לא, לא צריכי להך, דכל שמצודתו פרושה ולוכדת בשבת אסור משום שביתת כלים.}
\textblock{\textbf{ואונין של פשתן.} לרב יוסף דוקא בתנור מטלטל, הא בקרקע מותר שאין מוזהר על שביתת קרקעו.}
\textblock{\textbf{בכדי שיצודו.} ואם תאמר והיאך ידע מתי תבא החיה או העוף. ויש לומר דפורש במקום גדודי חיות דמסתמא יכנסו שם לזמן מועט. והכי איתא בירושלמי (בפרקין ה״ז) דמוקי [לה] בפורש בחורשין.}
\clearpage
\newsection{דף יח}
\textblock{ גמ׳:\textbf{ אבל דיו דלאו בר גיבול הוא מנתינת מים הוא דלחייב.} מהכא משמע דדיו לאו בר גיבול הוא, ולקמן בפרק במה מדליקין (שבת כג, א) משמע דבר גיבול הוא, דאמרינן התם כל השמנים יפים לדיו ואיבעיא להו לעשן או לגבל. ויש לפרש דהתם לאו גיבול ממש קאמר, אלא עירוב. ואיכא למידק דהכא משמע דטפי איכא לחיובי היכא דלאו בר גיבול הוא, וקיטמא דלאו בר גיבול הוא לכולי עלמא מנתינת מים מיחייב, וביו״ט פרק המביא (ביצה לב, ב) משמע איפכא, דאמרינן התם קיטמא שרי, ופירשו התם רוב המפרשים קיטמא שרי ליתן בה מים הואיל ולאו בר גיבול הוא. על כן פירש ר״ת דהכי קאמר קיטמא בלא מים לשרוק פי התנור שרי, וקא משמע לן דלא גזרינן דילמא אתי למיגבל.}
\textblock{\textbf{ואין נותנין חטים לתוך הרחים של מים אלא כדי שיטחנו.} פירוש: אפילו לרחים של מים, וכל שכן ברחים שמוליכין בהמות דאדם מצווה על שביתת בהמתו לכו״ע.}
\textblock{\textbf{והשתא דאמרת בית הלל אית להו שביתת כלים כו׳.} פירש רש״י ז״ל דהא על כרחיך הא ברייתא ב״ה היא מדשרי בהנך וקתני דריחים אסור משום שביתת כלים, מוגמר וגפרית מאי טעמא שרו. ולא ידעתי מה ראה רבנו ז״ל לפרש כן, דהא אפשר דברייתא לרב יוסף ב״ש היא, ופלוגתא דמתניתין בשביתת כלים היא דב״ש אית להו וב״ה לית להו וכדאיתא במסקנא. אלא מסתברא דמאי דאמרינן הכי בהדיא ב״ה אית להו שביתת כלים, היינו משום דקא סלקא דעתך דכיון דלאוקימתא דרבה אתיא ברייתא כב״ה, אם איתא דב״ה לית להו שביתת כלים לא הוה מפיק לה רב יוסף לברייתא לבר מהלכתא ולאוקומיה בשביתת כלים כב״ש, אלא מסתמא ס״ל לרב יוסף דב״ה אית להו שביתת כלים וברייתא ב״ה ולא ב״ש, ופלוגתא דמתני׳ באונין של פשתן וצמר ליורה לאו בשביתת כלים היא דהא אפילו ב״ה אית להו שביתת כלים, אלא פלוגתייהו משום גזירה היא במלאכות דאורייתא הנגמרות בשבת דב״ש אית להו וב״ה לית להו, וכיון שכן ברייתא דקתני פותקין מים לגנה ב״ה ולא ב״ש. ואפשר נמי דלרב יוסף ברייתא כולי עלמא היא וכולהו לית להו גזירה, אלא בשביתת כלים גופא פליגי, דב״ה לית להו שביתת כלים אלא היכא דקעביד הכלי מעשה וכדאמרינן נמי בגמ׳ בסמוך, וב״ש אית להו אפילו לא קא עביד מעשה כאונין של פשתן דמתניתין וכדאמרינן בגמרא למסקנא.\par \textbf{} ויש לפרש עוד דעל כרחיך ברייתא ב״ה ולא ב״ש, דאי פלוגתא דמתניתין בגזירה ברייתא לאו ב״ש מדשריא כל הנך, ואי פלוגתא דמתניתין בשביתת כלים ברייתא לאו ב״ש מדשריא מוגמר וגפרית וקס״ד השתא דמנחי בכלים, ואילו במתניתין קא אסרי אפילו אונין דלא קא עביד כלי מעשה, אלא על כרחין ב״ה היא. ומיהו אכתי לא ניחא לי דאי מדיוקא דמתני׳ קא דחינן לברייתא מדב״ש לידוק מינה נמי דליתא אפילו כב״ה, מדקא שרי במתני׳ מצודות חיה ואף על גב דקא עביד מעשה, דהשתא לא ס״ד דמיירי דוקא בלחי וקוקרי, ואי מתרצת לב״ה בלחי וקוקרי תרצה נמי לב״ש במוגמר וגפרית דמנחי אארעא, ויותר הוא רחוק לדחות מתניתין דמצודות ולהעמידה דוקא בלחי וקוקרי ולאפוקי מפשיטותא דמצודות כל מצודות במשמע, אבל במוגמר וגפרית הא לא דחקינן כלל כי מוקמינן לה דמנחא אארעא, דהא לא קתני אלא מניח מוגמר תחת הכלים ומתגמרין והולכין כל השבת, דמשמע לכאורה דכל עיקרא דמילתא לאו משום טרידתא דכלים המתגמרין אתיא לה אלא משום האי דיהיב המוגמר בידיה, והלכך לאו דחויה היא כי מוקמת לה במנחי אארעא. כך נראה לי.}
\textblock{\textbf{השתא דאמר רב הושעיא וכו׳.} וא״ת כיון דשביתת כלים ב״ש ולא ב״ה, מאי טעמא אפקה רב יוסף לברייתא מדב״ה ואוקמה כב״ש. תירצו בתוס׳ משום דלא ניחא ליה לדחוקי כולי האי לאוקומה בהשמעת קול. ולי נראה משום דברייתא קשיתיה דקתני כל הני ואי משום מלאכות הנגמרות בשבת ניתני חדא, ועוד קלור ואספלנית למאי תני פשיטא      דאפילו בשבת ליכא חיוב חטאת, הלכך אוקמה רב יוסף בשביתת כלים, וברייתא מנקטא סירכיה ואזלא, וסידורא דברייתא הכין, דמעיקרא אשמעינן דפותקין מים לגנה ואף על גב דאתעבידא מלאכה בגופה של קרקע שרי דלא אסרה תורה אלא בכלים ולא בקרקעות, והדר אשמעינן דאפילו בכלים לא אסרה אלא כשהמלאכה נעשית בגופו של כלי אבל לא כשהכלי מתפעל כמוגמר וגפרית שהן מונחים בקרקע והן מתגמרין ומתגפרין, והדר אשמעינן שריותא דבגופו של אדם דהיינו קילור, ולא גזרינן משום שחיקת סממנין אף על פי שהן מרפאין והולכין בשבת ואף על גב דבהנחתם בשבת גזרינן. והדר אשמעינן מאי דאסרה תורה, דהיא נתינת חיטין לריחים ומשום שביתת כלים. כך נראה לי.\par \textbf{} ולענין נתינת חטים לריחים, ר״ח ז״ל פסק הלכתא כרבה דאמר מפני שמשמעת את הקול, משום דרבה ורב יוסף הלכתא כרבה. והלכך אף על גב דקיימא לן דשביתת כלים לאו דאורייתא, בנתינת חטים לריחים אסור משום השמעת הקול וברייתא כולה בית הלל, ופלוגתא דבית שמאי ובית הלל דמתניתין במלאכות שחייבין עליהן חטאת הנגמרות בשבת היא ובגזירה פליגי, בית שמאי סברי גזרינן ובית הלל סברי לא גזרינן.\par \textbf{} ויש לי להביא ראיה לדברי ר״ח ז״ל, דהא תניא בתוספתא כדעתיה דרבה, דתניא התם בפרק קמא דמכלתין (ה״ט) אמרו להם ב״ש לב״ה אי אתם מודים שאין צולין בשר בצל וביצה ערב שבת עם חשיכה אלא כדי שיצולו אף דיו וסממנין וכרשינין כיוצא בהן, אמרו להם ב״ה אי אתם מודים שטוענין בקורת בית הבד ותולין עגולי הגת ערב שבת עם חשיכה אף דיו וסממנין וכרשינין כיוצא בהן, אלו עמדו בתשובתן ואלו עמדו בתשובתן, אלא שב״ש אומרים ששת ימים תעבוד ועשית כל מלאכתך (שמות כ, ט) מלאכתך גמורה מע״ש, ובית הלל אומרים ששת ימים תעשה מלאכה (שמות לה, ב) עשה אותה כל ששה, אלמא לאו בשביתת כלים פליגי כדברי רב יוסף אלא במלאכות הנגמרות בשבת כדברי רבה. ולקמן נמי בגמרא (יט, א) גבי טוענין בקורת בית הבד איכא ספרים דגרסי מאי שנא בכולהו דפליגי ומאי שנא הכא דלא פליגי ומשני התם אתי בהו לידי חיוב חטאת הכא לא אתי בהו לידי חיוב חטאת, וכן גירסת רש״י ז״ל, אלמא סוגיין בגמרא כרבה.\par \textbf{} אבל רב אלפסי ז״ל ורבותינו בעלי התוספות ז״ל פסקו הלכתא כרב יוסף דמוקי לה בשביתת כלים, ומדאמר רב הושעיא דבית הלל לית להו שביתת כלים אידחייא הא דאין נותנין מהלכתא דבית שמאי היא. ויש לסייע סברא זו מדאמרינן להדיא והשתא דאמר רב הושעיא מאן תנא שביתת כלים ב״ש היא מוגמר וגפרית מאי טעמא שרי ב״ש, אלמא לרב הושעיא לא מתוקמא ליה אלא בשביתת כלים וב״ש היא, דאי לא תימא הכי מאי קושיא דילמא ב״ש מוגמר וגפרית מיסר אסרי וברייתא ב״ה היא וטעמא דאין נותנין חטים לריחים משום השמעת קול כדרבה, אלא ודאי משמע לכאורה דרב הושעיא אברייתא קאי ועלה קאמר דמשום שביתת כלים היא וב״ש היא. אלא שיש לדחות דדילמא הכי קאמר, לטעמיה דרב יוסף דמוקי לה לברייתא משום שביתת כלים השתא דאמר רב הושעיא דשביתת כלים ב״ש היא ועל כרחיך ברייתא ב״ש היא, אם כן מוגמר וגפרית מאי טעמא שרו.}
\textblock{\textbf{גיגית נר וקדרה לב״ש אפקורי מפקר להו.} ואף על גב דבעלמא בעי הפקר בפני שלשה, הכא דלאפרושי מאיסור הוא לא בעי דמסתמא מפקר להו בגמר דעת. ולי נראה דמשום הפקר בית דין נגעו בה דלב בית דין מתנה      עליהם להפריש העם מאיסור, דאי לא תימא הכי ודאי כולי עלמא לאו דב״ש ידעי וכל שכן במקום ב״ה.\par \textbf{} קשיא לי בשלמא אליבא דרב יוסף ניחא דאיכא לתרוצי משום הפקר, אלא לרבה דמפרש טעמא דב״ש משום גזירה במלאכות הנגמרות בשבת, אי הכי גיגית וקדרה לב״ש מאי טעמא שרו. ויש לומר דאף לרבה הוי טעמא משום הפקר, כלומר: שהן מפקירין התבשיל עצמו ובדבר המופקר לא גזרו. ודחוק הוא. ואלא יש לומר דלרבה ניחא דכיון שאין האיסור אלא מדבריהם ומשום גזירה בנר וגיגית דלא אפשר לא גזרו, ובקדרה נמי לא החמירו לבשל לגמרי מבערב מפני שהיא גזירה שאין רוב הצבור יכולין לעמוד בה, ושפוד שאמרו במתניתין היינו משום דגזירתו קרובה ומצויה דשמא יחתה בגחלים, אבל לרב יוסף דמוקי לה בשביתת כלים ודאורייתא הוא לדידיה הוא דקשיא להו גיגית נר וקדרה היכי שרו לה ב״ש. כך נראה לי.}
\textblock{\textbf{מאן תנא הא דת״ר לא תמלא אשה קדרה עססיות ותורמסין וכו׳ ערב שבת עם חשיכה וכו׳ לימא ב״ש היא.} פירש הרמב״ן: שהתורמסין והעססיות אין צריכין בישול הרבה, ולפיכך העמידוה במסקנא ככולי עלמא ומשום גזירת חתוי, שאף על פי שלא בשלו כל עיקר הרי הן כתבשיל שלא בשל כל צרכו ודעתו עליהן לאכלן לאלתר. וכמדומה שהזקיקו לומר כן, משום דתנא כיוצא בו לא ימלא נחתום קיתון של מים, והמים ודאי אין צריכין אלא בשול מועט. אבל קשה לי, שאם כן היכי הוה מצי לאוקומה אפילו כבית שמאי מעיקרא, דמאי שנא מגיגית נר וקדרה, הכא נמי לישתרי דאף הני אפקורי מפקרי להו.\par \textbf{} אלא הפירוש הנכון כמו שפירש רש״י ז״ל, שהן צריכין בשול הרבה, לפיכך אף על פי שהן חיין גזרינן שמא יחתה בגחלים לאכלן למחר לפי שאין כל היום וכל הלילה די להם, וקיתון של מים כיון שהוא מניחן כדי ללוש בהן למוצאי שבת גזרינן שמא יחתה בהן בשבת כדי שיהיו לו חמין מוכנים ללוש בהן למוצאי שבת מיד. ולפירושו נראה לי שהיה סבור מתחילה דאתיא כב״ש דוקא דאית להן שביתת כלים, וקא סלקא דעתך דטעמא משום דכיון שהן קשין להתבשל והוא מניחן ערב שבת עם חשיכה אינו רוצה לאכלן בשבת אלא למוצאי שבת, ודומיא דקיתון של מים דבעי ליה למוצאי שבת, והלכך לא מפקרי להו, כלומר אין לב ב״ד מתנה להפקיר כלים כדי להתיר לבשל לצורך חול אלא כדי להתיר בשול הצריך לשבת, כך נראה לי. וגם זה ראיה למה שפירשתי (לעיל ד״ה גיגית) דאפקורי מפקרי להו ב״ד קאמר ולא בעל הנר והקדרה, שאם כן אף כאן נאמר שהוא מפקירן ולישרי ולא אתי לא כב״ש ולא כב״ה.}
\textblock{ ואסיקנא:\textbf{ גזירה שמא יחתה בגחלים.} ועססיות ותורמסין דוקא, ולא דמי לקדרה חייתא כמו שפירש״י ז״ל לפי שאין כל היום וכל הלילה די להם. אבל לרבנו האי גאון ז״ל מצאתי שכתב בפרק כירה דקדרה חייתא דשריא לאו חייתא ממש קאמר, שאילו צונן אסור להניחן בין השמשות והיינו טעמא דעססיות ותורמסין, אלא קדרה חייתא זו היא שחמה דלא התחילה לבשל, עד כאן. ואינו עולה יפה לפי דעתי.}
\textblock{\textbf{אי הכי מוגמר וגפרית ליגזור.} ואם תאמר לדידיה דלא מסיק אדעתיה האי טעמא דגזירת חתוי, היכי ניחא ליה מתניתין (לקמן שבת יט, ב) דקתני אין צולין בשר בצל וביצה ופת וחררה. יש לפרש דהוה סלקא דעתך דטעמא דהנך משום דלא מפסיק מידי בינם לבין הגחלים איכא למיגזר טפי דלמא אתי לחתויי.}
\textblock{\textbf{צמר ליורה ליגזור אמר שמואל ביורה עקורה.} ואפילו הכי בין לרבה בין לרב יוסף קא אסרי ב״ש, משום דכלי ראשון מבשל, וכדתנן (לקמן שבת מב, א) האלפס והקדרה שהעבירן מרותחין לא יתן לתוכן תבלין, והלכך איכא למיגזר משום שמא יבשל בשבת אליבא דרבה, ואיכא נמי משום שביתת כלים אליבא דרב יוסף.}
\textblock{      ואקשינן:\textbf{ ודילמא מגיס.} ואוקימנא:}
\textblock{\textbf{ ביורה עקורה וטוחה.} ומיהו בשקלט את העין איכא למימר דמותר אפילו בשאינה עקורה וטוחה דומיא דקדרה היכא דבשיל דשרי ולא גזרינן משום חתוי דגחלים, ומשום מגיס נמי ליכא דמשקלט את העין ליכא משום מגיס. ואיכא למימר דדוקא טוחה הא לאו הכי לעולם אסור גזירה משום מגיס, ולא דמי לקדרה דדרכן של סממנין להגיס תדיר כדי שלא יחרכו. וצריך עיון.\par \textbf{} ומהא דקא אמרינן ביורה עקורה וטוחה משמע דאפילו בעקורה איכא משום מגיס, דהגסה בכלי ראשון כבישול. וקשיא לי אם כן האלפס והקדרה שהעבירן מרותחין היאך מוציאין מהן בכף והלא מגיס. ויש לומר דבהגסה ראשונה הוא דאיכא משום מבשל לגבי קדרה מפני שמערב את הכל ואיכא משום קרובי בשולא אבל בשאר הגסות לא, דמראשונה ואילך ליכא בקדרה משום מבשל, והיינו דמפרשינן טעמא דעססיות ותורמסין משום מחתה בגחלים ולא אמרינן משום מגיס, והיינו נמי דמשהין על גבי כירה קטומה ולא חיישינן דילמא מגיס, אבל יורה של סממנין הוא דאיכא משום מגיס דדרכן בכך להגיס תדיר כדי שלא יחרכו כדכתיבנא. וכן ראיתי להרמב״ן ז״ל שכתב דמסתבר ליה דליכא משום מבשל מהגסה ראשונה ואילך. ועוד יש לי לומר דכל דבשיל כמאכל בן דרוסאי ליכא משום מגיס, דהא מגיס משום מבשל הוא דמחייב וכיון שהגיע למאכל בן דרוסאי תו ליכא משום מבשל, וכענין שאמרו בפרק כירה (לקמן שבת מ, ב) שמן אין בו משום בשול, שמותר לתת אותו אפילו במקום שהיד סולדת בו.}
\textblock{\textbf{האי קדרה חייתא אי בשיל שפיר דמי.} פירוש: בשיל כמאכל בן דרוסאי, בשיל ולא בשיל כלומר: שלא הגיע למאכל בן דרוסאי, וכדאמרינן נמי לקמן (שבת כ, א) חנניה אומר כל שהגיע למאכל בן דרוסאי מותר לשהותו על גבי כירה שאינה גרופה ואינה קטומה, כן פירש רבנו האי ז״ל בפרק כירה, וכן פירשו בתוס׳. אבל דעת הרב אלפסי והרמב״ם ז״ל (פ״ג מה׳ שבת ה״ד) בשיל ולא בשיל היינו משהתחיל לבשל ועד שנתבשל כל צרכו, ואפילו נתבשל כל צרכו אם הוא מצטמק ויפה לו אסור. וכן דעת הרמב״ן ז״ל.\par \textbf{} ומה שאמרו לקמן (שבת כ, א) גבי אין צולין בשר בצל וביצה אלא כדי שיצולו מבעוד יום, וכמה, א״ר אלעא אמר רב כדי שיצולו כמאכל בן דרוסאי, התם דוקא בשהם עצמן מונחין על גוף האש, וכן החררה פניה כנגד פני האש, דכיון שהגיעו למאכל בן דרוסאי אם יחתה בגחלים יחרך אותן, וכ״כ בפירוש הרמב״ם ז״ל (בפ״ג מהלכות שבת הט״ז). וזהו שכתב הרב אלפסי ז״ל לאותה דרב, אע״פ שפסק בפרק כירה שאסור לשהות על גבי כירה שאינה גרופה ואינה קטומה ודלא כחנניה.}
\textblock{\textbf{דברחא ולא שריק אסור.} פירוש: ואפילו חי, ובצלי מיירי דלא מסיח דעתיה מיניה משום דמתבשל מהרה ואתי לחתויי. וכן כתב הרב בעל ההלכות ז״ל דמכאן ואילך מיירי בצלי, והיינו דמייתינן עלה אין צולין בשר וביצה דמיירי בצלי.}
\textblock{\textbf{דגדיא ולא שריק דברחא ושריק רב אשי שרי ורב ירמיה מדיפתי אסר.} וקשיא לי, לרב ירמיה קשיא אונין של פשתן, דהתם לא שריק ומשום דקשו ליה זיקא שרו ב״ה. ושמא נאמר דאונין של פשתן קשי ליה זיקא טפי. ואי נמי מוקי לה איהו בשטח פי התנור.}
\textblock{\textbf{ולרב אשי הא תנן אין צולין וכו׳.} פירוש: דסתמא קתני בשר דמשמע כל בשר, ולר׳ ירמיה ניחא דאפשר לאוקומה בין בגדיא בין בברחא, אלא לרב אשי ליכא לאוקמה אלא בברחא דוקא, ותירץ הכי נמי דמוקי לה בברחא ולא שריק.}
\textblock{\textbf{כי פליגי דברחא ולא שריק רב אשי שרי ורב ירמיה אסר.} ואם תאמר לרב אשי קשיא מתניתין (לקמן שבת יט, ב), דקתני אין נותנין את הפת בתנור ולא חררה על גבי גחלים. וליכא למימר דמתניתין דוקא בתנור מגולה לגמרי, דאם כן כי אקשינן לקמן (שבת כ, א) גבי משלשלין את הפסח, אלא טעמא דבני חבורה זריזין הן הא לאו הכי אסור והא אמר מר גדיא בין שריק בין לא שריק שפיר דמי, מאי קושיא, לישני ליה התם במכוסה והכא במגולה, אלא ודאי סתמא תנור מכוסה הוא. ותירצו בתוס׳ דגבי פת שייך חתוי גחלים יותר מבשאר דברים. ואינו מחוור לי.\par \textbf{} ומסתברא לי דרב אשי גופיה לא שרי בברחא אלא משום דקשי ליה זיקא קצת, אלא דרב ירמיה אסר משום דלא קשי      ליה טובא, דאי לא תימא הכי תקשי לן מתני׳ לרב אשי, מאי שנא משלשלין את הפסח דנקט דמשמע טעמא משום דבני חבורה זריזין הן, צלי בעלמא נמי שרי, אלא היינו טעמא, דפסח כיון דלא מינתח לא קשי ליה זיקא כלל אבל ברחא קשה ליה קצת. וסעד לדבר דהא אמרינן והשתא דאמר מר כל מידי דקשי ליה זיקא לא מגלי ליה וכו׳, אלא ודאי דקשה ליה קצת כדאמרן. כך נראה לי.}
\textblock{\textbf{אמר רבינא האי קרא חייא שפיר דמי כיון דקשי ליה זיקא כבשרא דגדיא דמי.} כתב רב אלפסי זכרונו לברכה: מדקאמר כבשרא דגדיא דמי, שמע מינה דהלכתא כרב ירמיה דאסר בברחא משום דקאי רבינא כוותיה. אבל הרז״ה ז״ל כתב דהכא לרווחא דמילתא קאמר, לומר דקרא לכולי עלמא שריא, ומיהו בברחא נמי דילמא רבינא כרב אשי סבירא ליה, והלכך נקטינן כרב אשי דהוא בתראי. ומצאתי לרבנו האי גאון ז״ל, וזה לשונו: ולענין מאי דקשי ליה זיקא, מסקנא כלישנא בתרא וכרב אשי דבין גדיא ובין דבר אחר בין שריק בין לא שריק שרי, עד כאן.}
\clearpage
\newsection{דף יט}
\textblock{\textbf{מהו דתימא האי רמי עליה והאי לא רמי עליה קמ״ל.} פירוש: קמ״ל דאפילו הכי שרי. ולאו למימרא דעכו״ם רמי עליה, דהא ודאי לא רמי עליה כדאמרינן במסכת ביצה (כא, ב) גבי לכם ולא לעכו״ם, מרבה אני את הכלבים שמזונותם עליך ומוציא אני את העכו״ם שאין מזונותם עליך. וא״ת א״כ היאך נותנין לפניהם מזונות אפילו בחצר, והא אמרינן בפרק מי שהחשיך (לקמן שבת קנה, ב) אין נותנין מזונות לפני החזיר בחצר לפי שאין מזונותיו עליך. יש לומר שאני עכו״ם דמשום דרכי שלום חשבינן ליה קצת כמזונותיו עליך, וכדאמר בגיטין (סא, א) מפרנסין עניי עכו״ם עם עניי ישראל מפני דרכי שלום.\par \textbf{} ומיהו משמע מהכא דדוקא ליתן לו בחצר על דעת שיאכל בחצר, הא במפרש להוציא אסור. והדין נותן, דאפילו עם חשיכה לא התירו ב״ה להשאיל או ליתן לו כלי במתנה אלא כדי שיצא מפתח ביתו מבעוד יום. ובהדיא גרסינן בירושלמי (בפרקין ה״ח) אין נותנין לעכו״ם על מנת לצאת, נטל ויצא אין זקוק לו. והראב״ד ז״ל אסר אפילו בחצר כל היכא דיכול לאשתמוטי מיניה, ולא משום הוצאה אלא משום דלא טרחינן אלא לדברים שמזונותם עליך. והא דקיימא לן (ביצה כא, ב) מזמנין את הגוי בשבת, לאו דספי ליה אלא דמחוינן ליה ושקיל איהו לנפשיה, אי נמי דספינן לדידן ואכיל איהו בהדן.}
\textblock{ הא דאמרינן:\textbf{ לא ישכיר אדם כליו לעכו״ם בערב שבת וברביעי ובחמישי מותר.} איכא מאן דמפרש, דבין לבית שמאי בין לבית הלל אתיא, ובכלים שאין עושין בהם מלאכה כגון מטה ושלחן וכסא או חלוק, וטעמא דמילתא משום דנראה כעומד ונוטל שכר שבת, אבל ברביעי ובחמישי מותר. ודוקא בהבלעה כגון שמשכיר לו לשנה או לחודש, הא שכירות יום אסור, וכדתניא במציעא בפרק הזהב (בבא מציעא נח, א) השוכר את הפועל לשמור לו את הפרה לשמור את הזרעים אינו נוטל שכר שבת, הא שכיר חודש שכיר שנה נוטל ממנו שכר שבת. ואפילו הכי בערב שבת אסור, דנראה כמשכיר לו לצורך שבת, וכעומד ונוטל ממנו שכר שבת. ודוקא לעכו״ם אבל לישראל מותר אפילו בערב שבת, דלא חשיד למיתן ליה שכר שבת, וכיון שמשכיר לו לחודש או לשבת שרי. ואינו מחוור שאם אתה מתיר להשכיר לישראל משום דלא חשיד ליתן ליה שכר שבת, אפילו לעכו״ם שרי מהאי טעמא גופה דלא חשיד ישראל דשקיל מיניה שכר שבת.\par \textbf{} ואיכא מאן דמפרש לה בכלים שעושין בהן מלאכה ומשום שביתת כלים נגעו בה ובית שמאי היא, וכן נראה מדברי רב אלפסי ז״ל שלא כתבה בהלכות. וזה אינו מחוור כלל, דאם כן הוה להו לבעלי הגמרא לפרושי בהדיא דבית שמאי הוא, ולא למיקבעה בגמ׳ סתם דמשמע דהלכתא היא, ובית שמאי במקום בית הלל אינה משנה. ועוד דאם כן אפילו ברביעי ובחמישי ואפילו בראשון נמי אסור, וכדתניא לעיל (שבת יח, ב) לא ימכור אדם חפצו לעכו״ם, וסתמא קתני ואפילו ביום ראשון.\par \textbf{} והמחוור שבכולן, דמיירי בכלים שעושין בהן מלאכה ובית הלל היא, ומשום דנראה כעומד ונוטל שכר שבת, ודוקא לעכו״ם, אבל לישראל דלא עביד בהו מלאכה ודאי שרי.\par \textbf{} אי נמי טעמא דמילתא משום דנראה כשלוחו של ישראל במלאכה שהוא עושה בהן, ואף על גב דשרו בית הלל בעורות לעבדן בשנתנן לו עם השמש, התם היינו טעמא משום דקצץ [עם] עכו״ם, דכיון דקצץ עכו״ם בדנפשיה קא טרח ואין לישראל שום ריוח במה שהעכו״ם עושה בהן בשבת, אבל שכירות כלים יש לו ריוח לישראל בעשיית מלאכתו בשבת ואפילו בהבלעה, דהדבר ידוע שאילו לא היה רשאי העכו״ם לעשות בהן מלאכה בשבת לא היה נותן לו כל כך, והלכך מתהני ביה ישראל ואסיר דמחזי כשלוחו של ישראל, ואפילו הכי דוקא בערב שבת, אבל ברביעי ובחמישי מותר דכולי האי לא אחמור כיון דעכו״ם במלאכתו הוא עוסק, ומאן דחזי נמי סבר לצורך מחר בלחוד הוא דמוגר ליה, הלכך שרי.}
\textblock{      \textbf{אמר רב ששת ה״ק אם לא קצץ בית שמאי אומרים כדי שיגיע לביתו ובית הלל אומרים כדי שיגיע לבית הסמוך לחומה.} קשיא לי, למה ליה לאדכורי הכא בית שמאי, דהא לבית שמאי ליכא מידי בין קצץ ללא קצץ. ויש לומר משום דבעי לפרושי דלבית הלל בשלא קצץ בכדי שיגיע לחומה שרי, נקט נמי בית שמאי.}
\textblock{\textbf{אין מפליגין בספינה פחות משלשה ימים קודם השבת.} פירש ר״ח ז״ל: דוקא בספינה גוששת למטה מעשרה, ומשום איסור תחומין נגעו בה, אבל למעלה מעשרה דליכא משום תחומין שרי. והקשו עליו בתוס׳, דאם כן כי איבעיא להו בעירובין בפרק מי שהוציאוהו (עירובין מג, א) אם יש תחומין למעלה מעשרה או לא, הוה להו לאתויי הא דיש תחומין למעלה מעשרה, ולישני במהלכת ברקק.\par \textbf{} והרב אלפסי ז״ל הקשה, דאם כן מאי איריא שלשה אפילו טפי נמי, ועוד לדבר מצוה אמאי שרי והא העמידו דבריהם במקום עשה, וכדתנן (פסחים צא, ב) אונן טובל ואוכל את פסחו לערב אבל לא בקדשים, ואמרינן עלה גבי פסח לא העמידו דבריהם במקום כרת גבי קדשים העמידו דבריהם במקום עשה. ולדידי נמי קשיא, דבהדיא גרסינן בירושלמי (בפירקין ה״ח) אין מפליגין בים הגדול לא בערב שבת ולא בחמישי בשבת, וים הגדול למעלה מעשרה הוא. ופירשה רב אלפסי ז״ל, משום עונג שבת, כדאיתא בהלכות. ור״י בעל התוס׳ ז״ל פירשה, משום שאסור לעבור במים במעבורת ואפילו בתוך התחום משום דהוי ליה כשט, ואסור לשוט גזירה שמא יעשה חבית של שייטין. אי נמי אסור לעבור במעבורת שמא יסייע וקא עביד עובדין דחול. אי נמי דמחזי כמאן דממטי לה לספינה ארבע אמות בכרמלית. ותדע לך, מדגרסינן לקמן בפרק תולין (שבת קלט, ב) חזי מר האי צורבא [מרבנן] דאזיל ונאים במברא, ועבר בהאי גיסא ואמר למינם קא מכוונא ואזיל וסייר פירי, אמר ליה הערמה (מדרבנן) [בדרבנן] היא וצורבא מרבנן לא אתי למיעבד לכתחילה, אלמא לכתחילה שלא בהערמה אסור, ובהערמה דוקא לצורבא מרבנן הא לכולי עלמא אסור. ומיהו ברביעי שרי דכולי האי לא אחמור רבנן.\par \textbf{} והא דגרסינן בירושלמי אין מפליגין בים הגדול לא בערב שבת ולא בחמישי בשבת בית שמאי אוסרין אפילו ברביעי ובית הלל מתירין, לאו למימרא שיהו בית הלל מתירין ואפילו בערב שבת, אלא ברביעי קא מיפלגי, דבית שמאי אוסרין אפילו ברביעי ובית הלל מתירין ברביעי דהיינו שלשה ימים קודם השבת.\par \textbf{} והיכא דהפליגה ספינתו אפילו חוץ לאלפים אמה, מותר להלך בכל הספינה, הואיל ושבת באויר מחיצות מבעוד יום, כדאיתא בעירובין בריש פרק מי שהוציאוהו (עירובין מב, ב).\par \textbf{} ואותן קרונות שמושכין אותן בהמות, אע״פ שנכנס שם קודם ג׳ ימים לשבת אפילו מתחילת יום ראשון אסור מן הטעם שאסרו לרכוב בשבת, גזירה שמא יחתוך זמורה להנהיג הבהמה.}
\textblock{ הא דתנן:\textbf{ אמר רבן שמעון בן גמליאל נוהגין היו של בית אבא לתת כלי לבן שלהם לכובס ג׳ לפני השבת.} לא שהלכה כן, אלא להחמיר על עצמן כדברי בית שמאי.\par \textbf{} והא דשרו בית הלל לתת עורות לעבדן וכלים לכובס עם השמש, דוקא בקצץ ובקבולת דעכו״ם בדנפשיה קא טרח, הא בשלא קצץ אסור דבמלאכת ישראל קא טרח והוי ליה כשלוחו של ישראל, וכדתניא אין משלחין אגרות ביד עכו״ם בערב שבת אלא אם כן קצץ, אבל בקצץ ודאי הוא דשרי. והא דגרסינן במסכת מועד קטן פרק מי שהפך (מועד קטן יב, א) אמר שמואל מקבלי קבולת חוץ לתחום מותר תוך התחום אסור. הני מילי בקבולת מלאכת קרקע דיש לו קול וניכר שהוא של ישראל, אבל בשל כלים מותר.\par \textbf{} ודוקא בביתו של עכו״ם, אבל בביתו של ישראל בין כך ובין כך אסור, כדגרסינן בירושלמי (בפירקין ה״ח): תניא אומנין שהיו עושין עם ישראל בתוך ביתו אסור, בתוך בתיהם מותר, במה דברים אמורים בקבולת אבל בשכירות אפילו בתוך בתיהם אסור, במה דברים אמורים בתלוש מן הקרקע אבל במחובר לקרקע אסור.}
\textblock{ הא דתניא:\textbf{ רבי ישמעאל אומר יגמור משתחשך.} לאו למימרא יגמור ממש בידים, דהא לא שרי בית הלל אלא טעינת קורה עם השמש אבל לטעון משחשיכה לא, וכל שכן שלא יסחוט בידים, דאע״ג דמלאכה דאורייתא ליכא כיון שריסקן מבערב מ״מ מדרבנן אסור. וכן פירש רש״י ז״ל, ובהדיא תניא לקמן בפרק תולין (שבת קמ, א) שום שרסקו מערב שבת למחר נותן לתוכו פול וגריסין, ולא ישחוק אלא מערב.}
\textblock{\textbf{דאסיר כי פליגי במחוסרין שחיקה.} ובודאי רבי יוסי ברבי חנינא דאוקמה למתניתין כרבי ישמעאל לית ליה דרבי יוחנן, דהא מתני׳ לכולי עלמא במחוסרין דיכה הן, אלא ודאי לית ליה דר׳ יוחנן. וקיימא לן כר׳ יוסי ברבי חנינא דעביד עובדא כר׳ ישמעאל, ומעשה רב. והתימה מן הרב אלפסי ז״ל, שהביא הא דר׳ יוסי בר׳ חנינא והא דרבי יוחנן. ועוד יש תימה בדבריו, שהוא ז״ל פסק בפרק חבית (לקמן שבת קמה, ב) כר״א ור״ש שאמרו בזיתים וענבים שרסקן מערב שבת ויצאו מעצמן שהוא מותר, ואע״פ שהן כמחוסרין דיכה כדאיתא התם. וזו מן התשובות שהשיב עליו הראב״ד ז״ל.}
\textblock{\textbf{שמן של בדדין.} פירש רש״י ז״ל: שמן המשתייר בזוית הבד. והקשו בתוס׳ מה ענין [מוקצה] בכאן. ופירשו הם ז״ל דקאי אדלעיל דאיירי בטעינת קורת בית הבד, וקאמר שאותו שמן שזב והולך מתחת הקורה בשבת רב אסר משום נולד, דבין השמשות לא היה ראוי לא למשקה ולא לאכילה, ולא דמי לקדרה חייתא דההיא מכל מקום בין השמשות היתה בעולם, אבל המשקין האלו בין השמשות לא היו בעולם. אי נמי יש לומר דהתם היה ראוי לכוס, אבל זיתים וענבים מרוסקין לא חזו למידי.}
\textblock{\textbf{הנהו כרכי דזוגי רב אסר ושמואל שרי.} פירש רש״י ז״ל: מחצלאות שמכסין בהם את הסחורה. ורב אלפסי ז״ל פירש שכרוכות ועומדות לסחורה. והקשה ר״ת דבפרק מפנין (לקמן שבת קכח, א) אמרינן דרב במוקצה לאכילה סבר ליה כרבי יהודה ובמוקצה לטלטול סבירא ליה כרבי שמעון. על כן פירש רבנו תם ז״ל שהן מחצלאות שמכסין בהן דג מליח, והן מוקצות מחמת מיאוס. והא דדייק בפרק מי שהחשיך (לקמן שבת קנו, ב) מהא דכרכי דזוגי דרב סבירא ליה כרבי יהודה גבי בהמה שמתה, מפרש רבנו תם ז״ל דנבלה שנתנבלה בשבת אפילו לכלבים אינה ראויה דמוקצה מחמת איסור היא, ולהכי דייק שפיר דרב סבר לה כרבי יהודה מהא דכרכי דזוגי, דמוקצה מחמת איסור חמיר ממוקצה מחמת מיאוס, כדמוכח בפרק מי שהחשיך (שבת קנז, א) גבי חד אמר במוקצה מחמת מיאוס הלכה כר׳ שמעון.}
\textblock{ [מתני׳:] הא דתנן:\textbf{ רבי אליעזר אומר כדי שיקרמו פניה התחתונים.} ר״א להקל הוא דאתא. והכי איתא בירושלמי (בפרקין ה״י) דגרסינן התם מודה רבי אליעזר בלחם הפנים שאין קרוי לחם עד שיקרמו פניו בתנור, דאלמא הכא הוא דלא בעי אלא קרימה כל דהו הא בלחם הפנים בעי קרימה מעליא.}
\clearpage
\newsection{דף כ}
\textblock{ גמרא: \textbf{קרא כי אתא לאיברים ופדרים הוא דאתא.} כלומר, לאיברים דחול שמשלה בהן האש מאמש.\par \textbf{} ולא לעשות להם מערכה בפני עצמן קאמר, דהא אמרינן לקמן (שבת כד, ב) עולת שבת בשבתו (במדבר כח, י) ולא עולת חול בשבת, אלא להוסיף בה אש ולחתות בהן ולהבעיר באותה מערכה עצמה שלהן.\par \textbf{} והקשו בתוס׳ היכי נפקא לן כל הני דרשות ממושבותיכם, דהכא דרשינן ליה לאיברים ופדרים ובריש מסכת יבמות (ו, ב) דרשינן מיניה שאין מיתת בית דין דוחה את השבת, ובקדושין פרק קמא (לז, ב) דרשינן מיניה דלא תבעי קדוש בית דין כמועדות. ותירצו דתרי מושבותיכם כתיבא, דבפרשת אמור אל הכהנים כתיב (ויקרא כג, ג) שבת היא לה׳ בכל מושבותיכם, ומיניה דרשינן בפרק קמא דקדושין דשבת לא בעיא קדוש בי״ד כמועדות, ואף על גב דרש״י ז״ל לא פירש כן שם והכא דרשינן אברים ופדרים מיתורא, מדהוה ליה למכתב בכל מושבות וכתב בכל מושבותיכם.}
\textblock{ הא ד\textbf{אמר רבי יוחנן עצים של בבל אינם צריכין רוב.} ואמרינן:}
\textblock{\textbf{ מאי היא, אילימא סילתי השתא פתילה אמר עולא המדליק צריך שידליק ברוב היוצא סילתי מיבעיא.} ואוקימנא:}
\textblock{\textbf{ בשוכא דארזא.} קשיא לי, אכתי תקשי ליה מדעולא, דהא ודאי פתילה עדיפא משוכא דארזא, דפתילה נאחזת בה האור ובשוכא מסכסכת בה האור, כדאיתא לקמן בריש פרק במה מדליקין דטפי עדיפא פתילה משוכא דארזא. ורב אלפס לא הביא בהלכות האי דרבי יוחנן. ומשמע ודאי דהלכתא היא, דהא ליכא מאן דפליג עלה. הדרן עלך פרק יציאות השבת}
\textblock{\textbf{לכש שוכא דארזא.} ואסיקנא: בעמרניתא דאית ביה. ואם תאמר אם כן מסיפא נפקא, דהא קתני כל היוצא מן העץ אין מדליקין בו. יש לומר איצטריך משום דאין למדין מן הכללות ואפילו במקום שנאמר בהן חוץ. אי נמי יש לומר דלא מיקרי יוצא מן העץ אלא דבר שיוצא מן העץ על ידי כתישה כגון פשתן, אבל עמרניתא דבשוכא באפי נפשה קאי ולא מימעיט מדבר היוצא מן העץ.}
\textblock{}
\textblock{\textbf{אנן שירא פרנדא קרינא ליה.} והא דאמרינן בסוטה בסופו (מח, ב), משחרב בית המקדש בטל שירא פרנדא, לא בטל לגמרי קאמר, אלא שנתמעט ואינו מצוי כבתחילה, וכדאמרינן נמי התם, משחרב בית המקדש בטל זכוכית לבנה, ולא נתבטל לגמרי, כדאיתא בברכות פרק אין עומדין (ברכות לא, א) תבר קמייהו כסא דזכוכית לבנה, ואמרינן בשלהי אלו מציאות (ב״מ כט, ב) מי שיש לו מעות של רבית ורוצה לאבדן ישתמש בזכוכית לבנה.}
\textblock{\textbf{שעוה איצטריכא ליה מהו דתימא לפתילה נמי לא קא משמע לן.} יש לפרש מהו דתימא שאפילו עשה מן השעוה כעין פתילה, כלומר שעשה מהן כעין נרות של שעוה שלנו שקורין קנצילי״ש יהא אסור להדליקן, קא משמע לן מכאן ואילך פסול שמנים, לומר שאינו אסור אלא בשנתן השעוה והזפת בנר כמין שמן, אבל כשחברן יחד ועשה מהן פתילה מותר דנמשך הוא יפה אחר הנר. והוא הדין דהוה מצי למימר זיפתא איצטריכא ליה, אלא לפי שדרך לעשות פתילות משעוה ואין דרך לעשות כן מזפת נקט הכא שעוה.\par \textbf{} ומכאן יש להתיר נרות של שעוה בשבת. וכן התירו בתוס׳. וכן נראה מדברי רש״י ז״ל. אבל ראיתי בתשובת רבנו שרירא גאון ז״ל שכתב בתשובה וז״ל: ואבוקה של שעוה ששאלת, לא חזי לנא מאן דאדליק בשעוה בבי שמשי, וכל רבוותא דחזינון הוו אוסרין, עד כאן. וכן אסרו חכמי נרבונא, ופירשו כן מהו דתימא לפתילה נמי לא שאם עשה נר של שעוה כעין שלנו ונתנו בפך עם שמן שעתה הפתילה שבתוך השעוה דולקת גם מחמת השמן יהא אסור, קא משמע לן מכאן ואילך פסול שמנים, שלא אסרו אלו אלא בשאין שם שמן אלא הן, הא יש בו שמן מותר. ואין פירושם מחוור בעיני, דאם כן היינו שעוה ודבר אחר, ונראה שכל השנויין במשנתנו אפילו נתן בו שמן כל שהוא אסור, דבכולהו אמר רבה דאסור שאין מדליקין בהן בתערובת לפי שאין מדליקין בהן בפני עצמם. ודוחק הוא לומר דעל ידי פתילה ממשכי בעינייהו וגזרו פתילה אטו בלא פתילה, ועל ידי תערובת שמן לא גזרו, כדאמרינן (לקמן שבת כא, א) בדרב ברונא דבסמוך, דאם כן ליכללינהו רב ברונא לשעוה וזפת וחלב בפתילה בהדי חלב מהותך וקרבי דגים שנימוחו. ורבנו הרב נ״ר מן המתירין.}
\clearpage
\newsection{דף כא}
\textblock{\textbf{מאי לאו להדליק לא להקפות.} וקיימא לן כרבה דאסור. ואני תמה על מה סמכו בדורות הללו, לכרוך גמי על צמר גפן ומדליקין בהו בעששיות של זכוכית. ואפשר לומר, שלא אסרו אלא אותם שמנו חכמים, לפי שהן ראוין להדליקן בפני עצמן, אלא שהאור מסכסכת בהן ועשויות לכבות מהרה, וכן כל כיוצא בהן שהן דולקין והולכין בפני עצמן כאגוז של בית רשב״ג, לפיכך גזרו ע״י תערובת אטו בעינייהו, אבל דבר שאין ראוי להדליקו בפני עצמו כורכין עליו דבר אחר שמדליקין, דמאי אמרת דילמא מדליקין בו בפני עצמו, הא אין מדליקין בו בפני עצמו לעולם, שאין דולק לעולם, וכענין שאמרו למעלה (כ, ב) האי עץ בעלמא הוא, כלומר: ואין עושין ממנו פתילה לעולם.}
\textblock{\textbf{חלב מהותך וקרבי דגים שנימוחו.} פירוש: חלב מחוי והוא כעין שמן. וכן מפורש בירושלמי (בפירקין ה״א). ורש״י ז״ל פירש חלב מהותך מבושל. ואינו מחוור, דהוה ליה לרב למינקט מבושל כלישנא דמתניתין. ועוד דהכא משמע דחלב מהותך נמשך הוא אחר הפתילה ואפילו לחודיה בלא תערובת שמן מדינא שרי אלא דגזרינן ליה אטו שאינו מהותך, ואילו חלב מבושל משמע דמחמת עצמו אסור ככל הנך שמנים דמתניתין. ועוד דאמרינן הכא שמנים שאמרו חכמים אין מדליקין בהן לא יתן לתוכו שמן כל שהוא וידליק לפי שאין מדליקין בהן בפני עצמן, ומשמע דכללא הוא לכולה מתניתין, דאי לא הוה ליה למימר חוץ מחלב מבושל. אלא מאי מהותך מחוי שלא נקרש. וכן מפורש בירושלמי, דגרסינן התם: רב ברונא אמר חלב מטיף לתוכו שמן כל שהוא ומדליק, רבי יוסי בעי, מה אנן קיימין, אם במחוי אפילו לא נתן לתוכו שמן, ותני שמואל כן כל שמתיכין אותו ואינו קרוש מחוי הוא ומדליקין בו, ואם בשאינו מחוי אפילו נתן לתוכו שמן, אתא רב נחוניא רב ברונא בשם רב חלב מהותך וקרבי דגים מדליקין בהן, עד כאן בירושלמי, אלמא מחוי שאינו נקרש הוא שמדליקין בו, אבל מחוי שנקרש אין מדליקין בו, אלא שבירושלמי הקלו במחוי אפילו בלא שמן ולא גזרו אטו שאינו מחוי ובגמרין גזרו.}
\textblock{      \textbf{וקרבי דגים שנמוחו.} גם כאן פירש״י ז״ל (לקמן שבת כד, ב בד״ה שמן): דהיינו שמן דגים. וגם זה אינו מחוור, דשמן דגים שרי במתניתין בלא נתינת שמן. אלא שמן דגים הוא היוצא מן הבשר ונמשך הוא אחר הפתילה, וקרבי דגים הן שנמוחו הם עצמן ואינו נמשך היטב אחר הפתילה. וכן פירשו בתוס׳. ולקמן (שבת כו, א ד״ה סומכוס) נאריך בו יותר בס״ד.}
\textblock{ הא דאקשינן:\textbf{ מבלאי מכנסי כהנים ומהמיניהן היו מפקיעין ומהן מדליקין.} איכא למידק מאי קא מקשה מינה, והלא יש בו פשתן, דשש כתנא, והוה ליה כרך דבר שמדליקין בו על דבר שאין מדליקין בו, ואע״ג דבשבת אסור, היינו משום גזירה דילמא אתי לאדלוקי בעינייהו, אבל במקדש ליכא למיגזר. ויש לומר דפשתן שבהן דבר מועט הוא, שאין בהן אלא רביע פשתן ושלושת החלקים תכלת וארגמן ותולעת שני, ודבר מועט של פשתן כזה בטל הוא לגבי שאר המינין כאילו אין בו כלל.}
\textblock{ הא דאמרינן:\textbf{ קסבר כבתה זקוק לה ומותר להשתמש לאורה.} איכא למידק ומנא לן מהא דמותר להשתמש לאורה, דילמא היינו טעמא משום דאם כבתה זקוק לה והלכך אין מדליקין בהן בשבת דדילמא כביא ואי אפשר לו ליזקק לה. ויש לומר דמדאיצטריך לומר בין בחול בין בשבת שמעינן לה, שאם איתא דטעמא דשבת משום דאם כבתה זקוק לה הוא, לימא סתמא אין מדליקין בהן בחנוכה, ואנא ידענא דכל שכן בשבת, דאי אפשר שיזקק לה, אלא ודאי הכי קאמר אין מדליקין בהן בחול משום דכבתה זקוק לה ודילמא פשע ולא מדליק, וכל שכן בשבת דאיכא משום טעמא אחרינא דילמא אתי להטויי, ושמע מינה דמותר להשתמש לאורה. אי נמי יש לומר, דלאו מדיוקא דלישניה דרב הונא נפקא ליה, אלא סברא דנפשיה הוא דקאמר, דכיון דאיכא למימר דאית ליה לרב הונא הכי, מסתמא לא עבדינן דליפלוג עליה רב חסדא דאית ליה הכין על כרחין מדקאמר דמדליקין בהן בחול ואין מדליקין בהן בשבת.}
\textblock{\textbf{אמר ר׳ ירמיה מאי טעמא דרב קסבר כבתה אין זקוק לה ואסור להשתמש לאורה.} וטעמא דמילתא, משום דעל ידי נס שנעשה במנורה תקנו, והלכך עשאוה כמנורה שאסור להשתמש לאורה. ואי נמי כיון דלמצוה הדליקה איכא משום בזויי מצוה.\par \textbf{} ואם תאמר רב הונא ורב חסדא היכי שרו, וכי לית להו בזויי מצוה אסור, והא אבא דכולהו דם (לקמן שבת כב, א). ועוד קשה עלייהו מהא דתניא לקמן (שם ע״ב) מעשר שני אין שוקלין כנגדו דינרי זהב, וכמו שהקשה ממנה לקמן (שם) למאן דאמר מדליקין בקינסא דלא חייש לבזויי מצוה. יש לומר דסבירא ליה לרב הונא ולרב חסדא דכל שאינו משתמש בגופו כשוקל כנגדו או שמדליקין ממנו בקנסא לאו בזויי מצוה הוא, ואף על גב דמשתמש לאורה, דהא לא משתמש בגופיה ממש, ולא כל המשתמשין שוין ולא כל הבזויין שוין, וכדמוכח שמעתין דלקמן (שם).\par \textbf{} והלכתא כמאן דאמר אסור להשתמש לאורה. דהא כי אתא רבין נמי אמרה משמיה דרבי יוחנן וקבלה מיניה אביי, וא״ל אי זכאי קבלתיה מעיקרא. ורבא נמי דהוא בתרא הכי סבירא ליה, כדמוכח בסמוך דאמר וצריך נר אחרת להשתמש לאורה. והר״ז ז״ל כתב דההיא אפילו מאן דאמר מותר להשתמש לאורה מודה בה, דאי ליכא נר אחרת הרואה אומר לצרכו הוא דאדלקה. ואינו מחוור בעיני (כלום) [כלל], דהא ודאי מאן דאמר מותר, מותר להשתמש לאורה כל תשמיש קאמר, ואמאי ניחוש לנכנסין ויוצאין דאמרו לצרכו הוא דאדלקה. ועוד נראה לי, דעל כרחין רבא אסור להשתמש לאורה אית ליה, דאמרינן לקמן (שבת כג, ב) אמר רבא פשיטא לי נר ביתו ונר חנוכה, נר ביתו עדיף משום שלום ביתו, ואם איתא דמותר להשתמש לאורה, עושה נר אחד ועולה לו לכאן ולכאן, ומניחה על שלחנו ודיו, כדרך שאמרו בשעת הסכנה מניחו על שלחנו ודיו. ומינה נמי משמע, דאסור להשתמש לאורה כל תשמיש ואפילו תשמיש דמצוה כגון סעודת שבת וקריאת התורה, ורב אסי נמי אוסיף בה לקמן (שבת כב, א) אפילו תשמיש כל דהו דאינו נראה כנהנה, דאמר אסור להרצות מעות כנגד נר חנוכה, כלומר: אפילו מרחוק כנגדה שאינו נראה כנהנה ממנה, והיינו דמתמה עליה שמואל וכי נר קדושה יש בו. וכ״פ הרב אלפסי ז״ל כדברי כולם להחמיר.}
\textblock{\textbf{ואע״ג דאפליגו רב ושמואל במדליקין מנר לנר, ואוקימנא פלוגתייהו באכחושי מצוה ובדקא מדליק משרגא     } לשרגא, וקיימא לן כשמואל דשרי, אבל קינסא לכולי עלמא אסור משום בזויי מצוה, דאלמא היכא דאיכא משום בזויי מצוה בלחוד הוא דאסור, הא היכא דליכא משום בזויי מצוה שרי, והכא בתשמיש דמצוה לאו בזויי מצוה איכא. יש לומר דכל שמשתמש בו לא מיפרסמא ניסא דהרואה אומר לצרכו הוא דאדלקה, והלכך אפילו ליכא משום בזויי מצוה כל היכא דלא מיפרסמא ניסא אסור, והכי נמי משום בזיון מצוה כהרצאת מעות מרחוק כנגדה אסור אף על גב דליכא משום ביטול פרסומי ניסא. אי נמי יש לומר, דכל שאדם משתמש ממצוה זו לאחרת בזויי מצוה היא זו, דמחזי כמאן דלא חביבא ליה הך מצוה ועביד מצוה זו תשמיש למצוה אחרת, ולהדליק מנר לנר דוקא הוא דשרי שמואל, משום דכולה מצוה חדא.}
\textblock{ הא דאמרינן:\textbf{ מצותה משתשקע החמה עד שתכלה רגל מן השוק.} מסתברא דלאו עכובא היא לומר דקודם לכן אם רצה להדליק אינו מדליק, דהא ודאי אילו רצה להדליק מדליק סמוך לשקיעת החמה דהא איכא פרסומי ניסא, וכענין שאמרו לקמן (שבת כג, ב) בנר שבת דעמוד האש משלים לעמוד הענן, ולומר דבמדליק סמוך לשקיעה איכא היכר דלצורך מצות שבת מדליקו, והכא נמי דכוותה, אלא דעיקר מצותה לחייבו להדליק אינה אלא משתשקע החמה. והראיה הדלקת נר חנוכה בערב שבת, דעל כרחין מקדים עם חשכה לרבה דאמר בשלהי פרקין (לד, ב) משתשקע החמה בין השמשות, דבשלמא לרב יוסף דאמר משמיה דרב יהודה אמר שמואל, משתשקע החמה עד שהכסיף העליון והשוה לתחתון יום, לא הוה ראיה, דאפשר לו להדליק בערב שבת משתשקע החמה עד שישוה העליון לתחתון, אלא לרבה דאמר משתשקע החמה הוי בין השמשות, תקשי דהא תנן (שם, א) ספק חשיכה ספק אינה חשיכה אין מדליקין את הנרות, וקיימא לן כוותיה הלכה למעשה, אלא ודאי מצוה קאמר ואילו רצה להקדים עם חשיכה מקדים.\par \textbf{} אלא מיהו נראה מדברי הרב בעל הלכות, דדוקא קאמר משתשקע החמה, ובערב שבת נמי מדליק משתשקע החמה כרב יוסף, שהוא ז״ל כתב בהלכות חנוכה הא דאמר רב יהודה אמר שמואל (לקמן שבת לה, ב) כוכב אחד יום שנים בין השמשות שלשה לילה, ואלמלא כתבה לכוונה זו, למה כתבה בהלכות חנוכה ומאי שייכא דההיא בחנוכה. ואין צורך לכך כמו שכתבתי.}
\textblock{ והא נמי דקתני:\textbf{ עד שתכלה רגל מן השוק.} ופרישנא דאי לא אדליק מדליק, לאו למימרא דאי לא אדליק בתוך שיעור זה אינו מדליק, דהא תנן (מגילה כ, א) כל שמצותו בלילה כשר כל הלילה, אלא שלא עשה מצוה כתקנה, דליכא פרסומי       כולי האי, ומיהו אי לא אדליק מדליק ולא הפסיד, אלא כעושה מצוה שלא כתקנה לגמרי. וכ״כ מורי הרב ז״ל בהל׳.\par \textbf{} ופירשו בתוס׳, דלא אמרו עד שתכלה רגל מן השוק, אלא בדורות הללו שמדליקין בחוץ, אבל עכשיו שמדליקין בבית בפנים, כל שעה ושעה זמניה הוא, דהא איכא פרסומי ניסא לאותם העומדים בבית. ומיהו לכתחילה מצוה להדליק משתשקע החמה מיד, דזריזין מקדימין למצות, ואמרינן לקמן (שבת כג, ב) רב הונא הוה רגיל וחליף אפתחא דר׳ אבין נגרא, חזא דהוי רגילי בשרגי אמר תרי גברי רברבי נפקי מהכא.}
\textblock{\textbf{אי נמי לשיעורא.} פירש הרב אלפסי ז״ל בהלכות, שאם היתה דולקת והולכת עד השיעור הזה, ורצה לכבותה או להשתמש לאורה הרשות בידו. ונראה מדבריו שאם כבתה שמותר להסתפק ממותר השמן שבנר, וכל שכן הוא שלא היקצה אותו אלא למצותו וכל זמן שדולקת והולכת הא כבתה מותר, דומיא דעצי סוכה ונוי סוכה שמותרין לאחר החג. אבל מקצת מן הגאונים ז״ל אמרו ואם כבתה ונשאר שמן בנר ביום א׳, מוסיף עליו ומדליק ביום ב׳, וכן בשאר הימים, ואם נשאר בה ביום אחרון עושה לו מדורה ושורפו במקומו שהרי הוקצה למצותו. ולפי דבריהם, יש לחלק בין זה לעצי סוכה ונוייה, דהתם לאו מקצה להו אלא לימי החג, לפי שעושין להשאיר אחר החג, ולא מקצה להו למצותן לגמרי, אבל שמן ופתילה שעשויין להתבער לגמרי כי יהיב להו בנר לגמרי מקצה אותן למצותן, דאין אדם מצפה מתי תכבה נרו, ואם נשתיירו הרי הן אסורים שהרי הקצה אותן לגמרי למצותן, ודומה לעצי סוכה ונוייה שנפלו בתוך החג שאסורים כדמוכח במסכת ביצה בפרק המביא (ל, ב).}
\textblock{\textbf{נר חנוכה מצוה להניחה על פתח ביתו מבחוץ.} פירש רש״י ז״ל: לא ברשות הרבים אלא בחצרו, שבתיהן היו פתוחין לחצר, ואם היה דר בעליה שאין לו מקום בחצר להניחה שם מניחה בפנים כנגד חלון הסמוכה לרשות הרבים. ואינו מחוור בעיני. מדתנן (ב״ק סב, ב) ומייתינן לה בסמוך הניח חנוני את נרו מבחוץ, החנוני חייב, רבי יהודה אומר בנר חנוכה פטור מפני שהוציאו ברשות, ואם אין מניחה אלא בפנים, וזה הניחה בחוץ ממש למה פטור, ומי הרשהו, ואם אתה אומר דוקא בשהניחה כנגד חוץ ולעולם בפנים בחצרו, אם כן אפילו חנוני בכי הא אמאי חייב, והלא ניזק ברשות המזיק הוזק, ואדרבא בעל הגמל חייב. אלא מסתברא מבחוץ ממש קאמר.}
\textblock{\textbf{שהיה מונח בחותמו של כהן גדול.} ואף על גב דעכו״ם עשאום כזבים וכלי חרס מטמא בהסט. איכא למימר שמצאוהו בקרקע שאין לחוש להסט. אי נמי מגולה וקודם גזירה היה.}
\textblock{\textbf{אמר רבינא משמיה דרבא זאת אומרת נר חנוכה צריך להניחה בתוך עשרה, דאי סלקא דעתך למעלה מעשרה, לימא ליה, היה לך להניחה למעלה מגמל ורוכבו.} ותמיה לי אכתי בתוך עשרה מנא לן, לימא צריך להניחה למטה מגמל ורוכבו, דהא גמל ורוכבו למעלה מעשרה טפחים הוי טובא. וי״ל דקסבר דכיון דאפקת ליה מעשרים אמה כסוכה וכמבוי, מוקמינן ליה לעשרה שהוא הכשר סוכה נמי שנתנו לו חכמים שיעור אחד ידוע מן השיעורין הקבועים בשאר המצות, וכיון דמפקת לה מעשרים שהוא למעלה הרבה מגמל ורוכבו אוקמוה אעשרה. ודחינן דילמא לא נתנו בו חכמים שיעור, אלא מניחו באיזה מקום שירצה והוא שיהיה למטה מעשרים דלהוי ליה היכירא ופרסומי ניסא, ומיהו אם הניחו למטה מגמל ורוכבו והדליק פשתנו פטור, דלא אטרחוהו רבנן דלא ליתי לאימנועי.\par \textbf{} ולענין פסק הלכה קיימא לן כרבינא דאמר משמיה דרבא, דלא שבקינן מאי דאיפשיטא להו לרבא ורבינא ונקטינן מאי דאידחי בגמרא בדרך דילמא בעלמא. וכן פסק רבנו חננאל ז״ל. וכן פסק רבנו הרב ז״ל.}
\clearpage
\newsection{דף כב}
\textblock{\textbf{אבוהון דכולהו דם.} והא דלא קאמר אבא דכולהו עצי סוכה, ואף על גב דנפקא לן מחג הסוכות, מה חג לה׳ (ויקרא כג, לד) אף עצי סוכה לה׳, כדאיתא בריש פרק המביא כדי יין (ביצה ל, ב). יש לפרש משום דאי מהתם הוה אמינא דשאני התם שיש בטול מצוה כשמסתפק מהם וסותר את הסוכה, אבל בסיפוק נויין והרצאת מעות דליכא משום ביטול מצוה אימא דשרי, אלא דמדם ילפינן לכולהו.}
\textblock{\textbf{שממנו היה מדליק ובו היה מסיים.} פרשתי באר היטב בתשובת שאלה (ח״א סי׳ עט וסי׳ שט) בס״ד.\par \textbf{}   בענייני הטבת הנרות לפנינו בחלק התשובות למסכת תמיד.}
\textblock{\textbf{חזינא אי הדלקה עושה מצוה מדליקין מנר לנר.} דהא ילפינן ממנורה דכל דליכא בזויי מצוה שרי ולא חיישינן לאכחושי מצוה, וכיון שכן הכא ליכא משום בזויי מצוה דבשעה שמדליק מדליק ממצוה למצוה. ואי הנחה עושה [מצוה] הוה ליה כמדליק בקינסא, דבשעה שמדליק לא מקיים המצוה והלכך אסור, ואפילו בפתילות ארוכות אי אפשר דהא בעינן הנחה לבתר הדלקה, ולקמן נמי (כג, א) הכי גרסינן: מכבה מגביהה ומדליקה וחוזר ומניחה מיבעי ליה, דלמאן דאמר הנחה עושה מצוה בעינן הנחה לבתר הדלקה.\par \textbf{} וקיימא לן דהדלקה עושה מצוה, והלכך מדליקין מנר לנר. ודוקא בפתילות ארוכות, אבל בקינסא אסור לכ״ע דהא איכא בזויי מצוה.}
\clearpage
\newsection{דף כג}
\textblock{ הא דאמרינן:\textbf{ אלא למאן דאמר הנחה עושה מצוה מכבה ומגביהה וכו׳.} איכא למידק למה מכבה, יגביהנה ויניחנה לשם מצות נר חנוכה, דמי גרעה דלוקה מהדליקה חרש שוטה וקטן דלמאן דאמר הנחה עושה מצוה מהני אם הניחה בן דעת כדמוכח בסמוך. ויש לומר דמיגרע גרע, דהתם כיון דבשעת הדלקה הוא הא מוכחא מילתא דלשם נר חנוכה הודלקה וכשהניחה בן דעת דיו ושפיר דמי, אבל הכא שהיתה דולקת מאתמול ליכא פרסומי ניסא דהרואה אומר לצרכו הוא דאדלקה.}
\textblock{\textbf{הרואה מברך שתים.} מסתברא בשלא הדליק ולא הדליקו עליו בתוך ביתו ואינו עתיד להדליק אותה הלילה, הא לאו הכי א״צ לברך, דלא מצינו יוצא מן המצוה וחוזר ומברך על הראיה. ויש מרבוותא ז״ל דפירשו אע״פ שמדליקין עליו בתוך ביתו צריך לברך על הראיה. ואין להם על מה שיסמוכו.}
\textblock{\textbf{רבא אמר רוב עמי הארץ מעשרין הן.} פירש רש״י ז״ל: אפילו ספק דדבריהם בעי ברכה, ודמאי אפילו ספק נמי לא הוי אלא חומרא בעלמא, דרוב עמי הארץ מעשרין הן. משמע מדבריו דרבא פליג אדאביי. ואיכא למידק, דהא בהדיא אמרינן בברכות (כא, א) ספק קרא קריאת שמע ספק לא קרא קריאת שמע אינו חוזר וקורא, ספק אמר אמת ויציב ספק לא אמר חוזר ואומר, דאמת ויציב דאורייתא קריאת שמע דרבנן, אלמא בספק דדבריהם אינו מברך. ואולי נאמר לדברי רש״י דלא אמר רבא הכי אלא בספק דדבריהם שיש לו עיקר בדאורייתא, כגון יום טוב שני שודאו דאורייתא ומשום ספק יום ראשון הוא, ואף על גב דברכות מדרבנן מכל מקום כיון דעיקרו משום ספק של תורה מברכין, והוא הדין דמאי שהיינו מברכין עליו משום שודאו דאורייתא, אלא כיון שרובן מעשרין ורובא ככולא בעלמא הוא לא בעי ברכה, אבל קריאת שמע דעיקרו מדבריהם ספקו אינו חוזר וקורא.\par \textbf{} אבל הרב אלפסי ז״ל כתב בפרק רבי אליעזר דמילה (לקמן שבת קלה, א) גבי נולד כשהוא מהול, שצריך להטיף ממנו דם ברית ואינו מברך עליו, וכן כתב משמיה דרבנו האי גאון ז״ל, ואף על פי שעיקר מילה דאורייתא. ויש לפרש לפי דבריהם דרבא לא פליג אדאביי ופירוקא אחרינא הוא דמפרק בדמאי, ולעולם ביום טוב שני טעמא כדי שלא יזלזלו בו כטעמיה דאביי, וכיון שכן ספק נטל לולב ספק לא נטל אפילו ביום טוב ראשון חוזר ונוטל ואינו מברך, לפי שהברכה מדבריהם וכל ספק בדבריהם אינו מברך. וצריך עיון.}
\textblock{ הא ד\textbf{אמר רבא נר ביתו וקדוש היום נר ביתו עדיף משום שלום ביתו.} תמיה לי דהא ודאי נר ביתו לצורך סעודה קאמר, ואם כן למה ליה משום שלום ביתו תיפוק ליה משום דאפשר לו לקיים את שתיהם דמקדש אריפתא. ואולי נאמר דמצוה מן המובחר לקדש אחמרא וכדאמרינן (פסחים קו, א) קדשהו על היין בכניסתו, אלא דכי חביבא ליה ריפתא אפשר לקדש אריפתא משום חביבותא, ומשום חבוב מצוה מקדש אמידי דחביב ליה טפי, הא היכא דלא חביבא ליה ריפתא טפי מצוה בחמרא, ואפילו הכי נר ביתו עדיף משום שלום ביתו ומקדש אריפתא. והיינו נמי דאיצטריך למימר גבי נר חנוכה וקדוש היום נר חנוכה עדיף משום פרסומי ניסא, הא לאו הכי קדוש היום עדיף, משום דקדוש על היין מצוה מן המובחר, ומשום דאפשר לקדש אריפתא הוא דעדיף נר חנוכה, הא לאו הכי קדוש היום עדיף דהוי דאורייתא. וכן כתב הרב בעל ההלכות ז״ל נר חנוכה עדיף משום פרסומי ניסא דאפשר דמקדש אריפתא.\par \textbf{} עוד כתב הרב בעל ההלכות ז״ל: והיכא דקא בעי אדלוקי נר שבת ונר חנוכה, ברישא מדליק דחנוכה והדר מדליק של שבת, דאי מדליק דשבת ברישא איתסר ליה לאדלוקי דחנוכה משום דקבלה לשבת עליה. ודבריו תמוהין דהדלקה אינה עושה קבלה, דהא תניא בשלהי פרקין (לה, ב) שש תקיעות בערב שבת וכו׳ שלישית הדליק המדליק וכו׳ ושוהה כדי לצלות דג קטן או כדי להדביק פת בתנור, ואמרינן נמי התם אמר רבי יוסי בר׳ יהודה שמעתי שאם בא להדליק אחר שש תקיעות מדליק וכו׳, אלמא הדלקת הנר אינה קבלה. והדין נותן, שאם אין אתה אומר כן מכי אדליק נר ראשון קבלה ואיתסר ליה להדליק נר שני, ונמצא שאסור להדליק שתי נרות בערב שבת. אלא ודאי ליתא, ואדרבא מדליק דשבת ברישא הואיל ועדיף ותדיר. וכן דעת הרמב״ן ז״ל.}
\textblock{ הא דאמר אביי לרבא:\textbf{ אלא מעתה ביום טוב לישתרי.} איכא למידק מאי קא מקשה ליה מיום טוב, דילמא טעמא משום דאין שורפין קדשים ביום טוב, וכדמפרש בגמרא עלה לקמן, ורבא נמי אמאי לא מפרש ליה הכין. ויש לומר דאביי ורבא הכא סבירא להו דלא אסרה תורה אלא שריפת קדשים בלבד לפי שהיא הבערה שלא לצורך היום, אבל שמן תרומה שנטמאת כיון שיש לו בו היתר מותר לו לשורפו דהבערה לצורך היום היא. וא״ת והא אוקימנא לקמן (שבת כד, ב) טעמא דמתניתין (שם), משום שאין שורפין קדשים ביום טוב, ואיבעיא להו מנא הני מילי ואמר אביי אמר קרא (במדבר כח, י) עולת שבת בשבתו ולא עולת חול בשבת ולא עולת חול ביום טוב, אלמא אביי נמי אית ליה דאין שורפין תרומה טמאה ביום טוב. יש לומר דאביי לאו אמתניתין קאי, אלא אאין שורפין קדשים ביום טוב, והא דאוקימנא לקמן טעמא דמתניתין לפי שאין שורפין קדשים ביום טוב, היינו אליבא דרב חסדא, משום דקיימא לן כוותיה דהא תניא כוותיה (לקמן שבת כד, א).\par \textbf{} אי נמי יש לומר, דאביי אית ליה דטעמא דמתניתין משום דאין שורפין קדשים ביום טוב, אלא שמדד במדה גדושה וקסבר דאפילו בשבת דעלמא אסור ואף על פי שמדליקה בחול לפי שהיא דולקת והולכת בשבת, ויש מי שסובר כן בירושלמי (פ״ב, ה״א), ומכיון דשמעיה לרבה דאוקי למתניתין בגזירת שמא יטה, ממילא ידע דרבה לית ליה בשמן תרומה שנטמא משום איסור שריפת קדשים, ומשום הכי אקשי ליה לדבריך ביום טוב לישתרי.\par \textbf{} אי נמי יש לומר דאביי כרב חסדא סבירא ליה דביום טוב שחל להיות בערב שבת עסקינן, וטעמא דכולה מתניתין משום דאין שורפין קדשים ביום טוב, אלא מדשמעיה לרבה דדחיק ומוקי לה בההיא גזירה דהוא דבר רחוק שיהא מטה כדי שימהר מעט בשריפה מה שהוא מותר ליהנות בו בשריפתו כשמן דעלמא, ודאי דרבה לית ליה בשמן שריפה טעמא משום שריפת קדשים, דאי אית ליה טפי הוה ליה לאוקומה ביום טוב שחל להיות בערב שבת וכרב חסדא, ומשום הכי אקשי ליה. והיינו נמי דסייעיה נמי לרב חסדא, מדתניא כל אלו שאמרו אין מדליקין בהן בשבת מדליקין בהן ביום טוב חוץ משמן שריפה לפי שאין שורפין קדשים ביום טוב, ומסתמא לכולה מילתא דרב חסדא קא מייתי מינה סייעתא, ולומר דלית ביה משום גזירת הטיה כלל ובשבת דעלמא שרי, והיינו משום האי טעמא דאמרינן דכיון דקא מפרש בברייתא דטעמא משום שריפת קדשים, ממילא אדחי ודאי ההוא טעמא דגזירת הטיה דטעמא קלישתא הוא, ולא מוקי לה רבה למתניתין בההוא טעמא אלא מדלא אשכח טעמא אחרינא לאוקומי ביה למתניתין.\par \textbf{} וכיון שכן קיימא לן דבשבת דעלמא מדליקין בו, ואף על גב שדולקת והולכת בשבת, מלאכה היא שנעשית מאליה ולא חיישינן לה, וכדרב חסדא דסבירא ליה הכין. וכתב רבנו הרב ז״ל דמשום כך הכניסו בגמרא הך פיסקא דולא בשמן שריפה באמצע הלכות חנוכה, משום דשייכא בהדלקת נר חנוכה בשבת לרבה.}
\clearpage
\newsection{דף כד}
\textblock{\textbf{ימים שאין בהם קרבן מוסף כגון שני וחמישי של תעניות ומעמדות.} והוא הדין לכל שאר ימים שאין בהם קרבן מוסף כגון חנוכה ופורים. ובהדיא תניא בתוספתא (ברכות פ״ג הי״ד) כל יום שאין בו קרבן מוסף כגון חנוכה ופורים, שחרית ומנחה מתפלל י״ח ואומר מעין המאורע בהודאה, ואם לא אמר אין מחזירין אותו.}
\textblock{ גירסא דוקיא:\textbf{ י״ט שחל להיות בשבת המפטיר בנביא במנחה וכו׳.} ואע״פ ששנינו במגילה (כא, א) בשבת במנחה קוראין ג׳ ואין מפטירין בנביא, מקומות מקומות יש, ועדיין בפרס מפטירין בנביא במנחה בשבת. וכן השיב הגאון ז״ל בתשובה. ומגיהי ספרים הגיהו יוה״כ שחל להיות בשבת, מפני שהוקשה להם מה ששנינו שם במגילה. וכן כתב בעל המאור ז״ל.}
\textblock{\textbf{ולית הלכתא ככל הני שמעתתא אלא כי הא דא״ר יהושע בן לוי.} פירש רש״י ז״ל: דהא דרב גידל נמי מידחיא, והמפטיר בנביא בראש חודש שחל להיות בשבת מזכיר של ראש חודש, ואף על גב דהפטרה בראש חודש ליכא כלל מ״מ יום הוא שנתחייב בהפטרה. וכן פסק אלפסי ז״ל. אבל ר״ת ז״ל וכן בעל המאור ז״ל אמרו דלא אידחיא הא דרב גידל כלל דהפטרה בראש חודש ליכא כלל, ואיתא להא דרבי יהושע בן לוי ואיתא להא דרב גידל, ולא מידחו אלא הא דרב הונא ורב יהודה ורב אחדבוי דפליגי אדרבי יהושע בן לוי.\par \textbf{} ורבנו הרב ז״ל הכריע כדברי רש״י והרב אלפסי ז״ל, מדאמרינן סתם לית הלכתא ככל הני שמעתתא, ואם איתא דאיתא לדרב גידל, כיון דלדעתיה דאביי אף היא דמיא לדרב הונא ובין איתיה ביומיה בין ליתיה ביומיה כלל כולהו חדא אורחא אית להו, הוה ליה לבעל הגמרא לפרושי בהדיא לית הלכתא כרב הונא ורב יהודה ולא כרב אחדבוי, כי היכי דלא ניטעי בהא דרב גידל לאפוקה מהלכתא. ועוד דאם איתא דהא דרב גידל משום דאין נביא בראש חודש כלל הוא, אם כן משמע דהיכא דאיתיה כגון מוספין דאיתיה ערבית ושחרית ומנחה מודה בה רב דמזכיר, והא ודאי ליתא, דהא רב אחדבוי משמיה דרב אמר איפכא, ובכדי לא משוינן חלוקא ביני אמוראי ונימא תרי אמוראי נינהו אליבא דרב, אלא ודאי טעמיה דרב גידל כטעמיה דרב אחדבוי דתרווייהו משמיה דרב קאמרי, וכיון דאידחי הא דרב אחדבוי אידחיא נמי דרב גידל, דיום הוא שנתחייב בכך. ואף על פי שיש להשיב, הלכה למעשה הכי שפיר טפי למיעבד, משום דמ״ד מזכיר חייב להזכיר קאמר, ולמ״ד אינו מזכיר לאו אסור להזכיר קאמר אלא אין צריך להזכיר קאמר ואם בא להזכיר מזכיר, וכדאמרינן לעיל (בע״א) גבי הזכרת חנוכה בברכת המזון אינו מזכיר ואם בא להזכיר מזכיר בהודאה.\par \textbf{} עוד כתב הרב רבנו ז״ל, דאף על גב דאידחיא הא דרב גידל מכל מקום אינו חותם בה בשל ראש חודש כלל, אלא שאומר מעין המאורע בעבודה ואינו חותם אלא בשל שבת בלבד. וטעמא קאמר דהיאך אפשר דיפה כח ראש חודש בהפטרה דליתא אלא בשביל שבת יותר משלש תפלות הקבועות בראש חודש, אלא ודאי אינו חותם אלא בשל שבת בלבד. אבל המפטיר בנביא במנחה ביום הכפורים שחל להיות בשבת, חותם אף בשל שבת כדרך שהוא חותם בהפטרת שחרית.\par \textbf{} וראש חודש שחל להיות בשבת, אמרי רבוותא ז״ל שאינו מזכיר שבת ביעלה ויבא, וכן כתב רבנו ז״ל, לפי שלא מצינו הזכרה אחר חתימה.}
\textblock{\textbf{לא באליה ולא בחלב, חכמים היינו תנא קמא איכא בינייהו דרב ברונא אמר רב.} דאמר לעיל (שבת כא, א) חלב מהותך נותן לתוכו שמן כל שהוא ומדליק.}
\textblock{\textbf{ולא מסיימי.} פירוש: ולא שיהא מהותך כחלב דמתניתין וכמו שפירש רש״י דחלב מהותך היינו חלב מבושל כדכתבינן לעיל (שם), אלא הכי קאמר דתנא קמא אמר לא בחלב וסתמא קאמר, ואיכא למימר דבכל ענין קאמר ואפילו מחוי ואפילו על ידי תערובת שמן, ודלא כרב ברונא, וחכמים לא אסרו אלא חלב מבושל שנקרש, אבל מחוי על ידי תערובת מישרא שרי, והיינו דרב ברונא, אי נמי איכא למימר איפכא, דרבנן קמאי לא אסרי אלא חלב כמות שהוא, אי נמי קרוש שאינו נמשך אחר הפתילה, ואי נמי מחוי ובלא שמן משום גזירה דאינו מחוי, אבל מחוי על ידי תערובת שרי, והיינו דרב ברונא, ורבנן בתראי סברי דאפילו מבושל דמחוי ואפילו על ידי תערובת אסור, ודלא כרב ברונא, והיינו דלא מסיימי.\par \textbf{} ואם תאמר נימא כל תנא בתרא לטפויי מילתא קאמר, וכדאמרינן בפרק המוכר פירות (ב״ב צג, ב). תירצו בתוס׳ משום דלכאורה הוה משמע דת״ק אסר בכל ענין דומיא דשאר שמנים דאיירינן לעיל מיניה כגון שעוה וזפת, ותנא בתרא לא קאי אלא אנחום דהתיר במבושל וקאמר ליה איהו דאפילו מבושל אסור, אבל במחוי ובנתינת שמן אפשר דשרי לדידיה.}
\textblock{ הא דאיבעיא לן:\textbf{ מנא הני מילי.} איכא למידק ומנא לן דשרי, דהא לא הותרה הבערה אלא או לצורך אכילה או לצורך מצות היום, הא שלא לצורך מצות היום כלל אסור. איכא למימר, משום דסלקא דעתך אמינא דהוי יום טוב לא תעשה גרידא, ואתי עשה דשריפת קדשים ודחי לא תעשה. ותדע לך מדקא מתרץ רב אשי דהוי יום טוב עשה ולא תעשה ואיתרצא להו בהכין, שמע מינה מעיקרא לא הוה סלקא דעתך הכי. ואם תאמר אם כן רבא למה ליה לאהדורי בתר פירוקא אחרינא, לימא כרב אשי, דהא משמע בפרק קמא דביצה (ח, ב) דרבא סבירא ליה הכי דהוי יום טוב עשה ולא תעשה, מדאקשינן התם גבי כסוי הדם דאמרינן אתי עשה ודחי לא תעשה, מכדי יום טוב עשה ולא תעשה הוא ולא אתי עשה ודחי את לא תעשה ועשה, אלא אמר רבא אפר כירה מוכן הוא לודאי כו׳, אלמא אית ליה לרבא דיום טוב עשה ולא תעשה. ויש לפרש דרבא לא הדר ביה התם מהך קושיא אלא משום דמסתבר ליה טעמא בתרא. אי נמי דמכל מקום סבירא ליה לרבא דאין עשה דכסוי דוחה יום טוב, דדריש ליה מלבדו (שמות יב, טז), כך תירצו בתוס׳.\par \textbf{} ואיכא דקשיא ליה, ותהוי נמי שריפת קדשים עשה וי״ט לא תעשה גרידא אכתי היכי הוה דחי האי עשה ללא תעשה דהא אפשר להמתין למחר ולקיים עשה, כדריש לקיש (לקמן שבת קלג, א) דאמר כל מקום שאתה מוצא עשה ולא תעשה אם אתה יכול לקיים את שניהם מוטב ואם לאו יבא עשה וידחה לא תעשה. ותירץ הרמב״ן ז״ל, דלא אמר ר״ל הכי אלא היכא דאפשר לקיים את שניהם עכשיו בלא דחיית לא תעשה, כגון ההיא דיבמות דאמרינן בפרק כיצד (כ, ב) גמרא אלמנה לכהן גדול, גבי אלמנה מן האירוסין לכהן גדול, כדר״ל דאמר ר״ל כל מקום וכו׳ והכא נמי אפשר בחליצה דמקיים עשה ול״ת, וכגון ההיא דמנחות דאמרינן בפרק התכלת (מנחות מ, א) גבי סדין בציצית שאפשר לעשות לו כן ממינו ולא עבדינן צמר, וכן כל כיוצא באלו, אבל היכא דלא אפשר לקיים עשה היום אלא בדחיית לא תעשה אי אפשר לקיים את שניהם מיקרי, שהרי אי אפשר לקיימן עכשיו.}
\textblock{\textbf{ובודאי הכי משמע כדבריו כדמוכח בריש פרק קמא דיבמות (ו, א), דאמרינן התם, אלא אתיא מבנין בית המקדש, } דתניא יכול יהא בנין בית המקדש דוחה את השבת תלמוד לומר (ויקרא יט, ל) את שבתותי תשמורו, הא לאו הכי דחיא, מאי לאו בבונה וסותר. ואם איתא היאך אפשר לומר כן דהא בנין בית המקדש אפשר למחר ואפשר לקיים את שניהם ואם כן דכוותיה היאך אפשר לומר דדחיא. ועוד דדחינן (שם) לאו בלאו דמחמר, ואקשינן אלא הא דקיימא לן דאתי עשה ודחי לא תעשה ליגמר מהכא דלא דחיא, אמאי, לימא שאני הכא משום דאפשר כדר״ל הא בעלמא דלא אפשר דחיא. וכי האי גוונא נמי איכא למידק מדאתינן התם (ע״ב) למיפשטה ממיתת בית דין, ואף על גב דמיתת בית דין אפשר למחר.\par \textbf{} ומיהו אכתי איכא למידק אשמעתין, בשלמא קדשים איכא עשה דאורייתא בשריפתן אלא תרומה מנא לן, דכל הני קראי דקא מייתי בקדשים קאי ולא בתרומה טמאה. אלא איכא למימר דאסמכוה רבנן אקדשים. וכן נראה מדברי רש״י ז״ל. והדין נותן הואיל ואפשר ליהנות ממנה בשעת שריפתה יהיה מותר, ואטו משום דאיכא מצוה בשריפתה מיגרע גרע, אם כן אכתי שמן שריפה למה אסרוהו. ויש לפרש דגזירה גרידתא היא דגזרינן תרומה טמאה אטו קדשים, שאם אתה מתירו בתרומה טמאה אפילו לצורך אתי למשרף קדשים שנטמאו שלא לצורך, וקדשים אסורים דבר תורה. אי נמי יש לפרש שעשו חכמים שריפת תרומה שנטמאת כשריפת קדשים משום דדמיא לקדש, אי נמי משום תקלה, וכיון שעשו אותה כשריפת קדשים של מצוה אף לאיסור עשאוה כמותן. וכי בעינן מנא הני מילי אשריפת קדשים ממש בעינן.}
\textblock{ הא דאמרינן:\textbf{ בא הכתוב ליתן בקר שני לשריפתו.} משמע דשריפתו אינו אלא עד בקר שני ואפילו רצה לשורפו בליל מוצאי יום טוב אינו רשאי. וכן מוכח בירושלמי (פ״ב ה״א). וכן כתב רש״י ז״ל והכי משמע הנותר ממנו לראשון עד בקר שני המתינוהו ותשרפוהו. וטעמא דמילתא לפי שאין שורפין קדשים בלילה, שכן מצינו בשלמים שאינן נאכלין לאור שלישי ואפילו הכי אינן נשרפין עד יום שלישי, כדאמרינן בריש מסכת פסחים (ג, א) והנותר (מזבח השלמים) [מבשר הזבח] וגו׳ באש ישרף (ויקרא ז, יז) ביום אתה שורפו ואי אתה שורפו בלילה. וגרסי׳ נמי בירו׳ (בפירקין ה״א) והנותר ממנו וגו׳ באש תשרופו (שמות יב, י) אחר שני בקרים אחר בוקרו של חמשה עשר ואחר בוקרו של ששה עשר וכתיב (ויקרא ז, יז) והנותר מבשר הזבח ביום השלישי באש ישרף. מהו להצית את האור במדורת חמץ, מאן דיליף לה מן הנותר אסור, ומאן דלא יליף לה מן הנותר מותר. ואי קשיא לך כיון דקדשים אין נשרפין בלילה, מאי שנא יום טוב שאין מדליקין בו בשמן שריפה, אפילו בחול נמי אין מדליקין. אין ודאי קושיא היא, ובירושלמי מקשו לה, ומתרץ אמר רבי יוחנן ירדו לה בשיטת רבי ישמעאל דרבי ישמעאל אמר תינוק שעבר זמנו נימול בין ביום בין בלילה, ומקשו מה אית לך למימר שמן שריפה שעבר זמנה נשרפת בין ביום בין בלילה, אמר רבי יודה בן פזי מכיון שנטמא כמי שעבר זמנה.}
\textblock{\textbf{עולת שבת בשבתו (במדבר כח, י) ולא עולת חול בשבת ולא עולת חול ביום טוב.} ואף על גב דאמרינן ביומא פרק טרף בקלפי (יומא מו, א) איברי חול שניתותרו עושה להן מערכה בפני עצמן וסודרן אפילו בשבת, ורב הונא אמר דאינו עושה להן מערכה בפני עצמן אלא סודרן במערכה הגדולה, לא קשיא, דהתם באברים שמשלה בהן האור מאמש שכבר נעשה לחמו של מזבח והוו להו כקרבן היום, והלכך דחו מבמועדו (במדבר שם, ב) ואפילו בשבת. כן תירץ ר״ת ז״ל ורבנו יצחק בר׳ אשר ז״ל.}
\textblock{\textbf{לבדו ולא מילה שלא בזמנה דאתיא בק״ו.} פירש רש״י ז״ל דהוה אמינא דמילה שלא בזמנה תדחי יום טוב מק״ו דצרעת, מה צרעת שדוחה את העבודה ועבודה דוחה את השבת מילה שדוחה את הצרעת בין בזמנה בין שלא בזמנה אינו דין שתדחה את השבת, והכי איתא לקמן בפרק רבי אליעזר דמילה (שבת קלב, א-ב). ואינו מחוור, דהא רבא לית ליה התם האי ק״ו, דקסבר צרעת דדוחה את העבודה אינו מתורת דחוי אלא משום דגברא לא חזי.\par \textbf{} ובתוס׳ פירשו ק״ו ומה נדרים ונדבות שלא נכרתו עליהן שלש עשרה בריתות קרבין ביום טוב מילה שנכרתו עליה שלש עשרה בריתות לא כל שכן, וקא סבר רבא כמאן דאמר נדרים ונדבות קרבין ביום טוב. ואינו מחוור בעיני, דאכתי איכא למיפרך, מה לנדרים ונדבות שכן יש בהן צורך היום שהרי נאכלין לבעלים, או שיש בהן צורך היום שאינו בדין שתהא כירתך פתוחה וכירת רבך סתומה (ביצה כ, ב), תאמר בקדשים שאי אפשר ליהנות מהם.\par \textbf{} ומקצת מרבותינו הצרפתים יש דלא גרסי ק״ו כלל, וטעמא קאמרי דהא לרבא משמע דקסבר דיום טוב לא תעשה גרידא הוא מדלא מתרץ לה כרב אשי, וכיון שכן בלאו ק״ו ודאי דחיא, דהוה ליה מילה עשה ויום טוב לא תעשה ואתי עשה ודחי לא תעשה. אלא שהיא בכל הספרים, ואף בספרי הגאונים ז״ל מצאתיה.\par \textbf{} ועוד תמיהא לי מאי קאמר רבא ומאי קא מייתי מילה שלא בזמנה, דהא אנן שריפת קדשים קא מיבעיא לן, ואם לומר דכשם שמילה שלא בזמנה אינה דוחה אף קדשים שלא בזמנן אינן דוחין, הוה ליה למימר בהדיא.\par \textbf{} וראיתי לרבנו שרירא גאון ז״ל בתשובה דקל וחומר הוא דשקלינן ממילה שלא בזמנה לקדשים שניתותרו, וז״ל: האי ק״ו דשקלינן ממילה שלא בזמנה לנותר, ומה מילה דאיתותר לאחר זמנה לא חזיא ביום טוב, כ״ש נותר דלא שרפינן ביום טוב, ואיכא דגרסי לה להא מילתא בגמרא, עכ״ל התשובה, ובזה נסתלקו כל הקושיות. אלא דלשון דאתיא בק״ו לכאורה אמילה קאי ולא אשריפת קדשים.}
\clearpage
\newsection{דף כה}
\textblock{\textbf{כשם שמצוה לשרוף את הקדשים שנטמאו כך מצוה לשרוף את התרומה שנטמאת.} פירש רש״י: כי היכי דלא ליתי בה לידי תקלה. ולדבריו לאו דוקא שריפה כל איבוד כן, ואפילו זורה לרוח או שמריצה לפני כלבו, וכן פירש הוא ז״ל בפרק אלו עוברין (פסחים מו, א) גבי הא דתנן לא יקרא לה שם עד שתאפה. והקשו עליו בתוס׳ דבשלהי תמורה (לג, ב) קתני תרומה טמאה גבי אלו הן הנשרפין וקתני (שם לד, א) הנשרפין לא יקברו. והא דאמרינן בשלהי פרק קמא דפסחים (כ, ב) גבי תרומה טמאה תעשה זילוף, ואיכא למידק דהא תנן בתמורה הראוי לקבורה קבורה, פירוש משקין. יש לומר דלאו אתרומה קאי אלא אערלה וכלאי הכרם קאי (ולא אתרומה), ולעולם תרומה הראוי לשריפה שריפה דוקא. וטעמא משום דדמיא לקודש הטעינוה חכמים שריפה כקודש, אי נמי מדאורייתא טעונה שריפה מדאיקרי קודש.}
\textblock{\textbf{אמר רבא כשם שמצוה לשרוף את הקדשים שנטמאו כך מצוה לשרוף את התרומה שנטמאת.} לא ידעתי מה היתה תשובתו של רבא בהא, דלא הוה ליה למימר אלא משום דר׳ ינאי א״נ משום דר׳ אלעזר וצ״ע.}
\textblock{ הא דאמרינן:\textbf{ תתן לו (דברים יח, ד) ולא לאורו מכלל דבת אורו היא.} איכא למידק מידי ולא לאורו כתיב, אימא דאינו ראוי לו ולא לאורו, ואתא למעוטי דאין שם תרומה חל עליה, כדדרשינן בפרק ראשית הגז (חולין קלז, א) תתן לו ולא לשקו מכלל דאין ראשית הגז נוהג אלא ברחלים. ותירץ רבנו תם ז״ל דכל חד וחד מידריש לפי עניניה, דהכא ליכא למימר הכין, דהא ילפינן ביבמות בפרק האשה רבה (יבמות פט, א) דהתורם מן הטמא על הטהור שהיא תרומה, ותנן (תרומות פ״ב, מ״ב) התורם מן הטמא על הטהור בשוגג תרומתו תרומה, וכדרבי אילעא דאמר (יבמות שם בע״ב) מנין לתורם מן הרעה על היפה שתרומתו תרומה דכתיב (במדבר יח, לב) ולא תשאו עליו חטא וגו׳, הלכך דרשינן מיניה לו ולא לאורו מכלל דבת אורו היא.}
\textblock{ הא דאמרינן:\textbf{ ואינהו סבור משום כסות לילה.} פירש רש״י ז״ל (בד״ה ב״ש): דטעמייהו דב״ש דפטרי משום דלא דרשי סמוכין, ומתכלת בלחוד הוא דפטרי אבל במינו חייב, וב״ה מחייבי אפילו בתכלת דדרשי סמוכין, ואינהו הטילו לבן בלחוד ולא תכלת, דאינהו סבור אף על גב דקיימא לן כבית הלל, מכל מקום חכמים גזרו ביה משום כסות לילה. והקשו עליו בתוספות (בד״ה סדין), דהא קיי״ל דאפילו מאן דלא דריש סמוכין בכל התורה כולה במשנה תורה דריש בריש פ״ק דיבמות (ד, א). ועוד מדקאמר סדין בציצית ולא קאמר סדין בתכלת משמע דלגמרי פטרי ליה ואפילו ממינו.\par \textbf{} ור״ת ז״ל פירש (שם), דאפילו לב״ש דבר תורה חייב ואפילו בתכלת, אלא דפטרי ליה לגמרי ואפילו במינו משום דלא אפשר להטיל בו תכלת, וכדמפרש התם במנחות (מ, ב) טעמא משום כסות לילה אי נמי גזירה שמא יקרע סדינו בתוך שלש ויחזור ויתפרנו והתורה אמרה (דברים כב, יב) תעשה ולא מן העשוי, לפיכך פטרוהו מהכל ואפילו להטיל בו מינו, דתכלת מעכב את הלבן, ובית הלל לית להו גזירה כלל ומחייבי ליה אפילו בתכלת, ואינהו בהא סבור כבית שמאי משום כסות לילה. וכן דעת הרב בעל המאור ז״ל. ולשון אינהו סבור אינו הולמו.}
\textblock{\textbf{הדלקת נר בשבת חובה.} איכא מאן דדייק מדקרי לה חובה שמע מינה אינה טעונה ברכה, כדאמרינן (חולין קה, א) מים ראשונים מצוה, מים אחרונים חובה ואין טעונין ברכה. ור״ת ז״ל כתב דמים אחרונים לאו משום דאינן טעונין ברכה קרי להו חובה, אלא משום דהן חמורין יותר מן הראשונים לפי שהרגו את הנפש, ועוד שהרי משום מלח סדומית שיש בהם משום סכנת נפשות קרי להו חובה, ומיהו כיון דלית ביה משום חשש איסור אלא משום שמא יבא לידי סכנת נפשות אינן טעונין ברכה, דאינן אלא להצלה בעלמא, אבל הדלקת נר בשבת דאיכא משום מצות עונג שבת טעון ברכה, ולהכי קריא לה חובה משום דלגבי רחיצת ידים ורגלים בחמין חובה דרחיצת חמין בערב שבת אינה מצוה כ״כ. וכן כתב רב עמרם בסידורו הידוע, המדליק בשבת מברך אשר קדשנו במצותיו וצונו וכו׳. ואומר ר״ת ז״ל שמצוה להדליק לשם נר שבת, שאם היתה דולקת והולכת זקוק לכבותה ולחזור ולהדליק, וכדאמרינן לעיל (שבת כג, ב) א״ל ההוא סבא תנא ובלבד שלא יקדים ובלבד שלא יאחר.}
\textblock{\textbf{מטה מוצעת ואשה מקושטת לתלמידי חכמים.} פירוש מקושטת מלשון קשט עצמך (בבא מציעא קז, ב וש״נ), כלומר: מתוקנת והגונה במעשיה וראויה לתלמידי חכמים, וכענין שאמר רבי עקיבא בסמוך כל שיש לו אשה נאה במעשיה.}
\textblock{\textbf{בית הכסא סמוך לשלחנו.} פירוש: סמוך בזמן, וכדתנן (יומא כ, א תמיד כו, א) או סמוך לו בין מלפניו בין מלאחריו.}
\textblock{\textbf{אמר ליה אביי ולימא מר מפני שהוא עף אמר ליה חדא ועוד קאמינא.} איכא למידק אי מפני שהוא עף מאי שנא שבת אפילו בחול נמי, כדאמרינן בסמוך אין מדליקין בנפט לבן בחול ואין צריך לומר בשבת. ונראה לי דהכי קאמר ליה, ולימא מר מפני שהוא עף ואין מדליקין בו כלל קאמר, דהא לא קתני אין מדליקין בצרי בשבת אלא סתמא ויש במשמע שאין מדליקין בו כלל, אמר ליה חדא ועוד קאמינא, חדא מפני שהוא עף ואפילו בחול אסור ועוד יש איסור נוסף בשבת מפני שמסתפק ממנו.}
\clearpage
\newsection{דף כו}
\textblock{\textbf{אין מדליקין בטבל טמא בחול ואין צריך לומר בשבת.} פירש רש״י ז״ל: וכל שכן בטהור. והקשו בתוס׳ הא מהיכא תיתי, אי משום דאמרינן מה תרומה טהורה אין לך בה אלא משעת הרמה ואילך, הא ודאי ליכא למימר דאין לך בה הדלקה אלא משעת הרמה קאמר, דהא לאחר הרמה ודאי אסור להדליקה. ועוד דלמאן דאמר חולין הטבולין לתרומה לאו כתרומה דמו מאי איכא למימר. אלא טבל טמא דוקא קאמר, והכי קאמר מה תרומה טהורה אין לך אכילה אלא משעת הרמה ואילך אף תרומה טמאה אין לך בה אכילה דהיינו שריפתה אלא משעת הרמה ואילך, ולעולם טבל טהור מותר להדליק בו, דכיון דשייכא בה אכילה ממש אין אסור בה אלא אכילה ממש.\par \textbf{} ותמיהא לי מאי האי דקתני ואין צריך לומר בשבת, דאיזה איסור נוסף בו בשבת, דאי משום דאין שורפין קדשים ביום טוב ובשבת, תינח למאן דאמר חולין הטבולין לתרומה כתרומה, אלא למאן דאמר לאו כתרומה מאי איכא למימר. וצל״ע.}
\textblock{\textbf{וכן היה רבי שמעון בן אלעזר אומר צרי אינו אלא שרף מעצי הקטף.} פירש רש״י ז״ל: לשון ראשון, שהוא נתינת טעם למה אין מדליקין בו, מפני שהוא יוצא מן העץ, וכל היוצא מן העץ אין מדליקין בו מפני שאינו נמשך אחר הפתילה. ולזה הפירוש קשיא, דאם כן למה להו לאביי ורבא לעיל לפרושי טעמיה דר׳ שמעון מפני שהוא עף וכן מפני שריחו נודף וגזירה שמא יסתפק ממנו, והא ר׳ שמעון בן אלעזר עצמו מפרש טעמו. ועל כן היה יותר נראה כלשון שני שפרש״י ז״ל, שאינו נתינת טעם אלא הכי קאמר ועוד היה ר׳ שמעון בן אלעזר אומר דבר אחר בצרי. אלא שמצאתי בירושלמי (בפרקין ה״ב) מפורש כלשון הראשון, דגרסינן התם תניא רבי שמעון בן אלעזר אומר אין מדליקין בצרי מפני שהוא שרף. והלכך יש לי לומר דאביי ורבא טעמיה דנפשייהו קאמרי ולא טעמא דרבי שמעון בן אלעזר, משום דקיימא לן שאינו אסור אלא שמנים שמנו חכמים בלבד, הא שאר שמנים אפילו יוצא מן העץ שרו, ובהא קאמר רבא דאפילו לרבנן אין מדליקין בו מפני שהוא עף או משום שמא יסתפקו ממנו, כך נראה לי. וכן נראה מדברי הרב אלפסי ז״ל שהביא הא דאביי ורבא ולא הביא הא דר׳ שמעון בן אלעזר עצמו.}
\textblock{\textbf{סומכוס היינו תנא קמא איכא בינייהו דרב ברונא אמר רב ולא מסיימי.} פירש רש״י ז״ל איכא בינייהו דרב ברונא דאמר צריך ליתן בתוכו שמן כל שהוא, אי נמי דאכשר חלב מהותך בשמן כל שהוא, דחד מהני תנאי סבירא ליה דחלב מהותך ניתר בתערובת כרב ברונא אבל שמן דגים אפילו בעיניה ודלא כרב ברונא, ואידך סבר דחלב כלל כלל לא ודלא כרב ברונא ושמן דגים ע״י תערובת וכדרב ברונא. ואינו מחוור. דאם כן ליכא חד מינייהו דאית ליה דרב ברונא, אלא חד מינייהו פליג עליה בחדא ואידך פליג עליה בחדא. אלא ודאי שמן דגים לאו היינו קרבי דגים, וחד מהני תנאי סבר דשמן דגים אפילו בעיניה, אבל קרבי דגים אפילו ע״י תערובת והיינו דרב ברונא, ואידך סבר דשמן דגים על ידי תערובת אין בעיניה לא, אבל קרבי דגים כלל כלל לא ודלא כרב ברונא. וכן כתב ר״ח ז״ל דשמן דגים וקרבי דגים תרי מילי נינהו. וכן דעת רבותינו הצרפתים ז״ל (בד״ה איכא).}
\textblock{\textbf{אמר אביי רבי שמעון בן אלעזר ותנא דבי רבי ישמעאל אמרו דבר אחד.} והא דקאמר רבי שמעון בן כל היוצא מן העץ לאו דוקא, דהוא הדין לכל שאר בגדים חוץ מצמר ופשתים ואף על גב דאין יוצאין מן העץ כתנא דבי רבי ישמעאל, אלא מדאיצטריך למימר חוץ מפשתן דיוצא מן העץ נקט כל היוצא מן העץ. ואם תאמר ומנין לו לאביי הא, דילמא רבי שמעון בן אלעזר דוקא קתני. לא היא, דהא אי אפשר למעט שום בגד אלא מדפרט הכתוב (ויקרא יג, מז) בנגעים צמר ופשתים, וכיון שכן וקא דריש צמר ופשתים למעט שאר יוצאין מן העץ על כרחין אף שאר בגדים נתמעטו.}
\textblock{\textbf{רבא אמר שלשה על שלשה בשאר בגדים איכא בינייהו.} רבא סבר דרבי שמעון דאמר אין בו משום שלש על שלש דוקא קתני שלש על שלש, דאי לא ליתני סתם כתנא דבי רבי ישמעאל, ואביי סבר דלאו דוקא, אלא דאגב אורחיה אתא לאשמועינן דשיעור פשתים נמי שלש על שלש. ואם תאמר ומאי דוחקיה דאביי לאפוקה לדר׳ שמעון מפשטה ואמר דשלש על שלש לאו דוקא. יש לומר משום דאביי מיסבר סבר דאם איתא דנהגא טומאת שרצים בשאר בגדים כנ״ל על כרחין אפילו שלש על שלש מטמא בהן, (ד)מדגלי רחמנא בנגעים בצמר ופשתים דשלש על שלש כבגד גדול, וק״ו הדברים ומה נגעים קלים שאין שאר הבגדים מטמאין בהם, עשה בהם שלש על שלש כשלשה על שלשה, שרצים חמורים ששאר בגדים מטמאין בהם, אינו דין שנעשה בהן שלש על שלש כשלשה על שלשה.\par \textbf{} ואם תאמר והא אביי הוא דאמר לקמן (שבת כז, א) דלא אמרינן מדגלי רחמנא בנגעים הוא הדין לשרצים. התם הוא למאי דסבירא ליה דאין שאר בגדים מטמאין כלל בשרצים, והלכך כיון דבשרצים לא מטמא אלא צמר ופשתים בלחוד, אי לא דרבי בהו נמי קרא שלש על שלש בצמר ופשתים לא ילפינן להו ממאי דגלי רחמנא בנגעים, דאדרבא השתא איכא למימר דנגעים חמירי שכן שתי וערב טמא בהן.\par \textbf{} ולרבא אפשר נמי דרבי שמעון בן אלעזר אית ליה כתנא דבי רבי ישמעאל, ונלמוד סתום מן המפורש, ומה כאן צמר ופשתים אף כל צמר ופשתים, ואי לאו דכתב רחמנא (ויקרא יא, לב) גבי שרצים או בגד, לא מטמא בהו שאר בגדים כלל, אלא דרבינהו קרא מאו בגד, וכיון דלא אתי אלא מרבויא לא מרבינן בהו אלא בגד שלם דהיינו שלשה על שלשה הא שלש על שלש לא, ולא אמרינן בהו מדגלי רחמנא בנגעים דשלש על שלש כשלשה על שלשה אף בשרצים נעשה בהן שלש על שלש כשלשה על שלשה, דע״כ לא אמרינן מדגלי רחמנא אלא במה שנתפרש עיקר טומאתו בנגעים כצמר ופשתים, דכיון דטמאן הכתוב בנגעים אף על פי שטיהר בהן שאר בגדים נעשה בהן שלש על שלש כשלשה על שלשה, אלמא גלי לן בהאי רחמנא דבגד שלש על שלש דצמר ופשתים חשיב כבגד שלשה על שלשה, אבל בשאר בגדים דלא הוו מטמאי כלל אי לאו רבויא דאו, כי מרבינן בהו לא מרבינן אלא בגד שלם דהיינו שלשה על שלשה.\par \textbf{} ואם תאמר ולרבא מנא ליה דרבי שמעון בן אלעזר דאית ליה דאף כל צמר ופשתים כתנא דבי רבי ישמעאל, דילמא לרבי שמעון בן אלעזר לא ילפינן סתום ממפורש, ובנגעים דאית בהו משום שלש על שלש בצמר ופשתים מרבויא דוהבגד (ויקרא יג, מז), ובשרצים מדגלי רחמנא בנגעים, ושאר בגדים בנגעים כלל כלל לא דתרי מיעוטי כתיבי, ושאר בגדים דשרצים מדכתיב או בגד, וכאן גלי לן הכתוב דבגד האמור בנגעים לא בנה אב שהרי בגד האמור בשרצים אפילו שאר בגדים במשמע, וכפשטה דאידך תנא דבי רבי ישמעאל אליבא דאביי דמשמע דבכל בגדים האמורים בתורה סתם סבירא ליה דכל בגדים במשמע מדגלי רחמנא בבגד שבשרצים דכל בגדים מטמאין בהן מדכתיב בהו או בגד, והיינו דאיצטריך רבא לרבות שאר בגדים בשרצים מאו בגד ולא קאמר מדכתיב בגד סתם, דהכי קאמר, בגד אילו לא נאמר אלא בגד הייתי אומר דאין לי אלא בגד צמר ופשתים, מדפרט לך הכתוב באחד מהם בגד צמר ופשתים דההוא בנה אב, מנין לרבות שלשה על שלשה בשאר בגדים, כלומר: מנין דבגד סתם לאו היינו צמר ופשתים דוקא אלא כל בגדים במשמע, תלמוד לומר או בגד, כלומר: כתב רחמנא או לגלויי דלא אמרינן דבגד סתם היינו צמר ופשתים, וצמר ופשתים דפרט הכתוב בנגעים לחדושו בא ולומר דאין שאר בגדים מטמאין בנגעים כלל, ומיהו שלש על שלש בשאר בגדים לא, משום דבגד סתם לא משמע פחות משלשה על שלשה.\par \textbf{} ויש לומר מדאמר אביי ר׳ שמעון ותנא דבי ר׳ ישמעאל אמרו דבר אחד, ואמר רבא שלשה על שלשה איכא בינייהו, משמע דרבא נמי מודה דאמרו דבר אחד דבגד סתם אינו בכל מקום אלא צמר ופשתים, אלא בשרצים דרבינהו רחמנא מאו בגד, הא בשאר מקומות דלא רבי בהו קרא כטומאת שכבת זרע ואי נמי בציצית אינו אלא צמר ופשתים. והצריכו לרבא לומר כן ואף על גב דאיהו לא סבירא ליה הכין אלא כרבנן דפליגי עליה דתנא דבי ר׳ ישמעאל וכדמחייב איהו שאר בגדים בציצית כדאיתא בסמוך ובריש פרק קמא דיבמות (ד, ב) ובמנחות פרק התכלת (מנחות לט, ב), משום דקשיתיה ההוא טעמא דאמרן דכל מאן דאית ליה דבגד סתם היינו אפילו שאר בגדים על כרחין אפילו שלש על שלש מטמא בהו כדגלי רחמנא בנגעים שעשה בבגדים המטמאין בהן שלש על שלש כשלשה על שלשה וקל וחומר הדברים כמו שכתבנו.}
\textblock{\textbf{שלשה על שלשה דחזי בין לעניים בין לעשירים אתי בקל וחומר.} פירוש: לאו קל וחומר ממש קאמר, אלא לרווחא דמלתא נקט, דשלשה בטעמא מיתרבו דבגד שלם נינהו דכיון דחזו בין לעניים בין לעשירים מאי שנא מבגד גדול, שאם אי אתה אומר כן אף אני אומר אמה ואמתיים. ותדע לך מדאמרינן בסמוך תרי מיעוטי כתיבי חד למעוטי שלש על שלש וחד למעוטי שלשה על שלשה, ואם איתא אכתי איצטריך קרא אחרינא למעוטי בגד שלם. ועוד מדאמר רבא בגד אין לי אלא בגד שלשה על שלשה בשאר בגדים מנין ת״ל או בגד, ואם איתא נוקמיה לבגד גדול אבל שלשה      שלשה לא. ועוד לאביי דאמר האי או בגד מיבעי ליה לרבות שלש על שלש בצמר ופשתים דמטמא בשרצים, הוה ליה למימר לרבות שלשה על שלשה בצמר ופשתים, דהא בשרצים לא כתיב שתי וערב, ואכתי שלשה על שלשה בצמר ופשתים דמטמא בשרצים מנא ליה. אלא דטעמא כדאמרן דשלשה על שלשה בטעמא אתו, דהיינו שלשה על שלשה היינו בגד שלם כיון דחזי בין לעניים בין לעשירים.\par \textbf{} ומיהו קשיא לי הא, מדאמרינן ואימא לרבות שלשה על שלשה, ואהדר ליה ולאו ק״ו הוא, ואי איתא מאי קאמר ואימא לרבות שלשה על שלשה, דהא אינהו לא צריכי קרא דהיינו בגד. ויש לומר דמעיקרא לא איתברר להו דלהוי שלשה על שלשה כבגד אלא מקל וחומר דשתי וערב, אבל השתא דכתיב שתי וערב ומייתי להו לשלשה על שלשה מינייהו בק״ו, ממילא שמעינן דבכל דוכתא שלשה על שלשה כבגד שלם כיון דחזו בין לעניים בין לעשירים כשתי וערב, אבל שלש על שלש אף על גב דרבייה רחמנא בנגעים מוהבגד, לאו גלויי מילתא בעלמא הוא שיהא כבגד שלם, כיון דלא חזי לעניים ולעשירים כבגד גדול, אלא היכא דרבייה רבייה, הא בעלמא לא. כך נראה לי.}
\textblock{\textbf{טעמא דכתב קרא הא לא כתב קרא לא אמרינן.} ודוקא שלש על שלש הא פחות מכן לא מטמא. ואיכא למידק דהא אמרינן לקמן (שבת סג, ב) דאריג כל שהוא טמא. תירצו בתוס׳ (ע״א ד״ה אין) דהתם בשאין דעתו לארוג בו יותר. והא דאמרינן נמי לעיל דשלש על שלש אין מסככין בו דמשמע הא פחות מכן מסככין, איכא למידק דהא אמרינן בפ״ק דסוכה (טז, א) דמטלניות שאין בהן שלש על שלש אין מסככין. י״ל דההיא בבא מבגד גדול, והכא בשלא ארג בו יותר עדיין.}
\textblock{\textbf{ואימא לרבות שלשה על שלשה בשאר בגדים.} ואיכא למידק והיאך אפשר לומר כן, והלא כבר אמרו דשלש על שלש בצמר ופשתים לא מיטמו אלא מרבויא דוהבגד דמקל וחומר לא אתי, וכשאתה מרבה ממנו שלשה על שלשה בשאר בגדים, אם כן שלש על שלש אפילו בצמר ופשתים לא מיטמו דמנא תיתי, והלכך צמר ופשתים ושאר בגדים שוין הן לגמרי, ואם כן צמר ופשתים למאי אתא. ואם תאמר דהיינו מאי דקא מתרץ ואזיל אמר קרא (ויקרא יג, מז) צמר ופשתים צמר ופשתים אין מידי אחרינא לא, הא לא אפשר, דאם כן מאי קא פריך תו ואימא כי מיעט שאר בגדים משלש על שלש אבל שלשה על שלשה מיטמו, ומנא לן דמיטמו כיון דאיצטריך והבגד לרבות שלש על שלש בצמר ופשתים.\par \textbf{} ורש״י ז״ל פירש: דקסבר דשלש על שלש בצמר ופשתים אתי בקל וחומר. ואינו מחוור, דהא אמרינן דשלש על שלש בצמר ופשתים לא אתי בקל וחומר ומינה קא סליק. ואפילו אתה אומר דהשתא קא הדר ביה, הוה ליה לפרושי הכין בהדיא, ואימא לרבות שלשה על שלשה בשאר בגדים, דאילו שלש על שלש בצמר ופשתים מקל וחומר קא אתי.\par \textbf{} ופירשו בתוס׳ דהשתא משמע ליה לתלמודא דוהבגד דאתא לרבות כל שהוא אפילו של שלש על שלש דחזי מיהא לעניים והוא הדין לשלשה על שלשה דשאר בגדים, ומיהו שלש על שלש דשאר בגדים אי אפשר לרבות דאם כן צמר ופשתים למאי אתא, ומאן דמתרץ לה אמר קרא בגד צמר ובגד פשתים הוה טעי וסבר דמקשה לא הוה ידע דצמר ופשתים כתיב, והיינו דמהדר ליה אמר קרא בגד צמר בגד פשתים, והיינו דהדר פריך ואימא כי מיעט קרא, כלומר: הכי קאמינא אימא כי אמעיט משלש על שלש, וביאר לו עכשיו מה שהקשה לו מעיקרא. ואינו מחוור כלל.\par \textbf{} ועוד קשה למה שפירש רש״י ז״ל אמר קרא בגד צמר ופשתים לאחר שכלל פרט לומר דאפילו בגד גדול דוקא צמר ופשתים, ואם כדבריו מאי קא מקשה תו ואימא כי מיעט שאר בגדים משלש על שלש, דהא אחר שכלל בגד שהוא גדול פרט.\par \textbf{} ויש לפרש ואימא לרבות שלשה על שלשה בשאר בגדים הא שלש על שלש לא מיטמו כלל ואפילו בצמר ופשתים, דהא אמאי דקאמרינן דכולי עלמא מיהא שלש על שלש בצמר ופשתים מטמא מנא לן אתינן. ומשום הכי קא דייק ואזיל אכתי מנא לן, דאימא דוהבגד לא לרבות שלש על שלש קא אתי דאיהו לא חזי אלא לעניים דטפי משמע דנרבה מיניה בגד שלם בשאר בגדים דחזי בין לעניים בין לעשירים דהאי והבגד מרבה בגד שלם, ואהדר ליה אמר קרא בגד צמר בגד פשתים, ואם כן למה פרט בהן הכתוב כיון דליכא חילוק בין צמר ופשתים לשאר בגדים ואיצטריך למיכתב והבגד לרבות שאר בגדים, לא ליכתוב רחמנא צמר ופשתים ולא ליכתוב והבגד, הדר אקשי, אין ודאי מדכתיב צמר ופשתים למעט בא והבגד לרבות שלש על שלש בצמר ופשתים, אלא דאכתי אימא דכי מיעט רחמנא שאר בגדים משלש על שלש דרבינהו רחמנא בצמר ופשתים, ומאי דרבי בצמר מיעט בשאר בגדים, ואהדר ליה תרי מיעוטי כתיבי, חד למעוטי שלש על שלש וחד למעוטי שלשה על שלשה דהיינו בגד שלם.}
\clearpage
\newsection{דף כז}
\textblock{\textbf{נפקא ליה מאו בגד (ויקרא יא, לב) דתניא.} פירש רש״י ז״ל: דלא גרסינן דתניא, דאי ברייתא היא היכי פליג עלה אביי. ואין צורך לכך, דאביי מוקי לה כאידך תנא דבי רבי ישמעאל כדאמר אביי גופיה בסמוך.}
\textblock{\textbf{ורבא גלי רחמנא בנגעים והוא הדין בשרצים.} קשיא לי אם כן ליטמא שתי וערב בשרצים, מדגלי רחמנא בנגעים. וניחא לי דבגד כתיב ולא שתי וערב, אבל שלש על שלש בגד איקרי אע״ג דלא חזי לעשירים. ועוד דאי אפשר לרבות שתי        וערב בשרצים, דדומיא דשק קאמר דבעינן מיטלטל מלא וריקן.}
\textblock{\textbf{ואביי איכא למיפרך.} ודוקא לאביי אליבא דתנא דבי רבי ישמעאל ורבי שמעון בן אלעזר, אבל לאידך תנא דבי רבי ישמעאל מודה אביי דדרשינן ליה כדדריש ליה רבא, ורבא נמי דריש ליה לתנא דבי רבי ישמעאל כדדריש אביי, ולא פליגי אביי ורבא אלא אליבא דרבי שמעון בן אלעזר.}
\textblock{\textbf{הדר ביה רבא מההיא.} איכא למידק הא אמרינן בריש פרק קמא דיבמות (ד, ב) והא תנא דבי רבי ישמעאל לית ליה דרבא בההיא דהכנף מין כנף וכו׳, ומנא ליה דלית ליה, הא רבא אית ליה דאף כל קאי אשרצים, דאי לאו אף כל הוה אמינא דמיירי גבי שרצים בכל בגדים. ותירצו בתוס׳ דשמעתא דהתם אזלא כרב נחמן בר יצחק דאמר הכא דאף כל למעוטי שאר בגדים מציצית, ולדידיה תנא דבי רבי ישמעאל לית ליה הא דרבא אלא כל שאר מינין פטורין מן הציצית.\par \textbf{} ומיהו אכתי קשיא דבפרק התכלת (מנחות לט, ב) משמע דתנא דבי רבי ישמעאל מפיק מדרבא, משמע דרבא אית ליה כתנא דקתני השיראין והכלך והסריקין כולן חייבין בציצית, וקא סבר דמדאורייתא קאמר מדקתני בסיפא כולן צמר ופשתים פוטרין בהם, וכדאיתא התם דפריך מינה לרב נחמן דאוקמה מדרבנן, ומסיק התם דרב נחמן כתנא דבי רבי ישמעאל ורבא כתנא דברייתא, ואם כן תהדר קושיין לדוכתיה.\par \textbf{} ולדידי לא קשיא לי, דהא איכא למימר דלרבא לתנא דבי רבי ישמעאל נמי שאר מינין חייבים ואף כל למעוטי שרצים הוא דאתא כי הכא, אלא רב נחמן סבר דאף כל למעוטי לגמרי כל בגדים שנאמרו בתורה סתם ולית ליה כאידך תנא דבי רבי ישמעאל כדמשמע התם בפרק התכלת, ואי נמי אף כל לציצית הוא דאתא כרב נחמן בר יצחק. והיינו דכי אמר רב נחמן התם בריש פרק התכלת השיראין פטורין מן הציצית איתיביה רבא מדתניא השיראין והכלך והסריקין כולן חייבין בציצית, ואם איתא דתנא דבי רבי ישמעאל אית ליה פטורא בשאר בגדים אפילו לדעתיה דרבא, אם כן מאי פריך ליה לרב נחמן מההיא ברייתא דהא רב נחמן על כרחין כתנא דבי רבי ישמעאל סבירא ליה, אלא דקסבר רבא דאף תנא דבי רבי ישמעאל לא אמר אף כל אלא למעוטי שרצים וברייתא תנא דבי רבי ישמעאל היא וכדהדרנא ואמר הכא רבא, ורב נחמן נמי כתנא דבי רבי ישמעאל מוקי לה ומדרבנן, וקא דייק לה רב נחמן מדקתני בסיפא דברייתא דשאר מינין במינן פוטרין ואי מדאורייתא צמר ופשתים בלחוד הוא דפטרי, ופרקינן אי משום הא לא קשיא דכתיב הכנף מין כנף וכו׳, כלומר: ולעולם שאר מינין פוטרין מדאורייתא במינן ותנא דבי רבי ישמעאל נמי היא, וכדתניא בהדיא תנא דבי רבי ישמעאל בגד אין לי אלא בגד צמר ופשתים מנין לרבות השיראין וכו׳, ורב נחמן סבר דהאי תנא מפיק מאידך תנא דבי רבי ישמעאל ואיהו סבר כאידך תנא.\par \textbf{} והרמב״ן ז״ל תירץ, דכל הני אמוראי כולהו סבירא להו דתנא דבי רבי ישמעאל אף כל למעוטי כל מקום שנאמר בתורה בגדים סתם קאמר, וכללא הוא, ולא מיתפקא מכללא אלא משום שרבתה אותם תורה בטומאה דריבה הכתוב מאו בגד ובציצית ליכא ריבוי ואפילו לרבא. והא דהכנף מין כנף לא מרבינן מיניה שאר בגדים לתנא דבי רבי ישמעאל אלא מין כנף, לומר דבעינן כולן ממין כנף כדאיתא בפרק התכלת (מנחות לח, א), וכי אית ליה לרבא דשאר מינין חייבין בציצית לאו ממין כנף יליף אלא בלאו הכי סבירא ליה דשאר מינין חייבין וקסבר דפליגי רבנן עליה דתנא דבי רבי ישמעאל, והיינו דאקשינן התם בריש פרק קמא דיבמות והא תנא דבי רבי ישמעאל לית ליה דרבא, דאלמא מיפשט פשיטא להו דלית ליה לרבא דתנא דבי רבי ישמעאל.\par \textbf{} ולענין פסק הלכה בציצית, רב אלפסי ז״ל (בהלכות קטנות הלכות ציצית) פסק כרב נחמן דאמר התם בפרק התכלת שיראין פטורין מן הציצית מדאורייתא דתניא כותיה דתנא דבי רבי ישמעאל, אבל הם ושאר המינין חייבין בציצית מדרבנן. וכן דעת הרמב״ן ז״ל דכיון דלא אשכחן תנא דפליג עליה דתנא דבי רבי ישמעאל בהדיא ואידך תנא איכא למימר דלא מפקא מיניה כדאמרן הלכתא כותיה, ורב נחמן הכי סבירא ליה כדאיתא בפרק התכלת. ואע״ג דאמרינן בריש מסכת יבמות (ה, א) הא תינח לתנא דבי רבי ישמעאל לרבנן מאי איכא למימר, אליבא דרבא ורב יהודה קאמרינן דסברי דפליגי רבנן עליה, אבל לדידן לא פליגי ולא מוקמינן פלוגתא ביני תנאי שלא לצורך.\par \textbf{} ויש מי שאומר דהא דתנא דבי רבי ישמעאל אינה הלכה משום דאיתוקם בשיטה, דאמר אביי רבי שמעון בן אלעזר ותנא דבי רבי ישמעאל אמרו דבר אחד. ויש מי שאומר דאביי לפרושי מלתייהו הוא דאתא ולא לאוקמינהו בשיטה, ויש כיוצא בה בפסחים פרק כל שעה (פסחים לב, ב) דאמר אביי רבי עקיבא ור׳ יוחנן בן נורי ורבי אלעזר כולהו סבירא להו חמץ בפסח אסור בהנאה והא ודאי אפילו לאביי הלכה פסוקה היא ולא מיקריא שיטה. ורבותינו בעלי התוספות זכרונם לברכה פסקו הלכה כרבא משום דבתרא הוא.\par       \textbf{} ומצאתי בפירושי רב האי גאון ז״ל שהביא ראיה שהלכה כתנא דבי רבי ישמעאל, מדאמרינן לקמן (שבת כח, א-ב) אלא אמר רבא מברניש דכת״ק דרבי ישמעאל דאמר צמר ופשתים מידי אחרינא לא אתיא בקל וחומר מנוצה של עזים שאין מטמא בנגעים כו׳, וזה לשון הגאון ז״ל: ולפי דרכנו למדנו שהכריע רבא מברניש כתנא קמא דרבי ישמעאל דאמר צמר ופשתים אין מידי אחרינא לא, והוציא את דברי תנא דבי רבי ישמעאל מן ההלכה שהוא מביא נוצה של עזים והשיראין והכלך והסריקון, עד כאן. ולא ירדתי לסוף דעתו, דלגבי נגעים ליכא מאן דפליג דדוקא צמר ופשתים דתרי מיעוטי כתיבי ולא נחלקו אלא בשרצים, ובשרצים הוא דקא מרבה אידך תנא שאר בגדים מרבויא דאו בגד.}
\textblock{ הא דאמרינן:\textbf{ רבי שמעון בן אלעזר וסומכוס אמרו דבר אחד.} קשיא לן דהא רבי שמעון בן אלעזר כל היוצא מן העץ מסככין בו חוץ מפשתן קאמר, דמשמע בכל ענין ואפילו לא נטוה [אין] מסככין בו, וסומכוס לא פסיל אלא בטווי דוקא. ותירצו בתוס׳ דפשתן משמע דדייק ונפיץ, וקסבר רבי שמעון בן אלעזר דאע״ג דאינו טווי פסול לסכך, וטעמא דידיה משום דכיון דאילו נטוה מטמא בנגעים ופסול דבר תורה השתא דקרוב לטויה פסול מדבריהם, והיינו כסומכוס, דכיון דאמר סומכוס סככה בטווי פסולה מפני שמטמאה בנגעים משמע דפסולה דאורייתא קאמר, וכיון דסבר ליה דבטווי פסולה דאורייתא ודאי פסיל לה כשאינו טווי מדרבנן מפני שהוא קרוב לטויה, הלכך תרווייהו אמרו דבר אחד, דרבי שמעון מדפסיל בדייק ונפיץ בודאי סבירא ליה דבטווי מטמא בנגעים כסומכוס, וסומכוס נמי דמפסיל נמי בטווי דבר תורה בדייק ונפיץ פסיל מדרבנן כיון דקרוב לטויה.\par \textbf{} אלא דאכתי קשיא לי ומנא להו דפשתן דרבי שמעון בן אלעזר היינו דייק ונפיץ, דלמא דלא דייק ונפיץ אי נמי דייק ולא נפיץ. ושמא משום דקתני כל היוצא מן העץ, משמע שכבר הופרש ויצא מן העץ לגמרי דהיינו דייק ונפיץ.\par \textbf{} ומכל מקום עדיין קשה דהכא משמע לפי פירושם דדוקא רבי שמעון בן אלעזר וסומכוס פסלי לה, הא רבנן פליגי עלייהו ומכשרי ליה, ואילו בסוכה (יב, ב) אמרינן באניצי פשתן פסולה בהוצני פשתן כשרה והושני פשתן איני יודע, ומפרשי בגמ׳ דלא פשיטא להו להכשיר אלא בדלא תרי ולא דייק ולא נפיץ, אבל תרי וכל שכן דדייק ולא נפיץ פסולה, ואף על פי שאינו ראוי ליטמא בנגעים כלל.\par \textbf{} ויש מפרשים דאדרבא רבי שמעון בן אלעזר וסומכוס אמרו דבר אחד להקל, דדוקא הראוי לטמא בנגעים בלבד אין מסככין בו הא שאר מינין אף על פי שהוא טווי מסככין בו, אבל לרבנן אפילו אינו טווי דאין ראוי לקבל טומאה ואפילו תרי ולא דייק ואי נמי אפילו שאר מינין אין מסככין בהן, דלאו פסולת גורן ויקב נינהו, ואי נמי מדרבנן.}
\textblock{\textbf{אונין של פשתן.} פירש רש״י ז״ל: אונין פשתן שלא נטוה. ואינו מחוור, דמהיכא תיתי דליטמא אפילו בנגעים דהא לא כתיב אלא שתי וערב. ובערוך פירש שנטוה, ושתי וערב דקאמר הכא היינו של צמר משום דבצמר ניכר בין שתי לערב אבל בפשתן לא מינכר, וכדמשמע מדאמרינן בפרק קמא דע״ז (יז, ב) כשנתפס רבי אלעזר למינות אמרו לו אמאי קרו לך רבי אמר להו רבן של תרסיים אני, אייתיאו ליה תרי קיבורי אמרו ליה הי דשתיא הי דערבא, אתרחיש ליה ניסא אתאי זיבורא איתיב אדערבא אתאי זיבורתא אותיב ליה אדשתיא, אלמא לא מינכר בפשתן, אבל בצמר מינכר שזה טווי כלפי היד וזה כלאחר היד, ודרך היה לשלקו אע״פ שאין רגילות עכשיו.}
\textblock{ מתני׳:\textbf{ כל היוצא מן העץ אין מדליקין בו אלא פשתן.} פירש רש״י ז״ל: כגון קנבוס וצמר גפן. ואינו מחוור, דהא אמרינן בריש פרקין (שבת כא, א) פתילות שאמרו חכמים אין מדליקין בהם מה טעם מפני שהאור מסכסכת בהן, ואילו צמר גפן אין לך פתילה שמושכת השמן ושהאור נדלקת בה יפה כמותה. ונראין דברי ר״ת ז״ל שפירש דקנבוס וצמר גפן לאו יוצאין מן העץ נינהו אלא מיני זרעים נינהו, כדאמרינן (מנחות טו, ב) לא אסרה תורה אלא קנבוס ולוף אבל שאר מיני זרעים מדרבנן הוא דאסירי, ועוד דאמרינן בפרק כיצד מברכין (ברכות מ, א) כל היכא דאי שקלת ליה לפירא הדר אילנא ומפיק גוזא בורא פרי העץ אלמא קנבוס וצמר גפן דכי שקלת ליה לפירא לא הדר מפיק לאו עץ נינהו, ולא הוצרך תנא דמתניתין להתיר פשתן אלא מפני שקראו הכתוב עץ דכתיב (יהושע ב, ו) ותטמנם בפשתי העץ.}
\textblock{\textbf{והא דגרסינן בירושלמי (בפרקין ה״ג) אמר רב שמואל בר רב יצחק כתיב (שמות כז, כ) להעלות נר תמיד, שיערו לומר אין לך עושה שלהבת אלא פשתן בלבד. יש לפרש דדוקא למנורה קאמר, דבדידה כתיב להעלות נר תמיד. אלא דקשיא לי, דהא משמע בריש פרקין (שבת כא, א) דלא אסרו למקדש אלא מה שאסרו בשבת, וטפי פשיטא להו בשבת       } ממקדש, וכדתני רמי בר חמא פתילות שאמרו אין מדליקין בהן בשבת אין מדליקין בהם במקדש. ונראה לומר דההיא דירושלמי לא בא למעט אלא כל היוצא מן העץ כגון עמרניתא דארזא ודערבתא (לעיל שבת כ, ב) דעלה קתני לה.}
\textblock{ הא דאמרינן:\textbf{ כל היוצא מן העץ אינו מטמא טומאת אהלים אלא פשתן.} יש מי שאומר דדוקא בשעושה ממנו אהל וחברו לקרקע דהוי ליה כבית, הא במטלטל כל המאהילין מטמאין. וקשיא לי, דהא ממשכן גמרינן ליה ומשכן מטלטל הוה, ובשמעתין קרשים כעין קרשי המשכן מטהרין ואינהו מטלטלין הוו ועל אדנים היו עומדים. ושמא אפילו מטלטל כיון שעשה ממנו אהל אינו מטמא דגמרינן לה בגזרה שוה וגזרת הכתוב הוא. וכן נראה מדברי רבותינו הצרפתים זכרונם לברכה שכתבו בתוספות אינו מטמא טומאת אהלים אלא פשתן ואף על פי שמתחלה היה בגד כיון שיחדו לאהל בטלי ליה מתורת בגד. ואחר כך מצאתי בתוס׳ כלשון הראשון דכשחברו בקרקע קאמר.}
\clearpage
\newsection{דף כח}
\textblock{ הא דאקשינן:\textbf{ אי מה להלן קרשים אף כאן קרשים.} פירש רש״י ז״ל: דהכי פריך מה להלן איקרו קרשים אהל אף כאן יהא קרוי אהל. והקשו בתוס׳ והלא קרשים לא עשו מהן אלא דפנות ולא סכך ואינו קרוי אהל אלא סכך. ופירשו הם ז״ל דהכי קאמר: אי מה להלן היו קרשים עם הפשתן הכא נמי לא ליטמא טומאת אהלים עד שיהו שם קרשים ופשתים. וכן מצאתי לרבנו שרירא גאון ז״ל בתשובה. ואינו מחוור, דאי משום הא לא איריא דהיינו דאקשינן לעיל בסמוך אי מה להלן שזורין וחוטן כפול ששה פירוש והוא הדין לכל שאר הדברים שהיו במשכן כגון תכלת וארגמן ותולעת שני וקרשים, וכי משני אהל אהל ריבה הוא הדין לכולהו. ותירצו בתוס׳ דדלמא קרשים שאני משום דהן העיקר שהן מעמידין הכל ולפיכך לא יהא קרוי אהל אלא אם כן יש בו קרשים ולא ריבה אהל אלא לענין שאר דברים.\par \textbf{} ודוקא ליטמא הוא דאמרינן דאינו מטמא אלא צמר ופשתים, דכתיב (במדבר יט, יח) והזה על האהל, אבל להביא את הטומאה כל שהוא טפח על טפח מביא את הטומאה, דאתי מקל וחומר דמצורע, כדיליף לה בספרי (פרשת חוקת פיסקא ד) וכמו שפירש רש״י ז״ל.}
\textblock{\textbf{השתא עור בהמה טהורה לא מטמא.} פירוש: ואע״פ שהיה במשכן דהא איכא עורות אילים.}
\textblock{\textbf{עור בהמה טמאה מבעיא.} פירוש: דאפילו תמצא לומר דתחש טמא היה מכל מקום לא יטמא.}
\textblock{\textbf{עור בהמה טמאה ושלקה ביד כהן כו׳.} פירש רש״י ז״ל דלהכי איצטריך לרבות עור בהמה טמאה, משום דקתני בתורת כהנים (פרשת תזריע פרק יד ה״א) צמר (ויקרא יג, מז) אין לי אלא מין בהמה דקה ונאכל, מין בהמה דקה ואינו נאכל בהמה גסה ונאכל בהמה גסה ואינו נאכל עד שתהא מרבה להביא עורות שרצים מנין, תלמוד לומר (שם פסוק מח) בעור.}
\textblock{\textbf{אלא גמר משרצים דתניא אין לי אלא עור בהמה טהורה וכו׳.} ואם תאמר ומנא ליה דקרא איירי טפי בעור בהמה טהורה. תירצו בתוס׳ כמו שתירץ רש״י ז״ל לעיל בנגעים, משום דכתיב (ויקרא יא, לב) שק והוה אמינא מה שק מין בהמה דקה ונאכל אף עור מין בהמה דקה ונאכל. והכי תניא בתורת כהנים (פרשת שמיני פרק ח ה״ד) בהדיא דשק הוי דבר הבא מן העזים. ואע״ג דמרבה בתורת כהנים (שם) אף שעשאה מחזיר, אף על פי כן הוה אמינא כי הוקש עור לשק הני מילי למשמעות דשק דהיינו צמר של עזים אבל לא למאי דמרבינן משק בהמה טמאה. ובאהל גופיה כתיב ביה עור בהמה טהורה דהיינו עורות אילים לפיכך היה אומר דאף כאן עור בהמה טהורה.}
\textblock{\textbf{אלא הא דתני רב יוסף לא הוכשר למלאכת שמים אלא עור בהמה טהורה בלבד למאי הלכתא.} פירש רש״י ז״ל: דאי למילף דתחש טהור הוה מאי דהוה הוה. ואם תאמר והא מבעיא לן בסמוך מאי הוי עלה. יש לפרש דמדלא קאמר הכין בהדיא, וקאמר בהאי לישנא לא הוכשרו למלאכת שמים, משמע דאתא לאשמועינן שום דבר הוראה.\par \textbf{} ובתשובת רבנו האי גאון ז״ל מצאתי שפירש וז״ל: השתא קא מבעיא לן, הואיל וקם לו קל וחומר דרב הונא מברניש דלית ליה פירכא ולמדנו ממנו דעור מטמא טומאת אהלים, הלכך כמה דגמרינן אהל המת ממשכן בגזירה שוה דאהל אהל מה להלן באהל דמשכן צמר ופשתים אף באהל המת צמר ופשתים הם, הכי גמרינן אהל דמשכן מאהל המת מה להלן באהל [המת] עור בהמה טמאה אף באהל דמשכן עור בהמה טמאה, הרי מן הדין הזה הוכשר לאהל משכן עור בהמה טמאה. ואם כן הוא, אלא הא דתני רב יוסף לא הוכשר למלאכת שמים אלא עור בהמה טהורה בלבד למאי הלכתא, הואיל ולאהל משכן הוכשר עור בהמה טמאה, ומפרקינן לתפילין, עד כאן.}
\textblock{הא דאמרינן:\textbf{ לכרכן בשערן ולתפרן בגידין.} איכא למידק והא איהו עור בהמה טהורה קאמר דמשמע עור ממש. וצ״ע.}
\textblock{\textbf{נהי דגמירי שחורות, טהורות מי גמירי.} איכא למידק וליליף נמי רצועות מתורת ה׳ בפיך (שמות יג, ט), דהא ברצועות נמי איכא דל״ת ויו״ד. ויש לומר דהנהו לאו כתיבה הן אלא קשר בעלמא ואינו חשוב כתב. ובתוס׳ אמרו לקמן בפרק במה אשה (שבת סב, א ד״ה שי״ן) דדל״ת ויו״ד לאו הלכה גמירי להו מסיני, ולא גרסינן (בגמרא להלן שם) אלא שי״ן שבתפילין הלכה למשה מסיני, ודייקי לה מהא. ואין צורך. וכבר הארכתי בה במגילה בריש פרק בני העיר (מגילה כו, ב, ד״ה תשמישי) בסייעתא דשמיא.}
\textblock{ הא דאמרינן הכא:\textbf{ דכולי עלמא אית להו דרבי יהודה וכולי עלמא אית להו דעולא.} הוא הדין דהוה מצי למימר דפליגי בדרבי יהודה, ואי נמי בדעולא, דר׳ אליעזר אית ליה ורבי עקיבא לית ליה, אלא משום דקיימא לן כרבי יהודה ביום טוב וכן כדעולא דאמר צריך להדליק ברוב היוצא וקיימא לן נמי כרבי עקיבא בהא דפתילת הבגד, לא אוקימנא פלוגתייהו בהכין כי היכי דלא תקשי לן הלכתא אהלכתא. ואי נמי משמע ליה דבין רבי אליעזר בין רבי עקיבא בשקפלה דוקא קאמרי, ואי בשקפלה דוקא על כרחין רבי עקיבא נמי ס״ל כרבי יהודה. כך נראה לי.}
\clearpage
\newsection{דף כט}
\textblock{\textbf{מסיקין בכלים ואין מסיקין בשברי כלים דברי רבי יהודה רבי שמעון מתיר. מסיקין בתמרים אכלן אין מסיקין בגרעיניהם.} ואיכא למידק מדאמרינן לקמן בפרק נוטל (שבת קמב, ב) למימרא דרבא כרבי יהודה סבירא ליה, והא אמר רבא לשמעיה טוי לי בר אווזא ושדי מעיה לשונרא, ומשני התם כיון דמסרחי דעתיה עלויה מעיקרא, ואם כן הכא נמי מעיקרא דעתיה אגרעינין מאתמול לאחר שיאכל התמרים. ודחקו בתוס׳ דהתם בשנשחטה מאמש, ואף על פי שהיה ראוי למאכל באותה שעה דעתו היה שאם יסריח למחר שיאכילם לשונרא, וכיון שהיה דעתו עליהם ועוד שלא היה האווז צריך לבני מעים אינן כנולד אלא כמאכל בפני עצמו והויא לה הכנה מעלייתא, אבל גרעיני תמרים שהאוכל צריך להם הוי ליה כמאכל אחד והכנתו אינה הכנה מעלייתא והוי ליה כנולד ואסור.}
\textblock{\textbf{והא דרב לאו בפירוש אתמר אלא מכללא אתמר.} ואע״ג דמהא לא שמעינן אלא גרעיני תמרים דמעיקרא מכסו והדר מיגלו, אבל קליפי אגוזים דאיכא למימר דכיון דמעיקרא מיגלי והדר מיגלי לא הוי כנולד, וכדעביד מיניה גמרין צריכותא בסמוך. אפילו הכי איכא למימר דמחדא שמעינן אידך, דלאו טעמא הוא לחלק ביניהן בכך, אלא דלרוחא דמלתא אשמעינן כולהו תלמודא כי היכי דלא ניטעי בה.}
\textblock{\textbf{כי אדליק בה פורתא הוה ליה שבר כלי וכי קא מהפך באיסורא קא מהפך.} מכאן נראה לי דאין איסור ליהנות מן המוקצה, דלא קשיא ליה אלא היכי מהפך בהו, כלומר: שאסור לטלטל, הא אי לא מהפך בהו אף על פי שהתבשיל מתבשל בו אין בכך כלום, והיינו נמי דאמרינן לעיל (שבת כח, ב) גבי פתילת הבגד, הכא בשלש על שלש מצומצמות עסקינן וכולי עלמא אית להו דרבי יהודה וכולי עלמא אית להו דעולא דאמר צריך להדליק ברוב היוצא וכי קא מדליק בשבר כלי קא מדליק, כלומר: ואסור להשתמש בידים בשברי כלים, דאלמא אי לית להו דעולא שפיר דמי, ואף על גב דממילא דולק והולך ונעשה שבר כלי ונהנה ממנו ומשתמש לאורו, שלא אסרו אלא לטלטלו או לאכלו ואפילו להשתמש בו בידים כגון הדלקה או לסמוך בו כרעי המטה ואפילו במקומו שאינו מזיזו ומטלטלו, אבל הנאה הבאה ממילא שפיר דמי. אלא אם כן תדחה דשאני הכא דמיקלא קלי איסורא, הא במוקצה בעין אסור ליהנות ממנו. והראשון עיקר.}
\textblock{ והא דאמרינן:\textbf{ מרבה עליהם עצים מוכנים.} לא מתורת ביטול נגעו בה דכל דבר שהוא בעין וניכר בשעת תשמישו אינו בטל, אלא הכא אינו צריך לבטול אלא צריך רוב כדי שלא יראה כמטלטל את האיסור אלא כמטלטל בהיתר, וכי מטלטל בשבר כלי הוי ליה כטלטול מן הצד. כך נראה לי.      מזה כתבתי במקומו בפרק קמא דביצה (ד, ב) בסייעתא דשמיא.}
\textblock{\textbf{בין מן המוכן בין שלא מן המוכן.} פירש רש״י ז״ל (לקמן ע״ב): דמוכן דרבי אליעזר היינו קופסא ושאינו מוכן תלאו במגוד והניחו אחורי הדלת, ומוכן דרבי יהושע תלאו במגוד והניחו אחורי הדלת ושאינו מוכן זרקו לאשפה. ואינו מחוור. חדא דלמה לן לאורוכי קופסא ואשפה, כיון דאינהו לא פליגי בהו. ועוד דסתמא מוכן ושאינו מוכן דרבי אליעזר היינו מוכן ושאינו מוכן דרבי יהושע. ובתוס׳ פירשו דמוכן דתרווייהו תלאו במגוד ושאינו מוכן הניחו אחורי הדלת, והא דקא אמרינן ורבי אליעזר אמאי קרי ליה שלא מן המוכן דלגבי קופסא לאו מוכן הוא ולא קאמר דלגבי תלאו במגוד לאו מוכן הוא, משום דלא עדיף כולי האי תלאו במגוד מהניחו אחורי הדלת דלקרייה לגביה שלא מן המוכן, והא נמי דאמרינן בדרבי יהושע ואמאי קרי ליה מן המוכן דלגבי אשפה מוכן הוא ולא קאמר דלגבי הניחו אחורי הדלת מוכן הוא, מהאי טעמא נמי הוא, דלא גרע אחורי הדלת כולי האי לגבי תלאו במגוד דלימא הכין.}
\textblock{\textbf{ובלבד שלא יתכוין בחמה מפני החמה ובגשמים מפני הגשמים.} ואף על גב דמודה רבי שמעון בפסיק רישיה ולא ימות, הכא בהיה לבוש יפה שאינו צריך לאותו בגד כלאים לא לחמה ולא לגשמים.}
\clearpage
\newsection{דף ל}
\textblock{ הא דאמרינן:\textbf{ במאי אי בחולה שיש בו סכנה מותר מיבעי ליה.} פירש רש״י ז״ל: דהוא הדין דמיבעי ליה בשאר פטורין דמתניתין כגון עכו״ם ולסטים. ובנמוקי הרמב״ן ז״ל כתוב: ויש לפרש שיודע היה שעכו״ם ולסטים יש בהן סכנה שכן דרכם לעולם, וספיקן נמי מתירין, ולא הוה קשיא ליה בהו ליתני מותר משום דקא סלקא דעתך דמשום חולה נקט בשארא פטור, ומשום הכי קשיא ליה, אי בחולה שיש בו סכנה היה לו לשנות בכולן מותר, אי בחולה שאין בו סכנה לא הוה ליה למתנייה בהדי שארא והוה ליה למיתני בהאי חייב ובשארא מותר. ואני מצאתי בירושלמי (בפרקין ה״ה) כלשון רש״י ז״ל, דגרסינן התם גבי מתניתין דהכא א״ר שמואל בר רב יצחק כיני מתניתין מפני עכו״ם של סכנה ומפני לסטים של סכנה ר׳ יוסי בעי אי מפני עכו״ם של סכנה ליתני מותר.}
\clearpage
\newsection{דף לד}
\textblock{\textbf{אמרו לו שנים צא וערב עלינו.} פירוש: עירובי חצרות, וכן פירש ר״ח ז״ל ועיקר. דאילו בעירובי תחומין הא תנן ספק חשיכה ספק אינה חשיכה אין מערבין. ואין נראה לומר דהני מילי לכתחילה אבל בדיעבד עירובו עירוב, דהא תנן בפרק בכל מערבין (עירובין לה, א) נתגלגל חוץ לתחום נפל עליו גל או נשרף תרומה ונטמאת מבעוד יום אינו עירוב, משחשיכה הרי זה עירוב, אם ספק רבי מאיר ורבי יהודה אומרים הרי זה חמר גמל, רבי יוסי ורבי שמעון אומרים ספק עירוב כשר, ועד כאן לא מכשרי רבי יוסי ורבי שמעון אלא כגון תרומה ונטמאת דאמרינן העמד תרומה בחזקתה והשתא הוא דנטמאת, וכדאיתא התם (לו, א) ספק בתרומה טהורה עירב ספק בתרומה טמאה עירב אין זה ספק עירוב כשר, וכן בנתגלגל חוץ לתחום דאמרינן השתא הוא דנתגלגל, אבל הניח עירובו בין השמשות דספיקא הוא וליכא חזקה לא, וכן כתב רבנו הרב ז״ל. ורש״י ז״ל שפירשה בעירובי תחומין לא מחוור.}
\textblock{\textbf{בין השמשות ספיקא דרבנן הוא ולקולא.} פירוש: לאו למימרא דבין השמשות גופיה דרבנן, אלא ספק בשל תורה הוא, והכא הכי פירושו: הנחת עירוב בין השמשות ספיקא דרבנן, דעירוב דרבנן הוא וספיקא דרבנן לקולא.}
\textblock{ כך היא גירסת הגאונים ז״ל וכן היא בהלכות רב אלפסי ז״ל:\textbf{ אמר רבא מפני מה אמרו אין טומנין בדבר המוסיף הבל ואפילו מבעוד יום, גזירה שמא ירתיח, אמר ליה אביי אי הכי בין השמשות נמי לגזור, סתם קדירות בין השמשות רותחות הן. ואמר (רבא) [רבה] מפני מה אמרו אין טומנין בדבר שאינו מוסיף הבל משחשיכה, גזירה שמא יטמין ברמץ, ויטמין, גזירה שמא יחתה בגחלים.} ופירשה הרמב״ם ז״ל מפני מה אמרו אין טומנין בדבר המוסיף מבעוד יום וכדתנן (לקמן שבת מז, ב) אין טומנין לא בגפת ולא בזבל וכל הטמנה מבעוד יום הוא, גזירה שמא ירתיח, כלומר: שמא מתוך שיטמין בדבר המוסיף תעלה קדירתו רתיחה ויצטרך לגלותה ולהסיר המרותחת משתחשך ויחזור ויכסה ונמצא מטמין בדבר המוסיף בשבת ואסור. אי הכי אפילו בין השמשות נמי לא יטמין בדבר המוסיף, אמר ליה סתם קדירות בין השמשות כבר נחו מרתיחתן, כלומר: כבר נגמרה המרותחת ושוב לא תעלה מרותחת וליכא למגזר. ואמר רבא מפני מה אמרו אין טומנין בדבר שאינו מוסיף משחשיכה וכדתנן במתניתין ספק חשיכה ספק אינה חשיכה טומנין הא ודאי חשיכה אין      כלומר: כלל כלל ואפילו בדבר שאינו מוסיף, גזירה שמא יטמין ברמץ, כלומר: שהוא אפר וגחלים מעורבין יחד ואף הוא אינו מוסיף הבל ואתי לחתויי.\par \textbf{} וזה קשה הרבה, חדא דהיאך אפשר דמבעוד יום אסור להטמין בדבר המוסיף ובין השמשות מותר. ועוד שהוא אומר דסתם קדרות בין השמשות כבר נחו מרתיחתן, ואנו סתם קדרות רותחות הן קאמרינן. ואלו מתשובות הראב״ד ז״ל שהשיב עליו בהשגות.\par \textbf{} ועוד קשיא לי, ומאין לו דבין השמשות מותר בדבר המוסיף שהיה מקשה כל כך להדיא אי הכי בין השמשות ליתסר. ואי משום הא דתנן במתניתין דהכא ספק חשיכה ספק אינה חשיכה טומנין וסתמא קתני טומנין בכל דבר ואפילו בדבר המוסיף, הא ליתא, דספק חשיכה ספק אינה חשיכה לאו דוקא ספק חשיכה הא קודם לכן לא, אלא אדרבה רבותא קא משמע לן דאפילו בין השמשות טומנין וכל שכן מבעוד יום ודומיא דמערבין, ואם איתא אפילו בדבר המוסיף יטמין סמוך לחשיכה וכל שכן מבעוד יום, אלא ודאי מתניתין בדבר שאינו מוסיף דוקא.\par \textbf{} והראב״ד ז״ל (בהשגות שם) פירש לפי הגירסא הזאת, גזירה שמא ירתיח, דכיון דהטמין בדבר המוסיף גלי אדעתיה דרותח קא בעי לה לאורתא וזימנין דמפסיק רתיחה משום דאריך זמניה ומרתח לה משחשיכה, אי הכי בין השמשות נמי ליתסר בדבר שאינו מוסיף, דכיון דשהה מלהטמין מחזי דרותח קא בעי ליה לאורתא, אמר ליה סתם קדרות בין השמשות רותחות הן לאורתא ולא תפסוק רתיחתן. עד כאן.\par \textbf{} ועדיין אין זה נכון בעיני, דכיון דבדבר המוסיף עסקינן, היכי אקשינן סתם אי הכי בין השמשות ליגזר, כלומר: בדבר שאינו מוסיף, דאם איתא הוה ליה לפרושי הכי בהדיא. אלא שיש לי לומר בזה, דכיון דאמר בין השמשות לא יטמין תו לא אצטריך למימר בהדיא בדבר שאינו מוסיף דכבר מבואר וידוע דכל הטמנה אינה אלא בדבר שאינו מוסיף. ואינו מספיק.\par \textbf{} ורש״י ז״ל גריס בקמייתא בדבר שאינו מוסיף ובאחרונה בדבר המוסיף. והוא הנכון, (ורמץ) [דרמץ] דבר המרתיח הוא וכל דבר המרתיח מוסיף הבל.}
\textblock{\textbf{ואי זהו בין השמשות משתשקע החמה וכל זמן שפני מזרח מאדימין וכו׳.} איכא למידק הני תנאי במאי פליגי, דהא ודאי עד צאת הכוכבים הוי יממא וכדאמרינן בריש פרק קמא דמסכת ברכות (ב, א-ב), ודייקי לה מדכתיב (נחמיה ד, טו) ואנחנו עושים במלאכה וכו׳, ותנן בפרק שני דמגילה (כ, א) וכולן שעשו משעלה עמוד השחר כשר, ודייקי לה בגמרא (שם ע״ב) מנא הני מילי אמר רבי זירא אמר קרא ואנחנו עושים במלאכה וגומר וכתיב (שם פסוק טז) והיה לנו הלילה משמר והיום מלאכה, וכי היכי דנפקא לן מהתם דמעלות השחר הוי יממא הכי נמי נפקא מינה דעד צאת הכוכבים הוי יממא. ותו דתניא לקמן (שבת לה, ב) רבי נתן אומר כוכב אחד יום, שנים בין השמשות, שלשה לילה.\par \textbf{} ויש לומר דבהא דאמרינן עלה דההיא פליגי, לא כוכבין הנראין ביום ולא כוכבים שאין נראין אלא בלילה אלא כוכבים בינונים, דמר סבר כל שנראין עד שהכסיף והשוה לתחתון כוכבים בינונים, ומר סבר עד חצי מיל, ומר סבר אפילו לאחר זמן. וההיא נמי דאמרינן בפסח שני (פסחים צד, א) דמשקיעת החמה ועד צאת הכוכבים מהלך ד׳ מילין נמי לא הוי שיעורא ברירא להו, דבההיא מספקא להו איזהו מהלך אדם בינוני.}
\textblock{\textbf{הכסיף התחתון ולא הכסיף העליון נמי בין השמשות.} קשיא לי אי הכי למה ליה פני מזרח מאדימין, לימא משתשקע החמה ועד שיכסיף העליון והשוה לתחתון. ויש לומר דאי אמר הכי לא ידעינן מאי הכסיף דקאמר, והשתא הכי קאמר כל זמן שפני מזרח מאדימין, כלומר: ואפילו נסתלקה אדמימות מן התחתון עדיין בין השמשות עד שיכסיף העליון שאין שם אדמימות כלל, ומכלל זה דהכספה שאמרנו היינו הסתלק האדמימות. ולרב יוסף דאמר כל זמן שפני מזרח מאדימין יום, תחתון של פני מזרח קאמר. כך נראה לי.}
\textblock{\textbf{ואזדו לטעמייהו דאתמר שיעור בין השמשות בכמה רבה אמר רב יהודה אמר שמואל שלשה חלקי מיל.} איכא למידק אשמעתין, דהכא משמע דמשקיעת החמה ועד צאת הכוכבים ליכא אלא תלתא רביעי מיל, ואילו בפסח שני (פסחים שם) תניא משקיעת החמה ועד צאת הכוכבים ארבעת מילין, דתניא התם רבי יהודה אומר עוביה       רקיע אחד מעשרה ביום, תדע כמה מהלך אדם בינוני ביום עשר פרסאות ומעלות השחר ועד הנץ החמה ארבעת מילין משתשקע החמה ועד צאת הכוכבים ארבעת מילין, אלמא אפילו לדעת רבי יהודה הוי משתשקע החמה ועד צאת הכוכבים שהוא לילה ארבע מילין, נמצאת אומר דשיעור בין השמשות לרבי יהודה ד׳ מילין, והכא אמר רבה שלשה רביעי מיל.\par \textbf{} ותירץ ר״ת ז״ל בספר הישר (חלק החידושים סי׳ רכא): דב׳ שקיעות הן, השקיעה הראשונה היא שתשקע החמה בעובי הרקיע עד גמר השקיעה שהלכה כל עובי הרקיע, ואז היא יציאת הכוכבים והוא שיעור ד׳ מילין, אבל השקיעה האמורה כאן היא סוף השקיעה, לפי שכל זמן שהיא כנגד חלונה ועדיין אינה מהלכת אחרי הכפה פני מזרח מאדימין כנגד מקומה של חמה, והוא מה שאמרו כאן כל זמן שפני מזרח מאדימין. וזהו שאמרו כאן משתשקע החמה כלומר: מששקעה כבר, ושם בפסחים אמרו משקיעת החמה כלומר: ששקיעה עצמה בכלל. והיינו נמי דאמרינן לקמן (שבת לה, ב) אמר ליה רבא לשמעיה אתון דלא קים לכו בשיעורא דרבנן אדאיכא שמשא בריש דקלי אדליקו שרגא, דאלמא למאן דקים ליה בשיעורא דרבנן אע״ג דליכא שמשא אפילו בריש דקלא אלא דנסתלקה זריחתה מן הארץ לגמרי אפילו הכי תלינן שרגא דאכתי יממא הוא.\par \textbf{} ולזה הפירוש הסכים הרמב״ן ז״ל. אלא שהוקשה לו הא דאמרינן בסמוך (לה, א) הרוצה לידע שיעורו של רבי נחמיה יניח חמה בראש הכרמל וירד ויטבול בים ויעלה וזה שיעורו של רבי נחמיה, ופירש רש״י ז״ל: שחמה סמוך לשקיעתה נראית על ראשי ההרים ובתוך שירד ויטבול בים ויעלה הוי לילה, ומשמע מיהא דמתחלת שקיעת החמה אנו משערין. ותירץ הוא ז״ל, אם אדם מניח חמה בראש הכרמל וירד ויטבול עדיין יום הוא ועלתה לו טבילה, ומאותו זמן ואילך מתחיל בין השמשות דרבי נחמיה, והרוצה לידע מאיזה זמן מתחיל שיעורו של רבי נחמיה קאמר. והביא ראיה מן הירושלמי (ברכות פ״א ה״א), דגרסינן התם רבי שמואל בר חייא רבי יודן בשם רבי חנינה התחיל לו גלגל חמה לשקוע אדם עומד בראש הכרמל ויורד וטובל בים הגדול ועולה ואוכל בתרומתו חזקה ביום טבל.\par \textbf{} ואין פירושו בזה מחוור בעיני. שהרי לרבה התחלת בין השמשות דרבי יהודה ורבי נחמיה אחד הוא, ואם כן למה ליה למימר הרוצה לידע באיזה זמן מתחיל שיעורו של רבי נחמיה, לימא הרוצה לידע שיעורו של רבי יהודה ורבי נחמיה, כלומר: זמן התחלת בין השמשות שלהם.\par \textbf{} אלא נראה לי שבתוך שיעור זה נכלל כל בין השמשות דרבי נחמיה, ולפי שאין הכל בקיאין בתחלת השקיעה האחרונה כדי שיעמוד ממנה על סוף השקיעה ושיהא מותר לו לאכול בתרומתו, קא יהיב השתא שיעורא דכולי עלמא בקיאין בו, דמכי הוי שמשא בראש הכרמל ירד משם ויטבול בים ויעלה וכשיעלה שם ידע שכבר נשלם זמן בין השמשות ומותר לו לאכול בתרומתו, ויעלה דקאמר אין פירושו ויעלה מטבילתו, אלא יעלה דומיא דירד, כלומר: ירד מראש הר הכרמל ויטבול ויעלה אל ראש ההר ויאכל שם, וזה שיעורו של בין השמשות דרבי נחמיה, כלומר: סוף שיעורו.\par \textbf{} אבל לדידי קשיא לי מה שאמרו בירושלמי בפרק קמא דמסכת ברכות (ה״א) דגרסינן התם: רבי חנניה חבריהון דרבנן בעי, כמה דתימא ערבית נראו שלשה כוכבים אף על פי שהחמה נתונה באמצע הרקיע, פירוש: באמצע עובי הרקיע, ויימר אף בשחרית כן, פירוש: אף על פי שהתחילה החמה לצאת ועדיין הוא בעובי הרקיע נדון אותו כלילה ומפני מה אמרו מעלות השחר הוה יום, אמר רבי בא כתיב (בראשית יט, כג) השמש יצא על הארץ ולוט בא צוערה וכתיב (ויקרא כב, ז) ובא השמש וטהר הקיש יציאתו לביאתו, מה ביאתו משיתכסה מן הבריות כלומר: שנתכסה זריחתו מן הבריות והתחיל להשתקע ברקיע אף יציאתו משיתודע לבריות דהיינו משהתחיל ליכנס ברקיע, אמר רבי בא כתיב (בראשית מד, ג) הבוקר אור התורה קראה לאור בוקר, תני רבי ישמעאל בבקר בבקר (שמות טז, כא) כדי ליתן תחום לבוקרו של בוקר, אמר ר׳ יוסי בר׳ בון אם אומר את ליתן עוביו של רקיע ללילה בין בערבית בין בשחרית, נמצאת אומר שאין היום והלילה שוין, ותני באחד בתקופת ניסן ובאחד בתקופת תשרי היום והלילה שוים, אמר רבי חייא נלפינה מדרך הארץ שרי מלכא נפיק אף על גב דלא נפיק אמרין דנפיק שרי מלכא עייל לא אמרין דעל עד שעתא דעייל, עד כאן בירושלמי, אלמא משמע דבערבית משעה שהתחיל להשתקע בעובי הרקיע הוי לילה. וצריך לי עיון.}
\clearpage
\newsection{דף לה}
\textblock{\textbf{חלתא בת תרי כורי שרי לטלטולה בת תלתא כורי אסור לטלטולה.} קשיא לי דהוה ליה למימר בת תרי כורי שרי לטלטולה בת תרי כורי ומשהו אסור לטלטולה. ונראה דמשום דרב יוסף שרי אף בת תלתא נקט רבה תלתא אסור, ורב יוסף נמי הואיל ואמר רבה תרי ותלתא אמר איהו תלתא וארבעה והוא הדין לתלתא ומשהו.}
\textblock{\textbf{אמר רבה בר בר חנה אמר רבי יוחנן הלכה כרבי יהודה לענין שבת.} כלומר: בכניסת שבת, והיינו לחומרא אבל ביציאתו לא אלא כרבי יוסי. וכי קאמר וכרבי יוסי לענין תרומה, הוא הדין דהוה מצי למימר לענין מוצאי שבת.\par \textbf{} ומהא דרבי יוחנן לא שמעינן אי כרבה אי כרב יוסף. והרב אלפסי ז״ל פסק כרבה משום דקיימא לן (ב״ב קיד, ב) דכל היכא דפליגי רבה ורב יוסף הלכתא כותיה חוץ משדה ענין ומחצה, ועוד דספיקא דאורייתא הוא ולחומרא נקטינן.\par \textbf{} ומורי הרב ז״ל פסק כרב יוסף, דכי קיימא לן כרבה הני מילי כל היכא דפליגי בסברא דנפשייהו ומשום דאמרינן בהוריות (יד, א) רב יוסף סיני רבה עוקר הרים, אבל הכא מפי השמועה הם חולקים, מר אמר לה משמיה דרב יהודה ומר אמר לה משמיה דרב יהודה. ועוד דסוגיין הכא כרב יוסף, דאמרינן אביי חזייה לרבא דקא דוי למערב אמר ליה והתניא כל זמן שפני מזרח מאדימין אמר ליה פנים המאדימין את המזרח, והיינו כרב יוסף דקא דוי למערב למחזי אם פניו מאדימין ואכתי יממא הוא כרב יוסף, דאי כרבה ולמיחזי אם שקעה חמה, מאי קא אמר ליה אביי והתניא כל זמן שפני מזרח מאדימין, והלא פני מזרח המאדימין דאדכר בברייתא אינו סימן ליום וגם לא סימן לסוף בין השמשות, דהא בין השמשות מושך והולך עד שיכסיף העליון והשוה לתחתון. ומיהו אף על גב דמתחזי מהאי עובדא דכדרב יוסף סבירא ליה לא עבדינן ביה עובדא לקולא אלא לחומרא. אלו דבריו. ועוד יש להביא ראיה מברייתא דקא מייתי בירושלמי בפרק קמא דברכות (ה״א) כל זמן שפני מזרח מאדימות זהו יום, הכסיפו זהו בין השמשות, השחירו זהו לילה, והא אתיא כרב יוסף.\par \textbf{} ועם כל זה יש לי ללמד זכות על דברי הרב אלפסי ז״ל, חדא דהא רבי יוחנן ספוקי מספקא ליה בבין השמשות כמאן, והלכך פסק כרבי יהודה בהכנסת השבת שהוא השיעור המוקדם וכרבי יוסי לענין תרומה ומוצאי שבת שהוא השיעור המאוחר שבכולן, וכיון שכן על כרחין לחומרא אזלינן בכולהו. ועוד דאם איתא לימא הלכה כרבי נחמיה לענין שבת, דלדידיה התחלת בין השמשות משתשקע החמה ואף על פי שפני מזרח מאדימין, אלא ודאי רבי יוחנן כרבה סבירא ליה, וכי נקט רבי יהודה הוא הדין דהוה מצי למימר כרבי נחמיה דהתחלת בין השמשות שוה לשניהם, אלא משום דרבי יהודה קדים בברייתא נקט רבי יהודה, ועוד דמשום דרבי יהודה שעוריה משיך טפי מדרבי נחמיה ואפילו הכי לא קיימא לן כותיה לא באפוקי שבתא ולא לגבי תרומה נקטינן לדרבי יהודה.\par \textbf{} ומאי דקשיא ליה לרבנו הרב ז״ל, יש לי לומר דהא דקא דוי למחזי אם הכסיפו פני מזרח דהיינו סלוק אדמימות מן המזרח לגמרי דהכספת פנים היינו סלוק אדמימות כמו שכתבתי למעלה (לד, ב ד״ה הכסיף) וכדי לערב או להטמין קא דוי ביה, ואמר ליה מי סברת דסלוק אדמימות מן המזרח בלבד בעינן סלוק אדמימות כל פני מערב בעינן. כך נראה לי.}
\textblock{\textbf{דלא אכלי כהני תרומה עד דשלים בין השמשות דרבי יוסי.} איכא למידק ומאי נפקא מינה דהא לבתר דשלים בין השמשות דרבי יהודה מתחיל בין השמשות דרבי יוסי, ובין השמשות דרבי יוסי כהרף עין ואין אדם יכול לעמוד בו, ובשיעורא זוטא כי האי מאי נפקא לן בין סוף בין השמשות דרבי יהודה לסוף בין השמשות דרבי יוסי.\par \textbf{} ויש לומר דכי אמרינן (בעמוד א) דשלים בין השמשות דרבי יהודה והדר מתחיל בין השמשות דרבי יוסי, לאו למימרא דכי שלים דרבי יהודה הדר מתחיל מיד דרבי יוסי, אלא למימרא דלא עריב דרבי יוסי בגו בין השמשות דרבי יהודה אלא לבתר דרבי יהודה הוי דרבי יוסי, ולעולם לא מתחיל מיד אלא לאחר זמן.}
\textblock{\textbf{וכתב מורי הרב ז״ל דמסתברא דבין השמשות דרבי יוסי אינו מתאחר לאחר בין השמשות דרבי יהודה כשיעור טבילה, מדאמר רב יהודה אמר שמואל (שם) בין השמשות דרבי יהודה לרבי יוסי כהנים טובלין בו, ומדאמרי הכי ולא אמרי לבתר דשלים בין השמשות דרבי יהודה לרבי יוסי כהנים וכו׳ אלמא אין הפסק ביניהם כשיעור טבילה שלם, ושיעור טבילה פחות מנ׳ אמה דתנן (זבים פ״א מ״ה) ראה אחת מרובה כשלש כמגדיון לשילה שהן [כדי] שתי טבילות ושני ספוגין הרי זה זב גמור ותניא בתוספתא בתחלת מסכת זבין       } (ה״ד) רבי אומר משל למה הדבר דומה לחבל של מאה אמה ראה בתחילת מאה ובסוף חמישים אמה ובסוף מאה אמה הרי זה זב גמור.}
\textblock{\textbf{סילק המסלק.} פירש רש״י ז״ל: קדרות הראויות להסתלק למאכל הלילה. ונראה לי לפי פירושו, דאתיא הא דתנא דבי רבי ישמעאל כרבי יהודה, דאית ליה לקמן פרק כירה (שבת לז, א) דאפילו לשהות אין משהין אלא על גבי כירה גרופה וקטומה, ומתניתין (דלקמן שבת לו, ב) לשהות תנן, אבל לחנניה (דלעיל שבת כ, א) לא היו צריכין לסלק כלל. ואפשר היה לפרש דסילק המסלק והטמין המטמין חדא היא, כלומר סילק המסלק מאכל הראוי לצורך מחר והטמינו, משום דטומן אפילו בדבר שאינו מוסיף הבל אסור להניחו על גבי דבר המוסיף הבל, וכדתניא (לקמן שבת מז, ב עי״ש) קופה שטמן בה אסור להניחה על גבי גפת של זיתים, ולא מפקא האי תנא דבי רבי ישמעאל מדחנניה. כך נראה לי.}
\textblock{ הא דתניא:\textbf{ ושוהה כדי להדביק פת בתנור ולצלות דג קטן.} קשיא לי, דהא משמע דאפילו ליכא שיעורא אלא כדי להדביק הפת בתנור ושיקרמו פניה קודם חשיכה לבד שרי ואף על פי שרדייתה בשבת, והכין נמי משמע במתניתין דפרק קמא (יט, ב) דקתני ואין נותנין פת כו׳ אלא כדי שיקרמו פניה, ואילו בפרק קמא בשמעתא קמייתא (ג,ב. ד,א) משמע דרדיית הפת מדרבנן מיהא אסירא, וכדבעא רב ביבי בר אביי הדביק פת בתנור התירו לו לרדותה קודם שיבא לידי איסור סקילה או לא.\par \textbf{} וראיתי לר״ח ז״ל במסכת ראש השנה ריש פרק יום טוב שחל להיות בשבת, דרדיית הפת שאסרוה היינו בפת רכה שהוא כעין עיסה שרדייתו היינו עריכתו, והיינו דרב ביבי דקודם שיבא לידי חיוב חטאת קאמר, כלומר דעדיין לא קרמה פניה, הא כשנאפה היטב שרי ואפילו לכתחילה. והוצרך לפרש כן שם, מפני שהתירו שם (ר״ה כט, ב) תקיעת שופר אי לאו משום גזירה שמא יעבירנו ד׳ אמות ברשות הרבים, ודייקינן לה מדכתיב (ויקרא כג, ז) כל מלאכת עבודה לא תעשו יצאו תקיעת שופר ורדיית הפת שהיא חכמה ואינה מלאכה, דאלמא אינה מלאכה ומשרא שריא ואפילו לכתחילה. וליתא, חדא דהא דקאמרינן התם היינו לומר דמדאורייתא לא אסירא ונפקא מינה דבמקום מצוה שרי. ותדע לך דהא ודאי תקיעת שופר שלא במקום מצוה אסירא וכדאיתא בשמעתין דהכא, ובפרק בתרא דראש השנה (לג, א) גם כן גבי מתעסקין עמהן עד שילמודו, ואמרינן נמי בפרק כסוי הדם (חולין פד, ב) תקיעת שופר תוכיח שספיקה דוחה יום טוב ואין ודאה דוחה שבת, ואע״פ שאינה מלאכה אלא חכמה. ועוד דהא תניא לקמן בפרק כל כתבי (שבת קיז, ב) שכח פת בתנור וקדש עליו היום מצילין מזון שלש סעודות וכו׳ דאלמא בשכח בלבד התירו לרדות הא במדביק לכתחילה לא התירו. וצריך לי עיון.\par \textbf{} אחר כך מצאתי להרמב״ן ז״ל בספר המלחמות (לעיל שבת ד, א), דהכא ברודה בסכין ולצורך שבת, וכההיא דשכח פת בתנור, שלא גזרו בשבת כלאחר יד לצורך השבת בין בבא להציל בין בעלמא. ועדיין אינו מחוור בעיני כל הצורך, חדא דדחקינן הכא ומוקמינן בכדי מזון שלש סעודות דוקא, ועוד היא צריכה לי עיון.}
\clearpage
\newsection{דף לו}
\textblock{\textbf{לא קשיא הא ר׳ יהודה הא ר׳ שמעון הא ר׳ נחמיה.} פירוש: הא דתניא שופר מטלטל וחצוצרות אין מטלטלין, ר׳ יהודה היא דשרי דבר שמלאכתו להיתר קצת ואע״פ שעיקר תשמישו לאיסור כגון קורנס של נפחים לפצע בו אגוזים, ודוקא לצורך גופו או לצורך מקומו, אבל מחמה לצל או כדי שלא יגנב לא, ואסר כלי המיוחד לאיסור ואפילו לצורך גופו ולצורך מקומו, כמטה שיחדה למעות והניח עליה מעות דאין מטלטלין אותה כלל כדאיתא לקמן (שבת מד, ב) ואוקימנא כר׳ יהודה. והלכך שופר שמגמעין בו מים לפעמים לצורך גופו ולצורך מקומו שרי, אבל חצוצרה שאינה ראויה לגמע בה מים כלל ומיוחדת היא לתקיעה דאסירא הויא לה כמטה המיוחדת למעות ואסירא. והא דתניא כשם שמטלטלין את השופר כך מטלטלין את החצוצרות, ר׳ שמעון היא דשרי לטלטל אפילו כלים המיוחדים לתשמישי איסור לצורך גופן ולצורך מקומן. והא דתניא שאין מטלטלין לא את השופר ולא את החצוצרות, ר׳ נחמיה היא דאמר אפילו טלית אין ניטל אלא לצורך תשמישו בלבד, כלומר: לצורך תשמישו המיוחד לו.\par      \textbf{} ואם תאמר מכל מקום לכולי עלמא כלי שמלאכתו לאיסור אין מטלטלין אותו אלא לצורך גופו ולצורך מקומו הא שלא יגנב לא, ותדע לך מדגרסינן בפרק כל הכלים (לקמן שבת קכד, ב) רבי אלעזר אומר אפילו מכבדות של תמרה מטלטלין, והוינן בה במאי עסקינן אילימא מחמה לצל בהא לימא רבי אלעזר אפילו מכבדות של תמרה מטלטלין, ואי סלקא דעתך דר׳ שמעון בכהאי גוונא שרי, מאי קושיא דלמא ר׳ אלעזר כר׳ שמעון סבירא ליה, אלא שמע מינה דכלי שמלאכתו לאיסור אפילו לרבי שמעון אין מטלטלין אותו מחמה לצל ולא כדי שלא יגנב, וכיון שכן מאי דוחקין דאוקימנא ההיא דשאין מטלטלין לא את השופר ולא את החצוצרות כר׳ נחמיה, לוקמה ככולי עלמא, דהא ההיא כדי שלא יגנב היא וכדאמרינן בהדיא, אלא מקום יש לו בראש גגו ששם מניח שופרו. ויש לומר דלישנא קא דייק דקתני לפי שאין מטלטלין דמשמע דאין מטלטלין כלל.}
\textblock{ הא דאמרינן:\textbf{ למאי נפקא מינה למחט שנמצאת בעובי בית הכוסות.} כתבתיה בפרק אלו טריפות (חולין נא, א ד״ה מחט) בסייעתא דשמיא.גבבא אין מוסיפין הבל.}
\textblock{\textbf{עד שיגרוף.} פירש רש״י ז״ל: משום דמוסיף הבל, וכבר (פרש״י ז״ל) [פרשינן] בפרקין דלעיל (שבת לד, ב) גזירה שמא יחתה בגחלים. ופירש כן הוא ז״ל מפני שהוא מפרש ענין הפרק הזה אף בהטמנה, והוא סבור שכל גרוף וקטום וכן קש וגבבא אין מוסיפין הבל.\par \textbf{} ואני תמיה לדבריו, שהוא ז״ל גורס לעיל בסוף פרק במה מדליקין (שם) מפני מה אמרו אין טומנין בדבר המוסיף מבעוד יום גזירה שמא יטמין ברמץ, ופירש הוא ז״ל רמץ גחלים ואפר מעורבין, והיינו כקטומין, אלמא אף קטומה מוסיף הבל. ועוד קשה עליו דהא מהדרינן לאוקומה למתניתין להחזיר אבל לשהות משהין ומתניתין חנניה היא, ואי בהטמנה לא התיר חנניה לשהות בהטמנה בכירה שאינה גרופה ואינה קטומה ואפילו כמאכל בן דרוסאי, משום דהויא ליה הטמנה בגחלים ממש דמוסיפין הבל ואסור ואפילו מבעוד יום. ואף רש״י ז״ל בעצמו פירש כן לקמן (שבת לז, א) גבי תוכה וגבה, דתוכה אסור כשאינה גרופה וקטומה משום דהויא ליה הטמנה ברמץ דאסרינן מבעוד יום.\par \textbf{} אלא עיקר משנתינו זו אינה הטמנה, אלא כמו שפירשו רב האי גאון ור״ח ז״ל. וז״ל רב האי גאון ז״ל: אין טומנין את הקדרה בדבר שמוסיף הבל בחומה ואפילו מבעוד יום, אבל שהוי על גבי כירה אינה הטמנה שלא הניח את הקדרה עצמה בתוך אש שאין קטום אלא לתלותה עליו ויש ביניהם ריוח או להניח כסא בתוך אש והקדרה עליו וכך אנו עושין, עד כאן. וממה שכתב שלא התירו להניח את הקדרה בתוך אש שאין קטום, משמע דסבירא ליה להגאון ז״ל דבתוך אש קטום טומנין שאינו מוסיף הבל, וכאותה גירסא דגרסי הגאונים ז״ל בשלהי פרק במה מדליקין (לעיל שם) מפני מה אמרו אין טומנין בדבר שאינו מוסיף הבל משחשיכה גזירה שמא יטמין ברמץ, דאלמא רמץ דבר שאינו מוסיף הוא. ומיהו תימה כיון שאין הקטימה אלא קטימה כל שהוא וכדאיתא בירושלמי (פ״ג ה״א) וכדמוכח נמי בגמרא (לקמן שבת לז, א) דאמר גחלים שעממו הרי הן כקטומין, היאך אפשר שלא יוסיפו הבל, ומכל מקום למדנו מדבריו שאין משנתינו בהטמנה אלא בשהוי ובשאין הקדרה נוגעת כלל באש. ור״ח ז״ל כתב כן, שענין משנתנו אינה הטמנה אלא כעין כסא של ברזל והקדרה יושבת עליו והיא תלויה באבנים או בכיוצא בהן, אבל הטמנה על גבי גחלים דברי הכל אסור, דקיימא לן הטמנה בדבר המוסיף מבעוד יום אסור.}
\textblock{\textbf{עד שיגרוף או עד שיתן את האפר.} גרסינן בירושלמי (פ״ג ה״א): הגורף עד שיגרוף כל צרכו, מן מה דתני הגורף צריך לטאט בידו, הדא אמרה עד שיגרוף כל צרכו, הקוטם עד שיקטום כל צרכו, מן מה דתני מלבה עליה נעורת של פשתן הדא אמרה אפילו לא קטם כל צרכו, כלומר שאין צריך לקטום עד שלא יהא האש ניכר בו אלא כיון שקטם קצת מוכחא מילתא שהוא מתיאש ממנו ואינו רוצה בחתוי. והכי נמי משמע בגמרא, מדאמרינן (לקמן שם) גחלים שעממו הרי הן כקטומין, והוא הטעם שאמרנו דכיון שעממו ולא חשש ללבותן מוכחא מילתא שאין קפיד (בחתוי) [בחיתויו], ועוד אמרו (שם) קטמה ונתלבתה סומכין לה ומקיימין עליה, כלומר שהיא כקטומה.\par \textbf{} ומכל מקום לענין גורף, נראה מהירושלמי שהוא צריך לגרוף לגמרי עד שלא ישאר בו אש כלל. ואם תאמר כיון שהוא צריך לגרוף לגמרי, מפני מה אסרו (לח, ב) בתנור גרוף. פירש הרמב״ם ז״ל (פ״ג מהל׳ שבת ה״ו) מפני שאי אפשר שלא ישאר בו ניצוץ אחד וממנו ראוי לחתות ולהסיק. ולדבריו צריכין אנו לומר, דבכירה גרופה התירו מפני שאף על פי שנשארו בה ניצוצות מתוך שהבלה מועט אינה ראויה להתחמם בניצוצות מועטין הנשארים ולא יהיב דעתיה ולא אתי לחתויי, אבל תנור מתוך שהבלו רב יהיב דעתיה ומחתה. ודומיא דקש וגבבא, דבכירה שרי אפילו בשאינה גרופה ובתנור אסיר.\par \textbf{} והר״ז הלוי ז״ל פירש מדוחק קושיא זו, דגריפה שאמרו לא שיהא צריך לגרוף ולהוציא חוץ מן הכירה, אלא שיגרוף ויסלק גחלים לצד אחד ויניח הקדרה במקום הגרוף, והלכך בתנור הגרוף איכא למיגזר דילמא אתי לחתויי בגחלים שבתנור. ואינו מחוור בעיני כלל, דאם איתא מאי קא מיבעיא לן בגמרא (לקמן שבת לז, א), בכירה שאינה גרופה וקטומה מהו לסמוך, דודאי טפי איכא למיגזר לחתויי בגחלים בשהקדרה וגחלים בכירה עצמה ממאי דאיכא למיגזר בשהגחלים בתוכה וקדרה סמוכה לה מחוץ, וכיון שכן כיון דבגרופה מתחת קדרה שרי כל שכן כשהקדרה מחוץ לגמרי. ועוד דהא לאביי משמע לקמן (שבת לח, ב) דכופח שהסיקוהו בגפת ובעצים אי גרופה שרי, ואפילו הכי בשאינה גרופה אין סומכין לה, וזה דבר רחוק. ודוחק הוא לומר דטפי עדיף גרופה משום דמעשיו מוכיחים עליו, דכיון שגרף וסלק לצד אחד לא יהיב דעתיה תו לאחתויי ואסוחי אסח דעתיה מיניה אבל שאינה גרופה אפילו בסמיכתה איכא למגזר.\par \textbf{} והנכון שנאמר דגרופה ממש בעינן, אלא דבתנור כיון שהבלו רב ומרתיח אינו נראה כגרוף אלא כמי שיש שם גחלים, ואתי לאשהויי או לאהדורי בכירה שאינה גרופה ואתי לחתויי, אי נמי אתי לאשהויי או לאהדורי בתנור גופיה דאינו גרוף.\par \textbf{} והאי תבשיל דתנן במתניתין, יש לי לומר דלמאן דאמר לשהות תנן, אם הסיקוה בקש ובגבבא אפילו אינה גרופה ואי נמי בגפת ובעצים בגרופה וקטומה, משהין עליה אפילו לא הגיע למאכל בן דרוסאי, דהא טעמא משום חתויי גחלים ובקש ובגבבא ואי נמי בגפת ובעצים כשגרף וקטם ליכא בכירה משום חתוי גחלים, וכל שכן במאכל בן דרוסאי וכל שכן במצטמק ויפה לו, אבל למ״ד להחזיר תנן אבל לשהות משהין אף על פי שאינה גרופה ואינה קטומה, האי תבשיל דמתניתין דוקא בשהגיע למאכל בן דרוסאי וכדתני בהדיא חנניה כל שהגיע למאכל בן דרוסאי משהין על גבי כירה שאינה גרופה וקטומה, אבל בשיל ולא בשיל שלא הגיע בשולו למאכל בן דרוסאי לא אלא אי שדא גרמא חיה ואי נמי בקדרה חייתא כדאיתא שלהי פרק קמא (לעיל שבת יח, ב). ודוקא כשהסיקוה בגפת ובעצים, אבל בקש ובגבבא לעולם משהין בין הגיע למאכל בן דרוסאי בין לא הגיע, דהא ליכא משום חתוי בקש ובגבבא שבכירה ולא שבכופח, אבל להחזיר לעולם אינו מחזיר אלא כשהגיע למאכל בן דרוסאי הא קודם לכן לא דנמצא מבשל בשבת.}
\textblock{}
\textblock{\textbf{ויש לחלוק ולומר דכל שלא הגיע למאכל בן דרוסאי אין משהין ואפילו בגרופה וקטומה ולא בשהסיקוה בקש ובגבבא, משום דכיון דבשיל קצת ולא הגיע למאכל בן דרוסאי אפילו בקש ובגבבא יהיב דעתיה ומחתה, וכענין שאמרו בעססיות ותורמוסין (שם) דכיון דצריכין בשול גדול יהיב דעתיה ומחתה, ואפילו בקש ובגבבא שייך חתוי בכי הא דהא גזרינן ביה בתנור (לקמן שבת לח, ב). והיינו דכי אקשי בשלהי פרק קמא (שם) צמר ליורה ליגזור (איצטריכא) [איצטריכינן] לאוקומה ביורה עקורה ולא אוקמינן בגרופה וקטומה ואי נמי בקש ובגבבא, משום דצמר ליורה קודם שיקלוט את העין הרי הוא כתבשיל שלא הגיע למאכל בן דרוסאי. ומיהו הראשון נראה עיקר, משום דבגרוף מיהא בכירה ליכא למיחש ולמיגזר למידי. וההיא דצמר ליורה לא      } ניחא ליה לאוקומה בכירה גרופה כיון דכולהו אינך מיירי בתנור, והלכך ניחא ליה טפי לאוקומה אף בתנור כיון דאשכח ביה נמי היתירא. ולשון תבשיל כולל הוא אף תבשיל שלא הגיע למאכל בן דרוסאי.}
\textblock{\textbf{בית שמאי אומרים חמין אבל לא תבשיל.} מדקאסרי בית שמאי תבשיל שמע מינה דאפילו בגרופה אית ליה גזירת חתוי, וכיון שכן נראה לי דחמין שהתירו דוקא בשהוחמו כל צרכן בערב שבת מפני שמצטמק ורע להם הא קודם לכן לא, דהא אפילו בחמין איכא למיגזר משום חתוי גחלים וכדתניא (לעיל שבת יח, ב) לא ימלא נחתום חבית של מים ויניחנה בתוך התנור.\par \textbf{} ואפשר דהוא הדין לתבשיל שמצטמק ורע לו, ולא אסרו אלא בתבשיל המצטמק ויפה לו כסתם תבשיל דמצטמק ויפה לו וכדמשמע בגמרא (לקמן שבת לח, א), דאקשינן מדרבי מאיר דאמר שכח קדרה על גבי כירה ובשלה בשבת חמין שהוחמו כל צרכן ותבשיל שבשל כל צרכו בין בשוגג בין במזיד יאכל אאידך דרבי מאיר דתניא לעיל מינה (לז, א) שתי כירות המתאימות וכו׳ משהין על גבי גרופה וקטומה ואין משהין על גבי שאינה גרופה וקטומה ומה הן משהין בית הלל אומרים חמין אבל לא תבשיל, דשמע מינה דסתם תבשיל היינו מצטמק ויפה לו, דאי לא מאי קושיא דלמא התם במצטמק ויפה לו לפיכך אין משהין והכא במצטמק ורע לו ולפיכך מותר. אלא שיש לדחות דהתם הא פליג בה רבי יהודה ואמר חמין מותרין מפני שמצטמקין ורע להם ולפיכך מותר ותבשיל שבשל כל צרכו אסור מפני שמצטמק ויפה לו, דאלמא רבי מאיר במצטמק ויפה לו נמי שרי. אבל מדאקשינן מדרבי יהודה אדרבי יהודה איכא למידק דדלמא הכא ביפה לו וכדאמר(י׳) בהדיא והתם במצטמק ורע לו, אלא שמע מינה דסתם תבשיל מצטמק ויפה לו. ועוד מדקתני תבשיל שבשל כל צרכו אסור מפני שמצטמק ויפה לו ולא קתני תבשיל שמצטמק ויפה לו אסור, אלמא כל שמצטמק ויפה לו נקיט בלשון תבשיל וכל מצטמק ורע לו נקיט בלשון חמין, כלומר: חמין ותבשיל הדומה לו שמצטמק ורע לו. ועוד דכל היכא דתני חמין ותבשיל משמע דתרי גווני נקיט, כלומר: מצטמק ורע לו ומצטמק ויפה לו, דאי לא ליתני חדא והוא הדין לאידך. ומדבית הלל דברייתא אליבא דרבי מאיר שמעינן לבית שמאי דמתניתין, דכי היכי דלבית הלל אליבא דרבי מאיר משמע דלא אסרו אלא תבשיל המצטמק ויפה לו הכי נמי לבית שמאי דמתניתין דאית להו כבית הלל דברייתא לא אסרו אלא במצטמק ויפה לו, הא מצטמק ורע לו הרי הוא כחמין ושרי. כך נראה לי.\par \textbf{} ורש״י ז״ל שפירש חמין דלא צריכי בישולי דליכא למיגזר משום שמא יחתה אבל לא תבשיל דניחא ליה בבישוליה ואתי לחתויי, לאו למימרא דליכא במים משום חתוי, דהא תניא (לעיל שבת יח, ב) לא ימלא נחתום קיתון של מים ויתננו בתוך התנור ערב שבת עם חשיכה ופירשו בשלהי פרק קמא (לעיל שם) גזירה שמא יחתה בגחלים, אלא טעמיה כדפרישית, דבחמין דהוחמו כל צרכן קאמר דלא אתי לחתויי דמצטמק ורע להן, אבל תבשיל אפילו בשל כל צרכו אתי לחתויי דמצטמק ויפה לו.\par \textbf{} ובתוס׳ (ד״ה חמין) פירשו דלבית שמאי אפילו חמין שלא הוחמו כל צרכן שרי ואף על גב דמתבשלין ויפה להם, משום דכיון דאין יפה להם אלא עד שיוחמו כל צרכן אבל משם ואילך לא ועוד דקלין להתבשל, אף על גב דהשתא יפה להם לא אתי לחתויי דבלא חתוי מתבשלין היטב.\par \textbf{} ומכל מקום לבית הלל שרי בכל ענין, בין חמין שהוחמו כל צרכן ותבשיל שבשיל כל צרכו ובין לא בשיל כל צרכו, ואפילו לא בשיל כמאכל בן דרוסאי וכל שכן מצטמק ויפה לו, דבכירה גרופה וקטומה לא חיישינן לחתויי כלל.}
\clearpage
\newsection{דף לז}
\textblock{ גמרא:\textbf{ אי אמרת בשלמא להחזיר תנן היינו דשני בין מתוכה לעל גבה.} פירש רש״י ז״ל: אי אמרת בשלמא להחזיר תנן, רב חמא בר גוריא אשריותא דלשהות בשאינה גרופה קאי, והיינו דבתוכה אסיר משום דהוה ליה מטמין בדבר המוסיף הבל אבל על גבה שרי, אלא אי אמרת לשהות תנן דאפילו בלשהות אי גרוף וקטום אין אי לא לא, אפילו בתוכה נמי לישתרי דהא הוה ליה מטמין בדבר שאינו מוסיף הבל.\par \textbf{} ואינו מחוור כלל. חדא דאיהו אזיל לטעמיה דמפרש לה למתניתין בהטמנה, וליתא כדכתיבנא לעיל (שבת לו, ב ד״ה עד שיגרוף). ועוד דאפילו קטומה דבר המוסיף הוא. ועוד דמה לו לדחוק ולפרש דרב חמא קאי אשריותא דלשהות בשאינה גרופה, דהא בשריותא דחזרה נמי איכא לאפלוגי בין תוכה לעל גבה וכדאמרינן בסמוך. אלא הכי פירושה: בשלמא אי להחזיר תנן ברישא היינו דאיכא לאפלוגי בין תוכה לגבה, אלא אי אמרת דלשהות תנן בשהייה מה לי תוכה מה לי גבה, אלמא נותנין דמתניתין להחזיר הוא ועלה קאי רב חמא, ופריק אין הכי נמי דאשריותא דלהחזיר קאי ומיהו לאו ארישא דמתניתין קאי אלא אסיפא קאי.}
\textblock{\textbf{ואם תאמר אי מהימן ליה רב לידוק מאידך דרב, דהא אמרינן לקמן (עמוד ב) רב ושמואל דאמרי תרווייהו מצטמק ויפה לו אסור, וכל שכן כמאכל בן דרוסאי דמצטמק ויפה לו. ויש לומר דרב גופיה כרבי יהודה סבירא ליה דפליג       } אדחנניה ואפילו תמצא לומר דמתניתין להחזיר תנן, אלא הכא משום דרב קא מפרש לה למתניתין קא מייתי מינה, דהא לא שנו קאמר ואפירושא דמתניתין קא מהדר היכי תנן.\par \textbf{} ואם תאמר לידוק מדרבי יוחנן דפסק (לקמן שבת מו, א) כסתם מתניתין ואפילו הכי אמר רב שמואל בר רב יהודה אמר רבי יוחנן (לקמן עמוד ב) תבשיל שבשל כל צרכו מותר, דאלמא כמאכל בן דרוסאי אסור ודלא כחנניה. יש לומר משום דאמוראי נינהו אליבא דרבי יוחנן דאיכא מאן דאסר ואיכא מאן דשרי, דהא איכא רב ששת משמיה דרבי יוחנן דאמר (שם) דמתניתין להחזיר תנן.}
\textblock{\textbf{רבי יהודה אומר מה הן משהין בית שמאי אומרים חמין אבל לא תבשיל ובית הלל אומרים חמין ותבשיל.} מסתברא לי דלרבי יהודה משהין אפילו בשאינה גרופה חמין שהוחמו כל צרכן ותבשיל שנתבשל כל צרכו המצטמק ורע לו, משום דמחדא נפקא לן אידך, דכיון דבגרופה משהין אפילו מצטמק ויפה לו משום דליכא משום חתוי ולא גזרינן אטו שאינה גרופה, ממילא בשאינה גרופה משהין מצטמק ורע לו דליכא למיגזר שמא יחתה, דלא אשכחן תנא ולא אמורא דשרי לשהות בגרוף וקטום ואסר בשאינו גרוף במצטמק ורע לו. והא דתניא לקמן (שבת לח, א) שכח קדרה על גבי כירה ובשלה בשבת וכו׳ רבי יהודה אומר חמין שהוחמו כל צרכן מותרין מפני שמצטמק ורע להן וכו׳, דאלמא משמע לכאורה דדוקא שכח ובדיעבד אבל לכתחילה אפילו הוחמו כל צרכן אסור. לא היא, דהתם משום תבשיל נקטה ולומר דאפילו תבשיל שנתבשל כל צרכו כיון דמצטמק ויפה לו אסור ואפילו בדיעבד, אבל לעולם מצטמק ורע לו דליכא למיגזר משום חתוי אפילו בשאינה גרופה שרי. והיינו דאמרינן בשלהי פרק קמא (לעיל שבת יח, ב) האי קדרה חייתא אי נמי בשיל שפיר דמי, דההיא משמע דלכולי עלמא אמרינן לה. וכן כל מידי דקשי ליה זיקא דשריא התם, משום דליכא למיחש לחתויי.\par \textbf{} ועוד מסתייע לי הדין סברא, דרב ושמואל (לקמן עמוד ב) אמרי הכין בהדיא, דאסרי מצטמק ויפה לו ושרו מצטמק ורע לו ואפילו בשאינו גרוף ואינו קטום, ואתיא דרב ושמואל כרבי יהודה וכמתניתין למאן דאמר לשהות תנן. אבל אילו היינו אומרים דלרבי יהודה כל דבר אסור בשאינו גרוף ואפילו מצטמק ורע לו, א״כ רב ושמואל דאמרי כמאן. ולקמן (עמוד ב ד״ה הא דאמר) נכתוב זה יותר בארוכה בס״ד.}
\textblock{\textbf{אלא אי אמרת להחזיר תנן מתניתין מני לא רבי מאיר ולא רבי יהודה.} קשה לי מאי קשיא ליה מתניתין מני, ואמאי דחיק ואוקי לה כרבי יהודה וסבר ליה כוותיה בחדא ופליג עליה בחדא, לימא הא מני חנניה היא. ונראה לי משום דרבי מאיר ורבי יהודה איירו בהדיא בחמין ותבשיל ובשהייה ובחזרה כדאיירי בהו מתניתין, וחנניה לא איירי בהדיא בהו אלא בשהייה בלבד, משום הכי קא מהדר לאוקומה כחד מהני תנאי.}
\textblock{\textbf{איבעיא להו מהו לסמוך.} פירוש: האי בעיא לכולהו לישני היא, למאן דאמר לשהות תנן קא בעי מהו לשהות בסמיכה, ולמאן דאמר להחזיר תנן קא מיבעיא ליה מהו לסמוך בחזרה בשאינה גרופה. ואף על גב דקא מייתי משתי כירות המתאימות דמיירי בשהייה, מינה נשמע לחזרה למאן דאמר להחזיר תנן. וכן אתה מפרש בכולהו אינך דקא מייתי. ודלא כדברי הרב אלפסי ז״ל דאייתי ראיה מהא לפסוק כמאן דאמר לשהות תנן.}
\textblock{\textbf{ואע״ג דקא סליק הבלא מאידך.} הא דאיצטריך לאורוכי בלישניה ולמימר הכין, מסתברא משום דאיהו חייש קצת למאי דמהדר ליה דלמא שאני התם דכיון דמדליא שליט בה אוירא.}
\textblock{\textbf{שמעת מינה מצטמק ויפה לו מותר.} פירש רש״י ז״ל: דאי מצטמק ורע לו מאי אתא לאשמועינן. והקשו עליו בתוס׳ דדלמא איצטריך לאפוקי מדרבי מאיר דאסר לעיל בשאינה גרופה אפילו חמין דמצטמק ורע להם. ועוד הקשו למאי דקא סלקא דעתין דמצטמק ויפה לו מותר אפילו בלא גרופה וקטומה, אם כן אמאי קתני קטומה. ועוד כי משני שאני הכא דקטמה מאי קא פריך אי הכי מאי למימרא, דטובא קא משמע לן דאיצטריך לאפוקי ממאי דקא סלקא דעתין מעיקרא דהדרא למלתא קמייתא אלא אף על גב דהובערה לא הדרא למלתא קמייתא. ועוד קשיא לי, היכי ניחא ליה טפי במאי דהדר משני הובערה אצטריכא ליה, דהיינו פירוקא קמא דקאמר שאני הכא דקטמה, כלומר: וקא משמע לן דלא הדרא למלתא קמייתא. ועוד דכי אקשינן אי הכי מאי למימרא, אדרבה הוה ליה לאקשויי, אי הכי אמאי קתני חמין שהוחמו כל צרכן ותבשיל שבשל כל צרכו אפילו לא הוחמו כל צרכן ואפילו תבשיל שלא נתבשל כל צרכו נמי. ובתוס׳ דחקו בפירושה. וגם מן הגדולים ראיתי שדחקו בה הרבה.}
\textblock{\textbf{ולי נראה, דמעיקרא קא סלקא דעתך דהא דנקט קטמה הא קא משמע לן, דאף על גב דקטמה ונתלבתה היא כקטומה לגמרי כי הובערה הדרא למלתא קמייתא, והוה ליה למיתני קטמה והובערה אין משהין עליה אלא חמין שהוחמו כל      } צרכן ותבשיל שנתבשל כל צרכו, אלא דנקט ליה בלשון היתר משום דמלתא אגב אורחיה קא משמע לן דכל שבשל כל צרכו משהין עליה אף על פי שחזרה למילתא קמייתא, אלמא מצטמק ויפה לו מותר. ועוד דכל היכא דנקט חמין ותבשיל היינו תבשיל המצטמק ויפה לו, וכדכתיבנא לעיל (שבת לו, ב ד״ה בית) דחמין הוו כלל למצטמקים ורע להם ותבשיל השנוי עמהם הוי כלל למצטמקים ויפה להם. ועוד דאי במצטמק ורע לו לא איצטריך לאשמועינן כדפירש רש״י ז״ל, דאי לאפוקי מדרבי מאיר לא איצטריך, דרבי מאיר ורבי יהודה הלכה כרבי יהודה (עירובין מו, ב), ועוד דפליג אמתניתין. ודחי שאני הכא דקטמה, כלומר: לא כמו שאתה סבור דאיסורא דהובערה איצטריכא ליה, אלא אדרבה משום שריותא דקטמה שרינן בה במצטמק ויפה לו, והיינו דנקט לה בלשון היתר ולא נקט לה בלשון איסור דלימא אין משהין עליה אלא חמין שהוחמו כל צרכן. וקא סלקא דעתך השתא דהכי קאמר, דשאני הכא דקטמה וקמ״ל דאע״ג דהובערה הרי היא כקטומה גמורה וכקטומה ונתלבתה, והלכך כי אקשינן אי הכי מאי למימרא עדיפא מינה הוה מצי לאקשויי אי הכי מאי שנא תבשיל שבשל כל צרכו אפילו לא בשל נמי ואפילו לא בשל כמאכל בן דרוסאי דבקטומה לא חיישינן לחתויי, אלא אורחא דתלמודא בהכין דבמלתא דאתחיל בה שקיל וטרי, והלכך אקדים ואקשי במאי דאיירינן דהיינו שריותא דמצטמק ויפה לו, כלומר: הא דאתא לאשמועינן דמצטמק ויפה לו שרי מאי למימרא, ובמאי דפריק ליה אתריצו להו כולהו קושיי.}
\textblock{\textbf{הובערה אצטריכא ליה, מהו דתימא הדרא למלתא קמייתא קא משמע לן.} כלומר: לא כמו שאתה סבור, דשאני הכא דקטמה דקאמינא לא שתהא כקטומה גמורה אלא כקטומה ואינה קטומה, ואיצטריכא ליה לאשמועינן, דאי לא אשמעינן הוה אמינא כיון דהובערה הדרא למילתא קמייתא ולא תהא כקטומה כלל ויהא אסור בה מצטמק ויפה לו, קא משמע לן כיון דקטמה אף על גב דהובערה משהין עליה מצטמק ויפה לו והוא שבשל כל צרכו, משום דהובערה כקטומה ואינה קטומה גמורה. כך נראה לי.}
\textblock{\textbf{ורב ששת דיוקא דמתניתין קא משמע לן.} פירש רש״י ז״ל: דיוקא דמתניתין דפרק קמא (לעיל שבת יט, ב), דלא מיפרשא בהדיא דאי קרמו פניה שרי. ואינו מחוור, דכיון דקתני אלא אם כן קרמו בודאי מיפרשא בהדיא דאם קרמו פניה שרי. אלא דיוקא דמתניתין דפירקין דהכא (לעיל שבת לו, ב) קא משמע לן, דלא מיפרשא אי להחזיר תנן ואי לשהות תנן, ואדרבא סתמא כמאן דאמר לשהות תנן דהא למאן דאמר להחזיר תנן איצטריכינן לחסורה.}
\textblock{ הא ד\textbf{אמר רב שמואל בר רב יהודה אמר רבי יוחנן כירה שהסיקוה בגפת ובעצים משהין עליה חמין שהוחמו כל צרכן ותבשיל שבשל כל צרכו ואפילו מצטמק ויפה לו.} תמיהא לי דאמר כמאן לא כחנניה ולא כרבי מאיר ולא כרבי יהודה, דאילו לחנניה אפילו לא בשל כל צרכו אלא כמאכל בן דרוסאי, ואילו לרבי מאיר אפילו חמין שהוחמו כל צרכן בגרופה אין בשאינה גרופה לא, ולרבי יהודה מצטמק ויפה לו מיהא בשאינה גרופה לא. ובשלמא רב ושמואל דאסרי מצטמק ויפה לו וקא שרו מצטמק ורע לו לא קשיא לי דאינהו דאמרי כרבי יהודה וכדכתבינן לעיל (עמוד א ד״ה רבי) דלרבי יהודה מצטמק ורע לו אפילו בשאינה גרופה שרי, אלא רבי יוחנן דאמר כמאן דלא אשכחן תנא דאית ליה הכין, דאם איתא כי אמרינן בריש פרקין (לעיל שבת לו, ב) הא[י] לא יתן לא יחזיר אבל לשהות משהין אף על פי שאינו גרוף ואינו קטום ומני חנניה היא וכו׳, אכתי מנא ליה דחנניה היא ולא רבנן דלמא רבנן וכי קתני דאין מחזירין אבל לשהות משהין חמין ותבשיל ה״מ חמין שהוחמו כל צרכן ותבשיל שבשל כל צרכו אבל כמאכל בן דרוסאי לא ודלא כחנניה אלא כרבנן, אלא ודאי משמע דליכא תנא דאית ליה הכין. ועוד קשיא לי, דהא אמר רבי יוחנן (לקמן שבת מו, א) הלכה כסתם משנה וא״כ היאך אמר הכין, דבין למ״ד מחזירין בין למ״ד משהין מתניתין דלא כי הא דרבי יוחנן. ואע״ג דאמרינן בעלמא (יבמות טז, ב וש״נ) דאמוראי נינהו ואליבא דרבי יוחנן, לעולם קא בעי לה ומפרק לה.}
\textblock{\textbf{וראיתי בפירושי רבנו האי גאון ז״ל ענין שיראה ממנו דרב ששת משמיה דרבי יוחנן ורב שמואל בר יהודה משמיה דרבי יוחנן לא פליגי, אלא מר אמר חדא משמיה דרבי יוחנן ומר אמר חדא משמיה דרבי יוחנן, שכך כתב וזה לשונו: והלכה כמאן דאמר להחזיר תנן, ונסתייעה בפירוש דאמר רב ששת אמר רבי יוחנן כירה שהסיקוה בגפת ובעצים משהין עליה חמין שלא הוחמו כל צרכן ותבשיל שלא בשל כל צרכו ויתר דבריו, וסייעה עוד מדאמר רבא תרווייהו תננהי וכו׳, ואף רב שמואל בר יהודה אמר רבי יוחנן כמות זה בחמין שהוחמו כל צרכן ותבשיל שבשל כל צרכו ואפילו במצטמק ויפה לו וכו׳. ובסוף דבריו כתב עוד: נמצא המותר בשני פנים, (שחמו) [שהוחמו] ולא בשלו כל עיקר, ושבשלו כמאכל בן דרוסאי ואפילו בשלו כל צרכן. אלמא מדקאמר       } ואפילו בשלו כל צרכן, משמע שהוא ז״ל סבר דרב שמואל בר יהודה משמיה דרבי יוחנן רבותא אשמעינן דאפילו נתבשל כל צרכו ומצטמק ויפה לו שרי אע״ג דאיכא למגזר כיון שבחתוי מועט כדי לצמקו סגי ליה דלמא יהיב דעתיה ומחתה, ורב ששת אשמעינן בלא נתבשל כל צרכו ורב שמואל בר יהודה אוסיף אפילו נתבשל כל צרכו ומצטמק ויפה לו וכל שכן כשלא נתבשל כל צרכו.\par \textbf{} ומסתייעא הדין סברא, מדרב דאסר הכא מצטמק ויפה לו ובשלהי פרק קמא גמרא אין צולין בשר בצל וביצה אלא כדי שיצולו מבעוד יום (לעיל שבת כ, א) איבעיא לן ועד כמה, אמר רב כדי שיצולו מבעוד יום כמאכל בן דרוסאי, דאלמא טפי אסיר נתבשל כל צרכו וצריך להצטמק ויפה לו מלא נתבשל כל צרכו בין מצטמק ויפה לו בין רע לו. ואף על פי שיש בזה תירוץ אחר לדעת הרב אלפסי ז״ל וכמו שכתבתי שם (לעיל שבת יח, ב ד״ה האי) גבי קדרה חייתא.\par \textbf{} ואם תאמר אם כן למאן דאמר מצטמק ויפה לו אסור, כלומר: משום שקרוב לצימוקו יהיב דעתיה ומחתה, אם כן אפילו מצטמק ורע לו כל שהוא קרוב לבישולו אסור, דדילמא יהיב דעתיה ומחתה כדי לגמור אותו בשול מועט. יש לומר דכיון דמצטמק ורע לו לא מחתה דמסתפי דילמא מצטמק ומתפסד, וכמו שכתבתי למעלה גבי חמין.\par \textbf{} ותבשיל דמתניתין למאן דאמר להחזיר תנן אבל להשהות משהין כל תבשיל במשמע, בין הגיע למאכל בן דרוסאי בין נתבשל כל צרכו ואפילו מצטמק ויפה לו, ורבי יוחנן לטעמיה דאמר הלכה כסתם משנה, ורב סבר דמתניתין בתבשיל שנתבשל אפילו כל צרכו ומצטמק ורע לו ואי נמי במצטמק ויפה לו ובשלא נתבשל כל צרכו, אבל בנתבשל כל צרכו ומצטמק ויפה לו אסור.\par \textbf{} ומיהו לא מחוור, דחנניה כללא קא כאיל כל שנתבשל כמאכל בן דרוסאי משהין, ולאו דוקא כמאכל בן דרוסאי ממש הא פחות מכן ויתר על כן לא, אלא כל שהוא כמאכל בן דרוסאי ולמעלה מותר. ועוד דמאי דאיבעיא לן בשלהי פרק קמא כדי שיצולו עד כמה ואמר רב כמאכל בן דרוסאי, לאו דוקא ולומר דיתר מכן אסור, דאם איתא הוה ליה לתנא דמתניתין (לעיל שבת יט, ב) למיתני כדי שיצולו והוא שלא יתבשל כל צרכו.\par \textbf{} אלא אפשר דרב כרבי יהודה סבירא ליה דתבשיל המצטמק ויפה לו בשאינה גרופה לא, וההיא דשלהי פרק קמא דבצל וביצה משום דלא אתי לחתויי מפני שפניהם על פני האש ואי מחתה בהו חריך להו. וכן כתב הרמב״ם ז״ל (פ״ג מהל׳ שבת הט״ז) וזו היא דעתו של הרב אלפסי ז״ל. ובמצטמק ורע לו דמשמע דשרי רב בשנתבשל כל צרכו ואפילו לכתחילה, בהא נמי כרבי יהודה סבירא ליה, למאי דאתחזי לי אליבא דרבי יהודה כדכתיבנא לעיל (שבת לז, א ד״ה רבי יהודה). ואם תמצי לומר דרבי יהודה אפילו במצטמק ורע לו בשאינו גרוף לכתחילה אסר, וכסתמא דברייתא דרבי יהודה דשתי כירות המתאימות (לעיל שבת לז, א) דמשמע דעל גבי כירה שאינה גרופה כלל כלל לא, איכא למימר דרב תנא הוא ופליג וכן שמואל, ואי נמי שמואל כרב. כך נראה לי.}
\textblock{\textbf{אתון דמקרביתו לרב ושמואל עבידו כרב ושמואל.} כלומר: משום דברים המותרים ואחרים נהגו בהן איסור (פסחים נ, ב), אנן נעביד כרבי יוחנן דשרי מצטמק ויפה לו.}
\textblock{\textbf{כיון דמסוכן הוא אפילו בשבת נמי שרי.} תמיהא לי אי משום הא אפשר בקטום דבקטימה כל שהוא סגי, וההוא ודאי מוסיף הבל הוא וכל שכן שמשמר חומו של תבשיל.}
\textblock{\textbf{דהא רב נחמן בר יצחק מארי דעובדא הוא ומשהי ליה ואכיל.} הכא גרסינן רב נחמן בר יצחק, ולקמן גבי הלכתא מצטמק ויפה לו אסור גרסינן רב נחמן סתם דהוא רב נחמן בר יעקב דהוא רביה דרבא, כי היכי דלא תקשי דרב נחמן אדרב נחמן, וכן הגירסאות בספרים שלנו. ויש ספרים דגרסי בתרווייהו רב נחמן בר יצחק, ולפי אותן ספרים נצטרך לומר, דההיא דאמר רב נחמן מצטמק ויפה לו אסור לאו אליבא דנפשיה קאמר אלא אליבא דמאן דאמר מצטמק ויפה לו אסור, והכי קאמר הלכתא למאן דאמר מצטמק ויפה לו אסור כל דאית ביה קמחא מצטמק ורע לו הוא וכו׳, וכאילו הלכתא אכללא דמלתא קאי. ואינו מחוור כלל. וגירסת הספרים שלנו יותר נכונה.}
\textblock{ הא ד\textbf{בעו מיניה מרבי חייא שכח קדירה על גבי כירה ובשלה בשבת מהו.} יש לפרש דבשלא הגיע למאכל בן דרוסאי, וכן פירש רש״י ז״ל, וכן פירשה רב האי גאון ז״ל. ולכולי עלמא אתיא, בין לחנניה בין לרבנן, והיינו דקאמר ובשלה בשבת דמשמע שנתבשלה בשבת דמעיקרא לא נתבשלה, ואילו הגיע למאכל בן דרוסאי לא נתבשלה אלא מערב שבת דכל שהגיע למאכל בן דרוסאי אין בו משום      עכו״ם לכולי עלמא, וכן בשר בצל וביצה צולין אותם ערב שבת כל שהגיע למאכל בן דרוסאי קודם חשיכה (כדלעיל שבת כ, א). ואף על גב דאייתינן עלה הא דתניא שכח קדרה על גבי כירה וכו׳ במה דברים אמורים בחמין שלא הוחמו כל צרכן ותבשיל שלא בשל כל צרכו, דמשמע דנתבשל כמאכל בן דרוסאי אלא שלא נתבשל כל צרכו, כלומר: כמאכל כל אדם. ההיא רבי מאיר ורבי יהודה היא דפליגי עליה דחנניה, ולדידהו כל שלא נתבשל כל צרכו לא שנא הגיע למאכל בן דרוסאי לא שנא לא הגיע למאכל בן דרוסאי, דלא נתבשל כל צרכו כללא הוא להגיע ולשלא הגיע, ומדרבי מאיר ורבי יהודה נשמע לדידן, דכי היכי דלדידהו בין לא הגיע בין הגיע כיון שלא נתבשל כל צרכו אסור, לדידן נמי קודם שהגיע למאכל בן דרוסאי אסור. ועוד דבשלא הגיע למאכל בן דרוסאי אין הפרש בין רבי יהודה ורבי מאיר לחנניה, ומשלא הגיע למאכל בן דרוסאי קא מייתי ראיה. והלכך מבעיא זו ומכל הסוגיא שנאמרה עליה ליכא ראיה לדברי האומרים דלשהות תנן.}
\clearpage
\newsection{דף לח}
\textblock{\textbf{למחר נפק דרש להו המבשל בשבת בשוגג יאכל במזיד לא יאכל דברי רבי מאיר כו׳.} קשיא לי בשלמא לרב נחמן בר יצחק דאמר לאיסורא שפיר קא מייתי מדרבי מאיר דאפילו תימא דהלכה כרבי יהודה מכל שכן קא מייתי, כלומר: דאפילו רבי מאיר דמיקל טפי בההיא בהא לא שנא לאיסורא, אלא לרבה ורב יוסף דאמרו להתירא מאי קא מייתי מדרבי מאיר, והא קיימא לן כרבי יהודה דאסר ביומיה אפילו בשוגג, ואפילו לרב דמורי להו לתלמידיה כרבי מאיר מכל מקום בפירקא דריש כרבי יהודה (חולין טו, א), והכא היכי דריש כרבי מאיר. ויש לומר דכיון דמבשל ממש מורה לתלמידים כרבי מאיר, בשוכח קדירה ונתבשלה ממילא מדרש נמי דרשי כרבי מאיר.}
\textblock{\textbf{דרבי מאיר אדרבי מאיר לא קשיא הא לכתחילה הא דיעבד.} איכא למידק אכתי קשיא דרבי מאיר אדרבי מאיר, דהא בברייתא דלעיל (שבת לז, א) קא שרי לכתחילה אפילו חמין שלא הוחמו כל צרכן, דסתם חמין שלא הוחמו כל צרכן נינהו, ואילו הכא קא אסר במזיד אפילו בדיעבד כל שלא הוחמו כל צרכן. ויש לומר דאשנויא דרבי יהודה קא סמיך בהא, דכאן בקטומה כאן בשאינה קטומה.\par \textbf{} והא נמי דאקשינן דרבי יהודה אדרבי יהודה ושנינן כאן בגרופה כאן בשאינה גרופה, על כרחין נמי אית לן למימר דאשנויא דרבי מאיר סמיך, דאי לא אכתי תקשי ליה מברייתא דלעיל דלא שרי רבי יהודה מידי לשהות בשאינה גרופה, ואילו הכא קא שרי מיהא חמין שהוחמו כל צרכן, אלא על כרחין אשנויא דרבי מאיר סמיך בהא כאן לכתחילה כאן בדיעבד. וכן תירצו לה בתוס׳.\par \textbf{} אבל לדידי לא קשיא לי בדרבי יהודה, דאפילו לכתחילה נמי שרי כל מצטמק ורע לו אפילו בשאינה קטומה וכמו שכתבתי למעלה (לז, א ד״ה רבי).}
\textblock{ הא ד\textbf{איבעיא לן עבר ושהה מאי.} קשיא להו לרבוותא ז״ל, מאי קא מבעיא להו, אי בדלא בשיל כמאכל בן דרוסאי הא פשטינן לעיל לאיסורא כרב נחמן בר יצחק ואפילו בשוכח וכל שכן בעובר ומשהה, ואי במצטמק ויפה לו פלוגתא דרבי מאיר ורבי יהודה היא, ואי הלכתא כמאן קמבעיא ליה ליבעי הכי בהדיא גבי ההיא ברייתא, ועוד דרבי יהודה ורבי מאיר הלכה כרבי יהודה (עירובין מו, ב).\par \textbf{} והראב״ד ז״ל פירש דבמצטמק ויפה לו ודאי קא מיבעיא ליה ובמשהה בשוגג קאמר, וקא מיבעיא ליה רבי יהודה דברייתא אמזיד דרבי מאיר פליג או דילמא אפילו אשוגג כדמחמיר במבשל ממש, וכי אמרינן נמי קנסו אף על השוכח, דוקא שוכח אבל שוגג לא.\par \textbf{} ומיהו עבר ושהה לא משמע דוקא שוגג ולא מזיד, דאדרבה לכאורה משמע במזיד, וכדאמרינן בעלמא עבר ואפה מאי (ביצה יז, א) עברה ולשה מאי (פסחים מב, א), ואי הכא בשוגג דוקא הוה ליה למימר שגג ושהה מאי. ועוד מאי קא מייתי מדרבי יוסי דמצא ביצים מצומקות ואסר להם, דילמא התם במשהין במזיד. ויש לומר בזו דנשתהו בשוגג משמע, ועוד דסתמא דמילתא לא שהו במזיד אלא שוגגין היו כלומר: כסבורין שהיה מותר לעשות כן, ואף על פי כן אסרוה להם, ופריק לא אסר להם לאותה שבת אלא לשבת אחרת כלומר: שיזהרו שלא לעשות כן לשבת אחרת.\par \textbf{} והרב אלפסי ז״ל גריס: עבר ושכח, וכן נראה שהרמב״ם ז״ל (פ״ג מהלכות שבת ה״ט) פירשה בשכח ושהה. ולדבריהם כי אמרינן קנסו אף השוכח, דוקא בשלא נתבשל כל צרכו אבל מצטמק ויפה לו דלמא בשוכח לא אסרו. והקשה עליהם הראב״ד ז״ל מהא דמצא ביצים מצומקות ואסר להם, וההיא אי שכוחין היו מאי שנא שכחת אותה שבת (לשכחת) [משכחת] שבת אחרת, אלא דשוגגין היו והודיע אותן שהוא אסור ולא יעשו כן עוד.\par      \textbf{} ומורי הרב ז״ל פירש דהך בעיא אמילתיה דרב נחמן בר יצחק קיימא דאמר תרווייהו לאיסורא, אי נמי אגזירתא, וקא מיבעיא להו דילמא עד כאן לא קנסו לא בשוגג ולא במזיד אלא בבשיל ולא בשיל ומשום דנתבשל לגמרי בשבת, אבל בנתבשל כל צרכו אע״ג דמצטמק ויפה לו לא קנסו. והא דתניא רבי יהודה אומר חמין שהוחמו כל צרכן מותרין תבשיל שנתבשל כל צרכו אסור מפני שמצטמק ויפה לו, לא שמיע ליה למאן דבעי לה כי היכי דלא שמיע להו לרבה ולרב יוסף דאמרי להתירא. ואף רש״י ז״ל פירש עבר ושהה במזיד. ונכון הוא.\par \textbf{} אלא דקשיא לי אם כן אמאי לא פשטוה מדרבי יהודה וסלקא בבעיא ולא איפשיטא, ומי נימא דכולהו לא שמיע להו ברייתא, וזה דבר של תימה, ועדיין הדבר צריך תלמוד.\par \textbf{} ומכל מקום נראה מכאן דאפילו מצטמק ויפה לו אסור לשהות על גבי כירה שאינה גרופה וקטומה ודלא כחנניה. וכתב רבנו האי גאון ז״ל דאליבא דמאן דאסר מצטמק ויפה לו קא מיבעיא ליה, אבל אנן לא סבירא לן כותיה. ומורי הרב ז״ל הכין כתב, דמשום דרב ושמואל סבירא להו (לעיל שבת לז, ב) מצטמק ויפה לו אסור ומאן דמקרב לגבייהו מיבעיא ליה למיעבד הכין וכדקאמר ליה רב עוקבא ממישן לרב אשי (שם), משו״ה איצטריך למידע אם עבר ושהה מאי אי שרי או אסור.\par \textbf{} ולענין פסק הלכה: הרב רבנו אלפסי ז״ל פסק כמאן דאמר לשהות תנן, דאפילו לשהות אי גרוף וקטום אין ואי לא לא, וכל שכן להחזיר. והלכך כירה שאינה גרופה וקטומה אסור להשהות עליה תבשיל שלא בשל כל צרכו וחמין שלא הוחמו כל צרכן ואפילו מצטמק ורע לו, אבל תבשיל שבשל כל צרכו וחמין שהוחמו כל צרכן משהין ואף על פי שאינה גרופה ואינה קטומה ודוקא מצטמק ורע לו, אבל מצטמק ויפה לו אסור עד שיגרוף או עד שיתן את האפר.\par \textbf{} וראיותיו שהביא בהלכות אין הכרח באחת מהן, דכל אותן בעיות שנעשו במהו לסמוך שנסמך עליהן הרב ז״ל אין בהם כדי הכרח, דאפילו למאן דאמר להחזיר תנן איכא למיבעיא נמי לענין סמיכה בחזרה, וכמו שכתבנו לעיל במקומה (שבת לז, א ד״ה איבעיא), וההיא דלקמן (עמוד ב) נמי דאביי דאמר אילימא על גבה ובמאי אי בשאינה גרופה על גבה מי שרי, לא מכרעת אי בשהיה ואי בחזרה. וכי תימא מכל מקום איכא לסיועה להא דרבנו אלפסי ז״ל מדאצרכינן (לעיל שם) לדחוקי ולחסורי למתניתין אליבא דחנניה, ולרבי יהודה אתיא מתניתין כפשטה. לא היא, דאדרבה דוק מינה מדדחקינן לחסורי מתניתין כי היכי דתיקום כחנניה שמע מינה דכחנניה קיימא לן ומשום הכי מהדרינן לאוקומה לסתמא דמתניתין כותיה, דאי לא לא טרחינן ולא משכנינן נפשין אדידיה לדחוקי מתניתין ולחסורי ולפרוקי.\par \textbf{} ורבנו האי גאון ז״ל פסק כחנניה דכל שהוא כמאכל בן דרוסאי משהין על גבי כירה שאינה גרופה ואינה קטומה בין מצטמק ורע לו בין מצטמק ויפה לו, מדמחסרינן ומפרקינן לה אליבא דחנניה וסלקא בההוא פירוקא, ומדרב ששת אמר רבי יוחנן ומדסייעוה רבא ואמר תרווייהו תננהי, ומדאמרינן מכדי תרווייהו תננהי רבי יוחנן מאי אתא לאשמועינן, ופרקינן דיוקא דמתניתין אתא לאשמועינן, אלמא פשיטא להו דמתניתין דיקא הכין וכל שהגיע למאכל בן דרוסאי שרי. ואף רב שמואל בר יהודה אמר בכמות זה אף בחמין שהוחמו כל צרכן ומצטמק ויפה לו, ולא פליג אאידך דרב ששת אלא לחדותי אתא אף בנתבשל כל צרכו ומצטמק ויפה לו, וכדכתיבנא לעיל (שם ד״ה הא דאמר). ואף על גב דרב ושמואל (לעיל שם) אסרי במצטמק ויפה לו וכל שכן בשלא נתבשל כל צרכו, אפילו הכי רב ושמואל ורבי יוחנן הלכה כרבי יוחנן, ועוד דהא סייעה רבא דהוא בתרא ואמר תרווייהו תננהי, ועוד דמר עוקבא ממישן אמר אנן נעביד כרבי יוחנן, ואף על גב דאמר ליה לרב אשי אתון דמקרביתו לרב ושמואל עבידו כרב ושמואל, משום דהוה כדברים המותרים ואחרים נהגו בהן איסור. וכללא דרב נחמן דאמר (לעיל שם) דכל דאית בה מיחא וכו׳, לדבריהן דרב ושמואל קאמר ולית הלכתא אלא כרבי יוחנן וכ[ד]סייעיה מר עוקבא ממישן, ובעיא דעבר ושהה אליבא דמאן דאסר מצטמק ויפה לו היא ואנן לא סבירא לן כוותיה. אלו דברי הגאון רבנו האי ז״ל.\par \textbf{} ועוד יש לומר, דאפילו תמצי לומר דרב נחמן דאמר הלכתא מצטמק ויפה לו אסור הכי סבירא ליה לדידיה, מכל מקום אנן כרבא ורב עוקבא ממישן דאמר ליה לרב אשי הלכה סבירא לן דבתראי נינהו. ועוד דהא רב נחמן [בר יצחק] מארי דעובדא עביד עובדא לנפשיה (לעיל שם) ומעשה רב. ואף על גב דרבי יוסי אסר להו בצפורי, לא עדיף מדרבי יהודה ולא קיימא לן כוותיה.\par \textbf{} וכן פסקו גם בתוס׳ (לעיל שם ד״ה אמר). ואף רש״י ז״ל (לעיל שם ד״ה ורב) מן המתירין. ועליהם סמכו במקומות אלו לנהוג היתר בדבר, וכדאין הם וראיותיהם לסמוך עליהם.}
\textblock{\textbf{ומי שהטיל פשרה בין רב ששת אליבא דרבי יוחנן ורב שמואל בר יהודה, לומר דרב ששת איירי במצטמק ורע לו ולפיכך אפילו לא נתבשל כל צרכו שרי ורב שמואל בר יהודה איירי במצטמק ויפה לו ולפיכך אסר אלא אם כן נתבשל כל צרכו אינו אלא מן המתמיהין, חדא דסתם תבשיל מצטמק ויפה לו הוא, ועוד דכל דנקט ליה תבשיל בהדי חמין משמע לכאורה דתבשיל מצטמק ויפה לו, ומעובדא דרבי מוכח דביצה יפה לו, ואפילו הכי תנא במתניתין דפרק קמא (לעיל שבת יט, ב) אין צולין בשר בצל וביצה אלא כדי שיצולו מבעוד } יום ופירש רב (לעיל שבת כ, א) כמה יצולו משהגיע למאכל בן דרוסאי, והתימה הגדול שבעל הפירוש הזה נסמך על ההיא דרב דבשר בצל וביצה לפסוק כחנניה, ואף על פי כן אסר מצטמק ויפה לו בשלא נתבשל כל צרכו. ועוד דאי מתניתין להחזיר תנן אבל לשהות משהין אף על פי שאינה גרופה ואינה קטומה ומתניתין חנניה היא, אם כן אפילו מצטמק ויפה לו ושלא נתבשל כל צרכו שרי, דהא מתניתין תבשיל סתמא קתני ומשמע כל תבשיל. ועוד דהא למאן דאמר מתניתין לשהות תנן אתינן לאוקומה (לעיל שבת לז, א) מתניתין כרבי יהודה, ורבי יהודה תבשיל המצטמק ויפה לו קאמר מדאותביה מדידיה לדידיה מברייתא דתבשיל אסור מפני שמצטמק ויפה לו, אלמא תבשיל דקתני באידך ברייתא קמאה נמי מצטמק ויפה לו הוא, ואפילו הכי אמרינן דמתניתין מני רבי יהודה היא. אלא שאין לאותו הפירוש עיקר.\par \textbf{} והנכון שבכלל הדברים מה שכתב רבנו האי גאון ז״ל, והוא המסכים לפרק כל הקושיות וליישב את הכל על אופניו, ונשאר המנהג על בוריו. והלכך כל שנתבשל כמאכל בן דרוסאי בין נתבשל כל צרכו בין לא נתבשל כל צרכו ואפילו מצטמק ויפה לו, משהין בין בכירה שאינה גרופה בין בתנור, דבשהיה אין הפרש בין תנור לכירה, והנהו דפרק קמא (לעיל שבת יח, ב) דשרינן בנתבשל בתנור נינהו. וקדירה חייתא נמי שרי דלא אתי לחתויי, אבל בשיל ולא בשיל דהיינו שלא הגיע למאכל בן דרוסאי אסור, ואי שדא ביה גרמא חיא שפיר דמי. וכדברי רבנו האי גאון ז״ל (הובא לעיל (שבת יח, ב) ד״ה מאן) קדירה חייתא ואי נמי גרמא חיא דלא הוחמו קודם חשיכה אסור, והיינו ברייתא דעססיות ותורמוסין וקיתון של מים דאסרינן עם חשיכה בפרק קמא (יח, ב). אבל לדידן לא קשיא דעססיות שלנו דקשין הן אצל בישול ואפילו בעי להו למחר לא מסח דעתיה מלחתויי לפי שהן צריכות בישול רב ואפילו כל הלילה אינו מספיק להם, וקיתון של מים נמי כיון שהם קלים להתבשל הוו להו חיים גמורים כתבשיל דבשיל ולא בשיל, וכדכתבינן לעיל במקומה. וכן מאי דקשה ליה זיקא כגדיא דמנתח וכולהו הנך דפרק קמא (לעיל שם) שרי, כיון דקשי ליה זיקא לא מגלי להו.}
\textblock{\textbf{לדברי האומרים מחזירין, מחזירין אפילו בשבת.} פירש רש״י ז״ל: שבת, יום שבת, שלא תאמר לא התירו אלא סילק בלילה והחזיר, אבל יום מכיון שסילק נראה כמי שאין דעתו להחזיר אלא לאכילה דהשתא בעי לה וכי החזיר דמי למניח לכתחילה ואסור, קא משמע לן. וקשיא לי דאם כן מה ראיה יש מדרבי יוחנן דלמא התם בשסילק בלילה. ויש לי לומר לדברי רש״י זכרונו לברכה דכיון שסלקו כדי למזוג לו ומזגו כל שכן הוא דמחזי כמי שגומר בדעתו שלא להחזירו, ולא שנא יום ולא שנא לילה. ויש מי שפירש שבת היינו משתחשך, שלא תאמר לא התירו אלא בין השמשות. ובתוס׳ פירשו דמתניתין מבעוד יום סמוך לחשיכה, בכדי שלא יוכל להתחמם קודם חשיכה אילו היתה צוננת קרי מחזיר, ובכדי שיוכל להתחמם קודם חשיכה קרי משהה. ותדע לך מדקתני עד שיגרוף ולא קתני אלא אם כן גרף, דאלמא משמע דחזרה בשעה הראויה לגריפה היא, ואתו בגמרא למימר דאפילו להחזיר בשבת ממש שרי.}
\textblock{\textbf{מכלל דעודן בידו אף על פי שאין דעתו להחזיר מותר.} פסק רבנו האי גאון ז״ל כלישנא קמא ולחומרא, וזה לשונו: ודאמרת כי עודן בידו מותר להחזיר דוקא דעתו להחזיר הוא דשרי אבל אין דעתו להחזיר עקרה דעתיה מנהון, ואם הניחן על גבי קרקע אפילו דעתו להחזיר אסור, כיון דאיכא תרי לישני בדאורייתא עבדינן לחומרא. והכין נמי (בבי) [בעי] דאמרינן בהו תיק״ו אסורין דאורייתא נינהו. עד כאן. וכן כתב ר״ח ז״ל. ולאו דאורייתא ממש קאמרי דהני דרבנן נינהו, דכשהגיע למאכל בן דרוסאי היא מתניתין, דאי לא אפילו דעתו להחזיר ואפילו עודן בידו אסור להחזירו דמבשל הוא, אלא ודאי בנתבשל הוא וליכא אלא איסורא דרבנן, אבל דעתן של גאונים ז״ל דכיון דאסרוהו משום       כמבשל וקרוב הדבר לבא לידי איסורא דאורייתא הויא לה כדאורייתא, ולפיכך הלכו בה להחמיר כשל תורה. וכן הסכימו כל הפוסקים לפסוק בה להחמיר, הרב אלפסי ז״ל והרמב״ם ז״ל (פ״ג מהל׳ שבת ה״י) והר״ז הלוי ז״ל.}
\textblock{ הא דדייק אביי:\textbf{ במאי אילימא בשאינה גרופה וקטומה אלא כירה כי אינה גרופה וקטומה מי שרי וכו׳.} איכא למידק מאי קאמר, דלמא בגרופה וקטומה והרי הוא כתנור דאף על גב דגרוף וקטום על גביו אסור, דאי בכירה שפיר דמי, וכאוקמתא דרב אדא בר מתנה. יש לומר דקסבר אביי דכופח גרוף וקטום אפילו על גביו שפיר דמי, דלא עדיף גרוף וקטום מקש וגבבא דכל שבקש וגבבא שרי אף גרוף וקטום שרי, והלכך כופח דבקש וגבבא הרי הוא ככירים ואפילו על גבה שרי הוא הדין לגרוף וקטום, משום הכי סבירא ליה דמתניתין דכופח לית לה אוקמתא אלא כשאינו גרוף וקטום.\par \textbf{} ולהאי פירושא לישנא דקאמר אילימא בשאינה גרופה וקטומה לאו דוקא, אלא הכי הוה ליה למימר, ובמאי עסקינן אילימא על גביו ובשאינו גרוף וקטום הוא כלומר: ועל כרחין בשאינו גרוף וקטום הוא, ולא דק בלישנא. ולפום הדין פירושא נמי שמעינן דתנור אסור לעולם בין בגרוף בין בשאינו גרוף, דכיון דאסרינן ליה אפילו לסמוך ואפילו הוסק בקש ובגבבא דהא תניא כוותיה דאביי, הוא הדין לגפת ועצים ואפילו גרוף וקטום. וכן נראה עיקר.\par \textbf{} נמצא עכשיו לפי פירוש זה, דיש בדברי אביי להקל ולהחמיר ובדרב אדא נמי להקל ולהחמיר, אלא במה שאביי מחמיר מיקל רב אדא ובמה שמיקל אביי מחמיר רב אדא, דאביי מחמיר בסמיכת תנור ומיקל בכופח גרוף וקטום אפילו על גביו, וחלופא בדרב אדא שהוא מחמיר בכופח גרוף וקטום על גביו לשוויה כתנור ומיקל בסמיכת תנור.\par \textbf{} וקיימא לן כאביי באיסור סמיכת תנור דהא תניא כוותיה, אבל לא קיימא לן כוותיה להקל בכופח גרוף וקטום על גביו, אלא הרי הוא כתנור וכפירושא דרב אדא, דבהא לא תניא כוותיה דאביי. וכן פסק רב אלפסי ז״ל.\par \textbf{} ולענין סמיכת כופח, לכולי עלמא בגרוף וקטום שרי, דהא לאביי אפילו על גביו שרי בגרוף וקטום. ואף על גב דקתני במתניתין בגפת ובעצים הרי הוא כתנור וקא מפרש אביי הרי הוא כתנור לענין סמיכה, לאו לגמרי לאשווי לתנור ואפילו בסמיכה בגרוף וקטום אלא לסמיכת שאינו גרוף וקטום, דכיון דקתני בקש ובגבבא הרי הוא ככירה מינה שמעינן דהרי הוא כתנור דקאמר היינו כדמפרש בגפת ובעצים בשאינו גרוף וקטום הרי הוא כתנור לענין סמיכה. וכל שכן דלרב יוסף ורב אדא שהוא מותר, דלדידהו אפילו סמיכת תנור שרי. ואפשר נמי דאביי הכין סבירא ליה, והא דלא אוקמה למתניתין בגרופה וקטומה כרב אדא, משום דבעי לאוקומה למתניתין אפילו בשאינה גרופה, דהא דקתני בגפת ובעצים הרי הוא כתנור דאי ככירה שרי משמע ליה בכל ענין, דאלמא כירה שרי בכל ענין והיינו לסמוך, ומכלל דתנור אסור לעולם ואפילו לסמוך ואפילו גרוף וקטום.\par \textbf{} ולמאי דכתבינן לעיל (עמוד א) דסוגיא דגמרא מכרעת דמתניתין (דלעיל שבת לו, ב) להחזיר תנן אבל לשהות משהין, תנור נמי דוקא להחזיר הוא דאסור בין לתוכו בין על גביו בין סמוך לו, והוא הדין לכופח שהוסק בגפת ובעצים לתוכו ועל גביו אבל לסמוך סומכין ואפילו בחזרה, אבל לשהות משהין בכל מקום ואפילו בשאינו גרוף וקטום, וכדכתבינן לעיל.}
\textblock{ מתני׳:\textbf{ אין נותנין ביצה בצד המיחם.} פירוש: דתולדת האור [כאור]. ואסיקנא בגמ׳ שאם גלגל חייב חטאת.}
\textblock{\textbf{ולא יפקיענה בסודרין ורבי יוסי מתיר.} פירוש: להפקיענה בסודרין, אבל בצד המיחם דתולדת האור הוא לכולי עלמא אסור, וכדתנינן (לקמן שבת קמה, ב) חוץ מקוליס האספנין שהדחתו זה הוא גמר מלאכתו, ואמרינן נמי בהדיא בגמרא (שם) בתולדת האור כולי עלמא לא פליגי דאסור.}
\clearpage
\newsection{דף לט}
\textblock{ גמרא:\textbf{ כל שבא בחמין מלפני השבת שורין אותו בחמין בשבת.} פירוש: שבא בחמין מלפני השבת ונתבשל כמאכל בן דרוסאי דשוב אין בו משום בשולי עכו״ם      והלכך שורין אותו אפילו בחמין בכלי ראשון, אבל לא נתבשל כמאכל בן דרוסאי והוי דבר שמתבשל בכלי ראשון כתבלין וכיוצא בזה אסור, וכדתנן בפירקין (לקמן שבת מב, א) האלפס והקדרה שהעבירן מרותחין לא יתן לתוכן תבלין.}
\textblock{\textbf{וכל שלא בא בחמין מלפני השבת מדיחין אותו בחמין.} פירוש: שמשליכין עליו מים חמים ומדיחין אותו בהם. ולמאן דסבירא ליה דעירוי דכלי ראשון לאו ככלי ראשון, הכא נמי שרי להדיחו בעירוי דכלי ראשון. ואי עירוי דכלי ראשון ככלי ראשון, הכא לא שמדיחין אותו בקלוח מים שהוא מערה מכלי ראשון קאמר, אלא לאחר שעירה אותן בתוך כלי שני מדיחו בתוכן, וכדתנן (לקמן שבת מב, ב) אבל נותן הוא לתוך הקערה ולתוך התמחוי. ולקמן גבי ההיא מתניתין דהאלפס והקדירה (שם בעמוד א בד״ה האלפס) נאריך בה בסייעתא דשמיא.}
\textblock{\textbf{רבה אמר גזירה שמא יטמין ברמץ ורב יוסף אמר גזירה שמא יזיז עפר ממקומו.} והא דרבה ורב יוסף לרבי יוסי בלחוד הוא דאיצטריך אבל לרבנן לא איצטריך, דאינהו אפילו להפקיעה בסודרין אסרי משום תולדת האור ואף על גב דליכא לא משום גזירת רמץ ולא משום גזירת שמא יזיז עפר ממקומו. ואף על פי שהביא הרב אלפסי ז״ל הא דרבה ורב יוסף, לברור פירושא דמתניתין בלבד הוא שכתב כן.}
\textblock{ הא דאמרינן:\textbf{ מאי בינייהו עפר תיחוח.} פירש רש״י ז״ל: דלמאן דאמר שמא יטמין ברמץ איכא, ולמאן דאמר שמא יזיז עפר ממקומו ליכא, דהא בעפר תיחוח ליכא משום גומא.\par \textbf{} ואיכא למידק מדאמרינן בפרק קמא דביצה (ז, ב. ח, א) גבי דקר נעוץ, והא קא עביד כתישה, בעפר תיחוח, והא קא עביד גומא, הא מני רבי שמעון היא דאמר כל מלאכה שאינה צריכה לגופה פטור עליה, אלמא בעפר תיחוח איכא משום גומא. ויש לומר דהכא בעפר תיחוח הרבה בעומק ומן הצדדין דבעפר תיחוח ממש ליכא משום גומא, אבל התם בעפר תיחוח מועט שכשיסתלק העפר התיחוח נעשה גומא בדפנות של קרקע קשה.\par \textbf{} ואינו מחוור בעיני כל הצורך. דאם כן למה דחק והעמידה שם אליבא דרבי שמעון בלבד, ואפשר נמי דלית הלכתא כרבי שמעון, לוקמה בעפר תיחוח הרבה דליכא משום גומא לכולי עלמא. ועוד היכי קא מקשינן להדיא והא קא עביד גומא, דמנא ליה דבעפר תיחוח מועט קאמר דלמא בדאיכא טובא דליכא גומא, דעפר תיחוח סתם טובא משמע וכדהכא.\par \textbf{} ומצאתי כאן לרבנו האי גאון ז״ל שפירש בהיפך, דלמאן דאמר גזירה שמא יטמין ברמץ ליכא דרמץ בעפר לא מיחלף, אבל למאן דאמר גזירה שמא יזיז עפר ממקומו איכא, והשתא אתיא ההיא דמסכת ביצה שפיר.\par \textbf{} אלא דקשיא לי דהא אותבינן ממעשה שעשו אנשי טבריא ואמרינן בשלמא למאן דאמר גזירה שמא יטמין ברמץ היינו דדמיא להטמנה, ואם איתא אפילו למאן דאמר שמא יטמין ברמץ היכי ניחא והא לא דמיא לרמץ, דמי דמיא הטמנה במים טפי לרמץ מהטמנה בעפר תיחוח. וצ״ע.\par \textbf{} ולרבה דאמר גזירה שמא יטמין ברמץ בדין הוא דאפילו מבעוד יום נמי ליתסר, וכדאסרינן נמי בכל דבר המוסיף הבל אפילו מבעוד יום וכטעמא דמעשה שעשו אנשי טבריא שאסרו להם חכמים משום הטמנה לדעת רב חסדא ואף על גב דהתם מבעוד יום היה, אלא היינו טעמא דלא מיתסרא מבעוד יום לרבה משום דכל שהוחם בערב שבת מן החמה מצטנן הוא לגמרי בליל שבת והוה ליה בשבת דבר שאינו מוסיף הבל.\par \textbf{} ואם תאמר ורב יוסף מאי טעמא לא אמר כרבה, אטו לית ליה גזירת הטמנה בדבר המוסיף הבל. כתב מורי הרב ז״ל דקסבר שלא גזרו אלא בדבר המוסיף מצד עצמו כגון גפת ומלח וכיוצא בהן, אבל הכא שאין חומן מצד עצמן אלא מחום השמש וכשתסלקם ממקומן יתקררו הלכך אפילו במקומן לא גזרו.\par \textbf{} ואם תאמר והא חול דבר המוסיף הבל הוא ולכולי עלמא ליתסור משום הטמנה ואפילו מבעוד יום, וכדתנן (לקמן שבת מז, ב) אין טומנין לא בסיד ולא בחול. יש לומר התם בתבשיל רותח אבל צונן לא, וכדאמרינן עלה בפרק לא יחפור (ב״ב יט, א) דחמימי חמים דקרירי קריר.\par \textbf{} ויש לי לומר לפי פירושו של רבנו האי גאון ז״ל, דרב יוסף לא פליג עליה דרבה דאיהו נמי אית ליה גזירת הטמנה, אלא שהוסיף בטעמו לומר דאפילו בדבר שאין בו משום הטמנה כעפר תיחוח אפילו הכי אסור משום שמא יזיז עפר ממקומו. ואף על גב דבמתניתין אבק דרכים קתני דבדידיה לכולי עלמא איכא משום גזירת הטמנה, ניחא ליה למינקט טעמא דשייך בעפר ואבק ולאשמעינן דאבק לאו דוקא.\par \textbf{} והא דאקשינן לרב יוסף ממעשה שעשו אנשי טבריה, ואמרינן בשלמא למאן דאמר גזירה שמא יטמין ברמץ היינו דדמיא להטמנה אלא למאן דאמר גזירה שמא יזיז עפר ממקומו מאי איכא למימר, דאלמא רב יוסף לית ליה כלל בכי הא גזירת הטמנה, דאי לא לימא ליה רב יוסף תרתי אית ליה והתם משום הטמנה. לא היא, אלא משום דרב יוסף פריש טעמא דהא מתניתין דלא יטמיננה בחול משום שמא יזיז עפר ממקומו, ועלה אייתי הא דאנשי טבריא על כרחין בתר טעמא דההיא גרירא, והלכך אקשינן ליה מיניה, ולעולם רב יוסף מודה הוא לרבה דמודה רב יוסף בכל דבר דאיכא למיגזר משום הטמנה, אלא דהא מתניתין עיקרה משום שמא יזיז עפר הוא דנפקא מינה אפילו לעפר תיחוח.\par \textbf{} ועם פירושו של רבנו האי גאון ז״ל גם כן איתרצא לי קושיא אחריתי דקשיא לי, רבה אמאי נקט רמץ דמוסיף הבל הוא הא משחשיכה אפילו בדבר שאינו מוסיף הבל נמי אסור, אלא דרבה לאו משום דבר המוסיף הבל נקט לה אלא משום דאבק דמי לרמץ. ואפשר נמי דלפירושו של רש״י ז״ל כן, ואבק ועפר תרווייהו לרמץ דמי ובדידיה מיחלפו.\par \textbf{} ולכולי עלמא אפילו תימצי לומר דבעפר תיחוח ממש שייך איסור גומא בפירות לא שייך, דאם כן אסור ליטול קמח ביום טוב ללוש אליבא דרבי יהודה דומיא דדקר נעוץ בעפר תיחוח. וגרסינן הכא בירושלמי (ה״ג): תמן אמרין חמה מותרת תולדות חמה אסורין ורבנן דהכא אמרין בין חמה בין תולדת חמה מותרין, ואמרינן מתני׳ פליגא על רבנין דהכא לא יטמיננה בחול ובאבק דרכים, ופרקינן שניא היא שהוא עושה חריץ, אילו אסר בקמח יאות, אלמא בקמח אין בו משום חריץ ומשום גומא.}
\textblock{\textbf{רשב״ג אומר מגלגלין ביצה על גבי גג רותח.} והא דרשב״ג סבירא ליה כרבי יוסי, דאילו לרבנן דאסרי אף תולדות חמה אף זה אסור. ובהדיא גרסינן בירושלמי (שם): מתני׳ פליגא על רבנין דהתם, דתני רשב״ג אומר מגלגלין ביצה על גבי גג רותח וכו׳, מה עבדין ליה רבנן דתמן, פתרין לה חלוקין על רשב״ג.}
\textblock{\textbf{אמר רב חסדא ממעשה שעשו אנשי טבריא ואסרו להם רבנן בטלה הטמנה בדבר המוסיף הבל ואפילו מבעוד יום.} איכא למידק למה לן ממעשה שעשו אנשי טבריא, והא בהדיא תנן (לקמן שבת מז, ב) אין טומנין לא בגפת ולא במלח וכו׳, וההיא ודאי אפילו מבעוד יום קאמר, מדתנן (לקמן שבת מט, א) טומנין בגיזי צמר ואין מטלטלין אותן, ומדקתני אין מטלטלין אותן שמע מינה דטומנין דקאמר היינו מבעוד יום, דאי לאחר שחשיכה היאך טומנין והלא אין מטלטלין אותן, אלא ודאי מבעוד יום, ומה שהתיר בגיזי צמר אסר בגפת דהיינו אפילו מבעוד יום.\par \textbf{} ויש מפרשים דאי ממתניתין הוה אמינא הני מילי בין השמשות וכרבי דאמר (לעיל שבת ח, ב) כל שהוא משום שבות לא גזרו בבין השמשות, והלכך בגיזי צמר אף על פי שאין מטלטלין אותן בשבת בבין השמשות התירו, אבל בגפת שהוא מוסיף הבל לא. ואם תאמר כיון דבין השמשות לא גזרו בשל דבריהם, אם כן אפילו בדבר המוסיף הבל יטמין כדרך שהוא מטמין מבעוד יום. איכא למימר דבאיסור הטמנה החמירו. כך תירצו בתוס׳.\par \textbf{} ויש מפרשים דהכי קאמר ממעשה שעשו אנשי טבריה ואסרו להם חכמים בטלה הטמנה, כלומר: שעל אותו מעשה גזרו עליה, דומה למה שאמרו מעדותו של רבי יוחנן בן גדגדא נשנית משנה זו.\par \textbf{} ומהא שמעינן דמטמין בדבר המוסיף התבשיל אסור אפילו בדיעבד. וכתב הרמב״ן ז״ל ודוקא בצונן שנתחמם או נצטמק ויפה לו, אבל בעומד בחמימותו כשעה ראשונה אין לחוש.}
\textblock{\textbf{ואם תאמר היכי מפרש רב חסדא מעשה דאנשי טבריא משום הטמנה, והא ההוא טעמא דהטמנה ליתיה אלא לרבה בלבד ואליבא דרבי יוסי אבל לרבנן לא נאסר אלא משום דגזרינן בתולדת חמה, ואפילו לרבי יוסי נמי אליבא דרב יוסף      } ליתיה אלא משום תולדת האור. ויש לומר דרב חסדא כרבה סבירא ליה דמשום גזירת הטמנה היא, ורבנן נמי משום תרי טעמי אסרי להו משום דהוחמו בשבת ומשום הטמנה בדבר המוסיף הבל. דקסבר רב חסדא דאסיפא קאי כיון דלא תני לה בתר ההיא דרישא ובין רבנן בין רבי יוסי מודו בה, וכיון דאמאי דקא מודי ביה רבי יוסי לרבנן קא מייתי לה, הא משמע דבין רבי יוסי בין רבנן חד טעמא אית להו בההוא מעשה. דאי לא, לא ניחא למימר דהכי קאמר לא יטמיננה באבק דרכים לרבנן משום תולדת חמה ולרבי יוסי משום הטמנה, ומעשה שעשו אנשי טבריא ואסרו להם חכמים, לרבנן משום תולדת חמה ולר׳ יוסי משום הטמנה, דזה דבר רחוק שיביא מעשה על הודאת בעלי המחלוקת ושיהא הטעם מוחלק לגמרי. ולפי פירוש זה ההיא דאבק דרכים, נמי משום תרתי אסור לרבנן.\par \textbf{} ועוד תדע לך, דאי לאו הכי מאי קא מייתי עלה מעשה שעשו אנשי טבריא, דמבעוד יום היה, וכדאמר רב חסדא ממעשה שעשו אנשי טבריא וכו׳ בטלה הטמנה בדבר המוסיף הבל מבעוד יום, ואילו חלוקת רבי יוסי ורבנן במטמין בשבת היא אבל מערב שבת לכולי עלמא שרי. אלא כדאמרן, דההיא דחול ואבק דרכים לכולי עלמא משום הטמנה אית בה לדעת רבה ורב חסדא דמפרשי לה דאסיפא קאי, ולומר דלכולי עלמא אסור ואפילו מבעוד יום ואפילו בתולדת חמה בדבר שאינו מצטנן בלילה כחמי טבריא, והוא הדין לחול ולאבק דרכים בשבת דאיכא משום הטמנה, ומה שלא אסרו בהן מבעוד יום מפני שהן מצטננין בלילה לגמרי כמו שכתבנו למעלה (בד״ה הא).\par \textbf{} ויש לדחות, דמדקאמרו להם חכמים בשבת כחמין שהוחמו בשבת ואסורין קא מייתו ראיה, ולומר דמינה שמעינן דחמין שהוחמו בתולדת חמה אסורין בין בשתיה בין ברחיצה. ומכל מקום אינו מחוור, דאם כן נצטרך לומר דרבנן דרבי יוסי לית להו לגמרי כרבנן דאמרו להם, דרבנן דטבריא על כרחין משום הטמנה אסרו, דאי משום איסור תולדת חמה בלבד והלא מערב שבת הביאוהו ומשהה על גבי תולדת חמה לא שמענו מי שמחמיר עליו לשהות, וכל שכן למאן דאמר להחזיר תנן, ועוד דאילו כן נלמוד ממנה דלשהות תנן. ואם תאמר דלא אסרו להם אלא סילון הפתוח שהמים נגרין תמיד ואלו ששותין בשבת עכשיו הוחמו לאחר חשיכה, אבל סילון הסתום שבאו מימיו בתוכו מערב שבת מותר, לא היא, דמידי הוא טעמא אלא לרב חסדא, הא איהו קאמר דממה שאסרו להם חכמים בטלה הטמנה בדבר המוסיף מבעוד יום, שמע מינה שהוא סובר דאפילו אותן שנשתהו שם מבעוד יום אסור כנ״ל. ומשום הכי כתב הרב אלפסי ז״ל לחלוקתן דרבה ורב יוסף, אף על פי שפסק בתולדת חמה כרבנן.\par \textbf{} ויש מפרשים דקסבר רב חסדא דאפילו לרבנן דאסרי תולדות חמה הני מילי לבשל בהם לכתחילה בשבת, אבל מותר להניח קיתון של מים צונן מערב שבת באמבטי של חמי טבריא. דמאי אמרת גזירה שמא יעשה כן מערב שבת בתולדת האור, אפילו עביד הכי ליכא איסורא דאורייתא, ועוד אפילו על גבי האש ממש התירו (לעיל שבת יח,ב) לשהות קדירה חיה מיהא לכולי עלמא, הלכך כשאסרו להם חכמים לא משום תולדת חמה אסרו אלא משום הטמנה בדבר המוסיף הבל. ואינו מחוור בעיני. דאם כן למה הביאוהו גבי לא יטמננה בחול ובאבק דרכים דלרבנן לא הוי טעמא אלא משום תולדת חמה, אלא אם תאמר כלשון הראשון שכתבתי דרבנן נמי בההיא דחול משום תרי טעמי אסרי לה.}
\textblock{\textbf{אמר עולא הלכה כאנשי טבריא.} פירוש: דקסבר עולא דלא אסרו הטמנה מבעוד יום בתולדת חמה, שלא אסרו אלא במטמין בשבת עצמה, ומה שאסרו להם חכמים לא משום הטמנה אסרו אלא מפני שהוחמו בשבת בחמי חמה, ולאו אסיפא קאי אלא ארישא קאי, ורבי יוסי לא מודה להו בהא ורבנן הוא דמייתו לה. והכין משמע בירושלמי (דפרקין ה״ג) דגרסינן התם: תמן אמרין חמה מותרת תולדות חמה אסורין ורבנן דהכא אמרין בין חמה בין תולדת חמה מותרת, מתניתא פליגא על רבנן דהכא לא יטמיננה בחול ובאבק דרכים וכו׳ כדאיתא התם עד על דעתיהון דרבנין מתמן מעשה שעשו אנשי טבריא סלקא על דעתיהון דרבנין דהכא לא סלקת מתניתא כמעשה שעשו אנשי טבריא, כלומר: לא עלתה הלכה כמעשה שעשו אנשי טבריא. ועוד יש לומר עולא כרב יוסף סבירא ליה דרבי יוסי אפילו בשבת עצמה לא אסר אלא משום שמא יזיז עפר ממקומו, הא לאו הכי הטמנה גופה בתולדות חמה מותרת, וסבירא ליה לדידיה דהלכתא כרבי יוסי, אבל רבנן אסרו משום תולדות חמה ועוד משום הטמנה ואפילו במטמין בהן       יום, אבל לעולם האי מעשה דאנשי טבריא רבנן הוא דמייתו לה ולא הודה בו רבי יוסי.}
\textblock{\textbf{אמר ליה רב נחמן כבר תברינהו רבנן לסילונייהו.} ומדשתיק ליה עולא לרב נחמן ולא אהדר ליה שמע מינה קבלה מיניה, והלכך הלכתא כרבנן, והכין פסק ר״ח והרב אלפסי ושאר הגאונים ז״ל. ועוד דהוו להו רב חסדא ורב נחמן תרי לגבי עולא, וקיימא לן כרבים. וכן הסכימו לדעתם מורי הרב ז״ל והרמב״ן ז״ל.\par \textbf{} ועוד כתב הרב מורי ז״ל, דרב נחמן דאמר ליה לעולא כבר תברינהו רבנן לסילונייהו לא שמעינן ליה דפליג אדרבי יוסי, דהא איכא למימר בהבאת סילון של צונן בתוך אמה של חמין דוקא קא אסרינן משום דדמיא להטמנה וכרבה דאמר גזירה שמא יטמין ברמץ, ואפילו הכי הלכה כרבנן דאסרי לבשל בתולדת חמה, דאמר רבינא לקמן (שבת מ, ב) המבשל בחמי טבריא חייב, ואסיקנא מאי חייב דקאמר מכת מרדות מדרבנן, וכיון דאין בהם משום בשול מן התורה שמעינן מינה דתולדות חמה נינהו ואפילו הכי אסור לבשל בהן מדרבנן. וקימא לן כרבה דאמר גזירה שמא יטמין ברמץ, הלכך אסור להטמין קיתון של צונן באמבטי של חמי טבריא מע״ש אע״ג דתולדות חמה נינהו, אבל להניח על גביו בלא הטמנה מערב שבת מותר ובשבת אסור. ע״כ.}
\textblock{\textbf{אילימא רחיצת כל גופו אלא חמין שהוחמו בשבת וכו׳ והתניא חמין שהוחמו מערב שבת וכו׳.} איכא למידק למה ליה לאקשויי מברייתא ומדיוקא דהא הוחמו מערב שבת מותרין, לידוק ממתניתין גופה דקתני אסורין ברחיצה ובשתיה, דאלמא אסורין בכל הנאה ליומן ואפילו לרחוץ בהן מקצת גופו ואין צריך לומר כל גופו. ויש לי לומר דניחא ליה לאתויי מרחיצה ממש. ואם תאמר לידוק כולה מילתא מסיפא דקתני ביום טוב כחמין שהוחמו ביום טוב, ובמאי, אילימא רחיצת כל גופו, הא הוחמו מערב יום טוב שרי, והתניא לקמן (שבת מ, א) מרחץ שפקקו נקביו מערב יום טוב למחר נכנס ומזיע ויוצא ומשתטף בבית החיצון דאלמא משמע דבבית הפנימי לא, אלא פניו ידיו ורגליו, לימא תנן סתמא כבית שמאי. ותירצו בתוס׳ דניחא ליה למידק מרישא כל מאי דמצי למידק מינה.}
\textblock{ הא ד\textbf{תנן בית שמאי אומרים לא יחם אדם חמין לרגליו אלא אם כן ראויין לשתיה.} שהוחמו לשתות קאמר, אבל אם הוחמו לרחיצה בלבד אסורין. ותדע לך, דהא בעינן לאוקומה מתניתין כבית שמאי ואילו במתניתין תנן ביום טוב כחמין שהוחמו ביום טוב ואסורין ברחיצה ומותרין בשתיה, אלמא ראוין לשתיה אלא כיון שהוחמו לדעת רחיצה אסורין. ועוד דבית שמאי לית להו מתוך שהותרה הבערה לצורך הותרה נמי שלא לצורך (ביצה יב, א-ב) וכל שכן דלית להו הואיל. אלא הכי קאמר אלא אם כן ראוין לשתיה, וראויין דהכא פירושו מוכנין.}
\textblock{\textbf{ובית הלל מתירין.} משמע דאפילו בית הלל לא שרו אלא לרגליו כלומר: למקצת גופו, אבל לרחוץ בהן כל גופו אסור, והכין נמי מוכחא כולה שמעתין דהכא.}
\textblock{\textbf{ואיכא למידק, דהא ודאי מדשרו בית הלל להחם לרגליו מסבר סברי דהנאת הגוף הרי הוא כאוכל נפש ובכלל אשר יאכל לכל נפש (שמות יב, טז) הוא, ומהאי טעמא נמי הוא דשרו (ביצה כא, ב) מדורה להתחמם כנגדה מפני שהיא שוה לכל נפש וכל הנאה השוה לכל נפש הרי היא בכלל אשר יאכל לכל נפש כדאיתא בריש פרק קמא (דביצה) [דכתובות ז, א], וזיעה נמי דבר תורה מותרת קודם גזירה כדאיתא בהדיא במסכת ביצה בפרק המביא (לב, א) וכדאיתא נמי לקמן בסמוך (שבת מ, א), אלא שגזרו עליה מפני הבלנין שהיו מחמין ביומן והיו אומרים מערב שבת הוחמו וגזרו עליהן אפילו להזיע כדאיתא בסמוך, וכיון שכן ליכא איסורא דאורייתא כלל ברחיצת כל הגוף וכל שכן בזיעה, אם כן מפני      } מה אסרוה ביום טוב דהא ליכא גזירה כלל דמאי איסורא אתי מינה. ואם תאמר גזירה משום שבת, הא ליתא דליכא למיגזר כלל יום טוב אטו שבת בדברים של אוכל נפש.\par \textbf{} ויש מפרשים, דכיון דאיכא ביום טוב נמי איסור במדיח וסך קרקע (לקמן שבת מ, ב) ואי נמי משום סחיטת אלונטית (לקמן שבת קמז, א) עשו יום טוב כשבת. ואינו מחוור בעיני כלל, דאי משום הא נגזור אפילו בחמי טבריא, שהגזירה ההיא שוה היא בחמי טבריא כמו בחמי האור.\par \textbf{} אבל בתוס׳ אמרו דרחיצת כל הגוף אסורה דבר תורה דאינה צורך כל נפש, אלא דומיא דמוגמר שנאסר מן הטעם הזה בפרק קמא דכתובות (שם), אבל רחיצת פניו ידיו ורגליו הוי צורך כל נפש, וכן הזיעה שוה לכל נפש דמשום רפואה היא, וכיון דרחיצת כל הגוף אסורה דבר תורה משום לתא דידה אסרו את הזיעה. והביאו ראיה מן הירושלמי (דפרקין ה״ג) דבעי התם מותר לשתות ואסור לרחוץ, ומתרץ משום דכתיב אשר יאכל לכל נפש הוא לבדו יעשה לכם, והתם בירושלמי התירו רחיצת כל גופו אבר אבר.\par \textbf{} ולפי דברי רבותינו הצרפתים ז״ל נצטרך לומר, דרחיצת כל גופו בפעם אחת מפני שאינה שוה לכל נפש אסורה דבר תורה, אבל רחיצת כל גופו אבר אבר מותר דבר תורה, משום דכיון דרחיצת אבר אחד צורך כל נפש ומותר לא חלקנו באברים שנאמר רגליו בלבד מותר אבל זרועו או שוקו אסור, וכיון שכן נמצא בשעה שהוא רוחץ אבר זה בהיתר הוא רוחץ וכשחוזר ורוחץ את השני גם הוא בהיתר הוא רוחץ, אלא שאסור מדרבנן. והיינו דלקמן בסמוך (שבת מ, א) משמע בברייתא דרחיצת כל גופו אבר אבר אינה אסורה אלא מדבריהם, מדקתני ואין צריך לומר חמין שהוחמו ביום טוב, כלומר: דאסורין ברחיצת כל הגוף ואפילו אבר אבר, דאלמא משמע דאסור מדרבנן קאמר ברישא.}
\textblock{\textbf{לא ישתטף אדם כל גופו בין בחמין בין בצונן דברי רבי מאיר וכו׳.} שמועה זו כולה נתחבטו בה הראשונים ורבו פירושיה. והנכון שנאמר, כי חמין אלו שנחלקו עליהם בברייתא הם שהוחמו מערב שבת. וכן נראה לי לדקדק מן התוספתא (פ״ד ה״ג) ששנו שם כברייתא זו ושם מצאתי סיפא דברייתא זו: אמר רבי יהודה מעשה בביתוס בן זונין שהיו ממלאין לו דלי של צונן מערב שבת ונותנין עליו בשבת כדי שיקרה אלמא חמין שאסר רבי יהודה היינו אפילו הוחמו מערב שבת. והכי פירושה: דרבי מאיר דאסר בין בחמין בין בצונן משום דאית ליה גזירת מרחצאות אפילו בצונן, לפי שדרכן של רוחצין להשתטף בצונן אחר רחיצה, וכדאמרינן לקמן (שבת מא, א) רחץ בחמין ולא נשתטף בצונן דומה לברזל שהכניסוהו לאור ולא הכניסוהו לצונן, ורבי יהודה נמי אית ליה גזירת מרחצאות ודוקא בחמין אבל בצונן לא, משום דהרבה פעמים אדם נותן עליו צונן כדי להקר, ורבי שמעון לית ליה גזירת מרחצאות כלל בחמין שהוחמו מערב שבת, ומתניתין רבי שמעון היא, ואסורין ברחיצה דקתני לא רחיצה ממש אלא שיטוף, וחמין שהוחמו בשבת אסורין הא מערב שבת מותרין ואפילו להשתטף בהן כל גופו, וברייתא דקתני חמין שהוחמו מערב שבת למחר רוחץ בהן פניו ידיו ורגליו אבל לא כל גופו ההיא ברחיצה ממש ודברי הכל, ואם תמצי לומר בשטיפה רבי מאיר ורבי יהודה היא ודלא כרבי שמעון.}
\textblock{ הכי גרסינן:\textbf{ אמר רב חסדא מחלוקת בכלי אבל בקרקע דברי הכל מותר.} וכן היא ברוב הספרים, וכן היא גירסתן של גאונים ז״ל. והכי פירושא: מחלוקת בנוטל בכלי ונותן על גביו, לפי שדרכן של רוחצין בכך לאחר שהזיעו נותנין עליהם מים חמין בכלי, וכדמוכח בברייתא דמרחץ שפקקו נקביו (כ)דמייתינן בסמוך (מ, א), אבל בקרקע אין דרך הרוחצין להשתטף במים הנתונין בקרקע.}
\textblock{\textbf{והא מעשה שעשו אנשי טבריא בקרקע הוה.} כלומר: שהיו אותן מי הסילון נגרין בעומק שבקרקע, ואפילו הכי אסרו חכמים להשתטף בהם בשבת.\par \textbf{} ואם תאמר מאי קושיא ההיא בחמין שהוחמו בשבת אבל הוחמו מערב שבת מותרין. לא היא, דהכא קא מקשה הכי, והא מעשה שעשו אנשי טבריא ואסרו להם חכמים מפני שהוחמו בשבת ואוקימנא כרבי שמעון, והוחמו מערב שבת לרבי מאיר ורבי יהודה הויא כהוחמו בשבת לרבי שמעון, והלכך כדרך שאסרו לרבי שמעון בשבת ואפילו בקרקע הכי נמי לרבי מאיר ורבי יהודה אפילו כשהוחמו מערב שבת. ואם תאמר ליהדר ולימא מתניתין בקרקע וככולי עלמא. יש לומר דאמאי דאוקימנא לה כרבי שמעון סמיך ולא בעינן דליפלוג רב חסדא אמאי דאמר רב איקא. ועוד דמקצת ספרים יש דגרסי רב איקא אמר רב, ולא פליג רב חסדא עליה דרביה, ולפיכך ניחא לן טפי לאפוכה לדרב חסדא ואמר מחלוקת בקרקע אבל בכלי דברי הכל אסור.}
\clearpage
\newsection{דף מ}
\textblock{\textbf{דאי מכללא הני מילי במתניתין אבל בברייתא לא.} איכא למידק מי קאמר במשנתינו כל מקום קאמר, וכדאמרינן בכתובות בפרק שני דייני גזירות (כתובות קט, א) גבי כל מקום שאמר רבן גמליאל רואה אני את דברי אדמון הלכה כמותו, ואבעיא להו התם במתניתין או אפילו בברייתא, ואהדרינן מי קאמר במשנתינו כל מקום קאמר, וכן במנחות פרק הקומץ רבה (מנחות לא, א-ב) אמר רב אשי אמר לי מר זוטרא קשי בה רב חנינא מסורא פשיטא מי קאמר במשנתינו כל מקום קאמר. ויש לומר דהכא הכי קאמר: דלמא אמר רבי יוחנן בפירוש במשנתינו, דלא שמיע ליה לרב יוסף מימרא בלישנא דוקא, ואהדר ליה אנא בפירוש שמיע לי.\par \textbf{} וכיון דלא איפסיקא להו מימרא היכי אתמר אי במשנתינו או בכל מקום, לא עבדינן עובדא כדברי המכריע בברייתא. וכן פסקו הגאונים ז״ל. וכן כתב הרב אלפסי ז״ל בפרק קמא דקידושין (כד, ב).\par \textbf{} ואם תאמר היכי פסק רבי יוחנן כרבי יהודה, והא אוקימנא מתניתין כרבי שמעון, ואיהו קאמר (לקמן שבת מו, א) הלכה כסתם משנה. ואף על גב דאמרינן בעלמא (יבמות טז, ב וש״נ) אמוראי נינהו אליבא דרבי יוחנן, מכל מקום לא שתיק גמרא בשום מקום דלא ליקשי לה אלא מקשה לה ומפרק לה אמוראי נינהו. ויש לומר דרבי יוחנן סתמא אחרינא אשכח, דתנן לקמן בפרק חבית (שבת קמז, א) הרוחץ במי מערה ובמי טבריא מסתפג אפילו בעשר אלונטיות ולא יביאם בידו, ורבי יוחנן (שם עמוד ב) מתני לה להא כבן חכינאי, ודייקינן עלה בגמרא (בע״א) קתני מי מערה דומיא דמי טבריא, מה מי טבריא בחמין אף מי מערה בחמין, הרוחץ דיעבד אין לכתחילה לא, מכלל דלהשתטף אפילו לכתחילה, מני רבי שמעון היא, וכיון דההיא אתיא כרבי שמעון ותני לה כבן חכינאי הוה ליה מתניתין דהכא סתם ואחר כך מחלוקת, ואין הלכה כסתם. וזה דחוק בעיני, דאף על פי ששנה אותה בן חכינאי, מכל מקום אין מחלוקת שנויה בפירוש באותה משנה, והיאך נעשה זה כסתם ואחר כך מחלוקת.\par \textbf{} אבל בתוס׳ תירצו דכיון דאיכא מחלוקת בברייתא ואיכא הכרעה במקום מחלוקת דברייתא, ובהכרעה לא אמר רבי יוחנן שתהא הלכה כסתם משנה. וזה נכון. ומכל מקום מפני שהוא אומר בעלמא הלכה כסתם משנה הוצרך לפסוק כאן בפירוש הלכה כרבי יהודה, ולא סמך על מה שכבר אמר הלכה כדברי המכריע ואף על פי שרבי יהודה מכריע כאן.}
\textblock{\textbf{ואין צריך לומר חמין שהוחמו ביום טוב.} מדתני ואין צריך לומר חמין שהוחמו ביום טוב ולא תני והוא הדין לחמין שהוחמו מערב יום טוב, שמעינן מינה דדוקא הוחמו ביום טוב אבל הוחמו מערב יום טוב רוחץ בהם כל גופו אבר אבר. והרב אלפסי ז״ל התיר בשם גאון לרחוץ בהן כל גופו, וכמו שכתב במסכת ביצה (כא, ב) גבי מתניתין דלא יחם אדם חמין לרגליו. ואף הרמב״ן ז״ל נראה שהסכים לדבריו. ונראה לפי דבריהם, כי מה שפסק רבי יוחנן הלכה       יהודה דברייתא בשבת בלבד היא שנויה ולא איום טוב.}
\textblock{\textbf{למוצאי שבת רוחץ בו מיד.} הוא הדין דהוה מצי למיתני למחר נכנס ומזיע כדרך ששנו ביום טוב, דגזירת שבת ויום טוב בבת אחת הוה וגזירה חדא הויא וברייתא זו קודם שגזרו על הזיעה נשנית מדקתני מרחץ שפקקו נקביו מערב יום טוב למחר נכנס ומזיע. ותדע לך דחדא גזירה הואי, מדתניא בהא ברייתא משרבו עוברי עבירה התחילו לאסור, ואמרינן מאי עוברי עבירה ומייתו הא דרבי יהושע בן לוי בתחילה היו רוחצין בחמין שהוחמו מערב שבת ואע״ג דהא ברייתא דידן איירי ביו״ט כדקא אמרינן מעשה במרחץ בבני ברק שפקקו נקביו מערב יו״ט, אלמא שבת ויו״ט נגזרו כאחד, אלא משו״ה לא קתני בברייתא גבי שבת למחר נכנס ומזיע כדקתני ביום טוב, משום דאין דרך להזיע בלא שיטוף חמין לבסוף ובשבת אסור להכי נקט לה ביום טוב ולא בשבת, ואי נמי להכי תני הכין ביום טוב לאשמועינן דאפילו ביום טוב דוקא זיעה אבל לא רחיצה לפי שכבר גזרו על הרחיצה.}
\textblock{ והא דקתני\textbf{ ומשתטף בבית החיצון.} כתב מורי הרב ז״ל דבצונן קאמר, אבל לא בחמין ואע״פ שהוחמו מערב יום טוב. והביא ראיה מן התוספתא דתניא התם בהדיא (פ״ד ה״ב): מרחץ שסתמו נקביו מערב יום טוב נכנס ביום טוב ומזיע ויוצא ורוחץ בצונן, אמר רבי יהודה מעשה במרחץ של בני ברק וכו׳ עד ויוצאין ורוחצין בצונן אלא שהיו חמין שלה מחופין בנסרים. ומשום הכי דקדקו שהיו חמין שלה מחופין בנסרים, גזירה שלא יאמרו בחמין רחצו. ומה שהצריכו לפקוק נקבים מערב יום טוב אף על פי שלא היה משתטף כלל בחמין ואף על פי שלא גזרו עדיין על הזיעה, לפי שלא יבא לרחוץ בהן דכיון שמתחממין והולכין ביום טוב או בשבת יטעו לרחוץ בהם ולפיכך הצריכו לפקק את נקביו, הא לא פקק אינו נכנס כלל אפילו להזיע בלבד.}
\textblock{\textbf{התחילו הבלנין להחם חמין בשבת.} תמיהא לי והלא לא נחשדו ישראל על השבתות, וכדתניא (גיטין נד, א) נחשדו ישראל על השביעית ולא נחשדו על השבתות. ומצאתי בירושלמי (דפרקין ה״ג) שלא היו מחממין בשבת אלא שלא היו פוקקין נקבים מבערב וממלאין עצים מבערב, ובלשון הזה שנו אותה שם: בראשונה היו סותמין את החמין מערב שבת ונכנסין ורוחצין בשבת, ונחשדו להיות ממלאין אותו עצים והיא דולקת והולכת בשבת, אסרו להם רחיצה והתירו להם זיעה. אלא שעדיין קשה לי וכי מה איכפת להן אם היתה דולקת והולכת בשבת, והתניא (לעיל שבת יח, א) פותקין מים לגנה ערב שבת עם חשיכה ומתמלאת והולכת כל השבת כולה. ויש לומר דכיון שנקביו פתוחין והוא דולק והולך בעודו רוחץ בו ודאי אתי לחתויי, ולפיכך אסרו בדולק בשבת והתירו בפוקק, וחזרו ואסרו.}
\textblock{ הא ד\textbf{תנו רבנן מתחמם אדם כנגד המדורה וכו׳ ובלבד שלא ישתטף בצונן ויתחמם כנגד המדורה.} פירש ר״ת ז״ל: מפני שנראה כמפשיר מים שעליו. וסייעה מדגרסינן בירושלמי (דפרקין ה״ד) יורד אדם וטובל בצונן ועולה ומתחמם כנגד המדורה דברי רבי מאיר וחכמים אוסרין, יאות אמר רבי מאיר מה טעמון דרבנן, כההיא דאמר רבי זעירא בשם רב חייא מותר להפשיר במקום שהיד שולטת ואסור להפשיר במקום שאין היד שולטת.\par \textbf{} ויש מקשים לפירוש זה דהפשרה עצמה מותרת היא, שהרי שנינו (לקמן שבת מא, א) אבל נותן לתוכו מים כדי להפשירן, ותניא נמי בסמוך לא בשביל שיחמו אלא כדי שתפיג צינתן, והפגת צינתן היינו הפשר כדמוכחא כולה שמעתין, ושמע מינה דהפשר מותר אפילו כנגד המדורה ממש. ומיהו אינו קשה כל כך דמתניתין וברייתא לא שרו אלא בהפשר דלא אפשר למיתי לידי בישול כגון מים מרובין ואי נמי כנגד המדורה במקום שאין היד סולדת בו, אבל בירושלמי פירשוה לההיא בהפשר שבמקום שהיד סולדת בו. ותדע לך דאם כן לותביה בירושלמי לההיא סברא ממתניתין דאטו מי לא ידעי לה למתניתין, ואי נמי פליגא ההיא אמתניתין לימא מתניתין פליגא על רבנן.\par \textbf{} ויש מפרשים שאינו משום איסור עצמו של הפשר, אלא שנראה כרוחץ במים פושרין. ונכון הוא.}
\textblock{\textbf{מביא אדם קיתון של מים ומניחו כנגד המדורה לא בשביל שיחמו אלא בשביל שתפיג צינתן.} פירש רש״י ז״ל: לא שיניחם שם עד שיחמו אלא שתפיג צינתן במקצת. ונראה מתוך פירושו שאפילו כנגד המדורה ממש במקום הראוי לבשל מותר להפשיר, שלא אסרו אלא שלא יניחם כדי שיחמו כלומר: שיתבשלו אבל כדי הפשר מותר בכל מקום. ובודאי שכן נראה מלשון כדי שיחמו וכדי שתפיג צינתן. ולדבריו הא דאמר רב יהודה אמר שמואל אחד שמן ואחד מים יד סולדת בו אסור אין יד סולדת בו מותר, לא במקום שהיד סולדת ומקום שאין יד סולדת קאמר, אלא כשהניחן עד שתהא יד סולדת קאמר. ומכל מקום על גבי האש ממש אסור מפני שנראה כמבשל, דהא אפילו תבשיל שנתבשל כל צרכו אסור להחזיר בכירה שאינה גרופה וקטומה וכל שכן לתת לכתחילה ואפילו להפשיר, ולא התירו כאן אלא כנגד המדורה בלבד.\par \textbf{} וקשיא לי, דהא תנן (לקמן שבת מא, א) המיחם שפינהו לא יתן לתוכו צונן כדי שיחמו אבל נותן לתוכו או לתוך הכוס כדי להפשירן, ופירשה רב אדא בר מתנא המיחם שפינה ממנו מים לא יתן לתוכו מים מועטין כדי שיחמו אלא מים מרובין כדי להפשירן, ואם איתא דבכל מקום מותר להפשיר ובלבד שיזהר שלא יחם, אם כן אפילו מועטין יהא מותר ובלבד שלא יניחם שם כדי שיחמו. ואין מן הסברא לומר דמיחם שפינה ממנו מים אם נתן לתוכו מים מועטין מיד ישובו המים חמין ואי אפשר להפשיר בו מים מועטין, דמי עדיף מאש. ושמא יפרש רש״י ז״ל, דכיון שדרכן לתת לתוך המיחם או לתוך הכוס לעמוד שם ומסתפקין מהן מעט מעט, חיישינן שאם יתן לתוכן מים מועטין שמא יתעכבו שם עד שיחמו, אבל כנגד המדורה לא חיישינן דנגד המדורה פעמים שאדם מחמם שם פעמים שאדם מפשיר שם. ועוד יש לי לומר דאורחא דמלתא נקט, שדרך ליתן לתוכו פעמים להחם פעמים להפשיר, מועטין להחם מרובין להפשיר.\par \textbf{} ועדיין קשה לי לפי גירסת הספרים והיא גירסתן של גאונים ז״ל, הא דאמרינן בסמוך במעשה דר׳ יצחק בר אבדימי שנכנס אחר רבי לבית המרחץ ואמר לו טול בכלי שני ותן, ואמר שמע מינה תלת, שמע מינה שמן יש בו משום בישול, ושמע מינה כלי שני אינו מבשל, ושמע מינה הפשרו לא זהו בשולו, אלמא אף על פי שהפשרו לא זהו בשולו להפשירו בכלי ראשון לא, מפני שלא התירו להפשיר במקום הראוי לבשל גזירה שמא יבשל. אבל רש״י ז״ל ורבותינו בעלי התוס׳ ז״ל גורסים: ושמע מינה הפשרו זהו בשולו, ולפי גירסתם אינו קשה עליו מה שהקשיתי. אבל מכל מקום אין גירסתם עולה יפה, דאם איתא מה לי ראשון מה לי שני כל מקום שהוא מתבשל אסור, דלא אמרו כלי ראשון וכלי שני אלא מפני שזה מבשל וזה אינו מבשל. ותנינא (לקמן שבת קמה, ב) כל שבא בחמין מלפני השבת שורין אותו בחמין בשבת, וכל שלא בא בחמין מלפני השבת מדיחין אותו בחמין בשבת, חוץ מן המליח הישן וקוליס האספנין שהדחתן זהו גמר מלאכתן, אלמא כל שהוא מתבשל בין בכלי ראשון בין בכלי שני ואפילו בהדחה בעלמא חייב.\par \textbf{} ושמא נאמר, דכיון דההיא דרבי בחמי טבריא הוה כדאיתא בסמוך ובחמי טבריא אפילו מבשל לגמרי אינו מדאורייתא, לא אסרו אלא בכלי ראשון דבעלמא מבשל, אבל כלי שני כיון דבעלמא אינו מבשל ואף כאן ליכא אלא הפשר דבעלמא אינו בישול אע״ג דלגבי שמן הפשרו זהו בישולו לא גזרו עליו בכלי שני.\par \textbf{} ומכל מקום אינו מחוור, דאם כן האי מסקנא דאמרינן ושמע מינה תלת, פליגא אדרב יהודה אמר שמואל דאמר בשמן אין יד סולדת בו מותר, ואפילו כנגד המדורה, אלמא הפשרו לא זהו בשולו. ועוד דהיאך אפשר דשמעינן מינה הפשרו זהו בשולו, אדרבה נימא משום דסבירא ליה הפשרו לא זהו בשולו קאמר ליה דיכול ליתן בכלי שני דאי הפשרו זהו בשולו אף בכלי שני היה אוסר, וזה היה יותר פשוט וראוי לומר כדי שתהא תקנתן כעין דאורייתא, דכי היכי דבתולדות האור כל שהפשרו בשולו אסור בכלי שני בשלו בחמי טבריה נמי שאסרו מדבריהם כן.}
\textblock{\textbf{והנכון שנאמר, שלא התירו להפשיר אלא במקום שאין היד סולדת בו, אבל במקום שהיד סולדת בו אסור אפילו } להפשיר גזירה שמא יניח שם עד שיתבשל, ואפילו בכלי ראשון של חמי טבריא אסרו כן משום דכל מאי דתקון כעין דאורייתא תקון.\par \textbf{} והכי איתא בירושלמי דגרסינן התם בריש פרקין דכירה: אמר רבי זעירא בשם רב יהודה מותר להפשיר במקום שהיד שולטת בו ואסור להפשיר במקום שאין היד שולטת בו, ואפילו במקום שאין היד שולטת עד היכן, ר׳ יודה בן פזי ר׳ סימון בשם רבי יוסי בר חנינא עד כדי שהוא נותן ידו לתוכה והיא נכוית.\par \textbf{} ובהא מיתרצא לן הא דאמרינן לקמן בריש פרק במה טומנין (שבת מח, א) רבה ור׳ זירא איקלעו לבי ריש גלותא, חזיוה לההוא עבדא דאנח כוזא דמיא אפומא דקומקמא, נזהיה רבא, אמר ליה ר׳ זירא מאי שנא ממיחם על גבי מיחם, אמר ליה התם אוקומי קא מוקים הכא אולודי קא מוליד, ודייקי בה רבוותא ז״ל היכי דמי, אי אנחיה לבשל מאי טעמא דר׳ זירא, מי לא ידע מתניתין דקתני אין נותנין ביצה בצד המיחם בשביל שתתגלגל דתולדת האור לכולי עלמא אסור וחייב חטאת, ואי להפשיר מאי טעמא דרבה והא קתני המיחם שפינהו וכו׳ אבל נותן הוא לתוכו או לתוך הכוס כדי להפשירן, ותניא הכא ממלא אדם קיתון של מים ונותן כנגד המדורה. ועוד תניא בהדיא בתוספתא פ״ד (ה״ה) ממלא אדם כוס של יין ונותן על פי מיחם לא שיחם אלא שתפיג צינתו, ותירץ רבנו יצחק בר׳ שמואל הידוע בעל התוס׳ ז״ל, דלעולם להפשיר אלא שהיה ראוי להתבשל שם, ור׳ זירא סבר דכיון שאין דעתו רק להפשיר מותר, והיינו מיחם על גבי מיחם, ורבה סבר דהתם הוא דשרי מפני שכבר הוחמו ואינו יכול אלא להעמיד חומן, אבל הכא שהן ראויין להתבשל אפילו להפשיר אסור דכל שהוא מוליד חום כלומר: שמתחיל בו בשול במקום הראוי לבשל אסור גזירה שמא יבשל, ואף על פי שיש בזה תירוצים אחרים כמו שנכתוב שם בס״ד.\par \textbf{} ומכל מקום מסתברא דאינו אסור כנגד המדורה ולא בכלי ראשון אלא צונן שלא נתבשל וגזירה הפשר אטו בשול, אבל צונן שנתבשל אפילו בכלי ראשון או כנגד המדורה ואפילו במקום שהיד סולדת בו מותר, והוא שלא יתן על גבי האש או על גבי כירה ממש. ותדע לך, דהא תנינא (לעיל שבת לט, א) כל שבא בחמין מלפני השבת שורין אותו בחמין בשבת, כלומר: אפילו בכלי ראשון, ואע״ג דתולדת האור כאור והמבשל בה חייב חטאת כמבשל על האש ממש.\par \textbf{} ואע״ג שאסרו להחזיר על גבי כירה שאינה גרופה וליתן לכתחילה אפילו בגרופה וקטומה ואפילו דבר חם ומבושל לגמרי, התם היינו טעמא משום דמחזי כמבשל לפי שדרך בשול בכך, אבל כנגד המדורה או בכלי ראשון שאין דרכם של בני אדם לבשל כן רוב הפעמים אינו נראה כמבשל אלא כמפיג צינה. והלכך כל היכא דאיכא למיגזר באותו דבר ממש משום בשול, כמים או שמן שלא נתבשלו וכיוצא בהן אסור גזירה שמא ישהה ויבא לידי בשול, אבל מה שנתבשל לא גזרינן שמא יבא לבשל בעלמא.\par \textbf{} ופינה ממיחם למיחם דסלקא לן בתיקו לעיל (שבת לח, ב), ההיא במחזיר מיחם שני על גבי כירה היא ובמקום שהיה מיחם ראשון נתון.\par \textbf{} ולא עוד אלא אפילו תבשיל שלא נתבשל כלל אם הוא דבר קשה להתבשל, כגון בשר וכיו״ב שאינו מתבשל בכלי ראשון ולא כנגד המדורה, אפילו במקום שהיד סולדת בו מותר.\par \textbf{} וכללא דמלתא כל מקום שאין בא שם לידי בשול וכל דבר שאין בו משום בשול בין בכלי ראשון בין כנגד המדורה מותר ולא אמרו דסולדת אסור אלא במים ויין ושמן שהם (קשים) [קלים] להתבשל ובחמין שלא הוחמו כדי בישולן והראיה מדאמרינן לקמן בגמרא האלפס והקדרה שהעבירן מרותחין לא שאנו אלא תבלין אבל מלח לא דאפילו בכלי ראשון לא בשלה והיינו דרב נחמן דאמר רב נחמן מילחא צריכה בישולא כבשרא דתורא. ועוד מדאמרינן הכא שמן אף על גב שהיד סולדת בו מותר קסבר שמן אין בו משום בישול ואמרינן נמי גבי עובדא דרבי דאמר לו לרבי יצחק בר אבדימי טול בכלי שני ותן ש״מ שמן יש בו משום בשול דאלמא אם לא היה בו משום בישול אפילו כנגד מדורה במקום שהיד סולדת או בכלי ראשון מותר. וממה שאנו צריכין לדעת בכל מקום שאמרו כדי שתפיג צינתן היינו הפשר. והפשר היינו כל שלא הגיע לבשולו וכדמוכח בכולה שמעתין. ועוד תנן המחם שפינהו לא יתן לתוכו מים צונן בשביל שיחמו אלא נותן לתוכו או לתוך הכוס כדי להפשירן. ומכלל דברים אלו מותר לתת על גבי קדרה בשבת תבשיל שנתבשל מע״ש כל צרכו כגון פנאדיש וכיוצא בהן כדי לחממן ואפילו תהא היד סולדת בו ואף ע״פ שהקדרה נתונה על גבי האש לפי שאין דרך בשול בכך ואינו אלא כנותן בצד המיחם או כנגד המדורה והרי אין כאן משום חשש בישול אבל להטמין תחת הבגדים הנתונים על גבי מיחם ודאי מסתברא דאסור דהא נראין הדברים שאסור להטמין משחשכה אפילו תבשיל שנתבשל כל צרכו ואפילו בדבר שאינו מוסיף דפרקין דבמה טומנין ובמה אין טומנין בסתם תבשילין שנוי לא שנא הגיע למאכל בן דרוסאי ולא שנא קודם שהגיע למאכל בן דרוסאי ולא שנא נתבשל כל צרכו ובין מצטמק ויפה לו ובין מצטמק ורע לו מדאפליגי בפרקין דהכא באלין מילי בענין שיהוי ולא אפליגו בהו גבי הטמנה ש״מ לא שנא הכי לא שנא הכי לעולם אסור. ואע״ג דקי״ל דמותר להטמין את הצונן ואפשר דבין לשמרו מן החום ולהעמידו בצנתו ובין להפיג צנתו קצת אפי׳ הכי על גבי מחם אסור דכיון שהוא נותנו על גבי דבר חם ומטמין אפילו בדבר שאינו מוסיף ואפילו במה שהטמין בו המיחם מע״ש אסור. והיינו כרבא דנזהיה לההוא גברא דמנח כוזא דמיא אפומא דקומקמתא כדאיתא בריש פרק במה טומנין וכמו שפירש שם מורי הרב רבינו יונה ז״ל ונראין דבריו דכל שהוא מוליד חום בדרך הטמנה על גבי דבר חם כמטמין את הצונן על גבי מיחם אסור ומכל מקום להטיל עליה כלי לשמרה מן העכברים או שלא תתלכלך בעפרורית שרי שאין זה כמטמין להחם אלא שומר וכנותן כסוי על גבי קדרה וכ״ש אם הכלי שמוטל עליה רחב שלא יגע בה שאין זו הטמנה כנ״ל:}
\textblock{\textbf{דהא מעשה דרבי לאחר גזרה הוה.} כתוב בספר המאור איכא דקשיא ליה ההיא דגרסי׳ בפרק תפלת השחר פעם אחת התפלל רבי בע״ש וכו׳ ואוקימנא להזיע וקודם גזרה הכא נמי ממאי דבחמי טבריה ולאחר גזרה דלמא בחמי טבריה וקודם גזרה כי ההיא דהתם. ויש לומר אלו לא נכנס אלא להזיע לא היו נותנין לו פך שמן לסוך אלא ודאי לרחיצה ממש נכנס כההיא דאמרינן רחץ ולא סך מעיקרא דומה לנותן מים ע״ג חבית הלכך על כרחין בחמי טבריה ולאחר גזרה הוה מעשה ע״כ:}
\textblock{\textbf{לאפרושי מאיסורא שאני.} איכא למידק והיכי הוה סלקא דעתין דלישבקיה למעב׳ איסורא ולא אמר ליה איכא למימר משום דהוה מצי למימר כמאן דלא צריך להכי ולא לימא ליה אין מדיחין ואין סכין דלשון הוראה הוא וממילא שמעינן דלאפרושי מאיסורא שרי למימר אפילו בלשון הוראה וכן נמי שרי למימר עביד לי הכי והכי ואע״ג דממילא הוה הוראה כההיא דאמרינן טול בכלי שני ותן ולומר דבכלי ראשון אסור ובירושלמי התירו אפילו לשאול הלכות בית המרחץ בבית המרחץ דגרסינן התם שואלין הלכו׳ המרחץ בבית המרחץ והלכות בית הכסא בבית הכסא כהדא רשב״י על מסחי עם רבי מאיר אמר ליה מהו שנדיח אמר ליה אסור. מהו שנקנח אמר ליה אסור ולא כן שאל שמואל לרב מהו לענות אמן במקום המטונף אמר ליה אסור ואסו׳ דאמרית לן אסור אשכח תנ׳ שואלין הלכות המרחץ בבית המרחץ והלכות בית הכסא בבית הכסא. וכתב הרמב״ם ז״ל דמסתברא דגמ׳ דילן לא שריא אלא לאפרושי מאיסורא ומעשה דר׳ מאיר נמי לאו הכי הוה כדמייתי לה בסמוך אלא כדקא׳ הכא בקש להדיח לו וכו׳.}
\clearpage
\newsection{דף מא}
\textblock{מתני׳ \textbf{מוליאר הגרוף שותין ממנו בשבת.} פירש רש״י ז״ל לפי שאינו מוסיף הבל אלא משמר ומקיים חום שלהם שלא יצטננו אבל אנטיכי אע״פ שגרופה וקטומה אין שותין ממנה לפי שמוסיף הבל. וקא מקשו עלה בתוס׳ טובא חדא דאין להזכיר מוסיף הבל אלא גבי הטמנה שהיא אסורה גזירה שמא יטמין ברמץ. ועוד אם מחמת תוספת הבל למה המתין לאיסור עד השתא לימא הכין גבי תנור שהסיקוהו בקש וגבבה ולתני ואם נתן אין אוכלין ממנו בשבת כדתני הכא ועוד לימא תהוי תיובתיה מהכא למאן דאמ׳ להחזיר תנן אבל לשהות משהין אע״פ שאינה גרופה ואינה קטומה דהא קתני הכא דדוקא גרוף אבל שאינו גרוף אין שוהין. ועוד הך מתניתין במאי מיתוקמא דהא סתמא קתני אע״פ שגרופה אין שותין ממנה ולא פי׳ אי בשהחזיר אי נמי בששהא אי במזיד אי בשוכח דמשמע דבכל ענין אסור. ועוד דאי משום תוספת הבל קא אסרה מתניתין בשהיה תקשי לרבה ורב יוסף דאמרי לעיל שכח קדרה על גבי כירה ובשלה בשבת מותר בין בשוגג בין במזיד ומיירי בבשיל ולא בשיל. ועוד מאי שנא גרוף אפילו קטום נמי וכיון דבכלהו מתני׳ תני גרופה וקטומה עד שיגרוף או עד שיתן את האפר הכא נמי הוה ליה למיתני הגרוף או הקטום. ופירשו בתוספות דהכי קתני מוליאר הגרוף שותין ממנו במזיגה ביין כלומר לפי שאין המוליאר הגרוף נמי חם הרב׳ והלכך כשמוזגין יין במימיו אין היין מתבשל בתוכן אבל מים שבאנטיכי חמין ביותר ואם בא למזוג מהן ביין בין בכלי ראשון בין בכלי שני נמצא מתבשל ולפי פירוש זה דוקא גרוף התירו אבל קטום לא לפי שהקטימה אינה אלא כל שהיא ואפילו קטמה והובערה שרי. וא״כ עוד הוא מחמם ומוסיף חום והרי המים חמין וראוין לבשל כאלו לא גרוף ולא קטום ולא אמרו קטומה אלא להכר שלא יבא לידי חתוי כן יש לי לתרץ לפי פירושם. אבל מורי הרב רבינו יונה ז״ל הקשה עליהם והא חמין לתוך צונן בכל מקום מותר בין בכוס בין בספל בין באמבטי כדאיתא לקמן. ועוד היכי פסיק ותני אין שותין יתן לתוכו מים מרובין דומיא דמיחם שפינהו דמפרשי׳ בגמ׳ לא יתן לתוכו מים מועטין כדי שיחמו אלא נותן הוא לתוכו מים מרובין כדי להפשירן ולא גזרינן מרובין אטו מועטין. ועוד דלא הוה ליה למיתני אין שותין ממנו דחמין שבתוכו מותרין הן ושותין מהן והוה ליה למיתני אין מוזגין ממנו אי נמי לא יתן לתוכו כדרך ששנו במיחם ופירש הוא ז״ל שאסרו האנטיכי מפני שרגילין היו לתת לתוכו מים תדיר ולהוסיף בו מים ומפני שהגחלים מתחתיו ואי נמי מפני שדופנו עב חומו הולך ומוסיף והלכך אף ע״פ שהמים הצוננין מתפשרין בתוך המים החמין שבתוכו לשעתן לאחר שעה הם מתחממין והולכין בחום האנטיכי עד שמתבשלין שם וזה לא ידע ויחשוב מאחר שבשע׳ נתינתו לא נתבשלו ולא הוחמו אלא הופשרו הרי זה מותר ואינו אלא אסור כיון דלבסוף מתחממין בתוכן ולפיכן אסרו אפילו המים שהוחמו בו בע״ש שמא יוסיף בו מים בשבת. ומה שאסרו מוליאר שאינו גרוף מן הטעם שאמרו בירושלמי מפני שהרוח נכנסת לתוכו והגחלים בוערות כלומר שהמוליאר פתוח מתחתיו ונכנס׳ הרוח בתוכו ומבעיר את הגחלים בשבת זהו תורף פירושו של מורי הרב ז״ל ונכון הוא. וממה שאמרו בירושלמי באמת נראה שלאסור מים שהוחמו בו מע״ש אמרו מדגרסינן התם מוליאר הגרוף שותין ממנו בשבת גרוף אין שאינו גרוף לא א״ר אשיאן מפני שהגחלים נוגעות בגופו א״ר חנינא בריה דר׳ הלל מפני שהרוח נכנסת לתוכו והגחלים בוערות א״ר יוסא בר׳ בון מפני שהוא עשוי פרקים פרקים והוא מתירא שמא יתאכל דבקו והוא מוסיף מוי אנטיכי אע״פ שהיא גרופה אין שותין ממנה ר׳ חנינא ר׳ יסא ר׳ אחא אבא בר רב חנא בשם ר׳ יוחנן מפני שהיא מתחממת בכותליה רבנן דקסרין בשם רב חונא אם היתה גרופה ופתוחה מותר אבל לדידי קשיא לי אם לאסור מים שהוחמו בו מע״ש היה להם לסדר משנה זו קודם משנת אין נותנין ביצה בצד המחם לפי שמראש הפרק ועד אותה משנה מיירי בתבשילין וחמין שמע״ש אבל מאותה משנה ולמטה לא דברו אלא בחמין ותבשילין המתבשלין או מתחממין או מתפשרין בשבת כביצה בצד המיחם ולא יפקיענה בסודרין ולא יטמיננה בחול ובאבק דרכים ומעשה שעשו אנשי טבריא דאמרו להן חכמים כחמין שהוחמו בשבת, וכן המיחם שפינהו לא יתן לתוכו צונן, והאלפס והקדרה שהעבירן מרותחין, שכל אלו כשמתבשלין בשבת, והיאך אפשר שכל המשניות של מעלה ושל מטה בתבשילין ובחמין שבשבת וזו באה באמצע שהיא בחמין שהוחמו מערב שבת.\par \textbf{} ומשום הכי היה נראה לפרש דהכי קאמר: מוליאר הגרוף שותין ממנו בשבת, כלומר: אם נתן לתוכו מים בשבת ונפשרו שותין ממנו, אבל בשאינו גרוף לא ואפילו קטום, מפני שהגחלים נוגעות בגופו והוי ליה כנותן על גבי האש ממש ואסור אפילו בדיעבד ואע״פ שלא נתבשלו אלא שהופשרו, וכענין אותה דרבי יהודה דאסר אפילו בדיעבד ואפילו בשוכח תבשיל שנתבשל כל צרכו ומצטמק ויפה לו, ואע״פ שאין כאן משום בשול וכדתניא בברייתא דלעיל בריש פרקין (שבת לח, א) ומשום גזירה, והכא נמי גזרינן ואפילו בדיעבד. ואנטיכי אע״פ שגרופה וקטומה אין שותין ממנה כלל, לפי שאנטיכי מתוך שהוא חם ביותר מפני שנחושתו מחממתו לעולם אי אפשר להפשיר בו ואפילו נתן לתוכו מים מרובים כדי מילואו, וכאותה ששנינו בברייתא לקמן בסמוך (שבת מב, א) אבל באמבטי חמין לתוך צונן אבל לא צונן לתוך חמין, דמתוך שהוא חם ביותר אי אפשר לתת לתוכו מים כל כך שיתפשרו בתוכו אלא לעולם יבואו לידי חימום. ובדין הוא דליתני הכא מוליאר הגרוף נותנין לתוכו מים כדי להפשירן כדתני גבי מיחם שפינהו, אלא משום דסליק ממעשה שעשו אנשי טבריא שאמרו להם חכמים בשבת כחמין שהוחמו בשבת ואסורין ברחיצה ובשתיה ביום טוב כחמין שהוחמו ביום טוב ואסורין ברחיצה ומותרין בשתיה, תנא הכא נמי שותין ואין שותין. כך נראה לי. אלא שאין נראה כן מן הירושלמי וכמו שכתבתי למעלה.\par \textbf{} ויש לפרש פירוש הירושלמי כך, מוליאר הגרוף שותין ממנו בשבת, הא אינו גרוף לא, ואמאי לא יהא כמשהה על גבי כירה שאינה גרופה, ופירש ר׳ אשיאן מפני שהגחלים נוגעות בגופו, ולא התירו כירה שאינה גרופה ואפילו למאן דאמר להחזיר תנן אלא כשאין גחלים נוגעות בגופה של קדירה הא נוגעות לא מפני שהוא כטומן ואסור משום טומן בדבר המוסיף הבל. ומה שהתירו לסמוך לכירה שאינה גרופה וקטומה וכדתניא לעיל (שבת לז, א), התם הוא דמפסיק כלי אבל הכא שהגחלים נוגעות ממש בגופו של כלי שהמים בתוכו הרי זה כהטמנה, וכענין שכתבנו למעלה (לו, ב ד״ה עד) בשם רב האי ז״ל ור״ח ז״ל. ור׳ חנינא פירש מפני שהרוח נכנסת לתוכו והגחלים בוערות, כלומר: שכיון שהרוח נופח באש והולך ומתבער בשבת הרי זה כמבעיר כירה תחת קדירה, וגזרינן דלמא אתי איהו גופיה לחתויי ואפילו בקטומה. ושמא ר׳ חנינא חולק הוא על מה שפירש ר׳ אשיאן, ולומר שאין משום הטמנה במה שהגחלים נוגעות מצדו ולא מתחתיו ולא מכל צדדיו. ורבי יוסי בר׳ אבין פירש מפני שהוא עשוי פרקים פרקים והוא מתירא שמא יתאכל דבקו והוא מוסיף מוי, ולפיכך גזרו ואסרו עליו אפילו אותן שהוחמו מערב שבת כדי שלא יבא להשהות בו ויבא לידי בשול.\par \textbf{} ומכל מקום אנו למדין מן הירושלמי, דאין משהין בכירה חלולה שהרוח נכנסת בתוכה, ואם עבר ושהה אסור, מפני שהוא נראה כמבעיר ממש תחת קדירה וגזירה שמא יחתה בגחלים ואפילו בכירה קטומה. ואם כן כירות הללו שלנו שהן פתוחות מתחת ונוקבין אותן בקרקעיתן כדי שתהא הרוח מנשבת בהן ותבער בגחלים, אסור לשהות בהם ואפילו בקטומה. ושמא לא אמרו אלא במוליאר מפני שהוא חלול ממש והרוח נכנסת בו וגחלים בוערים ממנו, הא בכירה שלנו אין הבערת הרוח ניכרת בהן כל כך ולא גזרינן. ואי נמי איכא למימר דר׳ אשיאן ורבי יוסי בר׳ בון לא הודו לו ולא גזרו בהכין. וכל הני מילי מדרבנן נינהו והלכה כדברי המיקל, ופוק חזי מאי עמא דבר, וכאותה שאמרו בירושלמי (פאה פ״ז ה״ה) כל הלכה שהיא רופפת בידך ואין אתה יודע מה טיבה פוק חזי מאי צבורא נהיג ונהג כן.}
\textblock{ גמרא:\textbf{ הא קא מצרף הא מני ר׳ שמעון היא וכו׳.} ואם תאמר לימא ליה כשלא הגיע לצירוף וכולי עלמא, וכדמשני ביומא (לד, ב) גבי עששיות של ברזל מטילין לו. ונראה לי משום דמתניתין סתמא קתני המיחם שפינהו. ועוד דאם איתא, כי מפליג בין להחם בין להפשיר ליפלוג וליתני בדידה, במה דברים אמורים בשלא הגיע לצירוף אבל הגיע לצירוף לעולם אסור. ועוד דניחא ליה לאוקומה כרבי שמעון דהלכתא כוותיה, אבל ההיא דיומא על כרחין אוקמה בשלא הגיע לצירוף משום דרבי יהודה תני לה.\par \textbf{} ואם תאמר היכי מוקמינן לה כרבי שמעון, והא אביי ורבא דאמרי תרוייהו (לקמן שבת עה, א וש״נ) מודה רבי שמעון בפסיק רישיה ולא ימות. תירץ הראב״ד ז״ל, דהא דרב אדא וכן נמי ההיא דאביי בסמוך דאמר אבל פינה ממנו מים לא יתן לתוכו מים כל עיקר מפני שהוא מצרף דהא מני רבי יהודה היא, היינו מקמי דשמעוה מרבא, הא לבתר דשמעוה מרבא דאמר מודה רבי שמעון בפסיק רישיה ולא ימות סברוה, ומתניתין על כרחין בשלא פינה ממנו מים אבל פינה ממנו מים אפילו לרבי שמעון אסור. ומפורש הוא בפרק רבי אליעזר דמילה (לקמן שבת קלג, א) בשמעתא דמילה דוחה את הצרעת, דמעיקרא הוה סבירא ליה לאביי דרבי שמעון אפילו בפסיק רישיה ולא ימות שרי עד דשמעה מרבא וסברה.\par \textbf{} ואינו מחוור, דאם איתא הוה ליה לאקשויי הכא והא אביי ורבא דאמרי תרוייהו מודה רבי שמעון בפסיק רישיה ולא ימות כדמקשה בעלמא (כגון לקמן שבת עה, א). ועוד אם כן ליתא אפילו לדשמואל דאמר אפילו שיעור לצרף, ולא כן דעת הגאונים ז״ל אלא בין פינה ממנו מים בין שלא פינה ממנו מים נותן לתוכו כדי לצרף ואינו חושש, וכן פסק ר״ח ורבנו האי גאון ז״ל.\par \textbf{} אלא יש לומר דמיחם לאו פסיק רישיה ולא ימות הוא, דאפשר דלא הגיע לצירוף, מפני שהמים שבתוכו מונעין אותו להתחמם ולבא לידי צירוף. ואע״ג דקא מקשה להדיא והלא מצרף, לאו למימרא שמצרף בודאי אלא ה״ק והא שייך ביה צירוף והלכך איכא למיחש שמא יצרף. כך תירצו בתוס׳.}
\textblock{\textbf{מתקיף לה אביי מידי מיחם שפינה ממנו מים קתני, שפינהו קתני.} ואם תאמר והא כתיב (ישעיה נז, יד) פנו דרך, וכתיב (ויקרא יד, לו) ופנו את הבית פניתי (את) הבית (בראשית כד, לא), וכן רבים. יש לומר לשון תורה לחוד ולשון חכמים לחוד, וכדאמרינן (עירובין כ, ב) המפנה חפציו מזוית לזוית, וכן רבים. ואע״ג דאמרינן (כתובות עז, ב) פנו מקום לבר ליואי והמפנה את האוצר (ראה לקמן שבת קכו, ב), אפ״ה הוה ליה למיתני שפינה ממנו מים כי היכי דלא ניטעי בה, כיון דנפקא מינה לרבי שמעון ולרבי יהודה לענין צירוף.}
\textblock{\textbf{ומיחם שפינה ממנו מים לא יתן לתוכו מים כל עיקר מפני שהוא מצרף ור׳ יהודה היא.} ואם תאמר והא אמר אביי (לעיל שבת כב, א) כל מילי דמר עביד כרב בר מהני תלת, וחדא מינייהו גורר אדם מטה כסא וספסל ובלבד שלא יתכוין לעשות חריץ, ומסתמא אביי כרביה סבירא ליה, אם כן היכי מוקי לה למתניתין כרבי יהודה. יש לומר דאביי ודאי כרבי שמעון סבירא ליה, אלא דמתניתין קשיתיה מדקתני המיחם שפינהו ולא קתני המיחם שפינה ממנו מים, אלמא כשפינה ממנו מים אסור וכרבי יהודה, ומיהו איהו לאו כי הא מתניתין סבירא ליה אלא כרבי שמעון.\par \textbf{} והיינו דאצטריך אביי למימר אבל פינה ממנו מים לא יתן לתוכו מים כל עיקר, דאי משום דאתא אביי למימר מדנפשיה דכדי צירוף אסור ומשום דרבי יהודה דאמר דבר שאין מתכוין אסור, למה ליה למימר אבל פינה ממנו מים, אפילו בלא פינה ממנו מים מצי לאשמעינן, ולימא הני מילי שיעור להפשיר אבל שיעור לצרף, כלומר: מרובין כל כך המים שיצטננו המים שבמיחם לגמרי ויצרפו את המיחם אסור וכרב דבסמוך. אלא דאביי לאו מדעתיה דנפשיה קאמר לה אלא טעמא הוא דקא מפרש, דמהאי טעמא תנא תנא פינהו ולא תנא פינה ממנו מים, לאשמעינן דרבי יהודה היא ובפינה ממנו מים לא יתן לתוכו מים כל עיקר, והוא הדין לפינהו שלא יתן לתוכו מים מרובין כדי לצרף, דלהפשירן דמתניתין דוקא, ולומר דמועטין כדי שיעור להחם ואי נמי מרובין שיהא בהן כדי לצנן לגמרי מים שבתוך המיחם ולצרף את המיחם אסור וכרבי יהודה, ולא התיר אלא בשיעור בינוני שיהא בו שיעור להפשיר דוקא וכדרב.\par \textbf{} ופלוגתא דרב ושמואל באוקימתא דאביי שייכא, דאילו לאוקימתא דרב אדא כל מה שהוא נותן יש בו כדי לצרף, דכיון שנותן מים צונן לאחר שפינה ממנו מים חמין בין מרובין בין מועטין מצרפין, וכדמשמע מדפריך בהדיא לרב אדא והלא מצרף ולא שני בשאין בהם כדי לצרף, וכן נמי מדאמר אביי אבל פינה ממנו מים לא יתן לתוכו מים כל עיקר מפני שהוא מצרף.\par \textbf{} ושמואל דמפרש לה למתניתין כאביי ולא מפרש לה כרב אדא אע״ג דסבירא ליה כרבי שמעון, מהאי טעמא דאמרן הוא דלישנא דפינהו קשיתיה, ואלא מיהו סבירא ליה לשמואל דאע״ג דמתניתין פינהו קתני לאו בדוקא אלא הוא הדין לפינה ממנו מים, משום דלא קפיד תנא דמתניתין אלא שלא יתן כדי שיחמו הא כדי שיצטננו המים שהם במיחם שפיר דמי, ולא תנא כדי להפשירן אלא לאשמועינן דאפילו להפשיר שרי.\par \textbf{} ואפשר דאביי נמי הכי סבירא ליה, אלא משום דשמעיה לרב אדא דאמר דמתניתין דוקא כרבי שמעון ומתניתין שפינה ממנו מים קתני, אמר איהו דאדרבה מתניתין פינהו קתני ואי איכא למידק טפי איכא למידק דכרבי יהודה אתיא דדוקא פינהו אבל פינה ממנו מים כלל וכלל לא, ואלא מיהו בין דדיקא מתניתין הכי או לא איהו כרבי שמעון סבירא ליה.}
\textblock{\textbf{אמר רב לא שנו אלא שיעור להפשיר.} ואי קשיא לך דהיינו שיעור להפשיר והיינו שיעור לצרף, וכדמשמע לעיל מדאמר רב אדא אבל נותן לתוכו מים מרובין כדי להפשירן ואקשינן עלה והלא מצרף, אלמא כל שהן פושרין יש בהן כדי לצרף. לא היא, דכבר כתבנו דכל שהוא נותן לתוך המיחם שאין בו מים חמין יש בו צירוף בין שיתן בהן כדי להפשיר ואפילו אין בהם אלא כדי שיחמו, אבל בנותן לתוך מיחם שיש בו מים חמין לעולם אין המים חמין שבתוכו       אותו להצטרף עד שיצטננו לגמרי, וטעמא שהמים הצוננין הם שמצרפין אבל לא הפושרין.\par \textbf{} אבל רש״י ז״ל נראה שפירש שיטה זו בענין אחר, והוא ז״ל סבור דפלוגתא דרב ושמואל בדרב אדא שייכא, ושיעור לצרף היינו בשנותן לתוך המיחם מים מרובין כדי מילואו שיהא מלא על כל גדותיו, אבל כשאינו נותן כדי מילואו אינו מצרף. וכי אקשינן לעיל אדרב אדא והלא מצרף, אמים מרובין כדי להפשירן אקשינן, כלומר: דכיון דלא חלקת במים מרובין אלא אדרבא משמע דכל שהוא מרבה בשיעורן טפי עדיף ואפילו כדי מילואו של מיחם ואמאי והלא מצרף. זו היא שיטתו של רש״י ז״ל לפי מה שנראה מלשונו.}
\clearpage
\newsection{דף מב}
\textblock{\textbf{גחלת של מתכת אבל לא של עץ וכו׳.} ואסיקנא במלאכה שאינה צריכה לגופה סבר לה כרבי יהודה, והאי מתכוין לכבות הוא אלא שהיא מלאכה שאינה צריכה לגופה. ואם תאמר אם כן אפילו גחלת של מתכת ליתסר. כבר פירש רש״י ז״ל דבההיא לא שייך כבוי דאורייתא אלא דרבנן, והיכא דאיכא נזקא דרבים לא גזרו על השבות.\par \textbf{} ואם תאמר כיון דשמואל כרבי יהודה אמרה לשמעתיה, אכתי גחלת של מתכת היכי מכבה והלא מצרף. לא היא, דעד כאן לא אוקימנא לה כרבי יהודה אלא משום דאסר לכבות גחלת של עץ ובהא הוא דאמר כרבי יהודה, אבל מאי דשרי לכבות גחלת של מתכת ולא חייש לצרוף דלא כרבי יהודה אלא כרבי שמעון, דהא דבר שאין מתכוין הוא ובדבר שאין מתכוין כרבי שמעון סבירא ליה. ואי נמי צירוף לרבי יהודה מדרבנן היא ולא גזרו בה רבנן משום נזקא, וכדאמרינן במסכת יומא בפרק אמר להם הממונה (יומא לד, ב) אבל הכא צירוף [דרבנן] הוא. ושמא בכל צירוף קאמר ואפילו דכלים.\par \textbf{} ואם תאמר היכי מסקינן הכא דשמואל במלאכה שאינה צריכה לגופה כרבי יהודה סבירא ליה ואפילו במקום הזיקא לא שרינן ליה במלאכה דאורייתא, והא אמר שמואל (לקמן שבת קז, א. ומובא לעיל שבת ג, א) כל פטורי דשבת פטור אבל אסור בר מהני תלת וחדא מינייהו צידת נחש, וההיא כרבי שמעון היא וכדאמרינן בפרק שמנה שרצים גמרא הצדן שלא לצורך פטור (לקמן שבת קז, ב) איכא דמתני לה אהא הצד נחש בשבת וכו׳ מאן תנא אמר רב יהודה אמר רב רבי שמעון היא דאמר מלאכה שאינה צריכה לגופה פטור עליה. איכא למימר דשמואל בההיא לאו כרבי שמעון סבירא ליה, והא דחשיב לה בהדי פטור ומותר הכי קאמר, למאן דפטר בהו לאו פטור ואסור הוא אלא מותר לכתחלה, וכדדייק לה שמואל בפרק רבי אליעזר דאורג (לקמן שבת קז, א) מלישנא דמתניתין.\par \textbf{} אבל רבנו האי גאון ז״ל כתב דרבי יהודה אפילו במקום הזיקא אסר, אלא דשמואל לא סבירא ליה כוותיה במקום דאיכא הזיקא, ומשום הכי התיר לצוד את הנחש ולכבות גחלת של מתכת, אבל גחלת של עץ אסור לכבותה. ומה הפרש ביניהן, שגחלת של עץ אדמדמת היא וכיון שנראית פורשין ממנה עוברי דרך ואין באין לידי היזק, אבל גחלת של מתכת אף על פי שכבתה כיון שהיא חמה שורפת היא [ומכוה] ואינה נראית ומזקת. אלו דברי הגאון רבנו האי ז״ל. וכן כתבו הרב בעל ההלכות ז״ל ור״ח ז״ל.\par \textbf{} ומן התימה הוא, היאך התיר שמואל צידת נחש שהיא מלאכה דאורייתא משום הזיקא. ויש לומר דכיון דדרכו להזיק ורבים ניזוקין בו כסכנת נפשות חשיב ליה שמואל, דאי אפשר לרבים ליזהר ממנו דאם זה יזהר זה לא ישמר ממנו, משא״כ בגחלת של עץ דהיא אינה הולכת ומזקת וכל אחד יכול להשמר ממנה, אבל רבי יהודה לעולם אסר ואפילו במקום נזק.\par \textbf{} ומדבריהם למדנו דרבי יהודה אפילו במלאכה דרבנן לא שרי מלאכה שאינה צריכה לגופה ואפילו במקום הזיקא דרבים, שהם ז״ל כתבו בגחלת של מתכת קא שרי בה שמואל ופליג בה אדרבי יהודה, וכבוי גחלת של מתכת ודאי מדרבנן, וכדאמרינן פרק רבי אליעזר דמילה (לקמן שבת קלד, א) ממתקין את החרדל בגחלת ואוקימנא בגחלת של מתכת אבל לא בשל עץ משום דהא לאו דאורייתא והא דאורייתא.}
\textblock{\textbf{בדבר שאין מתכוין סבר לה כרבי שמעון במלאכה שאינה צריכה לגופה סבר לה כרבי יהודה.} ומהא שמעינן דלאו הא בהא תלי, אלא שהמקשה היה סבור דמילתא דתליא הא בהא היא. ואי נמי יש לומר דקסבר דכיון דפלוגתא דרבי יהודה ורבי שמעון היא ובדבר שאין מתכוין סבר לה כרבי שמעון הכי נמי בפלוגתייהו דמלאכה שאינה צריכה לגופה סבר לה כרבי שמעון, ואסיקנא דלא.}
\textblock{\textbf{נותן אדם חמין לתוך צונן אבל לא צונן לתוך חמין.} פירש רש״י ז״ל משום דתתאה גבר. ואינו מחוור דאם כן      כיצד צולין (פסחים עו, א) דפליגי בהא מילתא רב ושמואל חד אמר תתאה גבר וחד אמר עילאה גבר, לותביה מהא למ״ד עילאה גבר. ועוד דהא אפילו למאן דאמר תתאה גבר מודה דכדי קליפה אסור, דאדמקרר ליה תתאה לעילאה מיבלע בלע, ואם כן אפילו חמין לתוך צונן הרי מבשל כדי קליפה וליתסר. אלא ודאי נראה דבמים שהן מתערבין אלו לתוך אלו ליכא למימר בהא תתאה ועילאה.\par \textbf{} ובתוס׳ פירשו שדרך המערה שנותן המועט לתוך המרובה, והלכך חמין מועטין לתוך צונן מרובין אין בהם כח לבשל, אבל צוננין מועטין לתוך חמין מרובין מתבשלין. וגם זה אינו מחוור בעיני, דהכא סתמא קתני לא שנא מרובין ולא שנא מועטין, ולתוך הכוס ואפילו לתוך המיחם רגילות הוא לתת צונן הרבה כדי להפשיר מפני שהן עשוין לשתיה ואין אדם שותה חמין גמורין אלא פושרין, והוה ליה למיתני הכי צונן לתוך חמין כדי שיחמו אסור וכדי להפשירן מותר, וכדתני במתניתין.\par \textbf{} אלא נראה דכלים אלו בין כוס בין אמבטי וספל כולן כלי שני הם. וכן פירשו הגאונים ז״ל. וכן כתב הרמב״ם ז״ל. והכי נמי משמע מדאקשינן לקמן אלא לרב נחמן רחיצה בחמין מי ליכא, ואי איתא משכחת לה בכלי שני. ואע״ג דאמבטי דלעיל (שבת מ, ב) גבי עובדא דרבי משמע דכלי ראשון הוא מדאמר ליה טול בכלי שני ותן, אמבטי דהכא כלי שני הוא, דהא אמבטאות הרבה יש והכא דבר הלמד מענינו אמבטי דומיא דכוס. ומהתם נמי משמע לי דהכא אפילו בנותן מים מרובין ואפילו כדי מילואו של אמבטי ושל ספל קאמר, דאי לא הא משכחת לה בנותן מים מרובין כדי להפשירן, שהרי אין אדם רוחץ במים חמין שהוא מסלד בהן.\par \textbf{} והכא הכי פירושא: נותן אדם חמין כל שהן לתוך הצונן, לפי שהן מתקררין בערוין ומתערבין ממש בתוך הצונן ואין כח בהן לבשל, ואפילו בבאין מכלי ראשון, דאפילו תמצא לומר בעלמא דערוי דכלי ראשון ככלי ראשון, הני מילי כגון שמערה על גבי תבלין וכיוצא בהן שאין חמין מתערבין בתוכן אלא מכין עליהם בקילוחן ומבשלין, אבל מים במים הן מתערבין ממש בתוך הצונן ואין בהן כח לבשל אדרבא כל שהן נופלין הן מצטננין, וכל שכן הכא בכוס שהוא כלי שני שאינו מבשל. אבל צונן לתוך חמין אסור, משום דגזור בית שמאי כלי שני אטו כלי ראשון הואיל ואפשר דאתי בכלי ראשון לידי איסורא דאורייתא דכלי ראשון מבשל. ובית הלל מתירין אפילו בצונן לתוך חמין, דכלי שני אינו מבשל, ואפילו נתן לתוכן צוננין מועטין דאין באין לתוכן לידי בשול אלא לידי הפשר. והא דתנן במתניתין אבל נותן לתוכו או לתוך הכוס כדי להפשירן ומפרשינן עלה בגמרא אבל נותן לתוכו מים מרובין כדי להפשירן, משום מיחם הוא דאמרינן אבל בכוס כל שהוא נותן לתוכו היינו להפשירן. אבל באמבטי שעומדין לרחיצת אדם מחממין הרבה, והלכך צונן לתוך חמין אסור לעולם שאי אפשר לבא לידי הפשר אלא לידי בישול ואפילו נתן באמבטי כדי מילואו, ואף על פי שהוא כלי שני ובעלמא אינו מבשל הכא שאני לפי שדרכן לחממן יותר מדאי. וספל כיוצא בו למאי דהוה סבירא ליה לרב יוסף מעיקרא, אבל במסקנא לא מדתני רבי חייא ספל אינו כאמבטי, דמים שבספל אינן חמין כל כך ואינו אלא ככוס, והואיל וכלי שני הוא לעולם מותר שאינו בא אלא לידי הפשר, ואפילו היו מים שהיד סולדת בהן לאחר שנתערבו צונן לתוכן, אינו מחמת הצונן שבו להיות יד סולדת אלא מחמת הראשונים שלא נצטננו.}
\textblock{\textbf{ולמאי דסליק אדעתין מעיקרא דספל הרי הוא כאמבטי ואמר רב נחמן הלכה כרבי שמעון בן מנסיא וכו׳.} מהא משמע דרב יוסף כרב נחמן סבירא ליה, דאי לא מאי קושיא, משכחת לה לרב יוסף בחמין לתוך צונן כבית הלל בין בספל בין באמבטי ולרב נחמן בספל, אלא משמע דכהדדי סבירא להו.\par \textbf{} ואם תאמר אם כן למאי דמתרץ דרבי שמעון בן מנסיא ארישא קאי, למה ליה לרב יוסף דאמר ספל הרי הוא כאמבטי, לימא הרי הוא ככוס דכולהו אסירי. ועוד לישתוק מינה לגמרי, דמי תיסק אדעתין דגרע טפי ספל מכוס. תירצו בתוס׳ דלאו אליבא דרבי שמעון בן מנסיא איצטריכא ליה למתנייה ולאו אליבא דהלכתא קאמר לה, אלא אליבא דבית הלל דשרי בכוס ואסרי באמבטי.\par \textbf{} ולענין פסק הלכה איכא למימר דאמבטי אפילו חמין לתוך צונן אסור כרבי שמעון בן מנסיא וכרב נחמן דפסק הלכתא כוותיה, דהא דדחינן ומוקמינן לה לדרבי שמעון בן מנסיא ארישא היינו למאי דקא סלקא דעתין מעיקרא דספל כאמבטי כי היכי דנשכח רחיצה דשריא בשבת, אבל למסקנא דאסקינן דספל אינו כאמבטי תו לא צריכינן לאוקומה לדרבי שמעון ארישא אלא אסיפא, אבל ברישא דכוס מודה רבי שמעון דשרי וכמתניתין דתני אבל נותן הוא לתוך הכוס, ורבא דלא קפיד בספל לא פליג אדרב נחמן.}
\textblock{\textbf{אבל הגאונים ז״ל לא פסקו כן, דסבירא להו דהא דרבי שמעון בן מנסיא לעולם ארישא קאי. ונראה דקא סברי מדקאמר מי סברת אסיפא קאי ארישא קאי דאלמא קושטא דמלתא הכין, דאי לא לימא דלמא ארישא קאי. ועוד דמדאצטריך רב הונא בריה דרב יהושע למימר חזינא ליה לרבא דלא קפיד אמנא משמע דאיכא מאן דפליג בה, והיינו דר״ש בן מנסיא ורב נחמן דסבירא ליה כותיה, דאי לא למה ליה למימר חזינא ליה דלא קפיד פשיטא, הלכך דרבי שמעון בן מנסיא ליתא, דאע״ג דפסק רב נחמן כותיה הא       } רבא דהוא בתרא דלא קפיד בה ועביד בה עובדא. וספל הרי הוא ככוס, ואע״ג דמערה לתוכו צונן לתוך חמין דלא מפסיק כלי שרי וכדתני רבי חייא, וכל שכן כוס וכסתמא דמתניתין דנותן לתוך הכוס, ואפילו באמבטי נמי כבית הלל בחמין לתוך צונן.}
\textblock{ מתני׳:\textbf{ האלפס והקדרה שהעבירן מרותחין לא יתן לתוכן תבלין אבל נותן הוא לתוך הקערה או לתוך התמחוי.} פירש רש״י ז״ל: לתוך הקערה או לתוך התמחוי אחר שעירה לתוכן את החמין, אבל לתת לתוך הקערה ולערות עליהן מן האלפס לא דערוי דכלי ראשון ככלי ראשון, ולפיכך אסור לערות מים מכלי ראשון על התרנגולת כדי למלגה מפני שהוא כמבשל הדם בתוכה. ולדבריו הא דאמרינן בפרק בתרא דע״ז (עד, ב) נעוה ארתחו, אפילו בעירוי דכלי ראשון הוא, ולא בעי שירתיחו תחתיו. וכן פירש ר״ת ז״ל.\par \textbf{} אבל ר״ח ז״ל פירש שם: ארתחו הרתיחוה באש, כלומר: נעוה של חרס רתחה מתחתיו באש. וכן פירש רבנו שמואל ז״ל נותן לתוך הקערה ולתוך התמחוי ומערה עליהם דעירוי דכלי ראשון לאו ככלי ראשון, דלא אשכח איסורא אלא בנותן בתוך הקדרה ובתוך האלפס דמשמע לתוך הקערה לעולם מותר, אלא דרישא דוקא וסיפא בין בנותן תבלין לתוכה ומערה עליהן בין במערה לתוך הקערה ואחר כך נותן את התבלין. ועוד מביא ראיה מדקיימא לן תתאה גבר, ואי אמרת ערוי מבשל אם כן עלאה גבר. ועוד דכיון דקיימא לן תתאה גבר היאך נתיר להגעיל בערוי דכלי ראשון כמו שפירשו הם בנעוה ארתחו, והא כוליה כלי בלע האיסור והערוי אינו מפליט אלא כדי קליפה, אלא ודאי אין הגעלת כלי שנתבשל בו איסור בערוי רותחין מכלי ראשון. ואולי הם ז״ל לא אמרו כן אלא ביין נסך שאין בליעתו אלא כדי קליפה.\par \textbf{} ונשאלה שאלה לגאון ז״ל, והשיב דערוי דכלי ראשון אינו מבשל מהא דהאלפס והקדרה, מדלא מפליג בקערה עצמה. והא דאמרינן בזבחים פרק דם חטאת (זבחים צה, ב) תנו רבנן אשר תבשל בו ישבר (ויקרא ו, כא) אין לי אלא שבשל בו, עירה לתוכו רותח מנין תלמוד לומר בו ישבר, אלמא דעירה לתוכו רותח חשיב בשול, לא היא, דהתם היינו טעמא משום שבלע כלי חטאת ואע״פ שלא נתבשל בו. ותדע לך דהתם קרי ליה לערוי בלוע בלא בשול, מדבעינן התם תלאו באויר תנור מהו, תא שמע אחד שבישל בו ואחד שעירה לתוכו רותח, ומהדר בלוע בלא בשול לא קמבעיא לי כי קמבעיא לי בשול בלא בלוע, אלמא עירה לתוכו חשיב בלוע בלא בשול.\par \textbf{} ור״ת ז״ל הביא ראיה ממה שאמרו בירושלמי כאן (ה״ה) ובפרק קמא דמעשרות (ה״ד) דערוי דכלי ראשון ככלי ראשון ומבשל, דגרסינן התם: מהו ליתן תבלין מלמטן ולערות עליו מלמעלה, רבי יונה אמר אסור וערוי ככלי ראשון הוא ופליג עלה רבי יוסי, חיליה דרבי יונה מן הדא אחד שבשל ואחד שעירה לתוכו רותח וכה הוא [אמר הכין], אמר ר׳ יוסי תמן כלי חרס בולע תבלין אינן מתבשלין, התיב רבי יוסי בר׳ בון והתני אף בכלי נחשת כן אית לך מימר כלי נחשת בולע, פירוש התיב רבי יוסי בר׳ בון אי טעמא משום דכלי חרס קל לבלוע בפחות מכדי בשול סמנין, והתניא אף בכלי נחשת ויש לך לומר שכלי נחשת קל לבלוע ברתיחה מועטת של כלי שני, הא אינו אלא מפני שערוי ככלי ראשון והלכך אף בכלי נחשת כן, והוא הדין לתבלין שהן מתבשלין בערוי דכלי ראשון.\par \textbf{} ומיהו אינה ראיה, דאדרבה יש ללמוד משם דאינו ככלי ראשון, מדמסיק התם מהו לערות עם הקלוח אמר רבי חנינא בריה דרבי הלל מחלוקת דרבי יונה ורבי יוסי, רבי יצחק בן גרפתא בעא קומי רבי מנא עשה כן בשבת חייב משום מבשל עשה כן בבשר וחלב חייב משום מבשל, א״ל כיי דאמר רבי זעירא איזהו חלוט ברור כל שהאור מהלך תחתיו וכה איזהו תבשיל ברור כל שהאור מהלך תחתיו, עד כאן בירושלמי. אלמא ערוי אינו ככלי ראשון ואינו מבשל, וכרבי יוסי דפליג עליה דרבי יונה ואע״ג דעירה לתוכו. [והא דאמרינן] אף בכלי נחשת כן וצריך מריקה ושטיפה, התם לאו משום בשול הוא אלא משום בלוע, וכדאיתא בגמרין בזבחים בפרק דם חטאת (זבחים צה, ב).}
\textblock{ גמרא:\textbf{ נשברה לו חבית בראש גגו.} פירש רש״י ז״ל: חבית של טבל שאינה ראויה לטלטל. והקשו עליו בתוספות אם כן ליתני של טבל, כדתני במתניתא דלקמן (שבת מג, א) חבית של טבל שנשברה. ועוד דהא קתני סיפא (לקמן שבת קיז, ב) נזדמנו לו אורחים מביא כלי אחר וקולט כלי אחר ויצרף, ואי בטבל אורחים מאי עבידתייהו. אלא ודאי במעושר, ואפילו הכי אסור משום דהצלה שאינה מצויה לא התירו אפילו בדבר הראוי אלא אם כן הוא לצורך השבת. ותדע לך מדאקשינן (לקמן שבת מג, א) דלף ולא מפרקינן בדלף הראוי כדלקמן (שם), אלמא אפילו בדלף הראוי לא התירו. ואי קשיא לך בדבר הראוי אמאי לא. תירץ הרמב״ן ז״ל משום דבכל הצלה איכא למיחש למאי דאמרינן בפרק כל כתבי הקדש (לקמן שבת קיז, ב) מתוך שאדם בהול על ממונו אי שרית ליה אתי לאתויי דרך רשות הרבים, ואף על פי כן התירו בכלי אחד משום איבוד ממון, ואי אמרת דלא חיישי רבנן להצלה      מצויה היה להם לאסור הכל משום גזירה דשמא יביא כלי, עד כאן.}
\textblock{\textbf{ניצוצות שכיחי.} פירש רש״י ז״ל (ד״ה שאינה): אבל טפטוף דשמן לא שכיח והצלה שאינה מצויה היא. אבל בתוס׳ פירשו דלהכי לא שרינן נתינת כלי תחת הנר לקבל את השמן, לא משום דטפטוף לא שכיח, אלא אדרבא מתוך שהיא עשויה לטפטף רגילין הן להניח כלי תחת הנר בערב שבת, והלכך נתינה דבשבת הויא הצלה שאינה מצויה. ואם תאמר אם כן אף בניצוצות נאמר כן דהא אמרינן דשכיחי. יש לומר שכיחי קצת ומשום הכי נותנין בשבת, אבל לא שכיחי כל כך כשמן שיהא מצוי מערב שבת ושתהא נתינה בשבת הצלה שאינה מצויה בהן.}
\clearpage
\newsection{דף מג}
\textblock{ הא דאמרינן:\textbf{ טבל מוכן הוא אצל שבת, שאם עבר ותקנו מתוקן.} לאו למימרא שיהא מוכן גמור וראוי לטלטלו, דהא תנן לקמן בפרק מפנין (שבת קכו, ב) אין מטלטלין את הטבל. אלא הכי קאמר: דאילו עבר ותקנו מתוקן אשתכח דקליש איסוריה ולא חשבינן ליה כמבטל כלי מהיכנו.\par \textbf{} ואם תאמר אם כן נמצאת מבטל ואין מבטלין כלי מהיכנו, שהרי אף בביצה אם עבר וסילק הביצה מתוכו חזר הכלי להתירו. ויש לומר דאינו דומה לטבל שאפשר שיפקיע איסורו ממש, מה שאין כן בביצה שאי אפשר להפקיע איסורו ולעולם הכלי בטל מהיכנו כל זמן שהביצה בתוכו.\par \textbf{} ואם תאמר לרב יוסף אתיא מתניתין דלא כרבי שמעון, דלדידיה מותר השמן שבנר לאחר שכבה מותר כדאיתא לקמן (שבת מד, א). יש לומר שאני התם דבעוד שהוא דלוק מיהא הויא איסורא דאורייתא, ואף על גב דשמן המטפטף גופיה דרבנן מכל מקום בדאורייתא החמירו.}
\textblock{\textbf{איתביה כל הני תיובתא ושני בצריך למקומו.} יש מפרשים דר׳ יצחק נמי אית ליה הא דרב יוסף דאמר אין מבטלין כלי מהיכנו, ודייקי לה מדאוקימנא בפרק משילין (ביצה לו, א) הא דתנן (שם לה, ב) נותנין כלי תחת הדלף אליבא דר׳ יצחק בדלף הראוי ולא מוקמינן לה בצריך למקומו, ועוד מדמוקי מתניתין (דלקמן שבת מז, ב) דנותנין כלי תחת הנר לקבל ניצוצות בצריך למקומו על כרחין מתניתין נמי דאין נותנין כלי תחת הנר לקבל את השמן אף היא אפילו צריך למקומו, דאי לא ליפלוג וליתני בדידה, ואפילו הכי תנן דאין נותנין, וטעמא לפי שאין מבטלין כלי מהיכנו. והא דר׳ יצחק עדיפא מדרב יוסף, דרב יוסף לית ליה אין כלי ניטל אלא לדבר הניטל, ור׳ יצחק סבירא ליה תרתי דאין מבטלין כלי מהיכנו ועוד שאין כלי ניטל אלא לדבר הניטל. והיינו דרב הונא, דאית ליה כדר׳ יצחק כדאמרינן בסמוך (בעמוד ב) זילו אמרו לרב יצחק כבר תרגמה רב הונא לשמעתיך בבבל, ופשיטא להו בפרק מי שהחשיך (לקמן שבת קנד, ב) דאית ליה אין מבטלין כלי מהיכנו. וכיון שכן כל הנך מתני׳ דמוקמינן להו בצריך למקומו, לאו למימרא דסגי ליה בהכין, דאכתי איצטריך לן למימר בהו כל הני פירוקין דפריק בהו רב יוסף משום דלא יבטל כלי מהיכנו. וכן כתב מורי הרב רבנו יונה ז״ל.}
\textblock{\textbf{תא שמע אבל כופה עליה כלי בשביל שלא תשבר, הכא נמי בצריך למקומו.} פירשו בתוס׳ שהמקשה היה רוצה לידע אם יצטרך להעמידה בצריך למקומו, דהא בהנהו דמייתי בסמוך לא בעי לשנויי הכין. ולי נראה דלישנא דברייתא קשיתיה, דמדקתני אין נוטלין אותה לכסות בה את הכלי אבל כופה עליה כלי משמע דהכי קאמר אף על פי שאין מטלטלין אותה מטלטלין את הכלי מחמתה, אלמא כלי ניטל לדבר שאינו ניטל, ומשני בצריך למקומו.}
\textblock{\textbf{ לא נצרכה אלא לאותן שתי חלות.} ואם תאמר והא תניא לקמן בפרק המצניע (שבת צה, א) הרודה חלות דבש בשבת בשוגג חייב חטאת דברי רבי אליעזר, ואפילו רבנן דפליגי עליה מודו דאסור משום שבות, וא״כ לכו״ע הוי דבר שאינו ניטל. י״ל דהכא בחלות רדויות ומונחות בכוורת.}
\textblock{\textbf{לעולם רבי יהודה ומאי שלא יתכוין וכו׳.} ואם תאמר אמאי דחק ומוקי לה כרבי יהודה, לוקמה כרבי שמעון ואע״ג דלא חשיב עלייהו, ואתיא ברייתא כפשטה. יש לומר דמהדר לאוקומה ככולי עלמא. אבל הר״ז ז״ל כתב דר׳ יצחק בשיטת רבי יהודה אמרה, דלדידיה דמחמיר במוקצה נחמיר ביה נמי שלא לטלטל אפילו כלי המוכן מחמתו, אבל לרבי שמעון דמיקל במוקצה נקל נמי לטלטל את הכלי אפילו לדבר שאינו ניטל לדידיה במוקצה מחמת חסרון כיס וביצה שנולדה בשבת וכיוצא בהן, ומשום הכי איצטרכינן לשנוייה אליבא דרבי יהודה. אבל מורי הרב ז״ל כתב דר׳ יצחק אפילו כרבי שמעון, ולמר כדאית ליה ולמר כדאית ליה. ובתוס׳ תירצו כעין התירוץ הראשון, ופירשו דהכי פירושא, אליבא דמאן איצטריכת לשנויי בשחשב עליהן על כרחין זה אינו אליבא דרבי שמעון דלדידיה לית ליה מוקצה אלא לרבי יהודה אימא סיפא וכו׳, וכן מוכח בביצה בריש פרק משילין דגרסינן התם עלה דהא (לו, א) במאי אוקימתא כרבי יהודה אימא סיפא, ומ״מ היה יכול לתרץ הדרי בי ואפילו לא חישב ור״ש היא.}
\textblock{\textbf{רב אשי אמר מי קתני בימות החמה וכו׳. } פירשו בתוס׳ דאף התירוץ הראשון השיבו רב אשי לרב עוקבא כדמוכח במסכת ביצה בפרק משילין (שם), דבהדיא גרסינן התם אמר ליה לא נצרכה אלא לאותן שתי חלות ואיבעית אימא מי קתני בימות החמה וכו׳.\par \textbf{} והא דרב אשי לא דייקא דאית ליה כר׳ יצחק, דלתרוצה אליביה קא אתא ודלמא לדידיה לא סבירא ליה, והלכך ליכא מיניה ראיה לפסוק כר׳ יצחק. ורבנו אלפסי ז״ל כתב בפרק מי שהחשיך (לקמן שבת קנד, ב) דליתא לדר׳ יצחק. וכן הסכים לדעתו כאן הר״ז הלוי ז״ל. אבל רבי מורי הרב ז״ל פסק כר׳ יצחק כמו שכתב בהלכותיו.}
\textblock{\textbf{כבר תרגמה רב הונא לשמעתיך בבבל.} איכא למידק והא רב הונא הוא דאמר בפרק מי שהחשיך (שם) היתה בהמתו טעונה כלי זכוכית מביא כרים וכסתות ומניח תחתיה, אלמא לית ליה לרב הונא הא דר׳ יצחק, אלא מטלטלין כלי לדבר שאינו ניטל. תירצו בתוס׳ דהתם הוא דוקא משום דלהפסד מרובה חששו, כדי שלא תתקלקל הבהמה ולא ישתברו הכלים. וכן כתב הר״ז הלוי ז״ל שם בפרק מי שהחשיך. ואף על גב דאוקימנא התם בשליפי זוטרי דאפשר לשומטן מתחתיהן ולא הוי מבטל כלי מהיכנו, אבל בשליפי רברבי לא משום דהוי מבטל כלי מהיכנו דאלמא להפסד מרובה נמי לא חששו, טלטול כלי לדבר שאינו ניטל אינו חמור כל כך והתירוהו במקום הפסד מרובה, אבל ביטול כלי מהיכנו חמור יותר ואפילו במקום הפסד מרובה, ואף על פי ששניהם מדבריהם. ולעולם רב הונא כר׳ יצחק סבירא ליה, וכדאמר רב ששת הכא דתרגמה דר׳ יצחק בבבל. וכזה כתב הראב״ד ז״ל בפרק מי שהחשיך, וגם הר״ז הלוי ז״ל כאן ובפרק מי שהחשיך.\par \textbf{} אבל הרמב״ן ז״ל הקשה דלדברי ר׳ יצחק אפילו משום הפסד מרובה לא התירו לטלטל אלא לדבר הניטל וכדמוכח בפרק משילין, דתנן התם (ביצה לה, ב) ומכסין את הפירות בכלים מפני הדלף וכן כדי יין וכן כדי שמן, ואתמר עלה (שם לו, א) אמר עולא אפילו אוירא דלבני ור׳ יצחק אמר פירות הראוין, ואזדא ר׳ יצחק לטעמיה וכו׳, ואוקמא עולא למתניתין בדטבלא, ואקשינן בשלמא למאן דאמר בדטבלא היינו דקתני כדי יין וכדי שמן אלא למאן דאמר פירות הראוין הא תנא ליה פירות, כדי יין וכדי שמן איצטריכא ליה מהו דתימא להפסד מרובה חששו להפסד מועט לא חששו קא משמע לן, והא פירות דהפסד מרובה הוא ואפילו הכי אין כלי ניטל אלא בראוין, ועוד דחבית ולבנים וכוורת דכל הני הפסד מרובה הוא, ואפילו הכי אקשינן מינייהו לר׳ יצחק ואיצטריך לתרוצינהו, שמע מינה דר׳ יצחק אפילו במקום הפסד מרובה אמרה לשמעתיה. ועוד דכיון דתרוייהו של דבריהם נינהו, מאי טעמא שרו משום הפסד מרובה לטלטל כלי ולא התירו לבטל כלי מהיכנו. אלא ודאי שמעינן מהתם דרב הונא לית ליה הא דר׳ יצחק, וכמו שכתב שם הרב אלפסי ז״ל, והכא דילמא בספרי דיוקני לא גרסי רב הונא.\par \textbf{} והראב״ד ז״ל פירש כאן, דהכא הכי קאמר, כבר תרגמה רב הונא לשמעתיך בבבל, שלא אמרו אין כלי ניטל אלא למעט צורך מת, הא במקום הצלה כולן ניטלין. וזה דחוק דאדרבה התירו יותר במת מבשאר הצלות, וכדאמרינן (לקמן שבת קמב, ב) לא אמרו ככר או תינוק אלא לגבי מת בלבד, ובסמוך לקמן (שבת מד, א) אמר רבי יוחנן הלכה כרבי יהודה בן לקיש במת, כלומר: אבל בממון לא.\par \textbf{} והרב אלפסי ז״ל לא כתב הא דרב ששת כלל, לומר דאתיא כר׳ יצחק ולא קיימא לן כוותיה. ואי קשיא לך לפסק הרב אלפסי זכרונו לברכה הא דתני שילא מרי. יש מי שפירש דההיא דוקא כשאפשר לעשותו בשביל חי וכדתני ואזיל, הא לא אפשר עושין אפילו בשביל מת ממש.}
\textblock{\textbf{ומכל מקום למדנו מדבריהם, דכל היכא דאפשר בלאו הכי אין מטלטלין כלי לדבר שאינו ניטל. ואפשר      } דשמעתין מוכחת דלכולי עלמא אין כלי ניטל שלא לצורך, דהא לרבה לא התירו לטלטל להצלה שאינה מצויה ואפילו בדברים הניטלין, ולר׳ יצחק נמי דוקא בצריך למקומו הא בלאו הכין לא, ולא מצינו מי שהתיר לטלטל להצלה שאינה מצויה ואפילו דברים הניטלים.\par \textbf{} ומיהו יש בדבר לדון ולהקל, דמה שלא התירו כאן לטלטל אפילו לצורך דבר הניטל כגון מי שנשברה לו חבית בראש גגו אלא דוקא בהצלה מצויה, היינו דוקא במקום הצלה, שבהצלות החמירו כדי שלא יבא לטלטל ולהביא אפילו מרשות הרבים כדי להציל מתוך שהוא בהול על ממונו, וכדאמרינן בפרק כל כתבי הקדש (לקמן שבת קיז, ב), מכדי בהתירא קא טרח ליציל טפי, אמר רבה מתוך שהוא בהול על ממונו אי שרית ליה אתי לכבויי, אמר ליה אביי אלא הא דתניא נשברה לו חבית בראש גגו מביא כלי ומניח תחתיה ובלבד שלא יביא כלי אחר ויקלוט כלי אחר ויצרף התם מאי גזירה איכא, התם נמי איכא גזירה שמא יביא כלי דרך רשות הרבים. ומכל מקום המחמיר תבוא עליו ברכה.}
\textblock{\textbf{מביאין מחצלת ופורשין עליהן.} ואע״ג דקיימא לן דאין עושין אהל עראי בתחלה ביום טוב וכל שכן בשבת, כדאיתא לקמן בפרק תולין (שבת קלז, ב) ובעירובין פרק מי שהוציאוהו (עירובין מד, א). הני מילי כדרך שעושין אהלים ממטה למעלה אבל מלמעלה למטה כדהכא שרי, וכההיא נמי דפרק המביא כדי יין (ביצה לב, ב. לג, א) דאמר רב יהודה האי מדורתא מלמעלה למטה שרי ממטה למעלה אסור וכן ביעתא וכן פוריא וכן חביתא.}
\textblock{\textbf{זה זוקף מטתו.} איכא למידק, דהכא לא שריא אלא בשביל חי, ואילו במסכת עירובין פרק מי שהוציאוהו (שם) תניא זוקף את המטה ופורס עליה סדין שלא תפול החמה על המת ועל האוכלין. ויש לומר דעל המת ועל האוכלין דוקא קאמר כלומר: כשיש שם אוכלין, אבל על המת בלבד לא. והא דתניא זוקף את המטה ופורס עליה סדין דאפילו מלמטה למעלה, הא אוקמינן לה התם בדופן רביעי שאינו אלא תוספת בעלמא.}
\textblock{\textbf{מר סבר טלטול מן הצד שמיה טלטול.} איכא למידק היכי אמר רב הכא דשמיה טלטול, והא אמרינן לקמן בפרק תולין בסופו (שבת קמא, א) אמרי בי רב טלטול מן הצד לא שמיה טלטול, והא פרכינן לקמן מדאמרי בי רב לרב, ובפרק השואל (ב״מ קב, ב) ובפרק בית כור (ב״ב קה, ב) משמע [דרב] ואמרי בי רב לא פליגי. ועוד מי נימא פליגא דרב אמתניתין דתנן בפרק נוטל (לקמן שבת קמב, ב) מעות שעל הכר נוער את הכר והן נופלות מאליהן, ותנן (שם) אבן שעל פי החבית מטה על צדה והיא נופלת מאליה. ואע״ג דהתם בשוכח אבל במניח נעשה בסיס לדבר האסור (כדאיתא בגמרא שם), הכא אי אפשר דמיירי במניח דאם כן מאי טעמא דמאן דשרי. ואפילו מת מבערב והוא מונח שם והולך אפילו הכי לא נעשית המטה בסיס לו, לפי שאין המטה צריכה למת אלא אדרבא צריך הוא להטילו על גבי קרקע כדי שימתין, וכדתנן (לקמן שבת קנא, א) שומטין את הכר מתחתיו ומטילין אותו על החול כדי שימתין. וקיימא לן כרבי אלעזר בן תדאי דאמר פגה שטמנה בתבן וחררה שטמנה בגחלים תוחב להן כוש או כרכר והן ננערות מאליהן, ואמר רב נחמן הלכה כרבי אלעזר בן תדאי כדאמר בריש פרק כל הכלים (לקמן שבת קכג, א), ועד כאן לא פליגי רבנן עליה אלא כשאין מקצתן מגולין אבל כשמקצתן מגולין מודו, אם כן רב דאמר כמאן.\par \textbf{} ותירצו בתוס׳ דלא אמרו טלטול מן הצד לא שמיה טלטול אלא כשהוא צריך לטלטל האיסור מחמת דבר המותר, כפגה שטמנה בתבן וכאבן שעל פי החבית ומעות שעל הכר וכיוצא באלו שאינו צריך לטלטל את האיסור אלא מחמת שצריך למטה אבל הכא שהוא צריך לטלטל את המת מחמת עצמו של מת שלא יסריח בחמה ושלא ישרף בדליקה אסור, אלא דשמואל שרי דקסבר דכל טלטול מן הצד שרי. וקיימא לן כרב דאסר. ומכל מקום למדנו מדבריהם שאם היה צריך לאותה מטה שהוא מוטל עליה מותר ואפילו לרב.\par \textbf{} אבל הרב אלפסי ז״ל כתב דכי אמרינן טלטול מן הצד שמיה טלטול הני מילי כגון טלטול אבנים ומת שהוא כאבן וכיוצא בדברים אלו, אבל טלטול אוכלין המעורבין עם דברים אסורין והוא צריך לאוכלן בשבת כגון פגה שטמנה בתבן ופוגלא שטמנה בעפר וכיוצא בהן בכי הא לא שמיה טלטול. ולא ירדתי לסוף דעתו, שהרי שנינו (לקמן שבת קמא, א) הקש שעל גבי המטה לא ינענענו בידו אבל מנענעו בגופו, והקש הרי כאבן הוא ולמה התירו בו טלטול מן הצד. ודברי התוס׳ נראין בעיני עיקר. ואולי אף הרב אלפסי ז״ל לזה נתכוין לומר, דלא אסרו אלא טלטול אבנים ומת וכיוצא בהם בזמן שהן מטלטלין לצורך עצמן, אבל טלטול דבר האסור מחמת דבר המותר כפגה ופוגלא וכיוצא בהן מותר.}
\clearpage
\newsection{דף מד}
\textblock{\textbf{כי שרי רבי שמעון בנר זוטרא דדעתיה עלויה אבל הני דנפישי לא.} ואם תאמר נר נמי שמן שבו מיהא ליתסר, דהא לענין שמן הרי הוא כגדול שאין דעתו עליו, שדרכו דולק והולך עד שיכלה שמן שבו. יש לומר כיון דיהיב דעתיה על הנר עצמו ומצפה אימתי תכבה נרו אף על המותר שבשמן נותן דעתו, אבל בגדולים לגמרי מסלק דעתו מהן.}
\textblock{\textbf{אמר ר׳ זירא פמוט שהדליקו עליו לדברי המתיר אסור לטלטלו.} פירש רש״י ז״ל: פמוט מנורה קטנה ולדברי המתיר דהיינו רבי מאיר אסור לטלטלו, אבל לרבי שמעון ודאי מותר דאפילו בנר של חרס שהדליק בו באותה שבת התיר. ואינו מחוור, מדאמרינן בלישנא בתרא פמוט שהדליקו עליו דברי הכל אסור, ומדאמרינן לדברי הכל משמע דאף רבי שמעון בכלל. אלא נראה כדברי ר״ח ז״ל שפירש פמוט מנורה גדולה ולדברי המתיר היינו בין רבי מאיר בין רבי שמעון.}
\textblock{\textbf{ומה נר דלהכי עבידא כי לא אדליק בה שרי לטלטלה.} פירש רש״י ז״ל: מדקתני אבל לא ישן ולא קתני אבל לא את המיוחד לכך אלמא לא מיתסר בהזמנה אלא בהדלקה, ובלישנא בתרא דאוקימנא בשהניח עליו מעות, פירש ולא דמי לנר של מתכת שאם לא הדליקו בו באותו שבת מותר וכדתניא כל הנרות של מתכת מטלטלין אותן חוץ מן הנר שהדליקו בו באותה שבת משום דההוא בשלא יחדו. ואיכא דקשיא ליה אם כן מאי קא דייק בלישנא קמא מדקתני אבל לא ישן דדוקא ישן אבל בהזמנה לא, דדלמא הוא הדין להזמנה והכא רבותא קאמר אבל לא ישן אף על פי שלא הזמינו משום דהוה ליה מוקצה מחמת מיאוס מה שאין כן בשל מתכת.\par \textbf{} על כן פירשו בתוס׳ דנר של מתכת אף על פי שיחדו לכך מותר לטלטלו אם לא הדליקו עליו באותה שבת, מפני שאין מדליקין בו אלא לפעמים, והלכך דוקא הדליקו בו באותה שבת הא לא הדליקו בו באותה השבת שרי שהרי לא נעשה בשבת זו בסיס לדבר איסור, ולא דמיא למטה שיחדה למעות מפני שהיא עשויה להניח עליה מעות בכל שעה ולפיכך הרי היא כאילו הניחו שם מעות בשבת, והוא הדין לנר של חרס שאף הוא מדליקין בו תמיד ומשום הכי ישן אסור ולעולם הרי הוא כבסיס לדבר האיסור.\par \textbf{} ואם תאמר אם כן מאי טעמא אסרינן לנר של חרס ישן משום מוקצה מחמת מיאוס, אפילו בלא מיאוס נמי ליתסר כיון דמיוחד הוא לאיסור דומיא דמטה. נראה לי משום דעיקרא דמילתא משום מיאוסו הוא, דמשום שהוא נמאס בהדלקתו ואינו ראוי למלאכה אחרת אדם מדליק בו בכל שעה, אבל של מתכת שאינו מאוס אדם חס עליו ואינו מדליק בו בכל שעה.\par \textbf{} ומיהו לענין עיקר קושיא הראשונה נראה לומר דאינה קושיא, דהא מדקתני מטלטלין נר חדש אבל לא ישן ולא קתני אין מטלטלין נר ישן אבל מטלטלין את החדש נראה דעיקר משום שריותא דנר חדש הוא דאתא, וכיון דלא תנא איסורא אלא בישן שמע מינה דכל חדש מותר ואפילו מיוחד לכך מדלא קתני אבל לא את המיוחד לכך, כדברי רש״י ז״ל.\par \textbf{} ומיהו בנר של מתכת נראין דברי התוספות דאפילו יחדו והדליק בו לשבת אחרת מותר, שהרי פמוט ודאי לכך הוא עשוי וכיון שהדליק עליו לכך הוא מיחדו, ואפילו הכי אם לא הדליקו עליו באותה שבת לכולי עלמא שרי.}
\textblock{\textbf{אלא אי איתמר הכי איתמר אמר רב יהודה אמר רב מטה שיחדה למעות.} קשיא לי דהוה ליה למימר הכי איתמר אמר רב הונא מטה שיחדה למעות וכו׳, דהא מימרא דאתינן עלה רב הונא הוא דאמרה ולא רב יהודה וכדי אמרה ולא משמיה דרב. ונראה לי דהא דרב יהודה אמר רב ידיעא להו דאיתמר הכין, ומשום דרב הונא ורב יהודה תרווייהו תלמידי דרב הוו אמרינן דהא דרב הונא דרב היא, והכי קאמרינן אלא אי איתמר דרב הונא הכי איתמר כדאמר רב יהודה משמיה דרב ותרווייהו מיניה דרב שמיעא להו.}
\textblock{\textbf{מוכני שלה בזמן שהיא נשמטת אין חבור לה וכו׳.} פירש רש״י ז״ל: דגבי שידה תנן לה במסכת כלים (פי״ח מ״ב) והיא עגלה של עץ מוקפת מחיצות למרכב אנשים, ופי׳ מוכני אופן, ופי׳ אינה מצלת עמה באהל המת כגון שגובה המוכני מגין על הכלים שעל דפני השידה ואין מצלת מפני שהמוכני כלי לעצמו ומקבל טומאה. והקשו עליו בתוס׳ טובא, חדא שאם מוכני הוא האופן לא היה לו לומר מוכני בלשון יחיד אלא בלשון רבים, שאין עגלה הולכת באופן אחד. ועוד שאין האופן עשוי להניח עליו מעות. ועוד כי מה שפירש שאם היו כלים על דפני השידה אין האופן מגין עליהם, אינו מחוור, שאע״פ שהמוכני כלי לעצמו הוה ליה פשוטי כלי עץ ואינו מקבל טומאה וכל שאינו מקבל טומאה חוצץ בפני הטומאה. ועוד כי מה שפירש גם כן שהשידה עשויה למרכב אנשים, אינו מחוור, דאם כן למה נמדדת שהרי כל הראוי למדרס טמא אף על פי שהוא מחזיק יותר מארבעים סאה, דלא גמרינן משק בכלים העשוין למדרסאות וכדאיתא בבכורות בפרק אלו מומין (לח, א), וכיון דמטמא מדרס מטמא טומאת מת. ועוד שכל שהוא מקבל שום טומאה אינו מציל באהל המת.\par \textbf{} אלא הפירוש המחוור כמו שפירשו ר״ח ור״ת ז״ל, שהמוכני הוא בסיס שהשידה יושבת עליו מלשון את כנו (עיין שמות ל, כח) כן של זהב (עיין יומא לז, ב), והשידה הזו עשויה לשמור בתוכה כלים או פירות ומכוסה היא מלמעלה, ולפיכך בזמן שהיא גדולה אינה מקבלת טומאה שהוקש לשק, ואם השידה נקובה כנגד דפני המוכני בזמן שאינה נשמטת הוה ליה כלי אחד ואין המוכני מקבל טומאה ולפיכך חוצץ בפני הטומאה, אבל בזמן שהיא נשמטת הוה ליה כלי בפני עצמו ומטלטל הוא מלא וריקן ולפיכך אינו חוצץ בפני הטומאה.}
\textblock{\textbf{ואין גוררין אותה [בשבת] בזמן שיש עליה מעות.} ואם תאמר בזמן שיש עליה מעות אף על פי שאינה נשמטת ממנו מפני מה גוררין אותה. תירץ רבנו אפרים ז״ל בשיש בשידה פירות או כלים, והלכך בשאינה נשמטת הוה ליה הכל כלי אחד ונעשה בסיס לדבר האסור ולדבר המותר ומותר לגוררה, וכדתנן (לקמן שבת קמא, ב) נוטל אדם כלכלה מלאה פירות והאבן בתוכה ובתוס׳ פירשו דבזמן שהיא נשמטת אינו בטל לגבי השידה, וכיון שנעשה בסיס לדבר האסור אסור לגוררה, אבל בזמן שאינה נשמטת הרי הוא בטל לגבי השידה, ואף על פי שנעשה הוא בסיס לדבר האסור לא חשבינן השידה לבסיס לו מפני שהמוכני טפלה לשידה.}
\textblock{\textbf{הא אין עליה מעות שריא אע״ג דהוו עליה כל בין השמשות.} ואם תאמר מאי קושיא, דלמא בשוכח, דכל זמן שהוא עליה אסור לטלטלה טלטול גמור אבל אם נטלו מותר לטלטלה כדאיתא בפרק נוטל (לקמן שבת קמב, ב) גבי מעות שעל הכר. תירצו בתוס׳ דאין גוררין אותה בזמן שיש עליה מעות משמע להו דאין גוררין אותה כלל ואפילו לצורך גופה ולצורך מקומה, והלכך על כרחין לא מיירי בשוכח אלא במניח, דאי בשוכח אפילו בעודן עליו מותר לצורך מקומו, כדאמרינן התם לא שנו אלא לצורך גופו אבל לצורך מקומו מטלטלן ועודן עליו.}
\textblock{\textbf{הא מני רבי שמעון היא.} ואם תאמר והא מודה רבי שמעון בכוס וקערה ועששית. ויש לומר דהכא נמי דעתו עליה ומצפה אימתי יפלו או ינטלו המעות מעליה, והוה ליה דומיא דנר קטן.}
\clearpage
\newsection{דף מה}
\textblock{\textbf{הכי נמי מסתברא דרב כרבי יהודה סבירא ליה.} תימא למה ליה למידק מדוכתא אחרינא מהא דמטה גופה שמעינן לה, מדקאמר יחדה והניח עליה מעות אסור לטלטלה ואילו לרבי שמעון אפילו הניח עליה מעות שריא, דומיא דנר שלא הדליקו עליה באותה שבת (לעיל שבת מד, א), ודומיא דחצוצרות דשרי רבי שמעון ואפילו תקע בה (כדאיתא לעיל שבת לו, א). יש לומר דמטה שאני דאיכא למימר דילמא אפילו רבי שמעון מודה בה, דכיון שהוא מיחדה למעותיו אדם קובע לה מקום ומקצה אותה לגמרי מדעתו מחמת חסרון כיס. והרמב״ן ז״ל תירץ דמשום דהך מימרא תירוצא הוא דקאמרינן אלא אי איתמר הכי איתמר בעינן לאתויי ראיה ממקום אחר. ואינו מחוור בעיני כלל, דעל כרחין ההיא ודאי רב יהודה בלשונה אמרה משמיה דרב דהוא רביה דרב הונא נמי ומיניה שמיעא ליה, וכמו שכתבתי למעלה (מד, ב ד״ה אלא).}
\textblock{\textbf{כי לית ליה לרבי שמעון מוקצה היכא דלא דחייה בידים היכא דדחייה בידים אית ליה.} ואם תאמר הא נר הא דחייה בידים ואפילו הכי שרי רבי שמעון. יש לומר דשאני הכא דאכתי קאי בדחויו והוה ליה ככוס וקערה. ואם תאמר אם כן לידוק מכוס וקערה דאסירי. יש לומר שאני התם דמקצה ליה מדעתיה לגמרי דלא מסיק אדעתיה דכבה וכל היכא דדולק והולך אי אפשר לטלטלה, אבל הכא דאפשר לטלטלה אע״ג דקאי בדחויו קמא דלמא רבי שמעון משרא שרי.}
\textblock{\textbf{אין מוקצה לרבי שמעון אלא שמן שבנר בשעה שהוא דולק.} פירוש: שמן המטפטף. ותדע לך מדקא סלקא דעתך השתא דהוקצה למצותו והוקצה לאיסורו דוקא בעינן, ואילו שמן שבנר ממש אפילו ליכא משום מצוה על כרחין אסר רבי שמעון, דהא תניא (ביצה כב, א) המסתפק ממנו חייב משום מכבה, ועוד דכל היכא דהוא דולק הרי הוא בסיס לדבר האסור, וכדאמרינן בשלהי פרקין (לקמן שבת מז, א) הנח לנר שמן ופתילה הואיל ונעשו בסיס לדבר האסור.}
\textblock{\textbf{והא ממאי דרבי שמעון היא דתני ר׳ חייא בר יוסף וכו׳.} ואם תאמר כיון דאי לא הא מאידך ברייתא דעטרה בקרמין לא מצי למידק, למה ליה לאתויי, לידוק מהא בלחוד דתני ר׳ חייא בר יוסף. יש לומר משום דאמרינן (חולין קמא, א-ב) כל ברייתא דלא מיתניא בי רבי חייא ורבי הושעיא לאו ברייתא היא איצטריכא ליה לאתויי ההיא ברייתא דסככה כהלכתה דמיתניא סתם דמיתניא בי רבי חייא ורבי הושעיא. ואם תאמר היכי מצי לאוקומה לההיא כרבי שמעון, והא מדקתני אסור להסתפק עד מוצאי יום טוב האחרון אלמא אית ליה להאי תנא מגו דאיתקצאי לבין השמשות איתקצאי לכולי יומא, ואילו לרבי שמעון לית ליה, דהא נר שכבה לדידיה מותר ומוכני נמי היכא דלית ביה מעות אע״ג דהוה עליה בין השמשות. תירץ ר״ת ז״ל דהא דאוקימנא לה כרבי שמעון לאו כולה ברייתא מוקמינן לה כוותיה אלא מאי דקתני דאסור להסתפק ממנו, ואתא כולה כרבי שמעון לבר מהא דקתני עד מוצאי יום טוב האחרון.}
\textblock{\textbf{ואם התנה עליהן הכל לפי תנאו.} הקשו בתוס׳ ז״ל (לעיל שבת מד, א ד״ה שבנר) מאי שנא נר שכבה דאסור לרבי יהודה מסוכה רעועה דמועיל בה תנאי הואיל ודעתו עלויה מאתמול (ביצה ל, ב), ולרבי שמעון נמי יועיל לשמן המטפטף בשעה שהוא דולק. והם ז״ל דחקו בתירוצין. אבל הרמב״ן ז״ל כתב דבירושלמי (דפרקין ה״ז עי״ש) הביאו ברייתא דמפרשת בהדיא דתנאי מועיל בנר, דגרסינן התם: תני אם התנה עליו מותר, מה אנן קיימין אין כרבי מאיר אפילו התנה יהא אסור ואין כרבי שמעון אפילו לא התנה יהא מותר אלא כרבי יהודה נר מאוס הוא, הוי מאן דתנא אם התנה עליו יהא מותר רבי שמעון דתני אבל כוס וקערה ועששית אף על פי שכבו אסור ליגע בהן רבי טבי בשם רב חסדא אפילו רבי שמעון דו אמר תמן מותר מודה הוא הכא שהוא אסור וכו׳. ופירש הוא ז״ל שהם הקשו שם בירושלמי על ברייתא זו דתני בנר שבת אם התנה עליו מותר מני אי רבי מאיר אין תנאי מועיל בו לפי שהם סבורים שם בירושלמי לרבי מאיר דכל המיוחד לאיסור אסור, ואי רבי יהודה כל שכן שאין אדם יכול להתנות על הדבר המאוס שלא יהא מיאוסו מקצהו, ובסוף העמידוהו כרבי שמעון בכוס וקערה ועששית דמודה בהו רבי שמעון דאסירי ואם התנה עליהן מותר, והוא הדין בנר של מתכת      מאיר ולרבי יהודה דמהני בהו תנאה דומיא דסוכה. עד כאן.}
\textblock{\textbf{לעולם רבי יהודה היא ואוכל אצטריכא ליה וכו׳.} תמיהא לי אמאי דחיק ומוקי לה הכין, לוקמה כרבי מאיר דלית ליה אלא מוקצה שדחאו בידים. ואפשר דמשום דליתא לדרבי מאיר אלא או כרבי יהודה או כרבי שמעון מהדר תלמודא לאוקומה כחד מינייהו, וכענין שאמרו בריש פרק קמא דגיטין (ד, א) מעיקרא מאי טעמא לא מוקמינן כרבי יהודה, מהדרינן אדרבי מאיר דסתם מתניתין רבי מאיר, מהדרינן אדרבי אלעזר דהלכתא כוותיה בגיטין.}
\textblock{\textbf{ואי בעית אימא לדבריו דרבי שמעון קאמר ליה וליה לא סבירא ליה.} פירוש: מהא לא תשמע מינה כלום, ואלא מיהו רבי כרבי שמעון סבירא ליה וכדאמר רבי יהושע בן לוי בסמוך (מו, א) פעם אחת הלך רבי לדיוספרא והורה במנורה כרבי שמעון בנר.}
\textblock{\textbf{והאמר רב נחמן מאן דאית ליה מוקצה אית ליה נולד.} איכא למידק והא רבי יוחנן הוא דלית ליה כרב נחמן בההיא אוקמתא בריש פרק קמא דביצה (ב, א), דאיהו סבר (שם ג, א) דטעמא דביצה שנולדה ביום טוב משום גזירת משקין שזבו. ויש לומר שאלו מן הדברים שנאמרו בגמרא לרווחא דמלתא לתרוצי אליבא דכולי עלמא.}
\textblock{\textbf{ורבי יוחנן אמר אנו אין לנו אלא בנר כרבי שמעון.} פירוש: אין לנו שרבי שמעון מתיר אלא בנר וכטעמא דאמרן לפי שהוא מצפה אימתי תכבה נרו, אבל במנורה מודה רבי שמעון וכטעמא דמפרש בסמוך ואזיל, ומיהו רבי יוחנן לית ליה אפילו בנר כרבי שמעון.}
\clearpage
\newsection{דף מו}
\textblock{\textbf{הלכך דחוליות בין גדולה בין קטנה אסורה.} איכא למידק דבפרק שני דביצה (כא, ב) תנן שלשה דברים רבן גמליאל מחמיר כדברי בית שמאי וקא חשיב חדא מינייהו אין זוקפין את המנורה ביום טוב, ואוקימנא בגמרא (שם כב, א) במנורה של חוליות משום דמיחזי כבונה בית שמאי סברי יש בנין בכלים ויש סתירה בכלים ובית הלל סברי אין בנין בכלים ואין סתירה בכלים, אלמא שרי אפילו לזקוף מנורה של חוליות וכל שכן דשרי לטלטל. ויש לומר דההיא במנורה רפויה, וכדברי רבן שמעון בן גמליאל דאמר בסוף פירקין (לקמן שבת מז, ב) אם היה רפוי מותר גבי מטה של חוליות. ובתוס׳ הקשו לתירוץ זה דאם כן תקשי למאן דאסר לקמן ברפוי. ולדידי נמי קשיא לי דאי ההיא ברפויה תקשי לן, דהא בהדיא רבן גמליאל מחמיר כדברי בית שמאי, ואילו במטה רפויה מתיר רבן שמעון בן גמליאל אפילו לכתחלה, ומי נימא דמיקל רבן שמעון בן גמליאל במה שהיו בית אביו מחמירין. אלא שבזו יש לי לומר דהכא לאחרים והתם לעצמן, וכמו שאמרו לו שם (ביצה כא, ב) ומה נעשה לבית אביך שהן מחמירין על עצמן ומקילין אצל אחרים. ותירצו הם ז״ל דמנורה דהתם מיירי במנורה של חוליות מחוברין יחד ואינן מפורקין ופעמים מטין אותן ופעמים זוקפין אותם, והא דהכא בשל חוליות מפורקין.}
\textblock{\textbf{אבל הרב אלפסי ז״ל לא הביא בהלכותיו כלל הא דרבי יוחנן דהכא, לומר שאין הלכה כן, משום דמשמע ליה דרבי יוחנן אתיא כמאן דאמר יש בנין בכלים, והלכך במחזיר כל צרכו ואפילו שלא תקע חייב חטאת ומשום הכי קא אסר הכא לטלטלה משום דמיחזי כבונה, וקסבר רבי יוחנן דההיא דשרו בית הלל התם ברפויה ובשאינו מחזיר כל צרכו, ומשום שמחת יום טוב התירו אפילו לזוקפה והוא שלא החזיר כל צרכו אבל בשבת אסור לטלטל, אבל אנן דקיימא לן דאין בנין בכלים אלא אם כן תוקע וכדאמרינן בריש פרק כל הכלים (לקמן שבת קכב, ב) גבי דלת של שידה תיבה ומגדל שרינן לטלטל אפילו בשבת, ואי רפויה שריא אפילו להחזיר וכדאמרינן לקמן (שבת מז, ב) גבי מטה גללניתא. ומכל מקום אינו מחוור, דהא משמע דמסקנא דגמרא כי הא דרבי יוחנן, מדקא מתמהינן אי הכי מאי טעמא דריש לקיש, ועוד מדמסקינן       } וכיילינן הלכך דחוליות בין גדולה בין קטנה אסורה, גדולה נמי ואית בה חידקי אסורה גזירה אטו גדולה של חוליות. ובפרק קמא דביצה (י, א ד״ה הכי) כתבתי יותר בסייעתא דשמיא בשמעתא דמוחלפת השיטה.}
\textblock{ הא דאקשינן הכא:\textbf{ מי אמר רבי יוחנן הכי והאמר רבי יוחנן הלכה כסתם משנה ותנן (כלים פי״ח מ״ב) מוכני שלה וכו׳.} ואיצטריך רבי זירא לתרץ תהא משנתינו בשלא היו עליה מעות כל בין השמשות שלא לשבור דבריו של רבי יוחנן. תמיהא לי דהא אדרבה אקשינן בשלהי מי שהחשיך (לקמן שבת קנו, ב. קנז, א) ומי אמר רבי יוחנן הלכה כרבי שמעון והאמר רבי יוחנן הלכה כסתם משנה ותנן (ביצה לא, א) אין מבקעין עצים מן הקורה ולא מן הקורה שנשברה ביום טוב, ואקשינן עליה נמי מן סתמות אחרות דתנן (שם כט, ב. ל, א) מתחילין בערימת התבן אבל לא בעצים שבמוקצה, ותנן (שם מ, א) אין משקין ואין שוחטין את המדבריות אבל משקין ושוחטין את הבייתות, ואצטריך לשנויי רבי יוחנן סתמא אחרינא אשכח דתנן (לקמן שבת קמג, א) מגביהין מעל השלחן עצמות וקליפין וכו׳. וכיון שכן מאי קא מקשה הכא ממוכני שלה דהא איכא סתמי אחריני טובא כרבי יהודה, ומאי קא מתרץ לה נמי שלא לשבור דבריו של רבי יוחנן דהא איכא אחרינא דתקשי ליה. ועוד קשיא לי אמאי לא מייתי התם הא דמוכני.\par \textbf{} ויש לומר דהכא במוקצה מחמת איסור בלבד הוא דקאמר רבי יוחנן הכא דהלכה כרבי יהודה, אבל בשאר מוקצה כרבי שמעון סבירא ליה וכדאמרינן התם, והלכך לא מצי לאותביה מהנך סתמי דמייתינן התם דמוקצה מחמת איסור חמיר טפי. והתם נמי לא אותבינן עליה הא דמוכח מהאי טעמא גופא. והכא נמי לא ניחא ליה לשנויי רבי יוחנן סתמא אחרינא אשכח, דתנן אין משקין ושוחטין את המדבריות וכל שכן הכא דמוקצה מחמת איסור, שלא לשבור נמי דברי רבי יוחנן דפסק בשאר מוקצה כרבי שמעון.\par \textbf{} וכן הלכה דבשאר מוקצה הלכה כרבי שמעון, ובמוקצה מחמת איסור בלחוד הלכה כרבי יהודה, וכדאיתא התם בשלהי פרק מי שהחשיך.}
\textblock{\textbf{שרגא דמשחא שרי לטלטולה דנפטא אסור.} פירש רש״י ז״ל: דנפטא אסור אפילו לרבי שמעון. וכן נראה ודאי דסתם שרגא דחרס משמע, וכדתניא (לעיל שבת מד, א) מטלטלין נר חדש אבל לא ישן [דברי רבי יהודה] ותניא (שם עי״ש) רבי יהודה אומר כל הנרות של (חרס) [מתכת] מטלטלין חוץ מן הנר שהדליקו עליו באותה שבת, אלמא סתם נר היינו של חרס עד שיפרוט של מתכת, והלכך אי כרבי יהודה אפילו דמשחא נמי אסיר.\par \textbf{} ואי קשיא לך נמי דהא רבה כרבי יהודה סבירא ליה, דהא אמר אביי (לעיל שבת כב, א) כל מילי דמר עביד כרב ורב כרבי יהודה סבירא ליה (לעיל שבת מד, ב). יש לומר התם לאפוקי מדשמואל, אבל הכא תליא באשלי רברבי דרבי יהושע בן לוי פסק כרבי שמעון (לעיל שבת מה, ב).}
\textblock{\textbf{מי יימר דנפיל ביה מומא ואם תמצי לומר נפיל ביה מומא מי יימר וכו׳.} כל הני מי יימר לאו דוקא, אלא לרווחא דמלתא נקט להו. ותדע לך, דהא בההיא דקא מייתי מפירין נדרים ליכא אלא חד מי יימר דמיזדקק לה בעל, ובנשאלין לנדרים מי יימר דמיזדקק ליה חכם.}
\textblock{\textbf{מתיב רמי בר חמא מפירין בשבת לצורך השבת ואמאי לימא מי יימר דמיזדקק לה בעל.} קשיא לי הכא מאי מוקצה איכא, ואע״ג דלא הוי חזי לדידה משום נדרה הא הוי חזי לאחריני וכל דחזי לאחריני לאו מוקצה הוא, וכדתנן לקמן בפרק מפנין (שבת קכו, ב) מפנין תרומה ואמרינן עלה בגמרא (שם קכז, ב) פשיטא, ופרקינן לא צריכא בתרומה ביד ישראל, מהו דתימא כיון דלא חזיא ליה אסור קא משמע לן כיון דחזי לכהן שפיר דמי. ויש לומר דהכא נמי בטלטול ודאי שרי, דהא לא אקצייה מדעתיה לגמרי ואפילו לטלטול כיון דחזי לאחריני, אלא מאכילה בלחוד הוא דאקצייה לדידיה, ומאכילה קא מקשה ליה דומיא דפירות שאין במינן במחובר שהבא בשביל ישראל זה מותר לישראל אחר ומי שבאו בשבילו מותר לטלטלן (ביצה כד, ב. כה, א). ואם תאמר והא כיון דמקצה ליה מאכילה אף בטלטול אסור, דמאי שנא מגרוגרות וצימוקין דמשום דלא חזו ליה לאכילה ומקצה להו מאכילה חשבינן להו מוקצה גמור ואסור לטלטולי להו. יש לומר דשאני התם דלא חזו לאכילה לא לדידיה ולא לאחרים. ואם תאמר והא משמע לקמן בסמוך (שבת מז, א) דכל מידי דלא חזי למריה אע״ג דחזי לאחריני חשבינן ליה מוקצה גמור ואסור לטלטלו ואפילו לצורך      דחזי להו, וכדאמרינן קורטין בי רבי מי חשיבי ותניא בגדי עניים לעניים בגדי עשירים לעשירים. יש לומר דשאני התם דכיון דגבי עשירים לא חזי למידי מצד גריעותיה אקצויי מקצה ליה לגמרי ולא מסקי אדעתיהו ליתן לעניים דחזי להו, אבל הכא דאפילו לדידיה חזי ליה אלא דאריה דאיסורא רכיב עליה וחזי לאחריני לא מקצה ליה לגמרי.}
\textblock{\textbf{כל היכא דקא מכוין איכא איסורא דאורייתא כי לא מכוין גזר רבי שמעון.} איכא דקשיא ליה והא הכא אפילו מכוין לכבות ליכא איסורא דאורייתא לרבי שמעון, דהא מלאכה שאינה צריכה לגופה היא. ויש לומר דמשכחת לה בפתילה שצריך להבהבה. ואי נמי בכל פתילה, דכיון דאיכא דוכתא דמחייב עליה מדאורייתא גזר אפילו שלא במתכוין ואפילו בדוכתא דלא אתי בה לידי איסורא דאורייתא.}
\clearpage
\newsection{דף מז}
\textblock{\textbf{מחתה באפרה.} ואם תאמר והא אפר כירה מוכן הוא. יש לומר הכא בשאין צריכין לו ואין דעתו עליה, וכמו שפירש רש״י ז״ל.}
\textblock{\textbf{בגדי עניים לעניים [בגדי עשירים לעשירים].} פירוש: שלשה על שלשה דהיינו בגדי עשירים, מטמאין טומאת מדרס לעשירים וכל שכן לעניים. אבל בגדי עניים דהיינו פחות משלשה על שלשה, לעניים הוא דמטמא טומאת מדרס אבל לעשירים לא מטמא טומאת מדרס. אבל מכל מקום מטמא הוא לשאר טומאות אפילו לעשירים, דלא אימעוט אלא מטומאת מדרס בלבד. ותדע לך, דהא תנן הבגד מטמא משום שלשה על שלשה למדרס ומשום שלש על שלש לטמא מת ואיתא נמי בפרק כ״ז מכלים (מ״ב), ועל כרחין היינו לעשירים דאילו לעניים אפילו בפחות מכן מטמא טומאת מדרס כדתנן התם בפרק כ״ח (מ״ח) בגדי עניים אע״פ שאין בהם שלשה על שלשה טמאים טומאת מדרס, אלמא לטמא מת אפילו לעשירים מטמא בשלש על שלש. וטעמא דמדרסאות משום דמדרס ביחוד תליא מילתא ואין עשירים מיחדין בפחות משלשה על שלשה לישיבה, אבל עניים מיחדין אותה ועשיר שיחדה בטלה דעתו אצל עשירים, אבל שלש על שלש לשאר טומאות טמא דהא חזי. והיינו דבפרק במה מדליקין (לעיל שבת כו, ב) לא מפלגינן בשלש על שלש בין לעניים בין לעשירים.\par \textbf{} ויש מפרשים: בגדי עניים בגדים גסים, בגדי עשירים בגדים דקים.}
\textblock{\textbf{הוה מטלטל כנונא אגב קיטמא ואע״ג דאיכא עליה שברי עצים.} כלומר: משום דאפר כירה מוכן הוא, והוה ליה בסיס לדבר המותר ולדבר האסור ומותר. ואם תאמר אם כן כי אקשינן עליה מושוין שאם יש שם שברי פתילה שאסור לטלטלה ליסייעיה ממתניתין דכלכלה. יש לומר דשברי עצים דמו לשברי פתילה, דדין הוא שלא יהו שברי עצים בטלים לגבי כנונא כי היכי דאין שברי פתילה בטלים לגבי נר ושמן לפי שהן חשובין לגבן, מה שאין כן בכלכלה שהפירות חשובין והאבן אינה חשובה. ועוד דעיקר הטלטול אינו אלא מחמת הפירות ולא מחמת האבן ואין האבן עשויה לינטל בכלכלה, אבל עצים בכנונא ושברי פתילה בנר מקומן הוא ועשוין הן לינטל באותן כלים.\par \textbf{} ואיכא למידק דהכא משמע דדוקא אגב קיטמא שרי הא לאו הכי נעשה בסיס לדבר האסור דהיינו שברי העצים, ואילו בפרק שני דביצה בסופו (כא, ב) גבי אין מזמנין את הגוי ביום טוב אמרינן רב אחא בר יעקב אמר אפילו בשבת נמי אין מזמנין גזירה משום שיורי כוסות, ופריך ונטלטלינהו אגב כסא דהא אמר רבא כי הוינא בי רב נחמן הוה מטלטלין כנונא אגב קיטמא ואע״ג דאיכא עליה שברי עצים, ומאי קושיא שאני כנונא הואיל ואיכא קיטמא דנעשה בסיס אף לדבר המותר דד(ו)מיא לכלכלה והאבן בתוכה, אבל בכוס דליכא אלא שיורי כוסות בלחוד דאסורין אסור לטלטלן. ותירצו בתוס׳ דהכי פריך, דכי היכי דמטלטלינן כנונא אגב קיטמא ואין הכנונא והקיטמא בטלין לגבי שברי עצים שבה משום דלא חשיבי, הכי נמי שיורי כוסות לא חשיבי למיהוי כוס בטל לגבייהו.\par \textbf{} ור״ת ז״ל תירץ דהא דאמרינן הכא דמטלטלין כנונא אגב קיטמא, היינו לפי שאין יכול לנער שברי העצים ממנה כדי שלא יתפזר האפר שבתוכה, וכן כלכלה מלאה פירות והאבן בתוכה בשאינו יכול לנער האבן כדי שלא יפלו הפירות ויטנפו, דהכי מוקמינן לה בפרק נוטל (לקמן שבת קמב, א) בפירי דמטנפי, ומשום הכי בעינן שיהא בו אפר וכן פירות בכלכלה שאילולי הן היה יכול לנער את האבן ושברי העצים ולא בעינן להו כדי להתיר הטלטול מחמת היתר הפירות והאפר, אבל שיורי כוסות אע״פ שאין דבר עמהם מטלטלין אותן אגב כוסות לפי שאינו יכול לנערם לפי שהן דבוקין אל הכוס. ואין תירוצו זה מחוור כל הצורך. דהא משמע דעיקר ההיתר משום היתר הפירות, כלומר: מפני שנעשה בסיס לדבר המותר ולדבר האסור ומותר.}
\textblock{\textbf{אינהו דאמור כי האי תנא וכו׳.} איכא למידק דרב ושמואל דוקא בתוקע קאמרי, דאי לא בין לתנא קמא בין לרבן שמעון בן גמליאל פטור אבל אסור ואפילו במחזיר כל צרכו, דליכא מאן דאית ליה כבית שמאי דאמרי במסכת ביצה (כא, ב. כב, א) דיש בנין בכלים גבי זקיפת מנורה. ורבי אבא ורב הונא בר חייא על כרחין ברפוי וכשאינו תוקע, דהא אוקימנא להו כרבן שמעון בן גמליאל דאמר אם היה רפוי מותר, וכיון שכן מאי קא מקשה להו רב יהודה מדרב ושמואל אינהו לחוד ורב ושמואל לחוד. ותירצו בתוס׳ דהכי קא מקשה להו רב יהודה, מדרב ושמואל אמרי דבתוקע חייב חטאת על כרחין בשאינו תוקע ואפילו ברפוי לכתחילה מיהא אסור, וכדאמרינן לקמן (עמוד ב) גבי מטה גללניתא, מאי דעתיך בנין מן הצד הוא נהי דאיסורא דאורייתא ליכא איסורא דרבנן מיהא איכא. ואינהו סבור אע״ג דבתקע חייב חטאת לא תקע מותר לכתחילה ולא גזרינן, וכרבן שמעון בן גמליאל דאמר אם היה רפוי מותר.}
\textblock{\textbf{מפני שמקרב את כיבויו.} פירוש: אם אתה מתירו לעשות כן בערב שבת אתי למיעבד אפילו בשבת, ופעמים שעם נתינתו יפלו הניצוצות והוא מגביה הכלי ומכבה אותן.\par \textbf{} ואם תאמר ומאי שנא ממצודות חיה דגים ועופות ושריית דיו וסממנים וכל הנך דפרקין קמא (לעיל שבת יז, ב) דשרו בית הלל עם חשיכה ולא גזרינן דילמא אתי למיעבד בשבת. יש לומר התם דאב מלאכה היא לא חיישינן דמיבדל בדילי אינשי מינייהו דכולי עלמא ידעי דאסיר וזהירי בהו, אבל בנתינת מים תחת הנר לא רמי אנפשיה ולא חשיבא ליה מלאכה וסבר דשרי ויהיב ומקרב כיבויו. ולפי פירוש זה ליכא למיחש לנתינת מים מערב שבת בתוך עששית של זכוכית שקורין לנפאדיש, דהא ליכא למיגזר לשמא יתן לתוכו בשבת דהא מיזהר זהירי בהו אינשי.\par \textbf{} ויש מפרשים מקרב את כיבויו לפי שהמים תחת הניצוצות ונופלות לתוכו, ואינו דומה אפילו למחיצת שלג וברד דשרו רבנן וכדאיתא בירושלמי (דפרקין ה״ח), וכדמשמע נמי בפרק כל כתבי הקדש (לקמן שבת קכ, ב) גבי טלית שאחז בו האור דשרו רבנן דרבי יוסי ליתן עליה מים מצד אחד ואם כבתה כבתה. וטעמא משום דהכא אין דבר מפסיק בין המים לניצוצות וכל זמן שהן נופלות הרי הן נופלות לתוך המים וכבות, אבל התם שאין המים תחתיו אלא מן הצד ואין האש נופלת לתוך המים להדיא לא חשבינן ליה מקרב כיבוי. ולפירוש זה איכא למידק אם כן היאך נותנין מים בעששיות שהרי ניצוצות נופלות בהן להדיא, ולא עוד אלא להדליק בהן כל עיקר היאך מותר שהרי ניצוצות נופלות לתוך השמן וכבות. ויש לומר דהתם אינו מתכוין לכבות אלא להעלות השמן ושיהא הנר דולק יפה והלכך לכולי עלמא שרי ואפילו לרבי יוסי, אבל הכא דמתכוין לכיבוי אסור.}
\textblock{\textbf{במה טומנין ובמה אין טומנין.} פירשו בשם רבנו שמואל ז״ל: דכולה פרקין דהכא מיירי בבשיל ולא בשיל, אבל בחייתא אי נמי בבשיל כל צרכו שרי, דהא אמרינן בשלהי פרק קמא (לעיל שבת יח, ב) דבקדרה חייתא אי נמי בבשיל לא גזרינן שמא יחתה בגחלים, ואיסור הטמנה אינו אלא משום גזירת חתוי כדאיתא בשלהי פרק במה מדליקין (לעיל שבת לד, ב). אבל ר״ת ז״ל ורבנו יצחק הזקן ז״ל לא הודו לו, משום דסתם קדרות בין השמשות הם מבושלות ובסתם קדרות איכא לאוקומי מתניתין. ועוד נראה לי דאילו יש חילוק בהטמנה כמו בשהיה, היה להם לבעלי הגמרא לפרש הענין כך כמו      שפירשו בפרק כירה והיה להם לחלק כאן כמו שחלקו שם בנתבשל כל צרכו או מצטמק ורע לו ובלא נתבשל כל צרכו ומצטמק ויפה לו, אלא שאין בהטמנה חלוק כלל.\par \textbf{} ונתן ר״ת ז״ל טעם לדבר, לפי שהשהיה מן הסתם אינה אלא לאכול לערב וחומה משתמר עד שעת אכילה, ועוד כיון שהוא מניחה מגולה ושליט בה אוירא אין חתויו מועיל לו כל כך ואין החום נתפש בה והלכך אף הוא אינו בא לידי חתוי, אבל הטמנה שהוא מניחו למחר אע״פ שהוא טמון אם לא יחתה בו מתקרר והולך ובמעט חתוי חומו נתפש ומועיל והלכך אתי לחתויי, ולפיכך גזרו שלא להטמין בדבר המוסיף שמא יטמין ברמץ ואתי לחתויי.}
\textblock{ גמרא:\textbf{ קופה שטמן בה אסור להניחה על גבי גפת של זיתים.} מסתברא מדקאמר קופה שטמן בה אסור להניחה ולא קאמר קדרה שטמנה בקופה אסור להניחה על גבי גפת של זיתים, שמע מינה שאף על פי ששולי הקדרה טמונין בתוך הקופה אפילו הכי אסור להניח הקופה על גבי דבר המוסיף הבל ואע״פ שמפסיק הקופה בינו ובין הקדרה, ושמע מינה שאין קטימה מועיל להטמנה וכל שכן הוא.\par \textbf{} ואפשר שאפילו אין הקופה דבוקה ממש על גבי גפת של זיתים אלא שהיא תלויה עליו אסור, דכיון שטמנה גילה בדעתו שהוא רוצה להניחה למחר ומקפיד עליה שישתמר חומה ואתי לחתויי, דכיון דמשום גזירת חתוי הוא מה לי נוגעת בדבר המוסיף מה לי אינה נוגעת כל זמן שגילה בדעתו שהוא מקפיד שישתמר חומה. ועוד נראה לי דעל כרחין הכא בכל ענין מיירי ואפילו בשאינה נוגעת על גבי הגפת ממש, שאילו בנוגעת ממש למה לי משום שטמן בקופה אפילו לא טמן כלל יהא אסור שהרי כל זמן שפני הקדרה דבוקין על פני דבר המוסיף הבל זו היא הטמנה וכמו שכתבנו בריש פרק כירה (לעיל שבת לו, ב ד״ה עד) משמן של גאונים ז״ל שהטמנה במקצת הטמנה היא ואין הדבר תלוי בהטמנת כל הקדרה, ותדע לך מדקתני במתניתין (לקמן שבת מט, א) כיצד הוא עושה נוטל את הכסוי והן נופלות, אלמא כסוי הקדרה היה מגולה ואף על פי כן לא התירו אלא בדבר שאינו מוסיף אבל בדבר המוסיף אסור, ואם תאמר דוקא בשכולה טמונה חוץ מפיה, אם כן נתת דבריך לשיעורין. אלא נראה דמה שלא אסרו כאן אלא בשטמן בקופה היינו אפילו בשאין שולי הקדרה והקופה דבוקים על פני הגפת אלא אפילו בתלויה עליו מפני שהוא מעלה הבל ומרתיח קצת למה שהוא עליו וכל זמן שהוא טומן מגלה בדעתו שהוא מקפיד להיות חומו משתמר ואתי לחתויי, אבל כל זמן שאינו מניח ממש עליו ואף הוא אינו מקפיד עליו להטמינה כבר גילה בדעתו שהוא מתיאש ממנו ואינו מקפיד אם אין תבשילו חם ולא אתי לחתויי.\par \textbf{} ודוקא כשהדבר שהוא מטמין בתוכו הוא דבוק עם הקדרה שזו היא הטמנה, אבל אם הוא רחוק ממנו ויש ריוח ביניהם אין זו הטמנה, אלא כאותה שהתירו בפרקין קמא (לעיל שבת יח, ב) קדרא חייתא ובשיל לתוך התנור, שלא גזרו אלא במה שנותנין סביב לקדרה כגון רמץ וגזירה משום שמא יטמין ברמץ, אבל בתנור וכיוצא בו כגון כלי חרס רחבים שאין דופניהם דבוקים לדופני הקדרה ויש אויר ביניהם מותר. וכן אמרו בתוס׳.\par \textbf{} ויש מפרשים: קופה שטמן בה אסור להניחה על גבי גפת של זיתים ודוקא בששולי הקופה נוגעין בגפת, דהוה ליה כטומן בדבר שאינו מוסיף וטמן את הכל בדבר המוסיף. ואם כן לדבריהם הא דר׳ זירא אתא לאשמועינן דאפילו טמן בקופה שדפני הקופה מבדילין בין הקדרה והגפת אסור. ואינו נראה כן, אלא דוקא בשטמן קאמר וכמו שכתבתי. כך נראה לי.}
\textblock{\textbf{לעולם אימא לך דשומשמין נמי אסור וכו׳.} וכן הלכה. וכתב מורי הרב רבנו יונה ז״ל דכיון דלא איפשיטא בעיין מוקמינן מתניתין על סתם משמעותיה, דקתני ולא בגפת סתמא דמשמע אפילו דשומשמין. ועוד מדנקט ר׳ זירא לענין אסוקי הבלא גפת של זיתים ולא גפת סתם כמתניתין, שמע מינה דמתניתין דקתני גפת סתם אפילו דשומשמין קאמר. וליכא למימר דאשמועינן ר׳ זירא דגפת דקתני במתניתין גפת של זיתים תנן, דהא ודאי לא שמעינן דינא דהטמנה מדינא דאסוקי הבלא. עד כאן. ואף על גב דאתינן למיפשט מינה איפכא, מדנקט גפת של זיתים שמע מינה דשומשמין לא, הכין ודאי הוה סלקא דעתין מעיקרא, אבל השתא דדחינן הני מילי לענין אסוקי הבלא מסתבר ודאי הכין, מדנקיט לה ר׳ זירא באסוקי הבלא ואיצטריך לפרושי בהדיא גפת של זיתים לומר דלענין אסוקי דוקא גפת של זיתים אבל לענין הטמנה אפילו דשומשמין.}
\newchap{פרק \hebrewnumeral{4} במה טומנין}
\clearpage
\newsection{דף מח}
\textblock{}
\textblock{\textbf{אמר ליה התם אוקומי קא מוקים הכא אולודי קא מוליד.} ואם תאמר דקארי לה מאי קארי לה, והא קתני בהדיא בההיא ברייתא גופא (לקמן שבת נא, ב) דמיחם על גבי מיחם לא בשביל שיחמו אלא שיהיו משומרין. יש לומר בהא דר׳ זירא לא הוה ידיע ליה לההיא ברייתא ולא מינה קא מותיב, אלא הכי קאמר ליה: מאי שנא ממיחם על גבי מיחם דהא מעשים בכל יום דמניחין ולא מיחה אדם בדבר. ואינו מחוור בעיני, דהוה ליה למימר והא מעשים בכל יום דמניחין מיחם על גבי מיחם, שכן דרך התלמוד לומר בכיוצא באלו. אלא נראה לי שר׳ זירא לא היה יודע כל אותה ברייתא אלא רישא דברייתא, ויש כיוצא בזה בהרבה מקומות בתלמוד שהמקשה יודע תחלת הברייתא או סופה ואינו יודע כולה ומקשה ממנה, וכיוצא בה אחת בפרק לא יחפור (ב״ב כו, א) גבי אין פורשין נשבים ליונים.}
\textblock{\textbf{אמר ליה התם אוקומי קא מוקים.} איכא למידק היכי דמי, אם הניח כדי להפשיר, מאי טעמיה דרבא והא קתני בפרק כירה (לעיל שבת מא, א) אבל נותן הוא לתוכו או לתוך הכוס כדי להפשירן ותניא (לעיל שבת מ, ב) מביא אדם קיתון של מים ומניחו כנגד המדורה לא בשביל שיחמו אלא כדי שתפיג צינתן והפגת צינה היינו הפשר וכדכתיבנן לעיל (שבת מ, ב ד״ה         מביא) ואי להחם מאי טעמיה דר׳ זירא דהא קתני התם דאסור, וי״ל דהכא בשהניחו כדי להפשיר אלא שהיה הקומקום כל כך חם שאילו היה מניחו שם זמן רב היה מתחמם בו, ומפני זה אסרו רבה ולומר שלא התירו כנגד המדורה אלא במקום שאי אפשר לבא לידי בישול וכמו שכתבתי למעלה בפרק כירה (שם), וכדמשמע נמי בירושלמי שכתבתי שם משום דגזרינן שמא ישהה ויבוא לידי בישול והלכך גזרו אפילו בתחלת בישול, ורבי זירא סבר דכיון דסוף סוף אינו מתכוין להחם אלא כדי שתפיג צינתן שפיר דמי, כן תירץ ר״י הזקן הידוע בעל התוספות ז״ל.\par אבל מורי הרב רבינו יונה ז״ל הקשה אי באיסור הנחה בלא הטמנה קא מיירי מאי שייכא הכא, התם בפרק כירה הוה ליה לאתויי גבי ברייתא דמביא אדם קיתון של מים ומניחו כנגד המדורה, ועוד הא דתניא דמניחין מיחם על גבי מיחם דמייתינן עלה בהטמנה היא, דאי לא אמאי עריב לה ותני בהדי דיני הטמנה וכדתניא לקמן בשלהי פרקין (נא, ב) ליתנייה בסיפא דברייתא ולא לפסיק בדיני הטמנה, ולפום הכי פירש הוא ז״ל דהכא בהנחת כוזא אפומא דקומקמא כדי להטמין שניהם בדבר שאינו מוסיף הבל היה מעשה ובשבת היה, וכגון שכסה הקומקום מבעוד יום דמותר לגלותו ולכסותו בשבת וכדאיתא לקמן (נא, א), והיה רוצה להטמין הכוזא אפומא דקומקמא דקיימא לן מותר להטמין את הצונן ואפילו להפיג צינתן כדבעינן למיכתב לקמן (שם, ד״ה אמר), ומשום הכי הוה סלקא דעתיה דרבי זירא שאפילו על גבי מיחם נמי שרי, ורבה אסר דלא התירו להטמין את הצונן אלא בדבר שאינו מוסיף הבל אבל על גבי מיחם אסור מפני שתוספת חם המיחם ניכר בו והוה ליה כמטמין על גבי דבר המוסיף הבל, שלא התירו במיחם על גבי מיחם אלא מיחם דוקא שאין חומו של תחתון ניכר בעליון שאינו אלא משמר חומו, והביא ראיה מן התוספתא (פ״ד הי״ד) דקתני בה בהדיא טומנין מיחם על גבי מיחם וקדירה על גבי קדירה כו׳.}
\textblock{\textbf{מאי שנא מפרונקא.} פירש ר״ח ז״ל דאמר רבא בפרק תולין (קלט, ב) האי פרונקא אפלגא דכובא שרי, דלא הוי אהל ולא חיישינן שמא יסחוט.}
\textblock{\textbf{אי משום הא לא איריא הכי קאמר אם לא טמן בהם אין מטלטלין אותן.} איכא למידק ליסייעיה ממתניתין (לקמן מט, א) דקתני בהדיא בגיזי צמר ואין מטלטלין אותן כיצד הוא עושה נוטל את הכסוי והם נופלין מאיליהן, אלמא אף על פי שטמן בהן אין מטלטלין אותן וכדדייקינן מינה לקמן (מט, ב), וי״ל דאי מהתם הוה דחי ליה דלמא גיזי צמר חשיבי טפי אבל מהאי ברייתא דקתני בהדיא ולא במוכין הוה מייתי ליה, ואף על גב דלקמן לא משמע דגיזי צמר חשיבי טפי, מכל מקום הכא הוה אפשר לדחויי הכין, והא דאביי הלכתא היא ובשל הפתק וכדאמר רבינא לקמן (נ, א) כי היכי דלא נפיק לה לדאביי לבר מהלכתא.}
\textblock{\textbf{מאי [לפנינו: מה] בין זה למגופת חבית.} פירש רש״י ז״ל (ד״ה מה) דתניא בפרק חבית (קמו, א) רבן [צריך להוסיף: שמעון בן] גמליאל אומר מביא אדם חבית של יין ומתיז ראשה בסייף ומניחה לפני האורחים בשבת ואינו חושש, והקשו בתוספות (ד״ה וכי) דאם כן מאי קא פריך דהא טעמא דרבן [צריך להוסיף: שמעון בן] גמליאל התם משום דלעין יפה קא מכוין ולא לפיתחא אבל הכא לפיתחא קא מכוין ואסור, ועוד מאי קא משני זה חיבור וזה אינו חיבור, התם נמי חיבור הוא דהא מגופה של חבית עצמה שרי להתיז משום דלעין יפה קא מכוין, וכדמוכח התם דבעו מיניה מרב ששת מהו למיברז חביתא בבורטיא ואמר להו אסור, ופריך עלה מהא דרבן שמעון בן גמליאל דשרי להתיז ראשה בסייף, ומשני התם לעין יפה קא מכוין אבל הכא לפתחא קא מכוין דאי לעין יפה ליפתחיה מפתח, ושמעינן מינה דמגופה של חבית קא מתיז, מדקא מייתי לה על הא דמיברז חביתא בבורטיא, דההיא על כרחין בגופה של חבית עצמה היא דאי במגופה בהדיא שרי לה במתניתין דקתני התם אין נוקבין מגופה של חבית דברי רבי יהודה וחכמים מתירין, אלא ודאי בגוף החבית אסר רבי זירא ודכוותה שרי רבן שמעון בן גמליאל ואף על גב דחבור הוא משום דלעין יפה קא מכוין, והוה ליה לשנויי הכא בשמעתין התם לעין יפה קא מכוין הכא לפתחא קא מכוין.\par ורבינו תם ז״ל (ד״ה וכי) פירש וכי מה בין זה לנקיבת מגופת חבית דתנן בפרק חבית (לקמן קמו, ב) אין נוקבין מגופה של חבית דברי רבי יהודה וחכמים מתירין, וא״ת מאי קשיא ליה התם הוה ליה פתח שאינו עשוי להכניס ולהוציא ואפילו לרבי יהודה אינו חייב חטאת, אבל הכא דעשוי להכניס ולהוציא לכולי עלמא אסור וחייב חטאת, י״ל דהכי קא מקשי אם איתא דפותח בית הצואר חייב חטאת, התם במגופה אף על גב דאינו עשוי להכניס ולהוציא הוה ליה למיסר מיהא דרבנן כיון דאילו היה עשוי להכניס ולהוציא היה חייב חטאת, ומשני זה חיבור שהוא הכל כלי אחד, אבל מגופה אינה מן החבית ואפילו היה אותו נקב עשוי להכניס ולהוציא לא היה חייב חטאת לפי שאין המגופה מן גופה של חבית עצמה.}
\textblock{\textbf{תנן שלל של כובסין [לפנינו נוסף: ושלשלת של מפתחות] והבגד שהוא תפור בכלאים חיבור עד שיתחיל להתיר.} לאו הכי תני לה במתניתין אלא הכי תנינן לה בפרק בתרא דמסכת פרה (פי״ב מ״ט) שלל הכובסין והבגד שהוא תפור בכלאים חיבור לטומאה ואינו חיבור להזייה, אלא משום דעיקר ברייתא שנויה במשנתינו קאמר הכא תנן, ומשום דבמתניתין לא מפרש עד מתי הוי חיבור ובברייתא קא מפרש עד שיתחיל להתיר דמינה קא דייקינן דאפילו שלא בשעת מלאכה הוי חיבור, משום הכי קא מייתי ברייתא ולא מייתי מתניתין.}
\textblock{\textbf{מאן תנא הא מילתא דאמור רבנן כל המחובר לו הרי הוא כמוהו.} פירוש היינו ההיא דשלל של כובסין וכן פירש רש״י ז״ל.}
\textblock{\textbf{[אמר רב יהודה אמר רב רבי מאיר היא].} ורבנן פליגי עליה, ואף על גב דגבי מספורת של פרקים הוי חיבור לטומאה אפילו לרבנן, שאני התם ששניהם עושין מעין מלאכה אחת מה שאין כן בשלל של כובסין שאין בגד זה עושה מלאכה אחת עם בגד זה, וכן בית הפך ובית התבלין.}
\textblock{\textbf{עבדו בהו רבנן היכרא כי היכי דלא ליתי לשרוף עלייהו תרומה וקדשים. } ובשלל של כובסין איכא היכרא טובא כיון דלא הוה חיבור להזאה, אבל כלי חרס דלאו בני הזאה נינהו איצטריכו למעבד בהו האי היכרא דלא ליטמו באויר.}
\textblock{\textbf{מה נפשך אי חיבור הוא אפילו להזאה [לפנינו נוסף: נמי] אי לאו חבור הוא אפילו לטומאה נמי.} ואיכא למידק אמאי לא אקשי נמי הכין גבי שלל של כובסין דבדידיה נמי קתני במתניתין חיבור לטומאה ואינו חיבור להזאה, ורבא נמי דקא מתרץ דבר תורה בשעת מלאכה חיבור בין לטומאה בין להזאה, אמאי לא מתרץ כדמתרצינן בההיא דבית הפך עבדו רבנן היכרא כי היכי דלא לשרוף עליה תרומה וקדשים, וי״ל דרבא פשיטא ליה דבשעת מלאכה הוי חיבור דבר תורה הואיל ושניהם עושין מלאכה אחת מה שאין כן בבית הפך ובשלל של כובסין, ומהאי טעמא נמי לא אקשינן ליה גבי שלל של כובסין, דכיון שאינן עושין מלאכה אחת פשיטא ליה דאפילו בשעת מלאכה דהיינו בשעת כיבוס אינו חיבור דבר תורה, ומה שאינו חיבור להזאה היכרא הוא דעבדי בה.}
\clearpage
\newsection{דף מט}
\textblock{\textbf{וגזרו טומאה שלא בשעת מלאכה משום טומאה שבשעת מלאכה.} ואם תאמר, ליגזור נמי במקל שעשאו יד לקרדום. י״ל שאני מקל דעשוי אדם לזורקו לבין העצים שלא בשעת מלאכה ואין אדם מצניעו מה שאין כן במספורת של פרקים.}
\textblock{\textbf{מרטא דביני אטמא.} לשון רבינו האיי גאון ז״ל יש בין יריכות הבהמה במקום שמשתין צמר לח ואינו יבש לעולם ואפילו אורגין אותו ואפילו בלה נמצא לח ולא יבש.}
\textblock{\textbf{תפילין צריכין גוף נקי כאלישע בעל כנפים מאי טעמא [לפנינו: מאי היא] אמר אביי שלא יפיח בהן}. פירוש צריכין גוף נקי היודע ליזהר שלא יפיח בהן כלומר שיזהר לסלקם בשעה שצריך להפיח, וכן פירש רש״י ז״ל (ד״ה שלא), וכן פירשו גם בתוספות (ד״ה אביי).}
\textblock{\textbf{רבא אמר שלא ישן בהן.} פירוש אביי לא פליג אדרבא דאף הוא סבר דאסור לישן בהם בין שינת קבע בין שינת עראי וכדתניא בסוכה פרק הישן (כו, א), ומוקמינן לה התם דמנחי ארישיה, אלא דאביי לא חייש לשמא ישן בהם דהרבה פעמים שאין אדם יכול להעמיד עצמו שלא ישן, ורבא בעי שיהא יכול להעמיד על עצמו שלא ישן בהם, וטעמא דמילתא דאסור לישן בהם כדי שלא יסיח דעתו מהן ואי נמי שלא יפיח בהן, אבל משום חשש קרי לא דבעל קרי אינו אסור בהן, וכדמוכח בסוכה בפרק הישן (שם, ב) דאמר רבי יוסי ילדים לעולם חולצים מפני שרגילים בטומאה ופריך לימא קסבר רבי יוסי בעל קרי אסור בתפלין, ומשני הכא בילדים ונשותיהן עמהם עסקינן שמא יבא לידי הרגל דבר, אלמא משום קרי לא אסרינן.}
\textblock{\textbf{נטלם מראשו.} וא״ת והיכי עביד הכי והא אמרינן (סנהדרין עד, ב) בשעת השמד אפילו אמצוה קלה יהרג ואל יעבור, וי״ל דהני מילי במצות לא תעשה אבל במצות עשה לא דשב ואל תעשה שאני דלא מוכחא מילתא.\par וא״ת אם כן היכי מסר נפשיה ומנח והלא מתחייב בנפשו, וי״ל דאע״פ שאינו מחויב ליהרג על קיומה אפילו הכי רשאי למסור נפשו עליה ומקבל שכר, שכל מצוה שמסרו ישראל נפשן בשעת השמד מוחזקת בידם ומקבלין עליה שכר הרבה. ואמרו (בבראשית) [צ״ל: בויקרא] רבה (פרשה לב, א) מה לך יוצא ליסקל על שמלתי את בני מה לך יוצא ליצלב על שנטלתי את הלולב. ואפילו על התפלה שהיא מצוה מדרבנן, מסר דניאל עצמו עליה.}
\textblock{\textbf{הביאו לן שלחין ונשב עליהן.} פירוש: בשבת, וכן פירשו כל הגאונים ז״ל. ורש״י ז״ל שפירש בחול היה מעשה, אינו מחוור, דאם כן מאי קא מייתי מינה ראיה דילמא עין יפה נהג בהן מפני האורחים. ועוד רבי ישמעאל אדמייתי מהא ליתי מאידך דאבוה דאמר לקמן בהדיא דמותר. אלא ודאי בשבת היה מעשה, וניחא ליה לאתויי מהא דעבד עובדא בנפשיה.}
\textblock{\textbf{כנגד מלאכה ומלאכות שבתורה ארבעים חסר אחת.} פירש ר״ח ז״ל: שכתובין בתורה מלאכה ומלאכות ס״א, טול מהם ג׳ מלאכתו הכתובים בויכולו (בראשית ב, ב-ג) שאינן צווי וד׳ שכתוב בהן ועשית ויעש תעשה ולרגל המלאכה אשר לפני (בראשית לג, יד), וי״ד שכתוב בהן כל מלאכת עבודה הרי כ״א, נשארו מ׳. ובכללם ששת ימים תאכל מצות וגו׳ לא תעשה מלאכה (דברים טז, ח), ולמה זה כי בפירוש אמרו בתלמוד ארץ ישראל בפרק כלל גדול (במכילתין פ״ז ריש ה״ב) שבא להשלים ל״ט מלאכות שבתורה. ויש מי שהוציא מכלל אלו הארבעים ויבא הביתה לעשות מלאכתו (בראשית לט, יא) נשארו ל״ט, ויש מי שהוציא והמלאכה היתה דים (שמות לו, ז) ולדברי הכל נשארו ל״ט.\par \textbf{} ואי קשיא לך למאן דיליף ממלאכות שנכתבו בתורה אטו מדעת עצמן בררו להן חכמים המלאכות. ועוד דבהדיא תנן בפרק הזורק (לקמן שבת צו, א) המושיט חייב שכן היתה עבודת הלוים. ויש לומר דכולהו ילפי ממשכן מדנסמכה פרשת שבת למלאכת המשכן, אלא דמר יליף כולה מילתא ממשכן, ומר לא יליף מניינן ממשכן אלא ממלאכה ומלאכות, וחשיבותן כלומר אבות המלאכות בלחוד ילפינן ממשכן. ואם תאמר אם כן מאי קאמרינן בסמוך תניא כמאן דאמר ממשכן דתניא אין חייבין אלא במלאכה שכיוצא בה היתה במשכן, והא כולי עלמא מודו בהא. יש לומר דמדקתני הם זרעו ואתם לא תזרעו קא מסייע ליה, דמינה משמע דעיקרא דמילתא מהתם יליף וממלאכות שבמשכן אזהרינהו רחמנא.}
\clearpage
\newsection{דף נ}
\textblock{\textbf{אלא אי איתמר הכי איתמר לא שנו אלא שלא יחדן להטמנה אבל יחדן להטמנה מטלטלין אותן.} כתב רבנו האי גאון ז״ל: והלכה כמו שאמרנו משמיה דרבא בשנייה שאם יחדן להטמנה מטלטלין אותן, וקאי רבינא כוותיה דקאמר משנתנו בשל הפתק, עד כאן. ולא הבנתי דבריו, דהא משמע דרבינא קאי כלישנא קמא דרבא ולתרוצה למתניתין דלא תקשי אדרבא הוא דאתו, דאילו בשייחדן להטמנה אפילו בשל הפתק נמי מטלטלין אותן. אבל הרב אלפסי ז״ל הביא אוקמתא בתרייתא דרבא ולא הביא דברי רבינא. והוא מן התימה, דרבינא בתראה הוא וקיימא לן כוותיה. וכן פסק הרב רבנו יונה ז״ל דבשל הפתק אם יחדן להטמנה מטלטלין אותן, דהא דתני ר׳ יוסי בן שאול אמר רבי לא שנו אלא שלא יחדן להטמנה בשל הפתק הוא כרבינא, אבל במוכין וגיזי צמר שאינן של הפתק אפילו לא יחדן להטמנה אם טמן בהן מטלטלין אותן כדאמר רבא בקמייתא. ואף על גב דרבא אמתניתין קאי כדמשמע מדקאמר לא שנו, ומתניתין הא אוקימנא בשל הפתק. יש לומר דמתניתין בכל מוכין וגיזי צמר קא מיירי, ורישא הכי קאמר טומנין בגיזי צמר ואין מטלטלין אותן אם לא טמן בהם, ואם טמן בהם מטלטלין אותן אם אינן של הפתק, ואם של הפתק אין מטלטלין אותן. והרב בעל ההלכות ז״ל גם הוא כתב דמוכין שטמן בהן מטלטלין אותן. ודאמר (רבא) [אביי] לעיל (שבת מח, א) וכי מפני שאין לזה קופה של תבן מפקיר קופה של מוכין ההיא נמי בשל הפתק.}
\textblock{\textbf{ורב אסי אמר יושב אף על פי שלא קשר ואף על פי שלא חשב.} מדקאמר ישב ואף על פי שלא חשב מסתברא דמוסיף הוא על דברי שמואל, דטפי עדיף חשב ולא ישב מישב ולא חשב, ומדסייעיה רב אשי לרב אסי (מדתניא) [מדתנן] הקש שעל המטה מנענעו בגופו וכו׳ שמע מינה הלכתא כרב אסי וכל שכן אם חשב עליהן כשמואל ואף על פי שלא ישב. וכן פסק רבנו האי גאון ז״ל, וזה לשונו: הלכה כרבן שמעון בן גמליאל שאם חשב אף על פי שלא קשר מותר, ומאי ניהו קשר שקשר חוטי החריות אחד באחד. ואף על גב דרב סבר לה כתנא קמא, הכא קמו להון שמואל ורבה בר בר חנה בחד טעמא והלכה בהדיא איתמרא קמיה דרב, ודרב אסי הלכה ניהו דהא סייעה רב אשי ממתניתין ואמר שמע מינה. עד כאן. ומורי הרב רבנו יונה ז״ל כתב כן ומסתברא לן דרב אסי אפילו ישב קאמר וכ״ש חשב דמיחדי לישיבה טפי. והא דתניא יוצאין בפיקרין ובציפה בזמן שצבען בשמן וכרכן במשיחה, ואילו מחשבה לא מהניא. התם משום דלא מוכחא מילתא דמיחדי למכתו ומיחזי כמוציא בשבת, אבל יצא בהן מבעוד יום מוכחא מילתא.\par \textbf{} ופירוש ציפה ופיקרין פירש רבנו האי גאון ז״ל פאה נכרית שעושה איש קרח לכסות את ראשו כמי שיש לו שער. ורש״י ז״ל פירש בענין אחר.}
\textblock{\textbf{דתנן רבי שמעון אומר נזיר חופף ומפספס אבל לא סורק.} לאו רבי שמעון תני(א) במתניתין אלא סתמא קתני, והכי תני(א) לה בנזיר פרק שלשה מינין (נזיר לט, א. מב, א) נזיר שגלח בין בזוג בין בתער או שפספס כל שהוא חייב נזיר חופף ומפספס אבל לא סורק רבי ישמעאל אומר לא יחוף באדמה מפני שמשרת את השער, אלא דבגמרא (שם) אמרינן עלה הא מני רבי שמעון היא דאמר (לעיל שבת כב, א) דבר שאין מתכוין מותר ולפיכך הכניסו כאן רבי שמעון בלישנא דמתניתין משום דרבי שמעון היא. והרבה כיוצא בזה בתלמוד. ואחרת יש בריש פרקין (שבת מח, ב) ההיא דתנן שלל של כובסין וכמו שכתבתי עליה למעלה (שם ד״ה תנן).}
\textblock{ והא דתנן:\textbf{ אבל לא סורק.} אמרינן עלה התם (בנזיר שם) משום דכל הסורק להסיר נימין המדולדלות קא מכוין.\par \textbf{} ופירוש חופף אפילו בנתר ובחול, ותדע מדקתני רבי ישמעאל אומר לא יחוף באדמה מפני שמשרת את השער.}
\textblock{\textbf{עפר לבינתא.} כתב רבנו האי גאון ז״ל יש שמפרש לבנה ממש, ויש אומרים לבינתא של לבונה כנרד לובאן בלשון ערב, וכך אנו סוברים.}
\textblock{\textbf{מהו לפצוע זיתים בשבת.} פירש רש״י ז״ל: כדי למתק מרירתן. ואינו מחוור. דאם כן מאי קא מתמה וכי בחול מי התירו אדרבא אפילו בשבת יהא מותר דתקון אוכל כזה למתקו למה יאסר, והכא לאו משום סחיטה קא בעי דאם כן הוה ליה למיבעי הכין בהדיא ולא היה לו לשאול כן בכאן אלא לקמן בפרק חבית בהלכות סחיטה. אבל רבנו האי גאון ז״ל פירש לפצוע זיתים לממשא ידא. וזה נכון.}
\textblock{\textbf{והכא בחוששין קא מיפלגי מר סבר חוששין שמא נתקלקלה הגומא.} פירש רש״י ז״ל: אי שרית ליה ליטול בקופה זקופה פעמים שנתקלקלה הגומא ואתי נמי להחזירה. והקשה עליו מורי הרב ז״ל דאם כן חוששין שמא תתקלקל הגומא הוה ליה למימר. אלא הכי קאמר מר סבר חוששין שמא נתקלקלה הגומא שאם נטלה בקופה זקופה לא יחזיר עד שיתברר לו שלא נתקלקלה הגומא הא סתמא דילמא נתקלקלה הגומא, ומר סבר סתמא לא חיישינן. כן פירש מורי הרב ז״ל. וכפירושו מצאתי לרבנו האי גאון ז״ל.}
\textblock{\textbf{אמר שמואל האי סכינא דביני אורבי דצה שלפה והדר דצה שפיר דמי.} פירש רש״י ז״ל, דהא דאסרי רב הונא ושמואל היכא דלא דצה משום דמזיז עפר ממקומו. ואינו מחוור. דאם כן קשיא דשמואל אדשמואל, דהא אית ליה לשמואל דטלטול מן הצד לאו שמיה טלטול, כדאיתא לעיל בפרק כירה (שבת מג, ב) דאמר שמואל מת המוטל בחמה הופכו ממטה למטה קסבר טלטול מן הצד לא שמיה טלטול. ועוד דטלטול בכהאי גוונא לא אשכחן אמורא דאסר, דהא איפסיקא הלכתא כרבי אלעזר בן תדאי (לקמן שבת קכג, א) וכמה סתומות נסתמו כן, וכמו שכתבתי שם בפרק כירה באותה שמועה. אלא טעמא דרב הונא ושמואל הכא משום עשיית גומא. וכן נראה מדברי רבנו האי גאון ז״ל שפירש גזראתא דקני שפיר דמי שאם יש בין נדבכי הבנין קנים או שהיה הבנין בקנים מותר לנעצה לכתחילה שאין לקנים גומא. עכ״ל.}
\textblock{ יש ספרים דגרסי:\textbf{ אם היו מקצת עליו מגולין.} וכתב רבנו האי גאון ז״ל דטעות הוא, שאם האמהות טמונין והעלין מגולין הרי זו זריעה מעולה, אלא דוקא אם היו מקצתן מגולין. נראה שהוא ז״ל אינו גורס אותו כלל אלא כך אם היו מקצתן מגולין. ויש גורסים אם היו מקצת העליון מגולין. והכל עולה לטעם אחד. ולפי גירסא זו וגירסת רבנו האי גאון ז״ל הא דנקט מקצתן מגולין בין לענין שבת בין לענין כלאים ושביעית ומעשר נקט לה.\par \textbf{} אבל רש״י ז״ל גורס: אם היו מקצת העלין מגולין, ומשום שבת נקט לה אבל לענין כלאים ושביעית ומעשר לא. וטעמא דאין בהם משום כלאים ושביעית ומעשר פירש בירושלמי (כלאים פ״א ה״ט) באגודה, כלומר: שעשה מהן אגודה וטמנה. ודוקא טומן שאינו רוצה בהשרשתן אבל מתכוין לזריעה לא, וכדאיתא התם בירושלמי. ודוקא בשלא השרישו הא השרישו אסור ליטלן בשבת משום תולש, ומדקתני ולא משום מעשר [משמע] דמיירי בשהשרישו וניתוספו שאם לא ניתוספו אפילו השרישו ואפילו זרען לשם זריעה גמורה אין בו משום מעשר, ואף רש״י ז״ל כן פירש, ומיהו בשבת דוקא בדלא אשרוש וכדאמרן.}
\textblock{\textbf{אבל רבנו האי גאון ז״ל פירש: ולא משום מעשר דאי איכא התם שמנה ראשי לפתות לא מצטרפי לחיובינהו להנך במעשר, כלומר: שאם יש כאן שמנה ראשי לפתות שלא נתעשרו וטמן עם אלו ואחר כך עקרן אינן כלפתות חדשות להצטרף עם אלו. ולדבריו נראה לי דצריך להעמידה בשרבו הגידולין על העיקר, שאם לא כן היאך יצטרף הלפת עצמו למעשר השמנה שלא נתעשרו כלל ובעיא היא בנדרים פרק הנודר מן הירק (נדרים נז, ב) בגידולין שרבו על העיקר אם מבטלין       } את העיקר אם לא הא לא רבו פשיטא דלא, [ו]אתינן למיפשטה התם מדאמר רבי יצחק אמר רבי יוחנן ליטרא בצלים שתקנה וזרעה מתעשרת לפי כולה, אלמא אתיאן גידולין ומבטלין עיקר, ודחינן דילמא לחומרא, והכא אם היה כל שיעור הלפתות מצטרף עם שמנה החדשות הוי חומרא דאתי לידי קולא וכמעשר מן הפטור על החיוב ומהחיוב על הפטור. ומצאתי בפירושי המשנה לרבנו שמשון ז״ל (כלאים פ״א מ״ט ד״ה ונטלין) שהקשה עליו למה שפירש דמעשר בעי צירוף, שלא מצינו כן בשום מקום אלא גבי מעשר בהמה, דמעשר ירק פשיטא דמחייב על כל לפת ולפת.\par \textbf{} יש מי שכתב דלית הלכתא כי הא מתניתין דקתני אם היו מקצתן מגולין ניטלין בשבת דאלמא בשאינן מגולין כלל אינן ניטלין, ואנן הא קיימא לן כר׳ אלעזר בן תדאי דאמר לענין פגה שטמנה בתבן וחררה שטמנה בגחלים תוחב לה בכוש או בכרכר והן ננערות דטלטול מן הצד לא שמיה טלטול (לקמן שבת קכג, א). אבל מורי הרב ז״ל כתב דהלכתא היא, וטעמא דכיון שהיתה גומא זו מכוסה ולא היתה ניכרת שם כל עיקר מיחזי השתא כעושה גומא מה שאם כן בתבן וגחלים דלית בהו משום עשיית גומא בפרק אלו קשרים.}
\clearpage
\newsection{דף נא}
\textblock{\textbf{אמר רב יהודה אמר שמואל מותר להטמין את הצונן.} פירש רש״י ז״ל: להטמין שלא יחמו ולא גזרינן אטו הטמנה כדי שיחמו. ונראה מדברי הרב ז״ל שלהטמין את הצונן כדי לחממו אסור ואע״פ שהוא מטמין בדבר שאינו מוסיף הבל. אבל הרמב״ם ז״ל (פ״ד מהל׳ שבת ה״ד) פירש אפילו לחממו. וכן נראה מדברי הגאונים ז״ל. וכן ודאי נראין הדברים, דכולה פירקין בהטמנה דלהחם קא מיירי ולא בהטמנה דלהעמיד צנה ושלא יתחמם, ועוד מדדחינן אי ממתניתין הוה אמינא הני מילי דבר שאין דרכו להטמין, ואם להטמין שלא יחמו אין לך דבר שדרכו להטמין שלא יתחמם יותר מן המים בימות החמה.\par \textbf{} וכתב הרמב״ן ז״ל דדוקא בדבר שאינו מוסיף אבל בדבר המוסיף אסור ואפילו מבעוד יום ואף על פי שהוא צונן גמור. והוצרך לפרש כן, כדי שלא תאמר דכדרך שהתירו להטמין אפילו משחשיכה בדבר שאינו מוסיף, דאלמא שאני צונן גמור מחמין שהרי אסרו להטמין את החמין משחשיכה ואפילו בדבר שאינו מוסיף הבל, הכי נמי לישרי להטמין הצונן מבעוד יום אפילו בדבר המוסיף שלא גזרו בצונן כלל לחתוי גחלים. והביא ראיה ז״ל מדרב חסדא ממעשה שעשו אנשי טבריא (לעיל שבת לט, א-ב). ודבר ברור הוא.}
\textblock{\textbf{אע״פ שאמרו אין טומנין את החמין בדבר המוסיף הבל אם בא להוסיף מוסיף, כיצד רבן שמעון בן גמליאל אומר נוטל את הסדינין ומניח את הגלופקרין וכו׳.} כתב מורי הרב ז״ל: איכא מרבוותא ז״ל דלא גריס כיצד, משום דקשיא ליה הרי אין דברי רבן שמעון בן גמליאל פירוש לדבר זה דרבן שמעון בן גמליאל לגלות ולחזור ולכסות הוא מתיר וברישא דברייתא לא התרנו אלא להוסיף. ומביא ראיה מן התוספתא (פ״ד הי״ב) דלא קתני בה כיצד. ולפי דבריהם הא דרבן שמעון בן גמליאל פליגא, והלכתא כתנא קמא שלא התיר אלא להוסיף. ובכולהו ספרי גרסינן כיצד, ומיתחזי לן דהכי פירושא, כיצד אם בא להוסיף מוסיף מי אמרינן לא התרנו להוסיף אלא אם כן כסה מתחילה בדבר המשמר חומו בתוכו כגון גלופקרין וכיוצא בו או דילמא אפילו בדבר קל כגון סדינין שאינן משמרין חומו כיון דמהני כל שהוא כסהו מבעוד יום חשבינן ליה, ופרשוה מתוך דברי רבן שמעון בן גמליאל שאמר נוטל את הסדינין ומניח את הגלופקרין וכולה זו ואין צריך לומר זו קתני. ומצינו בתלמוד ירושלמי (פ״ד ה״ג) שהוצרכו לפרש דבר זה, והכי גרסינן התם: תני אין טומנין משחשיכה אבל מוסיפין עליו כסות וכלים, כמה יהא עליהם ויהא מותר לכסותן, רבי זריקא בשם רבי חנינא אמר אפילו מפה, אמר רבי זעירא ובלבד דבר שהוא מועיל, אמר רבי חנינא כל הדברים מועילין, אמר רבי מתניא ויאות מאן דנסיב מרטוט ויהיב ליה על רישיה בשעת צינתא דילמא לא כביש צינתא. עד כאן לשון מורי ז״ל.}
\textblock{\textbf{לא אסרו אלא באותו מיחם אבל פינן ממיחם למיחם מותר.} תוספתא (פ״ד הי״ב): לא אסרו להטמין אלא במיחם שהוחמו בו מערב שבת אבל מפנה הוא לתוך מיחם אחר או לתוך קיתון אחר ומטמין.}
\textblock{ הא דתניא:\textbf{ טמן וכיסה בדבר הניטל וכו׳.} פירש רש״י ז״ל: טמן קרוי מה שנותן סביבות הקדרה וכיסוי קרי מה שנותן על פיה, והלכך כשטמן בדבר שאינו ניטל אם כיסה בדבר שניטל מגלה הקדרה ואוחזה, אבל טמן בדבר הניטל וכיסה בדבר שאינו ניטל אם אין מקצת פיה מגולה לא יטול דאין לו במה יאחזנה, וליטול מלמטה ולנער את הכיסוי דומיא דהאבן שעל פי החבית (דלקמן שבת קמב, ב) אי אפשר, משום דהוה ליה מניח ולא התירו אלא בשוכח אבל במניח נעשה בסיס לדבר האסור (כדאיתא לקמן שם), והיכא דהוה מקצת פיה מגולה אין זה טלטול שמצדדו והכיסוי נופל מאליו. ויש להקשות דאם כן אפילו מקצת הכיסוי מגולה היאך יטלטל והלא נעשה הכיסוי בסיס לדבר האסור, דאטו נר שעל גבי הטבלא פנוי מי מנער את הטבלא ומעות שעל גבי הכר כל הכר מגולה ובמניח לא יגע בו (שם).}
\textblock{\textbf{ושמעתי משמו של הראב״ד ז״ל דהיכא דמקצת פי הקדרה מגולה לא נעשה האוכל בסיס לדבר האסור, והקדרה נעשה בסיס לדבר האסור ולדבר המותר, כגון כלכלה מלאה פירות והאבן בתוכה שמותר לטלטלה אפילו במניח, כדאיתא בפרק נוטל. וגם זה אינו מחוור בעיני. שאפילו כשאין מקצתה מגולה האוכל עצמו לא נעשה בסיס, שאין הדבר שאינו ניטל עומד ממש על האוכל אלא על כיסוי הקדרה, והקדרה היא שנעשית בסיס לדבר המותר ולדבר האסור ולעולם יהא מותר. ועוד מאי שנא מפגה שטמנה בתבן וחררה שטמנה בגחלים דתוחב לה בכוש או בכרכר כדברי רבי      } אלעזר בן תדאי בפרק כל הכלים (לקמן שבת קכג, א) וקיימא לן כותיה כדאיתא (שם).\par \textbf{} ונראין דברי הר״ז הלוי ז״ל שפירש שהכיסוי הוא תחת ההטמנה והוא כיסוי לקדרה או לתבשיל בלא קדרה, ואם היתה קדרה מגולה כיון שאין הכיסוי מגיע עד הארץ אף על פי שכיסה בדבר שאינו ניטל נוטל ומחזיר שמנער את הקדרה והכיסוי נופל, ואין אומרין בזה נעשה בסיס לדבר האסור לפי שאינו אלא לצורך שעה ודעתו היה מאתמול ליטול ממנה בשבת, ואם לא היתה הקדרה מגולה כל עיקר כגון שהכיסוי שלה מקיף אותה מכל דפנותיה ומגיע עד לקרקע אינו נוטל ומחזיר. ע״כ.\par \textbf{} אלא שקשה לי קצת. שהיה לו לומר כיסה בדבר שאינו ניטל וטמן בדבר הניטל לפי שהכיסוי קודם להטמנה. ועוד דמאי שנא דנקט איסור בכיסוי והיתר בהטמנה, לימא נמי איפכא שאפילו כיסה בדבר הניטל וטמן בדבר שאינו ניטל אם מגעת ההטמנה עד לקרקע אינו נוטל. אלא שבזו יש לומר שאין דרך להטמין עד לקרקע או להניח פיה מגולה בלא הטמנה.\par \textbf{} גם הטעם שאמר שאינו בסיס לדבר האסור מפני שאינו אלא לצורך שעה אינו מחוור בעיני, שהרי אסרו (לעיל שבת מז, א) נר ושמן ופתילה מפני שנעשו בסיס לדבר האסור ואף על פי שאין הנר בסיס לכל היום אלא לצורך מקצת הלילה שהרי אדם מצפה מתי תכבה נרו (לעיל שבת מו, ב). אבל נראה כמו שכתב הוא ז״ל בפרק כל הכלים (לקמן שם) לפי שאין הקדרה נעשית בסיס להטמנה ולכיסוי ואינה תשמיש להם אלא אדרבא הם תשמישין לקדרה, וה״ל כפגה שטמנה בתבן וחררה שטמנה בגחלים שהתבן והגחלים תשמיש לפגה וחררה ואין הפגה והחררה תשמיש להם.}
\textblock{ הא דתניא:\textbf{ אין מרסקין לא את השלג ולא את הברד.} פירש רש״י ז״ל: משום דקא מוליד בשבת ודמיא למלאכה שבורא את המים האלו, אבל נותן לתוך הכוס אע״פ שנמוח מאיליו. ומה שכתב הרב ז״ל משום דקא מוליד לאו למיסר משום נולד קאמר, אלא שהאיסור הוא משום סרך מלאכה לפי שהוא כבורא ומוליד את המים הללו. אבל בספר התרומה (סי׳ רלד ורלה) כתוב שהוא אסור משום נולד, ולפיכך אסר ליתן קדרה או פנאדה שקרש שמנוניתא כנגד המדורה, משום דמעיקרא עב וקפוי ועכשיו נמחה ונעשה צלול והוה ליה נולד. ולדבריו אפילו בחמה אסור דהוה ליה נולד. ואינו מחוור. דאם כן למה התירו לתת לתוך הכוס, דמכל מקום הרי הוא נפשר בתוך הכוס והרי הוא נולד בשבת ואסור. ועוד דהא פירות דלאו בני סחיטה נינהו סוחטין לכתחילה (לקמן שבת קמג, ב. קמד, ב), ואפילו תותים ורמונים היוצא מעצמן אם לאוכלין מותר (לקמן שבת קמג, ב). אלא לעולם לא אסרו אלא לרסק ביד מפני סרך מלאכה.\par \textbf{} ולי נראה דמשום גזירת סחיטה בפירות העומדין למשקה נגעו בה מפני שהברד והשלג למימיהן הן עומדין, ולפיכך לתת לתוך הכוס מותר שאינו נראה כסוחט. ועוד הקלו בו לרסק בתוך הכוס כפירות דלאו בני סחיטה והתירו לרסק אפילו ביד לתוך הכוס וכדתני בתוספתא (פ״ד הט״ו) אבל מרסק הוא לתוך הקערה. וטעמא דמילתא, לפי שאף על פי שנקרש ונעשה עב הכל יודעין שאין בו אוכל ושמימיו נסחטין מתוכו אלא מים הן מתחלתן ועד סופן אלא שנקרשו לפי שעה ולפיכך הקילו בהן לסוחטן לתוך הקערה או לתוך הכוס, אלא שהחמירו בהן לסחטן ולרסקן בפני עצמן. כך נראה לי.}
\textblock{\textbf{ארבע בהמות יוצאות באפסר וכו׳ מאי לאו למעוטי גמל בחטם לא למעוטי נאקה באפסר.} כלומר: לא למעוטי שמירה מעולה קא אתי, אלא למי שדרכן להשתמר בשמירה מעולה דמותר לצאת בפחותה ממנה קא אתי, ולאשמועינן שיש מהן להקל, כגון סוס וחמור שדרך הסוס בשיר וחמור דהיינו לובדקים וכדמפורש במתני׳ בסמוך שדרכו לצאת בפרומביא ואפילו הכי שרי בפחותה ממנה דהיינו אפסר, אבל לעולם גמל בחטם דהיינו שמירה יתירתא שרי, ויש מהן להחמיר כגון נאקה באפסר לפי שאינה שמירה לו כלל. וקשיא לי דהוה ליה למימר לא למישרי סוס וחמור באפסר, דאילו נאקה באפסר הא משמע דפשיטא להו בלא הא דרבי ישמעאל דכיון דלא מינטרא ביה כלל משוי הוא והא לא אצטריכא אלא למישרי סוס וחמור באפסר ולומר דשמירה הויא להו. ויש לומר דאין הכי נמי דהכין עיקרא דצריכותא דרבי ישמעאל לפום מאי דדחינן השתא, אלא משום דאמר איהו מאי לאו למעוטי גמל בחטם נקט איהו נמי למעוטי. ואם תאמר אמאי לא מוכח מסוס וחמור דאף על גב דסגי להו באפסר אפילו הכי שריא בהו שיר ופרומביא. יש לומר דסוס דרכו בשיר ואע״ג דסגי ליה נמי באפסר אין זה דרכן של בריות לדקדק ולצמצם בשמירות לומר דאין משמרין אלא במה שהן נשמרין לא יותר מכן ולא פחות מכן, אבל חטם לגמל שמירה יתירתא היא מאד שאין דרכן של בריות לעשותה לסתם גמלים והיא היא דקא מיבעיא לן.}
\textblock{\textbf{קדמיה חמריה דלוי וכו׳.} הקשה ר״ת ז״ל דהכא משמע דמכבדין בדרכים, והכין נמי משמע ביומא בפרק אמר להם הממונה (יומא לז, א) דאמרינן התם שלשה שהיו מהלכין בדרך הרב באמצע גדול בימינו קטן בשמאלו וכו׳, ואילו בברכות פרק שלשה שאכלו (ברכות מו, ב) אמרינן אין מכבדין לא בדרכים ולא בגשרים. ותירץ דהתם בשאין הולכים לענין אחד והלכך אין אחד מהם צריך להמתין ולכבד את חברו, אבל כאן בשהולכין לענין אחד.}
\newchap{פרק \hebrewnumeral{5} במה בהמה}
\clearpage
\newsection{דף נב}
\textblock{}
\textblock{\textbf{אמר ליה הכי אמר אבוך משמיה דשמואל הלכה כחנניה.} תמיהא לי למאי איצטריך ליה להא דשמואל, דהא משמע דלכולי עלמא חמור שעסקיו רעים יוצא בפרומביא, ואפילו גמל אם עסקיו רעים וצריך לכך יוצא אף בחטם. ותדע לך דהא לרב דאסר פרה ברצועה שבין קרניה אקשינן עליה מפרה במוסרה ופרקינן במורדת, ועגלי דרב הונא נמי נפקי באפסריהן ואף על גב דרב הונא תלמידיה דרב דאסר פרה ברצועה דאלמא משום דעגלי סתמן מורדין שרי, וחמור שעסקיו רעים נמי דכוותה. ולוי נמי הכי בעי מיניה חמור שעסקיו רעים מהו לצאת בפרומביא דאלמא אף למאן דאסר בנטירותא יתירתא קא מיבעיא ליה. ויש לומר דרבותא קאמר ליה וקושטא דמילתא קאמר דסבירא ליה דאפילו בעלמא כל נטירותא יתירתא שפיר דמי.}
\textblock{\textbf{אמר ליה אביי אדרבה תסתיים דשמואל דאמר בין לנוי בין לשמר אסור.} איכא למידק מאי שנא דסמיך אהא טפי מההיא דרב הונא בר חייא. ויש לומר משום דבכולי תלמודא אמרינן דרב יהודה דסמכא דדייק וגמיר טפי, וכדאמרינן בפרק קמא דחולין (יח, ב) דאפילו ספיקי דגברי גמיר. ואם תאמר אם כן רב יוסף דגמיר מיניה דרב יהודה כדאיתא התם אמאי שביק דרביה וסמיך אדרב הונא בר חייא. יש לומר משום דקסבר דההיא דרב יהודה נמי לאו למעוטי גמל בחטם אלא למעוטי נאקה באפסר וכדאמרינן לעיל (שבת נא, ב). ולמסקנא נמי דאסיקנא (דרב ושמואל אמרו) [דשמואל אמר] לנוי אסור לשמר מותר אית לן נמי למימר הכין דההיא דארבע בהמות למעוטי נאקה באפסר. ואם תאמר אם כן אביי אמאי דחייה הא דרב הונא בר חייא ומשוי ליה טועה כיון דאיכא לאוקומה להא ולהא. יש לומר דסבירא ליה לאביי דלמעוטי נאקה באפסר לא איצטריכא דפשיטא דכיון דלא מנטרא ביה משוי הוא וכדאמרינן לעיל בגמרא (שם), ולמשרי נמי סוס וחמור באפסר לא משמע ליה דאיצטריכא דפשיטא ליה.\par \textbf{} והר״ז הלוי ז״ל כתב דתרי גווני נטירותא יתירתא הוו, כל נטירותא יתירא שאין בני אדם עושין כן בחול כלל בשבת הוי משוי, וזהו גמל בחטם, ובהא אפילו שמואל מודה כאסהדותיה דמסהיד משמיה דרבי ישמעאל ברבי יוסי, וכל נטירותא יתירתא שמקצת בני אדם מצוין לעשותה בחול בשבת לא הוי משוי, וזהו חתול בסוגר כחנניה, מידי דהוה אלובדקים שיוצא באפסר ויוצא אף בפרומביא ולא אמרינן משוי הוא לדברי הכל, ובפרה ברצועה שבין קרניה איפליגו בה רב ושמואל רב מדמי לגמל בחטם ושמואל מדמי לחתול בסוגר לחנניה ולובדקים בפרומביא לדברי הכל והלכתא בהא כרב דאסר, ואף על פי שדחה הרב אלפסי ז״ל הא דאמרינן הלכה כחנניה לדידן לא מידחיא וכדפרישנא. עד כאן.\par \textbf{} והקשו עליו דהא בגמרא מדמו ההיא דחנניה לגמל בחטם דהא מוקמינא לה כתנאי, ולתנא קמא דחנניה גמל בחטם אסור ולחנניה מותר. ועוד דפרה נמי מקצתן מוציאין אותה במוסירה ואפילו לרב, וכדמתריצנא לה לההיא דפרה במוסירה אליבא דרב במוליכה מעיר לעיר ואי נמי במורדת. ומדעתי שאינו קשה עליו כל כך, דשמא בההיא דחנניה ותנא קמא לא גריס הר״ז הלוי ז״ל כתנאי אלא תנו רבנן ומילתא באנפי נפשה היא דקא מייתי תלמודא, ומקצת ספרים נמי הכין כתיב בהו. וההיא נמי דפרה במוסירה לא מוכחא דמצויה מוסירה קצת בפרה, דההיא היינו במוליכה מעיר לעיר אבל בעיר לא שייכא כלל.\par \textbf{} אלא נראה דהא קשיא עליה, דאם איתא מאי קאמר רב יוסף תסתיים דשמואל הוא דאמר לשמר מותר דאמר רב הונא בר חייא אמר שמואל הלכה כחנניה [דהא בין לרב בין לשמואל קיימא לן כחנניה] ובפרה הוא דפליגי אי דמיא לההיא או לא, ואם תאמר דלרב יוסף לא שאני ליה בין הכין להכין, אם כן תקשי לן נמי לדידיה דשמואל אדשמואל והדרא קושיין לדוכתין.\par \textbf{} והרב אלפסי ז״ל ושאר הגאונים ז״ל פסקו הלכה כרב, מדאשכחן אביי ורבא ורבינא דבתראי נינהו דקא מתרצי הא דפרה במוסירה לאוקומה אליבא דרב שמע מינה דכוותיה סבירא להו. אבל הראב״ד ז״ל כתב דיותר ראוי לפסוק כשמואל, מדחזינן לרבה בר רב הונא דקא פשיט לבעיא דלוי בריה דרב הונא בר חייא מדשמואל ולא פשיט לה מדרב אלמא הלכתא כשמואל. ועוד דהא דתני דבי מנשה מסייעא ליה לדשמואל, דתני דבי מנשה עז שחקוק לה בקרניה יוצאה באפסר בשבת, והא ודאי עז באפסר נטירותא יתירתא. ואי משום פירוקי דאביי ורבא ורבינא דפריקו אליבא דרב, האי סוגיא בעלמא היא ואין דוחין מילתא ברירא ושמעתא ברירא מקמי סוגיא. אלו דברי הרב ז״ל. ונראין דבריו, אף על פי שאותה שהביא מדתני דבי מנשה לא מכרעא כחד ואפשר דכולי עלמא מודו בה, דעז שאני משום דמנתח ומורד והוה ליה כעגלי דרב הונא וכפרה מורדת.}
\textblock{\textbf{שאני פרה דדמיה יקרין.} כלומר: ודרכן של בעלים לשמרה שמירה מעולה כזו אף בחול, וכיון דבחול דרכה בכך   בשבת אינו משוי. ומינה שמעינן לסוס ופרד בזמן הזה, דכיון דדרכן לצאת בחול בפרומביא, אפילו בשבת אינו משוי ואפילו לרב.}
\textblock{\textbf{במתניתא תנא יוצאין כרוכין לימשך.} ושמואל מתרץ לה לטעמיה כרוכין ובלבד שישאר קצתו לימשך בו, ורב הונא מתרץ לה לטעמיה כרוכין רפויין לימשך בהן.}
\textblock{\textbf{בבאין מנוי אדם לנוי בהמה.} פירש רש״י ז״ל: שנטמאו בעודן לאדם. ולחנם פירש כן, דאפילו לאחר שבאו לנוי בהמה מקבלין טומאה, שאין עולין מידי דין קבלת טומאה בלא שינוי מעשה, וכן פירש רש״י ז״ל עצמו בפרק במה אשה יוצאה (לקמן שבת נח, ב בד״ה של בהמה) גבי זוג בהמה ועשאו לדלת. והכין מוכח בפרק קמא דסוכה (יג, ב. יד, א) גבי קצר לאכילה ונמלך עליהן לסיכוך דאפילו לבטל קבלת טומאה מכאן ולהבא לא מהניא מחשבה. וכופת שאור שיחדה לישיבה דאמרינן (חולין קכט, א) דע״י יחוד טהורה מלקבל טומאה, התם נמי בשיש בו שינוי מעשה.}
\textblock{\textbf{וטובלין במקומן.} איכא למידק אמאי לא אקשי הכא טבילה מאן דכר שמה, כדאקשינן בריש פרק דבמה אשה יוצאה (לקמן שבת נז, א) גבי הא דתנן לא תטבול בהם עד שתרפם. ותירץ ר״ת ז״ל (בע״א בתוד״ה וטובלי) דכיון דאיירי הכא בדין של כלים איזה מותר לצאת בו ואיזהו אסור אגב גררא נמי איירי בדין טבילתן, דהכי אורחא דתלמודא כההיא דהקיטע יוצא בקב שלו ואם יש לו בית קבול כתיתין טמא (דלקמן (שבת סה, ב). סו, א), אבל בפרק במה אשה יוצאה דמיירי בדין כלים מאי שייך למיתני התם דין טבילת האשה.}
\textblock{\textbf{וטובלין במקומן והאיכא חציצה.} פירש רש״י ז״ל: שהטבעת תקועה בשיר בחוזק ואין המים נכנסין שם. ויש לומר לפי פירושו שפעמים מוציאין הטבעת מתוכו ומשתמשין בו לעצמו, דאם לא כן הרי הכל כלי אחד ואין כאן חציצה. אלא דקשיא לי דבמקומן אצואר בהמה משמע. והכי איתא בירושלמי (פ״ה ה״א) דאקשינן התם איתמר לא תטבול בהם עד שתרפם והכא איתמר הכין ופריק לא קשיא כאן ברפוים כאן באפוצים.}
\textblock{\textbf{הכא במאי עסקינן בשריתכן.} ואם תאמר לוקמה ברפוין וכדאמרינן במסקנא בחלולין. תירץ רבנו תם ז״ל דאגב אורחיה ניחא ליה לאשמועינן דכרב יוסף סבירא ליה, ואפילו למסקנא דאמרינן דאתיא נמי כרבי יצחק נפחא, איכא למימר דהיא גופא קא משמע לן דמעשה לתקן לאו מעשה הוא.}
\textblock{\textbf{היא של אלמוג וחותמה של מתכת טהורה.} דהוי ליה פשוטי כלי עץ. ואם תאמר והא איכא בית קבול מושב החותם. יש לומר בית קבול העשוי למלאות הוא.}
\clearpage
\newsection{דף נג}
\textblock{\textbf{אלא לאו שאינה קשורה לו מערב שבת.} פירש מורי הרב ז״ל: כדי שלא ישתמש בצדדי הבהמה. וכן נראה מן הירושלמי (פ״ה ה״ב). ורש״י ז״ל פירש (לקמן שבת נד, ב ד״ה כדאמרן) בענין אחר.}
\textblock{\textbf{לא יצא הסוס בזנב שועל ולא בזהרורית שבין עיניו.} הכא משמע דכל דלנוי בהמה אסור, וכדאמרינן נמי לעיל (שבת נב, א) גבי פרה ברצועה שבין קרניה דבין לרב בין לשמואל לנוי אסור. ואיכא למידק מדאמרינן לקמן (שבת נד, ב) אין חמור יוצא בזוג אע״פ שהוא פקוק משום דמיחזי כמאן דאזיל לחינגא, אלמא משום דמיחזי כמאן דאזיל לחינגא אסור הא משום נוי מותר. ויש לומר דלרבותא נקט לההוא טעמא, דאפילו תמצא לומר דלנוי מותר הכא אסור. ויש מפרשים דלנוי הרגיל לכל בחול אפילו בשבת שרי והיינו טעמא דזוג, אבל רצועה בקרני פרה וזהרורית שבין עיני הסוס אינו רגיל בכל אלא מקצת אנשים עושים כן לעתים ולפיכך אסור דהוי ליה כמשוי.}
\textblock{\textbf{לא יריצנה בחצר בשביל שתתרפה ורבי יאשיה מקל.} שמעתי משם הראב״ד זכרונו לברכה דמדקתני רבי יאשיה מקל ולא קתני רבי יאשיה מתיר שמע מינה אין מורין כן. אבל ה״ר אשר פירש דאדרבא מדדרש רבא הלכה כרבי יאשיה, שמע מינה הלכה ומורין כן דאי לאו הכי לא הוה דריש לה, וכדאמרינן בפרק קמא דחולין (טו, א) כי מורי להו רב לתלמידיה מורי להו כרבי מאיר כי דריש להו בפרקא דריש להו כרבי יהודה.}
\clearpage
\newsection{דף נד}
\textblock{\textbf{ליבש אבל לא לחלב.} פירש מורי הרב ז״ל: מפני שאין דרכן לצאת כן בחול מפני שבשעה שהחלב מתכנס בדדין הן חולבין אותן, אבל בשבת כדי שלא יטפטף ויאבד קושרין להם בגד כדי שלא יטפטף והוה ליה משוי.}
\textblock{\textbf{כי אתא רבין אמר רבי יוחנן הלכה כתנא קמא.} פירש רש״י ז״ל: כתנא קמא דמתניתין דשרי לגמרי. אבל רבנו האי גאון ז״ל פירש כתנא קמא דברייתא דלא שרי אלא ליבש אבל לא לחלב. וכן נראה.}
\newchap{פרק \hebrewnumeral{6} במה אשה}
\clearpage
\newsection{דף נז}
\textblock{}
\textblock{ מתני׳:\textbf{ לא בחוטי צמר ולא בחוטי פשתן ולא ברצועה שבראשי הבנות. } רבותינו בעל התוס׳ ז״ל פירשו דלא מיירי בשקלעה בהן שערה, דאם כן מאי איריא משום דילמא מתרמיא לה טבילה של מצוה ואתיא לאתויינהו תיפוק ליה משום שהיא סותרת, וכדאמרינן בפרק המצניע (לקמן שבת צה, א) הגודלת חייבת משום בונה וכיון שכן הסותרת חייבת משום סותר, ואפילו למאן דאמר (לקמן שבת צד, ב) דאינה חייבת משום בונה מכל מקום מדרבנן איכא, ולא עוד אלא כיון דאילו סותרת חייבת לא חיישינן לטבילה דהא לא שלפה, והלכך יוצאה בהן בשקלעה שערה.}
\textblock{ גמרא:\textbf{ דילמא מתרמיא לה טבילה של מצוה ושריא להו ואתיא לאתויינהו ד׳ אמות ברשות הרבים.} ואם תאמר אם כן אפילו בחגור שבמתניה לא תצא שמא תתירנו ותוליכנו. יש לומר שלא אסרו להן חכמים אלא תכשיטין קטנים וכיוצא בהן שאינה מרגשת בהן בכל שעה וחיישינן דילמא תשתלי אבל דברים גדולים לא. וכן כתבו בתוספות.\par \textbf{} עוד אמרו בתוספות דכל מה שהתירו חכמים לצאת בו מותר להתיר ולקשור אפילו ברשות הרבים ולא חיישינן דילמא אתי לאתויינהו. וטעמא דמילתא, דכיון שהתירו חכמים לצאת בהם מפני שאין כאן צורך התרה, לא שייך בהו שלא תתיר כדי שלא תוליך, דאי אמרת שהיא שכחה אף היא מתרת ומוליכה דהא שכחה היא, ואי אינה שכחה אף היא אינה מוליכה. והיינו דתניא גבי קמיע מומחה לקמן (שבת סא, א) וקושר ומתיר אפילו ברשות הרבים. והא דאמרינן התם (בעמוד ב) אי אמרת קמיע יש בו משום קדושה זימנין דמצריך לבית הכסא ואתי לאתויינהו ד׳ אמות ברשות הרבים, הכי קאמר: וכיון דאפשר דעל כרחיה אתי למישלף דילמא אף הוא שוכח ומעביר.}
\textblock{ גירסת הגאונים ז״ל:\textbf{ הכא בקרטלא עסקינן.} ובמשנתנו גרסינן ולא בקטלא, דקטלא איכא משום תכשיט ולא משום דילמא מזדמנא טבילה של מצוה מדלא ערבה ותני לה בהדי חוטי צמר וחוטי פשתן, וקרטלא לית בה משום תכשיט אבל אית בה משום טבילה של מצוה. אבל רש״י ז״ל גריס קטלא. ופירושו כמו השנוי במשנתנו. ובודאי דבקטלאות יש בהן משום חציצה, וכדתניא בתוספתא דמקואות (פ״ו ה״ד עיי״ש) השירים והטבעות והקטלאות אוצין חוצצין רפוין אינן חוצצין. ואם תאמר אם כן אמאי לא ערבה ותני לה בהדי הנהו דרישא דמתניתין, ואם תאמר דקטלא אית בה משום תרתי משום חציצה ומשום תכשיט ולהכי תני לה בסיפא ולומר דאפילו משום תכשיט אסירא, אם כן מאי קא מקשה מיניה הכא דילמא הכא לאו משום חציצה אלא משום תכשיט. ויש לומר דתרי גווני קטלאות נינהו, חדא דנסכא וההיא הויא תכשיט ולית בה משום חציצה דמשום דקשה לא מיהדקא לה משום דכאיב לה והיינו ההיא דמתניתין, ואיכא דרצועה וההיא מהדקא לה כדי שתראה בעלת בשר דמשום שהיא רכה לא כאיב לה וההיא לית בה משום תכשיט ואית בה משום חציצה, והכא משום דתני לה בהדי חוטין שבאזן דלית בה משום תכשיט ולא איצטריך אלא לאשמועינן דלא צריכה למישלפה בעידן טבילה, שמע מינה דאבל לא בקטלא דקתני היינו נמי משום טבילה ולא משום תכשיט. וכן כתבו בתוס׳.}
\textblock{\textbf{אמר רב נחמן אמר שמואל מודים חכמים לרבי יהודה בחוטי שער.} ודוקא בראשי הבנות אינן חוצצין לפי שאין שער נקשר היטב על גבי שער, אבל על גבי בשר נקשר היטב וחוצצין דקשה על גבי רך חוצץ וכדאמרינן לעיל (בעמוד א). וקשיא לי דהא אמרינן בפרק בתרא דנדה (סז, א-ב) נימא אחת קשורה חוצצת שלש אינן חוצצות שתים איני יודע. וי״ל דה״נ בחוטין המשולשין. וכן מצאתי בירושלמי (דפרקין ה״א עי״ש), דגרסי׳ התם: שמואל אמר לית כאן של שער על דרבי יהודה אלא של צמר הא של שער ד״ה אינן חוצצין, רבי בא בשם רבי זעירא בשם רבנין נימא אחת חוצצת שתים ספק שלש אינן חוצצות, ר׳ יוסי בעי קשר נימא אחת לחברתה אחת היא לשתים שתים הן שתים לשלש שלש הן.}
\textblock{\textbf{כבלא דעבדא.} פירוש: ושפחות נמי יוצאות בהן, דהא בלא תצא אשה קתני לה. ואם תאמר אי כבלא דעבדא למה התירו (לקמן שבת סד, ב) בחצר, דבשלמא אי כיפה של צמר התירו לה בחצר משום שלא תתגנה על בעלה, אבל כבלא דעבדא אמאי. תירצו בתוס׳ דפעמים שאינו עושה רצון רבו וסבור רבו דמשום דלית ליה חותם הוא רואה עצמו כבן חורין ואתי לאינצויי. ויש מפרשים דבתכשיטין גזרו שאין דרך לשלפן מחצר לרשות הרבים [אבל] בסימני עבדים לא גזרו. וכן ודאי נראה, דחותם גנאי הוא לו וכל שאתה מצריכו להתירו נהנה הוא וזכור הוא ליטלו.\par \textbf{} הקשה הרמב״ן ז״ל למאן דאמר כיפה של צמר תנן אמאי אסורה והא תחת שבכה היא ושפיר קאמר רבי שמעון בן אלעזר, ותנן יוצאה אשה בטוטפת ובסרביטין הקבועין בה, וטעמא משום דלא מישלפה שבכה שאינה פורעת ראשה. ותירץ שהכיפה אינה כולה תחת השבכה אלא מקצתה כנגד גובה הראש ואפשר לה לשלוף הכיפה ועדיין השבכה מחפה את ראשה, ורבי שמעון מתיר כיון דעל השער ממש היא למטה מן השבכה.}
\textblock{\textbf{אין בה משום כלאים.} ומדקתני אין בה משום כלאים סתם ולא קתני אין בה משום כלאים דאורייתא, שמע מינה דאין בה משום כלאים כלל ואפילו דרבנן. ולא משום דאינו אריג דהא קתני (כלאים פ״ט מ״ט) הלבדין אסורין מפני שהם שועין ואף על פי שאינן טווין וארוגין, אלא טעמא דמילתא מפני שהם קשין וכל שהוא קשה ואינו כלאים דאורייתא לא גזרו ביה רבנן ואפילו בהעלאה אבל כלאים דרבנן ורכין      בהעלאה, אבל להציעו תחתיו מותר ובלבד שלא יהא בשרו נוגע בהן, קשים וכלאים דאורייתא בהעלאה אסור דבר תורה, דבגדי כהונה קשים הן ואפילו הכי לכהן הוא דשרי להו רחמנא לכולי עלמא אסור, ומשום הכי אמרינן בפרק קמא דערכין (ג, ב) ובמנחות (מג, א) דאיצטריך למיתני הכל חייבין בציצית כהנים לוים וישראלים וכהנים איצטריך ליה דכתיב (דברים כב, יא) לא תלבש שעטנז וסמיך ליה (שם פסוק יב) גדילים תעשה לך סד״א כל שישנו בלא תלבש שעטנז ישנו בגדילים והני כהני הואיל ואשתרי כלאים לגבייהו ליתנהו בציצית קמ״ל, אבל להציע תחתיו מותרין הואיל וקשין הן, אבל היכא דאית בהו תרתי כלאים דרבנן וקשין מותרין הן לגמרי וכדאמרינן בפ״ק דביצה (טו, א) האי נמטא גמדא דנרש שריא. ושם כתבתי יותר בס״ד.}
\clearpage
\newsection{דף נח}
\textblock{\textbf{ומי אמר שמואל הכי והאמר שמואל וכו׳, לא קשיא הא דעבד ליה רביה הא דעבד איהו לנפשיה.} והוא הדין דהוה מצי לתרוצי כאן בשל טיט כאן בשל מתכת וכדאיתא בסמוך, ואי נמי כאן בכסותו כאן בצוארו, אלא דניחא ליה לאוקומי למתניתין בכל החותמות ובכל מקום שהוא נתון בין בכסותו בין בצוארו.}
\textblock{ הא דאקשינן:\textbf{ במאי אוקימתא להא דשמואל בדעביד ליה רביה בכסותו אמאי לא.} ואוקימנא משום דילמא מפסיק ומקפל. הוא הדין לדעת המקשה דלית ליה פתרי אלא בהכין, אלא דסבירא ליה למקשה דבשלמא אי עביד ליה איהו לנפשיה איכא למיגזר דילמא מפסיק ומקפל, משום דמימר אמר אי לא מקפלנא השתא אמר רבי דאימלכי ושקלי ואי מקפלנא לא רמיא אנפשיה דמימר אמר לאו לכסויי עביד דהא מנפשיה עביד ליה, אבל אי עביד ליה רביה אי מפסיק לא מקפל ליה דמאי אהני ליה דאי חזי ליה רביה חשיד ליה דמשום דבעי לכסויי הוא דמקפל והלכך לא מיקפל, ואהדר ליה דאף בדעבד ליה רביה נמי איכא למיחש להכי.}
\textblock{\textbf{כולהו רבנן לא ליפקו בסרבלי חתימי.} פירש רש״י ז״ל שהוא סימן לריש גלותא. אבל רבנו האי גאון ז״ל פירש בענין אחר, וז״ל: כך פירש גאון אבי אבינו, שהיו המלכים טובעין בקרן כל טלית חשובה מטבע של מלך כדי שלא להבריחה מן המס וכדי שיודע שנלקחה מנתו ממנו, ע״כ אם תפול בשבת אפשר שיתירא ויקפל טליתו ויניחה על כתפו, ומי שאינו מתירא מן המלכות אינו פוחד. עד כאן. וזה נראה יותר.}
\textblock{\textbf{כל דבר אשר יבא באש (במדבר לא, כג) אפילו דבור אסור.} ואם תאמר והא קרא בגיעולי עכו״ם כתיב. איכא למימר אם אינו ענין לגיעול דלענין איסורי עכו״ם אין לחלק בין יש להן עינבל לשאין להן עינבל תנהו לענין טומאה, ואסיפיה דקרא קאי אך במי נדה יתחטא.}
\textblock{\textbf{הואיל והדיוט יכול להחזירו.} פירש רש״י ז״ל: לפיכך לא פרחה ממנו טומאתו הישנה, אבל אינו מקבל טומאה מכאן ולהבא. ואיצטריך לפרש כן משום דקתני אין להם עינבל טהורין, דאלמא אין טומאה יורדת להן כל שאין להן עינבל. ובאידך נמי דרבא דאמר הואיל וראוי להקישו על גבי כלי חרס פירש כן ואמר דהוי כניקב בפחות ממוציא רמון שלא פרחה ממנו טומאה ישנה וטהור מכאן ולהבא.\par \textbf{} אבל בתוס׳ לא הודו לו, אלא טומאתן עליהן לגמרי ואפילו לקבל טומאה מכאן ולהבא, דכיון שנגמרה מלאכתו שהרי היה להם עינבל וגם עכשיו הדיוט יכול להחזירו לאביי או שהוא ראוי להקישו על גבי כלי חרס לרבא לא יצאו מתורת כלי, אבל כל שלא היה להן עינבל אכתי לא ירדו לידי כלי וגולמי כלי מתכות הן ולפיכך טהורין לגמרי. והקשו בתוס׳ לפירוש רש״י ז״ל דאי בטומאה ישנה לא פליגי בה אדאביי דכל שהדיוט יכול להחזירו לא פרחה ממנו טומאה ולא פליגי אלא בלהבא, וכדמוכח בפרק שלשה שאכלו (ברכות נ, ב) דגרסינן התם מטה שאבדה חציה או שנגנבה חציה או שחלקוה אחין או שותפין טהורה החזירוה מקבלת טומאה מכאן ולהבא (כלים פי״ח מ״ט), ומפרש רבא התם דלמפרע לא משום דפרחה ממנו טומאה, והיינו דוקא באבדה [אבל לא אבדה לא פרחה ממנו טומאה] אפילו לרבא הואיל והדיוט יכול להחזירה אע״פ שאינה ראויה עכשיו למלאכתה הראשונה ואע״ג דהכא בעי הוא שיהא ראוי למלאכתו הראשונה, אלא על כרחין שלטומאה ישנה אף רבא מודה דלא פרחה ממנו כל שהדיוט יכול להחזיר אבל מכאן ולהבא הוא דבעי שיהא ראוי למלאכה הראשונה, אלא ודאי הכא לקבל טומאה מכאן ולהבא אמרו כדאמרן.}
\textblock{ הא ד\textbf{מותיב רבא הזוג והעינבל חבור.} הרבה פירושים נאמרו בה, והמחוור שבכולן הוא מה שפירש רבנו שמואל ז״ל. והכי קאמר: אביי אמר הואיל והדיוט יכול להחזירו הרי הוא כאילו לא ניטל, ואע״ג דהשתא בלא עינבל לא חזי למידי כיון דהדיוט יכול להחזירו ודבר קל הוא לחברם ולהפרידם כמו שירצה כמאן דמחברי דמו ואם נטמא העינבל נטמא אף הזוג כאילו הן מחוברין, ונפקא מינה דלכשיתחברו יטמא הזוג את הנוגע, ומיהו בלא חבור לא יטמא כי אם העינבל לבדו שנגע בטומאה. והכי מוכח לקמיה דקא מותיב ליה רבא לאביי וכי תימא הכי קאמר אע״ג דלא מחבר כמאן דמחבר דמי מכלל דאביי הכי סבירא ליה, דכיון דתלי טעמא בחזרת הדיוט אלמא סבירא ליה דלא חזי השתא למידי אלא משום סברא דחזרה מקבל טומאה מאחר דלא חזי למידי בלא חזרה. מותיב רבא הזוג והעינבל חבור שאם נטמא זה נטמא זה והני מילי כשמחוברין אבל שלא בשעת חבור אם נטמא זה לא נטמא זה ולא חיישינן להדיוט יכול להחזירו דהשתא מיהא לאו מחוברין נינהו, ומלשון המשנה הזאת אין לשמוע אם בשעת חבור אי לא כדפריך וכי תימא וכו׳ אבל דעתו של רבא להוכיח עליה מברייתא דלקמן דבשעת חבור מיירי ותקשי לאביי, ומינה הוה מצי לאותוביה לרבא ומשום       מיירי בזוג ועינבל בהדיא אלימא ליה לאקשויי מינה לאביי דאיירי נמי בזוג ועינבל. וכי תימא הכי קאמר הזוג והעינבל לעולם חבור אפילו כשנתפרדו ואע״ג דלא מחברי כמאן דמחברי דמי ואם נטמא זה נטמא זה משום טעמא דהדיוט יכול להחזירו ומחוסר חזרה לאו מחוסר מעשה הוא, והתניא וכו׳ מספורת של פרקים ואיזמל של רהיטני שתי ברייתות הן במסכת כלים (בתוספתא כלים ב״מ פ״ג ה״ב וה״ו) כל אחת בפני עצמה, וברישא דידהו קתני בהדיא דכשאין מחוברין מקבלת טומאה כל אחת ואחת בפני עצמה, וגזרו על טומאה שלא בשעת מלאכה שיהא חבור לטומאה שיטמא זה בשביל זה משום דדמי לשעת מלאכה, אבל שלא בחיבורן שמעינן דאפילו גזירה דרבנן ליכא ואם נטמא זה לא נטמא זה, והעינבל לבדו לא יקבל טומאה כלל אע״ג דהדיוט יכול להחזירו, וקשיא לאביי. אלא אמר רבא הואיל וראוי וכו׳, ומספורת ואיזמל נמי בלא חבור ראוין למעין מלאכה ראשונה ומקבלין טומאה כל אחד ואחד בפני עצמו כדתניא בתוספתא.}
\clearpage
\newsection{דף נט}
\textblock{\textbf{ורבי יוחנן אמר הואיל וראוי לגמע בו מים.} פירשו בתוס׳: אע״פ שלא יחדו לגמע בו אלא מן הסתם כיון שהוא ראוי עדיין למלאכה אחרת מקבל טומאה אע״ג דלא יחדו. ותדע לך מדקאמר הואיל וראוי דאלמא בראויות תליא מלתא ואע״ג דלא יחדו. ועוד דאי ביחדו מאי טעמא דרבי יוסי ברבי חנינא דלא ליטמא משום גימוע. ועוד מה לי לא הוה ליה עינבל מעולם מה לי הוה ליה, הואיל והוא יחדו לגמוע בכל ענין הוה ליה לקבל טומאה.\par \textbf{} וקשיא להו דלקמן בפרק המצניע (שבת צה, ב) משמע דבעינן דוקא שיחדו למלאכה אחרת הא לא יחדו אע״ג דחזי ליה לא מקבל טומאה, דאמר רבא חמש מדות בכלי חרס וכו׳ ניקב כמוציא זית טהור מלקבל בו זיתים ועדיין כלי הוא לקבל בו רמונים, ופירש רש״י ז״ל: טהור הוא מלקבל טומאה עוד משום תורת סתם כלי חרס ששיעורו בזיתים ועדיין כלי הוא לקבל בו רמונים שאם יחדו שוב לרמונים מקבל טומאה מכאן ולהבא או אם תחילה היה מיוחד לרמונים לא נטהר, אלמא בעינן יחוד. ותירצו דהכא שאני הואיל ומעיקרא הוי כלי גמור וגם עתה יכול להחזיר בו העינבל איכא למימר דחשיב ליה כלי על ידי דבר מועט שעדיין ראוי לגמע בו מים קודם חזרת עינבל הלכך אפילו בלא יחוד סגי ליה.}
\textblock{\textbf{רבי מאיר סבר משוי הוא.} כלומר: מתוך כובדו. ואי נמי שאינו עשוי לתכשיטות אלא להראות עושר.}
\textblock{\textbf{ושמואל אמר מאן דרכה למיפק בכלילא אשה חשובה ואשה חשובה לא שלפא ומחויא.} ולאו למימרא דשמואל כרבי אליעזר דעיר של זהב סבירא ליה אלא מיסבר סבר דבהא כולי עלמא מודו בה, דכל היכא דלא הוי חשוב כולי האי לא ליפוק ביה אפילו חשובה דאי כולהו נפקי ביה גזרינן ביה דילמא נפקא ביה אשה חשובה ואשה שאינה חשובה נמי ושלפא ומחויא, וכל היכא דחשוב טובא כעיר של זהב סברי רבנן דכיון דחשוב כולי האי אף אשה חשובה נמי שלפא ומחויא ורבי אליעזר סבר כיון דחשובה כולי האי דנפקא בעיר של זהב אף היא חשובה לנהוג שררה בעצמה ולא שלפא ומחויא, אבל חשוב ואינו חשוב ככלילא דנסכא קסבר שמואל דכיון דחשובה כל כך דלא נפקא בה אלא אשה חשובה ואינה חשובה כל כך שתהא שולפת ומראה לכולי עלמא שפיר דמי.}
\textblock{\textbf{אמר מר יהודה אמר רב ששת קמרא שרי.} פירוש המיינא, מדקאמרינן בסמוך קמרא עלוי המיינא מאי אמר ליה תרי המייני קאמרת דאלמא קמרא נמי היינו המיינא, והמיינא היינו חגור וכדאמרינן בפרק קמא (לעיל שבת ט, ב) הני חברין בבלאי מכי שרו המיינא לא מטרחינן להו. וכן פירש רבנו האי גאון ז״ל.}
\textblock{\textbf{איכא דאמרי דאנסכא ואמר רב ששת מידי דהוה אאבנט של מלכים.} מכאן הביא ר״ת ז״ל ראיה דלא אסרו תכשיטין באיש אלא באשה, שהרי קמרא דאנסכא תכשיט(ין) הוא וקא שרי ליה משום [מידי] דהוי אאבנט של מלכים, והיינו נמי דבמתניתין לא איירי אלא בתכשיטי נשים. אלא דאיכא למימר שדברו בהווה דדרך נשים להתקשט ואין האנשים מתקשטים. ועוד ראיה מדקאמרינן לקמן (שבת סא, א-ב) גבי קמיע מומחה ובלבד שלא יקשרנו בשיר ובטבעת אלמא שיר וטבעת עצמן שרי. ונראה לי דאין זו ראיה, דאילו היה מותר לקשור בשיר ובטבעת היו טפילה לקמיע ושרי מידי דהוה אצלוחית של פליטון (לקמן שבת סב, ב). אבל בירושלמי (דפרקין ה״א) מפורש כדבריו, דגרסי׳ התם תכשיטין למה הן אסורין אמר ר׳ אבא ע״י שהנשים שחצניות היא מתרתן להראותן לחברתה והיא שכחה ומהלכת בהן ד׳ אמות. ואע״פ שאמרו שם בירושלמי במקום אחר (בסוף ההלכה הנ״ל) והאיש על ידי שאינו שוחץ מותר, נשמעינה מן הדא מעשה ברבן גמליאל ברבי שירד לטייל לתוך חצירו ומפתח של זהב בידו וגערו בו חביריו משום תכשיט הדא אמרה העשוי לתכשיט אסור הדא אמרה העשוי לכך ולכך הדא אמרה אחד האיש ואחד האשה הדא אמרה אפילו במקום שאמרו לא תצא ואם יצאת אינה חייבת אסור לצאת בו לחצר, דאלמא אפילו באיש גזרו. איכא למימר דפליגא אההיא דר׳ אבא, ובשל רבנן הלך אחר המיקל. ועוד דפשטא דמתניתין ושמעתתא דגמרין הכין דייקי, הלכך באיש מותר. והא דאמרינן לקמן (שבת סב, א) גבי טבעת שיש עליה חותם ושאין עליה חותם וחילופיהן באיש, לאו אחיוב חטאת ופטור אבל אסור קאי דנשמע מינה דטבעת שיש עליה חותם באיש פטור אבל אסור, אלא אחיוב חטאת בלחוד קאי, כלומר: חילופיהן      דכשאין בה חותם הוי משוי וחייב חטאת, אבל כשיש עליה חותם הוי תכשיט והלכך מותר אפילו לכתחילה. וטעמא משום דאנשים דעתם מיושבת עליהם ואינם מתפארים להראות תכשיטיהם כנשים. ואי קשיא לך הא דתנן (לקמן שבת סו, ב) ובני מלכים בזוגין וכל אדם אלא שדברו חכמים בהווה ואוקימנא בגמרא (לקמן שבת סז, א) באריג בכסותו הא לאו הכי אסור. יש לומר דמתניתין בקטנים שדעתן קלה עליהן כנשים, אבל גדולים אפילו אינו אריג בכסותן שרי, ומאי כל אדם דקתני כל אדם (גדול) [קטן]. אבל בירושלמי (דפרקין ה״ט עי״ש) אמרו מהו כל אדם הכל במשמע בין קטנים בין גדולים. ולפי זה יש לומר דטעמא דזוגין משום דחיישינן דילמא מחייכו עלייהו ואתי למישלף ולאתויי. וכן התיר הנגיד ז״ל בתכשיטי אנשים. אבל רבנו האי גאון ז״ל ור״ח ז״ל אסרו, וכך אמרו: והדין כך הוא שיהיה האיש מותר לצאת בטבעת שיש עליה חותם והאשה בטבעת שאין עליה חותם מפני שתכשיט הוא להם וכל תכשיט שרי, אלא מפני גזירה דרבא דאמר (לקמן שבת סב, א) פעמים שהאשה נותנת לבעלה והאיש נותן לאשתו ולפיכך אסרו אחד האיש ואחד האשה. והוא מן התימה שלא נאמרו לקמן דברי רבא אלא לתרוצא למתניתין דקתני בטבעת שיש עליה חותם חייבת חטאת ואמרינן בגמרא דאיש כשאין עליה חותם חייב חטאת, ואקשינן והא הוצאה כלאחר יד הוא, ופריק רבא דלאו כלאחר יד הוא להם דפעמים שהאיש נותן לאשתו להניחו בקופסא והאשה נותנת לבעלה להוליכו לאומן והן נושאין באצבע, ולא בא רבא לאסור שאין עליה חותם באשה ושיש עליה חותם באיש מההוא טעמא. ועכשיו שלא הוזהרו בכך בין אנשים ואפילו נשים, הנח להן לישראל מוטב שיהו שוגגין ואל יהו מזידין.}
\textblock{\textbf{הא יש עליה חותם חייבת.} הוא הדין דהוה מצי לאתויי מסיפא (לקמן שבת סב, א) דקתני בהדיא ולא בטבעת שיש עליה חותם ואם יצתה חייבת חטאת, אלא אורחא דתלמודא הוא בהרבה מקומות דכי קאי ברישא דייק מרישא ושביק סיפא דמפרשא בהדיא ואי נמי מסיפא אע״ג דמפרשא ברישא בהדיא. ולי נראה דהכא ניחא ליה למידק מרישא דבטבעת שאין עליה חותם פטורה אבל אסורה דאלמא תכשיט פטור אבל אסור, הלכך כיון דלבסוף לא סגיא דלא דייק מרישא ניחא ליה טפי למידק כולה מילתא מרישא.}
\textblock{ הא דקתני:\textbf{ רבי נחמיה מטמא.} אסיפא דוקא קאי, אבל ארישא דהיינו היא של מתכת וחותמה של אלמוג מטהר, דלדידיה הכל הולך אחר החותם. ובדין הוא דליתני ורבי נחמיה מחליף, והכי קתני בירושלמי בפרקין (ה״ג). והכין תני לה בתוספתא בדוכתא (כלים ב״מ פ״ג ה״ח) בהדיא.}
\clearpage
\newsection{דף ס}
\textblock{\textbf{ותיהוי כבירית טהורה ותשתרי.} לפי פירוש רש״י ז״ל נראה דבכדי נקט הכא טהורה, שאין עיקר הקושיא אלא תיהוי כבירית ותשתרי. ואינו נכון, דאם כן אמאי נקט טהורה ולאו אורחא דתלמודא הכין. ויש מפרשים דתרתי פריך, כלומר: תיהוי כבירית לגמרי שתהא טהורה שאינה לא כלי ולא תכשיט אלא לצניעותא בעלמא עבידא ואנן אמרינן בריש פרק במה בהמה (לעיל שבת נב, ב) דלענין טומאה זה וזה שוין, ותשתרי נמי לצאת בה הואיל ולצניעותא עבידא לא שלפא. ואינו נכון בעיני, דאי טהורה אמחט קאי הוה ליה למימר ותהוי כבירית טהורה ושריא אי נמי ותטהר ותשתרי, אלא אבירית משמע.\par \textbf{} ובתוס׳ פירשו דהכי קאמר: ותיהוי כבירית שהיא טהורה משום שאינה תכשיט אלא טבעת הכלים ועל ידי כך מותר לצאת בה בשבת דליכא למיחש למישלף ואחויי, הכא נמי שהיא טהורה שאינה אלא כטבעת הכלים תשתרי דאינה תכשיט. ואם תאמר אם כן היכי אמרינן לעיל דלענין טומאה דא ודא אחת היא, והא איכא לאיפלוגי בין אוגרת בה שערה לשאינה אוגרת. מסתברא דמשום דהא לא מיתניא הכין בברייתא לא מותיב לה, ואיהו נמי בשאין אוגרת בה שערה אהדר ליה וא״נ בשאינה אוגרת בה שערה בעא מיניה וכדאמרינן גבי אצבע קאמר ליה.}
\textblock{ והא דאמרינן:\textbf{ (הא תינח בחול) בשבת מאי איכא למימר.} פירש רש״י ז״ל: הרי בשבת אינה חולקת בה שערה. והקשו עליו בתוס׳ ז״ל חדא דמנא ליה דאסור לחלוק שער בשבת אדרבא משמע דשרי דהא תנן (נזיר מב, א) נזיר חופף ומפספס והוא הדין בשבת דקיימא לן כרבי שמעון דאמר דבר שאין מתכוין מותר, ועוד דאטו משום דאסור לחלוק בה שער הוא דאין יוצאין בה הא אילו היה מותר לחלוק בה שריא אדרבא הוה ליה כלי ואסור לצאת בו. אלא הכי פירושא: בשבת למאי חזיא והלא אינה תכשיט. ומפרקינן כמין טס יש לה בראשה והיא נוי לה שמניחתה כנגד פדחתה.}
\textblock{\textbf{ושמעו קול מעל גבי המערה.} פירוש: קול סנדל מסומר.}
\textblock{\textbf{לא סנדל מסומר ולא מנעל שאינו תפור.} ואם תאמר והא קתני התם (ביצה יד, ב) כל שנאותין בו ביום טוב משלחין אותו, ואסיק בגמרא (שם טו, א) כל שנאותין בו בחול משלחין אותו ביום טוב, וסנדל המסומר הא נאותין בו בחול. יש לומר דבסנדל המסומר החמירו הואיל ובאת בו תקלה ונגזרה בו גזירה, וזה הכלל דכל שנאותין בו אמנעל שאינו תפור קאי.}
\textblock{\textbf{אין בין יום טוב לשבת אלא אוכל נפש בלבד.} ואע״ג דבפרק כל הכלים לקמן (שבת קכד, א-ב) ובמסכת יום טוב פרק משילין (ביצה לז, א) מוקמינן לה כבית שמאי דאילו לבית הלל הא אמרי (ביצה יב, א) מתוך שהותרה הוצאה לצורך הותרה נמי שלא לצורך, מכל מקום טפי דמי יום טוב לשבת הואיל ולא הותרה בו מלאכה בשאר דברים שאין בהם צורך אוכל נפש.}
\clearpage
\newsection{דף סא}
\textblock{\textbf{אלא אפילו למאן דאמר שבת זמן תפילין הוא לא יצא דילמא מיפסקי ליה רצועה ואתי לאתויי ארבע אמות.} רש״י ז״ל לא גריס דילמא מיפסקא רצועה, דבלאו הכי איכא למיגזר דילמא איצטריך לבית הכסא, וכדתניא לקמן (שבת סב, א), וכדאמרינן בסמוך (לקמן ע״ב) לגבי קמיע של כתב, אלא הכי גריס דילמא אתי לאתויינהו ד׳ אמות ברה״ר [ו]פירש משום דזימנין דמיצריך לבית הכסא. אבל בתוס׳ יישבו גירסת הספרים, דנפקא מינה לתפילין של יד דליכא משום בית הכסא הואיל ומחופות עור, דתפילין של ראש היינו משום שי״ן דכתיב בהו כדאיתא לקמן (שבת סב, א).}
\textblock{\textbf{אלא כיון דאיתמחי גברא אף על גב דלא איתמחי קמיעא.} וכל שכן איתמחי קמיעא ואף על גב דלא איתמחי גברא. ואיתמחי גברא היינו שריפא שלשה אנשים בשלשה קמיעין מוחלקין, אבל לאדם אחד בשלשה מיני קמיעין סלקא בסמוך (בעמוד ב) בתיקו משום דדילמא האי גברא הוא דמקבל כתבא. ואיתמחי קמיע היינו שריפא שלשה בני אדם או אפילו שריפא לאדם אחד שלשה פעמים, וכדתניא איזהו קמיע מומחה כל שריפא ושנה ושלש. ואף על גב דאמר רב פפא (שם) פשיטא לי תלתא קמיעי לתלתא גברי תלתא תלתא זימני איתמחי גברא ואיתמחי קמיעא, משום גברא נקט הכין הא משום קמיעא אפילו לאדם אחד סגי.}
\clearpage
\newsection{דף סב}
\textblock{\textbf{ואמר אביי דלי״ת ויו״ד של תפילין הלכה למשה מסיני.} לא גרסי ליה בתוס׳ כלל, דדלי״ת ויו״ד לאו הלכה נינהו. ותדע לך מדאמרינן בפרק במה מדליקין (לעיל שבת כח, ב) לא הוכשרו למלאכת שמים אלא עור בהמה טהורה בלבד למאי הלכתא לתפילין, ואקשינן תפילין בהדיא כתיב בהו (שמות יג, ט) למען תהיה תורת ה׳ בפיך ופרקינן לעורן, ואקשינן והא אמר אביי שי״ן של תפילין הלכה למשה מסיני ופרקינן לרצועותיהן, ואי איתא לפרוך נמי רצועות והא אמר אביי דלי״ת ויו״ד של תפילין הלכה למשה מסיני כדאקשינן גבי לעורן משי״ן.}
\textblock{\textbf{אמר עולא וחילופיהן באיש.} פירש רש״י ז״ל: דאטבעת בלחוד קאי, אבל במחט איש ואשה כי הדדי נינהו. וכן פירש רב האי גאון ז״ל. וכן מוכחת כל הסוגיא. ורב אלפסי פירש בין בטבעת בין במחט, [ו]לא נראו דבריו.}
\textblock{\textbf{מכניסן זוג זוג אחד האיש ואחד האשה.} ואם תאמר דילמא משום דהוצאה כלאחר יד הוא באשה התירו. יש לומר אי משום הא מאי איריא זוג זוג אפילו טובא נמי.}
\textblock{\textbf{בירית באחת כבלים בשתים.} נראה לי דלרבין שני כלים הן, והבירית הוא טבעת הכלים ואין נותנין אלא באחת אבל כבלים הוא כלי אחר והוא עשוי לתכשיט ונותנין בשתים, דאילו (נתן) [נתנה] שתי בירית בשתי שוקיה אמאי טמאין ואמאי אין יוצאין דאטו משום דהוו שתים הוו להו מנא.}
\textblock{\textbf{יו״ד ה״א מלמעלה וקדש למ״ד מלמטה.} יש מפרשים שכתוב קדש ל בשיטה שניה בראש שיטה ושם מלמעלה בסוף שיטה ראשונה כזה: ואינו נראה שאין דרך השם קדש ל כתיבה וקריאה כן. אבל רבנו תם ז״ל פירש: שם מלמעלה בראש שיטה שניה וקדש ל מלמטה בסוף שיטה ראשונה כזה: ומפני שאדם פוגע תחילה קדש ל השם שיטה שניה קוראהו למעלה ולמעלה במקום זה הקדמה.}
\textblock{\textbf{מנין לאריג כל שהוא שהוא טמא מאו בגד (ויקרא יא, לב).} ודוקא בבגד שאין רצונו לארוג בו יותר ולא בא מבגד גדול דעכשיו יש לרבותו מאו בגד, כלומר שהוא בגד בפני עצמו, אבל הא דבעינן בעלמא שלש על שלש אי נמי שלשה על שלשה הני מילי בבא מבגד גדול. ואם תאמר והיכי מרבינן אריג כל שהוא מאו בגד, והא אפיקתיה בפרק במה מדליקין (לעיל שבת כז, א) לרבא שלשה על שלשה בשאר בגדים ולאביי לרבות שלש על שלש בצמר ופשתים דמטמא בשרצים. יש לומר דכוליה או בגד אייתר ליה כדאיתא לקמן בסמוך ודרשינן חדא מאו וחדא מבגד.}
\textblock{\textbf{מנין לתכשיט כל שהוא שהוא טמא מציץ.} ואם תאמר לימא מטבעת דזוטר טפי. יש לומר דניחא ליה לאתויי מציץ שהוא כלי קטן ותכשיטו משהו, וכל שהוא דקאמר אכולה מילתא קאי, כלומר: שהוא תכשיט שגופו כל שהוא ותכשיטו כל שהוא, אבל טבעת הוא עשוי לתכשיט ממש. כך תירצו בתוס׳.}
\textblock{\textbf{אריג ותכשיט כל שהוא טמא מכל כלי מעשה (במדבר לא, נא).} ואם תאמר למאי איצטריך, דהא אריג כל שהוא בפני עצמו טמא ותכשיט כל שהוא בפני עצמו טמא, וכיון שכן מאי למימרא דכשהן מחוברין ביחד שהוא טמא. ופירשו בתוס׳ דאריג שהוא תכשיט קאמר, דאילו אריג כל שהוא שאינו תכשיט רבינן מאו בגד, ותכשיט של מתכת כל שהוא רבינן מציץ, אבל הייתי אומר דדוקא של מתכת דומיא דציץ אבל דאריג לא, קא משמע לן דאפילו תכשיט כל שהוא של אריג טמא מכל כלי מעשה.}
\textblock{\textbf{ההוא במדין כתיב.} פירש רש״י ז״ל: בטומאת מת, דטומאה שנזכרה במדין בטומאת מת הוה וכדכתיב בה      לא, יט) תתחטאו אתם ושביכם דהיינו הזאה, והכי קאמר ההיא בטומאת מת כתיב דחמיר אבל בשאר טומאות דקילי מנא לן. ואינו מחוור. חדא דהוה ליה למימר בהדיא ההוא בטומאת מת כתיב. ועוד דהא טומאת מת וטומאת שרץ ילפינן מהדדי בסמוך (לקמן שבת סד, א) מבגד ועור בגד ועור (ויקרא יא, לב. במדבר לא, כ). ועוד דאם כן הא דמרבינן אריג כל שהוא מאו בגד דכתיב גבי שרצים ליפרוך נמי ההוא בשרצים כתיב בטומאת מת מנין, דבשרצים נמי איכא למיפרך מה לשרצים שכן טומאתו מרובה אי נמי מה לשרצים שכן בכעדשה וכדאמרינן בסמוך (לקמן שם).\par \textbf{} אלא הכי פירושא: ההוא בשלל של מדין כתיב גבי קרבן שהקריבו אנשי הצבא ולא משתעי מידי בטומאה וטהרה בההיא פרשתא, ואע״ג דכתיב בפרשה הראשונה דמשתעי בטומאה וטהרה תתחטאו אתם ושביכם, לאו בכולה שביה קא משתעי דהא איכא בהמה ונפש אדם שהביאו מן השבי.\par \textbf{} ותמיה לי דאם כן אצעדה וצמיד טבעת עגיל וכומז נמי בשלל מדין כתיבי (במדבר לא, נ) ולא בפרשת טומאה וטהרה שלמעלה, ואם כן מנא לן דמקבלי טומאה, והיכי אקשי הכי להדיא לעיל ואילו אצעדה טמאה. ונראה לי משום דהנהו נמי הוו בכלל כלים דכתיב גבי טומאת מת וכדכתיב (במדבר יט, יח) והזה על האהל ועל הנפשות ועל הכלים, אבל אריג ותכשיט כל שהוא לא הוי בכלל סתם כלים, והלכך איצטריך לרבויי מכל כלי.\par \textbf{} אחר כן מצאתי בספר הישר (חלק החידושים סי׳ רמו עי״ש) שנשמר מן הקושיא הזאת, וכן כתוב שם: וכל הני כלים דכתיבי נמי התם כגון אצעדה וצמיד ילפינן מכל כלי, וכיון דנפקא לן לגבי טומאת מת ילפינן מבגד או עור לגבי טומאת שרץ. עד כאן.}
\clearpage
\newsection{דף סד}
\textblock{\textbf{מנין לרבות קלקי וחבק.} ואם תאמר והלא אין מטלטלין מלא וריקן, ואם תאמר מפני שראוי לקפלו ולשים בו שום דבר, מכל מקום כיון שאינו עשוי לכך לא חשבינן ליה כמטלטל מלא וריקן, וכדתנן לקמן (שבת סו, א) גבי קב הקיטע ואם יש לו בית קבול ראש שוקו לא חשבינן ליה בית קבול כיון שאותו בית קבול אינו עשוי לטלטל בתוכו שום דבר, וכמו שפירש רש״י זכרונו לברכה שם. ותירצו בתוס׳ דאף קלקי וחבק נמי על כרחין אית לן למימר דעשוי לכל תשמיש. ולדידי קשיא לי דהא איצטריך למעוטי חבלים ומשיחות משום דאינן אריג ואף על גב דאין עשוין לטלטלן מלא וריקן ואינן ראוין לכך. וצריך לי עיון.}
\textblock{\textbf{מה לטומאת שרצים שכן מרובה.} פירש רבנו האי גאון ז״ל שהיא מצויה. ואינו מחוור, דאם כן הוה ליה למימר שכן מצויה. והנכון מה שפרש״י ז״ל, דכל שאר הטומאות הקלות שאינן צריכות שבעה כטומאת מת קרי טומאת שרץ, וכדקרינן חטאת חלב כל חטאות שיש בהן כרת, ומרובה מרובה ממש קאמר.}
\textblock{\textbf{בגד ועור דכתיב (במדבר לא, כ) גבי מת למאי איצטריך שמע מינה לאפנויי.} ואם תאמר אם כן בגד ועור דכתיב גבי מת וגבי שרץ (ויקרא יא, לב) למאי איצטריכי הא כולהו אתו מהיקשא דשכבת זרע, דכל מה שנאמר בשרץ כאילו נכתב בשכבת זרע ומה שנכתב בשכבת זרע יליף מת מיניה והוא הדין למה שנכתב במת יליף שכבת זרע מיניה והדר יליף שרץ מיניה. ויש לומר דכל מילתא דאתיא בהיקשא טרח וכתב לה קרא.}
\textblock{ מתני׳:\textbf{ בטוטפת ובסרביטין בזמן שהן תפורין.} ואם תאמר והא מרישא (לעיל שבת נז, א) שמעת מינה [דקתני] ולא בטוטפת ולא בסרביטין בזמן שאינן תפורין הא תפורין יוצאין, והא נמי דקתני בכבול ובפאה לחצר מרישא שמעת מינה דקתני ולא בכבול ברשות הרבים הא בחצר יוצאה. ויש לומר דאורחא דתנא הכין דתני איסורא והדר תני התירא משום חדא דמיצטריך ליה למיתני בהיתר, וכדתנן (חולין מב, א) אלו טרפות ניקב הלב לבית חללו וכו׳, והדר תני (שם נד, א) אלו כשרות ניקב הלב ולא לבית חללו, וכן רבים.}
\textblock{\textbf{בכבול.} פירש רש״י ז״ל: דהאי כבול דסיפא לכולי עלמא היינו כיפה של צמר, מדמפרש טעמיה בגמרא כדי שלא תתגנה על בעלה. ואינו מחוור, דאם כן הוה להו למיפרך מינה לשמואל דאמר (לעיל שבת נח, א) בכבול דרישא (דמתניתין לעיל שבת נז, א) כבלא דעבדא. אלא האי נמי לשמואל כבלא דעבדא הוא, ולדידיה לאו משום האי טעמא שרו ליה אלא משום קטטה ואי נמי דלא גזרו בחותם שבעבדים דאינהו מידכר דכירי ושלפי ליה דניחא להו בהכין להעביר מהן חותם של עבדות, וכמו שכתבתי לעיל (שבת נז, ב ד״ה כבלא).}
\textblock{\textbf{במוך שהתקינה לנדתה.} פירש רש״י ז״ל: כדי שלא יטנפו בגדיה. ואינו מחוור, דאם כן אצולי טינוף לא חשבינן. וכדאמרינן בפרק קמא (לעיל שבת יא, ב) בכיס של זב. אבל בתוס׳ פירשו: כדי שלא יטפטף על בשרה ויתייבש עליה ויסרט את הבשר.}
\textblock{\textbf{שן תותבת שן של זהב.} פירש רש״י ז״ל: דכולה חדא מילתא היא, כלומר: שן תותבת שהיא שן של זהב. ואינו מחוור, דאם כן שן שן למה לי. ופירוש אחר פירש דתרתי קאמר, ושן תותבת אהתירא דרישא קאי, כלומר: ויוצאה גם כן בשן תותבת אבל בשן של זהב פליגי. וגם זה אינו מחוור, דאם כן ובשן תותבת הוה ליה למיתני.\par \textbf{} אלא תרתי נינהו ובתרוייהו פליגי, כלומר: שן תותבת פירוש שן נכרית תרגום תושב תותב וכן שן של זהב רבי מתיר דלא חיישינן לנפילה וכן נמי לא חיישינן למישלף ואחויי דכל מידי דמיגניא ביה לא אתי לאחויי וחכמים אוסרים דחיישינן דילמא נפלה ואתי לאתויי.\par \textbf{} ומיהו בברייתא (לקמן שבת סה, א) תניא דבשל כסף דברי הכל מותר, והיינו טעמא משום דכיון שאינה ניכרת בין השינים לא שלפה ומחויא, ואי נפלה נמי לא חשיבא ליה ולא אתי לאתויי דאזלא לבי אומנא ועבדא אחרינא דלא כסיפא ליה מילתא דסברה דאומנא לא מרגיש בה משום דדרכן של נשים לעשות תכשיטין של כסף בצורת כל הדברים, אבל שן דאי מתאביד האי לא משכחא אחרינא אי נפלה אתי לאתוייה.}
\textblock{ גמרא:\textbf{ בנדתה תהא עד שתבא במים.} ולזקנים הראשונים איכא למימר דנפקא לן מבמי נדה יתחטא (במדבר לא, כג) דאמרינן בשלהי עבודה זרה (עה, ב) מים שהנדה טובלת בהן, ורבי עקיבא סבירא ליה דדילמא ההיא לטהרות אבל לבעלה דחולין הוא לא צריכה.}
\clearpage
\newsection{דף סה}
\textblock{\textbf{עשתה לו בית יד מהו.} רש״י ז״ל ורבנו האי גאון ז״ל פירשו אמוך שהתקינה לנדתה. אבל הראב״ד ז״ל פירשה אמוך שהתקינה לאזנה ושבסנדלה, כלומר: אם עשתה לו בית יד צריך קשירה או לא. ונראה שפירש כן מדאמר רב נחמן בר הושעיא אמר רבי יוחנן עשתה לו בית יד מותר, דמשמע מהאי לישנא דמשום בית יד הוא דמותר, שאילו להתיר אפילו בבית יד היה לו לומר אפילו עשתה לו בית יד מותר. וכן נראה גם מהירושלמי (דפרקין ה״ה), דגרסינן התם א״ר זעירא עשה לו רב חייא בר אשי בית יד למוך שבאזנו.}
\textblock{\textbf{אמר אביי רבי ורבי אליעזר ורבי שמעון בן אלעזר כולהו סבירא להו דכל מידי דמיגניא לא אתיא לאחוייה.} תמיהא לי דהא אפילו לכולי עלמא לא אתיא לאחוייה, דהא בירית מהאי טעמא הוא דשרינן (לעיל שבת ס, א). ועוד דלישנא דמיגניא לא שייך שפיר בכיפה של צמר אלא משום צניעותא. ועל כן נראה לי דכל הני לא נפקן בהו אלא משום מום שן תותבת וכובלת וצלוחית של פליטון דאמרן לעיל (שבת סב, א) מאן דרכה למיפק בצלוחית אשה שריחה רע, והלכך סברי רבנן דכיון שמומה ניכר דכל הרואה אותה יוצאה בשן של זהב כבר הוא יודע ששינה חסר וכן הרואה אותה יוצאה בכובלת כבר יודע שהיא בעלת ריח רע, וכיון שאינה יכולה להעלים מומיה אף היא אינה מקפדת ושלפה ומחויא. ורבי סבר אע״פ שיכירו שהיא בעלת מום לא שלפה ומחויא, כי היכי דלא תתגנה בהראות מומיה בעין. וכן הטעם בכיפה של צמר שאין אשה מקשטת עצמה באותה כיפה אלא אשה שהיא קרחת מפאת ראשה, והיינו דתני לה בהדי פאה נכרית, והיינו דשרו לה טפי כבול בחצר כדי שלא תתגנה על בעלה משאר תכשיטין, ואפילו כובלת וצלוחית של פליטון לא התירו לה בחצר אע״פ שהוא גנאי לה וקרוב להתגנות בכך על בעלה, אלא שמע מינה דכיפה צריכה לה טפי לסלק גנותה. כך נראה לי.}
\textblock{ מתני׳: \textbf{בחוטין וכו׳ שבאזניהם.} כלומר: ולא חיישינן לטבילה משום דלא מיהדק ולא חייצי, כדפירש רש״י ז״ל. ולא כפירוש השני שפירש דבקליעת הראש קא מיירי ומשום דבקטנות לא חיישינן לטבילה, דאין פירוש זה מחוור כמו שדחה הוא ז״ל בפירושיו. ועוד דבהדיא קתני בברייתא דמייתינן בריש פרקין (לעיל שבת נז, א) הבנות יוצאות בחוטין שבאזניהם אבל לא בחבקין שבצואריהן, ומפרשינן לה התם משום טבילה. ועוד מדאמרינן בגמ׳ אבוה דשמואל לא שביק להו לבנתיה נפקן בחוטין שבאזניהן, ואקשינן עליה ממתניתין דקתני הבנות יוצאות בחוטין, ובנתיה דאבוה דשמואל נשואות הוו וכדקאמרינן עביד להו מקואות ביומי ניסן ומפצי ביומי תשרי, ואפילו הכי קא מקשה ליה ממתניתין.}
\textblock{\textbf{נשים המסוללות זו בזו פסולות לכהונה.} פירש רש״י ז״ל: לכהן גדול משום דכתיב ביה (ויקרא כא, יג) אשה בבתוליה יקח, שיהיו כל בתוליה קיימין. ואין זה מחוור, דבפרק הערל (יבמות עו, א) משמע דמשום זנות קא אסר ליה, דאמרינן התם לית הלכתא לא כברא ולא כאבא, כברא הא דאמרן, כאבא דאמר רב הונא נשים המסוללות זו בזו פסולות לכהונה, ואפילו לרבי אלעזר דאמר (שם סא, ב) פנוי הבא על הפנויה [שלא לשם אישות] עשאה זונה התם באיש אבל באשה פריצותא בעלמא הויא, אלמא משמע דטעמיה דרב הונא משום זונה קא אסר לה ואפילו לכהן הדיוט.}
\textblock{\textbf{ואי קשיא לך הא דאמרינן בפרק הבא על יבמתו (יבמות נט, ב) אנוסת עצמו ומפותת עצמו לא ישא ואם נשא [נשוי אנוסת חבירו ומפותת חבירו לא ישא ואם נשא] רבי אליעזר בן יעקב אומר הולד חלל, ואמרינן (שם ס, א עי״ש) מאי טעמא      } דרבי אליעזר אמר רב הונא אמר רב וכן אמר רב גדל סבר לה כרבי אלעזר דאמר פנוי הבא על הפנויה עשאה זונה, ואקשינן ומי סבר ליה [כותיה] והא קיי״ל דמשנת ר״א בן יעקב קב ונקי (שם מט, ב), ואילו בהא לית הלכתא כרבי אלעזר דאמר רב עמרם (שם סא, ב) אין הלכה כרבי אלעזר, ואי איתא דרב הונא בשיטת רבי אלעזר אמרה ומשום זונה קא אסר לה לכהונה מאי קא קשיא ליה התם דמוקי ליה לרבי אליעזר בן יעקב כרבי אלעזר היינו משום דלדידיה סבירא ליה כרבי אלעזר. ויש לומר דתלמודא קא קשיא ליה היכי קאמר הכין רבי אליעזר בן יעקב, דאי בשיטת רבי אלעזר אמרה תיקשי לן הלכתא אהלכתא. ואי נמי יש לומר דמדרב קא מקשה ליה לרב עמרם דרב רבהון דכולהו הוה.}
\textblock{\textbf{ומפצי ביומי תשרי.} פירש רש״י ז״ל: לשים תחת רגליהן מפני טיט הנהרות. וכן פירש הרב בפרק בתרא דנדה (סז, א) גבי אשה לא תטבול בנמל מפני הטיט. וכן היה נראה מאותה סוגיא שם לכאורה, מפני שאמרוה שם סמוך לנתנה תבשיל לבנה לא עלתה לה טבילה ואע״ג דהשתא ליכא אימר בדריוני נפל, וגם אותה שלא תטבול בנמל למד מן הענין, ושניהם משום חציצה. ויש מקצת ספרים שכתוב בהן שם גם בההיא דלא תטבול בנמל אע״ג דהשתא ליכא אימר בדריוני נפל, וזה בפירושו של רש״י ז״ל מפורש.\par \textbf{} ומכל מקום אינו נראה, לפי שאין טיט בנמל יותר מבשאר נהרות. ועוד שאין חוצץ אלא טיט היון דהיינו טיט הבורות וטיט היוצרים בלבד, כמו ששנינו במסכת מקואות בפרק ט׳ (מ״ב). ובגירסאות מדויקות ששם (בנדה) לא גריס אע״ג דליכא השתא אימר בדריוני נפל. והגאונים לא גרסי ליה כלל.\par \textbf{} והנכון כפירוש הגאונים ז״ל שפירשו דמשום צניעות היה עושה להן כן, מפני שהן מתיראות שמא יבא אדם ויסתכל בהן ולפיכך ימהרו לטבול ואינן טובלות כדרכן, ולפיכך היה עושה להן מפצי להיות כמסך להבדיל בין הנהר ובין העם לצניעות בעלמא, והוא הדין והוא הטעם לאותה שלא תטבול בנמל.}
\textblock{\textbf{סבר שמא ירבו נוטפין על הזוחלין.} ואם תאמר כי רבו מאי הוי. והא תנינא במסכת מקואות (פ״ה מ״ג) מעין שהוא מושך כנדל וריבה עליו והמשיכו הרי הוא כמות שהיה, ופרת נהר גדול ומושך הוא לעולם. יש לומר דלא ריבה ממש קאמר, אלא שהוסיף עליו, מכל מקום מי המעין רבים על הנוטפין. ויש מי שהקשה על זה דאם כן ליתני הוסיף ולא ליתני ריבה. ויש לומר עוד דההיא דמקואות בשריבה בתוך המעין בעצמו, כלומר: בתוך הגומא, וכיון שהוא נותן בתוך עיקר המעין עצמו הרי כל המים נחשבים כמי המעין, אבל במרבה בתוך הנהר המושך מתוך המעין נפסל ברוב השאובין או ברוב הנוטפין, ובמרבה המעין גם כן דוקא [במקום] זחילתן בין באורך בין ברוחב, אבל שלא במקום זחילתן כיון שזחילתן זו אינה אלא מחמת הנוטפין הרי הן כנוטפין ואין מטהרין אלא באשבורן, וכמו ששנינו שם בסיפא דההיא מתניתין דמקואות היה עומד וריבה עליו והמשיכו שוה למקוה לטהר באשבורן ולמעין להטביל בו בכל שהוא, ותנן נמי בפרק קמא דמסכת מקואות (מ״ז) למעלה מהן מעין שמימיו מועטין שרבו עליהם מים שאובין שוה למקוה לטהר באשבורן.\par \textbf{} ואם תאמר עוד, כי רבו נוטפין על הזוחלין נמי נימא קמא קמא בטיל, וכדאמרינן בפרק בתרא דעבודה זרה (עג, א) גבי יין נסך שנפל לבור, כי אתא רב דימי אמר רבי יוחנן המערה יין נסך מחבית לבור אפילו כל היום כולו ראשון ראשון בטל. וכלאים נמי היאך (נוספים) [נאסרים] בתוספת אחד ומאתים לימא ראשון ראשון בטל. ובנדרים נמי בפרק הנודר מן הירק (נדרים נז, ב) גבי בצל שעקרו בשביעית ונטעו בשמינית דאמרינן כשרבו גידוליו על עיקרו מותר דגידולי היתר מעלין את האיסור, ואמאי נימא ראשון ראשון בטל. תירץ ר״ת ז״ל דלא אמרו ביין נסך קמא קמא בטיל אלא במערה מעט מעט בהפסקות, אבל בלא הפסק לא אמרינן דליבטיל, והלכך בנוטפין שיורדין בלא הפסק ואי נמי בגידולי זרעים שמתרבין בכל שעה בלי הפסק הרי אנו רואין כאילו בא הכל ביחד ואינו בטל.\par \textbf{} אלא שקשה לתירוצו, הא [תנן] במסכת מקואות (פ״ז מ״ב) ומייתינן לה ביבמות פרק הערל (יבמות פב, ב) מקוה שיש בו ארבעים סאה נתן סאה ונטל סאה כשר עד רובו, וברובו מיהא פסול ולא אמרינן קמא קמא בטיל ואע״ג דקא יהיב לסירוגין. ויש לומר דהתם הוא מיירי במי פירות וכדמשמע הכי בהדיא במקומה במסכת מקואות, ואפשר דבמי פירות החמירו. ואם תאמר מאי קא מדמה לה התם ביבמות לתרומה בזמן הזה דרבנן, דהא פסול מי פירות למקוה דאורייתא, דהא ממעט להו בפרק כסוי הדם (חולין פד, א) מדכתיב מים יתירא. ותירצו בתוס׳ דהתם דוקא להטביל בו מחטין וצנורות דמדאורייתא לא בעו ארבעים סאה אלא מדרבנן בעלמא והלכך כיון שיש שם רובו של מקוה מים כשרים כשר, ומיהו בעינן דלישתארו רובו ומשום הכי מוכח מינה התם [ד]בדרבנן בעי רבייה.}
\textblock{\textbf{ועדיין הקשו בתוס׳ מהא דמסיק רב דימי בפרק שלישי דבכורות (כב, א) גבי הלוקח ציר מעם הארץ משיקו במים וטהור מה נפשך אי רובא מיא סלקא להו השקה אי רובא ציר לא בעי השקה, ומסיק לא שנו אלא לטבול בו פתן אבל לקדרה לא, מצא מין את מינו וניעור, אלמא אע״ג דנתבטלו המים בתוך הציר כי נפלה לקדרה ניעורו, וכל שכן דלא אמרינן ראשון ראשון בטל ואפילו בהפסקה. ותירצו דדילמא לגבי הא החמירו בטהרות יותר מן האיסורין. והא דתנן      } (תרומות פ״ה מ״ח) סאה של תרומה שנפלה למאה חולין ולא הספיק להעלותה עד שנפלה שם סאה אחרת לא תעלה, דאלמא אף באיסורין לא אמרינן קמא קמא בטיל. לא היא דהתם משום דלא נתבטלה לגמרי אלא צריך הוא להרים סאה של תרומה קודם שיהא מותר וכיון שצריך להרים לא שייך ביה למימר ראשון ראשון בטל.\par \textbf{} ואיכא למידק, אמאי אמרו שמא ירבו נוטפין על הזוחלין אפילו מחצה על מחצה נמי פסולין, וכדתנן במסכת עדיות (פ״ז מ״ג) העיד רבי (יצחק) [צדוק] על הזוחלין שרבו על הנוטפין שכשרין אלמא מחצה על מחצה פסולין. ויש לומר דהכא לאו דוקא שמא ירבו נוטפין ממש קאמר, אלא לפי שאי אפשר לצמצם לא דייק בלישנא. והרמב״ן ז״ל תירץ שהזוחלין שהלכו למקום הנוטפין ונתערבו הרי הן כנוטפין עד שירבו עליהן אבל הנוטפין שנתערבו בזוחלין הרי הן כזוחלין עד שירבו עליהן, והלכך מימי פרת ביומי ניסן שהנוטפין באין עליהן לא היו נפסלין אלא אם כן רבו עליהן הנוטפין.\par \textbf{} וכתבו בתוס׳ שהנוטפין הוא שאין מטהרין אלא באשבורן אבל מעין מטהר בין בזוחלין בין באשבורן, דאע״פ ששנינו (מקואות פ״ה מ״ה) הזוחלין כמעין והנוטפין כמקוה, לאו דוקא בזוחלין אלא משום דנוטפין דוקא במקוה ולא בזוחלין שנו שהמעין למעלה מהן שמטהר אפילו בזוחלין. ותדע לך מדתנן בפרק קמא דמקואות (מ״ז-ח שם) שוה למקוה לטהר באשבורן וכו׳ למעלה מהן מי מעין שהן מטהרין בזוחלין, ואם איתא דהאי באשבורן ולא בזוחלין (והאי) [וההיא] בזוחלין ולא באשבורן מאי למעלה מהן, האי בחד גוונא מטהרי והאי בחד גוונא מטהרי. (ותנן) [ותני] נמי (תוספתא מקואות פ״א ה״ז) מעין שהוא צר וחקקו ועשאו רחב מטהר באשבורן ואינו מטהר בזוחלין אלא במקום שיכולין לילך שם מתחילה, אלמא מעין נמי מטהר הוא באשבורן. ותנן נמי (פרה פ״ח מ״ח ומקואות פ״ה מ״ד) כל הימים כמקוה שנאמר (בראשית א, י) ולמקוה המים קרא ימים רבי יוסי אומר כל הימים מטהרין בזוחלין, ומסתמא משמע דר״י לא פליג אהא דמטהרין באשבורן אע״ג דחשיב להו כמעין ליטהר בזוחלין. ועוד מדאמרינן הכא אין המים מטהרין בזוחלין אלא פרת ביומי תשרי בלבד משמע דדוקא בזוחלין הוא דאין מטהרין בשאר הימים הא באשבורן מטהרין, ואיך יטהר פרת ביומי ניסן באשבורן ניחוש שמא לא רבו הנוטפין והוו להו זוחלין רובא ולא יטהרו באשבורן, אלא ודאי שהמעין מטהר בכל ענין.}
\textblock{\textbf{מסייע ליה לרב דאמר רב מטרא במערבא סהדא רבא וכו׳.} כבר כתבתיה בנדרים פרק אין בין המודר (נדרים מ, א) בסייעתא דשמיא.}
\textblock{\textbf{סיפא אתאן למטבע.} אבל שארא אפילו לכתחילה. ותמהני מאי שנא מטבע מאי שנא אבן, דאף היא אינה מטלטלת כמטבע ואין לה תורת כלי. ובירושלמי (פ״ו ה״ז) גרסינן: תני רבן שמעון בן גמליאל אומר לא שנו אלא מטבע ואבן הא באגוז מותר מפני שהוא מיטלטל, [אמר רב אדא בר אהבה אתיא דרבן שמעון בן גמליאל כרבי מאיר כמה דרבי מאיר אמר דבר שהוא מיטלטל] מותר כן רבן שמעון בן גמליאל אומר דבר שהוא מיטלטל מותר.}
\textblock{מתני׳:\textbf{ הקיטע יוצא בקב שלו דברי רבי מאיר ורבי יוסי אוסר.} פירש רש״י ז״ל: דבמנעל ולא מנעל פליגי, דרבי מאיר סבר מנעל הוא ולפיכך יוצא בו, ורבי יוסי סבר לאו מנעל הוא ואינו לו אלא משוי ולפיכך אינו יוצא בו. ובודאי דלכאורה (כפירושו) [בפירוש] משמע מסוגיא דגמרא דהכא (סו, א), דאמרינן ואף שמואל הדר ביה דתנן (יבמות קא, א) חלצה בסנדל שאינו שלו בסנדל של עץ או בשל שמאל בימין חליצתה כשרה, ואמרינן מאן תנא אמר שמואל רבי מאיר היא דתנן הקיטע יוצא בקב שלו דברי רבי מאיר, אלמא מדאוקימנא הא דחליצה בפלוגתא דרבי מאיר ורבי יוסי שמע מינה דבמנעל ולא מנעל פליגי, דחליצה במנעל ולא מנעל תליא מילתא. ומדרב הונא נמי שמעינן לה, דתנן סנדל של סיידין טמא מדרס ואשה חולצת בו ויוצאין בו בשבת דברי רבי עקיבא ולא הודו לו, ואמר רב הונא מאן לא הודו לו רבי יוסי מאן הודו לו רבי מאיר.}
\textblock{\textbf{ואלא מיהו לא מחוור. דהא אסיקנא בפרק בתרא דיומא (עח, ב) ואמרינן [אלא אמר רבא] דכולי עלמא מנעל הוא ובהא פליגי מר סבר גזרינן דילמא משתמיט ואתי לאתויי ומר סבר לא גזרינן, וטעמא דשמואל ורב הונא דשמעתין דאוקימו } מתניתין דחליצה דלא כרבי יוסי היינו נמי מההוא טעמא דרבא, דכיון שקשה ומשתמיט לא קרינן ביה נעלו (דברים כה, ט) הראוי להלך בו לדעת רבי יוסי, ואע״ג דאית ביה שנצי, מתוך שהוא קשה ואין בו עור לא מיהדק ומשתמיט ואין ראוי להלך בו. והא דמשמע התם בפרק מצות חליצה (יבמות קג, א) דאי מחופה עור חולצת בו לכולי עלמא, התם הוא דכיון דמחופה עור אף הוא מתהדק ורצועותיו נקשרות היטב ולא משתמיט.\par \textbf{} אלא דאיכא למידק אם כן דשמואל ורב הונא בשיטת רבא אמרו, מאי טעמא אמר רבא מאן לא הודו לו רבי יוחנן בן נורי ולא אמר כרב הונא. ועוד התם בפרק מצות חליצה (יבמות קב, ב; קג, א) דתניא חלצה בסנדל של שעם ושל סיב ובקב הקיטע חליצתה כשרה וכו׳ במוך בסמיכת הרגלים ובאנפליא של בגד חליצתה פסולה, ואמרינן קב הקיטע מני רבי מאיר היא דתנן הקיטע יוצא בקב שלו דברי רבי מאיר באנפליא של בגד אתאן לרבנן, וסתמא דמילתא הני רבנן דקאמרינן היינו רבי יוסי דפליג בהדי רבי מאיר בקב הקיטע, ואם איתא דלא פליגי במנעל ולא מנעל אלא בקב דמשתמיט ואינו יכול להלך בו, אמאי מוקי אנפליא של בגד כרבנן הא בשארא לא פליגי וכולהו שוין בהו. ואיכא לתרוצי בהא, דהתם מתרץ לה אליבא דאביי דמוקי פלוגתייהו בפרק בתרא דיומא (שם) במנעל ולא מנעל, והיינו נמי דמתרץ לה אביי התם (ביבמות שם) ואמר מדסיפא רבנן רישא [נמי] רבנן ובמחופה עור, עד דאתא רבא ואוקמה כולה כרבי מאיר ומשום דלא מגין כלומר: אף סיפא רבי מאיר והוא הדין לרבי יוסי דבהא לא פליגי. ואלא מיהו אכתי קשיא הא דהכא, אמאי לא מוקי רבא לההיא דסנדל של סיידין כדאוקי לה רב הונא. ויש לומר דאף [ד]שמואל ורב הונא דהכא סבירא להו דבמנעל ולא מנעל פליגי ולפיכך לרבי יוסי אינה חולצת בו, אבל לרבא רבי יוסי נמי לגבי חליצה מודה הוא דחולצת, ומשום הכי אוקמה רבא לההיא דסנדל של סיידין דקתני בגוה ולא הודו לו כרבי יוחנן בן נורי, דאילו לרבי יוסי חולצת בו אע״פ שאינו יוצא בו בשבת.\par \textbf{} ואילו נפרש כן נצטרך לומר, דהא דאמר רבא התם בפרק מצות חליצה מדרישא רבי מאיר סיפא נמי רבי מאיר הוא הדין דהוה מצי למימר דכולה ברייתא כולי עלמא היא דלגבי חליצה לא פליגי, אלא כלפי מאי דסבירא להו מעיקרא דבמנעל ולא מנעל פליגי ותברוה רישא רבי מאיר וסיפא רבנן אמר להו רבא דאפילו כולה רבי מאיר היא. ואי נמי איכא למימר, דהתם לאו לאפוקי מדרבי יוסי אוקמה רבא כרבי מאיר, אלא לאפוקי מדרבי יוחנן בן נורי דלדידיה לא הוי מנעל כלל. וכן כתבתי שם ביבמות (ד״ה הא).\par \textbf{} ועדיין קשה לי ההיא דחולצת בסנדל של עץ דאוקי שמואל הכא כרבי מאיר, אמאי לא פליג בה רבא ונוקמה ככולי עלמא, כדפליג באוקמתא דרב הונא בההיא דסנדל של סיידין. ואולי נאמר דבשל עץ שהוא קשה ועשוי לישמט מודה רבא דרבי יוסי לטעמיה דאף לחליצה פסול דאינו ראוי להלך בו, וההיא דסנדל של עץ [ד]מתניתין וברייתא דהכא ובפרק מצות חליצה ודאי כרבי מאיר אתיין כפשטא דסוגיא דהכא ודהתם, אבל בשל סיידין בהא פליג רבא, או משום דהוי של שעם כדפירש הכא רש״י ז״ל ואינו קשה כל כך ולא משתמיט כשל עץ, או אפילו הוי של עץ כדפירש רבנו שמואל ז״ל אלא משום דאית ליה בסנדל ולא משתמיט כקב הקיטע. כך נראה לי.\par \textbf{} ואלא מיהו עדיין נצטרך לומר על כרחין דברייתא דהתם דקתני בשל סיב ובשל שעם ובקב הקיטע חליצתה כשרה ואוקימנא לה כרבי מאיר, לאו כולה רישא מוקמינן לרבא כרבי מאיר דוקא, אלא קב הקיטע בלחוד, אבל סנדל של שעם ושל סיב כולי עלמא היא.\par \textbf{} ואי קשיא לך כיון דאסיק רבא התם (ביומא) דכולי עלמא מנעל הוא, היכי נפיק ביה איהו ביום הכפורים. לא היא, דכולי עלמא דקאמר היינו רבי יוסי ורבי מאיר, אבל לרבי יוחנן בן נורי לאו מנעל הוא, וכדאמר רבא הכא מאן לא הודו לו רבי יוחנן בן נורי, דלדידיה לא הוה מנעל אלא של עור, כדכתיב (יחזקאל טז, י) ואנעלך תחש וכדאמרינן התם בפרק מצות חליצה (יבמות קב, ב), וקיימא לן כרבי יוחנן בן נורי. ואם תאמר מכל מקום היכי נפיק ביה ניגזור דילמא משתמיט ואתי לאתויי, דבהא לא אשכחן דפליג עליה דרבי יוסי אלא רבי מאיר ורבי מאיר ורבי יוסי הלכה כרבי יוסי (עירובין מו, ב). ולפיכך נראה לי כמו שכתבתי דבשל שעם ושל סיב אף רבי יוסי לא גזר משום דאינו קשה כל כך ולא עביד לאשתמוטי כשל עץ.}
\textblock{\textbf{ואם תאמר עוד, לאביי דסבירא ליה דבמנעל ולא מנעל פליגי דאלמא כל שאינו מנעל אין יוצאין בו דלאו תכשיט הוא אלא משוי, אם כן היכי נפיק ביה איהו, דאי מנעל הוא כרבי עקיבא ורבי מאיר אין יוצאין בו ביום הכפורים, ואי לאו מנעל הוא כרבי יוסי וכרבי יוחנן בן נורי, מכל מקום לא ליפוק דמשוי הוא. ונראה לי דלאביי בגזירה נמי פליגי אי גזרינן דילמא משתמיט או לא גזרינן, כלומר: דרבי יוסי תרתי אית ליה, דלאו מנעל הוא ועוד דעביד לאשתמוטי, דאי משום דלאו מנעל בלחוד הוא לא הוה אסר ליה, דכל מידי דמגין ודרכן של בריות לצאת בו בחול להנאת תשמישו לאו משוי הוא, מידי דהוה אאנפליא של בגד שאינו מנעל לכולי עלמא ויוצאין בו, ומשום דמשתמיט בלחוד נמי אי מנעל הוה לא הוה אסר ליה, דאי מלבוש הוא לא אסרו מלבושיו עליו ולא גזרו במלבושיו דילמא משתמיט, אלא משום דאינו מנעל ועוד דמשתמיט גזרינן לרבי יוסי, ורבי מאיר סבר דמנעל הוא ומשום גזירה דילמא משתמיט לא אסרינן ליה עליה כדאמרן,       } והלכך בשל סיב ובשל שעם דלא משתמיט כשל עץ כדאמרן שרי משום דלאו מנעל כרבי יוסי ורבי יוחנן בן נורי.\par \textbf{} ואכתי קשיא לי, אמאי איצטריך אביי התם ביומא לאוקומה לההיא דקב הקיטע בדאית ביה כתיתין ומשום תענוג, תיפוק לי דלכולי עלמא אסור ממה נפשך, לרבי מאיר משום מנעל ולרבי יוסי משום דלאו מנעל והוי משוי דעירוב הוצאה אף ליום הכפורים כשבת (כריתות יד, א).\par \textbf{} ומתוך הדוחק יש לי לומר, דודאי בין מאן דמייתי מינה ראיה בין אביי דדחי לה סבירא להו דבמנעל ולא מנעל פליגי, ומה״ט גופא אית להו לרבי מאיר ורבי יוסי למיסר ביוה״כ, אלא דמדקתני בההיא ברייתא ושוין שאסור לצאת בו ביוה״כ משמע להו דעוד מטעמא אחרינא דשניהם השוו בו לענין יום הכפורים ואינו בשבת אסרי ליה, ולכולי עלמא קאמר, כלומר: ושוין דבר מן הדין טעמא דמנעל ומשוי אסור, ואי סלקא דעתך דמשום ההוא טעמא דמנעל או משוי השוו בו לענין יום הכפורים דאסור, למה להו למימר דשוין ביוה״כ שאין יוצאין בו פשיטא דאסור לדידהו ממה נפשך כדאמרן, ומשום הכי אוקמה אביי בדאית ביה כתיתין ומשום תענוג, דקסבר בלא מנעל אסור משום עינוי, ולקרא דתענו את נפשותיכם (ויקרא טז, כט) סמכי ליה. ונפקא מינה להיכא דלאו מנעל וליכא נמי למיגזר דילמא משתמיט וכגון אנפליא של בגד דמיהדק שפיר ולא משתמיט ולאו מנעל הוא לכו״ע ואפילו הכי אי אית ביה כתיתין אסור משום תענוג, וכדקאמר התם ביבמות בפרק מצות חליצה ומהכא יליף לה.\par \textbf{} ועיקר בעיא דהתם דפרק בתרא דמסכת יומא (עח, א-ב) מהו לצאת בסנדל של שעם ביום הכפורים, כלומר: מי אמרינן דמנעל בלחוד אסרו או דילמא אף כל מידי דמגין כמנעל. ואתי למפשטה מברייתא דקב הקיטע דאינו מנעל לרבי יוסי ואפילו הכי קתני בברייתא ושוין שאסור לצאת בו ביום הכפורים אלמא כל מידי דמגין כמנעל אסרו, דמדאיצטרכינן למיתני ושוין שאסור לצאת בו ביום הכפורים על כרחין בלא טעמא דמנעל ומשוי קאמרינן ולכולי עלמא אסור, והלכך אף לדידן דקיימא לן דלאו מנעל הוא, ולית לן נמי דכל מידי דלאו מנעל אסור משום משוי דהא אנפליא של בגד אע״ג דלאו מנעל הוא לכולי עלמא כדאיתא בפרק מצות חליצה (יבמות קב, ב; קג, א) אפילו הכי יוצאין בה כדקתני התם בברייתא, אפילו כן מידי דמגין כמנעל אסור. ודחי אביי דאין ודאי משמע דקב הקיטע טעמא חדתא אית להו בגויה כדקאמרת, אלא מיהו לאו משום דמגין הוא וכדקא סלקא דעתך, אלא בדאית ביה כתיתין ומשום תענוג, והוא הדין לאנפליא דאינו מגין אי אית ביה כתיתין אסור לצאת בו ביום הכפורים, וכדאמר אביי נמי התם בפרק מצות חליצה (יבמות קב, ב), אבל ודאי אפילו לית ביה כתיתין לרבי מאיר אסור משום מנעל ולרבי יוסי אסור משום משוי ואי נמי משום דקשה ואין לו תוך ועביד לאשתמוטי, אבל לדידן דלא קיימא לן כרבי מאיר דאמר מנעל הוא ולא כרבי יוסי דאסר כל מידי דלאו נעל ומשוי שרי, ואי נמי טעמא דרבי יוסי משום דמשתמיט הני מילי בקב הקיטע דקשה ואין לו תוך כנעל דאיכא למיחש לאשתמוטי אבל בשל שעם דאינו קשה כל כך ואי נמי בשל עץ דיש לו תוך כנעל דלא עביד לאשתמוטי שרי. ואתא רבא ואוקמה בטעמא אחרינא ואמר דוקא נעל בלחוד הוא דאסור אבל מידי אחרינא לא ואפילו מגין כנעל ואפילו אית ביה כתיתין ואינהו דאסרי ליה ביום הכפורים משום מנעל הוא, הא לדידן קיימא לן כרבי יוחנן בן נורי דלאו מנעל הוא ומשום הכי שרי, והא דקתני בברייתא ושוין ביום הכפורים, לאו משום טעמא חדתא קאמר אלא לאשמעינן דפלוגתייהו לאו במנעל ולא מנעל הוא אלא לכולהו מנעל הוא ובשבת בגזירת דילמא משתמיט הוא [דפליגי], והלכך ביום הכפורים לכולהו אסור משום מנעל ולדידן דאינו מנעל שרי, ולאשתמוטי בשל שעם שאינו קשה כשל עץ או בשל עץ ויש לו תוך כנעל לא חיישינן דלא עבידי לאשתמוטי וכדאמרן. כך נראה לי.}
\clearpage
\newsection{דף סו}
\textblock{\textbf{סמוכות שלו טמאין מדרס.} פירש רש״י ז״ל: שהקב הוא חקוק להניח בו ראש שוקו, וסמוכות היינו סמוכות של הקיטע כי יש קיטע בשתי רגליו דהולך על שוקיו ועל ארכובותיו ועושה סמוכות של עור בשוק[י]ו. והקשו עליו בתוס׳ דאם כן אינו יכול להלך בלתי אותו קב ולמה אסר רבי יוסי לצאת בו, וכל שכן לרבא (ביומא עח, ב) דמפרש טעמא דרבי יוסי משום דלמא משתמיט ואתי לאתויי והלא אינו יכול להלך בלתי אותו הקב. ועוד דאם כן היכי מטהר ליה אביי מטומאת מדרס וקא מפרש טעמא משום דלא סמיך עלייהו, והלא כל גופו נשען עליו. ויש מי שתירץ שהוא נשען על מקלות שבידו ואין הקב צריך לו כל כך להשען עליו, ואינו אלא כמקל של זקנים כדאמר אביי בגמרא.\par \textbf{} ויש מי שפירש שהקב הוא כצורת רגל, ונותנו בראש שוקו כדי שיחשבו הרואים שהוא רגל, ועושה לו סמוכות ונותן ארכובותיו עליהן וכופה שוקו לאחוריו וקושר הסמיכה ליריכו והוא נשען עליו ומהלך בו, ולפעמים שהוא מסיר הסמיכה ונשען על הקב. ולפיכך נחלקו אביי ורבא, דלאביי טהורה מכלום לפי שאינו צריך לישען עליו אלא על הסמוכות, ואע״פ שלעתים מניח ראש שוקו על גבי הקרקע עם הקב דבר מועט הוא ואינו לו אלא כמקל של זקנים, ורבא אמר דטפי עדיף מיניה כדאיתא בגמ׳. והיינו נמי דאסר ליה רבי יוסי משום משוי או משום גזירת דילמא משתמיט ואתי לאתויי.\par \textbf{} ומכל מקום מדברי רש״י ז״ל נלמוד דדבר פשוט הוא דיכול להלך במקלות שבידו אף על פי שאינן קשורין בו, הואיל ואינו יכול להלך בענין אחר. וגם לפירוש השני יש להתיר כן, מדקתני סמוכות שלו ומשמע ששתי סמוכות יש לו וכגון שנקטעו שתי רגליו וכיון שכן אינו יכול להלך בלתי מקלות שבידו ואפילו הכי לא אסר להו. והא דקתני כסא סמוכות שלו טמאין מדרס ואין יוצאין בהן בשבת, כבר פירש רש״י ז״ל משום דאינהו תלו ולא מנחי אארעא זימנין דמשלפי. וכן התירו בתוס׳, ואמרו משמו של ר״ת ז״ל שהתיר לאדם אחד שנכווצו גידי שוקיו לצאת במקלו שבידו.}
\textblock{גמרא:\textbf{ סנדל של סיידין.} פירש רש״י ז״ל: שהוא של שעם, ומוכרי הסיד נועלין אותו כשמתעסקים בו מפני שהסיד מפסיד את העור. והקשה עליו ר״ת ז״ל דאם כן היכי מקשה עלה והלא לאו להלוכא עבידי, והלא הולך בו כשמתעסק בסיד. ועוד שהוא נשען עליו, ותניא בתורת כהנים (פרשת מצורע פרשת זבים פרק ב ה״ז) אין לי אלא יושב עומד נתלה ונשען מנין כו׳. ולפיכך פירש רבנו שמואל ז״ל דכלי הוא מכלי אומנות הסיידין לטוח בו קירות הבית, והוא מברזל ועשוי כעין סנדל שלמטה הוא ארוך ורחב כשולי סנדל, ובאמצעיתו יש לו בית קבול כדי אחיזת בית יד, ואותו בית יד מקיפו משפתו אל שפתו לשים בו כשהוא טח בו, ולעתותי ערב מניח רגלו בתוכו ומוליכו כך עד ביתו, ומשום הכי קאמר דלאו להלוכא קא עביד.\par \textbf{} ומיהו בתוספתא (כלים ב״ב פ״ד ה״ב) משמע כדפירש רש״י ז״ל דסנדל של סיידין היינו של קש, דתניא התם סנדל של קש רבי עקיבא מטמא מדרס ואשה חולצת בו וכו׳. ובירושלמי נמי גרסינן במסכת יבמות (פי״ב ה״ב עי״ש היטב): סנדל של עץ חבריא בשם רב והוא שיהו חבטיו של עור, תמן אמרין בשם רב והוא שיהו תרסיותיו של עור, רבי אילא בשם רבי יוחנן ואפילו כולו של עץ כשר, מתניתא מסייעא לרבי יוחנן סנדל של קש טמא מדרס והאשה חולצת בו דברי רבי עקיבא אבל חכמים לא הודו לו למדרס, עד כאן בירושלמי, והיא היא הברייתא השנויה כאן, אלמא סנדל של סיידין של קש הוא כדברי רש״י ז״ל.\par \textbf{} ומכל מקום מה שאמרו שם בירושלמי דחכמים דהיינו רבי יוחנן בן נורי לא הודה לו דוקא למדרס, אינו מסכים עם סוגית הגמ׳ שבכאן.\par \textbf{} ומיהו טעמא דמילתא לא כמו שפירש רש״י ז״ל דמשום דלא חשיב כשל עץ מטהר רבי יוחנן בן נורי, דהא אפילו כופת שאור שיחדה לישיבה טמאה מדרס, כדמוכח בפרק העור והרוטב (חולין קכט, א). אבל בתוס׳ פירשו טעמא דרבי יוחנן משום דאין תורת כלי עליו דחשיב ליה כדבר שאינו מתקיים וגרע טפי מכופת שאור. ורבי עקיבא סבר דכיון שמתקיימין לפי שעה כלים גמורים הם, ואפילו טומאת מת בלא יחוד לישיבה מטמא רבי עקיבא לכוורת ולשפופרת דהא סתמא קתני רבי עקיבא מטמא. והכי תניא בתוספתא דכלים בפרק מלא תרווד רקב (ב״מ פ״ז ה״ג) העושה כלי מדבר שהוא של מעמיד טמא מדבר שאינו של מעמיד טהור, הלפת ואתרוג והדלעת שחקקום התינוקות למוד בהן העפר טהורין, האלון והרמון והאגוז שחקקום תינוקות טמאין, ופירשוה הם ז״ל דבר המעמיד הוא דבר המתקיים, ודלעת דחשיב דבר שאינו מתקיים היינו דלעת לחה דומיא דלפת, אבל כלי עץ חשוב הוא.\par \textbf{} והא דאקשינן והא לאו להלוכא קא עביד, הכי קאמר: כיון שאינו נועלו להנאת הלוכו אלא כדי להציל מנעליו שלא ישרפו בסיד הרי הן כאילו אינן עשוין להלך, ומשני כיון שהוא מטייל בהן לעתותי ערב עד ביתו עבידי להלוך חשבינן להו.}
\textblock{\textbf{מאן הודו לו רבי מאיר.} איכא למידק והא קתני בההיא דאייתינן לעיל בסנדל של עץ חליצתה כשרה, דאלמא לכתחילה לא ואוקימנא כרבי מאיר. ויש לומר דהא דאמרינן מאן הודו לו רבי מאיר לאו אאשה חולצת בו קאי אלא איוצאין בו בשבת. ואי נמי אהכשרא דסנדל של סיידין קאי ולומר דהוי מנעל אף לרבי מאיר. ובתוס׳ אמרו דהא דקתני לעיל חליצתה כשרה הוא הדין לכתחילה, ואיידי דקא בעי למיתני סיפא באנפליא של בגד חליצתה פסולה ואפילו דיעבד תנא רישא חליצתה כשרה.}
\textblock{\textbf{אביי אמר טמא טומאת מת ואינו טמא טומאת מדרס.} איכא למידק אם כן סבירא ליה לאביי דלאו להלוכא עביד, וכיון שכן היאך אשה חולצת בו לרבי מאיר, דהא פסלינן בפרק מצות חליצה (יבמות קד, א) לכולי עלמא של זקן שעשאה לפי כבודו ומשום דלאו להלוכא עביד. ותירצו בתוס׳ דלגבי חליצה כל היכא דעביד קצת להלוך שרי אבל לטומאת מדרס לא מטמא אלא בדעביד להלוך ממש.}
\clearpage
\newsection{דף סז}
\textblock{\textbf{כל דבר שיש בו משום רפואה אין בו משום דרכי האמורי.} פירוש: כל שיש בו משום רפואה בין סם בין לחש, דאפילו לחשים שיש בהן ממש ומועילין לרפואה יש בהרבה מקומות בתלמוד שהן מותרין. ותניא בתוספתא פ״ח דמכלתין (הי״א עי״ש) אלו דברים מותרין היה מתחיל במלאכתו נותן שבח והודאה לפני המקום, בחבית או בעיסה מתפלל הוא שתכנס בהן ברכה ולא תכנס בהן מארה, לוחשין על המעים ועל הנחש ועל העקרב ולא מדרכי האמורי. ואפשר דאפילו לחשים שאין אנו יודעין אם יש בהן משום רפואה אם לאו מן הספק מותרין, ולא אסרו אלא אותן שהן בדוקין ודבר ברור שאין בו משום רפואה. וכן נראה שפירשו בתוס׳.}
\textblock{ מתני׳:\textbf{ היודע עיקר שבת ועשה מלאכות הרבה בשבתות הרבה חייב על כל שבת ושבת.} ואמרינן בכריתות בפרק אמרו לו (כריתות טז, ב) דימים שבינתיים הויין ידיעה לחלק. ופרש״י ז״ל כאן שאי אפשר שלא שמע בינתיים שאותו יום שבת היה. והקשו עליו בתוס׳ דידיעת היום לא מעלה ולא מוריד, דידיעת חטא בעינן, מדכתיב (ויקרא ד, כג) או הודע אליו חטאתו, ותדע לך מדאמרינן לקמן (שבת ע, ב; עא, א) קצר וטחן בשגגת שבת וזדון מלאכות כו׳ ונודע לו על קצירה וטחינה של שגגת שבת וזדון מלאכות כו׳ קצירה גוררת קצירה וטחינה גוררת טחינה, אלמא בקצר וטחן בשניה בזדון שבת אין אותה ידיעה מחלקת לחטאות דידיעת חטא בעינן. ועוד דהא דחייב על כל שבת ושבת ביודע עיקר שבת איצטריך למילף בגמ׳ (לקמן שבת סט, ב) מקראי, ואי כפרש״י ז״ל פשיטא קרא למה לי, אלא שמע מינה גזירת הכתוב הוא בשבת. ועוד דתנן בכריתות פרק אמרו לו (כריתות טז, א) אמר רבי עקיבא שאלתי את רבי אליעזר העושה מלאכות הרבה בשבתות הרבה מעין מלאכה אחת בהעלם אחד מהו וכו׳, ואמר (לחייב) [לי חייב] על כל אחת ואחת קל וחומר מנדה וכו׳, ופירש רב חסדא (שם עמוד ב) דשגגת שבת וזדון מלאכות בעא מיניה אי ימים שבינתיים הויין ידיעה לחלק או לא, ופשט ליה דהויין ידיעה לחלק, ואקשינן עליה (שם יז, א) אי ימים שבינתיים הויין ידיעה לחלק בעא מיניה ופשט ליה מנדה, נדה מאי ימים שבינתים אית בה, ופריק כגון שבא עליה וטבלה וראתה וחזר ובא עליה וטבלה וראתה וחזר ובא עליה דטבילות שבינתיים הויין ידיעה כימים שבינתיים, אלמא אע״פ ששהתה עשר שנים שלא טבלה לא אמרינן אי אפשר שלא שמע בינתיים שנדה היתה באותו זמן. ועוד דאפילו למה שתירץ כגון שטבלה בינתיים טבילות מאי ידיעה הויין, דילמא טבלה לטומאת נגיעתה בשרץ ונבלה או למת ומשום טהרות ואי נמי טבלה להקר ומאי ידיעה הויין.\par \textbf{} על כן פירשו הם ז״ל דגזירת הכתוב הוא בשבת דימי החול הוו כידיעה לחלק, וכן ימי טהרה הויין כידיעה לחלק, וילפינן לנדה [משבת] דימי היתר שבינתיים הויין ידיעה לחלק, ותרתי בעא מיניה ימים שבינתיים אי הוו ידיעה לחלק ועוד שבתות אי כגופין מחולקין דמו או לא, והא דפשט ליה מנדה שבתות כגופין מחולקין דמו פשט ליה אבל לענין ימים שבינתיים למהוי כידיעה אדרבא נדה משבת ילפינן. ולפי פירוש זה שפירשנוהו נראה שאפילו טבלה שלא בפניו והוא לא ידע אפי׳ הכי חייב על כל ביאה, דימי היתר שבינתיים מחלקין.\par \textbf{} אבל ראיתי בשם רבנו שמואל ז״ל שפירש ימים שבינתיים הויין ידיעה, דכיון שנודע לו יום החול הוי הפסק להעלמות, שהרי יצא מן הספק הראשון ויודע שעכשיו הוא חול, וכן בנדה מאחר שראה אותה טובלת וידע שטהרה ועכשיו אין לו ספק שיודע בודאי שהיא טהורה ידיעה היא. ולפי הפירוש הזה נראה שהוא צריך שידע בטבילתה.}
\textblock{\textbf{חייב על כל [אב] מלאכה ומלאכה.} פירוש: למאן דאמר התם במסכת כריתות (ראה בסוגיא שם) כגופין מחולקין דמו חייב על כל מלאכה ומלאכה של כל שבת ושבת, ולמאן דאמר לאו כגופין מחולקין דמו דוקא על כל אב מלאכה ומלאכה של כל השבתות חייב אחת, דהא ליכא למימר ימים שבינתיים הויין ידיעה לחלק לפי שאין השגגות בימים אלא במלאכות.}
\newchap{פרק \hebrewnumeral{7} כלל גדול}
\clearpage
\newsection{דף סח}
\textblock{}
\textblock{ גמרא:\textbf{ גבי שביעית נמי משום דקבעי למיתני עוד כלל אחר תנא כלל גדול.} כלל גדול שבשביעית הוא ששנינו בפרק שביעי (מ״א) כלל גדול אמרו בשביעית כל שהוא מאכל אדם ומאכל בהמה וממין הצובעים ואינו מתקיים בארץ, יש לו שביעית ולדמיו שביעית, יש לו ביעור ולדמיו ביעור. ועוד כלל אחר היא, ששנינו שם (במ״ב) ועוד כלל אחר אמרו כל שהוא מאכל אדם ומאכל בהמה וממין הצובעים ומתקיים בארץ, יש לו שביעית ולדמיו שביעית, אין לו ביעור ולא לדמיו ביעור, ואע״ג דבכלל גדול קא חשיב ארבע, ובעוד כלל אחר נמי קא חשיב ארבע, היינו גודליה דקמא משום דבקמא קתני יש לו ביעור ובבתרא קתני אין לו ביעור. והא דקתני התם (פ״ח, מ״א) כלל גדול אמרו בשביעית כל המיוחד למאכל אדם אין עושין ממנו מלוגמא, ואע״ג דלא קתני כלל אחר בתריה. איכא למימר דאכלל האמצעי קאי, דכיון דתניא עוד כלל אחר אמרו, תנא בתריה כלל גדול. וכללות ששנינו במעשר הוא ששנינו בפ״ק דמעשרות (מ״א) כלל אמרו במעשרות כל שהוא אוכל ונשמר וגדולו מן הארץ חייב במעשר,      כלל אחר אמרו כל שתחילתו אוכל וסופו אוכל, אע״פ שהוא שומרו להוסיף אוכל חייב קטן וגדול.}
\textblock{\textbf{שבת ושביעית דאית בהו אבות ותולדות.} אבות שבשביעית הנהו דכתיבי באורייתא, ואידך דרבנן קרי להו תולדות. ואם תאמר אם כן במעשר איכא אבות ותולדות, כגון תירוש ויצהר דחייבין דאורייתא ואידך דרבנן נינהו. תירצו בתוספות דלא שייך אבות ותולדות אלא במידי דאית ביה אסור מלאכה. ולא הבנתי, דהא גבי נזקין תנן אבות ותולדות. ואם תאמר עוד אם כן בכלל שני דשבת ובכלל שני דשביעית ליתני נמי גדול מהאי טעמא. ויש לומר דאכלל גדול דרישא קיימי, כלומר: עוד כלל אחר גדול אמרו.}
\textblock{\textbf{ולבר קפרא דתני כלל גדול במעשר מאי אבות ותולדות איכא.} ואם תאמר לבר קפרא טפי ניחא, דלדידיה לאו משום אבות ותולדות, אלא בכולהו משום דקא בעי למיתני עוד כלל אחר. יש לומר דבר קפרא לא תני אלא חד. ואי נמי ניחא ליה לפרושי נמי הא דבר קפרא כטעמא דמתניתין.}
\textblock{\textbf{ואילו פאה ליתא בתאנה וירק.} פירש ר״ת ז״ל: דמדאורייתא דגן תירוש ויצהר דוקא איתנהו בפאה, משום דפאת שדך בקצרך כתיב (ויקרא כג, כב) וקציר דגן משמע, וכתיב (ויקרא יט, י) וכרמך לא תעולל, וכתיב (דברים כד, כ) כי תחבוט זיתך לא תפאר אחריך, הא כל שאר האילנות ושאר הדברים אינן חייבים דבר תורה. והא דדרשינן להו מקראי בתורת כהנים (קדושים פרק א, פי׳ ז), אסמכתא בעלמא נינהו, דהכי נמי מרבה בספרי (בחוקתי פרשתא ח פ׳ י״ב) כל מילי לגבי מעשר, אע״ג דכתיב בהו בהדיא מעשר דגנך תירושך ויצהרך (דברים יד, כג). ובתאנה וירק היינו טעמא דלא תקון בהו פאה, משום דלית בהו ריוח לעניים, אדרבא מתבטלין ממלאכתן על כך כיון שאין ידוע זמן לקיטתן, ובירק נמי כיון שאינו מכניסו לקיום דבר מועט הוא, ואי אפשר שלא יתבטל עליו יותר ממה שמרויח, ועוד דכיון שאין עושין ממנו גורן אין העם יודע מתי ילקטנו ויתבטל עליו [שיהיה] יושב וממתין. והא דנקט תאנה לאו דוקא, אלא הוא הדין לכל שאר האילנות שאין נלקטין כאחד, והיינו דלא חשיב התם (פאה פ״א מ״ה) במחוייבי פאה אלא שמונה אילנות בלבד הא כל השאר פטורין, ותאנה לרבותא נקטה אע״ג דחייבת בבכורים.}
\textblock{\textbf{גר שנתגייר בין הגוים.} פירוש: בפני שלשה, אלא שלא הודיעוהו מצות שבת, אבל נתגייר בינו לבין עצמו אינו גר וכדאיתא ביבמות פרק החולץ (יבמות מו, ב).}
\textblock{ מהא דרב ושמואל דאוקימו מתניתין: ב\textbf{שהיתה עיקרה של שבת שכוחה ממנו.} נראה לי גם כן קושיא לפרש״י דמתניתין, דכיון שהיתה עיקר של שבת שכוחה ממנו מאי ימים שבינתיים איכא והאיך שמע שהיה שבת, ואפילו ללישנא בתרא מתניתין נמי בין שהכיר ולבסוף שכח בין שהיתה עיקרה של שבת שכוח ממנו. אבל לפירוש שני דאוקימו לה בגזירת הכתוב ניחא. ואע״ג דרש״י ז״ל חזר על צדדי הדבר כדי להולמו, אפילו הכי לא נתיישב.}
\textblock{\textbf{תנן כל השוכח עיקר שבת לאו מכלל דהוי ליה ידיעה.} פירוש: לישנא דשוכח קא דייק, דלא שייך שוכח אלא במי שידע, דאידך אינו יודע קרי ליה.}
\textblock{\textbf{ליתני הכיר ולבסוף שכח וכל שכן הא.} איכא למידק דלמא רבותא קמ״ל, דאפילו יודע עיקר שבת אינו חייב על כל אב מלאכה ומלאכה וכדבעינן נמי למימר בסמוך אלא על כל שבת ושבת. ויש לומר דהכי קאמר ליתני הכיר ולבסוף שכח, וכל שכן היודע עיקר שבת, דאי לאשמועינן קילותא, ליתני היודע עיקר שבת אינו חייב [אלא] על כל שבת ושבת, וכדתניא ברישא דמתניתין השוכח עיקר שבת אינו חייב אלא אחת, אבל השתא דתני חייב על כל שבת ושבת, משמע דצריכותא לחיובא הוא, ואם כן ליתני הכיר ולבסוף שכח וכל שכן היודע עיקר שבת, ומכל מקום אף הוא צריך למיתני היודע עיקר שבת לאשמועינן נמי פטורא, וא״כ ליערב וניתני הכיר ולבסוף שכח, וכן היודע עיקר שבת חייב על כל שבת ושבת, והיינו נמי דבתר דפריק מאי היודע שידעה ושכחה, הדרינן ואקשינן ליה, אבל לא שכחה מאי חייב על כל מלאכה ומלאכה, אדתני היודע שהוא שבת כו׳, ליתני היודע עיקר שבת, כלומר אם כן מדלא תני תרתי הכיר ולבסוף שכח, והיודע עיקר שבת, אלמא דוקא שידעה ושכחה חייב על כל שבת ושבת בלבד, אבל היודע עדיין עיקר שבת חייב על כל אב מלאכה ומלאכה.}
\textblock{ והא דאמרינן:\textbf{ ליתני היודע עיקר שבת וכל שכן הא.} פירוש: ליתני הכיר ולבסוף שכח חייב על כל שבת ושבת והיודע עיקר שבת חייב על כל מלאכה ומלאכה, אבל אי תנא היודע עיקר שבת לחודא הוה אמינא דהיינו אפילו ידעה ולבסוף שכחה, דהא תני ליה השתא ומפרשינן ליה הכין.}
\textblock{ ודרב ושמואל הכי אתמר:\textbf{ רב ושמואל דאמרי תרוייהו אפילו תינוק שנשבה בין העכו״ם, וגר שנתגייר בין העכו״ם, כהכיר ולבסוף שכח דמי וחייב.} ואם תאמר אם כן ליערבינהו וליתנינהו בהדיא, וכדאקשינן לעיל למאי       גבי הכיר ולבסוף שכח. יש לומר דשוכח עיקר שבת כולל שני אלה.}
\textblock{\textbf{אמר לו הן וכל שכן שהוספת.} פירוש: שאף זה קרוי שוגג, אבל לאו למימרא דסבירא ליה למונבז שאינו שוגג אלא כשהיתה לו ידיעה בשעת מעשה, דהא תניא בסמוך שגג בזה ובזה זה שוגג האמור בתורה ואוקים לה כמונבז דפוטר, אלמא דכשלא היתה לו ידיעה בשעת מעשה מחייב. ויש לפרש דשלא היתה לו ידיעה כלל נפקא ליה לפטורא מבנין אב דמזיד קרוי חוטא ושוגג קרוי חוטא, והלכך כל שלא היתה לו ידיעה כלל לא דמי כלל למזיד ולפיכך פטור, ולחייבו אפילו בשהיתה לו ידיעה בשעת מעשה נפקא ליה מהיקשא דתורה אחת יהיה לכם לעושה בשגגה והנפש אשר תעשה ביד רמה (במדבר טו, ל), ואי ליכא אלא חד הוה אמרינן דאינו קרוי שוגג אלא עד שתהא לו ידיעה אפילו בשעת מעשה, השתא דאיכא בנין אב ואיכא הקישא מחד ילפינן לחיוב ומחד ילפינן לפטור. והיינו נמי דאיבעיא להו מאי טעמא דמונבז, כלומר: אי בכולה מלתא אבנין אב סמכינן אם כן אפילו לא היתה לו ידיעה בשעת מעשה ליפטר, ופשטוה דידיעה דשעת מעשה נפקא לן מהיקשא, ואהני בנין אב ואהני היקש וכדאמרן, ורש״י ז״ל פירש בענין אחר.}
\clearpage
\newsection{דף סט}
\textblock{\textbf{מה להלן דבר שזדונו כרת ושגגתו חטאת וכו׳.} איכא למידק, לימא מה התם דבר שזדונו סקילה. יש לומר דמהיקשא גופא דתורה אחת יהיה לכם לעושה בשגגה והנפש אשר תעשה ביד רמה (שם) קאמר, כלומר: בפירוש הוקשה כל התורה לכרת שבעבודה זרה, כלומר: לכרת הקישה ולא לסקילה. ואם תאמר מונבז הא מנא ליה. יש לומר כל שכן לדידיה דכולה נפקא לה מהכא דאין היקש למחצה, אלא דרבנן דמונבז סברי דאי אפשר להקישן לגמרי כמונבז, לפי שאין מקישים שוגג למזיד, ואחרים אומרים דמונבז נפקא ליה מגזירה שוה דעליה, וכדנפקא לן בריש פרק קמא דיבמות (ט, א) דגרסי׳ התם נאמר כאן חטאו עליה, ונאמר להלן לצרור לגלות ערותה עליה, מה להלן דבר שזדונו כרת ועל שגגתו חטאת, אף כאן דבר שזדונו כרת ועל שגגתו חטאת.}
\textblock{\textbf{עד שישגוג בלאו שבה.} פירוש: אף על פי שהוא יודע איסור עשה שבה כשב מידיעתו קרינן ליה אצל לאו שבה, דדלמא אי הוה ידע ליה ללאו הוה הדר ביה, והיינו דסמיך ליה ריש לקיש אקרא דאשר לא תעשנה.}
\textblock{\textbf{דידע לה בתחומין ואליבא דרבי עקיבא.} והוא הדין דהוה מצי למימר דידע לה לשבת בעשה דשבתון ותשבות, לפי מה שכתבנו. והכי נמי הוה מצי למימר דידע ליה בהבערה ואליבא דרבי יוסי דאמר (לקמן שבת ע, א) הבערה ללאו יצאתה, ואי נמי במחמר אחר בהמתו דלית ביה אלא לאו לכולי עלמא (לקמן שבת קנג, ב). ואי נמי איכא למימר דמשום הכי לא אוקמה בהבערה ואליבא דרבי יוסי, משום דעל כרחין דלא כרבי יוסי, דקא מני מבעיר מאבות מלאכות. ובמחמר נמי לא מוקי לה, דלא ניחא ליה למימר דידע ליה בשביתת בהמתו ולא ידע ליה בשביתת עצמו.}
\textblock{ והא דאמר ר׳ יוחנן\textbf{ שאם עשאן כולן בהעלם אחד חייב על כל אחד.} פירוש: לא לחלק אתא לאשמועינן, דמרישא שמעת מינה דקתני היודע שהוא שבת ועשה מלאכות הרבה חייב על כל אב מלאכה ומלאכה, אלא הא קא משמע לן מנינא דשגגת כרת שמה שגגה, אע״ג דידע ליה לשבת בלאו וכדדייקינן מינה הכא. ואי נמי דידיעת תחומין שמה ידיעה וכדמפרקינן לה אליבא דריש לקיש.}
\textblock{\textbf{הכל מודים בשבועת בטוי שאינו חייב עד שישגוג בלאו שבה.} פירש״י ז״ל: דאם נשבע שלא אוכל ואכל אינו חייב אלא א״כ שכח שבועתו כשאכל, דהשתא שגג בלאו שבה. והזקיקו לרש״י ז״ל לפרש כן, לפי שהוא ז״ל סבור שאילו שגג לגמרי בלאו שאינו יודע שיש בו איסור, לא מחייב קרבן דהאדם (ויקרא ה, ד) פרט לאנוס קרינן ביה, דכיון דלא ידע ליה כלל אנוס הוא וכמ״ש הוא ז״ל בפי׳ בסמוך. וגם הרמב״ם ז״ל כן הוא סבור וכמ״ש בפ״א מהלכות שבועות (הי״ג). וכן יראה לכאורה מהא דתניא בסמוך (בע״ב), איזהו שבועת ביטוי לשעבר, אמר יודע אני ששבועה זו אסורה אבל איני יודע אם חייבין עליה קרבן אם לאו, ולא אוקמה באוקימתא רויחא וכגון דאמר מותר ששגג בלאו ממש שבה.}
\textblock{\textbf{ואינו מחוור. דלא ממעטינן מהאדם בשבועה פרט לאנוס אלא אכל ושכח ונשבע שלא אכל, וכעובדא דתלמידי דרב (שבועות כו, א) דמר אמר שבועה הכי אמר רב ומר אמר שבועה הכי אמר רב, דסבור שקיים שבועתו אלא שלבו אנסו וכדאמר ליה רב את לבך אנסך, אבל טועה באיסורא דסבר מותר זה אינו אונס אפילו בשבועה אלא חייב כמו שהוא חייב בכל שאר איסורין שבתורה, כדתניא לעיל (שבת סח, ב) גבי תינוק שנשבה בין העכו״ם וחייב על הדם אחת ועל החלב אחת ועל העבודה זרה אחת, וכן כשהיתה עיקר של שבת שכוחה      } ממנו, וכל שכן אם הכיר ולבסוף שכח שהוא חייב לכולי עלמא. וכן כתב רבנו שמואל ז״ל. וההיא דלקמן אינה ראיה כדכתבינן לקמן, אלא הכי פירושא: עד שישגוג בלאו שבה, שסבור שלא הזהירה התורה על שבועת בטוי ואומר מותר.}
\textblock{ הכי גריס רש״י ז״ל:\textbf{ הא מני מונבז היא.} ופירש הוא ז״ל: אבל רבנן לית להו שבועת בטוי לשעבר כלל. והקשו עליו בתוס׳, דרבנן דמונבז מאן רבי עקיבא, ור״ע הא שמעינן ליה בהדיא בשבועות (כה, א) דשבועות בטוי ישנה לשעבר ולהבא. ואחרים גורסים הא מני אילימא מונבז פשיטא, השתא בכל התורה כולה דלאו חדוש הוא אמר מונבז שגגת קרבן שמה שגגה הכא דחדוש הוא לא כל שכן, אלא לאו רבנן ותיובתא דאביי תיובתא. וזה הנכונה שבגירסאות. אבל רש״י ז״ל הקשה דמאי פשיטותא, אדרבה איצטריך לאשמועינן דחלוקה שגגה זו משאר שגגות דאילו בכל התורה כולה למונבז בין שגג בקרבן בין שגג בלאו וכרת שמה שגגה, וכאן לשעבר שגג בלאו שבה לאו שגגה היא דהאדם בשבועה פרט לאנוס. ואין קושייתו מחוורת וכמו שכתבנו לעיל דלא אמעיט מן האדם בשבועה פרט לאנוס אלא בשאכל ושכח וסבור שלא אכל ונשבע שלא אכל, אבל אומר מותר לאו בכלל אנוס הוא, והלכך הא דקתני בברייתא איזו הוא שבועת וכו׳ כפשטא.\par \textbf{} בחידושי רבנו שלפנינו למכילתין יש דילוג מדף סט ע״ב עד דף צ ע״ב. אמנם המעתיקים או המדפיסים כתבו בסוף פרק כלל גדול: ״לא חבר הרב ז״ל יותר בפרק זה גם בפרק המוציא ובפרק ר׳ עקיבא לא חבר כלום״ אך אין דבר זה מתקבל על הדעת, ואכן מצאנו בספרי הראשונים שאחרי רבנו, שמעתיקים מחידושיו גם לפרקים אלו, ומה שעלה בידינו הצגנום כאן על מקומם עם ציון המקור.}
\textblock{\textbf{אימא אף משום זורע.} ומורי נרו יאיר מפרש דרב פריש לן דכי קתני כולן מלאכה אחת הן דמשמע דהוו אבות, היינו משום זורע ונוטע, אבל כולהו אידך תולדות נינהו.}
\textblock{\textbf{מחק אות גדולה ויש במקומה לכתוב שתים חייב.} פירוש: משום דהוה ליה כמוחק שתי אותיות, ואע״ג דלא בעינן כדי שיהא מקום לכתוב שתי אותיות גדולות כאותה שמחק דהא לא אפשר, מכל מקום בעינן כעין אות שמחק שיהו בינונית, ומאותה כתיבה שמחק, שאם היא כתיבה אשורית בעינן נמי אשורית, דאי לא אינו חייב לפי שאין לך מוחק אות קטנה אשורית שלא יהא בה כדי לכתוב שתי אותיות בכתיבה שאינה אשורית כגון מיש״ק שלנו, וא״כ מאי גדולה, ולא קתני גדולה אלא לאפוקי שיהא מקום כעין אותה שמחק לכתוב שתי אותיות בינוניות מאותה אות שמחק, שאם הוא אל״ף שיהא מקום לכתוב שני אלפין בינונים, ואם הוא בי״ת שיהא מקום לכתוב שני ביתין בינונים או אחרות דומיא דהני, דאי לא אין לך מוחק אות אפילו בינונית שלא יהא בה לכתוב שני ווי״ן או שני יודי״ן ומאי גדולה דקתני, דאין לומר מאי גדולה הוא הדין בינונית, ולאפוקי וי״ו או יו״ד קא אתי שאם מחקם אינו חייב משום דלא הוו גדולות, דזה דחוק הוא מאד, ועוד דתינח באותיות בכתיבה מאי איכא למימר.\par \textbf{} ומיהו מכל מקום לא בעינן שיהו שם דומיא דשם משמעון או יה מיהודה דלא נאמרו אלו אלא כדי לחייב משום מלאכת מחשבת בלבד, דסלקא דעתך אמינא כיון דאין במחשבתו אלא לכתוב שמעון או יהודה לא ליחייב, קא משמע לן כיון דשם הוא חייב, אבל בכותב שתי אותיות באיזה אות שתהיה חייב, דהא הוו בקרשים ז״ח ט״י כדי לזווגן ואף על פי שאינן שם. כן ראיתי לתלמיד אחד בשם רבו הרשב״א ז״ל.}
\newchap{פרק \hebrewnumeral{8} המוציא יין}
\clearpage
\newsection{דף עט}
\textblock{}
\textblock{\textbf{ורבי יהודה סבר אין כותבין שובר.} [פירש רש״י ז״ל (ד״ה והכא בסופו): ואע״פ שכבר פרעו צריך להחזיר לו מעותיו אם לא ימצא השטר, ודין זה מחודש שלא נמצא בשום מקום אלא בזה. וכתב הרא״ה ז״ל דהוא הנכון  שבפירושים]. וראיתי לאחד מן הגדולים שכתב בשם הרשב״א ז״ל דנפקא מינה לדידן דקיימא לן כרבנן דכותבין שובר, שאם יש נאמנות בשטר כגון שכתוב בו ויהא נאמן כשני עדים כשרים לעולם, שאין כותבין שובר, ואם לא יתן לו שטרו מחזיר לו מעותיו ואכיל הלה וחדי, ואינו מספיק לא חותם ידו ולא העדאת עדים לפי שכבר האמינו כשני עדים. ומיהו אם כתב לו בבית דין שובר הא ודאי אין מחזיר לו כלום, אלא אדרבה מגבין אותו לכתחילה בבית דין כיון דקיימא לן כותבין שובר. עד כאן.}
\newchap{פרק \hebrewnumeral{9} אמר רבי עקיבא}
\textblock{}
\textblock{\textbf{אלא [זה] זבוב בעל עקרון.} תימה למה הביאו זו הברייתא, אי למימר דזבוב עבודה זרה היא, פשיטא, דהא קרא כתיב (מלכים-ב א, ב) בבעל זבוב אלהי עקרון. ואיכא למימר אי מקרא הוה אמינא גוף גדול הוא אלא שנקרא כן, אבל השתא דכתיב (שופטים ח, לג) וישימו, משמע שמשימין אותה בתוך חיקם, אלמא גוף קטן הוא. ואין זה נכון דאף על פי כן אפשר שהיא [גדולה] מכזית. ורש״י ז״ל פירש שזבוב הוא כמשמעו אבל לכך צריך לברייתא לומר שהיא עבודה זרה גמורה ואדוקין בה, מלשון ברית שהיא לשון אהבה וחיבה, כלומר: שלא תאמר בעקרון הוא שעשו כך לזכר עבודה זרה בעלמא, ואינה עבודה זרה גמורה ואינן אדוקין בה. ובירושלמי (פ״ט, ה״א) מצאתי: אבל בע״ז שלימה אפילו כל שהוא דא״ר יוסי בר בון בשם ר׳ חמא בר גוריא הבעל ראש גויה היה ובאטין היה, וישימו להם בעלי ברית לאלהים, פירוש ברית הוא ראש הגויה שבו הברית, והוא נמי לשון בעל, ונראה שהגירסא (כאפין) [באפין] היה שהיה קטן כגרעין של אפונין [כן הוא בירושלמי כת״י ליידן, וכן הוא בירושלמי ע״ז פ״ג, ה״ו. ועיין רמב״ן]. ולפי זה לא מייתי ראיה משום לשון זבוב אלא מלשון בעל ברית, לומר שהיו אדוקין בצורת ראש הגויה שנקרא בעל, והא דקתני זה זבוב בעל עקרון, מפני שנקרא בעל אומר כן, או שהוא דבר קטן כזבוב. הרשב״א ז״ל. [מלקוטי רבנו בצלאל אשכנזי שבת, נדפס בסוף ספר הבתים. וכן הוא בחידושי הרמב״ן, עיש״ה].}
\textblock{ מתני׳:\textbf{ מנין לערוגה שהיא ששה על ששה טפחים שזורעין בתוכה חמשה זרעונים וכו׳.} צריך להרחיק בין זרע לזרע ג׳ טפחים כשיעור יניקת זרעים (ב״ב יז, א), ואע״ג דכל זרע יונק עד ג׳, לא חשו לערוב יניקות, והיינו דקאמר לא ינקי מהדדי ולא קאמר לא ינקי אהדדי דלא חשו אלא שלא ינקו זרע זה מהזרע האחר עצמו. [ביאורי מהר״י קורקוס הל׳ כלאים פ״ד ה״ט בשם הרשב״א].}
\textblock{\textbf{הרוצה למלאות כל גינתו ירק עושה ערוגה ועוגל בה חמשה.} [פירש רש״י עוגל בתוכה חמשה עגולין של ה׳ אמות וזורען עד לאחד. ופריך והאיכא ביני ביני, כלומר, הרי יש גבול העגולה, ומשני אה״נ במחריב הביניים, וכל גינתו דקאמר היינו תוכן של ערוגות ואגבול לא קפיד] והקשה הרשב״א על זה דאיך עלה בדעתו שיזרע הגבול והלא אין דרך לזורעה ותו דהיה לו להקשות שאם יזרע הגבול הוי ערבוב. לכן פירש ולא פריך אלא על חצי טפח שבין עוגל למרובע, דבשלמא קרנות זרע להו כיון שמה שבפנים הוי בעוגל איכא היכירא, אבל אותו חצי טפח איך יזרע אותו שלא יהיה ניכר. ומשני אה״נ דיניח אותו חרב, וכל גינתו קרי ליה דלחצי טפח לא קפיד ולא חשיב ליה. ורב אשי אמר אם היה העוגל זרוע שתי זורע אותו חצי טפח ערב וכו׳. [מהר״י קורקוס שם הל׳ טו].}
\textblock{\textbf{אבל חכמים אומרים שש עונות שלמות.} הכי גרסינן: אבל חכמים אומרים שלש עונות, זוהי גירסת הרשב״א. [כסף משנה הלכות אבות הטומאה פ״ה הי״א].}
\clearpage
\newsection{דף פז}
\textblock{\textbf{שלשה דברים עשה משה מדעתו.} והרשב״א ז״ל כתב קצת דומה לזה. וכתב אחר כך וז״ל: וצריך עיון מנא לן שהסכים הקב״ה עמו אי משום אשר [שברת] הרי כמה כתובים בתורה שאינן לשון אושר.\par \textbf{} ושמעתי משום דכתיב (דברים י, ב) אשר שברת ושמתם, לוחות ושברי לוחות מונחים בארון, ואלמלא היה בשבירתן חטא אין קטיגור במקום סניגור אלא מלמד שהיתה שבירתן חביבה לפניו, ומדרש אגדה (שמו״ר פמ״ו, פ״ג) יהושע ושבעים זקנים תפסו בידיו שלא ישברם ולא יכלו לו, אמר הקב״ה תהא שלום באותה יד דכתיב (דברים לד, יב) ולכל היד החזקה אשר עשה משה, ואפשר דמשום הכי דריש האי אשר לשון אשרי. [הכותב בעין יעקב].}
\textblock{\textbf{מאי דריש וכו׳.} וקשיא אי מדרש דרש לאו מדעתיה היה, ולא הסכים על ידו היה, והרבה כיוצא בו [עשו] משה ושאר נביאים. אלא ודאי לשלישי אמר, אלא גלוי היה לפניו דיוסיף משה, ולפיכך מה שאמר לו ליום השלישי חזר ואמר היום ומחר כדי שיהא ברצונו של משה רבנו להתלות במדרשו ולא יהא כמעביר על דבריו במה שכתוב ליום השלישי אבל לא שיהא משה רבנו ע״ה מוכרח לדרוש כן, שאפילו בחצי היום שייך [למימר היום ומחר]. הרשב״א בחידושיו. [לקוטי רבנו בצלאל אשכנזי. ועי׳ רמב״ן].}
\textblock{\textbf{ואתה פה עמוד עמדי (דברים ה, כח).} [וא״ת מנלן דפירש משה מדעתו קודם ושוב הסכים הקב״ה על ידו, שמא זה היה צווי גמור שצוה לו לפרוש. תוספות]. והרשב״א תירץ אילולי שמדעתו עשה, שכינה למה אמרה לו כן לאחר מתן תורה והלא אף מתחילה היה מדבר עמו בכל שעה, אלא ודאי שלא נצטוה עוד עד שפירש הוא מדעתו מקל וחומר. ומיהו קל וחומר גופיה לאו דוקא, דהא מצי משמש וטובל ומדבר בכל יום, אלא הוא מדעתו נשא קל וחומר בעצמו להתקדש שיהא ראוי לדיבור בכל עת. [לקוטי רבנו בצלאל אשכנזי. ועי׳ רמב״ן].}
\textblock{\textbf{ומה פסח.} קשיא לי מאי מדעתו הוה, והלא ק״ו דרש. אלא ק״ו בעלמא עשה לומר שלא יהא נתפש בשבירת הלוחות אבל אינו קל וחומר גמור, דאדרבא צריכים הם לתורה כדי שיחזרו ויעשו תשובה לכן אמר לו יישר כחך ששברת. חידושי הרשב״א. [לקוטי רבנו בצלאל אשכנזי וכן הוא ברמב״ן].}
\clearpage
\newsection{דף פח}
\textblock{\textbf{מכאן מודעא רבה וכו׳ קימו מה שקבלו כבר.} הא דאמרינן באגדה מכאן מודעא רבה, ומתרץ קבלוה בימי אחשורוש, ק״ל וכי מה קבלה זו עושה מסופו של עולם לתחילתו, אם קודם אחשורוש לא היו מצווין למה נענשו, ואם נאמר מפני שעברו על גזירת מלכם, א״כ בטלה מודעא זו, ועוד למה הצריכם לקבלה וברית. ונ״ל דמתחילה אע״פ שהיה להם מודעא מ״מ לא נתן להם את הארץ אלא כדי שיקיימו את התורה, כמו שמפורש בתורה בכמה פרשיות וכתיב ויתן להם ארצות גוים (תהלים קה, מד) וגו׳ בעבור ישמרו (שם שם, מה) וגו׳, והם עצמם בתחלה לא עזבו בדבר כלל ולא אמרו במודעא כלום, כאלו מסרו נפשם מעצמם אמרו כל אשר דבר ה׳ נעשה ונשמע (שמות יט, ח), ולפיכך כשעברו על התורה עמד והגלם מן הארץ. [אבל] משגלו יש מודעא על הדבר דכתיב (יחזקאל ב, לב) והעולה על רוחכם היה לא תהיה אשר אתם אומרים נהיה כגוים וגו׳ וכדאמרינן באגדת חלק (סנהדרין קה, א) עבד שמכרו רבו כלום יש לו עליו, לפיכך כשבאו לארץ בביאה שניה בימי עזרא עמדו מעצמן וקבלוה ברצון שלא יטענו עוד שום תרעומת והיינו בימי אחשורוש שהוציא ממות [לחיים] והיה זה חביב להם יותר מגאולת מצרים. [הכותב בעין יעקב בשם הרשב״א. ועי׳ בעץ יוסף שם].}
\textblock{ מתני׳:\textbf{ המצניע לזרע ולדוגמא ולרפואה כו׳.} תמיהא לי למה הוצרך לחזור ולשנות כאן, שהרי כבר שנינו כן בפרק כלל גדול (שבת עה, ב) דתנן התם כל שאינו כשר להצניע, ואין מצניעין כמוהו, והוציאו בשבת אינו חייב אלא המצניעו. ויש לומר דאתא לאשמועינן דאפילו שכח למה הצניעו והוציאו, חייב שעל דעת ראשונה הוא עושה, וכדאוקים לה אביי.}
\textblock{ גמרא:\textbf{ למה ליה למיתני המצניע ליתני המוציא.} כלומר: דאפילו לא החשיבו מתחלה להצנעה אלא שהוציאו עכשיו לזריעה חייב דהא אחשביה, ודוקא במוציא לזריעה וכיוצא בזה, לפי שאין אדם טורח לזרוע נימא אחת, וכדאמרינן בפרק המוציא (עט, א) אמר רב פפא הא דזריע הא דלא זריע, לפי שאין אדם טורח להוציא נימא אחת לזריעה, אבל אי טרח ואפקיה לכך חייב דדבר חשוב הוא לכך, מה שאין כן במוציא לאכילה שאפילו הצניעו והוציאו לאכילה אינו חייב בפחות מגרוגרת. ותדע לך מדאמרינן לקמן (שבת צא, א) בעי רבא הוציא חצי גרוגרת לזריעה ותפחה ונמלך עליה לאכילה מהו, דאלמא דוקא תפחה, הא לא תפחה אף על פי שחשבה ממש לאכילה פטור ומן הטעם שאמרנו. כך נראה לי.}
\newchap{פרק \hebrewnumeral{10} המצניע}
\clearpage
\newsection{דף צא}
\textblock{}
\textblock{\textbf{מתקיף לה רב יצחק בריה דרב יהודה וכו׳.} פירש רש״י ז״ל: דאדאביי קאי דאוקי מתניתין בששכח למה הצניעו, ואפילו הכי אזלינן בתר מחשבתו הראשונה. ואינו מחוור. שאם כן היה לו לבעל הגמרא לסדר דברי ר׳ יצחק אחר דברי אביי ולא יפסיק ביניהם בהא דרב יהודה. ובתוספות פירשו דאהא דרב יהודה אביו קאי, משום דפירש כל שהוא היינו אפילו חטה אחת הואיל ועל ידי מחשבתו אתה מחייבו בדבר שאינו ראוי להוציאו אפילו לזריעה שאין אדם טורח להוציא נימא אחת לזריעה, אלא שאנו הולכין לגמרי אחר מחשבתו, אם כן חשב להוציא כל ביתו הכי נמי. ופרקינן התם בטלה מחשבתו אצל כל אדם, כלומר: שאין כח בידו שלא להחשיב מה שהוא חשוב אצל הכל, אבל להחשיב מה שאינו חשוב אצל הכל לטרוח בכך לכתחלה הרשות בידו, כיון שאף השיעור הזה ראוי הוא וחשוב אצל הכל אלא שאין טורחין בכך לכתחלה.}
\textblock{\textbf{פשיטא זיל הכא איכא שיעורא וזיל הכא איכא שיעורא.} קשיא לי, מאי קושיא שהרי דרך התלמוד הוא כן בהרבה מקומות לפתוח בדבר פשוט כדי לשאול אחריו מה שצריך לשאול, ואף כאן מפני שרצה רבא לשאול אחריו הוציא חצי גרוגרת לזריעה. והוציא כגרוגרת לאכילה, עשה להן פתח מזה אע״פ שאינו צריכה, ולמה הוצרך לתת טעם מהו דתימא בעינן עקירה והנחה בחדא מחשבה, בלאו הכי ניחא. ובפרקין (צב, ב) תנינן המוציא ככר לרשות הרבים חייב, והא אינה צריכה אלא משום דבעי למיתני הוציאוהו שנים פטורין. וניחא לי דמרא דשמעתא קמייתא ר׳ נחמן, ואיהו לא בעי הנך בעייתא, אלא רבא הוא דאיבעיא ליה, ומשום הכי אקשינן לר׳ נחמן, למה ליה לאשמועינן הכי פשיטא.}
\textblock{ בהא ד\textbf{בעי רבא הוציא חצי גרוגרת לזריעה ותפחה ונמלך עליה לאכילה מהו.} הקשה הרמב״ן ז״ל, מאי קא מיבעיא ליה, דהא איהו הוא דאמר במסכת מנחות בפ׳ כל המנחות באות מצה (מנחות נד, א), כל היכא דמעיקרא הות ביה והשתא לית ביה הא לית ביה, וכל היכא דמעיקרא לית ביה והשתא אית ביה מדרבנן, כי פליגי דמעיקרא הוה ביה וצמק וחזר ותפח, מר סבר יש דיחוי באיסורין, ומר סבר אין דחוי באיסורין, וכיון שכן גבי שבת נמי בדלית ביה ותפח היכי מחייב.\par \textbf{} ותירץ הוא ז״ל, דהכא בגרוגרת ממש עסקינן, ואמרינן התם שאני גרוגרת הואיל ויכול לשלקן ולהחזירן לכמות שהיו, וכיון שכן כשתפחה יש בה שיעור, ומיהו בעוד שלא תפחה לא חשיבא שיעור ואפילו למאן דאמר התם (שם נה, א) גבי תרומה דשיעור הוא. ואינו מחוור בעיני כל הצורך, דמאי שנא שבת מאי שנא תרומה, הא אפילו קודם ששלקה חשבינן ליה התם לענין תרומת מעשר כתאנה מדתורמין גרוגרת על התאנים במנין, ואף על גב דהשתא בציר ליה שיעורא.\par \textbf{} ולי נראה דאינו דומה שיעור הוצאת שבת לשאר האיסורין, דאילו פגול ונותר וחלב לא חייבה התורה בהם אלא בשיעור אכילה דהיינו כזית, וכיון שכן כל שאין בו כזית אף ע״פ שתפח ונראה ככזית הרי אין בו כזית ולא נהנה גרונו בכזית, אבל הוצאות שבת בחשיבותא תליא מלתא, וכל שדרך הבריות להחשיב ולהצניע חייבין עליו בהוצאתו, ועל כן אין שיעור הוצאה אחד לכל הדברים אלא כל אחד ואחד לפי מה שהוא חשוב, עד שהעלו הענין שאפילו מה שאינו חשוב לכל ואין מצניעין כמוהו והצניע המצניע חייב, וכיון שכן אף פחות מכגרוגרת שתפחה ונראה (ככזית) [כגרוגרת] מחשיבין אותו ומצניעין כמוהו ולפיכך חייב. כך נראה לי.}
\textblock{\textbf{זרק כזית תרומה לבית טמא מהו.} ואסיקנא כגון שהיה פחות מכביצה אוכלין וכזית זה מצטרף לכביצה, ומשום דאיסור שבת ואיסור טומאה בהדי הדדי אתיין, ומשום הכי נקט זרק ולא נקט הוציא, דאילו הוציא קדים ליה איסור שבת לטומאה, דכשיצרף ידו למטה משלשה מיד הוה ליה כמונח לענין שבת, ואפילו לרבנן דבעי הנחה על גבי משהו, הני מילי בזורק אבל במעביר לא וכדאמרינן לעיל בפרק המוציא (שבת פ, א), ולענין טומאה לא נטמא עד שנתחבר ממש לאוכלין. ונקט נמי תרומה דאילו חולין לא עבד ולא מידי, שהרי החולין מותרין הן באכילה אף לאחר שנטמאו.}
\textblock{\textbf{ואם תאמר אע״ג דמצטרף לענין טומאה היאך יתחייב לענין שבת, דהא בעינן עקירה והנחה בדבר חשוב ואילו בהנחה איכא דבר חשוב כיון דמצטרף לאוכלין, אבל בשעת עקירה דלא היה שם אלא כזית אינו חשוב. יש לומר עקירה צורך הנחה היא, וכיון דבשעת הנחתו יהא חשוב, אף עכשיו חשבינן ליה כדבר חשוב, דהא לצורך אותו הנחה עקרה, ודאי חייב, ואילו זרקו לבית טהור פשיטא ליה דחייב הואיל וזר לוקה עליו בכזית גם ראוי לאכילת כהנים, אבל כשנטמא בטל חשיבותו דאף זר אינו לוקה עליה, וכדאמרינן במסכת      } יבמות בריש פרק הערל (יבמות עג, ב) דהאוכלה בטומאת עצמה אינו לוקה עליה, וליכא אלא איסור עשה דכתיב (דברים טו, כב) בשעריך תאכלנו לזה ולא לאחר ולאו הבא מכלל עשה עשה, והיינו דלא בעי לה הכא אלא שזרקו לבית טמא ומשום צירוף אוכלין שבה, והלכך בשעת עקירה ודאי שיעור הראוי לענין שבת הוה ביה, כן תירץ רבנו שמואל.\par \textbf{} אבל הרמב״ן ז״ל הקשה עליו שאם מפני שראויה היתה לאכילת כהנים אם כן אף החולין ראוין לאכילת כל אדם, וכן בהא נמי דמשני בלחם הפנים מדאפקיה איפסיל ליה, אכתי קשיא דהא חשיב הוא דזר וכהן לוקין עליו בכזית. ועוד שאם כן כל האוכלין האסורין ליחייב בכזית. ונראה לי שעיקר טעם רבי שמואל ז״ל אינו אלא מפני שהיא ראויה לאכילת כהנים כלומר: מחייבת להם הכתוב ושמרתם את משמרת תרומותי (במדבר יח, א), מה שאין כן בחולין, ועם דברי רבי שמואל ז״ל עלה לנו תירוץ למה נקט כזית, הואיל ואינו מתחייב אלא משום צירופו ואם כן אפילו פחות מכזית, אבל עם דברי הרב ז״ל אינו קשה, דמשום עקירה נקט כזית דפחות מכזית אינו כלום.\par \textbf{} ועוד יש לומר דדוקא כזית משום דחשיב בעלמא ללקות עליו, מה שאין כן בפחות מכזית, דאפילו לרבי יוחנן דאמר (יומא עג, ב) חצי שיעור דאורייתא איסורא בעלמא הוא דאיכא. ואע״ג דלגבי תרומה טמאה ליכא חילוק בין כזית לפחות מכזית, דאפילו בכזית ליכא אלא עשה כדאמרן, ובפחות מכזית נמי איכא איסורא דאורייתא מדרבי יוחנן דאמר חצי שיעור דאורייתא וקיימא לן כותיה, מכל מקום כיון דבעלמא חמירא כזית ללקות עליו ופחות מכזית לעולם לא, הכא נמי כזית הוי חשוב פחות מכזית לא הוי חשוב.}
\textblock{ הא דנקט הכא:\textbf{ כגון דאיכא אוכלין פחות מכביצה.} ולא אמר דאיכא פחות מכביצה תרומה, מסתברא דאפילו איכא פחות מכביצה אוכלין חולין, והדין נותן דהא תלמודא משום צירוף הוא דבעי, ואפילו על ידי צירוף דחולין מיטמא כזית של תרומה. אבל ראיתי לרבינו האי גאון ז״ל שכתב אי לענין טומאה כביצה אוכלין בעינן, שאם טימא תרומה פחות מכביצה פטור, ואהדרינן לעולם לענין שבת וכגון שזרקו לאותו בית טמא תרומה פחות מכביצה ואיתא התם וזרק עכשיו כזית שהוא משלימו לההוא דאיתיה התם לכביצה. ע״כ. ואיני יודע למה הוצרך לפרש כך, דאפילו כזית מקבל טומאה על ידי צירוף ומתחייב הוא עליה משום מטמא תרומה.\par \textbf{} ומיהו דוקא בשלא היה שם אלא פחות מכביצה אוכלין, שלא ירדה להן טומאה אלא ע״י צירוף הכזית דאז מקבל טומאה אף כזית זה, אבל אם יש שם כביצה אוכלין כבר נטמאו בלא צירוף כזית זה, ואע״פ שנצטרף להם כזית זה לא נתוסף להם טומאה שכבר שבעו להם טומאה, וכיון שכן אין מצטרפין עם זית זה שיקבל טומאה על ידיהם כאילו הוא משיעור הכביצה, ואינו אלא כזית העומד לעצמו שאינו מקבל טומאה ולפיכך נקט הכא פחות מכביצה וכענין שאמרו במנחות בפרק הקומץ רבה גמ׳ (כד, א) שתי מנחות שלא נקמצו ונתערבו, בעי רבא עשרון שחלקו ונטמא אחד מהן והניחו בכסא וחזר טבול יום ונגע באותו טמא מהו, מי אמרינן שבע ליה טומאה או לא, כלומר: שבע ליה טומאה אותו חצי שנטמא כבר ולפיכך אינו מצטרף לטהור לטמאו, ואע״פ שאם היו שניהם טהורין ונגע באחד מהם פשיטא ליה דמצטרף.}
\textblock{ מהא דאמרינן הכא:\textbf{ למאי אי לענין טומאה כביצה אוכלין בעינן.} שמעינן דאין אוכלין מקבלין טומאה פחות מכביצה, והכי נמי משמע מדתנן במסכת אהלות (פי״ג, מ״ה) אלו ממעטין בחלון פחות מכביצה אוכלין, וקתני סיפא (במ״ו) זה הכלל, הטהור ממעט והטמא אינו ממעט, אלמא פחות מכביצה אינו מקבל טומאה לא מדברי תורה ולא מדברי סופרים שאילו כן לא היה ממעט.}
\textblock{\textbf{והא דתניא בספרא (ויקרא יא, לד) מכל האוכל מלמד שהוא מטמא בכל שהוא. יש מפרשים אותה בצירוף כלי כגון ההיא דתניא (פסחים מד, א) המקפה של תרומה והשום והשמן של חולין ונגע טבול יום במקצתן פסל את כולן. והא דתניא בפרק קמא דחולין גמרא (כד, ב) הטהור בכלי חרס       } תוכו אע״פ שלא נגעו כו׳, נאמר תוכו לטמא ונאמר תוכו ליטמא מה תוכו האמור לטמא אע״ג שלא נגע, תוכו האמור ליטמא גם כן אע״פ שלא נגע, ואמרינן עלה התם מנא לן אמר רבי יונתן התורה העידה על כלי חרס ואפילו מלא חרדל, כלומר ואילו מחמת נגיעה לא היה אפשר שיטמא אלא לכל היותר בגרגיר שלישי, אלא שכל גרגיר וגרגיר מיטמא מאויר כלי, דאלמא אפילו כל שהוא מקבל טומאה. יש לומר דלאו דוקא מלא חרדל, דהתם לא איירי בהאי דינא דכביצה ופחות מכביצה, אלא הכי קאמר התורה העידה על כלי חרס ואפילו מלא חתיכות כמלא חרדל, ולעולם כשיש בהן חתיכות גדולות כביצה. ורש״י ז״ל כתב בהרבה מקומות (פסחים לג, ב ועוד) דכל שהוא אוכלין מקבלין טומאה, אבל לטמא אחרים בעינן כביצה. ושמעתין דהכא לכאורה הויא תיובתיה, וההיא נמי דאמר באהלות הויא תיובתיה.}
\textblock{\textbf{ואי סלקא דעתך אגד כלי שמיה אגד, קדים ליה איסור גניבה לאיסור שבת.} כתב רש״י ז״ל: במסכת כתובות (לא, ב) פריך דאפקיה היכא, אי דאפקיה לרשות הרבים איסור שבת איכא איסור גניבה ליכא, ומוקי לה בצדי רשות הרבים, אי נמי דצירף ידו למטה משלשה וקבלה. ולפי דבריו הכי פירושא, קדים ליה איסור גניבה דמכיון שיצא מקצת הכלי בחוץ וקבלו הוא בידו, כל פרוטה ופרוטה שבכיס קנהו בהגבהה, כיון שיצא דרך פיו שהוא יכול להוציא המעות משם, ואי נמי אפילו דרך שוליו משום דאי בעי מפקיע לחלמה. ואוקימנא בנסכא, ומהא שמעינן דבכליו של מוכר היכא דלא משתקיל ליה לא קני ואפילו בהגבהה, ויש מפרשים דהכא סבירא לן כמאן דמוקי לה התם בדאפקיה לרשות הרבים, ולענין מיקנה ממש לא קני, ומיהו לענין גניבה מיחייב, מכיון דאפקיה מרשות בעלים ואי בעי שקיל ליה.}
\textblock{ כך היא גירסת הספרים:\textbf{ והא איכא מקום חלמה דאי בעי מפקע לה, בנסכא, וכיון דאיכא שנצין מפיק ליה עד פומיה ושרי ושקיל, ושנצין אגיד מגואי, דליכא שנצין, אי בעית אימא דאית ליה ומכרכי.} אבל רבנו האי גאון ז״ל גריס: והא איכא שנצין, דליכא שנצין, אי נמי אית ומכרכי. ופירש משמא דרבוותא ז״ל דקושיא היא אמאן דאיצטריך לאוקומיה בדמפיק ליה דרך שוליו ובנסכא, כלומר למה לן לדחוקי ולאוקמה בהכין, אפילו תימא דמפיק דרך פיו לא מצי שקיל עד דנפק כוליה משום דשנצין שבפיו קשורין שכן דרך אותן כיסין, ופריק דליכא שנצין, כלומר דניחא ליה לאוקומה בכל כיס אפילו דלית ליה שנצין, ואי נמי דאית ליה ומכרכי עלויה.}
\clearpage
\newsection{דף צב}
\textblock{\textbf{איפוך.} יש מפרשים דלגמרי מהפכין דרבא לדאביי ודאביי לרבא, כלומר: דאביי אמר ביד פטור בכלי חייב ורבא אמר ביד חייב בכלי פטור, ואע״ג דאמר רבא לקמן (שבת ק, א) בעינן הנחה על גבי משהו, הני מילי בזורק אבל במעביר לא וכדאמרינן לעיל בפרק המוציא (שבת פ, א) גמרא דיו לכתוב בו שתי אותיות. ואיכא דקשיא ליה, דהא ר׳ אבהו אוקי מתניתין דפשט העני את ידו לפנים ונתן לתוך ידו של בעל הבית, בשלשל ידו למטה משלשה (לעיל שבת ה, א), ואפילו הכי קתני נטל בעל הבית מתוכה שניהם פטורים, ורבא לא פליג עליה, אלא דקשיא ליה איכפל תנא לאשמועינן כל הני. ויש לומר דרבא אעיקר דינא נמי פליג עליה, אלא דלא אשכח קושיא דלדחייה מההיא סברא, אלא משום דאי אפשר דליכפל תנא לאשמועינן כל הני. ואי נמי יש לומר דבדינא דרישא בלחוד דהיינו נתן לתוך ידו של בעל הבית הוא דלא פליג רבא עליה דבדידיה איירי רבי אבהו, אבל בסיפא דהיינו נטל בעל הבית מתוכה, אי למטה משלשה כרישא חיובי מחייב רבא ופליג אדרבי אבהו, ומשום דרבא סבר אגד יד לא שמיה אגד, ורבי אבהו סבר דאגד יד נמי שמיה אגד, ואי נמי דידו בתר גופו גרירא.\par \textbf{} וכתב רב האי גאון ז״ל דלכולהו ביד חייב וליכא מאן דפטר, אלא מאן דפטר בשלא שלשל ידו למטה מג׳, ומאן דלא פטר בשלשל ידו למטה מג׳, וכדאוקי׳ מתניתין רבי אבהו דנתן לתוך ידו של בעל הבית, ואע״ג דהא אקשינן עליה איכפל תנא לאשמעינן כל הני, ולא הביא במתניתין, מימרא מיהא לא בטל הוא ולא נפק הדין טעמא מהלכתא. ונראה דהוא ז״ל מפרש דהא דפרקינן הכא, התם למעלה משלשה הכא למטה משלשה אוקימנא לכולהו, כלומר: התם למעלה משלשה, והוא הדין לדאביי, ורבא למטה משלשה, ורבי אבהו רישא דמתניתין בלחוד הוא דמוקי בלמטה משלשה, אבל סיפא דמתניתין בלמעלה משלשה הוא וכדאוקימנא לה הכא. וזה דחוק.\par \textbf{} ויש מפרשים דכלי בלבד הוא דמפכינן, אבל יד כדקאי קאי, אביי אמר בין ביד בין בכלי חייב, ורבא אמר בין ביד בין בכלי פטור, ורבא בעיקרא דמימרא פליג על ר׳ אבהו, וכי אקשינן הכא וביד חייב אדאביי בלבד הוא דאקשינן.}
\textblock{ מתני׳:\textbf{ לאחר ידו ברגלו ובפיו.} פירוש: שלא דרך אכילה, הא בשעה שהוא אוכל ונתכוון להוציא כך חייב, דמחשבתו משויא ליה מקום. וכדתנן בכריתות (יג, ב) יש אוכל אכילה אחת וחייב עליה ארבע חטאות ואשם אחד כו׳, ר״מ אומר אם היה שבת והוציאו בפיו חייב, ואמרינן עלה לקמן בשלהי פרק הזורק (שבת קב, א) אמאי והא אין דרך הוצאה בכך, ופרקינן כיון דקא מכוין מחשבתו משויא ליה מקום וחייב.}
\textblock{\textbf{ובמרפקו.} פירש רש״י ז״ל: מרפקו שחיו שקורין אישילא״ש. והקשו עליו דמרפק היינו אציל וכדמתרגמינן על כל אצילי יָדַי (יחזקאל יג, יח) מרפק ידיה, ועוד דתנן בערכין פרק האומר משקלי (יט, א), משקל ידי עלי ר״י אומר מביא חבית מים ומכניס עד מרפקו ושוקל, ותניא בברייתא (בע״ב)      יהודה אומר ממלא חבית מים ומכניס עד האציל, אלמא היינו מרפק היינו אציל, ואציל היינו קובד״ו, וכדאמרינן במסכת סופרים (פ״ג, הי״א) לא יניח אדם ספר תורה על ברכיו ויניח אצילו עליו, ואין דרך להניח השחי אלא הקובד״ו. ועוד דבההיא דערכין תניא וברגל עד הארכובה, דאלמא מרפק ביד היינו כנגד ארכובה ברגל, וארכובה היינו ראש השוק, וכולה שמעתא דערכין ודאי הכין מכרעת, ותנן במסכת אהלות (פ״א, מ״ח) גבי רמ״ח אברים באדם שנים בקנה שנים במרפק אחד בזרוע ארבע בכתף אלמא מרפק למטה מן הזרוע.\par \textbf{} והרמב״ן ז״ל תירץ שכל אותו עצם מתחילתו ועד השחי נקרא מרפק, ואציל דהתם בערכין, היינו עד תחילת הפרק, וההיא נמי דרמ״ח איברים היינו תחילתו מפני שיש בו שני עצמות שקורין חכמי הטבע קובד״י עליון וקובד״י תחתון, ומפני שרצה לחלק אותו (מסוף) הפרק הוציא תחילתו בלשון מרפק וסופו בלשון זרוע, ולפיכך לענין משנתנו ראוי יותר לפרש שחי כמו שפירש רש״י ז״ל, לפי שכן דרך אדם להוציא תחת שחיו ולא להוציא בקובד״י.}
\textblock{ גמרא: גירסת הספרים:\textbf{ ומשכן גופיה מנא לן דכתיב (שמות כו, טז) עשר אמות אורך הקרש, וכתיב (שם מ, יט) ויפרש את האהל על המשכן, ואמר רב משה רבנו וכו׳.} כלומר: דכל שבטו כמוהו, וכדאמרינן בסמוך ואינו נכון. אבל רב האי גאון ז״ל לא גריס ליה כלל, דכיון דגמרינן למזבח שהוא עשר, ממילא ידעינן דאף הלוים קומתן עשר, מדכתיב גבי מזבח גופיה (במדבר ד, יד) ופרשו עליו כסוי עור תחש.}
\textblock{\textbf{מדקאמר להו אי אתם מודים בלאחריו ובא לו לאחריו שהוא חייב מכלל דפטרי רבנן.} פירוש: לאו ממאי דקאמר להו אי אתם מודים קא דייקי, דאדרבא בכל מקום אמאי דלא פליגי קאמרי הכי, ולומר כמו שאתם מודים בזה אודו לי בזה, אלא מדקאמר הם לא מצאו תשובה לדברי קא דייק, דקא סלקא דעתך דהכי קאמר הם לא מצאו תשובה וחלוק לחלוק ביניהם ואעפ״כ לא חזרו בהן, דאלמא סברי דאף במוציא [לאחריו ובא לו לאחריו הוי] כלאחר ידו ופטורין.}
\textblock{\textbf{אף מקבלי פתקין כן.} פירש רש״י ז״ל: פתקי המלך פעמים מוציאין אותן למוסרן לרץ זה וכשאינו מוצאו לזה מוסרו לרץ אחר ומפני שדבר המלך נחוץ, ואע״פ שלא נתקיימה מחשבתו שהוציא על מנת שיתנו לרץ זה ונתנו לרץ אחר. ואינו מחוור. דאינו דומה לנתנו לאחריו ובא לו לפניו או לאחריו [ובא לפניו], אבל רב האי גאון ז״ל פירש לבלרי מלכות חוגרין בסינר, ותולין פתקין במתניהם פעמים שבאים לפניהם פעמים שבאים לאחור ולא אכפת להון. וזה נכון.}
\textblock{\textbf{זה יכול וזה אינו יכול.} פירוש: כל יכול שבכאן הוא שיכול להוציאו באותו ענין שהוא אוחזו עכשיו וכל שאינו יכול פירושו שאינו יכול להוציאו באותו ענין שהוא אוחזו עכשיו, ואע״פ שיש בכחו לתקנו ולהוציאו בענין אחר, דמכל מקום כיון שהוא אוחזו עכשיו בענין שהוא אינו יכול להוציאו, לא איכפת לן אם יש בו כח להוציאו בענין אחר.}
\clearpage
\newsection{דף צג}
\textblock{\textbf{לכך נאמר (ויקרא ד, כז) בעשותה, יחיד שעשאה חייב שנים שעשאוה פטורין.} עיקר דרשא העושה את כולה ולא העושה את מקצתה, אלא הא דאמר הכא בעשותה יחיד שעשאה חייב, פירושו הוא לומר בעשותה העושה את כולה הוא לבדו ולא העושה את מקצתה וחברו מקצתה, כלומר: אף על פי שעשאוה כולה בין שניהם.}
\textblock{ רש״י ז״ל גריס:\textbf{ רבי שמעון לטעמיה דאמר יחיד שעשאה בהוראת בית דין חייב.} ואין הגירסא מחוורת. דאדרבה בהוריות (ג, ב) תנינן הורו ב״ד וידעו שטעו וחזרו בהן בין שהביאו כפרתן בין שלא הביאו כפרתן, והלך היחיד ועשה על פיהן ר״ש פוטר ר״א אומר ספק. ויש מתרצים דשאני התם דכיון שחזרו בהם וידעו שטעו, ידיעתן כידיעתו וכמי שנזכר הוא דמי שהרי עליהן היה סומך. והר״ם בר׳ יוסף       תירץ דהתם הוא דאיכא כפרה הכא ליכא כפרה. ואינו מחוור. דמכל מקום מנא לן דר׳ שמעון אית ליה הכי, והיכן אמרה דנימא רבי שמעון לטעמיה. אבל רבנו האי גאון והגאונים ז״ל גורסים, ורבי שמעון יחיד שעשאה בהוראת ב״ד לא צריך קרא, כלומר: דפשיטא משום דאנוס הוא.}
\textblock{\textbf{הי מנייהו מחייב וכו׳ רב המנונא אמר משום דקא מסייע בהדיה.} יש מקשים והיאך אפשר שזה שיכול ועושה המלאכה פטור, וזה שאינו יכול ואינו עושה המלאכה יהא חייב, והיאך אמר כן רב המנונא. ותירץ הר״מ בר יוסף ז״ל, משום דמספקא לן דשמא אותו שיכול חייב, לפי שהוא עושה עיקר המעשה, או שמא אותו שאינו יכול חייב, לפי שהוא עושה בכל כחו, ומחייבים ליה משום דמדמינן ליה לזה אינו יכול וזה אינו יכול דחייבין לרבי מאיר ורבי יהודה, (והלכך) [והיכול] לפטור מדמינן ליה לזה יכול ולזה יכול דפטור לרבי יהודה, ופשיט רב חסדא דזה שיכול יש לחייב מפני שהוא עיקר, ואמר ליה רב המנונא, דאדרבה זה שאינו יכול ומסייע בכל כחו יש לחייב, ואמר ליה מסייע אין בו ממש. ומכל מקום בין רב חסדא בין רב המנונא שניהם מודים דאין לחייב אלא האחד לבדו.\par \textbf{} ואינו מחוור. שהטעם בזה אינו יכול וזה אינו יכול אינו מפני שמסייע בכל כחו, אלא מפני שתלויה בכל אחד. ועוד קשיא לי דאי משום הא, שאינו יכול לר׳ שמעון אמאי חייב, דהא לר׳ שמעון זה אינו יכול וזה אינו יכול פטורין. ועוד יכול אמאי פטור לרבי מאיר, דהא זה יכול וזה יכול לרבי מאיר שניהם חייבין, וסתמא דמלתא הכא אינו חייב אפילו לרבי מאיר אלא האחד, וכולהו שוו בהא, דאינו סברא שיהיו שניהם חייבין לרבי מאיר ולא יהא חייב אלא האחד לרבי יהודה ורבי שמעון, שכיון דכיילינן ואמרינן זה יכול וזה אינו יכול דברי הכל חייב, משמע דכולן שוין בהא בין בחיוב בין במהות החיוב.\par \textbf{} ויש מפרשים דמשום דקתני דברי הכל חייב, הוה משמע להו דאינו חייב אלא האחד, ומשום כך שאלו איזה ראוי לחייב יותר, ואמר רב חסדא זה שיכול, שהמלאכה נעשית על ידו, דזה שאינו יכול בטל הוא אצל היכול, אע״ג דמחייב ר׳ יהודה בזה אינו יכול וזה אינו יכול הכא לא, ורב המנונא אמר דאפילו שאינו יכול חייב לרבי יהודה מפני שמסייע, וכדמחייב ר״י בזה אינו יכול וזה אינו יכול, ואמר ר״ח מסייע אין בו ממש, ואינו דומה לזה אינו יכול וזה אינו יכול, דהתם המלאכה נעשית ע״י שניהם ואי אפשר להעשות ע״י האחד, ולפיכך אין אחד מהם מסייע אלא שניהן עושין מעשה. וגם זה אינו מחוור בעיני כל הצורך מן הטענה שכתבתי. דאי אפשר לחלק בדבר זה ולומר דאליבא דר״י קאמר רב המנונא ולאו אליבא דהנך, דלדידהו מאי איכא למימר, וכולהו שוין בדבר זה.\par \textbf{} ומסתברא ודאי דלרב המנונא אף מי שאינו יכול קאמר דחייב וכדברי הפירוש האחרון, אלא דהכי פירושא: רב חסדא אמר יכול חייב לכולי עלמא, ואפילו לרבי יהודה ור׳ שמעון מפני שהמלאכה כולה נעשית על ידו, ואינו דומה לזה יכול וזה יכול, דהתם כל אחד עושה מלאכה וראוי לעשות את כולה וכשלא עשאה כולה התורה פטרתו, אבל זה יכול וזה אינו יכול יחיד שעשאה קרינן ביה, ומי שאינו יכול אינו אלא מסייע ומסייע אין בו ממש ואפילו לר״מ, ואינו דומה לזה אינו יכול וזה אינו יכול, דהתם כל אחד עושה מעשה ואין אחד מהם מסייע אלא עושה מעשה כחבירו. ורב המנונא אמר שאפילו המסייע יש לחייב לכולי עלמא, לר״מ מפני שעושה מעשה בכל כחו, ור״מ בכל ענין מחייב, וזה גם כן עושה הוא מעשה בכל כחו, והוי ליה כזה אינו יכול וזה אינו יכול, והוא הדין לרבי יהודה מן הטענה הזו לדעת רב המנונא, ולרבי שמעון דפטר בכל ענין בין בשניהם יכולין בין זה לעצמו וזה לעצמו בין בזה אינו יכול וזה אינו יכול, סבירא ליה לרב המנונא דהכא מודה, משום דכשזה יכול וזה יכול והוי ליה לחד למעבד כולה ולא עבד, קרינן ביה העושה כולה ולא העושה מקצתה, והוא הדין והוא הטעם לזה אינו יכול וזה אינו יכול, כיון שאין אחד מהם יכול לעשותה, דלא קרינן בהו יחיד שעשאה, אבל בזה יכול וזה אינו יכול, היכול חייב על כרחו דבדידיה קרינן ביה יחיד שעשאה, וכיון שהיכול חייב, אי אפשר לפטור את השני משום שנים שעשאוה, שאם כן אף היכול ליפטר, וכיון שכן מחייבין אותו מיהא משום מסייע, ואמר ליה רב חסדא מסייע אין בו ממש ופטור.}
\textblock{\textbf{היה יושב על גבי מטה וארבע טליות וכו׳ ורבי שמעון מטהר.} מה שפירש רש״י ז״ל בפירוש שני דטעמא דר׳ שמעון דבעיא שיעור זיבה לזה ושיעור זיבה לזה, כלומר: שיהא זב מכביד על כל אחד ואחד, אינו מחוור, דהא אתיא למפשט מהא דמתניתין דלא בעיא שיעור לכל אחד ואחד, ואם איתא תנאי היא דהא ר׳ שמעון בעי שיעור זיבה לכל אחד ואחד, אלא הלשון הראשון עיקר, דר״ש לטעמיה בזה אינו יכול וזה אינו יכול.}
\textblock{\textbf{היה רוכב על גבי בהמה וכו׳ טהורות מפני שיכולה לעמוד על שלש.} פירש רש״י ז״ל: דכל אחד ואחד הוי ליה רביעי ואינו אלא כמסייע, וסתמא כר׳ יהודה דזה יכול וזה יכול קרי מסייע. ומסתברא דאפילו ר״מ מודה בהא דעד      לא קאמר ר״מ אלא בזה יכול וזה יכול, דכיון דכל אחד יכול אנו רואין [כאילו] כל אחד עושה מעשה, דאיזה מהם יקרא עושה ואיזה מהם יקרא מסייע, וכן בזה אינו יכול וזה אינו יכול, כיון ששניהם צריכין לה הרי כל אחד עושה מעשה ואפילו לרבי יהודה, אבל כאן כיון שאין הרביעי צריך לשלשה ואותו הרביעי אינו מסויים נמצא שכל אחד מהם ראוי להקרא מסייע, ואינו כזה יכול וזה יכול שאין האחד ראוי ליקרא אלא מסייע. ולפי מה שכתבתי אני פטורין לכ״ע ואפילו לר״מ.}
\textblock{\textbf{וכיון דזימנין עקרה הא וזימנין עקרה הא, ליהוי כזב המתהפך.} קשיא לי מאי כיון, דכשתמצא לומר דלא עקרה כל שכן דהיה לנו לטמא אליבא דמ״ד מסייע יש בו ממש, ולא הל״ל כיון, אלא ה״ל למימר אע״ג דזימנין עקרה לה לגמרי ליהוי כזב המתהפך, ונ״ל דה״פ דאם איתא דמסייע יש בו ממש, אלא דהכא אתי לטהורי׳ משום דזימנין דעקרה ליה לגמרי, אדרבה כ״ש שיש לטמויינהו לא מחמת מסייע אלא מחמת שהזב נשען עליו ממש דהוי כזב המתהפך, ואלומי הוא דאלים לקושייה.}
\textblock{ גירסת הספרים:\textbf{ אמר מר זה יכול וזה יכול רבי מאיר מחייב איבעיא להו בעינן שיעור לכל אחד ואחד וכו׳.} ורב האי גאון ז״ל דחאה לגירסא זו בשתי ידים, שאי אפשר דאבעי להו דלא כהלכתא, וגריס אמר מר זה אינו יכול וזה אינו יכול, ומשום דמחייב בה ר׳ יהודה וקיימא לן כותיה, ועוד הביא ראיה מדאמרינן אף אנן נמי תנינא, צבי שנכנס לבית ולא יכול אחד לנעול ונעלו שנים חייבין, דהיינו כר׳ יהודה דאמר בזה אינו יכול וזה אינו יכול דחייב, אלמא דבזה אינו יכול וזה אינו יכול הוא דאיבעיא להו. ואין צריך למחוק הגירסא דמדר״מ נשמע לרבנן, ומשום הכי אצטרכינן למבעי אהא דר״מ משום דדבר רחוק הוא שיהא שיעור גדול כל כך שלא יהא האחד יכול להוציאו ולא יהא בו שיעור לזה ולזה. ולא עוד אלא שיראה כדבר שאי אפשר, דשיעורי שבת אינן גדולים כל כך, שימצא דבר שיהא שיעורו גדול כל כך שלא יוכל אחד להוציאו ואפילו היה תשש ודל כח, אלא שאפשר להעמידו כגון שהוציאו בכלי ואינן צריכין לכלי, ומתוך כבדו של כלי אין אחד יכול להוציאו.}
\textblock{\textbf{שמע מינה אוכל שני זיתי חלב בהעלם אחד חייב שתים.} כלומר: ש״מ דאיכא תנא דסבירא ליה הכין, והיינו סומכוס דמספקא לן אליביה בפרק אותו ואת בנו (חולין פב, ב) ולא אפשיטא התם, דלמא נפשוט מהא. ותמיהא לי היכי מדמי לה לאוכל שני זיתי חלב בהעלם אחד, דהא הכא בבת אחת מפיק להו ובכי האי גוונא ליכא לחיובא תרתי ואפילו תמצא לומר בקצר וקצר או אכל ואכל בהעלם אחד שהוא חייב תרתי, דאטו מי שבלע בבת אחת כדי שני זיתים מי איכא למימר שיהא חייב שתים. ותדע לך דהא איכא בכריתות פרק אמרו לו אכלת חלב, (טו, א) דר״א מחייב תרתי בקוצר כגרוגרת וחזר וקצר כגרוגרות בהעלם אחד, ואפילו הכי בקצר כשני גרוגרות בבת אחת אינו חייב אלא אחת, וכדאמרינן התם גבי קצר וקצר היכי משכחת לה דלא מחייב, כגון שקצר שתי גרוגרות בבת אחת אבל קצר כגרוגרת וחזר וקצר כגרוגרת חייב שתים. ויש לומר דכיון שהן שני שמות דהיינו אוכלין וכלי, וגופין מחולקין, אפילו בבת אחת חשבינן ליה כזה אחר זה, וכדאמרינן התם הכא במאי עסקינן בדלית המודלה על גבי תאנה וקצצן בבת אחת דמשום הכי מחייב ר״א הואיל ושמות מחולקין וגופות מחולקין, דכותיה גבי קצר וקצר היכי משכחת לה דלא מיחייב כגון שקצר שתי גרוגרות בבת אחת.\par \textbf{} ואכתי קשיא לי אמאי לא אוקמה כר״א דאית ליה התם בכריתות בפרק אמרו לו (שם) הבא על אחותו שהיא אחות אביו שהיא אחות אמו וחזר ובא עליה חייב על כל אחת ואחת, ואיבעיא להו התם טעמא דרבי אליעזר משום דעבד תרתי, ומינה דקצר וקצר דעבד תרתי חייב שתים, או דלמא טעמא דרבי אליעזר התם משום דאי אפשר לערבן לביאות בבת אחת אבל בשתי גרוגרות דאפשר לערבן אין חייב אלא אחת, ואסקה רבה לטעמיה דר״א התם משום דעבד תרתי, והלכך קצר כגרוגרת וחזר וקצר כגרוגרת בהעלם אחד חייב שתים, וכל שהוא חייב בקצר וקצר מחייב בקוצר בבת אחת שתי גרוגרות בשמות מחולקין וגופות מחולקין, כגון דלית המודלה על תאנה וקצצן בבת אחת וכדאיתא התם, והכא בכלי ואוכלין דכותה היא.\par \textbf{} ויש לומר דפלוגתא היא התם, דרב יוסף אמר דטעמיה דר׳ אליעזר התם משום דאי אפשר לערבן, אבל הכא דאפשר לערבן אינו חייב אלא אחת, והלכך לא בעי לעיולי נפשיה בפלוגתא. ובירושלמי נמי משמע דאית להו כההיא סברא דר׳ יוסף דגרסינן התם בפרקין דאמר ר׳ עקיבא (שבת פ״ט ה״ז) גבי מתניתין דהמוציא קופת הרוכלין אף על פי שיש בה מינין הרבה אינו חייב אלא חטאת אחת, וקשיא אילו הוציא והוציא בהעלם אחד כלום הוא חייב אלא אחת, למי נצרכה לר׳ אליעזר, שלא תאמר מינין הרבה יעשו כהעלמות הרבה ויהא חייב על כל אחת ואחת, לפום כן צריך מימר אינו חייב אלא אחת, ואע״ג דרבה ורב יוסף הלכתא כרבה, הכא לא מעייל נפשיה בפלוגתייהו, ועוד דפשטא דברייתא התם כרב יוסף דיקא. כך נראה לי.}
\textblock{ ה״ג:\textbf{ הכא במאי עסקינן כגון ששגג על האוכלין והזיד על הכלי.} וחייב אף על הכלי דקתני כרת או מיתה קאמר, ועל האוכלין פטור מחטאת משום דאינו שב מידיעתו שהרי מזיד על הכלי. ואני תמיה לרב ששת רישא דקתני חייב על האוכלין ופטור על הכלי, מאי קאמר דאי במזיד בשניהם מאי נפקא מינה דהא נסקל על האוכלין, ואי במזיד על האוכלין ושוגג על הכלי אפילו היה צריך לכלי נמי פטור אף על הכלי, דאינו שב מידיעתו שהרי מזיד על האוכלין, ואי בשוגג בזה ובזה אפילו היה צריך נמי לכלי פטור, דאכל שני זיתי חלב בהעלם אחד הוא. ואולי נאמר דהיא גופא קמ״ל      שני זיתי חלב בהעלם אחד אינו חייב אלא אחת, ולאפוקי מדר׳ אליעזר, ואליבא דרבה דאמר (כריתות טו, א) קצר וקצר בהעלם אחד חייב שתים. אי נמי יש לומר כגון ששגג על האוכלין שהיה סבור שלא אסרה תורה הוצאת האוכלין, והזיד על הכלי כלומר: שנתכוין להוציאו, שאסור להוציאו לצורך עצמו אבל לעשותו טפלה מותר, והלכך אע״פ שנתכוין להוצאת הכלי פטור עליו ולפיכך חייב על האוכלין, והכי קאמר: המוציא אוכלין בכלי ושגג על האוכלין והזיד על הכלי חייב על האוכלין, מפני שהוא פטור על הכלי, ואתקיף לה רב אשי והא אף על הכלי קתני, כלומר: דמשמע דחייב הוא שתים בין על הכלי בין על האוכלין.\par \textbf{} ור״ח ז״ל גריס אמר רב ששת כגון שהזיד על האוכלין ושגג על הכלי. וחיוב הכלי דסיפא כחיוב האוכלין דרישא, ותרווייהו לחטאת. ואינו מחוור. דאי אפשר לחייבו חטאת על שגגת הכלי כיון שהזיד באוכלין, וכדאמרן שהרי אינו שב מידיעתו. ורבנו האי גאון ז״ל גורס כגירסת הספרים וז״ל: אמר רב ששת הכא במאי עסקינן כגון ששגג על האוכלין והזיד על הכלי, פירוש והכי קאמר חייב על זה חטאת ועל זה כרת, מתקיף לה רב אשי והא אף על הכלי קתני, דאלמא האי כי האי. ע״כ. והוא מן התימא האיך אפשר שיתחייב חטאת על האוכלין כיון שהזיד על הכלי. ואולי נאמר דראוי לחטאת קאמר אלא שדבר אחר גרם לו להפטר. ובזה אפשר להעמיד אף גירסת ר״ח ז״ל.}
\clearpage
\newsection{דף צד}
\textblock{\textbf{ובפלוגתא דרבי יוחנן ור׳ שמעון בן לקיש.} פירש רש״י ז״ל: דרב ששת דלא אוקמה בנודע לו וחזר ונודע לו, ורב אשי דאוקמה בהכין, פליגי בפלוגתא דר״י וריש לקיש, דרב ששת כר״ל ולפום כך לא מצי לאוקומה בנודע לו וחזר ונודע לו, ורב אשי כרבי יוחנן. ואינו מחוור. דמנא ליה דרב ששת כר״ל [כיון] דקיימא לן כר׳ יוחנן אלא דלא אסיק אדעתיה לאוקומה בהכין, ואי נמי דלא קשיא ליה האי אף על הכלי. אלא הכי פירושא: רב אשי אמר בנודע לו וחזר ונודע לו וכפלוגתא דר׳ יוחנן ורשב״ל, למר בנודע לו וחזר ונודע לו מיד, דידיעות לבד מחלקות, ולמר בנודע לו לאחר הפרשה, ואי נמי לאחר הקרבה למר כדאית ליה ולמר כדאית ליה בפרק כלל גדול (שבת עא, ב).}
\textblock{\textbf{דבי וייאדן.} פירש רש״י ז״ל: ציידי עופות, וקושרן ומניחן על הסוס כשהן חיין. ומה שאמר קושרן לא דק דמודה ר׳ נתן בכפות. ורבי האי גאון ז״ל פירש בזיארן, הנץ וכיוצא בו שהמלכים צדין בהם עופות קורין אותן באז, והאחד העוסק בהן קורין אותו בלשון פרסי באזאר, ובזמן שהן יותר מאחד קורין להן בזיארן, ויש להם סוסים מיוחדים הנקראין סוסיא דבאזאר״ן.}
\textblock{\textbf{פוטר היה רבי שמעון אף במוציא את המת לקוברו.} פירוש: משום דאף זו מלאכה שאינה צריכה לגופה, לפי שאין לו הנאה לא בהוצאתו ולא בקבורתו, אבל ההנאה היא מניעת הטומאה שהוא מונעו ממנו בהוצאתו, וכן (לקמן שבת קז, ב) צידת נחש שלא ישכנו נקראת מלאכה שאינה צריכה לגופה, שאינו אלא מניעת היזק ואין לו הנאה בגופה של מלאכה, וכן (לעיל שבת לא, ב) כבוי פתילה וגחלים מלאכה שאינו צריכה לגופה, שאין הנאה לו בגופו של כבוי, אלא בכבוי של פתילה שצריך להבהבה וגחלים לעשות מהם פחמים שהוא צריך להדליקם ולכבותם, והכבוי בעצמו הוא התיקון ממש.\par \textbf{} ואם תאמר אם כן תופר יריעה שנפל לה דרנא (לעיל שבת עה, א) לא יהא חייב שאינו צריך לגופה של תפירה, שברצונו לא נפל בה דרנא ולא היה צריך לאותה מלאכה, יש לומר דמכל מקום כיון שנפל שם דרנא צריך הוא לאותה תפירה ונהנה הוא ממנה, כתולש מן המחובר דברצונו לא היה מחובר ואינו צריך שיהא מחובר כדי שיתלוש ממנה, ואף על פי כן מאחר שהוא מחובר צריך הוא לתלוש ונהנה הוא מאותה תלישה ממש, ומלאכה הצריכה ממש הוא. והמוציא ריח רע כל שהוא, ששנינו בפרק א״ר עקיבא (לעיל שבת צ, א) שהוא חייב, אף רבי שמעון מודה בה ובשהוציאו לצורך רפואה לגמר בריח רע, והרבה עושין כן. וכן פירשה הר״ז הלוי ז״ל.}
\textblock{\textbf{מי קאמינא לרשות הרבים לכרמלית קאמינא.} יש מפרשים דדוקא על ידי ככר או תינוק, דכיון דאפשר לתקן הטלטול על ידי ככר או תינוק מתקנין, ולא התירו להוציאו לכרמלית אלא מפני שאי אפשר בלאו הכי ומשום כבוד הבריות התירו. והרמב״ן ז״ל כתב דנראה דכיון שצריך להוציאו לכרמלית, שרי אפילו בלא ככר, כדי שלא ירבה בהוצאה, שהוצאת התינוק עצמה אסורה היא לכרמלית ואם המת הותר מפני כבודו יתירו בהוצאת התינוק והככר. ויש לי לומר דכל שהוא בתוך רשות היחיד אי אפשר להוציא בלא ככר או תינוק, מפני שיש לתקן איסור הטלטול על ידי כך, וכיון דכל שאר טלטול המת שברשות היחיד לא התירו בלא ככר ותינוק, נראה שלא התירו כאן בלאו הכין, ואם באנו לומר דכשיגיע בצד הכרמלית יסלק הככר ויוציאנו בלא ככר, יראה כחוכא דעד כאן לא התרנו לטלטל בלא ככר, ועכשיו אתה אוסרו בככר ומתירו בלא ככר, הלכך לא אפשר בלא ככר או תינוק.}
\textblock{ומכל מקום יש לעיין דמהכא משמע דאפילו לרבי שמעון אסור להוציאו לרשות הרבים עצמה, ואמאי והלא ליכא בהוצאת המת אפילו מרשות לרשות לר״ש אלא איסורא בעלמא, ומשום כבוד הבריות נתיר לר״ש כמו שאנו מתירין לרבי יהודה בכרמלית ועל ידי תינוק. ובשם ר״ח ז״ל [אמרו] שלא התירו כבוד הבריות אלא בדבר שעיקרו מדברי סופרים כגון כרמלית, אבל לרשות הרבים לא, דהכא מי מפיס במלאכה שלו אם הוא צריך לגופו אם לאו, שהרי אפשר בהוצאת המת שיהא בה צורך לגופה של מלאכה כגון אותה שאמרו בירושלמי (פ״י ה״ה) במת עכו״ם שהוציא לכלבו, ואפילו אמירה לעכו״ם שהוא שבות שאין בה מעשה לא התירו כל שיש בה חלול שבת באבות מלאכות, וכל שהן בדררא דאיסורא דאורייתא גנאי הוא למת ואין ההיתר משום כבודו. וכן כתב הרמב״ן ז״ל. והראב״ד ז״ל פירשה לזו בשאינו מוציא לכבודו של מת, חדא דיותר הוא מתכבד בבית ממה שהוא מתכבד בכרמלית, ואם מפני שלא יסריח בחום הבית הוציאוהו לכרמלית, והלא ממטה למטה (לעיל שבת מג, ב) לא התירו להפכו כדי שלא יסריח וכל שכן להוציאו לכרמלית, אלא הכא משום כבוד החיים הוא שהתיר, מפני שהיה מסריח וכדי לסלק מהם ריחו כדי שלא יהיו מלוכלכין בו התירו.\par \textbf{} ולענין מחלוקת רבי יהודה ור״ש במלאכה שאינה צריכה לגופה פסק ר״ח ז״ל הלכתא כרבי שמעון. מדאשכחן לרבא דהוא בתרא דסבירא ליה כותיה בפרק נוטל אדם את בנו (לקמן שבת קמא, ב). ורבנו האי גאון ז״ל גם הוא נראה מדבריו שהוא פוסק כן, וכן פסק הראב״ד ז״ל. אבל הרב אלפסי ז״ל נראה שפוסק כרבי יהודה, מפני שכתב בפרק כירה (שבת מב, א) ההוא דאמר שמואל מכבין גחלת של מתכת ברשות הרבים אבל לא של עץ, ואמרינן התם דשמואל סבירא ליה כרבי יהודה במלאכה שאינה צריכה לגופה, וכן השמיט הרב ז״ל בפרק שמונה שרצים (קז, ב) אותן ג׳ שמועות דצידת נחש ומפיס מורסא וצידת צבי דאוקימנא כר׳ שמעון. וכן פסק הרמב״ם ז״ל (פ״א, ה״ז) כרבי יהודה. והרמב״ן ז״ל כתב לקמן בסוף פרק כל כתבי הקדש (שבת קכא, ב) שאף הרב אלפסי ז״ל כר״ש הוא פוסק, שהרי הביא משנתינו שבפרק במה מדליקין (שבת כט, ב) דקתני חוץ מן הפתילה מפני שהוא עושה פחם ואוקמה כרבי שמעון, והני מתניתין דמפיס מורסא וצידת נחש כבר כתבן בפרק האורג (שבת קז, א) דאלמא הכי סבירא ליה, ומה שכתב בפרק כירה (שבת מב, א) ההוא דשמואל במלאכה שאינה צריכה לגופה [ד]סבירא ליה כרבי יהודה, בא ללמדנו דלמאי דסבירא לן כר״ש אפילו של עץ נמי שרי.}
\textblock{\textbf{והא דתנן פטור, דאפיק חצי זית מכזית ומחצה.} איכא למידק מאי דוחקייהו לאוקומה הכין, לימא דלא הוה ביה אלא חצי זית ואפקיה. יש לומר דמן המת משמע דפריש מן המת. ואי נמי יש לומר דאי הכי פשיטא ולא צריכא למימר.}
\clearpage
\newsection{דף צה}
\textblock{\textbf{שגג בשבת חייב חטאת הזיד ביום טוב לוקה את הארבעים.} הא דקתני ביום טוב לוקה את הארבעים, לא קאי אחולב ואמחבץ ומגבן דאוכל נפש נינהו, ואוכל נפש מדאורייתא מישרא שרי ואפילו בשאפשר לעשותו מערב יום טוב, דלא אמרו אפשר ולא אפשר אלא במכשירין, וכדאיתא בביצה (כח, ב) ובמגילה (ז, ב) משום דכתיב (שמות יב, טז) אך הוא לבדו יעשה לכם. והא דאמרינן בפרק ר׳ אליעזר דמילה (שבת קלד, א) אסור לגבן, הני מילי מדרבנן אבל מדאורייתא שרי, וכל שכן לר׳ אליעזר דאית ליה (לקמן שבת קלז, ב) אפילו במכשירין דבין אפשר בין לא אפשר שרו, וכל שכן בחולב דאוכל נפש ממש ואי אפשר מערב יום טוב, והא דהכא ר״א תני לה. ומיהו אפשר להעמידה לזו אפילו בחולב ומחבץ ומגבן, וחולב בחלב טמאה, ומחבץ ומגבן בשרף ערלה ובכל דבר שאסור באכילה, ומכל מקום הרודה חלת דבש ביום טוב לוקה את הארבעים ואכולה סיפא קאי, דאע״ג דאוכל נפש הוא הא מקישו הכתוב ליער לומר שהוא אסור משום תולש, ומחובר אסור אפילו ביום טוב דבר תורה, ואפילו באוכל נפש וכדאמרינן בשמעתא קמייתא דביצה (ג, א) פירות הנושרין אסורין גזירה שמא יעלה ויתלוש, ובירושלמי (ביצה פ״א, ה״י) ממעטו מדכתיב הוא לבדו יעשה לכם, מיעוטין הן, דגרסינן התם: תני אין בוררין ולא טוחנין ולא מרקדין, והבורר והטוחן והמרקד בשבת נסקל, ביום טוב סופג את הארבעים, ומקשינן והא תניא בורר כדרכו בחיקו ובתמחוי, ומתרץ אמר רבי חנינא ענתנה דר״ג היא דאמר אף מדיח ושולה, והא תני (תוספתא פ״ב הי״ב) של בית ר״ג היו שוחקין פלפלין ברחים שלהן, מותר לטחון ואסור לבור, אמר רבי יוסי ברבי בון לא הותרה טחינה כדרכה. פירוש: לומר שאין שחיקת הפלפלים ברחים שלהם לר״ג טחינה כדרכה, מנין שאין בוררין ולא טוחנין ולא מרקדין, רבי יוסי בשם ר״ל אך אשר יאכל לכל נפש הוא לבדו יעשה לכם (שמות יב, טז), מן ושמרתם את המצות (שם יז), כלומר: מלישה ואילך למעט טחינה והרקדה וכ״ש תלישה וקצירה, תני חזקיה ופליג אך הוא לבדו, הרי אלו מעוטין, מכאן שלא יקצור ושלא יטחון ושלא ירקד ביו״ט, וברירה איכא בינייהו.}
\textblock{\textbf{ומכל מקום לדברי הכל נתמעטו טחינה והרקדה, שלא התירה התורה ביום טוב אלא מלאכות של יומן כגון שחיטה ואפיה ובישול, אבל אלו תקון הן ודרך בני אדם לתקן מהם לימים רבים, וכדי שלא יתקן ביום טוב מחמת חול, לא התירה התורה אותן ביום טוב אפילו לצורך היום. ולא התירה אלא תיקון אוכל בדבר שהוא ברשותו, אבל לעקור דבר מגדולו לא, ואפילו פירות הנפגמין מיום טוב לחברו כגון תאנים ותותים, וכל שכן צידת בעלי חיים שהוא בכלל, והיינו טעמא דאין צדין דמתניתין. וטעם זה כתב בו הרמב״ן ז״ל ביום טוב פרק אין צדין (ביצה כד, ב) בספר המלחמות. ועוד יש לומר טעם בענין איסור הצידה כדי שלא ילך בחורשין ויצוד, ונמצא בטל משמחת יום טוב. ובפ״ק       } דביצה גרסינן בירושלמי (הל׳ יב) מהו להעמיד חלב, אם אתה אומר כן, אף הוא מעמיד מיום טוב לחול.\par \textbf{} אבל בתוס׳ אמרו דאף מחובר באוכל נפש מותר דבר תורה, והכא לא קאי אחלת דבש, אלא אהנך תרתי דהיינו מכבד ומרבץ. והא דאמרינן בפרק קמא דביצה (ג, א) גזירה שמא יעלה ויתלוש, איירי סמוך לחשיכה, ואפשר הכא נמי ברודה סמוך לחשיכה קאמר, ואי נמי ברודה בפירוש לצורך חול, ואליבא דמ״ד התם בפרק אלו עוברין (פסחים מח, א) דר׳ אליעזר לית ליה הואיל, אבל למאן דאמר התם דר״א אית ליה הואיל אי אפשר לאוקומה בהכין, אלא אפילו בעושה לצורך חול לא לקי. ולפי דברי התוס׳ איכא למימר דאפילו אבבא דרישא קאי, ואכולהו קתני ביום טוב במזיד לוקה את הארבעים, ומיירי בחולב ומחבץ סמוך לחשיכה. ומכל מקום אין דרכם מחוורת, דודאי הכין משמע דמחובר אסור ביום טוב דבר תורה.}
\textblock{\textbf{חולב חייב משום מפרק.} ואיכא למידק דהא אמרינן בפרק כלל גדול (לעיל שבת עג, ב. עה, א) דאין דישה ואין עמור אלא בגדולי קרקע. ופירש ר״ת ז״ל בספר הישר בכתובות בפרק אעפ״י (כתובות ס, א) דתרי גווני מפרק נינהו, מפרק בגדולי קרקע הוי תולדה דדש, ומפרק דלאו גדולי קרקע כגון הכא הוי תולדה דממחק, שממחק הדד ומפרק ממנה החלב. אבל רש״י ז״ל פירש שהוא תולדת דש. ולדידיה תקשי לן הא דהא אין דישה אלא בגדולי קרקע. גם לפירושו של ר״ת ז״ל קשה מדאמרינן בפרק חבית (לקמן שבת קמד, ב) חולב אדם לתוך הקדרה אבל לא לתוך הקערה, ואם איתא דחולב חייב משום ממחק, אפילו כשחולב נמי לתוך הקדרה ליחייב דהא ממחק הדד, דלענין ממחק מה לי לתוך הקדרה מה לי לתוך הקערה. אבל לפרש״י ז״ל ניחא דכל שהוא מפרק בתוך הקדרה הוי ליה כאוכל הבא לתוך אוכל ואינו כדישה, אלא שאף לדברי רש״י נמי קשיא אידך, דאין דישה אלא בגידולי קרקע.\par \textbf{} ויש לומר דר״א לית ליה הכין, ולא קיימא לן כותיה, ולפי זה חולב אינו אסור דבר תורה. ואי קשיא לך דהא אמרינן בכתובות (ס, א) פרק אע״פ, גונח יונק חלב כדרכו ואינו חושש, מאי טעמא מפרק כלאחר יד, ובמקום צער לא גזרינן ביה רבנן, דאלמא חולב ביד הוי ליה מפרק גמור ואסור ד״ת. ויש לומר דלדבריו דר״א קאמר, ולומר דאפילו לר״א דמחייב אי יונק בפיו שרי דמפרק כלאחר יד הוה, ולעולם לא סבירא לן כותיה וכן כתב רב האי גאון ז״ל, אבל ר״ח גאון ז״ל כתב דחולב מפרק גמור הוא וחייב, מההיא דמס׳ כתובות (שם) ואיתא נמי ביבמות (קיד, א). ולדבריו בהמה נמי גדולי קרקע מיקרי, וכדאמרינן בעלמא (עירובין כז, ב) גבי מעשר שני מה הפרט מפורש פרי וגדולי קרקע והתם בקר וצאן כתיב (דברים יד, כו).}
\textblock{\textbf{טעמא מאי אמור רבנן דלמא אתי לאשוויי גומות.} פירוש: רבנן דאמרי אינו אלא משום שבות, אבל לרבי אליעזר דמחייב אי אפשר לומר כן, דהיכי מייתי לידי חטאת וכל שכן לידי מיתה משום גזירת אשוויי גומות, עד שיודע לך שהשוה אותן ממש, אלא טעמא דר״א דחשיב מכבד ומרבץ מתקן ממש כבונה, ואי נמי מחייב משום מכה בפטיש.}
\textblock{\textbf{והאידנא דסבירא לן כר׳ שמעון שרי אפילו לכתחילה.} פירש רש״י ז״ל וכן בתוס׳ דלא קאי אלא אמרבץ בלבד, משום דלא הוה פסיק רישיה ולא ימות, אבל מכבד אסור אפילו לר״ש, ולא פליגי רבנן הכא ואמרי אינו חייב אלא משום שבות אלא במרבץ בלבד, אבל במכבד כולי עלמא מודו, והא דקתני אחד זה ואחד זה אשבת ויום טוב קאי. וכן מפורש בתוספתא (פ״י, ה״י) וחכמים אומרים אחד שבת ואחד יום טוב אינן אלא משום שבות. ומסתייעא הדין סברא מדאמרינן בפרק מפנין (לקמן שבת קכז, א) אמר שמואל מאי אבל לא את האוצר שלא יגמור את האוצר כולו, משום דאתי לאשוויי גומות, אבל אתחולי מתחלינן, ומני ר׳ שמעון היא דלית ליה מוקצה, אלמא אפילו ר״ש מודה בכבוד הבית דאסור, דהוי ליה פסיק רישיה ולא ימות, ואפילו ביום טוב אסרו לכבד את הבית ואפילו בין המטות, כי הוא אחד      דברים (ביצה כב, ב) שהיה עושה ר״ג להקל ואין חכמים מודים לו. ועוד דאמרינן בפרק כל הכלים (לקמן שבת קכד, ב) ר״א מתיר אף בשל תמרה, ומתמהינן במאי אילימא מחמה לצל, בהא לימא ר״א אף של תמרה, ומאי תמיהא אי סבירא ליה כר״ש דשרי. אבל הרי״ף ז״ל התיר בפרק כל הכלים (שם) לכבד אפילו במכבדות של תמרה משום דאמרינן הכא והאידנא דסבירא לן כר״ש שרי אפילו לכתחילה, ולימד דאפילו אמכבד קאמרינן. וכן פסק רבנו האי גאון ז״ל כאן.\par \textbf{} ובעל הלכות גדולות כך פסק, דאפילו מכבד בדהוצי שרי, והא דאמרינן בריש פרק מפנין (שם) אבל לא את האוצר ומשום אשוויי גומות, לאו משום דאתי לאשוויי גומות מתוך כבוד קאמר, דהתם לאו במכבד אלא בגומר בלבד בלא כבוד איירי, ואף על פי כן אסור משום שמא יראה שם גומות וישוה אותם לדעת קאמר, לפי שלא ראה אותם עד עכשיו ודרך מכבדי אוצר בכך לפי שנעשה כולו גומות ועשוין למלאות מפני האורחין. וההיא דביצה (שם) דהתיר ר״ג לכבד בין המטות ולא הודו לו חכמים, השיב הרי״ף ז״ל בתשובה דהיינו חכמים היינו ר׳ יהודה, ולא קיימא לן כותיה אלא כר״ש, וההוא דמכבדות של תמרה דפרק כל הכלים (שם) גירסא אחרת יש לגאונים ז״ל בה, והכי גרסי: ואי מחמה לצל בהא לימא ר״א אף של תמרה, סבירא ליה כר״ש דאמר דבר שאין מתכוין מותר.}
\textblock{\textbf{ורמינהו רבי שמעון אומר אין בין נקוב לשאינו נקוב אלא להכשר זרעים בלבד.} איכא למידק מאי קושיא, אדרבה ברייתא כמתניתין דלענין שבת נקוב כשאינו נקוב, דהא כייל ר״ש בברייתא לכל נקוב כשאינו נקוב חוץ מהכשר זרעים בלבד. ויש מפרשים דדינא דמתניתין לא קשיא ליה, אלא לישנא דמתניתין קא קשיא ליה, דמדקתני במתניתין ר״ש פוטר בזה ובזה משמע להו דר״ש עושה נקוב כשאינו נקוב לכל דבר, דאי לפטור בלבד התולש מעציץ נקוב, הוי ליה למיתני התולש מעציץ נקוב חייב ור״ש פוטר, אלא להכי תנא פוטר בזה ובזה, לאשמועינן דזה כזה לכל דבר, והיינו נמי דאקשי ואזיל למימרא דר״ש נקוב כשאינו נקוב משוי ליה, ואילו לא קשיא ליה אלא מאי דפטר התולש בשבת מעציץ נקוב, הכי הוי ליה למימר אלמא לר״ש נקוב כתלוש הוא, ומשום הכי קשיא ליה דהא בברייתא קתני מיהא דלענין הכשר זרעים לא הוי נקוב כשאינו נקוב.\par \textbf{} ואינו מחוור בעיני כלל. חדא דפוטר בזה ובזה לא משמע שיהיה כלל גדול כל כך שלא יהא חלוק נקוב משאינו נקוב בשום מקום, דהא לא מפיק ברייתא מיניה אלא הכשר זרעים שיהא ראוי עליו להקשות. ועוד דאי כדקאמרת דלישנא דמתניתין כללא רבה לכל מילי, מאי קא מתרץ לכל מילי ר״ש כתלוש משוי לה, ושאני לענין טומאה דהתורה ריבתה טהרה אצל זרעים, דהא ודאי טעמא ברירא קאמר לענין ברייתא אמאי חלק ר״ש בין הכשר זרעים לשאר דברים, אלא לענין קושיתנו לא אנח לן כלום, דאכתי תקשי לן כללא דמתניתין, כיון שאי אפשר למהוי נקוב כשאינו נקוב לענין הכשר זרעים מיהא, אם כן היכי קתני במתניתין פוטר בזה ובזה, דמיניה שמעינן דזה כזה לגמרי. ועוד דדוחק הוא לומר דמאן דמקשה מברייתא לא הוי ידע טעמא דברייתא גופא, מאי שנא הכשר זרעים משאר מילי, אמאי בזה כנקוב ובעלמא כשאינו נקוב.\par \textbf{} וכן אין לפרש דהכי קשיא ליה דברייתא קתני אין בין נקוב לשאינו נקוב, וסבירא ליה דלחומרא קאמר דלכל מילי שאינו נקוב כנקוב, דאם כן הוי ליה למיתני אין בין שאינו נקוב לנקוב. ועוד דבהדיא תניא במס׳ כלאים (פ״ז, מ״ח) עציץ נקוב מקדש ושאינו נקוב אינו מקדש, ר״ש אומר זה וזה (אסורין) [אוסרין] ולא מקדשין, אלמא ר״ש לקולא אזיל בנקוב, וכל שכן דלא אזיל בשאינו נקוב לחומרא.\par \textbf{} והפירוש המחוור מה שפירש בתוס׳ דדינא דמתניתין קא קשיא ליה, היכי פטר ר״ש בתולש מן הנקוב דהא לגבי הכשר משוי ליה לנקוב כמחובר, וברייתא גופא לא הויא קשיא דקא סלקא דעתך דאין בין נקוב לשאינו נקוב דקתני, היינו לכל מילי דתלוי בארץ כגון כלאים ומעשרות ושביעית ופאה, משום דכתיב בהו ארצך או אדמתך או שדה וכרם, ומשום דסבירא ליה לר״ש דאפילו נקוב לאו שדה וכרם מקרי ולאו אדמתך מקרי, אבל לכל שאר מילי דלא תלוי בארץ אלא בתלוש ומחובר כהכשר זרעים יש הפרש בין נקוב לשאינו נקוב, והא דקתני אלא להכשר זרעים בלבד, לאו הכשר זרעים ממש קאמר בלבד, אלא הכשר זרעים והדומה להם בלבד קאמר, והלכך שבת נמי ליחייב, וקא סלקא דעתיה דמקשה הכין משום דלא הוי ידע לפרושי מאי שנא הכשר זרעים משאר מילי, ופריק לא כדקס״ד, אלא הכשר זרעים דוקא קאמר כפשטא דברייתא וטעמא כדמשני וכו׳.}
\textblock{\textbf{ועדיין כלי הוא לקבל רמונים.} פירוש: אם יחדו לרמונים, אבל סתם כלי חרס שיעורו כמוציא זיתים וכדתנן במס׳ כלים (פ״ג, מ״א) וכמו שכתב רש״י ז״ל.}
\textblock{\textbf{ובמוקף צמיד פתיל עד שיפחת רובו.} ואם מוקף צמיד פתיל בפיו מציל על מה שבתוכו אע״פ שלא הקיף צמיד פתיל בנקבו כיון שלא נפחת רובו דכלי פתוח כתיב כך פרש״י ז״ל, ואע״ג דלענין צמיד פתיל בעינן סתימה מעליא וכדמוכח התם במשנת מסכת כלים, הני מילי דרך פיו ואפילו נקב כל שהוא, ומן הצדדין כשנפחת בכדי שיעורו, ברברבי רובו ובזוטרי במוציא רמון, אבל בפחות מכן לא בעינן סתימה כלל.}
\textblock{\textbf{אבל ר״ת ז״ל הקשה היאך אפשר שיהיה נקוב במוציא זית או במוציא רמון ויהא מציל באהל המת, והא ראוי       } בנגיעה ממקום הנקב. ולפיכך פירש הוא ז״ל דלעולם בשסתם הנקב היטב, אלא דעד רובו א״נ עד מוציא רמון קרוי כלי, ומציל בצמיד פתיל, אבל יותר מכן אינו כלי, ואפילו הקיפו צמיד פתיל אינו מציל, דהוי ליה כאוכלין שגבלן בטיט דאמרינן בזבחים (ג, ב) דאינו מציל באהל המת, ואם תאמר א״כ מאי קא מבעיא ליה התם לרבא במסכת ב״ק (קה, א) גבי חבית שנקבה וסתמוה שמרים אגף חציה והניח חציה מהו, דאלמא לא בעי סתימה גמורה. יש לומר דהכי קאמר סתם חציה והניח חציה סתום מן השמרים מהו שיצטרפו אותן הסתימות, אבל הניח חציה בלא סתימה כלל פשיטא דאינה מצלת.}
\clearpage
\newsection{דף צו}
\textblock{\textbf{לא קשיא הא ברברבי הא בזוטרי.} פירש רש״י ז״ל: ברברבי ברובו, בזוטרי כמוציא רמון. וכן משמע לכאורה מדאמר רבא נקב כמוציא רמון טהור מכלום, ואם הוקף בצמיד פתיל עד שיפחת רובו, דאלמא רובו יותר ממוציא רמון. והקשו עליו בתוס׳ מדתנן במסכת כלים (כלים ב״ק פ״ז, ה״ו) חבית שנקבה וסתמוה שמרים הצילה, והרי חבית כלי גדול הוא ואם הולכים בו אחר רובו היאך אפשר לשמרים לסתום רובו, וההיא גבי צמיד פתיל תנינן לה במסכת כלים פרק אלו כלים מצילין בצמיד פתיל (פ״י, מ״א), ובתוספתא (פ״י, מ״א) קתני עלה ר׳ יהודה אומר אין צמיד פתיל מבפנים, כיצד חבית שנקבה וסתמוה שמרים אינה מצלת וחכמים אומרים מצלת. על כן פירשו בתוס׳ ברברבי כמוציא רמון בזוטרי ברוב, וההיא בשנקבה כמוציא רמון, ואפשר לשמרים לסתום כמוציא רמון.\par \textbf{} ולפי דבריהם נראה לפרש דהא דאמר רבא ובצמיד פתיל עד שיפחת רובו לאו אמוציא רמון דסליק מיניה קאי, אלא מלתא באנפי נפשה קאמר, כלומר: ויש מדה אחרת בכלי חרס שאינה לטהרת עצמו אלא לטהרת כלים שבתוכו, שאע״פ שניקב מציל בצמיד פתיל עד שיפחת רובו, ולעולם בזוטרי ובפחות ממוציא רמון, ומשום דשיעור רובו הוי דינא, ומוציא רמון חומרא להחמיר עליו אפילו במעוטו, לפום כן נקט רבא רובו משום דאיהו דינא. ומיהו לעיקר קושית התוס׳ שהקשו ממתניתין דחבית שנקבה, כתב הרמב״ן ז״ל דאינה קושיא, דחבית כלי העשוי למשקין הוא, וכל העשוי למשקין שיעורו בכונס משקה בצמיד פתיל, וכדתנן במס׳ כלים (פ״ט, מ״ח) נקב העשוי לאוכלין שיעורו בזיתים, העשוי למשקים שיעורו במשקין, העשוי לכך ולכך מטילין אותו לחומרא בצמיד פתיל בכונס משקה.}
\textblock{\textbf{פליגי בה תרי אמוראי במערבא חד אמר כמוציא רמון וכו׳.} פירש רש״י ז״ל: דבהכשר זרעים פליגי. והקשו עליו בתוס׳ דהא בהדיא תנן במסכת עוקצין (פ״ב, מ״י) גבי הכשר כמה הוא שיעורו של נקב כדי שיהא בו שורש קטן. והם פירשו לענין צמיד פתיל ובסתם כלי. והרמב״ן ז״ל הקשה דסתם כלי חרס לאוכלין הוא, ושיעורו בזיתים או יותר ושורש קטן פחות מכן הוא. ופירש הוא ז״ל דלהכשר זרעים הוא, כדברי רש״י ז״ל, ולר״ש קאמרי דאיהו מיקל בנקב ואפילו הכי בהכשר בעי טפי. [ולדבריו] אני תמיה היאך נחלקו סתם אמוראי אליבא דר״ש דלית הלכתא כותיה. ולדברי כולם אני תמיה אם להכשר זרעים או לצמיד פתיל הוא, למה הכניסו דברי רבי זירא באמצע, דלמעלה אחר דברי רבא ורב אסי היה לו לאמרה. וצריך עיון.}
\textblock{מתני׳:\textbf{ שתי גזוזטראות ברשות הרבים זו כנגד זו המושיט והזורק מזו לזו פטור.} כלומר: שלא היתה הושטה כזו במשכן.}
\textblock{}
\textblock{\textbf{שתיהן בדיוטא אחת הזורק פטור והמושיט חייב שכך היתה עבודת הלוים.} קשיא לי והא הושטת קרשים על גבי עגלות, כשתי דיוטות ברשות הרבים זו כנגד זו היתה, שהרי העגלות לאורך רשות הרבים היו, ורחבן לרוחב הרשות וארכן של קרשים הנתונין על רוחב העגלות לרוחב הרשות היה, כדאמרינן בגמרא שמהן למדנו רוחב רשות הרבים שהוא שש עשרה אמה, ואם כן על כרחנו לרוחב רשות הרבים היו מושיטין הקרשים, וכיון שכן אדרבה מושיט מזו לזו כשהן זו כנגד זו ברשות הרבים ליחייב (כרשות הרבים) שכן היתה עבודת הלוים, ובששתיהן בדיוטא אחת כלומר: לאורך רשות הרבים ליפטר שלא היתה עבודת הלוים כן.\par \textbf{} עד שראיתי לרבנו האי גאון ז״ל ענין יראה שאין ההקפדה כשהגזוזטראות לרוחב הרשות או לארכו, אלא בששתי הגזוזטראות עשויות לעשות דיוטא אחת כעגלות ששתיהן נעשות דיוטא אחת, או שתי דיוטות שלא כעגלות של מדבר, ומפני שאין דרך לקרות רה״ר כשמושיט מזו לזו בשני צדי רה״ר פטור, אבל כשהן לאורך הרשות שדרך לעשותן לדיוטא אחת חייב. וזה לשונו: שתי גזוזטראות זו כנגד זו ברה״ר, גזוזטרא רה״י היא, וכששתים זו כנגד זו ברה״ר ומופלגות זו מזו ואינן דומות לדיוטא אחת, אלא זו רשות לעצמה והיא רה״י וזו רשות לעצמה והיא רה״י ורה״ר ביניהם מזו לזו וכו׳, אבל אם הגזוזטראות שתיהן העומדות ברה״ר מוכנות להיות דיוטא אחת ורשות אחד, ובאיזה צד שמרכיבין נסר מזו לזו על גבי רשות הרבים כדי להיות שתיהן דיוטא אחת ורשות אחד, אם עשאן כן מבעוד יום הרי הן כרשות אחת ומותר לטלטל מזו לזו על גבי אותו הנסר שהוא מתוח מזו לזו על גבי רשות הרבים, כדתנן התם (עירובין עח, ב) לענין חריץ שבין שתי חצרות, עמוק עשרה ורחב ארבעה מערבין שנים ואין מערבין אחת, ותנן עלה נתן עליה נסר שהוא רחב ארבעה טפחים, וכן שני גזוזטראות זו כנגד זו, מערבין אחת ואין מערבין שנים, פחות מכאן מערבין שנים ואין מערבין אחד, והכי קאמר בב׳ גזוזטראות כחריץ, אם נתן על שני הגזוזטראות נסר שהוא רחב ארבעה, וגזוזטראות גבהן עשרה הרי הן כרשות אחד, ואם הנסרים מוצבין ביניהן מבעוד יום אלא שקדש היום ולא חוברו, אסור להושיט מזו לזו ובין זולתו, ואם הושיט חייב, כי כך היו העגלות שנמסרו במדבר ללוים. שתי עגלות זו אחר זו ברשות הרבים, והיו מותחין קרשים ומציבין קרש על שתי עגלות ראשו זה על זו וראשו זה על זו כנסר שפירשנו ענינו למעלה, וזהו היו שתיהן בדיוטא אחת המושיט מזו לזו חייב, כי כן היה דרכן של לוים בחול שהיו מושיטין את הקרשים מזו לזו וזו היא מלאכתן, ולפיכך העושה כן בשבת חייב, אבל לא היו זורקין כדי שלא לנהוג בזיון בקרשים, ולפיכך הזורק מזו לזו פטור אבל אסור. והזהרו בפירוש זו כי בטורח גדול נגלה לנו. ע״כ.}
\textblock{ גמרא:\textbf{ הוצאה גופה היכא כתיבא.} לא הוי ליה למימר אלא הוצאה גופה במשכן היכי הואי, כדבעי בפרק כלל גדול (לעיל שבת עד, ב) קשירה במשכן היכי הואי, אלא משום דהוצאה מלאכה גרועה בעיא היכא כתיבא, דאי לא, לא מחייבי עלה אע״ג דהואי במשכן וכדכתבינן בריש מכלתין. ומשום הכי איכא דגריס הכא וממאי דבשבת קאי וכו׳ גמרי העברה העברה וכו׳, דאי לא לא מחייבינן עלה אע״ג דהות במשכן. אלא מיהו רבינו האי גאון ז״ל כתב דלא גרסינן ליה וטעותא הוא, דמעיקרא לא איסתפקא להו אלא מנא ליה דהויא מלאכה, ומהדרינן משום דכתיב (שמות לו, ו) אל יעשו עוד מלאכה ויכלא העם מהביא, דאלמא הבאה דהיינו הוצאה קרויה מלאכה. וכן כתבה הרמב״ם ז״ל בזה הלשון בספרו (פי״ב, ה״ח). והוא הנכון.}
\textblock{ הכי גרסינן:\textbf{ הך דהואי במשכן חשיבא הך דלא הואי במשכן לא חשיבה, הך דכתיבא קרי לה אב הך דלא כתיבא קרי לה תולדה.} ופירושו, הואי במשכן קאמר משום שאר מלאכות, וכתיבא קאמר משום הוצאה, דהכנסה נמי הואי במשכן שהיו הלוים מכניסין הנדבה ממחנה לויה ללשכות, אלא הך דכתיבא בהדיא ומפרש בה קרא קרי ליה אב. ובשאר מלאכות נמי אף על גב דאיכא מינייהו במשכן ואעפ״כ קרי להו תולדות כשובט ומדקדק (לקמן שבת צז, ב), היינו משום דבכלל האחרות הן לגמרי ואינו בדין שיהיו מחולקין מהן לגמרי לענין חטאת.\par \textbf{} ואיכא למידק אמאי לא קאמר דנפקא מינה נמי אפילו לר״א דאי עבד אב ואב דידיה בהעלם אחד, כגון דש ודש אינו חייב אלא אחד, ואילו עביד אב ותולדה כגון דש ומפרק מחייב תרתי. ולאו מלתא דאם כן ליקרינהו לכולהו אבות או מלאכות. ואכתי איכא למידק דדלמא להכי קרי ליה תולדות דאי מתרו ליה משום אב דידיה מחייב, וכדאמרינן (לקמן שבת קלח, א) משום מאי מתרינן ביה משום בורר, ואילו הוו כולהו אבות ומתרה ביה משום אב אחר כגון שמתרה למפרק משום דש לא מחייב. ויש לומר דלר׳ אליעזר כיון שהן מוחלקות לחטאות מן האבות להתחייב בהן במקום אב, אינו בדין שיתחייב על תולדה כי מתרה ביה משום אב דידיה.}
\textblock{\textbf{זרק ארבע אמות ברשות הרבים מנא ליה דמחייב.} פירוש: דקא סלקא דעתך (דזריקה ברשות הרבים מנא ליה       דזריקה מרשות לרשות הוא דהויא תולדה דהוצאה דהיא היא נמי מרשות לרשות, אבל זריקה ברשות הרבים [מנא ליה דמחייב, ואסיקנא דהא] נמי גמרא גמרינן לה, כלומר: דהויין תולדות דהוצאה ולא שיהיה מלאכה בפני עצמה, דאם כן הויין להו ארבעים או ארבעים ואחת ואנן ארבעים חסר אחת תנן, ואם כן לדידן מנא לן דמחייב על זריקת ועל העברת ארבע אמות ברשות הרבים ולקמן נמי (צז, ב) קרינן ליה לזריקת ארבע אמות תולדה דהוצאה, דאתיא למפשט דר׳ יהודה מחייב אתולדה במקום אב מדתניא מרשות הרבים לרשות היחיד ועבר ארבע אמות ברשות הרבים ר׳ יהודה מחייב וחכמים פוטרין, וקא סלקא דעתך דר׳ יהודה מחייב תרתי קאמר, דאלמא זריקת ארבע אמות ברשות הרבים תולדה דהוצאה היא. ובספר המאור כתוב לפי שארבע אמות בכל מקום קונות לו וכרשותו דמיין, וכשמוציא חוצה להן בזורק או במעביר כמוציא מרשות היחיד לרשות הרבים דמיין.}
\textblock{\textbf{א״ר יאשיה שכן אורגי יריעות זורקין מחטיהן.} תמיהא לי לרבי יאשיה ולרב חסדא דילפי ליה מאורגי יריעות או מתופרי יריעות, אם כן ארבעים הוי דמאן דיליף לה מהכא לאו תולדה היא. ובירושלמי (ה״א) גרסינן ר׳ יודה עביד ארבע אמות בר״ה מלאכה בפני עצמה, על דעתיה דרבי יודה ארבעים מלאכות נינהו ונתנינהו, לא אתיא במתניתין אלא מילין דכל עמא מודו בהון, ר׳ זעירא משום רבי יוחנן מתופרי יריעות למד ר׳ יודה, שהרי תופרי יריעות מזרקין מחטיהן אלו לאלו, וזה כדברי שכל מי שלמד אותה ממשכן אינו עושה אותה תולדה אלא אב, וארבעים הן. ואי אפשר לתרץ לרבי יאשיה ולרב חסדא כמו שתירצו בירושלמי דאנן ארבעים חסר חד תנן, ואם כן זורק ומעביר ארבע אמות ברשות הרבים לא מיחייב, דאי מחייבת להו אף אתה אומר דארבעים הן, ואי אמרת דאינן אלא ארבעים חסר אחת על כרחין מעביר וזורק ארבע אמות ברשות הרבים פטור, דמנא ליה דמחייב, וצריך לי עיון. אלא משום דלא קמה, לא דייקינן בה כולי האי.}
\textblock{ הא דאמרינן:\textbf{ והא במקום פטור קא אזלא.} מסתברא דלאו דוקא במקום פטור, אלא כרמלית קאמר דמקום אורגי יריעות רשות הרבים מקורה היה והיינו כרמלית, ועוד דיריעה עצמה רחבה ד׳ באמה והויא כרמלית, ואדר׳ יאשיה ורב חסדא נמי הוי ליה לאקשויי הכי והא כרמלית היא, כדאמרן דמקורה הוא, אלא דעדיפא מינה אקשי ליה. כך נראה לי.}
\textblock{\textbf{אימא אינו חייב אלא על אחת מהן.} כבר כתבתיה בריש פרק קמא דמכלתין (ו, ב) בסייעתא דשמיא.}
\clearpage
\newsection{דף צז}
\textblock{\textbf{בעי רבה למטה מעשרה פליגי וכו׳ אבל למעלה מעשרה דברי הכל פטור.} פירש רש״י ז״ל: ואי בהכי פליגי, הא דקתני שני גזוזטראות בדיוטא אחת הזורק פטור כולי עלמא היא, כלומר: דאיכא לאוקומה בלמעלה מעשרה וכולי עלמא. ואם תאמר אם כן מאי כיצד, פירשו בתוס׳ דהאי כיצד לאו אפלוגתא דר׳ עקיבא ורבנן קא מהדר, אלא מלתא באפה נפשה היא, ואגזוזטראות קאי, כלומר: כיצד דין שני גזוזטראות, ודכותיה ביבמות פרק ר״ג (יבמות נ, א עיי״ש) וחכמים אומרים יש גט אחר גט ויש חליצה אחר חליצה כיצד כו׳, ולא קאי ארישא, אלא כיצד דין יבם ויבמה קתני כדאיתא התם. ומיהו יש ספרים דלא גרסי הכא כיצד, ויש לפרש דאי מוקמינן פלוגתייהו בלמטה מעשרה אבל בלמעלה דברי הכל פטור לא גרסינן כיצד. ואי מוקמה פלוגתייהו אפילו בלמעלה מעשרה גרסינן כיצד, וסיפא רבנן הוא ולא רבי עקיבא.\par \textbf{} ובירושלמי (שם) נמי איתא הכין: שמואל אמר לא שנו אלא למטה מעשרה הא למעלה אסור. פירוש: אסור בלבד אבל פטור מחטאת, ור׳ אלעזר אמר אפילו למעלה מעשרה, דא״ר אילא בשם ר״א מעגלות למד רבי עקיבא ועגלות לא למעלה מעשרה אינון, שתי גזוזטראות אית דתני כיצד ואית דלא תני כיצד, על דעתיה דר״א אית כאן כיצד על דעתיה דשמואל לית כאן כיצד, וזה מבואר כמו שאמרנו.}
\textblock{\textbf{ובמאי אילימא במעביר למטה מעשרה הוא דמחייב למעלה מעשרה לא מחייב והא אמר ר״א המוציא משוי למעלה מעשרה טפחים חייב.} פירוש: לאו למימרא שהמעביר מרשות היחיד לרשות היחיד דרך רשות הרבים ולא הניחו ברשות הרבים שיהא חייב, אלא הכי קאמר: במאי אילימא במעביר דוקא אבל לא בזורק ומשום דקלוטה לאו כמי שהונחה דמי, אלא דמעביר הוא דחייב ומשום דסבירא ליה לרבי עקיבא דמהלך כעומד דמי, אם כן אפילו למעלה מעשרה נמי וכר״א. ובתוס׳ במסכת עירובין (לג, א) מצאתי דמעביר מרשות היחיד לרשות היחיד דרך רשות הרבים חייב, ודייקי לה מהא. ואין זה מחוור לי כלל. ושם הארכתי בפרק בכל מערבין (שם) גבי נתנו באילן בס״ד.}
\textblock{\textbf{ופליגא דרב חלקיה בר טובי.} פירש רבנו האי גאון ז״ל: דרב חלקיה פליגא אתרוייהו, דאדר׳ אלעזר פליגי דקאמר      מחייב למעלה מעשרה, ורב חלקיה קאמר דלמעלה מעשרה כולי עלמא לא פליגי דפטור, ועוד דאע״ג דבין לר״א בין לרב המנונא רבנן פטרי בלמעלה מעשרה כר׳ חלקיה, מכל מקום לא שמעינן להו דקא מפלגי לרבנן בין תוך שלשה ובין למעלה משלשה, ומשמע נמי מברייתא דרב המנונא דאפילו תוך שלשה קא פטרי רבנן, מדקתני ועבר ברשות הרבים עצמה, וכדרבא דאמר לקמן (שבת ק, א) גבי חולית הבור והסלע, תוך שלשה לרבנן צריך הנחה על גבי משהו, ואילו לרב חלקיה תוך שלשה דברי הכל חייב.}
\textblock{\textbf{וכגון דאמר עד דנפקא לרשות הרבים תנוח.} פירוש: לא שאמר סתם בלשון הזה עם יציאתה תנוח, דאם כן תנוח ותעמוד משמע, והרי לא נחה אעפ״י שלענין שבת הויא כמונחת. אלא הכי קאמר רוצה אני שתהא כמונחת לענין שבת עם יציאתה, שלא אתחייב בהוצאה אלא בהנחת יציאתה לרשות הרבים, ומשום הכי מחייב לר׳ יהודה כיון דסבירא ליה דקלוטה כמי שהונחה. הרמב״ן ז״ל.}
\textblock{ הא דאמרינן:\textbf{ ממאי דלמא לעולם דעבד חדא חדא.} תמיהא לי מאי קאמר, דהא כיון דאוקי פלוגתייהו דרבנן סברי תולדות נינהו ור׳ יהודה סבר אבות נינהו, אם כן אפילו עבדינהו תרוייהו בהדי הדדי מחייב ר׳ יהודה אתרוייהו משום דאבות נינהו. ויש לומר דכלפי מאי דקאמר ליה רב יוסף דקא עבדינהו תרוייהו בהדי הדדי, כלומר: דעיקר דברי רבי יהודה לחיוביה בדעבדינהו תרוייהו בהדי הדדי, ומשום אב ותולדתו, קא מהדר ליה איהו דלאו בדעבדינהו תרוייהו פליגי אלא אפילו בדעבדינהו חדא חדא, ואי אבות נינהו או תולדות נינהו פליגי. כך נראה לי.}
\textblock{\textbf{אתמר נמי רבה ורב יוסף דאמרי תרוייהו לא חייב רבי יהודה אלא אחת.} כלומר: דלא מחייב ר׳ יהודה אתולדה במקום אב, אבל רב יהודה משמיה דשמואל סבירא ליה דמחייב תרתי ומשום תולדה במקום אב. ואי נמי איכא למימר דשמואל סבירא ליה זורק ד׳ אמות ברשות הרבים אב חשיב ליה ומשום אב מחייב ליה, וכן כתב רבינו האי גאון ז״ל. והכי נמי איתא בירושלמי (ה״א) דמפרשי התם טעמיה דר׳ יהודה משום דזורק הוה ליה אב, והוא הירושלמי שכתבתי לעיל (שבת צו, ב) דגרסינן התם ר׳ יודה עבד ארבע אמות ברשות הרבים מלאכה בפני עצמו, על דעתיה דר׳ יודה ארבעים מלאכות נינהו, ר׳ זעירא בשם רבי יוחנן מתופרי יריעות למד ר׳ יודה שהיו זורקין את המחטין אלו לאלו. ותמיהא לי למה לא העמידוה כן בגמרין והוצרך לדחוק דרבנן פטרי לגמרי, ואם תאמר כדי שלא נעשה רבי יהודה חולק על מנין ארבעים חסר אחת, הא ליתא, דעל כרחין ר׳ יהודה פליג ולדידיה טפי הוו דהא איכא שובט ומדקדק דעביד להו אבות כדאיתא בסמוך. ויש לומר משום דפשטא דמלתא הכין מכרעא להו טפי דר״י מחייב חדא משמע וחכמים פוטרים נמי לגמרי משמע, דאפילו רב יוסף לא הוה מפיק ליה מהאי משמעותא אלא משום דלא הוה משכח ליה פתרי דהיאך אפשר דליפטרו רבנן לגמרי דהא קא מפיק, הא לאו הכי לישנא ודאי טפי משמע דפטרי לגמרי.\par \textbf{} ואיכא למידק אשמעתין, דהכא משמע לכולי עלמא דר׳ יהודה אית ליה קלוטה, ואילו בעירובין פרק המוצא תפילין (עירובין צז, ב) תנן היה קורא בראש הגג ונתגלגל הספר מידו, עד שלא הגיע לעשרה טפחים גוללו אצלו, משהגיע לעשרה טפחים הופכו על הכתב, רבי יהודה אומר אפי׳ אינו מסולק מן הארץ אלא מלא חוט גוללו אצלו, דאלמא אפי׳ תוך ג׳ לרבי יהודה בעינן הנחה ע״ג משהו. וי״ל דלית ליה לר׳ יהודה אלא כשרוצה שתנוח שם וא״נ היכא דאמר כ״מ שתרצה תנוח, אבל התם דאינו רוצה שינוח שם הספר דאינו רוצה שיתגלגל הספר מידו, בהא לית ליה קלוטה ובעינן הנחה על גבי משהו.\par \textbf{} והרמב״ן ז״ל תירץ דר׳ יהודה לית ליה קלוטה אלא היכא דראוי לנוח שם כגון זורק שנקלטה כולה באויר רשות הרבים, אבל במעביר אי נמי בתלוי ועומד שאין סופו לנוח שם, שדבר אחר גורם לו שלא ינוח כגון ההיא דעירובין דתלוי ועומד ואינו בעצמו קלוט באויר, לית ליה קלוטה.}
\textblock{\textbf{ואם תאמר בין להדין סברא בין לסברא קמא, רבא דאמר לקמן (שבת ק, א) דלעולם בעינן הנחה על גבי משהו ואפילו בזורק כמאן סבירא ליה. ועוד היכי אמרינן התם בפרק המוצא תפלין גבי ההיא דהיה קורא בספר לימא רבא כתנאי אמרה לשמעתיה, דהא רבא לאו כחד מנייהו וכדאמרן. יש לומר דסבר רבא דמדר׳ יהודה נשמע לרבנן, דכי היכי דר׳ יהודה      } דאית ליה קלוטה ברוצה שתנוח, כי נקלטה במקום שאינו רוצה שתנוח בעי הנחה על גבי משהו ואפילו תוך שלשה, לרבנן נמי דלית להו קלוטה בשום מקום הכי נמי בעו הנחה על גבי משהו ואפילו תוך שלשה, ו[ד]אמרינן התם לימא רבא כתנאי אמרה לשמעתיה, לאו כולה שמעתיה קאמר, אלא דסבר כר׳ יהודה בהא דבעי הנחה על גבי משהו כל היכא דלית ליה קלוטה. כן תירצו בתוס׳.}
\textblock{ הא דאמרינן:\textbf{ לאו היינו דאמר ליה רבינא לרבא ואמר ליה באומר כל מקום שתרצה תנוח.} פירשו רב האי גאון ז״ל ורש״י ז״ל דאפילו אקמייתא קאי, כלומר: אפילו נתכוון לזרוק שמונה וזרק ארבעה דפשיטא לך דאינו חייב שהרי לא נעשית מחשבתו, דהא אמר ליה רבא לרבינא דדוקא באומר כל מקום שתרצה תנוח, הא לאו הכי פטור משום דלא בעי לה הכא ולא נעשית מחשבתו, וגרסי׳ הכי אמאי הרי כתב שם משמעון: ולא גרסי אמר מר, וקושיא היא דקא מקשה אמאי דפטר אפילו בנתכוון לזרוק שמונה וזרק ארבעה, כלומר: מאי שנא מכותב שם משמעון, ופרקינן התם כל כמה דלא כתב שם לא מכתב ליה שמעון, אבל הכא כל כמה דלא זריק ארבע לא מזדריק תמניא, בתמיהא, כלומר: מי דמי זורק לכותב כותב כל כמה דלא כתב שם שי״ן מ״ם כל חדא באפי נפשא לא מיכתיב ליה שמעון, והלכך הרי אתעבידא מלאכה גמורה מדעתו, אבל זורק מי לא יכיל למיזרק תמניא בבת אחת כי לא נח בינייהו כלל, והלכך כי נח בסוף ד׳ לא מדעתו נח ולא אתעבידא מחשבתו.\par \textbf{} וכתב רב האי גאון ז״ל ומיפרשא בסוף פרק כיצד הרגל מועדת (בבא קמא כו, ב) נתכון לזרוק שמונה וזרק ארבע אי אמר כל שתרצה תנוח אין ואי לא לא. ותימא בעיני קצת שזו מחודשת כאן בתלמוד, דלאחר שאמר פשיטא לי מבעיא ליה וחולק התלמוד אפילו באותה שהיתה פשיטא לשואל.\par \textbf{} ויש מי שגורס אמר מר הרי כתב שם משמעון. ולפי גירסא זו נתכוון לזרוק שמונה וזרק ארבע חייב כדאפשיט ליה מעיקרא. והא דאמרינן: לאו היינו דאמר ליה רבא לרבה. אנתכוון לזרוק ד׳ וזרק ח׳ בלחוד קאי ובההיא הוא דפטרינן, אלא דהשתא הוא דקא מהדר ומדקדק אנתכון לזרוק שמונה וזרק ארבעה דפשיטא ליה ומדמינן לה לכתב שם משמעון, ומתמה ואזיל מי דמי התם כל כמה דלא כתב שם לא מכתב שמעון, אבל זורק אפשר שתנוח בסוף שמונה אע״ג דלא נח בסוף ארבע, [ופרקינן ה״נ כמה דלא זריק ארבע] לא מזדריק תמניא, כלומר: אי אפשר שתנוח בתמניא אלא אם כן עברו בתוך ארבע. ולפירוש זה הא דקא פשיט נתכוין לזרוק שמונה וזרק ארבע חייב הכא סלקא. וכן כתב הרמב״ם ז״ל (פי״ג, הכ״א). ולפי דבריו הא דאמרינן לעיל (בעמוד א) מידלי חד ומתתי חד גזירה דלמא נפלה ואתי לאתויי, דמשמע דוקא גזירה דלמא אתי לאתויי הא לדלמא נפלה בלחוד לא, התם היינו טעמא דנתכוון להיתר ועלה בידו איסור כנתכוון לזרוק שתים וזרק ארבע, אבל אם זרק מרשות הרבים לרשות הרבים ורשות היחיד באמצע ונתכוון לזרוק ארבע אמות ברשות הרבים ונפלה ברשות היחיד חייב, דהוי ליה ככותב שם משמעון דהוצאה וזריקה מלאכה אחת הן, דזריקה תולדה דהוצאה היא ואפילו זריקת ארבע אמות ברשות הרבים, כדאיתא לעיל (שבת צו, ב). וההיא דפרק כיצד הרגל (שם) דמייתי מינה ראיה רבנו האי גאון ז״ל, גם שם הפכו הגירסא וגרסי נתכוון לזרוק ארבע וזרק שמונה.}
\clearpage
\newsection{דף צח}
\textblock{\textbf{הא קא משמע לן רשויות מצטרפות ולא אמרינן קלוטה כמי שהונחה.} פירש רש״י ז״ל: רשויות מצטרפין, ולא כרבי יוסי דאמר בפרק המוציא (לעיל שבת פ, א) אם בהעלם אחד לרשות אחד חייב לשתי רשויות פטור. ואינו מחוור בעיני, דאם כן מלתא דפשיטא ליה לרבי יוסי לפטורא מתמהינן אנן ואמרינן פשיטא לחיובא, ועוד דשקלי וטרי בה אביי ורבא ורב אשי. ומסתברא דלא שייכי אהדדי, דהתם במוציא חצי זית וחזר ומוציא חצי זית, הלכך דינא הוא דלא יצטרפו אלא כשתי הוצאות      הן לגמרי, אבל הכא דבזריקה אחת דינא הוא שיצטרפו. כך נראה לי.}
\textblock{ והא נמי דאמרינן:\textbf{ וקמ״ל דקלוטה לאו כמי שהונחה.} פירשה רש״י ז״ל אפחות מארבע אמות, דמדפטור בפחות מארבע ואע״ג דעברה דרך רשות היחיד שמע מינה דקלוטה לאו כמי שהונחה. וגם זה אינו מחוור בעיני, דמנא לן דפטור משום דקלוטה לאו כמו שהונחה, דהא אי אפשר לומר קלוטה להתחייב עליה אלא באומר כל מקום שתרצה תנוח הא לאו הכי לא, וכיון שכן מנא לן דלית ליה קלוטה, דלמא אית ליה ושאני הכא דאפילו נחה ברשות היחיד ממש לא מחייב דהא לא אתעבידא מחשבתו, וכדאמרינן לעיל (שבת צז, א) כגון דמידלי חד ומתתי חד דזימנין דנפלה ואתי לאתויי, דאלמא דוקא משום גזירה דלמא אתי לאתויי הא משום גזירה דלמא נפלה לא, דאפילו נפלה לא מחייב כיון דלא נתכון שתנוח באותו רשות. ואם תאמר דהתם שאני דנתכוון לעשות היתר ועלה בידו איסור, מכל מקום לא עדיף מנתכוין לזרוק שמונה וזרק ארבע דפטור לדבריו ז״ל בעצמו אעפ״י שנתכוין לאיסור ועלה בידו איסור, וכיון שכן מנא לן דהכא בדאמר כל מקום שתרצה תנוח דנשמע מינה קלוטה לאו כמי שהונחה. ואם תאמר אם כן פשיטא למאי איצטריך ליה למיתני פחות מארבעה פטור. יש לומר דלמא משום רישא נקט לה.\par \textbf{} ורב האי גאון ז״ל פירשה ארישא, כלומר: ומדמחייב ליה בשזרק ארבע אמות שמע מינה נמי קלוטה לאו כמי שהונחה, ולא אמרינן כיון שקלטתו באמצע ארבע אמות כמי שנח ברשות היחיד דמי ונמצאו ארבע אמות ברשות הרבים מקוטעות ולא רצופות, עד כאן. פירוש לפירושו: דלמאן דאמר קלוטה כמי שהונחה אפילו זרק ארבע אמות פטור, דמשום זורק ארבע אמות ברשות הרבים לא מחייב, דהרי זו כמי שהונחה ברשות היחיד כיון דרשות מוחלק הוא, ומשום מוציא ומכניס ליכא לחיוביה, דהא לא אתעבידא מחשבתו.}
\textblock{ גירסת הגאונים ז״ל:\textbf{ כי קאמר רב בדרא תתאה.} פירוש: בשורה התחתונה שיש ריוח בין קרש לקרש, אבל לאחר שנתן השורות העליונות אי אפשר לצמצם הקרשים זה על זה ונוטים אילך ואילך ומכסין את הריוח. ורש״י ז״ל גריס בדראתה כמו שכתב בפירושיו.}
\textblock{\textbf{אמר רב כהנא באטבי.} מלשון כמלא אטבא. ופירש רבינו האי גאון ז״ל: כי קאמר רב דוקא בזמן דלית להו אלא אטבי בלחוד, וקודם נתינת הקרשים, ואקשינן היכא מונח להו אגבה דעגלה עגלה מקורה היא, כלומר: כי אין מניחין האטבי אלא בזמן נתינת הקרשים, דאז מניחין את האטבי על גבי עץ העגלה כדי לאחוז את הקרשים כדי שלא ישמטו, ובאותו העת עגלה מקורה היתה בקרשים, נתבטלו בהא דברי רב חסדא.}
\textblock{ ואסיקנא בה:\textbf{ אמר שמואל ביתדות.} כלומר: שהיו מניחין יתדות ועמודים של קלעים על חלל העגלה תחלה בשתי וערב, והא הן דרא תתאה ועליהן מניחין את הקרשים, ובזמן שהיתדות בלבד על העגלה אמר רב שתחתיה רשות הרבים. ורש״י ז״ל פירש בענין אחר. ויש מי שפירש אטבי היכא מנח להו אגבה דעגלה, כלומר: שאין דרך להניח אלא על צפוי העגלה כשהעגלה מקורה נועצין בגבה האטבי, ואם אתה מעמידה באטבי עגלה גופא מקורה הואי, אמר שמואל בזמן אטבי בלחוד וכדרב חסדא, ואפ״ה לא במקורה בנסרים אלא בשיש אגבה דעגלה יתידות בלבד שעליהן תוחבין האטבי. וא״ת מפני מה דחקו להעמידה בכך, לוקמה בשאין שם כלום. תירץ רבינו האי גאון ז״ל, דתחתיהן ליכא למימר אלא בזמן שיש עליהן שום דבר ולא בזמן שהן מגולות לגמרי.}
\textblock{\textbf{דשפי ליה כטריז.} כך גריס רבינו האי גאון ז״ל, ופירש דטריז בלשון פרסי חתיכות בגד שתופרין בשפת החלוק כמין כיס, לשים בו מעות שקורין אלגי״ב, והוא רחב מלמטה והולך ומתקצר עד כאצבע מלמעלה.}
\clearpage
\newsection{דף צט}
\textblock{\textbf{אמר רבא וצידי עגלה כמלא עגלה.} פרש״י ז״ל: בין שני צידי עגלה דנמצאת כל העגלה מן הקצה אל הקצה עם הצדדים ואופניהם חמש אמות. ומה שאמרו בין עגלה לעגלה ה׳ אמות, לא מחמת אורך הצדדין, אלא מחמת אורך הקרשים, ובין עגלה לעגלה דקאמר לא מגוף העגלה קאמר, אלא מאופן עגלה לאופן עגלה קאמר, כי הוא ז״ל פירש שהקרשים אינן מוטלין על שתי עגלות, אלא על עגלה אחת, והקרש עודף מכאן ומכאן, דנמצא רוחב העגלה אוכל מן הקרש שתי אמות וחצי, ונמצא הקרש עודף בין שני צדדין       אמות ומחצה, צא מהם שתי אמות ומחצה לצידי העגלה, נשארו שני אמות ומחצה מכאן ושתי אמות ומחצה מכאן חוץ לצדדין, וכן הקרש האחר הנתון על עגלה השנית, ושתי העגלות היו הולכות זו בצד זו לרוחב הדרך, נמצא שיש בין אופן עגלה לאופן עגלה חמש אמות [או] יותר כדי שלא יגיע קרש שבזו לקרש שבזו ויעכבו את הליכתן, וכי קאמר כמלא אורך עגלה לא דק.\par \textbf{} והרב ז״ל הקשה על דברי עצמו, דאם כן היכי ילפינן לרוחב רשות הרבים שהוא שש עשרה אמות מעגלות, והלא יותר מעשרים ואחד הן, שהרי הקרשים ארכן עשר אמות והן מוטלין על דפני העגלה לרחבן ונמצא ארכן של קרשים לרוחב הדרך, הרי עשרים אמה לבד מאמה שהיה צריך למשוך קרשי כל עגלה ועגלה לצד החיצון של עגלה כדי שלא יגעו הקרשים זו בזו ויעכבו הלוך העגלות. ותירץ הוא ז״ל שאין מחשבין מן הקרשים, לפי שהקרשים למעלה מעשרה הן, ולגבי רשות הרבים לית ליה למיחשב אלא מקום העגלות ורוחב שביניהן.\par \textbf{} ור״ת ז״ל השיב עליו כמה תשובות: חדא דכיון דלא היה הקרש שוכב אלא על עגלה אחת, היאך הקשו באמתא ופלגא סגיא, והלא קרשים גדולים ועבים כל כך, היאך אפשר להם במושב אמה וחצי, והיאך שיער המקשה ודקדק כל כך שאפשר להן באמה וחצי, ועוד שאפילו תחשוב מקום העגלות בלבד היה טפי מט״ו אמה, דאי אפשר לקרבן כל כך כדי שלא יעכבו הקרשים הלוך העגלות, ועוד אם מן העגלות אנו מחשבין אף אנו לא נחשוב אלא מקום העגלות עצמן ואינן אלא עשר לדברי רש״י ז״ל, כי מקום פנוי שבין עגלה לעגלה אינו אלא מחמת אורך הקרשים, והקרשים אינן מתחשבין לפי דבריו, ועוד דמכל מקום אע״פ שהקרשים למעלה, צריכין היו להניח מקום פנוי בין האהלים יותר מעשרים ואחד אמה כדי שיוכלו הקרשים לעבור שם, ונמצא הדרך רחב כרוחב הקרשים ומקום בן לוי.\par \textbf{} ועוד אם לא היה הקרש מוטל אלא על עגלה בפני עצמה, מנא ליה שהיו שתי עגלות הולכות ביחד זו בצד זו, דלמא זו אחר זו ממש היו הולכות, או דלמא אחת מקדמת ואחת מאחרת, או שמא ארבעתן היו הולכות זו בצד זו ונמצא הרשות רחב מאד, ועוד קשיא לי לדידי דבכל השמועה אינו קורא עגלה אלא להעגלה ולשאר קורא צדדין, ואם כן כשאמרו בין עגלה לעגלה כמלא אורך עגלה היינו בין עגלה עצמה לעגלה עצמה, כלומר: מן החלל ועד החלל ולא מן האופן לאופן. ועוד כיון דאמר בין עגלה לעגלה כמלא אורך עגלה, סתמא דמלתא שאין אתה אומר אורך עגלה אלא לחלל אורך העגלה, כך בין עגלה לעגלה היינו מחלל העגלה לחלל העגלה.\par \textbf{} אלא הפירוש הנכון כמו שפירשו הגאונים ז״ל. שהקרש היה מוטל ראשו האחד על עגלה זו וראשו השני על העגלה השניה וראש הקרשים מצומצמין על דופן חלל העגלות, וצידי עגלה כמלא רוחב עגלה, כל צד וצד קאמר שנמצא מן האופן ועד האופן שבע אמות ומחצה, והיינו דאמרינן דבין עגלה לעגלה כאורך עגלה, לפי שמחלל עגלה זו עד סוף האופן שתי אמות וחצי וכן לעגלה השניה, ונמצאו אופני שתי העגלות שעליהן שוכבין הקרשים סמוכין זה לזה, ונמצא בין סוף עגלה זו לסוף עגלה זו ט״ו אמות.}
\textblock{\textbf{מסייע ליה לרבי יוחנן דאמר בור וחוליתה מצטרפין לעשרה.} וכי היכי דמצטרפי לעשרה, הוא הדין דמצטרפי רחב כותלי החוליא לחלל הבור לרוחב ארבעה. וכן פרש״י ז״ל בערובין פרק המוצא תפילין (עירובין צט, ב). ובירושלמי משמע דדוקא כשהעומד רבה על החלל אבל אם החלל רבה על העומד לא מצטרפי, דגרסי׳ התם בפרקין דהכא (ה״ב): א״ר יוחנן העומד והחלל מצטרפין בארבעה והוא שיהיה העומד רבה על החלל, ר׳ זעירא בעי עד שיהיה עומד שכאן ועומד שכאן רבה, אמר ר׳ יוסא פשיטא לר׳ זעירא שאין עומד מצד אחד מצטרף, פשיטא ליה שיהא עומד מצד אחד רבה, לא צורכא דלא אפילו עומד השני. ע״כ בירושלמי. אע״פ שיש בו קצת גמגום ונראה כטעות ידי סופר, מכל מקום למדנו משם דדוקא כשהעומד רבה על החלל. וכן נראה לי לדקדק מדאביי דאמר לקמן (שבת ק, א) בור עמוק עשרה ורחב שמונה חלקו במחצלת פטור דהנחת חפץ וסילוק מחיצה בהדי הדדי אתו, ואם איתא דכל מחיצה מצטרפת לחלל הבור, אם כן לא סילקה מחצלת זו מרוחב הבור כלום, שאף היא מצטרפת לחלל, אלא כמו שכתבתי, ויותר מזה כתבתי בריש פרק קמא דמכלתין (ח, א ד״ה רחבה) גבי זרק כוורת בסייעתא דשמיא.}
\textblock{\textbf{או דלמא כיון דממקום פטור קא אתא לא.} איכא למידק ותפשוט מדבן עזאי, דתניא בריש פרק קמא דמכילתין (ו, א) אחד המוציא ואחד הזורק ואחד המכניס חייב, בן עזאי אומר המוציא והמכניס פטור המושיט והזורק חייב, פירוש: מחנות לפלטיא דרך סטיו. ועוד דהא אמר רבא לעיל (שבת ח, ב) המעביר ד׳ אמות דרך עליו חייב, ואמרינן (צב, א) המוציא משוי למעלה מעשרה טפחים חייב שכן היה משא בני קהת. ועוד תניא (קא, א) רבי יוסי בר יהודה אומר נעץ קנה ברשות הרבים ובראשו טרסקל וזרק ונח על גביו חייב, ואוקימנא בריש מכלתין (ה, א) למעלה מעשרה. ועוד תניא (ק, א) עמוד ברשות הרבים גבוה עשרה ורחב ארבעה ואין בעיקרו ארבעה ויש בקצר שלו שלשה וזרק ונח על גביו חייב, וכל הני ממקום פטור קא אתיין. ומיהו הא דעמוד וההיא דר׳ יוסי נראה דלא קשיא ליה, דדחי להו כדדחי מתניתין דחולית הבור והסלע, ומוקי לה במחט או אית להו מורשא, וההיא דרבא נמי לא תקשי ליה, דמאמורא לא פשט ליה. אלא ברייתא דבן עזאי תקשי, וכן משא בני קהת אמאי לא פשטו לה מינייהו.}
\textblock{\textbf{והרמב״ן ז״ל תירץ דכל היכי דעקר ממקום חיוב ונח במקום פטור ובאויר של מקום חיוב ממש לא מיבעיא ליה דחייב, כי קא מיבעיא ליה בשנח בעמוד העומד ברשות הרבים שאוירו הוא אויר של רשות הרבים והוי מקום פטור,      } דכי אמרינן דרשות היחיד עולה עד לרקיע, הני מילי רשות היחיד גמורה במחיצות כגון חצר אבל עמוד לא, וכן דעת רבינו האי גאון ז״ל בזה. וכיון שכן כשזרק על גבי עמוד דממקום פטור קא אתיא, ובמקום פטור הוא עומד, כלומר: שכולו עומד באויר מקום פטור אלא שהוא נח על מקום חיוב, פטור, אבל נעץ קנה ובראשו טרסקל שיש לו מחיצות הרי כולו נח במקום חיוב, ועמוד גבוה עשרה ורוחב ארבעה ואין בעיקרו ארבעה, דלמא בשיש לו מחיצות, וההיא דבן עזאי במקום חיוב ממש נח, וכן מעביר דרך עליו ומוציא משוי למעלה מעשרה הנחתה במקום חיוב הוא, הלכך חייב.\par \textbf{} ולא מחוור בעיני, דהא מוציא משוי למעלה מעשרה אם עמד לפוש מי לא מחייב, וכן מעביר דרך עליו ועמד בסוף ארבעה מי לא מחייב. ובתוס׳ אמרו דשאני מעביר ומוציא שכן היו במשכן, ואינו מחוור בעיני ועדיין צריכה תלמוד.\par \textbf{} ומכל מקום אני תמה כל שהוא מניח על גבי הסלע או על גבי החוליא למה יתחייב, והא אגד כלי במקום פטור כיון דאויר העמוד מקום פטור, וכדאמרינן לעיל בריש פרק קמא (שבת ח, א) גבי כוורת. ויש לומר דלא אמרו אלא בשהכלי עצמו אוכל כל גובה הרשות ועוד אגדו במקום פטור כגון שנעץ קנה גבוה יותר מעשרה ברשות הרבים, אבל כשאין גופו של כלי גבוה כל כך שיהא אוכל כל גובה הרשות לא, והיינו נמי דכי אמרינן לעיל בפרק המצניע (לעיל צב, א) המוציא משוי למעלה מעשרה טפחים חייב שכן משא בני קהת, אמרינן והתם מנא ליה, [ו]איצטרכינן למילף ממזבח או מארון דמדלי קרקעיתו מן הרשות יותר מעשרה, ואמאי הא לא הוי ליה למימר אלא שהיה גבהו של מזבח או של ארון הנישא על הכתף, למעלה מעשרה טפחים חייב.}
\textblock{\textbf{כותל ברשות הרבים גבוה עשרה ואינו רחב ארבע ומוקף לכרמלית ועשאו רשות היחיד.} מסתברא דהא דאיצטריך למימר כל הני, משום דמעיקרא היה מקום פטור, ועכשיו כשגדר הכרמלית ונעשית ממילא על ידי מחיצה זו רשות היחיד, הלכך מיבעיא ליה כיון דלא נתחדש שום דבר בגופו של כותל זה ומעיקרא היה מקום פטור, אי אזלינן ביה בתר מעיקרא, או דלמא כיון דמכל מקום על ידה נעשה רשות היחיד אף היא מותרת כרשות היחיד, אבל אם מתחלה גדר כרמלית בארבע מחיצות, פשיטא ליה דאע״פ שאין המחיצות רחבות ד׳ חייב, כמחיצות רה״י ברה״י כשהוא מניח בראש המחיצה דהוי ליה כחורי רשות היחיד שהן כרשות היחיד.}
\textblock{\textbf{הכא מבטל ליה התם לא מבטל לה.} הקשו בתוספות: וכי לא מבטל לה מאי הוי הא ליכא ד׳ אמות, דאטו אם היה בידו חפץ של ד׳ אמות והניחו מי חייב, והלא כל שאינו כולו חוץ לד׳ אמות פטור דהוי ליה אגדו מבפנים. ועוד מנא ליה למקשה דאיירי בד׳ אמות מצומצמות, והלא כל מקום ששנינו זורק או מוציא, בשזרק החפץ כולו מיירי או הוציאו כולו חוץ לד׳ אמות הוא, ומאי שנא הא. ותירצו הם דהכא נמי בשזרק חוץ לארבע אמות כדין כל שיעורי שבת, והכי קא פריך אי אמרת דהנחה וסילוק מחיצה בהדי הדדי אתו, נמצא דבילה זו ממעטת רוחב הרשות לגבי דבילה אחרת או חפץ אחר לפי שאי אפשר לזורק אחר להתחייב, דהוי כאילו הוסיף במקום הנחת הדבילה בעובי המחיצה ואינו רשות היחיד להתחייב עליו, א״כ ג״ז לא יתחייב עליו. ואהדר ליה דכיון דלא מבטל לה אינה ממעטת ולא מבטלא מחיצתה בהנחתה.}
\textblock{\textbf{זרק דף ונח על גבי יתדות מהו.} וסבירא ליה לרבא דאמרינן פי תקרה יורד וסותם דאי לא לא הוי רה״י וכדאמרי׳ נמי לקמן (שבת קא, א) גבי נעץ קנה ברה״ר ובראשו טרסקל. ור״ת ז״ל פירש בערובין (כה, א בתוס׳ ד״ה אכסדרא. וצד, ב) דלא אמרינן פי תקרה יורד וסותם בד׳ מחיצות, ולדבריו איכא לאוקומה בדאית לה שתי מחיצות. ועדיין צריך לי תלמוד שהרי מחיצה שהגדיין בוקעין בה הוא, ואינו מחיצה.}
\clearpage
\newsection{דף ק}
\textblock{\textbf{חלקה במחצלת פטור לאביי דפשיטא ליה דמחצלת מבטלא מחיצתא כל שכן חוליא דמבטלא מחיצתא.} מהאי לישנא משמע לי דהא דאביי אע״ג דלא מבטל לה למחצלת, דאי כשבטלה מאי כל שכן דקאמר היא [היא], ולרבי יוחנן דבעי בחוליא, מחצלת אמאי פשיטא ליה. וקשיא לי דאם כן אפילו זרק לתוכה מחצלת לאביי ליפטר, דומיא דזרק חוליא. ועוד קשיא לי דאם כן מתניתין דהזורק באויר למטה מעשרה טפחים כזורק בארץ דאקשינן עלה והא לא נח ואיצטריך ר״י לאוקומה בדבילה שמינה, לאביי היכי מוקי לה, דאי בדבילה שמינה אמאי חייב, דהא הנחה וסילוק מחיצה בהדי הדדי קא אתו ואע״ג דלא מבטל לה התם, וכי תימא דמוקי לה בחור, הא אביי גופיה [הוא] דקא דחי ליה לרבא בפרק קמא דמכלתין (ז, ב) דאי אפשר לאוקומא בחור משום דצרור וחפץ מיהדר וקאתי, ואי נמי משום דמתניתין אי אפשר דמיירי בדאית בה חור, מדקתני רישא למעלה מעשרה טפחים כזורק באויר וכדאיתא התם. ומתוך הדוחק יש לי לומר דחלקה במחצלת כשמבטלה שהרי מתכוין הוא לעשות מחיצה, אבל זרק מחצלת שאינו אלא להצניעה שם חייב, דמה שהוא מניח ברשות אינו מבטל הרשות, ומתניתין דהזורק למטה מעשרה טפחים, נמי בדבילה שמינה כדרבי יוחנן משום דלא מבטל לה. והא דאמרינן כל שכן חוליא, ורבי יוחנן דמבעיא ליה חוליא ומחצלת פשיטא ליה. היינו משום דאין דרכן של בריות לבטל מחצלת בבור לעשותה מחיצה גמורה קיימת, ולמחר הוא עשוי לסלקה. וא״ת אי הכי ליפלוג ולימא אפילו בזורק מחצלת, ולימא זרק מחצלת חייב ואם בטלה פטור. יש לומר דאורחא דמלתא נקט, דאדם עשוי לעשות מחיצה במחצלת, ואינו עשוי למעט את העומק במחצלת.}
\textblock{\textbf{מלאה פירות.} פירוש: מאמש וזרק על גביהן היום, אבל לא קאמר דאם זרק בה פירות שיבטלו מחיצתה. וא״ת פירות דמאמש נמי למה מבטלין מחיצתה, והא דעתו לפנותן ואינן שם אלא להצניען. פירש רבנו האי גאון ז״ל בשאין דעתן לפנותן אלא שבטלן שם, ולדבריו אף ע״ג דאין אדם עשוי לבטל פירות בבור ובטלה דעתו אצל כל אדם, לגבי שבת הוי מיעוט. ובתוספות פירשו בפירות של טבל שאינו יכול לסלקן משם כל אותה שבת, ובטלין הם ליומן שם. ויש מפרשים דאע״ג דלא בטלן כלל משום דפירות מבטלין מחיצות, דמחיצות נכרות בעינן בעומק ורוחב בשעת זריקה והכא ליכא, אבל מים לא מבטלי היכר מחיצות.}
\textblock{\textbf{רבי שמעון אומר אם יש במקום שזרק עומק עשרה ורחב ארבעה חייב.} ואיכא למידק אדמסייע ליה מדר״ש לותביה מדרבנן דפטרי, דאלמא מיא מבטלי מחיצתא. ויש מי שפירש דת״ק ור״ש לא נחלקו בהא דלכולי עלמא לא מבטלי מחיצתא, אלא בעיקרא דמלתא פליגי, דת״ק סבר דבור עמוק עשרה ורחב ארבעה בכרמלית הרי הוא ככרמלית, ור״ש סבר רשות היחיד הוי. ונראה לי טעמא משום דאי עיקר פלוגתייהו במיא אי מבטלי מחיצתא או לא, ליפלוג בבור מלא מים דמתברר פלוגתייהו ביה טפי. וזהו דעת הרמב״ם ז״ל שכן כתב (בפי״ד מהלכות שבת ה״ו), דבור עמוק עשרה ורחב ארבעה בכרמלית דינו ככרמלית. אבל הראב״ד ז״ל תפש עליו ואמר, דלכולי עלמא בור שבכרמלית רשות היחיד הוא, אלא הכא שאני שהמים צפין עליו והרי הכל כמקום אחד, וזהו הטעם שנחלק ת״ק על ר׳ שמעון, אבל בור שביבשה רשות היחיד הוא לכולי עלמא. ואומר אני שהרב ר׳ משה ז״ל סמך על התוספתא דמכלתין דתניא בפרק י״א (ה״ב) איסרטא על גבי בקעה זרק מן האיסרטא לבקעה פטור, ר״ש אומר אם יש במקום שזרק עומק עשרה טפחים הרי זה חייב. אלא שיש לדקדק עליו מהא דתניא וקא מייתי לה בריש פרק קמא דמכלתין (ח, ב) גבי תשמיש על ידי הדחק, נתכוון לשבות בר״ה והניח עירובו בבור למעלה מעשרה טפחים עירובו עירוב, למטה מעשרה טפחים אין עירובו עירוב, ואתיא לפרושי הא דקתני למעלה מעשרה טפחים בבור דלית ביה עשרה טפחים, אלמא תשמיש על ידי הדחק שמיה תשמיש, ומשני ליה הוא ועירובו בכרמלית, אבל למטה מעשרה טפחים כלומר: בעומק עשרה טפחים אין עירובו עירוב דהוא במקום אחד ועירובו במקום אחר, אלמא בור עשרה ורוחב ד׳ בכרמלית רשות היחיד הוא. כדברי הראב״ד ז״ל.\par \textbf{} ויש לי לתרץ לדעת הרמב״ם ז״ל דהא מני ר״ש היא, דנלמוד סתום מן המפורש בתוספתא. אלא דקשה לי קצת דאם כן הוי ליה למימר התם והא מני ר״ש היא כיון דלרבנן לא הוי דינא הכי, וכדקאמרינן בפירוקא אחרינא דפריק התם דקאמרינן וזימנין משני ליה הוא ברה״ר ועירובו בכרמלית ורבי הוא דאמר כל שהוא משום שבות לא גזרו עליו בין השמשות. ועוד יש לדקדק על דברי הרמב״ם ז״ל מהא דתניא בעירובין פרק הדר (עירובין סז, ב) סלע שבים גבוה עשרה ורחב ארבע אין מטלטלין לא מתוכו לים ולא מן הים לתוכו, דאלמא אע״פ שהים ככרמלית מקיף את הסלע והיא בתוכו אפילו הכי אין לסלע דין כרמלית אלא הרי היא רשות היחיד גמורה. וזו גם כן אחת מן התשובות שהשיב עליו הראב״ד ז״ל בהשגות.\par \textbf{} ואף בזו יש לי ללמוד עליו זכות, דהתם הוא מפני שמוחלק מן הים לגמרי וניכר הוא שאינו ממנו שזה ים וזה סלע, ואע״פ שהים מקיף אותו מכל צדדיו, אינו נעשה בכך ככרמלית, שאם כן אף כל העולם יהא כרמלית, שהרי מקיף אותו אוקיאנוס, וכאותה שאמרו בפרק עושין פסין (עירובין כב, ב) אי הכי כולי עלמא לא לחייבו דהא מקיף ליה אוקיאנוס, והוא הדין והוא הטעם לספינה שבים בזמן שגבוהה עשרה שאסור לטלטל מתוכה לים כדאמרינן לקמן (בעמוד ב).}
\textblock{\textbf{אלא שלפי טעם זה היינו צריכין לומר, דתל גבוה עשרה ורחב ארבעה בכרמלית שביבשה דינו ככרמלית, שאף הוא      } אינו ניכר שיהא מובדל ומוחלק מן הכרמלית. וההיא דאמרינן תל גבוה עשרה ורחב ארבעה מטלטלין בתוכו עד בית סאתים, בתל שברשות הרבים מיתניא. ואי נמי יש לומר דבתל כולי עלמא לא פליגי ואפילו רבנן מודו ביה שאפילו בכרמלית הוי רשות היחיד, שכל שהוא גבוה ניכר שהוא מקום בפני עצמו ומוחלק מן הכרמלית לגמרי, אבל בור אינו ניכר כל כך בתוך הכרמלית ומכלל הכרמלית הוא נחשב שאינו נחלק ממנו, והיינו דלא חלק ר״ש בתוספתא אלא בעומק משום דבגובהה אפילו רבנן לא פליגי עליה, והוא הדין והוא הטעם לספינה שבים. אח״כ באה לידי תשובת הרב ז״ל (הרמב״ם) בעצמו שהשיב לחכמי לוניל שהקשו לו על זה. וכן השיב להם: יראה לי מדבריכם שיש בספר שלכם חסרון דברים, וזהו נוסח הספר בור שבכרמלית הרי הוא ככרמלית, אפילו עמוק מאה אמה אם אין בו ארבעה. עכ״ל הרב ז״ל. וגם זה ענין מגומגם בעיני. והלא מקום פטור אחד מן הד׳ רשויות הוא, ולמה לא יחלק גם כן רשות לעצמו ויהיה בור זה מקום פטור, וכי יהיה דינו חמור כשהוא בכרמלית מבור שהוא בר״ה שאם אין בו ד׳ על ד׳ דהוי מקום פטור.}
\textblock{ הא ד\textbf{אמר רבי יהודה אמר רב אמר רב חייא זרק למעלה מעשרה והלכה ונחה בחור כל שהוא.} פירש רבינו האי גאון ז״ל דדוקא בחור גבוה עשרה מצומצמין, אבל למעלה מעשרה לא נחלקו ר״מ ורבנן בהא, כיון דממקום פטור קא אתי ליה. ולא ירדתי לסוף דעת הרב ז״ל, דאדרבה למעלה מעשרה למעלה ממנו כמה משמע, (ואם כן) [דאי לא כן] הוי ליה למימר זרק בחור גבוה עשרה.}
\textblock{ הא דקתני:\textbf{ מבוי ששוה לתוכו ונעשה מדרון לר״ה וכו׳.} במתלקט עשרה מתוך ארבע היא, וכההיא דמסיים ר״ח ב״ג המתלקט עשרה מתוך ארבע וזרק ונח על גביו חייב, ובהדיא קתני בה בתוספתא (פי״א, ה״ד) מבוי ששוה לרשות היחיד ועשוי מדרון לרשות הרבים, אם יש גובה עשרה טפחים בתוך ד׳ אמות אין צריך לחי וקורה וכו׳ שוה לרשות הרבים ועשוי מדרון לרשות היחיד, אם יש גבוה עשרה טפחים בתוך ד׳ אמות אין צריך לחי וקורה וכו׳, תל ברשות הרבים ר׳ חנינא ב״ג אומר אם גובהו עשרה טפחים בתוך ד׳ אמות נטל הימנו ונתן על גביו חייב. וזו היא הברייתא עצמה שהביא כאן בגמרא, אלא שדרך בעלי הגמרא לקצר בלשון הברייתות בהרבה מקומות ולשנות בלשון, ובלבד שתהא הברייתא קיימת.}
\textblock{ (הא דקתני) [מתני׳:]\textbf{ זרק בתוך ארבע ונתגלגל חוץ לארבע.} מסתברא דזורק לתומו על דעת שתנוח באיזה מקום שתרצה, שאם בנתכוון לזרוק בתוך ארבעה, הוי ליה למיתני נתכוון לזרוק שתים וזרק ארבע. ועוד דלא הוה לה דומיא דסיפא, דכל עצמה לא באה אלא לבאר דין נתגלגל. וכיון שכן, רישא דוקא בשנח משהו בתוך ארבע וחזר ונתגלגל, הא לאו הכי אמאי פטור הא נחה מכחו חוץ לארבע. וכן פירשו בתוספות דדוקא בשנח תוך ארבע. ולפירוש זה הוא הדין דהוה מצי לאקשויי רישא אמאי והא לא נח ולוקמה בשנח ע״ג משהו, אלא דניחא ליה לאהדורי טפי בתר פטורא. כך נראה לי.}
\textblock{ הא ד\textbf{אמר ר׳ יוחנן והוא שנח על גבי משהו.} נראה לי פירושו והוא שנח משהו ועל גבי משהו, ולומר דאפילו העמידו הרוח, וכדתניא אחזתו הרוח משהו אע״פ שחזרה והכניסתו חייב.}
\textblock{\textbf{אמר ליה רבינא למרימר לאו היינו מתניתין וא״ר יוחנן והוא שנח על גבי משהו.} פירוש: ומאי איצטריך רבא לאשמועינן, ותמיהא לי ומנא ידע רבינא דשמעה רבא להא דרבי יוחנן. ויש לומר דהכי קאמר ולאו היינו מתניתין דפירש רבי יוחנן עלה והוא שנח על גבי משהו, אם כן רבא אמאי עבדא שמועה באפי נפשה, יפרש כך אמתניתין כדפירש בה רבי יוחנן, ואהדר ליה דלא שמעינן ממתניתין מאי דשמעינן מהא, דאילו פירש כן אמתניתין הוי אמינא דוקא במתגלגל שאין סופו לנוח, כלומר: שאין ראוי לנוח שם ואפילו נפלה שם, והלכך לא עדיף תוך שלשה דידה מממשו, אבל מקום שסופו לנוח בו, כלומר: שראוי לנוח בו אילו נפלה בו הוי אמינא דאף תוך שלשה שלו כממשו, קא משמע לן.\par \textbf{} ובהא נמי מתרצא לי קושיא אחריתי, דקשיא לי היכי פשיטא ליה כולי האי לרבינא דבעינן הנחה על גבי משהו דקשיא ליה נמי אמאי איצטריך רבא לאשמועינן, דהא איכא רב חלקיה בר טובי דאמר לעיל (שבת צז, א) תוך שלשה כולי עלמא לא פליגי דחייב, וברייתא מסייעא ליה דתניא בהדיא תוך שלשה דברי הכל חייב. ועוד דפשטא דמתניתין בעירובין פרק המוצא תפילין (עירובין צז, ב) רבנן דר״י דלא כרבא, דתנן היה קורא בראש הגג ונתגלגל הספר מידו למעלה מעשרה טפחים גוללו אצלו וכו׳, ר״י אומר אפילו אינו מסולק מן הארץ אלא כמלא החוט גוללו אצלו, ואמרינן עלה בגמ׳ מאי טעמא דר״י דבעינן הנחה על גבי משהו [וכו׳] לימא כתנאי אמרה לשמעתיה. והוצרכנו לדחוק ולומר דכולה ר״י היא, ואביי התם לית ליה הכין, ורב חלקיה נמי אפשר דלית ליה הכין, אם כן מלתא דצריכא לאשמועינן היא. אלא דבמה שפירשתי ניחא דלא קשיא ליה אלא אמאי לא פירש הכי אמתני׳ וכדר׳ יוחנן, ועבדה שמועה באפי נפשה.}
\textblock{\textbf{ומכל מקום איכא למידק לרבא דהא קתני בברייתא דלעיל (שבת צז, א) בהדיא תוך שלשה ד״ה חייב. ויש לומר דרבא סבר דההיא ברייתא משבשתא היא, דהא קתני באידך ברייתא הכא בזרק חוץ לארבע אמות ודחפתו הרוח והכניסתו פטור, אחזתו הרוח משהו אע״פ שחזר והכניסתו חייב, ומתניתין       } דהכא נמי דייקא לה הכין, דעל כרחין רישא דקתני תוך ארבע ונתגלגל חוץ לארבע בשנח תוך ארבע היא כמו שכתבנו, דאפילו עברה תוך שלשה ליכא מאן דאמר שיהא כמונחת, דבזורק ארבע אמות ברשות הרבים ואפילו תוך ג׳ ליכא מאן דפטר, דלפוטרו ברשות אחת ליכא מ״ד, אם כן סיפא נמי דקתני חוץ לארבע ונתגלגל לתוך ארבע חייב דוקא בשנחה, הלכך משבש לה רבא לההיא ברייתא דרב חלקיה מקמי דיוקא דהא מתניתין ומקמי הא ברייתא דאחזתו הרוח.\par \textbf{} ואם תאמר מאי קא מקשה ליה התם בערובין (צח, ב) לימא כתנאי אמרה לשמעתיה ומאי קא מתרץ להו כולה רבי יהודה, דהא על כרחין רבא כתנאי אמרה, דהא תנא דרב חלקיה לא בעי הנחה על גבי משהו. יש לומר דהכי קאמר לימא כתנאי דמתניתין אמרה, ומאי חזא דשבק רבנן ואמר כרבי יהודה, ורב חלקיה דלמא לא תקשי ליה מתניתין, דאיכא למימר דלצדדין קתני, ואי נמי שאני התם דמתגלגל שאין סופו לנוח כדדחי ליה נמי אמימר לרבינא, ומשבש לה להאי ברייתא דאחזתו הרוח מקמי ההיא ברייתא דקתני תוך שלשה דברי הכל חייב, ומשום דפשטא דמתניתין דעירובין רבנן פליגי עלה דר׳ יהודה ולא בעו הנחה על גבי משהו, ואע״ג דהתם אפילו קלוטה תוך עשרה אית להו לפי פשטא דמתניתין, מכל מקום שמעינן מינה דלרבנן הנחה לא בעי וכדאקשינן מיניה לרבא ואמרינן לימא רבא כתנאי אמרה לשמעתיה. ומסתברא דהלכתא כרבא, מדפשיטא ליה לרבינא דמתניתין דייקא כוותיה, ורבי יוחנן נמי הכי משמע ליה ופשטא דמלתא נמי משמע דמרימר אית ליה הכין. ור״ח ז״ל פסק כרב חלקיה.}
\textblock{\textbf{ובדין הוא דזיז נמי לא ליבעי אלא כי היכי דלהוי ליה היכרא.} תמיהא לי כיון דאית ליה לרב הונא דמארעא משחינן מי דחקו להצריך אפילו זיז, דהא מקום פטור הוא ומקום פטור אינו צריך דבר אחר להתירו אלא משתמש לכתחילה עם רה״י ועם רה״ר. ומסתברא דמתניתין קשיתיה, דקתני הזורק מן הים לספינה ומן הספינה לים פטור, כלומר: אבל אסור. ופירש הוא (ז״ל) דלאו משום זורק מכרמלית לספינה אלא משום דלא ליתי למיחלף בכרמלית בעלמא, והלכך היכרא בעי ובהיכרא סגי ליה. ואם תאמר אם כן לכולי עלמא לישתרי ואפילו לכשתמצא לומר דממיא משחינן, דהא מכניס מכרמלית למקום פטור וממקום פטור לספינה. יש לומר דסבירא ליה לרב חסדא ולרבה בר רב הונא כמאן דאמר בעירובין (פז, ב) דאפילו ברשויות דרבנן אמרינן ובלבד שלא יחליפו, ולהדין סברא אין חילוק בין שיהא הזיז למטה מעשרה בין שיהא למעלה מעשרה שהוא מקום פטור לכולי עלמא, אפילו הכי אסור כדאמרן דאפילו ברשויות דרבנן אמרינן ובלבד שלא יחליפו.}
\textblock{\textbf{עושה מקום ארבע וממלא.} ואיכא למידק האי מקום ארבע היכי דמי, אי למעלה מעשרה היינו רשות היחיד, ואי למטה מעשרה היינו כרמלית, ולעולם מטלטל הוא מכרמלית לרשות היחיד. פירש רבנו האי גאון ז״ל בשם הגאונים ז״ל: שאותו מקום כגון תיבה פחותה או סל פחותה, ונמצא סיוע בתלמוד ירושלמי (בפרקין ה״ה) דגרסינן התם: אמר רב המנונא נסר שהוא נתון לספינה ואין בו רוחב ארבע, מותר לישב בו ולעשות צרכיו בשבת, אמר ר׳ מונא אילו אמר תיבה פחותה יאות, אמר ר׳ בון מאן דבעי למעבד תקנה לאילפא מוציא נסר חוץ לשלשה שאין בו רוחב ארבע ואת רואה את המחיצות כאילו עולות דאמר ר׳ יעקב בר אחא בשם רב המנונא כל שלשה ושלשה שהן סמוכין למחיצה כמחיצה הן, ר׳ יצחק בר׳ אלעזר מפקד ליה לר׳ יהושע בר שמיי דהוה פריש, מעבדא ליה סל פחות וכו׳. ע״כ בירושלמי. דאלמא משמע שהוא צריך לפחות אותו ולעשות אותו כמין תיבה פחותה או סל פחותה.}
\textblock{\textbf{ואתיא כאותה דתניא בעירובין בפרק כיצד משתתפין (שם) רבי חנניא בן עקביא אמר גזוזטרא שיש בה ארבע אמות על ארבע אמות, חוקק בה ארבעה על ארבעה וממלא, כלומר דאמרינן כוף וגוד, ואע״ג דתנא דמתניתין דהתם לית ליה כוף וגוד ולא שרו עד דאיכא מחיצה ממש (על) [של] עשרה טפחים, וכדתנן התם (שם) גזוזטרא שהיא למעלה מן המים אין ממלאין ממנה אלא אם כן עשה לה מחיצה גבוהה עשרה טפחים בין מלמעלה בין מלמטה, וקיימא לן כתנא דמתניתין. שאני התם דהוא בחצר ויכול לעשות מחיצה ממש לא התירו לו אלא במחיצה גמורה של עשרה, אבל בספינה איכא למימר מודו ליה לרבי חנינא בן עקיבא, ועוד קולא      } אחרת התירו בספינה שלא הצריכו דף ארבע אמות על ארבע אמות ולחקוק בו ארבעה על ארבעה, אלא אפילו ארבע על ארבע ובחוקק בה משהו סגי ליה ודולה דרך החקק. אבל ר״ת ז״ל פירש דעושה מקום ארבע דקאמר הכא, היינו מקום חקק של ארבע קאמר, ולעולם בדף של ארבע אמות כדי שיהא עשרה טפחים לכל רוח משפת החקק ועד שפת הדף שנוכל לומר בו כוף כאותה ממש של רבי חנינא בן עקיבא. ואין [הלשון] שבכאן נראה כן.}
\textblock{\textbf{גמירי דאין ספינה מהלכת בפחות מעשרה טפחים.} כלומר: שיהא למטה ממקום שקיעתה עשרה טפחים, דאי משפת המים אכתי היאך ידלה וישקיע דלי במים, נמצא דולה מכרמלית ממש, דמארעא ועד מקום דלי ליכא עשרה.}
\textblock{ הא ד\textbf{אמר רב הונא כרמלית מארעא משחינן.} לא פליגא אדאביי דאמר לעיל (בע״א) מיא לא מבטלי מחיצתה, דהתם הוא ברה״י שעולה עד לרקיע וכיון דמיא לא מבטלי מחיצתה נמצא זה משתמש ברה״י, אבל בכרמלית שאוירו הוא מקום פטור כי שקיל למעלה מעשרה טפחים אי מארעא משחינן על כרחי׳ פטור, ולא משום דמיא מבטלי מחיצתא, אלא משום דסבירא ליה דמיא נמי רשות הן לעצמן כל שהן בכרמלית אבל ברשות היחיד לא.}
\textblock{\textbf{דלמא מורשא אית ליה.} פירש ר״ח ז״ל: סמוך לספינה במקום שהוא דולה.}
\clearpage
\newsection{דף קא}
\textblock{\textbf{אלא לאו אחודה ושמע מינה כחו בכרמלית לא גזרו ביה רבנן.} וא״ת אדמסייע ליה מדר׳ יהודה, לותביה מדרבנן דאמרי לא מתוכה לים ולא מן הים לתוכה. תירצו בתוספות דרבנן ורבי יהודה לא פליגי אלא בגבוהה עשרה ועמוקה עשרה, ובהא פליגי דרבנן סברי דלא מטלטל להדיא מתוכה לכרמלית, ואע״פ שמטלטל מרשות היחיד לכרמלית דרך מקום פטור שהאויר הוי מקום פטור כשמעלה ידו על דפני הספינה לשפוך לים, דסבירא להו לרבנן דאין משתמשין להדיא מרשות היחיד לכרמלית דרך מקום פטור, ואפילו למ״ד התם בערובין (פז, ב) דלא אמרינן ברשויות דרבנן ובלבד שלא יחליפו, הני מילי במניח [במקום] פטור אבל בלא הנחה לא. ור׳ יהודה סבר דבגבוהה עשרה ודאי משתמש, דטלטול גמור דרך מקום פטור הוא משתמש, אלא אפילו בשאינה גבוהה עשרה יש לעשות תקנה כגון ששופך דרך חודה. והביאו ראיה לפי׳ זה מן התוספתא (פי״א, ה״ח) דמתפרשת שם מחלוקתן בגבוהה עשרה, דתניא התם: ספינה שבים גבוהה עשרה טפחים, אין מטלטלין לא מתוכה לים ולא מן הים לתוכה, ר״י אומר עמוקה עשרה ואינה גבוהה עשרה וכו׳. אלמא פלוגתייהו דוקא בגבוהה עשרה היא. ולפי פירוש זה הא דבעי רב הונא זיז כל שהוא ורבה ורב חסדא מקום ארבעה לרבנן בלחוד הוא, אבל לר״י אינו צריך שהרי יכול להשתמש להדיא דרך מקום פטור.}
\textblock{\textbf{אלמא כחו בכרמלית לא גזרו בהו רבנן.} הקשו בתוס׳ דבפרק כיצד (שם פח, א) גבי גזוזטרא שהיא למעלה מן הים אמרינן, אמר רבה בר רב הונא לא שנו אלא למלאות אבל לשפוך אסור, וקא יהיב טעמא התם משום שהמים נתזין מכחו חוץ לד׳ אמות, ואמאי והא כחו בכרמלית הוא. ותירצו דהתם דבחצר בדין הוא דליגזור, דאי שרית ליה אתי לאתויי לצד אחר של ביתו דהוא רשות הרבים גמורה, אבל הכא בספינה ומכל צדדיה ליכא אלא כרמלית לא.}
\textblock{\textbf{הני ביציאתא דמישן.} לפי מה שפירש רש״י ז״ל שהן קצרות למטה שאין להם ארבעה, הקשו בתוס׳ אם כן כיון שאינה רשות היחיד אפילו כרמלית לא הוי, דאין כרמלית פחות מארבעה כדאמרינן בריש פרק קמא דמכלתין (ז, א) דהקילו בו מקולי רשות היחיד, אם כן מקום פטור הוא ומשתמש בכולה.\par \textbf{} ועוד דמאי קאמר אי מלינהו קני ואורבני לית לן בה, דכיון שיכול למלאותה אפילו לא מילא, כדאמרינן בעירובין פרק חלון (עירובין עח, א) גבי עמוד ברשות הרבים גבוה עשרה ורוחב ד׳, דבעא מיניה רב אחא בריה דרבא לרב אשי מלאו כולו ביתדות מהו, אמר ליה לא שמיע לך דאמר ר׳ יוחנן בור וחולייתה מצטרפין לעשרה, אמאי והא לא משתמש ליה, אלא מאי אית לך למימר דמנח מידי ומשתמש ליה הכא נמי דמנח מידי ומשתמש, דאלמא כל שיכול להניח ולהשתמש בו ברוחב ארבעה, אפילו קודם הנחה חשוב כרשות היחיד. וזו נראה בעיני שאינו קושיא, דהתם הוא דרחב ארבעה אלא שהיתדות התחובות בו אינן רחבין ארבעה, וכיון שהעיקר שהן תחובין בו ישנו רחב ארבעה אין היתדות מבטלות הרשות הואיל ואף בהן יכול להניח מידי ומשתמש, מה שאין כן בבא לעשות רשות לכתחילה שאילו מילא רשות הרבים יתדות ברחב ארבעה לא יעשה רשות היחיד בכך משום דאי בעי מנח עליה מידי, ובור וחולייתה כיון שיש בהן ארבעה ויכול להשתמש אי מנח ביה מידי שפיר דמי, אבל כאן שהיא קצרה מלמטה מארבעה אינה רשות היחיד אף על פי שיכול למלאותה עד שימלאנה. אבל מה שהקשו עליו בראשונה קשה.\par     \textbf{} ופירשו הם ז״ל כמו שפירשו רבינו האי גאון ז״ל ורבנו חננאל ז״ל, דפירשו ביציאתא דמישן ספינות קטנות עשויות לשמש את הגדולות, ומתוך שהם קלות ומהלכות על בצעי המים קרי להו ביציאתא, מישן שם מקום, ומפני שהוא מקום אגמי מים עושין שם ביצאתא להלך באגמים, ויש להן דפנות ומחיצות וקרקעיתן עשויות נסרים נסרים וביניהם חלל והים והנהר נכנסים לתוכה באגמים והיושבין בתוכה במים יושבין ואפילו תתהפך במים אינה נטבעת, ואמר ר״ה כיון שהיא פרוצה לים שהוא כרמלית דינה ככרמלית, ולא אמרן אלא שאין בה בפחות משלשה נסר רחב ארבעה אבל אם יש בפחות משלשה נסר רחב ארבעה, רשות היחיד הוא ומטלטלין בכולה, ואם אין בפחות משלשה נסר ארבעה ומילא החלל שבין נסר לנסר קני ואורבני מותר לטלטל בתוכה, ופריך רב נחמן אמאי אמרת אם אין בפחות משלשה רחב ארבעה אין מטלטלין אלא בד׳ אמות, נימא גוד אחית מחיצתא שרואין דפני הספינה כאילו הן מחוברות עם הנסר שהוא קרקעה. זהו תורף פירוש רבינו האי גאון ז״ל. ויש לעיין אם כפירושו למה צריך נסר רחב ארבעה, דכיון דאין בה בין נסר לנסר שלשה הרי זה כלבוד.}
\textblock{\textbf{ויש בקצר שלו שלשה.} פירש רש״י ז״ל: גובה שלשה, ואפילו שלשה קאמר וכל שכן פחות מכן דהוי ליה לבוד. אבל בתוס׳ אמרו דלא יתכן, דאם כן הוי ליה כנעץ קנה ברשות הרבים דפליגי רבנן עליה דר׳ יוסי, משום דהוי ליה מחיצה שהגדיים בוקעין בה. והם ז״ל פירשו דבעובי קאמר, כלומר: אם בצד הקצר שלו יש שלשה טפחים בעובי.}
\textblock{\textbf{לא נצרכא אלא לערב ולטלטל מזו לזו.} וכתב הראב״ד ז״ל דקא משמע לן דאע״פ שהן גבוהות עשרה טפחים ואין ביניהם לא פתח ולא סולם, עירוב מועיל בהם ומטלטלין מזו לזו, מה שאין כן בשני חצרות שאין מעורבות זו בזו אלא או דרך פתח (או דרך סולם) שיש בו ארבעה על ארבעה טפחים בתוך עשרה, או דרך סולם גבוה עשרה ורחב ארבעה, וספינות אלו אע״פ שאין בהן אחד מכל אלו מותר לערב ולטלטל מזו לזו. וטעמא דמלתא מפני שהן כבתים ולא כחצרות, ואמרינן בעירובין (עו, ב) ביתא כמאן דמליא דמי. ובתוספות אמרו דהא קא משמע לן, דסלקא דעתך אמינא דלא תקנו ערוב אלא בדבר הקבוע תדיר כגון בתים וחצרות, קא משמע לן.}
\textblock{\textbf{כי אתמר דרב נחמן אמזיד אתמר.} פירש רש״י ז״ל: על נגללו במזיד דקתני בהא ברייתא אתמר, וכי קתני חזרו להיתרן הראשון אשארא. והקשו עליו בתוס׳ דאם כן למה שנה בברייתא מזידין. על כן פירש ר״ת ז״ל דבעלמא אתמר כלומר: במחיצה שנעשית לכתחילה במזיד, אבל זו אינה נעשית לכתחילה שאינה באה אלא להתיר ערובו כבתחילה וכל שעה היא רשות היחיד הלכך אף מזיד מועיל דלא קנסינן בכי הא כיון שאין זה אלא מעמיד דבר על חזקתו הראשונה.}
\textblock{\textbf{קשרה בדבר המעמידה מביא לה טומאה, בדבר שאין מעמידה אין מביא לה טומאה, ואמר שמואל והוא שקשרה בשלשלת של ברזל לענין טומאה בחלל חרב אמר רחמנא (במדבר יט, טז) חרב הרי הוא כחלל.} כך היא גירסת הספרים וכן גריס רש״י. ופירש הוא ז״ל: בדבר המעמידה דבר שדרכו להעמידה בו דהיינו שלשלת של ברזל, ודבר שאין מעמידה שאין דרכו להעמידה בהן כגון חבלים שאינן של ברזל, ולומר שאם קשרה בשלשלת של ברזל וראשו האחר באהל המת, הרי השלשלת מטמאת את הספינה דהרי הוא אבי אבות הטומאה כחלל עצמו, וספינה זו ספינת הירדן שמקבלת טומאה (לעיל שבת פג, ב), והרי הספינה חוזרת ומטמאה כלי שבתוכה דהרי היא אב הטומאה, אבל דבר שאין רגיל להעמידה כגון חבלים אעפ״י שנגעו במת אינן אלא אב הטומאה, והן מטמאות את הספינה להיות ולד הטומאה, ואינה מטמאה את הכלים שבתוכה שאין מקבלין טומאה אלא מאב הטומאה.\par \textbf{} ואין פירושו מחוור. דאם כן למה לי קשרה אפילו נגעה נמי, ועוד דמביא לה טומאה באהל משמע, דאין הלשון הזה בשום מקום בנגיעה אלא במביא דרך אהל, ועוד דבדבר המעמידה אינו משמע כן, דאי בדבר הרגיל קאמר שהוא שלשלת של ברזל לא הוי ליה לשמואל למימר והוא שקשרה בשלשלת של ברזל, אלא אמר שמואל מאי דבר המעמידה שלשלת של ברזל. ועוד מה ענין זה לענין עירוב שיצטרך שמואל לאשמועינן דלא ניטעי בה, דנבעי נמי שלשלת של ברזל.}
\textblock{\textbf{ולפיכך פירש ר״ת ז״ל בשלשלת טמאה שנגעה במת או שהאהילה עליו, וחזרה הספינה והאהילה על השלשלת ועל הכלים וטמאם לפי שחרב הרי הוא כחלל, ודוקא קשרה כלומר: שקשר השלשלת בספינה לפי שאינה נפרדת ממנה      } ולעולם הספינה מאהלת עליה ועל הכלים ונמצאת הטומאה קבועה בה ולפיכך מביא את הטומאה, שאילו לא היתה השלשלת קשורה לספינה נמצאת הספינה לעתים נשמטת ממנה ולא חשבינן הטומאה כקבועה בה ואינה מביאה את הטומאה, וכדתנן (אהלות פ״ח, מ״ה) אלו לא מביאין ולא חוצצין, הזרעים וכו׳ והעוף הפורח והטלית המנפנפת וספינה שהיא שטה על פני המים, אבל עכשיו שהיא קשורה עם השלשלת אעפ״י שהיא שטה על פני המים מכל מקום הטומאה קבועה בספינה. והא דתנן (שם) וספינה שהיא שטה על פני המים אינה מביאה את הטומאה, התם כשהטומאה והכלים במקום אחד, והספינה והעוף שטין עליהן. ואם תאמר גם לפירוש זה, למאי אצטריך שמואל [לאפוקי] מההיא, מה ענין שבת אצל טומאה. תירץ הרמב״ן ז״ל, כי היכי דלא תיסק אדעתין דאפילו במאהלת על המת בעי שמואל קשרה בשלשלת משום דלהוי אהל קבוע והכא נמי ליבעי קביעותא ואי לא קביעי כמפורדות דמיין, קמ״ל דמשום דלהוי שלשלת גופא מטמאה קאמר, והוא הדין לכל דבר המעמיד דהוי אהל.}
\clearpage
\newsection{דף קב}
\textblock{ הכי גרסינן בתוספות (ד״ה אמר):\textbf{ זרק בפי הכלב.} דאילו לגירסת הספרים דגרסי ונחה בפי הכלב, משמע שלא נתכון שתנוח בפי הכלב ואם כן לא הוי כמקום ארבעה, אבל בשזרק לנוח בפי הכלב מחשבתו משויא ליה מקום.}
\textblock{\textbf{אם היתה שבת והוציאו.} פירש רש״י ז״ל דקסבר דאין עירוב והוצאה ליום הכפורים. ואינו מחוור, דזעירי דהוה בעי למידק מיניה הכין במקומה בפרק אמרו לו אכלת חלב (כריתות יד, א) איתותב, ואסיקנא דזעירי ברייתא היא, אלא הא אסיקנא התם דלהכי נקט אם היה שבת לטפויי איסורי, דאף משום שבת (היה) [חייב].}
\newchap{פרק \hebrewnumeral{12} הבונה}
\textblock{}
\textblock{\textbf{וליטעמיך אימא סיפא רבי יוסי אומר אפילו העלה וכו׳.} איכא למידק מאי קאמר, אי למימר דרבי יוסי פליג ושמואל כר׳ יוסי לא הוה ליה למימר וליטעמיך, אלא שמואל כרבי יוסי סבירא ליה אי נמי לימא תנאי היא. ואי לאקשויי מדרבי יוסי למאן דמותיב מת״ק מאי קושיא, הא ת״ק פליג בה. ובתוספות אמרו דלא גרסינן בה רבי יוסי, אלא הכי גרסינן: וליטעמיך אימא סיפא אפילו העלה על גבי דימוס, כלומר: וכולה חדא תנא, ומשני תלתא בנייני הוו. אלא מיהו קשה דבכולהו ספרי גרסינן בה רבי יוסי. ויש לומר לפי שיטתן דכולה רבי יוסי היא וכך היה מקובל בידן. ומכל מקום אי אפשר, דבירושלמי (בפרקין הל׳ א׳) עביד לה פלוגתא דר״י ורבנן דגרסינן התם: תני אחד מביא את האבן ואחד מביא את הטיט המביא את הטיט חייב, ר״י אומר שניהן חייבין, סבר ר״י אבן בלא טיט בנין, הכל מודים שאם נתן את הטיט תחילה ואחר כך נתן את האבן שהוא חייב.\par \textbf{} והרמב״ן ז״ל פירש דהכי קאמר: וליטעמיך שאתה תמה על שמואל דמחייב בצרור בלבד על גבי עפר, בא ותמה על רבי יוסי דמחייב אפילו בהנחה בלבד. אלא תלתא בנייני הוו, תתאה מציעא ועילאה, תתאה בצרורי ועפרא ולכולי עלמא חייב, והיינו דשמואל, מציעאה באבנא וטינא והא נמי כולי עלמא הוא, ועילאה בהנחה בעלמא לרבי יוסי, ובההיא הוא דפליגי רבנן עליה דרבי יוסי.}
\textblock{\textbf{האי מאן דעייל שופתא בקופינא רב אמר חייב משום בונה.} איכא למידק, והא אין בנין בכלים כדאמרינן בפרק שני דביצה (כב, א). ויש מפרשים דכי אמרינן אין בנין וסתירה בכלי, הני מילי בכלי שנתפרק כגון מנורה של חוליות, וכדשרו ב״ה (שם) לזקוף את המנורה ביום טוב, אבל לעשות כלי לכתחילה אין לך בנין גדול מזה, ואין זה נקרא בנין בכלים שהרי אינו כלי אלא עושה כלי. ובזה ניחא לי הא דאמרינן לעיל בפרק המצניע (שבת צה, א) דמגבן חייב משום בונה, ואם אין בנין בכלים היאך אפשר שיש בנין באוכלין, אלא שהעושה דבר מתחילתו חייב משום עושה כלי והוא הבונה.}
\textblock{\textbf{ולפי זה כתב הרמב״ן ז״ל דאפשר שכל כלי שצריך אומן בחזרתו מחייב משום בונה דהוה ליה כעושה כלי מתחילתו, שהרי משעה שנתפרק ואין ההדיוט יכול להחזירה בטל מתורת כלי. והיינו דגזרינן בפרק כירה (לעיל שבת מו, א) במנורה של חוליות ואסרינן ליה אפילו לטלטל, גזירה שמא תפול ותתפרק לגמרי ויחזירנה ונמצא עושה כלי לכתחילה. אבל אין נראה כן דעת הרב אלפסי ז״ל, לפי שהשמיטה מן ההלכות לומר שאינה הלכה, דאזלא כמאן דאמר יש בנין בכלים, ובמקומה (שם ד״ה הלכך) כתבתיה וכן בפרק קמא דמסכת ביצה (יא, ב) גבי תריסי חנויות בסייעתא דשמיא. ולפי הסברא הזו הא דאמרינן במאן דעביד חלתא שהוא חייב שלש עשרה חטאות כדאיתא בפרק כלל גדול (שבת עד, ב), חד       } מינייהו משום בונה, והכין איתא בירושלמי (פ״ז, הל׳ ב׳), והא דפליג הכא שמואל ואמר משום מכה בפטיש, משום דסבירא ליה דבדבר מועט כזה ליכא בנין אלא משום גמר מלאכה דהיינו מכה בפטיש.\par \textbf{} ורבינו האי גאון ז״ל כתב, רב אמר משום בונה ואף על גב דכלי הוא קסבר רב יש בנין בכלים, ולפי זה הא דתנן התם (ביצה כא, ב) בית הלל מתירין כלומר: לזקוף את המנורה, ומפרשינן בגמרא דביש בנין בכלים או אין בנין בכלים פליגי, לית ליה לרב הכין, אלא קסבר דאפילו בית הלל סבירא להו יש בנין בכלים, והא דשרו זקיפת המנורה, היינו משום דלא מחזי להו זקיפא לחוד כבנין כיון שלא נתפרקה כלל. והכין נמי סבירא ליה לרבי יוחנן (לעיל שבת מו, א) במה שאסר לטלטל את המנורה בשבת, והיינו נמי דקאמר התם בפרק קמא דביצה (יא, ב) גבי מתניתין דתריסין מוחלפת השיטה. אבל אנן דקיימא לן אין בנין בכלים, מפרשינן טעמא דבית הלל משום דאין בנין בכלים.\par \textbf{} ובירושלמי (פ״ז הל׳ ב׳) איתא: מה בנין היה במשכן, שהיו נותנין את הקרשים על גבי האדנים, ולא לשעה היתה, מכיון שהיו נוסעים וחונים על פי הדבור כמו שהוא לעולם, הדא אמרה בנין לשעה בנין, הדא אמרה אפילו נתנן על גבי דבר אחר, הדא אמרה אפילו מן הצד, הדא אמרה בנין על גבי כלים בנין, אדנים כקרקע הן.}
\clearpage
\newsection{דף קג}
\textblock{\textbf{אלא לשמואל לאו גמר מלאכה הוא.} פירוש: משום דקודח הוא שקודח כדי לשים בו מסמר או דבר אחר, וכיון שכן אינה גמר מלאכה אלא שמתקן תיקון כדי להגמר בדבר אחר דהיינו המסמר.}
\textblock{\textbf{דחק קפיזא בקבא.} פירש רש״י ז״ל: שחקק כדי קפיזא בבקעת גדולה הראויה לחוק בו קבא. ורב האי גאון ז״ל פירש: כגון שעשה סימן לידע היכא דוכתא דקפיזא, כמו שנתות היו בהין (מנחות פז, ב).}
\textblock{\textbf{אלא אביי ורבא דאמרי תרוייהו שכן מרדדי טסין למשכן עושין כן.} יש לפרש דאביי ורבא על רבה ורב יוסף פליגי, שאמרו דאינו מחמת שמאמן ידו אלא שכן היתה במשכן, והיינו דאמרינן אלא. אבל רבינו האי גאון ז״ל פירש דאביי ורבא לפרש דברי רבה ורב יוסף אתו, דאינו כמו שאתה סבור דמאמן את ידו לעשות מלאכה קא אמרי, אלא מפני שמאמן את ידו לכוין להכות באותה מלאכה עצמה שהוא עושה וזה מכלל המלאכה היא כדי שיעשה מלאכתו בתיקון, ואינו דומה לנטל את המגל לקצור ואינו קוצר, אלא התחלת המלאכה הוא כדאמרן. ובירושלמי (בפרקין הל׳ א) מקשו על הא דרבן שמעון בן גמליאל, דאמרינן התם וקשיא על הא דרבן שמעון בן גמליאל נטל לקצור ולא קצר שמא כלום הוא, אמר רב אדא אתיא דרבן שמעון בן גמליאל כר״י, דתניא השובט והמקשקש על האריג הרי זה חייב, מפני שהוא כמשיב ידו, ואף הכא מפני שהוא כמשיב ידו.}
\textblock{\textbf{אם לאכילה כגרוגרת אם לבהמה כמלא פי גדי אם להסיק כדי לבשל ביצה קלה.} פירוש: אם ראוי לאכילה שיעורו כגרוגרת, ואם אינו ראוי לאכילה, כמלא פי גדי, ואם הן יבשין שאינם ראוין לגדי אלא להסיק, שיעורו כדי לבשל ביצה קלה. אבל אם ראוי לאכילה אף על פי שמוציאו להסיק, שיעורו כגרוגרת, וכדאמרינן בפרק כלל גדול (לעיל שבת עו, א) המוציא תבן כמלא פי פרה לגמל, כולי עלמא לא פליגי דחייב דהא חזי לפרה.}
\textblock{\textbf{לא צריכא דעביד בארעא דלאו דיליה.} פירש בערוך (ערך פסק) דכל מידי דלית ליה הנאה, אף על גב דפסיק רישיה ולא ימות שרי. ומייתי ראיה מהא, ומההוא דאי אית ליה הושענא אחריתי (סוכה לג, ב). ואף רבינו האיי גאון ז״ל כתב כלשון הזה: ואוקימנא דקא עביד בארעא דלא דיליה, שאף על פי שהקרקע מתיפה (אי) [אין] עדיפות לזה התולש בתיקונו, ואיכא ודאי למימר אף על גב שהוא יודע שהקרקע מתיפה אין זאת במחשבתו ואינו רוצה, [לפיכך] אינו חייב אלא על כשיעור שחישב עליו. עד כאן. ובפ״ק דכתובות (ו, א) בשמעתא דמהו לבעול בתחילה בשבת, הארכתי בה.}
\textblock{\textbf{משני סימניות.} פירש רבינו האיי גאון ז״ל: שאינן אותיות ידועות בכתב שהן נקראות, אלא סימנין בעלמא, כגון נוני״ן הפוכין דכתיבי (במדבר י, לה-לו) גבי ויהי בנסוע הארן כדמפרש בגמ׳ דכל כתבי הקדש (לקמן שבת קטו, ב), וסימניות דכתיבי בהדדי בתהלים (קז, כב-ל) וכדמפרש בגמרא דארבעה ראשי שנים הן (ר״ה יז, ב). אבל רש״י ז״ל פירש:      סימניות, האחד בדיו והאחת בסיקרא. ואינו מחוור בעיני, דבבא זו אינה אלא בצורת האותיות, ובבא שניה במינין שהן נכתבין, ולמטה הוה ליה למיתני הא.}
\textblock{\textbf{המעבד כל שהוא.} לאו דוקא, דהא אמרינן בפרק המוציא (עט, א) שיעור המעבד כדי לכתוב קמיע. אלא לאפוקי מדרבי שמעון דאמר עד שיעבד את כולו, נקט איהו כל שהוא.}
\textblock{\textbf{זאת אומרת סתום ועשאו פתוח כשר.} פירוש: וכן פשוט שעשאו כפוף, דהא תנן נמי דן מדניאל. ואיכא למידק מנא ליה הא, דלמא כשכתב שם במ״ם סתומה ודן בנו״ן פשוטה. ומסתברא דהא ליתא, דאם כן לא זהו שם משמעון אלא תיבה אחרת היא, ולא מקצת התיבה הראשונה היא זו, ואפשר נמי דפטור הוא דמלאכה אחרת היא לכשתמצא לומר דסתום שעשאו פתוח ופתוח שעשאו סתום פסול. ועוד דהא מפרשינן לעיל בפרק הזורק (שבת צז, ב) משום דכל אימת דלא מיכתיב שם לא מכתיב שמעון, כלומר: וזו היתה כוונתו מתחלה, ואם איתא הא לא מיכתב שמעון בשם הזה. כך נראה לי.}
\textblock{\textbf{שלא יכתוב אלפי״ן עייני״ן ועייני״ן אלפי״ן.} פירש רש״י ז״ל: כי הוצרך לומר כן מפני שדומין בקריאתן, שיש בני אדם שקורין לאלפי״ן עייני״ן ולעייני״ן אלפי״ן. ואינו מחוור בעיני, דזה אינו בכלל כתיבה תמה, אלא מהפך לגמרי הוא הענין ומאבדו, שאילו כתב במקום את השמים ואת הארץ, עת השמים ועת הארץ היה מחריב העולם. אלא שלא יכתוב אל״ף כענין שיראה ממנה צורת עי״ן הפוכה, שאם ירחיק הרגל השמאלי מן הקו האמצעי יראה כאילו עי״ן הפוכה כזה: ומכאן יש ללמוד שצריך לזקוף ראש הקו האמצעי בכתיבת האלף ולא כזה: שאילו לא כן לא היתה כצורת העי״ן אפילו כשירחיק ממנה הרגל השמאלי. ואגב דאמר שלא יכתוב אלפי״ן עייני״ן אמר נמי ולא עייני״ן אלפי״ן, אף על פי שאי אפשר לכתוב עי״ן בצורת אל״ף, וכמו שאמר שלא יכתוב טיתי״ן פפי״ן בגררא דלא יכתוב פפי״ן טטי״ן.}
\textblock{\textbf{חיתי״ן ההי״ן.} כלומר: שלא ירחיק הקו האחרון, מגג האות כעין ה״א, וכן צריך להיות שני חטוטרות לחי״ת בגגה כגון ולא כגון זה: ותדע מדתנן בפרקין (דף קד, ב) נתכוון לכתוב חי״ת וכתב שני זייני״ן, ותניא נמי בגמרא כגון שנטלו לגג של חי״ת ועשאו שני זייני״ן, דאלמא לשני ראשי הגג יש כמין חטוטרות כצורת הזי״ן.}
\textblock{\textbf{זייני״ן נוני״ן נוני״ן זייני״ן.} כלומר: נו״ן פשוטה, ומכאן צריך לעשות לנו״ן חטוטרת כזה: שהיא כצורת זי״ן, ולא שיעשה אותה בראש כפוף כגון ראשו של וא״ו כזה: שאילו כן היה לו לומר ווי״ן נוני״ן נוני״ן ווי״ן.}
\textblock{\textbf{טטי״ן פפי״ן.} כלומר: שלא ירחיק הקו האחרון מן האות כגון זה: ומכאן דקדק ר״ת ז״ל שצריך לכוף הקו האמצעי בתוך הטי״ת כזה שאם לא יכוף אותו אפילו כשירחיק הקו האחרון לא ידמה לפ״א, שהרי הקו האמצעי של פ״א כפוף לתוך האות.}
\clearpage
\newsection{דף קד}
\textblock{\textbf{מ״ם וסמ״ך שבלוחות בנס היו עומדין.} מכאן שצריך לחבר ראש הקו האחרון עם הגג כגון זה: ועכשיו הם עומדות בנס, אבל אילו היה מרחיק ואפילו קצת הקו מן הגג כזה: לא היתה עומדת בנס.}
\textblock{\textbf{אלא סתום ועשאו פתוח גרועי גרעה דאמר מר מנצפ״ך צופים אמרום.} מהכא משמע דמנצפ״ך היו האותיות הפתוחות והפשוטות, ובריש פרק קמא דמגילה (ג, א) משמע בהפך שהן הסתומות והכפופות. וכבר פירש כאן ושם רש״י ז״ל, דקאי הכא וקא מקשה, וקאי התם וקא מקשה, ושיטת התלמוד היא בכמה מקומות. ויש לפרש דהתם על דרך ממה נפשך קא מקשה, כלומר אי מנצפך סתומות נינהו קשיא דלאו צופים אמרום שבלוחות היו דאמר       חסדא מם וסמך שבלוחות בנס היו עומדות, ואי פשוטות קשיא מהא שאין נביא רשאי לחדש דבר. וכבר כתבתי שם בארוכה בסייעתא דשמיא.}
\textblock{\textbf{מאי טעמא כרעא דקו״ף תליא.} מכאן שצריך להרחיק הרגל מן הגג כזה: ולא דבוק כזה:}
\textblock{\textbf{ומאי טעמא שקר אחדא כרעא קאי.} מכאן שצריך לעשות סוף השי״ן חד כזה: ולא רחב כזה}
\textblock{ מתני׳:\textbf{ ובכל דבר שהוא רושם.} ואמרינן בגמרא לאתויי מאי לאתויי הא דתני רב חיננא כתבו במי טריא ואפצא כשר, תני ר׳ חייא כתבו באבר בשחור ובשיחור כשר, ומהכא משמע דהלכות שבת כהלכות גיטין, ובעינן שיכתוב בדבר של קיימא ועל גבי דבר של קיימא, וכמו ששנינו לענין גיטין (גיטין יט, א), ובהדיא שנינו בתוספתא כן בכאן (פי״ב, ה״ו) המקרע על העור כתבנית כתב פטור, הרושם על העור כתבנית כתב חייב, בקליפי אגוזים בקליפי רמונים בדם הקרוש ובחלב הקרוש על עלה זית ועל עלי חרוב ועל עלי דלעת ועל כל דבר של קיימא חייב, על עלי כרשין ועל עלי בצלים ועל עלי חזרין ועל עלי ירקות ועל כל דבר שאינו של קיימא פטור, זה הכלל כתב דבר של קיימא בדבר שאינו של קיימא או דבר שאינו של קיימא על בדבר של קיימא פטור, עד שיכתוב דבר שהוא של קיימא על דבר שהוא של קיימא. ובקורע ורושם על העור (דבר של קיימא) מצאתי חלוף בין התוספתא והירושלמי כאן בגמ׳ (ה״ד), (ו)בתוספתא קורע פטור ורושם חייב, ובירושלמי קורע חייב ורושם פטור.}
\textblock{\textbf{כתב על גבי כתב פטור.} ודוקא דיו על גבי דיו, אבל דיו על גבי סיקרא אמרינן בגיטין פרק המביא תניינא (יט, א) שהוא חייב שתים, אחת משום מוחק ואחת משום כותב, ובסיקרא על גבי דיו פליגי בה התם, אמרי לה חייב ואמרי לה פטור, אמרי לה חייב מוחק הוא, ואמרי לה פטור מקלקל הוא. ובירושלמי בפרקין דהכא (ה״ד) גרסינן והוא שכתב דיו על גבי דיו וסיקרא על גבי סיקרא, אבל אם כתב דיו על גבי סיקרא וסיקרא על גבי דיו חייב, רבי יצחק בר משרשיא בשם רבנן דתמן חייב שתים משום כותב ומשום מוחק, וכתב ולא המטיף רבי יודן בר שלום ור׳ מתנא, חד אמר בשלא עירב נקודות וחורנא אמר אפי׳ עירב נקודות, וכתב לא השופך וכו׳.}
\textblock{\textbf{באבר כשר.} ואמרינן בגיטין (שם) והא תניא פסול, לא קשיא הא באברא הא במיא דאברא.}
\textblock{\textbf{וחכמים אומרים אין השם מן המובחר.} פירוש: לדידהו אינו כתב כלל דהא תנן כתב על גבי כתב פטור, אלא לטעמיה דרבי יהודה קאמרי ליה, כלומר: אפילו לדידך דהוי כתב, אודי מיהא לגבי כתיבת השם שאינו מן המובחר, והכין איתא בירושלמי (ה״ה) בשיטתו השיבוהו בשיטתך שאתה אומר כתב הוא, אף הוא אינו מן המובחר.}
\textblock{\textbf{אמר רב אשי אפילו תימא רבנן להשלים שאני.} ולכאורה הוה משמע דדוקא להשלים את הספר אבל להשלים את השם לא, דאם איתא ליתני להשלים את השם ואנא ידענא דכל שכן להשלים את הספר. אבל בתוספתא (פי״ב, ה״ב) גרסינן כתב אות אחת להשלים את השם, אות אחת להשלים את הספר חייב. ואיתא נמי בירושלמי בפרק האורג (ה״א).}
\textblock{\textbf{כתב אות אחת בטבריא ואות אחת בציפורי חייב.} פירוש: כתב אות אחת במגילה בטבריא ואות אחת באותה מגילה בצפורי. אבל רב האי גאון ז״ל כתב: כך היו הראשונים ז״ל מפרשים כי טבריא וצפורי סמוכות זו לזו, והיה שער של זו מסוייד בסיד עד קצה העיר, וכן שער של זו מסוייד בסיד עד קצה העיר, ואם תכתוב אות גדולה עד קצה זו, כאשר יכתב על השער (ים) [השני] אות גדולה כמוה עד קצה זו לזו, אע״פ שאין קרובות זו לזו הרי הרואה את זו ומהלך עד שרואה את זו, יודע הוא כי תואמות הן ומכוונות זו לזו, והיינו דאמרינן כתיבה היא זו אלא שהיא מחוסרת קריבה ע״כ. ודבר רחוק הוא זה. אבל מצאתי בירושלמי (דפרקין ה״ד) ענין יורה על פירוש זה, דגרסינן התם ר׳ יעקב רבי יוסי בשם ר׳ אלעזר כתב אות אחת בטבריא ואחת בציפורי חייב, והא תניא ואם אינן נהגין זה עם זה פטור, אמר ר׳ בא בר ממל תפתר בסיד דק.}
\textblock{ מהא דאמרינן:\textbf{ הכא במאי עסקינן כגון שנטל לגגו של חי״ת ועשאו שני זייני״ן.} שמעינן דכי האי גוונא לא מיקרי חק תוכות, שאינו אלא כמפריד בין שני אותיות      וכן שנטלו לגגו של דלי״ת ועשאו רי״ש אינו אלא כדיו שנפל על גבי האות ומעבירו.}
\textblock{\textbf{תנא נתכוון לכתוב אות אחת ועלו בידו שתים חייב.} איכא למידק ולהוי כנתכוון לזרוק שתים וזרק ארבע דפטור. ויש לומר דהכא בשרוצה שתים אלא שעכשיו היה מתכוין לכתוב האחת מהן ועלו בידו שתים, וכגון דלא נתכוון לכתוב שתי אותיות ידועות אלא איזה אותיות שיזדמן לו, וכההיא דאמרינן לעיל (שבת צז, ב) כל מקום שתרצה תנוח.\par \textbf{} ובהא נמי מיתרצא מתניתין דקתני נתכוון לכתוב חי״ת ועלו בידו שני זייני״ן פטור, דאף בהא איכא למידק מאי שנא נתכוון לכתוב חי״ת ועלו בידו שני זייני״ן, אפילו נתכוון לכתוב זיי״ן וחי״ת וכתב טי״ת וכ״ף פטור, שלא נתכוון לאלו, מידי דהוי אנתכוון לכבות זה וכבה זה דפטור כדאיתא בכריתות (כ, א), אלא דמתניתין בנתכוון לכתוב שתים איזו שתים שיעלו בידו, אבל עכשיו היה כותב חי״ת ועלו בידו שני זייני״ן, ומכל מקום בדעתו הוא שאם יזדמנו לו דרך מקרה אותיות הצריכות זיון שיזיינם, ומשום הכי אמרינן הא דבעי זיוני הא דלא בעי זיוני.}
\clearpage
\newsection{דף קה}
\textblock{\textbf{חביב נתתיך באומות.} דאנו דורשין ה״א בחי״ת, וכמו שאמרו בירושלמי פרק כלל גדול (ה״ב) גבי אבות מלאכות ארבעים חסר אחת, מנין לאבות מלאכות מן התורה, רבי חנינא דצפורין בשם ר׳ אבהו אל״ף חד, למ״ד תלתין, ה״א חמשה, דבר חד, דברים תרין, מכאן לאבות מלאכות ארבעים חסר אחת שכתוב בתורה, רבנן דקסרין אמרין מן אתרה לא חסרה כלום, אל״ף חד, למ״ד תלתין, ח׳ תמניא, לא מתמנעין רבנן דרשין בין ה״א לחי״ת.}
\textblock{מתני׳:\textbf{ ואחד על האריג.} יש לפרש אחד על שנים, שהרי עם האחד הזה הרי היא מלאכה מתקיימת, ויש לפרש ואחת על שלשה. ופלוגתא היא בירושלמי, דגרסינן התם (ה״א) אשכחת על דר״א פעמים שלשה בתחלה, פעמים שנים על גבי אחד ארוג מאתמול, פעמים אחד על גבי שנים ארוגים מאתמול, רבנן דקסרין בעיין מהו אחד על האריג אחד על גבי שלשה, רבנן דהכא אמרין אחד על גבי שנים.}
\textblock{\textbf{וחכמים אומרים בין בתחלה בין בסוף שיעורו שני חוטין.} פירוש: בין בתחילת השיעור בין בהשלמת השיעור, אבל בסוף ממש דהיינו סוף היריעה אפילו באחד חייב, כדתניא לעיל בשלהי פרקין דהבונה (שבת קד, ב) כתב אות אחת והשלימה לספר ארג חוט אחד והשלימו לבגד חייב, ואוקימנא אפילו לרבנן דלהשלים שאני.}
\newchap{פרק \hebrewnumeral{13} האורג}
\textblock{}
\textblock{ גמרא:\textbf{ ואע״פ שחילל את השבת יצא ידי קריעה.} כלומר: אף ע״פ שעשה הקריעה באיסור אין צריך לחזור ולקרוע, ובירושלמי (דפרקין ה״ג) מקשו עלה, בעון קומי רבי בא הכמה דאתמר תמן השוחט חטאתו בשבת כיפר ומביא אחרת, ואימא אף הכא לא יצא ידי קריעה, אלא כר׳ שמעון דר״ש אומר עד שיהא לו צורך בגופו של דבר, אמר לון תמן הוא גורם לעצמו ברם הכא את גרמת לו, אמר ר׳ יוסא ואפילו תמן את גרמת לו שאלולא שאמרת לו [יביא] היאך היה היה מתכפר לו, הוי סובר מימר דר״ש הוא, חבריא בעון קומי ר׳ יוסא, לא כן א״ר יוחנן בשם ר״ש בן יוצדק מצה גזולה אין אדם יוצא בה ידי חובתו בפסח, א״ל תמן גופיה עבירה ברם הכא היא עבר עבירה, כך אנו אומרים הוציא מצה מרה״י לר״ה אינו יוצא בה ידי חובתו בפסח.}
\textblock{\textbf{חמתו אחמתו נמי לא קשיא הא רבי יהודה הא רבי שמעון.} נראה מדברי רש״י ז״ל, דהשתא דאתינן להכי, מתו       דמתניתין היינו ר״ש, ואפילו במת דידיה דבעי אבולי עליה, [דהיא] מלאכה שאינה צריכה לגופה כהוציא את המת לקברו, אלא דמעיקרא קא מהדר לאוקומה כולה כחד תנא, ולאוקמינהו מתניתין וברייתא כר׳ יהודה, וכיון דלא אפשר ליה לאוקומיה חמתו דמתניתין ודברייתא כחד תנא, הדרינן ומוקמינן כולה מתניתין דהיינו מת וחמתו כר״ש, ודברייתא כר״י. ובתוס׳ מפרשים דלעולם מתו אמתו צריכין לתרוצי כדתריצנא, דאי במתו ממש שראוי להתאבל עליו אפילו ר״ש מודה דחייב דמלאכה הצריכה לו מקרי הואיל ומצוה קא עביד. ואינו מחוור, דא״כ אף המוציא את המת לקוברו יהא חייב, דמצוה קא עביד, דדוחק הוא לומר דהתם לא רמי עליה ממש להוציאו, דאפשר ע״י אחרים, אבל קריעה הוא חייב בכך, דמת מצוה מאי איכא למימר, ואנן המוציא את המת סתמא תנן.\par \textbf{} ויש מפרשים משום דצריך ללבוש בגד קרוע כל שבעה וקרעו לפניו, אף לר״ש מקרי מלאכה הצריכה לגופה. וגם זה אינו מחוור בעיני, דהא משמע הכא דאף אדם כשר אי נמי שאר בני אדם דלאו כשרין אי קאי בשעת יציאת נשמה הרי הן כמתו, והנהו אינן חייבין ללבוש בגד קרוע אלא מאחין הן מיד. ועוד דאפילו אחיו ואחותו אם החליפו אינן חייבין לקרוע אלא אותו בגד שנתקרע אסור באחוי, וכן הוא כל זמן שלובשו צריך שיהא קרעו לפניו, הא אם רצה להחליף מחליף ואינו קורע (מו״ק כד, א), ודברי רש״י ז״ל נראה לי עיקר.}
\textblock{\textbf{הא ר׳ שמעון והא ר׳ יהודה.} ואיכא למידק היכי מוקמינן מתניתין כר״ש, והא מתניתין קתני וכל המקלקלין פטורין ואוקימנא לה לקמן על כרחין כר״י. ואיכא למימר דאין הכי נמי דהוה מצי לאקשויי הכין אלא דעדיפא מיניה אקשי ליה, ולמסקנא דאסיקנא דקרע למירמא אימתא על אנשי ביתיה, כולהו רבי יהודה נינהו, ומתניתין בקורע למעבד נחת רוח ליצרו, ומתניתא למירמא אימתא אאנשי ביתיה, אבל רש״י ור״ח ז״ל פירשו, דרישא דמתני׳ ר״ש, וסיפא דהיינו המקלקלין פטורין ר״י, אבל צריך לעיין דהא חד בבא הוא ובחד פטורא כייל להו.}
\clearpage
\newsection{דף קו}
\textblock{\textbf{כל המקלקלין פטורין חוץ מחובל ומבעיר.} אוקימנא כר״ש, ומתניתין דכייל ותני כל המקלקלין ולא מפקינן חובל ומבעיר ר״י היא. ומפני שלא מצינו למחלוקת הזה עיקר בשום מקום, דקדק רש״י ז״ל דנפקא להו ממחלוקתן במלאכה הצריכה לגופה, ויסוד הענין לפי שכל המלאכות ממשכן גמרינן להו וכולן צריכות היו לגופן ותיקון היה בהם, ולפיכך סבר ר״ש דכל מלאכה שאין לו תיקון וצורך גוף המלאכה ממש, פטור עליה, וכיון שכן מילה בשבת מן הסתם מותרת שהרי יש בו חבלה וקלקול אצל התינוק, ותיקון המילה שהוא עושה המצוה אינו תיקון בגוף הדבר אלא שהיא מצוה ותיקון מצוה אין מתחשב לו, אע״פ שהוא מתקנו לתרומה ולקדשים ולאכילת פסחים, אין אותו תיקון בגוף המילה אלא אצל אחרים ומלאכה שאי״צ לגופה היא, וכיון שכן לא היתה התורה צריכה להתיר מילה בשבת, אלא לאשמועינן דאע״ג דמקלקלין בעלמא פטורים המקלקל בחבורה חייב, וחידוש הוא שחדשה תורה במילה. וכן הבערה דבת כהן מן הסתם מותרת היתה, לפי שכל מבעיר מקלקל הוא והתיקון הוא אצל אחרים ומלאכה שאי״צ לגופה היא, ולפיכך כשאסר הכתוב הבערה דבת כהן, לחדוש בא לאשמועינן דמקלקל בהבערה חייב.}
\textblock{\textbf{ור׳ יהודה סבר דמלאכה הצריכה בלבד בעינן, כלומר: שיהא בה תיקון מצד אחד או בגופה של מלאכה או אפילו אצל אחרים, ולפיכך מילה אסורה היתה שהרי יש בה תיקון אצל אחרים כמו שאמרנו, ולפיכך הוצרך הכתוב להתירה, וכן הבערה דבת כהן מן הדין אסורה היא דהרי יש בה תיקון האבר שהוא מצרפו, והרי הוא כבשול סמנים, ולא אסרה התורה הבערת בת כהן אלא מצד תיקון האבר, אבל הבערה גופה דבת כהן קלקול גמור הוא ומותר היה לולא התיקון שיש בה מצד אחר שהוא בישול הפתילה. (וצריך עיון לרבי שמעון בבישול הפתילה שהיא מלאכה הצריכה לגופיה לרב אשי [ולרבי שמעון] אין לומר כן, וכן בפרק קמא דיבמות (ו, ב) אומר כן. וצריך עיון בהבערה לחלק דוקא לרבי יהודה, וכן [קבורת] מת מצוה לרבי שמעון לידחי [שבת]). ולעולם מקלקלין גמורין אפילו בחובל ומבעיר נמי פטורין, ולא משכחת לה לרבי יהודה דחייב אלא בשיש בו תיקון אצל אחרים, חובל בצריך לדם לתת לכלבו ומבעיר בצריך לאפרו. וא״ת מכל מקום למה הוצרך הכתוב לאסור הבערה בבת כהן. ויש לומר דלא תבערו לרבי יהודה לחלק, ומושבות דמיניה דרשינן הבערה דבית דין כדאיתא בריש פרק קמא דיבמות (ו, ב) איצטריך, כי היכי דלא נידחי מיתת בית דין שבת מקל וחומר כדאיתא התם ביבמות. ועדיין צריך לתרץ (שיטת) [קושית] הראב״ד ז״ל דקשה (עליו) [ליה] אם כן לרבי יהודה שאר מיתות ב״ד יהיו מותרות בשבת, דהא מקלקל גמור הוא. ויש לומר דלרבי יהודה נמי שאר מיתות אסורות, דיש בהם תיקון מצד שהן מתבערין מתוך העדה, והויה לה כהוצאת כזית מן המת וכעדשה מן השרץ (לעיל שבת צג, ב) דחשיב תיקון הרחקת הטומאה וסילוק הנזק. והא דאמרינן מה לי בישול פתילה לרבותא נקטיה, ולא צריך. ולרבי שמעון לא חשיבא ליה בישול הפתילה תיקון. ומיהו אפשר דרב אשי לאו [משום] דלא סבירא ליה כר״ש, אלא קא מפרש טעמא דר׳ יהודה. והתם נמי ביבמות [מייתי] הדין טעמא, ואמר אימת דשמעת ליה לרבי יוסי דהבערה ללאו יצאת הני מילי הבערה דעלמא, הבערה דב״ד בישול פתילה הוא ואמר רב אשי מה לי בישול פתילה מה לי בישול סמנין, התם הוא משום דבין לרבי יוסי בין לר׳ נתן לית להו דרבי שמעון, דהא לר״ש      } הבערה לגופה איצטריכא ולחדושא ולא יצתה לא לחלק ולא ללאו, ונמצאת הבערה בבת כהן אסורה לר׳ שמעון מדגלי בה קרא, ושאר מיתות ממילא, ואיצטריכא מושבות כי היכי דלא נדחי מק״ו מעבודה.\par \textbf{} ואכתי קשיא לי כיון דקיום מצוה לא חשיבא לה תיקון בגופו של דבר ומלאכה שלא לצורך הוא, אם כן קבורת מת מצוה תדחי שבת, ובשלמא לרבי יהודה כיון שיש תיקון אצל המת אסור, וכ״ש דליכא חפירה בקרקע קלקול, אלא לרבי שמעון קשיא. ויש לומר דלרבי שמעון אין הכי נמי דשרי. והא דקאמרינן התם קבורת מת מצוה תוכיח שאינה דוחה את השבת, דלא כרבי שמעון דלרבי שמעון ודאי דוחה, ותדע דבפרק המצניע (שבת צד, א-ב) פוטר היה רבי שמעון במוציא את המת לקברו, וכללא כייל אפילו במת מצוה. זה נראה לי לפי שיטת רש״י ז״ל. ורבי שמואל ז״ל דקדק מתוך דבריו וכו׳ וקשה עליו וכו׳. ועוד קשה דמי עדיף מקלקל ממתקן, מתקן צריך לגופה מקלקל לא צריך לגופה, ובספר הישר (סי׳ רצ) פירש וכו׳.\par \textbf{} ויש לי לדקדק, אם במקלקל בחבורה חשיב לצורך גופה, אם כן כל המקלקלין פטורים חוץ מחובל ומבעיר הכי קאמר כל המקלקלין ואפילו יש בהן צורך פטורים, והיכי משכחת לה לפטורא, והראב״ד ז״ל כתב כללא דמלאכה הצריכה לגופה וכו׳.}
\textblock{\textbf{ואין נותנין לפניהן מזונות.} פירש רש״י ז״ל: כיון דמוקצין הן לא מצי מיטרח עלייהו. והקשו עליו בתוספות דהא תנן (לקמן שבת קנו, ב) מחתכין את הדלועין לפני הבהמה ואת הנבלה לפני הכלבים, ובפרק מפנין (שבת קכו, ב) אמרו מטלטלין את הלוף מפני שהוא מאכל לעורבים, ומאן דאסר התם לא אסר אלא משום שאין דרכם של ישראל לגדל עורבים. ועוד קשיא אפילו אווזין ותרנגולין היאך נותנין לפניהם מזונות. ופירשו הם ז״ל דאינה אלא גזירה שמא יבא לידי צידה, והלכך כל שהוא מחוסר צידה דאורייתא אסור, וכל שאין מחוסר צידה דאורייתא מותר. והביאו ראיה מדאמרינן בפ׳ אין צדין (ביצה כד, א) היכי דמי מחוסר צידה אמר רב יהודה [אמר שמואל] כל שאומר הבא מצודה ונצודנו, א״ל רב יוסף והא אווזין ותרנגולים שאומר הבא מצודה ונצודנו ותניא הצד אווזין ותרנגולים ויונים הדריסיאות פטור. ואיכא למידק בההיא מאי קושיא, דהא פטור אבל אסור, דכל פטורא דשבת פטור אבל אסור, וההיא גבי שבת מתניא. אלא הכי פירושו דאמתניתין דקתני כל שמחוסר צידה אסור קאי, ופירושו אסור ליתן לפניהם מזונות, וה״ק איזה מחוסר צידה שגוזרין עליו שלא ליתן לפניו מזונות, ואמר רב יהודה אמר שמואל כל שאומר הבא מצודה ונצודנו, ולפי׳ אקשי ליה רב יוסף דאלמא כל שאומר הבא מצודה מחוסר צידה דאורייתא כיון דגזרינן, והא אווזין ותרנגולים שאומר הבא מצודה ונצודנו ואפילו הכי הצדן פטור אלמא לאו מחוסר צידה דאורייתא, וכיון שכן היכי גזרינן ביה שלא ליתן לפניהם מזונות שמא יצוד. אבל בירושלמי דפרקין (ה״ז) מפורש לפי שאין עושין תקנה לדבר שאינו מן המוכן, וכדברי רש״י ז״ל. ויש לפרש דההיא דכלבים ועורבין לא קשיא, לפי שמוכנין הן למלאכתנו וברשותינו הם ומזונותיהן עלינו, אבל מה שאינו ברשותינו ואינן ניצודין אסור לפי שאין מזונותן עלינו, ומה שהוא בביבר קטן הרי הוא כניצוד וברשותינו הוא, לפיכך נותנין לפניהם מזונות שאף הם מזונותן עליך.}
\textblock{\textbf{אמר ליה אביי הלכה מכלל דפליגי.} איכא למידק ולימא ליה אין, דהא ר׳ יהודה פליג בכל ביברין. ולאו מלתא היא דלאפוקי מדר׳ יהודה לא צריך דפשיטא דהלכה כחכמים, ועוד שלא היה לו לומר הלכה כרשב״ג אלא הלכה כחכמים, ולא הוצרך רב יוסף לפסוק כרשב״ג אלא לאפוקי מחכמים ולפיכך הקשה הלכה מכלל דפליגי.}
\clearpage
\newsection{דף קז}
\textblock{\textbf{אף אנן נמי תנינא אף על פי שעמד הראשון והלך לו הראשון חייב והשני פטור מאי לאו פטור ומותר לא פטור אבל אסור.} תמיהא לי והיאך אפשר לומר כן שיהא השני אסור, אטו אלו צד הראשון ובא שני ונטלו אחר שניצוד מה איסור יש בדבר. ועוד תמיהא לי מאי קאמר ר׳ אבא נכנסה לו צפור תחת כנפיו יושב ומשמרה, דאלמא דוקא יושב ומשמר אבל נוטל לא, ואמאי אפילו ליטול למה לא כיון שכבר ניצוד. ויש לומר דמדאורייתא ודאי מותר, אלא שאסור לטלטל משום טלטול דמוקצה, ומכל מקום אתא לאשמועינן דיושב ומשמר ולא גזרינן אטו צד לכתחילה, ומהאי טעמא נמי הוא דהוה מצי למימר השתא דשני פטור אבל אסור לישב שם אחר עמידת הראשון מפני שנראה כאילו הוא צדו עכשיו וגזרינן אטו צד לכתחילה.}
\textblock{ תוספתא (פי״ג ה״ו):\textbf{ ישב אחד על הפתח ובא אחר וצדו מבפנים, היושב על הפתח חייב והצדו פטור, הא למה זה דומה לנועל את ביתו לשמרו ונמצא צבי שמור בתוכו.} פירוש: כאילו יש צבי ניצוד בתוך הבית והבית סגור, ובא זה ונועל עוד בפניו במנעול, אף על פי שגורם שמירה על שמירתו מותר, ולא אמרינן הרי זה כצדו שאלולי לא נעל הוא במנעול שמא היו דלתות נפתחות וצבי נמלט, כך שני זה אחר שישב הראשון, ואי נמי כשהיה שם צבי קשור ובא זה ונעל שאף על פי שנעל הצבי אין אומרים לזה שיפתח ביתו כדי שיצא הצבי, לפי שבשעה שנעל בהיתר נעל לשמור מה שבתוכו, ואפילו נתכוון לשמור הצבי שגם הוא מכלל חפצי ביתו הוא ונועל לשמרו לכתחילה.}
\textblock{    תוספתא (שם):\textbf{ ישב אחד על הפתח ונמצא צבי בתוכו, אע״פ שמתכוון לישב עד שתחשך פטור מפני שקדמה צידת צבי למחשבה, אין לך שהוא חייב אלא המתכוון לצוד, אבל קדמה צידה למחשבה פטור.} ואפשר דפטור דקתני הכא פטור אבל אסור הוא, כיון שעל ידי מעשיו הוא ניצוד ולא היה נצוד מתחילה, והיינו דקא תני במתניתין לנועל את ביתו לשומרו ונמצא צבי שמור בתוכו וכו׳, והכא קתני ונמצא צבי בתוכו. ויש לומר דפטור ומותר קאמר, לומר שאינו צריך לפתוח כיון שלא נתכוון לצידה ונמצא ניצוד בתוכו בלא מתכוון ואינו מוסיף עכשיו בצידתו למה יפתח ומאי דהוה הוה, דכיון דלא עביד איסורא דאורייתא בצידתו בתחילתו הרי הוא מותר בסופו.\par \textbf{} ובירושלמי (דפרקין ה״ו) נראה שהתירו לנעול לכתחילה ביתו לשמור ביתו וצבי שבתוכו, דכיון שהוא צורך ביתו אע״פ שעל ידי כך ניצוד הצבי ממילא מותר, ובלבד שלא יתכוון לשמור את הצבי בלבד, דהכי גרסינן בירושלמי בפרקין דהכא: ר׳ יוסא בר׳ בון בשם ר׳ חונא היה צבי רץ כדרכו ונתכוון לנעול כדרכו בעדו ובעד הצבי מותר, ראה תינוק מבעבע בנהר ונתכוון להעלותו ולהעלות נחיל של דגים עמו מותר, ר׳ יוסי בר׳ בון בשם ר׳ חונא היה מפקח בגל ונתכוון להעלותו ולהעלות צרור של זהובים עמו מותר. ולפי זה הא דאמרינן ונתכוון לנעול בעדו, לא בעדו בלבד קאמר, אלא אם נתכוון לנעול אף בעדו קאמר, ולומר שאילו צריך לנעול בעדו מותר אע״פ שמתכוון שיהא הצבי ניצוד בתוכו (ע״כ).}
\textblock{ מתני׳:\textbf{ הצדן והחובל בהן חייב.} פירוש: החובל בהן עד שנצרר הדם, דכיון שיש להם עור הוא מעכב את הדם מלצאת ואלמלא העור מעכבו היה יוצא, וכיון שכן אין חבורה חוזרת וחייב משום נטילת נשמה שבאותו מקום כי הדם הוא הנפש, אבל שאר שרצים ורמשים אילו נעקר הדם ממקומו לגמרי עד שתהא חבורה אף הוא היה יוצא בחוץ כיון שאין להם עור שיעכבנו, אבל אילו יצא לחוץ אף בשאר שקצים היה חייב ומשום נטילת נשמה. אבל רש״י ז״ל פירש, דאיסור החבלה משום צובע, ולדבריו פטור של שאר שקצים, משום שאין להם עור שיצטבע בדם. ואינו מחוור, לפי שבפרק אלו טריפות (חולין מו, ב) אמרינן בהדיא דאף שאר שקצים ורמשים אם יצא מהן דם חייב, ומפרש נמי טעמא דשאר שקצים משום דלא הדרא בריא ומשום נטילת נשמה, ועוד מה צורך בצביעת עורן, וכאותה שאמרו בפרק כלל גדול (שבת עה, א) שוחט מפני מאי מחייב, רב אמר משום צובע, אמר רב מלתא דאמרי אימא בה מילתא דלא ליתו דרי בתראי וליחכו עלה, צובע מאי ניחא ליה, ניחא דליתווס בית השחיטה דמה דליחזו אינשי וליזבנו מיניה, ואם כן הכא מאי ניחא ליה.\par \textbf{} אלא מיהו בירושלמי בפרק כלל גדול (ה״ב) משמע כדברי רש״י ז״ל, דגרסינן התם מה צביעה היתה במשכן, שהיו משרבטין בבהמה ועורות אלים מאדמים, אמר ר׳ אסי הדא אמרה העושה חבורה ונצרר הדם חייב משום צובע, המאדים אודם בשפה חייב, והמוציא את הדם חייב משום נטילת נשמה שבאותו מקום. ואולי אם הוא צריך לצבוע אותו מקום לשום ענין קאמר שהוא חייב אף משום צובע, ולהתחייב שתים משום צובע ומשום נטילת נשמה, וכדאמרינן בפרק כלל גדול (שבת עה, א) משום צובע אין משום נטילת נשמה לא, אימא אף משום צובע.}
\textblock{}
\textblock{\textbf{והצדן לצורך חייב שלא לצורך פטור.} פירש רש״י ז״ל: שלא לצורך לפי שאין במינן ניצוד. ואינו מחוור, דאם כן אפילו לצורך נמי פטור. אלא משמע דשמונה שרצים האמורים בתורה אין דרכן להזיק וכל צידתן היא לצורך, אבל שאר שקצים ורמשים שדרכן להזיק פעמים אדם צדן שלא לצורך, כלומר שלא יזיקו ומשום הכי תני פלוגתא בסיפא, והוא הדין לרישא. הרמב״ן ז״ל.}
\textblock{\textbf{חיה ועוף שברשותו הצדן פטור.} ירושלמי (ה״א): לא אמר אלא שברשות, הא אינו ברשות אדם חייב, אמר ר׳ יוסי הדא אמרה שור שמרד הצדו בשבת חייב.}
\textblock{ גמרא:\textbf{ אמר ר׳ יהודה מאן ת״ק רבי יהודה דאזיל בתר גישתא.} פירש רש״י ז״ל: דלית ליה אלה הטמאים, אלא משום דאזיל בתר גישתא אמר דהני לית להו עורות, הלכך מתניתין דהחובל בהן ודאי דלא כותיה. והקשו עליו בתוס׳ דהא רבנן נמי אזלי בתר גישתא מדלא מנו תנשמת באלו שעורותיהן כבשרן דאלו שעורותיהן כבשרן ואמרינן התם בגמרא (חולין קכב, ב) מ״ט לא תני נמי תנשמת ופרקינן רבנן אזלי בתר גישתא ולא פליגי בהדי ר״י אלא בגישתא דהלטאה, ואפילו הכי אמר רב לעיל עד כאן לא פליגי רבנן עליה דרבי יוחנן בן נורי אלא לענין טומאה דכתיב (ויקרא יא, לא) אלה הטמאים לכם. ולפיכך פירשו הם ז״ל דכולהו אית להו דרשא דאלה הטמאים לכם, ולומר דעורות יש להם, אלא שהתורה ריבתה אותן לטומאה כבשר, אלא דקרא מיעט וריבה דכתיב אלה הטמאים, אלה מיעוט הוא וה׳ דהטמאים ריבה, והלכך כל דאית ליה גישתא דעור ממעטינן מאלה ואינו מטמא כבשר, וכולהו אינך מטמאין כבשר דמרבינן מהטמאים, ולית להו למינהו הפסיק הענין, אלא כיון דמיעט הכתוב וריבה אזלינן בהו בתר גישתא, אלא דבגישתא דהלטאה פליגי רבנן ורבי יהודה. והם ז״ל חזרו והקשו לפירושם, דאם כן בסמוך דקאמר מאן רבנן דר׳ יוחנן בן נורי, ר׳ יהודה דאזיל בתר גישתא מכל מקום מנא ליה דפליג רבי יהודה, לימא נמי דעד כאן לא פליג אלא לענין טומאה משום דכתיב אלה הטמאים, אבל לענין שבת לא פליגי כדקאמר לרבנן, ולא העלו בו דבר.\par \textbf{} ולי נראה דהאי סוגיא לא אזלא כההיא דפרק העור והרוטב (שם) דהתם מפרש טעמא דרבנן משום דאזיל בתר גישתא, ומאן דאזיל בתר גישתא לא דריש כלל אלה הטמאים ולמינהו הפסיק הענין כדאיתא התם, ואיהו נמי תני תנשמת כדקאמרינן התם רב תנא הוא ותנא תנשמת, ושמא סבירא ליה לרב דרבנן נמי תנו תנשמת, דכל מאן דאית ליה אלה הטמאים תני אפילו תנשמת, דאי לא לא לכתביה רחמנא לתנשמת בהדי דאנקה והכח, לכתביה גבי החלד והעכבר, והא דקאמר רב הכא דרבנן לא פליגי אלא לענין טומאה ומשום דכתיב אלה הטמאים, אזיל לטעמיה דהתם, אבל למ״ד התם דרבנן לא תנו תנשמת, לית ליה אלה הטמאים, אלא כולה מלתא משום דאזלי בתר גישתא, בין לענין טומאה בין לענין שבת.}
\textblock{ הא דאמרינן:\textbf{ במושיט ידו למעי בהמה ודלדל עובר שבמעיה דחייב משום עוקר דבר מגדולו.} הקשה הרמב״ן ז״ל, דהא תולש כנף מן העוף (לעיל שבת עד, ב) וכן גוזז כשהן חיין, לא מחייבינן תרתי חדא משום עוקר דבר מגדולו וחדא משום תולש או גוזז, וכן (לעיל שבת צד, ב) הנוטל שערו או שפמו וצפרניו, דאלמא ליכא משום עוקר דבר מגדולו אלא בגדולי קרקע כמו שאין דישה אלא בגדולי קרקע (לעיל שבת עה, א), והכי נמי משמע בבכורות (כה א) גבי תולש צמר מבכור, דתולש לאו היינו גוזז, וכנגדו ביום טוב מותר דלית ביה משום עוקר דבר מגדולו. ולפיכך פירש הוא ז״ל דדלדל את העובר משום נוטל נשמה הוא חייב, והכי קאמר אע״ג דהאי עובר לית ליה בדידיה נשמה, כיון דגידולו תלוי בנשמת אמו העוקרו חייב משום נוטל נשמתו ממנו, דלאו מי אמר רב ששת בגדולי קרקע דמאן דתליש כשותא מהיזמי שיניקתו תלויה בהיזמי כדאמרי׳ בעירובין בפרק בכל מערבין (עירובין כח, ב) דקטלין ליה להיזמתא ויבשה כשותא, ומחייב משום עוקר דבר מגדולו דהוא משום נטילת נשמה, דומיא דחובל שהוא משום נטילת נשמה מאבר אחד, וסירכא דלישנא נקט, וכן נראה מדברי הרמב״ם (פי״א, ה״א) מהלכותיו. ומה שאמרו בירושלמי בפרק כלל גדול (ה״ב) רבנן דקסרין אמרין ההיא דצד נונא וכל דבר שהוא מבדילו מן חיותו חייב משום קוצר, לא אתי בשיטתא דגמרא דילן.}
\textblock{\textbf{הצד פרעוש בשבת רבי אליעזר מחייב חטאת ורבי יהושע פוטר.} פירוש: פטור אבל אסור, ובתוס׳ [אמרו] שאם יתירא שמא ישכנו וכל שכן אם נושך אותו, יכול להסירו דאינו אלא כמתעסק דהוה כקוץ ברשות הרבים.}
\clearpage
\newsection{דף קח}
\textblock{ ה״ג ר״ח ז״ל:\textbf{ אמר ליה שמואל לקרנא גברא רבה אתי ממערבא וחש במעיה וחזינהו שמואל למיא דקא דלו לקבליה ועכירי.} פירוש: דשמואל שמע דאדם גדול בא בספינה, והיה רואה המים עולין למעלה ע״י הרוח והיו עכורות מטיט ועפרורית והיה יודע שהיה שותה מאותו נהר ואותו המים משלשלין, לכך אמר דחש במעיה.}
\textblock{ מתני׳:\textbf{ אין עושין הילמי בשבת אבל עושין מי מלח.} ירושלמי (דפרקין, ה״ב):מה בין הילמי מה בין מי מלח, הילמי צריכה אומן מי מלח אין צריך אומן.}
\clearpage
\newsection{דף קט}
\textblock{\textbf{עלין אין בהן משום רפואה.} פירש רש״י ז״ל: באכילתן. ואינו מחוור דדבר למד מענינו הוא וכולה שמעתא ברפואת העין מיירי, ועוד דאם כן כי קאמר רב ששת גרגירא אפילו לדידי מעלי לי, למיסר אכילתה, ואי אמרת אכילה דהא תנן (בעמוד ב) כל האוכלין אוכל אדם לרפואה, וגרגירא מיכל אכלי לה אינשי ומאכל בריאים הוא. אלא לשים על גבי העין קאמר.}
\textblock{\textbf{שירקא טחיא שרי.} פירש רש״י: לטוח צלי בבצים טרופות, ובלבד שלא יהא חם כל כך שיתבשלו בו. ואינו מחוור, דאם כן פשיטא ומאי קא משמע לן, דהא אין כאן הכשר אוכלין כל כך דנטעה בו לאיסור, ואיך יאסור רב חייא בר אבא, ועוד דמה דומה לנתינת מים צלולין ויין צלול לתוך המשמרת דקא מייתינן עלה. ור״ח ז״ל פירש שהוא מעי האבטיח, וקא שרי ליה לסנן אותו במסננת כמים צלולין ויין צלול בתוך המשמרת, ועושין כן מפני שהוא טוב לשלשל בני מעים.}
\textblock{\textbf{פיעפועי ביעי.} פירש רש״י ז״ל: לטרוף בצים בקערה, והוא אסור משום מראית העין שנראה כמי שרוצה לבשל בקדרה. וגם זה אינו מחוור, דדבר הלמד מענינו הוא, ובענין רפואה עסקינן. ור״ח ז״ל פירש דמין ירק הוא כמו פעפעין וחלוגלוגות דעירובין ואמרינן מערבין בפעפועין בפרק בכל מערבין (עירובין כח, א), ופירש בירושלמי בסוף מסכת פאה (פ״ח, ה״ד) קקולי. ורש״י ז״ל פירש שם בעירובין שהם ירקות ששמם נוטלש. אבל קשה לי דכיון דמערבין בהם אלמא אכלי להו אינשי, אם כן אמאי אסור דהא תנן [כל אוכלין] אוכל אדם לרפואה.}
\textblock{\textbf{נותן אדם מים צלולין ויין צלול לתוך המשמרת בשבת.} ודוקא לתוך המשמרת לפי שאין אדם מקפיד עליה ולא אתי לידי סחיטה, וכאותו שאמרו בפרק במה טומנין (לעיל שבת מח, א) (אבל) [גבי] דסתודר אפומא דחביתא אסור, ומאי שנא מפרונקא, התם לא קפיד הכא קפיד.}
\textblock{\textbf{במה דברים אמורים בשלא נשתהא שם אבל נשתהא שם אסור.} פירוש: אמי משרה וימה של סדום קאי, אבל במי טבריא אפילו נשתהא, תדע דהא קתני בברייתא רוחצין במי טבריא ובים הגדול, אבל לא במי משרה, ואי בלא נשתהא אפילו במי משרה נמי.}
\textblock{\textbf{ים הגדול אים הגדול לא קשיא, הא ביפין שבו הא ברעים שבו.} ומסתברא לי דמילתא פסיקתא קתני, יפין שבו אפילו נשתהא, רעין שבו אסור אפילו לא נשתהא, מדלא הזכיר כאן שהוי כלל, והא דקתני בברייתא קמייתא לא בים הגדול ולא במי משרה, לאו למימרא שיהא זה כזה ותרוייהו בדאשתהי, אלא תרוייהו לאיסורי האי כדיניה והאי כדיניה. ויש מפרשים דוקא בדאישתהי מדתני לה בהדי מי משרה, ולפי דבריהם יש לי לומר דמה שלא הזכיר כאן אשתהי ולא אשתהי, משום דסמיך אמאי דאמרן מעיקרא דאוקים ההיא דאין רוחצין בים הגדול דוקא בדאשתהי, ואכתי על ההיא תירוצא קיימא, אלא דאסיקנא השתא דלאו כולא מלתא תליא בשהוי בלחוד, אלא אף ביפין שבו ורעין שבו. וכסברא ראשונה נראה מדברי הריא״ף ז״ל.}
\textblock{ מתני׳:\textbf{ אין אוכלין איזוב יון בשבת.} תוספתא (פי״ג, ה״ז): אין לועסין מצטיכא בשבת, אימתי בזמן שמתכוון לרפואה ואם מפני ריח הפה הרי זה מותר, לא ישוף אדם סם יבש בשיניו בשבת, אימתי בזמן שמתכוון לרפואה, ואם מפני ריח הפה הרי זה מותר.}
\textblock{\textbf{הני מילי היכא דקא מכוון הכא לא מכוון.} כתב רב אחא בשאלתות שלו (פ׳ אמור סי׳ קה): אסור למשתא סמא דעקראתא אפילו מעלי לגופא, דכתיב (ויקרא כב, כד) ובארצכם לא תעשו. ואע״ג דלא מכוון וקיימא לן כרבי שמעון בדבר שאין מתכוון, הני מילי לענין שבת, אבל במילי דעלמא כר׳ יהודה קיימא לן. ולדבריו הא דקאמר הני מילי היכא דקא מכוון הכא לא מכוון, פירושו ומותר כרבי שמעון, דהא בפלוגתא דרבי שמעון ור׳ יהודה שייכא.}
\textblock{\textbf{והקשו עליו בתוס׳ דודאי כל מאן דאית ליה דבר שאינו מתכוון מותר בכל מילי אית ליה הכין, לא שנא שבת ולא שנא מילי דעלמא. ותדע לך מדאמר רבי יוחנן בפרק המוציא יין (שבת פא, ב) אסור לקנח בחרס בשבת, וקא מפרש עולא טעמא התם משום השרת נימין ואע״ג דלא מיכוין אסור, ואקשינן ומי אמר ר״י הכין, הא אמר ר״י הלכה כסתם משנה ותנן נזיר חופף ומפספס אבל לא סורק, ומאי קושיא דלמא בעלמא אית ליה דבר שאין מתכוון מותר אבל לענין שבת לא, ובבכורות      } פרק כל פסולי המוקדשין (בכורות לד, א) גבי מקיז דם לבכור, אמר רב יהודה אמר שמואל הלכה כרבי שמעון, ופריך אטו עד השתא לא אשמעינן שמואל דבר שאין מתכוון מותר, ואם איתא, מאי קא פריך, עד השתא אשמעינן במילי דשבת, והשתא אשמעינן אפילו בשאר מילי.\par \textbf{} ולדידי נמי קשיא, דאם איתא דהא בפלוגתא דרבי יהודה ורבי שמעון בדבר שאין מתכוון שייכא, כי קאמר דא״ר יוחנן הרוצה שיסרס תרנגול יטול כרבלתו, הוה ליה למימר דאמר ר׳ יוחנן הלכה כסתם משנה, ותנן נזיר חופף ומפספס, ואפילו תמצי לומר דניחא לי לאתויי טפי הא דתרנגול דאיירי בסרוס ממש, אכתי מאי קא מייתי מההיא, דההיא אפילו במתכוון היא, וכדאמרינן הרוצה שיסרס תרנגול. ועוד כי דחי לה רב אשי מינה ואמר, והאמר ר׳ יוחנן רמות רוחא הוא דנקיט ליה, ליהדר וליתי׳ ראיה מאידך דרבי יוחנן, דאמר הלכה כסתם משנה ותנן נזיר חופף ומפספס.\par \textbf{} ובתוס׳ אמרו דהא לאו בדרבי שמעון ורבי יהודה שייכא, דבהא אפילו רבי שמעון מודה בה משום דפסיק רישיה ולא ימות הוא, ואפילו לפירוש הערוך (ערך פסק) דאמר דאפילו בפסיק רישיה פליג רבי שמעון בכל שאינו נהנה ממנו, וכדמוכח ההיא דלעיל דריש פרק הבונה (לעיל שבת קג, א) גבי תולש עולשין, דאקשינן והא אביי ורבא דאמרי תרווייהו מודה רבי שמעון בפסיק רישיה ולא ימות, ופרקינן התם דעביד בארעא דלאו דידיה, איכא למימר דהני מילי לענין שבת משום דכתיב ביה מלאכת מחשבת, אבל במילי דעלמא כל שפסיק רישיה ולא ימות מודה ביה רבי שמעון. ותדע לך מדפריך תלמודא בזבחים בפרק כל התדיר (זבחים צא, ב) גבי מתנדב יין מזלפו על גבי האשים, והא מכבה, ואע״ג דהתם לא מכוון לכבוי ולא ניחא ליה.\par \textbf{} ומכל מקום אכתי קשיא לי, מאי קא מייתי מדרבי יוחנן דאמר הרוצה שיסרס תרנגול דהתם הא מכוון. ויש לומר דלרבותא נקטיה דאיהו סבירא ליה דלא אסרה תורה אלא בנוגע ממש במקום סירוס, וכדכתיב (ויקרא כג, כד) ומעוך וכתות ונתוק וכרות, אבל בשאינו נוגע באבר עצמו מותר וכדבר דאתי ממילא היא, והיינו דקאמר הני מילי היכא דקא מכוון ליגע באבר אבל היכא דלא מכוון ליגע באבר עצמו לא כדר״י ואף על גב דמתכוון לסרוסי וכל שכן הכא דלא מכוון.}
\clearpage
\newsection{דף קיא}
\textblock{\textbf{אלא בזקן.} קשיא לי ומי גרע זקן מסריס. ואיכא למימר דאין הכי נמי דמסרס אחר מסרס גזירת הכתוב הוא, וטעמא דקרא משום דכל שכיוצא בו כלומר בחור שכמותו מוליד וזה מסרסו אין ניכר שהיה סריס מתחילתו ונראה כמסרסו עכשיו ולפיכך אסריס הקפיד הכתוב, אבל זקן שאין ראוי להוליד לפי דרכן של בריות שוב לא הקפידה התורה בסירוס דידיה.}
\textblock{\textbf{בזקנה אי נמי בעקרה.} כלומר: דליכא משום ביטול פריה ורביה, ומשום סירוס נמי ליכא דבאשה ליכא משום סירוס, דאדם [זכר] הוא דאסר רחמנא.}
\textblock{\textbf{כל חייבי טבילות טובלין כדרכן בין בתשעה באב בין ביום הכיפורים.} כתבו בתוס׳ דטובלין אפילו למ״ד טבילה בזמנה לאו מצוה, דהא רבי יוסי ברבי יהודה לית ליה טבילה בזמנה, כדאיתא בנדה (ל, א) בשמעתא דטועה וכדאמרינן נמי לקמן בפרק כל כתבי הקדש (שבת קכא, א), ואפילו הכי קתני בברייתא (שם) הזב והזבה והמצורע והמצורעת ובועל נדה וטמא מת טבילתן ביום נדה ויולדת טבילתן בלילה בעל קרי טובל והולך כל היום כולו ר׳ יוסי אמר מן המנחה ולמעלה אינו צריך לטבול, כלומר ביום הכפורים, אלמא קודם המנחה טובל, ואפילו מן המנחה ולמעלה אינו צריך לטבול קאמר אבל לא יטבול לא קאמר. וטעמא דמלתא, דאף על גב דאסור ברחיצה ואפילו להושיט אצבעו במים, הני מילי שלא לצורך אבל לצורך מותר וכדאמרינן (יומא עז, ב) מי שיש לו חטטין בראשו סך כדרכו ואינו חושש, ואמרינן נמי (שם) שומרי פירות עוברין עד צוארן במים.}
\textblock{\textbf{למימרא דרב כר׳ שמעון סבירא ליה והא אמר רב שימי וכו׳.} ואיכא למידק מאי קושיא, אטו מאן דאית ליה בחדא כר׳ שמעון, מי אית ליה בכולה שבת כותיה, והא שמואל אית ליה כותיה בדבר שאין מתכוון, ובמלאכה שאינה צריכה לגופה לית ליה כותיה, כדאיתא בפרק כירה (שבת מב, א) גבי המיחם שפינהו, ואמוראי טובא דסבירא להו כר׳ שמעון בחדא ובחדא כר׳ יהודה. ויש מפרשים דקבלה היתה בידם כל מאן דאית ליה כר׳ שמעון בהא אית ליה בכל קולי שבת כותיה. ואינו מתיישב על הלב. ובתוס׳ פירשו, דר׳ שמעון שרי מטעם דבר שאין מתכוון, ותנא קמא אסר אפילו מתכוון לתענוג ולא לרפואה, וכי קאמר ר׳ שמעון כל ישראל בני מלכים הם, לא מחמת מעלתם קאמר, אלא כלפי דאסר תנא קמא לכולי עלמא חוץ מבני מלכים, קאמר ליה לדבר זה כל ישראל מותרים בו כבני מלכים דכל אדם סך בו לתענוג, ולפיכך פריך ליה מהא דדבר שאין מתכוון. ואינו מחוור בעיני, דהא קתני במתניתין בני מלכים סכין שמן ורד על מכותיהן, ועלה קאמר ר׳ שמעון כל ישראל בני מלכים דאלמא אפילו במתכוון לרפואה שרי ליה ר׳ שמעון. ועוד תנן בפרק במה אשה יוצאה (שבת סו, ב) ובני מלכים בזוגין ולא בני מלכים בלבד אמרו אלא אפילו כל אדם אלא שדברו חכמים בהוה, ואמרינן עלה בגמרא (סז, א) מאן תנא רבי שמעון היא דאמר כל ישראל בני מלכים, דאלמא טעמא דרבי שמעון משום דכל ישראל ראוין לכך מחמת מעלתם קאמר ולא משום דבר שאין מתכוון. ובספר המאור כתוב דהכי קים להו דבכל השבת כולה לא סבר לה כר׳ שמעון, ע״כ, כלומר ומשום הכי קא מתמה ומי סבר לה כלל כר׳ שמעון.}
\textblock{\textbf{האי מסוכרייא דנזייאתא אסור להדוקיה ביומא טבא.} פירש רש״י ז״ל: העץ שתוחבין בברזא שבצדי החבית כורכין עליו מטלית או נעורת של פשתן וכשמהדקו בברזא אתי לידי סחיטה, וכדאסרינן לקמן (שבת קמא, א) להדוקי אודרא אפומא דשישא. ור״ת ז״ל כתב בס׳ הישר (סי׳ ר״ל) דאין האיסור משום סחיטה, דאם כן הוה ליה למימר לא ליהדק דילמא אתי לידי סחיטה, וכדאמרינן התם לא ליהדוק אודרא אפומא דשישא דלמא אתי לידי סחיטה, אלא טעמא הכא משום דלמא מבטל ליה ומשוי ליה דופן, כדאמרינן (לקמן שבת קכה, ב) כיון דהדקיה שוייה דופן לקרוייה.\par \textbf{} ועוד דתרי גווני סחיטה נינהו חד משום ליבון וחד משום מפרק בידים ואינו משום לבון, דדרכן של בני אדם ללבן (ביין) ולכבס כליהם במים, וההיא בין אזלי לאיבוד מים הנסחטים בין דלא אזלי לאיבוד אסור משום סחיטה, והיינו דאסר רבה לההוא דסדר דסתודר אפומא דחביתא כדאיתא בריש פרק במה טומנין (לעיל שבת מח, א) וההיא במים היא, והיינו נמי דאמרינן בפרק אלו קשרים (לקמן שבת קיג, ב) ליעבר זימנין דמיתווסן מאניה במיא ואתיא לידי סחיטה, אבל ביין ליכא משום סחיטה כדאמר בפרק תולין (לקמן שבת קלט, ב) מסננין את היין בסודרין, דאלמא לא חיישינן דלמא אתי לידי סחיטה כיון דאזיל לאיבוד, ואי משום ליבון לא מלבני אינשי כלים בשאר משקין, דאדרבא מתלכלכין ולעולם איכא ריחא וחזותא. אבל כשאין הדבר הנסחט הולך לאיבוד איכא משום מפרק, והיינו דאסרינן לקמן בפרק תולין (שבת קמא, א) להדוקי אודרא אפומא דשישא דלמא אתי לידי סחיטה והשמן הנסחט נופל לתוך הכלי.\par \textbf{} והרב בעל הערוך ז״ל (ערך סבר וערך פסק) גם כן כתב כן, דכל היכא דאזיל לאיבוד דלא ניחא ליה אף על גב דהוה פסיק רישיה שרי, והביא ראיה מההיא דאמרינן לעיל בפרק הבונה (שבת קג, א) לא צריכה דעביד בארעא דלאו דיליה, ועוד מדאמרינן בפרק לולב הגזול (סוכה לג, ב) גבי אם היו ענביו מרובין מעליו וכו׳ לא ילקט, דאי אית ליה הושענא אחריתי שרי. ועוד בפרק במה מדליקין (לעיל שבת כט, ב) מוכרי כסות מוכרין כדרכן ואין חוששין משום כלאים, ועוד ביומא (לד, ב) דאמרינן עששיות של ברזל מטילין לתוך הצונן ואף על גב דמצרף, וכל הני פסיק רישיה נינהו. ומשום הכי פירש הוא ז״ל משום דלמא מבטל לה ומשוי ליה דופן, כמו שפירש רבנו תם ז״ל. ואי נמי דהא מסוכריא הויא אפומא דנזייאתא ומשום סחיטה, כההיא דאודרא אפומא דשישא, ובריש פרק קמא דכתובות (ו, א) כתבתיה בארוכה בסייעתא דשמיא.}
\textblock{ הא ד\textbf{בעי רב אחדבוי עניבה לר׳ מאיר מאי.} ולא איפשיטא. תימה אמאי לא פשטוה מההיא דתניא לקמן (שבת קיג, ב) חבל דלי לא יהא קושרו אלא עונבו, וההיא ר״מ היא מדפריך מינה בפרק קמא דפסחים (יא, א) לרבנן דר׳ יהודה דהיינו ר״מ, גבי ר׳ יהודה אומר בודקין אור י״ד.}
\newchap{פרק \hebrewnumeral{15} ואלו קשרים}
\clearpage
\newsection{דף קיב}
\textblock{}
\textblock{ הא דאמרינן:\textbf{ פטור אבל אסור קא קשיא לי.} יש מפרשים אפילו אי אמרת לי פטור אבל אסור הוה קשיא לי. ואינו מחוור, דהוה ליה למימר פטור אבל אסור קא מבעיא לי, אלא נראה דאברייתא קאי, והכי פירושא, פטור אבל אסור דקתני בברייתא דלעיל קשיא לי דלא אשכחת ליה טעמא דלא ידענא היכי אוקמיא, דאי בכי האי קשיא לי אמאי, אבל ברייתא דקתני חייב חטאת שפיר מתוקמא ליה בקיטרא דעבדי אושכפי.}
\textblock{\textbf{מאי שנא מדרב ירמיה התם מינטר הכא לא מינטר.} איכא למידק דודאי מדקאמר ליה מאי שנא מדר׳ ירמיה משמע דבחד גוונא מיירי, ואם כן אי לאו מנא הוא רבי ירמיה      שרי לטלטולה ואפילו בכרמלית דלא מינטר, דהא לא שרו טלטול במקום פסידא. ועוד אמאי לא שרא ליה ואע״ג דלאו מנא הוא, דהא כיון דאפסיק השתא בשבת ואכתי חזי למלאכה קצת דהא חזי לכסויי בה מנא ולמסמך בה כרעי המטה, ותנן (לקמן שבת קכד,ב) כל הכלים הניטלין בשבת שבריהן ניטלין ובלבד שיהו עושין מעין מלאכה. ואי אפשר לומר דאינהו כר״י סבירא להו דבעי מעין מלאכתו הראשונה, דאם כן רבי אבהו היכי שרי אפילו בכרמלית, ועוד דאי אפשר דסבירא ליה בהא כר״י דהא טעמא דרבי יהודה התם משום נולד, ואנן לא קיימא לן כותיה בנולד בשבת.\par \textbf{} וכתב הראב״ד ז״ל דאפשר דר׳ יוסף סבירא ליה כר׳ יהודה, ור׳ יוחנן סבירא ליה הכין כדאיתא בשמעתין מדמתרץ לה אליבא דר׳ יהודה כו׳, ור׳ אבהו לא סבירא ליה כוותיה והיינו דשרא ליה לר׳ ירמיה. ואינו מחוור, דהא משמע דר׳ יוסף ור׳ אבהו לא אפליגו במידי בהא, מדקא מקשה ליה מאי שנא מדרבי ירמיה וקא מהדר ליה התם מינטר והכא לא מינטר, ולא אמר ליה איהו סבר מנא ואנא סברי לאו מנא. ועוד תירץ הראב״ד ז״ל דרב יוסף ור׳ אבהו סבירא להו כר׳ יהודה דוקא בסנדל שנפסקה אחד מאזניו דלא מבטיל ליה איניש ממלאכתו בהכי לחודיה לכסויי ביה מנא או לסמוך בו כרעי המטה אלא דעתו לתקוני ולאהדורי למלאכתו, הלכך כיון דלא חזי ליה לנועלה לכולה שבת מוקצה הוא, ומשום הכי אמר ליה רב יוסף שבקיה דלאו מנא הוא, ודוקא בחצר דמינטר דמקצה ליה איניש מדעתיה עד דמתקן ליה, אבל בכרמלית דלא מינטר לא מקצה ליה איניש אלא דעתיה עליה לטלטלה על ידי גמי לח.}
\textblock{\textbf{אימר דאמרינן חלצה בשל שמאל בימין חליצתה כשרה היכא דלמילתיה מנא הוא הכא למילתיה לאו מנא הוא.} ואיכא למידק דאם כן מאי קא מקשה כי מוקמינן לה כרבנן אבל לא לחליצה דלאו מנא הוא, והא תנן חלצה בשל שמאל בימין וכו׳, לימא התם משום דלמילתיה מנא הכא למילתא לאו מנא הוא, דאפילו לרבנן לא משוו ליה מנא, אלא משום דחזי לאפוכא משמאל לימין, אלמא למילתא דהיינו לשמאל לאו מנא הוא. ויש לומר דלרבנן כיון דחזי לאפוכא לימין ולכך קאי ולא לשמאל שפיר חזי לחליצה, דהשתא של שמאל דקאי לשמאל אפילו הכי חלצה בה בימין חליצתה כשרה, האי דכל עצמו לא הוי מנא לרבנן אלא משום דחזי לימין וקאי להפוכא לימין לא כל שכן, אבל לר׳ יהודה דמשום דאפשר להפוכא לימין לא משוי ליה מנא עד דמתקן ליה ומהפך לה ממש, איכא למימר הכין, דכיון דאיהו השתא לאו מיניה למילתיה ולא קאי להפוכי, אע״ג דאפשר דמתקן ליה ומהפך ליה לימין השתא מיהא לאו מנא.}
\textblock{\textbf{אבל טמא מגע מדרס.} פירש רש״י ז״ל: מחמת שנגע בעצמו כשהיה אב הטומאה. ואינו מחוור, חדא דהא במסכת מנחות פרק הקומץ רבה (מנחות כד, א) גבי שתי מנחות שלא נקמצו, אבעיא ליה לרבה עשרון שחלקו ונטמא אחד מהם והניחו בכסא וחזר טבול יום ונגע באותו טמא מהו, מי אמרינן שבע ליה טומאה או לא, ולא פשטוה מהא, דלא אמרינן שבע ליה טומאה. ועוד דהתם אתיא למיפשטא מדתנן סדין שנטמא מדרס ועשאו וילון, טהור מן המדרס אבל טמא מגע מדרס, אמר ר׳ יוסי וכי באיזה מדרס נגע זה, אלא שאם נגע בו הזב טמא מגע הזב, כי יגע בו הזב מיהא ואפילו לבסוף טמא מדרס ומגע הזב, אמאי לימא שבע ליה טומאה אלמא לא מטמיא ליה מגע מדרס משום מדרס הראשון ומשום דנגע בעצמו.}
\textblock{\textbf{ועוד דאתי נמי למיפשטא מסיפא דההיא, דקתני מודה ר׳ יוסי בשני סדינין המקופלין ומונחין זה על גב זה וישב זב עליהן שהעליון טמא מדרס והתחתון טמא מגע מדרס, אלמא       } התחתון בלחוד הוא דאמרינן דטמא מגע מדרס, משום דנגע בעליון, אבל עליון לא טמא מגע מדרס משום דנגע בעצמו וגם לא משום דנגע בתחתון, דאלמא בעליון אמרינן שבע ליה טומאה, וטעמא דמלתא משום דאפילו לרבא לא איבעיא ליה היכא דמגעו לא מהניא מידי בההיא שעתא דנגע בו, דכבר היה טמא מדרס, דעד כאן לא איבעיא ליה אלא בעשרון שחלקו ונטמא חציו והניחו בכסא וחזר טבול יום ונגע בטמא דמהניא נגיעתו לטמא חציו הטהור, ובכי הא הוא דאיסתפקא ליה דלמא לא אמרינן שבע ליה טומאה, והלכך בסדין העליון לא אפשר לטמויי מגע מדרס מחמת נגיעת עצמו ולא מחמת נגיעתו בתחתון, דטומאת מגעו לא מהניא מידי, אבל התחתון שהיה טהור ובאה לו בהדדי טומאת מדרסו ומגעו בסדין העליון, היה סבירא ליה דלא אמרינן ביה שבע ליה טומאה, ומכל מקום למדנו דלכולי עלמא אמרינן שבע ליה טומאה לגבי מגע עצמו.\par \textbf{} ובתוס׳ פירשו, טמא מגע מדרס מחמת האזן הראשונה שהיתה טהורה לגמרי, ונטמאת במגע מחמת חבורה, ואף על גב דגבי מדרס מיטמא מחמת חבור הסנדל דהוי כסנדל עצמו, מכל מקום מקבלת טומאת מגע דלא שייך לומר שבע ליה טומאה הואיל וטומאת מדרס דידה לאו מחמת עצמה באה לו אלא מחמת שנתחברה בסנדל וחזרה להיות כסנדל, ועוד דבהדדי קא אתו לה, וכדבעי׳ למימר התם גבי ההיא דשני סדינין דהתחתון טמא מדרס ומגע מדרס משום דבבת אחת אתו ליה, ואגב אזן דחיילי עליה שתיהם בבת אחת חייל נמי אסנדל, כיון דאזן וסנדל הכל כלי אחד.\par \textbf{} ובתוס׳ עירובין (כד, א ד״ה אבל) מצאתי תירוץ אחר בשם רבינו שמשון ז״ל, שבדבר המקבל מגע מעצמו לא שייך למימר שבע ליה טומאה, דכיון דמגופיה אתיא ליה טומאה אלימא למיחל טפי מטומאה דאתיא ליה מעלמא, ולהכי קיימא טומאת מגע מדרס דסנדל כל שעתא ושעתא, וכן בשר דאבר מן החי דאיתא בפרק העור והרוטב (חולין קכט, ב) דאמרינן התם גבי חתך בשר מאבר מן החי, חשב עליו ואח״כ חתכו טמא טומאת אוכלין, שאחר שנעשה אוכל במחשבתו קבל טומאה מטמאה, וכר״מ דאמר בית הסתרים מטמא, ולא אמרינן שבע ליה טומאה דאבר מן החי. וצריך לי תלמוד, דאם כן אף סדין העליון יהא טמא אף מגע מדרס מחמת נגיעתו בעצמו.}
\clearpage
\newsection{דף קיג}
\textblock{ מתני׳:\textbf{ קושרין דלי בפסיקיא אבל לא בחבל.} כתבו בתוס׳ דלא מיירי בדליים שלנו שאינם קבועים בבור, אלא בדליים הקבועים, אבל שלנו אפילו חבל דעלמא דלאו דגרדי אינו קשר של קיימא כיון דאינו עשוי לעמוד שם זמן מרובה.}
\textblock{\textbf{כלל אמר ר׳ יהודה כל קשר שאינו של קיימא אין חייבין עליו.} איכא למידק מדקתני אין חייבין עליו משמע הא איסורא איכא, ואם כן היכי קתני ר״י מתיר דמשמע לכתחילה, ועוד דהא אפילו רבנן לא מחייבי ביה חטאת אלא מיסר אסרי. ויש לומר דר״י לאו אדרבנן קאי אלא מילתא באפי נפשה היא, ונקט האי לישנא לומר הא של קיימא חייבין ואפילו בעניבה וכדאמרינן בגמרא דלר״י עניבה גופא קשירה היא, ולאפוקי מדר״מ אתי דאמר עניבה לאו קשירה היא.}
\textblock{ גמרא:\textbf{ אם היו מקצת עלין שלו מגולין וכו׳.} כך גריס רש״י ז״ל: והגאונים ז״ל אינם גורסין אלא אם היה מקצת עליון שלו מגולה, וכבר כתבתיה בארוכה שלהי פרק במה טומנין (לעיל שבת נ, ב) בסייעתא דשמיא.}
\textblock{\textbf{שלא יהא דבורך של שבת כדבורך של חול.} פירש רש״י ז״ל: כגון מקח וממכר וחשבונות. והקשו עליו דהיינו ממצוא חפצך. אבל בירושלמי (דפרקין ה״ג) אמרינן שאפילו דבור בעלמא לרבות בו כדרך שהוא מרבה לדבר בחול אסור, דגרסינן התם: אמר ר׳ חנינא בדוחק התירו שאלת שלום בשבת, אמר ר״ש בר אבא ר״ש בן יוחי כד הוי      לאמיה משתעי סגיין הוה אמר לה אימא שבתא היא, תני אסור לשאול צרכיו בשבת, ר׳ זעירא שאל לר׳ חייא בר אבא מהו למימר רוענו זוננו פרנסנו, א״ל טופס ברכה כך הוא.}
\textblock{\textbf{וליתקע כי היכי דלידעי דחלבי שבת קרבין ביום הכיפורים.} כלומר: ליתקע כשחל יום הכיפורים בערב שבת, דהא מקילתא לחמירתא תקעינן כי היכי דלבטלו ולא ליקרבי חלבי יום הכיפורים בשבת, ומינה ידעי דחלבי שבת קרבין ביום הכיפורים, דיום הכיפורים קיל משבת. ואם תאמר אמאי לא אמר כי היכי דלידעו דאין חלבי יום הכיפורים קרבין בשבת. יש לומר דאי משום הא לא הוו צריכי למיתקע, דכי לא תקעו נמי אמרינן משום ששניהם שוין הוא ואין של זה קרב בזה.\par \textbf{} ואם תאמר לימא וליבדיל כלומר כשחל במוצאי שבת כי היכי דלידעו דחלבי שבת קרבין ביום הכיפורים, והא עדיפא טפי דהוה קרוב. ויש לומר [דבזה] ליכא פרסום לרבים דכל חד וחד מבדיל בתוך ביתו וליכא פרסום, מה שאין כן בתקיעה. ותדע לך, מדאמרינן לקמן ושבות קרובה התירו, והתנן יום טוב שחל להיות במוצאי שבת מבדילין ולא תוקעין, ואמאי ליתקע כי היכי דלידעו דשרי בשחיטה לאלתר, ואם איתא אמאי והא מבדילין, אלא שמע מינה דליכא הוכחה אלא בתקיעה.\par \textbf{} ואם תאמר היכי קאמר הכא ליתקע כי היכי דלידעי דחלבי שבת קרבין ביום הכיפורים, כלומר לשנה אי מקלע למוצאי שבת, והא שבות רחוקה היא ושבות רחוקה לא התירו. ואפשר היה לומר דהכא כדינו תוקעין, דמקילתא לחמירתא תוקעין, וממילא ידעי דכי מיקלע במוצאי שבת חלבי שבת קרבין בו. ואם תאמר לימא אין דוחין שבות להתיר, כמו שאמרו לקמן בסמוך. גם בזה יש לומר כמו שתרצנו דמקילתא לחמירתא תקעינן, ויש עוד לתרץ דלא רמי אנפשיה הכין, אלא מכי משנינן לקמן אין דוחין שבות להתיר.}
\textblock{\textbf{ושבות קרובה התירו והתנן יום טוב שחל להיות במוצאי שבת מבדילין ולא תוקעין, ואמאי ליתקע כי היכי דלידעי דשרי בשחיטה לאלתר.} פירוש: ולאו למימרא דבשאר מוצאי שבת תוקעין להתיר את העם במלאכה, דהא קתני רישא (חולין כו, ב) כל מקום שיש הבדלה אין תקיעה, ובברייתא דפרק במה מדליקין (שבת לה, ב) לא תני אלא תקיעות דערב שבת, ואם איתא לא לישתמיט תנא דליתני בשום דוכתא תקיעה דמוצאי שבת. אלא הכא משום דראוי לתקוע כי היכי דלידעו דשרי בשחיטה כדי שיתעסקו בשמחת יום טוב, אבל במלאכת חול מאי איכפת לן אי עבדי לאלתר אי לא עבדי דנתקע כי היכי דלידעו דשרי במלאכה. אבל הרמב״ם ז״ל (פ״ה, ה״כ) למד מכאן שתוקעין בכל מוצאי שבת להתיר במלאכה, ואינו מחוור כדאמרן.}
\textblock{\textbf{יום הכיפורים שחל להיות בשבת מותר בקניבת ירק.} פירש רש״י ז״ל: להסיר העלין הרעים. ואינו מחוור, דאם כן היינו בורר. ובתוס׳ נראה שפירשו לחתוך הירק. ולא דקדקו, דאם כן הוי ליה טוחן. אבל בירושלמי (דפרקין הלכה ג׳) פירשו הדחת ירק, שכן אמרו שם ויבדיל שכן הוא מותר להדיח כבשין ושלקין, וכן שנו בתוספתא (פי״ג, הי״ח) ופירוש עגמת נפש שחששו לעגמת נפש, והתירו עכשיו לקנב כדי שלא יצטרך לערב לתקן את הכל וירעב ותהיה נפשו עגומה, ורש״י ז״ל לא פירש כן, וזה נכון. ומה שהתירו מן המנחה ולמעלה, יש מפרשים משום דקודם לכן נראה כמתקן לצורך היום, אבל מן המנחה ולמעלה דרכן של בני אדם לתקן מאכלן לצורך הערב. ויש מפרשים דקודם המנחה נפשו מתאוה לאכול וחוששין דלמא אתי למיכל, אבל מן המנחה ולמעלה דומה כמי שיש לו פת בסלו ואינו מתאוה לאכול ולא חיישינן דלמא אתי למיכל.}
\clearpage
\newsection{דף קטו}
\textblock{ מתני׳:\textbf{ כל כתבי הקדש מצילין אותן מפני הדליקה.} פירוש: אבל לא דבר אחר, וכתבו בתוס׳ בשם ר״ת ז״ל דדוקא בחצר שנפלה שם הדליקה, דאיכא למיגזר דלמא אתי לכבויי, אבל בחצר אחרת מותר להציל כל דבר דליכא למיחש לכבויי.}
\textblock{\textbf{מצילין מפני הדליקה.} פירוש: אבל לא לכבות, ובירושלמי (דפרקין, ה״א) אמרו שאם אינו יכול להציל אלא בכבוי מציל לר״ש דאמר מלאכה שאינה צריכה לגופה פטור עליה, ואע״ג דמשום שבות איכא, אין לך דבר של שבות עומד בפני כתבי הקדש.}
\textblock{}
\textblock{ גמרא:\textbf{ מאי לאו קורין בהם נביאים ואין קורין בהם כתובים.} קשיא לי למה ליה למימר הכי, והא עיקר קושיא ליתא, אלא מדקתני אף על פי שכתובין בכל לשון. ומדוחק יש לי לומר דמשום דלרב הונא איצטריך לפרושי כי היכי דלא תקשי לך דהא קתני בהדיא בין שקורין בהן בין שאין קורין בהן, דאלמא אף על פי שלא נתנו לקרות בהן מצילין, והכי קאמר מדקתני אין קורין בהן, לא קשיא דמאי אין קורין כתובים, אבל אף על פי שכתובין בכל לשון, קשיא.}
\textblock{\textbf{רב הונא אמר אין מצילין דהא לא ניתנו לקרות בהן.} ופסק הרב אלפסי ז״ל כרב הונא דרב חסדא תלמידו הוה. והר״ז הלוי כתב דעכשיו אי אפשר לומר (ד)[ב]תרגום כן, דהא ניתן ליקרות בו וכן הברכות, משום דבציר ליבא ושרי משום עת לעשות לה׳ הפרו תורתך, וכדאמרינן (גיטין ס, א. תמורה יד, ב) ר׳ יוחנן וריש לקיש מעייני בספרא דאגדתא בשבת, משום עת לעשות לה׳. ואף הרמב״ן ז״ל כתב כן בתרגום דילן. והרב בעל ההלכות ז״ל כתב דמצילין ספרא דאפטרתא מפני הדליקה, ושרי לטלטולי ולמקרא ביה מדר״י, ובמסכת סופרים (פרק י״ז הלכה א׳) אמר דמצילין ספרא דאגדתא. ומכל מקום כל הכתובין ביוונית מצילין, דהא לכולי עלמא לעז יווני כשר, ותנן (מגילה ח, ב) רבן שמעון בן גמליאל אומר אף ספרים לא התירו לכתוב אלא בלעז יווני. והקשה הראב״ד ז״ל דהא תניא גפטית לגפטיים מדי למדיים, דאלמא אף כתובין בכל לשון מותרין לקרוא בהן הלועזות ולמה לא יצילו. ותירץ דבמקומן מצילין אותן.}
\textblock{ה״ג רש״י ז״ל וכן היא במקצת הספרים:\textbf{ מאי תניא דתני׳ אין בין ספרים למגילה, אלא שהן נכתבין בכל לשון ומגילה עד שתהא כתובה אשורית ובדיו.} ופירש הוא ז״ל: דאלו דברים יש בין ספרים למגילה, שהמגילה צריכה שתהא כתובה אשורית ובדיו, אבל בספרים אע״פ שכתובין בכל לשון ושלא בדיו כשרין. ואינו נכון, דהא ריש גלותא ודאי לא מבעיא ליה היכא דנתנו ליקרות בהן דלכולי עלמא מצילין כדאמרינן לעיל, ועוד דלישנא דקאמר תבעי למאן דאמר מצילין, תבעי למאן דאמר אין מצילין קא דייק הכי, ועוד דהא בהדיא תניא בפרק הבונה (שבת קג, ב) כתבה כשירה או שכתב את השירה כיוצא בה ושלא בדיו הרי אלו יגנזו, אלמא דיו בעיא. ועוד דהא כולהו מכתיבה כתיבה גמירי, כדאיתא במגילה (יט, א) דאמרינן כתיב הכא (וכתבתם) [ותכתב אסתר המלכה (אסתר ט, כט)] וכתיב התם (ירמיה לו, יט) ואני כותב (בספר) [על הספר] (ו)בדיו, אלמא כל היכא דכתיב ביה כתיבה צריך דיו, ובמסכת סופרים (ריש פ״א) אמרו כן בהדיא.\par ובתוס׳ רצו להעמיד הגירסא ואמרו דלא לענין הכשר קריאה תניא אלא לענין הצלה, כלומר אין ביניהם למגילה לענין הצלה, אלא ששאר הספרים ניצולין אע״פ שאינן כתובין אשורית ובדיו, ומגילה אין מצילין אותה עד שתהא אשורית ובדיו, וטעמא משום דשאר ספרים לפי שיש בהן הזכרות, אבל מגילה דלא כתיבה בה הזכרות לא עד שתהא כתובה כהלכתה. וקשיא לי דלפי דבריהם לא מוכח מינה אלא למאן דאמר מצילין, דהא תניא אלא שהן נכתבין בכל לשון, דאלמא אף בכל לשון מצילין, ואם כן רבה ב״ר הונא היכי קאמר ליה אי תניא תניא, דמי איכא למימר דכר׳ חסדא סבירא ליה ולא כאבוה. ויש לי לתרץ דכיון דאיהו בעי אפילו למאן דאמר מצילין, ואכולה אהדר ליה רבה בר רב הונא אין מצילין, כלפי הא אמר ליה ריש גלותא והא תניא מצילין ונקט מינה מיהא חדא דלמאן דאמר מצילין בהא נמי מצילין, ואמר ליה רבה אי תניא תניא, דלמאן דאמר מצילין בהא נמי      מצילין. ומיהו אכתי לא מחוור, דלישנא דברייתא לא משמע דמיירי לענין הצלה אלא לענין הכשר כתיבה. על כן נראה לי דלא גרסינן לה כלל, והא דרב המנונא דאמר תנא מצילין ברייתא היא דמפורשת בתוספתא (פי״ד, ה״ג).}
\textblock{הא דאמרינן:\textbf{ הני מילי בדיו דמקיים אבל הכא כיון דלא מקיים לא.} קשיא לי דהני נמי מקיימו, שהרי שנינו למעלה בפרק הבונה (שבת קד, ב) דהכותב בשבת באחת מאלו חייב, וטעמא לפי שהיא מלאכה המתקיימת, ובהדיא תניא עלה בתוספתא (פי״ב, ה״ו) זה הכלל כתב דבר של קיימא בדבר שאינו של קיימא, או דבר שאינו של קיימא בדבר של קיימא פטור, עד שיכתוב דבר של קיימא בדבר של קיימא, אלמא הני נמי מקיימי. ויש לומר דמקיימי קצת עד שדרכן של בני אדם לכתוב בהן דברים שאינן עשוין לקיימן לעולם, אלא זמן אחד כספרי הזכרונות וכיוצא בהן, והלכך לענין שבת מלאכת מחשבת היא, אבל לענין ספרים דבעינן דבר המתקיים לעולם, הני לא מקיימי, ומקיים ולא מקיים לענין ספרים קאמר. כך נראה לי.}
\textblock{ הא ד\textbf{בעא מיניה רב הונא בר חלוב מרב נחמן ספר תורה שאין בו ללקט שמונים וחמש אותיות, כגון פרשת ויהי בנסוע הארון, מצילין אותו מפני הדליקה.} תמיהא לי אמאי לא פשטה מדקתני בברייתא בהדיא, ספר תורה שבלה אם יש בו ללקט פ״ה אותיות כפרשת ויהי בנסוע הארון מצילין פחות מכן אין מצילין. ועוד דמלישנא דבעיא משמע דאיהו הוה ידע לה לברייתא, דהא נקט לישנא דברייתא ספר תורה שאין בו ללקט כפרשת ויהי בנסוע וכו׳. ויש לומר דאההיא [ברייתא] קאי והלכתא קא בעי מיניה אי ההיא ברייתא או לא ומשום דקשיא ליה אידך דקתני תרגום שכתבו מקרא וכו׳ מצילין אותו מפני הדליקה ואין צריך לומר תרגום שבתורה וכדקא פריך ליה מיניה ואזיל, ופריק ליה דאין מצילין, ואי משום ההיא, כי תניא ההיא להשלים.}
\clearpage
\newsection{דף קטז}
\textblock{\textbf{ההוא דגייז ושדי.} קשיא לי ותיפוק ליה מיהא משום גויל שבין שיטה לשיטה, דאי אמרת דגייז נמי ושדי הא לאו ספר הוא זה ואנן ספר תורה שנמחק תנן. ויש לומר דבין שיטה לשיטה הרי הוא כמקום הכתב, שאף הוא לא קדוש אלא אגב הכתב, דאינו מניחו אלא להפריש ולהבדיל בין הכתב שיהא ניכר ונקרא, אלא שבין דף לדף הוא משייר לנוי וליופי ומעצמו הוא קדוש, והיינו דלא אדכר בשום מקום בכלל שאלותיו שבין שיטה לשיטה, אלא שלמעלה ושלמטה שבין פרשה לפרשה שבין דף לדף שבתחילת הספר ושבסוף הספר, אבל שבין שיטה לשיטה לא מיבעיא ליה.}
\textblock{\textbf{רב אשי אמר לעולם כדאמרן ושמואל דאמר כרבי נחמיה.} פירש רש״י ז״ל: והא דפסקי סדרא בנהרדעא בכתובי, היינו משום דכיון דרבנן פליגי עליה דרבי נחמיה דרבים נינהו, הנהיגן הוא כדברי חכמים. אבל הר״ם בר יוסף ז״ל פירש דרבי נחמיה תרתי אית ליה משום ביטול בית המדרש ומשום שטרי הדיוטות והלכך אפילו שלא בזמן בית המדרש אסור משום שטרי הדיוטות, אבל למפטר בבי כנישתא במנחתא שרי, משום דאינהו רגילי לבתר אפטרתא דדרשי בה דבכי הא ליכא משום שטרי הדיוטות וכדתניא אבל שונין בהן ודורשין בהן.}
\textblock{ הא דאמרינן:\textbf{ התם טלטול הכא מלאכה.} תמיהא לי מאי קאמר התם טלטול, דלא הוה ליה למימר אלא התם הוצאה דרבנן והכא מלאכה, דהא לאו משום טלטול הוא אלא מחמת הוצאה למבוי. ואולי משום דטלטול מחמת הוצאה הוא דאסרו כדאיתא בפרק קמא דביצה (דף יב, א) נמצא שהטלטול וההוצאה כענין אחד, ולפיכך אינו מקפיד מלהזכיר ההוצאה בלשון טלטול. ואי נמי מפני שהתחלת ענין זה הוא הטלטול שמטלטלו כדי להוציאו, לפיכך נקט טלטול.}
\textblock{\textbf{פליגי בטלטול ופליגי במלאכה.} פירוש דרבנן סברי דמותר לטלטלו ואפילו לאחר הפשטת כולו, ורבי ישמעאל בנו של ר״י בן ברוקה אסר אפילו לא נפשט אלא עד החזה, ואפילו לטלטלו מחמה לצל. ופירש רש״י ז״ל משום דעור לאו בר טלטול הוא, ולא דמי לשלחין דאמרינן בפרק במה טומנין (שבת מט, ב) שמטלטלין אותם, דהתם עור דבהמה גסה דחזי למשטחיה ולמיזגא עליה, והכי קאמרי ליה רבנן לא נטלטל עור אגב בשר, דבשר ראוי הוא לטלטלו מחמה לצל משום כבוד שמים שלא יסריח, והלכך אף העור נטלטל אגב הבשר כשם שמטלטל תיק עם הספר, ופריק מי דמי התם הספר מותר הוא לטלטלו ולפיכך אף התיק נעשה בסיס לדבר המותר ומטלטל אף להצלה אגב הספר, אבל בשר לר׳ ישמעאל צורך הדיוט הוא והוא עצמו אסור לטלטלו, והלכך אפילו היה העור מותר לטלטלו מצד עצמו, מחמת הבשר היה נאסר לפי שנעשה בסיס לדבר האסור, וכל שכן שהעור נמי אסור לטלטלו וכדאמרינן.\par \textbf{} ואינו מחוור, דאם כן היאך חוזר ואומר אם מטלטלין תיק הספר עם הספר ואע״פ שיש בתוכו מעות לא נטלטל עור אגב בשר, כיון דאסיק דבשר עצמו אסור לטלטלו מצד עצמו. ועוד דמה שכתב הוא ז״ל, דעור בהמה דקה אסור לטלטלו דלא התירו אלא עור בהמה גסה, אינו מחוור, דהא תנן בפרקין (קכ, א) פורשין עור של גדי על גבי שידה תיבה ומגדל מפני שהוא מחרך.}
\textblock{\textbf{והרבה פירושים נאמרו כאן, והנכון מה שפירש בו הרמב״ן ז״ל, דעור מותר היינו שלחין דהא חזי למזגא עלייהו, ובשר אסור לדברי ר׳ ישמעאל דהא לא חזי עד לערב שאינו נאכל אלא בלילה וצלי ולית ביה כבוד שמים בנאכל       } להדיוט, וכמו שפירש רש״י. והשתא לדבריו קאמרי ליה, והכי קאמרי ליה לא נטלטל עור בבשרו וכדקאמרינן (לעיל שבת מז, א) מטלטלינן כנונא אגב קיטמא דפירושו בקיטמיה, כלומר אע״ג דאית ביה קיטמא, והכי נמי אע״פ דאית ביה בשר דאגיד ביה, ואמרינן מי דמי התם העור הוא שנעשה בסיס לבשר ונעשה כמוהו וכיון דבשר אסור אף העור נעשה כמוהו ואסור כמוהו.}
\clearpage
\newsection{דף קיז}
\textblock{\textbf{ותרווייהו אליבא דרבי אליעזר.} איכא למידק מאי דוחקיה דמפיק להו מהלכתא. ויש לפרש משום דקשיא ליה היאך אפשר דבן בתירה מתיר בלא לחי כלל, דהא איכא לאחלופי ברשות הרבים.}
\textblock{\textbf{ועוד לרבנן נציל אוכלין ומשקין.} ולאו כדרך שהוא מציל כתבי הקודש קאמר, דכתבי הקודש מצילין אפילו לשאינו משותף, ואילו אוכלין אינו מציל אלא במשותף דומיא דחצר המעורבת, אלא הכי קאמר כיון דמוקמת ליה במבוי הראוי לשתוף מאי טעמא קתני בסיפא גבי הצלת אוכלין לחצר המעורבת, לימא למבוי המשותף וליפלוג וליתני במבוי עצמו ולימא כל כתבי הקודש מצילין, להיכן מצילין למבוי, ואם היה משותף מציל לתוכו אפילו אוכלין ומשקין. ואי נמי יש לפרש דהכי קאמר, כיון דמוקמת ליה במבוי גמור אלמא אף אתה מצריכו כל הלכות מבוי לומר שישנו אף משותף, ואם כן אף אוכלין ומשקין מציל.}
\textblock{\textbf{אמר ליה אביי לדידך נמי לרבנן נציל לתוכו אוכלין ומשקין.} איכא למידק והא רבה גופיה דקא מותיב הכי לרב חסדא, והיאך הוא לא נשמר ממנה. ועוד קשיא לי, רבה מאי דוחקיה דלא אוקמיה למתניתין כהלכתא וכבית הלל ובשיש לו שלש מחיצות ולחי אחד, דאי משום בן בתירה, השתא בשתי מחיצות ולחי אחד קא שרי לפום מאי דמוקי לה רבה השתא, כל שכן בשלש מחיצות גמורות. ועם מה שתירץ הרמב״ן ז״ל בקושיא הראשונה עלתה לי תירוצא אף לקושיא זו, שהוא ז״ל תירץ דכיון דאין לו אלא שתי מחיצות ושתי לחיים אינו חשוב כל כך דנציל לתוכו אוכלין ומשקין ואפילו במשותף, משום דדמי לרשות הרבים, ומתוך שהוא בהול על ממונו אי שרית ליה אתי להציל אף לרשות הרבים, ואמר ליה אביי לא, דאם איתא, אף אוכלין ומשקין שרי כיון דמבוי גמור הוא, ואף אני אומר דהיינו נמי דלא אוקמה רבה כב״ה, משום דמודה רבה דאי בשיש לו שלש מחיצות ולחי אחד מבוי גמור הוא ובההוא אפילו אוכלין ומשקין מציל.}
\textblock{\textbf{אלא אמר רב אשי שלש מחיצות ולחי אחד זהו מבוי שאינו מפולש כו׳ ואפילו לר״א דאמר בעי שני לחיים וכו׳.} פירש רש״י ז״ל: ותרווייהו כר״א, ולפי פירושו קיימא לן כבן בתירה, דכיון דהקילו טפי בהצלת כתבי הקודש מהצלת אוכלין, ורבנן דבעו לגבי אוכלין שלש מחיצות ושתי לחיים כר״א, אפילו הכי שרו בכתבי הקודש אפילו בלחי אחד, אף אנו נאמר דכיון דקיימא לן כבית הלל דאפילו אוכלין מצילין למבוי שיש לו שלש מחיצות ולחי אחד, לגבי כתבי הקודש נקל טפי ואפילו אין לו אלא שלש מחיצות בלחוד בלא לחי נציל.\par \textbf{} וכן כתב הר״ז הלוי ז״ל שגם הוא מפרש כדברי רש״י ז״ל, דתרווייהו אליבא דר״א. אבל אין פירושו מחוור, דאם כן הוי ליה לרב אשי למימר ותרווייהו אליבא דר״א כמו שאמר רב חסדא, ורבה נמי דאמר ותרווייהו אליבא דרב יהודה. אלא רב אשי כב״ה מוקי לה, אלא דכי היכי דלא תיקשי אי הכי אפילו אוכלין ומשקין נציל, קאמר דבהא אפילו ר״א מודה בה, משום דבעי תנא למיתנייה אפילו לר״א, לא מצי למיתני דנציל לתוכו אוכלין ומשקין, והשתא אתיא מתניתין כהלכתא, וכן פסק הרב אלפסי ז״ל בהלכות. ואף ר״ח ז״ל פסק כן הלכה כתנא קמא דמתניתין.}
\textblock{ מהא דתניא:\textbf{ הציל פת נקיה אין מציל פת הדראה.} דקדקו בתוס׳ לאסור גם ביום טוב (שני) למי שיש לו פת נקיה שלא יאפה פת הדראה. ואינו מחוור בעיני, דהא לגבי הצלה אסרו אפילו הדברים המותרים בעלמא מפני שהוא בהול על ממונו וחששו דלמא אתי לכבויי, אבל ביום טוב אם הטיבתו פת הדראה אופה ואינו נמנע.}
\textblock{\textbf{מצילין מיום הכיפורים לשבת אבל לא משבת ליום הכיפורים.} מדקאמר אבל לא משבת ליום הכיפורים, ולא קאמר ולא מיום הכיפורים למוצאי יום הכיפורים, שמע מינה מצילין. והכין נמי איתא בהדיא בירושלמי (דפרקין ה״ג) דגרסינן התם: ביום הכיפורים מאי אית לך על דעתין דרבנן לא יציל כלום, כל עמא מודו שמצילין מזון סעודה אחת מפני הסכנה וגרסי׳ תו התם מצילין לחולה ולזקן כבינוני, ולרעבתן כבינוני.}
\textblock{\textbf{והא תנא דבי שמואל כל מלאכת עבודה לא תעשו יצאו תקיעת שופר ורדיית הפת.} איכא למידק מאי קא מייתי מיום טוב לשבת, ביום טוב שאני דכתיב ביה מלאכת עבודה, אבל בשבת דכתיב לא תעשה כל מלאכה, הכל בכלל איסור. ועל כן פירש רבי שמואל ז״ל דלא גרסינן כל מלאכת      אלא לא תעשה כל מלאכה. ואין זה נכון, חדא דאין הספרים מודין לו, ועוד מנא לן דכל מלאכה מוציא תקיעת שופר ורדיית הפת, אדרבה מרבה הוא כל מלאכה לאיסור, אלא ודאי כל מלאכת עבודה גרסינן. וקושיא מעיקרא ליתא, דכיון דאינה מלאכה לגבי יום טוב גם לגבי שבת לא חשיבא מלאכה, דהא לא התיר הכתוב ביום טוב אלא אוכל נפש בלבד וכדאמרינן (מגילה ז, א) אין בין יום טוב לשבת אלא אוכל נפש בלבד, ואי נמי מכשירי אוכל נפש, וא״נ לבית הלל (ביצה יב,א) כל מה שאפשר להתיר משום מיגו שהותר באוכל נפש, והלכך כשמוציא הכתוב תקיעת שופר ורדיית הפת מכלל איסור מלאכת יום טוב שמעת מיניה דאינה מלאכה.\par \textbf{} אבל הרמב״ן ז״ל פירש הענין, משום דגבי חג המצות כתיב (שמות יב, טז) כל מלאכה לא יעשה בהם אך אשר יאכל לכל נפש וגו׳ מפני שכתוב בהן כל מלאכה הוצרך להוציא בפירוש אוכל נפש, אבל בשאר החגים דלא כתיב בהם אלא כל מלאכת עבודה, לא הוצרך הכתוב להוציא בפירוש אוכל נפש, שאין בכלל מלאכת עבודה אוכל נפש שהיא אינה מלאכת עבודה אלא מלאכת הנאה, ובמקום אחר כתיב בחג המצות לא תעשה מלאכה (דברים טז, ח), ומפני שלא אמר כל לא הוצרך הכתוב להתיר שם אוכל נפש בפירוש, ועוד במקום אחר כתיב (ויקרא כג, ה) בחג המצות כל מלאכת עבודה לא תעשו, ומכאן למדו להתיר תקיעת שופר ורדיית הפת, דהאי מלאכת עבודה למעוטי מאי, אי למעוטי אוכל נפש, הרי התירו בפירוש בחג המצות, אלא ודאי להוציא אלו שאינן מלאכה בא, אלמא אע״פ שאסר בו הכתוב בפירוש כל מלאכה חוץ מאוכל נפש, אפילו הכי תקיעת שופר ורדיית הפת הותרו בו דאלמא אינו בכלל כל מלאכה, וכיון שכן אף בשבת ויום הכיפורים שכתוב בהן לא תעשה כל מלאכה מותר דאינן בכלל כל מלאכה.}
\textblock{\textbf{רבי זירא בצע אכוליה שירותיה.} פירש רש״י ז״ל: חתיכה גדולה שיש בה די לכולה שירותיה, וכן פירש גם בברכות (לט, ב). ואינו מחוור בעיני, דהא אינו ענין שמועתנו, ועוד דאדרבה כל שהוא מרבה בחתיכה הוי עין יפה, וכדאמרינן התם (מו, א) בעל הבית בוצע כדי שיבצע בעין יפה, והיכי קרי ליה הכי רעבתנותא. אלא נראה לי דבצע על כל הככרות קאמר, ומשום דקאמרינן דנקט תרתי ובוצע חדא. קאמר הכא ר׳ זירא בצע אכולהו ככרות דמנחי קמיה, והיינו דאמרינן דמחזי כרעבתנותא.}
\textblock{\textbf{כמה סעודות חייב אדם לאכול בשבת, שלש.} כתב הרב בעל ההלכות ז״ל דאין חיוב סעודות אלו להינתן בשחרית ובצהרים ובמנחה, אלא אפילו מפסיק באמצע סעודתו בברכת המזון וחוזר ואוכל שפיר דמי, ולדבריו הא דאמרינן בסמוך קערות שאכל בהן שחרית מדיחן לאכול בהן בצהרים בצהרים מדיחן לאכול בהן במנחה, לא שיהיה החיוב בחילוק אכילות בזמנים אלו, אלא אורחא דמילתא נקט.\par \textbf{} וכן יש שאומרים שאפילו עביד אותן שלש סעודות במיני תרגימי שפיר דמי, והא דאקשינן לקמן (שבת קיח, א) מהא שיש לו מזו ארביסר סעודות, ואמרינן לרבנן חמיסר הויין ואי ר׳ חדקא שתסרי הויין, התם נמי משום דאורחא דאינשי דלא קבעי סעודה אלא על הפת קאמר הכין, דאי בעי למיכל נהמא מחייבינן למיתן ליה, והוא הדין למדיר את אשתו ובפרק אע״פ (כתובות סד, ב) גבי המדיר, הארכתי בזה יותר בסייעתא דשמיא.}
\clearpage
\newsection{דף קכ}
\textblock{ מתני׳:\textbf{ מצילין סל מלא ככרות וכו׳.} ואומר לאחרים חבואו והצילו לכם. ירושלמי (דפרקין ה״ד): שכך הוא דרכו להזמין אורחים בשבת.}
\textblock{\textbf{ופושט וחוזר ולובש ואמר בואו והצילו עמי.} ירושלמי (שם ה״ה): שכן דרכם להשאיל כלים.}
\textblock{ גמרא:\textbf{ פירש טליתו וקיפל.} כלומר: שהביא כלים והניחן בתוך טליתו וקיפל טליתו עליהם, ואסיקנא דכבא להציל דמי, כיון שלא הוציא כל אחד בפני עצמו. ומכאן הביאו התוס׳ קושיא לדברי רש״י ז״ל, שפירש בסוף פרק קמא דמכלתין (יט, ב) הא דאמרינן תרי תלמידי חד מציל בחד מאנא וחד מציל בד׳ וה׳ מאני, ובפלוגתא דרב הונא ורבי אבא בר זבדא, שהיה מקפל ד׳ וה׳ כלים ומניחן בכלי אחד ומוציאן. ואינו דהא אסיקנא הכי דכבא להציל דמי, ואפילו רב הונא שרי. ועל כן פירשו הם שם שהיה מציל מזון הרבה בכלים מוחלקין (הרבה) ומוציא וחוזר ומוציא, ובאותה חצר, כרב אבא בר זבדא.\par \textbf{} אבל הרב רבינו משה ב״ן ז״ל פירש כאן, פירס טליתו והביא כלים ושפכן לתוך טליתו, ולא שהיה מוציא כלים מקופלין בתוך טליתו, ומשום דרב הונא אסר בכלים המקופלים והתיר במציל ואפילו הרבה, שאל ר׳ הונא בריה דרב יהושע במביא כלים ושופך לתוך טליתו ומקפל ומוציא מאי, משום דהא דמיא למציל בחדא משום דאינו מוציא אלא בכלי אחד, ובחדא למקפל שמביא כלים הרבה ומקפל לתוך טליתו.}
\textblock{\textbf{לובש ומוציא ופושט וחוזר ולובש ומוציא.} והא דלא שרינן הכי באוכלין ומשקין, משום דהתם כיון שבידו הוא מוציא, מתוך שהוא בהול על ממונו אי שרית ליה אתי לכבויי, אבל הכא כיון שאין אתה מתירו אלא דרך לבישה, רמי אנפשיה ומידכר.}
\textblock{ הא ד\textbf{א״ר יהודה טלית שאחז בה האור מצד אחד נותנין עליה מים מצד אחר.} השמיטה הרב אלפסי ז״ל, והביא ברייתא דקתני טלית שאחז בה האור פושטה ומתכסה בה, לומר שאין הלכה כרב יהודה. ואע״פ שאמרו דרב יהודה דאמר כרבי שמעון בן ננס, וקיימא לן כר׳ שמעון בן ננס. יש לומר שהרב ז״ל סבור דרב יהודה הוא דסבירא ליה דר״ש בן ננס פליג אפילו בהא אע״ג דמקרב את כבויו. והא דתנן בפרק כירה (שבת מז, ב) נותנין כלי תחת הנר לקבל ניצוצות ובלבד שלא יתן לתוכו מים, ר׳ יוסי היא ולא רבנן וכדקא סלקא דעתך התם למימר, אבל אנן דסבירא לן כאוקימתא דרב אשי       לההיא אפילו לרבנן ואומר דמקרב את הכבוי הוא ואפילו לרבנן אסור וכדדייקא בברייתא כדאיתא התם, שמעינן נמי דעד כאן לא שרי ר״ש בן ננס הכא אלא במחיצת כלים דאפשר דלא מתבקעי, אבל במחיצה של מים לא שרי, ובהא לא פליגי.\par \textbf{} וכן תירצו הראב״ד והרמב״ן ז״ל להעמיד דברי הרב אלפסי ז״ל. אבל אינו מיושב בעיני כל הצורך, דאם איתא, הוי ליה הכא נמי לאקשויי, אימר דשמעת ליה לר״ש בן ננס גורם כבוי, מקרב כבוי מי שמעת ליה, והא תנן נותנין כלי תחת הנר וכו׳ אבל לא יתן לתוכו מים, וכי תימא רבי יוסי היא וכולה כדשקיל וטרי בה התם בסוף כירה (שבת מז, ב), אלא שמקרב את כבוי שאמרו שם, פירוש אחר יש לו וכבר כתבתיו במקומו בסייעתא דשמיא.}
\textblock{\textbf{אימר דאמר ר׳ שמעון בן ננס מפני שהוא מחרך, גרם כבוי מי אמר.} וא״ת והא דקתני עושין מחיצה בכלים בין ריקנים בין מלאים, ואי לא שרי גורם לכבויי, מלאין אמאי. ויש לומר דקא סלקא דעתך דמקשה דמלאין דנקט לאשמעינן דלא גזרינן שאינן עשויין להשתבר אטו עשויין להשתבר, ולעולם מלאין בשל חרס דכפר שיחין וכלי כפר חנניה דאינן עשויין להשתבר.}
\textblock{ הא דתניא:\textbf{ נר שעל גבי הטבלא מנער את הטבלא והיא נופלת ואם כבתה כבתה.} ואוקימנא דוקא בשוכח אבל במניח נעשה בסיס לדבר האסור. וא״ת מכל מקום הרי שופך הוא שמן שבנר על כרחו וחייב משום מכבה וכדתניא (ביצה כב, א) המסתפק ממנו חייב משום מכבה, ואם הוא מטה השמן כלפי הנר חייב משום מבעיר וכדתניא הנותן שמן בנר חייב משום מבעיר, וכיון דאי אפשר בלאו הכי אסור אפילו לרבי שמעון דהא מודה בפסיק רישיה ולא ימות. ומיהו לדברי בעל הערוך ז״ל שכתבנו למעלה (קיא, א) דפירש דלא מודה ר׳ שמעון בפסיק רישיה דלא ניחא ליה, הכא נמי איכא למימר דלא ניחא ליה בהכא דלא מתהני מיהא מידי. ואי נמי יש לי לומר בנר של שעוה וכיוצא בו, דהכא לאו משום הכי אתי ליה אלא משום שריותא דאם כבתה כבתה.}
\textblock{\textbf{נר שאחורי הדלת פותח ונועל.} פירש רש״י ז״ל: ואם כבתה ברוח כבתה. והקשו עליו בתוס׳ דאם כן מאי נועל. ואינה קושיא לפי דעתי דנועל לאו דוקא ומשום פותח נקט לה, ואורחא דמלתא נקט כלומר אינו נמנע מלפתוח ולנעול כדרכו. והם ז״ל פירשו שהנר עומד על הדלת ומשום נדנוד הדלת בפתיחתו או בנעילתו יפול ויכבה, ומשום בסיס ליכא דאיכא למימר בשוכח. וא״נ שאין הדלת בטלה לגבי הנר להיות בסיסו. אבל לשני הפרושים האלו קשה לי, דהיכי קרי ליה להאי פסיק רישיה דדלמא לא מכביא בנפילתו או בנדנודו ולא ברוח הנכנס לה מכנגד הפתח. ור״ח ז״ל פירש שהנר קבוע אחורי הדלת ובפתיחתו יכבנו.}
\textblock{\textbf{א״ל בהא אפילו ר׳ שמעון מודה.} קשיא לי לפי שיטת הרב בעל הערוך מה הנאה יש לו בכבויו של נר. ושמא בפתילה שצריך להבהבה, ואי נמי דקא מתהני בשיור השמן והפתילה, ומכל מקום שמעינן מיהא דאפילו במלאכה דרבנן ובשאינו מתכוין אי פסיק רישיה הוא אסור אפילו לר׳ שמעון, דהא הכא אינו מתכוון ומלאכה שאינה צריכה לגופה היא, ואפילו הכי אמרינן דמודה ר׳ שמעון בפסיק רישיה.}
\textblock{\textbf{דהא אביי ורבא דאמרי תרוייהו.} אביי לא הוה אסר הכין מעיקרא עד דשמעה מרבא, וכדאיתא לקמן בפרק רבי אליעזר דמילה (שבת קלג, א).}
\textblock{\textbf{והא איפכא שמעינן להו.} לאו איפכא ממש קאמר, דהא כולהו תנאי דברייתא מיסר אסרי, אלא משום דת״ק דמתניתין לקולא ורבי יוסי לחומרא ובברייתא רבי יוסי לקולא ורבנן לחומרא קאמר הכין.}
\textblock{\textbf{לעולם יורד וטובל ובלבד שלא ישפשף.} תמיהא לי דהא מכיון שהוא מכניס ידו במים הרי זה כמקרב את כבויו. ויש לומר דלא קרינן מקרב את כבויו אלא בכענין נותן מים בכלי שתחת הנר, ואי נמי בנותן בצד הטלית שאחז בו האור, משום דאי יפלו שם ניצוצות או תגיע שם דליקה      תכבה, אבל כאן אפשר דלא ימחק שאילו ודאי נמחק היינו כמשפשף, שהרי הוא נותן ידו במים.}
\clearpage
\newsection{דף קכא}
\textblock{\textbf{שמעת מינה קטן אוכל נבלות בית דין מצווין עליו להפרישו.} וא״ת והא לרבי שמעון מלאכה שאינה צריכה לגופה היא וליכא אלא איסורא דרבנן, והתם ביבמות בפרק חרש (קיד, א) אמרינן דלכולי עלמא באיסורי דרבנן אין בית דין מצווין להפרישו. יש לומר דדלמא מתניתין כר׳ יהודה, אי נמי כר׳ שמעון ובצריך לפחמין, כך תירצו בתוס׳.\par \textbf{} ולי נראה דאין צורך לכך, דהכא אתמוהי קא מתמה, והכי קאמר שמעת מיניה קטן אוכל נבלות בית דין מצווין להפרישו ואפילו בדרבנן, ואסיקנא בקטן העושה על דעת אביו, וכיון שכן אפילו בדרבנן נמי נמי מפרישין דהוה ליה כמו שצוהו אביו לעשות. ואפילו לפי מה שכתבתי במסכת יבמות (שם) דלדידן דקיימא לן דאפילו בדאורייתא אין בית דין מצווין להפרישם, באיסורי דרבנן ספינן ליה בידים. הא לא קשיא לי, דכיון דעביד לצורך אביו ולא לצורכו אסור, אבל לצורך עצמו שרי. וכבר כתבתיה שם בארוכה בסייעתא דשמיא. וכן נראה לי קצת ממה שאמרו כאן בירושלמי דגרסינן בירושלמי (הלכה ז): אבל קטן שבא לכבות אין שומעין לו, לא כן תני ראו אותו יוצא ומלקט עשבים אין את זקוק לו, תמן יש לו צורך בעשבים ברם הכא אין לו צורך בכבוי. אלא דלפי זה לא היינו שומעין לו אפילו בשאין עושה על דעת אביו, וזה שלא כדרך גמרתנו.}
\textblock{ הכי גריס רש״י ז״ל:\textbf{ צואה של קטן והא חזיא לכלבים, ותנן מחתכין את הנבלה לפני הכלבים, וכי תימא נולד הוא, והא תניא נהרות המושכין ומעינות הנובעין הרי הן כרגלי כל אדם, אלמא כיון דהכי אורחייהו דעתיה עלויה, הכא נמי כיון דאורחיהו בהכי דעתיה עלויה.} ויש מקשים עליו מדאמרינן בריש פרק קמא דביצה (ב, א) במאי אי בתרנגולת העומדת לגדל ביצים נולד הוא, ואי איתא מאי קושיא, הא כיון דאורחייה הכי דעתיה עלויה. ויש מפרשים דהתם בעומדת לגדל ביצים לאפרוחים קאמר. ואינו מחוור, דאם כן מאי קאמר בתרנגולת העומדת לגדל ביצים, דמשמע דעיקרא דמלתא דתרנגולת תליא אי עומדת לאכילה או לגדל ביצים, ועוד דתניא התם (ד, א) ביצה תאכל אגב אימה, ואמרינן עלה הכא במאי עסקינן כגון שלקחה סתם, נשחטה הוברר הדבר דלאכילה עומדת, לא נשחטה הוברר הדבר דלגדל ביצים עומדת, ואם איתא אפילו לא נשחטה היאך הוברר הדבר דלגדל ביצים לאפרוחים עומדת, דלמא לגדל ביצים לאכילה עומדת שכן הרבה עושין כן. אלא יש לתרץ לדברי רש״י ז״ל, דהתם כיון דאגידא באימא כגופה דמיא, וכיון דאימא אסירא איהי נמי אסירא.\par \textbf{} אלא דאכתי איכא למידק מדאפליגי ב״ש וב״ה (לקמן שבת קמג, א) בקליפין ועצמות, וחד מינייהו אמר מסלק את הטבלא ונוערה אבל קליפין ועצמות גופייהו לא ומשום מוקצה, ואמאי והא כיון דדעתיה למיכלינהו לפירי דעתיה אקליפין למיהב לבהמה, דהתם בראוין לבהמה קא מיירו. ותו אפר שהוסק ביום טוב דאמרינן בפרק קמא דביצה (ח, א) דאסור, והא דעתיה אעצים המוכנים לתבשילו, וא״כ נימא דדעתיה נמי אאפרא. ויש לומר דכל שעומד למאכל אדם ולתשמישו, לא יהיב דעתיה אלא לכך ולא מסיק אדעתיה פסולת דידיה, דאין דעתיה דאיניש אלא אמאי דחזי ליה, אבל בדלא חזי, מסיק אדעתיה לאזמוני לבהמה.\par \textbf{} אבל [בספרים] לא גרסינן הכי כדגריס רש״י ז״ל, אלא הכי גרסינן: צואה של קטן גרף של רעי הוא. וכן נראה גירסת הגאונים, והכי איתא בהלכות הרב אלפסי ז״ל. אבל בירושלמי (ה״ח) מצאתי כדברי רש״י ז״ל, דגרסינן התם: צואה של קטן ולא מאכל של תרנגולין אינון, מר עוקבא אמר תיפתר באלין רברביא שלא יבואו לידי מירוח.}
\textblock{\textbf{ברצין אחריו ודברי הכל.} הרב בעל ההלכות וכן הראב״ד והר״ז הלוי ז״ל פירשוה אברייתא, כלומר ברייתא דקתני חמשה נהרגין ודברי הכל, כלומר דאפילו רבי יהודה מודה בה, אבל שאר המזיקין שאינן מצויין כל כך להזיק, אפילו ברצין אחריו אין נהרגין לר׳ יהודה, אבל לר״ש אפילו שאר המזיקין נהרגין ואפילו אין רצין אחריו, דכיון דליכא אלא איסורי דרבנן במקום הזיקא שרי, ור׳ יהושע בן לוי כר״ש.}
\textblock{\textbf{ויש לדקדק שאם כפירושם מאי קאמר ודברי הכל, דלא הוה ליה למימר אלא ברצין אחריו ור׳ יהודה היא, דהא ר״ש אפילו בשאין רצין אחריו שרי. ועוד תמיהא לי היאך אפשר       } שיאמר ר׳ יהודה דנחש רודפו להכישו אסור להורגו, והלא פקוח נפש הוא, אלא אדרבי יהושע בן לוי קאי, כלומר אנא מתרצנא ליה לר׳ יהושע בן לוי דרצין אחריו קאמר ודברי הכל, וברייתא ר׳ שמעון היא ואפילו בשאין רצין אחריו. וכן נראה מדברי הרב אלפסי ז״ל. ואפילו גרסינן דאנא מתרצנא לה, כלומר דנוקמא דר׳ יהושע בן לוי ברצין אחריו, ומתרצנא (ליה) [לה] ברייתא דנוקמא כר׳ שמעון ואפילו בשאין רצין. ומיהו אי גרסינן מתריצנא ליה אתי שפיר טפי.\par \textbf{} ומיהו קיימא לן כר״ש במלאכה שאינה צריכה לגופה, משום דסתמא דמתניתא כותיה, ורבא דהוא בתראה נמי כותיה, ואפילו הכי בשאר המזיקין כשאין רצין אחריו משנה בהריגתן כל מה שאפשר לשנויי, אם אפשר לצודן צדן ואינו הורגן, דהא תנן כופין קערה על עקרב שלא תשוך דאלמא כופין ולא הורגין. ואע״ג דצידה נמי אב מלאכה הוא, מיהו לא מינכרא מלתא כולי האי, ואי לא אפשר ליה בהכין דורסן לפי תומו, והיינו דרב ששת דאמר לקמן נחש דורסו לפי תומו. נמצאת אומר ברצין אחריו כולהו נהרגין להדיא ואפילו לר״י משום פקוח נפש, בשאין רצין אחריו חמשה המנויין בברייתא נהרגין להדיא לר״ש מפני שהזיקן מצוי, שאר המזיקין בשאין רצין אחריו אין נהרגין להדיא ואפילו לר״ש, אלא דורסן לפי תומו.}
\textblock{ הא ד\textbf{אמר רב ששת נחש דורסו לפי תומו.} פירש רש״י ז״ל: שאינו צריך ליזהר ממנו שלא ידרסנו אלא דורסו לפי תומו. ונראה מדבריו דלא התירו אלא משום דבר שאין מתכוון וכר״ש. ואינו מחוור, חדא דהא קאמר ר׳ ינאי צרעה אני הורג דמשמע לכתחילה ומתכוון. ועוד מדאמר ליה אבא בר מרתא לדבי ריש גלותא לא צריכיתו, הכי אמר רב יהודה רוק דורסו לפי תומו, ומנא ליה שיזדמן להם לדורסו בלא מתכוין. ועוד דמכל מקום צריכים היו לכוף עליו כלי עד שיזדמן להם לדורסו לפי תומו. ועוד דאי אפשר לומר דהא קמ״ל דדוקא לפי תומו אבל במתכוון לא, דהא משמע דלהיתרא קא אתא לאשמועינן, דאי לא לימא נחש אינו הורגו אלא דורסו לפי תומו. אלא הכי פירושה דורסו אפילו במתכוון, אלא שהוא עושה לפי תומו שמראה עצמו כאילו אינו מתכוון, דכל כמה דאפשר לשנויי משנינן, וכדכתבינן לעיל.}
\textblock{\textbf{אמר ליה רבי זירא בניטלין באדם אחד או אפילו בשני בני אדם, אמר ליה כאותן של בית אביך.} כלומר: אפילו ניטלין בשני בני אדם. ובירושלמי (בפרקין, ה״א) נחלקו בשלשה, דאיכא מאן דאמר התם כלי שניטל בשנים מטלטלין אותו, אבל שלשה וארבעה וחמשה אין מטלטלין אותו, ואמר ר׳ זירא מכיון דאת אמר בשנים מותר, מעתה אפילו בארבעה וחמשה. וכן הלכתא, מדאמרינן בעירובין (קב, א) ההיא שריתא דהוה בי ר׳ פדת דהוו שקלי לה בי עשרה ושדו ליה אדשא.}
\clearpage
\newsection{דף קכב}
\textblock{\textbf{כאותן של בית אביך.} פירש רש״י ז״ל: דקטנות היו. ואינו מחוור, דהא בפרק כירה (שבת מו, א) לא אסרו אפילו גדולה אלא בדאית בה חידקי וגזירה משום דחוליות, ומשמע התם אפילו גדולה ממש הניטלת בשתי ידים.}
\textblock{ מתני׳:\textbf{ מילא מים לבהמתו משקה אחריו ישראל ואם בשביל ישראל אסור.} כתבו בתוס׳ בשם ר״ת דדוקא להשקות לבהמתו אסור משום דאהנו ליה מעשיו דהא אי אפשר לבהמה שתרד לבור, אבל לשתות מהם ישראל שרי, דהא לא אהני ליה מידי דאי בעי מטפס ויורד מטפס ועולה וכן כתב גם הר״ז הלוי ז״ל בשם חכמי נרבונה ז״ל. אבל ר׳ יצחק ז״ל הקשה עליו, מדתניא (להלן בגמ׳) עכו״ם שלקט עשבים לבהמתו מאכיל אחריו ישראל, ואם בשביל ישראל אסור, והתם מאי אהני ליה, והא אמרינן מעמיד אדם בהמתו על גבי עשבים, ומפרש במכילתא (שופטים פ״כ, י״ב) למען ינוח שורך וחמורך, הנח לה והיא תולשת וזו היא נייחא שלה. ודחק ר״ת ז״ל כגון דקאי בתרי עברי דנהרא, דאי אפשר לה למיזל. והר״ז הלוי ז״ל דחק יותר והעמידה בעשבים מפוזרין לכאן ולכאן, שאין דרך להאכיל בהמה בדקאי לה באפה ואזלא. וזה יותר דחוק, והא דאמרינן דקאי לה באפה לאו אמחובר קאמרינן, אלא אמוקצה דתלוש ומשום גזירה דלמא מאכיל לה בידים וכמו שפרש״י ז״ל. ולפי דבריהם הא (דתניא) [דתנן] מילא מים לבהמתו, מן הבור דוקא קאמר, הא בממלא מן העין או מן הנהר אפילו בשביל ישראל מותר. ואינו מחוור, דאם כן היה לו למיתני בהדיא מילא מים לבהמתו מן הבור.\par \textbf{} ובתוס׳ אמרו עוד דאפילו לדברי האוסרים, יש מי שאומר דדוקא שתיה אבל שאר תשמישין מותר, וקא מייתו ראיה מההיא דאמרינן ביבמות (קיד, א) במפתחות בי מדרשא דאתאבידו ברשות הרבים, זיל דבר תמן טליא וטליתא דאי משכחו להו מייתי להו, אם איתא כי מטו להו מאי הוי, והא לא אפשר לאשתמושי בהו. והם הקשו על זה מנר וכבש ומרחץ דאסרינן הכא כשנעשה על דעת ישראל, וההיא דיבמות בקטן העושה לדעת עצמו. ולדידי נראה דאפילו לדבריהם מאן לימא לן דלהשתמש בהו השתא קאמר התם, דלמא כי היכי דלא ליזלו לאיבוד ולאשתמושי בהו לאורתא. אלא שאין צורך לכך אלא אפילו ליומן. וכמו שאמרו בתוס׳.}
\textblock{ הא דתניא:\textbf{ עכו״ם שלקט עשבים ואם בשביל ישראל אסור.} וכן נמי בנר וכבש ומרחץ. איכא למידק בהו, והלא המבשל בשבת בשוגג לר״מ יאכל (חולין יד, א), והלכתא כותיה לדעת רובן של פוסקים אלא דלא דרשינן ליה בפרקא. ויש לומר דשאני ישראל דליכא למיגזר ביה שוגג אטו מזיד, דאפילו שרית ליה שוגג מידע ידע דאסור לעשות כן לכתחילה, אבל במלאכות הנעשות על ידי עכו״ם אי שרית ליה בדיעבד, אתי למימר ליה, דמלאכות הנעשות על ידי עכו״ם קילי ליה, ואף על גב דאי אמר ליה נמי ליכא איסורא דאורייתא, לאו גזירה לגזירה      אלא כולה חדא גזירה היא וכאותה שאמרו בשמעתא קמייתא דביצה (ג, א).\par \textbf{} ואדרבא איכא למידק ליקט לעצמו היאך מאכיל אחריו ישראל והדליק לעצמו היאך משתמש בו ישראל, והא תניא (ביצה כד, ב) עכו״ם שהביא דורון לישראל אם יש במינו במחובר לקרקע אסור ולערב אסור בכדי שיעשו, ומי לא עסקינן אפילו רואה אותו לוקח מן השוק ומביא לו, דלא ליקטן בשביל ישראל, ועוד מאי שנא מפירות הנושרין או ממשקין שזבו. ויש לומר דמשקין שזבו ופירות הנושרין דממילא קא אתו אי שרית ליה אתי למעבד איהו בידים דמיחלף ליה, אבל כל מידי דלא אתי ממילא אלא תלי במעשה כגון אפיה ובישול ונר וכבש ונעשה על ידי עכו״ם בשביל עכו״ם, שרי אפילו לישראל דליכא למיגזר בהא מידי, דישראל בעכו״ם לא מיחלף, אבל כשעשאן בשביל ישראל אי שרית ליה לישראל אתי למימר ליה כמו שכתבנו.\par \textbf{} ומיהו פירות שתלשן עכו״ם ואי נמי משקין הבאין על ידי סחיטת עכו״ם אסורין, דאי שרית ליה אתי למיכל אפילו כשנשרו או כשזבו מעצמן וכולה חדא גזירה היא, וכדאמרינן התם בביצה (ג, א) משקין שזבו טעמא מאי גזירה דלמא יסחוט, היא גופה גזירה ואנן ניקום וניגזר גזירה לגזירה, אין כולה חדא גזירה היא. ומהא שמעינן דעכו״ם שאפה או שבשל מידי דלאו מוקצה ואי נמי מידי דלית ביה משום בשולי עכו״ם לעצמו בשבת אוכל אחריו ישראל, אבל במידי דאתי ממילא לא.\par \textbf{} ואם תאמר אם כן פירות האדמה שאינן נתלשין ממילא, (א״כ) לישתרו בשתלשן עכו״ם לעצמו. יש לומר דשאני התם דכיון דלא עבידי דאתו אסורין משום מוקצה, ואפילו לר׳ שמעון משום דכגרוגרות וצימוקין נינהו לדידיה דאסוחי אסח דעתיה מינייהו כיון דלא תלשן מערב שבת, ואפילו דעכו״ם נמי אסירי, דכיון דבשלו אסירי מהאי טעמא אפילו בשל עכו״ם אסור כדי שלא תחלוק. וכן כתב רש״י ז״ל בפרק אין צדין (ביצה כד, ב) גבי מימרא דרב פפא דעכו״ם שהביא דורון לישראל. ועשבים דשרינן להאכיל אחריו ישראל לבהמתו, דאין מוקצה לבהמה אלא משום טלטולו של ישראל, והא אוקימנא לה הכא בדקאי באפה ואזלא ואכלה.}
\textblock{\textbf{רבא אמר אפילו תימא בפניו נר לאחד נר למאה.} ומיהו ודאי מודה לאביי דכל שלא בפניו אפילו לאפושי בשביל ישראל שרי, והיינו ברייתא דעכו״ם שלקט עשבים לבהמתו, ובמכירו ובשלא בפניו נמי שרי והיינו דאביי, ואי מידי דליכא לאפושי בשביל ישראל כנר וכבש אפילו בפניו שרי והיינו דרבא. אבל אם עשאו בשביל ישראל לעולם אסור, ואפילו עשאו בשביל ישראל אחר אסור.\par \textbf{} ואם רוב עכו״ם מותר, והיינו מרחץ המרחצת בשבת ונר הדולק במסיבה. וטעמא משום דאמרינן דמסתמא עיקר כונתו ומעשיו אדעתא דרובא הוא, וכאילו לא נעשה בשביל המיעוט דאין הולכין אלא אחר הרוב. ואי מחצה על מחצה נינהו, כיון דאיכא למימר דאדעתא דישראל עבד אסור, וכדתניא בברייתא בנר הדלוק במסיבה. והני מילי בשלא עשה לצורך עצמו, אבל עשאו לצורך עצמו אפילו היו שם כמה ישראלים מותר, דלעולם עיקר דעתו אינה אלא לעצמו, וכדמוכח בעובדא דבסמוך, [ד]ההוא מעשה היכי דמי אי רובא גוים היו אמאי אהדרינהו שמואל לאפיה מינייהו, ואי רובא ישראל הוו ואי נמי מחצה על מחצה כי אייתי שטרי וקא קרי בהו מאי הוי, אלא שמע מינה דכל שהוא עצמו משתמש בו שרי. וכן פירש רש״י ז״ל.\par \textbf{} וכן היא בירושלמי (בפרקין, ה״ט) דגרסינן התם: נכרי שהדליק לצורכו ולצורך ישראל, נשמיענו מן הדא, שמואל איתקביל גבי חד פרסי ואיטפי בוצינא, אזל ההוא פרסי בעי מדלקתה והפך שמואל אפוי, כיון דחמיניה מתעסק בשטרותיו ידע דלאו בגיניה הוא וחזר שמואל אפוי, א״ר יעקב בר אחא הדא אמרה לצרכו ולצורך ישראל אסור, אמר רבי יוסי שניא היא שאין מטריחין את האדם לצאת מביתו למה הפך שמואל אפוי. ע״כ בירושלמי. ונראה דלא גרסינן אסור אלא מותר, מדהדרינהו שמואל לאפוי קא מייתי ראיה, והיינו דקא דחי ליה ר׳ יוסא דשניא היא דאין מטריחין על האדם לצאת מביתו, ואקשי ליה אי מהאי טעמא, אם כן מעיקרא אמאי אהדרינהו שמואל לאפוי.}
\textblock{\textbf{ומכאן נראה לי גם כן שאפילו ישראל יושב ובא גוי והדליק את האש בשבת סמוך לו, אם אין הגוי מתחמם כנגדו      } אסור לישראל להתעכב ולישב שם, דהא סבר ר׳ יוסא למימר דאין מטריחין על האדם לצאת מביתו, ואהדרו ליה אם כן שמואל למה הפך לאפוי. ולפי מה שכתבתי דכשהגוי העושה משתמש בו, אפילו עושה לצרכו ולצורך ישראל מותר, אני מגמגם קצת, מהא דאקשינן מדר״ג והא ר״ג מכירו הוה, והוצרך אביי לאוקומי בשלא בפניו, ורבא משני מטעמא דנר לאחד נר למאה, דאלמא לאביי אי בפניו אסור, ולרבא במידי דאיכא לאפושי בשביל ישראל אסור, ואע״ג דההוא לצורך ישראל ולצרכו הוה, ומעתה ההיא דשמואל לא הויא ראיה, דההיא נר הוה ונר לאחד נר למאה. ויש לפרש דשאני רבן גמליאל דכיון דנשיא ישראל הוה, עיקרו משום דידיה. ומכל מקום לפי תירוץ זה בעבד ושפחה המביא מים לצרכו ולצורך ישראל אדוניו אסור, דעיקרו משום אדוניו הוא. ויש לומר עוד דשמא אותו גוי שעבד את הכבש, לא מאותן שהיו בספינה הצריכין לירד היה אלא אחר שלא היה בספינה, אלא כדי שירדו משם הבאים בספינה.\par \textbf{} ובתוס׳ התירו בגוי המלקט עשבים ומאכיל לבהמתו של ישראל, שאין ישראל חייב למונעו, מדתנן (לעיל שבת קכא, א) נכרי שבא לכבות אין אומרים לו כבה ואל תכבה. ואפשר שהוא כן, אלא שאין ראייתם מחוורת בעיני, דדלמא התם משום דבדליקה התירו, ותדע שאילו בעלמא מותר, רבי יוסף בן סימאי למה מנען (לעיל שם), אלא שהיה סבור דכי היכי דבעלמא אסור בדליקה נמי אסור, ושלחו ליה דלא. ומכל מקום במערים אסרו בתוספות, מדאמרינן בפרק הפועלים (ב״מ צ, א) גבי תורא דגנבין ארמאי, הערמה איתעבידא בהו ליזבון לגוים.}
\textblock{\textbf{אמר רב יהודה קורנס של אגוזים לפצוע בו את האגוזים, הא של נפחים לא, דקסבר רב יהודה כל דבר שמלאכתו לאיסור כלל כלל לא.} והא דקתני בסיפא (קכד, א) כל הכלים ניטלין לצורך ושלא לצורך, יש לפרש לדרב יהודה כולה בכלים שמלאכתן להיתר היא, ולצורך לצורך גופו, ושלא לצורך לצורך מקומו, ורבי יוסי פליג ושרי כל הכלים ואפילו שמלאכתן לאיסור, חוץ מן המסר הגדול ויתד של מחרישה, ואתא ר׳ נחמיה למימר דאפילו מלאכתו להיתר אינו ניטל אלא לצורך תשמישו בלבד.\par \textbf{} וא״ת כיון דלרב יהודה צורך מקומו שלא לצורך קרי ליה, רישא דמתניתין אמאי נקט לפצוע בו את האגוזים דמשמע דלצורך גופו דוקא, אבל שלא לצורך דהיינו לצורך מקומו לא, הוה ליה למיתני קורנס של אגוזים לצורך מקומו דהוי רבותא טפי. ותירץ הרמב״ן ז״ל, דנקט לפצוע בו את האגוזים לאשמועינן דכלי שמלאכתו לאיסור אפילו לפצוע בו את האגוזים לא, והדר תנן דכלי שמלאכתו להיתר אפילו לצורך מקומו נמי שרי, כי היכי דלא תימא דדוקא לפצוע בו את האגוזים אבל לצורך מקומו לא, קא משמע לן דכל שמלאכתו להיתר אפילו לצורך מקומו שרי. אבל לרבה בר נחמני ולרבא בריה דרב יוסף בר חמא, כולה מתניתין אתיא כפשטא, דלרבה דסבירא ליה דכלי שמלאכתו לאיסור לצורך גופו אין לצורך מקומו לא, רישא דמתניתין בקורנס של נפחים והלכך נקט לפצוע בו את האגוזים ולומר הא לצורך מקומו לא, וסיפא בכלי שמלאכתו להיתר, ולצורך, לצורך גופו, שלא לצורך, לצורך מקומו. ולרבא בריה דרב יוסף בר חמא דאית ליה דכלי שמלאכתו לאיסור בין לצורך גופו בין לצורך מקומו מותר, לא קפיד תנא דמתניתין בין צורך גופו בין צורך מקומו, דכולה חדא מלתא היא, וסיפא דקתני בין לצורך בין שלא לצורך, היינו לצורך גופו וצורך מקומו, שלא לצורך מחמה לצל וכמו שמפרש הוא בגמ׳.}
\textblock{\textbf{ויש מקשים היכי קאמר רב יהודה דדבר שמלאכתו לאיסור אינו ניטל כלל, והא איהו הוא דאמר בשלהי פרק כירה (לעיל שבת מו, א) גבי נר, דמשחא שרי דנפטא אסיר. ותירצו דשאני נר שהוא אינו משמש מלאכת איסור, אלא שנעשה בסיס לנר ושמן פתילה, והיינו נמי טעמא דפמוטות. אבל מכל מקום עדיין קשה, דהא איהו שרי כלי קיואי לעיל בריש פרק אלו קשרים (שבת קיג, א), וכלי קיואי כלים שעושין      } בהן מלאכת איסור הן. ויש לומר דהנהו לבתר דשמעה להא מרבה אמרינהו, דהא אותביה רבה ואיתותב.\par \textbf{} ומיהו עדיין קשה לי. רב יהודה אפילו מקמי דשמעיה מרבה היכי אסיק אדעתיה הכין, דהא תניא (לקמן שבת קכג, ב) התירו וחזרו והתירו וחזרו והתירו והיכי מתוקמא ליה. ויש לומר דברייתא לא שמיעא ליה. ואי נמי יש לומר דר״י ודאי מסבר הוה סבר דאפילו כל שמלאכתו לאיסור לצורך שרי, אלא דמתניתין הוא דמפרש, דקסבר דת״ק לא שרי אלא כלים שמלאכתן להיתר, מדקתני רבי יוסי אומר כל הכלים ניטלין חוץ מן המסר הגדול ויתד של מחרישה, דמשמע הא לת״ק לא, והא מני רבי נחמיה היא, וברייתא דלא כרבי נחמיה, דהא לא מיתוקמא ליה, ולעולם רב יהודה כרבה סבירא ליה, ובהכי מתרצין לן כולהו קושיין, והכין נמי מוקמינן לה למתניתין כרבי נחמיה, ומיהו לא מתוקמא כרבי נחמיה לפום קושטא, משום דקשיא רחת ומזלג וכדאותביה רבה לרב יהודה.}
\newchap{פרק \hebrewnumeral{17} כל הכלים}
\clearpage
\newsection{דף קכג}
\textblock{}
\textblock{ הא ד\textbf{אותביה אביי לרבה ממדוכה.} איכא למידק והא אביי נמי כרבה סבירא ליה דדבר שמלאכתו לאיסור לצורך גופו מיהא שרי וכדאיתא לקמן, והיכא לא אסיק אדעתיה לאוקומה כרבי נחמיה. ויש לומר דהא דלקמן לבתר דשמעה מרבה רביה הוה. ואי נמי לאפוקי מיניה תירוצה הוי בעי דלמא אית ליה לרבה תירוצה דלא שמיע ליה. ומיהו רבה נמי הוה אפשר ליה לאוקמיה בצורך מקומו, אלא משום דאין מטלטלין אותה לגמרי משמע ליה וכדמשמע ליה נמי לרבי אלעזר לקמן דאוקמה קודם התרת כלים, ולרבי נחמיה נמי ודאי לא מטלטלא ליה כלל למאי דסבירא ליה לרבה כלי שמלאכתו לאיסור לצורך מקומו לא, ואפילו לדידן כיון דסבירא ליה לרבי נחמיה דאין כלי ניטל אלא לצורך תשמישו, ותשמישה של מדוכה לאיסור הוא, הלכך אינו ניטל כלל.}
\textblock{ הא דתנן:\textbf{ מדוכה בזמן שיש בה שום מטלטלין אותה.} יש מי שפירש דהוא הדין על ידי ככר, ואע״ג דאמרינן (לעיל שבת מג, ב) לא אמרו ככר או תינוק אלא למת בלבד, הני מילי היכא דליכא תורת כלי כמת, אבל כל מידי דאית ליה תורת כלי מיטלטל הוא על ידי ככר. וליתא. אלא אין לך מוקצה מיטלטל על ידי ככר, ולא אמרו כאן אלא בשום וכיוצא בו הנדוכין, והיינו טעמא משום דכיון דאין זו אסורה אלא מחמת מלאכה, ועכשיו משמש היתר במלאכתו, הרי מה שאוסרה הוא מתירה, והוי ליה כקדרה מטלטלת עם התבשיל.}
\textblock{\textbf{סיכי זיירי ומזורי.} פירש רש״י ז״ל: בלשון אחד שהן כלי קיואי. ואינו מחוור, דהא אמרינן לעיל בפרק אלו קשרים (שבת קיג, א) דכלי קיואי לכולי עלמא שרו, חוץ מכובד העליון וכובד התחתון, ואפילו הנהו איכא מאן דשרי, ורבי יוחנן דבעא לרבי יהודה בר ליואי ליפשטה מהא דרב, ואיכא למימר דלמא לא שמיעא ליה, והני דהכא היינו כובד העליון וכובד התחתון, ואינו נכון. ויש מפרשים דכלי כובסין הן, וכן פירש בערוך (ערך זיירי).}
\textblock{\textbf{אסובי ינוקא.} פירש רש״י: להחליק סדר אבריו ביד, דכשהוא נולד אבריו מתפרקין וצריך ליישבן, ולא דמי ללפופי דשרי לדברי הכל בשלהי פרק חבית (שבת קמז, ב), דהתם בבגדים הוא מחבש אותו וממילא הוי מתוקן, אבל אסובי ביד דמי לתקוני מנא. ויש שהקשו עליו, דאם איתא לא הוה ליה לדמויי לאפיקטוזין, דטפי הוה ליה לדמויי למעצבין את התינוק דאמרינן לקמן (שבת קמז, א) דאסיר. ור״ח ז״ל פירש       גרמא דפומא דנפלא לינוקי, ופעמים שמקיא על ידי כך והיינו דמדמי לה לאפיקטוזין. וכן פירש גם הר״ז הלוי ז״ל. ויש מקשים דהא רפואתו לאו משום הקאתו היא, והיא גופא נמי שריא דהא אינו מתכוין. ויש מפרשים כמו שפר״ח ז״ל, אלא שאינו בא מצד הקאתו, אלא כך מדמה אותו להקאתו דהיינו אפיקטוזין, דכמו שחתימת האיצטומכא אסור לעשות לה רפואה באפקטוזין לפתחה כך סתימת הגרון אסור לעשות לה רפואה באסובי כדי לפתחה. וכן פירשו בתוס׳.}
\textblock{\textbf{עד שאמרו כל הכלים ניטלין בשבת ואפילו בשני בני אדם.} ואיני יודע במה הן נחלקין אביי ורבא, דאביי אמר דוקא באדם אחד ובניטל אפילו בשתי ידים, ורבא אמר אפילו בשני בני אדם. ויש לפרש דאביי סבר דכיון דאיכא לאוקמה באדם אחד, לא מוקמינן לה בשני בני אדם, דתפשת מועט תפשת תפשת מרובה לא תפשת, ורבא סבר דכיון דקתני כל הכלים אפילו בשני בני אדם משמע. ואי נמי בסברא בעלמא הוא דפליגי, דאביי סבר דכל שאינו ניטל באדם אחד, מקצה אותו מדעתו ואין מטלטלין אותו, ורבא סבר דאפילו בשני בני אדם, ובעיא היא לעיל בפרק כל כתבי הקודש בסופו (שבת קכב, א) גבי קרונות, ואסיקנא התם דאפילו בשני אדם, ושני בני אדם לאו דוקא אלא אפילו ארבעה וחמשה, ופלוגתא היא בירושלמי (פי״ז, ה״א) דגרסינן התם שהוא ניטל בשנים מטלטלין אותו בשנים, בשלשה ובחמשה אסור, א״ר זעירא מכיון דתימא בשנים מותר, אף בארבעה וחמשה מותר.}
\textblock{\textbf{איתביה אביי מדוכה אם יש בה שום מטלטלין אותו ואם לאו אין מטלטלין.} תמיהא לי מאי קא מותיב מינה אביי לרבא, דהא לדידיה נמי על כרחך אין מטלטלין אותה דקתני, היינו לצורך מקומה, אבל לצורך גופה מטלטלין אותה. ויש לומר דמשום דסבירא ליה לאביי דאין מטלטלין אותה כלל משמע, משום הכי קא מותיב ליה לרבא מיניה, דלמא נפיק מיניה פירוקא חדתא. ועוד יש לי לומר דלאביי משמע ליה דאין מטלטלין אותה כלל קאמר וכדאמרן, והכי קאמר ליה בשלמא לדידי דאמינא דלצורך מקומו שלא לצורך קרינן ליה, ומינה דלר׳ נחמיה אפילו מלאכתו להיתר לצורך מקומו לא שרי, מוקמינן לה למדוכה כר׳ נחמיה וכדאוקי לה רבה לעיל, אלא לדידך דאמרת דצורך גופו וצורך מקומו כהדדי ניהו, אם כן כמאן מוקמת לה, ואע״ג דהשתא לא איירי אינהו כלל בדר׳ נחמיה, מכלל פלוגתייהו נפקא להו כדאמרן. ואי נמי עיקר תיובתיה דאביי לקמן (שבת קכד, א) הות גבי פירושא דמתניתין דהא פליגי התם בהדיא בדר׳ נחמיה, ולאו דוקא מותיב אביי לרבא הכא והתם, דבודאי כל יומא לא פריך ליה מינה, אלא דתלמודא הוא דמסדר להו לאקשויי הכא והתם. כך נראה לי.}
\clearpage
\newsection{דף קכד}
\textblock{\textbf{אמר רבא ממאי דלמא לעולם אימא לך לאחר התרת כלים וכו׳.} בהאי פורתא לא מיעפש כלום, והלכך לאו צורך גופו מיקרי אלא הוי ליה כמחמה לצל. ואם תאמר אם כן לישתרי דהא כלי שמלאכתו להיתר אפילו מחמה לצל שרי כדרבא. ויש לומר דהכא רבה גרסינן, דהוא רבה בר נחמני דאסר אפילו כלי שמלאכתו להיתר מחמה לצל.}
\textblock{\textbf{ואתא רבי נחמיה למימר לצורך גופו ולצורך מקומו אין, מחמה לצל לא.} פירש רש״י ז״ל: דהאי צורך גופו וצורך מקומו דר׳ נחמיה, היינו לצורך תשמיש המיוחד לו, הא לאו הכי לא, וכדאמרינן (לקמן שבת קמו, א) דלר׳ נחמיה אין כלי ניטל אלא לצורך תשמישו ואפילו טלית ואפילו תרווד, ותניא (לעיל שבת לה, ב) אין מטלטלין לא השופר ולא את החצוצרות, ואוקימנא לה כר׳ נחמיה כדאיתא בשלהי פרק במה מדליקין (שבת לו, א) ואע״ג דשופר הא חזי למלאכת היתר לגמע בו מים לתינוק כדאיתא התם, ולקמן בסמוך תניא אין מסיקין לא בכלים ולא בשברי כלים, ואוקימנא כרבי נחמיה, אע״ג דהסקה ביום טוב מלאכת היתר הוא אלא שאינה מלאכה המיוחדת להן, אלא על כרחין האי צורך גופו דר״נ דהכא אינו כצורך גופו דרבנן, אלא צורך תשמישו המיוחד לו, כגון טלית ללבוש תרווד לנער בו את הקדרה סכין לחתוך בה, וצורך מקומו נמי היינו בצורך תשמיש המיוחד לו לר׳ נחמיה אליבא דרבא, דכשם שמייחדין להשתמש בהם כך מייחדין להסתלק אחר תשמישן. זו היא שיטת רש״י ז״ל. וכ״כ (ר״ת) [ר״ח] ז״ל. ולפי דבריו הא דחבית שנשברה, דקתני שובר אדם את החבית (קמו, א) ומוקמינן לה בדרוסות ואליבא דר׳ נחמיה, צריכין אנו לפרש כגון ששברה במקצוע של דבילה, ושלא כדברי רש״י ז״ל עצמו שפירש שם בקרדום וסייף, ואגב חורפיה לא עיין בה. ומעתה אתיא שפיר ההיא דאמרינן בפרק בכל מערבין (עירובין לד, ב) גבי נתנו למגדל דקטיר במתנא ובעי למיפסקיה ואליבא דרבי נחמיה, והשתא הוי טעמא לפי שאינו עשוי למיפסק אלא לחתוך בו בשר ולפיכך לא הוי תשמיש המיוחד לסכין, וכן הטעם להא דאמרינן לקמן (שבת קמו, א) גבי חותמות אבל לא מפקיע ולא חותך.}
\textblock{\textbf{אבל ר״ת ז״ל הקשה עליו, דהא מדפליג ר׳ נחמיה בסיפא דמתניתין דהכא, לכאורה משמע דברישא לא פליגי אלא כולהו תנאי מודו בה, דאלמא לכל צורך גופו שיהא מלאכת היתר שרי לטלטלה ואפילו לרבי נחמיה. ואפשר לי לתרץ בהא, דדלמא נטר להו רבי נחמיה עד סיפא ופליג אכולהו. אלא הא קשיא לי קצת, מדקאמר ת״ק כל הכלים ניטלין לצורך ושלא לצורך, ורבי נחמיה פליג בה ואמר אין ניטלין אלא לצורך, לכאורה משמע דלא פליגי כלל אלא בשלא לצורך הא לצורך שוין בו, דאם איתא הוי ליה למימר אלא לצורך תשמישו. ואפשר דמשום דת״ק סתים לישניה      } דאמר לצורך, ולא פירש מאי לצורך, אמר איהו נמי לצורך, ולעולם האי לצורך למר כדאית ליה ולמר כדאית ליה.\par \textbf{} ור״ת ז״ל הקשה עוד, דדוחק הוא לחלק במימרא אחת במימריה דרבה, בין גופו ומקומו דרבנן לגופו ומקומו דר׳ נחמיה. וגם בזה יש לי לומר דהכא סמיך אלישנא דמתניתין וכדאמרן. ונמצא לפי דברי רש״י ז״ל, דכלי שמלאכתו לאיסור אינו ניטל כלל לצורך גופו לפי שאינו מיוחד אלא למלאכת איסור, אבל לצורך מקומו ניטל שהרי נטילתו בתשמיש המיוחד לו, ואע״ג דבכולה שמעתין שריא טפי לצורך גופו מצורך מקומו, הכא לא קשיא, דבדין הוא דאפילו לצורך גופו יטלטלנו, אלא שאי אפשר דלא משכחת לה צורך גופו. ולי נראה דיש לנו כיוצא בזה בפרק נוטל (שבת קמב, ב), גבי כר שהניח עליו מעות דלצורך גופו אסור ולצורך מקומו מותר.\par \textbf{} אבל ר״ת פירש דלדברי רש״י ז״ל, דלרבי נחמיה אין כלי שמלאכתו לאיסור מיטלטל כלל ואפילו לצורך מקומו דאינו תשמיש המיוחד לו, דדוקא בכלי שמלאכתו להיתר שרי לפי שמייחדו להסתלק אחר תשמישו, וזה הואיל ואינו משתמש בו אף הוא אינו מסלקו. והראשון נראה יותר. והיינו נמי דרבא לא אשכח פתרי במדוכה אלא במחמה לצל, ולא מצי לאוקמה כרבי נחמיה, ואילו מצי לאוקמה כרבי נחמיה טפי הוה ניחא ליה לאוקמה כותיה משום דאין מטלטלין אותה לכאורה לגמרי משמע וכדמקשי ליה אביי מינה. וכמו שכתבתי למעלה.\par \textbf{} ויש מפרשים, דלרבי נחמיה אפילו כלי שמלאכתו לאיסור משכחת ליה לצורך תשמישו, דלא אסר רבי נחמיה אלא במשנה אותו למלאכה אחרת, כגון קורנס של נפחים לישב עליו או לכסות בו את הכלי ואף על גב דישיבה וכיסוי מלאכה (גופו) של היתר הוא, אבל לפצוע בו אגוזים תשמישו המיוחד לו הוא שהרי מיוחד הוא להכות בו, ואף על גב שמשנה אותו מהכאת מלאכת איסור להכאת מלאכת היתר, אין זה שינוי תשמישו. וכן מכבדות של תמרה העשוין לכבד את הבית שהיא מלאכת איסור למאן דאית ליה הכין, שרי לטלטולה לכבד בה את המטות ואת השלחן שהן מלאכות של היתר. ומדוכה לרבי נחמיה דלא שרי לפצוע בתוכה אגוזים, היינו דדיכה לחוד ופציעה לחוד ולאו מלאכה אחת הן, וכן נמי ההיא דעירובין דמיפסק מיתנא בסכין לאו מלאכתו היא, דחתיכה לחוד והפסקה לחוד.\par \textbf{} ויש מקשים לפירוש זה דאם כן יפסקנו במספרים, וכן חותמות שבקרקע יחתוך במספרים. ומצאתי לר״ח ז״ל שם בערובין בפרק בכל מערבין (עירובין לה, א) פירוש במנעל דקטיר במתנא עסקינן דבעי סכינא למיפסקיה ונאבד הסכין, כלומר דאילו לא נאבד הסכין עירובו עירוב לכולי עלמא דחתיכה תשמישו המיוחד לו לסכין הוא, אלא כשנאבד הסכין ואינו יכול לחתכו אלא צריך קרדום או מרא שאין תשמישן לכך, והלכך לרבנן עירובו עירוב דכל הכלים ניטלין לצורך גופו, ור׳ אלעזר סבר כרבי נחמיה. ופירוש נכון הוא.\par \textbf{} ור״ת ז״ל פירש, דלרבי נחמיה אפילו כלי שמלאכתו לאיסור לצורך גופו מותר, ודוקא לצורך תשמישו שרגיל לו לפעמים לעשות בו בחול ואע״פ שאינו מיוחד לכך, ותרווד וטלית דקאמר היינו לשנותו למלאכה שאינו רגיל בה בחול כלל, ורישא דמתניתין דכל הכלים לכולי עלמא, ואפילו רבי נחמיה מודה בה דכולהו הני דמתניתין תשמישן הוא בהכי בחול לעתים, והיינו דשרי רבי נחמיה [ב]ההיא דחבית לטלטל אפילו קרדום לחתוך בו את החבית הואיל וצריך לחתוך בו הגרוגרות הדרוסות, אבל ההיא דעירובין (לה, א) דאסר לטלטל סכין למיפסק מיתנא, וקרדום לשבור בו את החבית, ושופר וחצוצרות לשתות, וסכין להפקיע ולחתוך בו חותמות, ולהסיק בכלים, היינו דכל אלו משום שאינם רגילין כלל אלא באקראי בעלמא ולא מסיק אדעתיה עלוייהו, ולפיכך מחמיר רבי נחמיה דאין זה קרוי תשמישו. ורבנן סברי אפילו אינו תשמישו לצורך גופו קרי׳ ליה, ואע״פ שאינו ראוי לאותה מלאכה כגון חצוצרות לגמע בו מים לתינוק, ורבי יהודה סבר דלצורך גופו ואפילו למלאכה שאינו רגיל בה אלא באקראי בעלמא שרי לצורך גופו ולצורך מקומו, דסתמא דהא מתניתין בין לר״ש בין לרבי יהודה היא. ורב נמי דסבירא ליה במוקצה כר׳ יהודה, סבירא ליה כרבא כדאיתא לקמן בסמוך. ודוקא למלאכה שהוא ראוי לה, אבל למלאכה שאינו ראוי לה אסור, והיינו דאסר חצוצרות בשלהי פרק במה מדליקין (לעיל שבת לה, ב).\par \textbf{} וקשיא לי טובא להדין פירושא, חדא דאפילו לדידיה ז״ל תיקשי ליה מאי דאקשי איהו לרש״י ז״ל, דצורך גופו וצורך מקומו לרבנן לאו היינו לצורך גופו וצורך מקומו דר׳ נחמיה. ועוד דהא מדאמר רב יהודה בריש פרקין (שבת קכב, ב) קורנס של אגוזים לפצוע בו אגוזים אבל קורנס של נפחים לא, ואקשי ליה רבה רחת ומזלג, ואתא אביי לסיועיה לרב יהודה ואותביה לרבה ממדוכה, ופירקה רבה ואמר הא מני ר׳ נחמיה היא, אלמא משמע דלר׳ נחמיה קורנס של נפחים לפצוע בו אגוזים לא שרי וכרב יהודה, ומתניתין דלא כר׳ נחמיה. ומכאן קשיא לי גם כן לפירוש האחר שכתבתי. ודברי רש״י ז״ל נראין עיקר.}
\textblock{ ירושלמי (פט״ז, ה״א):\textbf{ מדוכה אם יש בה שום מטלטלין אותה, ואם לאו אין מטלטלין אותה, } רשב״ג אומר אם היתה מדוכה קטנה נותנה על גבי השלחן הרי היא כקערה ומטלטלין אותה. ע״כ. ומסתברא דלא פליג בה ת״ק.}
\textblock{ מדאקשו:\textbf{ לרבה אליבא דר״נ הני קערות היכי מטלטלין להו.} משמע דלרבא לא קשיא מינה מידי, משום דלרבא מטלטל להו לצורך מקומן, ומהא נמי שמעינן דכל שמטלטל לצורך מקומו, אין אומרין כיון שסלקו מאותו מקום שהוא צריך לו שומטו מיד ומניחו במקומו, אלא כיון שמטלטלו לצורך מקומו מניחו באיזה מקום שירצה, דאי לא, לרבא נמי תיקשי היכי מטלטלין להו ומפקינן להו מכוליה ביתא, והכי נמי מוכח כל ההיא סוגיא דפרק כירה (לעיל שבת מז, ב) גבי נותן כלי לקבל בו הניצוצות.}
\textblock{ הא ד\textbf{תנן בית שמאי אומרים אין מוציאין לא את הקטן ולא את הספר תורה וכו׳ ובית הלל מתירין.} כבר כתבתיה בארוכה בסייעתא דשמיא במסכת ביצה (יב, א ד״ה אלא).}
\textblock{ הא דאמר ליה רבא לרב מרי:\textbf{ חזו לאורחים.} לדבריו קאמר ליה, דאילו לרבא אפילו לא חזו כלל ולא צריך להו מטלטלין מחמה לצל, וכדאמר ליה לבסוף.}
\textblock{\textbf{בהא לימא ר׳ אלעזר אף של תמרה.} ונראה כי בה״ג גריס: הוא דאמר כר״ש דאמר דבר שאין מתכוון מותר. ואכתי קשיא לי קצת דמכל מקום היכי מתמה ואמר בהא לימא ר״א אף בשל תמרה, כיון דאיכא למימר דכר״ש סבירא ליה. ושמא לא גריס ליה להאי לישנא כולה, אלא ה״ג בשל תמרה ור״א כר״ש סבירא ליה. (ולדידן) [ולדבריו] גירסא קיימא דאף של תמרה שרי, ומכבדין את הבית דלא פסיק רישיה ולא ימות הוא, דזימנין דלא משוי ליה גומות. וכתב הראב״ד ז״ל דדוקא כשכבד את הבית מערב שבת, הא לאו הכי אסור, דודאי אתי לאשוויי גומות, ודוקא בדליכא קליפי רמונים וכיוצא בהן שאינן ראוין לטלטל. ותמיהא לי שאם כן כלל וכלל לא יכבד דהא מזיז עפר ממקומו. והגאונים ז״ל כולם פסקו דמותר לכבד את הבית, ואפילו במכבדות של תמרה, וכבר כתבתיה בארוכה בסוף פרק המצניע (שבת צה, א) בסייעתא דשמיא.}
\textblock{\textbf{שברי זכוכית לכסות פי הפך.} ירושלמי (דפרקין, ה״ה): תמן תנינן (לעיל שבת פא, א) זכוכית כדי לגרוד בה בראש הכרכר, והכא הוא אמר הכין רבי אחא רבי מיאשה רבי כהן בשם רבנן דקסרין כאן בעבה כאן בחדה, ואית דבעי מימר כאן במטלטל כאן במוציא.}
\textblock{\textbf{חרס קטנה מותר לטלטלה בחצר.} פירוש: משום (דשייכי) [דשכיחי] בה כלים וחזיא לכסויי ביה מאנא. ודוקא חרס דאתי משיורי כלים, אבל אבן דלא אתיא משיורי כלים לא, והיינו טעמא דכל הני שמעתין.}
\clearpage
\newsection{דף קכה}
\textblock{ הא ד\textbf{אמר שמואל קרומית של קנים מותר לטלטלן בשבת.} איכא מאן דמפרש דדוקא בשנשברה בשבת הא בחול לא, משום דאין אדם מצניען אלא זורקן לאשפה או מיחדן להסקה, ומיהו בשנשברה בשבת והוכנו אגב אביהם והכי נמי בשבת חזיין למלאכתן ראשונה שרי, הא לא חזי למלאכתן ראשונה לא ואע״ג דחזו לכפורי ביה טנופא, כיון דבחול אדם זורקן לאשפה או מסיקן תחת תבשילו, והיינו דאיצטריך בר המדורי לפרשה, משום דמחצלת גופה למאי חזיא לכסויי בה עפרא האי נמי חזייא לכסויי ביה טנופא, ואע״ג דלא קיימא לן כרבי יהודה דבעי מעין מלאכתן ראשונה. ומחט של יד שנטל חררה או עוקצה שאסר רבא לעיל (שבת קכג, א) וקא יהיב טעמא לפי שאדם זורקה בין גרוטאות, ההיא בשניטל חררה ועוקצה מערב שבת, הא ניטלו בשבת שרי הואיל והוכנו אגב אביהם וחזיין למלאכתן הראשונה, דהא בחול נמי נוטלין בה את הקוץ. כך נראה לי לפי פירוש זה.}
\textblock{ הא ד\textbf{אמר ר׳ זירא אמר רב שיירי פרוזמיות אסור לטלטלן.} פירוש: שיירי טליתות, ואמר אביי במטלניות שאין בהן שלש על שלש דלא חזיא לא לעניים ולא לעשירים, נראה משום דפחות מכן אינן ראוין כלל וסתמא אדם זורקן לאשפה, וכדמשמע בפרק במה מדליקין (שבת כט, ב) גבי פתילת הבגד שקפלה כי לא תלאו במגוד ולא הניחו אחורי הדלת, מיהא לכולי עלמא הוי ליה כשברי עריבה שאין בה כדי לכסות בה פי החבית. אבל ראיתי בשם הראב״ד דהכא בשיירי טלית של מצוה עסקינן דלא מיכפר ביה טנופא, הא בשיירי טליתות דעלמא אפילו      מכאן שרי דלא גרע מחרס קטנה דחזיא לכפורי טנופא.}
\textblock{\textbf{מתקיף לה רב אשי אי הכי אדמיפלגי בשבריו ליפלגו בתנור גופה.} איכא למידק, וכי משום דטיהריה רחמנא לא מטלטלין ליה, מכל מקום מאנא הוא וכל הכלים ניטלין בשבת. ויש לומר דכיון דתנור למלאכתו הוא מיוחד ולמלאכתו לא חשיבא ליה מאנא אף הוא לא חשבינן ליה כמאנא כלל, ולפיכך אין מטלטלין אותו. ואי נמי יש לומר דלדבריו דרבא קאמר, כלומר לרבא דמדמי טלטול לטומאה אי הכי לפלוג בתנור גופיה.}
\textblock{\textbf{אמר רבינא כמאן מטלטלין האידנא כסויי דתנורא, כמאן כר״א.} והרב אלפסי ז״ל פסק כן דמטלטלין אותן אע״ג שאין להם בית אחיזה, וכן נראה מדאמר רבינא דמטלטלין. אלא שקשה עליו שהוא ז״ל פסק בשלהי פרקין כת״ק דמתניתין דאסר אפילו כיסויי כלים דחברינהו בארעא, והיינו כיסוי דתנורא, וזה מתפיסותיו של הר״ז הלוי ז״ל. ושמא הרב אלפסי ז״ל מפרש דחברינהו בארעא, כגון אותן חביות שחופרין גומא וטומנין כל גופן בתוכה ואין נראה מהם כלל אלא פיהם, דהשתא הוא דאיכא למגזר אטו בור ודות, אבל תנורין דעומדין על גבי קרקע אלא ששוליהן מחוברים בטיט לקרקע הרי הן ככלים, ומעדותו של ר׳ יוסי דאמר משמיה דר״א למדנו כן, וכי קאמר רבינא כמאן מטלטלינן כיסוי דתנורי, לאו דוקא דתנורי אלא דתנורי וכל דדמי להו. וכן פירש הרמב״ם ז״ל (פכ״ה הל׳ יג).\par \textbf{} והרמב״ן ז״ל תירץ, דרבינא דוקא דתנורי קאמר, וטעמא משום דהרי הוא כגופו של דבר, לפי שהתנור צריך לו שמסייע באפייתו, מה שאין כן בשאר כיסוי הכלים. ולא ירדתי לסוף דעתו דכל שכן שהוספת לאיסור לטלטלו בלא בית יד, דהא עיקר טעמא דבית יד כי היכא דלהוי בה היכרא דלא מבטל ליה לגבי קרקע והוי ליה כבונה. ודברי הרמב״ם ז״ל נראין עיקר.}
\textblock{\textbf{הכל מודים שאין עושין אהל ארעי בתחלה ביום טוב לא נחלקו אלא להוסיף.} וא״ת מנא ליה דפליגי בתוספת אהל ארעי, דלמא לא פליגי אלא בטלטול בעלמא. יש לומר מדאפליגו בפקק החלון ופליגי בנגר הנגרר, דאי בטלטול לבד ליפלגו בנגר הנגרר לבד, אבל השתא דפליגי בפקק החלון שמע מינה דגם בתוספות אהל פליגי, פליגי בפקק משום תוספת אהל ופליגי בנגר הנגרר משום טלטול כדאמרינן בסמוך כמחלוקת כאן כך מחלוקת בנגר הנגרר, דר״א סבר אם אין עליו תורת כלי אסור דהיינו קשור ותלוי, ורבנן שרו אע״פ שאינו תלוי והוא שיהא קשור. ועוד יש לומר דממתניתין גופיה נפקא להו, דקתני קשור ותלוי, וקשור בכולה שמעתין להתיר בטלטול ותלוי משום בנין.}
\textblock{\textbf{שאין עושין אהל עראי לכתחלה ביום טוב.} פרש״י ז״ל שאין קרוי אהל אלא בפורס על הגג לצל, אבל מן הצד צניעותא הוא ושרי, וכההיא דאמרינן בעירובין פרק כל גגות (עירובין צד, א) גבי פלוגתא דרב ושמואל בכותל שבין שתי חצרות דנפל דרב אסר לטלטל אלא בד׳ אמות, ושמואל שרי זה עד מקום עיקרו של כותל וזה עד מקום עיקרו של כותל, ואמרינן התם רב ושמואל הוו יתבי בההוא חצר נפל גודא דביני ביני, אמר להו שמואל נגודו לי גלימא, אהדרינהו רב לאפיה, ואמרינן עלה היכי עביד שמואל הכי והא הוא דאמר זה מטלטל עד מקום עיקר מחיצה וזה מטלטל עד עיקר מחיצה, ופרקינן שמואל דעבד לצניעותא בעלמא הוא דעבד, ולא אקשו והיכי עבד שמואל הכי דהא אין עושין אהל עראי בתחלה בשבת, דאלמא אין לחוש למחיצת עראי מן הצד.\par \textbf{} ורב דאהדריה לאפיה פירש רש״י ז״ל: משום דאסר להביא הגלימא, וכטעמיה דאסור לטלטל אלא בארבע אמות, ומתוך כך מוחק רש״י ז״ל בפרק קמא דסוכה (טז, ב) ובעירובין פרק כיצד משתתפין (עירובין פו, ב) מה שכתוב בספרים גבי ההיא דאמרינן פעם אחת שכחו ולא הביאו ספר תורה מערב שבת, למחר פרשו סדינין על גבי עמודים והביאו ספר תורה וקראו בו, וגרסינן בספרים פרשו לכתחילה סלקא דעתך והא אמר מר אין עושין אהל עראי וכו׳, ומחקה רש״י לגירסא זו דודאי במחיצה לא גזור, אלא הכי גרס: פרסו סלקא דעתך ותו לא, כלומר והיאך הביאום דרך רשות הרבים.\par \textbf{} ומה שאסר ר״א כאן בפקק החלון אע״פ שהוא תוספת מן הצד, היינו משום דהוא מחיצה קבועה, כך פירש רש״י ז״ל כאן. ואינו מביא כאן מחלוקת אלא לומר דר״א אוסר להוסיף בכל מחיצה שאסור לעשותה לכתחילה, בין מלמעלה ואפילו באהל עראי, בין מן הצד במחיצה קבועה, ורבנן מתירין בין מן הצד במחיצה קבועה כגון פקק החלון אע״פ שאינו תלוי, ובתוספת אהל אפילו מלמעלה כגון עובדא דעירובין (קב, א) דהנהו דכרי דהוה ליה לר״ה דביממא בעו טולא ובליליא בעו אוירא, אתא לקמיה דרב אמר ליה זיל כרוך בודייא ושייר בה טפח למחר מוסיף על אהל עראי ושפיר דמי.\par \textbf{} אבל ר״ת ז״ל הקשה עליו דלא אשכחן עיקר לפלוגתייהו דר״א ורבנן אלא בפקק החלון דהיינו מן הצד. ועוד דההיא נמי דעירובין דאהדרינהו לאפיה דהוצרך רש״י ז״ל לפרש משום דאיהו אסר להביא הגלימא משום דאסר לטלטל אלא בד׳ אמות, קשיא, דהא אפשר להביאה דרך מלבוש. וההיא נמי דפרשו סדינין ליכא למימר דקשיא ליה האיך הביאום, דדלמא בדרך מלבוש הביאום. ועוד קשיא הא דאמרינן בעירובין פרק מי שהוציאוהו (עירובין מד, א) גבי פעם אחת לא נכנסו לנמל, דאקשינן מתניתין אהדדי הא דתניא נפל דופנה לא יעמיד בה אדם ובהמה וכלים ולא יזקוף את המטה לפרוש עליה סדין, אהא דתניא עושה אדם חברו דופן כדי שיאכל וישתה וישן ויזקוף את המטה ויפרוש עליה סדין כדי שלא תפול החמה על המת ועל האוכלין, ופרקינן לא קשיא הא רבי אליעזר הא רבנן, דתנן פקק החלון ר״א אומר בזמן שקשור ותלוי וכו׳, ואקשינן עלה ולאו מי אמר רבה בר בר חנה הכל מודים שאין עושין אהל עראי בתחלה בשבת, לא נחלקו אלא להוסיף, ואסיקנא כלים אכלים לא קשיא הא בדופן שלישי הא בדופן רביעית, אלמא אף בדופן שייך אהל עראי.\par \textbf{} על כן פירש ר״ת ז״ל: דתוספות אהל עראי שייך בין למעלה בין מן הצד, והקשה הוא ז״ל גבי ההיא דמי שהוציאוהו (מד, א) אפילו דופן שלישית נמי לישתרי, דליהוי שתי מחיצות כטפח דהנהו דכרי דבי רב הונא ושלישית כתוספת ולישתרי כרבנן דשרו בתוספת אהל ארעי. ועל כן כתב הוא ז״ל, דכל מחיצה שהיא באה להתיר, כגון דופן שלישית בסוכה ואי נמי בדופן שבין שתי חצרות לרב, יש בו משום עשיית אהל והרי הן כטפח ראשון על גג, והיינו דאהדרינהו רב לאפיה, והיינו דאקשינן גבי פרשו סדינין והאמר מר אין עושין אהל עראי וכו׳, אבל דופן רביעית שאינה באה להתיר שאינה אלא לצל אינה אלא כתוספות אהל ושרי לרבנן, ולר״א בין בשלישית בין ברביעית אסור משום תוספות אהל.}
\textblock{ הא ד\textbf{א״ר אבא בר כהנא בין קשור בין שאינו קשור והוא שמתוקן.} פירוש: מתוקן ומוכן במחשבה קאמר, דהא אסקה ואמר אנא דאמרי כי האי תנא, כלומר כרשב״ג, ורשב״ג במחשבה בעלמא סגי ליה. וגבי רבי יוחנן הוא דאמרינן דפליג עליה בחדא דבעי תורת כלי, אבל רבי אבא משמע דלגמרי סבירא ליה כרשב״ג. והא דאייתינן בסמוך אמר ר״י בר שילא א״ר אסי א״ר יוחנן הלכה, לא גרסינן ביה ואמר ר״י, אלא אמר ר״י ומלתא באפי נפשה היא, ולא הוה אסוקי מלתא דר׳ אבא דאמר אנא דאמרי כי האי תנא. וכמו שכתב רש״י ז״ל.}
\clearpage
\newsection{דף קכו}
\textblock{\textbf{המונח כאן וכאן אסור.} פירש רש״י ז״ל: בשתוקעו בשבת הוי בונה ואסור מדאורייתא, ובתוס׳ פירשו דאין כאן אלא איסורא דרבנן, דומיא דנועלין במקדש אבל לא במדינה דליכא אלא איסורא דרבנן.}
\textblock{ הא ד\textbf{א״ר יוחנן והוא שיש תורת כלי עליהן.} פירש רש״י ז״ל: שיהא ראוי לתשמיש אחר. ור״ת ז״ל הקשה עליו בספר הישר (סי׳ רכ״ה) דמה לנו שיהא ראוי לתשמיש אחר, כיון שהוא ראוי לתשמיש זה. ועוד הא אנן תנן כל הכלים ניטלין בשבת, ולמה יצריך רבי יוחנן בנגר וכסויי הכלים יותר מבשאר כלים, ואילו היה צריך שיהא תורת כלי עליו לשאר דברים ולומר שיהא צריך כלי גמור בעלמא, בהא לא פליג ר׳ אליעזר ולא בעי בכי הא קשור ותלוי, ואין לך תנא דמחמיר ביותר מת״ק דרשב״ג ובקשירה בעלמא שרי. על כן פירש הוא ז״ל דעיקר דברי רבי יוחנן אינו אלא שיהא מתוקנים לכך, ולאו תורת כלי ממש בעי אלא מעשה שנראה וראוי ותיקונו מוכיח עליו שהוא מתוקן לכך או שיקשרנו. ובהא הוא דפליג אדרשב״ג, דאילו רשב״ג לא בעי לא מעשה ולא קשירה אלא מחשבה בעלמא, ורבי יוחנן פסיק כותיה דרשב״ג בחדא דלא בעי קשור ותלוי, ופליג עליה בחדא דבמחשבה לחוד לא שרי ליה בטלטול עד שיעשה בו מעשה ויתקננו לפום כך לפי מה שהוא צריך לו כנגר שיעשה לו קלוסטרא או תיקון אחר שיהא נגר המתוקן להיות נועל בו, ולא משום דבעי קלוסטרא לשוייה כלי בעלמא. ונגר שכאן היינו נגר שנלקח מן העצים שאין עליו שום היכר כלל, ועל כן הצריך שיתקננו במעשה גמור שיהא ראוי לנעול בו, וכסוי הכלים גם כן שבכאן הם גם כן דפים בעלמא, ועל כן הצריך בהם רבי יוחנן תורת כלי, כלומר שיעשה בו מעשה שיהא ראוי וניכר שהוא עומד לכך, ולא תורת כלי גמור שאם כן לא צריכא למימר.}
\textblock{\textbf{ומדברי הראב״ד ז״ל נראה דתורת כלי ממש בעינן, ואע״ג ששנינו כל הכלים ניטלין בשבת ודלתותיהן עמהם, שאני כלים וכסויין דמיירי דמחברי בארעא. וזה לשונו: לא הצריכו חכמים בית אחיזה ותורת כלי לכסוי אלא לכלים דחברינהו בארעא, אי נמי לכל מידי דהוי תוספת או שמוש למחובר, כגון נגר הנגרר או פקק החלון וקנה שהתקינו להיות פותח ונועל בו. וההיא (עירובין קב, א) שריתא ואסיתא דבי מר שמואל וכיוצא בהן משמשי מחובר דגזרינן בהו משום כסויי קרקעות והדומה להן, אבל כל דבר שהוא משמש את התלוש, אינו צריך בית אחיזה ולא תורת כלי אלא שיהא מתוקן, ואם תאמר אפילו מתוקן לא, דהא תנן (קכד, ב) שברי עריבה לכסות בה את פי החבית, וכי מי תיקנו לזה לכסויי      } חבית בשבת. התם כיון דבאין מכלי גדול כבר הוכנו על גבי אביהן. ע״כ.\par \textbf{} וצריך לי עיון דאם כן מאי קא מקשה ומי אמר רבי יוחנן הכי, והא אמר רבי יוחנן והוא שיש תורת כלי עליהן, וכי תימא הכי נמי דאיכא תורת כלי ומי בעי רשב״ג תורת כלי והתניא חריות של דקל כו׳, ומאי קושיא לעולם אימא לך רשב״ג בקנה שהתקינו להיות פותח ונועל בו תורת כלי בעי משום דמשמש מה שמחובר לקרקע, וחריות של דקל מטלטלין לגמרי. וצל״ע.}
\textblock{\textbf{ואמר רב יהודה בר שילא אמר רב אסי א״ר יוחנן והוא שיש תורת כלי עליהן.} איכא למידק והא רב אסי הוא דאמר לעיל (שבת קכה, ב) משמיה דרבי יוחנן דבהנחה בעלמא נעשית האבן כסוי לחבית ולא בעי מעשה ולא תורת כלי, ואמר נמי משמיה דר׳ יוחנן צאו ושפשפום אמר להן, ולא הצריכן מעשה אחר ולא תורת כלי. ותירץ ר״ת ז״ל דבפלוגתא דרבי אמי ורבי אסי דלעיל בשוכח ומניח, בדרבי אמי גרסינן אמר ר׳ יוחנן, אבל בדרבי אסי לא גרסינן בה ר׳ יוחנן, וכתב שכן מצא בספרים מדוקדקים, וההיא משמיה דידיה והא דהכא משמיה דרביה. ולספרים דגרסי בה רבי יוחנן, איכא למימר דאמוראי נינהו אליבא דר׳ אסי. וזה קשה קצת דאם כן לא הוה שתיק גמרא מיניה והוה מקשה לה ומתרץ לה הא דידיה הא דרביה אי נמי אמוראי נינהו אליבא דר׳ אסי.\par \textbf{} אבל ראיתי להרמב״ן ז״ל ענין יעלה לנו בו תירוץ לקושיא זו. והוא שאמר דפלוגתא דר׳ אמי ור׳ אסי באבן שעל פי החבית ובנדבך של אבנים לא דמיא כלל לכולהו אינך, דבין בחריות בין בנגר בין בכיסוי הכלים להתירן לגמרי בהתקנתן פליגי, והא דאבן שעל פי החבית וזו דנדבך אינו אלא להתיר אותן למלאכה שתקן בה מערב שבת להשתמש בה למחר, כלומר לסלק האבן למחר מעל פי החבית ולהחזירה על פי החבית, וכן אבני נדבך לסדרן לישיבה למחר בשפשוף זה שעשה להן מבערב שהוא מתקנן בכך לישיבתן, ואע״פ שאין עשוין להתיחד לכך ואין עליהן תורת כלי כלל. וזהו שדימו בגמרא מחלוקת זו דאבן שעל פי החבית למחלוקת של נדבך שאין אדם מיחדן לישיבה, כדמוכח בפרק כירה (שבת מג, א) דגבי לבנים מתרצים דאייתור מבניינא דחזיין למזגא עלייהו וגבי אבנים מתרצים תירוצא אחרינא, אלמא לא חזו למזגא עלייהו או משום דדמיהן יקרים או משום דישיבתן קשה כדאיתא בירושלמי אלו דבריו ז״ל.\par \textbf{} ומכלל דברים אלו יצא לנו דאפשר דאיתא להא דרב אסי משמיה דרבי יוחנן באבן שעל פי החבית ובאבני נדבך דלא בעינן תורת כלי לאותה מלאכה שהתחיל בה מבערב, ואיתא נמי להאי דהכא דאמר ר׳ אסי משמיה דר׳ יוחנן דבעינן תורת כלי להתירן לגמרי בהתקנתן, וזה יותר נכון. ובספר המאור [כתב], דר׳ יוחנן לא בעי אלא מעשה כל דהו, [ושפשוף] בדבר שצריך שפשוף, והנחה מבעוד יום על גבי הכלי הויא מעשה כל דהו.\par \textbf{} ולענין פסק הלכה, פסק הרב אלפסי ז״ל כר׳ אבא דאמר אע״פ שאינו קשור ואינו תלוי, והוא שיש תורת כלי עליו כר׳ יוחנן. ואע״פ שפסק בעירובין בפרק המוציא (קב, א) בנגר הנגרר כרבי יהודה, וכדפסק התם שמואל דאמרינן התם אמר ר״י אמר שמואל הלכה כרבי יהודה, ורבא נמי אמר התם והוא שקשור בדלת. לא קשיין הני פסקי אהדדי, דהתם שאין תורת כלי עליו, (דהתם) דאינו מתוקן אלא במחשבה גרידא, ולשוייה עליה תורת כלי בעינן קשירה. וכ״כ גם ר״ת בספר הישר (סי׳ רכה). והא דפסק שמואל התם כר״י דבעי קשירה, אע״ג דאיהו כרשב״ג סבירא ליה בחריות של דקל שגדרן לעצים כדאיתא בפרק במה טומנין (שבת נ, א). לא קשיא, דבכל מידי דמחבר בארעא מחמיר טפי.}
\textblock{ מתני׳:\textbf{ מפנין תרומה טהורה.} כתב רש״י ז״ל: אבל טמאה לא, דאפילו לבהמת כהן לא חזיא, ואע״ג דבחול חזיא ליה וכדאיתא בפרק כל שעה (פסחים לב, א) דאם רצה כהן מריצה לפני כלבו, בשבת ויום טוב לא חזיא, דזו היא ביעורה מן העולם או להסיק או לבהמה, ואין מבערין תרומה וקדשים ביום טוב. ואין דבריו נכונים, דמחו להו מאה עוכלי בעוכלא, דתרומה טמאה אינה ראויה לכלב דבת שריפה היא, ודרך שריפתה התירה לו הכתוב וכדאמרינן שלך תהא להסיקה תחת תבשילך, ואי נמי להדליקה בנר דזו נמי שריפה היא, ואמר רבא בפרק במה מדליקין (שבת כה, א) כשם שמצוה לשרוף קדשים שנטמאו כך מצוה לשרוף תרומה שנטמאת, ובהדיא שנינו בפרק בתרא דתמורה (לד, א) אלו הן הנשרפין חמץ בפסח ותרומה טמאה כו׳, ותניא בברייתא אוכלין בשריפה ומשקין בקבורה. והא דתניא בפסחים (לב, א) מריצה לפני כלבו, לאו בתרומה טמאה תניא, אלא בתרומת חמץ וכרבי יוסי הגלילי דאמר חמץ בפסח מותר בהנאה.}
\textblock{ גמרא: גירסת הספרים:\textbf{ ושמואל אמר ארבע וחמש כדאמרי אינשי ואי בעי אפילו טובא נמי מפנין.} וכתב הרב אלפסי ז״ל בהלכות, דיש מי שפוסק כרב חסדא, משום דסוגיא דשמעתא כותיה, וסתם מתני׳ כותיה, וגמרא קא שקיל וטרי בשיעורא דמתני׳, ואיכא מאן דפסק כשמואל דמוקי לה כר״ש דלית ליה מוקצה דקיימא לן כותיה. ע״כ. ואע״פ שלא הכריע הרב בפירוש כחד מנייהו, נראה שהוא ז״ל כרב חסדא פסק, שהרי כתב הא דבעו בגמרא הני ארבע וחמש דקאמרינן אע״ג דהוו אורחים טובא, או דלמא הכל לפי האורחים, וההיא ודאי לא שייכא אלא בדרב חסדא. ומכל מקום דוקא בפירושא דרישא דמתניתין איכא למימר דקיימא לן כדפריש לה רב חסדא, אבל באבל לא את האוצר על כרחין כפירושא דשמואל הוא, דהא רב חסדא מוקי לה כרבי יהודה, ואנן כרבי שמעון סבירא לן, ומאי לא את האוצר שלא יגמור את האוצר. ואלא מיהו דמתניתין לא דייקא הכי, דאי ארבע וחמש ארבע מחמש קאמר, הא בהדיא תני שלא יגמור את האוצר, ואם כן למה לי למיתני אבל לא יגמור את האוצר.\par \textbf{} ונראה לי דאי מפרשים לה כרב חסדא, כולה מתניתין כותיה מפרשא, ומאי לא את האוצר שלא יתחיל באוצר וסתמא כר״י, אלא אנן נקטינן מינה הלכתא בארבע וחמש דרישא, דבהא לא אשכחן דפליג ר״ש, אבל בלא יתחיל באוצר דהיינו סיפא לא, אלא כר׳ שמעון. ואע״ג דמקשה סתמי אהדדי בפרק מי שהחשיך (שבת קנו, ב) ולא אקשי מהא, משום דהא לא מכרעא, דאיכא לאוקומה אף כר׳ שמעון וכפירושא דשמואל. אבל רב אחא משבחא ז״ל גריס הא דשמואל ארבע וחמש כדאמרי אינשי, ומאי אבל לא את האוצר שלא יגמור את האוצר, ולא גריס בה ואי בעי אפילו טובא מפנין. וזו גירסא נכונה אילו הודו לה גירסת הספרים. והשתא אתיא כולה שמעתא אפילו כשמואל, וקיי״ל כותיה כיון דסוגיא דשמעתא לא קשיא ליה, חדא דמוקי לה למתניתין כר״ש, ועוד דשמואל קשיש מרב חסדא ועדיף מיניה.}
\newchap{פרק \hebrewnumeral{18} מפנין}
\clearpage
\newsection{דף קכז}
\textblock{}
\textblock{ הכי גריס רש״י ז״ל, וכן היא בספרים שלנו:\textbf{ תבואה צבורה בזמן שהתחיל בה מערב שבת מותר להסתפק ממנה בשבת.} ולפי גירסא זו הא דתנא עלה וכמה שיעור תבואה צבורה לתך, אליבא דר׳ אחא היא, ולא קיימא לן כותיה אלא כר״ש, ואפילו לתך ואפילו כורים מתחיל בה אפילו בשבת. והרב אלפסי ז״ל שכתבה בהלכות, גריס בה מותר לטלטלה. וכתב הרמב״ן ז״ל דלפי גירסא זו נראה שהרב ז״ל מפרשה כמתניתין במפנה מפני האורחים, אלא דבמתניתין לא שמענו אלא תבואה שבקופות אבל תבואה צבורה שבאוצר לא שמענו, ונחלקו בה בברייתא רב אחא אסר לטלטל ור״ש מתיר, ותנא כמה תבואה צבורה אתיא אליבא דר״ש, כלומר כמה שיעור תבואה צבורה מתיר ר״ש לטלטל לתך. ולפי דברי הירושלמי היינו נמי כשיעור ארבע וחמש קופות דמתניתין, דבירושלמי (בפרקין, ה״א) גרסינן התם, ר׳ זעירא שאל לר׳ יאשיה כמה שיעור קופות, אמר ליה נלמוד סתום מן המפורש, דתניא בשלש קופות של שלש סאין תורמין את הלשכה. וכן נראה לי אפילו מסוגיא דשמעתין דקופות שיעור ידוע יש להם, מדאיבעיא להו הני ארבע וחמש קופות בד׳ וה׳ אין בטפי לא או דלמא מעוטי במשוי עדיף, ואי קופות כל שהן בין קטנות בין גדולות קאמר, מאי קא מיבעיא להו, וכולה סוגיא מוכחא הכין. אבל הברייתא הראשונה דקתני אין מתחילין באוצר כר״י אתיא וליתא, ולא כתבה הרב ז״ל בהלכות אלא בגררא דתניא אידך. אבל התימא בדברי הרב ז״ל, שלא כתב כלל בעיא דמעוטי בהלוכא עדיף או מעוטי במשוי עדיף, וכל חד וחד מפני׳ לנפשיה ארבע וחמש קופות כשיעורא דמתניתין. וכן פירש הרמב״ן ז״ל.}
\textblock{\textbf{דמאי הא לא חזי ליה כיון דאי בעי מפקר ליה לנכסיה.} איכא למידק ואמאי איצטריך ליה למימר הכין, דאפילו לא חזו ליה כיון דחזי לעניים ולאכסניא שפיר דמי, וכדאמרינן גבי תרומה הואיל וחזיא לכהנים, ולקמן נמי גבי זכוכית מפני שהוא מאכל לנעמיות אמרינן אפילו אין לו נעמיות הואיל וראוי. על כן אמרו בתוס׳ דלא גרסינן ליה, אלא הכי גרסינן דמאי הואיל וחזי לעניים דתנן מאכילין את העניים דמאי.}
\textblock{\textbf{ויש מעמידים הגירסא ואומרים דגבי תרומה הואיל ועל כרחיה יהיב לכהנים, וזכוכית נמי הואיל ולא חזי אלא      } לנעמיות, וכן הלוף לעורבים, איכא למימר כיון דחזו להכי שרי, אלא דמאי שהוא מצניעו עד למחר לתקנו לאכילתו והשתא לא חזי ליה לא מטלטל ליה, אלא כיון דאי בעי חזי ליה השתא נמי מטלטל ליה דראוי הוא לו. ויש מי שאומר דשיטפא דגמרא היא דמשום דעל כרחין איצטריך למימר דחזי לעניים ולאתויי מתניתין דמאכילין את העניים דמאי ואת אכסניא דמאי, ואתמר בגמ׳ וכשיטה דהכא בדוקא בפסחים (לה, ב), ובברכות (מז, א), אמרו הכא נמי הכין ולא הוה צריך ליה.}
\clearpage
\newsection{דף קכח}
\textblock{ מתני׳:\textbf{ חבילי קש וחבילי עצים וחבילי זרדים אם התקינן למאכל בהמה מטלטלין אותן ואם לאו אין מטלטלין אותן.} קשיא לי כיון דראוין למה לי הכין, ומאי שנא משברי זכוכית דאע״ג דאין לו נעמיות ולא הכינן לכך מטלטלין אותן הואיל וראוי להן, והכי נמי יטלטל מיהא את התבן דראוי לבהמה שהיא מצויה. ויש לומר דשאני הני דסתמן להסקה קיימי וזו היא שאוסרתן עד שיכינם בפירוש למאכל בהמה, אבל שברי זכוכית שאינו עומד לדבר אחר בסתמא כיון דחזי מיהא לנעמיות מטלטלין אותן. ונראה לי דבהא נמי פליג תנא דמתניתין אתנא דברייתא בחרדל, דתנא דמתניתין אסר ותנא דברייתא שרי מפני שהוא מאכל ליונים, דתנא דמתניתין סבר סתמא למאכל אדם קאי, וכיון דהשתא לא חזי ליה אע״ג דהשתא הוא מאכל נמי ליונים לא מטלטלין ליה, דמידי דחזי ליה (ד)מקצה ליה מיונים, ותנא דברייתא סבר הואיל והוא מאכל ליונים שהוא ידוע ליונים דעתו נמי איונים, כלומר דלא מקצה להו מנייהו וקיי״ל כתנא דמתניתין.}
\textblock{\textbf{ומולל ואוכל.} תוספתא במסכת יום טוב (פ״א, הי״ג): המולל מלילות מערב שבת והמולל מלילות מערב יום טוב והמולל מלילות בשבת, מנפח על יד על יד ואוכל, אבל לא בקנון ולא בתמחוי. וזו מכרעת כדברי הרב אלפסי ז״ל שפסק שמותר למלול בשבת כביום טוב אלא שבשבת על ידי שינוי בראשי אצבעותיו, וביום טוב אפילו כדרכו, שלא כדברי הר״ז הלוי ז״ל.}
\textblock{ הכי גריס רש״י ז״ל:\textbf{ אתמר בשר חי מותר לטלטלו בשבת, בשר תפל רב הונא אמר מותר לטלטלו כו׳.} ואין גרסתו נכונה בעיני, שלא מצאתי בשום מקום בשר חי כנגד בשר תפל, אלא אדרבה בשר חי משמע תפל, ובכל מקום שמפריש בין מליח לשאינו מליח נקיט בשר מליח כנגד בשר תפל, וכדקתני בברייתא לקמן דג מליח מותר לטלטלו בשבת דג תפל אסור לטלטלו, בשר בין חי בין מליח מותר לטלטלו, ותנן (חולין קיג, א) טהור מליח וטמא תפל וכן רבים. ועוד דאי אתפל למה אסר רב חסדא והא חזי ליה לאומצא, וכדמשני איהו נמי בבר אווזא שאני בר אווזא דחזי לאומצא. ואם תאמר בר אווזא דוקא דרכיך אבל בשר דעלמא לא, לא היא, דאם כן אפילו לר׳ שמעון לא יטלטלו דהא לדידיה לא חזי ליה, ואי משום כלבים נמי לא שרי ליה דמאי דחזי ליה מקצה ליה מכלבים, והיינו טעמא דדג תפל דברייתא דלקמן. ועוד דבשר בהמה נמי חזי לאומצא, כדאמרינן (חולין קטו, ב) השוחט בשבת מותר לבריא באומצא, וסתמא לאו בעוף מיירי. ובפרק כל הבשר (חולין קיג, א) אמרינן, השובר מפרקתה של בהמה קודם שתצא נפשה גוזל את הרבים ומבליע דם באברים, ואיבעיא להו התם אי לימא גוזל את הרבים ומבליע דם נמי באברים ואי בעי למיכל מיניה באומצא לא מצי אכיל. וכל אומצא בלא מלח משמע בכל מקום.\par \textbf{} וגירסת הספרים נכונה דגרסי בשר תפוח. וכן היא בהלכות הריא״ף ז״ל. ואע״פ דקתני בברייתא בשר תפוח מטלטלין מפני שהוא מאכל לחיה ולא הקשו ממנה לרב חסדא, דלמא לא הוו ידעי לה, ואי נמי מוקי לה כר׳ שמעון ורב חסדא כר׳ יהודה סבירא ליה. אלא דקשיא אפילו לר׳ יהודה ליטלטל דהרי הוא מאכל לחיה כדאיתא בברייתא. וההיא ברייתא נמי קשיא לי אמאי נקט מפני שהוא מאכל לחיה, הוה ליה למימר לכלבים כלישנא דמתניתין (קנו, א) מחתכין את הנבלה לפני הכלבים, ועוד שהן מצוין יותר מן החיות. ועוד קשיא לי למה אמרו בין בברייתא בין בפלוגתא דרב הונא בשר תפוח, לנקוט בשר נבלה. ועל כן נראה לי דתפוח מחמת ארס קאמר, והיינו דרב חסדא (דאסור) [דאסר], דסבירא ליה דכיון דלא חזי אפילו לכלבים דאי אכלי מיניה מייתא, כל עצמן אינו משהה אותן לחיה כדי שלא יאכלו ממנו כלביו וימותו, ורב הונא אמר מותר דהא חזי לחיה וכדקא תני בברייתא, וברייתא לא שמיע ליה לרב חסדא והיינו דקתנו לה בברייתא בהדי מים מגולין. כך נראה לי.}
\textblock{\textbf{וראיתי בפירושי ההלכות לר׳ אבא מרי בר יוסף ז״ל דבנתפח בשבת פליגי, ורב חסדא אסר משום נולד כרבי יהודה, ורב הונא שרי כרבי שמעון דלית ליה נולד, אבל       } בנתפח מערב שבת לכולי עלמא שרי, והיינו ברייתא. ונכון הוא. אבל הרב אלפסי ז״ל אינו סובר כן, שהוא הביא סיוע לפסוק הלכה כר׳ הונא, מדמסייע ליה ברייתא דלקמן דבשר תפוח.}
\textblock{\textbf{במוקצה לאכילה סבר לה כרבי יהודה.} פירש רש״י ז״ל: באיסור אכילת מוקצה סבר לה כר׳ יהודה שאסור לאוכלו אבל שרי לטלטלו דדי לו שאסרתו עליו באכילה אלא מותר לטלטלו כר׳ שמעון. ויש מקשים והא אמרינן בשלהי פרק קמא (יט, ב) בכרכי דזוגי דרב אסור. ויש לומר דדבר שעומד לאכילה הוא דשרי לטלטלי משום דדי לו שאסרתו באכילה, אבל במה שאינו עומד לאכילה סבירא ליה כר״י.\par \textbf{} ועדיין קשה דהא אמר רב בפרק כירה (שבת מה, א) מניחין נר על גבי דקל בשבת ואין מניחין נר על גבי דקל ביום טוב ואוקימנא התם כר׳ יהודה, ואע״ג דאיכא שמן בנר דראוי לאכילה, ואפילו הכי אסור לטלטל. יש לומר שאף מותר השמן שבנר אינו ראוי לאכילה. וכל זה כפי גירסתו של רש״י ז״ל דגריס בשר תפל דאף הוא עומד לאכילה, אבל לגירסת הספרים והאלפסי ז״ל דגרסי בשר תפוח, אי אפשר לומר כן. [ויש לומר] ההיא דכרכי דזוגי מוקצה מחמת מיאוס הוא, וכדמפרש ר״ת ז״ל מחצלאות שפורשין על גבי דגים מלוחים וריחן רע ונמאסין, וההיא דנר מוקצה מחמת איסור, ובמוקצה מחמת מיאוס ומחמת איסור סבר לה נמי כותיה דרבי יהודה, וכדמוכח נמי בשלהי פרק בתרא דמכילתין (קנז, א).}
\textblock{\textbf{והא קא מבטל כלי מהיכנו.} ולא דמי לכופין את הסל לפני האפרוחין דשרי במתניתין, ואוקימנא בפרק כירה (שבת מג, א) אפילו כמ״ד אסור לבטל כלי מהיכנו, והיינו טעמא דהתם לא קיימי אפרוחין אלא לפי שעה ויכול הוא להפריחן, אבל הכא דלמא לא תעלה, ודלמא מבטל כלי מהיכנו קאמר. כך נראה לי.}
\textblock{ מתני׳:\textbf{ אין מילדין את הבהמה.} פירש רש״י ז״ל: משום טירחא. ולי נראה משום חלול דאורייתא. ותדע לך מדאקשינן בגמרא גבי מילדין את האשה ומחללין עליה את השבת, מכדי תנא ליה מילדין את האשה וקורין לה חכמה, מחללין עליה את השבת לאתויי מאי, ואם איתא דמילדין אין בו אלא טירחא בעלמא, מאי קא מקשה מחללין לאתויי מאי, לאתויי אפילו חלול דאורייתא, ואי משום דתנא וקורין לה חכמה, אף היא דרבנן דתחומין לאו דאורייתא.}
\textblock{ גמרא:\textbf{ אין סחיטה בשיער.} לא ידעתי לזה התירוץ, אם כן מה הפרש יש בין מביאה לה ביד למביאה לה בשיער דכולן משום שבות, ויותר שינוי יש במביאה בשערה ממביאה ביד.}
\clearpage
\newsection{דף קכט}
\textblock{\textbf{אמר ליה הלכתא כמר זוטרא ספק נפשות להקל.} ואפילו אמרה איני צריכה מחללין עליה את השבת, דסתם דברים הנעשין לחיה בשהקבר פתוח, אם אינן נעשין לזו יש בה משום סכנת נפשות, והא דקאמרה אין אני צריכה דלמא תונבא הוא דנקט לה. וכתב הרמב״ן ז״ל דנראין הדברים כשאין שם רופא וחכמה, אבל אם יש שם רופא וחכמה ואמרו אינה צריכה שומעין לה ולהם. ונראה דלעולם אמרה צריכה אני שומעין לה, דלב יודע מרת נפשו, כיון דכולהו נשי סתמא מסוכנות הן אצל אותן דברים. אבל לאחר שנסתם הקבר עד תשלום שבעה אמרה צריכה אני מחללין עליה השבת, ואף על גב דסתם נשי אינן מסוכנות לאותן דברים הצריכין לה תוך שבעה שומעין לה דלב יודע מרת נפשו, ונראה דאפילו יש שם רופא ואומר אינה צריכה, דאפשר וקרוב תוך שבעה שהיא צריכה, ולבשל לה בכלל דברים אלו. אבל לאחר שבעה עד תשלום שלושים אפילו אמרה צריכה אני, כיון שאין לה שום חולי אחר אלא צער לידה אין מחללין עליה, אבל עושין לה על ידי ארמאי. ונראה דדוקא על ידי ארמאי, אבל לא על ידי עבד ואפילו ערל, דחייב אדם על שביתת עבדו ובעבד ערל הכתוב מדבר, ולכולי עלמא לעשות מלאכתנו אסור כישראל, וכיון שעל ידי ישראל אסור אפילו על ידי עבד אסור.}
\textblock{וכתב הרמב״ן ז״ל שלשים אפילו אמרה צריכה אני אין מחללין עליה את השבת אלא עושין לה על ידי ארמאי, פירוש מתירין לה אמירה לעכו״ם במלאכה גמורה ואין מתירין לה אפילו שבות דרבנן על ידי ישראל, וכדאמרינן בביצה (כב, א) גבי אמימר דכחל מעכו״ם, מאי דעתיך משום דצרכי חולה נעשין על ידי עכו״ם הני מילי היכא דלא מסייע בהדיה, מר הא מסייע בהדיה דקא עמיץ ופתח. והא עמיץ      ופתח שבות בעלמא הוא, שאפילו היה מסייע ממש הכוחלת אינו אלא משום שבות כדתנן בפרק המצניע לעיל (שבת צד, ב) ואפילו הכי אסור. וכן נמי הא דתניא (קלד, א) אין טורפין יין ושמן לחולה בשבת, הכי נמי משמע דאיסורין דדבריהם אסור לעשות לחולה. וקשיא לי דהא אמרינן (כתובות ס, א) לגבי גונח יונק חלב, מפרק כלאחר יד הוא ובמקום צערא לא גזרו רבנן. ובפרק אין מעמידין (ע״ז כח, ב) אמרינן סבור מינה הני מילי היכא דשחיקי סמנין מאתמול אבל משחק ואתויי דרך רשות הרבים לא, אלמא מתירין סמנין שחוקין לחולה שאין בו סכנה, ואף על פי שיש בו מלאכה של דבריהם וכחל של עין עצמה, וכן אמרו (לקמן שבת קמח, א) מחזירין את השבר בשבת. ויש לומר כלאחר יד אע״פ שהוא שבות אינו דומה ולא נראה כעושה מלאכה שאין בו משום מראית העין התירו משום חולה, מה שאין כן בכוחל ומסייע לו שהיא מלאכה הנעשית כן בחול ודומה למחלל שבת, וכן נמי טריפת יין ושמן לחולה עובדין דחול נינהו ואסורין אע״פ שאין בהן מלאכה וכן זו ששנינו (לקמן שבת קמז, א) מי שנפרקה ידו או רגלו לא יטרפם בצונן, אבל רוחץ כדרכו ואם נתרפא נתרפא. וכל זה בחולה שאין בו סכנה כלל כגון כוחל עין שאינה מורדת דאמימר בסוף אוכלא ופצוחי עינא, אבל בחולי שיש בו סכנת אבר אחד עושין כדרכו מלאכה של דבריהם, וזהו עין שמרדה דקא סלקא דעתך היכא דשחקי סמנין שכוחלין אותה משום שהיא סכנת עורון, אבל למשחק ואתויי ברשות הרבים לא משום שאינה סכנת מיתה, וכן נראה מפירוש רש״י בפרק אין מעמידין (ע״ז שם), וזהו שאמרו הלכה מחזירין את השבר.\par \textbf{} והוי יודע שלא התירו דברים הללו כלל אלא בדבר שיש ממנו חולי לכל הגוף של אדם, כגון חיה שלשים אי נמי בסכנת אבר, וכן מכחל עינא בסוף אוכלא ופצוחי עינא שהוא נופל למשכב או מצטער החולה ממנו, אבל חושש והוא מתחזק כבריא אין מתירין לו אפילו שבות של דבריהם ואפילו על ידי עכו״ם, ולא עוד אלא דבר שאין בו מלאכה ולא כלום בעולם גזרו משום שחיקת סמנין, וזהו חושש בשיניו לא יגמע את החומץ, החושש במתניו לא יסוך יין, החושש בגרונו לא יערערנו בשמן (לקמן שבת קיא, א), וזהו חמין ושמן חוץ למכה שרי אבל תוך המכה אסור (לקמן שבת קלד, ב), וזהו חזרת רטיה במדינה שאסור (עירובין קב, ב), וזהו רפואות שאסורים משום שחיקת סמנין אפילו בדבר שהבריאין מותרים בו אם רצו כמו ששנינו במשנת פרק שמונה שרצים (לעיל שבת קט, ב), ואין צריך לפרש בזה יותר וכל המסכתא בשיטה זו הולכת ע״כ.}
\textblock{ בפלוגתא דאביי ורב הונא בריה דרב יהושע ב\textbf{מאימתי פתיחת הקבר.} דאביי אמר משעה שתשב על המשבר, ורב הונא בריה דרב יהושע אמר משעה שהדם שותת ויורד. כתב הרב אלפסי ז״ל מדקאמר משעה שהדם שותת ויורד ולא קאמר עד שיתחיל דם להיות שותת, שמעינן מינה דזמן של רב הונא קודם לישיבת המשבר, ופסק הלכה כאביי ואין מחללין עליה עד שתשב על המשבר ויהא הדם שותת ויורד. והוא מן התימה למה לא הלך גם בזו להקל דהא ספק נפשות להקל. והרמב״ם ז״ל (פרק ב׳, הלכה י״ג) פסק משעה שהדם שותת ויורד. גם מה שדקדק משעה שהדם שותת ויורד, יש לעיין דהא אביי ורב הונא אמאימתי פתיחת הקבר קיימי.}
\textblock{ הא ד\textbf{אמרי נהרדעי שבעה אמרה צריכה אני מחללין עליה, אמרה אין אני צריכה אין מחללין עליה.} קשיא דרישא אסיפא. ומיהו לפסק הלכה נקטינן כסיפא דדוקא אמרה אין צריכה אני, הא סתמא מחללין דספק נפשות להקל. וכן פסק הרב ז״ל בהלכות.}
\textblock{\textbf{אמר רב יהודה אמר שמואל לחיה שלשים למאי הלכתא אמרי נהרדעי לטבילה.} וא״ת מאי טעמא לא אוקמוה לחלל עליה שבת על ידי ארמאי, כדפריש לה אינהו לעיל. יש לומר דהיינו טעמא, דאם איתא דלענין חלול שבת קאמר לה הוה ליה למנקט שלשה ושבעה ושלשים כדאמרינן לעיל, ומדלא אמר הכי שמע מינה דלאו בההוא דינא קאי ואוקמוה לטבילה. כך נראה לי.}
\textblock{\textbf{מאה קרי בזוזא.} כך גריס רש״י ז״ל. ויש מקשים מאי שייך הכא, ומה צורך בדבר זה. ושמא מפני שדבר בענין הקזה אייתי הא, לומר שאין בו תועלת באכילתו. אבל הגאונים ז״ל וכן בערוך (ערך פגר) גרסי מאה קרני בזוזא. ופירש ר״ח: קרני כמו קרנא דאומנא לקמן (שבת קנד, ב), ופירוש מנהג הוא שהמקיז מאה ריבדי דכוסילתא נוטל זוז, וכן מגלח מאה ראשים נוטל זוז, אבל מגלח את השפה ולא כלום לא היה נוטל שכר, אלא המקיז או המגלח הראש מגלחה בלא שכר בהבלעה זו. וכן פירש הרב בעל הערוך זכרונו לברכה.}
\newchap{פרק \hebrewnumeral{19} רבי אליעזר דמילה}
\clearpage
\newsection{דף קל}
\textblock{}
\textblock{ מתני׳:\textbf{ רבי אליעזר אומר אם לא הביאו מערב שבת מביאו בשבת.} איידי דסליק מפרק מפנין דתני כל צרכי מילה עושין בשבת, התחיל כאן רבי אליעזר אומר אם לא הביא כלי, ולא הוצרך לפרש לאיזה דבר מביאו.}
\textblock{\textbf{מביאו בשבת.} הקשו בתוס׳ למה יביא איזמל אצל תינוק יביאו תינוק אצל איזמל דליכא אלא שבות, דאדם חי נושא את עצמו לכולי עלמא. ותירצו דהכא במאי עסקינן בשאין שם מניקה אלא אמו, ונמצא צריך לאמו לאחר מילה, ואמו מסוכנת שאינה יכולה לבא אצלו והם צריכין להחזירו אצל אמו, ואחר מילה הרי הוא חולה וככפות דמי ומודה רבי נתן בכפות (לעיל שבת צד, א). ואינו מחוור בעיני, דמכל מקום עכשיו אינו כפות למה נחלל עליו עכשיו יביא תינוק אצל איזמל דכיון דהשתא מיהא ליכא אלא שבות, ודלמא לאחר מילה תהא שם מניקה, ואפילו דלתות מדינה נעולות מכל מקום כל שאפשר עכשיו בלא חלול נעשה עכשיו בלא חלול, ולאחר מילה כיון דלא אפשר נביאהו אצל אמו דהא הותרה אצלו מלאכה זו. אלא יש לומר דאפילו קודם מילה הוי ככפות, דתינוק קטן כל כך אינו נושא את עצמו.\par \textbf{} ויש מי שתירץ גם בתוס׳ דלרבי אליעזר כיון שהתורה התירה מכשירין לא טרחינן ולא משנינן. ותדע לך מדאמרינן בגמרא מעשה היה ושכחו ולא הביא איזמל מערב שבת ולמחר הביאוהו שלא ברצון רבי אליעזר דשרי אפילו ברשות הרבים, אלא כר״ש דתנן ר״ש אומר אחד גגות ואחד חצרות וכו׳, אלמא לרבי אליעזר אע״פ שיכול להביאו דרך גגות וחצרות, אינו מביאו אלא דרך רשות הרבים דלא משנינן כלל. ועוד תדע לך דהא שרי רבי אליעזר להביאו בידו בהדיא ולא משנינן ביה להביאו בשערו או כלאחר יד כדרך שאמרו בחיה בפרק מפנין (לעיל שבת קכח, ב), אלמא לא משנינן כלל.\par \textbf{} ואם תאמר אמאי לא, והא (לקמן שבת קלג, א) א״ר שמעון בן לקיש כל מקום שאתה מוצא עשה ולא תעשה אם אתה יכול לקיים את שניהם מוטב ואם לאו יבא עשה וידחה את לא תעשה. יש לומר דשאני הכא דאי אפשר שלא לדחות מיהא לגבי מילה עצמה, והלכך אף היא ומכשיריה כמילה אריכתא דמו. ויש מי שאומר שאילו היה יכול להביאו דרך גגות במהרה כל כך כדרך שהוא מביאו ברשות הרבים אפילו רבי אליעזר מודה דמביא דרך גגות, אבל כשדרך הרבים קצרה ובלא טורח ובלא עיכוב ודרך גגות בעכוב קצת לא משנינן כלל, ובשערו נמי איכא עיכובא טפי והלכך לא טרחינן ולא משנינן. ומכל מקום לדברי כולם דוקא להביאו אפילו דרך רשות הרבים, אבל לעשות כלי לכתחילה היכא דאיכא כלי ואפשר לאתויי דרך גגות וקרפיפות ואפילו להביאו דרך רשות הרבים לא שרי ליה למעבד כלי, דהבאת כלי הוא לצורך מילה אבל עשיית כלי אינו לצורך מילה בלבד אלא אף לעשיית כלי.}
\textblock{ גמרא:\textbf{ כגון תפילין עדיין מרופה בידם דאמר ר׳ ינאי.} מסתברא דהא דר׳ ינאי כדי נקטה, דעיקרא דמלתא היינו הא דאמרינן פעם אחת גזרה מלכות הרשעה, דמינה שמעינן דלא מסר אדם עצמו עליה אלא אלישע בעל כנפים, ומשום כך עדיין רפויה בידם דמועטין שבישראל מניחין תפילין. והא דאייתי הא דר׳ ינאי משום דאייתי׳ לה בכוליה תלמודא בהדי הא דאלישע אייתי נמי הכא וכאתחולי מלתא היא. אבל רש״י ז״ל לא פירש כן (בד״ה דא״ר ינאי), ואין פירושו מחוור בעיני כאן.}
\textblock{\textbf{דתניא וחכמים אומרים כשם שאין מביאין אותו דרך רשות הרבים, כך אין מביאין אותו דרך גגות וחצרות ודרך קרפיפות.} כלומר: ודחינן מילה עד למחר, ודוקא בשאין שם אלא ישראל דאפילו שבות דישראל לא דחינן, אבל איכא עכו״ם אומר לו ומביאו דרך גגות, משום דאמירה לעכו״ם להביאו דרך גגות ליכא אלא שבות      ושבות דשבות לא גזרו במקום מילה, וכההיא דאמרינן התם בעירובין בפרק הדר (עירובין סז, ב) ההוא ינוקא דאשתפוך חמימיה אמר להו רבה זילו אמרו ליה לעכו״ם דליתי ליה מגו ביתאי. וזו כדעת הרב אלפסי ז״ל שכתב כאן בהלכות דלאו דרך ר״ה הביאו, דאמירה לעכו״ם בדבר שיש בו מעשה כלומר מלאכה דאורייתא כגון עשיית כלי או הוצאה לא התירו אפילו במילה, וכדאמרי התם בההיא עובדא דינוקא דאשתפוך חמימיה, דאותביה אביי לרב יוסף הזאה, ואהדר ליה רב יוסף ולא שני לך בין שבות שיש בו מעשה לשבות שאין בו מעשה, דהא מר נמי לא אמר ליה זיל אחים לי, אלמא אסור למימר זיל אחים לי אע״ג דישראל גופיה לא עבד מידי אלא אמירה בעלמא, כיון דאמר ליה למעבד מלאכה דאורייתא, והלכך אפילו למימר ליה לעכו״ם להביאו (אפילו) דרך רשות הרבים אסור.\par \textbf{} אבל הרב בעל הלכות ז״ל פירש: שבות שיש בו מעשה, כגון הזאה שהישראל עצמו הוא עושה מעשה, ואע״פ שאין בו אלא שבות, העמידו דבריהם במקום כרת, אבל אמירה לעכו״ם שאין בה לישראל שום מעשה שרי, ואע״ג דאיכא מלאכה גמורה אצל העכו״ם. ולא גריס התם דהא מר לא אמר ליה זיל אחים לי, דאפילו למימר ליה להחם שרי, ולפיכך התיר הרב בעל ההלכות ז״ל היכא דאיתבד איזמל או איפגום מקמי מילה, למימר ליה לעכו״ם לצבותי׳ או לאתויי, מההיא דאשתפוך חמימיה.\par \textbf{} ויש מקשים על דברי הגאון ז״ל, דהא משמע דאמירה לעכו״ם במלאכה דאורייתא, חמירא טפי ממעשה דישראל בדבר שאין בו אלא שבות, תדע לך דהא אמרינן בפרק המצניע (לעיל שבת צד, ב) ההוא שכבא דהוה בדרוקרת שרא להו רב נחמן לאפוקי לכרמלית, ואילו בר״ה (כ, א) אמרינן דבשבת ויום הכיפורים לא קברינן מיתא אפילו על ידי עממין. ויש מי שתירץ לדעת הגאונים דהתם משום כבודו של מת הוא, לפי שבזיון הוא אצל המת דליחללו עליה שבת או יום הכיפורים במלאכה גמורה, ותדע לך דהא גבי מת תנן (לקמן שבת קנא, א) הרי זה לא יקבר בו עולמית, ואילו בשאר מלאכות שרו לערב בכדי שיעשו. אבל בתוס׳ הכריעו כדברי הרי״ף מאידך עובדא דאתמר התם (עירובין סח, א) בינוקא דאשתפוך חמימיה דאמר להו לשיוליה לאימיה אי צריכא ליחם לה עכו״ם אגב אימיה, דאלמא דוקא אגב אימיה הא לאו הכי לא, ואף על גב דאמירה בעלמא היא דקא עביד ישראל, וזו ראיה גדולה לדברי הרב אלפסי ז״ל אילו היה מודה בה״ג בגירסא זו, אלא שהוא ז״ל לא יגרס וליחים ליה עכו״ם, אלא ליחים ליה סתם, כלומר ואפילו ישראל לפי שעל האם מחללין את השבת כל שאמרה צריכה אני, וכיון שכן אף להוסיף בשביל התינוק באותו מיחם מותר.\par \textbf{} ומיהו אכתי איכא למידק, דהא אמרי נהרדעי (לעיל שבת קכט, א) לחיה שלשה ושבעה ושלשים, שלשה אפילו אמרה אין צריכה אני מחללין עליה את השבת, שבעה אמרה צריכה אני מחללין לא אמרה צריכה אני אין מחללין, אבל עושין על ידי עכו״ם, ויום מילה דאישתפיך חמימיה לאחר שבעה הוא, ואפילו אמרה צריכה אני אין עושין על ידי ישראל, ואם כן על כרחין ליחים ליה עכו״ם, גרסינן. ויש לומר דהני שלשה ושבעה ושלשים מעת לעת נינהו, והלכך משכחת לה למילה בשבעה שלה וכגון שנולד סמוך לערב.\par \textbf{} ומכל מקום אכתי קשה דאי על ידי ישראל קאמר אפילו להוסיף באותו מיחם עצמו בשבת אסור, וכדאמרינן בפרק קמא דחולין (טו, ב) המבשל לחולה בשבת אסור לבריא גזירה שמא ירבה בשבילו, אלמא אפילו להוסיף אסור. ועוד דאמרינן במנחות פרק רבי ישמעאל (מנחות סד, א) חולה שאמדוהו לשתי גרוגרות, ויש שתי גרוגרות בשני עוקצים ושלשה גרוגרות בעוקץ אחד, הי מנייהו מייתינן, תרי מייתינן דממעט באוכלא, או דילמא תלתא מייתינן דממעט בבצירה, אלמא אפילו בבצירה אחת לרבות באוכלא אסור והכי נמי מאי שנא. ויש לומר דהכא איכא צורך מצוה ולא מחמרינן כולי האי, דהא מלאכה כי האי קילא, ותדע דהא ביום טוב שרי לכתחילה ואפילו לצורך חול, כדאמרינן בריש פרק יום טוב שחל להיות ערב שבת (ביצה יז, א) ממלאה אשה קדרה בשר אע״פ שאין צריכה אלא לחתיכה אחת, ממלא נחתום חבית מים אע״פ שאין צריך אלא לקיתון אחד.}
\textblock{\textbf{ויש מקשים על דברי רבנו האלפסי ז״ל מהא דאמרינן (גיטין ח, ב) הלוקח עיר בא״י כותבין עליו אונו ואפילו בשבת, ואע״ג דהכא איכא אמירה לעכו״ם בדבר שיש בו מלאכה גמורה. ויש מתרצים דמשום ישוב ארץ ישראל לא גזרו רבנן, ואין אומרים בדברים כאלו זו דומה לזו. ותדע לך שהרי יש מקומות שהתירו אפילו שבות הנעשה ע״י ישראל עצמו במקום פסידא, כגון צנור שעלו בו קשקשים דהתירו לו למעכן ברגלו כדאיתא בכתובות (ס, א), ולגבי דליקה (לעיל שבת קכב, א) לא התירו לומר לעכו״ם כבה, ואע״ג דליכא אלא שבות דשבות, דאפילו כבוי עצמו אינה מלאכה דאורייתא בכי הא דמלאכה שאינה צריכה לגופה היא, ולגבי מת נמי התירו שבות כההיא דשכבא דהוה בדרוקרא, ואילו לגבי מילה לא התירו אפילו שבות על ידי ישראל, ואף על פי שהמילה חמורה שנכרתו עליה י״ג בריתות. ומיהו לישנא דלא שני לך בין שבות שיש בו מעשה לשבות שאין בו מעשה דאמרינן התם בערובין פרק הדר (עירובין סח, א) מכרעא       } לי טפי כדברי הרב בעל ההלכות ז״ל, ושם כתבתי בסייעתא דשמיא.\par \textbf{} ומורי הרב ז״ל כתב בהלכותיו, דשבות שיש בו מעשה ושבות שאין בו מעשה דאמרינן התם, כולם באמירה לעכו״ם הן וכדעת הרב אלפסי ז״ל, אלא שאין פירוש מעשה מלאכה כמו שפירש הרב ז״ל, שאפילו במביא דרך גגות מלאכה הוא עושה, ואי משום דאינה מלאכה דאורייתא, היה לו לומר ולא שני לך בין שבות שיש בו מעשה דאורייתא לשבות שאין בו מעשה דאורייתא, אלא שיש בו מעשה פירוש שנתחדש בו מעשה בגופו של דבר, כגון עשיית כלי או אפיה ובשול ולהחם מים וכיוצא באלו שנתחדש ענין בגופן, ודמי האי מעשה ללישנא דאמרינן (ב״ק עא, א) לרבי יוחנן הסנדלר מעשה שבת דאורייתא, והתם לא קרי מעשה שבת אלא כגון אפיה ובישול וכיוצא בהן, אבל הבאת כלים ואוכלין מרשות לרשות ודאי לא מיתסרי אפילו לרבי יוחנן הסנדלר בהבאתן כיון שלא נתחדשה בהן הכנה בשבת ולפיכך החמירו באמירה לעכו״ם אפילו במקום מצוה היכא דקא מתקן ומחדש מידי בגוף הדבר משום דהוי טפי עובדין דחול, אבל בשבות שאין בו מעשה בגופו של דבר כגון הוצאה בלבד מרשות לרשות, אינו נראה כעובדין דחול ולא החמירו באמירתו. אלו דברי מורי הרב זכרונו לברכה.\par \textbf{} ולדידן דקיימא לן כר״ש באחד גגות ואחד חצרות ואחד מבואות ואחד קרפיפות רשות אחת לכלים ששבתו בתוכן, מביא את האיזמל אפילו על ידי ישראל, אבל שבת בתוך הבית אסור, והיינו דאמרינן (פסחים צב, ב) הזאה ערל ואיזמל העמידו דבריהם במקום כרת. וכתב הרמב״ן ז״ל שאין מתירין אפילו שבות שאין בו מעשה ועל ידי עכו״ם בשאר מצות אלא במילה בלבד לפי שנתנה שבת לידחות אצלה, אבל בשאר מצות אין דוחין שבות כלל, לפי שאין לנו היתר בשבות דשבות יותר משבות דמלאכה אלא בזו, וכן לענין חולה דאמרינן כל צרכי חולין נעשין על ידי ארמאי בשבת במלאכה גמורה, וכל שכן בשבות אמרו כדמוכח לעיל בפרק מפנין (שבת קכט, א) וכההיא דפרק הדר (עירובין סח, א) דמחממין לאימיה על ידי עכו״ם בשבת, ואמרינן נמי בפרק בתרא דביצה (כב, א) אמימר כחל עינא מעכו״ם בשבת, ואמר ליה רב אשי מאי דעתך משום דצרכי חולה נעשין על ידי עכו״ם, מר הא קא מסייע, אלמא אפילו בכוחל דהוא גופיה אין בו אלא שבות אינו נעשה ע״י עכו״ם אלא במקום חולי שאפילו המלאכה מותרת. והרמב״ם (הל׳ שבת פ״ו, ה״ט) התיר שבות דשבות לגבי שופר ולולב.}
\textblock{\textbf{ועוד דרבי אליעזר שמותי הוא.} פירש רש״י ז״ל: שברכוהו. ואינו מחוור, אטו משום שברכוהו לא תהא הלכה כמותו בכל מקום, אלא פירוש שמותי מדבית שמאי, וכן מפורש בירושלמי (תרומות פ״ה ה״ב).}
\clearpage
\newsection{דף קלא}
\textblock{\textbf{אלא אמר רב אשי מי גרם לחצרות שיאסרו בתים, ואינן.} ולית הלכתא כרב, אלא בין עירבו בין לא עירבו וכשמואל ור׳ יוחנן דאמרו הלכה כרבי שמעון דאמר(ו) בין עירבו בין לא עירבו, דלא גזרינן משום כלים ששבתו בתוך הבית. והלכך כיון דמשתרי לטלטולי מחצר למבוי וממבוי לחצר לכלים ששבתו לתוכן שרי נמי לטלטולי בכוליה מבוי, דהא בהא תליא. וכן כתב הרב מורי ז״ל. וכן כתבו בתוס׳.}
\textblock{\textbf{מאי טעמא אמר רב יוסף לפי שאין קבוע להן זמן.} איכא למידק מאי קאמר מאי טעמא, והא אמרינן דלא לכל אמר רבי אליעזר אלא היכא דגלי קרא בהדיא והני לית להו מקרא. ויש לומר דטעמא דקרא קא בעי. ויש מפרשים דלר״א נמי בעי מאי טעמא לא מייתי להו מסוכה, דמאי איכא למיפרך מה לסוכה שכן נוהגות בלילות כבימים, מזוזה הא נוהגת בלילות כבימים וציצית נמי נוהג הוא לרבנן אפילו בלילות דמוקמינן (לעיל שבת כז, ב) וראיתם אותו פרט לכסות סומא. וכן תירצו בתוס׳.}
\textblock{\textbf{הואיל ובידו להפקירן.} והא דלא אמר הואיל ובידו שלא ללובשו. פרש״י ז״ל דאפילו מונח בקופסא עבר עליה בעשה. ואינו מחוור, דאי משום הא אתיא דלא כהלכתא דכלי קופסא אינן חייבין בציצית דלאו חובת מנא הוא. ושמא משום דאיכא מאן דסבר ליה הכין במנחות (מד, א) לא נקט ליה בהכין ונקט טעמא דאתי לכולהו. וטעמא דטלית של הפקר פטור מן הציצית, דלא עדיף מטלית שאולה דאמרינן (חולין קי, ב) שהיא פטורה כל שלשים יום, וכן בית של הפקר פטור מן המזוזה כדאמרינן (מנחות מד, א) דשוכר פטור כל שלשים, ואע״ג דלאחר שלשים חייבין בין בציצית בין במזוזה, היינו מדרבנן דאילו מדאורייתא אין הפרש בין תוך שלשים לאחר שלשים.}
\textblock{\textbf{ואם תאמר כיון דפטרי׳ הכא בית שאינו שלו ממזוזה, הא דאמרינן בריש פרק ראשית הגז (חולין קלה, ב) מזוזה אע״ג דכתיב ביתך (דברים יא, כ) דידך אין ולא דשותפות כתב רחמנא למען ירבו ימיכם וימי בניכם (שם, שם, כא), ואקשינן ואלא ביתך למאי אתא, ופרקינן לכדרבה דאמר רבה דרך ביאתך מן הימין, לישני ליה ביתך למעוטי דר בבית      } שאינו שלו, וכדשני בציצית כסותך למה לי לכדרב יהודה דאמר טלית שאולה פטורה מן הציצית כל שלושים יום. יש לומר דאין הכי נמי אלא חד מתרי טעמי נקט.}
\textblock{ הא דאמרינן:\textbf{ דרבנן איצטריך סלקא דעתך אמינא נילף שבעת ימים מסוכות וכו׳.} קשיא לי אם כן לא לכתוב רחמנא ימים דהא כולה מלתא מביום הראשון נפקא לן. ויש לומר דאי לא כתב רחמנא ימים הוי אמינא ודאי דנילף שבעה שבעה מסוכה, ואי משום דכתיב ביום (הזה) [הראשון] הוי אמינא הני מילי בגבולין דלא חייב בהו רחמנא אלא יום אחד והכי נמי לא חייב בו אלא יום ולא לילה, אבל במקדש כי היכא דחייב ביה שבעה חייב ביה נמי לילות כימים, אבל השתא דכתיב גם במקדש ימים כמו שכתוב בגבולין, גלי לן קרא דכי היכי דיום שנאמר בגבולין דוקא ביום ולא בלילה, אף ימים שנאמר במקדש דוקא ימים ולא לילות.\par \textbf{} ואכתי איכא למידק דאם כן שפיר קאמרי ליה רבנן לר״א. ויש לומר דר״א סבר דכיון דאיכא למידרש ביום למילי אחריני דרשינן, וימים דכתיב גבי מקדש גלי נמי אגבולין, דכי היכי דבמקדש אינו ניטל אלא ביום אע״פ שהחמיר בו הכתוב להטעינו שבעה, כ״ש בגבולין שאינו ניטל אלא יום אחד שאינו ניטל בלילה אלא ביום, וגזירה שוה לא משבעת ימים היא אלא שבעה שבעה, ולא למילף לולב מסוכה אלא למילף סוכה מלולב, מה לולב מכשיריה דוחין את השבת אף סוכה מכשיריה דוחין את השבת, כדיליף מינה בסמוך ר״א, ולא הקישא למחצה הוא, דהא גלי קרא בלולב דאינו נוהג בלילות מדכתיב ביה ימים.}
\textblock{ הא דאמרינן:\textbf{ גמר שבעת ימים שבעת ימים מלולב.} איכא למידק דהא שבעת ימים דגבי לולב איצטריך, ואכתי מופנה משני צדדים ליתא אלא מופנה מצד אחד הוא, ואיכא למיפרך מה ללולב שכן טעון ארבעת מינין הוא. וכן נמי הא דאמרינן בסמוך יליף חמשה עשר חמשה עשר מחג הסוכות, לאו מופנה הוא. ויש לומר דגלוי מלתא בעלמא הוא, כיון דגלי קרא בכל הנך מצות דדחו שבת, אבל כי אתיא למילף במה מצינו פרכינן, משום דבמה מצינו כל דהו פרכינן (חולין קיד, א).}
\textblock{\textbf{נפקא לן מביום הכפורים וגמרי מהדדי.} ורבנן דלא ילפי, סבירא להו דלא גמרי אלא לסדר תקיעות, אבל לימים ולא ללילות לא גמרינן מהדדי, משום דכי היכי דהאי בראשון והאי בעשור, איכא למימר נמי דהאי ביום והאי בלילה.}
\textblock{ הא דאמרינן:\textbf{ מה להנך שכן עבר זמנן בטלין.} איכא למידק והא סוכה דאע״ג דדחית לה מקמי שבת, אינה בטלה דעבד לה בחולו של מועד. ויש לומר מכל מקום זמן יש לה דאי עבר זמנה אין לה תשלומין. ועוד יש לומר דלמאן איצטריך לר״א, ור״א סוכה לשבעה בעי, ואם אינו עושה אותה עכשיו אינו יכול לעשותה בחולו של מועד כדאיתא התם בסוכה (כז, ב).}
\clearpage
\newsection{דף קלב}
\textblock{\textbf{ומה מילה שהיא אחד מאבריו של אדם דוחה השבת וכו׳.} איכא למידק מאי קל וחומר, דלמא שאני מילה דנכרתו עליה י״ג בריתות, אי נמי משום דאית בה כרת בגדול, וכדפריך בריש פרק קמא דיבמות (ה, ב). תירצו בתוס׳ דהכא הכי פירושו ומה מילה שאינה אלא לקיום מצוה יחידית באבר אחד מאבריו של אדם דוחה את השבת ק״ו לפקוח נפש שהוא לקיום כל המצות שישמור ויעשה כל המצות.}
\textblock{\textbf{עצם כשעורה הלכה.} פירוש: לגלח בו את הנזיר, וכן נמי הא דיליף בקל וחומר לדם רביעית, היינו נמי לגלח בו את הנזיר, אבל לטומאה קרא כתיב וכדדרשינן או בעצם (במדבר יט, טז) זו עצם כשעורה. ודם רביעית נפקא לן מדכתיב (ויקרא כא, יא) על כל נפשות מת, ודרשינן מינה להביא דם רביעית הבא משני מתים, כדאיתא בחולין בפרק בהמה המקשה (חולין עב, א). והא דאמרינן בריש עירובין (ד, ב) ובכיצד מברכין (ברכות מא, א) שיעורין דאורייתא נינהו, דכתיב ארץ חטה ושעורה שעורה זה עצם כשעורה, ותסברא שיעורין מי כתיבי אלא הלכתא נינהו וקרא אסמכתא בעלמא הוא. לאו אעצם כשעורה קאי אלא (אשעורין) [אשארא].}
\textblock{ הא דאקשינן:\textbf{ אלא מעתה תפילין דכתיב בהו אות לידחו.} לאו אהנחתן קאמר דהנחתן לאו מלאכה היא, ולא עוד אלא אפילו למאן דאמר שבת לאו זמן תפילין הוא, אמרינן בפרק במה אשה יוצאה (שבת סא, א) דהיוצא בתפילין פטור משום דדרך מלבוש הוא, אלא אעשייתן קאמר דהיינו מכשיריו, וכן נמי הא דאקשו ציצית דכתיב בהו דורות לידחו שבת, מכשיריו קאמר. ואע״ג דרבנן אפילו במילה לא שרו מכשירין ולא ילפי מהא אלא גופו של מצוה, התם הוא      לאוקומה גזירה שוה לגופה של מילה, אבל תפילין וציצית לכשתמצא לומר דנפקי מגזירה שוה על כרחך לא איצטריך אלא למכשיריהן.}
\textblock{\textbf{אלא מעתה יהא זר כשר בהן ואונן כשר בהן.} איכא למידק והיאך אפשר לומר כן, והלא כהן כתיב בהו. ויש לומר דזר כדי נסבה ואגב גררא דאונן נסביה. ועוד יש לומר דטעמיה דקרא קא בעי. וכן פירשו משם ר״ח ז״ל. ויש לומר עוד דדלמא כהן דכתיב בהו לאו לעכב אלא [בדאיכא] כהן בכהן ולא זר הא ליכא כהן יהא זר כשר בו, כדי שלא תעכב כפרתן הואיל וחס רחמנא עלייהו.}
\textblock{\textbf{הדר אהדריה קרא.} פירוש: מדכתיב ביה ביום אהדריה קרא לכל תורת קרבנות לגמרי, משום דהוי ליה דבר שיצא לידון בדבר חדש שאי אתה רשאי להחזירו לכללו עד שיחזירנו הכתוב לכללו בפירוש, ולפום כן איצטריך ביום להחזירו לכללו לגמרי. ואם תאמר לרבי יוחנן, ר״א מכשירין מנא ליה. יש לומר דלר״א איכא הלכה וקרא, הלכה לגופה של מילה, וקרא למכשירין.}
\textblock{\textbf{תניא כותיה דר׳ יוחנן ודלא כרב אחא בר יעקב.} והוא הדין שהיא דלא כר״א, וכל שכן דהוי דלא כהני דאמרן הלכה, אלא דאינהו הא סלקי בתיובתא. ורבי שמואל ז״ל פירש בבבא בתרא דלאו כלום משמע, אלא כי אתמר בבי מדרשא בהאי לישניה תניא כותיה ולא כפלוני אמרי הכי, והכא נמי הכי אתמר דרב אחא שמיעא להו למאן דאותבה, ואמרה הכין.}
\textblock{\textbf{מאי טעמא דאתיא בק״ו שבת חמורה דוחה, צרעת לא כל שכן.} קשיא לי להאי לישנא דמייתי ליה בק״ו, דתינח מילה בזמנה דדחיה שבת, שלא בזמנה דלא דחיא שבת מנא ליה. ונראה לומר דכיון דאיכא במילה צד דהוו שבת נדחה, כולה מילה חשבי׳ כחמורה וכולה בכלל מקרייא דוחה, ואורחיה דתלמודא בהכין, וכההוא דריש פרק קמא דיבמות דאמרינן התם (ז, א) אשכחן דאתי עשה ודחי לא תעשה גרידא, שיש עמו כרת מנין, ואתא למפשטא מביניא דמילה ופסח דדוחין את השבת, ודחי׳ מה להנך שכן יש בה צד כרת, אלמא אע״ג דמילת קטן אין בה כרת ודחיא שבת, מכל מקום כיון שעיקר מילה היא חמורה שיש בה כרת אע״ג דמילת קטן אין בה כרת אין הולכין בה אחר הפרט אלא אחר הכלל דמילה גופה מכל מקום חמורה היא שיש בה צד כרת, והכי נמי דכותיה היא דעיקר מילה חמורה שיש בה צד דחוי שבת, והיינו נמי דלעיל אתיא לפרושא בדין ק״ו, ואע״ג דקתני בברייתא מילה בין בזמנה בין שלא בזמנה דוחה את הצרעת. אי נמי איכא למימר דלא קפדינן בה משום דתנא דברייתא סמיך אמסקנא דילפותא דמסיק ליה תנא משום דכתיב בה בשר, ובין בזמנה בין שלא בזמנה כתיב בשר. מכל מקום פשיטותא דמלתא כתירוצא קמא קאי שפיר טפי. כך נראה לי.}
\textblock{ הא דאמרינן:\textbf{ אתי עשה ודחי לא תעשה הני מילי לא תעשה גרידא, האי עשה ולא תעשה הוא.} קשיא לי ומעיקרא מאי ניחא ליה דודאי לא אתי עשה ודחי עשה ולא תעשה, דמאי אולמיה דהאי עשה מהאי עשה. ויש לומר דדלמא הוה סבירא ליה דכיון דעשה דמילה חמיר דאית בה כרת ועוד דנכרתו עליה שלשה עשר בריתות, דחי לעשה דצרעת, וכדמשמע בריש פרק קמא דיבמות (ו, א) דאי לאו דכתב רחמנא (ויקרא יט, ג) שבתותי תשמורו אני ה׳, ומינה דרשינן אתה ואביך חייבין בכבודי, הייתי אומר דאתי עשה דכבוד אב ואם ודחי לאו ועשה דמחמר ואף על גב דמחמר איכא לאו ועשה, והיינו טעמא משום דעשה דכבוד אב ואם חמיר טפי משום דהוקשה כבודם לכבוד המקום, וכן נמי במציעא בפרק אלו מציאות (בבא מציעא לב, א) גבי אמר לו אל תחזיר, וכן נמי אמרו בשלוח הקן (חולין קמא, א) דעשה דטהרת מצורע חמיר משום שלום ביתו, והכי נמי משום חומר עשה דמילה היה סבור מעיקרא דאתי ודחי לעשה ולא תעשה דצרעת ולבסוף הדר ביה. כך נראה לי.}
\textblock{\textbf{רבא אמר מילה בזמנה לא צריכה קרא, שבת חמורה דוחה צרעת לא כל שכן.} פירוש: והלכך בשר דכתיב גבי קטן לא צריך לגופו, ואם אינו צריך לגופו תנהו לענין בינוני ולרבא לא אתי ליה תנא בק״ו כלישנא קמא, דאם כן מאי או אינו דקתני, אלא כלישנא בתרא דאתו ליה מדין עשה ולא תעשה ולמילה שלא בזמנה.}
\textblock{\textbf{אמר ליה רב ספרא לרבא וממאי דשבת חמורה דלמא צרעת חמורה שכן דוחה את העבודה, אמר ליה התם לאו משום חומרא דצרעת אלא משום דאכתי גברא לא חזי.} מסתברא דכל מאן דאית ליה צרעת חמורה, על כרחך סבירא ליה דמכשירי מילה דוחין את השבת, דמילה גופה לא צריכה קרא דקל וחומר הוא, ומה צרעת דוחה שבת לא כל שכן (מילה), וכי איצטריך קרא דביום השמיני (ויקרא יב, ג) דמיניה דרשינן ביום ולא בלילה ביום ואפילו בשבת, למכשירין איצטריך. וסבירא ליה נמי      שלא בזמנו דלא דחיה שבת, היינו כטעמיה דרבא דלקמן דמייתי לה מדכתיב (שמות יב, יז) הוא לבדו ולא מכשיריו לבדו ולא מילה שלא בזמנה דאתיא בקל וחומר. וכל הני אמוראי דלעיל דמייתו לה למילה גופא דדחייא שבת, סבירא להו כרבא דאמר דשבת חמורה, וצרעת דדחייא עבודה משום דגברא לא חזי הוא, וכי איצטריך למילה גופה איצטריך. ותמיהא לי דאם כן רב ספרא כר״א סבירא ליה. ויש לומר דאיהו לאו משום דסבירא ליה הכין, אלא לאפוקי טעמא מיניה דרבא קאמר הכין. ואם תאמר אם כן דפלוגתא דר״א ורבנן בהא שייכא, כי קאמר הכא והא דרבא ורב ספרא תנאי היא ומייתי ליה פלוגתא דתנאי דברייתא, לוקמינהו בפלוגתא דר״א ורבנן דמתניתין. יש לומר דר״א נמי אפשר דלא סבירא ליה דצרעת חמורה, אלא איהו מקרא והלכתא מייתי להו, מילה הלכתא ומכשירין קרא וכדכתבינן לעיל. כך נראה לי.}
\textblock{\textbf{תינח נגעים טמאין נגעים טהורין מאי איכא למימר.} פירושו: דקים להו לרבנן דלא יקוץ אפילו נגעים טהורין משום עבודה, אבל אנן במקום דמדחייא עבודה לא אשכחן, דבשלמא בטמאין קיימא לן בפסחים (סז, א) דטומאת מת נדחית בקרבן צבור ולא בזבין ומצורעין ונידחין לפסח שני, אבל טהורין לא ידעינן מנא לן, אלא דהכי קים להו לרבנן.}
\textblock{ הא דאמר רב אשי: \textbf{היכא אמרינן דאתי עשה ודחי לא תעשה, כגון מילה בצרעת וכו׳ דבעידנא דעקר ליה ללאו קא מקיים עשה, הכא בעידנא דעקר ליה ללאו לא מקיים עשה.} ואיכא למידק היכי קא מסיק כן רב אשי טעמא משום דבעידנא דעקר לאו לא מקיים עשה, והא צרעת עשה ולא תעשה הוא ולא אתי עשה ודחי לא תעשה ועשה. ויש מפרשים דלאו עיקר טעמא הוא אלא לומר דאפילו לא הוי אלא לא תעשה גרידא, בכי הא לא אשכחן דמדחי. ואינו מחוור בעיני. ולפי מה שכתבתי אני דמעיקרא סבירא ליה לתנא דברייתא דעשה דמילה חמיר טפי ודחי עשה ולא תעשה דצרעת אתי שפיר, דניחא ליה לרב אשי למימר טעמא דסליק אפילו לכשתמצא לומר כדקא סלקא דעתך דתנא דברייתא למימר מעיקרא.}
\clearpage
\newsection{דף קלג}
\textblock{\textbf{אמר אביי לא נצרכה אלא לר׳ יהודה דאמר דבר שאין מתכוון אסור.} ומהא שמעינן דדבר שאין מתכוון לר׳ יהודה אסור מדאורייתא, מדאיצטריך רחמנא למישרי הכא.}
\textblock{ כך כתוב בספרים:\textbf{ [ו]לעשות, לעשות אי אתה עושה אבל אתה עושה בסיב שעל גבי רגליו.} וקשיא לי טובא לגירסא זו, חדא דאם כן לעשות הוי כמו מעשות כלומר השמר מעשות, ואי איתא לא אתי לעשה אלא ללא תעשה, ועשה בצרעת מנא ליה. ועוד מאי קא פריך בסמוך הא למה לי קרא דבר שאין מתכוון הוא, דהא קרא לאו למשתרי אתי אלא למיסר. ומיהו בהא איכא לתרוצי דעל כרחין לא אתי אלא למישרי, דאי למיסר כבר כתיב (דברים כד, ח) השמר בנגע הצרעת לשמור מאד, למה לי דכתב רחמנא לעשות, אלא לומר לך איני מזהירך אלא לעשות אבל עושה אתה בסיב. והגירסא הנכונה כמו שגורס רש״י ז״ל ולעשות לעשות בסיב שעל גבי רגליו, והשתא פריך שפיר למה לי קרא להתיר דבר שאין מתכוון הוא, והשתא אתי שפיר נמי לעשה, דהכי קאמר אתה רשאי לעשות בסיב שעל גבי רגליו ובמוט שעל כתפו, כלומר בשאין מתכוון לקוץ ולטהר, הא מתכוון לקוץ ולטהר אסור ולאו הבא מכלל עשה עשה.}
\textblock{\textbf{בתר דשמעה מרבא סברא.} ואיכא למידק בין לאביי בין לרבא, בין למאי דס״ל לאביי מעיקרא, למה לן קרא למישרי מילה בצרעת השתא לגבי הרשות שרי במקום מצוה לכ״ש, ולישתוק קרא מיניה. ונראה לי דאיצטריך משום דהכא בשאינו עושה ממש לשום קציצה אלא מלאכתו הוא עושה וממילא מיקץ קייץ, והלכך אי משום הא הוה אמינא דוקא בכי הא הוא דשרא רחמנא הא בקוצץ ממש בידים ומתכוון לקוץ אלא שאין מתכוון לשם קציצת טהרה לא, הלכך איצטריך למיכתב בשר להתיר במקום מילה, שהוא קוצץ בהרת בידים כנ״ל. והרמב״ן ז״ל תירץ דלא בא הכתוב אלא להתיר אפילו מתכוון לקוץ במקום מילה, פי׳ לפירושו לקוץ בהרתו לטהר הוא מתכוון.}
\textblock{\textbf{ואביי למאי דסליק אדעתיה מעיקרא האי בשר לר״ש מאי עביד ליה, אמר רב עמרם באומר לקוץ בהרתו הוא מתכוון.} ואיכא למידק אם כן מאי דוחקיה דאביי דהוה מוקים לה מעיקרא כר׳ יהודה דלא קיימא לן כותיה, לוקמיה אפילו לר״ש דקיימא לן כותיה בדבר שאין מתכוון. ועוד דהכין הוה שפיר טפי, דפריק לה אפילו למאי דסבירא ליה למ״ד דבעי לה דהא קאמר הא למה לי קרא דבר       מתכוון הוא, כלומר ודבר שאין מתכוון מותר. ויש לומר משום דימול בשר ערלתו משמע טפי במתכוון למול ולא לקוץ בהרתו. ומיהו לר״ש למאי דסבר ליה דשרי בעלמא אפילו בפסיק רישיה, על כרחין איצטריך ליה לאוקומיה באומר לקוץ הוא מתכוון.\par \textbf{} תמיהא לי מאן דמתני לה אההיא דר׳ יאשיה, אמאי קרי ליה דבר שאין מתכוון, הא מתכוון לקציצה אלא שאינו מתכוון לטהר אלא למול, ואין זה קרוי דבר שאין מתכוון אלא מלאכה שאינה צריכה לגופה, והא למה הדבר דומה למכבה גחלת של עץ ברשות הרבים (לעיל שבת מב, א) דמתכוון לכבות ולא מחמת שהוא צריך לגופו של כבוי אלא כדי שלא יזוקו בה רבים, וקרי׳ לה מלאכה שאינה צריכה לגופה ולא דבר שאין מתכוון, וכדאיתא בפרק כירה (לעיל שבת מא, ב) גבי המיחם שפינהו דאמרינן שמואל בדבר שאין מתכוון סבירא ליה כר״ש, במלאכה שאינה צריכה לגופה סבירא ליה כרבי יהודה. ויש לומר דאי כתב רחמנא בפירוש השמר בנגע הצרעת שלא תקוץ הכי נמי, אלא השמר בנגע הצרעת כתב רחמנא, כלומר שלא יטהרנו אלא בהוראת כהן, ואנן הוא דאמרינן דכיון שכן אסור לקוץ בהרת שמטהר הוא בקציצתו, והלכך כשזה קוצץ לשם מילה הוי ליה דבר שאין מתכוון אצל טהרה.}
\textblock{\textbf{לבדו ולא למילה שלא בזמנה, דאתיא בקל וחומר.} איכא גירסאות דמסיימי בה הכין: ומה צרעת שאינו נדחה מפני צורך הדיוט נדחה מפני מילה שלא בזמנה, יום טוב שנדחה מפני צורך הדיוט כגון שחיטה ובישול אינו דין שתהא מילה שלא בזמנה דוחה אותו. והיא גירסת גדולי הראשונים וכתובה בספר המאור. ואינה מחוורת בעיני, דשחיטה ובישול אינו צורך הדיוט גרידא כקציצת בהרת, אלא צורך מצוה, דשמחת יום טוב עשה הוא דכתיב (דברים טז, יד) ושמחת בחגיך. ולא גרסינן ליה כלל ואינה ברוב הספרים. גם רש״י ז״ל אינו גורס כן, שהוא ז״ל מפרש האי ק״ו משום דצרעת חמורה שדוחה את העבודה ועבודה דוחה את השבת ומילה שלא בזמנה דוחה אותה, יום טוב ושבת שנדחין מפני עבודה קל וחומר שנדחין מפני מילה אפילו שלא בזמנה. ומכל מקום איכא למידק, דהא רבא גופיה הוא דאמר לעיל איפכא ואמר דשבת חמורה, וצרעת דוחה את העבודה לאו משום דצרעת חמורה אלא משום דגברא דלא חזי. ועל כרחין נצטרך לומר דלעיל לא גרסינן רבא אלא רבה, דהוא רבה בר נחמני רביה דרבא.}
\textblock{\textbf{מאן תנא ר׳ ישמעאל בנו של רבי יוחנן בן ברוקה היא.} איכא למידק והא רבנן לא שרו התם הפשט כדרכו אלא בברזי, דליכא אלא משום שבות כדאמרינן בפרק כל כתבי הקודש, (קיז, א) והכא גבי ציצין שאינן מעכבין את המילה דשקיל להו כדרכו איכא איסורא דאורייתא. ויש לומר דהתם נמי רבנן בכל ענין שרו, ולטעמיה דר׳ ישמעאל קאמרי ליה לדידן בכל ענין שרי משום דכתיב (משלי טז, ד) כל פעל ה׳ למענהו שלא יהא קדשי שמים מוטלין כנבלה, אלא לדידך אודי לן מיהא דבברזי דליכא אלא שבות לישתרי כדשריא להציל תיק הספר עם הספר.}
\textblock{\textbf{אלא אמרי נהרדעי רבנן דפליגי עליה דרבי יוסי.} בלחם הפנים, וסלקא בהכין, וכיון שכן קיימא לן כהאי ברייתא דכל היכא דפירש אינו חוזר על ציצין שאין מעכבין, דרבנן ורבי יוסי הלכה כרבנן דרבים נינהו, ואף על גב דדחיה לה לברייתא מעיקרא מדר׳ ישמעאל ומדרבי יוסי, וכל שכן מדרבנן דפליגי עליה, השתא דאוקימנא כרבנן דלחם הפנים אידחי כל מאי דאמרינן מעיקרא ושמעינן דלא שייכי בהני פלוגתא דלעיל כלל. ונראה לי להביא ראיה, מדאמרינן לעיל דאפילו רבי יוסי דקדוש החדש מודה בה, ואילו הכא אמרי דאתיא דלא כרבי יוסי דלחם הפנים, ואם איתא דהני פלוגתא שייכן אהדדי אם כן תקשי לן דרבי יוסי אדרבי יוסי, שמע מינה דלא שייכן כלל אהדדי ואידחי כל מאי דאמרינן מעיקרא. ולפיכך פסקו הגאונים והרב אלפסי ז״ל כההיא ברייתא, משום דליכא מאן דפליג עלה אלא רבי יוסי דלחם הפנים לבד, ולית הלכתא כותיה אלא כרבנן דרבים נינהו.}
\textblock{\textbf{האי ר׳ אליעזר בן עזריה אומר מרחיצין אף מרחיצין מיבעי ליה.} קשיא לי והא ברייתא נמי דקתני בהדיא בדברי ת״ק ביום ראשון מרחיצין כדרכו, לא קתני בדר״א בן עזריה אף מרחיצין, אלא מרחיצין כלישנא דמתניתין. ויש לומר דרבא משבש לה לכולה ברייתא.}
\textblock{ הא דאמרינן:\textbf{ אי אמרת הרחצת מילה מי גרע מחמין שעל גבי המכה.} פירש רש״י ז״ל: מי גרע מחמין שעל גבי המכה דשרי, והכא אסרי רבנן בשלישי. ואינו מחוור בעיני, חדא דהא לא שייכא האי קושיא לתרגומה דרבא קמייתא ודתניא כותיה, דהא בהדיא תניא בברייתא ברבנן, ובשלישי מזלפין דאלמא רבנן לא אסרי בשלישי ליתן עליה חמין, והלכך אפילו תמצא לומר דלתרגומה דסבי פליגי רבנן לגמרי ביום שלישי, כלל [ו]כלל לא ואפילו בזילוף, כיון דללישנא בתרא דברייתא לא פליגי, הוה ליה לרבי יעקב למימר בהדיא דאי אמרת להרחצת מילה לתרגומא דסבי מי גרע ממכה. ועוד דהא אפילו לתרגומא דסבי, מנא ליה דפליגי לגמרי      דהא במתניתין לא פירשו רבנן לא יום ראשון ולא יום שלישי, אלא סתמא קתני מרחיצין בין לפני מילה בין לאחר מילה. ואי משום דקתני ר״א בן עזריה אומר מרחיצין ביום השלישי, דלכאורה משמע דרבנן לא שרו ביום שלישי כלל, דאי רבנן שרו לזלף אפילו ביום שלישי, ור״א בן עזריה שרי בכולהו להרחיץ ואפילו בשלישי, א״כ מאי שנא יום שלישי דנקט. לא היא דר״א לרבותא נקטה, לומר דאפילו בשלישי מרחיצין אותו, וכן נמי משמע מהא דאמר רבא, אי אמרת בשלמא ת״ק מזלפין קאמר, היינו דאמר ר״א מרחיצין, אלא אי אמרת ת״ק מרחיצין ביום ראשון קאמר ומזלפין ביום השלישי וכו׳, ר״א בן עזריה אף מרחיצין מיבעי ליה, והשתא אי אמרינן דלרבנן דלתרגומא דסבי ביום שלישי כלל [ו]כלל לא, כל שכן דהוי ליה למיתני אף, כלומר אף ביום השלישי מרחיצין.\par \textbf{} ועוד דאי רבנן לגמרי אסרי אף לזלף ביום שלישי, מנא ליה דר״א פליג בתרתי אדרבנן, ושרי ביום שלישי אפילו בהרחצה גמורה, דלמא לא שרי אלא בחדא, כלומר דמזלפין עליו ביד ביום שלישי, ומרחיצין דר״א בן עזריה כמרחיצין דרבנן. ועוד מאי קא מייתי מרב דאמר אין מונעין חמין מעל גבי המכה, דלמא רב דאמר כר״א דהלכתא כותיה. אלא נראה לי דהכי פירושא אי אמרת הרחצת מילה, אמאי אשמועינן תנא במילה ליתני מכה וכל שכן מילה דמי גרע מילה ממכה, ופרקינן מי לא שני לך בין חמין שהוחמו בשבת לחמין שהוחמו מערב שבת, והלכך נקט מילה, משום דבמילה שרי אפילו בשהוחמו בשבת, ואילו במכה ליכא מאן דשרו אלא כשהוחמו מערב שבת.}
\textblock{\textbf{הלכה כר״א בן עזריה בין בחמין שהוחמו בשבת בין בחמין שהוחמו מערב שבת בין בהרחצת כל גופו בין בהרחצת מילה.} פירוש: בשלישי וכל שכן בראשון ובשני דיותר מסוכן ביום ראשון ושני מביום שלישי, שהרי לתרגומא דברייתא רבנן דפליגי ביום שלישי, מודו ביום ראשון, והוא הדין נמי לתרגומא דסבי, וכל שכן לפירושו של רש״י ז״ל דפירש לתרגומא דסבי דרבנן לא שרו בשלישי אפילו לזלף, וביום הראשון שרו לזלף מיהא, אלמא לכולי עלמא טפי חמור יום ראשון מביום שלישי, ויום שני נמי דכל דלא סליק טפי מצטער ומסתכן. והא דכתיב (בראשית לד, כה) ביום השלישי בהיותם כואבים, לאו למימר דמסתכן דוקא ביום השלישי, אלא משום דבשלישי איכא חולשא טפי,והיינו דאתו עלייהו ביום השלישי ולא ביום הראשון ואף על גב דלכולי עלמא ביום ראשון מיהא מסתכן טפי, משום דאע״ג דמצטערי טפי מכל מקום ליכא חולשא.\par \textbf{} ולפני מילה נמי מרחיצין אותו אפילו בחמין שהוחמו בשבת על ידי עכו״ם שחממן מעצמו, ואפילו עבר ישראל ואמר ליה לעכו״ם זיל אחים לי, ואפילו עבר נמי וחממן ישראל, דכל הני משום קנס הוא דאסרינן ליה בעלמא, אבל הכא במקום מצוה לא קנסינן.\par \textbf{} ותדע לך דאפילו לפני מילה מרחיצין אפילו בחמין שהוחמו בשבת, דהא קתני במתניתין מרחיצין את המילה בין לפני מילה ובין לאחר מילה, אלמא לפני מילה ולאחר מילה חד דינא אית להו בהא. ובמאי דמרחיצין או מזלפין למר כדאית ליה ולמר כדאית ליה לאחר מילה הכי נמי מרחיצין או מזלפין נמי לפני מילה, ומתרגומא דרבא נמי שמעינן לה, דהא לתרגומיה ת״ק שרי ביום ראשון להרחיץ כדרכו בין לפני מילה בין לאחר מילה, ומינה נשמע לר״א בן עזריה דכי היכי דשרי רחיצה כדרכו לאחר מילה הכי נמי שרי לפני מילה. וכן פסק הרב אלפסי ז״ל משמא דרבוותא, וכן כתב מורי הרב רבי יונה ז״ל בהלכותיו.\par \textbf{} וא״ת היאך אפשר לומר דהרחצה דלפני מילה מותרת בחמין שהוחמו בשבת, וכן הרחצת כל גופו קודם מילה, והא מכשירי מילה נינהו, וחמין שהוחמו בשבת אסורין מדבריהם (לעיל שבת לט, ב) בין בשתיה בין ברחיצה, וכן נמי הרחצת כל גופו אסורה מדבריהם ואפילו אבר אבר (לעיל שבת מ, א), ואנן הא קיימא לן כרבנן דר״א דאסרי מכשירי מילה ואפילו בדבר שאין בו אלא שבות דרבנן. יש לומר דחמין שהוחמו בשבת קנס הוא דקנסינן, ובמקום מצוה לא גזרו, וכן הרחצת כל גופו לאו מלאכה הוא שהרי בצונן מותר, וכן פניו ידיו ורגליו אפילו בחמין שהוחמו מערב שבת, אלא שאסרוה מפני הבלנין שהיו מחמין בשבת, ובמקום מילה לא גזרו. וכן תירץ מורי הרב ר׳ יונה ז״ל.\par \textbf{} ואם תאמר עוד והא יהבינן טעמא בגמרא להרחצה דלאחר מילה בהרחצת כל גופו ובחמין שהוחמו בשבת מפני שסכנה היא לו, הא לפני מילה דאין שם סכנה ליתסר ותדחי מילה. תירץ מורי הרב ז״ל דמשום הכי נקט האי טעמא בגמרא, משום דלאחר מילה אי ליכא סכנה לא שריא ולא מידי ואפילו זילוף, אבל לפני מילה התירו לצורך המילה. ועוד דמשום הכי נקט בגמרא האי טעמא, לומר דעל כרחין חמין שהוחמו בשבת נמי שריא. ומעתה הרחצה שהתירו לאחר מילה התירו לפני המילה דחד דינא אית להו בהא, וכדמוכחא מתניתין כדאמרן.\par \textbf{} ולי נראה דכיון דהוצרכו להתיר שבות דרחיצה לגבי מילה, לא רצו לאסור אף לפני מילה דהויא לה כחוכא, לאחר מילה נתיר להדיא מפני שסכנה היא לו, הא אפילו קודם מילה יודעין בה הכל שעל כרחנו יבא לידי כך, ועכשיו נאסור, והלכך כל מה שהתירו בענין רחיצה לאחר המילה התירו אף לפני המילה.}
\textblock{\textbf{והיכא דאישתפוך חמימי ואיבדור סממני לאחר מילה, מחממין ושוחקין מפני הסכנה, ומיהו כל היכא דאפשר       } לשנויי משנינן, וכדתניא במתניתין לא שחק כמון מערב שבת לועס בשיניו, ולא יטרוף יין ושמן אלא נותן זה בפני עצמו וזה בפני עצמו, כלומר בכלי בלא בילה כדאיתא בגמרא.\par \textbf{} אבל אשתפוך קודם מילה ואי אפשר לו להרחיצו לאחר מילה אלא אם יאמר לעכו״ם להחם דקא מידחי ביה שבות דרבנן שיש בו מלאכה, ואי נמי בדליכא ויצטרך ישראל להחם לו, בזה ראיתי מחלוקת בין הגדולים, שהרמב״ן ז״ל כתב שאם יש לו חמין להרחיצו לפני המילה ואין לו להרחיצו לאחר מילה, שרוחצין אותו ומלין אותו שאין כאן מכשירין דוחין כלום, ואחר שמל הרי כאן סכנת נפשות שדוחה שבת, ואין אומרים תדחה מילה כדי שלא להביא אותו לסכנה ונדחה שבת, אלא מילה עצמה דוחה שבת וסכנת נפשות דוחה שבת, ואין למצוה אלא שעתה, ואין לדחות מילה מפני דחיית שבת שיבא אחר מכן מפני הסכנה. והביא ראיה ממתניתין דקתני לא שחק כמון מערב שבת לועס ואם לא התקין מערב שבת כורך על אצבעו ומביא ואפילו מחצר אחרת, ולא קתני כשאין לו כמון בביתו או שאי אפשר ללעוס בשיניו תדחה, ומדקתני תקנתא ולא קתני נמי דחייה שמע מינה כדאמרן. ועוד דכורך על אצבעו ומביא דרך רשות הרבים איסורא דאורייתא הוא דמשאוי הוא לו, אי נמי מחצר אחרת שלא נשתתפו נמי איסורא הוא ונדחה מילה מידי דהוה אאיזמל שאין מביאין אותו דרך שער בשינוי ואפילו במבוי שאינו משותף, אלא שאין צרכי סכנה שלאחר מילה דוחין אותה מתחלה, אלא מל ואחר כך מחלל לצורך הסכנה ע״כ.\par \textbf{} ולפי דברי הרב ז״ל הא דאמרינן בעירובין (סז, ב) בההוא ינוקא דאשתפוך חמימיה, דקודם מילה הוא. ואין נראה כן מדברי הגאונים ז״ל, אלא שנשפכו חמימי דלאחר מילה. ולא התירו הגאונים ז״ל אלא בדאישתפוך חמימי ואיבדור סממני לאחר מילה, הא קודם מילה תדחה, דעל כרחין אתי למידחי שבת במישחק סממני ומיחם חמימי. וכ״כ הר״ז הלוי ז״ל מפורש. ומסתברא לי כותייהו מדתני בפרק קמא דביצה (ב, א) השוחט בית שמאי אומרים יחפור בדקר ויכסה, וב״ה אומרים לא ישחוט אלא אם כן היה לו עפר מוכן מבעוד יום, ושוין שאם שחט שיחפור בדקר ויכסה. ואם כדברי הרמב״ן ז״ל, לעולם שוחטין ואע״פ שאין לו דקר נעוץ מפני מצות שמחת יום טוב, דבשעת שחיטה אין כאן דחוי כלל, ולאחר שחיטה הרי כאן מצות עשה דכסוי, ומצות עשה הוא דדוחה חפירת הדקר דאיסורא דרבנן הוא, ואין דוחין מצות שמחת יום טוב בשעתה שלא נבוא לידי איסורא דרבנן דלאחר שחיטה, אלא ודאי שמעינן דכל דבר שאנו יודעין בתחילת הענין שיבא לידי איסור מלאכה ואפילו דרבנן שאינה נתרת קודם זמן תדחה אותה מצוה כדי שלא נבוא לבסוף לידי כך וכל שכן במלאכה דאורייתא.\par \textbf{} ומה שהתירו לו לכרוך על אצבעו להביא מחצר שאינה מעורבת ולא דחו את המילה בכך, כדדחינן לה מפני הבאת האיזמל ואפילו דרך גגות וקרפיפות. יכילנא לשנויי, דאין הכי נמי אי אתידע מלתא מקמי מילה, ומתניתין בדלא ידעי עד לאחר מילה, ואפילו הכי כל היכא דאיכא לשנויי משנינן. ואלא מיהו אפילו תמצא לומר דבאתידע מקמי מילה מיירי, לא קשיא ולא מידי, דהתם שאני דכיון דעל כרחין מידחיא שבות דרבנן בלאחר מילה בהכין דהיינו נתינת אספלנית משום דהוה ליה הני אי אפשר לעשותה מערב שבת, אפילו אתידע מקמי מילה נמי לא דחי׳ מילה מקמי שבות דלאחר מילה דאין בה איסורא דאורייתא, והא דלא קתני נמי לא הביא כמון מערב שבת או שאי אפשר לו ללעוס תדחה מילה, אינה ראיה, דהא קתני רישא כלל אמר ר׳ עקיבא כל מלאכה שאפשר לעשותה מערב שבת אינה דוחה את השבת, והלכך תנא לאו כרוכלא תני ואזיל, ואם אי אפשר לעשותה מערב שבת, אלא הבאת כמון ואפילו מרשות הרבים שרי, כיון דאפשר להביאו בשער או לאחר ידו דאין בו אלא שבות דרבנן שאין בזה מלאכה דאורייתא, וכל שבות דלאחר מילה שאין בו מלאכה שרי כדאמרן, דמה לי דחיית שבות נתינת אספלנית, מה לי שבות אחר דלאחר מילה, כך נראה לי.}
\clearpage
\newsection{דף קלה}
\textblock{ גירסת הגאונים ז״ל:\textbf{ אמר רבה מנא אמינא לה דתניא ר״א הקפר ברבי אומר לא נחלקו ב״ש וב״ה וכו׳, לאו מכלל דת״ק סבר דברי הכל אין מחללין, אם כן ר״א הקפר טעמא דב״ש אתא לאשמועינן, דלמא הכי קאמר לא נחלקו ב״ש וב״ה בדבר זה.} ות״ק דקאמר הכא היינו ר״ש בן אלעזר, וכן נמצא בתוספתא (פט״ז, הל׳ ח). והא דקאמר לאו מכלל דת״ק סבר דברי הכל אין מחללין, יש מפרשים דאי אמרינן ת״ק דהיינו ר״ש בן אלעזר סבר דצריך להטיף ממנו דם ברית אפילו בשבת קאמר דודאי ערלה כבושה היא, היכי קאמר ר״א לא נחלקו ב״ש וב״ה בחול שצריך להטיף ממנו דם ברית, אמר ת״ק אפילו בשבת ואמר ליה איהו אפילו בחול,      שמע מינה דת״ק דהיינו ר״ש ב״א לא אמר אלא בחול ואפילו לב״ש, ולפיכך אמר ליה ר״א הקפר דבדבר זה (הוא דנחלקו) [לא נחלקו], דלכולי עלמא צריך ודאי להטיף ממנו דם ברית כדקאמר ר״ש בן אלעזר, אלא דמחלוקתן בשבת היא, דלב״ש מחללין עליו את השבת, ונמצא רבה דאמר בין כר״ש בן אלעזר בין כר׳ אלעזר הקפר, ורב יוסף דלא כחד.\par \textbf{} ויש מפרשים דהכי פירושא, לאו מכלל דת״ק סבר אין מחללין, דכיון דאשכחן לר״א הקפר דאמר דספק ערלה כבושה היא לב״ה, מסתמא אף ר״ש בן אלעזר הכי סבירא ליה, דמדר׳ אלעזר נשמע לר״ש, דבכדי לא נטיל מחלוקת ביניהם אלא במאי דאשכחן להו דפליגי דהיינו לב״ש דאינה משנה. ולי נראה דטעמא משום דאשכחן לרשב״א דמוסיף בהא את״ק, הכי נמי לר״א מוסיף אדר׳ שמעון, דכל תנא בתרא לטפויי מלתא קא אתי, והלכך ת״ק דר״ש אמר דאף בחול אין צריך להטיף ממנו דם ברית אלא לב״ש בלחוד, ור״ש אומר דאף לב״ה צריך להטיף בחול אבל לא בשבת משום דלכולהו ספק ערלה כבושה היא, ור״א הקפר על ר״ש מוסיף ואמר דלבית שמאי אפילו בשבת נמי דודאי ערלה כבושה היא לבית שמאי, ואקשינן אי הכי רבי אלעזר הקפר טעמא דבית שמאי אתא לאשמועינן, דאילו לענין דינא ר״א ור״ש הושוו לדעת אחת ולשניהם אין מחללין את השבת לבית הלל, ובטעמא דבית שמאי הוא דפליגי ומה בכך והא בית שמאי במקום בית הלל אינה משנה.\par \textbf{} ופרקינן דלמא הכי קאמר לא נחלקו ב״ש וב״ה בדבר זה, יש מי שפירש דלמא הכי קאמר ליה לר״ש, לא נחלקו ב״ש וב״ה בהטפת גר שנתגייר כשהוא מהול (כדקאמר) [כדקאמרת], דבההוא לא נחלקו שצריך להטיף ממנו דם ברית, אלא בהטפת נולד שהוא מהול פליגי ולחלל עליו את השבת, ולא א״ר אלעזר הכין לאשמועינן טעמא דב״ש בנולד כשהוא מהול, אלא לאשמועינן דבההיא הוא דפליגי ולא בגר שנתגייר כשהוא מהול. ויש מי שפירש דאת״ק דר״ש ב״א קאי כלומר דבין ר׳ שמעון בן אלעזר בין ר׳ אלעזר הקפר אמרו ליה לת״ק, לא נחלקו ב״ש וב״ה על נולד כשהוא מהול שצריך להטיף ממנו דם ברית, אלא על מה נחלקו, ר״ש ב״א אמר על גר שנתגייר כשהוא מהול, ורבי אליעזר הקפר קאמר בנולד כשהוא מהול עצמו ולחלל עליו את השבת.\par \textbf{} ורש״י ז״ל גורס: א״ר יוסף מנא אמינא לה דתניא ר״א הקפר אומר וכו׳, לאו מכלל דת״ק סבר דברי הכל מחללין, ודחינן דלמא ת״ק סבר ד״ה אין מחללין, ואקשינן אם כן ר״א הקפר טעמא דב״ש אתא לאשמועינן, ופרקינן דלמא הכי קאמר לא נחלקו ב״ש וב״ה בדבר זה, ואף לגירסא זו נמצא שהתלמוד דחה רב יוסף מאותה ברייתא דלא שמע מינה מידי. וגירסת הגאונים ז״ל נראית יותר נכונה.\par \textbf{} ולענין פסק, בנולד כשהוא מהול קיימא לן כר״ש בן אלעזר וכדפסק שמואל. ואע״ג דפסק רב כת״ק, הא איכא רבה ורב יוסף דפליגי אליבא דר״ש בן אלעזר ושמע מינה דסבירא להו כותיה. ובפלוגתא דרבה ורב יוסף קיימא לן כרבה דאמר דספק ערלה כבושה היא ואינה דוחה את השבת, דרבה ורב יוסף הלכה כרבה בר משדה ענין ומחצה (ב״ב קיד, ב), ואפילו אם תמצא לומר דלא אתמר כללא אלא במאי דפליגי בכולה מכלתין דבבא בתרא, מכל מקום לענין פסק הלכה הוי ליה ספיקא אי כרב יוסף, ונקטינן לחומרא ואין מחללין עליו את השבת.\par \textbf{} ולענין גר שנתגייר כשהוא מהול. יש אומרים דכיון דפסק שמואל הלכה כר״ש בן אלעזר, שמעינן מינה דאין צריך להטיף ממנו דם ברית, דסתמא כיון דר״ש בן אלעזר ור״א לא נחלקו בנולד כשהוא מהול אלא בגר שנתגייר, ואמר שמואל הלכה כר״ש בן אלעזר ולא פירש דדוקא בנולד כשהוא מהול, סתמא דמלתא בכל מאי דאמר ר״ש בן אלעזר פסק כותיה. ועוד יש מרבותינו ז״ל שאמרו שאילו לא פסק שמואל כר׳ שמעון בן אלעזר אלא בנולד כשהוא מהול בלבד אבל בגר שנתגייר כשהוא מהול לית ליה כותיה, הוי ליה לפסוק הלכה כרבי אלעזר הקפר דאוקי פלוגתייהו בנולד כשהוא מהול ולחלל עליו את השבת אבל בחול צריך להטיף, וכן גר שנתגייר כשהוא מהול לא נחלקו שצריך להטיף ממנו דם ברית. אבל הגאונים ז״ל וכן בהלכות הרב אלפסי ז״ל פסקו בגר שנתגייר כשהוא מהול שצריך להטיף ממנו דם ברית.\par \textbf{} ויש מביאים ראיה לדברי הגאונים ז״ל, משום דסוגיין בתלמודא הכין, מדאותביה רבי אלעזר בן פדא לר׳ יוחנן בפרק הערל (יבמות עב, ב) מדתניא אין לי אלא נמול ביום שמיני, נמול לתשעה נמול לעשרה לי״א לי״ב וגר שנתגייר כשהוא מהול מנין, תלמוד לומר וביום, אלמא גר שנתגייר צריך להטיף ממנו דם ברית. ועוד דקיימא לן כרבי יוסי דאמר בפרק החולץ (יבמות מו, ב) א גר עד שימול ויטבול, וכדאיפסקא התם הלכתא בהדיא, ומשמע דכללא הוא לכל הגרים בין ערלים בין כשהן נמולין. ועוד דתניא התם, הרי שבא ואמר מלתי ולא טבלתי מטבילין אותו, ור׳ יוסי אומר אין מטבילין אותו, ומפרשי רבוותא משום דחייש ר׳ יוסי דלמא ערבי מהול הוא וצריך להטיף ממנו דם ברית, וכן פירש שם רש״י ז״ל. וכבר איפסיקא התם הלכתא כרבי יוסי דאינו גר עד שימול ויטבול.\par \textbf{} ומה שאמר שמואל הלכה כר׳ שמעון, לאו אגר שנתגייר קאי אלא אנולד כשהוא מהול לבד, ואע״ג דר׳ שמעון תרתי קאמר והוא פסק סתם, אורחא דתלמודא בהכין למיפסק סתם ואע״ג דלית ליה כותיה אלא בחדא, וכההיא דאמרינן בפסחים (יג, א) מאי לאו לאכול לא לבער. ומיהו מה שהביאו ראיה מההיא דתניא בפרק הערל (שם) דאותביה ר״א לרבי יוחנן אין לי אלא נמול לשמנה וכו׳ גר שנתגייר כשהוא מהול מנין, אינה ראיה מחוורת בעיני, דאי מברייתא גופה קא מייתי ראיה, לאו ראיה היא דדלמא ההיא ר׳ אלעזר הקפר או תנא אחר הוא דאית ליה הכין, והתם נמי איכא תנאי אחרינא דתניא התם משוך ונולד כשהוא מהול וגר שנתגייר כשהוא מהול וקטן שעבר זמנו ושאר כל הנמולין, לאתויי מי שיש לו שתי ערלות אין נמולין אלא ביום, ר״א בר״ש אומר בזמנן אין נימולין אלא ביום כו׳. (ומשום) [ומאי] דמותיב מינה בר פדא לאו מגר שנתגייר קמותיב לר׳ יוחנן אלא מנמול לתשעה לעשרה וכו׳, כלומר שאע״פ שנמולין שלא בזמנן אין נמולים אלא ביום.\par \textbf{} וההיא דאינו גר עד שימול ויטבול, אינה ראיה גמורה, דדלמא התם בגר שלא נמול משום דאפשר למולו דכל דאפשר למולו צריך למולו כאבות, ולא גמרינן מאמהות שטבלו ולא מלו משום דאין דנין אפשר משאי אפשר, דזו היתה תשובתו של רבי יהודה, אבל בגר שנתגייר כשהוא מהול דאי אפשר למולו גמרינן ליה מאמהות ובטבילה סגי, ולא איפסיקא התם הלכתא כרבי יוסי אלא בההיא, אבל בההוא שבא ואמר מלתי ולא טבלתי לא איפסקא הלכתא כותיה. וכן השיב מורי הר״מ ז״ל.\par \textbf{} והר״ז הלוי ז״ל השיב דדלמא ההיא דאמר ר׳ יוסי אין מטבילין, לא משום חשש ערבי הוא, אלא משום חשש דלמא נולד כשהוא מהול הוה. ואין זה מחוור בעיני, דנולד כשהוא מהול מיעוטא הוא, ורבי יוסי לא חייש למעוטא (יבמות סז, א). וקבלת הגאונים ז״ל תכריע. וההיא דא״ר יוסי, משום נולד כשהוא מהול ליכא למימר, דמיעוטא הוא כדאמרן, ואע״ג דלא איפסיקא בההיא בהדיא הלכה כר׳ יוסי, מכל מקום קיימא לן (עירובין מו, ב) דר׳ יוסי ור׳ יהודה הלכה כר׳ יוסי. וההיא דאינו גר עד שימול ויטבול נמי ראיה, דאפילו גר שנתגייר כשהוא מהול לא דמי לאמהות, דהכא אפשר הוא בהטפת דם ברית.\par \textbf{} ור״ח ז״ל כתב דגר שנתגייר כשהוא מהול אין לו תקנה, אבל בניו נמולין לשמונה ונכנסין לקהל, דהא אתגייר בטבילה וכגר חשוב להכשיר זרעו, אבל בעצמו לא. ואינו מחוור דאי סבירא לן כר״ש בן אלעזר, אף הוא עצמו כשר ונכנס בקהל, דהא אין צריך להטיף ממנו דם ברית קאמר, ואי לאו כותיה סבירא לן הא סגי ליה בהטפת דם ברית, וכההיא דתניא התם בפרק הערל דכתבינן לעיל. ועוד הקשה עליו הרמב״ן ז״ל מההיא דאמר ר׳ יוסי התם אין מטבילין אותו, כלומר עד שיטיף, שאלמלא אין לו תקנה לעולם מטבילין אותו להכשיר זרעו אבל לא להכשיר עצמו. ואין נראה לי [לומר] דצריך להטיף ממנו דם ברית להכשיר זרעו, דהא משמע דהטפת דם ברית כמילה.\par \textbf{} ולענין הברכה כתב הרמב״ן ז״ל, שהמל את הגר ועבד שנתגייר כשהוא מהול, מברך אקב״ו להטיף דם ברית מן הגרים או מן העבדים, שאין הטפה זו מספק אלא ודאי, שאנו חייבים להטיף מהם דם ברית להכניסו בבריתו של אברהם אבינו ע״ה ובה נכנסין תחת כנפי השכינה, אבל נולד כשהוא מהול כיון דקיימא לן משום ספק ערלה כבושה הוא שמטיפין ממנו דם ברית, אין מברכין. וכן הורו הגאונים ז״ל וכ״כ הרב אלפסי בהלכותיו. והוא הטעם שאין מברכין להכניסו בבריתו של אברהם אבינו דשמא אין זו ערלה. אבל מורי הרב ז״ל כתב שצריך לברך, דכיון דאנן משום ספק בשל תורה מטיפין ממנו, בספיקא דאורייתא מברכין. ומחלוקתם תלויה בההיא דפרק במה מדליקין (שבת כג, א) דאמר אביי ודאי דדבריהם בעי ברכה ספק דדבריהם לא בעי ברכה, וכבר כתבתים שם בסייעתא דשמיא.}
\textblock{ הא ד\textbf{אמר (רב) [רבה] אמר רב אסי כל שאמו טמאה נמול לשמונה.} יש מי שפסק דליתא משום דאתי כתנאי, ואתיא דרבה אמר רב אסי כרב חמא דיחידאה הוא, וקיימא לן (לעיל שבת קל, ב) דאין הלכה כיחיד במקום רבים, ואי לאו דאיפליגו בה אמוראי הוה אמרינן דהלכתא כרב חמא ומשום דפסק רב אסי כוותיה, אבל השתא דאפליגו בה אמוראי דהיינו רב הונא וחייא בר רב, וכדאמרינן דפלוגתייהו בהא תליא, ואתיא מאן דאמר דבטומאת לידה תליא מלתא כרב חמא ומאן דאמר דלאו בטומאת לידה תליא מלתא כתנא קמא, קיימא לן כמאן דקאי כתנא קמא, ועוד דרבא      משרשיה ור׳ ירמיה שקלי וטרו אליבא דתנא קמא. וכן דעת הרמב״ן ז״ל.\par \textbf{} אבל מורי הרב ז״ל פסק כרב אסי, משום דכיון דאוקימנא לה בפלוגתא דרב הונא וחייא בר רב, הוי ליה מ״ד אין יוצא דופן נמול לשמונה ורב אסי רבים ומ״ד יוצא דופן נמול לשמונה יחיד, וקיימא לן כרבים. ואף על גב דאוקימנא לה בגמ׳ כתנאי וקם ליה רב אסי כיחידאה, אין לתלות דברייתא לא שמיעא להו לרב אסי ולמ״ד יוצא דופן אינו נמול לשמונה, אלא שמיעא להו ואיסתבר להו כרב חמא, וכל היכא דפליגי אמוראי חד פסיק כת״ק ותרי פסקי כיחידאה אין לנו להכריע מסברא כההוא דפסיק כרבים בלא ראיה, דהויא לה כפלוגתא דאמוראי וקיימא לן כהנהו אמוראי דהוו להו רבים אצל יחיד.\par \textbf{} והרב אלפסי ז״ל לא פסק כחד מנייהו, אלא שכתב בהלכות דברי כולם, וכולה שקלא וטריא דגמרא. ויש מי שסובר דמסתפקא מלתא להרב ז״ל ולפיכך לא פסק בה כלום, אלא מספיקא אין אחד מהם נמול אלא לשמונה ואין מחללין עליו את השבת דמטילין אותו לחומרא, וכן דעת הר״ז הלוי ז״ל. וכן דעת הרב בעל ההלכות ז״ל דמילתא ספיקא היא, ונמול לשמונה ואין מחללין עליו את השבת.}
\textblock{ הא ד\textbf{אמר רשב״ג כל ששהה שלשים יום באדם אינו נפל.} כללא כייל לכל הנולדים בין שנולד לשמונה ובין לשאר תינוקות, דכל דלא קים לן בגויה שכלו לו חדשיו אינו יוצא מכלל נפל אלא אם כן שהה שלשים יום, ואפילו סתם תינוקות ואפילו גמרו סימניו שערו וצפרניו. והיינו דאקשינן היכי מהלינן, ואי דוקא בבן ח׳ ואי נמי בסתם תינוקות ובשלא גמרו סימניו, לא הוי מקשי סתם היכא מהלינן אלא בן שמונה היכי מהלינן, אי נמי כל שלא גמרו סימניו היכי מהלינן. ובהא רשב״ג לחומרא ורבנן לקולא, וכדמשמע בשמעתין דרבנן לקולא ורשב״ג לחומרא, וכדמשמע נמי ריש פרק החולץ (יבמות לו, ב) גבי הא דמייתינן לקמן מת בתוך שלשים יום ועמדה ונתקדשה. ואפילו נולד לשמונה אילו שהה שלשים יום נמי יצא מכלל נפל לרשב״ג, דכללא כייל כל ששהה שלשים יום אינו נפל, משום דאיהו כרבי סבירא ליה דאמר כן ביבמות בפרק הערל (יבמות פ, ב) דסבירא להו לרבנן דלא משתהי בבטן כלל, ורבנן ורשב״ג בתרתי פליגי וכדאמרן. ומיהו רבי ורשב״ג לאו בחדא שטתא ממש קיימי, אלא רבי סמיך אגמר סימניו שערו וצפרניו, ורשב״ג לא סמיך אסימנים אלא אנשתהה שלשים יום. והכין איתא בתוספתא דמכלתין פרק (ט״ז, ד) אין עוקרין בהמה, דעביד לה פלוגתא דת״ק ורבי ורשב״ג. והא דאיבעיא לן לקמן מי פליגי רבנן עליה דרשב״ג ולא פשטוה מההיא ברייתא, דלמא לא שמיעא להו, ואי נמי דניחא ליה לאתויי ההיא דר״י אמר שמואל, דשמעינן מינה תרתי דפליגי רבנן עליה והלכתא כותיה, וכבר כתבתיה בארוכה בפרק הערל ביבמות (שם) בסייעתא דשמיא.}
\clearpage
\newsection{דף קלו}
\textblock{ הא ד\textbf{אמר אביי כתנאי.} פירש רש״י ז״ל: דאדרב אדא בר אהבה קאי, וכי אמרינן דכולי עלמא מת הוא, לומר דכולהו אית להו כרב אדא דמת הוא וכמחתך בשר בעלמא הוא. ויש מרבוותא ז״ל שהקשו עליו, דאם כן דכולהו סבירא להו דמחתך בשר בעלמא הוא, היכי רבי יוסי בר יהודה ור״א בר״ש סברי דהוי כטריפה, דהא מחתך מן הטריפה בשבת חייב משום חבורה, והשוחט את הטריפה חייב משום נטילת נשמה וכדאיתא בפסחים בפרק אלו דברים (פסחים עג, א), ולפיכך פירשוה דארשב״ג קאי.\par \textbf{} ואינו מחוור בעיני, דאם כן למסקנא דאמרינן דכולי עלמא מת הוא וכרשב״ג, א״כ ר׳ יוסי בר יהודה ור״א בר״ש לכל הפחות דלא עבדי לה מתה טפי מטריפה, אם כן תהדר לה קושיין לדוכתייה דלדידהו היכי מהלינן. וליכא למימר דלרשב״ג אליבא דידהו לא מהלינן אלא מי שכלו לו חדשיו, דאם כן לא הוה מקשינן היכי מהלינן, דמשמע דפשיטא להו דמהלי לכולי עלמא וליכא מאן דאסר אלא דטעמא איסתפקא להו היכי מהלינן, אלא נראה כמו שפירש רש״י ז״ל.\par \textbf{} והא דאמרינן כטריפה לאו למימרא כטריפה ממש, אלא לטהרה מידי נבלה בלחוד, וכדמפרשי אינהו טעמא דהא עיקר שחיטה אינה אלא להכשיר באכילה ואפ״ה אהני לה שחיטה לטהרה מידי נבלה, דלמאכל שיש לה רוח חיים מהני לה שחיטת סימניה לטהרה מידי נבלה, הכא נמי אע״ג דמת הוא שחיטת סימניו מטהרין אותו מידי נבלה, כן נראה לי לפי דברי רש״י ז״ל.}
\textblock{\textbf{ולכולי עלמא מהלינן ספקות ממה נפשך כרב אדא בר אהבה. והתימה מן הרב אלפסי ז״ל שכתב להא ברייתא דספק בן שבעה ספק בן שמונה אין מחללין עליו את השבת בפסק הלכה, ואילו בגמרא העמידוה כר״א ולמכשירי מילה. ונראה שהרב ז״ל סבור דלא קיימא לן כהא דרב אדא בר אהבה ודחייה בעלמא היא, ולא מהלי׳ לרשב״ג אלא מי שקים לן בגויה שכלו לו חדשיו. וכן כתב ר״ח ז״ל דלית הלכתא כרב אדא בר אהבה דשנויא בעלמא היא, והא       } אתי שפיר לפי הפירוש השני שכתבנו בהא דאביי וכמו שכתבתי, וצ״ע לפי דבריהם ז״ל.}
\textblock{ הא ד\textbf{אמר אביי כשפיהק ומת דברי הכל מת הוא.} פירש מורי הרב ז״ל: דלאו דוקא פיהק אלא הוא הדין לחלה ומת, דאם לא כן מאי קאמר אכלו ארי או שנפל מן הגג מר סבר חי הוא, דאלמא לא נחלקו אלא בשאכלו ארי, דאי לא לימא אבל חלה ומת. אבל הר״ז הלוי ז״ל כתב דדוקא פיהק ומת דמוכחא מלתא דאית ביה ריעותא, אבל חלה ומת הרי הוא כאכלו ארי דאפילו בן קיימא נמי יחלה וימות. והא דאביי לא כתבה הרב אלפסי ז״ל, לומר שאינה הלכה, דאידחייא לה מהא דאמר רבא מת בתוך שלשים יום ועמדה ונתקדשה אם אשת כהן היא אינה חולצת, ואוקימנא התם בריש פרק החולץ (יבמות לז, א) כרבנן דרבן שמעון בן גמליאל, דאלמא אפילו בשלא אכלו ארי פליגי רבנן.\par \textbf{} וקשיא לי קצת דאי איתא דההיא דרבא פליגא אדאביי, אמאי לא אקשו ליה מינה כדאקשי ליה מההיא דרב פפא ורב הונא בריה דרב יהושע. ואפשר לומר דלא שמיע להו ההיא דרבא. ועוד יש לומר דלא אקשינן הכי מדרב פפא ורב הונא בריה דרב יהושע, אלא לומר דמינה שמעינן דאביי לאו הכי אמרה אלא איפכא דמסתמא לא פליגי אהדדי, אלא אדרבא ודאי פליגא ולא אקשינן לאביי מדרבא. אבל מורי הרב ז״ל פסקה להא דאביי בפסקי הלכותיו, וכתב דההיא דרבא בשנפל מן הגג או כשאכלו ארי. והר״ז הלוי ז״ל פירש דההיא דרבא בשחלה ומת דהיינו כאכלו ארי, ולא אמרינן אלא כשפיהק ומת, וכמו שכתבנו למעלה.}
\clearpage
\newsection{דף קלז}
\textblock{ הא דאמרינן:\textbf{ ומאי שנא גבי מילה} ופרקינן דכתיב (בראשית יז, י) המול לכם כל זכר. משמע טעמא דכתיב כל זכר הא לאו הכי לא. וקשה דהא איצטריך למעוטי מערכין מדכתיב (ויקרא כז, ג ד) הזכר ואם נקבה, טעמא דכתיב ואם נקבה, הא זכר ונקבה משמע אפילו טומטום ואנדרוגינוס. ויש לומר דזכר ונקבה זכר ודאי נקבה ודאית משמע, ולא איצטריך גבי ערכין למכתב הזכר ואם נקבה, אלא משום דזכר ונקבה מבעי ליה לחלק בין ערך איש לערך אשה, וכדאיתא במסכת נדה פרק המפלת (נדה כח, ב) גבי המפלת טומטום ואנדרוגינוס דאמרינן התם וכל היכא דכתיב זכר ונקבה למעוטי טומטום ואנדרוגינוס, והא גבי ערכין דכתיב הזכר ואם נקבה, ותניא הזכר ולא טומטום ואנדרוגינוס, יכול לא יהא בערך איש אבל יהא בערך אשה ת״ל הזכר ואם נקבה, זכר ודאי נקבה ודאית ולא טומטום ואנדרוגינוס, טעמא דכתיב הזכר ואם נקבה הא לאו הכי מזכר ונקבה לא מימעיט, ההיא מבעי לי לחלק בין ערך איש לערך אשה.}
\textblock{\textbf{אבי הבן מברך אקב״ו להכניסו בבריתו של אברהם אבינו.} רבנו שמואל ז״ל היה אומר דצריך לברך כן קודם המילה, דכל המצות כולן מברך עליהן עובר לעשייתן. ור״ת ז״ל חלק עליו, שמשנה סדר הברייתא ששנה תחלה ברכת המל ואח״כ ברכת אבי הבן, ועוד דקתני העומדין שם אומרים כשם שנכנס לברית כו׳ דאלמא משמע שנכנס ונמול כבר. ומה שאומרים להכניסו ואמרינן בפסחים (ז, א) כולי עלמא לא פליגי בלבער דלהבא משמע. לא קשיא דלאו למימרא דלהבא דוקא משמע ולא לשעבר, אלא דטפי משמע להבא מלשעבר, אבל הכי נמי דמשמע לשעבר, ובעל ביעור הוא דפליגי, דמר סבר דמעיקרא בלחוד משמע ומר סבר אפילו להבא משמע, והלכך להכניסו שפיר משמע לשעבר. ולא דמי לכל המצות שמברך עליהן עובר לעשייתן, דהכא אינו אלא שבח והודאה בעלמא על שזיכהו להכניסו בבריתו של אברהם אבינו, כמו אשר קדש ידיד מבטן.}
\textblock{\textbf{ידיד מבטן.} פירש רש״י ז״ל: דהיינו יצחק שנתקדש מן הבטן. אבל ר״ת ז״ל כתב שזהו אברהם, כדדרשינן מה לידידי בביתי זה אברהם, כדאיתא בפרק כל המנחות באות מצה (מנחות נג, ב). ובברכה זו מזכיר שלש האבות, ידיד זה אברהם, חוק בשארו שם זה יצחק, וצאצאיו חתם באות ברית קדש זה יעקב. ואם תאמר מנא ליה באברהם שנתקדש מן הבטן. יש לומר דדלמא קים להו דילפי לה בגזירה שוה דידיעה ידיעה, באברהם כתיב כי ידעתיו, ולהלן כתיב (ירמיה א, ה) בטרם אצרך בבטן ידעתיך.}
\textblock{ ה״ג בהלכות הרי״ף ז״ל וכן הוא בספרים מדוייקים:\textbf{ המל את הגרים אומר אקב״ו למול הגרים ולהטיף מהם דם ברית, שאלמלא דם ברית לא נתקיימו שמים וארץ כו׳ בא״י כורת הברית.} ולאחר המילה אינו מברך כלום. ואם תאמר למה בברכת הגרים תקנו לומר למול ולא על המילה כמו שאר הנמולין, והיה להם לומר על מילת הגרים. כבר תירץ הרמב״ן ז״ל משום דמילת גר חובה גמורה היא על המל, ובעי לברוכי למול דהא לא סגיא דלאו איהו מהיל דלא אפשר למעבדה ע״י שליח, דכל דמהיל מצוה דנפשיה קא עביד. אלא דקשה לי קצת, דאם כן מאן דלא מהליה אבוה דבית דין מחויבים למימהליה כדרשינן בפרק קמא דקדושין (כט, א) מהמול לכם כל זכר, א״כ התם ליבריך למול, דכל מאן דמהיל ליה מצוה דנפשיה קא עביד.\par \textbf{} והתירוץ השני שתירץ הרב ז״ל טוב מן הראשון, שהוא ז״ל הקשה למה כללו כאן שתי הברכות של מילה באחת ולא ברכו בה תחלה וסוף כשאר הנמולין, ותירץ לפי שאינו גר עד שימול ויטבול לא ברכו עליה על המילה, כשם שאין מברכין על הטבילה אלא הוא בעצמו בעלייתו מברך עליה, אבל כללו הכל בברכה זו לפי שדם המילה בריתן של ישראל הוא ועל הברית אנו מברכין. וכענין זה היא ברכת אירוסין, שמברכין אקב״ו על העריות ואסר לנו את הארוסות. ומה שתקנו בה [לומר] להטיף מהם דם ברית, לפי שיש כמה גרים שמתגיירין כשהן נמולין ועיקר מילתן הטפת דם ברית, לפיכך תקנו לעולם כן. זו דעת הרמב״ן ז״ל וזה נכון.\par \textbf{} ויש ספרים שכתוב בהן, המל אומר על המילה והמברך אומר למול את הגרים. וכן כתב הרב בעל ההלכות ז״ל דתרתי תקנו תחלה וסוף כשאר המילות. והר״ז הלוי ז״ל הקשה עליו דהיאך יברך לבסוף למול, ולמול להבא משמע וכדאמרינן בלבער. וכבר כתבתי למעלה משם רבותינו הצרפתים ז״ל דלבער ולמול ולהכניסו לאו דוקא להבא, אלא דמשמע טפי להבא ומכל מקום לשעבר נמי משמע, מה שאין כן בעל ביעור דמשמע מעיקרא למ״ד דסבירא ליה הכין התם, והיינו דמברך אבי הבן להכניסו אחר מילה, והכא נמי כיון דלאחר המילה הוא מברך אין לטעות בו שיהא מברך להבא שכבר בירך לפניה ברכה אחרת אלא על המילה שכבר נעשית. ולשון אחר כתב הרמב״ן בזה, שנוסח ברכה הוא שקדשנו למול הגרים כשהן באין אלינו להתגייר ולהכניסם בברית. ומכל מקום גירסת הרי״ף ז״ל נראה יותר, וכך הסכימה דעת כל גדולי האחרונים ז״ל.}
\textblock{\textbf{דרבי אליעזר עדיפא מדרבי יהודה.} יש לפרש דרבי אליעזר לית ליה הוא לחלק, ויש לפרש דאית ליה וכרבי יהודה ממש, ומיהו דוקא במלאכות דאורייתא אבל במלאכות דרבנן שפיר דמי, כיון דהתירה התורה אפילו מכשירין שיש בהן מלאכה גמורה בשאי אפשר לעשותן אף רבנן התירו מלאכות שלהן אע״פ שאפשר לעשותן מערב יו״ט, כן כתבו בתוס׳. והראשון נראה לי עיקר, דאם איתא מנא ליה דר״א עדיפא מדר׳ יהודה דלמא אף ר״י יתיר במלאכות דרבנן ומדר״א נשמע ליה. ויש לדחות שהשיב לו כן, כלומר כשתמצא לומר דר״י לא יתיר בשאפשר לעשותן דר״א עדיפא.}
\textblock{\textbf{תלה כוזא בסיכתא הכי נמי דמחייב.} וא״ת כיון דמדמי אביי תלית משמרת לתלית כוזא, אם כן אף למה דקאמר איהו טעמא מדרבנן שלא יעשה כדרך שהוא עושה בחול, ליתסר נמי למיתלי כוזא בסיכתא. פירשו בתוס׳ דאי הוה מדאורייתא, בדאורייתא אין לחלק, אבל בדרבנן יש לחלק, דזה נראה כעובדין דחול וזה אינו נראה.}
\newchap{פרק \hebrewnumeral{20} תולין}
\clearpage
\newsection{דף קלח}
\textblock{}
\textblock{\textbf{אלא אמר אביי מדרבנן שלא יעשה כדרך שהוא עושה בחול.} פירש רש״י ז״ל דלאו משום אהל עראי הוא, אלא שלא יתקן לשמר בקבע כדרך שהוא עושה בחול, והראיה מדאמרינן לקמן (שבת קלט, ב) תולה אדם את המשמרת ביום טוב לתלות בו אגוזים ורמונים ותולה בו שמרים, ואי טעמא משום אהל מה לי תולה רמונים מה לי תולה לשמרים מכל מקום הא עביד אהל עראי, והקשה עליו הרמב״ן ז״ל דמדאקשינן לעיל לר״א אוסופי על אהל עראי אסור לעשות, ואיך לכתחלה שרי, אלמא טעמא דמשמרת משום אהל עראי הוא. ותירץ הוא ז״ל דדלמא סוגיא דלעיל בשיטת רב יוסף אתמר. ומכל מקום עדיין קשה דהא מדנקיט אביי חומרי מתניתא ותני משמרת בהדי כולה, משמע דמשמרת נמי משום אהל עראי הוא. על כן פירש הוא ז״ל דאביי ודאי משום אהל עראי קאמר, וכלפי מאי דקאמר ר׳ יוסף חייב חטאת, אמר אביי דליכא אלא שבות בעלמא דהיינו עובדא דחול. והאי דשרי׳ ברמונים משום דתולה לשמרים מותחה יפה וקובעה בכסבין והתולה לרמונים אין צריך לקובעה ולא למתחה יפה ולאו אהל הוא כלל.\par \textbf{} ואין זה מחוור בעיני כלל, דאם משום דמותחה ואינו מותחה או משום קביעותא הוא, מה לי רמונים מה לי שמרים, לימא ואם אינו מותחה יפה ואינו קובעה בכסבין שרי, וכדאמרינן נמי גבי כילה דאם אין לה גג טפח מותר לנטותה ואם יש לה טפח אסור. ועוד דאי היתרא דרמונים משום ההוא טעמא נינהו, למה לי לתלות בו רמונים תחלה וכדאמר רב עלה, ומה לי אי מוכחא מלתא דלתלות רמונים תלאה אי לאו מוכחא מלתא, הא מכל מקום לאו אהל הוא כלל.}
\textblock{\textbf{ומורי הרב ז״ל כתב בהלכותיו דכשהוא תולה לשמרים הוו כאהל לפי הוא צריך לחלל שתחתיה, אבל כשתולה אותם להניח עליה פירות מותר לפי שאינו צריך לחלל שתחתיה לדבר זה. ע״כ. ובודאי שאין משום אהל אלא בשצריך לחלל       } שתחתיה, והוא שיש לו שלש מחיצות. אלא שאני תמיה כיון שאינו אהל אלא כשצריך לחללו בשעת נטייתו, אפילו בתולה לשמרים אם אין כלי תחתיו להאהיל עליו לישתרי, דהוי ליה כההוא דאמרינן בביצה (לב, ב) בקדרה ומדורתא מלמעלה למטה שרי, וכההיא דאמרינן בפרק כירה (לעיל שבת מג, ב) מת המוטל בחמה באין שני בני אדם ויושבין בצדו, חם להם מלמטה זה מביא מטה וזה מביא מטה ויושבין עליהן, חם להם מלמעלה מביאים מחצלת ופורסין עליהם, זה זוקף מטתו ונשמט והולך לו וזה מביא וכו׳ ונמצאת מחיצה עשויה מאיליה, והכי נמי כיון שאינו אהל עד שיתן תחתיה כלי, כשהוא תולה את המשמרת ואח״כ נותן תחתיה כלי, הוי ליה אהל העשוי מלמעלה למטה ושרי. וא״ת משמרת אסרו לפי שדרכה בכך שתחלה תולין אותה ואח״כ נותנין כלי תחתיה, אם כן אע״פ שבשעה שתלאה לרמונים לא היה צריך לחלל שתחתיו, עכשיו שהוא תולה בה שמרים ונותן כלי תחתיה ליתסר. ואולי נאמר שלא אסרו אלא בשדעתו מתחלה לתת כלי תחתיה, אבל השתא דאין דעתו לתת בחלל שתחתיה כלי, נמצאת תלייתה בהיתר וכשנותן תחתיה כלי לבסוף אין עשייתו של אהל עכשיו. ואי נמי יש לומר דכל שהוא צריך לשים כלי תחתיו, מעכשיו הוא נקרא אהל שהרי צריך לאותו חלל כדי שישים תחתיו הכלי, אבל בתולה לרמונים אינו צריך לחלל כלל. ואפשר היה לי לומר כדברי רש״י, והא דמינקט אביי חומרי דמתני׳ ותני משמרת בהדי כילה, לאו משום דכולן חד טעמא אית להו, ואף רש״י סבור כן בכסא גליון שאינו משום אהל אלא משום גזירה שמא יתקע.\par \textbf{} והא דנקט לה בחומרתא חדא, משום דכולן כלים הן שגזרו חכמים בנטייתן, זה משום אהל, וזה משום עובדין דחול, וזה משום גזירה שמא יתקע. וכיוצא בזה יש לנו אחרת בפרק כל הצלמין (ע״ז מב, ב) דאמרינן התם מנקיט רב ששת חומרי מתני׳ ותני כל המזלות מותרין חוץ מחמה ולבנה, וכל הפרצופות מותר חוץ מפרצוף אדם, וכל הצורות מותרין חוץ מצורות דרקון, ואמרינן עלה כל המזלות וכו׳, במאי עסקינן אילימא בעושה כל המזלות מי שרי וכו׳ אלא פשיטא במוצא, אימא מציעתא כל הפרצופות מותרין חוץ מפרצוף אדם, ואי במוצא פרצוף אדם מי אסור, אלא פשיטא בעושה, אימא סיפא כל הצורות מותרים חוץ מצורת דרקון, אי בעושה צורת דרקון מי אסור, אלא פשיטא [במוצא וכו׳] רישא וסיפא במוצא מציעתא בעושה, אמר אביי אין רישא וסיפא במוצא מציעתא בעושה, רבא אמר כולה בעושה וכו׳, אלמא אביי לא קפיד למינקט כולהו בחד טעמא ממש, אלא כיון דשוו בחד צדדי התם דמנקט צורות שכולן אסורות ולא משום טעם אחד, וה״נ מינקט כלים שנטייתן אסורה ולא משום טעם אחד, כך נראה לי ליישב דברי רש״י ז״ל.}
\textblock{\textbf{הנוד והמשמרת.} אע״פ שאין בו משום אהל אלא בשצריך לחלל שתחתיו, נראה שהנוד היה צריך לפרסו ולהיות אויר תחתיו כדי שיכנס תחתיו הרוח ויצטנן.}
\textblock{\textbf{אבל מטה כסא וטרסקל ואסלא מותר לנטותן לכתחילה.} וכתבו בתוס׳ בענין בנין אוהלים דכללא דמלתא דלא אסרינן משום אהל אלא מידי דאית ליה מחיצות מלמטה ומשתמש באויר שתחתיו, אבל סדור השלחן שנותנים שתי ספסלים שיש להם רגלים דקים ועורכים עליו את השלחן, לא מיקרו מחיצה. ותמיה לי דגוד מי עביד ליה מחיצות גמורות רחבות, והלא אין פורסין אותו אלא על קונדסין, ופריסת השלחן אפשר שהוא מותרת מפני שאין משתמשין באויר שתחתיו, מה שאין כן במטה שמשתמשין בה באויר שתחתיה בנתינת סנדלין וכיוצא בהן.}
\textblock{\textbf{ומטה שלנו בזמן שהיא מסורגת בחבלים, אם יש בין חבל לחבל שלשה טפחים, אסור לפרוס עליהן סדין מפני שזה כעשיית אהל, וכן אסור לסלק מעליה הבגד התחתון מפני שהוא כסותר אוהלין, אבל אם אין בין חבל לחבל      } שלשה, הרי הן כלבוד ואין בו משום אהל, ובין כך ובין כך אילו היה כר או כסת או בגד פרוס עליה מערב שבת כשיעור טפח, למחר מותר לפרוס על המטה, משום דהוי כמוסיף על אהל עראי ושרי כרבנן וכעובדא דהנהו דכרי דהוו בי ר״ה דאיתא בעירובין (קב, א).}
\textblock{\textbf{איבעיא להו שימר מאי אמר רב הונא שימר חייב חטאת.} וכן הלכתא מדשקלי וטרי בה רבה ור׳ זירא בסמוך משום מאי מתרינן בה, ואע״ג דאתקיף רב ששת ואמר מי איכא מידי דרבנן מחייבי חטאת ור״א מתיר לכתחלה, הא אמר ר׳ יוסף דאין הכי נמי, ואייתי ראיה מעיר של זהב, ואע״ג דדחי לה אביי, דחויא בעלמא היא. ומיהו ר״א סבר דכיון דאינו בורר ולא מרקד בידים וממילא הוא משתמר, אינו כבורר ולא כמרקד ומותר אפילו לכתחלה. אבל מכל מקום תמיהא לי כיון דלרבנן חייב חטאת, הוי בורר גמור כדרכו, אם כן היאך הוו נותנין לתלויה ביום טוב, והרי לכולי עלמא אין בוררין את הקטנית ביום טוב בנפה ובכברה, וצריך לי עיון.}
\textblock{ הא ד\textbf{אתקיף רב ששת מי איכא מידי דרבנן מחייבי חטאת ור״א שרי לכתחלה.} איכא למידק והא תניא בפרק אלו קשרים (לעיל שבת קיג, א) חבל דלי שנפסק לא יהא קושרו אלא עונבו, רבי יהודה אומר כורך עליו פונדא או פסיקיא ובלבד שלא יענבנה, ואמרינן התם דטעמא דר״י משום דסבירא ליה דעניבה גופה קשירה הוא. ויש לומר דכל היכא דיחיד מחייב חטאת ורבים שרו לכתחלה לא קשיא ליה, אלא היכא דיחיד מחייב חטאת ויחיד אחר מתיר לכתחלה קשיא ליה, וכל שכן כשרבים מחייבי חטאת ויחיד מתיר לכתחלה.}
\textblock{\textbf{טלית כפולה לא יעשה ואם עשה פטור אבל אסור, היה כרוך עליה חוט או משיחה מותר לנטותה לכתחילה.} כתב הרי״ף ז״ל והוא שפירס ממנה מערב שבת על גבי הקינופות טפח, למחר הוי כמוסיף על אהל עראי, וכהני דכרי (עירובין קב, א) דהוה בי ר׳ הונא. אבל מורי הרב ז״ל כתב דאפילו בלא פריסת טפח, דאי בפורס טפח מאי שנא כרך עליה חוט או משיחה, תיפוק לי משום פריסת טפח.}
\textblock{\textbf{כילה מהו, אמר ליה אף מטה אסורה.} פירש ר״ת ז״ל: דרב סבר שהיה שואל אם יש לה צד איסור, ואחר כך שאל אם מטה אסורה לעולם, והשיב שאפילו כילה יש לה צד היתר, ואח״כ שאל אם מטה וכילה שוים, והשיב שאינן שוין.}
\textblock{\textbf{לא אמרן אלא דלא נחתא מפוריא טפח אבל נחתא מפוריא טפח אסור.} דכיון שהכילה פרוסה לצל הויא לה מטה כעשיית גג באמצע כילה, אבל מטה דידן אע״ג דנחתי סדינין מפוריא טפח לית לן בה, דכיון דפריסת סדינין על גבי מטה ואין בה משום אהל, לית לן בה. וכן כתב מורי הרב ז״ל.}
\textblock{\textbf{אלא לא קשיא הא דמיהדק הא דלא מיהדק.} פירוש: טעמא לאו משום אהל, אלא משום דלמא נפיל ואתי לאתויי. וכן פירש רש״י ז״ל, אבל רבינו חננאל ז״ל פירש לעולם משום אהל, וכשמיהדק הוי ליה כאהל, אבל גלימא דלא מיהדק לא הוי כאהל ושרי. ואינו מחוור. דאם כן הוי ליה למימר, לא קשיא הכא לא מיהדק התם מיהדק.}
\textblock{ הא ד\textbf{אמר רב לא שנו אלא בשני אדם אבל באדם אחד אסור.} פירש רש״י ז״ל: דבשני בני אדם אינו נמתח היטב, אבל באדם אחד שקושר על גבי קינוף זה וחוזר וקושר על גבי קינוף זה נמתח היטב. אבל מורי הרב ז״ל פירש משום דבאדם אחד הוי כעשיית אהל, שקושר וחוזר וקושר כדרך בנין שבונה מעט מעט, אבל בשני בני אדם שפורסין תחלה ואח״כ נותנין על גבי קנופות, אין זה כבנין שאין דרך בנין לעשותו כולו בבת אחת, והיינו דאמרינן בכילה דלא אפשר דלא ממתחא פורתא, ואינה נפרסת כולה בבת אחת. ולא ירדתי לסוף דעת מורי הרב ז״ל. דאם כן נתן דף על גבי מחיצות הכי נמי דשרי, וקדרה דאמרינן בביצה (לג, א) דמלמטה למעלה אסיר, אמאי והא נתינת הקדרה על גבי המחיצות בבת אחת הוא. ויש לפרש דהתם נמי דרך בנין בכך, כיון שהוא עושה גם מחיצותיה עכשיו.}
\clearpage
\newsection{דף קלט}
\textblock{ הא דאמרינן:\textbf{ ולישלח להו כדרמי בר יחזקאל.} ולא אמרינן ולישלח להו כילת חתנים. לא קשיא דאינהו בשיש לה גג קא מיבעיא להו. והא נמי דאמרינן ולשלח להו כר׳ טרפון, ולא אמרינן ולישלח להו כר׳ יאשיה דאמר אינו חייב עד שיזרע חטה ושעורה וחרצן במפולת יד, וקיימא לן כותיה כדאמרינן נהוג עלמא כתלתא סבי כר׳ יאשיה בכלאים כדאיתא בברכות פרק מי שמתו (ברכות כב, א) ובפרק קמא דקידושין (לט, א) לא צהריתו דקיימא לן כר׳ יאשיה. אינהו בזורע חטה וכשות וחרצן קא מבעיא להו, דמספקא להו אי מין אילן הוא או ירק הוא.}
\textblock{ הא דאמרן:\textbf{ וליתן ליה לתינוק ישראל.} פירשתיה ביבמות פרק חרש (קיד, א) גבי קטן אוכל נבלות אין בית דין מצווין להפרישו, בסייעתא דשמיא.}
\textblock{ הא דאמרינן הכא:\textbf{ אתי למיסרך.} ואמרינן נמי הכי בערובין (מ, ב) גבי זמן דיום הכפורים ליטעמיה לינוקא, אתי למיסרך.      משום דהוה עובדין דצריכין לדידן, ובהני הוא דוקא דחיישינן לדלמא אתי למיסרך, אבל במידי דצריך לתינוק לא חיישינן להכי, דהא התם לא מטעמיה ליה לינוקא, ואילו לדידיה מאכילין ומשקין לאלתר ואפילו טובא, וכבר כתבתי דבר זה והארכתי בדבר זה, בפרק חרש ביבמות (שם) בסייעתא דשמיא.}
\textblock{ הא דאמרינן:\textbf{ מת ביום טוב ראשון יתעסקו בו עממין.} פירש״י ז״ל: דוקא בדאשתהי, כגון דמת בשבת והיה יום טוב לאחר השבת, אי נמי כשיהא שבת מלאחריה דחיישינן דלא ישתהה, וראיתו מדאמר הכא במעשה דמעון ביום טוב הסמוך לשבת ולא ידענא אי מלפניה אי מלאחריה, דאלמא טעמא משום הא הוא דשרי להו רבי יוחנן. ולפי דברי הרב ז״ל הא דאמר מר זוטרא בפרק קמא דביצה (ו, א) לא אמרן אלא דאשתהי ורב אשי אמר אע״ג דלא אשתהי בי״ט שני הוא וכדמפרש רב אשי טעמא ואזיל, מאי טעמא דיום טוב שני לגבי מת כחול שויוה רבנן.\par \textbf{} וכבר הסכימו הגאונים ז״ל, דאפילו ביום טוב ראשון נמי אע״ג דלא אשתהי לא משהינן ליה, ורב אשי ומר זוטרא לא איפליגו אלא ביום טוב שני משום דשבות דאית ביה מלאכה ביד ישראל היא, ולזה הוא שהוצרך רב אשי לתת טעם, אבל ביום טוב ראשון דעל ידי עממין לכולי עלמא אע״ג דלא אשתהי לא משהינן ליה, ומעשה דמעון משום דשבת מלפניה או מאחריה הוא דשאלו הם, אבל רבי יוחנן התיר להם לגמרי. ונראה לי קצת ראיה מדאמרינן סתם, ופריך מהא דרבי יהודה בר שילא, ואם איתא מאי קושיא דלמא בני בשכר לאו בדאשתהי הוה, דהא סתם שאלו וסתם אסר להם. ומ״מ יש לדחות דמשום דאסר להם סתם לומר דלעולם אסור ואפילו בדאשתהי, לפיכך הקשו עליו למה לא התיר בדאשתהי מיהא כעובדא דמעון.\par \textbf{} ומכל מקום כך נהגו בכל המקומות להתעסק בו עממין ביום ראשון, וביום שני אפילו ישראל, אע״ג דלא אשתהו ואפילו לחצוב לו קבר. והא דאמרינן התם בפרק קמא דביצה (ו, א) אפילו למיגז ליה אסא ואפילו למיגד ליה גלימא, לאו דוקא הני ומשום דטרחתן מועט וכמו שפרש״י ז״ל במס׳ ביצה (שם), אלא אפילו הני קאמר, כלומר אע״פ שאפשר לו למת בלא הני, דהא אפשר בלא אסא ואפשר נמי לכרכו בתכריכיו, וכל שכן חפירת קבר שאי אפשר למת בלא כן.\par \textbf{} והא דתנן במסכת מועד קטן (ח, ב) אין חופרין כוכין וקברות במועד. כבר תירץ הרב אלפסי ז״ל, דהתם בחופר במועד לקבור בו מתים שימותו, וכן היה דרכן לחפור ולהכין כוכין קודם מעשה. ומה שהקשה עליו הראב״ד ז״ל א״כ למה התירו להרחיב ולהאריך במועד כיון שאין בו צורך המועד. ותירץ הרמב״ן ז״ל לפי שכל דבר שיש בו צורך לרבים התירו לתקנו, אע״פ שאסור להתחיל בו, וכמו ששנינו שם ומתקנין את קלקולי המים שברשות הרבים וחוטטין אותן, ואוקימנא כשאין הרבים צריכין להם, ואעפ״כ כיון שצרכי רבים הן מותר לתקן אבל לא להתחיל, והכא נמי צרכי רבים הוא. ואפילו בקבר בני משפחה ואפילו של יחיד נמי אפשר שהוא מותר, דכיון דמצוה הוא התירו בו מקצת מלאכה אפילו שלא לצורך המועד כמו שהתירו בצרכי רבים, וכל שכן שמא יצטרך לו במועד. וא״ת עוד והא תנן התם וארון עם המת בחצר ר״י אוסר אלא אם כן היו לו נסרים המנוסרים מערב יום טוב, ואם כן האיך נהגו עכשיו לחצוב קבר שלא בפני המת. יש לומר דההיא באדם שאינו מפורסם, הא באדם מפורסם מותר. והכין איתא בירושלמי. וכתב הרב אלפסי ז״ל בהלכות, וכיון שאין חופרין עכשיו אלא לצורך שעה הכל כאדם מפורסם, והכל יודעים דלצורך מת דביומו הוא דשרי.\par \textbf{} והא דאמרינן יתעסקו בו עממין. יש מי שאומר דדוקא בקבורתו, אבל טלטולו מותר על ידי ככר או תינוק, והוצאתו נמי מותרת כב״ה, דאמרינן מתוך שהותרה הוצאה לצורך הותרה נמי שלא לצורך. ויש מי שאוסר להוציאו, וכן מוכיח להם הלשון שאמרו יתעסקו בו עממין ולא אמרו יקברוהו וכו׳, דאלמא כל עסקיו על ידי עממין. וטעמא דכיון דאי אפשר לקבורה בישראל לא התירו הוצאה בישראל, שאם תתיר בהם מקצת מלאכה שמא יבואו לגמור מלאכה. ועוד שהן כעוסקין בקבורה עצמה ומסייעין בה שהוא חלול יום טוב לגמרי, והא דמיא לההוא דאקשינן בפרק כל הכלים (לעיל שבת קכד, ב) ובפרק קמא דביצה (עי״ש יב, א) אטו טלטול לאו צורך הוצאה הוא, ועוד שאין כאן משום כבודו של מת הואיל ובסוף יתעסקו בו עממין.}
\textblock{\textbf{ועוד כתב הרמב״ן ז״ל דהוצאת המת לקבורה כהוצאת אבנים לבנין, שאין לומר בו מתוך משום דלא שייך ביה צורך היום. ואם תאמר שהוא מצוה, הרי שריפת קדשים עשה ואינו דוחה יום טוב וכל שכן זה. ועוד דהא אתמר בפרק קמא      } דכתובות (ז, א) שאין מתוך אלא בהנאה השוה לכל נפש, והרי אין כאן הנאת נפש. ואין לומר מפני שאין (הנאה) [הוצאה] זו צריכה לגופה, אם כן בשבת יתירו, וכללו של דבר ששבת ויום טוב שוין בדבר זה, והמתיר בזה יתיר בזה. עד כאן.}
\textblock{ הא ד\textbf{אמר רב אשי והוא שתלה בה רמונים.} ואקשינן עלה ממערים ושותה מן החדש. איכא למידק אדרבא התם נמי שותה קאמרינן, אבל אומר לשתות ואינו שותה לא התירו. ותירץ הרמב״ן ז״ל דהכי קאמר והוא שתלה בה רמונים תחלה, הא תלה בה שמרים תחלה אף על פי שתלה בה רמונים לבסוף אסור, ואקשינן עלה ממערים ושותה מן החדש, אף על פי שמתחלה הוא טורח הוכיח סופו על תחלתו. ומתרץ התם לאו מוכחא מלתא דלחול עביד כיון דשתי לבסוף, אבל הכא מוכחא מלתא דלשמר תלה כיון דתלה בה שמרים תחלה, ומה שתולה בה רמונים לבסוף אמרינן הואיל ועשה בה מלאכתו משתמש הוא לשאר צרכיו.\par \textbf{} ואיני יודע טעם לקושיא זו. דודאי הא דאמר רב אשי והוא שתלה בה רמונים, שתלה תחלה קאמר, ולפיכך הקשה לו ממערים ושותה ואף על פי שיש לו מן הישן, ומוכיח דחדש לאחר המועד הוא דבעי ליה, ואפילו הכי כיון דשותה ממנו אמרינן דהוכיח סופו על תחלתו, והכא נמי אף על פי שתלה תחלה שמרים, כשתלה לבסוף רמונים הוכיח סופו על תחלתו, ופריק התם לא מוכחא מלתא, שאף על פי שיש לו ישן אין הכל יודעין בו שיש לו ישן ועוד דהרבה יש רוצים בחדש מן הישן, הכא מוכחא מלתא ואף על פי שיש לו שמרים אסור תחלה. וכן נראה לי מדברי רש״י ז״ל.}
\textblock{ [מתני׳:]\textbf{ ומסננין את היין בסודרין.} פירוש: יין עכור, דאי אפשר לפרש שמרים דהא יין קתני ואע״פ שמשנה כיון שהוא בורר גמור אסור, וכן אי אפשר לומר יין צלול, דאם כן אפילו במשמרת כזעירי, אלא על כרחין ביין עכור דכיון דאפשר למשתייה הכין אין כאן משום בורר, וכיון דמשנה קצת ומסננין בסודרין שרי.}
\textblock{ [מתני׳]: \textbf{ונותנים ביצה במסננת של חרדל.} פירש רש״י ז״ל: משום רבותיו ז״ל דנותנים ביצה כו׳ דוקא משום שינוי, ויורד לתוך הקערה שהוא כלי שני ומתלבן התבשיל. ואינו מחוור בעיני. שאילו דוקא במסננת של חרדל ומשום שינוי, אם כן מאי פריך אביי בפרק המילה (שבת קלד, א) מהא דתניא אין מסננין את החרדל במסננת שלה ואין ממתקין אותו בגחלת אמתניתין דהכא דתנן נותנין ביצה במסננת של חרדל, ופריק התם לא מחזי כבורר הכא מחזי כבורר. ואם איתא מאי קושיא התם במסננת שלו ומשום הכי אסור, והכא במסננת של חרדל דאיכא שינוי ומשום הכי מותר. אלא נראה כפירוש הראשון שפירש רש״י ז״ל, שנותנין אותה במסננת שהחרדל נתון בה לסנן, והחלמון שלה נוטף ומסנן עצמו והוי לחרדל למראה, והחלבון שהוא קשור נשאר למעלה, והיינו דאמרינן התם דלא מחזי כבורר, דאין כאן פסולת שהכל ראוי הוא, אלא שהוא רוצה לערב החלמון לבד עם החרדל לגוון.}
\textblock{ גמרא: הא ד\textbf{אמר רב חייא בר אשי אמר רב ובלבד שלא יגביה מקרקעיתו של כלי טפח.} משום שינוי קאמר הכי ולא משום אהל, דאם כן אפילו במסנן את היין בסודרין נימא הכין, אלא משום היכר, וכיון דהתם איכא היכר אחר דאינו עושה גומא שפיר דמי. מורי הרב ז״ל.}
\textblock{ הא ד\textbf{אמר רבא האי פרונקא אפלגא דכובא שרי אכולה כובא אסור.} פירש רש״י ז״ל: משום אהל. וא״ת וכי בכיסוי כלים יש משום אהל. תירצו בתוס׳ דשאני כובא שהוא רחב הרבה. ואינו מחוור בעיני, דאם כן הוה להו לפרושי כמה יהא רחב שיהא אסור וכמה קצר ויהא מותר, ונתת דבריך לשיעורין, ואפילו כיסוי קרקעות שהוא נראה יותר כאהל התירו (לעיל שבת קכו, א), ולא נתנו חכמים בו שיעור. ועוד אי משום אהל מאי שנא כולה מאי שנא פלגא, והא עובדא דרב הונא (עירובין קב, א) דהנהו דכרי משום תוספת אהל עראי התירו, אבל אילו רצה לפרוס על חצין של מחיצות היה אסור.}
\textblock{\textbf{ויותר נראה דברי הראב״ד ז״ל שפירש משום משמר, ואכולה כובא אסור משום דמחזי כמשמרת שכן דרכו בחול,       } אבל פלגא דכובא שרי דלאו היינו אורחיה, וכי הא מלתא אמר רב פפא בסמוך לא ניהדוק איניש סכתא אפומא דכובא משום דמחזי כמשמרת, וכיון דלא מהדק הוי ליה כפלגא דכובא ושרי. ע״כ. ולדברי רש״י ז״ל נצטרך לפרש שלא יהא הכובא מלאה שאם כן אין כאן אהל אלא בשחסר טפח דהוי אהל.}
\clearpage
\newsection{דף קמ}
\textblock{\textbf{התם מיחזי כי אולודי חיורא.} פירש רש״י ז״ל: שהסודר אסור דמחזי כאולודי חיורא שאדם מקפיד בו יותר. והקשה הרמב״ן ז״ל דאם איתא דרב הונא לאיסורא פשיט ליה, אם כן היכא אמר ליה רב נחמן ותפשוט ליה למר מסודרא, אדרבה כיון דרב חסדא פשיט ליה בכיתנא להתירא ורב הונא פשיט ליה בסודרא לאיסורא, ורב נחמן הוה סבירא ליה דחד דינא אית להו. לא הוה ליה למימר ותפשוט לה דאדרבה מקשא קשו פשטי אהדדי. על כן פירש הוא ז״ל דרב הונא פשיט ליה אף בסודרא להתירא, ואפילו הכי קא סלקא דעתך דכיתנא אסיר ופשט ליה רב חסדא להתירא. וכן נראה מדברי הרב אלפסי ז״ל שלא כתב להא דסודר, דאלמא תרווייהו להיתרא. ואינה קושיא לפי דעתי. דמשום דרב הונא רביה דרב חסדא אמר ליה ותפשוט ליה למר מסודרא כלומר לאיסורא, אע״ג דאמר לו רב חסדא דשרי.\par \textbf{} אלא הא קשיא לי דכיון דפשט ליה רב חסדא להתירא, אם איתא דרב הונא פשיט ליה בסודרא לאיסורא, כי אמר ליה רב נחמן ולבעי ליה למר סודרא היכי סתים ואהדר ליה האי בעאי מיניה מרב הונא ופשט לי, דמשמע פשט לי ביה כדפשט ליה רב חסדא בכיתנא, והוה ליה למימר ופשט לי לאיסורא. אלא שמצאתי במקצת ספרים שכתוב בהן כך ופשט לי לאיסורה, ואין רוב הספרים מסכימין לגירסא זו.}
\textblock{\textbf{תלאי דבשרא.} פרש״י ז״ל בשר מליח התלוי ליבשו, ומלא החבל קרוי תלאי דבשרא ושרי דנאכל חי באומצא, דכוורי אסור דאין נאכלין חיין. והקשו עליו בתוספות דאם כן לימא בשר מליח ולמה ליה למימר תלאי דבשרא. ועוד דאם כן בפרק מפנין (שבת קכח, א) הוה ליה לאתויי, דמייתי התם בשר תפוח בשר מליח ודג מליח ותפל. על כן פירשו בתוספות דתלאי היינו העץ שתולין בו בשר, כלומר דתורת כלי עליו דרגילות הוא להצניעו, אבל דכוורי לא אלא כל שנשתמש בו זורקו לבין העצים.}
\textblock{\textbf{אבל באבוס של קרקע דברי הכל אסור.} דגזרינן דלמא אתי לאשויי גומות ביד.}
\clearpage
\newsection{דף קמא}
\textblock{\textbf{הני פלפלי מידק חדא חדא בקתא דסכינא שרי.} יש מי שפירשה ביום טוב, והרב אלפסי פירש, בשבת. וכן הכריעו בתוספות, דאי ביום טוב מאי שנא מתבלין שנדוכין כדרכן ואפילו מלח נידוך בהצלאה בכל דבר (ביצה יד, א), ותנן נמי (ביצה כג, א) אין שוחקין פלפלין בריחים שלהן, דאלמא דוקא בריחים הוא דאסור משום דהוי טוחן הא כדרכן מותר. ובהדיא תנו בתוספתא (פט״ו, הי״ג) אין כותשין את המלח במדוך של עץ אבל מרסק הוא ביד של סכין ובעץ הפרור ואינו חושש, אין מרסקין דבילה וגרוגרות ואת החרובים לפני הזקנים אבל מרסק הוא ביד הסכין ובעץ הפרור ואינו חושש, אלמא אפילו בשבת בשינוי שרי, והכא לא אסר רבי יהודה אלא בפלפלי דצריכי למידק טפי, ואפילו הכי הלכתא כרבא דשרי ואפילו טובא נמי.}
\textblock{\textbf{אלא אמר רבא מקנחו בכותל ואין מקנחו בקרקע דלמא אתי לאשוויי גומות.} פירוש: לפי שאדם מחזיר למקום גומא לקנח בו רגליו, כי במישור אינו מקנחו היטב, ומתוך שהוא מקנחו שם שמא ישכח ויכוון להשוות את הגומא. אבל אין לפרשה משום דדבר שאין מתכוון אסור, דהא רבא כר״ש סבירא ליה כדאיתא בשלהי פ״ב דביצה (כג, א) וכן הא דאמר רבא בסמוך לא לצדד איניש כובא אארעא דלמא אתי לאשוויי גומות, מהאי טעמא נמי הוא לפי שהוא צריך למקום שוה שישב הכובא היטב, חוששין שמא ישוה הגומות ביד או בצדוד הכובא בכונה, וכמו שאמרו שם גבי אבוס דבשל קרקע לכולי עלמא אסור מהאי טעמא, וכמו שגם כן אסרו בעירובין (קד, א) נשים המשחקות באגוזים ובתפוחים דאתו לאשוויי גומות, כלומר במתכוון.}
\textblock{\textbf{והרב אלפסי ז״ל פסק בהא כרבא, דאסור לקנח על גבי קרקע. אבל הר״ז הלוי ז״ל פסק כרב פפא משום דבתרא הוא. והרמב״ן ז״ל השיג עליו דמכל מקום רבא רביה דרב פפא הוא, ואין הלכה כתלמיד במקום הרב. ולא ירדתי לסוף      } ראיתו. דהא בבתראי לא אמר הכי, דהא רב נחמן רביה דרבא הוא ואפילו כן הלכה כרבא כשלא היה יושב רבא לפני רב נחמן, וכל מקום דאיכא אמר ליה רבא לרב נחמן בלחוד חשבינן ליה תלמיד יושב לפני רבו ואין הלכה כמותו, אבל כשנחלקו שניהם כשני חולקים בעלמא הלכה כרבא, והכי נמי רב פפא הוא דנחלק על רבא, והלכתא כותיה דבתרא הוא.}
\textblock{ הא דתניא:\textbf{ לא תצא אשה במנעל המרופט ולא תחלוץ בו ואם חלצה חליצתה כשרה.} דוקא בשאינו מרופט ברובו, דאילו כן חליצתה פסולה. ותדע לך מדאמרינן ריש פרק מצות חליצה (יבמות קב, א) דאין חולצין במנעל מאי טעמא גזירה משום מנעל המרופט, ואם איתא דבכל מנעל מרופט חליצתה כשרה בדיעבד, היכי גזרינן ביה, אלא כדאמרן.}
\textblock{ מתני׳:\textbf{ נוטל אדם את בנו והאבן בידו.} בגמרא מוקי לה בתינוק שיש לו געגועין על אביו, וכלכלה דלא שייך (בם) [בה] האי טעמא, היינו משום שהוא צריך לפירות שבתוכה.}
\textblock{ גמרא:\textbf{ רבא כרבי נתן סבירא ליה דאמר חי נושא את עצמו.} איכא למידק דהא רבא גופיה דאמר לעיל פרק המצניע (שבת צד, א) דבאדם אפילו רבנן מודו ליה לרבי נתן, ולא נחלקו אלא בבהמה חיה ועוף. ויש לומר דהכא בתינוק קטן דכיון דמשרביט נפשיה דינו כבהמה. ואכתי איכא למידק מדאמרינן בפרק מפנין (שבת קכח, ב) האשה מדדה את בנה ברשות הרבים, ולא גזרינן דלמא אתי לטלטוליה משום דאדם נושא את עצמו, ולא פליגי רבנן עליה וכדפירש רש״י התם, דאלמא אפילו בקטן מודו רבנן. ויש לומר דהתם בתינוק הנוטל רגלו אחת ומניח אחרת דכיון דמכיר בהלוך לא משרביט נפשיה, אבל בגורר שאינו מגביה רגלו כלל משרביט נפשיה כבהמה ופלוגתא היא דר׳ נתן ורבנן, ומיהו בגורר אפילו ר׳ נתן מודה דלכתחלה אסור דומיא דבהמה, דלא פליג אלא בחיוב חטאת, ומיהו בקטן גמור שאינו מכיר בהלוך כלל וכלל כגון קטן בן שמונה או בן חודש וכיוצא בזה, אפשר דאפילו רבי נתן מודה דחייבין עליו חטאת, משום דאין לך כפות גדול מזה ומודה רבי נתן בכפות כדאמרינן לעיל בפרק המצניע (לעיל צד, א), וכן כתבתי למעלה בריש פרק ר׳ אליעזר דמילה (לעיל שבת קל, א).}
\textblock{ הא דאמרינן:\textbf{ כיס לגבי תינוק לא מבטל ליה.} לאו משום חשיבותיה דכיס קאמר, דאם איתא מאי קא פריך מנוטל אדם את בנו והאבן בידו, דאבן לא חשיבא ומבטל לגבי תינוק, ודבי רב ינאי למאי איצטריכו לדחוקה ולאוקומיה בתינוק שיש לו געגועין על אביו ודינר (לאו) משום דאי נפיל אתי לאתויי, לימא משום דדינר חשוב ולא מבטל ליה ואבן לא חשיבא ומבטל לה, אלא טעמא דמלתא משום דקסבר דכל דאינו צורך תשמישו אינו בטל לגמרי לגביה.}
\textblock{\textbf{אי הכי מאי איריא אבן אפילו דינר נמי.} פירשו בתוס׳: הכי קאמר אי אמרת בשלמא דבין כיס בין אבן בין דינר בטל לגבי תינוק וליכא חיוב חטאת, היינו דנוטל את בנו והאבן בידו ואפילו בשאין לו געגועין עליו, וטעמא דדינר משום דחיישינן דלמא נפל ואתי לאתויי, דהאי טעמא השתא נמי ידעינן לה, ומכל מקום חיוב חטאת ליכא. אלא אי אמרת דלא מבטל ליה לגבי תינוק וחיוב חטאת נמי איכא כאילו האב נוטל בידו, וטעמא דאבן משום דיש לו געגועין עליו, אי הכי אפילו דינר נמי, וקא סלקא דעתך דמקשה השתא דכיון דמשום געגועין שרית ליה אפילו מידי דאית ביה בר״ה חיוב חטאת לא חיישינן לדילמא נפיל ואתי לאתויי. ומשני דינר היינו טעמא משום גזירה דלמא נפיל, דכיון דאפשר דאתיא לידי טלטול בידים אפילו במקום געגועין לא שרו ליה. ולהאי פירושא אפילו להוליכו בידו בשיש בידו דינר אסור מהאי גזירה. וכן פירש רש״י.\par \textbf{} והרמב״ן ז״ל הקשה אם כן נאסור לעמוד בד׳ אמותיו. ואין קושיתו מחוורת בעיני דשמירתו ושמירת דינר על אביו. והוא ז״ל פירש אי אמרת בשלמא בתינוק שאין לו געגועין, היינו דשרי ליה אבן משום דמבטל לה לגבי תינוק, וכיס נמי מבטל ליה, וטעמא דדינר משום דכיון דאין דרכן של בני אדם למסור דינר לתינוק בלא כיס, אינו אלא כמוסרו לאביו שהוא נוטלו, והוי ליה כמאן דנקט ליה אב בידיה, אלא אי אמרת דכל שאינו לצורך תשמישו לא מבטל ליה והכא משום געגועין הוא דשרי, אפילו דינר נמי. ומפרקינן אבן אי נפלה לא אתי לאתויי, דינר אי נפל אתי לאתויי, דכיון שאף טלטולו ביד התינוק כטלטול דידיה דמי, אתי לאחלופי בטלטול בידים אי נפיל ליה, ולא רצו להתיר אותו טלטול קטן שהוא אינו מטלטל בעצמו משום געגועין, כדי שלא יבא לידי טלטול גדול. ומיהו כל היכא דלא מטלטל ליה לתינוק שרי, דלא גזרינן משום דלמא נפיל ואתי לאתויי, דכל היכא דלא עביד איהו מעשה ולא שייך בטלטול זה לא גזרינן, דמי גזרינן שמא יראה אדם תינוק ומטלטל דינר ולא יעמוד לו בקרוב ד׳ אמות, והאוחזו נמי לאו מידי עביד ביה.}
\newchap{פרק \hebrewnumeral{21} נוטל אדם את בנו}
\clearpage
\newsection{דף קמב}
\textblock{}
\textblock{ האי דאקשינן:\textbf{ למה לא תהוי הכלכלה בסיס לדבר האסור.} קשיא לי היכי מקשה הכא להדיא, דדלמא בשוכח הוא, דכולה מתניתין בשוכח דתנן (לקמן עמוד ב) האבן שעל פי החבית מטה על צדה ותנן מעות שעל הכר נוער את הכר, ואוקימנא בגמרא דוקא בשוכח אבל במניח נעשה בסיס לדבר האסור. ויש לומר דעל כרחנו משנתינו קשיא, דהא אפילו נוקמה בשוכח, תקשי לן אמאי נוטלה והאבן בתוכה, ינער את הכלכלה והיא נופלת, וכדתנן באבן שעל פי החבית ובמעות שעל גבי הכר, וכדאקשינן נמי לבתר דאוקמה ר׳ יוחנן במלאה פירות, וכיון שכן לא דק לאקשויי אי בשוכח לנערה ואי במניח נעשה בסיס. ומיהו לא אסיק אדעתיה לאוקומה בצריך למקומה, משום דכולה מתניתין לאו בצריך למקומה היא, דאם כן מטלטל הוא אפילו את החבית ואת הכר בעודן עליהן וכדאיתא בגמרא, ועוד דדומיא דנוטל אדם את בנו קתני לה דלגעגעו לבן ולא לצורך מקומו של תינוק. כך נראה לי.}
\textblock{\textbf{אמר רבה בר בר חנה אמר ר׳ יוחנן הכא בכלכלה מלאה פירות עסקינן.} כלומר: דהוי ליה בסיס לדבר המותר ולדבר האסור דשרי, ואקשינן ולישדינהו לפירי ולישדינהו לאבן. קשיא לי היכי דמי, אי באבן שאינה חשובה למה ליה דשדי לה דהא בטילה ולא חשיבא, וכדאמרינן גבי כנונא בשלהי פרק כירה (שבת מז, א) מטלטלין כנונא אגב קיטמא ואע״ג דאיכא שברי עצים משום דלא חשיבי. ואי באבן חשובה אע״ג דמלאה פירות תיתסר, וכדאמרינן גבי מוכני (מד, ב) בשיש עליה מעות שאסור לטלטלה אע״ג דהוי בסיס לדבר המותר ולדבר האסור, וכן נמי בתיק הספר עם הספר אע״ג שיש בתוכו מעות, דדוקא משום הצלת כתבי הקודש התירו כדאיתא פרק כל כתבי הקדש (קטז, ב). ויש לומר דמעות חשיבי ולא בטילי, ולעולם נעשה כלי בסיס להן אף על פי שהן מעורבין עם כלים, אבל אבן כיון שיש עמה פירות אין הכלי נעשה בסיס לה אלא לפירות, ומיהו כל היכא דמצי לנערה מנערה דחשיבא קצת. ועוד דיש לומר דאפילו דלא חשיבא כל היכא דמצי לנערה מנערה, וכנונא לא אפשר לנער שברי עצים שבתוכה, כדי שלא יתפזר האפר שבתוכה.}
\textblock{\textbf{הכא בכלכלה פחותה עסקינן.} ואיכא למידק, ולישקול להו לפירי ולישבוק לה לכלכלה וכדאמרינן בסמוך גבי תרומה טהורה, לא שנו אלא כשהטהורה למטה וטמאה למעלה, אבל טהורה למעלה שקיל לה לטהורה ושביק לה לטמאה. ויש לומר דהתם כשתרומה טהורה בכלי אחד וטמאה בכלי אחר ושניהם בתוך כלי גדול, דהא אפשר ליה דלשקול טהורה בכלי אלא שהוא כלי פתוח דאי מנער לה נפלה, אבל הכא אי שביק לכלכלה לית ליה כלי דליטלטלה בגויה. כך תירצו בתוס׳.}
\textblock{\textbf{מאן תנא דכל היכא דאיכא איסורא והתירא בהתירא טרחינן באיסורא לא טרחינן.} פירוש: לפי שכיון שהאבן מונחת על פי החבית שיש בה יין שהוא צריך לו ולא נעשית חבית בסיס לאבן, אלא דין הוא שתנטל האבן, והיה ראוי שתעשה אבן זו כפסולת שבאוכלין, שהוא עפרורית ופסולת וטנופת ואעפ״כ ניטלין בפני עצמן, הכא נמי תנטל בעצמה כדי שיהיו המשקין שבחבית מתוקנים לאכילה, דמאי שנא אבן שעל גבה מאבן שבתוך הפירות עצמן דנטלת כפסולת שבאוכלין, אלא מפני שפסולת מרובה אין טורחין אלא בהיתר, דהא צריך למישקלה לחבית כדמפרש ואזיל, בנמוקי הרמב״ן ז״ל.}
\textblock{\textbf{טעמא דחזי לאומצא הא לא חזי לאומצא לא אלמא רבא כר״י סבר ליה.} איכא למידק דהא מידי דחזי ליה לאיניש לא מקצה ליה לכלבים, וכיון דהשתא לא חזי ליה עד מוצאי שבת מודה ביה ר״ש דאסור לטלטלו, וכדאמרינן לעיל בפרק מפנין (שבת קכח, א) דג תפל אסור לטלטלו, ואם כן אי לא חזי לאומצא אפילו ר״ש מודה ביה. ויש לומר דהכי קאמר טעמא דחזי לאומצא, הא לא חזי לאומצא אלא לכלבים כגון בשר תפוח שתפח בו ביום לא מטלטלי׳ ליה והא ודאי כר״י.}
\textblock{\textbf{והא אמר ליה רבא לשמעיא טוי לי בר אווזא ושדי מעיה לשונרא.} פירש רש״י ז״ל: שנשחטה ביום טוב ובני מעיה לא חזי ביום טוב לאכילה דאין דרך לאכול בני מעים ביום טוב, ומשני כיון דמסרחן כלומר אי מצנע להו עד לערב מאתמול דעתיה עלויה. ואינו מחוור, דאם כן שאע״פ שראוין למאכל אדם כיון שאין דרכו לאכול אותן ביום טוב אסור לטלטלן, אם כן תקשי ליה מדידיה דאמר אי לאו דאדם חשוב אנא סכינא אבר יונה למאי והא חזי לאומצא, ואי איתא אע״ג דחזי לאומצא מכל מקום הא לא עבידי אינשי דאכלי בר יונה חי בשבת, אלא מענגו בבשר מבושל ומבושם. ועוד דהא שרינן לטלטל בשר חי ומליח, אע״פ שאין דרכן של בני אדם לאוכלן כך בשבת. ועוד קשה לדברי רש״י ז״ל שכתב דאיירי בשנשחטה ביום טוב דאע״ג דדעתיה עלויה מאתמול מכל מקום נולד הוא לר׳ יהודה, דומיא דגרעיני תמרה וכדתניא בפרק במה מדליקין (שבת כט, א) מסיקין בתמרים, אכלן אין מסיקין בגרעיניהן, ואע״ג דמאתמול ידע דבעי למיכלינהו וגרעיניהן לא חזי ליה ודעתיה עלייהו. ולפיכך פירשו בתוס׳ שם בפרק במה מדליקין דהכא בשנשחטה מערב יום טוב (הוא דמסרחו), ואע״ג דבערב יום טוב חזי מעיה לאדם והשתא הוא דאסרחו, כיון דידע דמסרחי דעתיה עלייהו מאתמול למישדינהו לשונרא.}
\textblock{\textbf{ואם תאמר מכל מקום תקשי מהא דתמרים שאפילו מאמש לא היו ראוין לאדם ודעתו עליהן להסקה, ואפילו הכי כיון דמעיקרא מיכסו והשתא מיגלו חשבינן להו נולד והכי       } נמי לא שנא. יש לומר דהתם כיון שהאוכל צריך לגרעין הרי הם כאוכל עצמו, ואע״פ שדעתו עליהן להסיקן תחת תבשילו כשיאכלם הוו להו כנולד, אבל בני מעים אחר שנשחט האווז אין האווז צריך אליהם, וכיון שדעתו להשליכם לכלבים כשיסריחו הוה לה הכנה מעלייתא.}
\clearpage
\newsection{דף קמג}
\textblock{ [מתני׳] הא דתנן:\textbf{ מגביהין מעל השלחן עצמות וקליפין.} פירש רש״י ז״ל: עצמות וקליפין שאינם ראוים למאכל בהמה, והקשו עליו בתוספות דהא תנן במתניתין שער של פולין של עדשים מפני שהוא מאכל של בהמה, ואוקימנא לה בגמרא כר״ש, אלמא הא לאו הכי לכולי עלמא אסור ואפילו לר״ש דלית ליה תורת כלי .}
\textblock{ והא דתנן:\textbf{ מסלק את הטבלא כולה ומנערה.} אוקימנא בגמרא כר׳ יהודה, ואע״ג דאית ליה לר׳ יהודה מוקצה ונולד הכא בצריך למקומה של טבלא. ואי נמי דהוי ליה כגרף של רעי. ואינו נאסר מתורת בסיס ואע״פ שהוא מניח מדעת על גבי הטבלא, דכיון דבאמצע שבת הוא ודעתו לסלקן לא הוי בסיס. וכבר כתבתיה בארוכה בריש פרק קמא דביצה בסייעתא דשמיא.}
\textblock{ הכי גריס רש״י ז״ל:\textbf{ מסייע ליה לר׳ יוחנן דאמר פירורין שאין בהן כזית אסור לאבדן ביד.} ואינו מחוור, דמאי סיעתא, דאי משום דקתני מעבירין ולא קתני זורקין, הא נמי תנן (קמג, א) מגביהין מעל השלחן עצמות וקליפין ולא קתני זורקין. ור״ח ז״ל גריס מותר לאבדן ביד, וכן בתוס׳, והא דקא דייק כן הוא מדקתני מפני שהוא מאכל בהמה, דמשמע דלאדם לא חזי, והלכך מותר לאבדן ביד כשאר מאכל בהמה. והכי נמי משמע בברכות פרק אלו דברים, דאמרינן התם (נב, ב): ובית הלל בשמש תלמיד חכם שנוטל פירורין שיש בהן כזית ומניח פירורין שאין בהן כזית, מסייע ליה לר׳ יוחנן דאמר פירורין שאין בהן כזית מותר לאבדן ביד.}
\textblock{ [גמרא] הא דאמרינן:\textbf{ הני גרעינין דתמרי ארמייתא שרי לטלטולינהו הואיל וחזיין אגב אימן ודפרסייתא אסיר.} פירשו בתוספות דלרבי יהודה קאמר, אבל לרבי שמעון דלית ליה נולד אפילו בפרסייאתא שריין דהא חזיין למאכל בהמה, ושמואל דמטלטל להו אגב ריפתא, אף על פי דאית ליה כר׳ שמעון כדאיתא בשלהי מכלתין (קנז, א), אחמורי הוא דמחמיר אנפשיה, וכההיא דאמר רבא לעיל (שבת קמב, ב) אי לאו דאדם חשוב אנא סכינא אבר יונה למה לי, ורבא דמטלטל להו אגב לקנא דמיא משום דאיהו כר׳ יהודה סבירא ליה כדאמרן לעיל, ואף על גב דמשמע לעיל דרבא משום דאדם חשוב הוא לא מטלטל אלא אגב סכינא מידי דשרי לטלטולי בעיניה, הני מילי בר יונה וכל דדמי ליה דאפשר למעביד ליה הכנה מאתמול, אבל הכא דאי אפשר למעבד לה הכנה אחריתי מאתמול אפילו לאדם חשוב שרי. וכן דעת מורי הרב רבי יונה ז״ל ודעת הר״ז הלוי ז״ל.\par \textbf{} ויש לפרש עוד דהכא שמואל אפילו כר״ש הוא דעבד, דהא דמשמע בפרק במה מדליקין (שבת כט, א) דלא אסר כהאי גוונא אלא מאן דסבר ליה כר׳ יהודה דאמר מסיקין בתמרים אבל אין מסיקין בגרעיניהן, הני מילי ביום טוב ומשום דרבי שמעון לית ליה נולד שרי לטלטוליה משום דחזו להסקה, אבל בשבת דלא חזו להסקה מוקצין נינהו דהא לא מחזי אלא להסקה, ואע״ג דחזי קצת לאכילת בהמה כיון דהשתא מיהא סתמן להסקה מוקצין נינהו, דבשבת להסקה לא חזי [ולכן] אסור לטלטלן, וכדתנן בפרק מפנין (שבת קכו, ב) חבילי קש וחבילי עצים וחבילי זרדים, אם התקינן למאכל בהמה מטלטלין אותן ואם לאו אין מטלטלין אותן, והא דשדינהו רב לחיותא בפרק במה מדליקין (שבת כט, א) ביום טוב בארמיאתה, מתוך שראויין להסקה. וכן דעת הרמב״ן ז״ל.\par \textbf{} והא דמטלטלי שמואל ורבא גרעינין אגב ריפתא ולקנא דמיא, לאו משום דפליגי אהא דאמר רב אשי לעיל לא אמרו ככר או תינוק אלא למת בלבד, דהא לא דמיא לההיא ובהא אפילו רב אשי מודה, דההיא דרב אשי במטלטל את ההיתר אגב האיסור, כגון מניח ככר על המת ומראה עצמו כרוצה לטלטל את ההיתר ועושה לו האיסור בסיס, והלכך אסור מפני שהוא מטלטל את האיסור ממש, אבל הא דשמואל ורבא שמניח האיסור על ההיתר ומטלטל גוף ההיתר וניטל גוף האיסור עמו, בזה כולי עלמא שרו.}
\textblock{\textbf{ואם תאמר מכל מקום יעשה הפת או לקנא דמיא בסיס לדבר האיסור. ויש לומר דאינו נעשה בסיס, דכיון שכל עצמו לא נעשה כן אלא לזרקן, דומה למה ששנינו מסלק את הטבלא ומנערה, כלומר במקום שהוא רוצה לזרוק שם הקליפין והעצמות. ואם תאמר והא אמרינן בפרק כירה (שבת מד, ב) במטה שלא יחדה למעות יש עליה מעות אסור לטלטלה. יש לומר דההיא בשיש עליה בין השמשות, ואי נמי במניח לדעת באמצע שבת על ידי עכו״ם או תינוק ולא היה דעתו לנער ממנה בשבת, ואפילו עשוין ליטול משם מעצמן, דבכהאי גוונא יש לנו להעמידה שם בפרק כירה והלכך לרבי יהודה אסור, אבל כל שהניחן באמצע שבת על דעת לנערן, אפילו רבי יהודה מודה וכל שכן      } ר׳ שמעון. ואי נמי יש לומר דהתם במטלטל לצורך גופה בלא ניעור ובשאינו צריך למקומו ועודן עליה עד המקום שהוא צריך להניחה שם. והראב״ד זכרונו לברכה אמר דהתם אף על פי שלא יחדה למעות בטלה היא לגבי מעות דחשיבי, אבל הכא גרעינין בטלי לגבי פת וקליפין לגבי טבלא.}
\newchap{פרק \hebrewnumeral{22} חבית}
\textblock{}
\textblock{\textbf{נתפזרו לו פירותיו בחצרו מלקט על יד על יד ואוכל אבל לא לתוך הסל ולא לתוך הקופה.} כתב מורי הרב ז״ל בהלכותיה כגון שנתפזרו לו בחצרו אחת הנה ואחת הנה, אבל במקום אחד מלקט לתוך הסל כמו שאמרנו בגמרא בענין הא דתנן בפרק נוטל (שבת קמב, א) כלכלה והאבן בתוכה ולישדינה לפירי ולישדיה לאבן ולנקטינהו. והרמב״ן ז״ל פירש כגון שנפלו לו בחצרו בתוך צרורות ועפרורית שבחצר, והיינו דנקט חצרו. וקצת נראה כן מן התוספתא (פי״ז, ה״ו) דתניא התם: פירות שנתפזרו מלקט אחד ואחד ואוכל, נתערבו לו עם פירות אחרים, בורר ומניח על השלחן בורר ומשליך לפני בהמתו, בררן אלו לעצמן ואלו לעצמן או שלקט מתוכן עפר וצרורות, הרי זה חייב.}
\textblock{\textbf{ומודים חכמים לרבי יהודה בשאר פירות.} ואי קשיא לך הא דתנן (ב, א) ביצה שנולדה ביום טוב בית הלל אוסרין, וקא מפרש טעמא התם (ג, א) רבי יצחק גזירה משום משקין שזבו, והא הכא מודו רבנן לרבי יהודה בשאר פירות דלאו בני סחיטה נינהו, וכל שכן בביצה דהוה להו למישרי דלא שייך בה סחיטה. יש לומר דלרבי יצחק סבירא ליה דאפילו בשאר פירות פליגי עליה, ומשמע דרבי יוחנן נמי הכי סבירא ליה הכא, ורבי יוחנן ורבי יצחק אמרו דבר אחד כדאיתא התם. ועוד יש לי תירוץ אחר בו כבר הארכתי במקומו בריש פרק קמא דביצה בסייעתא דשמיא (יג, א ד״ה כולהו).}
\textblock{ הא ד\textbf{תניא זיתים שמשך מהן שמן וענבים שמשך מהן יין וכו׳.} פירש רש״י ז״ל: שזב מהן כמו נהרות המושכין, ואינו מחוור, דאם כן הוה ליה למימר שנמשך מהן שמן, ואי נמי שמשכו שמן. ועוד דאם איתא ליקשי מינה לרב ועולא ורבי יוחנן דכולהו סבירא להו דחלוק היה רבי יהודה אפילו בזיתים וענבים, ואילו הכא קתני בהדיא זיתים שמשך מהן שמן וכו׳, בין לאוכלין בין למשקין היוצא מהן אסור דברי ר׳ יהודה, אלמא דמודה לחכמים בזיתים וענבים וכדאמר רב יהודה משמיה דשמואל. ולפיכך פירשו בתוס׳ שמשך בעל הבית מהן משקין תחלה ואחר כך הכניסן לאוכלין, ואמרי לך רב ועולא ורבי יוחנן דדוקא בשמשך מהן משקין תחלה ואח״כ הכניסן לאוכלין הוא דאסר רבי יהודה, משום דכיון דמעיקרא יהיב דעתיה למשקין וסחטן אע״פ שנמלך עליהן לבסוף לאוכלין אפילו הכי אי שרית ליה היוצא מהן יהיב דעתיה וסחיט, אבל אם הכניסן מתחלה לאוכלין בכי הא ודאי פליג רבי יהודה, ורב יהודה אמר שמואל סבר דהוא הדין למכניסן מתחלה אפילו לאוכלין.}
\textblock{\textbf{וסבר ר׳ יהודה סתם אסור והתנן חלב האשה וכו׳.} ואי קשיא לך ומנא ליה דהא מתניתין רבי יהודה היא. כבר נשמר רש״י ז״ל וכתב דכל שכן אי רבנן קתני לה, דהשתא אפילו לרבנן דמחמרי שרו בסתם ר׳ יהודה דמיקל לא כל שכן, ותיקשי בין לרבנן בין לרבי יהודה.}
\clearpage
\newsection{דף קמד}
\textblock{ והא דאמרינן נמי השתא\textbf{ ומה זיתים וענבים דבני סחיטה נינהו שלא לרצון ולא כלום הוא, תותים ורמונים דלאו בני סחיטה נינהו לא כל שכן.} פירש רש״י ז״ל: דהוא הדין דהוה ליה לאקשויי זיתים וענבים אזיתים וענבים, דקתני לעיל ברבי יהודה לאוכלין היוצא מהן (מותר) [אסור], אלא אלומי אלים ליה לקושיא וכו׳ דהכא אפילו בסתם נמי שרינן וכל שכן כשהכניסן לאוכלין, ובמאי דמפרקינן ומפרשינן לרצון בסתמא, שלא לרצון דגלי אדעתיה דאמר בפירוש לא ניחא לי, איתרצו כולהו קושיי, דטפי עדיף אומר בפירוש לא ניחא לי במשקין ממכניס לאוכלין.}
\textblock{\textbf{אבל הרמב״ן ז״ל פירש דמדרבנן לא קשיא ליה כלל, משום דכיון דאסרי רבנן אפילו בתותים ורמונים שהכניסן לאוכלין היוצא מהם אסור, שמע מינה דלאו משום דמשקין נינהו, אלא משום דגזרינן בהו משום שמא יסחוט, אטו זיתים וענבים דהא דמו למשקין, וכן בזיתים וענבים לרבי יהודה כיון דהכניסן אפילו לאוכלין אסר, אי אפשר דמשום דחשיב להו משקין הוא, אלא משום גזירה שכיון שדרכן לסחוט אותן וממליך עליהם אסורין משום שמא ימלך עליהן לסחטן, א״נ לרבי יהודה משום נולד, דאיהו אית ליה מוקצה דנולד בכל מקום, אלא בתותים ורמונים דשרי רבי יהודה לאוכלין ואסר למשקין ולסתם שמע מינה דמשקין נינהו בסתמא דסתם כמפרש למשקין דמי, ומשום הכי קא מקשה מק״ו דתותים ורמונים ומדרבי יהודה ולא מזיתים וענבים ומדרבנן. וא״ת לימא ליה מתניתין רבנן היא ולא תיקשי. לא קשיא דהא עדיף תירוציה טפי, ועוד דלגבי משקין ולא משקין לא פליגי,       } ומדרבי יהודה נשמע לרבנן דסתם משקין הוו, עד כאן. והוא הנכון שבשמועה זו.}
\textblock{\textbf{חלב האשה מטמא לרצון ושלא לרצון.} פירוש: כדמפרש ואזיל דדם מגפתה טמא.}
\textblock{\textbf{וחלב הבהמה מטמא לרצון ואינו מטמא שלא לרצון.} כתבו בתוספות דצריך עיון מאי שנא חלב בהמה דבעי תחלתו לרצון, וכן בסלי זיתים וענבים כדאיתא בסמוך, ומאי שנא מים דלא בעינן תחלת נביעתן לרצון, דמי גשמים לא בעינן רצון אלא בשעת (נביעתן ו)ירידתן על הפירות. ותירצו דהתם במסכת מכשירין (פרק ח׳, משנה ח׳) משמע דאף במי גשמים בעינן תחלת נביעתן לרצון. והכי תנינן לה בפרק בתרא דמסכת מכשירין בהא פלוגתא דרבי עקיבא ורבנן: אמר להם לא אם אמרתם בסלי זיתים וענבים שתחלתן אוכל וסופן משקה תאמרו בחלב שתחלתו וסופו משקה, אמר ר״ש עד כאן היתה תשובה, מכאן ואילך היינו משיבין לפניו, מי גשמים יוכיחו שתחלתן וסופן משקה ואינן מטמאין אלא לרצון, אמר לנו לא אם אמרתם במי גשמים שהרי אין רובן לאדם אלא לארצות ולאילנות, ורוב החלב לאדם. ומשמע מהכא דלרבנן כל המשקין אינן מטמאין אלא כשהיה תחלתן לרצון חוץ מחלב האשה בלבד, וטעמא משום דדם מגפתה טמא.}
\textblock{\textbf{שכן דם מגפתה טמא.} פירש רש״י ז״ל: וחלבה כדם מגפתה דדם נעכר ונעשה חלב, ודם מגפתה סתמא שלא לרצון והוא יטמא. והקשה עליו ר״ש ז״ל בפי׳ המשנה אשר לו במסכת מכשירין (פ״ו, מ״ח) דאם כן למה צריך תרי קראי במסכת נדה בפרק דם הנדה (נדה נה, ב) לדם וחלב ואפילו יהיו אותן קראי אסמכתא בעלמא. ועוד מה משיב רבי עקיבא מחמיר אני בחלב, שהרי דם וחלב חדא מלתא נינהו. ועוד בפרק כל היד (נדה יט, ב) אמרינן אין דם ירוק מכשיר לרבנן, ואמאי מי גרע מחלב. ועוד למה לי קרא בתוספתא דשבת (פ״ט, הי״ד) לדם הדוה שמכשיר מדכתיב (ויקרא יב, ז) מקור דמיה וכתיב (זכריה יג, א) יהי׳ מקור נפתח לבית וגו׳. ולפיכך הוא ז״ל פירש דמשום שיש חומרא באשה קאמר, שאפילו דם מגפתה טמא, ובשלא הקיז לשתיה קאמר דאי איתקז לשתיה אפילו בבהמה נמי מכשיר וכדתנן התם במכשירין, וטעמא משום דאחשביה.}
\textblock{\textbf{שאני סלי זיתים וענבים דכיון דלאיבוד קיימי מעיקרא אפקורי מפקר להו.} ואם תאמר והא הבוצר לגת הוכשר ואפילו בקופות שאינן מזופפות (לעיל שבת טז, א). ויש לומר דהתם משום גזירת בוצר בקופות מזופפות הוא, ובמפרש לבצור גזרו ולא בסתם.}
\textblock{\textbf{מי דמי ערביא אתרא הוא.} כלומר אתרא דגמלים היא וכיון שיש שם רוב גמלים שפיר קא מקיימי, דכל מקום שגמליהן מרובין צריכין לקוצין ומקיימין אותן, אבל רמונים אין דרכו של אדם לסוחטן, ובית מנשיא שסוחטין הרמונים בטלה דעתן אצל כל אדם.}
\textblock{\textbf{הכא נמי כיון דסחיט להו אחשבינהו.} קשיא לי דכיון דלרב נחמן לא חשבינן ליה משקה אלא משום דאחשביה, אם כן מאי טעמא דרבנן דאסרי אפילו כשהכניסן לאוכלין. ויש לי לומר דקסברי רבנן דכיון דיש מקצת בני אדם שסוחטין אותן כבית מנשיא, איכא למיגזר בהו דלמא האי נמי חשיב ליה משקה, ואי שרית ליה כשיצאו מעצמן ואפילו כשהכניסן לאוכלין, יהיב דעתיה וסחיט.}
\textblock{\textbf{הכא נמי כיון דסחיט להו אחשבינהו והוי ליה משקה.} וא״ת אם כן פגעין ועוזרדין נמי לא יסחוט והיכי שרי להו תנא דברייתא, דהא התם נמי כיון דסחיט להו אחשבינהו והוה ליה משקה. כבר פרש״י ז״ל דקסבר רב נחמן דברייתא הכי קאמר סוחטין בפגעין ובעוזרדין כדי למתקן אבל לא להוציא מהן משקה, אבל רמונים ותותים כיון דאיכא מאן דסחיט להו כבית מנשיא, יהיב דעתיה וסחיט להוציא מהן משקה. ונמצא לפי פרש״י ז״ל דאפילו פגעין ועוזרדין אסור לסוחטן בתוך קערה להוציא מהן משקה. ואינו מחוור.\par \textbf{} ויש מי שפירש דתותין ורמונים וכבשין סתמן [לאו] למשקה, הא בשסחטן ואחשבינהו (ו)הוי ליה משקה כיון דאיכא קצת בני אדם דסחטי להו כבית מנשיא, אבל פגעין ועוזרדין שאין דרכן של בני אדם לסוחטן אפילו סחיט להו האי בטלה דעתו אצל כל אדם ולא הוי משקה אלא כמפרק אוכל מתוך אוכל, והלכך סוחטין אותן לכתחלה ואפילו לתוך הקערה והיינו דשרי להו תנא דברייתא. וזה נכון. וכן פירש מורי הרב רבינו יונה ורמב״ן ז״ל. וכן נראה מדברי הרב אלפסי ז״ל שפסק כסתמא דברייתא דסוחטין לכתחלה בפגעין ובפרישין ובעוזרדין.}
\textblock{ הא ד\textbf{אמר רב פפא משום דהוי דבר שאין עושין ממנו מקוה לכתחילה.} לא למימר דאתי לאפלוגי אעיקר מאי דקא פסיק רב נחמן ואמר הלכה כשל בית מנשיא, דאפשר ודאי דהתם כיון דאחשבינהו הוה להו משקה, אלא אמאי דקא מדמי לה לההיא דאמר רב חסדא הוא דפליג לומר דההיא דרב חסדא טעמא אחרינא אית ליה, ורבא נמי דאית ליה כהאי סברא דרב פפא בסמוך לא פליג אהא דרב נחמן, אלא אאוקימתא דאביי בלחוד דלא משכח לאוקומי מתניתין אלא כר׳ יעקב, דרב נחמן ואביי קיימי בהא בחדא שיטתא דאינו פוסל את המקוה בשינוי מראה אלא משקה בלחוד, ורב פפא סבירא ליה דכל דבר שאין עושין ממנו מקוה לכתחילה פוסל בשינוי מראה, ובהא בלחוד הוא דפליגי אבל בדינא דרב נחמן לענין סחיטת תותים ורמונים ובמשקה היוצא מהם, לא פליג עליה כלל.}
\textblock{ הא ד\textbf{אמר רב יהודה אמר שמואל סוחט אדם אשכול ענבים לתוך הקדרה אבל לא לתוך הקערה.} אפילו      קאמר דדבר הלמד מענינו הוא, דהא הכא גבי שבת קיימינן ועד השתא בסחיטת פירות בשבת ובמשקין היוצא מהן עסקינן, ואף על גב דבמתניתין ובברייתא מודו בין רבנן בין רבי יהודה דאין סוחטין אפילו בתותים ורמונים לכתחילה וכל שכן בזיתים וענבים, ואפילו בשהכניסן לאוכלין ויצאו מעצמן קאמר לעיל רב יהודה גופיה משמיה דשמואל דמודה להו רבי יהודה לחכמים, הכא שאני שסחטן לתוך הקדרה שיש בה אוכל, וקסבר דמשקה הבא לאוכל הרי הוא כאוכל, והוי ליה כמפריד אוכל מתוך אוכל דשרי לכתחילה.\par \textbf{} ועוד דתניא לקמן (שבת קמה, א) סוחטין כבשים לצורך השבת אבל לא למוצאי שבת, וזיתים וענבים לא יסחוט ואם סחט חייב חטאת, ואוקימנא לה בין לרב בין לשמואל בין לרבי יוחנן דוקא כשסחטן לגופן, אבל למימיהן בין לרב בין לשמואל פטור אבל אסור, וזיתים וענבים חייב חטאת, ולרבי יוחנן בין כבשין בין שלקות למימיהן חייב חטאת כסוחט זיתים וענבים, ההיא נמי בסוחט לתוך הקערה היא מתניא, אבל לתוך הקדרה מותר. ותדע לך דהא שמואל דשרי הכא אפילו לכתחילה ואפילו זיתים וענבים לתוך הקדרה, קאמר התם דאפילו כבשין ושלקות למימיהן פטור אבל אסור וזיתים וענבים חייב חטאת. וזו היא ראיה שאין עליה תשובה, וכן דעת הרב אלפסי ז״ל בהלכות.}
\textblock{ הא דדייק רב חסדא מדשמואל ואמר\textbf{ מדברי רבנו נלמוד חולב אדם עז לתוך הקדרה אבל לא לתוך הקערה.} פירשו הגאונים ז״ל וכן הרב אלפסי ז״ל בהלכות דוקא ביום טוב אבל לא בשבת, ואע״ג דאוקימנא ההיא דשמואל אפילו בשבת, לא דייק מינה רב חסדא דתהוי נמי חליבת עז דהיא חיה ואי אפשר לאוכלה ביומה בשבת כסחיטת ענבים שאפשר לאוכלן כמות שהן, אלא לענין קדרה וקערה קאמר והאי כדיניה והאי כדיניה. ור״ת ז״ל כן כתב דהא דרב חסדא דוקא ביום טוב, ומשום דכיון דאפשר לו לאכול אפילו העז הוי ליה כמפריד אוכל מתוך אוכל, לכשתמצא לומר דמשקה הבא לאוכל כאוכל דמי, אבל בשבת כיון דאי אפשר לשחוט ולאכול העז בעצמה, אפילו תמצא לומר דמשקה הבא לאוכל כאוכל דמי. מכל מקום הוי ליה בשעה שהוא חולב כבורר אוכל מתוך הפסולת וחייב משום בורר.\par \textbf{} אבל גדולי האחרונים ומקצת מן הראשונים הסכימו דאפילו בשבת קאמר. וכן ודאי נראה, שאלולא כן הוה ליה לרב חסדא לפרושי כי היכא דלא נטעה במלתיה, כיון דההיא דשמואל בשבת היא אמורה, ואי משום טעמא דר״ת ז״ל, לא היא, דהכא לאו אוכל מתוך פסולת הוא דהא חזיא ליה למיכל אלא דמשום איסורא דשבת דרכיב עלה, ואפילו ביומיה נמי חזיא לעכו״ם ולחולה, הלכך הכא לאו בורר הוא אוכל מתוך פסולת אלא כמפרק אוכל מתוך אוכל, ומשום מוקצה נמי ליכא אף על גב דבהמה לא חזיא, דשמואל כר״ש סבירא ליה דלית ליה מוקצה ולא נולד. וכן כתב הרמב״ן זכרונו לברכה.\par \textbf{} ועוד כתב ז״ל והוא הנכון דאין מוקצה שקפץ מעצמו אסור אלא למלאכתו, כגון שוחט בשבת שאסור משום מוקצה הא כל שלא עבר על השבת ואפילו בשוגג ממילא הותרה, כדאמרינן (ביצה ב, ב) בביצה שנולדה בשבת שבת דעלמא תשתרי ולא אסרינן לה משום מוקצה. אלא שנראה להוסיף על דבריו דלאו דוקא שוחט, אלא גופה של בהמה אפילו מתה מעצמה אף ע״פ שלא עבר על השבת וממילא הותרה כדאמרינן בביצה, אפילו הכי אסורה להאכילה לכלבים ומשום מוקצה, וכדברי האומר (ביצה כז, ב) מודה היה רבי שמעון בבעלי חיים שמתו, אלא דהתם טעמא אחרינא היא, דכיון דאי אפשר לו להכינה ולהתירה ביומה לגמרי מקצה לה מדעתו דאין אדם יושב ומצפה מתי תמות בהמתו, אבל כל מידי דאפשר ומצוי לבא ביומו בלא איסור לא מקצה ליה מדעתו, והיינו טעמיה דביצה שנולדה ביום טוב דעלמא, והיינו טעמיה דחולב עז לתוך הקדרה.}
\clearpage
\newsection{דף קמה}
\textblock{\textbf{הוא מותיב לה והוא מפרק לה בסוחט לתוך הקערה.} והוא הדין דהוי ליה לתרוצי הכין בהדיא הא דמותיב רמי בר יחזקאל זב שחלב את העז בחולב לתוך הקערה, אלא לארווחה דמתניתין קאמר כלומר דאפילו בחולב לתוך הקדרה היא מתניתין, ומשום טפה המלוכלכת על פי הדד וקושטא דמלתא קא מתרץ.}
\textblock{ גירסתו של רש״י ז״ל:\textbf{ אמר רב פפא דכולי עלמא משקה הבא לאוכל לאו כאוכל.} ופשטא דמלתא לכאורה ודאי הכי אזלא שפיר, ומכל מקום לענין פסק הלכה לא קיימא לן הכי, אלא כרב יהודה אמר שמואל, ורבינא נמי הכי משמע דסבירא ליה מדמותיב ומפרק לה, וההוא דרמי בר יחזקאל נמי תרצוה בגמרא כשמואל דמשמע ודאי דבעלי הגמרא הכין סבירא להו, ורבי זירא אמר ר׳ חייא בר אשי משמיה דרב כותיה דשמואל דסוחט אדם אשכול של ענבים לתוך הקדרה. וגירסת הגאונים ז״ל נכונה מזו, שהם ז״ל גורסין דכולי עלמא דמשקה הבא לאוכל כאוכל דמי, וה״פ מר סבר דמשקה ההולך לאבוד משקה הוא, כלומר רבי יהודה דאמר הוכשר דאע״ג דמשקה הבא לאוכל בעלמא כאוכל, הכא כיון דהולך לאיבוד שהאש שורפו ואינו נשאר בתוך האוכל לאו כאוכל הוא אלא כמשקה דעלמא שאינו בא לתוך אוכל, ומר סבר אע״ג דהולך לאבוד לבסוף מכל מקום השתא מחליק בהם את הפת וליפותו הוא דקא עביד הלכך כמשקה הבא לאוכל דעלמא הוא ואוכל הוא. והשתא לא אתיא הא דרב פפא דלא כהלכתא.}
\textblock{\textbf{למימיהן פטור אבל אסור.} פירש רש״י: משום דאין המשקה הזה מגופן אלא משקה שנבלע בהם, והלכך אינו כמפרק אבל אסור גזירה אטו זיתים וענבים. והקשו עליו בתוס׳ מדאמרינן לעיל בפרק תולין (שבת קמא, א) לא ליהדוק אינש אודרא אפומא דשישא דלמא אתו לידי סחיטה, וכן אסרו לעיל (שבת קמג, א) ספוג בזמן שאין לו בית אחיזה. אלא פירושא משום דסבירא לן כתנא דבי מנשיא דאמר לקמן בסמוך דדבר תורה אינו חייב אלא דריכת זיתים וענבים בלבד משום דרובן למשקין, והלכך כשסוחטן הוי ליה כדריכת זיתים וענבים, אבל שאר פירות רובן לאכילה הן עומדין,       יהיב דעתיה וסחיט, בטלה דעתו אצל כל העולם ואפילו צריך לו לא חשיב משקה אלא אוכל.}
\textblock{\textbf{רב מתרץ לטעמיה ושמואל מתרץ לטעמיה.} ולענין פסק הלכה, כללא דכולה שמעתין סוחטין בפרישין ובפגעין ובעוזרדין וכל כיוצא בהן שאין דרכן של בני אדם לסוחטן, לפי שאינו אלא כמפריש אוכל אצל בני אדם, והלכך אילו רצה סוחט ואפילו לתוך הקערה דלאו משקה כלל. וזה כדעת הגאונים ז״ל ושלא כדברי רש״י ז״ל, והדברים נראין כן באמת כדברי הגאונים ז״ל.\par \textbf{} אבל דברים שדרכן של בני אדם לסחוט אותן לעתים אסור לכתחילה כתותים ורמונים שהרי יש קצת בני אדם שסוחטים אותם ואינו מפסידן, אבל היוצא מהן מותר והוא שהכניסן לאוכלין דקיימא לן בשאר פירות כר׳ יהודה וכדפסיק ר׳ יוחנן, ושום ובוסר ומלילות וחלות דבש דתנן (לעיל שבת יט, א) השום והבוסר והמלילות שרסקן מבעוד יום ר׳ ישמעאל אומר יגמור משתחשך רבי עקיבא אומר לא יגמור ואמרינן בשלהי פרק קמא דמכלתין (שם) במחוסרין דיכה כולי עלמא לא פליגי דאסור, היינו משום דהני טפי קיימי לסחיטה מפגעין ופרישין, והרי הם כתותים ורמונים, והלכך לכתחילה אסור, ועוד דלרבי יוחנן אפילו בשלקות וכבשין ודג לצירו לתוך הקערה חטאת הוא דמחייב, והלכך לסבריה דרבי יוחנן אפילו חיוב חטאת נמי איכא.\par \textbf{} זיתים וענבים היוצא מהם אסור ואפילו הכניסן לאוכלין, דקיימא לן בהא כרבנן וכדפסק ר׳ יוחנן, ועוד דאמר רב יהודה אמר שמואל דאפילו ר׳ יהודה מודה בהן לחכמים, אבל מותר לסחטן לכתחילה לתוך הקדרה, דמשקה הבא באוכל כאוכל דמי. ולענין כבשים ושלקות לגופן כולי עלמא לא פליגי דשרי, ואפילו לתוך הקערה למימיהן, אע״ג דר׳ יוחנן מחייב חטאת לא קיימא לן כותיה, אלא כרב ושמואל דתרווייהו אומרין אינו חייב חטאת, והוי ליה תרי לגבי ר׳ יוחנן דקיימא לן כותייהו. ועוד דתנא דבי מנשיא דבר תורה אינו חייב אלא על דריכת זיתים וענבים בלבד. אבל למימיהן דרב שרי לשלקין וכבשין פטור אבל אסור ושמואל אמר בתרוייהו פטור אבל אסור, קיימא לן בהא כשמואל משום דבענין איסורא שמואל ורבי יוחנן קיימי בחדא שיטתא וליכא בינייהו אלא חיוב חטאת, הלכך קיימא לן כותיה דשמואל דפטור אבל אסור מיהא.\par \textbf{} וכן פסק הרב אלפסי ז״ל, והלכך דג לצירו דקשרי רב ואפילו לתוך קערה לא קיימא לן כותיה, דרב אזיל בה לטעמיה דשרי שלקות אפילו לכתחילה בין לגופן בין למימיהן, והלכך אפילו דג לצירו לתוך הקערה אסור לתוך הקדרה מותר, ואין צריך לומר לחלוב לתוך הקערה בין בשבת בין ביום טוב אסור, ואפילו לינק מבהמה ביום טוב דהוי מפרק כלאחר יד אסור אלא במקום צער כגון גונח וכיוצא בו כדאיתא בכתובות בפרק אף על פי (כתובות ס, א), אבל לתוך הקדרה ביום טוב מותר דמשקה הבא לאוכל כאוכל דמי, וכדאמר רב חסדא מדברי רבינו נלמוד חולב אדם עז לתוך הקדרה אבל לא לתוך הקערה, ואפילו בשבת נראין הדברים שהוא מותר, אלא שכבר הורו הגאונים לאסור ולהם שומעין הלכה למעשה.}
\clearpage
\newsection{דף קמו}
\textblock{ מתני׳:\textbf{ שובר אדם את החבית לאכול ממנה גרוגרות.} פירש רש״י ז״ל: משום שאין במקלקל שום איסור שבת. ואינו מחוור בעיני, דאי משום מקלקל לכתחילה מי שרי. ויש לומר כיון דבעלמא במקלקל פטור אבל אסור הכא משום צורך שבת מותר לכתחילה. אלא דאכתי קשיא שהרי ביום טוב פרק המביא (ביצה לג, א) גבי ר׳ אליעזר אומר נוטל אדם קיסם וכו׳ אמרינן תני חדא קוטמו ומריח בו, ותניא אידך לא יקטמנו להריח בו, פירש עצי בשמים, ופריק אמר ר׳ זירא אמר ר׳ חסדא לא קשיא הא ברכין הא בקשין. ואקשינן מאי שנא מהא דתנן שובר אדם את החבית לאכול ממנה גרוגרות ובלבד שלא יתכוון לעשות כלי, ואם איתא מאי קושיא, שאני הכא דמקלקל הוא וכיון דבעלמא פטור אבל אסור הכא מותר לכתחילה.\par \textbf{} ובתוס׳ הקשו על משנתנו דהכא מההוא דתנן בעירובין בפרק בכל מערבין (עירובין לד, ב) נתנו במגדל ואבד המפתח עירובו עירוב, ר״א אומר אם אין ידוע שהמפתח במקומו אין עירובו עירוב, ואוקימנא פלוגתייהו במגדל של עץ דקטיר במתנא ובעי סכינא למיפסקיה, ורבנן סבור כל הכלים ניטלין בשבת לצורך גופן ור״א סבר כר׳ נחמיה דאמר אין כלי ניטל אלא לצורך תשמישו ואפילו טלית ואפילו תרווד, הא לאו הכי אפילו רבנן מודו דאין עירובו עירוב דהוא במקום אחד ועירובו במקום אחר ולא שרינן לשבור את המגדל. ומשום כך אמרו דמתניתין דהכא במוסתקי, כלומר בחבית של חתיכות מחוברות במוסתקי דהוא כלי רעוע ולא שייך בכי האי כלי בנין וסתירה, וכדאוקמוה התם במסכת ביצה פרק המביא, ואע״ג דלא מוקמינן לה התם בהכין אלא לר״א, מכל מקום גם לרבנן נמי על כרחין אצטרכינן לאוקומה בהכין, אע״ג דרבנן לא מחייבי חטאת בקוטם את הקיסם בין לחצות בו שיניו בין לפתוח בו את הדלת, מכל מקום הא אמרינן דבין כך ובין כך פטור אבל אסור, ואילו הכא שריא לכתחילה.}
\textblock{\textbf{ואף פירוש זה אינו מחוור בעיני, מדאמרינן בגמרא דבעו מיניה מרב ששת מהו למיברז חביתא בבורטיא, לפתחא קמכוון ואסור או דלמא לעין יפה קא מכוון ושפיר דמי. ועוד דאם איתא הא דאמרי (ביצה שם) רבה בר רב הונא ורבין בר רב אדא כי הוינן בי רב יהודה הוה מפשח ויהיב לנא אלוותא אלוותא אע״ג דחזיא לקתתא דנגרי וחציני, כמאן תרמייה, והתם אוקימנא להא כרבנן (והתם) [ו]מדמינן ההיא להא דחבית ולא מתוקמא לעולם אלא כרב יהודה, אלא משום דלא תקשי מתניתין לר״א כלומר דלא נימא דלית ליה לר״א מתניתין דשובר את החבית אצטריך לדחוקה ולאוקומה למתניתין במוסתקי, אבל ההיא דאלוותא אלוותא לעולם רבנן ולא ר״א, ואם איתא דמתניתין לרבנן נמי דוקא במוסתקי דרב יהודה כמאן. אלא נראה דבין כך ובין כך אפילו בלא מוסתקי לרבנן שרי דלעין יפה קא מכוון, אבל בנותן עירובו      } במגדל היינו טעמא משום דהתם במגדל גדול מיירי דהוי כמו אהל ושייך ביה בנין וסתירה, הא במגדל קטן לא שייך ביה טעמא דבנין וסתירה, והיינו טעמא דחבית. וכן פרש״י ז״ל שם בעירובין, וכן מוכח כל אותה סוגיא (ש)שם, דאין בנין בכלים ואין סתירה בכלים ומותר לשוברו, אלא טעמא דמגדל משום דאין לו דין כלי.}
\textblock{ [גמרא:] הא דמותיב מברייתא דקתני:\textbf{ מביא אדם חבית של יין ומתיז ראשה בסייף.} תמיהא לי אמאי לא מותיב ממתניתין דקתני שובר אדם את החבית לאכול ממנה גרוגרות. ויש לומר דאי ממתניתין הוה אמינא דוקא שובר דודאי לאו לפתחא קא מכוון, וכדקתני בהדיא ובלבד שלא יתכוון לעשות כלי.}
\textblock{\textbf{בית סתום אינו מטמא כל סביביו פרץ פצימיו מטמא כל סביביו.} פירש רש״י ז״ל: כדין קבר דגזרו חכמים לטמא את סביביו ד׳ אמות גזרו שמא יאהיל ולא יבין. ובתוס׳ הקשו עליו שהרי בסוטה בפרק משוח מלחמה (סוטה מד, א) אמרינן בהדיא דהיכא דמסיימי מחיצאתא לא גזרינן. ומשום הכי פירשו הם ז״ל כההיא דביצה (ביצה י, א) מת בבית ולה פתחים הרבה כולן טמאים, נפתח אחד מהן הוא טמא וכולן טהורין, והכי נמי בזמן שלא פרץ פצימיו הרי הוא כפתוח ואינו מטמא כלים שתחת זיזין שבסביבות הבית אלא מה שתחת הזיז שעל אותו הפתח, אבל פרץ פצימיו אז אינו ראוי ליצא באותו פתח יותר מבשאר מקומות, והלכך כל הכלים שתחת כל זיזין הסביבות טמאין.}
\textblock{ [מתני׳:]\textbf{ מי שנשרו כליו בדרך.} פירוש: מי שנפלו כליו במים מלשון פירות הנושרין ואינו לשון שריה, כמו שפרש״י ז״ל, ותדע לך מדאמרינן בריש פרק משילין (ביצה לה, ב) מאן דתני מנשירין לא משתבש כדתנן התם מי שנשרו כליו במים.}
\clearpage
\newsection{דף קמז}
\textblock{ [גמרא:]\textbf{ המנער טליתו.} פירש רש״י ז״ל מפני האבק ומשום מלבן. ובתוס׳ הקשו דלא מצינו לבון בכיוצא בזה. אבל ר״ח ז״ל פירש מנער את הטלית מן הטל כההיא דרבה דבסמוך ומשום מכבס, ולמאן דקפיד הוי פסיק רישיה, ולמאן דלא קפיד ליכא אלא דבר שאין מתכוון שאינו כבוס גמור.}
\textblock{\textbf{טלית מקופלת.} פירש רש״י ז״ל לאחר שנתנה על ראשו הגביה שיפוליה על כתפו חייב חטאת שאין זה דרך מלבוש. ומה שיצא רבי לשדה והניח שני צידי טליתו על כתפו, לאו בענין זה היה, דטלית מקופלת היינו כששני צידי הטלית שלפניה ושלאחריה מונחין על כתפו כגון שאנו מתעטפין בטליתות, אבל ההיא דרבי לא קפל אלא הצדדין שלפניו בלבד והיה סבור שזה מותר, וחזר בו מסהדותיהו דהני רבנן. ומכאן יש לאסור בחתנים שלנו שלא יצאו בטליתותיהם ברשות הרבים עד שישלשו. וכן אסר מורי הרב ז״ל.}
\textblock{ הא ד\textbf{אמר רבי אלעזר אסור לעשות מרזב בשבת.} ופירש רבי זירא כיסי בבלייתא, פירש הריא״ף: שטליתו מופשלת הנה והנה וחלל כנגד השדרה שהוא כעין מרזב, וכן פירש בערוך (ערך מרזב). והקשו בתוס׳ שאין זה דומה לכיס. ורש״י ז״ל פירש שמסלקין בגדיהן מן הארץ כשהן ארוכין וכופלין אותן כלפי מעלה פרוצ״ש בלע״ז ומחזיקין אותן בחוטין ונראין כיס ומרזב ואסור משום תקוני מנא, ורב פפא נתן כלל בדבר שכל שהוא לכנופי ולהעמיד כך תמיד מתוקנין אסור, אבל להתנאות לפי שעה מותר. וגם זה אינו מחוור בעיני. דאין ענין לשמועתנו, דהכא באיסור הוצאה איירינן לומר מה שאינו דרך מלבוש ואסור לצאת בו משום משוי, ולא איירי באיסור תקוני כלי. והם [התוס׳] פירשו כיסי בבלייתא שמקפלין טליתן מן הצדדין על כתיפן ועושין קפולין הרבה זה על גב זה זה קטן מזה ודומה למרזב שיש דפנות בין זה לזה וריוח בנתיים, ואינו דרך מלבוש אלא כמשוי ואסור לצאת בו, ואמר רב פפא אדעתא לכנופי כלומר שמקפלו כמו שיכול כדי לרוץ מהר אסור דזה דרך משוי, אבל להתנאות שאין מקפלו כל כך אלא מעט כדי להתנאות בטליתו שפיר דמי דדרך מלבוש בכך, וכן פירש מורי הרב ז״ל. ואין זה דומה לטלית מקופלת, כי טלית מקופלת היינו שראשי הטלית כלומר שפוליה מקופלים ומונחים על כתפו, אבל זה אינו מניח השיפולים ממש על כתפו אלא שמקפל צדדי הטלית מכאן ומכאן.}
\textblock{ [מתני׳:]\textbf{ הרוחץ במי מערה ובמי טבריא ונסתפג אפילו בעשר אלונטיאות לא יביאם בידו.} הא דקתני נסתפג, לאו דוקא נסתפג, אלא אפילו לכתחילה מסתפג בהן וכדקתני אבל עשרה בני אדם מסתפגין, ואין בין יחיד למרובין אלא שיחיד לא יביאנו דלמא משתלי וסחיט, וברבים לא חיישינן דכל חד וחד מדכר לחבריה, אבל לענין מסתפג הוא הדין ליחיד, דאי לא, ליתני לא יסתפג ואם נסתפג לא יביא ובהדיא תניא בברייתא מסתפג אדם באלונטית ומניחה בחלון.}
\textblock{\textbf{וקשיא לי למה לא אסרו להסתפג דלמא סחיט וכדאמרינן בפרק אלו קשרים (שבת קיג, ב) היה מהלך בשבת ופגע באמת המים אם יכול להניח רגלו הראשונה קודם שיעקור רגלו שניה מותר, ואם לאו אסור, ואתקיף עליה רבא היכי ליעביד, לינחות זמנין דמיתווסן מאניה מיא ואתי לידי סחיטה, והתם חיישינן אפילו אדלמא אתווסאן מאניה, כל שכן הכא דמסתפג בהן ממש. ועוד ההיא דריש פרק טומנין (מח, א) דפרס דסתודר אפומא דכובא, ונזהיה (רבא) [רבה] משום דדלמא סחיט. ויש לומר דלא אסרו להסתפג ברוחץ דאם כן אתה אוסר עליו אפילו רחיצה, שאין דרך לרחוץ בלא להסתפג ואין הדבר עומד, ובמקומות אחרים נמי התירו      } לצורך, כגון שומר פירות שהתירו לו לעבור עד צוארו במים (יומא עז, ב), ואינו צריך להגביה בגדיו, ואדרבא אמרו ובלבד שלא יוציא ידו מתחת שפת חלוקו.}
\textblock{\textbf{הרוחץ דיעבד אין לכתחילה לא.} כתב הריא״ף ז״ל: ואי קשיא לך הא דגרסינן בפרק שמנה שרצים (שבת קט, א) רוחצין במי גרר ובחמי טבריא, אלמא רוחצין לכתחילה. ותירץ דהרוחץ לאו אחמי טבריא קאי אלא אמי מערה, ולא תני טבריא אלא לגלויי אמי מערה, מה מי טבריא חמין אף מי מערה חמין, ומי מערה אמאי לא שרי לכתחילה משום דמטללא כדאמרינן בפרק שור שנגח את הפרה (ב״ק נ, ב) מערה מרבעא ומטללא, הלכך נפיש הבלא דידיה ואתי לידי זיעה ומשום הכי אסור לכתחילה. ואין זה מחוור. שהזיעה יותר מותרת מן הרחיצה, דלא נאסרה זיעה אפילו בחמי האור אלא משום גזירה דרחיצה כדאיתא בפרק כירה (שבת מ, א) וכיון דרחיצה דחמי טבריא מותרת כל שכן הזיעה בהן. ועוד דאי מי מערה דחמי טבריא קאמר, מאי קאמר מני רבי שמעון, דהא אפילו רבי יהודה לא פליג אלא בחמי האור, אלא מי מערה דחמי האור קאמר, שכן דרכן להחם מערב שבת וכדי להעמיד חומן מניחן במערה דמטללא, ולעולם מי טבריא אפילו מטללא נמי דלגמרי התירו חמי טבריא, וכמו שאמרו בפרק כירה (שם) ראו שאין הדבר עומד התירו להם חמי טבריא, ומשמע דלגמרי התירו. וכן דעת מורי הרב ודעת הרמב״ן ז״ל.}
\textblock{ הא דאמרינן:\textbf{ רבי היא דתניא אמר רבי כשהיינו למדין תורה אצל ר״ש כו׳.} איכא למידק ומאי ראיה והא התם אחריני הוו בהדיה, וכדתניא היינו מעלין שמן ואלונטית, ולא אסרו אלא ביחיד. ותירצו בתוס׳ ז״ל דמסתמא כיון דסבירא ליה כותיה בחצר וגג וקרפף שהוא כרשות אחד, מסתמא הוא הדין נמי דסבירא ליה כותיה בהא נמי וביחיד היה מביאו. ולי נראה דכיון שהיה עמו לא היו נוהגין בפניו שלא כדבריו שלא לכבוש את המלכה עמו בבית.}
\clearpage
\newsection{דף קמח}
\textblock{ [מתני׳:]\textbf{ ובלבד שלא יאמר לו הלויני.} ונשים שאינן מכירות בין הלויני להשאילני צריכות לומר תן לי וכיוצא בזה מלשונות שאין משמען לשון הלואה.}
\textblock{ [גמרא:]\textbf{ אמר ליה השאילני לא אתי למיכתב, הלויני אתי למכתב.} פירש רש״י ז״ל: משום דסתם הלואה שלושים יום, וכדי שלא ישכח כותב על פנקסו, אבל שאלה אין לה זמן אלא כל זמן שמשאיל רוצה ליטול את שלו נוטל, ולפיכך הצריכוהו לומר כאן השאילני אף על גב שאינו חוזר בעין וכל שאינו חוזר בעין לא שייך ביה לשון השאלה אלא לשון הלואה, דהכי קאמר ליה הריני מחזיר לך כל זמן שתרצה כמו שאלה. והא דאמרינן במנחות (מד, א) טלית שאולה כל שלושים יום פטורה מן הציצית, לאו משום דשאלה סתמא שלושים יום, אלא משום דדרכן של בני אדם להשאיל בפירוש כליהם עד שלושים יום, הלכך אף זה כשאינו מטיל בה ציצית לא אתי לידי חשדא, דמימר אמרו דטלית שאולה היא, הא לבתר מכאן אתי לידי חשדא, ולעולם המשאיל סתם אין לו זמן, וכשורא דמטללתא נמי דכל שלושים יום אין לו חזקה (ב״ב ו, ב), היינו משום דסתם בני אדם אינם מקפידין בכך עד שלושים יום.\par \textbf{} אבל ר״ת ז״ל וכן הרב בעל העיטור ז״ל סוברין דשאלה כהלואה בדבר זה דסתמן שלושים יום מההיא דמנחות, והכא הכי קאמר: שאלה כיון שהיא חוזרת בעיניה ואינו רואה אותה לא חייש לשכחה, מה שאין כן בהלואה דחייש לשכחה ואתי למיכתב, והכא נמי אע״פ דאינה חוזרת בעין, כיון דהוי ליה למימר הלויני ואמר השאילני מינכרא מלתא ולא אתי למיכתב. ודברי רש״י ז״ל נראין עיקר. וכן משמע בסמוך וכמו שאני עתיד לכתוב בסייעתא דשמיא.}
\textblock{\textbf{ומכל מקום מסתברא שאפילו לדברי רש״י ז״ל אסור לומר לו הלויני על מנת שאחזיר לך כל זמן שתרצה, דכיון      } דהלואה היא שנתנה להוצאה והוא אומר לו בלשון הלואה, אע״פ שהתנה עמו שיחזיר לו כל זמן שירצה, אפילו הכי אין דרכן של מלוים לתבוע ולדחוק את הלוה תוך שלושים, ואתי למכתב דלא מינכרא מלתא. ותדע לך, דאפילו כי אמר השאילני לא סגי ליה בהכין אי לאו משום דשרו ליה רבנן משום דלא ליתי לאימנועי משמחת יום טוב.}
\textblock{ ה״ג רש״י ז״ל:\textbf{ אמר ליה כיון דהשאילני שרו ליה רבנן הלויני לא שרו ליה רבנן מינכרא מלתא ולא אתי למכתב.} והיא גירסא דייקא. ויש ספרים שכתוב בהן: א״ל בחול דלא שנא כי אמר ליה הלויני לא שנא כי אמר ליה השאילני לא קפדינן עילויה, אתי למכתב, בשבת דהשאילני שרו ליה רבנן הלויני לא שרו ליה רבנן, מינכרא מלתא ולא אתי למיכתב. ונ״ל לפרשה הכין וכיון דבחול זימנין דבעי למימר הלויני, כלומר בדברים שהן הלואה שנתנו להוצאה, ואפילו הכי אמר השאילני והוא כותב על פנקסו, א״כ אף בשבת מאי היכרא אית ליה, דאין בין לשון הלואה ללשון שאלה ולא כלום, ובתרווייהו אתי למיכתב כדרך שהוא כותב בחול, א״ל בחול כיון שנתנה להוצאה מידע ידע דלהוצאה היא ואע״ג דא״ל לשון שאלה, והלכך הוא כותב כאילו אמר לו הלויני, אבל הכא דבדוקא א״ל השאילני דלא שרו ליה רבנן למימר ליה הלויני, מינכרא ליה מלתא ולא אתי למיכתב.}
\newchap{פרק \hebrewnumeral{23} שואל}
\textblock{}
\textblock{\textbf{בשבת הוא דאסור הא בחול שפיר דמי לימא מתניתין דלא כהלל.} ואיכא למידק למה ליה למימר מדיוקא, לימא בהדיא שואל אדם מחבירו כדי יין ושמן נימא מתניתין דלא כהלל, א״נ מדתנן וכן שואלת אשה מחברתה ככרות. וי״ל דמהני ליכא למשמע מידי, דדלמא כיון דהשאילני א״ל ושאלה כל אימת דבעי תבע לה לפי שעה לא חייש הלל, אבל הלויני דשלשים יום בכי הא חייש שמא יוקרו החטים. וזו ראיה לדברי רש״י ז״ל דשאלה אין לה זמן.}
\textblock{\textbf{לא תלוה אשה לחברתה ככר עד שתעשנה דמים.} הקשה הרמב״ן ז״ל אמאי לא תלוה דהא בשעת הלואה לאו מידי עבדי, ולא הוי ליה למימר אלא אשה שהלותה ככר מחבירתה והוקר נותנת לה דמיו. ותירץ הוא ז״ל דמתוך שהוא דבר מועט חוששין שאע״פ שהוקרו תתן לה הככר, דאומרת שמא תוחזק בין שכינותיה כצרת עין. ולפי דבריו נצטרך לפרש הא דפרקין, הא באתרא דקייץ דמייהו הא באתרא דלא קייץ דמייהו, דכיון דקייץ דמייהו אם יוקרו החטים תעשה עצמה כאילו אין לה ככר ותתן לה דמיו שהן קצובין וידועין בשעת ההלואה. ואין צורך לכך אלא מפני שסתם ככרות של בעל הבית אין נעשין במדה ידועה ואין להם קצב אחד לכל, והן אין מקפידין זה על זה להשיב לחמם במשקל, לפיכך חשש הלל שמא יוקרו החטים וכיון דלא קייץ דמיהם אינן יודעין כמה היה שוה ובאין לידי ריבית בסופן, ולפיכך אסור בתחלתן עד שתעשנה דמים, ואוקימנא באתרא דקייץ דמייהו, כלומר ואינן באין לידי ריבית שכבר ידוע להן דמיו. וכן נראה מפירוש רש״י כמו שפרשתי. ונראה לי ראיה דקאמר אתרא דקייץ דמייהו ואתרא דלא קייץ דמייהו, ולא קאמר הא בשיצא השער הא בשלא יצא השער.}
\textblock{ הא ד\textbf{אמר רב יוסף הלואת יום טוב לא נתנה ליתבע דאי נתנה ליתבע אתי למיכתב.} קשיא לי והא אמרינן דהשאילני מינכרא מלתא ולא אתי למיכתב, ורב יוסף אפילו אמר ליה בלשון שאלה קאמר, מדשקלינן וטרינן עלה למידק ממתניתין, ורבא נמי דאמר נתנה ליתבע מודה הוא דאיכא למיחש לדלמא אתי למיכתב, אלא משום דלא ליתי לאימנועי משמחת יום טוב שרו ליה רבנן. ויש לומר דכיון דאף זו הלואה היא ונתנה להוצאה ומידע ידע דלאו לאהדורי מיד שאיל מינה, אע״ג דאמר ליה בלשון שאלה איכא למיחש דלמא אתי למכתב, והלכך עד דאמר בלשון שאלה ועוד דגומר בעל הבית בדעתו שאם רצה הלה שלא להחזיר לו מדעתו לא יתבענו בבית דין דזו כעין מתנה היא, לא שרינן ליה.\par \textbf{} ולשון שאלה משום תרתי תקינו לה, חדא דליהוי ליה היכרא וליגמר בדעתיה, ועוד דאי בעי תבע ליה בין דידיה לדידיה מצי תבע כל אימת דבעי, ומשום הא לא אתי למיכתב, אבל בלשון הלואה איכא תרתי לריעותא, חדא דלא מינכרא ליה מלתא, ועוד שהיא לזמן מרובה ואפילו גמר בלביה דלא ליתבעיה בבי דינא אלא בין דידיה לדידיה אתי למיכתב. כך נראה לי.}
\textblock{\textbf{אי אמרת בשלמא דלא ניתנה ליתבע משום הכי מניח טליתו אצלו.} פירוש: ומיהו לא אתי למכתב כיון דמשכונו הוא בידו.}
\textblock{ ה״ג רש״י ז״ל:\textbf{ אם החדש מעובר משמט, ואי אמרת לא ניתנה ליתבע מאי משמט, שמוט ועומד הוא.} ואינו מחוור, דדלמא משמט דקאמר שצריך לומר לו משמט אני. אלא הכי גרסינן והיא גירסת הספרים, אי אמרת       ניתנה ליתבע משום הכי משמט, שלא יתבענו, ואפילו נתן לו מדעתו צריך לומר לו משמט אני, אלא אי אמרת לא ניתנה ליתבע מאי משמט, אי שלא לתובעו בבית דין שמוט ועומד הוא, ואי למימר ליה משמט אני, לא צריך דלא קרינן ביה לא יגוש וכל דלא קרינן ביה לא יגוש לא צריך למימר ליה משמט אני .}
\textblock{\textbf{שאני התם דאגלאי מלתא דחול הוא.} ומכאן נראה לי ללמוד דלכולי עלמא ביום טוב שני ניתנה ליתבע.}
\textblock{\textbf{רב אויא שקיל משכונא ורבה בר עולא מערים ערומי.} ולענין פסק הלכה: יש מי שפוסק כמ״ד ניתנה ליתבע, וגרסינן בה רבה, דהוא בר פלוגתיה דרב יוסף, והלכתא כרבה בר משדה ענין ומחצה (ב״ב קיד, ב), ואע״פ שיש לומר דלא נאמר אותו הכלל אלא במה שנחלקו בבבא בתרא, מכל מקום כל היכא דליכא ראיה לחד מינייהו דלתחזי מינה כחד מסתמא כרבה נקטינן, ומהא דרב אויא ורבא בר עולא ליכא למשמע מידי, דאינהו אחמורי מחמרי אנפשייהו, ואי נמי למיפק מידי בית דין טועין.\par \textbf{} ויש מי שגורס רבא, דהוא רבא בריה דרב יוסף בר חמא, ואפילו הכי יש פוסקין כמותו משום דבתרא היא וקיימא לן כותיה. ומכל מקום לכולי עלמא אי תפש לא מפקינן מיניה, כי האי דרבה בר עולא דמערים ושקיל מיניה, ועוד דאנן לא ניתנה ליתבע כלומר בבית דין קאמרינן הא בין דידיה לדידיה תבע, ואי תפיש תפיש, ולא עוד אלא דאי אזיל אידך תפיש מידי מיניה דמלוה כנגד מאי דתפיש הוא מיניה, מפקינן מיניה דלוה ויהבינן ליה, דהשתא לא הלואת יום טוב מגבינן אלא גזילה הוא דמפקינן מיניה, ואלא מיהו אי הדר לוה תפיש מיניה ההיא מידי גופא דתפיש מיניה מלוה לא מפקינן מיניה, דלא גזליה אלא דידיה הוא דקא שקיל.}
\textblock{ [מתני׳:]\textbf{ מונה אדם את אורחיו ואת פרפרותיו מפיו אבל לא מן הכתב.} כתבו בתוס׳ דלא אסרו מתוך הכתב אלא בקורא בפיו דומיא דמונה בפיו דקתני, אבל להסתכל בכתב ולמנות במחשבתו מותר דהא אית ליה היכרא. ומורי הרב ז״ל כתב דבין כך ובין כך אסור, דההיא ברייתא דמייתינן בשלהי שמעתין, דקתני כתב המהלך תחת הצורה ותחת הדיוקנאות אסור לקרותו בשבת, תני בה בתוספתא (פי״ח, ה״א) אסור להסתכל בו.}
\clearpage
\newsection{דף קמט}
\textblock{ [גמרא:]\textbf{ אלא איכא בינייהו דכתב אכותל ומתתאי.} קשה לי למה ליה למיהדר מאוקימתא קמייתא ולאוקמה באוקימתא אחריתי דהא תלמודא משום לתרוצי לרבה דלא ליפלגו עליה הוא דמהדר השתא בתר אוקימתא, וכיון שכן ההיא אוקימתא דלעיל שפיר אתיא, ולא הוי ליה אלא לאפוכי לסברא דידהו בלחוד ולימא אלא למ״ד שמא ימחוק איכא וכדרבה, ולמ״ד שמא יקרא גודא בשטרא לא מיחלף. ויש לומר דהא דקאמר השתא אלא דכתב אכותל ומתתאי לאו דוקא מתתאי ממש, אלא משום דמעיקרא נקטה ומידלי בדוקא קאמר הכא אפילו מתתאי, דלא תימא למ״ד גודא בשטרא לא מיחלף דוקא בדמדלאי דלא דמי לשטרא דנקיט ליה בידיה.}
\textblock{\textbf{לא לעולם דכתב אכותל ומידלי.} פירש הרב אלפסי דעל מאי בינייהו קא מהדר. ולפי דבריו נפרש שמועתינו כך: לעולם אימא לך דאיכא בינייהו דכתב ומידלי כדאוקמא מעיקרא, ולמ״ד שמא ימחוק ליכא, ולשמא יקרא נמי ליכא למיחש משום דקסבר גודא בשטרא לא מיחלף, ולמ״ד שמא יקרא איכא דקסבר גודא בשטרא מיחלף, אבל לשמא ימחוק ליכא, דאביי נמי לית ליה דרבה דאם איתא אכתי אמאי אמר שמא יקרא, והא דאמרינן הא דרבה תנאי היא, למאי דסבירא ליה לרבה קאמר, כלומר דרבה מוקי לפלוגתייהו בהכין, וכדאידך תנאי דאין רואין במראה דפליגי ודאי בדרבה, ורבה דאמר כת״ק דתרתי ברייתא. אבל רב ביבי ואביי לא שבקי ת״ק ואמרי כיחידאה, אלא סבירא ליה דלא בהכי פליגי, אלא בין ת״ק בין רבי אחא לית ליה דרבה בגודא, אלא אי מיחלף בשטרא או לא פליגי, דת״ק סבר מיחלף ור׳ אחא סבר לא מיחלף, וברייתא דקתני מונה אדם את אורחיו כמה בפנים וכמה בחוץ מכתב שעל גבי הכותל דאלמא גודא בשטרא לא מיחלף, ההיא בדחייק מיחק כדאוקימנא לה מעיקרא, ובכי הא ודאי לא מיחלף דלא דמי לשטרי כלל, אבל כי כתיב מיכתב דדמי לה קצת דהאי מיכתב      מיכתב, כתב בכתב מיחלף. זו שיטה על דרך הרב אלפסי ז״ל.\par \textbf{} ולדידי קשיא לי דאם איתא דהאי לא לעולם דכתיב אכותל ומידלי אאוקמתא קמייתא הדר, לא הוי ליה למימר לא לעולם אלא הכי הוי ליה למימר אלא לעולם. ועוד כיון דאתיא להאי אוקמתא דחייק אטבלא ואפנקס, מאי טעמא נדי מינה ושבקיה ליה והדריה לאוקימתא קמייתא כל היכא דאיכא לאחזוקיה. ועוד שזו סברא רחוקה מאד דעד השתא פשיטא לן בכולה שמעתין דגודא בשטרא לא מיחלף והשתא הדרינן דבגודא בשטרא אי מיחלף אי לא פליגי. ועוד דאנן תרתי אקשינן לעיל אההיא אוקימתא ולמ״ד שמא ימחוק ליחוש שמא יקרא כלומר דגודא בשטרא מיחלף ותו לשמא ימחוק לא חיישינן והא אנן תנן וכו׳, וכיון דהדרינן להאי אוקימתא ואתי לפרוקי מאי דאקשינן עליה מעיקרא ואמרינן ודקא קשיא לך דרבה, תנאי היא, הכי הוי ליה למימר ודקא קשיא ליה למאן דאמר שמא ימחוק, קסבר גודא בשטרא לא מיחלף. ואפילו תאמר דמשום דעד השתא נקטי׳ הא דרבה בקושטא והשתא אתי לחדותי ולדחוייה משום הכי איצטריך לפרוקי ולמימר והא דרבה תנאי היא, אבל אידך קושיא דלחוש שמא יקרא לא אצטריך לאהדורי עלה, משום דהיינו עיקר פלוגתייהו, ועוד דאדרבה בכולה שמעתין הוה נקטינן עד השתא דגודא בשטר לא מיחלף, לא היא דאדרבה כיון דמעיקרא (בהדי) [בהא] אוקימתא קשיא לן האי, והשתא עבדי עיקר פלוגתייהו בהא טפי הוי ליה לפרושי בהדיא.\par \textbf{} ונראה לי דהאי לא לעולם אברייתא דמונה אדם את אורחיו כמה בפנים קאי, ולאחזוקי אוקימתא דאוקימנא דאיכא בינייהו דחייק אטבלא והפנקס, ולמ״ד שמא ימחוק לשמא יקרא לא חיישינן דטבלא ופנקס דחייק לא מחלף בשטרא. ודקא קשיא לך עלה ברייתא דמונה אדם את אורחיו, אתי לתרוצי דהכי קאמר לעולם טבלא ופנקס לא מיחלף ודקא קשיא ליה ברייתא דקתני לא מן הכתב שעל גבי טבלא דלא משכחת לה פתרי אלא בדחייק מיחק, לא לעולם בדכתב מיכתב והיינו דטבלא ופנקס לא, דפנקס דכתב ומטלטל בכתב בעלמא מיחלף ועוד דשמא ימחוק, ודקא קשיא לך ברייתא אי הכי מאי שנא כותל מאי שנא פנקס ליחוש שמא ימחוק, בדמידלי. וברייתא דקתני אבל לא מכתב שעל גבי טבלא הוא הדין דהוה מצי לאיפלוגי בכותל עצמו בין מידלי למתתאי, אלא דאגב אורחיה קמ״ל דטבלא ופנקס בדכתב מכתב אף משום שמא יקרא איכא דמיחלף בשטרא מדשבקיה לכותל ונקיט פנקס, ונפקא מניה היכא דקרי אחריני בהדיה דליכא משום שמא ימחוק דכל חד וחד מדכר לחבריה, כדאמרינן גבי אין קורין לאור הנר בפ״ק דמכלתין (יב, א), וכדאמרינן (קמז, א) בעשרה מסתפגין באלונטית אחת בפרקין דחבית, והשתא אתיא כולה שמעתין שפיר.\par \textbf{} ומכל מקום לענין פסק הלכה נראה לי שהכל עולה לענין אחד, דכתיב בין בשטרא בין בפנקס לכולי עלמא אסור, או משום שמא יקרא או משום שמא ימחוק. ובכותל ומתתאי משום שמא ימחוק ובכותל ומדלאי נמי משום שמא יקרא כרבה, וכת״ק דמונה אדם את אורחיו ואת פרפרותיו מפיו אבל לא מן הכתב שעל גבי הכותל, וכת״ק דאין רואין במראה, ולאביי נמי ברייתא דמונה אדם את אורחיו כמה בפנים וכמה בחוץ בכתב שעל גבי הכותל בדחייק מיחק הוא, ואפשר נמי דרב ביבי כרבה סבר ליה אלא דמוקי ברייתא דמונה אדם את אורחיו כמה בפנים וכמה בחוץ כר׳ אחא, וטבלא בדחייק מיחק.\par \textbf{} ובהא פלוגתא דאביי ורב ביבי קיימא לן כאביי דאמר טבלא בשטרא מיחלף הואיל ותרווייהו מטלטלי, משום דכיון דפשטא דברייתא דמונה אדם את אורחיו כמה בפנים הכין דייקא, וסברא דתלמודא הכין אע״ג דשנינן ומוקמינן בדכתיב מיכתיב אשינויי דחיקי לא סמכינן, ולא שבקינן סברא דפשיטא ליה לתלמודא בדחייתא בעלמא. וכן פסק גם רבינו אלפסי ז״ל לפי שיטתו אע״פ שלא מן הטענה שכתבתי פסק כן.\par \textbf{} והר״ז הלוי ז״ל פירש שמועה זו בענין אחר וכתב דאביי לית ליה דרבה, והביא ראיה מהא דתנן בפרק המוצא תפילין (עירובין ק, א) הקורא בספר על האסקופה, ואוקמה אביי באסקופת כרמלית ורשות הרבים עוברת לפניה, ואקשינן עלה ופרקינן התם אלא הכא באסקופה ארוך עסקינן דאדהכי ואדהכי מידכר. ואינו נראה לי, דהא לאו אביי אמרה מדלא אמרינן אלא אמר אביי, אלא פירוקיה דאביי פרכינן ליה וגמרא פריק פירוקא אחרינא, ואפשר נמי דההוא פירוקא אתיא כרבא שלא כל הגזירות שוות בכל מקום אלא שלא גזרו משום כתבי הקודש.\par \textbf{} ולענין קריאה באגרות שלום בשבת, אסור גזירה משום קריאה בשטרי הדיוטות דהיינו שטרות של מקח וממכר, וכל שכן לדברי רש״י ז״ל שפירש (לעיל שבת קטז, ב) שטרי הדיוטות אגרות שלום.}
\textblock{\textbf{הכא במראה של מתכת עסקינן וכו׳ מפני שאדם עשוי להסיר בה נימין.} הר״ם בר יוסף פירש דעיקר גזירה משום מראה של מתכת היה, ואלא מיהו משום דידיה גזרו בכל מראה, והא דקאמר הכא בשל מתכת עסקינן אעיקר תקנתא קאי, כלומר דמשום לתא דידיה גזרו, וכההיא דאמרינן ריש פרק קמא דביצה (ב, ב) הכא ביום טוב אחר       עסקינן, ואפילו הכי שבת דעלמא ויום טוב דעלמא אסירי, אלא הכי קאמר משום גזירת יום טוב אחר השבת עסקינן. ואין כן דעת הריא״ף ז״ל, אלא דוקא במראה של מתכת אסרו, ותדע לך דלא גזרו מין אטו מין דאם כן היתה גזירה לגזירה, ותדע לך דהא רבא לא גזר מין אטו מין, דהא אמרינן (לעיל שבת יב, ב) אי אדם חשוב הוא מותר, ולא גזרינן חשוב אטו שאינו חשוב, וכן בשמש (שם) חלקו בין משחא לנפטא. ועוד הקשה עליו הרמב״ן ז״ל דאי איתא דת״ק בכל מראות אסר ור׳ לא אסר אלא בשל מתכת שאינה קבועה בלבד, הוי ליה למימר ור׳ מתיר בכל המראות חוץ משל מתכת שאינה קבועה, דהשתא דקאמר סתם ר׳ מתיר במראה הקבועה, משמע דלא נחלק אלא בקבועה בלבד הא בשאינה קבועה מודה לרבנן.\par \textbf{} ואם תאמר אין הכי נמי, אם כן אית ליה לר׳ נמי הא דרבה, דלא אמרינן אדהכי והכי מידכר, ואם כן בטלת כל דבריו, דאם כן בשל מתכת הקבועה נאסור דלא אמרינן אדהכי כו׳, ואפילו תמצי לומר דטעמא דקבועה משום דבין קבועה לשאינה קבועה לא טעו אינשי, מכל מקום נצטרך להודות דהא דרבה לאו כהני תנאי, ואנן כהני תנאי מוקמינן לה. וכן דעת מורי הרב ז״ל. ועל כן נהגו נשים להסתכל במראה של זכוכית בשבת שלא גזרו עליה כלל. והר״ז סבור כדעת הר״ם בר יוסף, ועם כל זה הראה פנים להתיר נשים בשל זכוכית.}
\textblock{ הא דאמרינן:\textbf{ מפיס אדם עם בניו כו׳.} כתב הראב״ד ז״ל דלית הלכתא הכי, אלא אפילו עם בניו אסור, ואפילו בחול משום קוביא ומאי דשרינן הכא מטעמיה דר״י אמר רב דאמר מותר אדם להלוות בניו ובני ביתו ברבית כדי להטעימן טעם רבית, הא אמרינן התם דאין הלכתא כותיה משום דלמא אתי למסרך.}
\clearpage
\newsection{דף קנ}
\textblock{ הא ד\textbf{אמר רבה בר בר חנה אמר רבי יוחנן הלכה כרבי יהושע בן קרחה.} קשיא לי דהא אוקימנא סתם מתניתין כרבי יהושע בן קרחה, ורבי יוחנן הלכה כסתם משנה אית ליה בכל מקום ואפילו תמצי לומר אמוראי נינהו, הוי ליה למימר הכי. ושמא נאמר דר״י לא מפרש למתניתין הכי ולא ידעינן היכי מפרש לה.}
\textblock{ הא ד\textbf{אמר רבי יהודה אמר שמואל מותר לאדם לומר לחבירו לכרך פלוני אני הולך למחר.} פירוש: לשם אני הולך בא עמי, דאי לא מאי קאמר, פשיטא.}
\textblock{\textbf{בשלמא קש משכחת לה במחובר.} פירוש: דלא מצי תליש משום דמזיז עפר ממקומו, ואי נמי משום דמיפה הקרקע וחייב משום חורש, כדאמרינן בריש פרק הבונה (שבת קג, א) התולש עולשין והמזרד זרדין אם ליפות את הקרקע כל שהן. ואי אפשר לפרש כאן משום עוקר דבר מגדולו, דקש יבש הוא ואינו יונק כלל מן הקרקע. וכן פירשו בתוספות.      משמע בפרק העור והרוטב (חולין קכז, ב) דאמרינן התם תאנים שצמקו באיביהן מטמא טומאת אוכלין, והתולש בשבת חייב, דוקא צמקו אבל יבשו לא מחייב, וכן מוכחא כולה שמעתא התם.}
\textblock{\textbf{משכחת לה בתבנא סריא.} כלומר: וכיון דלא מטלטל ליה משום מוקצה ואע״ג דלית ביה אלא איסורא דרבנן, אף הוא אינו זכאי באמירתו.}
\textblock{\textbf{על עסקי כלה ומת אין, על עסקי אחר לא, בשלמא אחר דומיא לכלה, משכחת לה למיגז ליה אסא.} כתבו בתוס׳, דמהא שמעינן דאפילו בדבר דאיכא איסור דאורייתא כגון למיגז ליה אסא אפילו לאחר, דוקא להחשיך על התחום אסור משום דמוכחא מלתא, אבל תוך התחום דלא מוכחא מלתא מותר. ואי קשיא לך הא דאמרינן בעירובין (לח, ב) לא יטייל אדם בתוך שדהו לראות מה היא צריכה, וכן לא יטייל לפתח מדינה כדי שיכנס מיד. התם הוא דבתוך שדה שהיא צריכה לניר ולעבוד דמינכרא מלתא, וכן מטייל לפתח מדינה שהמרחץ שם ומנכרא מלתא ואתי לידי חשדא.}
\textblock{\textbf{אמרינן המבדיל בין קודש לחול ומסלתינן סילתי.} פירש רש״י ז״ל: המבדיל בין קודש לחול להיכרא ללוות את המלך, ועבדינן צורכין ואח״כ אנו מברכין על הכוס. נראה שהוא ז״ל מפרש לה לזו בלא שם כלל, ואינו אומר אלא כך המבדיל בין קודש לחול ומיד עושה צרכיו ובלבד שהבדיל בתפילה. אבל הרי״ף פירשה בשם, ונראה שהוא ז״ל מפרש גם כן בכוס וכרב נתן בר אמי דתרגמא קמי דרב דוקא בין הגתות. ונראה לי כן ממה שכתב הרי״ף בפסחים בפרק ערבי פסחים (פסחים קד, א) דרב אשי ורב כהנא ורבא אית להו דאין צריך לומר אלא הבדלה אחת כהבדלתו של ר׳ יהודה הנשיא דאלמא זאת הבדלה הבדלה גמורה היא ובכוס, דאי לא, דלמא כי הכא דלא צריכא תלתא עד דמבדיל על הכוס. וכדברי רש״י ז״ל מסתברא, דאם איתא דרב אשי אתא לאשמועינן דעד דמבדיל על הכוס לא עבדינן צרכין, הוי ליה למימר כי הוינא בי רב כהנא אמרינן הבדלה על הכוס ואפכינן סלתין. כך נראה לי.}
\clearpage
\newsection{דף קנא}
\textblock{\textbf{אמר רב יהודה אמר שמואל מותר אדם לומר לחבירו שמור לי פירות (שבתחומי) [שבתחומך] ואני פירות (שבתחומך) [שבתחומי].} כלומר: אע״פ שהוא אינו יכול לילך שם דכיון שהוא מותר לישראל חברו לשמרן אין באמירתו כלום. וכתבו בתוס׳ דמהא שמעינן דישראל שקבל עליו שבת קודם שחשכה מותר לומר לישראל חבירו לעשות לו מלאכה פלונית, הואיל והיא נעשית בהיתר לעושה אותה. ודוקא לישראל חבירו מותר ואע״פ      אינו יכול לעשות, אבל לעכו״ם אסור לומר לו שמור לי פירותי שחוץ לתחום דכל דבר שאינו עושה אינו אומר לעכו״ם ועושה, ומסתברא באומר לו לעכו״ם לך ושמור לי פירותי שבמקום פלוני בדוקא, אבל אם העכו״ם כבר באותו תחום, כל שכן שמותר לומר לעכו״ם, דלא אמרו כל שאינו עושה אינו אומר לעכו״ם להחמיר עליו יותר מישראל חבירו.}
\textblock{ הא דתניא:\textbf{ אין מחשיכין על התחום להביא בהמה.} קשיא לי למה אינו מחשיך שהרי אם יש שם בורגנין מביא, ותנן אבל מחשיך הוא לשמור ומביא פירות, ואבא שאול נמי לא נחלק בה דאין מחשיכין להביא בהמה כלל, ודברי הכל היא, והיאך לא הקשו ממנה לר׳ יהודה אמר שמואל לעיל. ומקצת נוסחאות מצאתי, מחשיכין על התחום, והיא נראית עיקר. וכן מצאתי גם בתוספתא (פי״ח, ה״א), ולפי גירסת הספרים צריך לי עיון.}
\textblock{ הא ד\textbf{אמר אבא שאול אומר לו לך במקום פלוני ואם לא מצאת במקום פלוני הבא ממקום פלוני.} פירוש: לך למחר והביא למחר, אבל היום אסור דהא תלמודא אוסר אפילו הלך מעצמו והביא, כדתנן (לקמן): עכו״ם שהביא חלילין בשבת לא יספוד בהן ישראל.}
\textblock{ מתני׳:\textbf{ עכו״ם שהביא חלילין בשבת לא יספוד בהן ישראל.} פירש רש״י ז״ל: לא יספוד בהן לעולם, ומשום קנסא. ואינו מחוור, דאם איתא ליתני נמי בהא לא יספוד בהן עולמית וכדקתני בסיפא לא יקבר בו עולמית. ועוד דאי משום דנעשית בהן מלאכה דאורייתא קנסינן ואסרינן לעולם, אפילו כשהביאן ממקום קרוב יאסרו לעולם, שהרי נעשית בהן מלאכה דאורייתא שהכניסן מרשות הרבים לרשות היחיד.}
\textblock{\textbf{עשו לו ארון וכו׳ ואם בשביל ישראל לא יקבר בו עולמית.} ואוקימנא בגמרא דוקא בקבר העומד באסרטיא וארון העומד על הקבר דמקום פרהסיא וגנאי הוא למת שנקבר בקבר שנעשית בו מלאכה דאורייתא מחמתו באיסור, אבל כשאינו עומד באסרטיא מותר בכדי שיעשה.\par \textbf{} ויש מקצת מרבוותא שאמרו דוקא אותו ישראל הוא דלא יקבר בו עולמית בשעומד באסרטיא, ובשאינו עומד באסרטיא הוא שצריך להמתין בכדי שיעשה, הא לאחר מותר ליקבר בו מיד, שאין האיסור אלא כדי שלא יאמר לעכו״ם לעשות, הא ישראל אחר שלא נעשית מלאכה בשבילו מותר. והביאו ראיה ממה ששנו כאן בתוספתא (פי״ח , ה״ח): עכו״ם שהביא חלילין בשביל ישראל בשבת לא יספוד בהן אותו ישראל אבל ישראל אחר מותר, עשו לו ארון וחפרו לו את הקבר לא יקבר בהן אותו ישראל ולישראל אחר מותר. ע״כ בתוספתא. ומדלא קתני לא יקבר בו עולמית ועוד דקתני עכו״ם שהביא חלילין לא יספוד בהן אותו ישראל, משמע דלא יקבר ולא יספוד עד כדי שיעשו קאמר וכשאינו עומד באסרטיא, ואפילו הכי קתני אבל ישראל אחר מותר, דאלמא לא בעינן כדי שיעשו אלא לאותו ישראל ממש שנעשית מלאכה בשבילו.\par \textbf{} והא דאמר רב פפא בפרק אין צדין (ביצה כד, ב) עכו״ם שהביא דורון לישראל אם יש מאותו המין במחובר אסור ולערב מותרין בכדי שיעשו, ואם אין מאותו המין מחובר בתוך התחום מותרין חוץ לתחום הבא בשביל ישראל זה מותר בשביל ישראל אחר, אם יש מאותו המין במחובר לערב מותרין בכדי שיעשו דקאמר לאותו ישראל שהובאו בשבילו קאמר, אבל לאחר מותר מיד דכל שבא מחמת זה או שנעשה מחמת זה מותר לישראל אחר. והא דלא קאמר בה והבא בשביל ישראל זה מותר לישראל אחר כדקאמר בסיפא, היינו משום דביומן אסורין לכולי עלמא משום דאסורין משום מוקצה וכל שהוא אסור משום מוקצה אסור לזה כמו לזה שהובאו בשבילו, וכיון דביומא לא מצי פליג בין הבא בשבילו לישראל אחר, לא ניחא ליה למימר ולאיפלוגי בערב בכדי שיעשו לישראל זה ולישראל אחר מותר מיד, משום דהא דפליג בסיפא גבי תחומין היינו להתירו אפילו ביומו לישראל אחר. והא דתנן (לעיל שבת קכב, א) עכו״ם שהדליק את הנר משתמש לאורו ישראל ואם בשביל ישראל אסור, לאותו ישראל שבאותה מסיבה קאמר הא ישראל אחר מותר.}
\textblock{\textbf{וכן דעת הראב״ד בעירובין פרק מי שהוציאוהו בשמעתא דהשואל כלי מהעכו״ם (מז, ב), והביא ראיה מן המבשל בשבת דאמר ר״מ בשוגג יאכל במזיד לא יאכל, והוא דלא יאכל הא אחר יאכל, ואיהו ז״ל סבור דהלכתא כר״מ דהא כי מורה להו רב לתלמידיה מורי להו כר״מ (חולין טו, א), ועוד דהא רבי חייא בר אבא דריש להו בפרקא כר״מ (לעיל שבת לח, א), ורבה ורב יוסף ורב נחמן דבתראי נינהו כרבי חייא בר אבא סבירא להו ופרשי שמעתיה, וכיון שכן לא      } עדיף חיוב חטאת שנעשה על ידי עכו״ם בשביל ישראל מחיוב חטאת שנעשה במזיד על ידי ישראל. אלו הן דברי הרב ז״ל שכתב שם בעירובין.\par \textbf{} ולפי דבריהם הכי מפרשינן לה למתניתין דהכא, עכו״ם שהביא חלילין לא יספוד בהן שום ישראל למוצאי שבת עד שישהה כדי שיבואו מאותו מקום רחוק שדרכן להביאן משם, אלא אם כן באו ממקום קרוב וטעמא שלא הובאו בשביל ישראל ידוע, אלא מפני שאין דרכן של עכו״ם לספוד בחלילין סתמא דמלתא בשביל ישראל הובאו, ולא בשביל ישראל ידוע שאילו כן אותו ישראל היה אסור אבל לישראל אחר מותר דהבא בשביל ישראל זה וכו׳, אבל הכא דלא הובאו בשביל ישראל ידוע כל ישראל אסורין בו. ומיהו כיון שלא הביאן בפירוש בשביל ישראל מותרין לערב בכדי שיבואו ממקום רחוק ואעפ״י שהביאן בפרהסיא דומיא דחפרו את הקבר דאוקימנא באסרטיא, ואם באו ממקום קרוב מותרין כלומר ואפילו לאותו ישראל עצמו מותרין מיד, כלומר כשיעור שיבאו מאותו מקום קרוב. והא דאמרינן התם אם אין באותו המין במחובר לקרקע בתוך התחום מותרין, ביום טוב דוקא קאמר אבל בשבת אסורין ולערב בכדי שיבאו משם כיון שהביאן דרך רשות הרבים או בכרמלית. עשו לו ארון וחפרו (בו) [לו] את הקבר. כלומר: בשעשו ארון סתם וחפרו קבר סתם, סתמא דמלתא בשביל עכו״ם הוא שנעשו ולפיכך קובר בו ישראל למוצאי שבת מיד ממש בלתי שימתין כלל, ואם בשביל ישראל בפירוש לא יקבר בו אותו ישראל לעולם, ובשעומד באסרטיא כדאוקימנא לה בגמרא, הא בשאינו עומד באסרטיא מותר אפילו לאותו ישראל בכדי שיעשו אבל לישראל אחר מותר מיד, וכדתניא בתוספתא אבל ישראל אחר מותר. ואילו היה ראוי לקברו בשבת, אפילו בשבת היה מותר לקברו שם, דמה שנעשה בשביל ישראל זה מותר לאחר ולא דמי לפירות בזמן שיש במינן במחובר דאסורין לכל ביומן, דהתם היינו טעמא משום דהוו להו מוקצין לכל, ואפילו נשרו מעצמן או שנתלשו בשביל עכו״ם ממש אסרו והחמירו אפילו בספיקתן, וכדקיימא לן גבי ביצה שנולדה ביום טוב דאמרינן (ביצה ג, ב) וספיקא אסורה, משום דהוי ליה דבר שיש לו מתירין.\par \textbf{} ויש מפרשים דכל דבר שנעשית מלאכה בשבת בשביל ישראל אסור לכל ישראל עד כדי שיעשו, דההיא דאמר רב פפא אם יש באותו המין במחובר אסור ולערב בכדי שיעשו, לכל קאמר, דכי היכי דאסור דקאמר אסורין לכל קאמר, הא דאמר נמי ולערב אסור בכדי שיעשו אותן שאסורין להן ביומן קאמר. וכן כתב רש״י ז״ל בביצה פרק אין צדין (ביצה כה, א ד״ה חוץ לתחום) בההיא דרב פפא. והוא ז״ל כתב דטעמא משום דהן מוקצה ואפילו רבי שמעון מודה במוקצה כי האי דכגרוגרות וצימוקין דמי. ואין טעמו מחוור, דכיון דאסורין לערב בכדי שיעשו כבר נתברר דאין זה משום מוקצה ולא משום נולד שאילו כן היה לערב מותרין מיד דלא מצינו מוקצה שנאסר לערב, אלא טעמא משום שנעשית בהן מלאכה דאורייתא בשביל ישראל הוא.\par \textbf{} והיכא דליקט עכו״ם לעצמו דאסורין לישראל ביומן, וכדאיתא בעירובין (מ, א) גבי ההוא לפתא דאתא למחוזא אע״ג דלא נעשה בהן איסור שהרי העכו״ם לעצמו הוא עושה, התם טעמא משום דגזרינן בהו משום שמא יעלה ויתלוש, דאפילו בפירות הנושרין מעצמן גזרו. וכן נמי מתניתין דמצודות חיה ועוף שבפרק אין צדין (ביצה כד, א) שאסרו אפילו בשאין צריכין צידה, היינו טעמא נמי משום דגזרינן שמא יבואו לידי צידה, דעל כרחך אינו משום איסור מלאכה שנעשה בהן, דהתנן בפרק קמא דשבת (יז, ב) אין פורסין מצודות חיה ועופות ודגים אלא כדי שיצודו מבעוד יום ובית הלל מתירין.\par \textbf{} ומה שהתירו כשאין במינן במחובר אפילו ביומן לאחר, היינו דוקא באיסור תחומין, והקילו בהן לפי שאין איסורן שוה דלזה שהוא חוץ לתחום אסור ולזה שהוא בתוך התחום מותרין, וכיון שמותרין אפילו בתחומן לישראל זה התירו בהן אפילו למי שלא היה בתחומן והוא שלא בא בשבילו. וההיא דתניא בתוספתא אם בשביל ישראל לא יספוד בהן אותו ישראל אבל לישראל אחר מותר, לאחר שהמתין כדי שיבואו ממקומן קאמר, ובשהביאן בפרהסיא קאמר דלאותו ישראל שהובאו בפרסום בשבילו אסורין כדאמרינן בגמרא בקבר העומד באסרטיא, אבל לישראל אחר מותר לאחר כדי שיבאו ממקום רחוק, וכן בקבר, הא כדי שיעשו אסור לכל.\par \textbf{} ומכל מקום למדנו לדברי כולם דהא דקתני הכא לא יקבר בו עולמית, דוקא אותו ישראל שנעשה בשבילו אבל לישראל אחר מותר והוא שימתין בכדי שיעשו.}
\textblock{ גמרא:\textbf{ מאי מקום קרוב רב אמר מקום קרוב ממש ושמואל אמר חיישינן שמא חוץ לחומה לנו.} פירש רש״י ז״ל דשמואל לקולא, ואשכחן חוששין לקולא כאותה שאמרו בחגיגה (טו, א) חוששין שמא באמבטי עיברה, וכן חוששין משום זנות ואין חוששין משום קדושין (גיטין עג, ב). והכי פירושא: רב אמר מקום קרוב ממש, דאז מותר מיד כלומר וכשיעור שיבאו ממקום קרוב, אבל אם הדבר ספק אסור עד שימתין בכדי שיבואו ממקום רחוק, ושמואל אמר       לא נודע לו בבירור תולין הדבר להקל ואני אומר לא באו בשבת אלא מחוץ לחומה, שאני אומר חוץ לחומה לנו ומערב שבת הובאו שם, ומתניתין דקתני לא יספוד בהן אלא אם כן באו ממקום קרוב, הכי קאמר לא יספוד בהן ישראל אלא אם כן יש לתלות שבאו ממקום קרוב, ולאפוקי שאם נודע שבאו ממקום רחוק. ואולי הוצרך לומר כן ללמד שאפילו יש חלילין במקום קרוב שהיה יכול להביא משם וזה לא הביא אלא ממקום רחוק, אין אומרים לא נהנה זה אלא בכדי שיבואו ממקום קרוב שהרי מצוין לו במקום קרוב, ולפיכך לא ימתין אלא בכדי שיבואו ממקום קרוב כדי שלא יהנה בביאתן בשבת הא יותר מכאן לא שהרי לא נהנה, אפילו כן כיון שאלו באו ממקום רחוק לא יספוד בהן עד שימתין כדי שיבואו משם.}
\textblock{\textbf{ דיקא מתניתין כותיה.} יש גירסאות הרבה בפירוש רש״י, והגירסא הנכונה כך היא: אמר רבא דיקא מתניתין כותיה דשמואל, דתנן עיר שישראל ועכו״ם דרין בתוכה וכו׳, ומשנה היא במסכת מכשירין (פ״ב, מ״ה) וסייעתיה דשמואל מדרבי יהודה דאמר באמבטי קטנה אם יש שם רשות מותר לרחוץ בה מיד שאני אומר משחשכה הוחמו, דאלמא תולין להקל בהמתנה של ערב, ולא פליג ת״ק עליה דרבי יהודה. וברישא נמי דקתני מחצה על מחצה אסור לא פליג בה רבי יהודה, ולאו תיובתא דשמואל היא, דכיון דמרחץ גדולה היא כי מחממי לה אדעתא דכלהו מחממי לה והלכך צריך שימתין בכדי שיחמו חמין, והא דקתני באמבטי קטנה רוחץ בה מיד, לאו מיד ממש קאמר אלא בהמתנת זמן מועט שתוכל להתחמם ע״י רוב בני אדם ביחד.\par \textbf{} ואי קשיא לך ההיא דרב פפא דאמר אם יש מאותו המין במחובר לקרקע אסור ולערב בכדי שיעשו, דאלמא אפילו בהמתנת הערב לא תלינן להקל ואע״ג דאיכא למיתלי מערב יום טוב ואי נמי בשביל עכו״ם נתלשו. כבר תירץ מורי הרב ז״ל דכל שיש מאותו המין במחובר דרכן של בני אדם ללקט אותן ליומן ואין לוקטין מבערב להצניען למחרתו, וכיון שרובן עושין כן אין תולין בהן להקל.\par \textbf{} ולדידי קשיא לי, דהא משמע דעכו״ם שהביא דורון לישראל ואין מאותו המין מחובר לקרקע דאמרינן בתוך התחום מותרין, דוקא בשידוע שלא הביאן מחוץ לתחום, הא סתמא אין אומרים שמא חוץ לחומה לנו אלא חוששין שמא מחוץ לתחום באו, וכדאמרינן בעירובין [פרק] בכל מערבין (מ, א) גבי ליפתא דאתי למחוזא וחזיא רבא דכמישא ואמר הא ודאי מאתמול עקירא, מאי אמרת מחוץ לתחום אתאי הבא בשביל ישראל זה מותר לישראל אחר, וכל שכן האי דאדעתא דעכו״ם אתאי, דאלמא משמע דבסתמא חוששין שמא מחוץ לתחום אתאי, ורבא דקא מייתי הכא סייעתא לשמואל הוא דקאמר לה התם, וכיון שכן היאך אפשר דביומה מחזקינן ליה בסתמא דמחוץ לתחום אתאי והוא גופיה שרי לערב מיד משום דמחזקינן ליה דמתוך התחום אתאי. ומצאתי לרבותינו הצרפתים ז״ל במסכת עירובין (מ, א) דכל שהביא עכו״ם לישראל ביום טוב, מסתמא אנו מחזיקין אותו שמתוך התחום הביאו, ונסמכו על זו דשמואל דהכא, וכפירושו של רש״י ז״ל. ושמא הם סבורים דהא דאמר רבא התם בליפתא מאי אמרת חוץ לתחום אתאי דלרוחא דמלתא אומר כן, כלומר אפילו לרב דחייש לשמא ממקום רחוק באו כדאיתא הכא בההיא לא חיישינן, דהבא בשביל ישראל זה מותר לישראל אחר וכל שכן דרובן עכו״ם הוו ומסתמא מחמת עכו״ם הביאוה.\par \textbf{} ומיהו נראה דלא נחלקו רב ושמואל אלא במביא חלילין שדרכן להביאן חוץ לתחום, ואי נמי בעכו״ם שאינו שרוי עמו כגון ההיא עובדא דליפתא, דבכל הני איכא למיחש לשמא מחוץ לתחום הביאום, אבל בעכו״ם ששרוי עמו בעיר ופירות המצויין בעיר מפני מה ניחוש בהן לשמא מחוץ לתחום הביאום, אדרבה נימא כאן נמצאו וכאן היו, ועוד דספק בשל דבריהם הוא ולקולא, וכן מצאתי להראב״ד בעירובין (מז, ב) בשמעתא דנכסי העכו״ם קונין שביתה גבי השואל כלי מן העכו״ם ביום טוב.}
\textblock{\textbf{והגאונים ז״ל פירשו דשמואל להחמיר. והכי פירושא: רב אמר ממקום קרוב ממש, דכיון שראינוהו היום מביאן מתוך ביתו או ממקום קרוב די לו בכך, ושמואל אמר לעולם אסור עד שנדע בבירור שלא באו בשבת ממקום רחוק, הא בספק חוששין שמא בלילה הביאום ממקום רחוק אלא שלנו חוץ לחומה ועכשיו הוא שמביאן לתוך לחומה, וסייעיה רבא לשמואל מדתנן מחצה על מחצה אסור ולא תלינן להקל לומר      } דבשביל העכו״ם הדרין שם הוחמו, ואי נמי יש לפרש מדקתני מרחץ המרחצת בשבת אם רוב ישראל אסור קא מייתי ראיה, מדלא תלינן מאמש הוחמו חמיו ופקקו נקביו הוה. וכן פירש הרמב״ן ז״ל. והא דרבי יהודה באמבטי קטנה משום דאפשר שיחמו בעשרה בני אדם לאחר שחשכה, והרי המתין זה כדי שיחמו חמיו ומותר לכולי עלמא. ומורי הרב ז״ל הקשה לפירוש זה, שאם כן לא היה ליה לשמואל למימר חוששין שמא חוץ לחומה לנו שאין זה עיקר החשש, אלא כך היה ליה למימר שמא ממקום רחוק באו הלילה, ועיקר החשש היה לו להזכיר.}
\textblock{\textbf{ואמאי הכא נמי ימתין בכדי שיעשה.} פירש רש״י דארישא דמתניתין קאי דקתני עשו לו ארון וחפרו לו את הקבר יקבר בו ישראל כלומר מיד ממש, ואמאי ימתין בכדי שיעשו, דאף על גב דבשבילו עושה מן הסתם אינו ברור שהוא ודאי כן דאיכא לספוקי דלמא בשביל ישראל וימתין בכדי שיעשו, ופריק בעומד באסרטיא שניכר בודאי דמחמת עכו״ם נעשה, וכן ארון שעומד על קברו דעכו״ם, אבל סיפא דקתני אם בשביל ישראל לא יקבר בו עולמית אפילו אינו עומד באסרטיא קאמר. וכן נראה גם מדברי הרב אלפסי ז״ל שלא הביא בהלכות הא דאסרטיא אלא מתניתין לבדה כצורתה. ואינו מחוור, אלא אסיפא דקתני אם בשביל ישראל לא יקבר בו עולמית קאי, ופריק בעומד באסרטיא וגנאי הוא לו לישראל ליקבר בקבר שנתחלל שבת בשבילו. וכן פירש רש״י ז״ל משם הגאונים ז״ל.}
\clearpage
\newsection{דף קנג}
\textblock{ [מתני׳]:\textbf{ אין עמו נכרי מניחו על החמור.} ואוקימנא בגמרא כשהיא עומדת נוטלו הימנה וכשהיא מהלכת נותנו עליה, הא לאו הכי אסור משום מחמר, וכאן התירו לו טלטול משום גזירה דאדם בהול על ממונו ואי לא שרית ליה אתי לאתויי, וכשהגיע לחצר החיצונה חוזר לאיסורו ואין מטלטל אלא כלים הראויין לטלטל. וכתבו בתוס׳ בשם ר״ת ז״ל וכן בספר התרומות (סי׳ רכו) מכאן יש ללמוד למי שמתירא מן הלסטים או מן השלטון שהוא מותר לטלטל את המעות כדי שיחביאן, דמשום הפסד התירו לו לטלטלו כמו שהתירו כאן.\par \textbf{} אבל הרמב״ן ז״ל חולק בדבר, ואומר שלא התירו כאן אלא דוקא במה שהוא בידו, משום דקים להו לרבנן דאין אדם מעמיד עצמו על ממונו לזורקו מידו, אבל מה שאינו בידו לא התירו לו טלטולו משום הפסד ממונו. ותדע לך שהרי בדליקה לא התירו לו אפילו אמירה לעכו״ם, ואדרבה (לעיל שבת קכב, א) עכו״ם שבא לכבות אין אומרים לו כבה ואל תכבה, ותנן (לעיל שבת קטז, ב) מטלטלין תיק הספר עם הספר ואע״פ שיש בתוכו מעות, ומשום הצלת כתבי הקודש דאי אמרת לישדינהו אדהכי והכי נפלה דליקה כדאיתא התם בפרק כל כתבי. ועוד דאמרינן לעיל בפרק נוטל (שבת קמב, ב) פעם אחת שכחו ארנקי מלאה מעות בסרטיא, ואמר להם ר׳ יוחנן הניחו עליה ככר או תינוק וטלטלוה, ואסיק רב אשי התם דלא אמרו ככר או תינוק אלא למת בלבד, וקיימא לן כרב אשי דאסר לטלטלו, והכא קיימא לן במי שהחשיך לו בדרך שמטלטלו פחות פחות מארבע אמות, ואפילו ר׳ יוחנן לא אמר התם לטלטלו פחות פחות מארבע, אף על פי שרש״י ז״ל כן פירש שם, אלא במחיצה של בני אדם קאמר, דאי בפחות פחות מארבע מפני מה הוסיפו בהעברה להעביר הככר ללא צורך פחות פחות מארבע אמות שהוא אסור לכתחילה.}
\textblock{\textbf{ועוד כתב הרב ז״ל דדוקא למי שהחשיך לו בדרך, הא שוכח ויוצא בכיסו לדרך אסור. ומקצת דברי הרב ז״ל אינן מחוורין בעיני, לפי שאני סבור דבין החשיך לו בדרך בין שכח ויצא שלא מדעת בכיסו בין שכיסו בידו בין שהיה לפניו הכל מותר, דהא טעם תלמודא לפי שאין אדם מעמיד עצמו על ממונו הוא, ומה לי בידו מה לי בפניו והוא רואה אותו       } אבד הכל אחד. ותדע לך שהרי אמרו בגמרא לא שנו אלא כיסו אבל מציאה לא, וכן לא אמרן אלא דלא אתאי לידיה אבל אתאי לידיה ככיסיה דמי, דאלמא לא אסרו מציאה אלא בדלא אתי לידיה, ומינה דכיסיה בכענין זה דלא תפיס ליה בידיה שרי, וגם ביוצא בשוכח הוא הדין והוא הטעם, דאי לא שרית לי אתי׳ לאתויי. וגם מה שהביא הרב מדליקה שלא התירו בה אמירה לעכו״ם משום הפסד ממונו, אינה ראיה גמורה בעיני, דדלמא התם טעמא אחרינא משום דדליקה דולקת והולכת ואי שרית ליה איכא למיחש דלמא משתלי מתוך שהוא בהול על ממונו ואתי לכבויי עד שלא תגיע שם דליקה, אבל משום דלמא מציל ברשות הרבים ליכא. ותדע לך דהתם כל דאיכא למיחש לרשות הרבים אמרו כן, וכמו שאמרו בגמרא (לעיל שבת קיז, א) במציל למבוי המפולש דאי שרית ליה במבוי המפולש אתי לאתויי ברשות הרבים, אבל במציל למבוי שאינו מפולש או לחצר המעורבת לא חיישינן דלמא מציל ברשות הרבים אלא דלמא אתי לכבויי ומן הטעם שאמרנו, והלכך במתיירא מפני לסטים או שלטון כשאתה בא להתיר לו לטלטל ולהצניעו בביתו אי נמי בחצר המעורבת למה אתה חושש לו שיטלטל משום הפסד ממונו ומה בכך.\par \textbf{} וההיא דפרק נוטל (שבת קמב, ב) דמשמע ודאי שאוסר לטלטל פחות פחות מד׳ אמות ולפי המסקנא אפילו במחיצה של בני אדם אסור מפני איסור טלטול כרב אשי. התם איכא למימר משום דבעיר הוא ואפשר לו לישב ולשמור עד הערב, אבל בדרך שהוא בחזקת סכנה אי נמי אפילו בבית במתיירא מפני לסטים אני אומר שהוא מותר. ואי קשיא לך דהא נשברה לו חבית בראש גגו (לעיל שבת קיז, ב) שלא התיר[ו] לו להציל בשני כלים אלא מזון שלש סעודות וטעמא משום גזירה דשמא יביא כלים דרך רשות הרבים כדי להציל. התם נמי מפני שהוא אזיל לאיבוד ופעמים שאין עמו כלים רקים ואי שרית ליה אתי לאתויי כלים מר״ה ומציל, מה שאין כן לחוש כאן.\par \textbf{} ואלא מיהו אין דינם של רבותינו הצרפתים ז״ל נראה בעיני, דהכא לאו משום הפסד התירו לו לטלטלו ולהעביר פחות פחות מארבע אמות, אלא מפני שהוא במקום שאם לא התירו לו כן אינו מעמיד עצמו על ממונו ואתי לידי איסורא דאורייתא. ותדע לך, שהרי שנינו הגיע לחצר החיצונה נוטל כלים הנטלין בשבת ושאינן נטלין בשבת מתיר חבלים והשקין נופלין מאיליהן, ואמרינן בגמרא דאפילו בשליפא רברבי דקרני דאומנא דההפסד מרובה לא חששו ומאבדן ביד ואע״פ שעד עכשיו התרתו בטלטולן בעודו בדרך, דאלמא שמעינן מינה בהדיא דלא התירו לו כלל משום הפסד ממונו אלא מפני שהוא במקום שקרוב לבא בו לידי איסורא דאורייתא, אבל במי שמתירא מפני לסטים מאי איכא, אי לא שרית ליה אתי לטלטולי, אטו כדי שלא יבא לידי טלטול מי שרינן ליה טלטול, הלכך לא שרינן ליה כלל. כן נראה לי.}
\textblock{\textbf{איכא דאמרי בעי רבא מציאה הבא לידו מהו.} וסלקא בתיקו. ונראה דאע״ג דהוו תרי לישני, בדרבנן הוא, וכלישנא קמא נקטינן לקולא דהיכא דאתי לידיה מערב שבת אע״ג דלא טרח ביה ככיסיה דמי ושרי, ועוד דאפילו להאי לישנא בתרא בתיקו הוא דסלקא ושל דבריהם להקל, אלא היכא דלא אתיא לידיה אסור. ומשמע מפשוטן של דברים דמציאה דלא אתיא לידיה בכל ענין אסור, דהא נתינה לנכרי קיל מכולהו כדאיתא בסמוך, וכיון דאפילו לנכרי לא שרו ליה רבנן כל שכן על גבי חמור וחרש שוטה וקטן, וכל שכן להוליכה פחות פחות מארבע אמות שלא רצו לגלות אפילו בכיסו. וכן פסק הר״ח והראב״ד ז״ל.\par \textbf{} אבל הרמב״ם ז״ל (פ״ו, מה׳ שבת הכ״ב) כתב: במה דברים אמורים בכיסו, אבל במציאה לא יתן לעכו״ם אלא יוליכנה פחות פחות מארבע אמות. וכתב הרמב״ן ז״ל טעם לדבריו לפי שבכיסו הוא בהול על ממונו נחפז ללכת ושמא יוליכנו ארבע אמות, לפיכך לא התירו לו לכתחילה להוליכה פחות פחות מארבע אמות ולאסור לגמרי לא רצו שאין אדם מעמיד עצמו על ממונו, אבל גבי מציאה אסרו לו לתת לנכרי (ד)לפי שאדם מעמיד עצמו עליה, והתירו להוליכה פחות פחות מארבע אמות דלא אושא מלתא כולי האי, וכדרך שהתירו במוצא תפילין (עירובין צז, א) להכניסן פחות פחות מארבע אמות. ואינו נכון כלל דכיון שאמרו כאן בגמרא אבל מציאה לא, אם איתא דהולכה פחות פחות מארבע אמות שרינן בה לא הוו שתקי מינה ומכל מקום הוו אמרין אבל מוליכה פחות פחות מד׳ אמות, דלשון חכמים עושר.\par \textbf{} גם הרמב״ן ז״ל כתב שלא מצינו שהתירו זה במציאה אלא בתפילין, ופליגי בה רבי שמעון ורבנן, והטעם עצמו אינו נכון, וגם הראב״ד ז״ל השיג עליו בזה בהשגות שלו.}
\textblock{ תוספתא (פרק י״ח, הלכה ב׳):\textbf{ מי שהחשיך בדרך נותן כיסו לעכו״ם, אין עמו עכו״ם מניחו על החמור, הגיע לחצר החיצונה נוטל את הכלים הנטלין וכו׳, } במה דברים אמורים בעכו״ם שהוא מכירו, אבל בעכו״ם שאינו מכירו מהלך אחריו עד שהוא מגיע לביתו.}
\textblock{\textbf{מאי טעמא חמור אתה מצווה על שביתתו.} פירוש: לא שיהא בזה משום שביתה, דהא תנינן בסמוך שכשהיא מהלכת מניחו עליה וכשהיא עומדת נוטלו ממנה, אלא דשייכא בשביתה קאמר.}
\newchap{פרק \hebrewnumeral{24} מי שהחשיך}
\textblock{}
\textblock{\textbf{איכא דאמרי לחרש יהיב ליה ואיכא דאמרי לקטן יהיב ליה.} והלכך בשל דבריהם הוא, וכיון דלא אפשיטא יהיב למאן דבעי מינייהו. וקשיא לי דהא כיון דבגדול אסור וכדאמרינן בסמוך כל שבגופו חייב חטאת בחבירו פטור אבל אסור, אם כן היכי יהיב ליה לתינוק דאף ע״פ שאין בית דין מצווין להפרישו, למיספא ליה בידים אסור דקרינן ביה לא תאכלום לא תאכילום. ויש לומר דשאני הכא דאפילו בגדול ליכא אלא איסורא דרבנן ובשל דבריהם אפילו לכתחילה ספינן ליה, וכמו שכתבתי אני בפרק חרש ביבמות (קיג, ב) ובגיטין בפרק הניזקין (גיטין נה, א) בסייעתא דשמיא, ומכאן ראיה לדברי. ואף על גב דמשום דלא ליתי למיסרך אסור בכל מידי דקא עביד מחמת גדול וכדכתבינן התם, הכא במקום דאיכא למיגזר דאי לא שרית ליה אתי איהו למיעבד איסורא בידים שרינן ליה ולא חיישינן בכי הא לדלמא אתי למיסרך. כך נראה לי.}
\textblock{\textbf{והלא מחמר.} הוא הדין דהוי ליה לאקשויי והלא מצווה על שביתת בהמתו מדאורייתא, אלא דלמא מוקי לה בחמור שאינו שלו ושאלה ושכירות לא קנו. מיהו למסקנא כיון דלא עבדא עקירה והנחה, ליכא נמי משום שביתת בהמתו דבגופו נמי פטור. הרמב״ן ז״ל.}
\textblock{\textbf{כשהיא עומדת נוטלו הימנה.} הרמב״ם ז״ל (פרק כ מהל׳ שבת ה״ו) פירשה (כשהוא) [כשהיא] רוצה לעמוד קודם שתעמוד כדי שלא תהא שם לא עקירה ולא הנחה. ואינו נראה כן.}
\textblock{\textbf{היתה חבילתו מונחת לו על כתפו רץ תחתיה עד שמגיע לביתו.} והא דלא אמרינן הכי בכיס. כתב הרמב״ן ז״ל משום דבחבילה שהוא משוי כבד מידכר ולא אתי למיעבד עקירה והנחה, אבל בכיס שהוא דבר מועט וקל לא דכיר וחוששין דלמא נח ואתי למיעבד עקירה והנחה.}
\textblock{\textbf{כשהיא מהלכת נותנו עליה והיא עומדת נוטלו ממנה.} והוא הדין כשהוא נותנו לחרש שוטה וקטן, וכל שכן היא דאילו נותנו להן כשהן עומדין הא איתא בגדול בכיוצא בזה חיוב חטאת, וכל שבגופו איכא איסורא דאורייתא אי אמרינן לקטן למיעבד איכא נמי איסורא דאורייתא, דכתיב (ויקרא יא, מב) לא תאכלום קרינן ביה לא תאכילום, ואמרינן ביבמות פרק חרש (קיד, א) לא יאמר אדם לתינוק הבא לי חותם הבא לי מפתח אבל מניחו תולש מניחו זורק, ולעיל (שבת קכא, א) נמי תניא קטן שבא לכבות אין שומעין לו, ואוקימנא בגמרא בקטן העושה על דעת אביו, כלומר ועובר בזה משום לא תעשה [כל] מלאכה אתה ובנך (שמות כ, ט).}
\textblock{\textbf{כשהיא מהלכת נותנו עליה.} כתב הרמב״ם (שם): ואסור לו להנהיגה אפילו בקול כ״ז שהכיס עליה, כדי שלא יהא מחמר בשבת. ואינו מחוור בעיני כלל, שהרי כיון שאינו מניח עליה עד שתהא מהלכת אינו מחמר, ועוד דהא אמרינן הכא דכל שבחבירו פטור אבל אסור בחמורו מותר לכתחילה.}
\textblock{\textbf{כי קאמר רב הונא בקרנא דאומנא דלא חזיין ליה.} קשיא לי קרנא דאומנא נמי חזי לכסויי ביה מנא       ליה כשרגא דנפטא דשרינן אליבא דר״ש שלהי פרק כירה (שבת מו, א), ורב הונא במוקצה לטלטול כר״ש סבירא ליה כדאמרינן בפרק מפנין (שבת קכח, א). ויש לומר דקרנא דאומנא מאיס טפי משרגא דנפטא ולא חזי לכסויי ביה מנא כלל. ואי נמי משום דאי נפיל מיתבר לא עביד לכסויי ביה מנא.}
\textblock{\textbf{והא מבטל כלי מהיכנו, בשליפי זוטרי.} פירוש: אבל שליפי רברבי לא משום דאי שמיט כרים מתחתיהם מתוך כובדן מיתברי, והלכך אי אפשר לנערן ולסלקן עד לערב, והוי ליה מבטל כלי מהיכנו לכולי שבת ואסור. ואם תאמר שליפי רברבי נמי יביא כרים ויניח תחתיהן, ואי משום ביטול כלי מהיכנו לכשיצטרך לכרים יסלק שליפין ביד דהא כלי שמלאכתו לאיסור מותר לטלטלו לצורך מקומו וכדאמר רבא בפרק כל הכלים (שבת קכד, א), ורב דהוה רביה דרב הונא הכי סבירא ליה כדאיתא בפרק כל הכלים (שם), ובשלמא לסלקן מעל גבי חמור אסור, משום דהא גבה דחמרא לא צריך ליה כלל בשבת, אבל מעל גבי כרים מיהא שרי.\par \textbf{} וראיתי להר״ז הלוי זכרונו לברכה שכתב שכל שאינו ראוי בשום צד לטלטלו לצורך גופו, אין מטלטלין אותו לצורך מקומו. ואינו מחוור בעיני. שהרי לרבי נחמיה כלי שמלאכתו לאיסור אין ראוי לטלטלו לצורך גופו כלל, לפי שאין כלי ניטל לדידיה אלא לצורך תשמישו המיוחד לו בלבד, ואפילו כן לצורך מקומו מטלטלין אותו, וכפי פירוש של רש״י ז״ל כמו שכתבנו שם (לעיל שבת קכד, א) בסייעתא דשמיא, וכן נראה כדבריו. ונראה לי דאילו היו שליפי רברבי מוטלין על גבי כרים וכסתות אי נמי על גבי קרקע והוא צריך למקומן מטלטלין לצורך מקומו, אבל להביא כרים לכתחילה ולשים תחתיהם כדי שיטלטלם ביד לאחר מכן לצורך מה שעשינו לו מקום בידים לא, שאינו בדין דמה שהוא אסור לטלטלו נסבב ונגרום אנו לטלטלו ממש בידים. ועוד דמחזי כחוכא ליטלו מעל גבי חמורו אסרתו ואתה בא להתיר לטרוח ולהביא כלים ולטלטלו בידים לבסוף. ואינו דומה למה שהתיר רב הונא להביא כרים ולהניח תחת שליפי זוטרי, דהתם לא מטלטל ליה בידים, אבל מנער הוא את הכר והן נופלין מאליהן. כך נראה לי.}
\textblock{ הא דאקשינן:\textbf{ והא מבטל כלי מהיכנו.} ואוקימנא בשליפי זוטרי, כלומר שיכול לנערן, משמע דכל שיכול לנערן לא חשבינן ליה כמבטל כלי מהיכנו, דבטול לשעה אינו בטול אלא אם כן הוא מבטלו לכולי יומא. איכא למידק דהא אמר רב חסדא בפרק כירה (שבת מב, ב) אין נותנין כלי תחת תרנגולת לקבל ביצתה, וקא מפרש רב יוסף טעמא התם משום דקא מבטל כלי מהיכנו, והא ביצה כשליפי זוטרי הוא ויכול לנערה. ותירץ הרב רבינו זרחיה הלוי ז״ל דהתם מבטל כלי אצל הביצה לכולי יומא. ואינו מחוור בעיני כלל, דאם כן מאי קא מקשי עליה ממתניתין נותנין כלי תחת הנר לקבל ניצוצות, ואיצטריך לשנויי ניצוצות אין בהם ממש, וכן מנותנין כלי תחת הדלף, ומנא ליה דתנא במבטל ליה לכולי שבת מיירי. אלא יש לומר דלא התירו אפילו בטול לשעתו אלא להצלת הפסד מרובה, וכדאיתא בסמוך גבי עששיות מהו דתימא להפסד מועט נמי חששו קא משמע לן דלא התירו אלא משום הפסד מרובה.\par \textbf{} והא דמקשה התם מכופין את הסל לפני האפרוחין שיעלו וירדו, ופרקינן משום דמותר לטלטלו אחר שירדו, דאלמא לא חיישינן לבטול שעה. יש לומר דהתם נמי הפסד מרובה הוא שאם לא יעלו לקיניהם מתפזרים אילך ואילך ויאבדו. ואי נמי כיון דעשויין לירד מעצמן ואין צריך לטלטלו אפילו מן הצד לא חיישינן. ואם תאמר כיון שהכר מלאכתו להיתר למה לא יטלטלנו אפילו משום הפסד מועט, כיון שהוא יכול לנערו בכל עת שיצטרך לו. יש לומר דלענין הצלה חששו שמא יביא כלי דרך רשות הרבים, אלא שבמקום הפסד מרובה לא חששו.}
\textblock{ הא דשרי הכא\textbf{ רב הונא בשליפי זוטרי.} ולא אסר משום דאין כלי ניטל אלא לדבר שניטל, אע״ג דמשמע פרק כירה (שבת מג, ב) גבי נותנין כלי תחת הנר לקבל ניצוצות דרב הונא אית ליה דרבי יצחק. כבר כתבתי שם בארוכה בסייעתא דשמיא.}
\textblock{\textbf{היתה בהמתו טעונה טבל ועששית מתיר את החבלים.} פירוש: דוקא טבל שאין צריך כלי כגון תבואה      אבל חבית יין של טבל שהוא צריך לכלי מביא כרים ומניח תחתיה, ואמרינן עלה התם (לעיל שבת מג, א) טבל מוכן הוא אצל שבת שאם עבר ותיקנו מתוקן, והכא הכי קאמר היתה בהמתו טעונה טבל אינה נוטלו בעצמו לפי שאינו ראוי לטלטל וכדתנן בפרק מפנין (שבת קכו, ב) אבל לא את הטבל, ואם טעונה עששית דומיא דטבל דהיינו כולסא שאינו ראוי ביומיה לכלום אינו מביא כלים ומניח תחתיה.}
\textblock{\textbf{בכולסא.} הא דאוקמא בכולסא ולא אוקמא בקרני דאומנא ובשליפי רברבי, איכא למימר משום דדומיא דטבל קתני שאינו ראוי עד שיתקננו ואפילו בחול, ועששית נמי שאינו ראוי אפילו בחול עד שיתקננו.}
\textblock{\textbf{מכניס ראשו תחתיה ומסלקו לצד אחר.} ואע״ג דטלטול מקצת דבר המוקצה כטלטול כולו, וכדתנן לעיל בפרק שואל (שבת קנא, א) עושין כל צרכי המת סכין ומדיחין אותו ובלבד שלא יזיו בו אבר, הכא משום הפסד בהמתו, ועוד דטלטול מן הצד הוא שאינו מטלטלו ביד אלא בראשו התירו.}
\textblock{\textbf{והא איכא צער בעלי חיים.} כלומר: שהיא מדאורייתא. ופירש״י ז״ל שהוא דאורייתא ואתי דאורייתא ודחי בטול כלי מהיכנו דרבנן. ונראה שהוא ז״ל סבור דאהא דקאמרינן מטנפי וקא מבטל כלי מהיכנו קאי, אבל הר״ם ב״ן ז״ל הקשה עליו דאפילו תמצי לומר דצער בעלי חיים דאורייתא למה יבטל כלי מהיכנו לדחות ודאי של דבריהם בכדי, אדרבא יתיר חבלים ואע״ג דמצטרו זיקי, דאי משום הפסד הא לא שרינן לבטל כלי מהיכנו אפילו במקום הפסד מרובה בשליפי רברבי. אלא אעיקר עובדא קאי, כלומר מפני מה לא רצו לפרקה ומשום דלא ליצטרו זיקי, והא צער בעלי חיים דאורייתא וקא עביר אדאורייתא, ופריק קא סבר צער בעלי חיים דרבנן, ובמקום פסידא לא חששו לצער בעלי חיים. וזה נכון.}
\textblock{\textbf{שתים בידי אדם ואחת באילן כשרה ואין עולין לה ביום טוב.} פירוש: מפני שהיו רגילין להניח חפציהן על הסכך ונמצא משתמש באילן. אי נמי בשתחתית הסוכה נסמך על האילן דכל שעה שהולך בסוכה משתמש באילן, וכן פירשו בתוס׳ ז״ל. אבל אילו היה הסכך נסמך על האילן ואינו משתמש על גבי הסכך מותר, שאינו אסור להשתמש תחת צליו של אילן עצמו וכל שכן תחת צל הסוכה הנסמכת באילן.}
\clearpage
\newsection{דף קנה}
\textblock{\textbf{והלכתא צדדין אסורין צדי צדדין מותרין.} ואם תאמר אם כן למה לא מנו אותה עם יע״ל קג״ם דהלכתא כאביי (קידושין נב, א. ב״מ כב, ב). יש לומר דאביי בין בצדדין בין בצדי צדדין אסר, וכת״ק דאין עולין לה ביום טוב, ואנן פסקינן כר״ש בן אלעזר דשרי בצדי צדדין.}
\textblock{\textbf{מאי טעמא דרבי יהודה, קסבר שוויי אוכלין משוינן טרחא באוכלא לא טרחינן.} קשיא לי דהא תנן בפרק מפנין (שבת קכו, ב) חבילי קש וחבילי עצים וחבילי זרדין אם התקינו למאכל בהמה מטלטלין אותן ואם לאו אין מטלטלין אותן, אלמא כל שאינו מתקנו למאכל בהמה מערב שבת אפילו בטלטול אסורין, וכל שכן דלא מטלטלינן ומשוינן. ויש לומר דהתם בדלא בעי להו למאכל בהמה אלא לצורך מקומו, ואי נמי לצורך עצמן לישב עליהן.}
\textblock{ [מתני׳:]\textbf{ אין אובסין את הגמל.} ואי קשיא לך הא דאמרינן בעירובין פרק עושין פסין (עירובין כ, ב) גמל שראשו ורובו בפנים אובסין אותו בפנים. פירש רבנו שמואל דההיא הלעטה היא ולא דק בלישנא.}
\clearpage
\newsection{דף קנו}
\textblock{\textbf{הא דכתיב אפנקסיה דזעירי וכו׳ מהו לגבל אמר ליה אסור.} כתבה הרב אלפסי ז״ל, ונראה מדבריו שהלכה כן. והר״ז הלוי ז״ל כתב דלא קיימא לן הכי, דשמעתיה כרבי אזלא, ופליג עליה רבי יוסי בר יהודה, וקיימא לן כותיה. והרמב״ן ז״ל כתב, דהכא בגבול גמור קאמר, והכין תרצתא דשמעתא דגבול מורסן דתרנגולין לעולם אסור. קלי ושתית לאדם מותר על ידי שינוי ואף ע״ג דמגבל כל צרכו.וממרס ביד מורסן לשורים היינו דר׳      ומותר בשאינו מגבל כל צרכו אלא שתי וערב דלאו מערב שפיר וחוזר ומנערו לכלי וזהו פנקסו של זעירי, מהו לגבל כל צרכו לשורים אמר ליה אסור, מהו לפרק מכלי דהיינו ניעור ואמר ליה מותר, ואסיקנא אפילו כור ואפילו כורים על דרך זה. וכן זה שכתב רבינו ז״ל מה שאמרו הא בעבה הא ברכה, שאמרו בגמרא לתרץ ושוין, הא לדברי רבי יוסי בן יהודה שתיהן מותר, להתיר בלא שינוי ברכה נתכון, הא במשנה על יד על יד מותרין, ולפיכך כתבה רבנו הגדול ז״ל, עד כאן. ומדבריו למדנו, פירוש לפרק, דהיינו ניעור מכלי אל כלי, ולא כדברי רש״י דפירש לפרק מלפני זו ולתת בפני בהמה אחרת. ואין פירושו מחוור, דזו מתניתין היא בפרק תולין (לעיל שבת קמ, ב) נוטלין מלפני בהמה זו ונותנין לפני בהמה שפיה רע, אלא הפירוש הנכון הוא כפירוש זה, היינו לפרק ולנער מכלי אל כלי, ועוד דבר הלמד מענינו הוא.}
\textblock{\textbf{ואף רבי יוחנן סבר הלכה כרבי שמעון.} והא דלא אמר מהיכא שמעינן ליה לר׳ יוחנן, כדאמר ברב וזעירי ושמואל, משום דסוגיא דפרק כירה (לעיל שבת מו, א) קמא להו, דהתם שקלינן וטרינן טובא אי כר״ש סבירא ליה ואי כר״י סבירא ליה, והכא מסיק דכר״ש סבירא ליה.}
\clearpage
\newsection{דף קנז}
\textblock{ הא דאקשינן לרבי יוחנן מסתמא מתניתין\textbf{ דאין משקין ושוחטין את המדבריות.} הוה מצי לשנויי דמדבריות כגרוגרות וצמוקין נינהו דמודה בהו ר״ש, וכדאיתא דמשני לה הכין בפרק כירה (שם), אלא דהכא לא סבר לה כהא שנויא. והא דמקשינן הכא לרבי יוחנן מהני סתמא וסייעניה מסתמא אחרינא ולא מייתו מתניתין דמוכני בזמן שיש עליה מעות, כדמקשינן מינה עליה דרבי יוחנן בפרק כירה (שם) כבר כתבתיה שם בארוכה בסייעתא דשמיא.}
\textblock{ מהא דאפליגו כאן\textbf{ רב אחא ורבינא חד אמר בכל השבת הלכה כר״ש, לבד ממוקצה מחמת חסרון כיס, וחד אמר לבד ממוקצה מחמת מיאוס.} שמעינן מינה דבכל מאי דאיפלג ר״ש בשבת כגון מוקצה ונולד ודבר שאין מתכוון ומלאכה שאינה צריכה לגופה הלכה כר״ש, דאי במוקצה לבד קאמר, מאי בכל השבת דקאמר. ועוד דאפילו במוקצה לית הלכתא כותיה בכולהו, דהא איכא מוקצה מחמת איסור, ואליבא דחד מינייהו אפילו במוקצה מחמת מיאוס לא, ומאי בכל השבת. ועוד אילו כן, למ״ד התם דאין הלכה כמותו במוקצה מחמת חסרון כיס לבד, אבל במוקצה מחמת מיאוס נמי הלכה כמותו, למה פסקו כאן כר״ש והוצרכו להוציא ולמעט מוקצה מחמת חסרון כיס, לימא הלכה כר״מ דאית ליה הכין בפרק כירה. וכן כתב הרמב״ן. וכבר כתבתי למעלה (קכא, ב) שר״ח והראב״ד ז״ל כן כתבו דהלכה כר״ש אפילו במלאכה שאינה צריכה לגופה.}
\textblock{\textbf{אבל שלא לצורך לא אלמא הפרת נדרים מעת לעת.} כלומר: ואם לא יפר בשבת עדיין לא אבד את זכותו שהרי יכול הוא להפר למחר עד מעת לעת, והלכך לאו לצורך השבת הוא. ואם תאמר ואם נדרה ערב שבת בין השמשות מאי איכא למימר. יש לומר דהא דבר שאינו מצוי הוא ולדבר שאינו מצוי לא חששו. ויש מי שאומר דכי אמרינן נמי שלא לצורך לא הני מילי היכא דאפשר, כגון ששמע בשבת דאפשר למוצאי שבת, הא קודם השבת דלא אפשר למוצאי שבת מיפר אפילו בשבת ושפיר דמי.}

\addtocontents{toc}{\protect\end{multicols}}
\end{document}
