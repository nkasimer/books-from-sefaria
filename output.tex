\documentclass[12pt, openany]{book}
\usepackage[
paperheight=9in,
paperwidth=6in,
top=0.5in,
bottom=0.5in,
inner=0.7in,
outer=0.5in,
marginparsep=0.1in,
headsep=16pt
]{geometry}

\newcommand{\texttitle}{קיצור שלחן ערוך}\usepackage{titlesec}
\renewcommand{\partname}[1]{}
\usepackage{resources/unnumberedtotoc}

\usepackage{fancyhdr}
\pagestyle{fancy}
\fancyhf{}
\fancyhead[LO,RE]{\thepage}
\fancyhead[CO]{\chapname}
\fancyhead[CE]{\texttitle}

\usepackage{paracol}
\usepackage{anyfontsize}
\usepackage{ragged2e}
\usepackage{polyglossia}
\usepackage{multicol}
\usepackage{hyperref}
\usepackage[marginal]{footmisc}
\usepackage[titles]{tocloft}
\usepackage{xifthen}
\usepackage{graphicx}
\usepackage{dblfnote}\DFNalwaysdouble

\setdefaultlanguage{hebrew}
\setotherlanguage{english}
\usepackage{fontspec}
\setmainfont{Times New Roman}
\newfontfamily\englishfont{Times New Roman}
\setsansfont{Aharoni}

\newcommand{\sethebfont}{
\fontsize{11pt}{13.8pt} \selectfont
}

\newcommand{\LTRmark}{‎}

\newcommand{\hebeng}[2]{
	{\sethebfont #1}
	
	%\vspace{0.5\baselineskip}
	{\beginL\englishfont{\raggedright #2 \hfill} \LTRmark\endL}
	
	\vspace{\baselineskip}
}

\newcommand{\twocol}[1]{
	{\sethebfont \begin{multicols}{2}
			#1
	\end{multicols}}	
}

\newcommand{\textblock}[1]{
{\sethebfont #1\\}	
}

\setlength{\parskip}{6pt}
\setlength\parindent{0in}

\newcommand{\chapname}{}
\newcommand{\sectname}{}

\newcommand{\newchap}[1]{
	\addcontentsline{toc}{chapter}{#1}
	\renewcommand{\chapname}{#1}
		\begin{center}
			\textbf{%
\fontsize{16pt}{16pt}\selectfont
				#1}
		\end{center}
}

\let\footnoterule\relax
\setlength\premulticols{10\baselineskip}
\setlength{\columnsep}{0.25in}

\newcommand{\newsection}[1]{
	%\addcontentsline{toc}{section}{#1}
	\renewcommand{\sectname}{#1}	
	\vspace{-\baselineskip}
	\begin{center}
		\textbf{%
\fontsize{16pt}{16pt}\selectfont
			#1}
	\end{center}
	\vspace{-\baselineskip}
	\nopagebreak
}

\newcommand{\footnotecomment}[1]{
	\renewcommand\thefootnote{}
	\footnote{\textsf{#1}}}

\newcommand{\parencomment}[1]{\footnotesize (#1)}

\newcommand{\blockcomment}[2]{ 
\vspace{\baselineskip}
\newsection{#1}
\sethebfont	\textsf{#2}
\vspace{\baselineskip}}

\newcommand{\commenta}[1]{\footnotecomment{#1}\hspace{0em}}

\newcommand{\vsnum}[1]{(\hebrewnumeral{#1})\space}
\newcommand{\vsnumeng}[1]{(#1)\space}

\begin{document}
\frontmatter
\pagenumbering{roman}

\newcommand{\oneline}[1]{%
	\newdimen{\namewidth}%
	\setlength{\namewidth}{\widthof{#1}}%
	\ifthenelse{\lengthtest{\namewidth < \textwidth}}%
	{#1}% do nothing if shorter than text width
	{\resizebox{\textwidth}{!}{#1}}% scale down
}

\title{\oneline{\hspace*{0.5in}\texttitle\hspace*{0.5in}}}

\author{}

\date{}

\maketitle

\begin{minipage}[b][\textheight][b]{\textwidth}\englishfont\footnotesize
	\begin{english}
		\vfill
		The following book includes:
\begin{itemize}
\item[$\bullet$] Wikisource Kitzur Shulchan Aruch \textendash  Menukad
\begin{itemize}
\item[$\bullet$] License: CC-BY-SA
\item[$\bullet$] Source: \url{http://he.wikisource.org/wiki/%D7%A7%D7%99%D7%A6%D7%95%D7%A8_%D7%A9%D7%95%D7%9C%D7%97%D7%9F_%D7%A2%D7%A8%D7%95%D7%9A_%D7%9E%D7%A0%D7%95%D7%A7%D7%93_-_%D7%A0%D7%91}
\end{itemize}
\item[$\bullet$] Kitzur Shulchan Aruch, trans. Rabbi Avrohom Davis, Metsudah Pub., 1996
\begin{itemize}
\item[$\bullet$] License: CC-BY
\item[$\bullet$] Source: \url{https://www.judaicaplace.com/search/brand/Metsudah-Publications/}
\end{itemize}
\end{itemize}
		It was retrieved from Sefaria on \today\space \texthebrew{(\Hebrewtoday)}.  It was typeset and formatted by Ktavi.
		\clearpage
		
	\end{english}
\end{minipage}

\titleformat{\chapter}[hang]{\huge\bfseries}{\thechapter.}{1em}{}
\titlespacing*{\chapter}{0pt}{-3em}{1.1\parskip}
\titlelabel{\thetitle\quad}
%\addtocontents{toc}{\protect\vspace{-\baselineskip}}
\addtocontents{toc}{\protect\begin{multicols}{2}}
%\vspace*{-5\baselineskip}
{\small \tableofcontents}


\clearpage
\mainmatter
\pagenumbering{arabic}

\newchap{סימן נב}
\hebeng{עַל פֵּרוֹת הַגְּדֵלִים בָּאִילָן, מְבָרְכִין בּוֹרֵא פְּרִי הָעֵץ. וְעַל פֵּרוֹת הַגְּדֵלִים בַּאֲדָמָה, וְהֵם כָּל מִינֵי לִפְתָּן וִירָקוֹת וְקִטְנִיּוֹת וְטָאטָארְקֶע - תִּירָס וַעֲשָׂבִים, מְבָרְכִין בּוֹרֵא פְּרִי הָאֲדָמָה. וְלֹא נִקְרָא אִילָן, אֶלָּא זֶה שֶׁהָעֲנָפִים שֶׁלּוֹ נִשְׁאָרִים גַּם בַּחֹרֶף וּמוֹצִיא אַחַר כָּךְ עָלִים מִן הָעֲנָפִים, וַאֲפִלּוּ הֵם דַּקִּין כְּגִבְעוֹלֵי פִּשְׁתָּן. אֲבָל אִם הָעֲנָפִים כָּלִים לְגַמְרֵי בַּחֹרֶף וְאֵינוֹ נִשְׁאָר רַק הַשֹּׁרֶשׁ, לֹא נִקְרָא אִילָן, וּמְבָרְכִין עַל הַפֵּרוֹת בּוֹרֵא פְּרִי הָאֲדָמָה - ר״ב ר״ג}{Over fruits that grow on a tree you say the berachah \textit{Borei peri ha'eitz}, {[Who created the fruit of the tree]}. Over fruit {[or vegetables]} that grow in the ground, which include various condiments, vegetables, legumes, corn, and herbs, you say the berachah, \textit{Borei peri ha'adamah}, {[Who created the fruit of the ground]}. In order to be called a tree, it must have branches that survive the winter to produce leaves again, even though they are as thin as stalks of flax. However, {[a plant]} whose branches perish in the winter, although its roots remain intact, is not called a tree, and over its fruit we say, \textit{Borei peri ha'adamah}.}
\hebeng{עַל דָּבָר שֶׁאֵין גִּדּוּלוֹ מִן הָאָרֶץ, כְּמוֹ בָּשָׂר, דָּגִים, חָלָב, גְּבִינָה, וְכֵן עַל כָּל מִינֵי מַשְׁקִים, חוּץ מִן הַיַּיִן וְשֶׁמֶן זַיִת, מְבָרְכִין שֶׁהַכֹּל נִהְיָה בִּדְבָרוֹ. וְתֵבַת נִהְיָה, יֵשׁ לוֹמַר היוּ״ד בְּקָמַץ}{{[Before partaking of]} food which does not grow in the earth such as meat, fish, milk, and cheese, and, similarly, {[before drinking]} any beverage other than wine and olive oil, you say the berachah, \textit{Shehakol niyah bidevaro} {[through Whose word everything came to be]}. The word \textit{niyah}—נִהְיָה—should be pronounced with a \textit{kametz} under the \textit{yud}. \textit{Others maintain it should be pronounced with a \textit{segol} under the \textit{yud}—נִהְיֶה \textit{niyeh}. (\textit{Magen Avraham} 204: 14, \textit{Chayei Adam})}}
\hebeng{כְּמֵהִין וּפִטְרִיּוֹת - שוואמען אַף- עַל- פִּי שֶׁהֵן גְּדֵלִין מִלַּחְלוּחִית הָאָרֶץ, יְנִיקָתָן אֵינָהּ מִן הָאָרֶץ אֶלָּא מִן הָאֲוִיר, וְלָכֵן אֵינָן נִקְרָאִין פְּרִי הָאֲדָמָה, וּמְבָרְכִין עֲלֵיהֶן שֶׁהַכֹּל - ר״ד}{Even though mushrooms and truffles, grow in the moisture of the earth, they do not receive nourishment from the earth, but rather from the atmosphere. Therefore, they cannot be called fruit of the ground, and you should say \textit{Shehakol} over them.}
\hebeng{אֵין מְבָרְכִין בּוֹרֵא פְּרִי הָעֵץ אוֹ בּוֹרֵא פְּרִי הָאֲדָמָה, אֶלָּא עַל דָּבָר שֶׁהוּא טוֹב לְאָכְלוֹ חַי וְגַם הַדֶּרֶךְ הוּא לְאָכְלוֹ חַי. אֲבָל אִם אֵין הַדֶּרֶךְ לְאָכְלוֹ חַי אֶלָּא מְבֻשָּׁל, אַף- עַל- פִּי שֶׁהוּא טוֹב לְמַאֲכָל גַּם כְּשֶׁהוּא חַי, מִכָּל מָקוֹם אֵינוֹ חָשׁוּב כָּל כָּךְ, וְאֵין מְבָרְכִין עָלָיו בִּרְכָתוֹ אֶלָּא כְּשֶׁאוֹכְלוֹ מְבֻשָּׁל. אֲבָל אִם אוֹכְלוֹ חַי, אֵינוֹ מְבָרֵךְ עָלָיו אֶלָּא שֶׁהַכֹּל. וְכָבוּשׁ הֲרֵי הוּא כִּמְבֻשָּׁל. וְלָכֵן עַל כְּרוּב - קְרוֹיט כָּבוּשׁ מְבָרְכִין בּוֹרֵא פְּרִי הָאֲדָמָה. וְכֵן מָלִיחַ הוּא כִּמְבֻשָּׁל לְעִנְיָן זֶה}{We say the berachah \textit{Borei peri ha'eitz} and \textit{Borei peri ha'adamah} only over articles of foods which are best eaten raw and which, as a rule, are eaten raw. However, if it is not the custom to eat them raw, but only when cooked, even though they could also be eaten raw, nevertheless, since that is not the favored way {[they are eaten]}, their berachah is said only when they are eaten in their cooked state. But if you \textit{do} eat them raw, you should recite only \textit{Shehakol}. Pickled food is considered as cooked food. Consequently, before eating sauerkraut you should say \textit{Borei peri ha'adamah}. Similarly, salted food is considered the same as cooked food.}
\hebeng{עַל הַצְּנוֹן מְבָרְכִין בּוֹרֵא פְּרִי הָאֲדָמָה. וְכֵן עַל שׁוּמִים וּבְצָלִים כְּשֶׁהֵן רַכִּין וְדַרְכָּן לְאָכְלָן חַיִּין, אַף- עַל- פִּי שֶׁעַל פִּי הָרֹב אֵין אוֹכְלִין אוֹתָן רַק עִם פַּת, מִכָּל מָקוֹם גַּם אִם אוֹכְלָן בְּלֹא פַּת, מְבָרְכִין עֲלֵיהֶן בּוֹרֵא פְּרִי הָאֲדָמָה. אֲבָל אִם הִזְקִינוּ, שֶׁהֵם חֲרִיפִים מְאֹד, וְאֵין דַּרְכָּן לְאָכְלָן חַיִּין, מִי שֶׁאָכְלָן חַיִּין, מְבָרֵךְ עֲלֵיהֶם שֶׁהַכֹּל}{Before eating radishes, you should say \textit{Borei peri ha'adamah}. Likewise, over garlic and onions that are soft and can be eaten raw, although, as a rule, they are eaten with bread, nevertheless, even if you eat them without bread, you recite over them \textit{Borei peri ha'adamah}. \textit{Since in the Western countries, green onions (scallions) are generally eaten with bread, it is preferable that \textit{Shehakol} should be said before eating them (without bread). (\textit{Mishnah Berurah} 205: 5)} But if the {[garlic or onions]} have grown old which give them a very strong flavor, so that they are not usually eaten raw, then, if you do eat them raw, you should say the berachah \textit{Shehakol}.}
\hebeng{דְּבָרִים שֶׁהֵם טוֹבִים יוֹתֵר כְּשֶׁהֵם חַיִּין מִכְּשֶׁהֵם מְבֻשָּׁלִין, שֶׁהַבִּשּׁוּל מְגָ רֵעַ אוֹתָן, אֵין מְבָרְכִין עֲלֵיהֶן כְּשֶׁהֵן מְבֻשָּׁלִין אֶלָּא שֶׁהַכֹּל. וְאַף- עַל- פִּי שֶׁבִּשְּׁלָן עִם בָּשָׂר וְעַל יְדֵי הַבָּשָׂר נִשְׁתַּבְּחוּ, מִכָּל מָקוֹם אָז הַבָּשָׂר הוּא הָעִקָּר, וְאֵין מְבָרְכִין עֲלֵיהֶן אֶלָּא שֶׁהַכֹּל. אֲבָל אִם בִּשְּׁלָן בְּאֹפֶן שֶׁהֵן הָעִקָּר וּמִכָּל מָקוֹם נִשְׁתַּבְּחוּ, כְּגוֹן שֶׁטִּגְנָן בְּשֻׁמָּן אוֹ בִּדְבַשׁ וְכַיּוֹצֵא בּוֹ, מְבָרְכִין עֲלֵיהֶן הַבְּרָכָה הָרְאוּוּיָה לָהֶן, דְּמַה לִי אִם נִתְבַּשְּׁלוּ בְּמַיִם אוֹ בְּשֻׁמָּן וּדְבַשׁ - ר״ה}{{[If you eat]} articles of food that taste better raw than cooked, because cooking spoils them, then you should say over them when they are cooked, only the berachah \textit{Shehakol}. Even if they are cooked with meat and their taste improved because of the meat, nevertheless, since the meat is the main dish, we recite over them only the berachah, \textit{Shehakol}. However, if they were cooked in a way that made them into the main dish, or if {[the cooking]} improved them, as for example, when they were fried in oil or in honey, and the like, you should say their appropriate berachah, since it makes no difference whether they were cooked in water or in oil or in honey, {[as long as the food itself was improved by the cooking process]}.}
\hebeng{מִינֵי פֵּרוֹת הַגְּרוּעִים, הַגְּדֵלִים עַל אֲטָדִים וְקוֹצִים אוֹ בִּשְׁאָר אִילָנוֹת שֶׁיָּצְאוּ מֵאֲלֵיהֶן וְלֹא נַטְעוּ לְהוּ אֱנָֹשֵי, כְּמוֹ תַּפּוּחֵי יַעַר וְכַדּוֹמֶה, שֶׁכְּשֶׁהֵם חַיִּין אֵינָן רְאוּיִין לַאֲכִילָה, אַף- עַל- פִּי שֶׁבִּשְּׁלָן אוֹ טִגְּנָן בִּדְבַֹש וְסֻכָּר וְהֵן רְאוּיִין לַאֲכִילָה, אֵין מְבָרְכִין עֲלֵיהֶן אֶלָּא שֶׁהַכֹּל. אֲבָל לוּזִין - האַזעלנוס אַף- עַל- פִּי שֶׁגְּדֵלִים בַּיַּעַר, חֲשׁוּבִים הֵם, וּמְבָרְכִין עֲלֵיהֶם בּוֹרֵא פְּרִי הָעֵץ - ר״ג}{Inferior kinds of fruit that grow on brambles and thorn-bushes or on other trees which grow wild and are not planted, such as wild apples and the like, which when raw are not fit to eat, {[then]} even if they are cooked or fried in honey and sugar and made edible, you should recite \textit{Shehakol} over them. But hazel nuts, although they grow {[wild]} in the forest, are considered superior articles of food and you should recite \textit{Borei peri ha'eitz} over them.}
\hebeng{עֲשָׂבִים הַגְּדֵלִים מֵאֲלֵיהֶם בְּלִי זְרִיעָה, אַף- עַל- פִּי שֶׁהֵן רְאוּ יִין לֶאֱכֹל חַיִּין, וַאֲפִלּוּ בִּשְּׁלָן וְהוּא מַאֲכָל חָשׁוּב, מִכָּל מָקוֹם כֵּיוָן שֶׁאֵין זוֹרְעִין אוֹתוֹ, אֵינוֹ חָשׁוּב פְּרִי וּמְבָרְכִין עָלָיו שֶׁהַכֹּל. אֲבָל חַסָּה [סֶעלַאט] וְכַדּוֹמֶה שֶׁנִּזְרַע, מְבָרְכִין עָלָיו בּוֹרֵא פְרִי הָאֲדָמָה. וְגַם בַּעֲשָׂבִים הַגְּדֵלִים מֵאֲלֵיהֶן, אִם יֵשׁ בָּהֶם פֵּרוֹת חֲשׁוּבִים, כְּגוֹן יָאגְדֶּעס וּמָאלִינֶעס מְבָרְכִין עֲלֵיהֶם בּוֹרֵא פְּרִי הָאֲדָמָה}{Herbs which grow wild without cultivation are fit to eat raw. If they are cooked and made into a special dish, nevertheless, since they were not cultivated {[by man]}, they are not considered a fruit {[of the ground]} and you should say \textit{Shehakol} over them. But lettuce and similar {[vegetables]} that are planted require the berachah, \textit{Borei peri ha'adamah}. Furthermore, regarding herbs that grow wild, if they produce superior fruit, such as blackberries and raspberries, \textit{The raspberry and blackberry bushes that grow in our region remain intact over the winter (perennial plants), therefore, we should say \textit{Borei peri ha’eitz} over them. (\textit{Mishnah Berurah} 203: 1, \textit{Igros Moshe, Orach Chaim})} the berachah said over them is \textit{Borei peri ha'adamah}.}
\hebeng{דָּבָר שֶׁאֵינוֹ עִקַּר הַפְּרִי, אֵינוֹ חָשׁוּב כְּמוֹ הַפְּרִי עַצְמוֹ, אֶלָּא יוֹרֵד מַדְרֵגָה אַחַת, שֶׁאִם הוּא פְּרִי עֵץ, מְבָרְכִין עַל הַטָּפֵל בּוֹרֵא פְּרִי הָאֲדָמָה. וְאִם הוּא פְּרִי הָאֲדָמָה, מְבָרְכִין עַל הַטָּפֵל שֶׁהַכֹּל. וְלָכֵן אִילַן צְלָף - קאפערנבוים שֶׁהֶעָלִין שֶׁלּוֹ רְאוּיִין לַאֲכִילָה, וְיֵשׁ בֶּעָלִים כְּמִין תְּמָרִים בּוֹלְטִים, כְּמוֹ בֶּעָלִים שֶׁל עֲרָבָה, וְאֶבְיוֹנוֹת הֵן עִקַּר הַפְּרִי, וְקַפְרִיסִין הֵן הַקְּלִפָּה שֶׁסְּבִיב הַפְּרִי, כְּמוֹ קְלִפּוֹת הָאֱגוֹזִים, וְרָאוּיִים גַּם כֵּן לַאֲכִילָה, עַל הָאֶבְיוֹנוֹת שֶׁהֵן עִקַּר הַפְּרִי, מְבָרֵךְ בּוֹרֵא פְּרֵי הָעֵץ, וְעַל הֶעָלִין וְעַל הַתְּמָרוֹת וְעַל הַקַּפְרִיסִין, בּוֹרֵא פְּרִי הָאֲדָמָה, וְכֵן עֲלֵי וְרָדִים - ראזענבלעטער שֶׁנִּרְקְחוּ בִּדְבַשׁ וְסֻכָּר, מְבָרְכִין בּוֹרֵא פְּרִי הָאֲדָמָה, אַף- עַל- פִּי שֶׁגְּדֵלִים בָּאִילָן, מִפְּנֵי שֶׁאֵינָן עִקַּר הַפְּרִי. וְכֵן קְלִפּוֹת תּפּוּחֵי-זָהָב שֶׁנִּרְקְחוּ בִּדְבַשׁ וְסֻכָּר, מְבָרְכִין עֲלֵיהֶם בּוֹרֵא פְּרִי הָאֲדָמָה. וְעַל קְלִפּוֹת קִשּׁוּאִין שֶׁמְטַגְּנִין בִּדְבַשׁ וְסֻכָּר, מְבָרְכִין שֶׁהַכֹּל. וְעַל הַשַּׁרְבִיטִין מֵהַקִּטְנִיּוֹת שֶׁזּוֹרְעִין בַּשָּׂדוֹת, אַף- עַל- פִּי שֶׁהֵן מְתוּקִּים, אִם אֲכָלָן בְּלֹא הַקִטְנִיּוֹת, מְבָרְכִין עֲלֵיהֶם שֶׁהַכֹּל. וְאֵלּוּ שֶׁזּוֹרְעִים בַּגִּנּוֹת עַל דַּעַת לְאָכְלָן חַיִּין בְּשַׁרְבִיטֵיהֶן, אֲפִלּוּ כְּשֶׁאוֹכֵל הַשַּׁרְבִיטִין לְחוּד, יֵשׁ לְבָרֵךְ בּוֹרֵא פְּרִי הָאֲדָמָה - ר״ב ר״ד}{That which is not the main part of the fruit, is not considered as the fruit itself, but is one level lower. If it is the fruit of a tree, you say over the secondary part the berachah, \textit{Borei peri ha'adamah;} and if it is the fruit of the ground, you say over the secondary part the berachah, \textit{Shehakol}. Therefore, a caper tree whose leaves are edible, and where its leaves {[are joined with the stem]} there are small {[edible]} buds like the growths at the juncture of willow leaves, and the caper-berries form the main part of the fruit, while the caper flowers are only a peel around the fruit, like the shell of a nut, but are also edible, {[consequently]}, over the berries which constitute the fruit itself, you say the berachah \textit{Borei peri ha'eitz}, and over the leaves and the little buds and the flowers, {[you say]} \textit{Borei peri ha'adamah}. Likewise, over preserves made of rose leaves with honey and sugar, you say the berachah, \textit{Borei peri ha'adamah}, for although they grow on trees, they are not fruit. Similarly, over preserves made of orange peel, honey, and sugar, you say the berachah, \textit{Borei peri ha'adamah}. Over cucumber peel that was fried with honey and sugar, you say the berachah \textit{Shehakol}. Over the pods of peas that are cultivated in the fields, even though they taste sweet, yet, if you ate them without the peas, you should recite the berachah, \textit{Shehakol}. But those which have been grown in gardens for the purpose of eating them raw in their pods, even if you only eat the pods, you should say the berachah, \textit{Borei peri ha'adamah}.}
\hebeng{גַּרְעִינִין שֶׁל פֵּרוֹת אִם הֵם מְתוּקִּים, מְבָרֵךְ עֲלֵיהֶם בּוֹרֵא פְּרִי הָאֲדָמָה, אֲבָל גַּרְעִינִים הַמָּרִים אֵינָם נֶחְשָׁבִים כְּלָל, וְאִם אוֹכְלָן כָּךְ, אֵינוֹ מְבָרֵךְ עֲלֵיהֶם כְּלָל. וְאִם מִתְּקָן עַל יְדֵי הָאוֹר וְכַדּוֹמֶה, מְבָרֵךְ עֲלֵיהֶם שֶׁהַכֹּל}{Over the seeds of fruit that are sweet, \textit{Not to be taken literally—any seed that has a slightly pleasant taste is included. (\textit{Mishnah Berurah} 202: 23)} you say the berachah, \textit{Borei peri ha'adamah}. But bitter tasting seeds are worth nothing at all, and if you eat them, you do not say a berachah. But if you make them pleasant-tasting by roasting them, or in any other way, you say the berachah, \textit{Shehakol}, over them.}
\hebeng{שְׁקֵדִים הַמָּרִים, כְּשֶׁהֵם קְטַנִּים שֶׁאָז עִקַּר אֲכִילָתָן הִיא הַקְּלִפָּה שֶׁאֵינָהּ מָרָה, וְעַל דַּעַת כֵּן נוֹטְעִין אוֹתָן, מְבָרֵךְ עֲלֵיהֶן בּוֹרֵא פְּרִי הָעֵץ. וכְשֶׁהֵן גְּדוֹלִים שֶׁאָז עִקַּר הָאֲכִילָה הוּא מַה שֶּׁבִּפְנִים וְהוּא מַר, אִם אוֹכְלָן כָּךְ, אֵינוֹ מְבָרֵךְ כְּלָל. אֲבָל אִם מִתְּקָן עַל יְדֵי הָאוּר אוֹ דָּבָר אַחֵר, כֵּיוָן דַּפְרִי נִינְהוּ וְגַם עַל דַּעַת כֵּן נוֹטְעִין אוֹתָן, מְבָרֵךְ עֲלֵיהֶן בּוֹרֵא פְּרִי הָעֵץ)ר״ב. שְׁקֵדִים הַמְחֻפִּין בְּסֻכָּר אַף- עַל -פִּי שֶׁהַסֻכָּר הוא הָרֹב, מִכָּל מָקוֹם מְבָרְכִין עֲלֵיהֶן בּוֹרֵא פְּרִי הָעֵץ. וְקַלְמוּס הַמְחֻפֶּה בְּסֻכָּר מְבָרְכִין רַק שֶׁהַכֹּל כִּי הַקַּלְמוּס אֵינוֹ פְּרִי}{Small almonds are bitter and the principal food for which they are planted is the shell which is not bitter, and over these you recite the berachah, \textit{Borei peri ha'eitz}. But when they are large, the principal food {[are the kernels]} which are inside the shell. If these taste bitter, and you eat them {[raw]}, you do not say any berachah. But if you made them pleasant-tasting by roasting them or in any other way, since they are fruit and were planted for that purpose, you say the berachah, \textit{Borei peri ha'eitz} over them. Over sugar-coated almonds even if there is more sugar than almond {[in the confection]}, nevertheless, you say the berachah \textit{Borei peri ha'eitz}. Over sugar-coated calamus {[an aromatic root]}, you only say \textit{Shehakol}, since calamus is not a fruit.}
\hebeng{פֵּרוֹת שֶׁלֹּא נִגְמַר בִּשּׁוּלָן עַל הָאִילָן, אֲפִלּוּ בִשְׁלָן אוֹ טִגְּנָן בִּדְבַשׁ וְכַדּוֹמֶה, כְּמוֹ שֶׁהוּא הַדֶּרֶךְ לְטַגֵּן פֵּרוֹת שֶׁלֹּא נִגְמְרוּ, בִּדְבַשׁ אוֹ סֻכָּר, מְבָרֵךְ עֲלֵיהֶם שֶׁהַכֹּל. אַךְ עַל אֶתְרוֹג מְטֻגָּן בִּדְבַשׁ אוֹ בְּסֻכָּר יֵשׁ לְבָרֵךְ בּוֹרֵא פְּרִי הָעֵץ}{It is customary to fry unripe fruit in honey or sugar, you should say the berachah, \textit{Shehakol} over them. But over an \textit{esrog} (citron) that was fried in honey or sugar, you should recite the berachah \textit{Borei peri ha'eitz}.}
\hebeng{נוֹבְלוֹת, וְהֵן פֵּרוֹת שֶׁנִּשְׂרְפוּ מִן הַחֹם וְנָבְלוּ וְנָפְלוּ מִן הָאִילָן קֹדֶם שֶׁנִּתְבַּשְּׁלוּ, כֵּיוָן שֶׁהוּא דָּבָר שֶׁנִּתְקַלְקֵל אֵין מְבָרְכִין עָלָיו רַק שֶׁהַכֹּל. וְכֵן פַּת שֶׁעִפְּשָׁה וְתַבְשִׁיל שֶׁנִּתְקַלְקֵל קְצָת מְבָרְכִין עֲלֵיהֶן שֶׁהַכֹּל. אֲבָל אִם נִתְקַלְקְלוּ לְגַמְרֵי עַד שֶׁאֵינָן רְאוּיִין לַאֲכִילָה, אֵין מְבָרְכִין עֲלֵיהֶן כְּלָל. וְכֵן חֹמֶץ גָּמוּר - שְׁמְּבַעבֵּעַ כְּשֶׁשּׁוֹפְכִין אוֹתוֹ עַל הָאָרֶץ אֵין מְבָרְכִין עָלָיו כְּלָל. וְאִם עֵרְבוֹ בְּמַיִם עַד שֶׁרָאוּי לִשְׁתִיָּה מְבָרְכִין עָלָיו שֶׁהַכֹּל}{Over spoiled fruit, such as, fruit that has become parched by the heat and has fallen off the tree before becoming ripe, you should say the berachah, \textit{Shehakol}. Likewise, over moldy bread \textit{If you ate \textit{kedei sevi‘ah} (a quantity that satisfies your hunger) of it, you should say afterwards \textit{Birkas Hamazon;} if you ate less, you say \textit{Borei nefashos}. (\textit{Biur Halachah} 204: 1)} or over a slightly spoiled dish, the berachah, \textit{Shehakol}, should be said. But if they are completely spoiled, to the point that they are inedible, then you need not say any berachah. Similarly, over strong vinegar (which bubbles when it is poured on the ground), no berachah is said. But if you mix it with water, and it becomes fit to drink, you should say \textit{Shehakol} over it.}
\hebeng{וְיֵשׁ מִינֵי פֵּרוֹת שֶׁדַּרְכָּן בְּכָךְ שֶׁאֵינָן מִתְבַּשְּׁלִים לְעוֹלָם עַל הָאִילָן, אֶלָּא אַחַר שֶׁנּוֹטְלִין אוֹתָן מִן הָאִילָן מַנִּיחִין אוֹתָן בְּתוֹךְ קַשׁ וְתֶבֶן וְכַדּוֹמֶה וְעַל יְדֵי כָּךְ מִתְבַּשְּׁלִין, כְּגוֹן הָאַגָּסִּים הַקְּטַנִּים - אשריצן כֵּיוָן שֶׁדַּרְכָּן בְּכָךְ, מְבָרְכִין עֲלֵיהֶן בּוֹרֵא פְּרִי הָעֵץ - ר״ב ר״ד}{There are types of fruit which never ripen on the tree, but after they are picked from the tree they are stored {[and covered]} with straw and chaff and the like, and in that way they become ripe. Since that is their natural way of ripening, you recite the berachah, \textit{Borei peri ha'eitz}, over them.}
\hebeng{יֵשׁ מִינֵי פֵּרוֹת שֶׁאֵין בָּהֶם אֶלָּא שְֹרָף בְּעָלְמָא כָּנוּס בְּתוֹךְ הַחַרְצַנִּים - וְנִקְרָאִים קָאלִינֶעס וְאֵינָן רְאוּיִין לַאֲכִילָה אֶלָּא מוֹצְצִין אוֹתָן וְזוֹרְקִין הַקְּלִפּוֹת, מְבָרְכִין עַל מְצִיצָה זוֹ שֶׁהַכֹּל, - דְּכֵיוָן שֶׁעִקָּרוֹ אֵינוֹ אֶלָּא לַמַּשְׁקֶה הַיּוֹצֵא מִמֶּנּוּ אֵין עָלָיו שֵׁם פְּרִי כְּלָל, וַאֲפִלּוּ הוּא בּוֹלֵעַ גַּם הַקְּלִפָּה וְהַגַּרְעִין, אֵינוֹ מְבָרֵךְ רַק שֶׁהַכֹּל}{There are certain kinds of fruit that contain only juice that is concentrated in their seeds. ({[In Yiddish]} they are called \textit{kolines}.) {[These seeds]} are not fit to eat; and after the juice is extracted from them they are thrown away. Over this extraction, you recite \textit{Shehakol}, (since its main component is the juice that is extracted from it, in no way can it be considered a fruit). Even if you eat the skin and the seeds as well, you still should only say \textit{Shehakol}.}
\hebeng{אֵין מְבָרְכִין בּוֹרֵא פְּרִי הָעֵץ וּבוֹרֵא פְּרִי הָאֲדָמָה אֶלָּא כְּשֶׁנִּכָּר בְּמִקְצָת שֶׁהוּא פֶּרִי. אֲבָל אִם נִתְרַסֵּק עַד שֶׁאֵינוֹ נִכָּר כְּלָל מַה הוּא, כְּגוֹן, ]רִבָּה[ - לעקפאר - פאוועדלא, לאטווערג שֶׁמְבַשְּׁלִין מִשְּׁזִיפִין וְקִטְנִיּוֹת שֶׁרִסְּקָן לְגַמְרֵי וְכַדּוֹמֶה, מְבָרְכִין עֲלֵיהֶם שֶׁהַכֹּל. וּבְדִיעֲבַד אִם בֵּרַךְ עֲלֵיהֶם בְּרָכָה הָרְאוּיָה לָהֶן יָצָא. וְאִם רֹב דֶּרֶךְ אֲכִילַת אוֹתָן פֵּרוֹת הוּא עַל יְדֵי רִסּוּק שֶׁמְרַסְּקִין אוֹתָן לְגַמְרֵי, מְבָרְכִין אַף לְכַתְּחִלָּה בְּרָכָה הָרְאוּיָה לָהֶן - ר״ב ר״ד ר״ה}{We say neither the berachah \textit{Borei peri ha'eitz} nor \textit{Borei peri ha'adamah} unless we can at least slightly recognize the fruit. But if they are so crushed that they are unrecognizable, as for example, prune paste (\textit{lekvar}) which is made of cooked prunes, or mashed peas, etc., then we say \textit{Shehakol} over them. But if, after the fact, you said the berachah that was appropriate for their kind, you have fulfilled your obligation. However, if the accepted way of eating these fruits is in a crushed form, by completely mashing them, then you say even initially the berachah that was meant for them.}
\hebeng{אֹרֶז וְדֹחַן - הירז ורייז שֶׁנִּתְבַּשְּׁלוּ, אִם לֹא נִתְמַעֲכוּ, מְבָרֵךְ עֲלֵיהֶם בּוֹרֵא פְּרֵי הָאֲדָמָה. וְאִם נִתְמַעֲכוּ אוֹ שְׁטָחָן וְעָשָׂה מֵהֶן פַּת, יֵשׁ חִלּוּק בֵּין אֹרֶז לְדֹחַן, כִּי מִצַּד הַדִּין עַל הָאֹרֶז מְבָרֵךְ בּוֹרֵא מִינֵי מְזוֹנוֹת, וְעַל הַדֹּחַן שֶׁהַכֹּל. אֶלָּא שֶׁיֵּשׁ לָנוּ סָפֵק אֵיזֶהוּ אֹרֶז וְאֵיזֶהוּ דֹחַן, לָכֵן יְרֵא -שָׁמַיִם לֹא יֹאכַל בֵּין דֹּחַן בֵּין אֹרֶז שֶׁנִּתְמַעֲכוּ אֶלָּא בְּתוֹךְ הַסְּעוּדָּה. וּבִשְׁעַת הַדְּחַק שֶׁאֵין לוֹ פַּת מְבָרֵךְ בֵּין עַל הַדֹּחַן בֵּין עַל אֹרֶז שֶׁהַכֹּל וּלְאַחֲרֵיהֶם בּוֹרֵא נְפָשׁוֹת רַבּות. עַל פַּת הֶעָשׂוּי מִקִּטְנִיּוֹת - תִּרָס טענגרא קוקריטז. מאליי״ע טירקשען וויטץ, אֲפִלּוּ בִּמְקוֹמוֹת שֶׁדַּרְכָּן בְּלֶחֶם זֶה, מְבָרְכִין עָלָיו שֶׁהַכֹּל - עַיֵן פְּרִי מְגָדִים סִמָן ר״ח, מִֹשְבְּצוֹת סָעִיף קָטָן י״א, וְֹשִיוּרֵי בְּרָכָה סִימָן ר״ז. - דִּין עֵרַב קְמָחִין, עַיֵן בְֹּשֻלְחָן עָרוּךְ סִימָן ר״ח ס״ט}{Over rice and millet that have been cooked, if they have not been mashed, you say the berachah, \textit{Borei peri ha'adamah}. But if they were mashed or ground {[into flour]}, and you made bread out of them, there is a difference between rice and millet. {[Technically]}, according to the law, over rice, you should say \textit{Borei minei mezonos}, and the berachah, \textit{Shehakol}, over millet; but we are not certain whether the Hebrew word \textit{orez} means rice, \textit{\textit{Mishnah Berurah} rules that \textit{orez} denotes rice, therefore, \textit{Borei minei mezonos} should be said over it. (\textit{Mishnah Berurah} 208: 25)} and the Hebrew word \textit{dochan} means millet, {[or vice versa]}. Therefore, a God fearing person should eat mashed millet and rice only as part of a meal. However, in an emergency, if you have no bread, you should say over both rice and millet, the berachah, \textit{Shehakol}, and the after-berachah \textit{Borei nefashos rabbos}. Over bread made of cornmeal, even in places where such bread is the staple food, the berachah, \textit{Shehakol} should be recited.}
\hebeng{עַל הַסֻּכָּר מְבָרֵךְ שֶׁהַכֹּל. וְכֵן הַמּוֹצֵץ קָנִים מְתוּקִים, מְבָרֵךְ שֶׁהַכֹּל. וְכֵן ]קִנָּמוֹן וְֹשוֹּש[ צימרינד ולאקריטץ שֶׁכּוֹסְסִין וּבוֹלְעִין רַק טַעַם וּפוֹלְטִין הָעִקָּר מְבָרְכִין עֲלֵיהֶן שֶׁהַכֹּל}{Over sugar you say the berachah, \textit{Shehakol}. Likewise, if you chew sugar cane, you say \textit{Shehakol}. Also {[over]} cinnamon and licorice that you chew and only enjoy the flavor, and you discard the substance, you say the berachah, \textit{Shehakol}.}

\addtocontents{toc}{\protect\end{multicols}}
\end{document}
