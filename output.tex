\documentclass[12pt, openany]{book}
\usepackage[
paperheight=11in,
paperwidth=8.5in,
top=0.5in,
bottom=0.5in,
inner=0.7in,
outer=0.5in,
marginparsep=0.1in,
headsep=16pt
]{geometry}

\newcommand{\texttitle}{עבודה זרה}\usepackage{titlesec}
\usepackage{resources/unnumberedtotoc}

\usepackage{fancyhdr}
\pagestyle{fancy}
\fancyhf{}
\fancyhead[LO,RE]{\thepage}
\fancyhead[CO]{\chapname}
\fancyhead[CE]{\texttitle}

\usepackage{paracol}
\usepackage{anyfontsize}
\usepackage{ragged2e}
\usepackage{polyglossia}
\usepackage{multicol}
\usepackage{hyperref}

\setdefaultlanguage{hebrew}
\setotherlanguage{english}
\usepackage{fontspec}
\setmainfont{Frank Ruehl CLM}
\newfontfamily\englishfont{EB Garamond}

\newcommand{\sethebfont}{
\fontsize{10.5pt}{21.0pt} \selectfont
}

\newcommand{\hebeng}[2]{
	{\sethebfont #1\\}
	
	\begin{english}
		#2
	\end{english}
	\clearpage
}

\newcommand{\twocol}[1]{
	{\sethebfont \begin{multicols}{2}
			#1
	\end{multicols}}	
}

\newcommand{\textblock}[1]{
{\sethebfont #1\\}	
}

\setlength{\parskip}{8pt}

\newcommand{\chapname}{}
\newcommand{\sectname}{}

\newcommand{\newchap}[1]{
	\addcontentsline{toc}{chapter}{#1}
	\renewcommand{\chapname}{#1}
		\begin{center}
			\textbf{%
\fontsize{16pt}{16pt}\selectfont
				#1}
		\end{center}
}

\newcommand{\newsection}[1]{
	\addcontentsline{toc}{section}{#1}
	\renewcommand{\sectname}{#1}	
	\vspace{-\baselineskip}
	\begin{center}
		\textbf{%
\fontsize{16pt}{16pt}\selectfont
			#1}
	\end{center}
	\vspace{-\baselineskip}
	\nopagebreak
}

\begin{document}
\frontmatter
\pagenumbering{roman}

\title{\texttitle}

\author{}

\date{}

\maketitle

\begin{minipage}[b][\textheight][b]{\textwidth}\englishfont	
	\begin{english}
		\vfill
		The following book includes:
\begin{itemize}
\item[$\bullet$] Wikisource Talmud Bavli
\item[$\bullet$] License: CC-BY
\item[$\bullet$] Source: \url{http://he.wikisource.org/wiki/%D7%AA%D7%9C%D7%9E%D7%95%D7%93_%D7%91%D7%91%D7%9C%D7%99}
\end{itemize}
		It was retrieved from Sefaria on \today\space \texthebrew{(\Hebrewtoday)}.  It was typeset and formatted by Ktavi, using \LaTeX .
		\clearpage
		
	\end{english}
\end{minipage}


\tableofcontents

\clearpage
\mainmatter
\pagenumbering{arabic}

\newchap{פרק \hebrewnumeral{1}\quad לפני אידיהן}
\newsection{דף ב}
\twocol{
\par מתני׳ {\large\emph{לפני}} אידיהן של עובדי כוכבים שלשה ימים אסור לשאת ולתת עמהם להשאילן ולשאול מהן להלוותן וללוות מהן לפורען ולפרוע מהן רבי יהודה אומר נפרעין מהן מפני שמיצר הוא לו אמרו לו אע"פ שמיצר הוא עכשיו שמח הוא לאחר זמן:}
\twocol{{\large\emph{גמ׳}} רב ושמואל חד תני אידיהן וחד תני עידיהן מאן דתני אידיהן לא משתבש ומאן דתני עידיהן לא משתבש
\par מאן דתני אידיהן לא משתבש דכתיב (דברים לב, לה) כי קרוב יום אידם ומאן דתני עידיהן לא משתבש דכתיב (ישעיהו מג, ט) יתנו עידיהם ויצדקו}
\twocol{ומאן דתני אידיהן מאי טעמא לא תני עידיהן אמר לך תברא עדיף ומאן דתני עידיהן מאי טעמא לא תני אידיהן אמר לך מאן קא גרים להו תברא עדות שהעידו בעצמן הלכך עדות עדיפא
\par והאי יתנו עידיהם ויצדקו בעובדי כוכבים כתיב הא בישראל כתיב דאמר ריב"ל כל מצות שישראל עושין בעולם הזה באות ומעידות להם לעוה"ב שנאמר יתנו עידיהם ויצדקו אלו ישראל ישמעו ויאמרו אמת אלו עובדי כוכבים}
\twocol{אלא אמר רב הונא בריה דרב יהושע מאן דאמר עידיהן מהכא (ישעיהו מד, ט) יוצרי פסל כולם תוהו וחמודיהם בל יועילו ועידיהם המה
\par דרש ר' חנינא בר פפא ואיתימא ר' שמלאי לעתיד לבא מביא הקדוש ברוך הוא ס"ת [ומניחו] בחיקו ואומר למי שעסק בה יבא ויטול שכרו}
\twocol{
\par מיד מתקבצין ובאין עובדי כוכבים בערבוביא שנאמר (ישעיהו מג, ט) כל הגוים נקבצו יחדו [וגו'] אמר להם הקדוש ברוך הוא אל תכנסו לפני בערבוביא אלא תכנס כל אומה ואומה}
\newchap{פרק \hebrewnumeral{1}\quad לפני אידיהן}
\twocol{
\par וסופריה שנאמר (ישעיהו מג, ט) ויאספו לאומים ואין לאום אלא מלכות שנאמר (בראשית כה, כג) ולאום מלאום יאמץ ומי איכא ערבוביא קמי הקב"ה אלא כי היכי דלא ליערבבו אינהו [בהדי הדדי] דלישמעו מאי דאמר להו}
\twocol{[מיד] נכנסה לפניו מלכות רומי תחלה מ"ט משום דחשיבא ומנלן דחשיבא דכתי' (דניאל ז, כג) ותאכל כל ארעא ותדושינה ותדוקינה אמר רבי יוחנן זו רומי חייבת שטבעה יצא בכל העולם
\par ומנא לן דמאן דחשיב עייל ברישא כדרב חסדא דאמר רב חסדא מלך וצבור מלך נכנס תחלה לדין שנאמר (מלכים א ח, נט) לעשות משפט עבדו ומשפט עמו ישראל [וגו'] וטעמא מאי איבעית אימא לאו אורח ארעא למיתב מלכא מאבראי ואיבעית אימא מקמי דליפוש חרון אף}
\twocol{אמר להם הקב"ה במאי עסקתם אומרים לפניו רבש"ע הרבה שווקים תקנינו הרבה מרחצאות עשינו הרבה כסף וזהב הרבינו וכולם לא עשינו אלא בשביל ישראל כדי שיתעסקו בתורה
\par אמר להם הקב"ה שוטים שבעולם כל מה שעשיתם לצורך עצמכם עשיתם תקנתם שווקים להושיב בהן זונות מרחצאות לעדן בהן עצמכם כסף וזהב שלי הוא שנאמר (חגי ב, ח) לי הכסף ולי הזהב נאם ה' צבאות}
\twocol{כלום יש בכם מגיד זאת שנאמר מי בכם יגיד זאת ואין זאת אלא תורה שנאמר (דברים ד, מד) וזאת התורה אשר שם משה מיד יצאו בפחי נפש
\par יצאת מלכות רומי ונכנסה מלכות פרס אחריה מ"ט דהא חשיבא בתרה ומנלן דכתיב (דניאל ז, ה) וארו חיוא אחרי תנינא דמיא לדוב ותני רב יוסף אלו פרסיים שאוכלין ושותין כדוב ומסורבלין [בשר] כדוב ומגדלין שער כדוב ואין להם מנוחה כדוב}
\twocol{אמר להם הקב"ה במאי עסקתם אומרים לפניו רבש"ע הרבה גשרים גשרנו הרבה כרכים כבשנו הרבה מלחמות עשינו וכולם לא עשינו אלא בשביל ישראל כדי שיתעסקו בתורה
\par אמר להם הקב"ה כל מה שעשיתם לצורך עצמכם עשיתם תקנתם גשרים ליטול מהם מכס כרכים לעשות בהם אנגריא מלחמות אני עשיתי שנאמר (שמות טו, ג) ה' איש מלחמה כלום יש בכם מגיד זאת שנאמר (ישעיהו מג, ט) מי בכם יגיד זאת ואין זאת אלא תורה שנאמר וזאת התורה אשר שם משה מיד יצאו מלפניו בפחי נפש}
\twocol{וכי מאחר דחזית מלכות פרס למלכות רומי דלא מהניא ולא מידי מאי טעמא עיילא אמרי אינהו סתרי בית המקדש ואנן בנינן וכן לכל אומה ואומה
\par וכי מאחר דחזו לקמאי דלא מהני ולא מידי מ"ט עיילי סברי הנך אישתעבדו בהו בישראל ואנן לא שעבדנו בישראל מאי שנא הני דחשיבי ומאי שנא הני דלא חשיבי להו משום דהנך משכי במלכותייהו עד דאתי משיחא}
\twocol{אומרים לפניו רבש"ע כלום נתת לנו ולא קיבלנוה ומי מצי למימר הכי והכתי' (דברים לג, ב) ויאמר ה' מסיני בא וזרח משעיר למו וכתיב (חבקוק ג, ג) אלוה מתימן יבוא וגו' מאי בעי בשעיר ומאי בעי בפארן
\par א"ר יוחנן מלמד שהחזירה הקב"ה על כל אומה ולשון ולא קבלוה עד שבא אצל ישראל וקבלוה}
\twocol{אלא הכי אמרי כלום קיבלנוה ולא קיימנוה ועל דא תברתהון אמאי לא קבלתוה אלא כך אומרים לפניו רבש"ע כלום כפית עלינו הר כגיגית ולא קבלנוה כמו שעשית לישראל
\par דכתיב (שמות יט, יז) ויתיצבו בתחתית ההר ואמר רב דימי בר חמא מלמד שכפה הקב"ה הר כגיגית על ישראל ואמר להם אם אתם מקבלין את התורה מוטב ואם לאו שם תהא קבורתכם}
\twocol{מיד אומר להם הקב"ה הראשונות ישמיעונו שנא' (ישעיהו מג, ט) וראשונות ישמיענו שבע מצות שקיבלתם היכן קיימתם
\par ומנלן דלא קיימום דתני רב יוסף (חבקוק ג, ו) עמד וימודד ארץ ראה ויתר גוים מאי ראה ראה ז' מצות שקבלו עליהן בני נח ולא קיימום כיון שלא קיימום עמד והתירן להן איתגורי איתגור א"כ מצינו חוטא נשכר}
\twocol{אמר מר בריה דרבינא}
\newsection{דף ג}
\twocol{לומר שאף על פי שמקיימין אותן אין מקבלין עליהם שכר
\par ולא והתניא היה רבי מאיר אומר מנין שאפילו עובד כוכבים ועוסק בתורה שהוא ככהן גדול תלמוד לומר (ויקרא יח, ה) אשר יעשה אותם האדם וחי בהם כהנים לוים וישראלים לא נאמר אלא האדם הא למדת שאפילו עובד כוכבים ועוסק בתורה הרי הוא ככהן גדול}
\twocol{אלא לומר לך שאין מקבלין עליהם שכר כמצווה ועושה אלא כמי שאינו מצווה ועושה דאמר ר' חנינא גדול המצווה ועושה יותר משאינו מצווה ועושה
\par אלא כך אומרים העובדי כוכבים לפני הקב"ה רבש"ע ישראל שקיבלוה היכן קיימוה}
\twocol{אמר להם הקב"ה אני מעיד בהם שקיימו את התורה כולה אומרים לפניו רבש"ע כלום יש אב שמעיד על בנו דכתיב (שמות ד, כב) בני בכורי ישראל אמר להם הקב"ה שמים וארץ יעידו בהם שקיימו את התורה כולה
\par אומרים לפניו רבש"ע שמים וארץ נוגעין בעדותן שנאמ' (ירמיהו לג, כה) אם לא בריתי יומם ולילה חוקות שמים וארץ לא שמתי (דאר"ש) [ואר"ש] בן לקיש מאי דכתיב (בראשית א, לא) ויהי ערב ויהי בקר יום הששי מלמד שהתנה הקב"ה עם מעשה בראשית ואמר אם ישראל מקבלין את תורתי מוטב ואם לאו אני אחזיר אתכם לתוהו ובוהו}
\twocol{והיינו דאמר חזקיה מאי דכתיב (תהלים עו, ט) משמים השמעת דין ארץ יראה ושקטה אם יראה למה שקטה ואם שקטה למה יראה אלא בתחלה יראה ולבסוף שקטה
\par אמר להם הקב"ה מכם יבאו ויעידו בהן בישראל שקיימו את התורה כולה יבא נמרוד ויעיד באברהם שלא עבד עבודת כוכבים יבא לבן ויעיד ביעקב שלא נחשד על הגזל תבא אשת פוטיפרע ותעיד ביוסף שלא נחשד על העבירה}
\twocol{יבא נבוכד נצר ויעיד בחנניה מישאל ועזריה שלא השתחוו לצלם יבא דריוש ויעיד בדניאל שלא ביטל את התפלה יבא בלדד השוחי וצופר הנעמתי ואליפז התימני ואליהו בן ברכאל הבוזי ויעידו בהם בישראל שקיימו את כל התורה כולה שנאמר (ישעיהו מג, ט) יתנו עידיהם ויצדקו
\par אמרו לפניו רבש"ע תנה לנו מראש ונעשנה אמר להן הקב"ה שוטים שבעולם מי שטרח בערב שבת יאכל בשבת מי שלא טרח בערב שבת מהיכן יאכל בשבת אלא אף על פי כן מצוה קלה יש לי וסוכה שמה לכו ועשו אותה}
\twocol{ומי מצית אמרת הכי והא אמר רבי יהושע בן לוי מאי דכתיב (דברים ז, יא) אשר אנכי מצוך היום היום לעשותם ולא למחר לעשותם היום לעשותם ולא היום ליטול שכר
\par אלא שאין הקב"ה בא בטרוניא עם בריותיו ואמאי קרי ליה מצוה קלה משום דלית ביה חסרון כיס}
\twocol{מיד כל אחד [ואחד] נוטל והולך ועושה סוכה בראש גגו והקדוש ברוך הוא מקדיר עליהם חמה בתקופת תמוז וכל אחד ואחד מבעט בסוכתו ויוצא שנאמר (תהלים ב, ג) ננתקה את מוסרותימו ונשליכה ממנו עבותימו מקדיר והא אמרת אין הקדוש ברוך הוא בא בטרוניא עם בריותיו משום דישראל נמי זימני
\par דמשכא להו תקופת תמוז עד חגא והוי להו צערא והאמר רבא מצטער פטור מן הסוכה נהי דפטור בעוטי מי מבעטי}
\twocol{מיד הקב"ה יושב ומשחק עליהן שנאמר (תהלים ב, ד) יושב בשמים ישחק וגו' א"ר יצחק אין שחוק לפני הקב"ה אלא אותו היום בלבד
\par איכא דמתני להא דרבי יצחק אהא דתניא רבי יוסי אומר לעתיד לבא באין עובדי כוכבים ומתגיירין ומי מקבלינן מינייהו והתניא אין מקבלין גרים לימות המשיח כיוצא בו לא קבלו גרים לא בימי דוד ולא בימי שלמה}
\twocol{אלא שנעשו גרים גרורים ומניחין תפילין בראשיהן תפילין בזרועותיהם ציצית בבגדיהם מזוזה בפתחיהם
\par כיון שרואין מלחמת גוג ומגוג אומר להן על מה באתם אומרים לו על ה' ועל משיחו שנאמר (תהלים ב, א) למה רגשו גוים ולאומים יהגו ריק [וגו']}
\twocol{וכל אחד מנתק מצותו והולך שנאמר (תהלים ב, ג) ננתקה את מוסרותימו [וגו] והקב"ה יושב ומשחק שנאמר יושב בשמים ישחק וגו' א"ר יצחק אין לו להקב"ה שחוק אלא אותו היום בלבד
\par איני והא אמר רב יהודה אמר רב שתים עשרה שעות הוי היום שלש הראשונות הקב"ה יושב ועוסק בתורה שניות יושב ודן את כל העולם כולו כיון שרואה שנתחייב עולם כלייה עומד מכסא הדין ויושב על כסא רחמים}
\twocol{שלישיות יושב וזן את כל העולם כולו מקרני ראמים עד ביצי כנים רביעיות יושב ומשחק עם לויתן שנאמר (תהלים קד, כו) לויתן זה יצרת לשחק בו אמר רב נחמן בר יצחק עם בריותיו משחק ועל בריותיו אינו משחק אלא אותו היום בלבד
\par א"ל רב אחא לרב נחמן בר יצחק מיום שחרב בית המקדש אין שחוק להקב"ה ומנלן דליכא שחוק אילימא מדכתיב (ישעיהו כב, יב) ויקרא ה' אלהים צבאות ביום ההוא לבכי ולמספד ולקרחה וגו' דלמא ההוא יומא ותו לא}
\twocol{אלא דכתיב (תהלים קלז, ה) אם אשכחך ירושלם תשכח ימיני תדבק לשוני לחכי אם לא אזכרכי דלמא שכחה הוא דליכא אבל שחוק מיהא איכא אלא מהא (ישעיהו מב, יד) החשיתי מעולם אחריש אתאפק וגו'
\par ברביעיות מאי עביד יושב ומלמד תינוקות של בית רבן תורה שנאמר (ישעיהו כח, ט) את מי יורה דעה ואת מי יבין שמועה גמולי מחלב עתיקי משדים למי יורה דעה ולמי יבין שמועה לגמולי מחלב ולעתיקי משדים}
\twocol{ומעיקרא מאן הוה מיגמר להו איבעית אימא מיטטרון ואיבעית אימא הא והא עביד
\par ובליליא מאי עביד איבעית אימא מעין יממא ואיבעית אימא רוכב על כרוב קל שלו ושט בשמונה עשר אלף עולמות שנאמר (תהלים סח, יח) רכב אלהים רבותים אלפי שנאן אל תקרי שנאן אלא שאינן ואיבעית אימא יושב ושומע שירה מפי חיות שנאמר (תהלים מב, ט) יומם יצוה ה' חסדו ובלילה שירו עמי}
\twocol{אמר רבי לוי כל הפוסק מדברי תורה ועוסק בדברי שיחה מאכילין לו גחלי רתמים שנאמר (איוב ל, ד) הקוטפים מלוח עלי שיח ושורש רתמים לחמם אמר ריש לקיש כל העוסק בתורה בלילה הקב"ה מושך עליו חוט של חסד ביום שנאמר יומם יצוה ה' חסדו ובלילה שירו עמי מה טעם יומם יצוה ה' חסדו משום דבלילה שירו עמי
\par איכא דאמרי אמר ר"ל כל העוסק בתורה בעולם הזה הדומה ללילה הקב"ה מושך עליו חוט של חסד בעולם הבא הדומה ליום שנאמר יומם יצוה ה' חסדו וגו'}
\twocol{אמר רב יהודה אמר שמואל מאי דכתיב (חבקוק א, יד) ותעשה אדם כדגי הים כרמש לא מושל בו למה נמשלו בני אדם כדגי הים לומר לך מה דגים שבים כיון שעולין ליבשה מיד מתים אף בני אדם כיון שפורשין מדברי תורה ומן המצות מיד מתים דבר אחר מה דגים שבים כיון שקדרה עליהם חמה מיד מתים כך בני אדם כיון שקדרה עליהם חמה מיד מתים
\par איבעית אימא בעולם הזה ואיבעית אימא לעולם הבא איבעית אימא בעולם הזה כדר' חנינא דא"ר חנינא הכל בידי שמים חוץ מצנים פחים שנאמר (משלי כב, ה) צנים פחים בדרך עקש שומר נפשו ירחק מהם}
\twocol{ואיבעית אימא לעולם הבא כדרשב"ל דאמר רבי שמעון בן לקיש אין גיהנם לעתיד לבא אלא הקדוש ברוך הוא מוציא חמה מנרתיקה ומקדיר רשעים נידונין בה וצדיקים מתרפאין בה רשעים נידונין}
\newsection{דף ד}
\twocol{בה דכתיב (מלאכי ג, יט) [כי] הנה היום בא בוער כתנור והיו כל זדים וכל עושה רשעה קש ולהט אותם היום הבא אמר ה' צבאות אשר לא יעזוב להם שורש וענף לא שורש בעולם הזה ולא ענף לעולם הבא
\par צדיקים מתרפאין בה דכתיב (מלאכי ג, כ) וזרחה לכם יראי שמי שמש צדקה ומרפא בכנפיה וגו' ולא עוד אלא שמתעדנין בה שנאמר (מלאכי ג, כ) ויצאתם ופשתם כעגלי מרבק}
\twocol{דבר אחר מה דגים שבים כל הגדול מחבירו בולע את חבירו אף בני אדם אלמלא מוראה של מלכות כל הגדול מחבירו בולע את חבירו והיינו דתנן רבי חנינא סגן הכהנים אומר הוי מתפלל בשלומה של מלכות שאלמלא מוראה של מלכות איש את רעהו חיים בלעו
\par רב חיננא בר פפא רמי כתיב (איוב לז, כג) שדי לא מצאנוהו שגיא כח וכתיב (תהלים קמז, ה) גדול אדונינו ורב כח וכתיב (שמות טו, ו) ימינך ה' נאדרי בכח לא קשיא כאן בשעת הדין כאן בשעת מלחמה}
\twocol{רבי חמא בר' חנינא רמי כתיב (ישעיהו כז, ד) חימה אין לי וכתיב (נחום א, ב) נוקם ה' ובעל חימה לא קשיא כאן בישראל כאן בעובדי כוכבים רב חיננא בר פפא אמר חימה אין לי שכבר נשבעתי מי יתנני שלא נשבעתי אהיה שמיר ושית וגו'
\par והיינו דאמר רבי אלכסנדרי מאי דכתיב (זכריה יב, ט) והיה ביום ההוא אבקש להשמיד את כל הגוים אבקש ממי אמר הקב"ה אבקש בניגני שלהם אם יש להם זכות אפדם ואם לאו אשמידם}
\twocol{והיינו דאמר רבא מאי דכתיב (איוב ל, כד) אך לא בעי ישלח יד אם בפידו להן שוע אמר להן הקב"ה לישראל כשאני דן את ישראל אין אני דן אותם כעובדי כוכבים דכתיב (יחזקאל כא, לב) עוה עוה עוה אשימנה וגו' אלא אני נפרע מהן כפיד של תרנגולת
\par דבר אחר אפילו אין ישראל עושין מצוה לפני כי אם מעט כפיד של תרנגולין שמנקרין באשפה אני מצרפן לחשבון גדול [שנאמר אם בפידו] להן שוע [דבר אחר] בשכר שמשוועין לפני אני מושיע אותם}
\twocol{והיינו דאמר ר' אבא מאי דכתיב (הושע ז, יג) ואנכי אפדם והמה דברו עלי כזבים אני אמרתי אפדם בממונם בעוה"ז כדי שיזכו לעולם הבא והמה דברו עלי כזבים
\par והיינו דאמר רב פפי משמיה דרבא מאי דכתיב (הושע ז, טו) ואני יסרתי חזקתי זרועותם ואלי יחשבו רע אמר הקב"ה אני אמרתי איסרם ביסורין בעולם הזה כדי שיחזקו זרועותם לעוה"ב ואלי יחשבו רע}
\twocol{משתבח להו ר' אבהו למיני ברב ספרא דאדם גדול הוא שבקו ליה מיכסא דתליסר שנין יומא חד אשכחוהו אמרו ליה כתיב (עמוס ג, ב) רק אתכם ידעתי מכל משפחות האדמה על כן אפקוד עליכם את כל עונותיכם מאן דאית ליה סיסיא ברחמיה מסיק ליה אישתיק ולא אמר להו ולא מידי רמו ליה סודרא בצואריה וקא מצערו ליה
\par אתא רבי אבהו אשכחינהו אמר להו אמאי מצעריתו ליה אמרו ליה ולאו אמרת לן דאדם גדול הוא [ולא ידע למימר לן פירושא דהאי פסוקא]  אמר להו אימר דאמרי לכו בתנאי בקראי מי אמרי לכו}
\twocol{אמרו ליה מ"ש אתון דידעיתון אמר להו אנן דשכיחינן גביכון רמינן אנפשין ומעיינן אינהו לא מעייני
\par אמרו ליה לימא לן את אמר להו אמשול לכם משל למה"ד לאדם שנושה משני בנ"א אחד אוהבו ואחד שונאו אוהבו נפרע ממנו מעט מעט שונאו נפרע ממנו בבת אחת}
\twocol{א"ר אבא בר כהנא מאי דכתיב (בראשית יח, כה) חלילה לך מעשות כדבר הזה להמית צדיק עם רשע אמר אברהם לפני הקב"ה רבש"ע חולין הוא מעשות כדבר הזה להמית צדיק עם רשע
\par ולא והכתיב (יחזקאל כא, ח) והכרתי ממך צדיק ורשע בצדיק שאינו גמור}
\twocol{אבל בצדיק גמור לא והכתיב (יחזקאל ט, ו) וממקדשי תחלו ותני רב יוסף אל תקרי ממקדשי אלא ממקודשי אלו בני אדם שקיימו את התורה מאל"ף ועד תי"ו התם נמי כיון שהיה בידם למחות ולא מיחו הוו להו כצדיקים שאינן גמורים
\par רב פפא רמי כתיב (תהלים ז, יב) אל זועם בכל יום וכתיב (נחום א, ו) לפני זעמו מי יעמוד לא קשיא כאן ביחיד כאן בצבור}
\twocol{ת"ר אל זועם בכל יום וכמה זעמו רגע וכמה רגע אחת מחמש ריבוא ושלשת אלפים ושמונה מאות וארבעים ושמנה בשעה זו היא רגע ואין כל בריה יכולה לכוין אותה רגע חוץ מבלעם הרשע דכתיב ביה
\par (במדבר כד, טז) ויודע דעת עליון אפשר דעת בהמתו לא הוה ידע דעת עליון מי הוה ידע}
\twocol{מאי דעת בהמתו לא הוה ידע בעידנא דחזו ליה דהוה רכיב אחמריה אמרו ליה מאי טעמא לא רכבתא אסוסיא אמר להו ברטיבא שדאי ליה מיד ותאמר האתון הלא אנכי אתונך אמר לה לטעינא בעלמא
\par אמרה ליה אשר רכבת עלי אמר לה אקראי בעלמא אמרה ליה מעודך ועד היום הזה ולא עוד אלא שאני עושה לך רכיבות ביום ואישות בלילה כתיב הכא ההסכן הסכנתי וכתיב התם (מלכים א א, ב) ותהי לו סוכנת}
\twocol{אלא מאי ויודע דעת עליון שהיה יודע לכוין אותה שעה שהקב"ה כועס בה והיינו דקאמר להו נביא (מיכה ו, ה) עמי זכר נא מה יעץ בלק מלך מואב ומה ענה אותו בלעם בן בעור מן השטים ועד הגלגל למען דעת צדקות ה'
\par א"ר אלעזר אמר להן הקב"ה לישראל עמי ראו כמה צדקות עשיתי עמכם שלא כעסתי עליכם כל אותן הימים שאם כעסתי עליכם לא נשתייר מעובדי כוכבים משונאיהם של ישראל שריד ופליט והיינו דקאמר ליה בלעם לבלק (במדבר כג, ח) מה אקב לא קבה אל ומה אזעם לא זעם ה'}
\twocol{וכמה זעמו רגע וכמה רגע אמר אמימר ואיתימא רבינא רגע כמימריה ומנלן דרגע הוה ריתחיה דכתיב (תהלים ל, ו) כי רגע באפו חיים ברצונו ואיבעית אימא מהכא (ישעיהו כו, כ) חבי כמעט רגע עד יעבור זעם
\par אימת רתח אמר אביי בתלת שעי קמייתא כי חיורא כרבלתא דתרנגולא כל שעתא ושעתא מחוור חיורא כל שעתא אית ביה סורייקי סומקי ההיא שעתא לית ביה סורייקי סומקי}
\twocol{רבי יהושע בן לוי הוה מצער ליה ההוא מינא [בקראי יומא חד] נקט תרנגולא [ואוקמיה בין כרעיה דערסא] ועיין ביה סבר כי מטא ההיא שעתא אלטייה כי מטא ההיא שעתא נימנם
\par אמר שמע מינה לאו אורח ארעא למיעבד הכי [ורחמיו על כל מעשיו כתיב] וכתיב (משלי יז, כו) גם ענוש לצדיק לא טוב}
\twocol{תנא משמיה דר"מ בשעה שהמלכים מניחין כתריהן בראשיהן ומשתחוין לחמה מיד כועס [הקב"ה] אמר רב יוסף לא ליצלי איניש צלותא דמוספי בתלת שעי קמייתא דיומא ביומא קמא דריש שתא ביחיד דלמא כיון דמפקיד דינא דלמא מעייני בעובדיה ודחפו ליה מידחי
\par אי הכי דצבור נמי דצבור נפישא זכותיה אי הכי דיחיד דצפרא נמי לא כיון דאיכא צבורא דקא מצלו לא קא מדחי}
\twocol{והא אמרת שלש ראשונות הקב"ה יושב ועוסק בתורה איפוך
\par ואיבעית אימא לעולם לא תיפוך תורה דכתיב בה אמת דכתיב (משלי כג, כג) אמת קנה ואל תמכור אין הקב"ה עושה לפנים משורת הדין דין דלא כתיב ביה אמת הקב"ה עושה לפנים משורת הדין:}
\twocol{יום מעיד טרף בעגל סימן: גופא אמר רבי יהושע בן לוי מאי דכתיב (דברים ז, יא) אשר אנכי מצוך היום לעשותם היום לעשותם ולא למחר לעשותם היום לעשותם ולא היום ליטול שכרן
\par אמר רבי יהושע בן לוי כל מצות שישראל עושין בעולם הזה באות ומעידות אותם לעולם הבא שנאמר (ישעיהו מג, ט) יתנו עידיהם ויצדקו ישמעו ויאמרו אמת יתנו עידיהם ויצדקו אלו ישראל ישמעו ויאמרו אמת אלו עובדי כוכבים}
\twocol{ואמר רבי יהושע בן לוי כל מצות שישראל עושין בעולם הזה באות וטורפות אותם לעובדי כוכבים לעולם הבא על פניהם שנאמר (דברים ד, ו) ושמרתם ועשיתם כי היא חכמתכם ובינתכם לעיני העמים נגד העמים לא נאמר אלא לעיני העמים מלמד שבאות וטורפות לעובדי כוכבים על פניהם לעוה"ב
\par וא"ר יהושע בן לוי לא עשו ישראל את העגל אלא ליתן פתחון פה לבעלי תשובה שנאמר (דברים ה, כה) מי יתן והיה לבבם זה להם ליראה אותי כל הימים וגו'}
\twocol{והיינו דא"ר יוחנן משום ר"ש בן יוחאי לא דוד ראוי לאותו מעשה ולא ישראל ראוין לאותו מעשה לא דוד ראוי לאותו מעשה דכתיב (תהלים קט, כב) ולבי חלל בקרבי
\par ולא ישראל ראוין לאותו מעשה דכתיב מי יתן והיה לבבם זה להם ליראה אותי כל הימים אלא למה עשו}
\newsection{דף ה}
\twocol{לומר לך שאם חטא יחיד אומרים לו כלך אצל יחיד ואם חטאו צבור אומרים לו  כלך אצל צבור
\par וצריכא דאי אשמועינן יחיד משום דלא מפרסם חטאיה אבל צבור דמפרסם חטאיהו אימא לא ואי אשמועינן צבור משום דנפישי רחמייהו אבל יחיד דלא אלימא זכותיה אימא לא צריכא}
\twocol{והיינו דרבי שמואל בר נחמני א"ר יונתן מאי דכתיב (שמואל ב כג, א) נאם דוד בן ישי ונאם הגבר הוקם על נאם דוד בן ישי שהקים עולה של תשובה
\par וא"ר שמואל בר נחמני אמר ר' יונתן כל העושה מצוה אחת בעולם הזה מקדמתו והולכת לפניו לעולם הבא שנאמר (ישעיהו נח, ח) והלך לפניך צדקך וכבוד ה' יאספך וכל העובר עבירה אחת מלפפתו ומוליכתו ליום הדין שנאמר (איוב ו, יח) ילפתו ארחות דרכם וגו'}
\twocol{ר"א אומר קשורה בו ככלב שנאמר (בראשית לט, י) ולא שמע אליה לשכב אצלה להיות עמה לשכב אצלה בעוה"ז להיות עמה בעוה"ב
\par אמר ר"ל בואו ונחזיק טובה לאבותינו שאלמלא הן לא חטאו אנו לא באנו לעולם שנאמר (תהלים פב, ו) אני אמרתי אלהים אתם ובני עליון כלכם חבלתם מעשיכם אכן כאדם תמותון וגו'}
\twocol{למימרא דאי לא חטאו לא הוו מולדו והכתיב (בראשית ט, ז) ואתם פרו ורבו עד סיני בסיני נמי כתיב (דברים ה, כז) לך אמור להם שובו לכם לאהליכם לשמחת עונה
\par והכתיב (דברים ה, כו) למען ייטב להם ולבניהם וגו' לאותן העומדים על הר סיני}
\twocol{והאמר ר"ל מאי דכתיב (בראשית ה, א) זה ספר תולדות אדם וגו' וכי ספר היה לו לאדם הראשון מלמד שהראה לו הקב"ה לאדם הראשון דור דור ודורשיו דור דור וחכמיו דור דור ופרנסיו כיון שהגיע לדורו של ר"ע שמח בתורתו ונתעצב במיתתו אמר (תהלים קלט, יז) ולי מה יקרו רעיך אל [וגו']
\par וא"ר יוסי אין בן דוד בא עד שיכלו נשמות שבגוף שנאמר (ישעיהו נז, טז) [כי לא לעולם אריב ולא לנצח אקצוף] כי רוח מלפני יעטוף ונשמות אני עשיתי}
\twocol{לא תימא אנו לא באנו לעולם אלא כמי שלא באנו לעולם למימרא דאי לא חטאו לא הוו מייתי והכתיב פרשת יבמות ופרשת נחלות
\par על תנאי ומי כתיבי קראי על תנאי אין דהכי אמר רבי שמעון בן לקיש מאי דכתיב (בראשית א, לא) ויהי ערב ויהי בקר יום הששי מלמד שהתנה הקב"ה עם מעשה בראשית ואמר אם מקבלין ישראל את התורה מוטב ואם לאו אחזיר אתכם לתוהו ובוהו}
\twocol{מיתיבי (דברים ה, כו) מי יתן והיה לבבם זה להם לבטל מהם מלאך המות א"א שכבר נגזרה גזרה
\par הא לא קיבלו ישראל את התורה אלא כדי שלא תהא אומה ולשון שולטת בהן שנאמר (דברים ה, כו) למען ייטב להם ולבניהם עד עולם}
\twocol{הוא דאמר כי האי תנא דתניא רבי יוסי אומר לא קיבלו ישראל את התורה אלא כדי שלא יהא מלאך המות שולט בהן שנאמר (תהלים פב, ו) אני אמרתי אלהים אתם ובני עליון כלכם חבלתם מעשיכם אכן כאדם תמותון
\par ורבי יוסי נמי הכתיב למען ייטב להם ולבניהם עד עולם טובה הוא דהויא הא מיתה איכא (רבי יוסי) אמר לך כיון דליכא מיתה אין לך טובה גדולה מזו}
\twocol{ות"ק נמי הכתיב אכן כאדם תמותון מאי מיתה עניות דאמר מר ארבעה חשובים כמתים אלו הן עני סומא ומצורע ומי שאין לו בנים
\par עני דכתיב (שמות ד, יט) כי מתו כל האנשים ומאן נינהו דתן ואבירם ומי מתו מיהוי הוו אלא שירדו מנכסיהם}
\twocol{סומא דכתיב (איכה ג, ו) במחשכים הושיבני כמתי עולם מצורע דכתיב (במדבר יב, יב) אל נא תהי כמת ומי שאין לו בנים דכתיב (בראשית ל, א) הבה לי בנים ואם אין מתה אנכי
\par תנו רבנן (ויקרא כו, ג) אם בחקתי תלכו אין אם אלא לשון תחנונים וכן הוא אומר (תהלים פא, יד) לו עמי שומע לי [וגו'] כמעט אויביהם אכניע ואומר (ישעיהו מח, יח) לו הקשבת למצותי ויהי כנהר שלומך וגו' ויהי כחול זרעך וצאצאי מעיך וגו'}
\twocol{תנו רבנן (דברים ה, כז) מי יתן והיה לבבם זה להם אמר להן משה לישראל כפויי טובה בני כפויי טובה בשעה שאמר הקדוש ברוך הוא לישראל מי יתן והיה לבבם זה להם היה להם לומר תן אתה
\par כפויי טובה דכתיב (במדבר כא, ה) ונפשנו קצה}
\twocol{בלחם הקלוקל בני כפויי טובה דכתיב (בראשית ג, יב) האשה אשר נתתה עמדי היא נתנה לי מן העץ ואוכל
\par אף משה רבינו לא רמזה להן לישראל אלא לאחר ארבעים שנה שנאמר (דברים כט, ד) ואולך אתכם במדבר ארבעים שנה וכתיב (דברים כט, ג) ולא נתן ה' לכם לב וגו' אמר רבה ש"מ לא קאי איניש אדעתיה דרביה עד ארבעין שנין}
\twocol{א"ר יוחנן משום רבי בנאה מאי דכתיב (ישעיהו לב, כ) אשריכם זורעי על כל מים משלחי רגל השור והחמור אשריהם ישראל בזמן שעוסקין בתורה ובגמילות חסדים יצרם מסור בידם ואין הם מסורים ביד יצרם שנאמר אשריכם זורעי על כל מים ואין זריעה אלא צדקה שנאמר (הושע י, יב) זרעו לכם לצדקה וקצרו לפי חסד ואין מים אלא תורה שנאמר (ישעיהו נה, א) הוי כל צמא לכו למים
\par משלחי רגל השור והחמור תנא דבי אליהו לעולם ישים אדם עצמו על דברי תורה כשור לעול וכחמור למשאוי:}
\twocol{ג' ימים אסור לשאת ולתת עמהם וכו': ומי בעינן כולי האי
\par והתנן בארבעה פרקים בשנה המוכר בהמה לחבירו צריך להודיעו אמה מכרתי לשחוט בתה מכרתי לשחוט}
\twocol{ואלו הן עיו"ט האחרון של חג עיו"ט הראשון של פסח וערב עצרת וערב ר"ה וכדברי ר' יוסי הגלילי אף ערב יוה"כ בגליל
\par התם דלאכילה סגיא בחד יומא הכא דלהקרבה בעינן תלתא יומי ולהקרבה סגי בתלתא יומי והתנן שואלין בהלכות הפסח קודם הפסח שלשים יום רשב"ג אומר שתי שבתות}
\twocol{אנן דשכיחי מומין דפסלי אפילו בדוקין שבעין בעינן תלתין יומין אינהו דמחוסר אבר אית להו בתלתא יומי סגי
\par דא"ר אלעזר מנין למחוסר אבר דאסור לבני נח דכתיב (בראשית ו, יט) ומכל החי מכל בשר שנים מכל וגו' אמרה תורה הבא בהמה שחיין ראשי אברים שלה}
\twocol{האי מיבעי ליה למעוטי טריפה דלא טריפה מלחיות זרע נפקא הניחא למאן דאמר טריפה אינה יולדת}
\newsection{דף ו}
\twocol{אלא למ"ד טריפה יולדת מאי איכא למימר אמר קרא אתך בדומין לך ודלמא נח גופיה טריפה הוה תמים כתיב ביה
\par ודלמא תמים בדרכיו היה צדיק כתיב ביה}
\twocol{דלמא תמים בדרכיו צדיק במעשיו הוה לא ס"ד דנח גופיה טריפה הואי דאי ס"ד דנח טריפה הוה א"ל רחמנא כוותך עייל שלמין לא תעייל
\par והשתא דנפקא ליה מאתך לחיות זרע ל"ל אי מאתך הוה אמינא לצוותא בעלמא ואפילו זקן ואפילו סריס כתב רחמנא זרע}
\twocol{איבעיא להו שלשה ימים הן ואידיהן או דלמא הן בלא אידיהן
\par ת"ש ר' ישמעאל אומר שלשה לפניהם ושלשה לאחריהן אסור אי ס"ד הן ואידיהן רבי ישמעאל יום אידיהן חשיב להו מעיקרא וחשיב להו לבסוף}
\twocol{איידי דתנא שלשה לפניהם תנא נמי שלשה לאחריהם
\par ת"ש דאמר רב תחליפא בר אבדימי אמר שמואל יום א' לדברי רבי ישמעאל לעולם אסור ואי ס"ד הן ואידיהן האיכא ארבעה וחמשה דשרי}
\twocol{אליבא דרבי ישמעאל לא קמבעיא לי דהן בלא אידיהן כי קא מבעיא לי אליבא דרבנן מאי
\par אמר רבינא ת"ש ואלו הן אידיהן של עובדי כוכבים קלנדא סטרונייא וקרטסים ואמר רב חנין בר רבא קלנדא ח' ימים אחר תקופה סטרונייא שמונה ימים לפני תקופה וסימנך (תהלים קלט, ה) אחור וקדם צרתני}
\twocol{ואי סלקא דעתך הן ואידיהן עשרה הוו תנא כוליה קלנדא חד יומא הוא חשיב ליה
\par אמר רב אשי ת"ש לפני אידיהן של עובדי כוכבים שלשה ימים ואי ס"ד הן ואידיהן ליתני אידיהן של עובדי כוכבים שלשה ימים}
\twocol{וכי תימא האי דקתני לפני אידיהן למעוטי לאחר אידיהן ליתני אידם של עובדי כוכבים ג' ימים לפניהם אלא ש"מ הן בלא אידיהן ש"מ
\par איבעיא להו משום הרווחה או דלמא משום (ויקרא יט, יד) ולפני עור לא תתן מכשול}
\twocol{למאי נפקא מינה דאית ליה בהמה לדידיה אי אמרת משום הרווחה הא קא מרווח ליה אי אמרת משום עור לא תתן מכשול הא אית ליה לדידיה
\par וכי אית ליה לא עבר משום עור לא תתן מכשול והתניא אמר רבי נתן}
\twocol{מנין שלא יושיט אדם כוס של יין לנזיר ואבר מן החי לבני נח ת"ל (ויקרא יט, יד) ולפני עור לא תתן מכשול והא הכא דכי לא יהבינן ליה שקלי איהו וקעבר משום לפני עור לא תתן מכשול
\par הב"ע דקאי בתרי עברי נהרא דיקא נמי דקתני לא יושיט ולא קתני לא יתן ש"מ}
\twocol{איבעיא להו נשא ונתן מאי ר' יוחנן אמר נשא ונתן אסור ר"ל אמר נשא ונתן מותר איתיביה רבי יוחנן לריש לקיש אידיהן של עובדי כוכבים נשא ונתן אסורין מאי לאו לפני אידיהן לא אידיהן דוקא
\par א"ד איתיביה ר"ש בן לקיש לרבי יוחנן אידיהן של עובדי כוכבים נשא ונתן אסור אידיהן אין לפני אידיהן לא תנא אידי ואידי אידיהן קרי ליה}
\twocol{תניא כוותיה דר"ל כשאמרו אסור לשאת ולתת עמהם לא אסרו אלא בדבר המתקיים אבל בדבר שאינו מתקיים לא ואפילו בדבר המתקיים נשא ונתן מותר תני רב זביד בדבי רבי אושעיא דבר שאין מתקיים מוכרין להם אבל אין לוקחין מהם
\par ההוא מינאה דשדר ליה דינרא קיסרנאה לרבי יהודה נשיאה ביום אידו הוה יתיב ריש לקיש קמיה אמר היכי אעביד אשקליה אזיל ומודה לא אשקליה הויא ליה איבה א"ל ריש לקיש טול וזרוק אותו לבור בפניו אמר כל שכן דהויא ליה איבה כלאחר יד הוא דקאמינא:}
\twocol{להשאילן ולשאול מהן כו': בשלמא להשאילן דקא מרווח להו אבל לשאול מהן מעוטי קא ממעט להו אמר אביי גזרה לשאול מהן אטו להשאילן רבא אמר כולה משום דאזיל ומודה הוא:
\par להלוותם וללוות מהן: בשלמא להלוותם משום דקא מרווח להו אלא ללוות מהן אמאי אמר אביי גזרה ללוות מהן אטו להלוותם רבא אמר כולה משום דאזיל ומודה הוא:}
\twocol{לפורען ולפרוע מהן כו': בשלמא לפורען משום דקא מרווח להו אלא לפרוע מהן מעוטי ממעט להו אמר אביי גזירה לפרוע מהן אטו לפורען רבא אמר כולה משום דאזיל ומודה הוא
\par וצריכי דאי תנא לשאת ולתת עמהן משום דקא מרווח להו ואזיל ומודה אבל לשאול מהן דמעוטי קא ממעט להו שפיר דמי}
\twocol{ואי תנא לשאול מהן משום דחשיבא ליה מילתא ואזיל ומודה אבל ללוות מהן צערא בעלמא אית ליה אמר תוב לא הדרי זוזי
\par ואי תנא ללוות מהן משום דקאמר בעל כרחיה מיפרענא והשתא מיהא אזיל ומודה אבל ליפרע מהן דתו לא הדרי זוזי אימא צערא אית ליה ולא אזיל ומודה צריכא}
\twocol{רבי יהודה אומר נפרעין מהן כו': ולית ליה לרבי יהודה אף על פי שמיצר עכשיו שמח הוא לאחר זמן
\par והתניא רבי יהודה אומר אשה לא תסוד במועד מפני שניוול הוא לה ומודה ר' יהודה בסיד שיכולה לקפלו במועד שטופלתו במועד אע"פ שמצירה עכשיו שמחה היא לאחר זמן}
\twocol{אר"נ בר יצחק הנח להלכות מועד דכולהו מיצר עכשיו שמחה לאחר זמן רבינא אמר עובד כוכבים לענין פרעון לעולם מיצר
\par מתניתין דלא כר' יהושע בן קרחה דתניא ריב"ק אומר מלוה בשטר אין נפרעין מהן מלוה על פה נפרעין מהן מפני שהוא כמציל מידם}
\twocol{יתיב רב יוסף אחוריה דר' אבא ויתיב רבי אבא קמיה דרב הונא ויתיב וקאמר הלכתא כרבי יהושע בן קרחה והלכתא כר' יהודה
\par הלכתא כרבי יהושע הא דאמרן כר' יהודה דתניא הנותן צמר לצבע לצבוע לו אדום וצבעו שחור שחור וצבעו אדום}
\newsection{דף ז}
\twocol{ר"מ אומר נותן לו דמי צמרו רבי יהודה אומר אם השבח יתר על היציאה נותן לו את היציאה ואם היציאה יתירה על השבח נותן לו את השבח
\par אהדרינהו רב יוסף לאפיה בשלמא הלכה כרבי יהושע בן קרחה איצטריך ס"ד אמינא יחיד ורבים הלכה כרבים קא משמע לן הלכה כיחיד}
\twocol{אלא הלכה כרבי יהודה למה לי פשיטא דמחלוקת ואחר כך סתם הלכתא כסתם
\par מחלוקת בבבא קמא וסתם בבבא מציעא דתנן כל המשנה ידו על התחתונה וכל החוזר בו ידו על התחתונה}
\twocol{ורב הונא משום דאין סדר למשנה דאיכא למימר סתם תנא ברישא ואחר כך מחלוקת אי הכי כל מחלוקת ואחר כך סתם לימא אין סדר למשנה
\par ורב הונא כי לא אמרינן אין סדר בחדא מסכת' בתרי מסכתי אמרינן ורב יוסף כולה נזיקין חדא מסכת' היא}
\twocol{ואי בעית אימא משום דקתני לה גבי הלכתא פסיקתא כל המשנה ידו על התחתונה וכל החוזר בו ידו על התחתונה
\par ת"ר לא יאמר אדם לחבירו הנראה שתעמוד עמי לערב רבי יהושע בן קרחה אומר אומר אדם לחבירו הנראה שתעמוד עמי לערב אמר רבה בר בר חנה א"ר יוחנן הלכתא כרבי יהושע בן קרחה}
\twocol{ת"ר הנשאל לחכם וטימא לא ישאל לחכם ויטהר לחכם ואסר לא ישאל לחכם ויתיר
\par היו שנים אחד מטמא ואחד מטהר אחד אוסר ואחד מתיר אם היה אחד מהם גדול מחבירו בחכמה ובמנין הלך אחריו ואם לאו הלך אחר המחמיר ר' יהושע בן קרחה אומר בשל תורה הלך אחר המחמיר בשל סופרים הלך אחר המיקל א"ר יוסף הלכתא כרבי יהושע בן קרחה}
\twocol{ת"ר וכולן שחזרו בהן אין מקבלין אותן עולמית דברי ר"מ ר"י אומר חזרו בהן במטמוניות אין מקבלין אותן בפרהסיא מקבלין אותן א"ד עשו דבריהם במטמוניות מקבלין אותן
\par בפרהסיא אין מקבלין אותן ר"ש ור' יהושע בן קרחה אומרים בין כך ובין כך מקבלין שנאמר (ירמיהו ג, יד) שובו בנים שובבים א"ר יצחק איש כפר עכו א"ר יוחנן הלכתא כאותו הזוג:}
\twocol{{\large\emph{מתני׳}} רבי ישמעאל אומר שלשה לפניהם ושלשה לאחריהם אסור וחכ"א לפני אידיהן אסור לאחר אידיהן מותר
\par {\large\emph{גמ׳}} אמר רב תחליפא בר אבדימי אמר שמואל יום א' לדברי ר' ישמעאל לעולם אסור:}
\twocol{וחכ"א לפני אידיהן אסור לאחר אידיהן מותר כו': חכמים היינו ת"ק הן בלא אידיהן איכא בינייהו תנא קמא סבר הן בלא אידיהן ורבנן בתראי סברי הן ואידיהן
\par איבעית אימא נשא ונתן איכא בינייהו תנא קמא סבר נשא ונתן מותר ורבנן בתראי סברי נשא ונתן אסור}
\twocol{ואיבעית אימא דשמואל איכא בינייהו דאמר שמואל בגולה אין אסור אלא יום אידם תנא קמא אית ליה דשמואל רבנן בתראי לית להו דשמואל
\par איבעית אימא דנחום המדי איכא בינייהו דתניא נחום המדי אומר אינו אסור אלא יום אחד לפני אידיהן תנא קמא לית ליה דנחום המדי ורבנן בתראי אית להו דנחום המדי:}
\twocol{גופא נחום המדי אומר אינו אסור אלא יום אחד לפני אידיהן אמרו לו נשתקע הדבר ולא נאמר והאיכא רבנן בתראי דקיימי כוותיה מאן חכמים נחום המדי הוא
\par תניא אידך נחום המדי אומר מוכרין להן סוס זכר וזקן במלחמה אמרו לו נשתקע הדבר ולא נאמר}
\twocol{והאיכא בן בתירא דקאי כוותיה דתנן בן בתירא מתיר בסוס בן בתירא לא מפליג בין זכרים לנקבות איהו מדקא מפליג בין זכרים לנקבות כרבנן סבירא ליה ולרבנן נשתקע הדבר ולא נאמר
\par תניא נחום המדי אומר השבת מתעשר זרע וירק וזירין אמרו לו נשתקע הדבר ולא נאמר והאיכא ר"א דקאי כוותיה דתנן ר' אליעזר אומר השבת מתעשרת זרע וירק וזירין התם בדגנוניתא}
\twocol{אמר ליה רב אחא בר מניומי לאביי גברא רבה אתא מאתרין כל מילתא דאמר אמרי ליה נשתקע הדבר ולא נאמר אמר איכא חדא דעבדינן כוותיה דתניא נחום המדי אומר שואל אדם צרכיו בשומע תפלה
\par אמר בר מינה דההיא דתליא באשלי רברבי}
\twocol{דתניא ר' אליעזר אומר שואל אדם צרכיו ואחר כך יתפלל שנאמר (תהלים קב, א) תפלה לעני כי יעטוף ולפני ה' ישפוך שיחו וגו' אין שיחה אלא תפלה שנאמר (בראשית כד, סג) ויצא יצחק לשוח בשדה
\par ר' יהושע אומר יתפלל ואח"כ ישאל צרכיו שנאמר (תהלים קמב, ג) אשפוך לפניו שיחי צרתי לפניו אגיד}
\twocol{ור"א נמי הכתיב אשפוך לפניו שיחי הכי קאמר אשפוך לפניו שיחי בזמן שצרתי לפניו אגיד ור' יהושע נמי הכתיב תפלה לעני כי יעטוף הכי קאמר אימתי תפלה לעני בזמן שלפני ה' ישפוך שיחו
\par מכדי קראי לא כמר דייקי ולא כמר דייקי במאי קמיפלגי}
\twocol{כדדריש ר' שמלאי [דדריש ר' שמלאי] לעולם יסדר אדם שבחו של מקום ואח"כ יתפלל מנלן ממשה רבינו דכתיב (דברים ג, כד) ה' אלהים אתה החלות להראות את עבדך וגו' וכתיב בתריה אעברה נא ואראה את הארץ הטובה}
\newsection{דף ח}
\twocol{רבי יהושע סבר ילפינן ממשה ור"א סבר לא ילפינן ממשה שאני משה דרב גובריה וחכ"א לא כדברי זה ולא כדברי זה אלא שואל אדם צרכיו בשומע תפלה
\par אמר רב יהודה אמר שמואל הלכה שואל אדם צרכיו בשומע תפלה אמר רב יהודה בריה דרב שמואל בר שילת משמיה דרב אע"פ שאמרו שואל אדם צרכיו בשומע תפלה אבל אם בא לומר בסוף כל ברכה וברכה מעין כל ברכה וברכה אומר}
\twocol{א"ר חייא בר אשי אמר רב אע"פ שאמרו שואל אדם צרכיו בשומע תפלה אם יש לו חולה בתוך ביתו אומר בברכת חולים ואם צריך לפרנסה אומר בברכת השנים
\par אמר ר' יהושע בן לוי אע"פ שאמרו שואל אדם צרכיו בשומע תפלה אבל אם בא לומר אחר תפלתו אפילו כסדר יוה"כ אומר:}
\twocol{{\large\emph{מתני׳}} ואלו אידיהן של עובדי כוכבים קלנדא וסטרנורא וקרטיסים ויום גנוסיא של מלכיהם ויום הלידה ויום המיתה דברי רבי מאיר וחכמים אומרים כל מיתה שיש בה שריפה יש בה עבודת כוכבים ושאין בה שריפה אין בה עבודת כוכבים אבל יום תגלחת זקנו ובלוריתו ויום שעלה בו מן הים ויום שיצא מבית האסורין ועובד כוכבים שעשה משתה לבנו אינו אסור אלא אותו היום ואותו האיש בלבד:
\par {\large\emph{גמ׳}} אמר רב חנן בר רבא קלנדא ח' ימים אחר תקופה סטרנורא ח' ימים לפני תקופה וסימנך (תהלים קלט, ה) אחור וקדם צרתני וגו'}
\twocol{ת"ר לפי שראה אדם הראשון יום שמתמעט והולך אמר אוי לי שמא בשביל שסרחתי עולם חשוך בעדי וחוזר לתוהו ובוהו וזו היא מיתה שנקנסה עלי מן השמים עמד וישב ח' ימים בתענית [ובתפלה]
\par כיון שראה תקופת טבת וראה יום שמאריך והולך אמר מנהגו של עולם הוא הלך ועשה שמונה ימים טובים לשנה האחרת עשאן לאלו ולאלו ימים טובים הוא קבעם לשם שמים והם קבעום לשם עבודת כוכבים}
\twocol{בשלמא למ"ד בתשרי נברא העולם יומי זוטי חזא יומי אריכי אכתי לא חזא אלא למ"ד בניסן נברא העולם הא חזא ליה יומי זוטי ויומי אריכי דהוי זוטי כולי האי לא חזא
\par ת"ר יום שנברא בו אדם הראשון כיון ששקעה עליו חמה אמר אוי לי שבשביל שסרחתי עולם חשוך בעדי ויחזור עולם לתוהו ובוהו וזו היא מיתה שנקנסה עלי מן השמים היה יושב בתענית ובוכה כל הלילה וחוה בוכה כנגדו כיון שעלה עמוד השחר אמר מנהגו של עולם הוא עמד והקריב שור שקרניו קודמין לפרסותיו שנאמר (תהלים סט, לב) ותיטב לה' משור פר מקרין מפריס}
\twocol{ואמר רב יהודה אמר שמואל שור שהקריב אדם הראשון קרן אחת היתה [לו] במצחו שנאמר ותיטב לה' משור פר מקרין מפריס מקרין תרתי משמע אמר רב נחמן בר יצחק מקרן כתיב
\par אמר רב מתנה רומי שעשתה קלנדא וכל העיירות הסמוכות לה משתעבדות לה אותן עיירות אסורות או מותרות רבי יהושע בן לוי אמר קלנדא אסורה לכל היא רבי יוחנן אמר אין אסורה אלא לעובדיה בלבד}
\twocol{תנא כוותיה דר' יוחנן אע"פ שאמרו רומי עשתה קלנדא וכל עיירות הסמוכות לה משתעבדות לה היא עצמה אינה אסורה אלא לעובדיה בלבד
\par סטרנליא וקרטסים ויום גנוסיא של מלכיהם ויום שהומלך בו מלך לפניו אסור אחריו מותר ועובד כוכבים שעשה (בו) משתה לבנו אין אסור אלא אותו היום ואותו האיש}
\twocol{אמר רב אשי אף אנן נמי תנינא דקתני יום תגלחת זקנו ובלוריתו ויום שעלה בו מן הים ויום שיצא בו מבית האסורין אין אסור אלא אותו היום בלבד ואותו האיש
\par בשלמא אותו היום לאפוקי לפניו ולאחריו אלא אותו האיש לאפוקי מאי לאו לאפוקי משעבדיו ש"מ}
\twocol{תניא רבי ישמעאל אומר ישראל שבחוצה לארץ עובדי עבודת כוכבים בטהרה הן כיצד עובד כוכבים שעשה משתה לבנו וזימן כל היהודים שבעירו אע"פ שאוכלין משלהן ושותין משלהן ושמש שלהן עומד לפניהם מעלה עליהם הכתוב כאילו אכלו מזבחי מתים שנאמר (שמות לד, טו) וקרא לך ואכלת מזבחו
\par ואימא עד דאכיל אמר רבא אם כן נימא קרא ואכלת מזבחו מאי וקרא לך משעת קריאה הלכך}
\twocol{כל תלתין יומין בין א"ל מחמת הלולא ובין לא א"ל מחמת הלולא אסור מכאן ואילך אי א"ל מחמת הלולא אסור ואי לא אמר ליה מחמת הלולא שרי
\par וכי א"ל מחמת הלולא עד אימת אמר רב פפא עד תריסר ירחי שתא ומעיקרא מאימת אסור אמר רב פפא משמיה דרבא מכי רמו שערי באסינתי}
\twocol{ולבתר תריסר ירחי שתא שרי והא רב יצחק בריה דרב משרשיא איקלע לבי ההוא עובד כוכבים לבתר תריסר ירחי שתא ושמעיה דאודי ופירש ולא אכל שאני רב יצחק בריה דרב משרשיא דאדם חשוב הוא:
\par וקרטסים וכו': מאי קרטסים אמר רב יהודה אמר שמואל יום שתפסה בו רומי מלכות והתניא קרטסים ויום שתפסה בו רומי מלכות אמר רב יוסף שתי תפיסות תפסה רומי אחת בימי קלפטרא מלכתא ואחת שתפסה בימי יונים}
\twocol{דכי אתא רב דימי אמר תלתין ותרין קרבי עבדו רומאי בהדי יונאי ולא יכלו להו עד דשתפינהו לישראל בהדייהו והכי אתנו בהדייהו אי מינן מלכי מנייכו הפרכי אי מנייכו מלכי מינן הפרכי
\par ושלחו להו רומאי ליונאי עד האידנא עבידנא בקרבא השתא נעביד בדינא מרגלית ואבן טובה איזו מהן יעשה בסיס לחבירו שלחו להו מרגלית לאבן טובה}
\twocol{אבן טובה (ואינך) איזו מהן יעשה בסיס לחבירו אבן טובה לאינך אינך וספר תורה איזו מהן יעשה בסיס לחבירו אינך לספר תורה
\par שלחו להו [א"כ] אנן ספר תורה גבן וישראל בהדן כפו להו עשרין ושית שנין קמו להו בהימנותייהו בהדי ישראל מכאן ואילך אישתעבדו בהו}
\twocol{מעיקרא מאי דרוש ולבסוף מאי דרוש מעיקרא דרוש (בראשית לג, יב) נסעה ונלכה ואלכה לנגדך ולבסוף דרוש (בראשית לג, יד) יעבר נא אדני לפני עבדו
\par עשרין ושית שנין דקמו בהימנותייהו בהדי ישראל מנא לן דאמר רב כהנא כשחלה רבי ישמעאל בר יוסי שלחו ליה רבי אמור לנו שנים וג' דברים שאמרת לנו משום אביך}
\twocol{אמר להו מאה ושמנים שנה קודם שנחרב הבית פשטה מלכות הרשעה על ישראל פ' שנה עד לא חרב הבית גזרו טומאה על ארץ העמים ועל כלי זכוכית מ' שנה עד לא חרב הבית גלתה סנהדרין וישבה לה בחנות
\par למאי הלכתא א"ר יצחק בר אבדימי לומר שלא דנו דיני קנסות דיני קנסות סלקא דעתך והאמר רב יהודה אמר רב ברם זכור אותו האיש לטוב ורבי יהודה בן בבא שמו שאלמלא הוא נשתכחו דיני קנסות מישראל נשתכחו לגרסינהו}
\twocol{אלא בטלו דיני קנסות מישראל שגזרה מלכות הרשעה גזרה כל הסומך יהרג וכל הנסמך יהרג ועיר שסומכין בה תחרב ותחום שסומכין בו יעקר
\par מה עשה רבי יהודה בן בבא הלך וישב בין שני הרים גדולים ובין שתי עיירות גדולות בין ב' תחומי שבת בין אושא לשפרעם וסמך שם חמשה זקנים ר"מ ור' יהודה ור' יוסי ור"ש ורבי אלעזר בן שמוע ורב אויא מוסיף אף רבי נחמיה}
\twocol{כיון שהכירו בהם אויבים אמר להם בני רוצו אמרו לו רבי ואתה מה תהא עליך אמר להם הריני מוטל לפניהם כאבן שאין לה הופכין אמרו לא זזו משם עד שנעצו לגופו ג' מאות לולניאות של ברזל ועשאוהו לגופו ככברה
\par אמר רב נחמן בר יצחק לא תימא דיני קנסות אלא שלא דנו דיני נפשות}
\twocol{מ"ט כיון דחזו דנפישי להו רוצחין ולא יכלי למידן אמרו מוטב נגלי ממקום למקום כי היכי דלא ליחייבו
\par דכתיב (דברים יז, י) ועשית על פי הדבר אשר יגידו לך מן המקום ההוא מלמד שהמקום גורם:}
\twocol{מאה ושמנים ותו לא והתני רבי יוסי ברבי}
\newsection{דף ט}
\twocol{מלכות פרס בפני הבית שלשים וארבע שנה מלכות יון בפני הבית מאה ושמונים שנה מלכות חשמונאי בפני הבית מאה ושלש מלכות בית הורדוס מאה ושלש מכאן ואילך צא וחשוב כמה שנים אחר חורבן הבית
\par אלמא מאתן ושית הוו ואת אמרת מאה ושמונים הוו אלא עשרין ושית שנין קמו בהימנותייהו בהדי ישראל ולא אישתעבדו בהו ואמטו להכי לא קא חשיב להו כשפשטה מלכות הרשעה על ישראל}
\twocol{אמר רב פפא אי טעי האי תנא ולא ידע פרטי כמה הוה לישייליה לספרא כמה כתיב וניטפי עלייהו עשרין שנין ומשכח ליה לחומריה וסימניך (בראשית לא, מא) זה לי עשרים שנה אנכי בביתך
\par אי טעי ספרא נשייליה לתנא כמה חשיב ונבצר מינייהו עשרין שנין ומשכח ליה לחומריה וסימניך ספרא בצירא תנא תוספאה}
\twocol{תנא דבי אליהו ששת אלפים שנה הוי העולם שני אלפים תוהו שני אלפים תורה שני אלפים ימות המשיח בעונותינו שרבו יצאו מהן מה שיצאו מהן
\par שני אלפים תורה מאימת אי נימא ממתן תורה עד השתא ליכא כולי האי דכי מעיינת בהו תרי אלפי פרטי דהאי אלפא הוא דהואי}
\twocol{אלא (בראשית יב, ה) מואת הנפש אשר עשו בחרן וגמירי דאברהם בההיא שעתא בר חמשין ותרתי הוה
\par כמה בצרן מדתני תנא ארבע מאה וארבעים ותמניא שנין הויין כי מעיינת ביה מהנפש אשר עשו בחרן עד מתן תורה ארבע מאה וארבעים ותמניא שנין הויין}
\twocol{אמר רב פפא אי טעי תנא ולא ידע פרטיה כמה הוי לישייליה לספרא כמה כתיב וניטפי עלייהו ארבעין ותמני ומשכח ליה לחומריה וסימניך
\par (במדבר לה, ז) ארבעים ושמונה עיר ואי טעי ספרא נשייליה לתנא כמה קתני וניבצר מינייהו ארבעים ושמונה ומשכח ליה לחומריה וסימניך ספרא בצירא תנא תוספאה}
\twocol{אמר רב הונא בריה דרב יהושע האי מאן דלא ידע כמה שני בשבוע הוא עומד ניטפי חד שתא ונחשוב כללי ביובלי ופרטי בשבועי
\par ונשקל ממאה תרי ונשדי אפרטי ונחשובינהו לפרטי בשבועי וידע כמה שני בשבוע וסימניך (בראשית מה, ו) כי זה שנתים הרעב בקרב הארץ}
\twocol{אמר רבי חנינא אחר ארבע מאות לחורבן הבית אם יאמר לך אדם קח שדה שוה אלף דינרים בדינר אחד לא תקח במתניתא תנא אחר ארבעת אלפים ומאתים ושלשים ואחת שנה לבריאת עולם אם יאמר לך אדם קח לך שדה שוה אלף דינרים בדינר אחד אל תקח מאי בינייהו איכא בינייהו תלת שנין דמתניתא טפיא תלת שני
\par ההוא שטרא דהוה כתיב ביה}
\newsection{דף י}
\twocol{שית שנין יתירתא סבור רבנן קמיה דרב' למימר האי שטר מאוחר הוא ניעכביה עד דמטיא זמניה ולא טריף אמר רב נחמן האי ספרא דוקנא כתביה והנך שית שנין דמלכו בעילם דאנן לא חשבינן להו הוא קחשיב ליה ובזמניה כתביה
\par דתניא ר' יוסי אומר שש שנים מלכו בעילם ואח"כ פשטה מלכותן בכל העולם כולו:}
\twocol{מתקיף לה רב אחא בר יעקב ממאי דלמלכות יונים מנינן דלמא ליציאת מצרים מנינן ושבקיה לאלפא קמא ונקטיה אלפא בתרא והאי מאוחר הוא אמר רב נחמן בגולה אין מונין אלא למלכי יונים בלבד
\par הוא סבר דחויי קא מדחי ליה נפק דק ואשכח דתניא בגולה אין מונין אלא למלכי יונים בלבד}
\twocol{אמר רבינא מתניתין נמי דיקא דתנן באחד בניסן ר"ה למלכים ולרגלים ואמרינן למלכים למאי הלכתא אמר רב חסדא לשטרות
\par ותנן באחד בתשרי ר"ה לשנים ולשמיטין ואמרינן לשנים למאי הלכתא ואמר רב חסדא לשטרות קשיא שטרות אהדדי}
\twocol{ומשנינן כאן למלכי ישראל כאן למלכי עובדי כוכבים למלכי עובדי כוכבים מתשרי מנינן למלכי ישראל מניסן מנינן
\par ואנן השתא מתשרי מנינן ואי ס"ד ליציאת מצרים מנינן מניסן בעינן למימני אלא לאו ש"מ למלכי יונים מנינן ש"מ:}
\twocol{ויום גינוסיא של מלכיהם וכו': מאי ויום גינוסיא של מלכיהם אמר רב יהודה יום שמעמידין בו עובדי כוכבים את מלכם והתניא יום גינוסיא ויום שמעמידין בו את מלכם לא קשיא הא דידיה הא דבריה
\par ומי מוקמי מלכא בר מלכא והתני רב יוסף (עובדיה א, ב) הנה קטן נתתיך בגוים שאין מושיבין מלך בן מלך (עובדיה א, ב) בזוי אתה מאד שאין להן לא כתב ולא לשון אלא מאי יום גינוסיא יום הלידה}
\twocol{והתניא יום גינוסיא ויום הלידה לא קשיא הא דידיה הא דבריה
\par והתניא יום גינוסיא שלו יום גינוסיא של בנו ויום הלידה שלו ויום הלידה של בנו אלא מאי יום גינוסיא יום שמעמידין בו מלכם ולא קשיא הא דידיה הא דבריה}
\twocol{ואי קשיא לך דלא מוקמי מלכא בר מלכא ע"י שאלה מוקמי כגון אסוירוס בר אנטונינוס דמלך
\par א"ל אנטונינוס לרבי בעינא דימלוך אסוירוס ברי תחותי ותתעביד טבריא קלניא ואי אימא להו חדא עבדי תרי לא עבדי אייתי גברא ארכביה אחבריה ויהב ליה יונה לעילאי בידיה וא"ל לתתאה אימר לעילא דלמפרח מן ידיה יונה אמר שמע מינה הכי קאמר לי את בעי מינייהו דאסוירוס ברי ימלוך תחותי ואימא ליה לאסוירוס דתעביד טבריא קלניא}
\twocol{א"ל מצערין לי חשובי [רומאי] מעייל ליה לגינא כל יומא עקר ליה פוגלא ממשרא קמיה אמר ש"מ הכי קאמר לי את קטול חד חד מינייהו ולא תתגרה בהו בכולהו
\par ולימא ליה מימר [בהדיא] אמר שמעי (בי) חשובי רומי ומצערו ליה ולימא ליה בלחש משום דכתיב (קהלת י, כ) כי עוף השמים יוליך את הקול}
\twocol{הוה ליה ההוא ברתא דשמה גירא קעבדה איסורא שדר ליה גרגירא שדר ליה כוסברתא שדר ליה כרתי שלח ליה חסא
\par כל יומא הוה שדר ליה דהבא פריכא במטראתא וחיטי אפומייהו אמר להו אמטיו חיטי לרבי אמר [ליה רבי] לא צריכנא אית לי טובא אמר ליהוו למאן דבתרך דיהבי לבתראי דאתו בתרך ודאתי מינייהו ניפוק עלייהו}
\twocol{ה"ל ההיא נקרתא דהוה עיילא מביתיה לבית רבי כל יומא הוה מייתי תרי עבדי חד קטליה אבבא דבי רבי וחד קטליה אבבא דביתיה א"ל בעידנא דאתינא לא נשכח גבר קמך
\par יומא חד אשכחיה לר' חנינא בר חמא דהוה יתיב אמר לא אמינא לך בעידנא דאתינא לא נשכח גבר קמך א"ל לית דין בר איניש א"ל אימא ליה לההוא עבדא דגני אבבא דקאים וליתי}
\twocol{אזל ר' חנינא בר חמא אשכחיה דהוה קטיל אמר היכי אעביד אי איזיל ואימא ליה דקטיל אין משיבין על הקלקלה אשבקיה ואיזיל קא מזלזלינן במלכותא בעא רחמי עליה ואחייה ושדריה אמר ידענא זוטי דאית בכו מחיה מתים מיהו בעידנא דאתינא לא נשכח איניש קמך
\par כל יומא הוה משמש לרבי מאכיל ליה משקי ליה כי הוה בעי רבי למיסק לפוריא הוה גחין קמי פוריא א"ל סק עילואי לפורייך אמר לאו אורח ארעא לזלזולי במלכותא כולי האי אמר מי ישימני מצע תחתיך לעולם הבא}
\twocol{א"ל אתינא לעלמא דאתי א"ל אין א"ל והכתיב (עובדיה א, יח) לא יהיה שריד לבית עשו בעושה מעשה עשו
\par תניא נמי הכי לא יהיה שריד לבית עשו יכול לכל ת"ל לבית עשו בעושה מעשה עשו}
\twocol{א"ל והכתיב (יחזקאל לב, כט) שמה אדום מלכיה וכל נשיאיה א"ל מלכיה ולא כל מלכיה כל נשיאיה ולא כל שריה
\par תניא נמי הכי מלכיה ולא כל מלכיה כל נשיאיה ולא כל שריה מלכיה ולא כל מלכיה פרט לאנטונינוס בן אסוירוס כל נשיאיה ולא כל שריה פרט לקטיעה בר שלום}
\twocol{קטיעה בר שלום מאי הוי דההוא קיסרא דהוה סני ליהודאי אמר להו לחשיבי דמלכותא מי שעלה לו נימא ברגלו יקטענה ויחיה או יניחנה ויצטער אמרו לו יקטענה ויחיה
\par אמר להו קטיעה בר שלום חדא דלא יכלת להו לכולהו דכתיב (זכריה ב, י) כי כארבע רוחות השמים פרשתי אתכם מאי קאמר אלימא דבדרתהון בד' רוחות האי כארבע רוחות לארבע רוחות מבעי ליה אלא כשם שא"א לעולם בלא רוחות כך א"א לעולם בלא ישראל ועוד קרו לך מלכותא קטיעה}
\twocol{א"ל מימר שפיר קאמרת מיהו כל דזכי (מלכא) שדו ליה לקמוניא חלילא כד הוה נקטין ליה ואזלין אמרה ליה ההיא מטרוניתא ווי ליה לאילפא דאזלא בלא מכסא נפל על רישא דעורלתיה קטעה אמר יהבית מכסי חלפית ועברית כי קא שדו ליה אמר כל נכסאי לר"ע וחביריו יצא ר"ע ודרש (שמות כט, כח) והיה לאהרן ולבניו מחצה לאהרן ומחצה לבניו
\par יצתה בת קול ואמרה קטיעה בר שלום מזומן לחיי העוה"ב בכה רבי ואמר יש קונה עולמו בשעה אחת ויש קונה עולמו בכמה שנים}
\twocol{אנטונינוס שמשיה לרבי אדרכן שמשיה לרב כי שכיב אנטונינוס א"ר נתפרדה חבילה כי שכיב אדרכן אמר רב}
\newsection{דף יא}
\twocol{נתפרדה חבילה
\par אונקלוס בר קלונימוס איגייר שדר קיסר גונדא דרומאי אבתריה משכינהו בקראי איגיור הדר שדר גונדא דרומאי [אחרינא] אבתריה אמר להו לא תימרו ליה ולא מידי}
\twocol{כי הוו שקלו ואזלו אמר להו אימא לכו מילתא בעלמא ניפיורא נקט נורא קמי פיפיורא פיפיורא לדוכסא דוכסא להגמונא הגמונא לקומא קומא מי נקט נורא מקמי אינשי אמרי ליה לא אמר להו הקב"ה נקט נורא קמי ישראל דכתיב (שמות יג, כא) וה' הולך לפניהם יומם וגו' איגיור [כולהו]
\par הדר שדר גונדא אחרינא אבתריה אמר להו לא תשתעו מידי בהדיה כי נקטי ליה ואזלי חזא מזוזתא [דמנחא אפתחא] אותיב ידיה עלה ואמר להו מאי האי אמרו ליה אימא לן את}
\twocol{אמר להו מנהגו של עולם מלך בשר ודם יושב מבפנים ועבדיו משמרים אותו מבחוץ ואילו הקב"ה עבדיו מבפנים והוא משמרן מבחוץ שנאמר (תהלים קכא, ח) ה' ישמר צאתך ובואך מעתה ועד עולם איגיור תו לא שדר בתריה
\par (בראשית כה, כג) ויאמר ה' לה שני גוים בבטנך אמר רב יהודה אמר רב אל תקרי גוים אלא גיים זה אנטונינוס ורבי שלא פסקו מעל שולחנם לא חזרת ולא קישות ולא צנון לא בימות החמה ולא בימות הגשמים דאמר מר צנון מחתך אוכל חזרת מהפך מאכל קישות מרחיב מעיים}
\twocol{והא תנא דבי רבי ישמעאל למה נקרא שמן קישואין מפני שקשין לגופו של אדם כחרבות לא קשיא הא ברברבי הא בזוטרי:
\par יום הלידה ויום המיתה: מכלל דר"מ סבר לא שנא מיתה שיש בה שריפה ולא שנא מיתה שאין בה שריפה פלחי בה לעבודת כוכבים אלמא שריפה לאו חוקה היא מכלל דרבנן סברי שריפה חוקה היא}
\twocol{והא תניא שורפין על המלכים ולא מדרכי האמורי ואי חוקה היא אנן היכי שרפינן והכתיב (ויקרא יח, ג) ובחוקותיהם לא תלכו
\par אלא דכ"ע שריפה לאו חוקה היא אלא חשיבותא היא והכא בהא קמיפלגי ר"מ סבר לא שנא מיתה שיש בה שריפה ולא שנא מיתה שאין בה שריפה פלחי בה לעבודת כוכבים ורבנן סברי מיתה שיש בה שריפה חשיבא להו ופלחי בה ושאין בה שריפה לא חשיבא ולא פלחי בה}
\twocol{גופא שורפין על המלכים ואין בו משום דרכי האמורי שנאמר (ירמיהו לד, ה) בשלום תמות ובמשרפות אבותיך המלכים וגו' וכשם ששורפין על המלכים כך שורפין על הנשיאים
\par ומה הם שורפין על המלכים מיטתן וכלי תשמישן ומעשה שמת ר"ג הזקן ושרף עליו אונקלוס הגר שבעים מנה צורי והאמרת מה הן שורפין עליהם מיטתן וכלי תשמישן אימא בשבעים מנה צורי}
\twocol{ומידי אחרינא לא והתניא עוקרין על המלכים ואין בו משום דרכי האמורי אמר רב פפא סוס שרכב עליו
\par ובהמה טהורה לא והתניא עיקור שיש בה טריפה אסור ושאין בה טריפה מותר ואיזהו עיקור שאין בה טריפה}
\twocol{המנשר פרסותיה מן הארכובה ולמטה תרגמא רב פפא בעגלה המושכת בקרון:
\par יום תגלחת זקנו: איבעיא להו היכי קתני יום תגלחת זקנו והנחת בלוריתו או דלמא יום תגלחת זקנו והעברת בלוריתו ת"ש דתני' תרוייהו יום תגלחת זקנו והנחת בלוריתו יום תגלחת זקנו והעברת בלוריתו}
\twocol{אמר רב יהודה אמר שמואל עוד אחרת יש [להם] ברומי אחת לשבעים שנה מביאין אדם שלם ומרכיבין אותו על אדם חיגר ומלבישין אותו בגדי אדם הראשון ומניחין לו בראשו קרקיפלו של רבי ישמעאל
\par ותלו ליה [בצואריה] מתקל [ר'] זוזא דפיזא ומחפין את השווקים באינך ומכריזין לפניו סך קירי פלסתר אחוה דמרנא זייפנא דחמי חמי ודלא חמי לא חמי מאי אהני לרמאה ברמאותיה ולזייפנא בזייפנותיה ומסיימין בה הכי ווי לדין כד יקום דין}
\twocol{אמר רב אשי הכשילן פיהם לרשעים אי אמרו זייפנא אחוה דמרנא כדקאמרי השתא דאמרי דמרנא זייפנא מרנא גופיה זייפנא הוא
\par ותנא דידן מ"ט לא קחשיב לה להאי דאיתא בכל שתא ושתא קחשיב דליתא בכל שתא ושתא לא קחשיב}
\twocol{הני דרומאי ודפרסאי מאי מוטרדי וטוריסקי מוהרנקי ומוהרין הני דפרסאי ודרומאי דבבלאי מאי מוהרנקי ואקניתי' בחנוני ועשר באדר
\par אמר רב חנן בר רב חסדא אמר רב ואמרי לה א"ר חנן בר רבא אמר רב חמשה בתי עבודת כוכבים קבועין הן אלו הן בית בל בבבל בית נבו בכורסי תרעתא שבמפג צריפא שבאשקלון נשרא שבערביא כי אתא רב דימי הוסיפו עליהן יריד שבעין בכי נדבכה שבעכו איכא דאמרי נתברא שבעכו רב דימי מנהרדעא מתני איפכא יריד שבעכו נדבכה שבעין בכי}
\twocol{א"ל רב חנן בר רב חסדא לרב חסדא מאי קבועין הן א"ל הכי אמר אבוה דאימך קבועין הן לעולם תדירא כולה שתא פלחי להו
\par אמר שמואל בגולה אינו אסור אלא יום אידם בלבד ויום אידם נמי מי אסיר והא רב יהודה שרא ליה לרב ברונא לזבוני חמרא ולרב גידל לזבוני חיטין בחגתא דטייעי שאני חגתא דטייעי דלא קביעא:}
\twocol{{\large\emph{מתני׳}} עיר שיש בה עבודת כוכבים חוצה לה מותר היה חוצה לה עבודת כוכבים תוכה לה מותר מהו לילך לשם בזמן שהדרך מיוחדת לאותו מקום אסור ואם היה יכול להלך בה למקום אחר מותר:
\par {\large\emph{גמ׳}} ה"ד חוצה לה אמר רשב"ל משום ר' חנינא כגון עטלוזא של עזה וא"ד בעא מיניה רשב"ל מר"ח עטלוזא של עזה מהו א"ל לא הלכת לצור מימיך וראית ישראל ועובד כוכבים}
\newsection{דף יב}
\twocol{ששפתו שתי קדירות על גבי כירה אחת ולא חשו להם חכמים מאי לא חשו להם חכמים
\par אמר אביי משום בשר נבילה לא אמרינן דלמא מהדר אפיה ישראל לאחוריה ושדי עובד כוכבים נבילה בקדירה דכוותה ה"נ לא חשו להם חכמים משום דמי עבודת כוכבים}
\twocol{רבא אמר מאי לא חשו להם חכמים משום בישולי עובדי כוכבים
\par דכוותה ה"נ לא חשו להם חכמים משום יום אידם}
\twocol{רבה בר עולא אמר לא חשו להם חכמים משום צינורא
\par דכוותה ה"נ לא חשו להם חכמים משום לפני אידיהן:}
\twocol{מהו לילך לשם וכו': ת"ר עיר שיש בה עבודת כוכבים אסור ליכנס לתוכה ולא מתוכה לעיר אחרת דברי רבי מאיר וחכ"א כל זמן שהדרך מיוחדת לאותו מקום אסור אין הדרך מיוחדת לאותו מקום מותר
\par ישב לו קוץ בפני עבודת כוכבים לא ישחה ויטלנה מפני שנראה כמשתחוה לעבודת כוכבים ואם אינו נראה מותר נתפזרו לו מעותיו בפני עבודת כוכבים לא ישחה ויטלם מפני שנראה כמשתחוה לעבודת כוכבים ואם אינו נראה מותר}
\twocol{מעיין המושך לפני עבודת כוכבים לא ישחה וישתה מפני שנראה כמשתחוה לעבודת כוכבים ואם אינו נראה מותר פרצופות המקלחין מים לכרכין לא יניח פיו על פיהם וישתה מפני שנראה כמנשק לעבודת כוכבים כיוצא בו לא יניח פיו על סילון וישתה מפני הסכנה
\par מאי אינו נראה אילימא דלא מתחזי והאמר רב יהודה אמר רב כל מקום שאסרו חכמים מפני מראית העין אפילו בחדרי חדרים אסור אלא אימא אם אינו נראה כמשתחוה לעבודת כוכבים מותר}
\twocol{וצריכא דאי תנא קוץ משום דאפשר למיזל קמיה ומשקליה אבל מעות דלא אפשר אימא לא
\par ואי תנא מעות דממונא אבל קוץ דצערא אימא לא ואי תנא הני תרתי משום דליכא סכנה אבל מעיין דאיכא סכנה דאי לא שתי מיית אימא לא צריכא}
\twocol{פרצופות ל"ל משום דקבעי למיתני כיוצא בו לא יניח פיו על גבי הסילון וישתה מפני הסכנה
\par מאי סכנה עלוקה ת"ר לא ישתה אדם מים לא מן הנהרות ולא מן האגמים לא בפיו ולא בידו אחת ואם שתה דמו בראשו מפני הסכנה מאי סכנה סכנת עלוקה}
\twocol{מסייע ליה לרבי חנינא דאמר רבי חנינא הבולע נימא של מים מותר להחם לו חמין בשבת ומעשה באחד שבלע נימא של מים והתיר רבי נחמיה להחם לו חמין בשבת אדהכי והכי אמר רב הונא בריה דרב יהושע ליגמע חלא
\par אמר רב אידי בר אבין האי מאן דבלע זיבורא מחייא לא חיי מיהו לשקייה רביעתא דחלא שמגז אפשר דחיי פורתא עד דמפקיד אביתיה}
\twocol{ת"ר לא ישתה אדם מים בלילה ואם שתה דמו בראשו מפני הסכנה מאי סכנה סכנת שברירי ואם צחי מאי תקנתיה אי איכא אחרינא בהדיה ליתרייה ולימא ליה צחינא מיא ואי לא נקרקש בנכתמא אחצבא ונימא איהו לנפשיה פלניא בר פלניתא אמרה לך אימך אזדהר משברירי ברירי רירי ירי רי בכסי חיורי:
\par {\large\emph{מתני׳}} עיר שיש בה עבודת כוכבים והיו בה חנויות מעוטרות ושאינן מעוטרות זה היה מעשה בבית שאן ואמרו חכמים המעוטרות אסורות ושאינן מעוטרות מותרות:}
\twocol{{\large\emph{גמ׳}} אמר רשב"ל לא שנו אלא מעוטרות בוורד והדס דקא מתהני מריחא אבל מעוטרות בפירות מותרות מאי טעמא דאמר קרא (דברים יג, יח) לא ידבק בידך מאומה מן החרם נהנה הוא דאסור}
\newsection{דף יג}
\twocol{אבל מהנה שרי ורבי יוחנן אמר אפילו מעוטרות בפירות נמי אסור ק"ו נהנה אסור מהנה לא כ"ש
\par מיתיבי רבי נתן אומר יום שעבודת כוכבים מנחת בו את המכס מכריזין ואומרים כל מי שנוטל עטרה ויניח בראשו ובראש חמורו לכבוד עבודת כוכבים יניח לו את המכס ואם לאו אל יניח לו את המכס}
\twocol{יהודי שנמצא שם מה יעשה יניח נמצא נהנה לא יניח נמצא מהנה
\par מכאן אמרו הנושא ונותן בשוק של עבודת כוכבים בהמה תיעקר פירות כסות וכלים ירקבו מעות וכלי מתכות יוליכם לים המלח ואיזהו עיקור המנשר פרסותיה מן הארכובה ולמטה}
\twocol{קתני מיהת יניח נמצא נהנה לא יניח נמצא מהנה
\par אמר רב משרשיא בריה דרב אידי קסבר רשב"ל פליגי רבנן עליה דרבי נתן ואנא דאמרי כרבנן דפליגי עליה ור' יוחנן סבר לא פליגי}
\twocol{ולא פליגי והא תניא הולכין ליריד של עובדי כוכבים ולוקחין מהם בהמה עבדים ושפחות בתים ושדות וכרמים וכותב ומעלה בערכאות שלהן מפני שהוא כמציל מידם
\par ואם היה כהן מטמא בחוצה לארץ לדון ולערער עמהם וכשם שמטמא בחוצה לארץ כך מטמא בבית הקברות}
\twocol{בבית הקברות סלקא דעתך טומאה דאורייתא היא אלא בית הפרס דרבנן
\par ומטמא ללמוד תורה ולישא אשה א"ר יהודה אימתי בזמן שאין מוצא ללמוד אבל בזמן שמוצא ללמוד אינו מטמא}
\twocol{רבי יוסי אומר אפילו בזמן שמוצא ללמוד יטמא לפי שאין אדם זוכה ללמוד מכל
\par א"ר יוסי מעשה ביוסף הכהן שהלך אחר רבו לצידן ללמוד תורה ואמר רבי יוחנן הלכה כרבי יוסי}
\twocol{אלמא פליגי אמר לך רבי יוחנן לעולם לא פליגי
\par ולא קשיא כאן בלוקח מן התגר דשקלי מיכסא מיניה כאן בלוקח מבעל הבית דלא שקלי מיכסא מיניה}
\twocol{אמר מר בהמה תיעקר והא איכא צער בעלי חיים אמר אביי אמר רחמנא (יהושע יא, ו) את סוסיהם תעקר
\par אמר מר ואיזוהי עיקור מנשר פרסותיה מן הארכובה ולמטה ורמינהי אין מקדישין ואין מחרימין ואין מעריכין בזה"ז ואם הקדיש והחרים והעריך בהמה תיעקר פירות כסות וכלים}
\twocol{ירקבו מעות וכלי מתכות יוליכם לים המלח ואיזהו עיקור נועל דלת בפניה והיא מתה מאיליה
\par אמר אביי שאני התם משום בזיון קדשים ונשחטיה מישחט אתו בהו לידי תקלה}
\twocol{ולישויה גיסטרא אמר אביי אמר קרא (דברים יב, ג) ונתצתם את מזבחותם [וגו'] לא תעשון כן לה' אלהיכם
\par רבא אמר מפני שנראה כמטיל מום בקדשים נראה מום מעליא הוא ה"מ בזמן שבית המקדש קיים דחזי להקרבה השתא דלא חזי להקרבה לית לן בה}
\twocol{וליהוי כמטיל מום בבעל מום דאע"ג דלא חזי להקרבה אסור בעל מום נהי דלא חזי לגופיה לדמי חזי לאפוקי הכא דלא לדמי חזי ולא לגופיה חזי
\par אשכחיה רבי יונה לרבי עילאי דקאי אפיתחא דצור א"ל קתני בהמה תיעקר עבד מאי עבד ישראל לא קא מיבעיא לי כי קא מיבעיא לי עבד עובד כוכבים מאי א"ל מאי קא מיבעיא לך תניא העובדי כוכבים והרועי בהמה דקה לא מעלין ולא מורידין}
\twocol{א"ל ר' ירמיה לר' זירא קתני לוקחין מהן בהמה עבדים ושפחות עבד ישראל או דלמא אפי' עבד עובד כוכבים א"ל מסתברא עבד ישראל דאי עבד עובד כוכבים למאי מיבעי ליה כי אתא רבין אמר רבי שמעון בן לקיש אפילו עבד עובד כוכבים מפני שמכניסו תחת כנפי השכינה
\par אמר רב אשי אטו בהמה מאי מכניס תחת כנפי השכינה איכא אלא משום מעוטייהו וה"נ דממעטי שרי:}
\twocol{רבי יעקב זבן סנדלא ר' ירמיה זבן פיתא אמר ליה חד לחבריה יתמא עבד רבך הכי אמר ליה אידך יתמא עבד רבך הכי ותרוייהו מבעה"ב זבון וכל חד וחד סבר חבראי מתגר זבן דאמר רבי אבא בריה דרבי חייא בר אבא לא שנו אלא בלוקח מן התגר דשקלי מיכסא מיניה אבל בלוקח מבעה"ב דלא שקלי מיניה מיכסא מותר
\par א"ר אבא בריה דר' חייא בר אבא אילמלא היה ר' יוחנן הא זימנא באתרא דקא שקלי מיכסא אפי' מבעה"ב הוה אסר אלא אינהו היכי זבון מבעה"ב שאינו קבוע זבון:}
\twocol{{\large\emph{מתני׳}} אלו דברים אסורים למכור לעובד כוכבים אצטרובלין ובנות שוח ופטוטרות ולבונה ותרנגול הלבן רבי יהודה אומר מותר למכור לו תרנגול לבן בין התרנגולין ובזמן שהוא בפני עצמו קוטע את אצבעו ומוכרו לו לפי שאין מקריבים חסר לעבודת כוכבים
\par ושאר כל הדברים סתמן מותר ופירושן אסור ר"מ אומר אף דקל טב וחצב (ונקלב) אסור למכור לעובדי כוכבים: {\large\emph{גמ׳}} }
\newsection{דף יד}
\twocol{מאי איצטרובלין תורניתא ורמינהו הוסיפו עליהן אלכסין ואיצטרובלין מוכססין ובנות שוח ואי סלקא דעתך איצטרובלין תורניתא תורניתא מי איתא בשביעית
\par והתנן זה הכלל כל שיש לו עיקר יש לו שביעית וכל שאין לו עיקר אין לו שביעית אלא אמר רב ספרא פירי דארזא וכן כי אתא רבין א"ר אלעזר פירי דארזא:}
\twocol{בנות שוח: אמר רבה בר בר חנה אמר רבי יוחנן תאיני חיוראתא: ופטוטרות: אמר רבה בר בר חנה אמר רבי יוחנן בפטוטרותיהן שנו:
\par לבונה: אמר רבי יצחק אמר ר"ש בן לקיש לבונה זכה תנא ומכולן מוכרין להן חבילה וכמה חבילה פירש ר' יהודה בן בתירא אין חבילה פחותה משלשה מנין}
\twocol{וליחוש דלמא אזיל ומזבין לאחריני ומקטרי אמר אביי אלפני מפקדינן אלפני דלפני לא מפקדינן:
\par ותרנגול לבן: א"ר יונה א"ר זירא אמר רב זביד ואיכא דמתני אמר ר' יונה אמר ר' זירא תרנגול למי מותר למכור לו תרנגול לבן תרנגול לבן למי אסור למכור לו תרנגול לבן}
\twocol{תנן רבי יהודה אומר מוכר הוא לו תרנגול לבן בין התרנגולין היכי דמי אילימא דקאמר תרנגול לבן למי תרנגול לבן למי אפילו בין התרנגולין נמי לא
\par אלא לאו דקא אמר תרנגול למי תרנגול למי ואפילו הכי לרבי יהודה בין התרנגולין אין בפני עצמו לא ולת"ק אפילו בין התרנגולין נמי לא}
\twocol{אמר רב נחמן בר יצחק הכא במאי עסקינן כגון דאמר זה וזה
\par תניא נמי הכי אמר ר' יהודה אימתי בזמן שאמר תרנגול זה לבן אבל אם אמר זה וזה מותר ואפילו אמר תרנגול זה עובד כוכבים שעשה משתה לבנו או שהיה לו חולה בתוך ביתו מותר}
\twocol{והתניא עובד כוכבים שעשה משתה לבנו אינו אסור אלא אותו היום ואותו האיש בלבד אותו היום ואותו האיש מיהא אסור אמר רב יצחק בר רב משרשיא בטווזיג
\par תנן ושאר כל הדברים סתמן מותר ופירושן אסור מאי סתמן ומאי פירושן אילימא סתמא דקאמר חיטי חוורתא פירושן דקאמר לעבודת כוכבים}
\twocol{לא סתמן צריכא למימר דמזבנינן ולא פירושן צריכא למימר דלא מזבנינן אלא סתמן דקאמר חיטי פירושן דקאמר חוורתא
\par מכלל דתרנגול אפי' סתמן נמי לא אמרי לעולם סתמן דקאמר חיטי חוורתא פירושן דקאמר לעבודת כוכבים}
\twocol{ופירושן אצטריכא ליה סד"א האי גברא לאו לעבודת כוכבים קא בעי אלא מיבק הוא דאביק בעבודת כוכבים וסבר כי היכי דההוא גברא אביק ביה כ"ע נמי אביקו אימא הכי כי היכי דליתבו לי קמ"ל
\par בעי רב אשי תרנגול לבן קטוע למי מהו למכור לו תרנגול לבן שלם מי אמרינן מדקאמר קטוע לאו לעבודת כוכבים קבעי או דלמא איערומי קא מערים}
\twocol{את"ל האי איערומי הוא דקא מערים תרנגול לבן למי תרנגול לבן למי ויהבו ליה שחור ושקל ויהבו ליה אדום ושקל מהו למכור לו לבן מי אמרינן כיון דיהבו שחור ושקל אדום ושקל לאו לעבודת כוכבים קא בעי או דלמא איערומי קא מערים תיקו:
\par ר"מ אומר אף דקל וכו': א"ל רב חסדא לאבימי גמירי דעבודת כוכבים דאברהם אבינו ד' מאה פירקי הויין ואנן חמשה תנן ולא ידעינן מאי קאמרינן}
\twocol{ומאי קשיא דקתני ר"מ אומר אף דקל טב חצב ונקלס אסור למכור לעובדי כוכבים דקל טב הוא דלא מזבנינן הא דקל ביש מזבנינן והתנן אין מוכרין להם במחובר לקרקע א"ל מאי דקל טב פירות דקל טב וכן אמר רב הונא פירות דקל טב
\par חצב קשבא נקלס כי אתא רב דימי א"ר חמא בר יוסף קורייטי א"ל אביי לרב דימי תנן נקלס ולא ידעינן מהו ואת אמרת קורייטי ולא ידעינן מאי אהנית לן א"ל אהנאי לכו דכי אזלת התם אמרת להו נקלס ולא ידעי אמרת להו קורייטי וידעי וקא מחוו לך:}
\twocol{{\large\emph{מתני׳}} מקום שנהגו למכור בהמה דקה לעובדי כוכבים מוכרין מקום שנהגו שלא למכור אין מוכרין ובכל מקום אין מוכרין להם בהמה גסה עגלים וסייחים שלמין ושבורין ר' יהודה מתיר בשבורה ובן בתירא מתיר בסוס:
\par {\large\emph{גמ׳}} למימרא דאיסורא ליכא מנהגא הוא דאיכא היכא דנהיג איסור נהוג היכא דנהיג היתר נהוג}
\twocol{ורמינהי אין מעמידין בהמה בפונדקאות של עובדי כוכבים מפני שחשודין על הרביעה אמר רב מקום שהתירו למכור התירו לייחד מקום שאסרו לייחד אסרו למכור}
\newsection{דף טו}
\twocol{ור"א אומר אף במקום שאסרו לייחד מותר למכור מאי טעמא עובד כוכבים חס על בהמתו שלא תעקר ואף רב הדר ביה דאמר רב תחליפא א"ר שילא בר אבימי משמיה דרב עובד כוכבים חס על בהמתו שלא תעקר:
\par ובכל מקום אין מוכרין בהמה גסה כו': מ"ט נהי דלרביעה לא חיישינן מעביד ביה מלאכה חיישינן}
\twocol{וניעביד כיון דזבנה קנייה גזירה משום שאלה ומשום שכירות
\par שאלה קנייה ואגרא קנייה}
\twocol{אלא אמר רמי בריה דר' ייבא גזירה משום נסיוני דזמנין דזבנה לה ניהליה סמוך לשקיע' החמה דמעלי שבתא וא"ל תא נסייה ניהליה ושמעה ליה לקליה ואזלא מחמתיה וניחא ליה דתיזל והוה ליה מחמר אחר בהמתו בשבת והמחמר אחר בהמתו בשבת חייב חטאת
\par מתקיף לה רב שישא בריה דרב אידי ושכירות מי קניא והתנן אף במקום שאמרו להשכיר לא לבית דירה אמרו מפני שמכניס לתוכו עבודת כוכבים ואי ס"ד שכירות קניא האי כי קא מעייל לביתי' קא מעייל}
\twocol{שאני עבודת כוכבים דחמירא דכתיב (דברים ז, כו) ולא תביא תועבה אל ביתך
\par מתקיף לה רב יצחק ברי' דרב משרשיא ושכירות מי קניא והא תנן ישראל ששכר פרה מכהן יאכילנה כרשיני תרומה וכהן ששכר פרה מישראל אע"פ שמזונותיה עליו לא יאכילנה כרשיני תרומה}
\twocol{ואי ס"ד שכירות קניא אמאי לא יאכילנה פרה דידיה היא אלא ש"מ שכירות לא קניא והשתא דאמרת שכירו' לא קניא גזירה משום שכירות וגזירה משום שאלה וגזירה משום נסיוני
\par רב אדא שרא לזבוני חמרא אידא דספסירא אי משום נסיוני הא לא ידעה לקליה דאזלא מחמתיה ואי משום שאלה ושכירות כיון דלא דידיה היא לא מושיל ולא מוגר ועוד משום דלא ניגלי ביה מומא}
\twocol{רב הונא זבין ההיא פרה לעובד כוכבים אמר ליה רב חסדא מ"ט עבד מר הכי אמר ליה אימור לשחיטה זבנה
\par ומנא תימרא דאמרינן כי האי גוונא דתנן בש"א לא ימכור אדם פרה החורשת בשביעית וב"ה מתירין מפני שיכול לשוחטה}
\twocol{אמר רבה מי דמי התם אין אדם מצווה על שביתת בהמתו בשביעית הכא אדם מצווה על שביתת בהמתו בשבת
\par א"ל אביי וכל היכא דאדם מצווה אסור והרי שדה דאדם מצווה על שביתת שדהו בשביעית ותנן בש"א לא ימכור אדם שדה ניר בשביעית וב"ה מתירין מפני שיכול להובירה}
\twocol{מתקיף לה רב אשי וכל היכא דאין אדם מצווה שרי והרי כלים דאין אדם מצווה על שביתת כלים בשביעית ותנן אלו הן כלים שאין אדם רשאי למוכרן בשביעית המחרישה וכל כליה העול והמזרה והדקר
\par אלא אמר רב אשי כל היכא דאיכא למיתלא תלינן ואע"ג דמצווה וכל היכא דליכא למיתלי לא תלינן אע"ג דאינו מצווה}
\twocol{רבה זבין ההוא חמרא לישראל החשיד למכור לעובד כוכבים א"ל אביי מ"ט עבד מר הכי א"ל אנא לישראל זביני א"ל והא אזיל ומזבין ליה לעובד כוכבים לעובד כוכבים קא מזבין לישראל לא קא מזבין
\par איתיביה מקום שנהגו למכור בהמה דקה לכותים מוכרין שלא למכור אין מוכרין מ"ט אילימא משום דחשידי ארביעה ומי חשידי והתניא אין מעמידין בהמה בפונדקאות של עובדי כוכבים זכרים אצל זכרים ונקבות אצל נקבות ואין צ"ל נקבות אצל זכרים וזכרים אצל נקבות}
\twocol{ואין מוסרין בהמה לרועה שלהן ואין מייחדין עמהם ואין מוסרין להם תינוק ללמדו ספר וללמדו אומנות אבל מעמידין בהמה בפונדקאות של כותים זכרים אצל נקבות ונקבות אצל זכרים ואין צ"ל זכרים אצל זכרים ונקבות אצל נקבות
\par ומוסרין בהמה לרועה שלהן ומייחדין עמהם ומוסרין להם תינוק ללמדו ספר וללמדו אומנות אלמא לא חשידי}
\twocol{ועוד תניא אין מוכרין להם לא זיין ולא כלי זיין ואין משחיזין להן את הזיין ואין מוכרין להן לא סדן ולא קולרין ולא כבלים ולא שלשלאות של ברזל אחד עובד כוכבים ואחד כותי
\par מ"ט אי נימא דחשידי אשפיכות דמים ומי חשידי האמרת ומייחדין עמהן אלא משום דאתי לזבונה לעובד כוכבים}
\twocol{וכי תימא כותי לא עביד תשובה ישראל עביד תשובה והאמר רב נחמן אמר רבה בר אבוה כדרך שאמרו אסור למכור לעובד כוכבים כך אסור למכור לישראל החשוד למכור לעובד כוכבים רהיט בתריה תלתא פרסי וא"ד פרסא בחלא ולא אדרכיה
\par א"ר דימי בר אבא כדרך שאסור למכור לעובד כוכבים אסור למכור ללסטים ישראל ה"ד אי דחשיד דקטיל פשיטא היינו עובד כוכבים}
\twocol{ואי דלא קטיל אמאי לא לעולם דלא קטיל והב"ע במשמוטא דזימנין דעביד לאצולי נפשיה
\par תנו רבנן אין מוכרין להן תריסין וי"א מוכרין להן תריסין מ"ט אילימא משום דמגנו עלייהו אי הכי אפילו חיטי ושערי נמי לא אמר רב}
\newsection{דף טז}
\twocol{אי אפשר ה"נ
\par איכא דאמרי תריסין היינו טעמא דלא דכי שלים זינייהו קטלי בגוייהו ויש אומרים מוכרים להם תריסין דכי שלים זינייהו מערק ערקי אמר רב נחמן אמר רבה בר אבוה הלכה כיש אומרים}
\twocol{אמר רב אדא בר אהבה אין מוכרין להן עששיות של ברזל מ"ט משום דחלשי מינייהו כלי זיין אי הכי אפילו מרי וחציני נמי אמר רב זביד בפרזלא הינדואה והאידנא דקא מזבנינן א"ר אשי לפרסאי דמגנו עילוון:
\par עגלים וסייחים: תניא רבי יהודה מתיר בשבורה מפני שאינה יכולה להתרפאות ולחיות אמרו לו והלא מרביעין עליה ויולדת וכיון דמרביעין עליה ויולדת אתו לשהויה אמר להן לכשתלד אלמא לא מקבלת זכר:}
\twocol{בן בתירא מתיר בסוס: תניא בן בתירא מתיר בסוס מפני שהוא עושה בו מלאכה שאין חייבין עליה חטאת ורבי אוסר מפני ב' דברים אחד משום תורת כלי זיין ואחד משום תורת בהמה גסה
\par בשלמא תורת כלי זיין איכא דקטיל בסחופיה אלא תורת בהמה גסה מאי היא אמר רבי יוחנן לכשיזקין מטחינו ברחיים בשבת א"ר יוחנן הלכה כבן בתירא}
\twocol{איבעיא להו שור של פטם מהו תיבעי לרבי יהודה תיבעי לרבנן
\par תיבעי לרבי יהודה עד כאן לא קא שרי רבי יהודה אלא בשבורה דלא אתי לכלל מלאכה אבל האי דכי משהי ליה אתי לכלל מלאכה אסור}
\twocol{או דלמא אפילו לרבנן לא קא אסרי התם אלא דסתמיה לאו לשחיטה קאי אבל האי דסתמיה לשחיטה קאי אפילו רבנן שרו
\par ת"ש דאמר רב יהודה אמר שמואל של בית רבי היו מקריבין שור של פטם ביום אידם חסר ד' ריבבן שאין מקריבין אותו היום אלא למחר חסר ד' ריבבן שאין מקריבין אותו חי אלא שחוט חסר ד' ריבבן שאין מקריבין אותו כל עיקר}
\twocol{מ"ט לאו משום דלמא אתי לשהויי וליטעמיך שאין מקריבין אותו היום אלא למחר מאי טעמא אלא רבי מיעקר מילתא בעי וסבר יעקר ואתי פורתא פורתא
\par וכי משהי ליה בריא ועביד מלאכה אמר רב אשי אמר לי זבידא בר תורא משהינן ליה ועביד על חד תרין:}
\twocol{{\large\emph{מתני׳}} אין מוכרין להם דובין ואריות וכל דבר שיש בו נזק לרבים אין בונין עמהם בסילקי גרדום איצטדייא ובימה אבל בונין עמהם בימוסיאות ובית מרחצאות הגיע לכיפה שמעמידין בה עבודת כוכבים אסור לבנות:
\par {\large\emph{גמ׳}} אמר רב חנין בר רב חסדא ואמרי לה אמר רב חנן בר רבא אמר רב חיה גסה הרי היא כבהמה דקה לפירכוס אבל לא למכירה}
\twocol{ואני אומר אף למכירה מקום שנהגו למכור מוכרין שלא למכור אין מוכרין
\par תנן אין מוכרין להן דובין ואריות ולא כל דבר שיש בו נזק לרבים טעמא דאית ביה נזק לרבים הא לית ביה נזק לרבים שרי אמר רבה בר עולא בארי שבור}
\twocol{ואליבא דרבי יהודה רב אשי אמר סתם ארי שבור הוא אצל מלאכה
\par מיתיבי כשם שאין מוכרין להן בהמה גסה כך אין מוכרין להן חיה גסה ואפילו במקום שמוכרין להן בהמה דקה חיה גסה אין מוכרין להן תיובתא דרב חנן בר רבא תיובתא}
\twocol{רבינא רמי מתניתין אברייתא ומשני תנן אין מוכרין להן דובין ואריות ולא כל דבר שיש בו נזק לרבים טעמא דאית ביה נזק הא לית ביה נזק מוכרין
\par ורמינהי כשם שאין מוכרין בהמה גסה כך אין מוכרין חיה גסה ואפילו במקום שמוכרין בהמה דקה חיה גסה אין מוכרין ומשני בארי שבור ואליבא דר' יהודה רב אשי אמר סתם ארי שבור הוא אצל מלאכה}
\twocol{מתקיף לה רב נחמן מאן לימא לן דארי חיה גסה היא דלמא חיה דקה היא
\par רב אשי דייק מתניתין ומותיב תיובתא תנן אין מוכרין להן דובים ואריות ולא כל דבר שיש בו נזק לרבים טעמא דאית ביה נזק הא לית ביה נזק מוכרין}
\twocol{וטעמא ארי דסתם ארי שבור הוא אצל מלאכה אבל מידי אחרינא דעביד מלאכה לא תיובתא דרב חנן בר רבא תיובתא
\par וחיה גסה מיהת מאי מלאכה עבדא אמר אביי אמר לי מר יהודה דבי מר יוחני טחני ריחים בערודי}
\twocol{א"ר זירא כי הוינן בי רב יהודה אמר לן גמירו מינאי הא מילתא דמגברא רבה שמיע לי ולא ידענא אי מרב אי משמואל חיה גסה הרי היא כבהמה דקה לפירכוס
\par כי אתאי לקורקוניא אשכחתיה לרב חייא בר אשי ויתיב וקאמר משמיה דשמואל חיה גסה הרי היא כבהמה דקה לפירכוס אמינא ש"מ משמיה דשמואל איתמר כי אתאי לסורא אשכחתיה לרבה בר ירמיה דיתיב וקא"ל משמיה דרב חיה גסה הרי היא כבהמה דקה לפירכוס אמינא ש"מ איתמר משמיה דרב ואיתמר משמיה דשמואל}
\twocol{כי סליקת להתם אשכחתיה לרב אסי דיתיב וקאמר אמר רב חמא בר גוריא משמיה דרב חיה גסה הרי היא כבהמה דקה לפירכוס אמרי ליה ולא סבר לה מר דמאן מרא דשמעתתא רבה בר ירמיה א"ל פתיא אוכמא מינאי ומינך תסתיים שמעתא
\par איתמר נמי א"ר זירא אמר רב אסי אמר רבה בר ירמיה אמר רב חמא בר גוריא אמר רב חיה גסה הרי היא כבהמה דקה לפירכוס:}
\twocol{אין בונין כו': אמר רבה בר בר חנה א"ר יוחנן ג' בסילקאות הן של מלכי עובדי כוכבים ושל מרחצאות ושל אוצרות אמר רבא ב' להיתר ואחד לאיסור וסימן (תהלים קמט, ח) לאסור מלכיהם בזיקים
\par ואיכא דאמרי אמר רבא כולם להיתר והתנן אין בונין עמהן בסילקי גרדום איצטדייא ובימה אימא של גרדום ושל איצטדייא ושל בימה}
\twocol{ת"ר כשנתפס ר"א למינות העלהו לגרדום לידון אמר לו אותו הגמון זקן שכמותך יעסוק בדברים בטלים הללו
\par אמר לו נאמן עלי הדיין כסבור אותו הגמון עליו הוא אומר והוא לא אמר אלא כנגד אביו שבשמים אמר לו הואיל והאמנתי עליך דימוס פטור אתה}
\twocol{כשבא לביתו נכנסו תלמידיו אצלו לנחמו ולא קיבל עליו תנחומין אמר לו ר"ע רבי תרשיני לומר דבר אחד ממה שלימדתני אמר לו אמור אמר לו רבי שמא מינות בא לידך}
\newsection{דף יז}
\twocol{והנאך ועליו נתפסת אמר לו עקיבא הזכרתני פעם אחת הייתי מהלך בשוק העליון של ציפורי ומצאתי אחד ומתלמידי ישו הנוצרי ויעקב איש כפר סכניא שמו אמר לי כתוב בתורתכם (דברים כג, יט) לא תביא אתנן זונה [וגו'] מהו לעשות הימנו בהכ"ס לכ"ג ולא אמרתי לו כלום
\par אמר לי כך לימדני ישו הנוצרי (מיכה א, ז) כי מאתנן זונה קבצה ועד אתנן זונה ישובו ממקום הטנופת באו למקום הטנופת ילכו}
\twocol{והנאני הדבר על ידי זה נתפסתי למינות ועברתי על מה שכתוב בתורה (משלי ה, ח) הרחק מעליה דרכך זו מינות ואל תקרב אל פתח ביתה זו הרשות ואיכא דאמרי הרחק מעליה דרכך זו מינות והרשות ואל תקרב אל פתח ביתה זו זונה וכמה אמר רב חסדא ארבע אמות
\par ורבנן [האי] מאתנן זונה מאי דרשי ביה כדרב חסדא דאמר רב חסדא כל זונה שנשכרת לבסוף היא שוכרת שנאמר (יחזקאל טז, לד) ובתתך אתנן ואתנן לא נתן לך [ותהי להפך]}
\twocol{ופליגא דרבי פדת דא"ר פדת לא אסרה תורה אלא קריבה של גלוי עריות בלבד שנא' (ויקרא יח, ו) איש איש אל כל שאר בשרו לא תקרבו לגלות ערוה
\par עולא כי הוה אתי מבי רב הוה מנשק להו לאחתיה אבי ידייהו ואמרי לה אבי חדייהו ופליגא דידיה אדידיה דאמר עולא קריבה בעלמא אסור משום לך לך אמרין נזירא סחור סחור לכרמא לא תקרב}
\twocol{(משלי ל, טו) לעלוקה שתי בנות הב הב מאי הב הב אמר מר עוקבא [קול] שתי בנות שצועקות מגיהנם ואומרות בעוה"ז הבא הבא ומאן נינהו מינות והרשות איכא דאמרי אמר רב חסדא אמר מר עוקבא קול גיהנם צועקת ואומרת הביאו לי שתי בנות שצועקות ואומרות בעולם הזה הבא הבא
\par (משלי ב, יט) כל באיה לא ישובון ולא ישיגו אורחות חיים וכי מאחר שלא שבו היכן ישיגו ה"ק ואם ישובו לא ישיגו אורחות חיים}
\twocol{למימרא דכל הפורש ממינות מיית והא ההיא דאתאי לקמיה דרב חסדא ואמרה ליה קלה שבקלות עשתה בנה הקטן מבנה הגדול ואמר לה רב חסדא טרחו לה בזוודתא ולא מתה
\par מדקאמרה קלה שבקלות עשתה מכלל דמינות [נמי] הויא בה ההוא דלא הדרא בה שפיר ומש"ה לא מתה}
\twocol{איכא דאמרי ממינות אין מעבירה לא והא ההיא דאתאי קמיה דרב חסדא וא"ל [ר"ח זוידו לה זוודתא] ומתה מדקאמרה קלה שבקלות מכלל דמינות נמי הויא בה
\par ומעבירה לא והתניא אמרו עליו על ר"א בן דורדיא שלא הניח זונה אחת בעולם שלא בא עליה פעם אחת שמע שיש זונה אחת בכרכי הים והיתה נוטלת כיס דינרין בשכרה נטל כיס דינרין והלך ועבר עליה שבעה נהרות בשעת הרגל דבר הפיחה אמרה כשם שהפיחה זו אינה חוזרת למקומה כך אלעזר בן דורדיא אין מקבלין אותו בתשובה}
\twocol{הלך וישב בין שני הרים וגבעות אמר הרים וגבעות בקשו עלי רחמים אמרו לו עד שאנו מבקשים עליך נבקש על עצמנו שנאמר (ישעיהו נד, י) כי ההרים ימושו והגבעות תמוטינה אמר שמים וארץ בקשו עלי רחמים אמרו עד שאנו מבקשים עליך נבקש על עצמנו שנאמר (ישעיהו נא, ו) כי שמים כעשן נמלחו והארץ כבגד תבלה
\par אמר חמה ולבנה בקשו עלי רחמים אמרו לו עד שאנו מבקשים עליך נבקש על עצמנו שנאמר (ישעיהו כד, כג) וחפרה הלבנה ובושה החמה אמר כוכבים ומזלות בקשו עלי רחמים אמרו לו עד שאנו מבקשים עליך נבקש על עצמנו שנאמר (ישעיהו לד, ד) ונמקו כל צבא השמים}
\twocol{אמר אין הדבר תלוי אלא בי הניח ראשו בין ברכיו וגעה בבכיה עד שיצתה נשמתו יצתה בת קול ואמרה ר"א בן דורדיא מזומן לחיי העולם הבא [והא הכא בעבירה הוה ומית] התם נמי כיון דאביק בה טובא כמינות דמיא
\par בכה רבי ואמר יש קונה עולמו בכמה שנים ויש קונה עולמו בשעה אחת ואמר רבי לא דיין לבעלי תשובה שמקבלין אותן אלא שקורין אותן רבי}
\twocol{ר' חנינא ור' יונתן הוו קאזלי באורחא מטו להנהו תרי שבילי חד פצי אפיתחא דעבודת כוכבים וחד פצי אפיתחא דבי זונות אמר ליה חד לחבריה ניזיל אפיתחא דעבודת כוכבים
\par דנכיס יצריה א"ל אידך ניזיל אפיתחא דבי זונות ונכפייה ליצרין ונקבל אגרא כי מטו התם חזינהו [לזונות] איתכנעו מקמייהו}
\twocol{א"ל מנא לך הא א"ל (משלי ב, יא) מזמה תשמור עלך תבונה תנצרכה
\par א"ל רבנן לרבא מאי מזימה אילימא תורה דכתיב בה זימה ומתרגמינן עצת חטאין וכתיב (ישעיהו כח, כט) הפליא עצה הגדיל תושיה אי הכי זימה מבעי ליה ה"ק מדבר זימה תשמור עליך תורה תנצרכה}
\twocol{ת"ר כשנתפסו רבי אלעזר בן פרטא ורבי חנינא בן תרדיון א"ל ר' אלעזר בן פרטא לרבי חנינא בן תרדיון אשריך שנתפסת על דבר אחד אוי לי שנתפסתי על חמשה דברים
\par א"ל רבי חנינא אשריך שנתפסת על חמשה דברים ואתה ניצול אוי לי שנתפסתי על דבר אחד ואיני ניצול שאת עסקת בתורה ובגמילות חסדים ואני לא עסקתי אלא בתורה [בלבד]}
\twocol{וכדרב הונא דאמר רב הונא כל העוסק בתורה בלבד דומה כמי שאין לו אלוה שנאמר (דברי הימים ב טו, ג) וימים רבים לישראל ללא אלהי אמת [וגו'] מאי ללא אלהי אמת שכל העוסק בתורה בלבד דומה כמי שאין לו אלוה
\par ובגמילות חסדים לא עסק והתניא רבי אליעזר בן יעקב אומר לא יתן אדם מעותיו לארנקי של צדקה אלא א"כ ממונה עליו תלמיד חכם כר' חנינא בן תרדיון הימנוה הוא דהוה מהימן מיעבד לא עבד}
\twocol{והתניא אמר לו מעות של פורים נתחלפו לי במעות של צדקה וחלקתים לעניים מיעבד עבד כדבעי ליה לא עבד
\par אתיוהו לרבי אלעזר בן פרטא אמרו מ"ט תנית ומ"ט גנבת אמר להו אי סייפא לא ספרא ואי ספרא לא סייפא ומדהא ליתא הא נמי ליתא ומ"ט קרו לך רבי רבן של תרסיים אני}
\twocol{אייתו ליה תרי קיבורי אמרו ליה הי דשתיא והי דערבא איתרחיש ליה ניסא אתיא זיבוריתא אותיבא על דשתיא ואתאי זיבורא ויתיב על דערבא אמר להו האי דשתיא והאי דערבא
\par א"ל ומ"ט לא אתית לבי אבידן אמר להו זקן הייתי ומתיירא אני שמא תרמסוני ברגליכם [אמרו] ועד האידנא כמה סבי איתרמוס אתרחיש ניסא ההוא יומא אירמס חד סבא}
\twocol{ומ"ט קא שבקת עבדך לחירות אמר להו לא היו דברים מעולם קם חד [מינייהו] לאסהודי ביה אתא אליהו אידמי ליה כחד מחשובי דמלכותא א"ל מדאתרחיש ליה ניסא בכולהו בהא נמי אתרחיש ליה ניסא וההוא גברא בישותיה הוא דקא אחוי
\par ולא אשגח ביה קם למימר להו הוה כתיבא איגרתא דהוה כתיב מחשיבי דמלכות לשדורי לבי קיסר ושדרוה על ידיה דההוא גברא אתא אליהו פתקיה ארבע מאה פרסי אזל ולא אתא}
\twocol{אתיוהו לרבי חנינא בן תרדיון אמרו ליה אמאי קא עסקת באורייתא אמר להו כאשר צוני ה' אלהי מיד גזרו עליו לשריפה ועל אשתו להריגה ועל בתו לישב בקובה של זונות עליו לשריפה שהיה}
\newsection{דף יח}
\twocol{הוגה את השם באותיותיו והיכי עביד הכי והתנן אלו שאין להם חלק לעולם הבא האומר אין תורה מן השמים ואין תחיית המתים מן התורה אבא שאול אומר אף ההוגה את השם באותיותיו
\par להתלמד עבד כדתניא (דברים יח, ט) לא תלמד לעשות אבל אתה למד להבין ולהורות}
\twocol{אלא מאי טעמא אענש משום הוגה את השם בפרהסיא דהוי ועל אשתו להריגה דלא מיחה ביה מכאן אמרו כל מי שיש בידו למחות ואינו מוחה נענש עליו
\par ועל בתו לישב בקובה של זונות דאמר ר' יוחנן פעם אחת היתה בתו מהלכת לפני גדולי רומי אמרו כמה נאות פסיעותיה של ריבה זו מיד דקדקה בפסיעותיה והיינו דאמר ר' שמעון בן לקיש מאי דכתיב (תהלים מט, ו) עון עקבי יסבני עונות שאדם דש בעקביו בעולם הזה מסובין לו ליום הדין}
\twocol{בשעה שיצאו שלשתן צדקו עליהם את הדין הוא אמר (דברים לב, ד) הצור תמים פעלו [וגו'] ואשתו אמרה (דברים לב, ד) אל אמונה ואין עול בתו אמרה (ירמיהו לב, יט) גדול העצה ורב העליליה אשר עיניך פקוחות על כל דרכי וגו' אמר רבי [כמה] גדולים צדיקים הללו שנזדמנו להן שלש מקראות של צדוק הדין בשעת צדוק הדין
\par תנו רבנן כשחלה רבי יוסי בן קיסמא הלך רבי חנינא בן תרדיון לבקרו אמר לו חנינא אחי (אחי) אי אתה יודע שאומה זו מן השמים המליכוה שהחריבה את ביתו ושרפה את היכלו והרגה את חסידיו ואבדה את טוביו ועדיין היא קיימת ואני שמעתי עליך שאתה יושב ועוסק בתורה [ומקהיל קהלות ברבים] וספר מונח לך בחיקך}
\twocol{אמר לו מן השמים ירחמו אמר לו אני אומר לך דברים של טעם ואתה אומר לי מן השמים ירחמו תמה אני אם לא ישרפו אותך ואת ספר תורה באש אמר לו רבי מה אני לחיי העולם הבא
\par אמר לו כלום מעשה בא לידך אמר לו מעות של פורים נתחלפו לי במעות של צדקה וחלקתים לעניים אמר לו אם כן מחלקך יהי חלקי ומגורלך יהי גורלי}
\twocol{אמרו לא היו ימים מועטים עד שנפטר רבי יוסי בן קיסמא והלכו כל גדולי רומי לקברו והספידוהו הספד גדול ובחזרתן מצאוהו לרבי חנינא בן תרדיון שהיה יושב ועוסק בתורה ומקהיל קהלות ברבים וס"ת מונח לו בחיקו
\par הביאוהו וכרכוהו בס"ת והקיפוהו בחבילי זמורות והציתו בהן את האור והביאו ספוגין של צמר ושראום במים והניחום על לבו כדי שלא תצא נשמתו מהרה אמרה לו בתו אבא אראך בכך אמר לה אילמלי אני נשרפתי לבדי היה הדבר קשה לי עכשיו שאני נשרף וס"ת עמי מי שמבקש עלבונה של ס"ת הוא יבקש עלבוני}
\twocol{אמרו לו תלמידיו רבי מה אתה רואה אמר להן גליון נשרפין ואותיות פורחות אף אתה פתח פיך ותכנס בך האש אמר להן מוטב שיטלנה מי שנתנה ואל יחבל הוא בעצמו
\par אמר לו קלצטונירי רבי אם אני מרבה בשלהבת ונוטל ספוגין של צמר מעל לבך אתה מביאני לחיי העולם הבא אמר לו הן השבע לי נשבע לו מיד הרבה בשלהבת ונטל ספוגין של צמר מעל לבו יצאה נשמתו במהרה אף הוא קפץ ונפל לתוך האור}
\twocol{יצאה בת קול ואמרה רבי חנינא בן תרדיון וקלצטונירי מזומנין הן לחיי העולם הבא בכה רבי ואמר יש קונה עולמו בשעה אחת ויש קונה עולמו בכמה שנים
\par ברוריא דביתהו דר' מאיר ברתיה דר' חנינא בן תרדיון הואי אמרה לו זילא בי מלתא דיתבא אחתאי בקובה של זונות שקל תרקבא דדינרי ואזל אמר אי לא איתעביד בה איסורא מיתעביד ניסא אי עבדה איסורא לא איתעביד לה ניסא}
\twocol{אזל נקט נפשיה כחד פרשא אמר לה השמיעני לי אמרה ליה דשתנא אנא אמר לה מתרחנא מרתח אמרה לו נפישין טובא (ואיכא טובא הכא) דשפירן מינאי אמר ש"מ לא עבדה איסורא כל דאתי אמרה ליה הכי
\par אזל לגבי שומר דידה א"ל הבה ניהלה אמר ליה מיסתפינא ממלכותא אמר ליה שקול תרקבא דדינרא פלגא פלח ופלגא להוי לך א"ל וכי שלמי מאי איעביד א"ל אימא אלהא דמאיר ענני ומתצלת א"ל}
\twocol{ומי יימר דהכי איכא [א"ל השתא חזית] הוו הנהו כלבי דהוו קא אכלי אינשי שקל קלא שדא בהו הוו קאתו למיכליה אמר אלהא דמאיר ענני שבקוה ויהבה ליה
\par לסוף אשתמע מילתא בי מלכא אתיוה אסקוה לזקיפה אמר אלהא דמאיר ענני אחתוה אמרו ליה מאי האי אמר להו הכי הוה מעשה}
\twocol{אתו חקקו לדמותיה דר' מאיר אפיתחא דרומי אמרי כל דחזי לפרצופא הדין לייתיה יומא חדא חזיוהי רהט אבתריה רהט מקמייהו על לבי זונות איכא דאמרי בשולי עובדי כוכבים חזא טמש בהא ומתק בהא איכא דאמרי אתא אליהו אדמי להו כזונה כרכתיה אמרי חס ושלום אי ר' מאיר הוה לא הוה עביד הכי
\par קם ערק אתא לבבל איכא דאמרי מהאי מעשה ואיכא דאמרי ממעשה דברוריא:}
\twocol{תנו רבנן ההולך לאיצטדינין ולכרקום וראה שם את הנחשים ואת החברין בוקיון ומוקיון ומוליון ולוליון בלורין סלגורין הרי זה מושב לצים ועליהם הכתוב אומר (תהלים א, א) אשרי האיש אשר לא הלך וגו' כי אם בתורת ה' חפצו הא למדת. שדברים הללו מביאין את האדם לידי ביטול תורה
\par ורמינהי [הולכין] לאיצטדינין מותר מפני שצווח ומציל ולכרקום מותר מפני ישוב מדינה ובלבד שלא יתחשב עמהם ואם נתחשב עמהם אסור קשיא איצטדינין אאיצטדינין קשיא כרקום אכרקום}
\twocol{בשלמא כרקום אכרקום ל"ק כאן במתחשב עמהן כאן בשאין מתחשב עמהן אלא איצטדינין אאיצטדינין קשיא
\par תנאי היא דתניא אין הולכין לאיצטדינין מפני מושב לצים ור' נתן מתיר מפני שני דברים אחד מפני שצווח ומציל ואחד מפני שמעיד עדות אשה להשיאה}
\twocol{תנו רבנן אין הולכין לטרטיאות ולקרקסיאות מפני שמזבלין שם זיבול לעבודת כוכבים דברי ר' מאיר וחכמים אומרים מקום שמזבלין אסור מפני חשד עבודת כוכבים ומקום שאין מזבלין שם אסור מפני מושב לצים
\par מאי בינייהו אמר ר' חנינא מסורא נשא ונתן איכא בינייהו}
\twocol{דרש ר' שמעון בן פזי מאי דכתיב אשרי האיש אשר לא הלך בעצת רשעים ובדרך חטאים לא עמד ובמושב לצים לא ישב וכי מאחר שלא הלך היכן עמד ומאחר שלא עמד היכן ישב ומאחר שלא ישב היכן לץ
\par אלא לומר לך שאם הלך סופו לעמוד ואם עמד סופו לישב ואם ישב סופו ללוץ ואם לץ עליו הכתוב אומר (משלי ט, יב) אם חכמת חכמת לך ואם לצת לבדך תשא}
\twocol{א"ר אליעזר כל המתלוצץ יסורין באין עליו שנאמר (ישעיהו כח, כב) ועתה אל תתלוצצו פן יחזקו מוסריכם אמר להו רבא לרבנן במטותא בעינא מינייכו דלא תתלוצצו דלא ליתו עלייכו יסורין
\par אמר רב קטינא כל המתלוצץ מזונותיו מתמעטין שנאמר (הושע ז, ה) משך ידו את לוצצים אמר רבי שמעון בן לקיש כל המתלוצץ נופל בגיהנם שנאמר (משלי כא, כד) זד יהיר לץ שמו עושה בעברת זדון ואין עברה אלא גיהנם שנאמר (צפניה א, טו) יום עברה היום ההוא}
\twocol{אמר ר' אושעיא כל המתייהר נופל בגיהנם שנאמר זד יהיר לץ שמו עושה בעברת זדון ואין עברה אלא גיהנם שנאמר יום עברה היום ההוא אמר רבי חנילאי בר חנילאי כל המתלוצץ גורם כלייה לעולם שנאמר ועתה אל תתלוצצו פן יחזקו מוסריכם כי כלה ונחרצה שמעתי
\par אמר רבי אליעזר קשה היא שתחילת' יסורין וסופו כלייה דרש ר' שמעון בן פזי אשרי האיש אשר לא הלך לטרטיאות ולקרקסיאות של עובדי כוכבים ובדרך חטאים לא עמד זה שלא עמד בקנגיון ובמושב לצים לא ישב שלא ישב בתחבולות}
\twocol{שמא יאמר אדם הואיל ולא הלכתי לטרטיאות ולקרקסיאות ולא עמדתי בקנגיון אלך ואתגרה בשינה ת"ל ובתורתו יהגה יומם ולילה
\par אמר רב שמואל בר נחמני א"ר יונתן אשרי האיש אשר לא הלך בעצת רשעים זה}
\newsection{דף יט}
\twocol{אברהם אבינו שלא הלך בעצת אנשי דור הפלגה שרשעים היו שנאמר (בראשית יא, ד) הבה נבנה לנו עיר ובדרך חטאים לא עמד שלא עמד בעמידת סדום שחטאים היו שנאמר (בראשית יג, יג) ואנשי סדום רעים וחטאים לה' מאד
\par ובמושב לצים לא ישב שלא ישב במושב אנשי פלשתים מפני שלצנים היו שנאמר (שופטים טז, כה) ויהי כטוב לבם ויאמרו קראו לשמשון וישחק לנו}
\twocol{(תהלים קיב, א) אשרי איש ירא את ה' אשרי איש ולא אשרי אשה א"ר עמרם אמר רב אשרי מי שעושה תשובה כשהוא איש ר' יהושע בן לוי אמר אשרי מי שמתגבר על יצרו כאיש
\par במצותיו חפץ מאד אר"א במצותיו ולא בשכר מצותיו והיינו דתנן הוא היה אומר אל תהיו כעבדים המשמשין את הרב על מנת לקבל פרס אלא היו כעבדים המשמשין את הרב שלא על מנת לקבל פרס}
\twocol{כי אם בתורת ה' חפצו א"ר אין אדם לומד תורה אלא ממקום שלבו חפץ שנאמר (תהלים א, ב) כי אם בתורת ה' חפצו
\par לוי ור"ש ברבי יתבי קמיה דרבי וקא פסקי סידרא סליק ספרא לוי אמר לייתו [לן] משלי ר"ש ברבי אמר לייתו [לן] תילים כפייה ללוי ואייתו תילים כי מטו הכא כי אם בתורת ה' חפצו פריש רבי ואמר אין אדם לומד תורה אלא ממקום שלבו חפץ אמר לוי רבי נתת לנו רשות לעמוד}
\twocol{אמר ר' אבדימי בר חמא כל העוסק בתורה הקב"ה עושה לו חפציו שנאמר כי אם בתורת ה' חפצו אמר רבא לעולם ילמוד אדם תורה במקום שלבו חפץ שנאמר כי אם בתורת ה' חפצו
\par ואמר רבא בתחילה נקראת על שמו של הקב"ה ולבסוף נקראת על שמו שנאמר בתורת ה' חפצו ובתורתו יהגה יומם ולילה}
\twocol{ואמר רבא לעולם ילמד אדם תורה ואח"כ יהגה שנאמר בתורת ה' והדר ובתורתו יהגה
\par ואמר רבא לעולם ליגריס איניש ואע"ג דמשכח ואע"ג דלא ידע מאי קאמר שנאמר (תהלים קיט, כ) גרסה נפשי לתאבה גרסה כתיב ולא כתיב טחנה}
\twocol{רבא רמי כתיב (משלי ט, ג) על גפי וכתיב (משלי ט, יד) על כסא בתחלה על גפי ולבסוף על כסא
\par כתיב (משלי ח, ב) בראש מרומים וכתיב עלי דרך בתחלה בראש מרומים ולבסוף עלי דרך}
\twocol{עולא רמי כתיב (משלי ה, טו) שתה מים מבורך וכתיב ונוזלים מתוך בארך בתחלה שתה מבורך ולבסוף ונוזלים מתוך בארך
\par אמר רבא אמר רב סחורה אמר רב הונא מאי דכתיב (משלי יג, יא) הון מהבל ימעט וקובץ על יד ירבה אם עושה אדם תורתו חבילות חבילות מתמעט ואם קובץ על יד ירבה}
\twocol{אמר רבא ידעי רבנן להא מילתא ועברי עלה אמר רב נחמן בר יצחק אנא עבידתה וקיים בידי
\par אמר רב שיזבי משמיה דר"א בן עזריה מאי דכתיב (משלי יב, כז) לא יחרוך רמיה צידו לא יחיה ולא יאריך ימים צייד הרמאי}
\twocol{ורב ששת אמר צייד הרמאי יחרוך כי אתא רב דימי אמר משל לאדם שצד צפרין אם משבר כנפיה של ראשונה כולם מתקיימות בידו ואם לאו אין מתקיימות בידו
\par (תהלים א, ג) והיה כעץ שתול על פלגי מים אמרי דבי ר' ינאי כעץ שתול ולא כעץ נטוע כל הלומד תורה מרב אחד אינו רואה סימן ברכה לעולם}
\twocol{ אמר להו רב חסדא לרבנן בעינא דאימא לכו מלתא ומסתפינא דשבקיתו לי ואזליתו כל הלומד תורה מרב אחד אינו רואה סימן ברכה לעולם שבקוהו ואזול קמיה דרבא אמר להו הני מילי סברא אבל גמרא מרב אחד עדיף כי היכי
\par דלא ליפלוג לישני על פלגי מים א"ר תנחום בר חנילאי לעולם ישלש אדם שנותיו שליש במקרא שליש במשנה שליש בתלמוד}
\twocol{מי ידע איניש כמה חיי כי קאמרינן ביומי
\par (תהלים א, ג) אשר פריו יתן בעתו אמר רבא אם פריו יתן בעתו ועלהו לא יבול ואם לאו על הלומד ועל המלמד עליהם הכתוב אומר לא כן הרשעים כי אם וגו'}
\twocol{אמר רבי אבא אמר רב הונא אמר רב מאי דכתיב (משלי ז, כו) כי רבים חללים הפילה זה תלמיד שלא הגיע להוראה ומורה ועצומים כל הרוגיה זה תלמיד שהגיע להוראה ואינו מורה
\par ועד כמה עד מ' שנין והא רבא אורי התם בשוין}
\twocol{ועלהו לא יבול אמר רב אחא בר אדא אמר רב ואמרי לה אמר רב אחא בר אבא אמר רב המנונא אמר רב שאפילו שיחת חולין של ת"ח צריכה תלמוד שנאמר (תהלים א, ג) ועלהו לא יבול
\par וכל אשר יעשה יצליח א"ר יהושע בן לוי דבר זה כתוב בתורה ושנוי בנביאים ומשולש בכתובים כל העוסק בתורה נכסיו מצליחין לו כתוב בתורה דכתיב (דברים כט, ח) ושמרתם את דברי הברית הזאת ועשיתם אותם למען תשכילו את כל אשר תעשון}
\twocol{שנוי בנביאים דכתיב (יהושע א, ח) לא ימוש ספר התורה [הזה] מפיך והגית בו יומם ולילה למען תשמור לעשות ככל הכתוב בו כי אז תצליח את דרכיך ואז תשכיל משולש בכתובים דכתיב (תהלים א, ב) כי אם בתורת ה' חפצו ובתורתו יהגה יומם ולילה והיה כעץ שתול על פלגי מים אשר פריו יתן בעתו ועלהו לא יבול וכל אשר יעשה יצליח
\par מכריז רבי אלכסנדרי מאן בעי חיי מאן בעי חיי כנוף ואתו כולי עלמא לגביה אמרי ליה הב לן חיי אמר להו (תהלים לד, יג) מי האיש החפץ חיים וגו' נצור לשונך מרע וגו'}
\twocol{סור מרע ועשה טוב וגו' שמא יאמר נצרתי לשוני מרע ושפתי מדבר מרמה אלך ואתגרה בשינה ת"ל סור מרע ועשה טוב אין טוב אלא תורה שנאמר (משלי ד, ב) כי לקח טוב נתתי לכם תורתי אל תעזובו:
\par הגיע לכיפה מקום שמעמידין בה עבודת כוכבים: א"ר אלעזר אמר רבי יוחנן אם בנה שכרו מותר פשיטא משמשי עבודת כוכבים הן ומשמשי עבודת כוכבים בין לרבי ישמעאל בין לרבי עקיבא אינן אסורין עד שיעבדו}
\twocol{אמר רבי ירמיה לא נצרכה אלא לעבודת כוכבים עצמה הניחא למ"ד עבודת כוכבים של ישראל אסורה מיד ושל עובד כוכבים עד שתעבד שפיר אלא למ"ד של עובד כוכבים אסורה מיד מאי איכא למימר
\par אלא אמר רבה בר עולא לא נצרכה אלא במכוש אחרון עבודת כוכבים מאן קא גרים לה גמר מלאכה ואימת הויא גמר מלאכה במכוש אחרון מכוש אחרון לית ביה שוה פרוטה}
\twocol{אלמא קסבר ישנה לשכירות מתחלה ועד סוף:
\par {\large\emph{מתני׳}} ואין עושין תכשיטין לעבודת כוכבים קטלאות ונזמים וטבעות רבי אליעזר אומר בשכר מותר אין מוכרין להם במחובר לקרקע אבל מוכר הוא משיקצץ ר' יהודה אומר מוכר הוא על מנת לקוץ:}
\twocol{{\large\emph{גמ׳}} מנהני מילי אמר רבי יוסי בר חנינא}
\newsection{דף כ}
\twocol{דאמר קרא (דברים ז, ב) לא תחנם לא תתן להם חנייה בקרקע האי לא תחנם מיבעי ליה דהכי קאמר רחמנא לא תתן להם חן
\par א"כ לימא קרא לא תחונם מאי לא תחנם שמע מינה תרתי}
\twocol{ואכתי מיבעי ליה דהכי אמר רחמנא לא תתן להם מתנת של חנם אם כן לימא קרא לא תחינם מאי לא תחנם שמע מינה כולהו
\par תניא נמי הכי לא תחנם לא תתן להם חנייה בקרקע דבר אחר לא תחנם לא תתן להם חן דבר אחר לא תחנם לא תתן להם מתנת חנם}
\twocol{ומתנת חנם גופה תנאי היא דתניא (דברים יד, כא) לא תאכלו כל נבילה לגר אשר בשעריך תתננה ואכלה או מכור לנכרי אין לי אלא לגר בנתינה ולעובד כוכבים במכירה לגר במכירה מנין תלמוד לומר תתננה או מכור
\par לעובד כוכבים בנתינה מנין תלמוד לומר תתננה ואכלה או מכור לנכרי נמצא אתה אומר אחד גר ואחד עובד כוכבים בין בנתינה בין במכירה דברי ר' מאיר רבי יהודה אומר דברים ככתבן לגר בנתינה ולעובד כוכבים במכירה}
\twocol{שפיר קאמר ר"מ ור' יהודה אמר לך אי סלקא דעתך כדקאמר ר"מ לכתוב רחמנא תתננה ואכלה ומכור או למה לי שמע מינה לדברים ככתבן הוא דאתא
\par ור"מ ההוא לאקדומי נתינה דגר למכירה דעובד כוכבים ור' יהודה כיון דגר אתה מצווה להחיותו וכנעני אי אתה מצווה להחיותו להקדים לא צריך קרא:}
\twocol{ד"א לא תחנם לא תתן להם חן: מסייע ליה לרב דאמר רב אסור לאדם שיאמר כמה נאה עובדת כוכבים זו
\par מיתיבי מעשה ברשב"ג שהיה על גבי מעלה בהר הבית וראה עובדת כוכבים אחת נאה ביותר אמר (תהלים קד, כד) מה רבו מעשיך ה' ואף ר"ע ראה אשת טורנוסרופוס הרשע רק שחק ובכה רק שהיתה באה מטיפה סרוחה שחק דעתידה דמגיירא ונסיב לה בכה דהאי שופרא בלי עפרא}
\twocol{ורב אודויי הוא דקא מודה דאמר מר הרואה בריות טובות אומר ברוך שככה ברא בעולמו
\par ולאסתכולי מי שרי מיתיבי (דברים כג, י) ונשמרת מכל דבר רע שלא יסתכל אדם באשה נאה ואפילו פנויה באשת איש ואפי' מכוערת}
\twocol{ולא בבגדי צבע [של] אשה ולא בחמור ולא בחמורה ולא בחזיר ולא בחזירה ולא בעופות בזמן שנזקקין זה לזה ואפילו מלא עינים כמלאך המות
\par אמרו עליו על מלאך המות שכולו מלא עינים בשעת פטירתו של חולה עומד מעל מראשותיו וחרבו שלופה בידו וטיפה של מרה תלויה בו כיון שחולה רואה אותו מזדעזע ופותח פיו וזורקה לתוך פיו ממנה מת ממנה מסריח ממנה פניו מוריקות}
\twocol{קרן זוית הואי:
\par ולא בבגדי צבע [של] אשה: א"ר יהודה אמר שמואל אפילו שטוחין על גבי כותל א"ר פפא ובמכיר בעליהן אמר רבא דיקא נמי דקתני ולא בבגדי צבע אשה ולא קתני ולא בבגדי צבעונין שמע מינה}
\twocol{אמר רב חסדא הני מילי בעתיקי אבל בחדתי לית לן בה דאי לא תימא הכי אנן מנא לאשפורי היכי יהבינן הא קא מסתכל
\par ולטעמיך הא דאמר רב יהודה מין במינו מותר להכניס כמכחול בשפופרת הא קא מסתכל אלא בעבידתיה טריד ה"נ בעבידתיה טריד}
\twocol{אמר מר ממנה מת נימא פליגא דאבוה דשמואל דאמר אבוה דשמואל אמר לי מלאך המות אי לא דחיישנא ליקרא דברייתא הוה פרענא בית השחיטה כבהמה דלמא ההיא טיפה מחתכה להו לסימנין
\par ממנה מסריח מסייע ליה לרבי חנינא בר כהנא דא"ר חנינא בר כהנא אמרי בי רב הרוצה שלא יסריח מתו יהפכנו על פניו:}
\twocol{ת"ר (דברים כג, י) ונשמרת מכל דבר רע שלא יהרהר אדם ביום ויבוא לידי טומאה בלילה
\par מכאן א"ר פנחס בן יאיר תורה מביאה לידי זהירות זהירות מביאה לידי זריזות זריזות מביאה לידי נקיות נקיות מביאה לידי פרישות פרישות מביאה לידי טהרה טהרה מביאה לידי חסידות חסידות מביאה לידי ענוה ענוה מביאה לידי יראת חטא יראת חטא מביאה לידי קדושה קדושה מביאה לידי רוח הקודש רוח הקודש מביאה לידי תחיית המתים וחסידות גדולה מכולן שנאמר (תהלים פט, כ) אז דברת בחזון לחסידיך}
\twocol{ופליגא דרבי יהושע בן לוי דא"ר יהושע בן לוי ענוה גדולה מכולן שנאמר (ישעיהו סא, א) רוח ה' אלהים עלי יען משח ה' אותי לבשר ענוים חסידים לא נאמר אלא ענוים הא למדת שענוה גדולה מכולן:
\par אין מוכרין להן וכו': ת"ר מוכרין להן אילן על מנת לקוץ וקוצץ דברי ר' יהודה רבי מאיר אומר אין מוכרין להן אלא קצוצה שחת על מנת לגזוז וגוזז דברי רבי יהודה ר"מ אומר אין מוכרין להן אלא גזוזה קמה על מנת לקצור וקוצר דברי רבי יהודה ר' מאיר אומר אין מוכרין אלא קצורה}
\twocol{וצריכא דאי אשמעינן אילן בהא קאמר רבי מאיר כיון דלא פסיד משהי ליה אבל האי דכי משהי לה פסיד אימא מודי ליה לר' יהודה
\par ואי אשמעינן בהני תרתי משום דלא ידיע שבחייהו אבל שחת דידיע שבחייהו אימא מודי ליה לר' מאיר}
\twocol{ואי אשמעינן בהא בהא קאמר ר' מאיר אבל בהנך אימא מודי ליה לרבי יהודה צריכא
\par  איבעיא להו בהמה על מנת לשחוט מהו}
\twocol{התם טעמא מאי שרי ר' יהודה דלאו ברשותיה קיימי ולא מצי משהי להו אבל בהמה כיון דברשותיה דעובד כוכבים קיימא משהי לה או דלמא לא שנא
\par ת"ש דתניא בהמה ע"מ לשחוט ושוחט דברי רבי יהודה רבי מאיר אומר אין מוכרין לו אלא שחוטה:}
\twocol{{\large\emph{מתני׳}} אין משכירין להם בתים בארץ ישראל ואין צריך לומר שדות ובסוריא}
\newsection{דף כא}
\twocol{משכירין להם בתים אבל לא שדות ובחו"ל מוכרין להם בתים ומשכירין שדות דברי רבי מאיר רבי יוסי אומר בארץ ישראל משכירין להם בתים אבל לא שדות ובסוריא מוכרין בתים ומשכירין שדות ובחוץ לארץ מוכרין אלו ואלו
\par אף במקום שאמרו להשכיר לא לבית דירה אמרו מפני שהוא מכניס לתוכו עבודת כוכבים שנאמר (דברים ז, כו) לא תביא תועבה אל ביתך ובכל מקום לא ישכיר לו את המרחץ מפני שהוא נקרא על שמו:}
\twocol{{\large\emph{גמ׳}} מאי אין צריך לומר שדות אילימא משום דאית בה תרתי חדא חניית קרקע וחדא דקא מפקע לה ממעשר
\par אי הכי בתים נמי איכא תרתי חדא חניית קרקע וחדא דקא מפקע לה ממזוזה אמר רב משרשיא מזוזה חובת הדר הוא:}
\twocol{בסוריא משכירין בתים כו': מאי שנא מכירה דלא משום מכירה דארץ ישראל אי הכי משכירות נמי נגזור היא גופה גזרה ואנן ניקום וניגזור גזרה לגזרה
\par והא שכירות שדה דבסוריא דגזרה לגזרה היא וקא גזרינן התם לאו גזרה הוא קסבר כיבוש יחיד שמיה כיבוש}
\twocol{שדה דאית ביה תרתי גזרו ביה רבנן בתים דלית בהו תרתי לא גזרו בהו רבנן:
\par בחוץ לארץ וכו': שדה דאית ביה תרתי גזרו בהו רבנן בתים דלית בהו תרתי לא גזרו בהו רבנן:}
\twocol{רבי יוסי אומר בארץ ישראל משכירין להם בתים וכו': מ"ט שדות דאית בהו תרתי גזרו בהו רבנן בתים דלית בהו תרתי לא גזרו בהו רבנן:
\par ובסוריא מוכרין וכו': מ"ט קסבר כיבוש יחיד לא שמיה כיבוש ושדה דאית בה תרתי גזרו בה רבנן בתים דלית בהו תרתי לא גזרו בהו רבנן:}
\twocol{ובחו"ל מוכרין וכו': מאי טעמא כיון דמרחק לא גזרינן
\par אמר רב יהודה אמר שמואל הלכה כרבי יוסי אמר רב יוסף ובלבד שלא יעשנה שכונה וכמה שכונה תנא אין שכונה פחותה משלשה בני אדם}
\twocol{ולחוש דלמא אזיל האי ישראל ומזבין לחד עובד כוכבים ואזיל היאך ומזבין לה לתרי אמר אביי אלפני מפקדינן אלפני דלפני לא מפקדינן:
\par אף במקום שאמרו להשכיר: מכלל דאיכא דוכתא דלא מוגרי}
\twocol{וסתמא כרבי מאיר דאי ר' יוסי בכל דוכתא מוגרי:
\par ובכל מקום לא ישכור וכו': תניא רבן שמעון בן גמליאל אומר לא ישכור אדם מרחצו לעובד כוכבים מפני שנקרא על שמו ועובד כוכבים זה עושה בו מלאכה בשבתות ובימים טובים}
\twocol{אבל לכותי מאי שרי כותי אימר עביד ביה מלאכה בחולו של מועד בחולו של מועד אנן נמי עבדינן
\par אבל שדהו לעובד כוכבים מאי שרי מאי טעמא אריסא אריסותיה קעביד מרחץ נמי אמרי אריסא אריסותיה קעביד אריסא דמרחץ לא עבדי אנשי}
\twocol{תניא ר"ש בן אלעזר אומר לא ישכיר אדם שדהו לכותי מפני שנקראת על שמו וכותי זה עושה בו מלאכה בחוש"מ אבל עובד כוכבים מאי שרי דאמרי אריסא אריסותיה עביד א"ה כותי נמי אמרי אריסא אריסותיה עביד}
\newsection{דף כב}
\twocol{אריסותא לר"ש בן אלעזר לית ליה אלא עובד כוכבים מ"ט מותר דאמרינן ליה וציית כותי נמי אמרינן ליה וציית כותי לא ציית דאמר אנא גמירנא טפי מינך
\par א"ה מאי איריא מפני שנקראת על שמו תיפוק ליה משום (ויקרא יט, יד) לפני עור לא תתן מכשול חדא ועוד קאמר חדא משום לפני עור ועוד מפני שנקראת על שמו}
\twocol{הנהו מוריקאי דעובד כוכבים נקיט בשבתא וישראל בחד בשבתא אתו לקמיה דרבא שרא להו
\par איתיביה רבינא לרבא ישראל ועובד כוכבים שקיבלו שדה בשותפות לא יאמר ישראל לעובד כוכבים טול חלקך בשבת ואני בחול ואם התנו מתחלה מותר}
\twocol{ואם באו לחשבון אסור איכסיף לסוף איגלאי מלתא דהתנו מעיקרא הוו
\par רב גביהה מבי כתיל אמר הנהו שתילי דערלה הוה עובד כוכבים אכיל שני דערלה וישראל שני דהתירא אתו לקמיה דרבא שרא להו}
\twocol{והא אותביה רבינא לרבא לסיועי סייעיה והא אכסיף לא היו דברים מעולם
\par איבעיא להו סתמא מאי ת"ש אם התנו מתחילה מותר הא סתמא אסור}
\twocol{אימא סיפא אם באו לחשבון אסור הא סתמא מותר אלא מהא ליכא למשמע מינה:
\par \par \par {\large\emph{הדרן עלך לפני אידיהן}}\par \par }
\twocol{
\par }
\twocol{מתני׳ {\large\emph{אין}} מעמידין בהמה בפונדקאות של עובדי כוכבים מפני שחשודין על הרביעה ולא תתייחד אשה עמהן מפני שחשודין על העריות ולא יתייחד אדם עמהן מפני שחשודין על שפיכות דמים: {\large\emph{גמ׳}} }
\newchap{פרק \hebrewnumeral{2}\quad אין מעמידין}
\twocol{
\par ורמינהי לוקחין מהן בהמה לקרבן ואין חוששין לא משום רובע ולא משום נרבע ולא משום מוקצה ולא משום נעבד}
\twocol{בשלמא מוקצה ונעבד אם איתא דאקצייה ואם איתא דפלחיה לא הוה מזבין ליה אלא רובע ונרבע לחוש אמר רב תחליפא אמר רב שילא בר אבינא משמיה דרב עובד כוכבים חס על בהמתו שלא תעקר
\par התינח נקבות זכרים מאי איכא למימר אמר רב כהנא הואיל ומכחישין בבשר}
\twocol{אלא הא דתניא לוקחין בהמה מרועה שלהן ליחוש דלמא רבעה לה רועה שלהן מתיירא משום הפסד שכר
\par אלא הא דתניא אין מוסרין בהמה לרועה שלהן לימא רועה שלהן מתיירא משום הפסד שכרו}
\twocol{אינהו דידעי בהדדי מרתתי אנן דלא ידעינן בהו לא מרתתי אמר רבה היינו דאמרי אינשי מכתבא גללא בזע רגלא בחבריה ידע
\par אי הכי זכרים מנקבות לא ניזבון דחיישינן דלמא מרבעא ליה עילוה כיון דמיגרי בה מרתתא}
\twocol{אלא הא דתני רב יוסף ארמלתא לא תרבי כלבא ולא תשרי בר בי רב באושפיזא בשלמא בר בי רב צניע לה אלא כלבא כיון דמיגרה בה מרתתא
\par כיון דכי שדיא ליה אומצא ומסריך אבתרה מימר אמרי אינשי האי דמסריך אבתרה משום אומצא דקא מסריך}
\twocol{נקבות אצל נקבות מאי טעמא לא מייחדינן אמר מר עוקבא בר חמא מפני שהעובדי כוכבים מצויין אצל נשי חבריהן ופעמים שאינו מוצאה ומוצא את הבהמה ורובעה
\par ואיבעית אימא אפילו מוצאה נמי רובעה דאמר מר חביבה עליהן בהמתן של ישראל יותר מנשותיהן דא"ר יוחנן בשעה שבא נחש על חוה הטיל בה זוהמא אי הכי ישראל נמי ישראל שעמדו על הר סיני פסקה זוהמתן עובדי כוכבים שלא עמדו על הר סיני לא פסקה זוהמתן}
\twocol{איבעיא להו עופות מאי תא שמע דאמר רב יהודה אמר שמואל משום רבי חנינא אני ראיתי עובד כוכבים שלקח אווז מן השוק רבעה חנקה צלאה ואכלה וא"ר ירמיה מדיפתי אני ראיתי ערבי אחד שלקח ירך מן השוק וחקק בה כדי רביעה רבעה צלאה ואכלה}
\newsection{דף כג}
\twocol{רבינא אמר לא קשיא הא לכתחלה הא דיעבד
\par ומנא תימרא דשאני בין לכתחלה בין לדיעבד דתנן לא תתייחד אשה עמהם מפני שחשודין על העריות ורמינהו האשה שנחבשה בידי עובדי כוכבים ע"י ממון מותרת לבעלה ע"י נפשות אסורה לבעלה}
\twocol{אלא לאו ש"מ שאני לן בין לכתחלה לדיעבד ממאי דלמא לעולם אימא לך אפילו דיעבד נמי לא והכא היינו טעמא דמתיירא משום הפסד ממונו
\par תדע דקתני סיפא ע"י נפשות אסורה לבעלה ותו לא מידי}
\twocol{רבי פדת אמר לא קשיא הא רבי אליעזר הא רבנן דתנן גבי פרת חטאת ר' אליעזר אומר אינה נקחת מן העובדי כוכבים וחכמים מתירין מאי לאו בהא קמיפלגי דר"א סבר חיישינן לרביעה ורבנן סברי לא חיישינן לרביעה
\par ממאי דלמא דכ"ע לא חיישינן לרביעה והכא היינו טעמא דרבי אליעזר כדרב יהודה אמר רב דאמר רב יהודה אמר רב הניח עליהן עודה של שקין פסלה ובעגלה עד שתמשוך בה}
\twocol{מר סבר חיישינן ומר סבר לא חיישינן לא ס"ד משום ניחא פורתא לא מפסיד טובא
\par ה"נ לימא משום הנאה פורתא לא מפסיד טובא התם יצרו תוקפו}
\twocol{ודלמא דכולי עלמא לא חיישינן לרביעה והכא היינו טעמא דר"א כדתני שילא דתני שילא מ"ט דר' אליעזר (במדבר יט, ב) דבר אל בני ישראל ויקחו אליך בני ישראל יקחו ואין העובדי כוכבים יקחו
\par לא סלקא דעתך דקתני סיפא וכן היה רבי אליעזר פוסל בכל הקרבנות כולן ואי סלקא דעתך כדתני שילא בשלמא פרה כתיב בה קיחה אלא כולהו קרבנות קיחה כתיב בהו ודלמא עד כאן לא פליגי רבנן עליה דרבי אליעזר}
\twocol{אלא בפרה דדמיה יקרין אבל בשאר קרבנות מודו ליה
\par ואלא הא דתניא לוקחין מהן בהמה לקרבן מני לא ר' אליעזר ולא רבנן}
\twocol{ועוד תניא בהדיא מאי אותיבו ליה חברוהי לר' אליעזר (ישעיהו ס, ז) כל צאן קדר יקבצו לך יעלו (לרצון על) מזבחי
\par ע"כ לא פליגי אלא בחששא אבל היכא דודאי רבעה פסלה ש"מ דפרה קדשי מזבח היא דאי קדשי בדק הבית מי מיפסלא בה רביעה}
\twocol{שאני פרה דחטאת קרייה רחמנא
\par אלא מעתה תיפסל ביוצא דופן וכי תימא ה"נ אלמה תניא הקדישה ביוצא דופן פסולה ור"ש מכשיר}
\twocol{וכי תימא ר"ש לטעמיה דאמר יוצא דופן ולד מעליא הוא והא"ר יוחנן מודה היה ר"ש לענין קדשים שאינו קדוש
\par אלא שאני פרה הואיל ומום פוסל בה דבר ערוה וע"ז נמי פוסל בה דכתיב (ויקרא כב, כה) כי משחתם בהם מום בם ותנא דבי ר' ישמעאל כל מקום שנא' השחתה אינו אלא דבר ערוה וע"ז}
\twocol{דבר ערוה דכתיב (בראשית ו, יב) כי השחית כל בשר את דרכו על הארץ וע"ז דכתיב (דברים ד, טז) פן תשחיתון ועשיתם לכם פסל והא פרה נמי הואיל ומום פוסל בה דבר ערוה ועבודת כוכבים פסלי בה
\par גופא תני שילא מ"ט דרבי אליעזר דכתיב (במדבר יט, ב) דבר אל בני ישראל ויקחו בני ישראל יקחו ואין העובדי כוכבים יקחו אלא מעתה (שמות כה, ב) דבר אל בני ישראל ויקחו לי תרומה ה"נ דבני ישראל יקחו ואין העובדי כוכבים יקחו}
\twocol{וכי תימא ה"נ והאמר רב יהודה אמר שמואל שאלו את ר"א עד היכן כיבוד אב ואם אמר להם צאו וראו מה עשה עובד כוכבים אחד לאביו באשקלון ודמא בן נתינה שמו פעם אחת בקשו ממנו אבנים לאפוד}
\newsection{דף כד}
\twocol{בששים רבוא שכר רב כהנא מתני בשמונים רבוא והיו מפתחות מונחות תחת מראשותיו של אביו ולא צערו
\par אבני שהם הפסיק הענין והא ואבני מלואים כתיב דהדר ערביה}
\twocol{ועוד קתני סיפא לשנה אחרת נולדה לו פרה אדומה בעדרו נכנסו חכמי ישראל אצלו אמר להם יודע אני בכם שאם אני מבקש מכם כל ממון שבעולם אתם נותנין לי עכשיו איני מבקש מכם אלא אותו ממון שהפסדתי בשביל אבא
\par התם על ידי תגרי ישראל זבון}
\twocol{ורבי אליעזר לא חייש לרביעה
\par והתניא אמרו לו לר' אליעזר מעשה ולקחוה מן העובד כוכבים ודמא שמו ואמרי לה רמץ שמו אמר להן רבי אליעזר משם ראיה ישראל היו משמרין אותה משעה שנולדה רבי אליעזר תרתי אית ליה קיחה וחייש נמי לרביעה}
\twocol{אמר מר ישראל היו משמרין אותה משעה שנולדה וניחוש דלמא רבעי לאמא כי הוה מעברה דאמר רבא וולד הנוגחת אסורה היא וולדה נגחו וולד הנרבעת אסורה היא וולדה נרבעו אימא ישראל היו משמרין אותה משעה שנוצרה
\par וניחוש דלמא רבעוה לאמא מעיקרא דתנן כל הפסולין לגבי מזבח ולדותיהן מותרין ותני עלה ר' אליעזר אוסר}
\twocol{הניחא לרבא דאמר רבא אמר רב נחמן מחלוקת כשנרבעו כשהן מוקדשין אבל כשהן חולין דברי הכל מותרין
\par אלא לרב הונא בר חיננא אמר רב נחמן מחלוקת כשנרבעו כשהן חולין אבל כשהן מוקדשין דברי הכל אסורין מאי איכא למימר}
\twocol{אימא ישראל היו משמרין אותה לאמא משעה שנוצרה וניחוש דילמא רבעוה לאמא דאמא כולי האי לא חיישינן
\par אמר מר ישראל היו משמרין אותה משעה שנוצרה מנא ידעינן אמר רב כהנא כוס אדום מעבירין לפניה בשעה שעולה עליה זכר}
\twocol{א"ה אמאי דמיה יקרין הואיל ושתי שערות פוסלות בה ומאי שנא דידהו אמר רב כהנא במוחזקת
\par יתיב ר' אמי ורבי יצחק נפחא אקלעא דר' יצחק נפחא פתח חד מינייהו ואמר וכן היה ר"א פוסל בכל הקרבנות כולן}
\twocol{פתח אידך מינייהו ואמר מאי אותיבו ליה חברוהי לר"א (ישעיהו ס, ז) כל צאן קדר יקבצו לך וגו' אמר להן ר"א כולם גרים גרורים הם לעתיד לבא
\par אמר רב יוסף מאי קרא (צפניה ג, ט) כי אז אהפוך אל עמים שפה ברורה וגו' אמר ליה אביי ודלמא מעבודת כוכבים הוא דהדור בהו אמר ליה רב יוסף לעבדו שכם אחד כתיב}
\twocol{רב פפא מתני הכי ורב זביד מתני הכי ותרוייהו אמרי וכן היה רבי אליעזר פוסל בכל הקרבנות ותרוייהו אמרי ומאי אותיבו ליה חברוהי לרבי אליעזר כל צאן קדר יקבצו לך וגו' אמר רבי אליעזר כולם גרים גרורים הם לעתיד לבא
\par ומאי קראה כי אז אהפוך אל עמים שפה ברורה לקרוא כלם בשם ה' מתקיף לה רב יוסף ודלמא מעבודת כוכבים הוא דהדרי בהו א"ל אביי לעבדו שכם אחד כתיב}
\twocol{מיתיבי (שמות י, כה) ויאמר משה גם אתה תתן בידנו זבחים ועולות קודם מתן תורה שאני
\par ת"ש (שמות יח, יב) ויקח יתרו חתן משה עולה וזבחים לאלהים יתרו נמי קודם מתן תורה הוה הניחא למ"ד יתרו קודם מתן תורה הוה אלא למ"ד}
\twocol{יתרו לאחר מתן תורה הוה מאי איכא למימר אלא יתרו מישראל זבן
\par ת"ש (שמואל א טו, טו) ויאמר שאול מעמלקי הביאום אשר חמל העם על מיטב הצאן והבקר (המשנים והכרים ועל כל הצאן) למען זבוח לה' אלהיך מאי מיטב דמי מיטב}
\twocol{ומ"ש מיטב כי היכי דליקפץ עליהן זבינא
\par ת"ש (שמואל ב כד, כב) ויאמר ארונה אל דוד יקח ויעל אדוני המלך (את) הטוב בעיניו (ואת) [ראה] הבקר לעולה והמוריגים וכלי הבקר לעצים אמר רב נחמן ארונה גר תושב היה}
\twocol{מאי מוריגים אמר עולא מטה של טורביל מאי מטה של טורביל עיזא דקורקסא דדיישן אמר רב יוסף מאי קרא (ישעיהו מא, טו) הנה שמתיך למורג חרוץ חדש בעל פיפיות תדוש הרים ותדוק וגבעות כמוץ תשים
\par מיתיבי (שמואל א ו, יד) ואת הפרות העלו עולה לה' הוראת שעה היתה}
\twocol{ה"נ מסתברא דאי לא תימא הכי עולה נקבה מי איכא
\par ומאי קושיא דלמא בבמת יחיד וכדרב אדא בר אהבה דאמר רב אדא בר אהבה מנין לעולה נקבה שהיא כשרה בבמת יחיד שנאמר (שמואל א ז, ט) ויקח שמואל טלה חלב אחד ויעלהו עולה}
\twocol{ויעלהו זכר משמע אמר רב נחמן בר יצחק ויעלה כתיב
\par ר' יוחנן אמר גבול יש לה פחותה מבת ג' שנים נעקרת בת ג' שנים אינה נעקרת}
\twocol{איתיביה כל הני תיובתא שני להו פחותה מבת ג' שנים ת"ש ואת הפרות העלו עולה לה' בפחותה מבת שלש שנים
\par מתקיף לה רב הונא בריה דרב נתן א"כ היינו ואת בניהם כלו בבית פחותה מבת ג' שנים}
\twocol{(ופחותה מבת נ' שנים) מי קא ילדה והתניא פרה וחמור מבת ג' ודאי לכהן מכאן ואילך ספק אלא מחוורתא כדשנין מעיקרא:
\par (שמואל א ו, יב) וישרנה הפרות בדרך על דרך בית שמש וגו' מאי וישרנה א"ר יוחנן משום ר"מ שאמרו שירה ורב זוטרא בר טוביה אמר רב שישרו פניהם כנגד ארון ואמרו שירה}
\twocol{ומאי שירה אמרו א"ר יוחנן משום ר"מ (שמות טו, א) אז ישיר משה ובני ישראל ור' יוחנן דידיה אמר (ישעיהו יב, ד) ואמרתם ביום ההוא הודו לה' קראו בשמו וגו'
\par ור"ש בן לקיש אמר מזמורא יתמא (תהלים צח, א) מזמור שירו לה' שיר חדש כי נפלאות עשה הושיעה לו ימינו וזרוע קדשו ר' אלעזר אמר (תהלים צט, א) ה' מלך ירגזו עמים}
\twocol{ר' שמואל בר נחמני אמר (תהלים צג, א) ה' מלך גאות לבש ר' יצחק נפחא אמר רוני רוני השיטה התנופפי ברוב הדרך המחושקת בריקמי זהב המהוללה בדביר ארמון ומפוארה בעדי עדיים
\par רב אשי מתני לה להא דר' יצחק אהא (במדבר י, לה) ויהי בנסוע הארון ויאמר משה קומה ה' ישראל מאי אמרו אמר ר' יצחק רוני רוני השיטה וכו'}
\twocol{אמר רב כמאן קרו פרסאי לספרא דביר מהכא (שופטים א, יא) ושם דביר לפנים קרית ספר
\par רב אשי אמר כמאן קרו פרסאי לנידה דשתנא מהכא (בראשית לא, לה) כי דרך נשים לי}
\newsection{דף כה}
\twocol{(יהושע י, יג) וידום השמש וירח עמד עד יקום גוי אויביו הלא היא כתובה על ספר הישר מאי ספר הישר א"ר חייא בר אבא א"ר יוחנן זה ספר אברהם יצחק ויעקב שנקראו ישרים שנא' (במדבר כג, י) תמות נפשי מות ישרים
\par והיכא רמיזא (בראשית מח, יט) וזרעו יהיה מלא הגוים [אימתי יהיה מלא הגוים] בשעה שעמדה לו חמה ליהושע (יהושע י, יג) ויעמד השמש בחצי השמים ולא אץ לבוא כיום תמים}
\twocol{וכמה א"ר יהושע בן לוי עשרים וארבעה [שעי] אזיל שית וקם שית אזיל שית וקם שית כולה מלתא כיום תמים
\par ר' אלעזר אמר שלשים ושית אזיל שית וקם תריסר אזיל שית וקם תריסר עמידתו כיום תמים רבי שמואל בר נחמני אמר ארבעים ושמונה אזיל שית וקם תריסר אזיל שית וקם עשרים וארבעה [שנאמר] ולא אץ לבוא כיום תמים מכלל דמעיקרא לאו כיום תמים [הוה]}
\twocol{א"ד בתוספתא פליגי ר' יהושע בן לוי אמר עשרים וארבעה אזיל שית וקם תריסר אזיל שית וקם תריסר עמידתו כיום תמים ר"א אמר שלשים ושש אזיל שית וקם תריסר אזיל שית וקם עשרים וארבעה ולא אץ לבוא כיום תמים
\par ר' שמואל בר נחמני אמר ארבעים ושמונה אזיל שית וקם עשרים וארבעה אזיל שית וקם כ"ד מקיש עמידתו לביאתו מה ביאתו כיום תמים אף עמידתו כיום תמים}
\twocol{תנא כשם שעמדה לו חמה ליהושע כך עמדה לו חמה למשה ולנקדימון בן גוריון יהושע קראי נקדימון בן גוריון גמרא למשה מנלן אתיא אחל אחל כתיב הכא (דברים ב, כה) אחל תת פחדך וכתיב התם ביהושע אחל גדלך
\par ורבי יוחנן אמר אתיא תת תת כתיב הכא אחל תת פחדך וכתיב ביהושע (יהושע י, יב) ביום תת ה' את האמורי}
\twocol{ר' שמואל בר נחמני אמר מגופיה דקרא שמעת ליה (דברים ב, כה) אשר ישמעון שמעך ורגזו וחלו מפניך אימתי רגזו וחלו מפניך בשעה שעמדה לו חמה למשה
\par מיתיבי ולא היה כיום ההוא לפניו ואחריו איבעית אימא שעות הוא דלא הוו נפיש כולי האי ואיבעית אימא אבני ברד לא הוו דכתיב (יהושע י, יא) ויהי בנוסם מפני בני ישראל הם במורד בית חורון וה' השליך עליהם אבנים גדולות מן השמים עד עזקה וימותו}
\twocol{כתיב (שמואל ב א, יח) ויאמר ללמד בני יהודה קשת הנה כתובה על ספר הישר מאי ספר הישר א"ר חייא בר אבא א"ר יוחנן זה ספר אברהם יצחק ויעקב שנקראו ישרים דכתיב בהו (במדבר כג, י) תמות נפשי מות ישרים ותהי אחריתי כמוהו
\par והיכא רמיזא יהודה אתה יודוך אחיך ידך בעורף אויביך ואיזו היא מלחמה שצריכה יד כנגד עורף הוי אומר זו קשת}
\twocol{ר"א אומר זה ספר משנה תורה ואמאי קרו ליה ספר הישר דכתיב (דברים ו, יח) ועשית הישר והטוב בעיני ה' והיכא רמיזא ידיו רב לו ואיזו היא מלחמה שצריכה שתי ידים הוי אומר זו קשת
\par ר' שמואל בר נחמני אמר זה ספר שופטים ואמאי קרו ליה ספר הישר דכתיב (שופטים יז, ו) בימים ההם אין מלך בישראל איש הישר בעיניו יעשה והיכא רמיזא (שופטים ג, ב) למען דעת דורות בני ישראל ללמדם מלחמה ואיזו היא מלחמה שצריכה לימוד הוי אומר זו קשת}
\twocol{ומנלן דביהודה כתיב דכתיב (שופטים א, א) מי יעלה לנו בתחלה אל הכנעני להלחם בו ויאמר ה' יהודה יעלה
\par (שמואל א ט, כד) וירם הטבח את השוק והעליה וישם לפני שאול מאי והעליה ר' יוחנן אומר שוק ואליה מאי והעליה דמסמכא שוק לאליה}
\twocol{ורבי אלעזר אומר שוק וחזה מאי והעליה דמחית לה לחזה עילויה דשוק כי בעי אנופי ומנפי ליה ורבי שמואל בר נחמני אמר שוק ושופי מאי והעליה שופי עילויה דשוק קאי:
\par לא תתייחד אשה עמהם: במאי עסקינן אילימא בחד דכוותה גבי ישראל מי שרי והתנן לא יתייחד איש אחד עם שתי נשים}
\twocol{אלא בתלתא דכוותה גבי ישראל בפרוצים מי שרי והתנן אבל אשה אחת מתייחדת עם שני אנשים ואמר רב יהודה אמר שמואל לא שנו אלא בכשרים אבל בפרוצים אפילו עשרה נמי לא מעשה היה והוציאוה עשרה במטה
\par לא צריכא באשתו עמו עובד כוכבים אין אשתו משמרתו אבל ישראל אשתו משמרתו}
\twocol{ותיפוק ליה משום שפיכות דמים א"ר ירמיה באשה חשובה עסקינן דמירתתי מינה רב אידי אמר אשה כלי זיינה עליה
\par מאי בינייהו איכא בינייהו אשה חשובה בין אנשים ושאינה חשובה בין הנשים}
\twocol{תניא כוותיה דרב אידי בר אבין האשה אע"פ שהשלום עמה לא תתייחד עמהן מפני שחשודין על העריות:
\par לא יתייחד אדם עמהן: ת"ר ישראל שנזדמן לו עובד כוכבים בדרך טופלו לימינו ר' ישמעאל בנו של ר' יוחנן בן ברוקה אומר בסייף טופלו לימינו במקל טופלו לשמאלו}
\twocol{היו עולין במעלה או יורדין בירידה לא יהא ישראל למטה ועובד כוכבים למעלה אלא ישראל למעלה ועובד כוכבים למטה ואל ישוח לפניו שמא ירוץ את גולגלתו
\par שאלו להיכן הולך ירחיב לו את הדרך כדרך שעשה יעקב אבינו לעשו הרשע דכתיב (בראשית לג, יד) עד אשר אבא אל אדוני שעירה}
\twocol{וכתיב (בראשית לג, יז) ויעקב נסע סכותה
\par מעשה בתלמידי ר"ע שהיו הולכים לכזיב פגעו בהן ליסטים אמרו להן לאן אתם הולכים אמרו להן לעכו כיון שהגיעו לכזיב פירשו אמרו להן תלמידי מי אתם אמרו להן תלמידי ר"ע אמרו להן אשרי ר"ע ותלמידיו שלא פגע בהן אדם רע מעולם}
\twocol{רב מנשה הוה אזל}
\newsection{דף כו}
\twocol{לבי תורתא פגעו ביה גנבי אמרו ליה לאן קאזלת אמר להן לפומבדיתא כי מטא לבי תורתא פריש אמרו ליה תלמידא דיהודה רמאה את אמר להו ידעיתו ליה יהא רעוא דליהוו הנהו אינשי בשמתיה
\par אזלו עבדו גניבתא עשרין ותרתין שנין ולא אצלחו כיון דחזו אתו כולהו תבעו שמתייהו והוה בהו חד גירדנא דלא אתא לשרויה שמתיה אכלי' אריא היינו דאמרי אינשי גירדנא דלא טייזן שתא בציר משני}
\twocol{תא חזי מה בין גנבי בבל ולסטין דארץ ישראל:
\par {\large\emph{מתני׳}} בת ישראל לא תיילד את העובדת כוכבים מפני שמילדת בן לעבודת כוכבים אבל עובדת כוכבים מילדת בת ישראל בת ישראל לא תניק בנה של עובדת כוכבים אבל עובדת כוכבי' מניקה בנה של ישראל ברשותה:}
\twocol{{\large\emph{גמ׳}} ת"ר בת ישראל לא תיילד את העובדת כוכבים מפני שמילדת בן לעבודת כוכבים ועובדת כוכבים לא תיילד את בת ישראל מפני שחשודין על שפיכות דמים דברי רבי מאיר
\par וחכמים אומרים עובדת כוכבים מילדת את בת ישראל בזמן שאחרות עומדות על גבה אבל לא בינה לבינה ור"מ אומר אפי' אחרות עומדות על גבה נמי לא דזימנין דמנחא ליה ידא אפותא וקטלא ליה ולא מתחזי}
\twocol{כי ההיא איתתא דאמרה לחברתה מולדא יהודייתא בת מולדא יהודייתא אמרה לה נפישין בישתא דההיא איתתא דקא משפילנא מינייהו דמא כי אופיא דנהרא
\par ורבנן א"ל לא היא במילתא בעלמא הוא דאוקימתה:}
\twocol{בת ישראל לא תניק: ת"ר בת ישראל לא תניק בנה של עובדת כוכבים מפני שמגדלת בן לעבודת כוכבים ועובדת כוכבים לא תניק את בנה של בת ישראל מפני שחשודה על שפיכות דמים דברי ר"מ וחכ"א עובדת כוכבים מניקה את בנה של בת ישראל בזמן שאחרות עומדות על גבה אבל לא בינו לבינה ורבי מאיר אומר אפילו אחרות עומדות על גבה נמי לא דזימנין דשייפא ליה סמא לדד מאבראי וקטלא ליה
\par וצריכא דאי אשמעינן מילדת בההיא קאמרי רבנן דשרי דלא אפשר משום דאחרות רואות אותה אבל מניקה דאפשר דשייפא ליה סם לדד מאבראי וקטלא ליה אימא מודי ליה לרבי מאיר}
\twocol{ואי אשמעינן מניקה בההיא קאמר רבי מאיר דאסור משום דשייפא ליה סם לדד מאבראי וקטלא ליה אבל מילדת דלא אפשר היכא דאחרות עומדות על גבה אימא מודי להו לרבנן צריכא
\par ורמינהו יהודית מילדת עובדת כוכבים בשכר אבל לא בחנם אמר רב יוסף בשכר שרי משום איבה}
\twocol{סבר רב יוסף למימר אולודי עובדת כוכבים בשבתא בשכר שרי משום איבה א"ל אביי יכלה למימר לה דידן דמינטרי שבתא מחללינן עלייהו דידכו דלא מינטרי שבתא לא מחללינן
\par סבר רב יוסף למימר אנוקי בשכר שרי משום איבה אמר ליה אביי יכלה למימר אי פנויה היא בעינא לאינסובי אי אשת איש היא לא קא מזדהמנא באפי גברא}
\twocol{סבר רב יוסף למימר הא דתניא העובדי כוכבים ורועי בהמה דקה לא מעלין ולא מורידין אסוקי בשכר שרי משום איבה
\par אמר ליה אביי יכול לומר לו קאי ברי אאיגרא אי נמי נקיטא לי זימנא לבי דואר}
\twocol{תני רבי אבהו קמיה דר' יוחנן העובדי כוכבים ורועי בהמה דקה לא מעלין
\par ולא מורידין אבל המינין והמסורות והמומרים היו מורידין ולא מעלין}
\twocol{א"ל אני שונה (דברים כב, ג) לכל אבידת אחיך לרבות את המומר ואת אמרת היו מורידין סמי מכאן מומר
\par ולישני ליה כאן במומר אוכל נבילות לתיאבון כאן במומר אוכל נבילות להכעיס קסבר אוכל נבילות להכעיס מין הוא}
\twocol{איתמר מומר פליגי רב אחא ורבינא חד אמר לתיאבון מומר להכעיס מין הוי וחד אמר אפילו להכעיס נמי מומר אלא איזהו מין זה העובד אלילי כוכבים
\par מיתיבי אכל פרעוש אחד או יתוש אחד הרי זה מומר והא הכא דלהכעיס הוא וקתני מומר התם בעי למיטעם טעמא דאיסורא}
\twocol{אמר מר היו מורידין אבל לא מעלין השתא אחותי מחתינן אסוקי מיבעי אמר רב יוסף בר חמא אמר רב ששת לא נצרכא שאם היתה מעלה בבור מגררה דנקיט ליה עילא ואמר לא תיחות חיותא עלויה
\par רבה ורב יוסף דאמרי תרוייהו לא נצרכא שאם היתה אבן על פי הבאר מכסה אמר לעבורי חיותא עילויה רבינא אמר שאם היה סולם מסלקו אמר בעינא לאחותי ברי מאיגרא}
\twocol{ת"ר ישראל מל את העובד כוכבים לשום גר לאפוקי לשום מורנא דלא ועובד כוכבים לא ימול ישראל מפני שחשודין על שפיכות דמים דברי רבי מאיר
\par וחכמים אומרים עובד כוכבים מל את ישראל בזמן שאחרים עומדין על גבו אבל בינו לבינו לא ורבי מאיר אומר אפילו אחרים עומדים על גבו נמי לא דזימנין דמצלי ליה סכינא ומשוי ליה כרות שפכה}
\twocol{וסבר ר"מ עובד כוכבים לא ורמינהו עיר שאין בה רופא ישראל ויש בה רופא כותי ורופא עובד כוכבים ימול עובד כוכבים ואל ימול כותי דברי ר"מ רבי יהודה אומר ימול כותי ואל ימול עובד כוכבים
\par איפוך ר"מ אומר ימול כותי ולא עובד כוכבים ר' יהודה אומר עובד כוכבים ולא כותי}
\twocol{וסבר ר' יהודה עובד כוכבים שפיר דמי והתניא ר' יהודה אומר מנין למילה בעובד כוכבים שהיא פסולה שנא' (בראשית יז, ט) ואתה את בריתי תשמור
\par אלא לעולם לא תיפוך והכא במאי עסקינן}
\newsection{דף כז}
\twocol{ברופא מומחה דכי אתא רב דימי א"ר יוחנן אם היה מומחה לרבים מותר
\par וסבר רבי יהודה כותי שפיר דמי והתניא ישראל מל את הכותי וכותי לא ימול ישראל מפני שמל לשם הר גרזים דברי רבי יהודה}
\twocol{אמר לו רבי יוסי וכי היכן מצינו מילה מן התורה לשמה אלא מל והולך עד שתצא נשמתו
\par אלא לעולם איפוך כדאפכינן מעיקרא ודקא קשיא דרבי יהודה אדר' יהודה ההיא דרבי יהודה הנשיא היא}
\twocol{דתניא רבי יהודה הנשיא אומר מנין למילה בעובד כוכבים שהיא פסולה ת"ל ואתה את בריתי תשמור
\par אמר רב חסדא מאי טעמא דרבי יהודה דכתיב לה' המול ורבי יוסי המול ימול}
\twocol{ואידך הכתיב לה' המול ההוא בפסח כתיב ואידך נמי הכתיב המול ימול דברה תורה כלשון בני אדם
\par איתמר מנין למילה בעובד כוכבים שהיא פסולה דרו בר פפא משמיה דרב אמר ואתה את בריתי תשמור ורבי יוחנן המול ימול}
\twocol{מאי בינייהו ערבי מהול וגבנוני מהול איכא בינייהו מאן דאמר המול ימול איכא ומ"ד את בריתי תשמור ליכא
\par ולמאן דאמר המול ימול איכא והתנן קונם שאני נהנה מן הערלים מותר בערלי ישראל ואסור במולי עובדי כוכבים אלמא אף על גב דמהילי כמאן דלא מהילי דמו}
\twocol{אלא איכא בינייהו ישראל שמתו אחיו מחמת מילה ולא מלוהו למ"ד ואתה את בריתי תשמור איכא למאן דאמר המול ימול ליכא
\par ולמ"ד המול ימול ליכא והתנן קונם שאני נהנה ממולים אסור בערלי ישראל ומותר במולי עובדי כוכבים אלמא אע"ג דלא מהילי כמאן דמהילי דמו}
\twocol{אלא איכא בינייהו אשה למ"ד ואתה את בריתי תשמור ליכא דאשה לאו בת מילה היא ולמ"ד המול ימול איכא דאשה כמאן דמהילא דמיא
\par ומי איכא למאן דאמר אשה לא והכתיב (שמות ד, כה) ותקח צפורה צר קרי ביה ותקח והכתיב ותכרות קרי ביה ותכרת דאמרה לאיניש אחרינא ועבד ואיבעית אימא אתיא איהי ואתחלה ואתא משה ואגמרה:}
\twocol{{\large\emph{מתני׳}} מתרפאין מהן ריפוי ממון אבל לא ריפוי נפשות ואין מסתפרין מהן בכל מקום דברי רבי מאיר וחכמים אומרים ברה"ר מותר אבל לא בינו לבינו:
\par {\large\emph{גמ׳}} מאי ריפוי ממון ומאי ריפוי נפשות אילימא ריפוי ממון בשכר ריפוי נפשות בחנם ליתני מתרפאין מהן בשכר אבל לא בחנם}
\twocol{אלא ריפוי ממון דבר שאין בו סכנה ריפוי נפשות דבר שיש בו סכנה והאמר רב יהודה אפילו ריבדא דכוסילתא לא מתסינן מינייהו
\par אלא ריפוי ממון בהמתו ריפוי נפשות גופיה והיינו דאמר רב יהודה אפילו ריבדא דכוסילתא לא מתסינן מינייהו}
\twocol{אמר רב חסדא אמר מר עוקבא אבל אם אמר לו סם פלוני יפה לו סם פלוני רע לו מותר
\par סבר שיולי משאיל לו כי היכי דמשאיל לו משאיל לאיניש אחרינא ואתא ההוא גברא לאורועי נפשיה}
\twocol{אמר רבא א"ר יוחנן ואמרי לה אמר רב חסדא אמר ר' יוחנן ספק חי ספק מת אין מתרפאין מהן ודאי מת מתרפאין מהן
\par מת האיכא חיי שעה לחיי שעה לא חיישינן}
\twocol{ומנא תימרא דלחיי שעה לא חיישינן דכתיב (מלכים ב ז, ד) אם אמרנו נבוא העיר והרעב בעיר ומתנו שם והאיכא חיי שעה אלא לאו לחיי שעה לא חיישינן
\par מיתיבי לא ישא ויתן אדם עם המינין ואין מתרפאין מהן אפילו לחיי שעה}
\twocol{מעשה בבן דמא בן אחותו של ר' ישמעאל שהכישו נחש ובא יעקב איש כפר סכניא לרפאותו ולא הניחו ר' ישמעאל וא"ל ר' ישמעאל אחי הנח לו וארפא ממנו ואני אביא מקרא מן התורה שהוא מותר ולא הספיק לגמור את הדבר עד שיצתה נשמתו ומת
\par קרא עליו ר' ישמעאל אשריך בן דמא שגופך טהור ויצתה נשמתך בטהרה ולא עברת על דברי חביריך שהיו אומרים (קהלת י, ח) ופורץ גדר ישכנו נחש}
\twocol{שאני מינות דמשכא דאתי למימשך בתרייהו
\par אמר מר לא עברת על דברי חביריך שהיו אומרים ופורץ גדר ישכנו נחש איהו נמי חויא טרקיה חויא דרבנן דלית ליה אסותא כלל}
\twocol{ומאי ה"ל למימר (ויקרא יח, ה) וחי בהם ולא שימות בהם
\par ור' ישמעאל הני מילי בצינעא אבל בפרהסיא לא דתניא היה רבי ישמעאל אומר מנין שאם אומרים לו לאדם עבוד עבודת כוכבים ואל תהרג שיעבוד ואל יהרג ת"ל וחי בהם ולא שימות בהם יכול אפילו בפרהסיא ת"ל (ויקרא כב, לב) ולא תחללו את שם קדשי}
\twocol{אמר רבה בר בר חנה אמר רבי יוחנן כל מכה שמחללין עליה את השבת אין מתרפאין מהן ואיכא דאמרי אמר רבה בר בר חנה אמר ר"י כל}
\newsection{דף כח}
\twocol{מכה של חלל אין מתרפאין מהן מאי בינייהו איכא בינייהו גב היד וגב הרגל דאמר רב אדא בר מתנה אמר רב גב היד וגב הרגל הרי הן כמכה של חלל ומחללין עליהן את השבת
\par אמר רב זוטרא בר טוביה אמר רב כל מכה שצריכה אומד מחללין עליה את השבת אמר רב שמן בר אבא אמר ר' יוחנן והאי אישתא צמירתא כמכה של חלל דמי ומחללין עליה את השבת}
\twocol{מהיכן מכה של חלל פירש רבי אמי מן השפה ולפנים בעי רבי אליעזר ככי ושיני מאי כיון דאקושי נינהו כמכה דבראי דמו או דלמא כיון דגואי קיימי כמכה של חלל דמו
\par אמר אביי ת"ש החושש בשיניו לא יגמע בהן את החומץ חושש הוא דלא הא כאיב ליה טובא שפיר דמי דלמא תנא היכא דכאיב ליה טובא חושש נמי קרי ליה}
\twocol{ת"ש רבי יוחנן חש בצפדינא אזל לגבה דההיא מטרוניתא עבדה חמשא ומעלי שבתא א"ל למחר מאי אמרה ליה לא צריכת אי צריכנא מאי אמרה אשתבע לי דלא מגלית אישתבע לה לאלהא ישראל לא מגלינא גלייה ליה למחר נפק דרשה בפירקא
\par והא אישתבע לה לאלהא דישראל לא מגלינא אבל לעמיה ישראל מגלינא והאיכא חילול השם דגלי לה מעיקרא}
\twocol{אלמא כמכה של חלל דמיא אמר רב נחמן בר יצחק שאני צפדינא הואיל ומתחיל בפה וגומר בבני מעיים
\par מאי סימניה רמי מידי בי ככי ומייתי דמא מבי דרי ממאי הוי מקרירי קרירי דחיטי ומחמימי חמימי דשערי ומשיורי כסא דהרסנא מאי עבדא ליה א"ר אחא בריה דרבא מי שאור ושמן זית ומלח ומר בר רב אשי אמר משחא דאווזא בגדפא דאווזא}
\twocol{אמר אביי אנא עבדי כולהו ולא איתסאי עד דאמר לי ההוא טייעא אייתי קשייתא דזיתא דלא מלו תילתא וקלנהו אמרא חדתא ודביק ביה דדרי עבדי הכי ואיתסאי
\par ורבי יוחנן היכי עביד הכי והאמר רבה בר בר חנה אמר רבי יוחנן כל מכה שמחללין עליה את השבת אין מתרפאין מהן אדם חשוב שאני}
\twocol{והא רבי אבהו דאדם חשוב הוה ורמא ליה יעקב מינאה סמא אשקיה ואי לא רבי אמי ורבי אסי דלחכוהו לשקיה פסקיה לשקיה
\par דרבי יוחנן רופא מומחה הוה דרבי אבהו נמי רופא מומחה הוה שאני רבי אבהו דמוקמי ביה מיני בנפשייהו (שופטים טז, ל) תמות נפשי עם פלשתים}
\twocol{אמר שמואל האי פדעתא סכנתא היא ומחללין עליה את השבת מאי אסותא למיפסק דמא תחלי בחלא לאסוקי גרדא דיבלא וגירדא דאסנא או ניקרא מקילקלתא
\par אמר רב ספרא האי עינבתא פרוונקא דמלאכא דמותא היא מאי אסותא טיגנא בדובשא או כרפסא בטילייא אדהכי והכי ליתי עינבתא בת מינא וניגנדר עילוי חיורתי לחיורתי ואוכמתי לאוכמתי}
\twocol{אמר רבא האי סימטא פרוונקא דאשתא היא מאי אסותא למחייה שיתין איתקוטלי וליקרעיה שתי וערב והני מילי דלא חיור רישיה אבל חיור רישיה לית לן בה
\par רבי יעקב חש}
\twocol{בפיקעא אורי ליה רבי אמי ואמרי לה רבי אסי אורי ליה ליתי שב ביני אהלא תולנא וצייר ליה בחללא דבי צוארא וליכריך עילויה נירא ברקא וטמיש ליה בנטפא חיורא וליקליה ובדר ליה עילויה אדהכי והכי ליתי קשיתא דאסנא לינח פיקעא להדי פיקעא
\par וה"מ פיקעא עילאה פיקעא תתאה מאי לייתי תרבא דצפירתא דלא אפתח וליפשר ולישדי ביה}
\twocol{ואי לא לייתי תלת טרפא קרא דמייבשי בטולא וליקלי וליבדר עילויה ואי לא לייתי משקדי חלזוני ואי לא מייתי משח קירא ולינקוט בשחקי דכיתנא בקייטא ודעמר גופנא בסיתווא
\par רבי אבהו חש באודניה אורי ליה ר' יוחנן ואמרי ליה בי מדרשא מאי אורי ליה כי הא דאמר אביי אמרה לי אם לא איברי כולייתא אלא לאודנא ואמר רבא אמר לי מניומי אסיא כולהו שקיינו קשו לאודנא לבר ממיא דכולייתא לייתי כולייתא דברחא קרחא וליקרעיה שתי וערב ולינח אמללא דנורא והנהו מיא דנפקי מיניה לישדינהו באודניה לא קרירי ולא חמימי אלא פשורי}
\twocol{ואי לא לייתי תרבא דחיפושתא גמלניתא וליפשר ולישדי ביה ואי לא למלייה לאודניה מישחא וליעבד שב פתילתא דאספסתא וליתי שופתא דתומא וליתוב ברקא בחד רישא וליתלי בהו נורא ואידך רישא מותבא באודנא וליתוב אודניה להדא נורא ויזדהר מזיקא ונישקול חדא וננח חדא
\par לישנא אחרינא ואי לא לייתי שב פתילתא ביקרא ושייף ליה מישחא דאספסתא ונייתי חד רישא בנורא וחד רישא באודניה ונשקול חדא וננח חדא ויזדהר מזיקא}
\twocol{ואי לא לייתי אודרא דנדא דלא משקיף וננח בה ולתלייה לאודניה להדי נורא ומיזדהר מזיקא ואי לא לייתי גובתא דקניא עתיקא בר מאה שנין ולימלחיה מילחא גללניתא ולקלי ולידבק וסימנך רטיבא ליבשתא ויבשתא לרטיבא
\par אמר רבה בר זוטרא אמר רבי חנינא מעלין אזנים בשבת תני רב שמואל בר יהודה ביד אבל לא בסם איכא דאמרי בסם אבל לא ביד מ"ט מזריף זריף}
\twocol{אמר רב זוטרא בר טוביה אמר רב עין שמרדה מותר לכוחלה בשבת סבור מיניה הני מילי הוא דשחקי סמנין מאתמול אבל משחק בשבת ואתויי דרך רשות הרבים לא א"ל ההוא מרבנן ורבי יעקב שמיה לדידי מיפרשא מיני' דרב יהודה אפילו מישחק בשבת ואתויי דרך רשות הרבים מותר
\par רב יהודה שרא למיכחל עינא בשבת אמר להו רב שמואל בר יהודה מאן ציית ליהודה מחיל שבי לסוף חש בעיניה שלח ליה שרי או אסיר שלח ליה לכ"ע שרי לדידך אסיר}
\twocol{וכי מדידי הוא דמר שמואל היא ההיא אמתא דהואי בי מר שמואל דקדחא לה עינא בשבתא צווחא וליכא דאשגח בה פקעא עינא למחר נפק מר שמואל ודרש עין שמרדה מותר לכוחלה בשבת מאי טעמא דשורייני דעינא באובנתא דליבא תלו
\par כגון מאי אמר רב יהודה כגון רירא דיצא דמא דימעתא וקידחא ותחלת אוכלא לאפוקי סוף אוכלא ופצוחי עינא דלא}
\twocol{אמר רב יהודה זיבורא ודחרזיה סילוא וסמטא ודכאיב ליה עינא ואתי עילויה אישתא כולהו בי בני סכנתא חמה לחמה וסילקא לצינא וחילופא סכנתא חמימי לעקרבא וקרירי לזיבורא וחילופא סכנתא חמימי לסילוא וקרירי}
\newsection{דף כט}
\twocol{לחספניתא וחילופא סכנתא
\par חלא לסיבורי ומוניני לתעניתא וחילופא סכנתא תחלי וסיבורא סכנתא אישתא וסיבורא סכנתא כאיב עינא וסבורי סכנתא שני לדג דם שני לדם דג שלישי לו סכנתא}
\twocol{ת"ר המקיז דם לא יאכל חגב"ש לא חלב ולא גבינה ולא בצלים ולא שחלים אם אכל אמר אביי נייתי רביעתא דחלא ורביעתא דחמרא ונערבבינהו בהדי הדדי ונישתי וכי מפנה לא מפנה אלא למזרחה של עיר משום דקשה ריחא
\par א"ר יהושע בן לוי מעלין אונקלי בשבת מאי אונקלי אמר רבי אבא איסתומכא דליבא מאי אסותא מייתי כמונא כרוייא וניניא ואגדנא וציתרי ואבדתא}
\twocol{לליבא בחמרא וסימנך (תהלים קד, טו) ויין ישמח לבב אנוש לרוחא במיא וסימנך (בראשית א, ב) ורוח אלהים מרחפת על פני המים לכודא בשיכרא וסימנך (בראשית כד, טו) וכדה על שכמה
\par רב אחא בריה דרבא שחיק להו לכולהו בהדי הדדי ושקיל ליה מלא חמש אצבעתיה ושתי ליה רב אשי שחיק כל חד וחד לחודיה ושקיל מלא אצבעיה רבתי ומלא אצבעיה זוטרתי אמר רב פפא אנא עבדי לכל הני ולא איתסאי עד דאמר לי ההוא טייעא אייתי כוזא חדתא ומלייה מיא ורמי ביה תרוודא דדובשא דתלי לה בי כוכבי ולמחר אישתי עבדי הכי ואיתסאי}
\twocol{תנו רבנן ששה דברים מרפאין את החולה מחליו ורפואתן רפואה ואלו הן כרוב ותרדין ומי סיסין יבישה וקיבת והרת ויותרת הכבד ויש אומרים אף דגים קטנים ולא עוד אלא שדגים קטנים מפרין ומרבין כל גופו של אדם
\par עשרה דברים מחזירין את החולה לחליו וחליו קשה אלו הן האוכל בשר שור שומן בשר צלי בשר ציפרים וביצה צלויה ושחלים ותגלחת ומרחץ וגבינה וכבד וי"א אף אגוזים וי"א אף קשואין תנא דבי רבי ישמעאל למה נקרא שמן קשואין מפני שהן קשין לכל גופו של אדם כחרבות:}
\twocol{ואין מסתפרין מהן בכל מקום: ת"ר ישראל המסתפר מעובד כוכבים רואה במראה ועובד כוכבים המסתפר מישראל כיון שהגיע לבלוריתו שומט את ידו
\par אמר מר ישראל המסתפר מעובד כוכבים רואה במראה היכי דמי אי ברשות הרבים ל"ל מראה ואי ברשות היחיד כי רואה מאי הוי לעולם ברה"י וכיון דאיכא מראה מתחזי כאדם חשוב}
\twocol{רב חנא בר ביזנא הוה מסתפר מעובד כוכבים בשבילי דנהרדעא א"ל חנא חנא יאי קועיך לזוגא אמר תיתי לי דעברי אדר"מ
\par ואדרבנן לא עבר אימר דאמור רבנן ברה"ר ברה"י מי אמור והוא סבר שבילי דנהרדעא כיון דשכיחי רבים כרה"ר דמו:}
\twocol{ועובד כוכבים המסתפר מישראל כיון שהגיע לבלוריתו שומט את ידו: וכמה אמר רב מלכיה אמר רב אדא בר אהבה שלשה אצבעות לכל רוח ורוח
\par אמר רב חנינא בריה דרב איקא שפוד שפחות וגומות רב מלכיו בלורית אפר מקלה וגבינה רב מלכיה}
\twocol{אמר רב פפא מתני' ומתניתא רב מלכיה שמעתתא רב מלכיו וסימנא מתניתא מלכתא מאי בינייהו איכא בינייהו שפחות:
\par {\large\emph{מתני׳}} אלו דברים של עובדי כוכבים אסורין ואיסורן איסור הנאה היין והחומץ של עובדי כוכבים שהיה מתחלתו יין וחרס הדרייני ועורות לבובין רשב"ג אומר בזמן שהקרע שלו עגול אסור משוך מותר}
\twocol{בשר הנכנס לעבודת כוכבים מותר והיוצא אסור מפני שהוא כזבחי מתים דברי ר"ע ההולכין לתרפות אסור לשאת ולתת עמהן והבאין מותרין:
\par נודות העובדי כוכבים וקנקניהן ויין של ישראל כנוס בהן אסורין ואיסורן איסור הנאה דברי רבי מאיר וחכמים אומרים אין איסורן איסור הנאה: החרצנים והזגין של עובדי כוכבים אסורין ואיסורן איסור הנאה דברי ר"מ וחכ"א לחין אסורין יבישין מותרין}
\twocol{המורייס וגבינת בית אונייקי של עובדי כוכבים אסורין ואיסורן איסור הנאה דברי ר' מאיר וחכ"א אין איסורן איסור הנאה
\par אמר ר' יהודה שאל ר' ישמעאל את רבי יהושע כשהיו מהלכין בדרך אמר לו מפני מה אסרו גבינות עובדי כוכבים אמר לו מפני שמעמידין אותה בקיבה של נבילה}
\twocol{אמר לו והלא קיבת עולה חמורה מקיבת נבילה אמרו כהן שדעתו יפה שורפה חיה ולא הודו לו אבל אמרו אין נהנין ולא מועלין
\par אמר לו מפני שמעמידין אות' בקיבת עגלי עבודת כוכבים אמר לו אם כן למה לא אסרוה בהנאה}
\twocol{השיאו לדבר אחר אמר לו ישמעאל היאך אתה קורא כי טובים דודיך מיין או כי טובים דודיך
\par אמר לו כי טובים דודיך אמר לו אין הדבר כן שהרי חבירו מלמד עליו לריח שמניך טובים:}
\twocol{{\large\emph{גמ׳}} יין מנלן אמר רבה בר אבוה אמר קרא (דברים לב, לח) אשר חלב זבחימו יאכלו ישתו יין נסיכם מה זבח אסור בהנאה אף יין נמי אסור בהנאה
\par זבח גופיה מנלן דכתיב (תהלים קו, כח) ויצמדו לבעל פעור ויאכלו זבחי מתים מה מת אסור בהנאה אף זבח נמי אסור בהנאה}
\twocol{ומת גופיה מנלן אתיא שם שם מעגלה ערופה כתיב הכא (במדבר כ, א) ותמת שם מרים וכתיב התם (דברים כא, ד) וערפו שם את העגלה בנחל מה להלן אסור בהנאה אף כאן נמי אסור בהנאה
\par והתם מנלן אמרי דבי רבי ינאי כפרה כתיב בה כקדשים:}
\twocol{והחומץ של עובדי כוכבים שהיה מתחלתו יין: פשיטא משום דאחמיץ פקע ליה איסוריה אמר רב אשי הא אתא לאשמועינן חומץ שלנו ביד עובד כוכבים אין צריך חותם בתוך חותם אי משום אינסוכי לא מנסכי ואי משום איחלופי כיון דאיכא חותם לא טרח ומזייף
\par אמר רבי אילעא שנינו יין מבושל של עובדי כוכבים שהיה מתחלתו יין אסור פשיטא משום דאיבשיל פקע ליה איסורא אמר רב אשי הא אתא לאשמועינן יין מבושל שלנו ביד עובדי כוכבים אין צריך חותם בתוך חותם אי משום אינסוכי לא מנסכי ואי משום}
\newsection{דף ל}
\twocol{איחלופי כיון דאיכא חותם אחד לא טרח ומזייף
\par ת"ר יין מבושל ואלונתית של עובדי כוכבים אסורין אלונתית כברייתא מותרת ואיזו היא אלונתית כדתנן גבי שבת עושין אנומלין ואין עושין אלונתית ואיזו היא אנומלין ואיזו היא אלונתית אנומלין יין ודבש ופלפלין אלונתית יין ישן ומים צלולין ואפרסמון דעבדי לבי מסותא}
\twocol{רבה ורב יוסף דאמרי תרוייהו יין מזוג אין בו משום גילוי יין מבושל אין בו משום ניסוך איבעיא להו יין מבושל יש בו משום גילוי או אין בו משום גילוי ת"ש העיד רבי יעקב בר אידי על יין מבושל שאין בו משום גילוי
\par רבי ינאי בר ישמעאל חלש על לגביה ר' ישמעאל בן זירוד ורבנן לשיולי ביה יתבי וקא מבעיא להו יין מבושל יש בו משום גילוי או אין בו משום גילוי אמר להו ר' ישמעאל בן זירוד הכי אמר רשב"ל משום גברא רבה ומנו ר' חייא יין מבושל אין בו משום גילוי אמרו ליה נסמוך מחוי להו ר' ינאי בר ישמעאל עלי ועל צוארי}
\twocol{שמואל ואבלט הוו יתבי אייתו לקמייהו חמרא מבשלא משכיה לידיה א"ל שמואל הרי אמרו יין מבושל אין בו משום יין נסך
\par אמתיה דרבי חייא איגלויי לה ההוא חמרא מבשלא אתיא לקמיה דר' חייא אמר לה הרי אמרו יין מבושל אין בו משום גילוי שמעיה דרב אדא בר אהבה איגלי ליה חמרא מזיגא א"ל הרי אמרו יין מזוג אין בו משום גילוי}
\twocol{אמר רב פפא לא אמרן אלא דמזיג טובא אבל מזיג ולא מזיג שתי ומזיג ולא מזיג מי שתי והא רבה בר רב הונא הוה קאזיל בארבא והוה נקיט חמרא בהדיה וחזייה לההוא חיויא דצרי ואתי א"ל לשמעיה סמי עיניה דדין שקיל קלי מיא שדא ביה וסר לאחוריה
\par אחייא מסר נפשיה אמזיגא לא מסר נפשיה}
\twocol{ואמזיגא לא מסר נפשיה והא רבי ינאי הוה בי עכבורי ואמרי ליה בר הדיא הוה בי עכבורי הוו יתבי והוו קא שתו חמרא מזיגא פש להו חמרא בכובא וצרונהי בפרונקא וחזיא לההוא חיויא דשקיל מיא ורמא בכובא עד דמלא בכובא וסליק חמרא עילויה פרונקא ושתי
\par אמרי דמזיג איהו שתי דמזיגי אחריני לא שתי}
\twocol{אמר רב אשי ואיתימא רב משרשיא פירוקא לסכנתא אמר רבא הלכתא יין מזוג יש בו משום גילוי ויש בו משום יין נסך יין מבושל אין בו משום גילוי ואין בו משום יין נסך
\par שמעיה דרב חלקיה בר טובי איגליא ההוא קיסתא דמיא והוה ניים גבה אתא לגביה דרב חלקיה בר טובי א"ל הרי אמרו אימת ישן עליהן והני מילי ביממא אבל בליליא לא ולא היא לא שנא ביממא ול"ש בליליא אימת ישן עליהן לא אמרינן}
\twocol{רב לא שתי מבי ארמאה אמר לא זהירי בגילוי מבי ארמלתא שתי אמר סירכא דגברא נקיטא
\par שמואל לא שתי מיא מבי ארמלתא אמר לית לה אימתא דגברא ולא מיכסיא מיא אבל מבי ארמאה שתי נהי דאגילויא לא קפדי אמנקרותא מיהא קפדי א"ד רב לא שתי מיא מבי ארמאה אבל מבי ארמלתא שתי שמואל לא שתי מיא לא מבי ארמאה ולא מבי ארמלתא:}
\twocol{אריב"ל שלש יינות הן ואין בהן משום גילוי ואלו הן חד מר מתוק חד טילא חריפא דמצרי זיקי מר ירנקא מתוק חוליא רב חמא מתני לעילויא חד חמר ופלפלין מר אפסינתין מתוק מי בארג
\par אר"ש בן לקיש קרינא אין בו משום גילוי מאי קרינא א"ר אבהו חמרא חליא דאתי מעסיא אמר רבא ובמקומו יש בו משום גילוי מ"ט חמר מדינה הוא אמר רבא האי חמרא דאקרים עד תלתא יומי יש בו משום גילוי ומשום יין נסך}
\twocol{מכאן ואילך אין בו משום גילוי ואין בו משום יין נסך ונהרדעי אמרי אפי' לבתר תלתא יומי חיישינן משום גילוי מ"ט זימנין מיקרי שתי
\par ת"ר יין תוסס אין בו משום גילוי וכמה תסיסתו ג' ימים השחלים אין בהם משום גילוי ובני גולה נהגו בהן איסור ולא אמרן אלא דלית בהו חלא אבל אית בהו חלא מיגרי בהו}
\twocol{כותח הבבלי אין בו משום גילוי ובני גולה נהגו בו איסור אמר רב מנשי אי אית ביה נקורי חיישינן אמר רב חייא בר אשי אמר שמואל מי טיף טיף אין בו משום גילוי אמר רב אשי והוא דעביד טיף להדי טיף טיף
\par אמר רב חייא בר אשי אמר שמואל פי תאנה אין בו משום גילוי כמאן כי האי תנא דתניא רבי אליעזר אומר אוכל אדם ענבים ותאנים בלילה ואינו חושש משום שנאמר (תהלים קטז, ו) שומר פתאים ה'}
\twocol{אמר רב ספרא משום ר' יהושע דרומא שלשה מיני ארס הן של בחור שוקע של בינוני מפעפע ושל זקן צף למימרא דכמה דקשיש כחוש חיליה והתניא שלשה כל זמן שמזקינין גבורה מתוספת בהן אלו הן דג נחש וחזיר כח אוסופי הוא דקא מוסיף זיהריה קליש
\par של בחור שוקע למאי הלכתא דתניא חבית שנתגלה אע"פ ששתו ממנה תשעה ולא מתו לא ישתה ממנה עשירי מעשה היה ששתו ממנו תשעה ולא מתו ושתה עשירי ומת אמר ר' ירמיה זהו שוקע}
\twocol{וכן אבטיח שנתגלתה אע"פ שאכלו ממנה ט' בני אדם ולא מתו לא יאכל ממנה עשירי מעשה היה ואכלו ממנה תשעה ולא מתו ואכל עשירי ומת א"ר זהו שוקע
\par ת"ר מים שנתגלו הרי זה לא ישפכם ברשות הרבים ולא ירביץ בהן את הבית ולא יגבל בהן את הטיט ולא ישקה מהן לא בהמתו ולא בהמת חבירו ולא ירחץ בהן פניו ידיו ורגליו אחרים אומרים מקום שיש סירטא אסור אין סירטא מותר}
\twocol{אחרים היינו תנא קמא איכא בינייהו גב היד וגב הרגל ורומני דאפי
\par אמר מר לא ישקה מהן לא בהמתו ולא בהמת חבירו והתניא אבל משקהו לבהמת עצמו כי תניא ההיא לשונרא אי הכי דחבריה נמי דחבריה כחיש דידיה נמי כחיש הדר בריא דחבריה נמי הדר בריא זימנין דבעי לזבונא ומפסיד ליה מיניה}
\twocol{א"ר אסי א"ר יוחנן משום ר' יהודה בן בתירא שלשה יינות הן יין נסך אסור בהנאה ומטמא טומאה חמורה בכזית}
\newsection{דף לא}
\twocol{סתם יינם אסור בהנאה ומטמא טומאת משקין ברביעית המפקיד יינו אצל עובד כוכבים אסור בשתיה ומותר בהנאה
\par והתנן המפקיד פירותיו אצל עובד כוכבים הרי הן כפירותיו של עובד כוכבים לשביעית ולמעשר כגון שייחד לו קרן זוית}
\twocol{אי הכי בשתיה נמי לישתרי דהא רבי יוחנן אקלע לפרוד אמר כלום יש משנת בר קפרא תנא ליה ר' תנחום דמן פרוד המפקיד יינו אצל עובד כוכבים מותר בשתיה
\par קרי עליה (קהלת יא, ג) מקום שיפול העץ שם יהו שם יהו ס"ד אלא שם יהו פירותיו}
\twocol{א"ר זירא לא קשיא הא ר"א הא רבנן
\par דתניא אחד הלוקח ואחד השוכר בית בחצירו של עובד כוכבים ומלאוהו יין ומפתח או חותם ביד ישראל ר"א מתיר וחכמים אוסרין}
\twocol{א"ר חייא בריה דרבי חייא בר נחמני א"ר חסדא אמר רב ואמרי לה אמר רב חסדא אמר זעירי ואמרי לה א"ר חסדא אמר לי אבא בר חמא הכי אמר זעירי הלכה כר"א
\par אמר רבי אלעזר הכל משתמר בחותם אחד חוץ מן היין שאין משתמר בחותם אחד ור' יוחנן אמר אפי' יין משתמר בחותם אחד ולא פליגי הא כר"א הא כרבנן}
\twocol{איכא דאמרי א"ר אלעזר הכל משתמר בחותם בתוך חותם חוץ מן היין שאין משתמר בחותם בתוך חותם ורבי יוחנן אמר אפילו יין משתמר בחותם בתוך חותם ותרוייהו כרבנן מר סבר כי פליגי רבנן עליה דר"א בחותם אחד אבל בחותם בתוך חותם שרו ומר סבר אפילו חותם בתוך חותם אסרי
\par היכי דמי חותם בתוך חותם אמר רבא אגנא דפומא דחביתא שריקא וחתימא הוי חותם בתוך חותם ואי לא לא דיקולא ומיהדק הוי חותם בתוך חותם לא מיהדק לא הוי חותם בתוך חותם נוד בדיסקיא חתימת פיו למטה הוי חותם בתוך חותם פיו למעלה לא הוי חותם בתוך חותם וכי כייף פומיה לגיו וצייר וחתים הוי חותם בתוך חותם}
\twocol{ת"ר בראשונה היו אומרים יין של עין כושי אסור מפני בירת סריקא ושל ברקתא אסור מפני כפר פרשאי ושל זגדור אסור מפני כפר שלים חזרו לומר חביות פתוחות אסורות סתומות מותרות
\par מעיקרא מאי סבור ולבסוף מאי סבור מעיקרא סבור אין כותי מקפיד על מגע עובד כוכבים לא שנא פתוחות ולא שנא סתומות ולבסוף סבור כי לא קפיד אפתוחות אסתומות מקפיד קפיד}
\twocol{וסתומות מותרות ורמינהי
\par השולח חבית של יין ביד כותי ושל ציר ושל מורייס ביד עובד כוכבים אם מכיר חותמו וסתמו מותר אם לאו אסור}
\twocol{אמר רבי זירא לא קשיא כאן בעיר כאן בדרך
\par מתקיף לה רבי ירמיה מידי הנך דעיר לא בדרך אתו אלא אמר רבי ירמיה בין הגיתות שנינו כיון דכולי עלמא אפכי מירתת אמר השתא אי חזי לי מפסדו לי}
\twocol{אתמר מפני מה אסרו שכר של עובדי כוכבים רמי בר חמא אמר רבי יצחק משום חתנות רב נחמן אמר משום גילוי
\par אגילוי דמאי אילימא גילוי דנזייתא אנן נמי מגלינן ואלא דחביתא אנן נמי מגלינן לא צריכא באתרא דמצלו מיא}
\twocol{אלא מעתה ישן תשתרי דא"ר ישן מותר אין מניחו ליישן החמיץ מותר אין מניחו להחמיץ גזירה ישן אטו חדש
\par רב פפא מפיקין ליה לאבבא דחנותא ושתי רב אחאי מייתו ליה לביתיה ושתי ותרוייהו משום חתנות רב אחאי עביד הרחקה יתירתא}
\twocol{רב שמואל בר ביסנא איקלע למרגואן אייתו ליה חמרא ולא אשתי אייתו ליה שיכרא ולא אשתי בשלמא חמרא משום שימצא שיכרא משום מאי משום שימצא דשימצא
\par אמר רב האי שיכרא דארמאה שרי וחייא ברי לא נישתי מיניה מה נפשך אי שרי לכולי עלמא שרי אי אסיר לכולי עלמא אסיר}
\twocol{אלא רב סבר משום גילויא ואזיל מרורא דכשותא וקלי ליה זיהריה ודלקי מלקי ליה טפי וחייא ברי הואיל ולקי לא נישתי מיניה
\par אמר שמואל כל השרצים יש להן ארס של נחש ממית של שרצים אינו ממית אמר ליה שמואל לחייא בר רב בר אריא תא ואימא לך מילתא מעלייתא דהוה אמר רב אבוך הכי אמר אבוך הני ארמאי זוקאני דהוו שתו גילויא ולא מתו איידי דאכלי שקצים ורמשים חביל גופייהו}
\twocol{אמר רב יוסף}
\newsection{דף לב}
\twocol{האי חלא דשיכרא דארמאה אסור דמערבי ביה דורדיא דיין נסך אמר רב אשי ומאוצר שרי כיון דמערבי ביה מסרא סרי:
\par וחרס הדרייני: מאי הדרייני אמר רב יהודה אמר שמואל חרס של הדריינוס קיסר כי אתא רב דימי אמר קרקע בתולה היתה שלא עבדה אדם מעולם עבדה ונטעה ורמי ליה לחמרא בגולפי חיורי ומייצי להו לחמרייהו ומתברו להו בחספי ודרו בהדייהו וכל היכא דמטו תרו להו ושתו א"ר יהושע בן לוי וראשון שלנו כשלישי שלהן}
\twocol{איבעיא להו מהו לסמוך בהן כרעי המטה רוצה בקיומו ע"י ד"א שרי או אסור
\par ת"ש דר"א ורבי יוחנן חד אסר וחד שרי והלכתא כמאן דאסר}
\twocol{מיתיבי הדרדורין והרוקבאות של עובדי כוכבים יין של ישראל כנוס בהן אסור בשתיה ומותר בהנאה העיד שמעון בן גודא לפני בנו של ר"ג על ר"ג ששתה ממנו בעכו ולא הודו לו
\par נודות של עובדי כוכבים רשב"ג אומר משום רבי יהושע בן קפוסאי אסור לעשות מהן שטיחין לחמור והא הכא דרוצה בקיומו ע"י דבר אחר וקתני דאסור}
\twocol{וליטעמיך קנקנים של עובדי כוכבים ליתסרו למיזבן מאי שנא נודות ומ"ש קנקנים אמר רבא גזירה שמא יבקע נודו ויטלנו ויתפרנו על גבי נודו
\par ולמ"ד רוצה בקיומו על ידי ד"א אסור מ"ש קנקנים דשרו אמר לך התם ליתיה לאיסוריה בעיניה הכא איתיה לאיסוריה בעיניה:}
\twocol{ולא הודו לו: ורמינהי יין הבא ברוקבאות של עובדי כוכבים אסור בשתיה ומותר בהנאה העיד שמעון בן גודע לפני בנו של ר"ג על ר"ג ששתה ממנו בעכו והודו לו
\par מאי לא הודו לו דקאמר התם כל סייעתו אבל בנו מודי ליה איבעית אימא גודא לחוד וגודע לחוד:}
\twocol{ועורות לבובין: תנו רבנן איזהו עור לבוב כל שקרוע כנגד הלב וקדור כמין ארובה יש עליו קורט דם אסור
\par אין עליו קורט דם מותר אמר רב הונא לא שנו אלא שלא מלחו אבל מלחו אסור אימא מלחו העברתו:}
\twocol{רשב"ג אומר בזמן שהקרע שלו עגול אסור משוך מותר: אמר רב יוסף אמר רב יהודה אמר שמואל הלכה כרשב"ג
\par א"ל אביי הלכה מכלל דפליגי א"ל מאי נפקא לך מינה א"ל גמרא גמור זמורתא תהא:}
\twocol{בשר הנכנס לעבודת כוכבים מותר: מאן תנא אמר ר' חייא בר אבא א"ר יוחנן דלא כרבי אלעזר דאי כרבי אלעזר האמר סתם מחשבת עובד כוכבים לעבודת כוכבים:
\par והיוצא אסור מפני שהוא כזבחי מתים: מ"ט אי אפשר דליכא תקרובת עבודת כוכבים מני רבי יהודה בן בתירא היא}
\twocol{דתניא רבי יהודה בן בתירא אומר מנין לתקרובת עבודת כוכבים שמטמא באהל שנאמר (תהלים קו, כח) ויצמדו לבעל פעור ויאכלו זבחי מתים מה מת מטמא באהל אף תקרובת עבודת כוכבים מטמאה באהל:
\par ההולכין לתרפות אסורין לשאת ולתת עמהם: אמר שמואל עובד כוכבים ההולך לתרפות בהליכה אסור דאזיל ומודי קמי עבודת כוכבים בחזרה מותר מאי דהוה הוה}
\twocol{ישראל ההולך לתרפות בהליכה מותר דלמא הדר ביה ולא אזיל בחזרה אסור כיון}
\newsection{דף לג}
\twocol{דאביק בה מהדר הדר אזיל
\par והתניא ישראל ההולך לתרפות בין בהליכה בין בחזרה אסור אמר רב אשי כי תניא ההיא בישראל מומר דודאי אזיל}
\twocol{ת"ר עובד כוכבים ההולך ליריד בין בהליכה בין בחזרה מותר ישראל ההולך ליריד בהליכה מותר בחזרה אסור
\par מאי שנא ישראל דבחזרה אסור דאמרי עבודת כוכבים זבין דמי עבודת כוכבים איכא בהדיה עובד כוכבים נמי נימא עבודת כוכבים זבין דמי עבודת כוכבים איכא בהדיה}
\twocol{אלא עובד כוכבים אמרינן גלימא זבין חמרא זבין ישראל נמי נימא אימור גלימא זבין חמרא זבין אי איתא דה"ל הכא הוה מזבין ליה:
\par והבאין מותרין: ארשב"ל לא שנו אלא שאין קשורין זה בזה אבל קשורין זה בזה אסורין אימא דעתו לחזור:}
\twocol{נודות העובדי כוכבים וקנקניהם: ת"ר נודות העובדי כוכבים גרודים חדשים מותרין ישנים ומזופפין אסורין עובד כוכבים ריבבן ועיבדן ונתן לתוכן יין וישראל עומד על גביו אינו חושש
\par וכי מאחר דעובד כוכבים נותן לתוכן יין כי ישראל עומד [על] גביו מאי הוי אמר רב פפא ה"ק עובד כוכבים ריבבן ועיבדן וישראל נותן לתוכן יין וישראל אחר עומד על גביו ואינו חושש}
\twocol{ומאחר דישראל נותן לתוכן יין ישראל אחר עומד על גביו למה לי דלמא אגב טירדיה מנסך ולאו אדעתיה
\par רב זביד אמר לעולם כדקאמרת מעיקרא והכא בעידנא דקא שדי ליה נעשה כזורק מים לטיט אמר רב פפי ש"מ מדרב זביד האי עובד כוכבים דשדא חמרא לבי מילחי דישראל שרי}
\twocol{מתקיף לה רב אשי מי דמי התם קאזיל לאיבוד הכא לא קאזיל לאיבוד
\par בר עדי טייעא אנס הנהו זיקי מרב יצחק בר יוסף רמא בהו חמרא ואהדרינהו ניהליה אתא שאיל בי מדרשא א"ל רבי ירמיה כך הורה רבי אמי הלכה למעשה ממלאן מים שלשה ימים ומערן ואמר רבא צריך לערן מעת לעת}
\twocol{סבור מינה הני מילי דידן אבל דידהו לא כי אתא רבין א"ר שמעון בן לקיש אחד שלנו ואחד שלהם סבר רב אחא בריה דרבא קמיה דרב אשי למימר הני מילי נודות אבל קנקנים לא אמר ליה רב אשי לא שנא נודות ולא שנא קנקנים
\par ת"ר קנקנים של עובדי כוכבים חדשים גרודים מותרין ישנים ומזופפין אסורין עובד כוכבים נותן לתוכן יין ישראל נותן לתוכן מים עובד כוכבים נותן לתוכן יין ישראל נותן לתוכן ציר ומורייס ואינו חושש}
\twocol{איבעיא להו
\par לכתחלה או דיעבד ת"ש דתני רב זביד בר אושעיא הלוקח קנקנים מן העובדי כוכבים חדשים נותן לתוכן יין ישנים נותן לתוכן ציר ומורייס לכתחלה}
\twocol{בעא מיניה ר' יהודה נשיאה מרבי אמי החזירן לכבשן האש ונתלבנו מהו א"ל ציר שורף אור לא כ"ש אתמר נמי א"ר יוחנן ואמרי לה אמר ר' אסי א"ר יוחנן קנקנים של עובדי כוכבים שהחזירן לכבשן האש כיון שנשרה זיפתן מהן מותרין
\par אמר רב אשי לא תימא עד דנתרן אלא אפי' רפאי מירפא אע"ג דלא נתר קינסא פליגי בה רב אחא ורבינא חד אסר וחד שרי והלכתא כמאן דאסר}
\twocol{איבעיא להו מהו ליתן לתוכו שכר רב נחמן ורב יהודה אסרי ורבא שרי רבינא שרא ליה לרב חייא בריה דרב יצחק למירמא ביה שכרא אזל רמא ביה חמרא ואפילו הכי לא חש לה למילתא אמר אקראי בעלמא הוא
\par רב יצחק בר ביסנא הוה ליה הנהו מאני דפקוסנא מלינהו מיא אנחינהו בשימשא פקעו א"ל רבי אבא אסרתינהו עלך איסורא דלעולם אימור דאמור רבנן ממלינהו מיא אנוחי בשימשא מי אמור}
\twocol{א"ר יוסנא א"ר אמי כלי נתר אין לו טהרה עולמית מאי כלי נתר א"ר יוסי בר אבין כלי מחפורת של צריף
\par דבי פרזק רופילא אנס הני כובי מפומבדיתא רמא בהו חמרא אהדרינהו ניהלייהו אתו שיילוהו לרב יהודה אמר דבר שאין מכניסו לקיום הוא משכשכן במים והן מותרין}
\twocol{אמר רב עוירא הני חצבי שחימי דארמאי כיון דלא בלעי טובא משכשכן במים ומותרין אמר רב פפי הני פתוותא דבי מיכסי כיון דלא בלעי טובא משכשכן במים ומותרין
\par כסי רב אסי אסר ורב אשי שרי אי שתי בהו עובד כוכבים פעם ראשון כ"ע לא פליגי דאסור כי פליגי בפעם שני}
\twocol{איכא דאמרי פעם ראשון ושני כ"ע לא פליגי דאסור כי פליגי בפעם ג' והלכתא פעם ראשון ושני אסור ג' מותר
\par א"ר זביד האי מאני דקוניא חיורא ואוכמא שרי ירוקא אסור משום דמיצריף ואי אית בהו קרטופני כולהו אסירי דרש מרימר קוניא בין אוכמא בין חיורא בין ירוקא שרי}
\twocol{מאי שנא מחמץ בפסח דבעו מיניה ממרימר הני מאני דקוניא מהו לאשתמושי בהו בפיסחא ירוקא לא תיבעי לך דמצרפי ובלעי ואסירי כי תיבעי לך חיורי ואוכמי מאי
\par כי אית בהו קרטופני לא תיבעי לך דודאי בלעי ואסירי כי תיבעי לך דשיעי מאי אמר}
\newsection{דף לד}
\twocol{להו אנא חזינא להו דמדייתי וכיון דמדייתי ודאי בלעי ואסירי מ"ט התורה העידה על כלי חרס שאינו יוצא מידי דופנו לעולם
\par מ"ש מיין נסך דדרש להו מרימר כולהו מאני דקוניא שרי}
\twocol{וכ"ת חמץ דאורייתא יין נסך דרבנן והא כל דתקון רבנן כעין דאורייתא תקון זה תשמישו בחמין וזה תשמישו בצונן
\par ר"ע איקלע לגינזק בעו מיניה מתענין לשעות או אין מתענין לשעות לא הוה בידיה קנקנים של עובדי כוכבים אסורין או מותרין לא הוה בידיה במה שימש משה כל שבעת ימי המלואים לא הוה בידיה}
\twocol{אתא שאל בי מדרשא אמרי הלכתא מתענין לשעות ואם השלים מתפלל תפלת תענית והלכתא קנקנים של עובדי כוכבים לאחר י"ב חדש מותרין במה שימש משה שבעת ימי המלואים בחלוק לבן רב כהנא מתני בחלוק לבן שאין בו אימרא:
\par החרצנים והזגים של עובדי כוכבים וכו': ת"ר החרצנים והזגים של עובדי כוכבים לחין אסורין יבשים מותרים הי נינהו לחין והי נינהו יבשין אמר רב יהודה אמר שמואל לחין כל י"ב חדש יבשים לאחר י"ב חדש}
\twocol{אתמר אמר רבה בר בר חנה א"ר יוחנן כשהן אסורין אסורין אפילו בהנאה כשהן מותרין מותרין אפילו באכילה
\par א"ר זביד האי דורדיא דחמרא דארמאי בתר תריסר ירחי שתא שרי אמר רב חביבא בריה דרבא הני גולפי בתר תריסר ירחי שתא שרי אמר רב חביבא הני}
\twocol{אבטא דטייעי בתר תריסר ירחי שתא שרי אמר רב אחא בריה דרב איקא הני פורצני דארמאי בתר תריסר ירחי שתא שרי אמר רב אחא בריה דרבא הני גולפי שחימי ואוכמי בתר תריסר ירחי שתא שרי:
\par והמורייס: ת"ר מורייס אומן מותר ר' יהודה בן גמליאל אומר משום ר' חנינא ב"ג אף חילק אומן מותר}
\twocol{תני אבימי בריה דר' אבהו מורייס אומן מותר הוא תני לה והוא א"ל פעם ראשון ושני מותר שלישי אסור מ"ט פעם ראשון ושני דנפיש שומנייהו לא צריך למירמי בהו חמרא מכאן ואילך רמו בהו חמרא
\par ההוא ארבא דמורייסא דאתי לנמילא דעכו אותיב רבי אבא דמן עכו נטורי בהדה א"ל רבא עד האידנא מאן נטרה א"ל עד האידנא למאן ניחוש לה אי משום דמערבי ביה חמרא קיסתא דמורייס בלומא קיסתא דחמרא בד' לומי}
\twocol{א"ל ר' ירמיה לר' זירא דלמא איידי דצור אתו דשוי חמרא א"ל התם עיקולי ופשורי איכא:
\par וגבינת בית אונייקי: ארשב"ל מפני מה אסרו גבינת אונייקי מפני שרוב עגלים של אותה עיר נשחטין לעבודת כוכבים מאי איריא רוב עגלים אפילו מיעוט נמי דהא ר"מ חייש למיעוטא}
\twocol{אי אמרת רוב איכא מיעוט
\par אלא אי אמרת מיעוט כיון דאיכא רוב עגלים דאין נשחטין לעבודת כוכבים ואיכא נמי שאר בהמות דאין נשחטין לעבודת כוכבים ה"ל מיעוטא דמיעוטא ומיעוטא דמיעוטא לא חייש ר"מ}
\twocol{א"ל ר"ש בר אליקים לר"ש בן לקיש כי נשחטין לעבודת כוכבים מאי הוי והא את הוא דשרי
\par דאתמר השוחט את הבהמה לזרוק דמה לעבודת כוכבים להקטיר חלבה לעבודת כוכבים רבי יוחנן אמר אסורה קסבר מחשבין מעבודה לעבודה וילפינן חוץ מפנים}
\twocol{ורשב"ל אמר מותרת
\par א"ל תרמינך שעתך באומר בגמר זביחה הוא עובדה:}
\twocol{א"ר יהודה שאל ר' ישמעאל: אמר רב אחדבוי אמר רב המקדש בפרש שור הנסקל מקודשת בפרש עגלי עבודת כוכבים אינה מקודשת איבעית אימא סברא ואב"א קרא
\par איבעית אימא סברא גבי עגלי עבודת כוכבים ניחא ליה בנפחיה אבל גבי שור הנסקל לא ניחא ליה בנפחיה}
\twocol{איבעית אימא קרא כתיב הכא (דברים יג, יח) לא ידבק בידך מאומה וכתיב התם (שמות כא, כח) סקול יסקל השור ולא יאכל את בשרו בשרו אסור הא פרשו מותרת
\par אמר רבא תרוייהו תננהי מדקא"ל מפני שמעמידין בקיבת נבילה וקא מהדר ליה והלא קיבת עולה חמורה מקיבת נבילה}
\newsection{דף לה}
\twocol{מכלל דאיסורי הנאה שרו פרשייהו
\par ומדקא"ל מפני שמעמידין אותה בקיבת עגלי עבודת כוכבים וקא מהדר ליה א"כ למה לא אסרוה בהנאה מכלל דעבודת כוכבים אסור פרשייהו}
\twocol{ולהדר ליה משום דליתיה לאיסורא בעיניה
\par דהא מורייס לרבנן דלא אסרוהו בהנאה מ"ט לאו משום דליתיה לאיסורא בעיניה}
\twocol{אמרי הכא כיון דאוקמיה קא מוקים חשיב ליה כמאן דאיתיה לאיסוריה בעיניה:
\par השיאו לדבר אחר וכו': מאי (שיר השירים א, ב) כי טובים דודיך מיין כי אתא רב דימי אמר אמרה כנסת ישראל לפני הקב"ה רבש"ע עריבים עלי דברי דודיך יותר מיינה של תורה}
\twocol{מ"ש האי קרא דשייליה אר"ש בן פזי ואיתימא ר"ש בר אמי מרישיה דקרא קא"ל (שיר השירים א, ב) ישקני מנשיקות פיהו אמר ליה ישמעאל אחי חשוק שפתותיך זו בזו ואל תבהל להשיב
\par מ"ט אמר עולא ואיתימא רב שמואל בר אבא גזרה חדשה היא ואין מפקפקין בה מאי גזירתא אר"ש בן פזי אמר ריב"ל משום ניקור}
\twocol{ולימא ליה משום ניקור כדעולא דאמר עולא כי גזרי גזירתא במערבא לא מגלו טעמא עד תריסר ירחי שתא דלמא איכא איניש דלא ס"ל ואתי לזלזולי בה
\par מגדף בה ר' ירמיה אלא מעתה יבשה תשתרי ישן תשתרי דא"ר חנינא יבש מותר אין מניחו ליבש ישן מותר אין מניחו לישן}
\twocol{א"ר חנינא לפי שא"א לה בלא צחצוחי חלב ושמואל אמר מפני שמעמידין אותה בעור קיבת נבילה
\par הא קיבה גופא שריא ומי אמר שמואל הכי והתנן קיבת העובד כוכבים ושל נבילה הרי זו אסורה}
\twocol{והוינן בה אטו דעובד כוכבים לאו נבלה היא
\par ואמר שמואל חדא קתני קיבת שחיטת עובד כוכבים נבלה אסורה}
\twocol{ל"ק
\par כאן קודם חזרה כאן לאחר חזרה ומשנה לא זזה ממקומה}
\twocol{רב מלכיא משמיה דרב אדא בר אהבה אמר מפני שמחליקין פניה בשומן חזיר רב חסדא אמר מפני שמעמידין אותה בחומץ רב נחמן בר יצחק אמר מפני שמעמידין אותה בשרף הערלה
\par כמאן כי האי תנא (דתניא) ר"א אומר המעמיד בשרף הערלה אסור מפני שהוא פירי}
\twocol{אפי' תימא ר' יהושע עד כאן לא פליג ר' יהושע עליה דר"א אלא בקטפא דגוזא אבל בקטפא דפירא מודי
\par והיינו דתנן א"ר יהושע שמעתי בפירוש שהמעמיד בשרף העלין ובשרף העיקרין מותר בשרף הפגין אסור מפני שהוא פירי}
\twocol{בין לרב חסדא בין לרב נחמן בר יצחק תתסר בהנאה קשיא
\par דרש רב נחמן בריה דרב חסדא מאי דכתיב (שיר השירים א, ג) לריח שמניך טובים למה ת"ח דומה לצלוחית של פלייטין מגולה ריחה נודף מכוסה אין ריחה נודף}
\twocol{ולא עוד אלא דברים שמכוסין ממנו מתגלין לו שנאמר (שיר השירים א, ג) עלמות אהבוך קרי ביה עלומות ולא עוד אלא שמלאך המות אוהבו שנא' עלמות אהבוך קרי ביה על מות ולא עוד אלא שנוחל שני עולמות אחד העוה"ז ואחד העוה"ב שנא' עלמות קרי ביה עולמות:
\par {\large\emph{מתני׳}} ואלו דברים של עובדי כוכבים אסורין ואין איסורן איסור הנאה חלב שחלבו עובד כוכבים ואין ישראל רואהו והפת והשמן שלהן רבי ובית דינו התירו השמן}
\twocol{והשלקות וכבשין שדרכן לתת לתוכן יין וחומץ וטרית טרופה וציר שאין בה דגה כלבית שוטטת בו והחילק וקורט של חלתית ומלח שלקונדית הרי אלו אסורין ואין איסורן איסור הנאה:
\par {\large\emph{גמ׳}} חלב למאי ניחוש לה אי משום איחלופי טהור חיור טמא ירוק ואי משום איערובי ניקום דאמר מר חלב טהור עומד חלב טמא אינו עומד}
\twocol{אי דקא בעי לגבינה ה"נ הכא במאי עסקינן דקא בעי ליה לכמכא
\par ונשקול מיניה קלי וניקום כיון דבטהור נמי איכא נסיובי דלא קיימי ליכא למיקם עלה דמילתא}
\twocol{ואב"א אפי' תימא דקבעי לה לגבינה איכא דקאי ביני אטפי:
\par והפת: א"ר כהנא א"ר יוחנן פת לא הותרה בב"ד מכלל דאיכא מאן דשרי}
\twocol{אין דכי אתא רב דימי אמר פעם אחת יצא רבי לשדה והביא עובד כוכבים לפניו פת פורני מאפה סאה אמר רבי כמה נאה פת זו מה ראו חכמים לאוסרה מה ראו חכמים משום חתנות
\par אלא מה ראו חכמים לאוסרה בשדה כסבורין העם התיר רבי הפת ולא היא רבי לא התיר את הפת}
\twocol{רב יוסף ואיתימא רב שמואל בר יהודה אמר לא כך היה מעשה אלא אמרו פעם אחת הלך רבי למקום אחד וראה פת דחוק לתלמידים אמר רבי אין כאן פלטר כסבורין העם לומר פלטר עובד כוכבים והוא לא אמר אלא פלטר ישראל
\par א"ר חלבו אפילו למ"ד פלטר עובד כוכבים לא אמרן אלא דליכא פלטר ישראל אבל במקום דאיכא פלטר ישראל לא ורבי יוחנן אמר אפי' למ"ד פלטר עובד כוכבים ה"מ בשדה אבל בעיר לא משום חתנות}
\twocol{איבו הוה מנכית ואכיל פת אבי מצרי אמר להו רבא ואיתימא רב נחמן בר יצחק לא תשתעו בהדיה דאיבו דקאכיל לחמא דארמאי:
\par והשמן שלהן: שמן רב אמר דניאל גזר עליו ושמואל אמר}
\newsection{דף לו}
\twocol{זליפתן של כלים טמאים אוסרתן אטו כולי עלמא אוכלי טהרות נינהו אלא זליפתן של כלים אסורין אוסרתן
\par א"ל שמואל לרב בשלמא לדידי דאמינא זליפתן של כלים אסורין אוסרתן היינו דכי אתא רב יצחק בר שמואל בר מרתא ואמר דריש רבי שמלאי בנציבין שמן ר' יהודה ובית דינו נמנו עליו והתירוהו}
\twocol{קסבר נותן טעם לפגם מותר
\par אלא לדידך דאמרת דניאל גזר עליו דניאל גזר ואתא רבי יהודה הנשיא ומבטל ליה והתנן אין ב"ד יכול לבטל דברי ב"ד חבירו אלא א"כ גדול הימנו בחכמה ובמנין}
\twocol{א"ל שמלאי לודאה קא אמרת שאני לודאי דמזלזלו א"ל אשלח ליה איכסיף
\par אמר רב אם הם לא דרשו אנן לא דרשינן והכתיב (דניאל א, ח) וישם דניאל על לבו אשר לא יתגאל בפת בג המלך וביין משתיו בשתי משתאות הכתוב מדבר אחד משתה יין ואחד משתה שמן}
\twocol{רב סבר על לבו שם ולכל ישראל הורה ושמואל סבר על לבו שם ולכל ישראל לא הורה
\par ושמן דניאל גזר והאמר באלי אבימי נותאה משמיה דרב פיתן ושמנן יינן ובנותיהן כולן משמנה עשר דבר הן}
\twocol{וכי תימא אתא דניאל גזר ולא קיבל ואתו תלמידי דהלל ושמאי וגזור וקיבל א"כ מאי אסהדותיה דרב אלא דניאל גזר עליו בעיר ואתו אינהו וגזור אפילו בשדה
\par ור' יהודה הנשיא היכי מצי למישרא תקנתא דתלמידי שמאי והלל והתנן אין בית דין יכול לבטל דברי בית דין חבירו אלא אם כן גדול הימנו בחכמה ובמנין ועוד הא אמר רבה בר בר חנה אמר ר' יוחנן בכל יכול לבטל בית דין דברי בית דין חבירו חוץ משמונה עשר דבר שאפילו יבא אליהו ובית דינו אין שומעין לו}
\twocol{אמר רב משרשיא מה טעם הואיל ופשט איסורו ברוב ישראל שמן לא פשט איסורו ברוב ישראל דאמר רבי שמואל בר אבא אמר רבי יוחנן ישבו רבותינו ובדקו על שמן שלא פשט איסורו ברוב ישראל וסמכו רבותינו על דברי רשב"ג ועל דברי רבי אלעזר בר צדוק שהיו אומרים אין גוזרין גזירה על הצבור אא"כ רוב צבור יכולין לעמוד בה דאמר רב אדא בר אהבה מאי קרא
\par (מלאכי ג, ט) במארה אתם נארים ואותי אתם קובעים הגוי כולו אי איכא גוי כולו אין אי לא לא}
\twocol{גופא אמר באלי אמר אבימי נותאה משמיה דרב פיתן ושמנן יינן ובנותיהן כולן משמונה עשר דבר הן בנותיהן מאי היא אמר רב נחמן בר יצחק גזרו על בנותיהן נידות מעריסותן
\par וגניבא משמיה דרב אמר כולן משום עבודת כוכבים גזרו בהן דכי אתא רב אחא בר אדא א"ר יצחק גזרו על פיתן משום שמנן מאי אולמיה דשמן מפת}
\twocol{אלא על פיתן ושמנן משום יינן ועל יינן משום בנותיהן ועל בנותיהן משום דבר אחר ועל דבר אחר משום ד"א
\par בנותיהן דאורייתא היא דכתיב (דברים ז, ג) לא תתחתן בם דאורייתא ז' אומות אבל שאר עובדי כוכבים לא ואתו אינהו וגזור אפילו דשאר עובדי כוכבים}
\twocol{ולר"ש בן יוחי דאמר (דברים ז, ד) כי יסיר את בנך מאחרי לרבות כל המסירות מאי איכא למימר אלא דאורייתא אישות דרך חתנות ואתו אינהו גזור אפילו דרך זנות
\par זנות נמי בבית דינו של שם גזרו דכתיב (בראשית לח, כד) ויאמר יהודה הוציאוה ותשרף}
\twocol{אלא דאורייתא עובד כוכבים הבא על בת ישראל דמשכה בתריה אבל ישראל הבא על העובדת כוכבים לא ואתו אינהו גזור אפי' ישראל הבא על העובדת כוכבים
\par ישראל הבא על העובדת כוכבים הלכה למשה מסיני היא דאמר מר הבועל ארמית קנאין פוגעין בו}
\twocol{א"ל דאורייתא בפרהסיא וכמעשה שהיה ואתו אינהו גזור אפילו בצינעא בצינעא נמי בית דינו של חשמונאי גזרו
\par [דכי אתא רב דימי אמר ב"ד של חשמונאי גזרו] ישראל הבא על העובדת כוכבים חייב משום נשג"א}
\twocol{כי אתא רבין אמר משום נשג"ז
\par כי גזרו בית דינו של חשמונאי ביאה אבל ייחוד לא ואתו אינהו גזור אפי' ייחוד ייחוד נמי בית דינו של דוד גזרו}
\twocol{דאמר רב יהודה באותה שעה גזרו על ייחוד אמרי התם ייחוד דבת ישראל אבל ייחוד דעובדת כוכבים לא ואתו אינהו גזרו אפי' אייחוד דעובדת כוכבים
\par ייחוד דבת ישראל דאורייתא היא דאמר ר' יוחנן משום ר"ש בן יהוצדק רמז לייחוד מן התורה מנין שנאמר (דברים יג, ז) כי יסיתך אחיך בן אמך וכי בן אם מסית בן אב אינו מסית}
\twocol{אלא בן מתייחד עם אמו ואין אחר מתייחד עם כל עריות שבתורה
\par ייחוד דאורייתא דאשת איש ואתא דוד וגזר אפי' אייחוד דפנויה ואתו תלמידי בית שמאי ובית הלל גזור אפי' אייחוד דעובדת כוכבים}
\twocol{מאי על ד"א משום ד"א אמר רב נחמן בר יצחק גזרו על תינוק עובד כוכבים שיטמא בזיבה שלא יהא תינוק ישראל רגיל אצלו במשכב זכור
\par דא"ר זירא צער גדול היה לי אצל ר' אסי ור' אסי אצל ר' יוחנן ור' יוחנן אצל ר' ינאי ור' ינאי אצל רבי נתן בן עמרם ור"נ בן עמרם אצל רבי תינוק עובד כוכבים מאימתי מטמא בזיבה ואמר לי בן יומו וכשבאתי אצל ר' חייא אמר לי בן ט' שנים ויום אחד}
\twocol{וכשבאתי והרציתי דברי לפני רבי אמר לי הנח דברי ואחוז דברי רבי חייא דאמר תינוק עובד כוכבים אימתי מטמא בזיבה בן תשע שנים ויום אחד}
\newsection{דף לז}
\twocol{הואיל וראוי לביאה מטמא נמי בזיבה אמר רבינא הלכך הא תינוקת עובדת כוכבים בת ג' שנים ויום אחד הואיל וראויה לביאה מטמאה נמי בזיבה
\par פשיטא מהו דתימא האי ידע לארגולי והא לא ידעה לארגולי קמ"ל}
\twocol{מיסתמיך ואזיל ר' יהודה נשיאה אכתפיה דרבי שמלאי שמעיה א"ל שמלאי לא היית אמש בבית המדרש כשהתרנו את השמן אמר לו בימינו תתיר אף את הפת אמר לו א"כ קרו לן בית דינא שריא דתנן העיד רבי יוסי בן יועזר איש צרידה על אייל קמצא דכן ועל משקה בית מטבחיא דכן ועל דיקרב למיתא מסאב וקרו ליה יוסף שריא
\par אמר ליה התם שרא תלת ומר שרא חדא ואי שרי מר חדא אחריתי אכתי תרתין הוא דהויין א"ל אנא שראי אחריתי מאי היא}
\twocol{דתנן זה גיטך אם לא באתי מכאן עד שנים עשר חודש ומת בתוך שנים עשר חודש אינו גט ותני עלה ורבותינו התירוה לינשא ואמרינן מאן רבותינו אמר רב יהודה אמר שמואל בית דינא דשרו משחא
\par סברי לה כר' יוסי דאמר זמנו של שטר מוכיח עליו וא"ר אבא בריה דרבי חייא בר אבא ר' יהודה הנשיא הורה ולא הודו לו כל שעתו ואמרי לה כל סייעתו}
\twocol{א"ל רבי (אליעזר) [אלעזר] לההוא סבא כי שריתוה לאלתר שריתוה דלא אתי או דלמא לאחר שנים עשר חודש דהא איקיים ליה תנאיה
\par ותיבעי לך אמתני' דתנן הרי זה גיטך מעכשיו אם לא באתי מכאן עד שנים עשר חודש ומת בתוך שנים עשר חודש הוי גט דהא איקיים ליה תנאי}
\twocol{ותיבעי לך לאלתר הוי גיטא דהא לא אתא או דלמא לאחר י"ב חודש דהא איקיים ליה תנאיה אין ה"נ אלא משום דהוית בההוא מניינא
\par אמר אביי הכל מודים לכשתצא חמה מנרתיקה לכי נפקא קאמר לה וכי מיית בליליא גט לאחר מיתה הוא}
\twocol{על מנת שתצא חמה מנרתיקה מעכשיו קאמר לה וכי מיית בליליא הא ודאי תנאה הוי וגט מחיים הוא כדרב הונא דאמר רב הונא כל האומר על מנת כאומר מעכשיו דמי
\par לא נחלקו אלא באם תצא ר' יהודה הנשיא סבר לה כר' יוסי דאמר זמנו של שטר מוכיח עליו והוה ליה כמהיום אם מתי כמעכשיו אם מתי ורבנן לית להו דר' יוסי והוה ליה כזה גיטך אם מתי גרידא}
\twocol{גופא העיד יוסי בן יועזר איש צרידה על אייל קמצא דכן ועל משקה בי מטבחיא דכן ועל דיקרב למיתא מסאב וקרו ליה יוסף שריא מאי אייל קמצא רב פפא אמר שושיבא ורב חייא בר אמי משמיה דעולא אמר סוסביל
\par רב פפא אמר שושיבא וקמיפלגי בראשו ארוך מר סבר ראשו ארוך אסור ומר סבר ראשו ארוך מותר רב חייא בר אמי משמיה דעולא אמר}
\twocol{סוסביל בראשו ארוך כ"ע לא פליגי דאסור והכא בכנפיו חופין את רובו על ידי הדחק קמיפלגי מר סבר רובא כל דהו בעינן ומר סבר רובא דמנכר בעינן:
\par ועל משקה בי מטבחיא דכן: מאי דכן רב אמר דכן ממש ושמואל אמר דכן מלטמא אחרים אבל טומאת עצמן יש בהן}
\twocol{רב אמר דכן ממש קסבר טומאת משקין דרבנן וכי גזור רבנן טומאה במשקין דעלמא אבל במשקה בי מטבחיא לא גזרו רבנן
\par ושמואל אמר דכן מלטמא אחרים אבל טומאת עצמן יש בהן קסבר טומאת משקין דאורייתא לטמא אחרים דרבנן וכי גזרו רבנן במשקין דעלמא במשקין בי מטבחיא לא גזרו:}
\twocol{ועל דיקרב למיתא מסאב וקרו ליה יוסף שריא: יוסף אסרא מיבעי ליה ועוד דאורייתא היא דכתיב (במדבר יט, טז) וכל אשר יגע על פני השדה בחלל חרב או במת וגו'
\par דאורייתא דיקרב טמא דיקרב בדיקרב טהור ואתו אינהו וגזור אפילו דיקרב בדיקרב ואתא איהו ואוקמה אדאורייתא}
\twocol{דיקרב בדיקרב נמי דאורייתא הוא דכתיב (במדבר יט, כב) וכל אשר יגע בו הטמא יטמא
\par אמרוה רבנן קמיה דרבא משמיה דמר זוטרא בריה דרב נחמן דאמר משמיה דרב נחמן דאורייתא דיקרב בדיקרב בחיבורין טומאת שבעה שלא בחיבורין טומאת ערב ואתו אינהו וגזור אפילו שלא בחיבורין טומאת שבעה ואתא איהו ואוקמה אדאורייתא}
\twocol{דאורייתא מאי היא דכתיב (במדבר יט, יא) הנוגע במת לכל נפש אדם וטמא שבעת ימים וכתיב וכל אשר יגע בו הטמא יטמא וכתיב והנפש הנוגעת תטמא עד הערב הא כיצד
\par כאן בחיבורין כאן שלא בחיבורין}
\twocol{אמר להו רבא לאו אמינא לכו לא תתלו ביה בוקי סריקי ברב נחמן הכי אמר רב נחמן ספק טומאה ברשות הרבים התיר להן
\par והא הלכתא מסוטה גמרינן לה מה סוטה רשות היחיד אף טומאה רשות היחיד}
\twocol{הא א"ר יוחנן הלכה ואין מורין כן ואתא איהו ואורי ליה אורויי
\par תניא נמי הכי ר' יהודה אומר קורות נעץ להם ואמר עד כאן רשות הרבים עד כאן רשות היחיד כי אתו לקמיה דרבי ינאי אמר להו הא מיא בשיקעתא דבנהרא זילו טבולו:}
\twocol{והשלקות: מנהני מילי א"ר חייא בר אבא אמר רבי יוחנן אמר קרא (דברים ב, כח) אוכל בכסף תשבירני ואכלתי ומים בכסף תתן לי ושתיתי כמים מה מים שלא נשתנו אף אוכל שלא נשתנה
\par אלא מעתה חטין ועשאן קליות ה"נ דאסורין וכי תימא ה"נ והתניא חיטין ועשאן קליות מותרין אלא כמים מה מים שלא נשתנו מברייתן אף אוכל שלא נשתנה מברייתו}
\twocol{אלא מעתה חיטין וטחנן ה"נ דאסורין וכי תימא ה"נ והתניא חיטין ועשאן קליות הקמחים והסלתות שלהן מותרין אלא כמים מה מים שלא נשתנו מברייתן ע"י האור אף אוכל שלא נשתנה מברייתו ע"י האור
\par מידי אור כתיב}
\newsection{דף לח}
\twocol{אלא מדרבנן וקרא אסמכתא בעלמא
\par אמר רב שמואל בר רב יצחק אמר רב כל הנאכל כמות שהוא חי אין בו משום בישולי עובדי כוכבים בסורא מתנו הכי בפומבדיתא מתנו הכי אמר רב שמואל בר רב יצחק אמר רב כל שאינו נאכל על שולחן מלכים ללפת בו את הפת אין בו משום בישולי עובדי כוכבים}
\twocol{מאי בינייהו איכא בינייהו דגים קטנים וארדי ודייסא
\par אמר רב אסי אמר רב דגים קטנים מלוחים אין בהן משום בישולי עובדי כוכבים אמר רב יוסף אם צלאן עובד כוכבים סומך ישראל עליהם משום עירובי תבשילין ואי עבדינהו עובד כוכבים כסא דהרסנא אסור}
\twocol{פשיטא מהו דתימא הרסנא עיקר קמ"ל קימחא עיקר
\par אמר רב ברונא אמר רב עובד כוכבים שהצית את האור באגם כל החגבים שבאגם אסורין ה"ד אילימא דלא ידע הי טהור והי טמא מאי איריא עובד כוכבים אפילו ישראל נמי אלא משום בישולי עובדי כוכבים}
\twocol{כי האי גוונא מי אסיר והאמר רב חנן בר אמי א"ר פדת א"ר יוחנן האי עובד כוכבים דחריך רישא שרי למיכל מיניה אפילו מריש אוניה אלמא לעבורי שער קמיכוין ה"נ לגלויי אגמא קא מיכוין
\par לעולם דלא ידע הי טהור והי טמא ומעשה שהיה בעובד כוכבים היה}
\twocol{גופא אמר רבה בר בר חנה א"ר יוחנן האי עובד כוכבים דחריך רישא שרי למיכל מיניה אפילו מריש אוניה אמר רבינא הלכך האי עובד כוכבים דשדא סיכתא לאתונא וקבר בה ישראל קרא מעיקרא שפיר דמי פשיטא מהו דתימא לבשולי מנא קא מיכוין קמ"ל לשרורי מנא קא מיכוין
\par אמר רב יהודה אמר שמואל הניח ישראל בשר על גבי גחלים ובא עובד כוכבים והפך בו מותר היכי דמי אילימא דאי לא הפך ביה הוה בשיל פשיטא אלא לאו דאי לא הפך לא הוה בשיל אמאי מותר בישולי של עובדי כוכבים נינהו}
\twocol{לא צריכא דאי לא הפך הוה בשיל בתרתי שעי והשתא קא בשיל בחדא שעתא מהו דתימא קרובי בישולא מילתא היא קמ"ל
\par והאמר ר' אסי א"ר יוחנן כל שהוא כמאכל בן דרוסאי אין בו משום בישולי עובדי כוכבים הא אינו כמאכל בן דרוסאי יש בו משום בשולי עובדי כוכבים}
\twocol{התם כגון דאותביה בסילתא ושקליה עובד כוכבים ואותביה בתנורא
\par תניא נמי הכי מניח ישראל בשר על גבי גחלים ובא עובד כוכבים ומהפך בו עד שיבא ישראל מבית הכנסת או מבית המדרש ואינו חושש שופתת אשה קדירה על גבי כירה ובאת עובדת כוכבים}
\twocol{ומגיסה עד שתבא מבית המרחץ או מבית הכנסת ואינה חוששת
\par איבעיא להו הניח עובד כוכבים והפך ישראל מהו אמר רב נחמן בר יצחק ק"ו גמרו ביד עובד כוכבים מותר גמרו ביד ישראל לא כ"ש}
\twocol{איתמר נמי אמר רבה בר בר חנה א"ר יוחנן ואמרי לה אמר רב אחא בר בר חנה א"ר יוחנן בין שהניח עובד כוכבים והפך ישראל בין שהניח ישראל והפך עובד כוכבים מותר ואינו אסור עד שתהא תחלתו וגמרו ביד עובד כוכבים
\par אמר רבינא הלכתא הא ריפתא דשגר עובד כוכבים ואפה ישראל א"נ שגר ישראל ואפה עובד כוכבים א"נ שגר עובד כוכבים ואפה עובד כוכבים ואתא ישראל וחתה בה חתויי שפיר דמי}
\twocol{דג מליח חזקיה שרי ור' יוחנן אסר ביצה צלויה בר קפרא שרי ור' יוחנן אסר כי אתא רב דימי אמר אחד דג מליח ואחד ביצה צלויה חזקיה ובר קפרא שרו ורבי יוחנן אסר
\par ר' חייא פרוואה איקלע לבי ריש גלותא אמרו ליה ביצה צלויה מאי אמר להו חזקיה ובר קפרא שרו ור' יוחנן אסר ואין דבריו של אחד במקום שנים אמר להו רב זביד לא תציתו ליה הכי אמר אביי הלכתא כוותיה דרבי יוחנן אשקיוהו נגוטא דחלא ונח נפשיה}
\twocol{ת"ר הקפריסין והקפלוטות והמטליא והחמין והקליות שלהן מותרין ביצה צלויה אסורה שמן רבי יהודה הנשיא ובית דינו נמנו עליו והתירוהו
\par תניא היא המטליא היא פשליא היא שיעתא מאי שיעתא ארבב"ח אמר רבי יוחנן הא ארבעין שנין דנפיק האי עובדא ממצרים ורבה בר בר חנה דידיה אמר הא שתין שנין דנפיק האי עובדא ממצרים ולא פליגי מר בשניה ומר בשניה}
\twocol{מייתו ביזרא דכרפסא וביזרא דכיתנא וביזרא דשבלילתא ותרו להו בהדי הדדי בפשורי ושבקו ליה עד דמקבל ומייתי חצבי חדתי ומלו להו מיא ותרו בהו גרגישתא ומדבקין ביה ועיילין לבי בני אדנפקו מלבלבי ואכלי מינייהו וקיירי מבינתא דרישייהו עד טופרא דכרעייהו אמר רב אשי אמר לי רבי חנינא מילין ואמרי לה במילין
\par ת"ר הכוספן של עובדי כוכבים שהוחמו חמין ביורה גדולה אסור ביורה קטנה מותר ואיזו היא יורה קטנה א"ר ינאי כל שאין צפור דרור יכול ליכנס בתוכה}
\twocol{ודלמא אדמויי אדמוה ועיילוה אלא כל שאין ראש צפור דרור יכול ליכנס בתוכה
\par והתניא אחת יורה גדולה ואחת יורה קטנה מותר לא קשיא הא כמ"ד נותן טעם לפגם אסור הא כמ"ד נותן טעם לפגם מותר}
\twocol{אמר רב ששת האי מישחא שליקא דארמאי אסור אמר רב ספרא למאי ניחוש לה אי משום איערובי מיסרא סרי אי משום בישולי עובדי כוכבים נאכל הוא כמו שהוא חי אי משום גיעולי עובדי כוכבים נותן טעם לפגם הוא ומותר
\par בעו מיניה מרבי אסי הני אהיני שליקי דארמאי מאי חוליי לא תיבעי לך דודאי שרו מרירי לא תיבעי לך דודאי אסירי כי תיבעי לך מציעאי מאי אמר להו מאי תיבעי להו דרבי אסר ומנו לוי}
\twocol{שתיתאה רב שרי אבוה דשמואל ולוי אסרי בחיטי ושערי כ"ע לא פליגי דשרי בטלפחי דחלא כ"ע ל"פ דאסיר כי פליגי בטלפחי דמיא מר סבר גזרינן הא אטו הא ומר סבר לא גזרינן
\par ואיכא דאמרי בטלפחי דמיא כ"ע לא פליגי דאסיר כי פליגי בחיטי ושערי מר סבר גזרינן הא אטו הא ומר סבר לא גזרינן}
\twocol{אמר רב תרי מיני שתיתאה שדר ברזילי הגלעדי לדוד דכתיב (שמואל ב יז, כח) משכב וספות וכלי יוצר חטים ושעורים וקמח וקלי ופול ועדשים וקלי והשתא הוא דקא מפקי צני צני לשוקי דנהרדעא ולית דחייש להא דאבוה דשמואל ולוי:
\par וכבשין שדרכן לתת בתוכן יין: אמר חזקיה לא שנו אלא שדרכן אבל בידוע אסור אפילו בהנאה ומ"ש ממורייס דשרו רבנן בהנאה התם לעבורי זוהמא הכא למתוקי טעמא}
\twocol{ורבי יוחנן אמר אפילו בידוע נמי מותר ומאי שנא ממורייס לר"מ דאסיר בהנאה}
\newsection{דף לט}
\twocol{התם ידיע ממשו הכא לא ידיע ממשן:
\par וטרית טרופה וציר שאין בה דגה וכו': מאי חילק אמר רב נחמן בר אבא אמר רב זו סולתנית ומפני מה אסורה מפני שערבונה עולה עמה:}
\twocol{תנו רבנן אין לו עכשיו ועתיד לגדל לאחר זמן כגון הסולתנית והעפיץ הרי זה מותר יש לו עכשיו ועתיד להשיר בשעה שעולה מן הים כגון אקונס ואפונס כטספטייס ואכספטייס ואוטנס מותר
\par אכריז רבי אבהו בקיסרי קירבי דגים ועוברן ניקחין מכל אדם חזקתן אינן באים אלא מפלוסא ואספמיא כי הא דאמר אביי האי צחנתא דבב נהרא שריא}
\twocol{מ"ט אילימא משום דרדיפי מיא והאי דג טמא כיון דלית ליה חוט השדרה בדוכתא דרדיפי מיא לא מצי קאי והא קא חזינן דקאי
\par אלא משום דמליחי מיא והאי דג טמא כיון דלית ליה קלפי בדוכתא דמליחי מיא לא מצי קאי והא קחזינן דקאי אלא משום דלא מרבה טינא דג טמא אמר רבינא האידנא דקא שפכי ביה נהר גוזא ונהר גמדא אסירי}
\twocol{אמר אביי האי חמרא דימא שרי תורא דימא אסיר וסימניך טמא טהור טהור טמא
\par אמר רב אשי שפר נונא שרי קדש נונא אסיר וסימניך (שמות טז, כג) קדש לה' איכא דאמרי קבר נונא אסור וסימניך קברי עובדי כוכבים}
\twocol{רבי עקיבא איקלע לגינזק אייתו לקמיה ההוא נונא דהוה דמי לחיפושא חפייה בדיקולא חזא ביה קלפי ושרייה רב אשי איקלע לטמדוריא אייתו לקמיה ההוא נונא דהוה דמי לצלופחא נקטיה להדי יומא חזא דהוה ביה צימחי ושרייה
\par רב אשי איקלע לההוא אתרא אייתו לקמיה נונא דהוי דמי לשפרנונא חפייה במשיכלי חיורי חזא ביה קלפי ושרייה רבה בר בר חנה איקלע לאקרא דאגמא קריבו ליה צחנתא שמעיה לההוא גברא דהוה קרי ליה באטי}
\twocol{אמר מדקא קרי ליה באטי ש"מ דבר טמא אית ביה לא אכל מיניה לצפרא עיין בה אשכח ביה דבר טמא קרי אנפשיה (משלי יב, כא) לא יאונה לצדיק כל און:
\par והקורט של חילתית: מ"ט משום דמפסקי ליה בסכינא אע"ג דאמר מר נותן טעם לפגם מותר אגב חורפיה דחילתיתא מחליא ליה שמנוניתא והוה ליה כנותן טעם לשבח ואסור}
\twocol{עבדיה דר' לוי הוה קא מזבין חילתיתא כי נח נפשיה דר' לוי אתו לקמיה דרבי יוחנן אמרו ליה מהו למיזבן מיניה אמר להו עבדו של חבר הרי הוא כחבר
\par רב הונא בר מניומי זבן תכילתא מאנשי דביתיה דרב עמרם חסידא אתא לקמיה דרב יוסף לא הוה בידיה}
\twocol{פגע ביה חנן חייטא א"ל יוסף עניא מנא ליה בדידי הוה עובדא דזביני תכילתא מאנשי דביתיה דרבנאה אחוה דר' חייא בר אבא ואתאי לקמיה דרב מתנא לא הוה בידיה אתאי לקמיה דרב יהודה מהגרוניא אמר לי נפלת ליד הכי אמר שמואל אשת חבר הרי היא כחבר
\par תנינא להא דת"ר אשת חבר הרי היא כחבר עבדו של חבר הרי הוא כחבר חבר שמת אשתו ובניו ובני ביתו הרי הן בחזקתן עד שיחשדו וכן חצר שמוכרין בה תכלת הרי הן בחזקתן עד שתיפסל}
\twocol{ת"ר אשת עם הארץ שנשאת לחבר וכן בתו של עם הארץ שנשאת לחבר וכן עבדו של עם הארץ שנמכר לחבר כולן צריכין לקבל דברי חברות אבל אשת חבר שנשאת לעם הארץ וכן בתו של חבר שנשאת לעם הארץ וכן עבדו של חבר שנמכר לעם הארץ אינן צריכין לקבל דברי חברות לכתחלה דברי ר"מ
\par ר' יהודה אומר אף הן צריכין לקבל דברי חברות לכתחלה וכן היה ר"ש בן אלעזר אומר מעשה באשה אחת שנשאת לחבר והיתה קושרת לו תפילין על ידו נשאת למוכס והיתה קושרת לו קשרי מוכס על ידו}
\twocol{אמר רב חבי"ת אסור בחותם אחד חמפ"ג מותר בחותם אחד חלב בשר יין תכלת
\par אסורין בחותם אחד חילתית מורייס פת גבינה מותרין בחותם אחד}
\twocol{פת למאי ניחוש לה אי משום איחלופי קרירא בחמימא מידע ידיע דחיטי בדשערי נמי מידע ידיע אי כי הדדי כיון דאיכא חותם אחד לא טרח ומזייף
\par ורב מ"ש גבינה דלא טרח ומזייף חלב נמי לא טרח ומזייף אמר רב כהנא אפיק חלב ועייל חתיכת דג שאין בה סימן}
\twocol{היינו בשר תרי גווני בשר
\par ושמואל אומר בי"ת אסור בחותם אחד מח"ג מותר בחותם אחד בשר יין תכלת אסורין בחותם אחד מורייס חילתית גבינה מותרין בחותם אחד לשמואל חתיכת דג שאין בה סימן היינו בשר תרי גווני בשר לא אמרינן}
\twocol{ת"ר אין לוקחין ימ"ח מח"ג בסוריא לא יין ולא מורייס ולא חלב ולא מלח סלקונדרית ולא חילתית ולא גבינה אלא מן המומחה וכולן אם נתארח אצל בעל הבית מותר
\par מסייע ליה לרבי יהושע בן לוי דא"ר יהושע בן לוי שגר לו בעל הבית לביתו מותר מ"ט בעל הבית לא שביק היתירא ואכל איסורא וכי משגר ליה ממאי דאכיל משדר ליה:}
\twocol{ומלח סלקונדרית: מאי מלח סלקונדרית אמר רב יהודה אמר שמואל מלח שכל סלקונדרי רומי אוכלין אותה תנו רבנן מלח סלקונדרית שחורה אסורה לבנה מותרת דברי רבי מאיר רבי יהודה אומר לבנה אסורה שחורה מותרת רבי יהודה בן גמליאל משום רבי חנינא בן גמליאל אומר זו וזו אסורה
\par אמר רבה בר בר חנה אמר רבי יוחנן לדברי האומר לבנה אסורה קירבי דגים לבנים טמאים מעורבין בה לדברי האומר שחורה אסורה קירבי דגים שחורים טמאים מעורבין בה}
\twocol{לדברי האומר זו וזו אסורה זה וזה מעורבין בה אמר רבי אבהו משום רבי חנינא בן גמליאל זקן אחד היה בשכונתנו שהיה מחליק פניה בשומן חזיר:
\par הרי אלו אסורים: למעוטי מאי לחזקיה למעוטי בידוע לרבי יוחנן למעוטי מורייס וגבינת בית אונייקי וסתמא כר"מ:}
\twocol{{\large\emph{מתני׳}} ואלו מותרין באכילה חלב שחלבו עובד כוכבים וישראל רואהו והדבש והדבדבניות אע"פ שמנטפין אין בהן משום הכשר משקה וכבשין שאין דרכן לתת לתוכן יין וחומץ וטרית שאינה טרופה וציר שיש בה דגה ועלה של חילתית וזיתי גלוסקאות המגולגלין
\par ר' יוסי אומר השלחין אסורין החגבים הבאים מן הסלולה אסורין מן ההפתק מותרין וכן לתרומה:}
\twocol{{\large\emph{גמ׳}} תנינא להא דת"ר יושב ישראל בצד עדרו של עובד כוכבים ועובד כוכבים חולב לו ומביא לו ואינו חושש היכי דמי אי דליכא דבר טמא בעדרו פשיטא ואי דאיכא דבר טמא בעדרו אמאי
\par לעולם דאיכא דבר טמא וכי קאי חזי ליה וכי יתיב לא חזי ליה מהו דתימא כיון דיתיב לא חזי ליה ניחוש דלמא מייתי ומערב ביה קמ"ל כיון דכי קאי חזי ליה אירתותי מירתת ולא מיערב ביה:}
\twocol{והדבש: דבש למאי ניחוש לה אי משום איערובי מיסרא סרי אי משום בישולי עובדי כוכבים נאכל כמו שהוא חי אי משום גיעולי עובדי כוכבים נותן טעם לפגם הוא ומותר:
\par והדבדבניות אף על פי שמנטפות אין בהן משום הכשר משקה: ורמינהי הבוצר לגת שמאי אומר הוכשר הלל אומר לא הוכשר ואודי ליה הלל לשמאי}
\twocol{התם קא בעי ליה למשקה הכא לא קא בעי ליה למשקה:
\par וטרית שאינה טרופה: תנו רבנן איזו היא טרית שאינה טרופה כל שראש ושדרה ניכר ואיזו ציר שיש בה דגה כל שכילבית אחת או שתי כילביות}
\newsection{דף מ}
\twocol{שוטטות בו השתא כילבית אחת אמרת שרי שתי כילביות מבעיא לא קשיא כאן בפתוחות כאן בסתומות
\par איתמר רב הונא אמר עד שתהא ראש ושדרה ניכר רב נחמן אמר או ראש או שדרה מתיב רב עוקבא בר חמא ובדגים כל שיש לו סנפיר וקשקשת אמר אביי כי תניא ההיא בארא' ופלמודא דדמו רישייהו לטמאים}
\twocol{אמר רב יהודה משמיה דעולא מחלוקת לטבל בצירן אבל בגופן דברי הכל אסור עד שיהא ראש ושדרה ניכר אמר ר' זירא מריש הוה מטבילנא בצירן כיון דשמענא להא דאמר רב יהודה משמיה דעולא מחלוקת לטבל בצירן אבל בגופן דברי הכל אסור עד שיהא ראש ושדרה ניכר בצירן נמי לא מטבילנא
\par אמר רב פפא הלכתא עד שיהא ראש ושדרה ניכר של כל אחת ואחת מיתיבי חתיכות שיש בהן סימן בין בכולן בין במקצתן ואפילו באחד ממאה שבהן כולן מותרות ומעשה בעובד כוכבים אחד שהביא גרב של חתיכות ונמצא סימן באחת מהן והתיר רשב"ג את הגרב כולו}
\twocol{תרגמה רב פפא כשחתיכות שוות א"ה מאי למימרא מהו דתימא ניחוש דלמא אתרמי קמ"ל
\par ההוא ארבא דצחנתא דאתי לסיכרא נפק רב הונא בר חיננא וחזא ביה קלפי ושרייה א"ל רבא ומי איכא דשרי כה"ג באתרא דשכיחי קלפי נפק שיפורי דרבא ואסר שיפורי דרב הונא בר חיננא ושרי}
\twocol{א"ר ירמיה מדפתי לדידי אמר לי רב פפי כי שרא רב הונא בר חיננא בצירן אבל בגופן לא אמר רב אשי לדידי אמר לי רב פפא כי שרא רב הונא בר חיננא אפי' בגופן
\par ואנא לא מיסר אסרינא דקאמר לי רב פפא ולא מישרא שרינא דהא אמר (לי) רב יהודה משמיה דעולא מחלוקת לטבל בצירן אבל בגופן דברי הכל עד שיהא ראש ושדרה ניכר של כל אחד ואחד}
\twocol{יתיב רב חיננא בר אידי קמיה דרב אדא בר אהבה ויתיב וקאמר עובד כוכבים שהביא עריבה מלאה חביות ונמצאת כילבית באחת מהן פתוחות כולן מותרות סתומות היא מותרת וכולן אסורות א"ל מנא לך הא מתלתא קראי שמיע לי מרב ושמואל ורבי יוחנן
\par אמר רב ברונא אמר רב קירבי דגים ועוברן אין נקחין אלא מן המומחה רמי ליה עולא לרבי דוסתאי דמן בירי מדקאמר רב קירבי דגים ועוברן אין נקחין אלא מן המומחה מכלל דדג טמא אית ליה עובר ורמינהי דג טמא משריץ דג טהור מטיל ביצים}
\twocol{סמי מכאן עוברן א"ל רבי זירא לא תיסמי תרוייהו מטילי ביצים נינהו אלא זה משריץ מבחוץ וזה משריץ מבפנים
\par למה לי מומחה לבדוק בסימנין דתניא כסימני ביצים כך סימני דגים סימני דגים סלקא דעתך סימני דגים סנפיר וקשקשת כתיב בהו אלא כסימני ביצים כך סימני עוברי דגים}
\twocol{ואלו הן סימני ביצים כל שכודרת ועגולגלת ראשה אחד כד וראשה אחד חד טהורה ב' ראשיה חדין וב' ראשיה כדין טמאה חלמון מבחוץ וחלבון מבפנים טמאה חלבון מבחוץ וחלמון מבפנים טהור חלבון וחלמון מעורבין זה בזה זו היא ביצת השרץ אמר רבא כשנימוחו
\par ולרבי דוסתאי דמן בירי דאמר סמי מכאן עוברן}
\twocol{והתניא כסימני ביצים כך סימני עוברי דגים לאו תרוצי מתרצת לה כך סימני קירבי דגים
\par והיכי משכחת בסימני קירבי דגים שיהא כד וחד משכחת לה בשילפוחא}
\twocol{אם אין שם מומחה מאי אמר רב יהודה כיון דאמר אני מלחתים מותרין רב נחמן אמר עד שיאמר אלו דגים ואלו קירביהן אורי ליה רב יהודה לאדא דיילא כיון דאמר אני מלחתים מותרין:
\par ועלה של חילתית: פשיטא לא נצרכה אלא לקרטין שבו מהו דתימא ניחוש דלמא מייתי ומערב ביה קמ"ל דהא אישתרוקי היא דאישתרוק ואתא בהדה:}
\twocol{וזיתי גלוסקאות המגולגלין: פשיטא לא נצרכא אע"ג דרפי טובא מהו דתימא חמרא רמא בהו קמ"ל הני מחמת מישחא הוא דרפו:
\par ורבי יוסי אומר שלחין אסורין: היכי דמי שלחין א"ר יוסי בר חנינא כל שאוחזו בידו וגרעינתו נשמטת:}
\twocol{החגבין הבאין כו': ת"ר החגבין והקפריסין והקפלוטות הבאין מן האוצר ומן ההפתק ומן הספינה מותרין הנמכרין בקטלוזא לפני חנוני אסורין מפני שמזלף יין עליהן וכן יין תפוחים של עובדי כוכבים הבאין מן האוצר ומן ההפתק ומן הסלולה מותרין הנמכר בקטלוזא אסור מפני שמערבין בו יין
\par ת"ר פעם א' חש רבי במעיו אמר כלום יש אדם שיודע יין תפוחים של עובדי כוכבים אסור או מותר אמר לפניו ר' ישמעאל ב"ר יוסי פעם אחת חש אבא במעיו והביאו לו יין תפוחים של עובדי כוכבים של ע' שנה ושתה ונתרפא אמר לו כל כך היה בידך ואתה מצערני}
\twocol{בדקו ומצאו עובד כוכבים אחד שהיה לו שלש מאות גרבי יין של תפוחים של ע' שנה ושתה ונתרפא אמר ברוך המקום שמסר עולמו לשומרים:
\par וכן לתרומה: מאי וכן לתרומה אמר רב ששת וכן לכהן החשוד למכור תרומה לשם חולין לפניו הוא דאסור אבל הבא מן האוצר ומן ההפתק ומן הסלולה מותר אירתותי מירתת סבר שמעי ביה רבנן ומפסדו לי' מינאי:}
\twocol{\par \par {\large\emph{הדרן עלך אין מעמידין}}\par \par 
\par }
\twocol{מתני׳ {\large\emph{כל}} הצלמים אסורין מפני שהן נעבדין פעם אחת בשנה דברי רבי מאיר וחכמים אומרים אינו אסור אלא כל שיש בידו מקל או צפור או כדור רבן שמעון בן גמליאל אומר אף כל שיש בידו כל דבר:
\par {\large\emph{גמ׳}} אי דנעבדין פעם אחת בשנה מאי טעמא דרבנן א"ר יצחק בר יוסף א"ר יוחנן במקומו של ר"מ היו עובדין אותה פעם אחת בשנה ור"מ דחייש למיעוטא גזר שאר מקומות אטו אותו מקום ורבנן דלא חיישי למיעוטא לא גזרו שאר מקומות אטו אותו מקום}
\twocol{
\par אמר רב יהודה אמר שמואל באנדרטי של מלכים שנינו אמר רבה בר בר חנה אמר ר' יוחנן ובעומדין על פתח מדינה שנינו}
\newchap{פרק \hebrewnumeral{3}\quad כל הצלמים}
\newsection{דף מא}
\twocol{
\par אמר רבה מחלוקת בשל כפרים אבל בשל כרכים ד"ה מותרין מ"ט לנוי עבדי להו}
\twocol{ודכפרים מי איכא למ"ד לנוי קעבדי להו דכפרים ודאי למיפלחינהו עבדי להו
\par אלא אי אתמר הכי אתמר אמר רבה מחלוקת בשל כרכים אבל בשל כפרים ד"ה אסורים:}
\twocol{וחכ"א אינן אסורין וכו': מקל שרודה את עצמו תחת כל העולם כולו כמקל: צפור שתופש את עצמו תחת כל העולם כולו כצפור: כדור שתופש את עצמו תחת כל העולם כולו ככדור
\par תנא הוסיפו עליהן סייף עטרה וטבעת}
\twocol{סייף מעיקרא סבור לסטים בעלמא ולבסוף סבור שהורג את עצמו תחת כל העולם כולו
\par עטרה מעיקרא סבור גדיל כלילי בעלמא ולבסוף סבור כעטרה למלך טבעת מעיקרא סבור אישתיימא בעלמא ולבסוף סבור שחותם את עצמו תחת כל העולם כולו למיתה:}
\twocol{רבן שמעון בן גמליאל כו': תנא אפילו צרור אפי' קיסם
\par בעי רב אשי תפש בידו צואה מהו מי אמרינן כ"ע זילו באפיה כי צואה א"ד הוא מיהו דזיל באפי כ"ע כצואה תיקו:}
\twocol{{\large\emph{מתני׳}} המוצא שברי צלמים הרי אלו מותרין מצא תבנית יד או תבנית רגל הרי אלו אסורין מפני שכיוצא בהן נעבד:
\par {\large\emph{גמ׳}} אמר שמואל אפי' שברי עבודת כוכבים והאנן תנן שברי צלמים}
\twocol{ה"ה דאפי' שברי עבודת כוכבים והא דקתני שברי צלמים משום דקבעי למיתנא סיפא מצא תבנית יד תבנית רגל הרי אלו אסורין מפני שכיוצא בהן נעבד
\par תנן מצא תבנית יד תבנית רגל הרי אלו אסורין מפני שכיוצא בו נעבד אמאי}
\twocol{והא שברים נינהו תרגמה שמואל בעומדין על בסיסן
\par אתמר עבודת כוכבים שנשתברה מאיליה רבי יוחנן אמר אסורה רשב"ל אמר מותרת}
\twocol{רבי יוחנן אמר אסורה דהא לא בטלה רשב"ל אמר מותרת מסתמא בטולי מבטיל לה מימר אמר איהי נפשה לא אצלה לההוא גברא מצלה ליה
\par איתיביה ר' יוחנן לרשב"ל (שמואל א ה, ד) וראש דגון ושתי כפות ידיו כרותות וגו' וכתיב (שמואל א ה, ה) על כן לא ידרכו כהני דגון וגו'}
\twocol{אמר לו משם ראיה התם שמניחין את הדגון ועובדין את המפתן דאמרי הכי שבקיה איסריה לדגון ואתא איתיב ליה על המפתן
\par איתיביה המוצא שברי צלמים הרי אלו מותרין הא שברי עבודת כוכבים אסורין}
\twocol{לא תימא שברי עבודת כוכבים אסורין אלא אימא הא צלמים עצמן אסורין וסתמא כר' מאיר
\par ורבי יוחנן מדר"מ נשמע להו לרבנן לאו אמר ר' מאיר צלמים אסורין שברי צלמים מותרין לרבנן עבודת כוכבים נמי היא אסורה ושבריה מותרין}
\twocol{הכי השתא התם אימר עבדום אימר לא עבדום ואת"ל עבדום אימר בטלום עבודת כוכבים ודאי עבדוה מי יימר דבטלה הוי ספק וודאי ואין ספק מוציא מידי ודאי
\par ואין ספק מוציא מידי ודאי והתניא חבר שמת והניח מגורה מלאה פירות אפילו הן בני יומן הרי הן בחזקת מתוקנין}
\twocol{והא הכא דודאי טבילי ספק עשרינהו ספק לא עשרינהו וקאתי ספק ומוציא מידי ודאי
\par התם ודאי וודאי הוא דודאי עשרינהו כדרבי חנינא חוזאה דאמר רבי חנינא חוזאה חזקה על חבר שאינו מוציא דבר שאינו מתוקן מתחת ידו}
\twocol{ואבע"א מעיקרא לא טבילי ספק וספק הוא
\par אפשר דעבד כדר' אושעיא דאמר מערים אדם על תבואתו ומכניסה במוץ שלה כדי שתהא בהמתו אוכלת ופטורה מן המעשר}
\twocol{ואין ספק מוציא מידי ודאי והתניא אמר ר' יהודה מעשה בשפחתו}
\newsection{דף מב}
\twocol{של מציק אחד ברימון שהטילה נפל לבור ובא כהן והציץ לידע אם זכר אם נקבה ובא מעשה לפני חכמים וטיהרוהו מפני שחולדה וברדלס מצוין שם
\par והא הכא דודאי הטילה נפל ספק גררוהו ספק לא גררוהו וקאתי ספק ומוציא מידי ודאי}
\twocol{לא תימא הטילה נפל לבור אלא אימא הטילה כמין נפל לבור
\par והא לידע אם זכר אם נקבה הוא קתני}
\twocol{ה"ק לידע אם רוח הפילה אם נפל הטילה ואת"ל נפל הטילה לידע אם זכר אם נקבה
\par ואיבעית אימא כיון שחולדה וברדלס מצוין שם ודאי גררוהו}
\twocol{איתיביה מצא תבנית יד תבנית רגל הרי אלו אסורין מפני שכיוצא בהן נעבד אמאי הא שברים נינהו
\par הא תרגמה שמואל בעומדין על בסיסן}
\twocol{איתיביה עובד כוכבים מבטל עבודת כוכבים שלו ושל חברו וישראל אינו מבטל עבודת כוכבים של עובד כוכבים אמאי תיהוי כעבודת כוכבים שנשתברה מאליה
\par אמר אביי שפחסה וכי פחסה מאי הוי והא תנן פחסה אע"פ שלא חסרה בטלה}
\twocol{הני מילי דפחסה עובד כוכבים אבל פחסה ישראל לא בטלה
\par ורבא אמר לעולם כי פחסה ישראל נמי בטלה אלא גזרה דלמא מגבה לה והדר מבטיל לה והוי עבודת כוכבים ביד ישראל וכל עבודת כוכבים ביד ישראל אינה בטלה לעולם}
\twocol{איתיביה עכו"ם שהביא אבנים מן המרקוליס וחיפה בהן דרכים וטרטיאות מותרות וישראל שהביא אבנים מן המרקוליס וחיפה בהן דרכים וטרטיאות אסורות אמאי תיהוי כעבודת כוכבים שנשתברה מאליה
\par הכא נמי כדרבא}
\twocol{איתיביה עובד כוכבים ששיפה עבודת כוכבים לצרכו היא ושיפוייה מותרין לצרכה היא אסורה ושיפוייה מותרין וישראל ששיפה עבודת כוכבים בין לצרכו בין לצרכה היא ושיפוייה אסורין אמאי תיהוי כעבודת כוכבים שנשתברה מאליה
\par הכא נמי כדרבא}
\twocol{איתיביה רבי יוסי אומר שוחק וזורה לרוח או מטיל לים אמרו לו אף היא נעשה זבל וכתיב (דברים יג, יח) לא ידבק בידך מאומה מן החרם אמאי תיהוי כעבודת כוכבים שנשתברה מאליה
\par הכא נמי כדרבא}
\twocol{איתיביה רבי יוסי בן יסיאן אומר מצא צורת דרקון וראשו חתוך ספק עובד כוכבים חתכו ספק ישראל חתכו מותר ודאי ישראל חתכו אסור אמאי תיהוי כעבודת כוכבים שנשתברה מאליה
\par הכא נמי כדרבא}
\twocol{איתיביה רבי יוסי אומר אף לא ירקות בימות הגשמים מפני שהנבייה נושרת עליהן אמאי תיהוי כעבודת כוכבים שנשתברה מאליה
\par שאני התם דעיקר עבודת כוכבים קיימת}
\twocol{והא שיפויין דעיקר עבודת כוכבים קיימת וקתני לצרכה היא אסורה ושיפוייה מותרין
\par רב הונא בריה דרב יהושע אמר לפי שאין עבודת כוכבים בטלה דרך גדילתה}
\twocol{איתיביה ר' שמעון בן לקיש לר' יוחנן קן שבראש האילן של הקדש לא נהנין ולא מועלין בראשה של אשרה יתיז בקנה
\par קס"ד כגון ששברה ממנו עצים וקינתה בהן וקתני יתיז בקנה}
\twocol{הב"ע כגון דאייתי עצים מעלמא וקינתה בהן
\par דיקא נמי דקתני גבי הקדש לא נהנין ולא מועלין אי אמרת בשלמא דאייתי עצים מעלמא היינו דקתני גבי הקדש לא נהנין ולא מועלין לא נהנין מדרבנן ולא מועלין מדאורייתא דהא לא קדישי}
\twocol{אלא אי אמרת ששברה עצים ממנו וקינתה בהן אמאי לא מועלין הא קדישי
\par מידי איריא הכא בגידולין הבאין לאחר מכאן עסקינן וקא סבר אין מעילה בגידולין}
\twocol{ורבי אבהו א"ר יוחנן מאי יתיז יתיז באפרוחין
\par א"ל רבי יעקב לרבי ירמיה בר תחליפא אסברה לך באפרוחין כאן וכאן מותרין בביצים כאן וכאן אסורין אמר רב אשי ואפרוחין שצריכין לאמן כביצים דמו:}
\twocol{{\large\emph{מתני׳}} המוצא כלים ועליהם צורת חמה צורת לבנה צורת דרקון יוליכם לים המלח רבן שמעון בן גמליאל אומר שעל המכובדין אסורין שעל המבוזין מותרין:
\par {\large\emph{גמ׳}} למימרא דלהני הוא דפלחי להו למידי אחרינא לא ורמינהי השוחט לשום ימים לשום נהרות לשום מדבר לשום חמה לשום לבנה לשום כוכבים ומזלות לשום מיכאל שר הגדול לשום שילשול קטן הרי אלו זבחי מתים}
\twocol{אמר אביי מיפלח לכל דמשכחי פלחי מיצר ומפלחי הני תלתא דחשיבי ציירי להו ופלחי להו למידי אחרינא לנוי בעלמא עבדי להו
\par מנקיט רב ששת חומרי מתנייתא ותני כל המזלות מותרין חוץ ממזל חמה ולבנה וכל הפרצופין מותרין חוץ מפרצוף אדם וכל הצורות מותרות חוץ מצורת דרקון}
\twocol{אמר מר כל המזלות מותרין חוץ ממזל חמה ולבנה הכא במאי עסקינן אילימא בעושה אי בעושה כל המזלות מי שרי והכתיב (שמות כ, כג) לא תעשון אתי לא תעשון כדמות שמשי המשמשין לפני במרום
\par אלא פשיטא במוצא וכדתנן המוצא כלים ועליהם צורת חמה צורת לבנה צורת דרקון יוליכם לים המלח}
\twocol{אי במוצא אימא מציעתא כל הפרצופות מותרין חוץ מפרצוף אדם אי במוצא פרצוף אדם מי אסור והתנן המוצא כלים ועליהם צורת חמה צורת לבנה צורת דרקון יוליכם לים המלח צורת דרקון אין פרצוף אדם לא
\par אלא פשיטא בעושה וכדרב הונא בריה דרב יהושע}
\twocol{אי בעושה אימא סיפא כל הצורות מותרות חוץ מצורת דרקון ואי בעושה צורת דרקון מי אסיר והכתיב (שמות כ, כג) לא תעשון אתי אלהי כסף ואלהי זהב}
\newsection{דף מג}
\twocol{הני אין צורת דרקון לא
\par אלא פשיטא במוצא וכדתנן המוצא כלים ועליהם צורת חמה}
\twocol{רישא וסיפא במוצא ומציעתא בעושה
\par אמר אביי אין רישא וסיפא במוצא ומציעתא בעושה}
\twocol{רבא אמר כולה במוצא ומציעתא רבי יהודה היא דתניא רבי יהודה מוסיף אף דמות מניקה וסר אפיס מניקה על שם חוה שמניקה כל העולם כולו סר אפיס על שם יוסף שסר ומפיס את כל העולם כולו והוא דנקיט גריוא וקא כייל והיא דנקטא בן וקא מניקה:
\par תנו רבנן איזהו צורת דרקון פירש רשב"א כל שיש לו ציצין בין פרקיו מחוי רבי אסי בין פרקי צואר אמר ר' חמא ברבי חנינא הלכה כר"ש בן אלעזר}
\twocol{אמר רבה בר בר חנה אמר רבי יהושע בן לוי פעם אחת הייתי מהלך אחר ר' אלעזר הקפר בריבי בדרך ומצא שם טבעת ועליה צורת דרקון ומצא עובד כוכבים קטן ולא אמר לו כלום מצא עובד כוכבים גדול ואמר לו בטלה ולא בטלה סטרו ובטלה
\par ש"מ תלת ש"מ עובד כוכבים מבטל עבודת כוכבים שלו ושל חבירו וש"מ יודע בטיב של עבודת כוכבים ומשמשיה מבטל ושאינו יודע בטיב עבודת כוכבים ומשמשיה אינו מבטל וש"מ עובד כוכבים מבטל בעל כרחו}
\twocol{מגדף בה רבי חנינא ולית ליה לרבי אלעזר הקפר בריבי הא דתנן המציל מן הארי ומן הדוב ומן הנמר ומן הגייס ומן הנהר ומזוטו של ים ומשלוליתו של נהר והמוצא בסרטיא ופלטיא גדולה ובכל מקום שהרבים מצוין שם הרי אלו שלו מפני שהבעלים מתייאשין מהן
\par אמר אביי נהי דמינה מייאש מאיסורא מי מייאש מימר אמר אי עובד כוכבים משכח לה מפלח פלח לה אי ישראל משכח לה איידי דדמיה יקרין מזבין לה לעובד כוכבים ופלח לה:}
\twocol{תנן התם דמות צורות לבנות היה לו לר"ג בעלייתו בטבלא בכותל שבהן מראה את ההדיוטות ואומר להן כזה ראיתם או כזה ראיתם
\par ומי שרי והכתיב (שמות כ, כג) לא תעשון אתי לא תעשון כדמות שמשי המשמשים לפני}
\twocol{אמר אביי לא אסרה תורה אלא שמשין שאפשר לעשות כמותן
\par כדתניא לא יעשה אדם בית תבנית היכל אכסדרה תבנית אולם חצר תבנית עזרה שולחן תבנית שולחן מנורה תבנית מנורה אבל הוא עושה של ה' ושל ו' ושל ח' ושל ז' לא יעשה אפילו של שאר מיני מתכות}
\twocol{רבי יוסי בר יהודה אומר אף של עץ לא יעשה כדרך שעשו בית חשמונאי
\par אמרו לו משם ראיה שפודין של ברזל היו וחופין בבעץ העשירו עשאום של כסף חזרו והעשירו עשאום של זהב}
\twocol{ושמשין שאי אפשר לעשות כמותן מי שרי והתניא לא תעשון אתי לא תעשון כדמות שמשי המשמשים לפני במרום
\par אמר אביי}
\twocol{לא אסרה תורה אלא בדמות ד' פנים בהדי הדדי
\par אלא מעתה פרצוף אדם לחודיה תשתרי אלמה תניא כל הפרצופות מותרין חוץ מפרצוף אדם}
\twocol{אמר רב יהודה בריה דרב יהושע מפרקיה דרבי יהושע שמיע לי לא תעשון אתי לא תעשון אותי אבל שאר שמשין שרי
\par ושאר שמשין מי שרי והתניא (שמות כ, כג) לא תעשון אתי לא תעשון כדמות שמשי המשמשין לפני במרום כגון אופנים ושרפים וחיות הקדש ומלאכי השרת}
\twocol{אמר אביי לא אסרה תורה אלא שמשין שבמדור העליון
\par ושבמדור התחתון מי שרי והתניא אשר בשמים לרבות חמה ולבנה כוכבים ומזלות ממעל לרבות מלאכי השרת}
\twocol{כי תניא ההיא לעובדם
\par אי לעובדם אפילו שילשול קטן נמי אין הכי נמי ומסיפיה דקרא נפקא דתניא אשר בארץ לרבות ימים ונהרות הרים וגבעות מתחת לרבות שילשול קטן}
\twocol{ועשייה גרידתא מי שרי והתניא לא תעשון אתי לא תעשון כדמות שמשי המשמשין לפני במרום כגון חמה ולבנה כוכבים ומזלות
\par שאני ר"ג דאחרים עשו לו}
\twocol{והא רב יהודה דאחרים עשו לו וא"ל שמואל לרב יהודה שיננא סמי עיניה דדין
\par התם בחותמו בולט ומשום חשדא דתניא טבעת שחותמה בולט אסור להניחה ומותר לחתום בה חותמה שוקע מותר להניחה ואסור לחתום בה}
\twocol{ומי חיישינן לחשדא והא בי כנישתא דשף ויתיב בנהרדעא דאוקמי ביה אנדרטא והוו עיילי ביה אבוה דשמואל ולוי ומצלו בגויה ולא חיישי לחשדא רבים שאני
\par והא רבן גמליאל דיחיד הוה כיון דנשיא הוא שכיחי רבים גביה ואיבעית אימא דפרקים הואי}
\twocol{ואיבעית אימא להתלמד שאני דתניא (דברים יח, ט) לא תלמד לעשות אבל אתה למד להבין ולהורות:
\par רשב"ג אומר וכו': איזו הן מכובדין ואיזו הן מבוזין}
\twocol{אמר רב מכובדין למעלה מן המים מבוזין למטה מן המים ושמואל אמר אלו ואלו מבוזין הן אלא אלו הן מכובדין שעל השירין ועל הנזמים ועל הטבעות
\par תניא כוותיה דשמואל מכובדין שעל השירין ועל הנזמים ועל הטבעות מבוזין שעל היורות ועל הקומקמסין ועל מחמי חמים ושעל הסדינין ועל המטפחות:}
\twocol{{\large\emph{מתני׳}} רבי יוסי אומר שוחק וזורה לרוח או מטיל לים אמרו לו אף הוא נעשה זבל שנאמר (דברים יג, יח) לא ידבק בידך מאומה מן החרם:
\par {\large\emph{גמ׳}} תניא אמר להם רבי יוסי והלא כבר נאמר (דברים ט, כא) ואת חטאתכם}
\newsection{דף מד}
\twocol{אשר עשיתם את העגל לקחתי ואשרוף אותו באש ואכות אותו טחון היטב עד אשר דק לעפר ואשליך את עפרו אל הנחל היורד מן ההר
\par אמר לו משם ראיה הרי הוא אומר (שמות לב, כ) ויזר על פני המים וישק את בני ישראל לא נתכוין אלא לבודקן כסוטות}
\twocol{אמר להם רבי יוסי והלא כבר נאמר (דברי הימים ב טו, טז) וגם את מעכה אמו הסירה מגבירה אשר עשתה מפלצתה וגו' וידק וישרף בנחל קדרון אמר לו משם ראיה נחל קדרון אינו מגדל צמחין
\par ולא והתניא אלו ואלו מתערבין באמה ויוצאין לנחל קדרון ונמכרין לגננין לזבל ומועלין בהן מקומות מקומות יש בו יש מקום מגדל צמחין ויש מקום שאין מגדל צמחין}
\twocol{מאי מפלצתה אמר רב יהודה דהוה מפליא ליצנותא כדתני רב יוסף כמין זכרות עשתה לה והיתה נבעלת לו בכל יום
\par אמר להן רבי יוסי והלא כבר נאמר (מלכים ב יח, ד) וכתת נחש נחשת אשר עשה משה}
\twocol{אמרו לו משם ראיה הרי הוא אומר (במדבר כא, ח) ויאמר ה' אל משה עשה לך שרף לך משלך ואין אדם אוסר דבר שאינו שלו והתם בדין הוא דכתותי לא הוה צריך
\par אלא כיון דחזא דקא טעו ישראל בתריה עמד וכיתתו}
\twocol{אמר להם והלא כבר נאמר (שמואל ב ה, כא) ויעזבו שם את עצביהם וישאם דוד ואנשיו ומאי משמע דהאי וישאם דוד לישנא דזרויי הוא כדמתרגם רב יוסף (ישעיהו מא, טז) תזרם ורוח תשאם ומתרגמינן תזרינון ורוח תטלטלינון
\par אמרו לו משם ראיה הרי הוא אומר וישרפו באש ומדלא כתיב וישרפם וישאם ש"מ וישאם ממש}
\twocol{מכל מקום קשו קראי אהדדי
\par כדרב הונא דרב הונא רמי כתיב (דברי הימים א יד, יב) ויאמר דוד וישרפו באש וכתיב וישאם}
\twocol{לא קשיא כאן קודם שבא איתי הגיתי כאן לאחר שבא איתי הגיתי
\par דכתיב (שמואל ב יב, ל) ויקח את עטרת מלכם מעל ראשו ומשקלה ככר זהב ומי שרי איסורי הנאה נינהו אמר רב נחמן איתי הגיתי בא וביטלה}
\twocol{משקלה ככר זהב היכי מצי מנח לה אמר רב יהודה אמר רב ראויה לנוח על ראש דוד רבי יוסי ברבי חנינא אמר אבן שואבת היתה בה דהות דרא לה רבי אלעזר אמר אבן יקרה היתה בה ששוה ככר זהב:
\par (תהלים קיט, נו) זאת היתה לי כי פקודיך נצרתי מאי קאמר הכי קאמר בשכר שפקודיך נצרתי זאת היתה לי לעדות מאי עדותה א"ר יהושע בן לוי שהיה מניחה במקום תפילין והולמתו והא בעי אנוחי תפילין א"ר שמואל בר רב יצחק מקום יש בראש שראוי להניח בו שתי תפילין}
\twocol{(דברי הימים ב כג, יא) ויוציאו את בן המלך ויתנו עליו את הנזר ואת העדות נזר זו כלילא עדות א"ר יהודה אמר רב עדות הוא לבית דוד שכל הראוי למלכות הולמתו וכל שאינו ראוי למלכות אין הולמתו
\par (מלכים א א, ה) ואדניה בן חגית מתנשא לאמר אני אמלוך אמר רב יהודה אמר רב שמתנשא להולמו ולא הולמתו}
\twocol{(מלכים א א, ה) ויעש לו רכב ופרשים וחמשים איש רצים לפניו מאי רבותייהו תנא כולם נטולי טחול וחקוקי כפות הרגלים היו:
\par {\large\emph{מתני׳}} שאל פרוקלוס בן פלוספוס את ר"ג בעכו שהיה רוחץ במרחץ של אפרודיטי אמר ליה כתוב בתורתכם (דברים יג, יח) לא ידבק בידך מאומה מן החרם מפני מה אתה רוחץ במרחץ של אפרודיטי}
\twocol{אמר לו אין משיבין במרחץ וכשיצא אמר לו אני לא באתי בגבולה היא באה בגבולי אין אומרים נעשה מרחץ נוי לאפרודיטי אלא אומר נעשה אפרודיטי נוי למרחץ:
\par דבר אחר אם נותנים לך ממון הרבה אי אתה נכנס לעבודת כוכבים שלך ערום ובעל קרי ומשתין בפניה זו עומדת על פי הביב וכל העם משתינין לפניה לא נאמר אלא אלהיהם את שנוהג בו משום אלוה אסור את שאינו נוהג בו משום אלוה מותר:}
\twocol{{\large\emph{גמ׳}} והיכי עביד הכי והאמר רבה בר בר חנה אמר רבי יוחנן בכל מקום מותר להרהר חוץ מבית המרחץ ומבית הכסא
\par וכי תימא בלשון חול אמר ליה והאמר אביי דברים של חול מותר לאומרן בלשון קדש דברים של קדש אסור לאומרן בלשון חול}
\twocol{תנא כשיצא אמר לו אין משיבין במרחץ
\par א"ר חמא בר יוסף ברבי א"ר אושעיא תשובה גנובה השיבו ר"ג לאותו הגמון ואני אומר אינה גנובה}
\twocol{מה גנובתיה דקאמר לו זו עומדת על פי הביב וכל אדם משתין בפניה וכי משתין בפניה מאי הוי והאמר רבא פעור יוכיח שמפערין לפניו בכל יום ואינו בטל
\par ואני אומר אינה גנובה זו עבודתה בכך וזו אין עבודתה בכך}
\twocol{אמר אביי גנובתה מהכא דקאמר ליה אני לא באתי בגבולה והיא באה בגבולי וכי בא בגבולה מאי הוי והתנן עבודת כוכבים שיש לה מרחץ או גינה נהנין מהן שלא בטובה ואין נהנין מהן בטובה
\par ואני אומר אינה גנובה שלא בטובת רבן גמליאל כבטובת אחרים דמי}
\twocol{רב שימי בר חייא אמר גנובתה מהכא דקאמר לו זו עומדת על הביב וכל אדם משתינין בפניה וכי משתינין בפניה מאי הוי והתנן רק בפניה השתין בפניה גיררה וזרק בה את הצואה הרי זו אינה בטילה
\par ואני אומר אינה גנובה התם לפי שעתא הוא רתח עלה והדר מפייס לה הכא כל שעתא ושעתא בזלזולה קיימא}
\twocol{רבה בר עולא אמר גנובתה מהכא דקאמר ליה אין אומרין נעשה מרחץ נוי לאפרודיטי אלא נעשה אפרודיטי נוי למרחץ וכי אמר נעשה מרחץ לאפרודיטי נוי מאי הוי והתניא האומר בית זה לעבודת כוכבים כוס זה לעבודת כוכבים לא אמר כלום שאין הקדש לעבודת כוכבים
\par ואני אומר אינה גנובה נהי דאיתסורי לא מיתסרא נוי מיהא איכא:}
\newsection{דף מה}
\twocol{{\large\emph{מתני׳}} הנכרים העובדים את ההרים ואת הגבעות הן מותרין ומה שעליהן אסורין שנאמר (דברים ז, כה) לא תחמוד כסף וזהב עליהם
\par ר' יוסי הגלילי אומר (דברים יב, ב) אלהיהם על ההרים ולא ההרים אלהיהם אלהיהם על הגבעות ולא הגבעות אלהיהם}
\twocol{ומפני מה אשירה אסורה מפני שיש בה תפיסת ידי אדם וכל שיש בה תפיסת ידי אדם אסור
\par אר"ע אני אובין ואדון לפניך כל מקום שאתה מוצא הר גבוה וגבעה נשאה ועץ רענן דע שיש שם עבודת כוכבים:}
\twocol{{\large\emph{גמ׳}} ורבי יוסי הגלילי היינו תנא קמא אמר רמי בר חמא אמר ריש לקיש צפוי הר כהר איכא בינייהו תנא קמא סבר צפוי הר אינו כהר ומיתסר ור' יוסי הגלילי סבר צפוי הר הרי הוא כהר
\par רב ששת אמר דכולי עלמא צפוי הר אינו כהר}
\twocol{והכא באילן שנטעו ולבסוף עבדו קמיפלגי ת"ק סבר אילן שנטעו ולבסוף עבדו מותר ורבי יוסי הגלילי סבר אילן שנטעו ולבסוף עבדו אסור
\par ממאי מדקתני סיפא מפני מה אשירה אסורה מפני שיש בה תפיסת ידי אדם וכל שיש בו תפיסת ידי אדם אסור וכל שיש בו תפיסת אדם לאתויי מאי לאו לאתויי אילן שנטעו ולבסוף עבדו}
\twocol{ואף רבי יוסי בר' יהודה סבר אילן שנטעו ולבסוף עבדו אסור דתניא רבי יוסי בר' יהודה אומר מתוך שנאמר אלהיהם על ההרים ולא ההרים אלהיהם אלהיהם על הגבעות ולא גבעות אלהיהם שומע אני תחת כל עץ רענן אלהיהם ולא רענן אלהיהם
\par ת"ל (דברים יב, ג) ואשריהם תשרפון באש}
\twocol{אלא תחת כל עץ רענן ל"ל ההוא לכדר"ע הוא דאתא דאר"ע אני אובין ואדון לפניך כל מקום שאתה מוצא הר גבוה וגבעה נשאה ועץ רענן דע שיש שם עבודת כוכבים
\par ורבנן האי ואשריהם תשרפון באש מאי עבדי ליה מיבעי ליה לאילן שנטעו מתחילה לכך}
\twocol{ור' יוסי בר' יהודה נמי מיבעי ליה להכי ה"נ אלא אילן שנטעו ולבסוף עבדו מנא ליה נפקא ליה (דברים ז, ה) מואשריהם תגדעון איזהו עץ שגידועו אסור ועיקרו מותר הוי אומר אילן שנטעו ולבסוף עבדו
\par והא ואשריהם תשרפון באש קא נסיב לה תלמודא}
\twocol{אילו לא נאמר קאמר אילו לא נאמר תשרפון באש הייתי אומר אשריהם תגדעון באילן שנטעו מתחילה לכך השתא דכתיב ואשריהם תשרפון באש אייתר ליה ואשריהם תגדעון לאילן שנטעו ולבסוף עבדו
\par ורבנן האי ואשריהם תגדעון מאי עבדי ליה לכדר' יהושע בן לוי דא"ר יהושע בן לוי גידועי עבודת כוכבים קודמין לכיבוש ארץ ישראל כיבוש ארץ ישראל קודם לביעור עבודת כוכבים}
\twocol{דתני רב יוסף (דברים יב, ג) ונתצתם את מזבחותם והנח ושברתם את מצבותם והנח
\par והנח ס"ד שריפה בעי אמר רב הונא רדוף ואח"כ שרוף}
\twocol{ור' יוסי בר' יהודה האי סברא מנא ליה נפקא ליה (דברים יב, ב) מאבד תאבדון אבד ואח"כ תאבדון
\par ורבנן הא מיבעי ליה לעוקר עבודת כוכבים שצריך לשרש אחריה}
\twocol{ורבי יוסי בר' יהודה לשרש אחריה מנא ליה נפקא ליה (דברים יב, ג) מואבדתם את שמם מן המקום ההוא
\par ורבנן ההוא לכנות לה שם דתניא ר"א אומר מנין לעוקר עבודת כוכבים שצריך לשרש אחריה ת"ל ואבדתם את שמם}
\newsection{דף מו}
\twocol{אמר לו ר"ע והלא כבר נאמר (דברים יב, ב) אבד תאבדון אם כן מה ת"ל ואבדתם את שמם מן המקום ההוא לכנות לה שם
\par יכול לשבח לשבח ס"ד אלא יכול לא לשבח ולא לגנאי ת"ל (דברים ז, כו) שקץ תשקצנו ותעב תתעבנו כי חרם הוא}
\twocol{הא כיצד היו קורין אותה בית גליא קורין אותה בית כריא עין כל עין קוץ
\par תני תנא קמיה דרב ששת העכו"ם העובדים את ההרים ואת הגבעות הן מותרין ועובדיהן בסייף ואת הזרעים ואת הירקות הן אסורין ועובדיהן בסייף}
\twocol{א"ל דאמר לך מני רבי יוסי בר יהודה היא דאמר אילן שנטעו ולבסוף עבדו אסור
\par ולוקמה באילן שנטעו מתחלה לכך ורבנן לא ס"ד דקתני דומיא דהר מה הר שלא נטעו מתחלה לכך אף האי נמי שלא נטעו מתחלה לכך}
\twocol{איתמר אבני הר שנדלדלו בני רבי חייא ורבי יוחנן חד אמר אסורות וחד אמר מותרות מ"ט דמ"ד מותרות כהר מה הר שאין בו תפיסת ידי אדם ומותר אף הני שאין בהן תפיסת ידי אדם ומותרין
\par מה להר שכן מחובר בהמה תוכיח}
\twocol{מה לבהמה שכן בעלת חיים הר יוכיח
\par וחזר הדין לא ראי זה כראי זה ולא ראי זה כראי זה הצד השוה שבהן שאין בהן תפיסת ידי אדם ומותר אף כל שאין בהן תפיסת ידי אדם ומותר}
\twocol{מה להצד השוה שבהן שכן לא נשתנו מברייתן
\par אלא אתיא מבהמה בעלת מום ומהר}
\twocol{ואי נמי מבהמה תמה ומאילן יבש
\par ומאן דאסר להכי כתיב שקץ תשקצנו ותעב תתעבנו דאע"ג דאתיא מדינא להיתרא לא תתיא}
\twocol{תסתיים דבני ר' חייא דשרו דבעי חזקיה זקף ביצה להשתחוות לה מהו
\par קא סלקא דעתך להשתחוות לה והשתחוה לה וקא מיבעיא ליה האי זקיפתה אי הוי מעשה אי לא הוי מעשה אבל לא זקף לא מיתסרא ש"מ בני ר' חייא דשרו}
\twocol{לא לעולם אימא לך בני רבי חייא דאסרי דהשתחוה לה אע"ג דלא זקפה אסורה והכא במאי עסקינן כגון שזקף ביצה להשתחוות לה ולא השתחוה לה
\par ולמאן אי למאן דאמר עבודת כוכבים של ישראל אסורה מיד אסורה אי למאן דאמר עד שתיעבד הא לא פלחה}
\twocol{לא צריכא כגון שזקף ביצה להשתחוות לה ולא השתחוה לה ובא עובד כוכבים והשתחוה לה
\par כי הא דאמר רב יהודה אמר שמואל ישראל שזקף לבינה להשתחוות לה ובא עובד כוכבים והשתחוה לה אסורה וקא מיבעיא ליה לבינה הוא דמינכרא זקיפתה אבל ביצה לא או דלמא לא שנא תיקו:}
\twocol{בעי רמי בר חמא המשתחוה להר אבניו מהו למזבח
\par יש נעבד במחובר אצל גבוה או אין נעבד במחובר אצל גבוה}
\twocol{את"ל יש נעבד במחובר אצל גבוה מכשירי קרבן כקרבן דמו או לא
\par אמר רבא ק"ו ומה אתנן שמותר בתלוש להדיוט אסור במחובר לגבוה דכתיב (דברים כג, יט) לא תביא אתנן זונה ומחיר כלב לא שנא תלוש ולא שנא במחובר נעבד שאסור בתלוש להדיוט אינו דין שאסור במחובר לגבוה}
\twocol{א"ל רב הונא בריה דרב יהושע לרבא או חילוף ומה נעבד שאסור בתלוש אצל הדיוט מותר במחובר לגבוה שנא' (דברים יב, ב) אלהיהם על ההרים ולא ההרים אלהיהם לא שנא להדיוט ולא שנא לגבוה אתנן שמותר בתלוש להדיוט אינו דין שמותר במחובר לגבוה
\par ואי משום בית ה' אלהיך מיבעי ליה לכדתניא בית ה' אלהיך פרט לפרה שאינה באה לבית דברי ר"א וחכ"א לרבות את הריקועים}
\twocol{א"ל אנא קאמינא לחומרא ואת אמרת לקולא קולא וחומרא לחומרא פרכינן
\par א"ל רב פפא לרבא וכל היכא דאיכא קולא וחומרא לקולא לא פרכינן והא הזאה דפסח דפליגי ר' אליעזר ור"ע דר' אליעזר סבר לחומרא וקא מחייב ליה לגברא ור"ע לקולא ופטר וקא פריך ר"ע לקולא}
\twocol{דתנן השיב ר"ע או חילוף ומה הזאה שהיא משום שבות אינה דוחה השבת שחיטה שהיא דאורייתא לא כ"ש
\par התם ר' אליעזר גמריה ואייקר ליה תלמודא ואתא ר"ע לאדכוריה והיינו דא"ל רבי אל תכפירני בשעת הדין כך מקובל אני ממך הזאה שבות ואינה דוחה את השבת}
\twocol{בעי רמי בר חמא המשתחוה לקמת חטים מהו למנחות יש שינוי בנעבד או אין שינוי בנעבד
\par אמר מר זוטרא בריה דרב נחמן ת"ש כל האסורין לגבי מזבח ולדותיהן מותרים ותני עלה רבי אליעזר אוסר}
\twocol{ולאו אתמר עלה אמר רב נחמן אמר רבה בר אבוה מחלוקת כשנרבעו ולבסוף עיברו}
\newsection{דף מז}
\twocol{אבל עיברו ולבסוף נרבעו ד"ה אסורין וה"נ כעיברו ולבסוף נרבעו דמי
\par איכא דאמרי מחלוקת כשנרבעו ולבסוף עיברו אבל עיברו ולבסוף נרבעו ד"ה אסור והני נמי כי עיברו ולבסוף נרבעו דמי}
\twocol{הכי השתא התם מעיקרא בהמה והשתא בהמה דשא הוא דאחיזא באנפה הכא מעיקרא חיטי והשתא קמחא
\par בעי ר"ל המשתחוה לדקל לולבו מהו למצוה}
\twocol{באילן שנטעו מתחלה לכך לא תיבעי לך דאפילו להדיוט נמי אסור כי תיבעי לך באילן שנטעו ולבסוף עבדו
\par ואליבא דרבי יוסי בר יהודה לא תיבעי לך דאפי' להדיוט נמי אסור כי תיבעי לך אליבא דרבנן לענין מצוה מאי מי מאיס כלפי גבוה או לא}
\twocol{כי אתא רב דימי אמר באשירה שביטלה קמבעיא ליה יש דחוי אצל מצות או אין דחוי אצל מצות
\par תפשוט ליה מדתנן כיסהו ונתגלה פטור מלכסות כיסהו הרוח חייב לכסות ואמר רבה בר בר חנה א"ר יוחנן לא שנו אלא שחזר ונתגלה אבל לא חזר ונתגלה פטור מלכסות}
\twocol{והוינן בה כי חזר ונתגלה מאי הוי הואיל ואידחי אידחי
\par וא"ר פפא זאת אומרת אין דיחוי אצל מצות}
\twocol{דרב פפא גופיה איבעיא ליה מפשט פשיטא ליה לרב פפא דאין דיחוי אצל מצות לא שנא לקולא ולא שנא לחומרא
\par או דלמא ספוקי מספקא ליה ולחומרא אמרינן לקולא לא אמרינן תיקו}
\twocol{בעי רב פפא המשתחוה לבהמה צמרה מהו לתכלת
\par תכלת דמאי אי תכלת לכהנים היינו בעיא דרמי בר חמא ואי תכלת לציצית היינו בעיא דר"ל}
\twocol{אין ה"נ דלא הוה למיבעי ליה והאי דקא בעי ליה הא משום דאיכא מילי אחרנייתא צמרה מהו לתכלת קרניה מהו לחצוצרות שוקיה מהו לחלילין בני מעיה מהו לפארות
\par אליבא דמ"ד עיקר שירה בכלי לא תיבעי לך דודאי אסיר}
\twocol{כי תיבעי לך אליבא דמ"ד עיקר שירה בפה בסומי קלא בעלמא הוא ומייתינן או דלמא אפילו הכי אסיר תיקו
\par בעי רבה המשתחוה למעין מימיו מהו לנסכים מאי קא מיבעיא ליה אילימא לבבואה קא סגיד או דלמא למיא קא סגיד ותיבעי ליה ספל להדיוט}
\twocol{לעולם למיא קא סגיד והכי קמבעיא ליה למיא דקמיה קא סגיד וקמאי קמאי אזדו או דלמא לדברונא דמיא קא סגיד
\par ומי מיתסרי והא א"ר יוחנן משום ר"ש בן יהוצדק מים של רבים אין נאסרין לא צריכא דקא נבעי מארעא:}
\twocol{{\large\emph{מתני׳}} מי שהיה ביתו סמוך לעבודת כוכבים ונפל אסור לבנותו כיצד יעשה כונס בתוך שלו ארבע אמות ובונה
\par שלו ושל עבודת כוכבים}
\twocol{נידון מחצה על מחצה
\par אבניו עציו ועפרו מטמאין כשרץ שנאמר (דברים ז, כו) שקץ תשקצנו}
\twocol{ר"ע אומר כנדה שנאמר (ישעיהו ל, כב) תזרם כמו דוה צא תאמר לו מה נדה מטמאה במשא אף עבודת כוכבים מטמאה במשא:
\par {\large\emph{גמ׳}} והא קא מרווח לעבודת כוכבים א"ר חנינא מסורא דעבד ליה בית הכסא}
\twocol{והא בעי צניעותא דעבד ליה בית הכסא דלילה
\par והא אמר מר איזהו צנוע הנפנה בלילה במקום שנפנה ביום ואע"ג דאוקימנא בכדרך מיהו צניעותא בעי למעבד}
\twocol{דעבד ליה לתינוקות
\par א"נ דגדיר ליה בהיזמי והינגי:}
\twocol{{\large\emph{מתני׳}} שלשה בתים הן בית שבנאו מתחלה לעבודת כוכבים הרי זה אסור סיידו וכיידו לעבודת כוכבים וחידש נוטל מה שחידש הכניס לתוכה עבודת כוכבים והוציאה הרי זה מותר:
\par {\large\emph{גמ׳}} אמר רב המשתחוה לבית אסרו אלמא קסבר תלוש ולבסוף חברו כתלוש דמי והאנן בנאו תנן}
\twocol{בנאו אע"פ שלא השתחוה לו השתחוה אע"פ שלא בנאו א"ה הני שלשה ארבעה הוו
\par כיון דלענין ביטול בנה והשתחוה חד קא חשיב ליה:}
\twocol{{\large\emph{מתני׳}} שלש אבנים הן אבן שחצבה מתחלה לבימוס הרי זו אסורה סיידה וכיידה לשם עבודת כוכבים נוטל מה שסייד וכייד ומותרת העמיד עליה עבודת כוכבים וסילקה הרי זו מותרת:
\par {\large\emph{גמ׳}} א"ר אמי והוא שסייד וכייד בגופה של אבן}
\twocol{והא דומיא דבית תנן ובית לאו בגופיה הוא ומיתסר בית נמי איכא ביני אורבי
\par מי לא עסקינן דשייע והדר שייעיה}
\twocol{אלא כי אתמר דרבי אמי לענין ביטול אתמר ואע"ג דסייד וכייד בגופה של אבן כי נטל מה שחידש שפיר דמי
\par דמהו דתימא כיון שסייד וכייד בגופה של אבן כאבן שחצבה מתחלה לעבודת כוכבים דמיא ותיתסר כולה קמ"ל:}
\newsection{דף מח}
\twocol{{\large\emph{מתני׳}} שלש אשרות הן אילן שנטעו מתחלה לשם עבודת כוכבים הרי זו אסורה גידעו ופיסלו לשם עבודת כוכבים והחליף נוטל מה שהחליף העמיד תחתיה עבודת כוכבים ונטלה הרי זו מותרת:
\par {\large\emph{גמ׳}} אמרי דבי ר' ינאי והוא שהבריך והרכיב בגופו של אילן}
\twocol{והאנן גידעו ופיסלו תנן
\par אלא כי איתמר דרבי ינאי לענין ביטול איתמר דאף על גב דהבריך והרכיב בגופו של אילן כי נטל מה שהחליף שפיר דמי דמהו דתימא כיון דהבריך והרכיב בגופו של אילן כאילן שנטעו מתחלה דמי וליתסר כולה קמ"ל}
\twocol{אמר שמואל המשתחוה לאילן תוספתיה אסורה מתיב רבי אלעזר גידעו ופיסלו לעבודת כוכבים והחליף נוטל מה שהחליף גידעו ופיסלו אין לא גידעו ופיסלו לא
\par אמר לך שמואל הא מני רבנן היא ושמואל דאמר כרבי יוסי בר יהודה דאמר אילן שנטעו ולבסוף עבדו אסור}
\twocol{מתקיף לה רב אשי ממאי דרבי יוסי בר יהודה ורבנן בתוספת פליגי דלמא תוספת לדברי הכל אסור ובעיקרו פליגי
\par דרבי יוסי בר יהודה סבר עיקרו נמי אסור דכתיב (דברים יב, ג) ואשריהם תשרפון באש ורבנן סברי עיקר אילן שרי דכתיב (דברים ז, ה) ואשירהם תגדעון איזהו אילן שגידועו אסור ועיקרו שרי הוי אומר אילן שנטעו ולבסוף עבדו}
\twocol{וכי תימא הא דלא מתרצינן הכי איפוך רבנן לדרבי יוסי בר יהודה ודרבי יוסי בר יהודה לרבנן
\par אם כן גידעו ופיסלו מאן קתני לה לא רבנן ולא רבי יוסי בר יהודה אי רבנן בלא גידעו ופיסלו נמי תוספת אסורה אי רבי יוסי בר יהודה עיקר אילן נמי אסור}
\twocol{אי בעית אימא רבנן ואי בעית אימא רבי יוסי בר יהודה אי בעית אימא רבי יוסי בר יהודה כי קאמר רבי יוסי בר יהודה בלא גידעו ופיסלו עיקר אילן אסור בסתמא אבל גידעו ופיסלו גלי אדעתיה דבתוספת ניחא ליה בעיקר אילן לא ניחא ליה
\par איבעית אימא רבנן גידעו ופיסלו איצטריכא ליה ס"ד אמינא כיון דעבד ליה מעשה בגופיה עיקר אילן נמי ליתסר קמשמע לן:}
\twocol{{\large\emph{מתני׳}} איזו אשרה כל שיש תחתיה עבודת כוכבים ר"ש אומר כל שעובדין אותה ומעשה בצידן באילן שהיו עובדין אותו ומצאו תחתיו גל אמר להן ר"ש בדקו את הגל הזה ובדקוהו ומצאו בו צורה אמר להן הואיל ולצורה הן עובדין נתיר להן את האילן:
\par {\large\emph{גמ׳}} איזהו אשרה והא אנן שלש אשרות תנן ה"ק שתים לדברי הכל ואחת מחלוקת דר"ש ורבנן איזו היא אשרה שנחלקו בה ר"ש וחכמים כל שיש תחתיה עבודת כוכבים ר"ש אומר כל שעובדים אותה}
\twocol{איזו היא אשרה סתם אמר רב כל שכומרים יושבין תחתיה ואין טועמין מפירותיה ושמואל אמר אפילו אמרי הני תמרי לבי נצרפי אסור דרמי בי שיכרא ושתי ליה ביום אידם אמר אמימר אמרו לי סבי דפומבדיתא הלכתא כשמואל:
\par {\large\emph{מתני׳}} לא ישב בצילה ואם ישב טהור ולא יעבור תחתיה ואם עבר טמא היתה גוזלת את הרבים ועבר תחתיה טהור:}
\twocol{{\large\emph{גמ׳}} לא ישב בצילה פשיטא אמר רבה בר בר חנה א"ר יוחנן לא נצרכא אלא לצל צילה
\par מכלל דבצל קומתה אם ישב טמא לא דאפי' לצל קומתה נמי אם ישב טהור והא קמ"ל דאפילו לצל צילה לא ישב}
\twocol{איכא דמתני לה אסיפא ואם ישב טהור פשיטא אמר רבה בר בר חנה א"ר יוחנן לא נצרכא אלא לצל קומתה מכלל דלצל צילה אפילו לכתחלה ישב לא הא קמ"ל דאפילו לצל קומתה אם ישב טהור:
\par ולא יעבור תחתיה ואם עבר טמא: מ"ט אי אפשר דליכא תקרובת עבודת כוכבים}
\twocol{מני רבי יהודה בן בתירא היא דתניא רבי יהודה בן בתירא אומר מנין לתקרובת עבודת כוכבים שמטמאה באהל שנאמר (תהלים קו, כח) ויצמדו לבעל פעור ויאכלו זבחי מתים מה מת מטמא באהל אף תקרובת עבודת כוכבים מטמא באהל:
\par היתה גוזלת את הרבים ועבר תחתיה טהור: איבעיא להו עבר או עובר רבי יצחק בן אלעזר משמיה דחזקיה אמר עובר ורבי יוחנן אמר אם עבר}
\twocol{ולא פליגי הא דאיכא דירכא אחרינא הא דליכא דירכא אחרינא
\par א"ל רב ששת לשמעיה כי מטית להתם ארהיטני היכי דמי אי דליכא דירכא אחרינא ל"ל ארהיטני מישרא שרי ואי דאיכא דירכא אחרינא כי אמר ארהיטני מי שרי}
\twocol{לעולם דליכא דירכא אחרינא ואדם חשוב שאני:
\par {\large\emph{מתני׳}} זורעין תחתיה ירקות בימות הגשמים אבל לא בימות החמה והחזירין לא בימות החמ' ולא בימות הגשמים ר' יוסי אומר אף לא ירקות בימות הגשמים מפני שהנביה נושרת עליהן והוה להן לזבל:}
\twocol{{\large\emph{גמ׳}} למימרא דרבי יוסי סבר זה וזה גורם אסור ורבנן אמרי זה וזה גורם מותר
\par הא איפכא שמעינן להו דתנן רבי יוסי אומר שוחק וזורה לרוח או מטיל לים אמרו לו אף היא נעשה זבל ונאמר (דברים יג, יח) לא ידבק בידך מאומה מן החרם}
\twocol{קשיא דרבנן אדרבנן קשיא דר' יוסי אדרבי יוסי
\par בשלמא דרבי יוסי אדרבי יוסי לא קשיא התם דקאזיל לאיבוד מתיר הכא דלא קאזיל לאיבוד אסור}
\twocol{אלא דרבנן אדרבנן קשיא איפוך
\par ואיבעית אימא לא תיפוך דר' יוסי כדשנין דרבנן כדאמר רב מרי בריה דרב כהנא מה שמשביח בעור פוגם בבשר}
\twocol{הכא נמי מה שמשביח בנביה פוגם בצל
\par וסבר רבי יוסי זה וזה גורם אסור והתניא רבי יוסי אומר נוטעין יחור של ערלה ואין נוטעין אגוז של ערלה מפני שהוא פרי ואמר רב יהודה אמר רב מודה רבי יוסי שאם נטע והבריך והרכיב מותר}
\twocol{ותניא נמי הכי מודה רבי יוסי}
\newsection{דף מט}
\twocol{שאם נטע והבריך והרכיב מותר
\par וכי תימא שני ליה לר' יוסי בין שאר איסורין לעבודת כוכבים ומי שני ליה והתניא שדה שנזדבלה בזבל עבודת כוכבים וכן פרה שנתפטמה בכרשיני עבודת כוכבים תני חדא שדה תזרע פרה תשחט ותניא אידך שדה תבור ופרה תרזה}
\twocol{מאי לאו הא ר' יוסי והא רבנן לא הא ר"א והא רבנן
\par הי ר"א ורבנן אילימא ר"א ורבנן דשאור}
\twocol{דתנן שאור של חולין ושל תרומה שנפלו לתוך העיסה לא בזה כדי לחמץ ולא בזה כדי לחמץ ונצטרפו וחימצו
\par ר"א אומר אחר האחרון אני בא וחכ"א בין שנפל איסור לכתחלה ובין שנפל איסור לבסוף אינו אסור עד שיהא בו כדי לחמץ}
\twocol{ואמר אביי לא שנו אלא שקדם וסילק את האיסור אבל לא קדם וסילק את האיסור אסור
\par וממאי דטעמא דר"א כדאביי דלמא טעמא דר"א משום דאחר אחרון אני בא אי גמיר באיסורא אסורה ואי גמיר בהיתירא מותרין בין סלקיה ובין לא סלקיה}
\twocol{אלא ר"א ורבנן דעצים
\par דתנן נטל הימנה עצים אסורה בהנאה הסיק בה את התנור חדש יותץ ישן יוצן אפה בו את הפת אסורה בהנאה}
\twocol{נתערבה באחרים כולן אסורות בהנאה ר"א אומר יוליך הנאה לים המלח אמרו לו אין פדיון לעבודת כוכבים
\par רבנן דפליגי עליה דר"א מאן נינהו}
\twocol{אילימא רבנן דעצים אחמורי מחמרי
\par אלא רבנן דשאור אימר דשמעת להו לרבנן דמקילי בשאור בעבודת כוכבים מי מקילי}
\twocol{אלא לעולם הא רבי יוסי והא רבנן
\par ר' יוסי לדבריהם דרבנן אמר להו לדידי זה וזה גורם מותר}
\twocol{לדידכו דאמריתו זה וזה גורם אסור אודו לי מיהת אף ירקות בימות הגשמים
\par ורבנן כדאמר רב מרי בריה דרב כהנא}
\twocol{אמר רב יהודה אמר שמואל הלכה כרבי יוסי: ההוא גינתא דאיזדבל בזבלא דעבודת כוכבים שלח רב עמרם קמיה דרב יוסף א"ל הכי אמר רב יהודה אמר שמואל הלכה כרבי יוסי:
\par {\large\emph{מתני׳}} נטל ממנה עצים אסורין בהנאה הסיק בהן את התנור אם חדש יותץ ואם ישן יוצן אפה בו את הפת אסורה בהנאה}
\twocol{נתערבה באחרות כולן אסורות בהנאה ר"א אומר יוליך הנאה לים המלח אמרו לו אין פדיון לעבודת כוכבים:
\par נטל הימנה כרכור אסור בהנאה ארג בו את הבגד אסור בהנאה נתערב באחרים ואחרים באחרים כולן אסורין בהנאה ר"א אומר יוליך הנאה לים המלח אמרו לו אין פדיון לעבודת כוכבים:}
\twocol{{\large\emph{גמ׳}} וצריכא דאי אשמעינן קמייתא בהא קאמר ר' אליעזר משום דבעידנא דקא גמרה פת קלי לה איסורא אבל כרכור דאיתיה לאיסורא בעיניה אימא מודי לרבנן
\par ואי אשמעינן כרכור בהא קאמרי רבנן אבל פת אימא מודו ליה לר"א צריכא}
\twocol{א"ר חייא בריה דרבה בר נחמני אמר רב חסדא אמר זעירי הלכה כר"א איכא דאמרי אמר רב חסדא אמר לי אבא בר רב חסדא הכי אמר זעירי הלכה כר"א
\par אמר רב אדא בר אהבה לא שנו אלא פת אבל חבית לא ורב חסדא אמר אפילו חבית מותרת}
\twocol{ההוא גברא דאיתערב ליה חביתא דיין נסך בחמריה אתא לקמיה דרב חסדא אמר שקול ארבע זוזי ושדי בנהרא ונשתרי לך:
\par {\large\emph{מתני׳}} כיצד מבטלה קירסם וזירד נטל ממנה מקל או שרביט אפי' עלה ה"ז בטלה שיפה לצרכה אסורה שלא לצרכה מותרת:}
\twocol{{\large\emph{גמ׳}} אותן שפאין מה תהא עליהן פליגי בה רב הונא (ור' חייא) בר רב חד אמר אסורין וחד אמר מותרין
\par תניא כמ"ד מותרין דתניא עובד כוכבים ששיפה עבודת כוכבים לצרכו היא ושפאיה מותרין לצרכה היא אסורה ושפאיה מותרין וישראל ששיפה עבודת כוכבים בין לצרכה בין לצרכו היא ושפאיה אסורין}
\twocol{איתמר עבודת כוכבים שנשתברה רב אמר צריך לבטל כל קיסם וקיסם ושמואל אמר עבודת כוכבים אינה בטלה אלא דרך גדילתה
\par אדרבה דרך גדילתה מי מבטלא אלא ה"ק אין עבודת כוכבים צריכה לבטל אלא דרך גדילתה}
\twocol{לימא בהא קמיפלגי דמ"ס עובדין לשברין ומ"ס אין עובדין לשברין
\par לא דכ"ע עובדין לשברין והכא בשברי שברים קמיפלגי מ"ס שברי שברים אסורין ומ"ס שברי שברים מותרין}
\twocol{ואיבע"א דכ"ע שברי שברים מותרין והכא בעבודת כוכבי' של חליות ובהדיוט שיכול להחזירה קמיפלגי מ"ס כיון דהדיוט יכול להחזירה לא בטלה ומ"ס אין עבודת כוכבים בטלה אלא דרך גדילתה דהיינו אורחיה הא לאו גדילתה היא ואין צריכה לבטל:
\par \par \par {\large\emph{הדרן עלך כל הצלמים}}\par \par }
\twocol{
\par מתני׳ {\large\emph{רבי}} ישמעאל אומר שלש אבנים זו בצד זו בצד מרקוליס אסורות ושתים מותרות וחכ"א שנראות עמו אסורות ושאין נראות עמו מותרות:}
\twocol{{\large\emph{גמ׳}} בשלמא רבנן קסברי עובדין לשברים נראות עמו דאיכא למימר מיניה נפל אסורות שאין נראות עמו מותרות
\par אלא ר' ישמעאל מאי קסבר אי עובדין לשברין אפילו תרתי נמי ליתסר אי אין עובדין לשברים אפי' תלת נמי לא}
\twocol{
\par אמר רב יצחק בר יוסף א"ר יוחנן בידוע שנשרו ממנו דברי הכל אסורות ואפילו למ"ד אין עובדין לשברים ה"מ עבודת כוכבים דלאו היינו אורחיה אבל הכא דמעיקרא תבורי מיתברי היינו אורחיה כי פליגי בסתמא}
\newchap{פרק \hebrewnumeral{4}\quad רבי ישמעאל}
\newsection{דף נ}
\twocol{
\par במקורבות נמי דאיכא למימר מיניה נפל ד"ה אסורות כי פליגי במרוחקות}
\twocol{והא בצד מרקוליס קתני מאי בצד בצד ארבע אמות דידיה
\par רבי ישמעאל סבר עושין מרקוליס קטן בצד מרקוליס גדול שלש דדמיין למרקוליס אסורות שתים מותרות רבנן סברי אין עושין מרקוליס קטן בצד מרקוליס גדול לא שנא שלש ולא שנא שתים נראות עמו אסורות שאין נראות עמו מותרות:}
\twocol{אמר מר בידוע שנשרו ממנו דברי הכל אסורות ורמינהי אבנים שנשרו מן המרקוליס נראות עמו אסורות שאין נראות עמו מותרות ור' ישמעאל אומר שלש אסורות שתים מותרות אמר רבא לא תימא שנשרו אלא אימא שנמצאו
\par וסבר ר' ישמעאל שתים מותרות והתניא ר' ישמעאל אומר שתים בתפיסה לו אסורות שלש אפילו מרוחקות אסורות}
\twocol{אמר רבא לא קשיא כאן בתפיסה אחת כאן בשתי תפיסות וה"ד דאיכא גובהה ביני וביני
\par ומרקוליס כה"ג מי הוי והא תניא אלו הן אבני בית קוליס אחת מכאן ואחת מכאן ואחת על גביהן אמר רבא כי תניא ההיא בעיקר מרקוליס}
\twocol{בי ינאי מלכא חרוב אתו עובדי כוכבים אוקימו ביה מרקוליס אתו עובדי כוכבים אחריני דלא פלחי למרקוליס שקלינהו וחיפו בהן דרכים וסטרטאות איכא רבנן דפרשי ואיכא רבנן דלא פרשי
\par א"ר יוחנן בנן של קדושים מהלך עליהן ואנן נפרוש מהן מאן ניהו בנן של קדושים רבי מנחם ברבי סימאי ואמאי קרו ליה בנן של קדושים דאפי' בצורתא דזוזא לא מיסתכל}
\twocol{מ"ט דמאן דפריש סבר לה כי הא דאמר רב גידל א"ר חייא בר יוסף א"ר מנין לתקרובת עבודת כוכבים שאין לה בטילה עולמית שנאמר (תהלים קו, כח) ויצמדו לבעל פעור ויאכלו זבחי מתים מה מת אין לו בטילה לעולם אף תקרובת עבודת כוכבים אין לה בטילה לעולם
\par ומאן דלא פריש אמר בעינא כעין פנים וליכא}
\twocol{אמר רב יוסף בר אבא איקלע רבה בר ירמיה לאתרין ואתא ואייתי מתניתא בידיה עובד כוכבים שהביא אבנים מן המרקוליס וחיפה בהן דרכים וטרטיאות
\par מותרות ישראל שהביא אבנים מן המרקוליס וחיפה בהן דרכים וסרטיאות אסורות ולית נגר ולא בר נגר דיפרקינה}
\twocol{אמר רב ששת אנא לא נגר אנא ולא בר נגר אנא ופריקנא ליה מאי קושיא ליה דרב גידל בעינא כעין פנים וליכא
\par אמר רב יוסף בר אבא איקלע רבה בר ירמיה לאתרין ואתא ואייתי מתניתא בידיה מתליעין ומזהמין בשביעית ואין מתליעין ומזהמין במועד}
\twocol{כאן וכאן אין מגזמין וסכין שמן לגזום בין במועד בין בשביעית ולית נגר ולא בר נגר דיפרקינה
\par אמר רבינא אנא לא נגר אנא ולא בר נגר אנא ומפרקינא לה מאי קא קשיא ליה אילימא מועד אשביעית קא קשיא ליה מאי שנא שביעית דשרי ומ"ש מועד דאסור מי דמי שביעית מלאכה אסר רחמנא טירחא שרי מועד אפי' טירחא נמי אסור}
\twocol{ואלא זיהום אגיזום קא קשיא ליה מ"ש זיהום דשרי ומ"ש גיזום דאסור מי דמי זיהום אוקומי אילנא ושרי גיזום אברויי אילנא ואסור
\par ואלא זיהום אזיהום קא קשיא ליה דקתני מתליעין ומזהמין בשביעית ורמינהי מזהמין את הנטיעות וכורכין אותן וקוטמין אותן ועושין להם בתים ומשקין אותן עד ר"ה עד ר"ה אין בשביעית לא}
\twocol{ודלמא כדרב עוקבא בר חמא דאמר רב עוקבא בר חמא תרי קשקושי הוו חד לאברויי אילנא ואסור וחד לסתומי פילי ושרי ה"נ תרי זיהמומי הוי חד לאוקומי אילני ושרי וחד לאברויי אילני ואסור
\par ואלא סיכה אסיכה קא קשיא ליה דקתני סכין שמן לגזום בין במועד ובין בשביעית ורמינהי סכין את הפגין ומנקבין ומפטמין אותן עד ר"ה עד ר"ה אין בשביעית לא}
\twocol{מי דמי הכא אוקומי אילנא ושרי התם פטומי פירא ואסור
\par א"ל רב סמא בריה דרב אשי לרבינא בר ירמיה סיכה דמועד אזיהום דמועד קא קשיא ליה מכדי האי אוקומי והאי אוקומי מאי שנא האי דשרי ומאי שנא האי דאסור היינו דקא"ל לית נגר ולא בר נגר דיפרקינה}
\twocol{אמר רב יהודה אמר רב עבודת כוכבים שעובדין אותה במקל שבר מקל בפניה חייב זרק מקל בפניה פטור א"ל אביי לרבא מאי שנא שבר דהוה ליה כעין זביחה זרק נמי הוה ליה כעין זריקה אמר ליה בעינא זריקה משתברת וליכא
\par איתיביה ספת לה צואה או שנסך לפניה עביט של מימי רגלים}
\newsection{דף נא}
\twocol{חייב בשלמא עביט של מימי רגלים איכא זריקה משתברת אלא צואה מאי זריקה משתברת איכא בצואה לחה
\par לימא כתנאי שחט לה חגב ר' יהודה מחייב וחכמים פוטרים}
\twocol{מאי לאו בהא קמיפלגי דמר סבר אמרינן כעין זביחה ומר סבר לא אמרינן כעין זביחה אלא כעין פנים
\par לא דכ"ע לא אמרינן כעין זביחה אלא כעין פנים בעינן ושאני חגב הואיל וצוארו דומה לצואר בהמה}
\twocol{אמר ר"נ אמר רבה בר אבוה אמר רב עבודת כוכבים שעובדין אותה במקל שבר מקל בפניה חייב ונאסרת זרק מקל לפניה חייב ואינה נאסרת
\par א"ל רבא לר"נ מאי שנא שבר דהויא ליה כעין זביחה זרק נמי הויא ליה כעין זריקה א"ל בעינן זריקה משתברת וליכא}
\twocol{אלא מעתה אבני בית מרקוליס במה יאסרו א"ל אף לדידי קשיא לי ושאלתיה לרבה בר אבוה ורבה בר אבוה לחייא בר רב וחייא בר רב לרב וא"ל נעשה כמגדל עבודת כוכבים
\par הניחא למ"ד עבודת כוכבים של עובד כוכבים אסורה מיד אלא למ"ד עד שתעבד תישתרי דהא לא פלחה א"ל כל אחת ואחת נעשית עבודת כוכבים ותקרובת לחברתה}
\twocol{א"ה בתרייתא מיהא תשתרי אמר ליה אי ידעת לה זיל שקלה רב אשי אמר כל אחת ואחת נעשית תקרובת לעצמה ותקרובת לחברתה
\par תנן מצא בראשו כסות ומעות או כלים הרי אלו מותרין פרכילי ענבים ועטרות של שבלים ויינות שמנים וסלתות וכל דבר שכיוצא בו קרב לגבי מזבח אסור}
\twocol{בשלמא יינות שמנים וסלתות איכא כעין פנים ואיכא כעין זריקה משתברת אלא פרכילי ענבים ועטרות של שבלים לא כעין פנים איכא ולא כעין זריקה משתברת איכא
\par אמר רבא אמר עולא כגון שבצרן מתחלה לכך}
\twocol{א"ר אבהו א"ר יוחנן מנין לזובח בהמה בעלת מום לעבודת כוכבים שהוא פטור שנאמר (שמות כב, יט) זובח לאלהים יחרם בלתי לה' לבדו לא אסרה תורה אלא כעין פנים
\par הוי בה רבא במאי אילימא בדוקין שבעין השתא לבני נח חזיא לגבוה בבמה דידהו לעבודת כוכבים מיבעיא}
\twocol{אלא במחוסר אבר וכדרבי אלעזר דאמר ר' אלעזר מנין למחוסר אבר שהוא אסור לבני נח שנאמר (בראשית ו, יט) ומכל החי מכל בשר שנים מכל אמרה תורה הבא בהמה שחיין ראשי אברין שלה
\par האי ומכל החי מיבעי ליה למעוטי טריפה (בראשית ז, ג) מלהחיות זרע נפקא}
\twocol{הניחא למ"ד טריפה אינה יולדת אלא למ"ד טריפה יולדת מאי איכא למימר
\par אמר קרא אתך אתך בדומין לך ודלמא נח גופיה טריפה הוה תמים כתיב ביה}
\twocol{דלמא תמים בדרכיו צדיק כתיב ביה
\par דלמא תמים בדרכיו וצדיק במעשיו לא מצית אמרת דנח גופיה טריפה הוה דאי ס"ד נח טריפה הוה א"ל רחמנא דכוותך עייל שלמין לא תעייל}
\twocol{השתא דנפקא מאתך להחיות זרע למה לי אי מאתך ה"א לצוותא בעלמא ואפי' זקן ואפי' סריס קמ"ל להחיות זרע:
\par א"ר אלעזר מנין לשוחט בהמה למרקוליס שהוא חייב שנאמר (ויקרא יז, ז) ולא יזבחו עוד את זבחיהם לשעירים אם אינו ענין לכדרכה דכתיב (דברים יב, ל) איכה יעבדו הגוים האלה את אלהיהם תנהו ענין לשלא כדרכה}
\twocol{והא להכי הוא דאתא האי מיבעי ליה לכדתניא
\par עד כאן הוא מדבר בקדשים שהקדישן בשעת איסור הבמות והקריבן בשעת איסור הבמות}
\twocol{שהרי עונשן אמור שנאמר (ויקרא יז, ד) ואל פתח אהל מועד לא הביאו וגו' עונש שמענו אזהרה מנין ת"ל (דברים יב, יג) פן תעלה עולותיך
\par וכדר' אבין א"ר אילא דאמר ר' אבין א"ר אילא כל מקום שנאמר השמר ופן ואל אינו אלא בלא תעשה}
\twocol{מכאן ואילך הוא מדבר בקדשים שהקדישן בשעת היתר הבמות והקריבן בשעת איסור הבמות
\par שנאמר (ויקרא יז, ה) למען אשר יביאו בני ישראל את זבחיהם אשר הם זובחים שהתרתי לך כבר על פני השדה מלמד שכל הזובח בבמה בשעת איסור הבמות מעלה עליו הכתוב כאילו הוא זובח על פני השדה}
\twocol{והביאום לה' זו מצות עשה ומצות לא תעשה מנין ת"ל (ויקרא יז, ז) ולא יזבחו עוד את זבחיהם
\par יכול יהא ענוש כרת ת"ל (ויקרא יז, ז) חקת עולם תהיה זאת להם זאת להם ולא אחרת להם}
\twocol{אמר רבא קרי ביה ולא יזבחו וקרי ביה ולא עוד:
\par {\large\emph{מתני׳}} מצא בראשו מעות כסות או כלים הרי אלו מותרין פרכילי ענבים ועטרות של שבלים ויינות ושמנים וסלתות וכל דבר שכיוצא בו קרב ע"ג המזבח אסור:}
\twocol{{\large\emph{גמ׳}} מנהני מילי א"ר חייא בר יוסף א"ר אושעיא כתוב אחד אומר (דברים כט, טז) ותראו את שקוציהם ואת גלוליהם עץ ואבן כסף וזהב אשר עמהם וכתוב אחד אומר (דברים ז, כה) לא תחמוד כסף וזהב עליהם הא כיצד
\par עמהם דומיא דעליהם מה עליהם דבר של נוי אסור שאינו של נוי מותר אף עמהם דבר של נוי אסור ושאינו של נוי מותר}
\twocol{ואימא עליהם דומיא דעמהם מה עמהם כל מה שעמהם אף עליהם כל שעליהם א"כ לא יאמר עליהם
\par מעות דבר של נוי הוא אמרי דבי ר' ינאי בכיס קשור ותלוי לו בצוארו}
\twocol{כסות דבר של נוי הוא אמרי דבי ר' ינאי בכסות מקופלת ומונחת לו על ראשו כלי דבר של נוי הוא אמר רב פפא דסחיפא ליה משכילתא ארישיה
\par אמר רב אסי בר חייא כל שהוא לפנים מן הקלקלין אפי' מים ומלח אסור חוץ לקלקלין דבר של נוי אסור שאינו של נוי מותר א"ר יוסי בר חנינא נקטינן אין קלקלין לא לפעור ולא למרקוליס}
\twocol{למאי אילימא דאפי' פנים כחוץ דמי ושרי השתא פעורי מפערין קמיה מים ומלח לא מקרבין ליה אלא אפי' חוץ כבפנים דמי ואסור:
\par {\large\emph{מתני׳}} עבודת כוכבים שהיה לה גינה או מרחץ נהנין מהן שלא בטובה ואין נהנין מהן בטובה היה שלה ושל אחרים נהנין מהן בין בטובה ובין שלא בטובה עבודת כוכבים של עובד כוכבים אסורה מיד ושל ישראל אין אסורה עד שתיעבד:}
\twocol{{\large\emph{גמ׳}} אמר אביי בטובה בטובת כומרין שלא בטובה שלא בטובת כומרין לאפוקי טובת עובדיה דשרי
\par איכא דמתני לה אסיפא היה שלה ושל אחרים נהנין מהן בטובה ושלא בטובה אמר אביי בטובה בטובת אחרים שלא בטובה שלא בטובת כומרין}
\twocol{מאן דמתני אסיפא כ"ש ארישא ומאן דמתני ארישא אבל אסיפא כיון דאיכא אחרים בהדה אפי' בטובת כומרין נמי שפיר דמי:
\par עבודת כוכבים של עובד כוכבים אסורה מיד: מתני' מני ר"ע היא דתניא (דברים יב, ב) אבד תאבדון את כל המקומות אשר עבדו שם הגוים בכלים שנשתמשו בהן לעבודת כוכבים הכתוב מדבר}
\twocol{יכול עשאום ולא גמרום גמרום ולא הביאום הביאום ולא נשתמשו בהן יכול יהו אסורים ת"ל אשר עבדו שם הגוים שאין אסורין עד שיעבדו מכאן אמרו עבודת כוכבים של עובד כוכבים אינה אסורה עד שתיעבד ושל ישראל אסורה מיד דברי ר' ישמעאל
\par ר"ע אומר חילוף הדברים עבודת כוכבים של עובד כוכבים אסורה מיד ושל ישראל עד שתיעבד}
\twocol{אמר מר בכלים שנשתמשו בהן לעבודת כוכבים הכתוב מדבר הא מקומות כתיב אם אינו ענין למקומות דלא מיתסרי דכתיב (דברים יב, ב) אלהיהם על ההרים ולא ההרים אלהיהם}
\newsection{דף נב}
\twocol{תנהו ענין לכלים מכאן אמרו עבודת כוכבים של עובד כוכבים אינה אסורה אלא עד שתעבד ושל ישראל מיד
\par והא בכלים אוקימנא לה אמר קרא (דברים יב, ב) אשר אתם יורשים אותם את אלהיהם מקיש אלהיהם לכלים מה כלים עד שיעבדו אף אלהיהם נמי עד שיעבדו ור"ע דלא מקיש אמר לך את הפסיק הענין}
\twocol{ורבי ישמעאל אשכחן עבודת כוכבים של עובד כוכבים דאין אסורה עד שתעבד דישראל דאסורה מיד מנא ליה סברא הוא מדעובד כוכבים עד שתעבד דישראל אסורה מיד אימא דישראל כלל וכלל לא השתא גניזה בעיא איתסורי לא מיתסרא
\par ואימא כדעובד כוכבים אמר קרא (דברים ט, כא) ואת חטאתכם אשר עשיתם את העגל משעת עשייה קם ליה בחטא}
\twocol{אימא ה"מ למיקם גברא בחטא איתסורי לא מיתסרא אמר קרא (דברים כז, טו) ארור האיש אשר יעשה פסל ומסכה משעת עשייה קם ליה בארור
\par אימא ה"מ למיקם גברא בארור איתסורי לא מיתסרא תועבת ה' כתיב}
\twocol{ור"ע דבר המביא לידי תועבה
\par ור"ע עבודת כוכבים של עובד כוכבים דאסורה מיד מנא ליה אמר עולא אמר קרא (דברים ז, כה) פסילי אלהיהם תשרפון באש משפסלו נעשה אלוה}
\twocol{ואידך ההוא מיבעי ליה לכדתני רב יוסף דתני רב יוסף מנין לעובד כוכבים שפוסל אלוהו שנאמר פסילי אלהיהם תשרפון באש
\par ואידך נפקא ליה מדשמואל דשמואל רמי כתיב (דברים ז, כה) לא תחמוד כסף וזהב עליהם וכתיב ולקחת לך הא כיצד פסלו לאלוה לא תחמוד פסלו מאלוה ולקחת לך}
\twocol{ור"ע אשכחן עבודת כוכבים של עובד כוכבים דאסורה מיד דישראל עד שתעבד מנלן אמר רב יהודה אמר קרא (דברים כז, טו) ושם בסתר עד שיעשה לה דברים שבסתר
\par ואידך ההוא מיבעיא ליה לכדרבי יצחק דא"ר יצחק מנין לעבודת כוכבים של ישראל שטעונה גניזה שנאמר ושם בסתר}
\twocol{ואידך נפקא ליה מדרב חסדא אמר רב דאמר רב חסדא אמר רב מנין לעבודת כוכבים של ישראל שטעונה גניזה שנאמר (דברים טז, כא) לא תטע לך אשרה כל עץ אצל מזבח מה מזבח טעון גניזה אף אשרה טעונה גניזה
\par ואידך ההוא מיבעי ליה לכדר"ל דאמר ר"ל כל המעמיד דיין שאינו הגון כאילו נוטע אשרה בישראל שנאמר (דברים טז, יח) שופטים ושוטרים תתן לך בכל שעריך וסמיך ליה לא תטע לך אשרה כל עץ}
\twocol{אמר רב אשי ובמקום תלמידי חכמים כאילו נטעו אצל מזבח שנאמר אצל מזבח
\par בעי רב המנונא ריתך כלי לעבודת כוכבים מהו עבודת כוכבים דמאן אילימא עבודת כוכבים דעובד כוכבים בין לר' ישמעאל ובין לר"ע משמשי עבודת כוכבים הן ומשמשי עבודת כוכבים אין אסורין עד שיעבדו ואלא עבודת כוכבים דישראל}
\twocol{אליבא דמאן אילימא אליבא דר"ע השתא היא גופה לא מיתסר' עד שתעבד משמשיה מיבעיא ואלא אליבא דרבי ישמעאל דאמר אסורה מיד
\par מאי משמשין ממשמשין גמרינן מה התם עד שיעבדו אף הכא עד שיעבדו או דלמא מינה גמר מה היא אסורה מיד אף משמשיה אסורין מיד}
\twocol{מאי איריא דקא מיבעיא ליה ריתך כלי תיבעי ליה עשה
\par רב המנונא משום טומאה ישנה קמיבעיא ליה דתנן כלי מתכות פשוטיהן ומקבליהן טמאין נשתברו טהרו חזר ועשאן כלים יחזרו לטומאה ישנה}
\twocol{והכי קמיבעיא ליה כי הדרא טומאה ה"מ לטומאה דאורייתא אבל טומאה דרבנן לא או דלמא ל"ש ותיבעי ליה שאר טומאות דרבנן
\par חדא מגו חדא קמיבעיא ליה טומאה דרבנן מי הדרא או לא הדרא את"ל לא הדרא טומאה דעבודת כוכבים משום חומרא דעבודת כוכבים מי שויוה רבנן כטומאה דאורייתא או לא תיקו}
\twocol{בעי מיניה ר' יוחנן מר' ינאי תקרובת עבודת כוכבים של אוכלים מהו מי מהניא להו ביטול לטהרינהו מטומאה או לא
\par ותיבעי ליה כלים כלים לא קמיבעיא ליה כיון דאית להו טהרה במקוה טומאה נמי בטלה כי קמיבעיא ליה אוכלין}
\twocol{ותיבעי ליה עבודת כוכבי' גופה עבודת כוכבים גופה לא מיבעיא ליה
\par כיון דאיסורה בטיל טומאה נמי בטלה כי קא מיבעיא ליה תקרובת לעבודת כוכבים של אוכלין מאי כיון דאיסוריה לא בטיל כדרב גידל טומאה נמי לא בטלה או דלמא איסור דאורייתא לא בטיל טומאה דרבנן בטיל תיקו}
\twocol{בעא מיניה ר' יוסי בן שאול מרבי כלים ששימשו בהן בבית חוניו מהו שישתמשו בהן בבית המקדש
\par וקא מיבעיא ליה אליבא דמ"ד בית חוניו לאו בית עבודת כוכבים היא דתנן כהנים ששימשו בבית חוניו לא ישמשו במקדש שבירושלים ואינו צריך לומר לדבר אחר}
\twocol{כהנים הוא דקנסינהו רבנן משום דבני דעה נינהו אבל כלים לא או דלמא לא שנא
\par א"ל אסורים הן ומקרא היה בידינו ושכחנוהו}
\twocol{איתיביה (דברי הימים ב כט, יט) כל הכלים אשר הזניח המלך אחז במלכותו במעלו הכנו והקדשנו מאי לאו הכנו דאטבלינהו הקדשנו דאקדישננהו
\par א"ל ברוך אתה לשמים שהחזרת לי אבדתי הכנו שגנזנום והקדשנו שהקדשנו אחרים תחתיהם}
\twocol{לימא מסייע ליה מזרחית צפונית בה גנזו בית חשמונאי את אבני המזבח ששקצו אנשי יון ואמר רב ששת ששקצו לעבודת כוכבים
\par אמר רב פפא התם קרא אשכח ודרש דכתיב (יחזקאל ז, כב) ובאו בה פריצים וחללוה}
\twocol{אמרי  היכי נעביד ניתברינהו (דברים כז, ו) אבנים שלמות אמר רחמנא ננסרינהו (דברים כז, ה) לא תניף עליהם ברזל אמר רחמנא
\par ואמאי ליתברינהו ולישקלינהו לנפשייהו מי לא אמר רב אושעיא בקשו לגנוז כל כסף וזהב שבעולם משום כספא ודהבא של ירושלים והוינן בה ירושלים הויא רובא דעלמא}
\twocol{אלא אמר אביי בקשו לגנוז דינרא הדרייאנא טוריינא שיפא מפני טבעה של ירושלים עד שמצאו לה מקרא מן התורה שהוא מותר ובאו בה פריצים וחללוה
\par התם לא אשתמשו בהו לגבוה הכא כיון דאשתמש בהו לגבוה לאו אורח ארעא לאשתמושי בהו הדיוטא:}
\twocol{{\large\emph{מתני׳}} עובד כוכבים מבטל עבודת כוכבים שלו ושל חבירו וישראל אין מבטל עבודת כוכבים של עובד כוכבים המבטל עבודת כוכבים מבטל משמשיה ביטל משמשיה משמשין מותרין והיא אסורה:
\par {\large\emph{גמ׳}} מתני ליה ר' לר"ש ברבי עובד כוכבים מבטל עבודת כוכבים שלו ושל חבירו א"ל רבי שנית לנו בילדותך עובד כוכבים מבטל עבודת כוכבים שלו ושל ישראל דישראל מי קא מבטלה והא (דברים כז, טו) ושם בסתר כתיב א"ר הילל בריה דרבי וולס לא נצרכה שיש לו בה שותפות}
\twocol{בילדותו מאי קסבר ובזקנותו מאי קסבר בילדותו סבר ישראל אדעתא דעובד כוכבים פלח כיון דעובד כוכבים מבטל דנפשיה דישראל נמי מבטלה ובזקנותו סבר ישראל אדעתא דנפשיה פלח כי מבטל עובד כוכבים דנפשיה דישראל לא בטיל
\par איכא דמתני לה אסיפא ישראל אינו מבטל עבודת כוכבים של עובד כוכבים פשיטא א"ר הילל בריה}
\newsection{דף נג}
\twocol{דרבי וולס לא נצרכה שיש לו בה שותפות וקמ"ל ישראל הוא דלא מבטל דעובד כוכבים אבל עובד כוכבים דנפשיה מבטל
\par איכא דמתני לה אברייתא ר"ש בן מנסיא אומר עבודת כוכבים של ישראל אין לה בטילה עולמית מאי עולמית א"ר הילל בריה דר' וולס לא נצרכה אלא שיש לו לעובד כוכבים בה שותפות וקמ"ל דישראל אדעתא דנפשיה פלח:}
\twocol{{\large\emph{מתני׳}} כיצד מבטלה קטע ראש אזנה ראש חוטמה ראש אצבעה פחסה אע"פ שלא חיסרה ביטלה רק בפניה השתין בפניה גררה זרק בה את הצואה הרי זו אינה בטילה מכרה או משכנה רבי אומר ביטל וחכ"א לא ביטל:
\par {\large\emph{גמ׳}} כי לא חיסרה במאי ביטלה א"ר זירא שפחסה בפניה}
\twocol{רקק בפניה והשתין בפניה מנה"מ
\par אמר חזקיה דאמר קרא (ישעיהו ח, כא) והיה כי ירעב והתקצף וקלל במלכו ובאלהיו ופנה למעלה וכתיב בתריה ואל ארץ יביט והנה צרה וחשכה וגו' דאע"ג דקלל מלכו ואלהיו ופנה למעלה אל ארץ יביט:}
\twocol{מכרה או משכנה רבי אומר ביטל וכו': זעירי א"ר יוחנן ור' ירמיה בר אבא אמר רב ח"א מחלוקת בצורף עובד כוכבים אבל בצורף ישראל דברי הכל ביטל וחד אמר בצורף ישראל מחלוקת
\par איבעיא להו בצורף ישראל מחלוקת אבל צורף עובד כוכבים דברי הכל לא ביטל או דלמא בין בזה ובין בזה מחלוקת}
\twocol{ת"ש דא"ר נראין דבריי כשמכרה לחבלה ודברי חביריי שמכרה לעובדה
\par מאי לחבלה ומאי לעובדה אילימא לחבלה לחבלה ממש לעובדה לעובדה ממש מ"ט דמ"ד ביטל ומ"ט דמ"ד לא ביטל}
\twocol{אלא לאו לחבלה למי שעתיד לחבלה ומנו צורף ישראל לעובדה למי שעתיד לעובדה ומנו צורף עובד כוכבים וש"מ בין בזה ובין בזה מחלוקת
\par לא ה"ק א"ר נראין דבריי לחביריי כשמכרה לחבלה ומנו צורף ישראל שאף חביריי לא נחלקו עלי אלא כשמכרה לעובדה אבל לחבלה מודו לי}
\twocol{מיתיבי הלוקח גרוטאות מן העובדי כוכבים ומצא בהן עבודת כוכבים אם עד שלא נתן מעות משך יחזיר אם משנתן מעות משך יוליך לים המלח
\par אי אמרת בשלמא בצורף ישראל מחלוקת הא מני רבנן היא אלא אי אמרת בצורף עובד כוכבים מחלוקת אבל בצורף ישראל דברי הכל ביטל הא מני}
\twocol{שאני התם דאדעתא דגרוטאות זבין אדעתא דעבודת כוכבים לא זבין
\par תנו רבנן לוה עליה או שנפלה עליה מפולת או שגנבוה ליסטין או שהניחוה הבעלים והלכו למדינת הים}
\twocol{אם עתידין לחזור כמלחמת יהושע אינה בטילה
\par וצריכא דאי תנא לוה עליה מדלא זבנה לא בטלה אבל נפלה עליה מפולת מדלא קא מפני לה אימא בטולי בטלה צריכא}
\twocol{ואי תנא נפלה עליה מפולת משום דסבר הא מנחת כל אימת דבעינא לה שקילנא לה אבל גנבוה לסטים מדלא קא מהדר אבתרה בטולי בטלה צריכא
\par ואי תנא גנבוה לסטין משום דסבר אי עובד כוכבים שקיל לה מפלח פלח לה אי ישראל שקלה איידי דדמיה יקרין מזבין לה לעובד כוכבים ופלח לה אבל הניחוה הבעלים והלכו למדינת הים מדלא שקלו בהדייהו בטולי בטלוה צריכא}
\twocol{אם עתידין לחזור כמלחמת יהושע אינה בטילה מידי מלחמת יהושע מיהדר הדור ה"ק אם עתידין לחזור הרי הוא כמלחמת יהושע ואין לה בטילה
\par ולמה לי למיתלייה במלחמת יהושע מלתא אגב אורחא קמ"ל כי הא דאמר רב יהודה אמר רב ישראל שזקף לבינה להשתחות לה ובא עובד כוכבים והשתחוה לה אסרה}
\twocol{מנלן דאסרה א"ר אלעזר כתחילה של א"י דאמר רחמנא (דברים יב, ג) ואשריהם תשרפון באש מכדי ירושה היא להם מאבותיהם ואין אדם אוסר דבר שאינו שלו
\par ואי משום הנך דמעיקרא בביטולא בעלמא סגי להו}
\twocol{אלא מדפלחו ישראל לעגל גלו אדעתייהו דניחא להו בעבודת כוכבים וכי אתו עובדי כוכבים שליחותא דידהו עבדי ה"נ ישראל שזקף לבינה גליא דעתיה דניחא ליה בעבודת כוכבים וכי אתא עובד כוכבים ופלח לה שליחותא דידיה קעביד
\par ודלמא בעגל הוא דניחא להו במידי אחרינא לא אמר קרא (שמות לב, ד) אלה אלהיך ישראל מלמד שאיוו לאלוהות הרבה}
\twocol{אימא כל דבהדי עגל ניתסרו מכאן ואילך נישתרי מאן מוכח:
\par {\large\emph{מתני׳}} עבודת כוכבים שהניחוה עובדיה בשעת שלום מותרת בשעת מלחמה אסורה בימוסיאות של מלכים הרי אלו מותרות מפני שמעמידין אותה בשעה שהמלכים עוברים:}
\twocol{{\large\emph{גמ׳}} אמר רבי ירמיה בר אבא אמר רב בית נמרוד הרי היא כעבודת כוכבים שהניחוה עובדיה בשעת שלום ומותר אע"ג דכי בדרינהו רחמנא כשעת מלחמה דמי אי בעיא למיהדר הדור מדלא הדור בטולי בטלה:
\par בימוסיאות של מלכים הרי אלו מותרות: וכי מפני שמעמידין אותה בשעה שהמלכים עוברין מותרין}
\twocol{אמר רבה בר בר חנה אמר רבי יוחנן ה"ק מפני שמעמידין אותן בשעה שהמלכים עוברין ומלכים מניחין דרך זו והולכין בדרך אחרת
\par כי אתא עולא יתיב אבימסא פגימא א"ל רב יהודה לעולא והא רב ושמואל דאמרי תרוייהו בימוס שנפגם אסור ואפי' למ"ד אין עובדים לשברים ה"מ עבודת כוכבים דזילא ביה מלתא למפלח לשברים אבל האי לא איכפת ליה}
\twocol{א"ל מאן יהיב לן מעפרא דרב ושמואל ומלאינן עיינין הא רבי יוחנן ור"ל דאמרי תרוייהו בימוס שנפגם מותר ואפי' למ"ד עובדין לשברים ה"מ עבודת כוכבים דכיון דפלחה זילא ביה מילת' לבטולה אבל הני שקלי להאי ומייתו בימוס אחרינא
\par תניא כוותיה דר' יוחנן ור"ל בימוס שנפגם מותר מזבח שנפגם אסור עד שינתץ רובו ה"ד בימוס ה"ד מזבח א"ר יעקב בר אידי אמר ר' יוחנן בימוס אבן אחת מזבח אבנים הרבה}
\newsection{דף נד}
\twocol{אמר חזקיה מאי קרא (ישעיהו כז, ט) בשומו כל אבני מזבח כאבני גיר מנופצות לא יקומו אשרים וחמנים אי איכא כאבני גיר מנופצות לא יקומון אשרים וחמנים אי לאו יקומו
\par תנא נעבד שלו אסור ושל חבירו מותר ורמינהי איזהו נעבד כל שעובדים אותו בין בשוגג ובין במזיד בין באונס ובין ברצון האי אונס היכי דמי לאו כגון דאנס בהמת חבירו והשתחוה לה}
\twocol{אמר רמי בר חמא לא כגון שאנסוהו עובדי כוכבים והשתחוה לבהמתו דידיה מתקיף לה רבי זירא אונס רחמנא פטריה דכתיב (דברים כב, כו) ולנערה לא תעשה דבר
\par אלא אמר רבא הכל היו בכלל לא תעבדם וכשפרט לך הכתוב (ויקרא יח, ה) וחי בהם ולא שימות בהם יצא אונס}
\twocol{והדר כתב רחמנא ולא תחללו את שם קדשי דאפילו באונס הא כיצד הא בצנעא והא בפרהסיא
\par אמרו ליה רבנן לרבא תניא דמסייעא לך בימוסיאות של עובדי כוכבים בשעת הגזרה אף על פי שהגזרה בטלה אותן בימוסיאות לא בטלו}
\twocol{אמר להו אי משום הא לא תסייען אימר ישראל מומר הוה ופלח לה ברצון רב אשי אמר לא תימא אימר אלא ודאי ישראל מומר הוה ופלח לה ברצון
\par חזקיה אמר כגון שניסך לעבודת כוכבים יין על קרניה מתקיף לה רב אדא בר אהבה האי נעבד הוא האי בימוס בעלמא הוא ושרייה}
\twocol{אלא אמר רב אדא בר אהבה כגון שניסך לה יין בין קרניה דעבד בה מעשה וכי הא דאתא עולא אמר רבי יוחנן אף על פי שאמרו המשתחוה לבהמת חבירו לא אסרה עשה בה מעשה אסרה
\par אמר להו רב נחמן פוקו ואמרו ליה לעולא כבר תרגמה רב הונא לשמעתיך בבבל דאמר רב הונא היתה בהמת חבירו רבוצה בפני עבודת כוכבים כיון ששחט בה סימן אחד אסרה}
\twocol{מנא לן דאסרה אילימא מכהנים ודלמא שאני כהנים דבני דעה נינהו
\par ואלא מאבני מזבח ודלמא כדר"פ}
\twocol{ואלא מכלים דכתיב (דברי הימים ב כט, יט) ואת כל הכלים אשר הזניח המלך אחז במלכותו במעלו הכנו והקדשנו ואמר מר הכנו שגנזנום והקדשנו שהקדשנו אחרים תחתיהן והא אין אדם אוסר דבר שאינו שלו
\par אלא כיון דעבד בהו מעשה איתסרו להו הכא נמי כיון שעשה בה מעשה אסרה}
\twocol{כי אתא רב דימי א"ר יוחנן אע"פ שאמרו המשתחוה לקרקע עולם לא אסרה חפר בה בורות שיחין ומערות אסרה כי אתא רב שמואל בר יהודה א"ר יוחנן אע"פ שאמרו המשתחוה לבעלי חיים לא אסרן עשאן חליפין לעבודת כוכבים אסרן
\par כי אתא רבין אמר פליגו בה רבי ישמעאל ב"ר יוסי ורבנן חד אמר חליפין אסורין חליפי חליפין מותרין וחד אמר אפילו חליפי חליפין נמי אסורין}
\twocol{מ"ט דמ"ד חליפי חליפין אסורין אמר קרא (דברים ז, כו) והיית חרם כמוהו כל שאתה מהיה ממנו הרי הוא כמוהו ואידך אמר קרא הוא הוא ולא חליפי חליפין
\par ואידך ההוא מיבעי ליה למעוטי ערלה וכלאי הכרם שאם מכרן וקידש בדמיהן מקודשת}
\twocol{ואידך ערלה וכלאי הכרם לא צריכי מיעוטא דהויא להו עבודת כוכבים ושביעית שני כתובין הבאין כאחד וכל שני כתובין הבאין כאחד אין מלמדין
\par עבודת כוכבים הא דאמרן שביעית דכתיב (ויקרא כה, יב) כי יובל היא קדש תהיה לכם מה קדש תופס את דמיו ואסור אף שביעית תופסת את דמיה ואסורה}
\twocol{אי מה קדש תופס את דמיו ויוצא לחולין אף שביעית תופסת את דמיה ויוצאה לחולין ת"ל תהיה בהוייתה תהא
\par הא כיצד לקח בפירות שביעית בשר אלו ואלו מתבערין בשביעית לקח בבשר דגים יצא בשר נכנסו דגים בדגים יין יצאו דגים נכנס יין ביין שמן יצא יין ונכנס שמן הא כיצד אחרון אחרון נתפס בשביעית ופרי עצמו אסור}
\twocol{ואידך קסבר שני כתובין הבאין כאחד מלמדין ואיצטריך הוא למעוטינהו:
\par {\large\emph{מתני׳}} שאלו את הזקנים ברומי אם אין רצונו בעבודת כוכבים למה אינו מבטלה אמרו להן אילו לדבר שאין צורך לעולם בו היו עובדין היה מבטלו הרי הן עובדין לחמה וללבנה ולכוכבים ולמזלות יאבד עולמו מפני השוטים}
\twocol{אמרו להן א"כ יאבד דבר שאין צורך לעולם בו ויניח דבר שצורך העולם בו אמרו להן אף אנו מחזיקין ידי עובדיהן של אלו שאומרים תדעו שהן אלוהות שהרי הן לא בטלו:
\par {\large\emph{גמ׳}} ת"ר שאלו פלוסופין את הזקנים ברומי אם אלהיכם אין רצונו בעבודת כוכבים מפני מה אינו מבטלה אמרו להם אילו לדבר שאין העולם צורך לו היו עובדין הרי הוא מבטלה הרי הן עובדין לחמה וללבנה ולכוכבים ולמזלות יאבד עולם מפני השוטים אלא עולם כמנהגו נוהג ושוטים שקלקלו עתידין ליתן את הדין}
\twocol{דבר אחר הרי שגזל סאה של חטים [והלך] וזרעה בקרקע דין הוא שלא תצמח אלא עולם כמנהגו נוהג והולך ושוטים שקלקלו עתידין ליתן את הדין
\par דבר אחר הרי שבא על אשת חבירו דין הוא שלא תתעבר אלא עולם כמנהגו נוהג והולך ושוטים שקלקלו עתידין ליתן את הדין}
\twocol{והיינו דאמר ריש לקיש אמר הקב"ה לא דיין לרשעים שעושין סלע שלי פומבי אלא שמטריחין אותי ומחתימין אותי בעל כרחי
\par שאל פלוספוס אחד את ר"ג כתוב בתורתכם (דברים ד, כד) כי ה' אלהיך אש אוכלה הוא אל קנא מפני מה מתקנא בעובדיה ואין מתקנא בה}
\twocol{אמר לו אמשול לך משל למה"ד למלך בשר ודם שהיה לו בן אחד ואותו הבן היה מגדל לו את הכלב והעלה לו שם על שם אביו וכשהוא נשבע אומר בחיי כלב אבא כששמע המלך על מי הוא כועס על הבן הוא כועס או על הכלב הוא כועס הוי אומר על הבן הוא כועס
\par אמר לו כלב אתה קורא אותה והלא יש בה ממש אמר לו ומה ראית אמר לו פעם אחת נפלה דליקה בעירנו ונשרפה כל העיר כולה ואותו בית עבודת כוכבים לא נשרף}
\twocol{אמר לו אמשול לך משל למה"ד למלך ב"ו שסרחה עליו מדינה כשהוא עושה מלחמה עם החיים הוא עושה או עם המתים הוא עושה הוי אומר עם החיים הוא עושה
\par א"ל כלב אתה קורא אותה מת אתה קורא אותה א"כ יאבדנה מן העולם אמר לו אילו לדבר שאין העולם צריך לו היו עובדין הרי הוא מבטלה הרי הן עובדין לחמה וללבנה לכוכבים ולמזלות לאפיקים ולגאיות יאבד עולמו מפני שוטים וכן הוא אומר}
\newsection{דף נה}
\twocol{(צפניה א, ב) אסוף אסף כל מעל פני האדמה נאם ה' אסף אדם ובהמה אסף עוף השמים ודגי הים והמכשלות את הרשעים [וגו'] וכי מפני שהרשעים נכשלים בהן יאבדם מן העולם והלא לאדם הן עובדין (צפניה א, ג) והכרתי את האדם מעל פני האדמה [וגו']
\par שאל אגריפס שר צבא את ר"ג כתיב בתורתכם (דברים ד, כד) כי ה' אלהיך אש אכלה הוא אל קנא כלום מתקנא אלא חכם בחכם וגבור בגבור ועשיר בעשיר}
\twocol{אמר לו אמשול לך משל למה"ד לאדם שנשא אשה על אשתו חשובה ממנה אין מתקנאה בה פחותה ממנה מתקנאה בה
\par א"ל זונין לר"ע לבי ולבך ידע דעבודת כוכבים לית בה מששא והא קחזינן גברי דאזלי כי מתברי ואתו כי מצמדי מ"ט}
\twocol{אמר לו אמשול לך משל למה"ד לאדם נאמן שהיה בעיר וכל בני עירו היו מפקידין אצלו שלא בעדים ובא אדם אחד והפקיד לו בעדים פעם אחד שכח והפקיד אצלו שלא בעדים אמרה לו אשתו בוא ונכפרנו אמר לה וכי מפני ששוטה זה עשה שלא כהוגן אנו נאבד את אמונתינו
\par אף כך יסורין בשעה שמשגרין אותן על האדם משביעין אותן שלא תלכו אלא ביום פלוני ולא תצאו אלא ביום פלוני ובשעה פלונית ועל ידי פלוני ועל ידי סם פלוני כיון שהגיע זמנן לצאת הלך זה לבית עבודת כוכבים אמרו יסורין דין הוא שלא נצא וחוזרין ואומרים וכי מפני ששוטה זה עושה שלא כהוגן אנו נאבד שבועתנו}
\twocol{והיינו דא"ר יוחנן מאי דכתיב (דברים כח, נט) וחלים רעים ונאמנים רעים בשליחותן ונאמנים בשבועתן
\par א"ל רבא בר רב יצחק לרב יהודה האיכא בית עבודת כוכבים באתרין דכי מצטריך עלמא למטרא מתחזי להו בחלמא ואמר להו שחטו לי גברא ואייתי מטרא שחטו לה גברא ואתי מטרא}
\twocol{א"ל השתא אי הוי שכיבנא לא אמרי לכו הא מלתא דאמר רב מאי דכתיב (דברים ד, יט) אשר חלק ה' אלהיך אותם לכל העמים מלמד שהחליקן בדברים כדי לטורדן מן העולם
\par והיינו דאמר ריש לקיש מאי דכתיב (משלי ג, לד) אם ללצים הוא יליץ ולענוים יתן חן בא לטמא פותחין לו בא לטהר מסייעין אותו:}
\twocol{{\large\emph{מתני׳}} לוקחין גת בעוטה מן העובד כוכבים אף על פי שהוא נוטל בידו ונותן לתפוח ואינו עושה יין נסך עד שירד לבור ירד לבור מה שבבור אסור והשאר מותר
\par דורכין עם העובד כוכבים בגת}
\twocol{אבל לא בוצרין עמו ישראל שהוא עושה בטומאה לא דורכין ולא בוצרין עמו אבל מוליכין עמו חביות לגת ומביאין עמו מן הגת:
\par נחתום שהוא עושה בטומאה לא לשין ולא עורכין עמו אבל מוליכין עמו פת לפלטר:}
\twocol{{\large\emph{גמ׳}} אמר רב הונא יין כיון שהתחיל להמשך עושה יין נסך תנן לוקחים גת בעוטה מן העובד כוכבים ואע"פ שנטל בידו ונתן לתפוח א"ר הונא בגת פקוקה ומלאה
\par ת"ש ואינו עושה יין נסך עד שירד לבור ה"נ בגת פקוקה ומלאה}
\twocol{ת"ש ירד לבור מה שבבור אסור והשאר מותר אמר רב הונא לא קשיא כאן במשנה ראשונה כאן במשנה אחרונה
\par דתניא בראשונה היו אומרים בד"ד אין בוצרין עם העובד כוכבים בגת שאסור לגרום טומאה לחולין שבא"י ואין דורכין עם ישראל שעושה פירותיו בטומאה שאסור לסייע ידי עוברי עבירה אבל דורכים עם העובד כוכבים בגת ולא חיישינן לדרב הונא}
\twocol{וחזרו לומר דב"ד אין דורכין עם העובד כוכבים בגת משום דרב הונא}
\newsection{דף נו}
\twocol{ואין בוצרין עם ישראל שעושה פירותיו בטומאה וכ"ש שאין דורכין אבל בוצרין עם העובד כוכבים בגת שמותר לגרום טומאה לחולין שבא"י:
\par ואינו עושה יין נסך עד שירד לבור: והתניא יין משיקפה}
\twocol{אמר רבא לא קשיא הא ר"ע הא רבנן דתנן יין משירד לבור ר"ע אומר משיקפה
\par איבעיא להו קיפוי דבור או קיפוי דחבית}
\twocol{ת"ש דתניא יין משיקפה אע"פ שקפה קולט מן הגת העליונה ומן הצינור ושותה ש"מ קיפוי דבור קאמרינן ש"מ
\par והתני רב זביד בדבי רבי אושעיא יין משירד לבור ויקפה ר"ע אומר משישלה בחביות תרצה נמי להך קמייתא הכי יין משירד לבור ויקפה ר"ע אומר משישלה בחביות}
\twocol{ואלא מתניתין דקתני אינו עושה יין נסך עד שירד לבור לימא תלתא תנאי היא לא שאני יין נסך דאחמירו ביה רבנן
\par ולרבא דלא שאני ליה מוקים ליה כתלתא תנאי:}
\twocol{מה שבבור אסור והשאר מותר: אמר רב הונא לא שנו אלא שלא החזיר גרגותני לגת אבל החזיר גרגותני לגת אסור
\par גרגותני גופה במאי קא מיתסרא בנצוק ש"מ נצוק חיבור כדתני ר' חייא שפחסתו צלוחיתו ה"נ שפחסתו בורו}
\twocol{ההוא ינוקא דתנא עבודת כוכבים בשית שני בעו מיניה מהו לדרוך עם העובד כוכבים בגת אמר להו דורכין עם העובד כוכבים בגת והא קא מנסך בידיה דציירנא להו לידיה והא קא מנסך ברגל ניסוך דרגל לא שמיה ניסוך
\par ההוא עובדא דהוה בנהרדעא דדשו ישראל ועובד כוכבים לההוא חמרא ושהייה שמואל תלתא ריגלי מ"ט אילימא משום דקסבר}
\newsection{דף נז}
\twocol{דאי משכחנא תנא דאסר כרבי נתן אוסריניה אפי' בהנאה דתניא מדדו בין ביד בין ברגל ימכר ר' נתן אומר ביד אסור ברגל מותר
\par אימר דאמר ר' נתן ביד ברגל מי אמר אלא דאי משכחנא תנא דשרי כר"ש אישרייה אפי' בשתייה}
\twocol{ההוא עובדא דהוה בבירם דההוא עובד כוכבים דהוה קא סליק בדיקלא ואייתי לוליב' בהדי דקא נחית נגע בראשה דלוליבא בחמרא שלא בכוונה שרייה רב לזבוניה לעובדי כוכבים
\par אמרו ליה רב כהנא ורב אסי לרב והא מר הוא דאמר תינוק בן יומו הוא עושה יין נסך אמר להו אימור דאמרי אנא בשתייה בהנאה מי אמרי}
\twocol{גופא אמר רב תינוק בן יומו עושה יין נסך
\par איתיביה רב שימי בר חייא לרב הלוקח עבדים מן העובדי כוכבים שמלו ולא טבלו וכן בני השפחות שמלו ולא טבלו רוקן ומדרסן בשוק טמא ואמרי לה טהור}
\twocol{יינן גדולים עושים יין נסך קטנים אין עושים יין נסך ואלו הן גדולים ואלו הן קטנים גדולים יודעין בטיב עבודת כוכבים ומשמשיה קטנים אינם יודעין בטיב עבודת כוכבים ומשמשיה
\par קתני מיהת גדולים אין קטנים לא תרגמה אבני שפחות}
\twocol{הא וכן קאמר ארוקן ומדרסן
\par הניחא למאן דאמר טמא אלא למ"ד טהור מאי איכא למימר}
\twocol{הא קמ"ל עבדים דומיא דבני שפחות מה בני שפחות מלו ולא טבלו הוא דעושין יין נסך מלו וטבלו לא אף עבדים כן
\par לאפוקי מדרב נחמן אמר שמואל דאמר רב נחמן אמר שמואל הלוקח עבדים מן העובדי כוכבים אע"פ שמלו וטבלו עושין יין נסך עד שתשקע עבודת כוכבים מפיהם קמ"ל דלא}
\twocol{גופא אר"נ אמר שמואל הלוקח עבדים מן העובדי כוכבים אע"פ שמלו וטבלו עושין יין נסך עד שתשקע עבודת כוכבים מפיהם וכמה א"ר יהושע בן לוי עד שנים עשר חדש
\par איתיביה רבה לר"נ הלוקח עבדים מן העובדי כוכבים שמלו ולא טבלו וכן בני השפחות שמלו ולא טבלו רוקן ומדרסן}
\twocol{בשוק טמא ואמרי לה טהור יינן גדולים עושין יין נסך קטנים אין עושין יין נסך אלו הן גדולים ואלו הן קטנים גדולים שיודעין בטיב עבודת כוכבים ומשמשיה קטנים שאין יודעין בטיב עבודת כוכבים ומשמשיה
\par קתני מיהת מלו ולא טבלו אין מלו וטבלו לא תרגמה אבני שפחות}
\twocol{הא וכן קתני ארוקן ומדרסן
\par הניחא למאן דאמר טמא אלא למ"ד טהור מאי איכא למימר}
\twocol{הא קמ"ל עבדים דומיא דבני שפחות מה בני שפחות גדולים הוא דעושין יין נסך קטנים אין עושין יין נסך אף עבדים נמי גדולים עושין יין נסך קטנים אין עושין יין נסך
\par לאפוקי מדרב דאמר רב תינוק בן יומו עושה יין נסך קמ"ל דלא}
\twocol{ההוא עובדא דהוה במחוזא אתא עובד כוכבים עייל לחנותא דישראל אמר להו אית לכו חמרא לזבוני אמרו ליה לא הוה יתיב חמרא בדוולא שדי ביה ידיה שיכשך ביה אמר להו האי לאו חמרא הוא שקליה האיך בריתחיה שדייה לדנא
\par שרייה רבא לזבוני לעובדי כוכבים איפליג עליה רב הונא בר חיננא ורב הונא בריה דרב נחמן נפקי שיפורי דרבא ושרו ונפקי שיפורי דרב הונא בר חיננא ור"ה בר ר"נ ואסרי}
\newsection{דף נח}
\twocol{איקלע רב הונא בריה דר"נ למחוזא א"ל רבא לרב אליקים שמעיה טרוק טרוק גלי דלא ניתו אינשי דניטריד
\par על לגביה א"ל כי האי גוונא מאי א"ל אסור אפילו בהנאה והא מר הוא דאמר שיכשך אין עושה יין נסך אימר דאמרי אנא לבר מדמיה דההוא חמרא דמי דההוא חמרא מי אמרי}
\twocol{אמר רבא כי אתאי לפומבדיתא אקפן נחמני שמעתתא ומתניתא דאסיר
\par שמעתתא דההוא עובדא דהוה בנהרדעא ואסר שמואל בטבריא ואסר רבי יוחנן ואמרי ליה לפי שאינן בני תורה ואמר לי טבריא ונהרדעא אינן בני תורה דמחוזא בני תורה}
\twocol{מתניתא דאגרדמים עובד כוכבים שקדח במינקת והעלה או שטעם מן הכוס והחזירו לחבית זה היה מעשה ואסרוהו מאי לאו בהנאה לא בשתייה
\par אי הכי ליתני ימכר כדקתני סיפא חרם עובד כוכבים שהושיט ידו לחבית וכסבור של שמן היא ונמצאת של יין זה היה מעשה ואמרו ימכר תיובתא דרבא תיובתא}
\twocol{רבי יוחנן בן ארזא ור' יוסי בן נהוראי הוו יתבו וקא שתו חמרא אתא ההוא גברא אמרו ליה תא אשקינן לבתר דרמא לכסא איגלאי מילתא דעובד כוכבים הוא חד אסר אפי' בהנאה וחד שרי אפי' בשתייה אמר רבי יהושע בן לוי מאן דאסר שפיר אסר ומאן דשרי שפיר שרי מאן דאסר
\par מימר אמר סלקא דעתיה דרבנן כי הני שיכרא קא שתו אלא ודאי האי חמרא הוא ונסכיה מאן דשרי שפיר שרי מימר אמר ס"ד דרבנן כי הני חמרא קא שתו וא"ל לדידי תא אשקינן אלא ודאי שיכרא הוא קא שתו ולא נסכיה}
\twocol{והא קא חזי בליליא והא קא מרח ליה בחדתא
\par והא קא נגע ביה בנטלא וה"ל מגע עובד כוכבים שלא בכוונה ואסור לא צריכא דקא מוריק אורוקי וה"ל כחו שלא בכוונה וכל כחו שלא בכוונה לא גזרו ביה רבנן}
\twocol{בעא מיניה ר' אסי מר' יוחנן יין שמסכו עובד כוכבים מהו א"ל ואימא מזגו א"ל אנא כדכתיב קאמינא (משלי ט, ב) טבחה טבחה מסכה יינה א"ל לשון תורה לעצמה לשון חכמים לעצמו
\par מאי א"ל אסור משום לך לך אמרין נזירא סחור סחור לכרמא לא תקרב}
\twocol{רבי ירמיה איקלע לסבתא חזא חמרא דמזגי עובד כוכבים ואישתי ישראל מיניה ואסר להו משום לך לך אמרין נזירא סחור סחור לכרמא לא תקרב אתמר נמי א"ר יוחנן ואמרי לה א"ר אסי א"ר יוחנן יין שמזגו עובד כוכבים אסור משום לך לך אמרין נזירא סחור סחור לכרמא לא תקרב
\par ר"ל איקלע לבצרה חזא ישראל דקאכלי פירי דלא מעשרי ואסר להו חזא מיא דסגדי להו עובדי כוכבים ושתו ישראל ואסר להו}
\twocol{אתא לקמיה דרבי יוחנן א"ל אדמקטורך עלך זיל הדר בצר לאו היינו בצרה ומים של רבים אין נאסרין
\par רבי יוחנן לטעמיה}
\newsection{דף נט}
\twocol{דא"ר יוחנן משום ר"ש בן יהוצדק מים של רבים אין נאסרין הא דיחיד נאסרין
\par ותיפוק ליה דהא מחוברין נינהו לא צריכא דתלשינהו גלא}
\twocol{סוף סוף אבני הר שנדלדלו נינהו תסתיים דר' יוחנן דאמר אסורות
\par לא צריכא דטפחינהו בידיה}
\twocol{ר' חייא בר אבא איקלע לגבלא חזא בנות ישראל דמיעברן מעובדי כוכבים שמלו ולא טבלו חזא חמרא דמזגו עובדי כוכבים ושתו ישראל חזא תורמוסא דשלקי להו עובדי כוכבים ואכלי ישראל ולא אמר להו ולא מידי
\par אתא לקמיה דרבי יוחנן א"ל צא והכרז על בניהם שהן ממזרים ועל יינן משום יין נסך ועל תורמוסן משום בישולי עובדי כוכבים משום שאינן בני תורה}
\twocol{על בניהם שהם ממזרים ר' יוחנן לטעמיה דא"ר יוחנן לעולם אינו גר עד שימול ויטבול וכיון דלא טביל עובד כוכבים הוא ואמר רבה בר בר חנה א"ר יוחנן עובד כוכבים ועבד הבא על בת ישראל הולד ממזר
\par וגזור על יינם משום יין נסך משום לך לך אמרין נזירא סחור סחור לכרמא לא תקרב}
\twocol{ועל תורמוסן משום בישולי עובדי כוכבים לפי שאינן בני תורה טעמא דאינן בני תורה הא בני תורה שרי והאמר רב שמואל בר רב יצחק אמר רב כל שנאכל כמות שהוא חי אין בו משום בישולי עובדי כוכבים
\par ר' יוחנן כי הך לישנא ס"ל דאמר רב שמואל בר רב יצחק אמר רב כל שאינו עולה לשולחן של מלכים ללפת בו את הפת אין בו משום בישולי עובדי כוכבים טעמא דאינן בני תורה הא בני תורה שרי}
\twocol{בעו מיניה מרב כהנא עובד כוכבים מהו שיוליך ענבים לגת אמר להו אסור משום לך לך אמרין נזירא סחור סחור לכרמא לא תקרב איתיביה רב יימר לרב כהנא עובד כוכבים שהביא ענבים לגת בסלין
\par ובדודורין אע"פ שהיין מזלף עליהן מותר א"ל הביא קאמרת אנא לכתחלה קאמינא}
\twocol{ההוא אתרוגא דנפל לחביתא דחמרא אידרי עובד כוכבים ושקליה אמר להו רב אשי נקטוה לידיה כי היכי דלא לשכשיך ביה וברצוה עד דשייפא
\par אמר רב אשי האי עובד כוכבים דנסכיה לחמרא דישראל בכוונה אע"ג דלזבוניה לעובד כוכבים אחרינא אסור שרי ליה למישקל דמיה מההוא עובד כוכבים מאי טעמא מיקלא קלייה}
\twocol{אמר רב אשי מנא אמינא לה דתניא עובד כוכבים שנסך יינו של ישראל שלא בפני עבודת כוכבים אסור ורבי יהודה בן בבא ורבי יהודה בן בתירא מתירין משום שני דברים אחד שאין מנסכין יין אלא בפני עבודת כוכבים ואחד שאומר לו לא כל הימנך שתאסור ייני לאונסי
\par ההיא חביתא דחמרא דאישתקיל לברזא אתא עובד כוכבים אידרי אנח ידיה עילויה אמר רב פפא כל דלהדי ברזא חמרא אסיר}
\newsection{דף ס}
\twocol{ואידך שרי ואיכא דאמרי אמר רב פפא עד הברזא חמרא אסיר ואידך שרי
\par אמר רב יימר כתנאי חבית שנקבה בין מפיה בין משוליה ובין מצידיה ונגע בו טבול יום טמאה רבי יהודה אומר מפיה ומשוליה טמאה מצידיה טהורה מכאן ומכאן}
\twocol{אמר רב פפא עובד כוכבים אדנא וישראל אכובא חמרא אסיר מ"ט כי קאתי מכח עובד כוכבים קאתי ישראל אדנא ועובד כוכבים אכובא חמרא שרי ואי מצדד צדודי אסיר
\par אמר רב פפא האי עובד כוכבים דדרי זיקא וקאזיל ישראל אחוריה מליא שרי דלא מקרקש חסירא אסיר דלמא מקרקש כובא מליא אסיר דלמא נגע חסירא שרי דלא נגע}
\twocol{רב אשי אמר זיקא בין מליא ובין חסירא שרי מ"ט אין דרך ניסוך בכך
\par מעצרא זיירא רב פפי שרי רב אשי ואיתימא רב שימי בר אשי אסר}
\twocol{בכחו כולי עלמא לא פליגי דאסיר כי פליגי בכח כחו איכא דאמרי בכח כחו כולי עלמא לא פליגי דשרי כי פליגי בכחו הוה עובדא בכח כחו ואסר רב יעקב מנהר פקוד
\par ההוא חביתא}
\twocol{דאיפקעה לאורכה אידרי ההוא עובד כוכבים חבקה שרייה רפרם בר פפא ואי תימא רב הונא בריה דרב יהושע לזבוני לעובדי כוכבים וה"מ דפקעה לאורכה אבל לפותייה אפילו בשתיה שרי מ"ט מעשה לבינה קעביד
\par ההוא עובד כוכבים דאשתכח דהוה קאי במעצרתא אמר רב אשי אי איכא טופח להטפיח בעי הדחה ובעי ניגוב ואי לא בהדחה בעלמא סגי ליה:}
\twocol{{\large\emph{מתני׳}} עובד כוכבים שנמצא עומד בצד הבור של יין אם יש לו מלוה עליו אסור אין לו מלוה עליו מותר
\par נפל לבור ועלה מדדו בקנה התיז את הצרעה בקנה או שהיה מטפיח ע"פ חבית מרותחת בכל אלו היה מעשה ואמרו ימכר ור"ש מתיר נטל את החבית וזרקה בחמתו לבור זה היה מעשה והכשירו:}
\twocol{{\large\emph{גמ׳}} אמר שמואל והוא שיש לו מלוה על אותו יין
\par אמר רב אשי מתני' נמי דיקא דתנן המטהר יינו של עובד כוכבים ונותנו ברשותו והלה כותב לו התקבלתי ממך מעות מותר אבל אם ירצה ישראל להוציאו ואין מניחו עד שיתן לו מעותיו זה היה מעשה בבית שאן ואסרו}
\twocol{טעמא דאין מניחו הא מניחו שרי ש"מ מלוה על אותו יין בעינן ש"מ:
\par נפל לבור ועלה: אמר רב פפא לא שנו אלא שעלה מת אבל עלה חי אסור מ"ט אמר רב פפא דדמי עליה כיום אידם:}
\twocol{מדדו בקנה כל אלו היה מעשה ואמרו ימכר ור"ש מתיר: אמר רב אדא בר אהבה ינוחו לו לר"ש ברכות על ראשו כשהוא מתיר מתיר אפילו בשתיה וכשהוא אוסר אוסר אפילו בהנאה
\par א"ר חייא בריה דאבא בר נחמני אמר רב חסדא אמר רב ואמרי לה אמר רב חסדא אמר זעירי הלכה כר"ש איכא דאמרי אמר רב חסדא אמר לי אבא בר חנן הכי אמר זעירי הלכה כר"ש ואין הלכה כר"ש:}
\twocol{נטל חבית וזרקה [בחמתו] לבור זה היה מעשה [והכשירו]: אמר רב אשי כל שבזב טמא בעובד כוכבים עושה יין נסך כל שבזב טהור בעובד כוכבים אינו עושה יין נסך
\par איתיביה רב הונא לרב אשי נטל את החבית וזרקה בחמתו לבור זה היה מעשה בבית שאן והכשירו בחמתו אין שלא בחמתו לא}
\newsection{דף סא}
\twocol{התם דקאזיל מיניה ומיניה:
\par {\large\emph{מתני׳}} המטהר יינו של עובד כוכבים ונותנו ברשותו (ו)בבית הפתוח לרשות הרבים בעיר שיש בה עובדי כוכבים וישראלים מותר בעיר שכולה עובדי כוכבים אסור עד שישב ומשמר}
\twocol{ואין השומר צריך להיות יושב ומשמר אע"פ שהוא יוצא ונכנס מותר ר"ש בן אלעזר אומר רשות עובדי כוכבים אחת היא:
\par המטהר יינו של עובד כוכבים ונותנו ברשותו והלה כותב לו התקבלתי ממך מעות מותר אבל אם ירצה ישראל להוציא ואינו מניחו עד שיתן לו את מעותיו זה היה מעשה בבית שאן ואסרו:}
\twocol{{\large\emph{גמ׳}} בעיר שכולה עובדי כוכבים נמי והאיכא רוכלין המחזירין בעיירות אמר שמואל בעיר שיש לה דלתים ובריח
\par אמר רב יוסף וחלון כרה"ר דמי ואשפה כרה"ר דמי ודיקלא כרה"ר דמי}
\twocol{פסיק רישיה פליגי בה רב אחא ורבינא חד אסר וחד שרי מאן דאסר למה ליה דסליק התם ומאן דשרי זימנא דאבדה ליה בהמה וסליק לעיוני בתרה:
\par ת"ר אחד הלוקח ואחד השוכר בית בחצירו של עובד כוכבים ומילאהו יין וישראל דר באותה חצר מותר ואף על פי שאין מפתח וחותם בידו}
\twocol{בחצר אחרת מותר והוא שמפתח וחותם בידו
\par המטהר יינו של עובד כוכבים ברשותו וישראל דר באותה חצר מותר והוא שמפתח וחותם בידו א"ל רבי יוחנן לתנא תני אע"פ שאין מפתח וחותם בידו מותר}
\twocol{בחצר אחרת אסור אע"פ שמפתח וחותם בידו דברי ר"מ
\par וחכמים אוסרין עד שיהא שומר יושב ומשמר או עד שיבא ממונה הבא לקיצין}
\twocol{חכמים אהייא אילימא אסיפא תנא קמא נמי מיסר קא אסר ואלא ארישא דסיפא והא קאמר ליה ר' יוחנן לתנא תני אע"פ שאין מפתח וחותם בידו
\par ואלא אסיפא דרישא דקאמר ת"ק בחצר אחרת מותר והוא שמפתח וחותם בידו וחכמים אומרים לעולם אסור עד שיהא שומר יושב ומשמר או עד שיבא ממונה הבא לקיצין}
\twocol{ממונה בא לקיצין גריעותא הוא אלא עד שיבא ממונה שאינו בא לקיצין:
\par רשב"א אומר רשות עובדי כוכבים אחת היא: איבעיא להו ר"ש בן אלעזר להקל או להחמיר רב יהודה אמר זעירי להקל רב נחמן אמר זעירי להחמיר}
\twocol{רב יהודה אמר זעירי להקל והכי קאמר ת"ק כשם שברשותו אסור כך ברשות עובד כוכבים אחר נמי אסור וחיישינן לגומלין
\par ר"ש בן אלעזר אומר במה דברים אמורים ברשותו אבל ברשות עובד כוכבים אחר מותר ולא חיישינן לגומלין}
\twocol{רב נחמן אמר זעירי להחמיר וה"ק ת"ק במה דברים אמורים ברשותו אבל ברשות עובד כוכבים אחר מותר ולא חיישינן לגומלין ר"ש בן אלעזר אומר כל רשות עובדי כוכבים אחת היא
\par תניא כוותיה דרב נחמן אמר זעירי להחמיר אמר ר"ש בן אלעזר כל רשות עובדי כוכבים אחת היא מפני הרמאין}
\twocol{דבי פרזק רופילא אותיבו חמרא גבי אריסייהו סבור רבנן קמיה דרבא למימר כי חיישינן לגומלין הני מילי היכא דקא מותיב האי גבי האי אבל הכא כיון דאריסיה לאו דרכיה לאותוביה בי פרזק רופילא לגומלין לא חיישינן
\par אמר להו רבא אדרבה אפילו למ"ד לא חיישינן לגומלין ה"מ היכא דלא מירתת מיניה אבל הכא כיון דמירתת מיניה מחפי עליה זכותא}
\twocol{ההוא כרכא דהוה יתיב ביה חמרא דישראל אשתכח עובד כוכבים דהוה קאי ביני דני אמר רבא אם נתפס עליו כגנב חמרא שרי ואי לא אסור:
\par \par \par {\large\emph{הדרן עלך רבי ישמעאל}}\par \par }
\newchap{פרק \hebrewnumeral{5}\quad השוכר את הפועל}
\newsection{דף סב}
\twocol{
\par מתני׳ {\large\emph{השוכר}} את הפועל לעשות עמו ביין נסך שכרו אסור שכרו לעשות עמו מלאכה אחרת אע"פ שאמר לו העבר לי חבית של יין נסך ממקום למקום שכרו מותר השוכר את החמור להביא עליה יין נסך שכרה אסור שכרה לישב עליה אע"פ שהניח עובד כוכבים לגינו עליה שכרה מותר:}
\twocol{{\large\emph{גמ׳}} מ"ט שכרו אסור אילימא הואיל ויין נסך אסור בהנאה שכרו נמי אסור הרי ערלה וכלאי הכרם דאסורין בהנאה ותנן מכרן וקידש בדמיהן מקודשת
\par אלא הואיל ותופס את דמיו כעבודת כוכבים והרי שביעית דתופס' את דמיה ותנן האומר לפועל הילך דינר זה לקוט לי בו ירק היום שכרו אסור לקוט לי ירק היום שכרו מותר}
\twocol{א"ר אבהו א"ר יוחנן קנס הוא שקנסו חכמים בחמרין וביין נסך יין נסך הא דאמרן חמרין מאי היא דתניא החמרין שהיו עושין מלאכה בפירות שביעית שכרן שביעית
\par מאי שכרן שביעית אילימא דיהבינן להו שכר מפירות שביעית נמצא זה פורע חובו מפירות שביעית והתורה אמרה (ויקרא כה, ו) לאכלה ולא לסחורה}
\twocol{ואלא דקדוש שכרן בקדושת שביעית ומי קדוש והתניא האומר לפועל הילך דינר זה ולקוט לי ירק היום שכרו מותר לקוט לי ירק בו היום שכרו אסור
\par }
\twocol{אמר אביי לעולם יהבינן ליה שכר מפירות שביעית ודקא קשיא לך לאכלה ולא לסחורה דיהביה ניהליה בצד היתר כדתנן לא יאמר אדם לחבירו}
\newchap{פרק \hebrewnumeral{5}\quad השוכר את הפועל}
\twocol{
\par העלה לי פירות הללו לירושלים לחלק אבל אומר לו העלם לאוכלם ולשתותם בירושלים ונותנין זה לזה מתנה של חנם}
\twocol{ורבא אמר לעולם דקדוש בקדושת שביעית ודקא קשיא לך פועל פועל דלא נפיש אגריה לא קנסוהו רבנן חמרין דנפיש אגרייהו קנסו רבנן בהו ומתני' חומרא דיין נסך שאני:
\par איבעיא להו שכרו לסתם יינן מהו מי אמרי' כיון דאיסורא חמור כדיין נסך שכרו נמי אסור או דלמא הואיל וטומאתו קיל אף שכרו נמי קיל}
\twocol{תא שמע דההוא גברא דאגר ארביה לסתם יינן יהבו ליה חיטי באגרא אתא לקמיה דרב חסדא א"ל זיל קלינהו וקברינהו בקברי
\par ולימא ליה בדרינהו אתו בהו לידי תקלה וליקלינהו וליבדרינהו דלמא מזבלי בהו}
\twocol{ולקברינהו בעינייהו מי לא תנן אחד אבן שנסקל בה ואחד עץ שנתלה עליו ואחד סייף שנהרג בו ואחד סודר שנחנק בו כולם נקברים עמו
\par התם דקא קברי בבי דינא מוכחא מילתא דהרוגי בית דין נינהו הכא לא מוכחא מילתא אימר אינש גנב ואייתי קברא הכא}
\twocol{דבי רבי ינאי יזפי פירי שביעית מעניים ופרעו להו בשמינית אתו אמרו ליה לרבי יוחנן אמר להו יאות הן עבדין
\par וכנגדן באתנן מותר דתניא נתן לה ולא בא עליה בא עליה ולא נתן לה אתננה מותר}
\twocol{נתן לה ולא בא עליה פשיטא כיון דלא בא עליה מתנה בעלמא הוא דיהיב לה ותו בא עליה ולא נתן לה הא לא יהיב לה ולא מידי וכיון דלא נתן לה מאי אתננה מותר
\par אלא הכי קאמר נתן לה ואחר כך בא עליה או בא עליה ואחר כך נתן לה אתננה מותר}
\twocol{נתן לה ואחר כך בא עליה לכי בא עליה}
\newsection{דף סג}
\twocol{ליחול עלה איסור אתנן למפרע אמר רבי אליעזר כשקדמה והקריבתו
\par ה"ד אי דאמר לה קני ליך מעכשיו פשיטא דשרי דהא ליתיה בשעת ביאה ומתנה בעלמא הוא דיהיב לה}
\twocol{ואי דלא אמר לה קני ליך מעכשיו היכי מצי מקרבה (ויקרא כז, יד) ואיש כי יקדיש את ביתו קדש אמר רחמנא מה ביתו ברשותו אף כל ברשותו
\par אלא דאמר לה להוי גביך עד שעת ביאה ואי מיצטריך ליך קני מעכשיו}
\twocol{בעי רב הושעיא קדמה והקדישתו מהו כיון דאמר מר אמירתו לגבוה כמסירתו להדיוט כמאן דאקריבתיה דמי או דלמא השתא מיהא הא קאי ואיתיה בעיניה
\par ותפשוט מדרבי אליעזר דא"ר אליעזר שקדמה והקריבתו דוקא הקריבתו אבל הקדישתו לא}
\twocol{דרבי אליעזר גופיה קא מיבעיא ליה מאי מיפשט פשיטא ליה לרבי אליעזר דהקריבתו דוקא אבל הקדישתו לא דהא איתיה בשעת ביאה או דלמא הקריבתו פשיטא ליה והקדישתו מספקא ליה תיקו:
\par בא עליה ואחר כך נתן לה אתננה מותר: ורמינהי בא עליה ואחר כך נתן לה אפילו מכאן עד שלש שנים אתננה אסור}
\twocol{אר"נ בר יצחק אמר רב חסדא לא קשיא הא דאמר התבעלי לי בטלה זה הא דאמר לה התבעלי בטלה סתם
\par וכי אמר לה בטלה זה מאי הוי הא מחסר משיכה בזונה עובדת כוכבים דלא קניא במשיכה ואיבעית אימא לעולם בזונה ישראלית וכגון דקאי בחצירה}
\twocol{אי דקאי בחצירה בא עליה ואח"כ נתן לה הא קניא לה לא צריכא דשויה ניהלה אפותיקי דאמר לה אי מייתינא ליך זוזי מכאן עד יום פלוני מוטב ואי לא שקליה באתנניך
\par מתיב רב ששת אומר אדם לחמריו ולפועליו לכו ואכלו בדינר זה צאו ושתו בדינר זה ואינו חושש}
\twocol{לא משום שביעית ולא משום מעשר ולא משום יין נסך
\par ואם אמר להם צאו ואכלו ואני פורע צאו ושתו ואני פורע חושש משום שביעית ומשום מעשר ומשום יין נסך}
\twocol{אלמא כי קא פרע דמי איסור קא פרע הכא נמי כי קא פרע דמי איסורא קא פרע
\par תרגמה רב חסדא בחנוני המקיפו דמשתעבד ליה דכיון דאורחיה לאקופי קני ליה דינר גביה}
\twocol{אבל חנוני שאין מקיפו מאי מותר אי הכי אדתני צאו ואכלו בדינר זה צאו ושתו בדינר זה ליפלוג וליתני בדידה
\par במה דברים אמורים בחנוני המקיפו דמשתעבד ליה אבל חנוני שאין מקיפו מותר}
\twocol{ועוד חנוני שאין מקיפו מי לא משתעבד והאמר רבא האומר לחבירו תן מנה לפלוני ויקנו כל נכסאי לך קנה מדין ערב
\par אלא אמר רבא לא שנא מקיפו ולא שנא שאין מקיפו אע"ג דמשעבד ליה כיון דלא מייחד שיעבודיה לא מיתסר}
\twocol{אלא הכא אמאי חושש משום שביעית הא לא מייחד שיעבודיה הכא אמר רב פפא כגון שהקדים לו דינר
\par אמר רב כהנא אמריתה לשמעתא קמיה דרב זביד מנהרדעא א"ל אי הכי אדתני צאו ואכלו צאו ושתו ואני פורע צאו ואכלו צאו ושתו ואני מחשב מיבעי ליה א"ל תני צאו ואני מחשב}
\twocol{רב אשי אמר כגון שנטל ונתן ביד א"ל רב יימר לרב אשי אי הכי אדתני צאו ואכלו צאו ושתו טלו ואכלו טלו ושתו מיבעי ליה א"ל תני טלו ואכלו טלו ושתו
\par יתיב רב נחמן ועולא ואבימי בר פפי ויתיב רבי חייא בר אמי גבייהו ויתבי וקא מיבעיא להו שכרו לשבור ביין נסך מהו מי אמרינן כיון דרוצה בקיומו אסור או דלמא כל למעוטי תיפלה שפיר דמי}
\twocol{אר"נ ישבור ותבא עליו ברכה לימא מסייע ליה אין עודרין עם העובד כוכבים בכלאים}
\newsection{דף סד}
\twocol{אבל עוקרין עמו כדי למעוטי את התיפלה
\par סברוה הא מני ר' עקיבא היא דאמר המקיים בכלאים לוקה דתניא המנכש והמחפה בכלאים לוקה ר"ע אומר אף המקיים}
\twocol{מ"ט דר"ע אמר קרא שדך לא תזרע כלאים אין לי אלא זורע מקיים מנין ת"ל לא כלאים
\par ואילו למעוטי תיפלה שרי}
\twocol{לא הא מני רבנן היא
\par אי רבנן מאי איריא עוקרין אפי' קיומי נמי שפיר דמי הכא במאי עסקינן כגון דקא עביד בחנם ור' יהודה היא דאמר ליתן להם מתנת חנם אסור}
\twocol{מדרבי יהודה נשמע לר"ע לאו אמר ר' יהודה אסור ליתן להם מתנת חנם אבל למעוטי תיפלה שפיר דמי לר"ע נמי אע"ג דא"ר עקיבא המקיים בכלאים לוקה למעוטי תיפלה שפיר דמי ותו לא מידי
\par הדור יתבי וקמבעיא להו דמי עבודת כוכבים ביד עובד כוכבים מהו מי תופסת דמיה ביד עובד כוכבים או לא}
\twocol{אמר להו רב נחמן מסתברא דמי עבודת כוכבים ביד עובד כוכבים מותרין מדהנהו דאתו לקמיה דרבה בר אבוה אמר להו זילו זבינו כל מה דאית לכו ותו איתגיירו
\par מ"ט משום דקסבר דמי עבודת כוכבים ביד עובד כוכבים מותרין ודלמא שאני התם דכיון דדעתיה לאיגיורי ודאי בטלה}
\twocol{אלא מהכא ישראל שהיה נושה בעובד כוכבים מנה ומכר עבודת כוכבים והביא לו יין נסך והביא לו מותר אבל אם אמר לו המתן לי עד שאמכור עבודת כוכבים ואביא לך יין נסך ואביא לך אסור
\par מאי שנא רישא ומאי שנא סיפא אמר רב ששת סיפא משום דהוה ליה כי רוצה בקיומו}
\twocol{וכי רוצה בקיומו כה"ג מי אסיר והתנן גר ועובד כוכבים שירשו אביהן עובד כוכבים גר יכול לומר לו טול אתה עבודת כוכבים ואני מעות טול אתה יין נסך ואני פירות אם משבאו לרשות הגר אסור
\par אמר רבא בר עולא מתני' בעבודת כוכבים המתחלקת לפי שבריה}
\twocol{תינח עבודת כוכבים יין נסך מאי איכא למימר בחרס הדרייני
\par והלא רוצה בקיומו שלא יגנובו ושלא יאבדו אלא א"ר פפא ירושת הגר קאמרת שאני ירושת הגר דאקילו בה רבנן גזירה שמא יחזור לקלקולו}
\twocol{תניא נמי הכי בד"א שירשו אבל נשתתפו אסור
\par הדור יתבו וקמיבעיא להו גר תושב מהו שיבטל עבודת כוכבים דפלח מבטיל דלא פלח לא מבטיל או דלמא כל דבר מיני' מבטיל והאי בר מיניה הוא}
\twocol{אמר להו רב נחמן מסתברא דפלח מבטיל דלא פלח לא מבטיל
\par מיתיבי ישראל שמצא עבודת כוכבים בשוק עד שלא באתה לידו אומר לעובד כוכבים ומבטלה משבאתה לידו אינו אומר לעובד כוכבים ומבטלה מפני שאמרו עובד כוכבים מבטל עבודת כוכבים שלו ושל חבירו בין עובדה ובין שאין עובדה}
\twocol{מאי עובדה ומאי שאינו עובדה אילימא אידי ואידי עובד כוכבים היינו שלו ושל חבירו אלא לאו עובדה עובד כוכבים ומאי שאינו עובדה גר תושב וש"מ גר תושב נמי מבטל
\par לא לעולם אימא לך אידי ואידי עובד כוכבים ודקאמרת היינו שלו ושל חבירו רישא זה וזה לפעור וזה וזה למרקוליס סיפא זה לפעור וזה למרקוליס}
\twocol{מיתיבי איזהו גר תושב כל שקיבל עליו בפני ג' חברים שלא לעבוד עבודת כוכבים דברי ר"מ
\par וחכ"א כל שקיבל עליו שבע מצות שקבלו עליהם בני נח}
\twocol{אחרים אומרים אלו לא באו לכלל גר תושב אלא איזהו גר תושב זה גר אוכל נבילות שקבל עליו לקיים כל מצות האמורות בתורה חוץ מאיסור נבילות
\par מייחדין אצלו יין ואין מפקידין אצלו יין ואפי' בעיר שרובה ישראל אבל מייחדין אצלו יין ואפי' בעיר שרובה עובדי כוכבים שמנו כיינו}
\twocol{שמנו כיינו ס"ד שמן מי קא הוי יין נסך אלא יינו כשמנו
\par ולשאר כל דבר הרי הוא כעובד כוכבים רבן שמעון אומר יינו יין נסך ואמרי לה מותר בשתיה}
\twocol{קתני מיהא ולשאר כל דבריו הרי הוא כעובד כוכבים למאי הלכתא לאו דמבטל עבודת כוכבים כעובד כוכבים אר"נ בר יצחק לא ליתן רשות ולבטל רשות
\par וכדתניא ישראל מומר משמר שבתו בשוק מבטל רשות שאין משמר שבתו בשוק אין מבטל רשות מפני שאמרו ישראל נותן רשות ומבטל רשות}
\twocol{ובעובד כוכבים עד שישכור כיצד אומר לו רשותי קנויה לך רשותי מבוטלת לך קנה ואין צריך לזכות
\par רב יהודה שדר ליה קורבנא}
\newsection{דף סה}
\twocol{לאבידרנא ביום אידם אמר ידענא ביה דלא פלח לעבודת כוכבים א"ל רב יוסף והתניא איזהו גר תושב כל שקיבל עליו בפני ג' חברים שלא לעבוד עבודת כוכבים כי תניא ההיא להחיותו
\par והאמר רבה בר בר חנה א"ר יוחנן גר תושב שעברו עליו י"ב חדש ולא מל הרי הוא כמין שבעובדי כוכבים התם כגון שקיבל עליו למול ולא מל}
\twocol{רבא אמטי ליה קורבנא לבר שישך ביום אידם אמר ידענא ביה דלא פלח לעבודת כוכבים אזל אשכחיה דיתיב עד צואריה בוורדא וקיימן זונות ערומות קמיה א"ל אית לכו כה"ג לעלמא דאתי א"ל דידן עדיפא טפי מהאי א"ל טפי מהאי מי הוה א"ל אתון איכא עלייכו אימתא דמלכותא אנן לא תיהוי עלן אימתא דמלכותא א"ל אנא מיהא מאי אימתא דמלכותא איכא עלי
\par עד דיתבי אתא ההוא פריסתקא דמלכא א"ל קום דקבעי לך מלכא כי נפיק ואזיל א"ל עינא דבעי למיחזי לכו בישותא תיפקע א"ל רבא אמן פקע עיניה דבר שישך}
\twocol{אמר רב פפי איבעי ליה למימרא ליה מהאי קרא (תהלים מה, י) בנות מלכים ביקרותיך נצבה שגל לימינך בכתם אופיר אמר ר"נ בר יצחק איבעי ליה למימרא ליה מהכא (ישעיהו סד, ג) עין לא ראתה אלהים זולתך יעשה למחכה לו:
\par שכרו לעשות עמו מלאכה אחרת: ואע"ג דלא א"ל לעיתותי ערב}
\twocol{ורמינהי השוכר את הפועל ולעיתותי ערב אמר לו העבר חבית של יין נסך ממקום למקום שכרו מותר טעמא דא"ל לעיתותי ערב אין כולי יומא לא
\par אמר אביי כי תנן נמי מתניתין דאמר לעיתותי ערב תנן רבא אמר ל"ק הא דאמר ליה העבר לי מאה חביות במאה פרוטות הא דא"ל העבר לי חבית חבית בפרוטה}
\twocol{והתניא השוכר את הפועל ואמר לו העבר לי מאה חביות במאה פרוטות ונמצאת חבית של יין נסך ביניהן שכרו אסור חבית חבית בפרוטה ונמצאת חבית של יין נסך ביניהן שכרו מותר:
\par השוכר את החמור להביא עליה יין נסך שכרו אסור: הא תו ל"ל היינו רישא סיפא איצטריכא ליה שכרה לישב עליה אע"פ שהניח עובד כוכבים לגינו עליה שכרו מותר}
\twocol{למימרא דלגין לאו דינא הוא לאותוביה
\par ורמינהי השוכר את החמור שוכר מניח עליה כסותו ולגינתו ומזונותיו של אותו הדרך מכאן ואילך חמר מעכב עליו חמר מניח עליה שעורים ותבן ומזונותיו של אותו היום מכאן ואילך שוכר מעכב עליו}
\twocol{אמר אביי נהי דלגין דינא הוא לאותובי מיהא אי לא מותיב ליה מי אמרינן ליה נכי ליה אגרא דלגינתו
\par ה"ד אי דשכיח למזבן חמר נמי לעכב ואי דלא שכיח למזבן שוכר נמי לא לעכב}
\twocol{אמר רב פפא לא צריכא דשכיח למיטרח ולמזבן מאונא לאונא חמר דרכיה למיטרח ולמזבן שוכר לאו דרכיה למיטרח ולמזבן
\par אבוה דרב אחא בריה דרב איקא}
\twocol{הוה שפיך להו חמרא לעובדי כוכבים ואזיל מעבר להו מעברא ויהבו ליה גולפי באגרא אתו אמרו ליה לאביי א"ל כי קא טרח בהתירא קא טרח
\par והא רוצה בקיומו דלא נצטרו זיקי דמתני בהדייהו א"נ דמייתו פריסדקי בהדייהו}
\twocol{והא קא מעבר להו מעברא דקא טרח באיסורא דא"ל למברויא מעיקרא א"נ דנקיטי ביה קיטרי:
\par {\large\emph{מתני׳}} יין נסך שנפל ע"ג ענבים ידיחן והן מותרות ואם היו מבוקעות אסורות נפל ע"ג תאנים או על גבי תמרים אם יש בהן בנותן טעם אסור ומעשה בביתוס בן זונן שהביא גרוגרות בספינה ונשתברה חבית של יין נסך ונפל על גביהן ושאל לחכמים והתירום}
\twocol{זה הכלל כל שבהנאתו בנותן טעם אסור כל שאין בהנאתו בנותן טעם מותר כגון חומץ שנפל ע"ג גריסין:
\par {\large\emph{גמ׳}} מעשה לסתור חסורי מיחסרא והכי קתני אם נותן טעם לפגם הוא מותר ומעשה נמי בביתוס בן זונן שהיה מביא גרוגרות בספינה ונשתברה חבית של יין נסך ונפל על גביהן ובא מעשה לפני חכמים והתירום}
\twocol{ההוא כרי דחיטי דנפל עליה חביתא דיין נסך שרייה רבא לזבוניה לעובדי כוכבים
\par איתיביה רבה בר ליואי לרבא בגד שאבד בו כלאים ה"ז לא ימכרנה לעובד כוכבים ולא יעשנה מרדעת לחמור אבל עושה אותו תכריכין למת מצוה}
\twocol{לעובד כוכבים מ"ט לא דלמא אתי לזבוניה לישראל ה"נ אתי לזבוניה לישראל
\par הדר שרא למיטחינהו ולמפינהו ולזבונינהו לעובדי כוכבים שלא בפני ישראל}
\twocol{תנן יין נסך שנפל ע"ג ענבים ידיחן והן מותרות ואם היו מבוקעות אסורות מבוקעות אין שאין מבוקעות לא אמר רב פפא שאני חיטי הואיל ואגב צירייהו כמבוקעות דמיין}
\newsection{דף סו}
\twocol{חמרא עתיקא בענבי דברי הכל בנותן טעם חמרא חדתא בענבי אביי אמר במשהו ורבא אמר בנותן טעם
\par אביי אמר במשהו בתר טעמא אזלינן אידי ואידי חד טעמא הוא דהוה ליה מין במינו ומין במינו במשהו}
\twocol{ורבא אמר בנותן טעם בתר שמא אזלינן והאי שמא לחוד והאי שמא לחוד וה"ל מין בשאינו מינו ומין בשאינו מינו בנ"ט
\par תנן יין נסך שנפל ע"ג ענבים כו' קס"ד חמרא חדתא בענבי מאי לאו בנ"ט לא במשהו}
\twocol{הא מדקתני סיפא זה הכלל כל שבהנאתו בנותן טעם אסור כל שאין בהנאתו בנותן טעם מותר מכלל דבנותן טעם עסקינן
\par ואביי מתניתין בחמרא עתיקא בענבי}
\twocol{חלא דחמרא וחלא דשיכרא וחמירא דחיטי וחמירא דשערי אביי אמר בנותן טעם בתר טעמא אזלינן והאי טעמא לחוד והאי טעמא לחוד והוה ליה מין בשאינו מינו ומין בשאינו מינו בנותן טעם
\par ורבא אמר במשהו בתר שמא אזלינן והאי חלא מיקרי והאי חלא מיקרי והאי חמירא מיקרי והאי חמירא מיקרי וה"ל מין במינו וכל מין במינו במשהו}
\twocol{אמר אביי מנא אמינא לה דבתר טעמא אזלינן דתניא תבלין ב' וג' שמות והן מין אחד או מין ג' אסורין ומצטרפין ואמר חזקיה הכא במיני מתיקה עסקינן הואיל וראוין למתק בהן את הקדירה אי אמרת בשלמא בתר טעמא אזלינן כולי חד טעמא הוא אלא אי אמרת בתר שמא אזלינן האי שמא לחוד והאי שמא לחוד
\par ורבא אמר לך הא מני ר"מ היא דתניא רבי יהודה אומר משום רבי מאיר מנין לכל איסורין שבתורה שמצטרפין זה עם זה שנאמר (דברים יד, ג) לא תאכל כל תועבה כל שתיעבתי לך הרי הוא בבל תאכל}
\twocol{חלא לגו חמרא דברי הכל בנותן טעם חמרא לגו חלא אביי אמר במשהו ורבא אמר בנותן טעם
\par אביי אמר במשהו}
\twocol{ריחיה חלא וטעמא חמרא חלא והוה ליה מין במינו וכל מין במינו במשהו
\par רבא אמר בנותן טעם ריחיה חלא וטעמא חמרא חמרא והוה ליה מין בשאינו מינו וכל מין בשאינו מינו בנותן טעם}
\twocol{האי בת תיהא עובד כוכבים בדישראל ש"ד ישראל בדעובד כוכבים אביי אמר אסור רבא אמר מותר אביי אמר אסור ריחא מילתא היא רבא אמר מותר ריחא לאו מילתא היא
\par אמר רבא מנא אמינא לה דריחא ולא כלום הוא דתנן תנור שהסיקו בכמון של תרומה ואפה בו את הפת הפת מותרת לפי שאין טעם כמון אלא ריחא כמון ואביי שאני התם דמיקלא איסוריה}
\twocol{אמר רב מרי כתנאי הרודה פת חמה ונתנה ע"פ חבית של יין של תרומה ר"מ אוסר ור' יהודה מתיר רבי יוסי מתיר בשל חיטין ואוסר בשל שעורים מפני שהשעורים שואבות מאי לאו בהא קמיפלגי דמר סבר ריחא מילתא היא ומר סבר ריחא ולא כלום הוא
\par לרבא ודאי תנאי היא לאביי מי לימא תנאי היא}
\twocol{אמר לך אביי לאו מי איתמר עלה אמר רבה בר בר חנה אמר ר"ל בפת חמה וחבית פתוחה}
\newsection{דף סז}
\twocol{דברי הכל אסורה בפת צוננת וחבית מגופה דברי הכל מותרת לא נחלקו אלא בפת חמה וחבית מגופה בפת צוננת וחבית פתוחה והא דידי נמי כפת חמה וחבית פתוחה דמי:
\par זה הכלל כל שבהנאתו בנותן טעם כו': אמר רב יהודה אמר שמואל הכי הלכתא}
\twocol{ואמר רב יהודה אמר שמואל לא שנו אלא שנפל לתוך גריסין רותחין אבל נפל לתוך גריסין צוננין והרתיחן נעשה כמי שהשביח ולבסוף פגם ואסור
\par וכן כי אתא רבין אמר רבה בר בר חנה אמר ר' יוחנן לא שנו אלא שנפל לתוך גריסין רותחין אבל נפל לתוך גריסין צוננין והרתיחן נעשה כמי שהשביח ולבסוף פגם ואסור וכן כי אתא רב דימי כו' וכך היו עושין בערבי שבתות בציפורי וקוראין אותם שחליים}
\twocol{אמר ריש לקיש נותן טעם לפגם שאמרו לא שיאמרו קדירה זו חסירה מלח יתירה מלח חסירה תבלין יתירה תבלין אלא כל שאין חסירה כלום ואינה נאכלת מפני זה
\par ואיכא דאמרי אמר ריש לקיש נותן טעם לפגם שאמרו אין אומרין קדירה זו חסירה מלח יתירה מלח חסירה תבלין יתירה תבלין אלא השתא מיהא הא פגמה}
\twocol{אמר ר' אבהו אמר רבי יוחנן כל שטעמו וממשו אסור לוקין עליו וזהו כזית בכדי אכילת פרס
\par טעמו ולא ממשו אסור ואין לוקין עליו ואם ריבה טעם לפגם מותר}
\twocol{ולימא אם נתן טעם לפגם מותר הא קמשמע לן דאע"ג דאיכא מילי אחרנייתא דפגמה בהדיה והלכתא כלישנא בתרא דריש לקיש
\par אמר רב כהנא מדברי כולם נלמד נותן טעם לפגם מותר א"ל אביי בשלמא מכולהו לחיי אלא דר"ל אמרו קאמר וליה לא סבירא ליה}
\twocol{מכלל דאיכא למ"ד נותן טעם לפגם אסור
\par אין והתניא אחד נותן טעם לפגם ואחד נותן טעם לשבח אסור דברי ר"מ ר"ש אומר לשבח אסור ולפגם מותר}
\twocol{מ"ט דר"מ גמר מגיעולי עובדי כוכבים גיעולי עובדי כוכבים לאו נותן טעם לפגם הוא ואסר רחמנא ה"נ לא שנא
\par ואידך כדרב הונא בריה דרב חייא דאמר רב הונא בריה דרב חייא לא אסרה תורה אלא קדירה בת יומא דלא לפגם הוא ואידך קדירה בת יומא נמי אי אפשר דלא פגמה פורתא}
\twocol{ור"ש מאי טעמא דתניא (דברים יד, כא) לא תאכלו כל נבלה לגר אשר בשעריך כל הראויה לגר קרויה נבילה}
\newsection{דף סח}
\twocol{שאין ראויה לגר אינה קרויה נבלה
\par ור"מ ההוא למעוטי סרוחה מעיקרא ור"ש סרוחה מעיקרא לא צריכא מיעוטא עפרא בעלמא הוא}
\twocol{אמר עולא מחלוקת שהשביח ולבסוף פגם אבל פגם מעיקרא דברי הכל מותר
\par איתיביה רב חגא לעולא יין שנפל לתוך עדשים וחומץ שנפל לתוך גריסין אסור ור"ש מתיר והא הכא דפגם מעיקרא הוא ופליגי}
\twocol{אמר עולא חגא לא מידע ידע מאי קאמרי רבנן תיובתא קא מותיב הכא במאי עסקינן כגון שנפל לתוך גריסין צוננין והרתיחם נעשה כמי שהשביח ולבסוף פגם ואסור
\par ור' יוחנן אמר בפוגם מעיקרא מחלוקת}
\twocol{איבעיא להו בפוגם מעיקרא מחלוקת אבל השביח ולבסוף פגם דברי הכל אסור או דלמא בין בזו ובין בזו מחלוקת תיקו
\par אמר רב עמרם אפשר איתא להא דר' יוחנן ולא תניא לה במתניתין}
\twocol{נפק דק ואשכח דתניא שאור של חולין שנפל לתוך העיסה ויש בו כדי להחמיץ והחמיצה ואח"כ נפל שאור של תרומה או שאור של כלאי הכרם ויש בו כדי להחמיץ אסור ור"ש מתיר
\par והא הכא דפגם מעיקרא הוא ופליגי}
\twocol{א"ר זירא שאני עיסה הואיל וראויה לחמע בה כמה עיסות אחרות
\par ת"ש שאור של תרומה ושל חולין שנפלו לתוך העיסה בזה כדי להחמיץ ובזה כדי להחמיץ וחימצו אסור רבי שמעון מתיר נפל של תרומה תחלה ד"ה אסור נפל של חולין ואח"כ נפל של תרומה או של כלאי הכרם אסור ור"ש מתיר}
\twocol{והא הכא דפגם מעיקרא ופליגי וכי תימא ה"נ
\par כדרבי זירא ת"ש מסיפא היין שנפל לתוך עדשים וחומץ שנפל לתוך גריסין אסור ור"ש מתיר והא ה"נ דפגם מעיקרא ופליגי}
\twocol{וכי תימא ה"נ כדשני ליה עולא לרבי חגא כשהשביח ולבסוף פגם ומי פליגי כשהשביח ולבסוף פגם והא קתני נפל של תרומה תחלה דברי הכל אסור
\par אלא לאו ש"מ בפגם מעיקרא מחלוקת שמע מינה}
\twocol{הני תלתא בבי דקתני למה לי בשלמא בבא דסיפא קמ"ל בפוגם מעיקרא מחלוקת מציעתא נמי השביח ולבסוף פגם דברי הכל אסור
\par אלא רישא למה לי השתא ומה סיפא דלא קא משבח כלל אסרי רבנן רישא דקא משבח מיבעיא}
\twocol{אמר אביי רישא לר"ש אצטריך והכי קאמרי ליה רבנן לר"ש עיסה זו ראויה להחמיץ בשתי שעות מי גרם לה שתחמיץ בשעה אחת איסור
\par ור' שמעון כשהשביחו שניהם השביחו כשפגמו שניהם פגמו}
\twocol{לר"ש ליצטרף היתר ואיסור בהדי הדדי וליתסר
\par ר"ש לטעמיה דאמר אפי' איסור ואיסור נמי לא מיצטרפי}
\twocol{דתנן הערלה וכלאי הכרם מצטרפין ר"ש אומר אין מצטרפין
\par ההוא עכברא דנפל לחביתא דשיכרא אסריה רב לההוא שיכרא אמרוה רבנן קמיה דרב ששת נימא קסבר נט"ל אסור}
\twocol{אמר להו רב ששת בעלמא סבר רב נט"ל מותר והכא חידוש הוא דהא מימאס מאיס ובדילי אינשי מיניה ואפילו הכי אסריה רחמנא הלכך נט"ל נמי אסור
\par אמרו ליה רבנן לרב ששת אלא מעתה ליטמא לח ויבש אלמה תנן מטמאין לחים ואין מטמאין יבשים}
\twocol{ולטעמיך שכבת זרע תטמא לח ויבש אלמה תנן מטמאין לחין ואין מטמאין יבשין
\par אלא מאי אית לך למימר שכבת זרע אמר רחמנא בראויה להזריע ה"נ במותם אמר רחמנא כעין מותם}
\twocol{מתקיף לה רב שימי מנהרדעא ומי מאיס והלא עולה על שלחן של מלכים אמר רב שימי מנהרדעא לא קשיא הא בדדברא הא בדמתא
\par אמר רבא הלכתא נותן טעם לפגם מותר ועכברא בשיכרא לא ידענא מאי טעמא דרב אי משום דקסבר נותן טעם לפגם אסור ולית הלכתא כוותיה אי משום דקסבר נותן טעם לפגם מותר ועכברא בשיכרא אשבוחי משבח}
\twocol{איבעיא להו}
\newsection{דף סט}
\twocol{נפל לגו חלא מאי א"ל רב הילל לרב אשי הוה עובדא בי רב כהנא ואסר רב כהנא א"ל ההוא אימרטוטי אימרטט
\par רבינא סבר לשעורי במאה וחד אמר לא גרע מתרומה דתנן תרומה עולה באחד ומאה א"ל רב תחליפא בר גיזא לרבינא דלמא כתבלין של תרומה בקדירה דמי דלא בטיל טעמייהו}
\twocol{רב אחאי שיער בחלא בחמשין רב שמואל בריה דרב איקא שיער בשיכרא בשיתין
\par והלכתא אידי ואידי בשיתין וכן כל איסורין שבתורה:}
\twocol{{\large\emph{מתני׳}} עובד כוכבים שהיה מעביר עם ישראל כדי יין ממקום למקום אם היה בחזקת המשתמר מותר אם הודיעו שהוא מפליג כדי שישתום ויסתום ויגוב רשב"ג אומר כדי שיפתח את החבית ויגוף ותיגוב:
\par המניח יינו בקרון או בספינה והלך לו בקפנדריא נכנס למדינה ורחץ מותר}
\twocol{אם הודיעו שהוא מפליג כדי שישתום ויסתום ויגוב רשב"ג אומר כדי שיפתח את החבית ויגוף ותיגוב:
\par המניח עובד כוכבים בחנות אע"פ שיצא ונכנס מותר ואם הודיעו שהוא מפליג כדי שישתום ויסתום ויגוב רשב"ג אומר כדי שיפתח את החבית ויגוף ותיגוב:}
\twocol{היה אוכל עמו על השולחן והניח לגינין על השולחן ולגין על הדולבקי והניחו ויצא מה שעל השולחן אסור שעל הדולבקי מותר ואם אמר לו הוי מוזג ושותה אף שעל הדולבקי אסור חביות פתוחות אסורות סתומות מותרות כדי שיפתח ויגוף ותיגוב:
\par {\large\emph{גמ׳}} היכי דמי בחזקת המשתמר כדתניא הרי שהיו חמריו ופועליו טעונין טהרות אפילו הפליג מהן יותר ממיל טהרותיו טהורות ואם אמר להן לכו ואני בא אחריכם כיון שנתעלמה עינו מהם טהרותיו טמאות}
\twocol{מאי שנא רישא ומאי שנא סיפא אמר רב יצחק רישא במטהר חמריו ופועליו לכך
\par אי הכי סיפא נמי אין עם הארץ מקפיד על מגע חבירו אי הכי אפילו רישא נמי נימא הכי}
\twocol{אמר רבא
\par בבא להם דרך עקלתון אי הכי סיפא נמי כיון דאמר להם לכו ואני בא אחריכם סמכא דעתייהו:}
\twocol{המניח עובד כוכבים בחנותו כו' המניח יינו בקרון או בספינה כו': וצריכא דאי תנא עובד כוכבים דסבר דלמא אתי וחזי ליה אבל בקרון או בספינה אימא דמפליג לה לספינתיה ועביד מאי דבעי
\par ואי תנא בקרון או בספינה משום דסבר דלמא אתי באורחא אחריתי וקאי אגודא וחזי לי אבל עובד כוכבים בחנותו אימא אחיד לה לבבא ועביד כל דבעי קמ"ל}
\twocol{אמר רבה בר בר חנה א"ר יוחנן מחלוקת בשל סיד אבל בשל טיט דברי הכל כדי שיפתח ויגוף ויגוב
\par מיתיבי ארשב"ג לחכמים והלא סתומו ניכר בין מלמעלה ובין מלמטה}
\twocol{אי אמרת בשלמא בשל טיט מחלוקת היינו דקתני סתומו ניכר בין מלמעלה ובין מלמטה אלא אי אמרת בשל סיד מחלוקת בשלמא למטה ידיע אלא למעלה הא לא ידיע
\par רבן שמעון בן גמליאל הוא דלא' ידע מאי קאמרי רבנן וה"ק להו אי בשל טיט קאמריתו סתומו ניכר בין מלמעלה ובין מלמטה ואי בשל סיד קאמריתו נהי דלמעלה לא ידיע למטה מיהא ידיע ורבנן כיון דמלמעלה לא ידיע לא מסיק אדעתיה דאפיך וחזי ליה אי נמי זימנין דחלים}
\twocol{אמר רבא הלכה כרשב"ג הואיל ותנן סתמא כוותיה
\par דתנן היה אוכל על השולחן עמו והניח לגין על השולחן לגין על הדולבקי והניח ויצא מה שעל השולחן אסור מה שעל הדולבקי מותר ואם אמר לו הוי מזוג ושותה אף שעל הדולבקי אסור חביות פתוחות אסורות סתומות מותרות כדי שיפתח ויגוף ותיגוב}
\twocol{פשיטא מהו דתימא כולה רשב"ג קתני לה קמ"ל
\par וכי מאחר דקיימא לן כוותיה דרשב"ג דלא חייש לשתומא והלכתא כוותיה דרבי אליעזר דלא חייש לזיופא האידנא מאי טעמא לא מותבינן חמרא ביד עובדי כוכבים משום שייכא}
\twocol{אמר רבא זונה עובדת כוכבים וישראל מסובין אצלה חמרא שרי נהי דתקיף להו יצרא דעבירה}
\newsection{דף ע}
\twocol{יצרא דיין נסך לא תקיף להו זונה ישראלית ועובדי כוכבים מסובין חמרא אסור מ"ט הואיל וזילה עלייהו בתרייהו גרירא
\par ההוא ביתא דהוה יתיב ביה חמרא דישראל על עובד כוכבים אחדה לדשא באפיה והוה ביזעא בדשא אישתכח עובד כוכבים דקאי ביני דני אמר רבא כל דלהדי ביזעא שרי דהאי גיסא והאי גיסא אסור}
\twocol{ההוא חמרא דישראל דהוה יתיב בביתא דהוה דייר ישראל בעליונה ועובד כוכבים בתחתונה שמעו קל תיגרא נפקי קדים אתא עובד כוכבים אחדה לדשא באפיה אמר רבא חמרא שרי מימר אמר כי היכי דקדים אתאי אנא קדים ואתא ישראל ויתיב בעליונה וקא חזי לי
\par ההוא אושפיזא דהוה יתיב ביה חמרא דישראל אישתכח עובד כוכבים דהוה יתיב בי דני אמר רבא אם נתפס עליו כגנב שרי ואי לא אסיר}
\twocol{ההוא ביתא דהוה יתיב ביה חמרא אישתכח עובד כוכבים דהוה קאים בי דני אמר רבא אי אית ליה לאישתמוטי חמרא אסיר ואי לא חמרא שרי מיתיבי ננעל הפונדק או שאמר לו שמור אסור מאי לאו אע"ג דלית ליה לאישתמוטי לא בדאית ליה לאישתמוטי
\par ההוא ישראל ועובד כוכבים דהוו יתיבי וקא שתו חמרא שמע ישראל קל צלויי בי כנישתא קם ואזל אמר רבא חמרא שרי מימר אמר השתא מדכר ליה לחמריה והדר אתי}
\twocol{ההוא ישראל ועובד כוכבים דהוו יתיבי בארבא שמע ישראל קל שיפורי דבי שימשי נפק ואזל אמר רבא חמרא שרי מימר אמר השתא מדכר ליה לחמריה והדר אתי
\par ואי משום שבתא האמר רבא אמר לי איסור גיורא כי הוינן בארמיותן אמרינן יהודאי לא מנטרי שבתא דאי מנטרי שבתא כמה כיסי קא משתכחי בשוקא ולא ידענא דסבירא לן כרבי יצחק דא"ר יצחק המוצא כיס בשבת מוליכו פחות פחות מד' אמות}
\twocol{ההוא אריא דהוה נהים במעצרתא שמע עובד כוכבים טשא ביני דני אמר רבא חמרא שרי מימר אמר כי היכי דטשינא אנא איטשא נמי ישראל אחוריי וקא חזי לי
\par הנהו גנבי דסלקי לפומבדיתא ופתחו חביתא טובא אמר רבא חמרא שרי מ"ט רובא גנבי ישראל נינהו הוה עובדא בנהרדעי ואמר שמואל חמרא שרי}
\twocol{כמאן כרבי אליעזר דאמר ספק ביאה טהור
\par דתנן הנכנס לבקעה בימות הגשמים וטומאה בשדה פלונית ואמר הלכתי במקום הלז ואיני יודע אם נכנסתי לאותה שדה אם לא נכנסתי ר"א אומר ספק ביאה טהור ספק מגע טמא}
\twocol{לא שאני התם כיון דאיכא דפתחי לשום ממונא הוה ליה ספק ספיקא
\par ההיא רביתא דאישתכח דהות בי דני והות נקיטא אופיא בידה אמר רבא חמרא שרי אימר מגבה דחביתא שקלתיה ואע"ג דליכא תו אימר אתרמויי איתרמי לה}
\twocol{ההוא פולמוסא דסליק לנהרדעא פתחו חביתא טובא כי אתא רב דימי אמר עובדא הוה קמיה דרבי אלעזר ושרא ולא ידענא אי משום דסבר לה כרבי אליעזר דאמר ספק ביאה טהור אי משום דסבר רובא דאזלי בהדי פולמוסא ישראל נינהו
\par א"ה האי ספק ביאה ספק מגע הוא כיון דמפתחי טובא אימא אדעתא דממונא פתחו וכספק ביאה דמי}
\twocol{ההיא מסוביתא דמסרה לה איקלידא מפתחה לעובדת כוכבים א"ר יצחק א"ר אלעזר עובדא הוה בי מדרשא ואמרו לא מסרה לה אלא שמירת מפתח בלבד
\par אמר אביי אף אנן נמי תנינא המוסר מפתחות לע"ה טהרותיו טהורות לפי שלא מסר לו אלא שמירת מפתח בלבד השתא טהרותיו טהורות יין נסך מיבעיא}
\twocol{למימרא דטהרות אלימי מיין נסך אין דאיתמר חצר שחלקה במסיפס אמר רב טהרותיו טמאות ובעובד כוכבים אינו עושה יין נסך ורבי יוחנן אמר אף טהרותיו טהורות
\par מיתיבי הפנימית של חבר והחיצונה של ע"ה אותו חבר שוטח שם פירות ומניח שם כלים ואע"פ שידו של עם הארץ מגעת לשם קשיא לרב}
\twocol{אמר לך רב שאני התם שנתפס עליו כגנב
\par ת"ש רשב"ג אומר גגו של חבר למעלה מגגו של ע"ה אותו חבר שוטח שם פירות ומניח שם כלים ובלבד שלא תהא ידו של ע"ה מגעת לשם קשיא לרבי יוחנן}
\twocol{אמר לך רבי יוחנן שאני התם דאית ליה לאישתמוטי מימר אמר אימצורי קא ממצרא
\par ת"ש גגו של חבר בצד גגו של עם הארץ אותו חבר שוטח שם פירות ומניח שם כלים ואע"פ שידו של עם הארץ מגעת לשם קשיא לרב אמר לך רב לאו איכא ר"ש בן גמליאל דקאי כוותי אנא דאמרי כר"ש בן גמליאל:}
\twocol{{\large\emph{מתני׳}} בולשת שנכנסה לעיר בשעת שלום חביות פתוחות אסורות סתומות מותרות בשעת מלחמה אלו ואלו מותרות לפי שאין פנאי לנסך: {\large\emph{גמ׳}} }
\newsection{דף עא}
\twocol{ורמינהי עיר שכבשוה כרקום כל כהנות שבתוכה פסולות אמר רב מרי לנסך אין פנאי לבעול יש פנאי:
\par {\large\emph{מתני׳}} האומנין של ישראל ששלח להם עובד כוכבים חבית של יין נסך בשכרן מותר לומר תן לנו את דמיה משנכנסה לרשותן אסור:}
\twocol{{\large\emph{גמ׳}} אמר רב יהודה אמר רב מותר לאדם לומר לעובד כוכבים צא והפס עלי מנת המלך
\par מיתיבי אל יאמר אדם לעובד כוכבים עול תחתי לעוצר אמר ליה רב עול תחתי לעוצר קאמרת הא לא דמיא אלא להא אבל אומר לו מלטני מן העוצר:}
\twocol{{\large\emph{מתני׳}} המוכר יינו לעובד כוכבים פסק עד שלא מדד דמיו מותרין מדד עד שלא פסק דמיו אסורין:
\par {\large\emph{גמ׳}} אמר אמימר משיכה בעובד כוכבים קונה תדע דהני פרסאי משדרי פרדשני להדדי ולא הדרי בהו רב אשי אמר לעולם אימא לך משיכה בעובד כוכבים אינה קונה והאי דלא הדרי בהו דרמות רוחא הוא דנקיטא להו}
\twocol{אמר רב אשי מנא אמינא לה מדאמר להו רב להנהו סבויתא כי כייליתו חמרא לעובדי כוכבים שקלו זוזי מינייהו והדר כיילן להו ואי לא נקיטו בהדייהו זוזי אוזיפונהו והדר שקילו מינייהו כי היכי דתיהוי הלואה גבייהו דאי לא עבדיתו הכי כי קא הוי יין נסך ברשותייכו קא הוי וכי שקילתו דמי יין נסך קא שקילתו ואי ס"ד משיכה בעובד כוכבים קונה
\par מדמשכה עובד כוכבים קנייה יין נסך לא הוי עד דנגע ביה}
\twocol{אי דקא כייל ורמי למנא דישראל ה"נ לא צריכא דקא כייל ורמי למנא דעובד כוכבים
\par סוף סוף כי מטא לאוירא דמנא קנייה יין נסך לא הוי עד דמטי לארעיתיה דמנא ש"מ נצוק חבור}
\twocol{לא אי דנקיט ליה עובד כוכבים לכלי בידיה ה"נ לא צריכא דמנח אארעא
\par ותיקני ליה כליו שמעת מינה כליו של לוקח ברשות מוכר לא קנה לוקח}
\twocol{לא לעולם אימא לך קנה לוקח והכא במאי עסקינן כגון דאיכא עכבת יין אפומיה דכוזנתא דקמא קמא אינסיך ליה
\par וכמאן דלא כרשב"ג דאי רשב"ג האמר ימכר כולו לעובדי כוכבים חוץ מדמי יין נסך שבו}
\twocol{מידי הוא טעמא אלא לרב האמר רב הלכה כרשב"ג חבית בחבית אבל לא יין ביין
\par מיתיבי הלוקח גרוטאות מן העובדי כוכבים ומצא בהן עבודת כוכבים אם עד שלא נתן מעות משך יחזיר אם משנתן מעות משך יוליך לים המלח אי ס"ד משיכה בעובד כוכבים קונה אמאי יחזיר אמר אביי משום דמיחזי כי מקח טעות}
\twocol{אמר רבא רישא מקח טעות סיפא לאו מקח טעות אלא אמר רבא רישא וסיפא מקח טעות ורישא דלא יהיב זוזי לא מיתחזי כעבודת כוכבים ביד ישראל סיפא דיהיב זוזי מיתחזי כעבודת כוכבים ביד ישראל
\par א"ל מר קשישא בריה דרב חסדא לרב אשי ת"ש המוכר יינו לעובד כוכבים פסק עד שלא מדד דמיו מותרים ואי אמרת משיכה בעובד כוכבים אינה קונה אמאי דמיו מותרין הכא במאי עסקינן דאקדים ליה דינר}
\twocol{א"ה אימא סיפא מדד עד שלא פסק דמיו אסורין ואי דקדים ליה דינר אמאי דמיו אסורין
\par א"ל ולדידך דאמרת משיכה בעובד כוכבים קונה אמאי רישא דמיו מותרין וסיפא דמיו אסורין}
\twocol{אלא מאי אית לך למימר פסק סמכא דעתיה לא פסק לא סמכא דעתיה
\par לדידי נמי אע"ג דקדים ליה דינר פסק סמכא דעתיה לא פסק לא סמכא דעתיה}
\twocol{א"ל רבינא לרב אשי ת"ש דאמר ר' חייא בר אבא א"ר יוחנן בן נח נהרג על פחות משוה פרוטה ולא ניתן להישבון ואי אמרת משיכה בעובד כוכבים אינה קונה אמאי נהרג
\par משום דצעריה לישראל}
\newsection{דף עב}
\twocol{ומאי לא ניתן להישבון דאינו בתורת הישבון
\par אי הכי אימא סיפא בא חבירו ונטלה ממנו נהרג עליה בשלמא רישא משום דצעריה לישראל אלא סיפא מאי עביד}
\twocol{אלא ש"מ משיכה בעובד כוכבים קונה ש"מ
\par ההוא גברא דא"ל לחבריה אי מזביננא לה להא ארעא לך מזביננא לה אזל זבנה לאיניש אחרינא אמר רב יוסף קנה קמא}
\twocol{א"ל אביי והא לא פסק ומנא תימרא דכל היכא דלא פסק לא קנה דתנן המוכר יינו לעובד כוכבים פסק עד שלא מדד דמיו מותרין מדד עד שלא פסק דמיו אסורין
\par מאי הוי עלה מאי הוי עלה כדקאמרינן דלמא חומרא דיין נסך שאני}
\twocol{ת"ש דאמר רב אידי בר אבין עובדא הוה בי רב חסדא ורב חסדא בי רב הונא ופשטיה מהא דתנן משך חמריו ופועליו והכניסן לתוך ביתו בין פסק עד שלא מדד ובין מדד עד שלא פסק לא קנה ושניהן יכולין לחזור בהן
\par פרקן והכניסן לתוך ביתו פסק עד שלא מדד אין שניהן יכולין לחזור בהן מדד עד שלא פסק שניהן יכולין לחזור בהן}
\twocol{ההוא גברא דאמר ליה לחבריה אי מזביננא לה להא ארעא מזביננא לך במאה זוזי אזל זבנה לאיניש אחרינא במאה ועשרין אמר רב כהנא קנה קמא מתקיף לה רב יעקב מנהר פקוד האי זוזי אנסוהו והלכתא כרב יעקב מנהר פקוד
\par א"ל כדשיימי בתלתא אפילו תרי מגו תלתא כדאמרי בתלתא עד דאמרי בתלתא כדשיימי בארבעה עד דאמרי בארבעה וכ"ש היכא דא"ל כדאמרי בארבעה}
\twocol{א"ל כדשיימי בתלתא ואתו תלתא ושמוה וא"ל אידך ליתו תלתא אחריני דקים להו טפי אמר רב פפא דינא הוא דמעכב מתקיף לה רב הונא בריה דרב יהושע ממאי דהני קים להו טפי דלמא הני קים להו טפי והלכתא כרב הונא בריה דרב יהושע:
\par {\large\emph{מתני׳}} נטל את המשפך ומדד לתוך צלוחיתו של עובד כוכבים וחזר ומדד לתוך צלוחיתו של ישראל אם יש בו עכבת יין אסור המערה מכלי אל כלי את שעירה ממנו מותר ואת שעירה לתוכו אסור:}
\twocol{{\large\emph{גמ׳}} תנן התם הנצוק והקטפרס ומשקה טופח אינו חיבור לא לטומאה ולא לטהרה האשבורן חיבור לטומאה ולטהרה
\par אמר רב הונא נצוק וקטפרס ומשקה טופח חיבור לענין יין נסך}
\twocol{אמר ליה רב נחמן לרב הונא מנא לך הא אילימא מדתנן הנצוק והקטפרס ומשקה טופח אינו חיבור לא לטומאה ולא לטהרה לטומאה ולטהרה הוא דלא הוי חיבור הא לענין יין נסך הוי חיבור אימא סיפא האשבורן חיבור לטומאה ולטהרה לטומאה ולטהרה הוא דהוי חיבור הא לענין יין נסך לא הוי חיבור אלא מהא ליכא למשמע מינה
\par תנן נטל את המשפך ומדד לתוך צלוחיתו של עובד כוכבים וחזר ומדד לתוך צלוחיתו של ישראל}
\twocol{אם יש בו עכבת יין אסור הא עכבת יין במאי קא מתסרא לאו בנצוק ש"מ נצוק חיבור
\par תני ר' חייא שפחסתו צלוחיתו אבל לא פחסתו צלוחיתו מאי לא תפשוט דנצוק אינו חיבור לא פחסתו צלוחיתו תפשוט לך דאסור נצוק תיבעי}
\twocol{ת"ש המערה מכלי לכלי את שמערה ממנו מותר הא דביני ביני אסור ש"מ נצוק חיבור
\par אי נצוק חיבור אפילו דגויה דמנא נמי ליתסר הא לא קשיא דקא מקטיף קטופי מ"מ נצוק חיבור}
\twocol{ולטעמיך אימא סיפא את שעירה לתוכו הוא דאסיר הא דביני ביני שרי אלא מהא ליכא למשמע מינה
\par ת"ש המערה מחבית לבור קילוח היורד משפת חבית למטה אסור תרגמה רב ששת בעובד כוכבים המערה דאתי מכחו}
\twocol{אי עובד כוכבים המערה אפי' גוא דחביתא נמי מתסר כח דעובד כוכבים מדרבנן הוא דאסיר ההוא דנפק לבראי גזרו ביה רבנן ההוא דלגואי לא גזרו ביה רבנן
\par אמר להו רב חסדא להנהו סביתא כי כייליתו חמרא לעובדי כוכבים קטפי קטופי אי נמי נפצי נפוצי אמר להו רבא להנהו שפוכאי כי שפכיתו חמרא לא ליקרב עובד כוכבים לסייע בהדייכו דלמא משתליתו ושדיתו ליה עליה וקאתי מכחו ואסיר}
\twocol{ההוא גברא דאסיק חמרא בגישתא ובת גישתא אתא עובד כוכבים אנח ידיה אגישתא אסריה רבא לכוליה חמרא
\par א"ל רב פפא לרבא וא"ל רב אדא בר מתנה לרבא ואמרי לה רבינא לרבא במאי בנצוק שמעת מינה נצוק חיבור שאני התם דכולי חמרא אגישתא ובת גישתא גריר}
\twocol{אמר מר זוטרא בריה דרב נחמן קנישקנין שרי וה"מ דקדים פסק ישראל אבל קדם פסק עובד כוכבים לא רבה בר רב הונא איקלע לבי ריש גלותא שרא להו למשתא בקנישקנין}
\newsection{דף עג}
\twocol{איכא דאמרי רבה בר רב הונא גופיה אישתי בקנישקנין:
\par {\large\emph{מתני׳}} יין נסך אסור ואוסר בכל שהוא יין ביין ומים במים בכל שהוא יין במים ומים ביין בנותן טעם}
\twocol{זה הכלל מין במינו במשהו ושלא במינו בנותן טעם:
\par {\large\emph{גמ׳}} כי אתא רב דימי א"ר יוחנן המערה יין נסך מחבית לבור אפילו כל היום כולו ראשון ראשון בטל}
\twocol{תנן יין נסך אסור ואוסר בכל שהוא מאי לאו דקא נפיל איסורא לגו התירא לא דקא נפיל התירא לגו איסורא
\par ת"ש יין במים בנותן טעם מאי לאו דקא נפיל חמרא דאיסורא למיא דהתירא לא דקא נפיל חמרא דהתירא למיא דאיסורא}
\twocol{ומדרישא במיא דאיסורא סיפא נמי במיא דאיסורא וקתני סיפא מים ביין בנותן טעם אמר לך רב דימי כולה מתני' התירא לגו איסורא ורישא דקא נפיל חמרא דהתירא למיא דאיסורא סיפא דקא נפיל מיא דהתירא לחמרא דאיסורא
\par כי אתא רב יצחק בר יוסף א"ר יוחנן המערה יין נסך מצרצור קטן לבור אפילו כל היום כולו ראשון ראשון בטל ודוקא צרצור קטן דלא נפיש עמודיה אבל חבית דנפיש עמודיה לא}
\twocol{כי אתא רבין אמר רבי יוחנן יין נסך שנפל לבור ונפל שם קיתון של מים רואין את ההיתר כאילו אינו והשאר מים רבין עליו ומבטלין אותו
\par כי אתא רב שמואל בר יהודה א"ר יוחנן לא שנו אלא שנפל קיתון של מים תחלה אבל לא נפל שם קיתון של מים תחלה מצא מין את מינו וניעור}
\twocol{איכא דמתני לה אמתני' יין ביין כל שהוא אמר רב שמואל בר יהודה א"ר יוחנן לא שנו אלא שלא נפל שם קיתון של מים אבל נפל שם קיתון של מים רואין את ההיתר כאילו אינו והשאר מים רבין עליו ומבטלין אותו
\par מאי איכא בין לדמתני לה אמתני' בין לדמתני לה אדרבין מאן דמתני לה אמתני' לא בעי תחלה ומאן דמתני לה אדרבין בעי תחלה}
\twocol{איתמר יין נסך שנפל לבור ונפל שם קיתון של מים
\par אמר חזקיה הגדילו באיסור אסור הגדילו בהיתר מותר}
\twocol{ורבי יוחנן אמר אפי' הגדילו באיסור מותר
\par א"ל רבי ירמיה לרבי זירא לימא חזקיה ור' יוחנן בפלוגתא דר"א ורבנן קמיפלגי}
\twocol{דתנן שאור של חולין ושל תרומה שנפלו לתוך העיסה לא בזה כדי לחמץ ולא בזה כדי לחמץ ונצטרפו וחמצו
\par ר"א אומר אחר אחרון אני בא וחכ"א בין שנפל איסור בתחלה ובין בסוף אינו אסור עד שיהא בו כדי להחמיץ}
\twocol{ותסברא והאמר אביי לא שנו אלא שקדם וסילק את האיסור אבל לא קדם וסילק את האיסור אסור חזקיה דאמר כמאן
\par אלא הכא ברואין קמיפלגי לחזקיה לית ליה רואין לרבי יוחנן אית ליה}
\twocol{ומי אית ליה לרבי יוחנן רואין והא בעי מיניה ר' אסי מרבי יוחנן שני כוסות אחד של חולין ואחד של תרומה ומזגן ועירבן זה בזה מהו ולא פשט ליה
\par מעיקרא לא פשט ליה לבסוף פשט ליה אתמר נמי א"ר אמי א"ר יוחנן ואמרי לה א"ר אסי א"ר יוחנן ב' כוסות אחד של חולין ואחד של תרומה ומזגן ועירבן זה בזה רואין את ההיתר כאילו אינו והשאר מים רבין עליו ומבטלין אותו:}
\twocol{זה הכלל מין במינו במשהו שלא במינו בנותן טעם:
\par רב ושמואל דאמרי תרוייהו כל איסורין שבתורה במינן במשהו שלא במינן בנותן טעם}
\twocol{זה הכלל לאתויי מאי לאתויי כל איסורין שבתורה
\par ר' יוחנן ור"ל דאמרי תרוייהו כל איסורין שבתורה בין במינן בין שלא במינן בנותן טעם חוץ מטבל ויין נסך במינן במשהו ושלא במינן בנותן טעם וזה הכלל לאתויי טבל}
\twocol{תניא כוותיה דרב ושמואל תניא כוותיה דרבי יוחנן ור"ל
\par תניא כוותיה דרב ושמואל כל איסורין שבתורה במינן במשהו שלא במינן בנותן טעם}
\twocol{תניא כוותיה דר' יוחנן ור"ל כל איסורין שבתורה בין במינן בין שלא במינן בנותן טעם חוץ מטבל ויין נסך במינן במשהו שלא במינן בנותן טעם
\par בשלמא יין נסך משום חומרא דעבודת כוכבים אלא טבל מ"ט}
\twocol{כהיתירו כך איסורו דאמר שמואל חטה אחת פוטרת את הכרי ותניא נמי הכי במה אמרו טבל אוסר בכל שהוא במינו שלא במינו בנותן טעם:}
\newsection{דף עד}
\twocol{{\large\emph{מתני׳}} אלו אסורין ואוסרין בכל שהו יין נסך ועבודת כוכבים ועורות לבובין
\par ושור הנסקל ועגלה ערופה}
\twocol{וציפורי מצורע ושער נזיר ופטר חמור ובשר בחלב ושעיר המשתלח וחולין שנשחטו בעזרה הרי אלו אסורין ואוסרין בכל שהוא:
\par {\large\emph{גמ׳}} תנא מאי קחשיב אי דבר שבמנין קחשיב ליתני נמי חתיכות נבילה אי איסורי הנאה קא חשיב ליתני נמי חמץ בפסח א"ר חייא בר אבא ואיתימא ר' יצחק נפחא האי תנא תרתי אית ליה דבר שבמנין ואיסורי הנאה}
\twocol{וליתני אגוזי פרך ורימוני בדן דדבר שבמנין ואיסורי הנאה הוא
\par הא תנא ליה התם הראוי לערלה ערלה הראוי לכלאי הכרם כלאי הכרם}
\twocol{וליתני ככרות של בעה"ב לענין חמץ בפסח מאן שמעת דא"ל ר"ע הא תנא ליה התם ר"ע מוסיף אף ככרות של בעה"ב:
\par הרי אלו: למעוטי מאי למעוטי דבר שבמנין ולאו איסורי הנאה א"נ למעוטי איסורי הנאה ולא דבר שבמנין:}
\twocol{{\large\emph{מתני׳}} יין נסך שנפל לבור כולו אסור בהנאה רשב"ג אומר ימכר כולו לעובד כוכבים חוץ מדמי יי"נ שבו:
\par {\large\emph{גמ׳}} אמר רב הלכה כרשב"ג חבית בחביות אבל לא יין ביין ושמואל אמר אפי' יין ביין וכן אמר רבב"ח א"ר יוחנן אפי' יין ביין וכן א"ר שמואל בר נתן א"ר חנינא אפי' יין ביין וכן א"ר נחמן אמר רבה בר אבוה אפי' יין ביין}
\twocol{א"ר נחמן הלכה למעשה יי"נ יין ביין אסור חבית בחבית מותר סתם יין אפי' יין ביין מותר:
\par {\large\emph{מתני׳}} גת של אבן שזפתה עובד כוכבים מנגבה והיא טהורה ושל עץ רבי אומר ינגב וחכ"א יקלוף את הזפת ושל חרס אע"פ שקלף את הזפת הרי זו אסורה:}
\twocol{{\large\emph{גמ׳}} אמר רבא דוקא זפתה אבל דרך בה לא פשיטא זפתה תנן מהו דתימא הוא הדין אפילו דרך בה והאי דקתני זפתה אורחא דמלתא קתני קמ"ל
\par איכא דאמרי אמר רבא דוקא זפתה אבל דרך בה לא סגי לה בניגוב פשיטא זפתה תנן מהו דתימא ה"ה דאפילו דרך בה והאי דקתני זפתה אורחא דמלתא קתני קמ"ל דוקא זפתה אבל דרך בה לא סגי לה בניגוב}
\twocol{כי ההוא דאתא לקמיה דרבי חייא א"ל הב לי גברא דדכי לי מעצרתאי א"ל לרב זיל בהדיה וחזי דלא מצוחת עלי בי מדרשא אזל חזייה דהוה שיעא טפי אמר הא ודאי בניגוב סגי לה בהדי דקא אזיל ואתי חזא פילא מתותיה וחזא דהוה מלא חמרא אמר הא לא סגי לה בניגוב אלא בקילוף והיינו דא"ל חביבי חזי דלא מצוחת עלי בי מדרשא
\par ת"ר הגת והמחץ והמשפך של עובדי כוכבים רבי מתיר בניגוב וחכמים אוסרין ומודה רבי בקנקנים של עובדי כוכבים שהן אסורין ומה הפרש בין זה לזה זה מכניסו בקיום וזה אין מכניסו בקיום ושל עץ ושל אבן ינגב ואם היו מזופפין אסורין}
\twocol{והתנן גת של אבן שזפתה עובד כוכבים מנגבה והיא טהורה מתניתין דלא דרך בה ברייתא דדרך בה
\par אמר מר הגת והמחץ והמשפך של עובדי כוכבים רבי מתיר בניגוב וחכמים אוסרין והאנן תנן של חרס אע"פ שקלף את הזפת הרי זו אסורה אמר רבא סיפא דמתני' אתאן לרבנן}
\twocol{דרש רבא נעוה ארתחו רבא כי הוה משדר גולפי להרפניא סחיף להו אפומייהו וחתים להו אבירצייהו קסבר כל דבר שמכניסו לקיום אפילו לפי שעה גזרו ביה רבנן
\par במה מנגבן רב אמר במים רבה בר בר חנה אמר באפר רב אמר במים במים ולא באפר רבה בר בר חנה אמר באפר באפר ולא במים אלא}
\newsection{דף עה}
\twocol{רב אמר במים והוא הדין לאפר רבה בר בר חנה אמר לאפר והוא הדין למים ולא פליגי הא ברטיבתא הא ביבשתא
\par איתמר בי רב משמיה דרב אמרי תרתי תלת ושמואל אמר תלת ד'}
\twocol{בסורא מתנו הכי בפומבדיתא מתנו בי רב אמרי משמיה דרב תלת ד' ושמואל אמר ד' ה'
\par ולא פליגי מר קא חשיב מיא בתראי ומר לא קחשיב מיא בתראי}
\twocol{בעו מיניה מרבי אבהו הני גורגי דארמאי מאי
\par א"ל רבי אבהו תניתוה הרי שהיו גתיו ובית בדיו טמאין ובקש לעשותן בטהרה הדפין והעדשין והלולבין מדיחן והעקלין של נצרין ושל בצבוץ מנגבן ושל שיפה ושל גמי מישנן י"ב חדש רשב"ג אומר מניחן מגת לגת ומבד לבד}
\twocol{היינו תנא קמא איכא בינייהו חורפי ואפלי
\par ר' יוסי אומר הרוצה לטהרן מיד מגעילן ברותחין או חולטן במי זיתים רשב"ג משום ר' יוסי אומר מניחן תחת צינור שמימיו מקלחין או במעין שמימיו רודפין וכמה עונה}
\twocol{כדרך שאמרו ביין נסך כך אמרו בטהרות
\par כלפי לייא בטהרות קיימינן אלא כדרך שאמרו בטהרות כך אמרו ביין נסך}
\twocol{כמה עונה אמר רבי חייא בר אבא א"ר יוחנן או יום או לילה ר' חנא שאינה ואמרי לה ר' חנא בר שאינה אמר רבה בר בר חנה א"ר יוחנן חצי יום וחצי לילה
\par א"ר שמואל בר יצחק ולא פליגי הא בתקופת ניסן ותשרי הא בתקופת תמוז וטבת}
\twocol{אמר רבי יהודה הני רווקי דארמאי דמזיא מדיחן דעמרא מנגבן דכיתנא מישנן ואי איכא קטרי שרי להו הני דקולי וחלאתא דארמאי דחיטי בחבלי דצורי מדיחן
\par דצבתא לנגבן דכיתנא מישנן ואי אית בהו קיטרי שרי להו}
\twocol{איתמר עם הארץ שהושיט ידו לגת ונגע באשכולות רבי ורבי חייא חד אמר אשכול וכל סביבותיו טמאין וכל הגת כולה טהורה וחד אמר כל הגת כולה נמי טמאה
\par ולמ"ד אשכול וכל סביבותיו טמאים וכל הגת כולה טהורה מאי שנא מהא דתנן שרץ שנמצא ברחים אינו מטמא אלא מקום מגעו ואם היה משקין מהלך הכל טמא}
\twocol{התם לא מפסק ולא מידי הכא מפסקי אשכולות
\par אורו ליה רבנן לר' ירמיה ואמרי לה לבריה דרבי ירמיה כדברי האומר אשכול וכל סביבותיו טמאין וכל הגת כולה טהורה:}
\twocol{{\large\emph{מתני׳}} הלוקח כלי תשמיש מן העובדי כוכבים את שדרכו להטביל יטביל להגעיל יגעיל ללבן באור ילבן באור השפוד והאסכלא מלבנן באור הסכין שפה והיא טהורה:
\par {\large\emph{גמ׳}} תנא וכולן צריכין טבילה בארבעים סאה מנהני מילי אמר רבא דאמר קרא (במדבר לא, כג) כל דבר אשר יבא באש תעבירו באש וטהר הוסיף לך הכתוב טהרה אחרת}
\twocol{תני בר קפרא מתוך שנאמר (במדבר לא, כג) במי נדה שומע אני שצריך הזאה שלישי ושביעי ת"ל אך חלק
\par א"כ מה ת"ל במי נדה מים שנדה טובלת בהן הוי אומר ארבעים סאה}
\twocol{איצטריך למיכתב וטהר ואיצטריך למיכתב במי נדה אי כתב וטהר ה"א וטהר כל דהו כתב רחמנא במי נדה
\par ואי כתב רחמנא במי נדה הוה אמינא הערב שמש כנדה כתב רחמנא וטהר לאלתר}
\twocol{אמר רב נחמן אמר רבה בר אבוה אפי' כלים חדשים במשמע דהא ישנים וליבנן כחדשים דמו ואפילו הכי בעי טבילה מתקיף לה רב ששת אי הכי אפי' זוזא דסרבלא נמי א"ל כלי סעודה אמורין בפרשה
\par אמר רב נחמן אמר רבה בר אבוה לא שנו אלא בלקוחין וכמעשה שהיה אבל שאולין לא}
\twocol{רב יצחק בר יוסף זבן מנא דמרדא מעובד כוכבים סבר להטבילה א"ל ההוא מרבנן ורבי יעקב שמיה לדידי מפרשא לי מיניה דרבי יוחנן כלי מתכות אמורין בפרשה
\par אמר רב אשי הני כלי זכוכית הואיל וכי נשתברו יש להן תקנה ככלי מתכות דמו קוניא פליגי בה רב אחא ורבינא חד אמר כתחלתו וחד אמר כסופו והלכתא כסופו}
\twocol{איבעיא להו משכנתא מאי אמר מר בר רב אשי אבא משכן ליה עובד כוכבים כסא דכספא ואטבליה ואישתי ביה ולא ידענא אי משום דקסבר משכנתא כזביני דמיא אי משום דחזי לעובד כוכבים דדעתיה לשקועיה:
\par ת"ר הלוקח כלי תשמיש מן העובדי כוכבים דברים שלא נשתמש בהן מטבילן והן טהורין דברים שנשתמש בהן ע"י צונן כגון כוסות וקתוניות וצלוחיות מדיחן ומטבילן והם טהורין דברים שנשתמש בהן ע"י חמין כגון היורות הקומקמוסון ומחמי חמין מגעילן ומטבילן והן טהורין דברים שנשתמש בהן ע"י האור כגון השפודין והאסכלאות מלבנן ומטבילן והן טהורין}
\twocol{וכולן שנשתמש בהן עד שלא יטביל ושלא יגעיל ושלא ילבן תני חדא אסור ותניא אידך מותר
\par ל"ק הא כמאן דאמר נותן טעם לפגם אסור הא כמאן דאמר נותן טעם לפגם מותר}
\twocol{ולמאן דאמר נותן טעם לפגם מותר גיעולי עובדי כוכבים דאסר רחמנא היכי משכחת לה
\par אמר רב חייא בריה דרב הונא לא אסרה תורה אלא קדירה}
\newsection{דף עו}
\twocol{בת יומא דלאו נותן טעם לפגם הוא
\par מכאן ואילך לישתרי גזירה קדירה שאינה בת יומא משום קדירה בת יומא}
\twocol{ואידך קדירה בת יומא נמי מפגם פגמה
\par רמי ליה רב עמרם לרב ששת תנן השפודין והאסכלא מלבנן באור (והתני') גבי קדשים השפוד והאסכלא מגעילן בחמין}
\twocol{א"ל עמרם ברי מה ענין קדשים אצל גיעולי עובדי כוכבים הכא היתירא בלע התם איסורא בלע
\par אמר רבא סוף סוף כי קא פליט איסורא קא פליט אלא אמר רבא מאי הגעלה נמי שטיפה ומריקה}
\twocol{א"ל אביי מי דמי מריקה ושטיפה בצונן הגעלה בחמין אלא אמר אביי (איוב לו, לג) יגיד עליו רעו תנא הכא ליבון והוא הדין להגעלה תנא התם הגעלה וה"ה לליבון
\par א"ל רבא אי הכי לתנינהו לכולהו בחדא וליתני חדא באידך אחריתי ולימא יגיד עליו רעו}
\twocol{אלא אמר רבא קדשים היינו טעמייהו כדרב נחמן אמר רבה בר אבוה דאמר כל יום ויום נעשה גיעול לחבירו
\par תינח שלמים דכיון דלשני ימים מיתאכלי מקמי דניהוי נותר קא הוי גיעול אלא חטאת כיון דליום ולילה מיתאכלא כי מבשל בה האידנא חטאת הוי נותר כי הדר מבשל בה למחר או שלמים או חטאת קא פליט נותר דחטאת דהאידנא בחטאת ושלמים דלמחר}
\twocol{אמרי לא צריכא דכי מבשל בה חטאת האידנא הדר מבשל בה האידנא שלמים
\par דחטאת דלמחר ושלמים דאתמול בהדי הדדי קא שלים זמנייהו והדר מבשל שלמים דלמחר}
\twocol{א"ה הגעלה נמי לא ליבעי קשיא
\par רב פפא אמר האי קריד האי לא קריד}
\twocol{רב אשי אמר לעולם כדאמרן מעיקרא הכא התירא בלע הכא איסורא בלע
\par ודקא קשיא לך דבעידנא דקא פליט איסורא קא פליט בעידנא דקא פליט לא איתיה לאיסורא בעיניה:}
\twocol{ועד כמה מלבנן א"ר מני עד שתשיר קליפתן וכיצד מגעילן א"ר הונא יורה קטנה בתוך יורה גדולה
\par יורה גדולה מאי ת"ש דההוא דודא דהואי בי רב עקביה אהדר ליה}
\twocol{גדנפא דלישא אפומא ומליוה מיא וארתחה אמר רבא מאן חכים למעבד כי הא מילתא אי לאו רב עקביה דגברא רבא הוא קסבר כבולעו כך פולטו מה בולעו בנצוצות אף פולטו בנצוצות:
\par הסכין שפה והיא טהורה: אמר רב עוקבא בר חמא ונועצה עשרה פעמים בקרקע אמר רב הונא בריה דרב יהושע ובקרקע שאינה עבודה א"ר כהנא ובסכין יפה שאין בה גומות תניא נמי הכי סכין יפה שאין בה גומות נועצה עשרה פעמים בקרקע אמר רב הונא בריה דרב יהושע לאכול בה צונן}
\twocol{כי הא דמר יהודה ובאטי בר טובי הוו יתבי קמיה דשבור מלכא אייתו לקמייהו אתרוגא פסק אכל פסק והב ליה לבאטי בר טובי הדר דצה עשרה זימני בארעא פסק הב ליה למר יהודה א"ל באטי בר טובי וההוא גברא לאו בר ישראל הוא א"ל מר קים לי בגויה ומר לא קים לי בגויה
\par איכא דאמרי א"ל אידכר מאי עבדת באורתא:}
\twocol{\par \par {\large\emph{הדרן עלך השוכר את הפועל וסליקא לה מסכת עבודה זרה}}\par \par }

\end{document}
